Bestiario

\begin{multicols}{2}
Mostri
Le linee guida per comprendere le informazioni trovate
nelle statistiche dei mostri sono presentate di seguito.
Statistiche
Le statistiche di un mostro, a volte dette blocco
statistiche, forniscono le informazioni essenziali di cui
hai bisogno per dirigere il mostro.
Modificare le Creature
Nonostante la versatile collezione di mostri in questo documento,
potresti comunque trovarti in imbarazzo quando si tratta di
trovare la creatura perfetta per una tua avventura. Sentiti libro di
modificare le creature esistenti e trasformarle in qualcosa che ti
sia più utile, magari prendendo in prestito uno o due tratti da un
mostro diverso o usando una variante o archetipo, come quelli
presentati in questo documento. Tieni a mente che modificare un
mostro, anche applicando un archetipo, potrebbe cambiarne il
grado di sfida.
Taglia
Un mostro può essere di taglia Minuscola, Piccola,
Media, Grande, Enorme o Mastodontica. La tabella
Categorie di Taglia mostra quanto spazio una creatura
di una specifica taglia controlli in combattimento.
Categorie di Taglia
Taglia Spazio Esempio
Minuscola 75 x 75 cm Diavoletto, spiritello
Piccola 1,5 x 1,5 m Ratto gigante, goblin
Media 1,5 x 1,5 m Orco, lupo mannaro
Grande 3 x 3 m Ippogrifo, ogre
Enorme 4,5 x 4,5 m Gigante del fuoco, ent
Mastodontico 6 x 6 m o più Kraken, verme purpureo
Tipo
Il tipo di un mostro si riferisce alla sua natura basilare.
Certi incantesimi, oggetti magici, privilegi di classe e
altri effetti del gioco interagiscono in modi speciali con
le creature di un tipo specifico. Ad esempio, una freccia
ammazza draghi infligge danni extra non solo ai draghi
ma anche a tutte le altre creature del tipo drago, come i
draghi tartaruga e le viverne.
Il gioco comprende i seguenti tipi di mostri, che non
hanno regole specifiche.
Aberrazioni, creature totalmente aliene. Molte di esse
possiedono innate abilità magiche che attingono alla
mente aliena della creatura anziché dalle forze mistiche
del mondo. Esempi classici di aberrazioni sono aboleti,
osservatori, scortica mente e i batraci del caos.
Bestie, creature non umanoidi che sono una
componente naturale di un mondo fantasy. Alcune
possiedono poteri magici, ma la maggior parte è priva
di intelligenza e non ha alcuna forma di società o
linguaggio. Esempi classici di bestie sono tutte le
specie di animali comuni, i dinosauri e le versioni
giganti degli animali.
Celestiali, creature native dei Piani Superiori. Molti di
loro sono servitori delle divinità, impiegati come
messaggeri o agenti nel mondo dei mortali e per i piani.
I celestiali sono di natura buona, e quindi l’eccezionale
celestiale che devia dall’allineamento buono è una
orribile rarità. Esempi classici di celestiali sono angeli,
couatl e pegasi.
Costrutti, sono creati e non partoriti. Alcuni sono
programmati dai loro creatori per seguire una semplice
serie di istruzioni, mentre altri sono senzienti e capaci di
pensare per proprio conto. I golem sono i costrutti più
rappresentativi.
Draghi, sono grandi creature rettili di antica origine ed
enorme potere. I veri draghi, compresi i buoni draghi
metallici e i malvagi draghi cromatici, sono molto
intelligenti e possiedono doti magiche innate. In questa
categoria si collocano anche creature lontanamente
imparentate con i veri draghi, ma meno potenti, meno
intelligenti e meno magiche, come le viverne e gli
pseudodraghi.
Elementali, sono creature native dei piani elementali.
Alcune creature di questo tipo sono poco più che
masse animate del rispettivo elemento, e includono le
creature chiamate semplicemente elementali. Altre
creature possiedono forme biologiche infuse di energia
elementale. Le razze dei geni, compresi djinn ed efreet,
formano le civiltà più importanti dei piani elementali.
Altre creature elementali sono gli azer, i persecutori
invisibili e le bizzarrie d’acqua.
Fatati, sono creature magiche strettamente legate alle
forze della natura. Vivono in radure crepuscolari e
foreste nebbiose. Esempi di fatati sono driadi, pixie e
satiri.
Giganti, troneggiano sugli umani e i loro simili. Sono di
forma umana, sebbene alcuni abbiano più teste (ettin) o
deformità (fomori). Le sei varianti dei veri giganti sono
gigante di collina, gigante di pietra, gigante del gelo,
gigante del fuoco, gigante delle nuvole, gigante delle
tempeste. Oltre questi, anche ogri e troll sono giganti.
Immondi, creature perverse native dei Piani Inferi.
Alcune sono al servizio di divinità, ma molte di più
operano agli ordini di arcidiavoli e principi demoni. A
volte sacerdoti e maghi malvagi evocano gli immondi
nel mondo materiale perché eseguano le loro volontà.
Se un celestiale malvagio è una rarità, un immondo
buono è praticamente inconcepibile. Gli immondi
includono demoni, diavoli, segugi infernali e rakshasa.
Melme, sono creature gelatinose che difficilmente
hanno una forma fissa. Vivono principalmente
sottoterra, stabilendosi in grotte e sotterranei,
nutrendosi di rifiuti, carcasse o creature tanto sfortunate
da incapparvi. I protoplasmi neri e i cubi gelatinosi sono
tra le melme più riconoscibili.
Mostruosità, sono mostri nel senso più stretto del
termine – creature spaventose che non sono comuni,
né davvero naturali, e quasi mai benigne. Alcune sono
il risultato di esperimenti magici andati male (come
l’orsogufo), mentre altri sono il prodotto di terribili
maledizioni (tra cui annoveriamo il minotauro).
Sfuggono a qualsiasi categorizzazione, e in qualche
modo servono da categoria onnicomprensiva per quelle
creature che non corrispondono a nessun altro tipo di
mostro.
Non Morti, sono creature un tempo vive condotte ad un
orribile stato di non morte tramite la pratica della magia
negromantica o qualche blasfema maledizione. Tra i
non morti si annoverano cadaveri ambulanti, come
vampiri e zombi, e spiriti incorporei, come fantasmi e
spettri.
Piante, in questo contesto si tratta di creature vegetali,
non della normale flora. La maggior parte di esse sono
mobili, e alcune sono carnivore. L’esempio più classico 
 
di piante sono i tumuli ambulanti e gli ent. Anche le
creature fungoidi come le spore gassose e i miconidi
rientrano in questa categoria.
Umanoidi, sono la popolazione principale dei mondi di
gioco, civilizzati e selvaggi, comprendono gli umani e
un’ampia gamma di altre specie. Possiedono una
lingua e una cultura, poche o nessuna abilità magica
innata (sebbene molti umanoidi possano apprendere gli
incantesimi), e una forma bipede. Le razze più comuni
di umanoide sono quelle più adatte come personaggi
del giocatore: umani, nani, elfi e halfling. Quasi
altrettanto numerose, ma più brutali e selvagge, e quasi
tutte malvagie, sono le razze goblinoidi (goblin,
hobgoblin e bugbear), orchi, gnoll, lucertoloidi e coboldi.
Etichette
Un mostro può presentare una o più etichette indicate
tra parentesi, a seguire il suo tipo. Ad esempio un orco
ha il tipo umanoide (orco). Le etichette tra parentesi
forniscono ulteriori categorizzazioni per determinate
creature. Le etichette non hanno delle proprie regole
specifiche, ma alcuni elementi del gioco, come gli
oggetti magici, vi possono fare riferimento. Ad esempio,
una lancia particolarmente efficace contro i demoni,
funzionerebbe contro qualsiasi mostro che abbia
l’etichetta demone.
Allineamento
L’allineamento del mostro fornisce indicazioni sul suo
comportamento e come agisca in situazioni di
interpretazione o di combattimento. Ad esempio,
potrebbe risultare difficile discutere con un mostro
caotico malvagio, che potrebbe attaccare i personaggi
a vista, mentre un mostro neutrale potrebbe essere
disposto a negoziare.
L’allineamento specificato nel blocco statistiche del
mostro è la norma. Sentiti libero di deviarne e
modificare l’allineamento del mostro per adattarlo alle
necessità della tua campagna. Se vuoi un drago verde
di allineamento buono o un gigante delle tempeste
malvagio, non c’è nulla che te lo vieti.
Alcune creature possono avere qualsiasi
allineamento. In altre parole, sta a te scegliere
l’allineamento del mostro. La voce allineamento di un
mostro a volte riporta una tendenza o un’avversione per
legge, caos, bene o male. Ad esempio, un berserker
può essere di qualsiasi allineamento caotico (caotico
buono, caotico neutrale o caotico malvagio), come
appropriato alla sua natura selvaggia.
Molte creature di scarso intelletto non comprendono la
differenza tra legge, caos, bene o male. Non effettuano
scelte etiche o morali, ma piuttosto agiscono d’istinto.
Queste creature sono disallineate, che vuol dire che
non hanno un allineamento.
Classe Armatura
Un mostro che indossa un’armatura o trasporta uno
scudo ha una Classe Armatura (Difesa) che tiene conto
dell’armatura, lo scudo e della Destrezza. Altrimenti, la
Difesa di un mostro è basata sul suo modificatore di
Destrezza e l’armatura naturale, se la possiede. Se un
mostro possiede un’armatura naturale, indossa
armature o trasporta uno scudo, viene indicato tra
parentesi dopo il valore della sua Difesa.
Punti Ferita
Di solito quando scende a 0 punti ferita, un mostro
muore o viene distrutto.
I punti ferita di un mostro sono presentati sia come un
insieme di dadi che come valore medio. Ad esempio,
un mostro con 2d8 punti ferita ha di media 9 punti ferita
(2 x 4,5).
La taglia di un mostro determina il dado impiegato per
calcolare i suoi punti ferita, come mostrato sulla tabella
Dadi Vita per Taglia.
Dadi Vita per Taglia
Taglia del Mostro Dado Vita PF Medi per Dado
Minuscola d4 2,5
Piccola d6 3,5
Media d8 4,5
Grande d10 5,5
Enorme d12 6,5
Mastodontico d20 10,5
Anche il modificatore di Costituzione di un mostro
influenza il numero di punti ferita che possiede. Il suo
modificatore di Costituzione viene moltiplicato per il
numero di Dadi Vita che possiede, e il risultato viene
sommato ai suoi punti ferita. Ad esempio, un mostro ha
Costituzione 12 (modificatore +1) e 2d8 Dadi Vita, e
avrà quindi 2d8+2 punti ferita (media 11).
Velocità
La velocità di un mostro ti dice di quanto si possa
muovere durante il suo turno.
Tutte le creature possiedono una velocità di passeggio,
detta semplicemente velocità del mostro. Le creature
che non possiedono alcuna forma di spostamento
terreno hanno velocità di passeggio 0 metri.
Alcune creature possiedono uno o più dei seguenti
modi di movimento aggiuntivi.
Nuoto
Un mostro che possiede una velocità di nuoto non deve
spendere movimento extra per nuotare.
Scalata
Un mostro che possiede una velocità di scalata può
usare tutto o solo parte del suo movimento per
muoversi su superfici verticali. Il mostro non deve
spendere movimento extra per scalare.
Scavo
Un mostro che possiede una velocità di scavo può
usare la sua velocità per attraversare sabbia, terra,
fango, ecc. Un mostro non può scavare attraverso la
roccia solida a meno che non possieda un tratto
speciale che glielo permetta.
Volo
Un mostro che possiede una velocità di volo può usare
tutto o solo parte del suo movimento per volare. Alcuni
mostri hanno l’abilità di fluttuare, che li rende difficili da
abbattere. Il mostro smette di fluttuare quando muore.
Punteggi di Caratteristica
Ogni mostro possiede sei punteggi di caratteristica
(Forza, Destrezza, Costituzione, Intelligenza, Saggezza
e Carisma) e i corrispondenti modificatori.
Tiri Salvezza
La voce Tiri Salvezza è riservata alle creature
predisposte a resistere a certi effetti. Ad esempio, una
creatura che non viene facilmente affascinata o
spaventata può ottenere un bonus ai suoi tiri salvezza
di Saggezza. La maggior parte delle creature non gode
di bonus speciali ai tiri salvezza, nel qual caso questa
sezione è assente.
Un bonus ai tiri salvezza è la somma del modificatore
della caratteristica relativa del mostro e del suo bonus
di competenza, che viene determinato dal suo grado di
sfida (come mostrato sulla tabella Bonus di
Competenza per Grado di Sfida).
Bonus di Competenza per Grado di Sfida
Bonus di Bonus di
Sfida Competenza Sfida Competenza
0 +2 14 +5
1/8 +2 15 +5
1/4 +2 16 +5
1/2 +2 17 +6
1 +2 18 +6
2 +2 19 +6
3 +2 20 +6
4 +2 21 +7
5 +3 22 +7
6 +3 23 +7
7 +3 24 +7
8 +3 25 +8
9 +4 26 +8
10 +4 27 +8
11 +4 28 +8
12 +4 29 +9
13 +4 30 +9
Abilità
La voce Abilità è riservata a quei mostri che sono
competenti in una o più abilità. Ad esempio, un mostro
che è molto attento e furtivo potrebbe avere bonus alle
prove di Saggezza (Percezione) e Destrezza (Furtività).
Un bonus di abilità è la somma del modificatore di
caratteristica appropriato di un mostro e del suo bonus
di competenza, che viene determinato dal grado di
sfida del mostro (come mostrato sulla tabella Bonus di
Competenza per Grado di Sfida). Si possono applicare
anche altri modificatori. Ad esempio, un mostro
potrebbe avere un bonus più grande del previsto (di
solito il doppio del suo bonus di competenza) per
tenere conto della sua grande perizia.
Vulnerabilità, Resistenze e
Immunità
Alcune creature possiedono vulnerabilità, resistenze o
immunità ad un certo tipo di danno. Creature particolari
sono addirittura resistenti o immuni agli attacchi non
magici (un attacco magico è un attacco sferrato tramite
un incantesimo, un oggetto magico, o un’altra fonte di
magia). Inoltre, certe creature sono immuni a
determinate condizioni. Se un mostro è immune ad un
effetto di gioco che non viene considerato danno o
condizione, possiede invece un tratto speciale.
Sensi
La voce Sensi indica il punteggio passivo di Saggezza
(Percezione), oltre a qualsiasi senso speciale di cui il
mostro sia in possesso. I sensi speciali sono descritti di
seguito.
Percezione Tellurica
Un mostro con percezione tellurica può individuare e
trovare le origini delle vibrazioni entro uno specifico
raggio, purché il mostro e la fonte della vibrazione siano
in contatto con lo stesso terreno o sostanza. La
percezione tellurica non può essere impiegata per
individuare creature volanti o incorporee. Molte creature
scavatrici, come gli ankheg e i colossi di terra,
possiedono questo senso speciale.
Scurovisione
Una creatura con scurovisione può vedere nell’oscurità
fino ad una specifica gittata. Fino al limite della gittata, il
mostro può vedere a luce fioca come fosse luce
intensa, e nell’oscurità come fosse luce fioca. La
creatura non può discernere i colori al buio, solo
tonalità di grigio. Molte creature che vivono sottoterra
possiedono questo senso speciale.
Visione del Vero
Un mostro con la visione del vero può, fino ad una
specifica gittata, vedere attraverso l’oscurità normale e
magica, vedere creature e oggetti invisibili,
automaticamente individuare le illusioni e riuscire i tiri
salvezza contro di loro, e percepire la forma originale di
un mutaforma o di una creatura trasformata dalla
magia. Inoltre, la creatura può vedere nel Piano Etereo
fino alla stessa gittata.
Vista Cieca
Una creatura con vista cieca può percepire l’ambiente
circostante, senza fare affidamento alla vista, fino ad
una specifica gittata.
Le creature senza occhi, come i grimlock e le melme, e
le creature con ecolocazione o sensi potenziati, come i
pipistrelli e i draghi puri, possiedono questo senso.
Se un mostro è cieco di natura, la cosa viene annotata
tra parentesi, ad indicare che la gittata della sua vista
cieca definisce anche la gittata massima della sua
percezione.
 
Linguaggi
Le lingue che un mostro può parlare sono riportate in
ordine alfabetico. A volte un mostro può capire una
lingua ma non parlarla, e la cosa viene indicata a
questa voce. Una “-” indica che la creatura non parla né
comprende alcuna lingua.
Telepatia
La telepatia è un’abilità magica che permette ad un
mostro di comunicare mentalmente con un’altra
creatura nel raggio di azione specificato. La creatura
contattata non deve parlare la stessa lingua del mostro
per comunicare in questo modo, ma deve essere in
grado di comprendere almeno una lingua. Una creatura
senza telepatia può ricevere e rispondere a messaggi
telepatici ma non può iniziare o terminare una
conversazione telepatica.
Un mostro telepatico non ha bisogno di vedere la
creatura contattata e può terminare il contatto telepatico
in qualsiasi momento. Il contatto è infranto non appena
le due creature non si trovano più entro il raggio di
azione o se il mostro telepatico contatta un’altra
creatura a gittata. Un mostro telepatico può iniziare o
terminare una conversazione telepatica senza dover
usare un’azione, ma mentre il mostro è inabile, non può
dare inizio ad un contatto telepatico, e qualsiasi
contatto in corso viene terminato.
Una creatura nell’area di un campo anti-magia o in
qualsiasi altro posto in cui la magia non funziona non
può inviare o ricevere messaggi telepatici.
Sfida
Il grado di sfida di un mostro vi dice quanto sia grande
la minaccia che pone. Una compagnia di quattro
avventurieri equipaggiata in maniera appropriata e
riposata dovrebbe essere in grado di sconfiggere un
mostro dal grado di sfida pari al proprio livello medio
senza subire perdite. Ad esempio, una compagnia di
quattro personaggi di 3° livello dovrebbe ritenere un
mostro di grado di sfida 3 una degna sfida, ma non
letale.
I mostri che sono significativamente più deboli dei
personaggi di 1° livello hanno un grado di sfida inferiore
ad 1. I mostri con un grado di sfida 0 non presentano
problemi eccetto in grandi numeri; quelli privi di reali
attacchi non valgono punti esperienza, mentre quelli
che possono attaccare valgono 10 PE ciascuno.
Alcuni mostri presentano una sfida superiore a quelle
che anche una compagnia di 20° livello sia in grado di
gestire. Questi mostri hanno grado di sfida 21 o
superiore e sono progettati proprio per mettere alla
prova le capacità dei personaggi.
Punti Esperienza
Il numero di punti esperienza (PE) che vale un mostro è
basato sul suo grado di sfida. Di solito, vengono
assegnati PE per la sconfitta di un mostro, sebbene
l’Arbitro possa assegnare PE anche per l’aver
neutralizzato la minaccia posta dal mostro in qualche
altra maniera.
A meno che non sia indicato altrimenti, un mostro
richiamato da un incantesimo o altra abilità magica vale
i PE indicati nel suo blocco statistiche.
Punti Esperienza per Grado di Sfida
Sfida PE Sfida PE
0 0 o 10 14 11.500
1/8 25 15 13.000
1/4 50 16 15.000
1/2 100 17 18.000
1 200 18 20.000
2 450 19 22.000
3 700 20 25.000
4 1.100 21 33.000
5 1.800 22 41.000
6 2.300 23 50.000
7 2.900 24 62.000
8 3.900 25 75.000
9 5.000 26 90.000
10 5.900 27 105.000
11 7.200 28 120.000
12 8.400 29 135.000
13 10.000 30 155.000
Tratti Speciali
I tratti speciali (che compaiono dopo il grado di sfida di un
mostro ma prima di qualsiasi azione o reazione) sono
peculiarità che avranno probabilmente un ruolo in un
incontro di combattimento e che richiedono delle
spiegazioni.
Incantesimi
Un mostro con il privilegio di classe Incantesimi ha un
livello da incantatore e slot incantesimi, che impiega per
lanciare i suoi incantesimi di 1° livello o più alto. Il livello
dell’incantatore viene impiegato anche per i trucchetti
compresi dal privilegio.
Il mostro ha una lista di incantesimi conosciuti o preparati di
una specifica classe. La lista può includere anche incantesimi
forniti da un privilegio di quella classe, come il privilegio
Dominio Divino di un chierico. Il mostro è considerato
membro di quella classe quando entra in sintonia o usa un
oggetto magico che richieda l’appartenenza alla classe o
l’accesso alla sua lista degli incantesimi.
Un mostro può lanciare un incantesimo dalla sua lista ad un
livello più alto se ha lo slot incantesimo per farlo. Ad
esempio, un drow mago con l’incantesimo di 3° livello
invocare il fulmine può lanciarlo come incantesimo di 5°
livello, utilizzando uno dei suoi slot incantesimo di 5° livello.
Puoi anche cambiare gli incantesimi conosciuti o
preparati da un mostro, rimpiazzando qualsiasi
incantesimo sulla lista degli incantesimi di un mostro
con un incantesimo diverso dello stesso livello e della
stessa lista di classe. Se lo fai, potresti far sì che il
mostro rappresenti una minaccia inferiore o superiore a
quanto suggerito dal suo grado di sfida.
Incantesimi Innati
Un mostro con l’abilità innata di lanciare incantesimi ha il
tratto speciale Incantesimi Innati. A meno che non sia
indicato altrimenti, un incantesimo innato di 1° livello o più
alto viene sempre lanciato al più basso livello possibile, e
non può essere lanciato a livello più alto. Se un mostro ha
un trucchetto dove conta il livello e non ne viene fornito
alcuno, usare il grado di sfida del mostro.
Un incantesimo innato può essere sottoposto a speciali
regole o restrizioni. Ad esempio, un elfo oscuro mago può
eseguire in maniera innata l’incantesimo levitazione, ma
l’incantesimo ha la restrizione “solo personale”, ad indicare
che ha effetto solo sull’elfo oscuro mago.
Gli incantesimi innati di un mostro non possono essere
scambiati con altri incantesimi. Se gli incantesimi innati di
un mostro non necessitano di un tiro di attacco, non gli
viene fornito alcun bonus di attacco.
Azioni
Quando un mostro svolge le sue azioni, può scegliere tra
le opzioni della sezione Azioni del suo blocco statistiche o
impiegare una delle azioni disponibili a tutte le creature,
come Scattare o Nascondersi.
Attacchi da Mischia e a Distanza
L’azione più comune che un mostro effettuerà in
combattimento, sarà un attacco da mischia o a
distanza. Possono essere attacchi con incantesimi o
attacchi con armi, dove l’“arma” può essere un
manufatto o un’arma naturale, come gli artigli o la coda
chiodata.
Creatura contro Bersaglio. Il bersaglio di un attacco
da mischia o a distanza è di solito una creatura o un
bersaglio, la differenza nel fatto che un “bersaglio” può
essere una creatura o un oggetto.
Colpisce. Qualsiasi danno inflitto o altro effetto che
avviene come risultato di un attacco che colpisce il
bersaglio viene descritto nell’annotazione “Colpisce”.
Puoi scegliere se prendere il danno medio o tirare i
dadi; per questo motivo vengono presentati sia il danno
medio che una formula di dadi.
Manca. Se un attacco ha un effetto prodotto da un
colpo a vuoto, quell’informazione viene fornita
dall’annotazione “Manca”.
Danni. Se un mostro impugna armi manufatte, infligge
danni appropriati all’arma. I mostri più grossi di solito
impugnano armi di dimensioni superiori che infliggono
danni extra quando colpiscono. Raddoppiare i dadi
dell’arma se la creatura è Grande, triplicarli se Enorme
e quadruplicarli se Mastodontica.
Una creatura ha svantaggio ai tiri per colpire con un’arma
costruita per una taglia superiore alla sua. L’Arbitro può
decidere che le armi di due o più taglie più grandi di quella
dell’attaccante sono del tutto impossibili da usare.
Multiattacco
Una creatura che può effettuare più attacchi durante il
suo turno ha l’abilità Multiattacco. Una creatura non può
usare Multiattacco quando effettua un attacco di
opportunità, il quale deve essere un singolo attacco da
mischia.
Regole dell’Afferrare per i Mostri
Molti mostri possiedono un attacco speciale che gli permette di
afferrare rapidamente la preda. Quando un mostro colpisce con
un simile attacco, non deve effettuare un’ulteriore prova di
caratteristica per determinare se l’afferrare riesce, a meno che
l’attacco non dica altrimenti.
Una creatura afferrata dal mostro può usare la sua azione per
tentare di sfuggirgli. Per farlo, deve riuscire una prova di Forza
(Atletica) o Destrezza (Acrobazia) contro la DC di fuga nel
blocco statistiche del mostro. Se non viene fornita una DC di
fuga, assumere che la DC sia uguale a 10 + il modificatore di
Forza (Atletica) del mostro.
Munizioni
Un mostro porta con sé munizioni sufficienti per
effettuare i suoi attacchi a distanza. Puoi presumere
che un mostro abbia 2d4 proiettili per un attacco con
armi da lancio, e 2d10 proiettili per un’arma a proiettili
come un arco o una balestra.
Reazioni
Se un mostro può compiere qualcosa di speciale con le
sue reazioni, è riportato qui. Se una creatura non ha
reazioni speciali, questa sezione è assente.
Uso Limitato
Alcune abilità speciali hanno restrizioni sul numero di
volte che possono essere usate.
X/Giorno. L’annotazione “X/Giorno” indica un’abilità
speciale che può essere usata X volte prima che il
mostro debba terminare un riposo lungo per recuperare
gli usi consumati. Ad esempio, “1/Giorno” indica
un’abilità speciale che può essere usata una volta
prima che il mostro debba terminare un riposo lungo
per poterla riutilizzarla.
Ricarica X-Y. L’annotazione “Ricarica X-Y” indica che il
mostro può usare un’abilità speciale una volta e che
l’abilità ha una probabilità casuale di ricaricarsi ogni round
seguente di combattimento. All’inizio di ciascun turno del
mostro, tira un d6. Se il risultato è uno dei numeri
dell’annotazione di ricarica, il mostro recupera l’uso
dell’abilità speciale. L’abilità si ricarica subito quando un
mostro termina un riposo breve o lungo.
Ad esempio, “Ricarica 5-6” indica che un mostro può
usare la sua abilità speciale una volta. Poi, all’inizio del
turno del mostro, recupera l’uso dell’abilità se tira 5 o 6 su
di un d6.
Ricarica dopo un Riposo Breve o Lungo. Questa
annotazione indica che un mostro può usare un’abilità
speciale una volta e poi terminare un riposo breve o
lungo per utilizzarla di nuovo.
Equipaggiamento
Il blocco statistiche si riferisce all’equipaggiamento,
oltre le armi o le armature utilizzate dal mostro. Una
creatura che normalmente indossa abiti, come un
umanoide, si assume sia abbigliato in maniera
appropriata.
Puoi equipaggiare i mostri con ulteriore
equipaggiamento o ninnoli come preferisci, utilizzando il
capitolo “Equipaggiamento” come fonte di ispirazione, e
sei tu a decidere quanto dell’equipaggiamento del
mostro è recuperabile dopo che la creatura è stata
uccisa o se qualsiasi parte del suo equipaggiamento sia
ancora utilizzabile. Ad esempio, un’armatura 
 
ammaccata fatta per un mostro difficilmente sarà
utilizzabile da qualcun altro.
Se un mostro incantatore necessita di componenti
materiali per lanciare i suoi incantesimi, dai per
scontato che abbia le componenti materiali per lanciare
gli incantesimi nel suo blocco statistiche.
Creature Leggendarie
Una creatura leggendaria può fare cose che le normali
creature non possono fare. Le creature leggendarie
possono eseguire azioni speciali al di fuori del proprio
turno, e alcune possono estendere il proprio potere
all’ambiente, provocando l’avvenimento di effetti magici
straordinari nelle loro vicinanze.
Azioni Leggendarie
Una creatura leggendaria può effettuare un certo
numero di azioni speciali – dette azioni aggiuntive – al
di fuori del suo turno. Solo un’azione aggiuntiva può
essere usata alla volta e solo al termine del turno di
un’altra creatura. Una creatura leggendaria recupera
all’inizio del suo turno le azioni aggiuntive che ha
usato. Non è obbligata ad usare le sue azioni
leggendarie, e non può usare le azioni aggiuntive
mentre è inabile o altrimenti incapace di effettuare
azioni. Se sorpresa, non può usarle fin dopo il suo
primo turno di combattimento.
Se una creatura assume la forma di una creatura
leggendaria, magari tramite un incantesimo, non ne
ottiene però le azioni aggiuntive, le azioni da tana, o
gli effetti regionali.
La Tana di una Creatura
Leggendaria
Una creatura leggendaria può presentare una sezione
che ne descrive la tana e gli effetti speciali che vi può
creare mentre si trova lì, o per propria volontà o
semplicemente grazie alla sua presenza. Questa
sezione si applica solo alle creature leggendarie che
trascorrono molto tempo nelle loro tane ed è altamente
probabile che vi vengano incontrate.
Azioni da Tana
Se una creatura leggendaria ha un’azione da tana, può
usarla per imbrigliare la magia ambientale della sua
tana. Al conteggio di iniziativa 20, perdendo i pareggi,
la creatura può usare una delle sue opzioni di azioni da
tana. Non può farlo mentre è inabile o altrimenti
incapace di effettuare azioni. Se sorpresa, non può
farne uso fino a dopo il suo primo turno di
combattimento.
Effetti Regionali
La semplice presenza di una creatura leggendaria può
avere effetti strani e meravigliosi sull’ambiente, come
indicato in questa sezione. Gli effetti regionali
terminano all’istante o si dissipano col tempo una volta
morta la creatura leggendaria.

\textbf{Aboleth}\index{Aboleth}\\
Taglia: Grande - \hspace*{0pt}\hfill{Tipo: aberrazione}\\
Morale: malvagio\\
Potenza 5, Agilita' -1, Intelletto 4, Volontà 2, Magnetismo 4\\
Difesa: 20 - \hspace*{0pt}\hfill{Punti Ferita: 135}
Movimento 3 m, nuoto 12 m\\
Tiri Salvezza: Tempra +6, Riflessi +1, Arbitrio +6\\
Competenze: Consapevolezza +10, Cultura +12\\
Sensi: visione crepuscolare 36 m\\
Linguaggi: Acquan, telepatia 36 m\\
CR: 10\\
\textit{Anfibio}. L’aboleth può respirare aria e acqua.\\
\textit{Nube di Muco}. Mentre è sott’acqua, l’aboleth è avvolto da muco mutante. Una creatura che entri a contatto con l’aboleth, o che lo colpisca con un attacco da mischia mentre si trova entro 1 metro da esso, deve effettuare un tiro salvezza di Costituzione DC 14. Se lo fallisce, la creatura resta ammalata per 1d4 ore. La creatura ammalata può respirare solo sott’acqua.\\
\textit{Sonda Telepatica}. Se una creatura comunica telepaticamente con  l’aboleth, e l’aboleth può vederla, l’aboleth ne apprende i più grandi desideri.\\
\textit{Azioni Multiattacco}. L’aboleth effettua tre attacchi con i tentacoli\\
\textit{Tentacolo}. Attacco con arma da mischia: +9 a colpire, portata 3 m, un bersaglio.
Colpisce: 12 (2d6 + 5) danni da botta. Se il bersaglio è una creatura, deve riuscire un tiro salvezza di Costituzione DC 14 o divenire ammalato. La malattia non produce alcun effetto per 1 minuto e può essere rimossa da qualsiasi magia che curi le malattie. Dopo 1 minuto, la pelle della creatura ammalata diventa trasparente e viscida, la creatura non può recuperare punti ferita a meno che non sia sott’acqua, e la malattia può essere rimossa solo da guarire o un altro incantesimo cura malattie di 6° livello o più alto. Quando la creatura si trova al di fuori di un corpo d’acqua, subisce 6 (1d12) danni da acido ogni 10 minuti a meno che la sua pelle non vengabagnata prima che siano passati questi  10 minuti.\\
\textit{Coda}. Attacco con arma da mischia: +9 a colpire, portata 3 m, un bersaglio.
Colpisce: 15 (3d6 + 5) danni da botta.\\
\textit{Schiavizzare} (3/Giorno). L’aboleth prende a bersaglio una creatura che può vedere entro 9 metri da esso. Il bersaglio deve riuscire un tiro salvezza di Saggezza DC 14 o restare affascinato magicamente dall’aboleth finché l’aboleth muore o i due si trovano su piani di esistenza differenti. Il bersaglio affascinato è sotto il controllo dell’aboleth e non può effettuare reazioni. L’aboleth e il bersaglio possono comunicare telepaticamente tra di loro a qualsiasi distanza.  Ogni qualvolta il bersaglio affascinato subisce danni, può ripetere il tiro salvezza. Se lo riesce, l’effetto termina. Non più di una volta ogni 24 ore, può ripetere il tiro salvezza quando si trova almeno a 1,5 chilometri di distanza dall’aboleth.\\
L’aboleth può effettuare 3 azioni aggiuntive, scelte tra le opzioni seguenti. Può usare solo un’opzione aggiuntiva alla volta e solo al termine del turno di un’altra creatura. L’aboleth recupera le azioni aggiuntive spese all’inizio del proprio round.
\textit{Individuare}. L’aboleth effettua una prova di Consapevolezza.\\
\textit{Risucchio Psichico} (Costa 2 Azioni). Una creatura affascinata dall’aboleth subisce 10 (3d6) danni psichici, e l’aboleth recupera un numero di punti ferita pari al danno subito dalla creatura.\\
\textit{Spazzata di Coda}. L’aboleth effettua un attacco di coda.  \\

\subsubsection{Angeli}
\medskip\textbf{Angelo Deva}\index{Angelo Deva}
Taglia: Medio  - \hspace*{0pt}\hfill{Tipo celestiale}\\
Morale: buono\\
Potenza 4, Agilita' 4, Intelletto 4, Volontà 5, Magnetismo 5\\
Difesa: 21 - \hspace*{0pt}\hfill{Punti Ferita: 136}
Movimento 9 m, volo 27 m\\
Tiri Salvezza: Tempra +5, Riflessi +4, Arbitrio +9\\
Competenze: Consapevolezza +9
Resistenze ai Danni da Luce; da botta, perforante e tagliente di attacchi non magici\\
Immunità alle Condizioni affascinato, spaventato
Sensi: visione crepuscolare 36 m\\
Linguaggi tutte, telepatia 36 m\\
Sfida 10 (5.900 PE)\\
\textit{Armi Angeliche}. Gli attacchi con arma del deva sono magici. Quando il deva colpisce con qualsiasi arma, l’arma infligge 4d8 danni da Luce aggiuntivi (già compresi nell’attacco).
\textit{Incantesimi Innati}. La caratteristica da incantatore innato del deva è il Magnetismo (DC 17 per i tiri salvezza degli incantesimi). Il deva può lanciare in maniera innata i seguenti incantesimi, con l’uso delle sole componenti verbali:\\
\textit{A volontà}: individuazione del bene e del male\\
\textit{1/giorno}: comunione, rianimare morti\\
\textit{Resistenza alla Magia}. Il deva ha +1d6 ai tiri salvezza contro incantesimi e altri effetti magici.\\
\textit{Azioni Multiattacco}. Il deva effettua due attacchi da mischia.\\
\textit{Mazza}. Attacco con arma da mischia: +8 a colpire, portata 1 m, un bersaglio.
Colpisce: 7 (1d6 + 4) danni da botta più 18 (4d8) danni da Luce. \\
\textit{Tocco Guaritore (3/Giorno)}. Il deva entra a contatto con un’altra creatura. Il bersaglio recupera magicamente 20 (4d8 + 2) punti ferita ed è libero da qualsiasi cecità, malattia, maledizione, sordità o veleno.\\
\textit{Mutare Forma}. Il deva può trasformarsi magicamente in un umanoide o bestia il cui grado di sfida sia pari o inferiore al proprio, o tornare alla sua vera forma. Alla morte ritorna alla sua vera forma. Qualsiasi equipaggiamento stia indossando o trasportando viene assorbito o trasportatonella nuova forma (a scelta del deva).  Nella nuova forma, il deva mantiene le sue statistiche di gioco e la facoltà di parlare, ma la sua Difesa, metodi di movimento, Potenza,
Agilità e sensi speciali vengono rimpiazzati da quelli della nuova forma, e ottiene qualsiasi statistica o capacità (eccetto i privilegi di classe, azioni aggiuntive e azioni da tana) possedute dalla sua nuova forma e non dalla sua originale.

\medskip\textbf{Angelo Planetar}\index{Angelo Planetar}\\
Taglia: Grande  - \hspace*{0pt}\hfill{Tipo celestiale}\\
Morale: buono\\
Potenza 7, Agilità 5, Intelletto 4, Volontà 6, Magnetismo 7\\
Difesa: 24 - \hspace*{0pt}\hfill{Punti Ferita: 200}
Movimento 12 m, volo 36 m\\
Tiri Salvezza: Tempra +12, Riflessi +5, Arbitrio +11\\
Competenze: Consapevolezza +11
Resistenze ai Danni da Luce; da botta, perforante e tagliente di attacchi non magici\\
Immunità alle Condizioni affascinato, spaventato\\
Sensi visione del vero 36 m, Percezione passiva 21\\
Linguaggi tutte, telepatia 36 m\\
Sfida 16 (5.900 PE)\\
\textit{Armi Angeliche}. Gli attacchi con arma del planetar sono magici. Quando colpisce con qualsiasi arma, l’arma infligge 5d8 danni da Luce aggiuntivi (già compresi nell’attacco).\\
\textit{Consapevolezza Divina}. Il planetar riconosce immediatamente le bugie.\\
\textit{Incantesimi Innati}. La caratteristica da incantatore innato del planetario è il Magnetismo (DC 20 per i tiri salvezza degli incantesimi).\\
Il planetario può lanciare in maniera innata i seguenti incantesimi, senza bisogno di componenti materiali: 
\textit{A volontà}: individuazione del bene e del male, invisibilità (solo personale)\\
\textit{3/giorno}: barriera di lame, colpo infuocato, dissolvi il bene e il male, rianimare morti\\
\textit{1/giorno}: comunione, controllare tempo atmosferico, piaga degli insetti\\
\textit{Resistenza alla Magia}. Il planetar ha +1d6 ai tiri salvezza contro incantesimi e altri effetti magici.\\
\textit{Azioni Multiattacco}. Il planetar effettua due attacchi da mischia.\\
\textit{Spadone}. Attacco con arma da mischia: +12 a colpire, portata 1,5 m, un bersaglio.
Colpisce: 21 (4d6 + 7) danni taglienti più 22 (5d8) danni da Luce.\\
\textit{Tocco Guaritore} (4/Giorno). Il planetar entra a contatto con un’altra creatura. Il bersaglio recupera magicamente 30 (6d8 + 3) punti ferita ed è libero da qualsiasi cecità, malattia, maledizione, sordità o veleno.\\

Angelo Solar
Grande celestiale, legale buono
FORZA 26 (+8)
DESTREZZA 22 (+6)
COSTITUZIONE 26 (+8)
INTELLIGENZA 25 (+7)
SAGGEZZA 25 (+7)
Carisma 30 (+10)
Classe Armatura 21 (armatura naturale)
\hspace*{0pt}\hfill{Punti Ferita}: 243 (18d10 + 144)
Velocità 15 m, volo 45 m
Tiri Salvezza Intelligenza +14, Saggezza +14, Carisma +17
Abilità Percezione +14
Resistenze ai Danni da Luce; da botta, perforante e tagliente
di attacchi non magici
Immunità ai Danni necrotico, veleno
Immunità alle Condizioni affascinato, avvelenato, sfinimento,
spaventato
Sensi visione del vero 36 m, Percezione passiva 24
Linguaggi tutte, telepatia 36 m
Sfida 21 (33.000 PE)
Armi Angeliche. Gli attacchi con arma del solar sono magici.
Quando colpisce con qualsiasi arma, l’arma infligge 6d8 danni
radianti aggiuntivi (già compresi nell’attacco).
Consapevolezza Divina. Il solar riconosce immediatamente le
bugie.
Incantesimi Innati. La caratteristica da incantatore innato del
solar è il Carisma (DC 25 per i tiri salvezza degli incantesimi). Il
solar può lanciare in maniera innata i seguenti incantesimi, senza
bisogno di componenti materiali:
A volontà: individuazione del bene e del male, invisibilità (solo
personale)
3/giorno: barriera di lame, colpo infuocato, dissolvi il bene e il
male, resurrezione
1/giorno: comunione, controllare tempo atmosferico
Resistenza alla Magia. Il solar ha +1d6 ai tiri salvezza
contro incantesimi e altri effetti magici.
Azioni
Multiattacco. Il solar effettua due attacchi con lo spadone.
Spadone. Attacco con arma da mischia: +15 a colpire, portata
1,5 m, un bersaglio.
Colpisce: 22 (4d6 + 8) danni taglienti più 27 (6d8) danni da Luce.
Arco Lungo dell’Uccisione. Attacco con arma a distanza: +13 a
colpire, gittata 45/180 m, un bersaglio.
Colpisce: 15 (2d8 + 6) danni perforanti più 27 (6d8) danni da Luce.
Se il bersaglio è una creatura con 100 punti ferita o meno, deve
riuscire un tiro salvezza di Costituzione DC 15 o morire.
Spada Volante. Il solare libera il suo spadone perché fluttui
magicamente in uno spazio non occupato entro 1,5 metri da lui.
Se il solare può vedere la spada, con un’azione bonus le può
ordinare mentalmente di volare per un massimo di 15 metri ed
effettuare un attacco contro un bersaglio o ritornare nella mano
del solare. Se la spada fluttuante è bersaglio di un effetto, si
considera come se fosse impugnata dal solare. Se il solare muore,
la spada fluttuante cade a terra.
Tocco Guaritore (4/Giorno). Il solare entra a contatto con
un’altra creatura. Il bersaglio recupera magicamente 40 (8d8 + 4)
punti ferita ed è libero da qualsiasi cecità, malattia, maledizione,
sordità o veleno.
Azioni Leggendarie
Il solare può effettuare 3 azioni aggiuntive, scelte tra le opzioni
seguenti. Può usare solo un’opzione leggendaria alla volta e solo
al termine del turno di un’altra creatura. Il solare recupera le
azioni aggiuntive spese all’inizio del proprio round.
Esplosione Incandescente (Costa 2 Azioni). Il solare emette
energia magica divina. Ogni creatura di sua scelta, in un raggio di 3
metri, deve effettuare un tiro salvezza di Destrezza DC 23, subendo
14 (4d6) danni da fuoco più 14 (4d6) danni da Luce se fallisce il tiro
salvezza, o la metà se lo riesce.
Sguardo Accecante (Costa 3 Azioni). Il solare prende a
bersaglio una creatura entro 9 metri e che possa vedere. Se il
bersaglio può vedere il solare, il bersaglio deve riuscire un tiro
salvezza di Costituzione DC 15 o restare accecato finché un
incantesimo come ristorare inferiore non rimuoverà la cecità.
Teletrasporto. Il solare si teletrasporta magicamente fino a 36 metri
di distanza, insieme a tutto l’equipaggiamento che sta indossando o
trasportando, in uno spazio non occupato e che può vedere.
 
Ankheg
Grande mostruosità, disallineato
FORZA 17 (+3)
DESTREZZA 11 (+0)
COSTITUZIONE 13 (+1)
INTELLIGENZA 1 (-5)
SAGGEZZA 13 (+1)
Carisma 6 (-2)
Classe Armatura 14 (armatura naturale), 11 mentre è prono
\hspace*{0pt}\hfill{Punti Ferita}: 39 (6d10 + 6)
Velocità 9 m, scavo 3 m
Sensi scurovisione 18 m, percezione tellurica 18 m, Percezione
passiva 11
Linguaggi -
Sfida 2 (450 PE)
Azioni
Morso. Attacco con arma da mischia: +5 a colpire, portata 1,5
m, un bersaglio.
Colpisce: 10 (2d6 + 3) danni taglienti più 3 (1d6) danni da acido.
Se il bersaglio è una creatura di taglia Grande o inferiore, è
afferrata (DC 13 per fuggire). Fino al termine dell’afferrare,
l’ankheg può mordere solo la creatura afferrata e ha vantaggio ai
tiri di attacco contro di essa.
Spruzzo Acido (Ricarica 6). L’ankheg sputa acido in una linea
lunga 9 metri e larga 1,5 metri, purché non stia afferrando
nessuna creatura. Ogni creatura su quella linea deve effettuare un
tiro salvezza di Destrezza DC 13, e subire 10 (3d6) danni da
acido se fallisce il tiro salvezza, o la metà di questi danni se lo
riesce.
Arpia
Media mostruosità, caotico malvagio
FORZA 12 (+1)
DESTREZZA 13 (+1)
COSTITUZIONE 12 (+1)
INTELLIGENZA 7 (-2)
SAGGEZZA 10 (+0)
Carisma 13 (+1)
Classe Armatura 11
\hspace*{0pt}\hfill{Punti Ferita}: 38 (7d8 + 7)
Velocità 6 m, volo 12 m
Sensi Percezione passiva 10
Linguaggi Comune
Sfida 1 (200 PE)
Azioni
Multiattacco. L’armatura effettua due attacchi: uno con gli
artigli e uno con il randello.
Artigli. Attacco con arma da mischia: +3 a colpire, portata 1,5
m, un bersaglio.
Colpisce: 6 (2d4 + 1) danni taglienti.
Randello. Attacco con arma da mischia: +3 a colpire, portata 1,5
m, un bersaglio.
Colpisce: 3 (1d4 + 1) danni da botta.
Canto Ammaliatore. L’arpia canta una melodia magica. Ogni
umanoide e gigante entro 90 metri dall’arpia e che possa udire la
canzone deve riuscire un tiro salvezza di Saggezza DC 11 o
restare affascinato fino al termine della canzone. L’arpia deve
effettuare un’azione bonus durante il suo prossimo turno per
continuare a cantare. Può smettere di cantare in qualsiasi
momento. Il canto ha termine se l’arpia è inabile.
Mentre è affascinato dall’arpia, un bersaglio è inabile e ignora le
canzoni di altre arpie. Se il bersaglio affascinato si trova a più di
1,5 metri dall’arpia, il bersaglio deve muoversi durante il proprio
turno per dirigersi verso l’arpia usando la via più diretta. Egli
non eviterà attacchi di opportunità, ma prima di muoversi in un
terreno pericoloso, come lava o un pozzo, e prima di subire
danno da qualsiasi fonte che non sia l’arpia, il bersaglio potrà
ripetere il tiro salvezza. Una creatura può ripetere il tiro salvezza
al termine di ciascun proprio round. Se il tiro salvezza ha
successo, l’effetto ha termine per quel bersaglio.
Un bersaglio che riesce il tiro salvezza è immune al canto di
quell’arpia per le successive 24 ore.
Azer
Media elementale, legale neutrale
FORZA 17 (+3)
DESTREZZA 12 (+1)
COSTITUZIONE 15 (+2)
INTELLIGENZA 12 (+1)
SAGGEZZA 13 (+1)
Carisma 10 (+0)
Classe Armatura 17 (armatura naturale, scudo)
\hspace*{0pt}\hfill{Punti Ferita}: 39 (6d8 + 12)
Velocità 9 m
Tiri Salvezza Costituzione +4
Immunità ai Danni fuoco, veleno
Immunità alle Condizioni avvelenato
Sensi Percezione passiva 11
Linguaggi Ignan
Sfida 2 (450 PE)
Armi Riscaldate. Quando l’azer colpisce con un’arma da mischia
in metallo, infligge 3 (1d6) danni da fuoco aggiuntivi (già inclusi
nell’attacco).
Corpo Riscaldato. Una creatura che entri a contatto con l’azer o
lo colpisca con un attacco da mischia mentre si trova entro 1,5
metri da lui subisce 5 (1d10) danni da fuoco.
Fuoco Vivente. Un azer non ha bisogno di cibo, bevande o di
dormire.
Illuminazione. L’azer irradia luce intensa in un raggio di 3 metri
e luce fioca per ulteriori 3 metri.
Azioni
Martello da Guerra. Attacco con arma da mischia: +5 a colpire,
portata 1,5 m, un bersaglio.
Colpisce: 7 (1d8 + 3) danni da botta, o 8 (1d10 + 3) danni
contundenti se usato a due mani per effettuare un attacco da
mischia, più 3 (1d6) danni da fuoco.
Basilisco
Media mostruosità, disallineato
FORZA 16 (+3)
DESTREZZA 8 (-1)
COSTITUZIONE 15 (+2)
INTELLIGENZA 2 (-4)
SAGGEZZA 8 (-1)
Carisma 7 (-2)
Classe Armatura 15 (armatura naturale)
\hspace*{0pt}\hfill{Punti Ferita}: 52 (8d8 + 16)
Velocità 6 m
Sensi scurovisione 18 m, Percezione passiva 9
Linguaggi -
Sfida 3 (700 PE)
Sguardo Pietrificante. Se una creatura comincia il suo turno entro 9
metri dal basilisco e i due si possono vedere vicendevolmente, se non
è inabile il basilisco può obbligare la creatura ad effettuare un tiro
salvezza di Costituzione DC 12. Se la creatura fallisce il tiro
salvezza, inizia magicamente a trasformarsi in pietra ed è intralciata.
La creatura deve ripetere il tiro salvezza al termine del suo prossimo
turno. Se lo riesce, l’effetto termina. Se lo fallisce, la creatura è
pietrificata finché non viene liberata dall’incantesimo ristorare
superiore o altra magia.
Una creatura che non sia sorpresa, può distogliere lo sguardo per
evitare il tiro salvezza all’inizio del suo turno. In quel caso, non
potrà vedere il basilisco fino all’inizio del suo prossimo turno,
quando potrà distogliere nuovamente lo sguardo. Se nel
frattempo dovesse guardare il basilisco, dovrebbe
immediatamente effettuare il tiro salvezza.
Se il basilisco si trova entro 9 metri dal suo riflesso a luce intensa
e lo vede, lo scambia per un rivale e diventa il bersaglio del
proprio sguardo.
Azioni
Morso. Attacco con arma da mischia: +5 a colpire, portata 1,5
m, un bersaglio.
Colpisce: 10 (2d6 + 3) danni perforanti più 7 (2d6) danni da veleno.
 
Behir
Enorme mostruosità, neutrale malvagio
FORZA 23 (+6)
DESTREZZA 16 (+3)
COSTITUZIONE 18 (+4)
INTELLIGENZA 7 (-2)
SAGGEZZA 14 (+2)
Carisma 12 (+1)
Classe Armatura 17 (armatura naturale)
\hspace*{0pt}\hfill{Punti Ferita}: 168 (16d12 + 64)
Velocità 15 m, scalata 12 m
Abilità Furtività +7, Percezione +6
Immunità al Danno fulmine
Sensi scurovisione 27 m, Percezione passiva 16
Linguaggi Draconico
Sfida 11 (7.200 PE)
Azioni
Multiattacco. Il behir effettua due attacchi: uno con il morso e
uno per stritolare.
Morso. Attacco con arma da mischia: +10 a colpire, portata 3 m,
un bersaglio.
Colpisce: 22 (3d10 + 6) danni perforanti.
Stritolare. Attacco con arma da mischia: +10 a colpire, portata
1,5 m, una creatura di taglia Grande o inferiore.
Colpisce: 17 (2d10 + 6) danni da botta più 17 (2d10 + 6)
danni taglienti. Il bersaglio è afferrato (DC 16 per fuggire) Se il
behir non sta già stritolando un’altra creatura, il bersaglio è
afferrato e intralciato fino al termine dell’afferrare.
Inghiottire. Il behir effettua una attacco di morso contro un
bersaglio di taglia Media o inferiore che sta afferrando. Se
l’attacco colpisce, il bersaglio è inghiottito, e l’afferrare ha
termine. Il bersaglio inghiottito è accecato e intralciato, ha
copertura totale contro gli attacchi e altri effetti all’esterno del
behir, e subisce 21 (6d6) danni da acido all’inizio di ciascun
turno del behir. Il behir può inghiottire solo una creatura alla
volta.
Se il behir subisce 30 o più danni in un singolo turno da una
creatura che ha inghiottito, deve riuscire un tiro salvezza di
Costituzione DC 14 al termine di quel turno o vomitare la
creatura, che ricade prona in uno spazio entro 3 metri dal behir.
Se il behir muore, una creatura inghiottita non è più intralciata da
esso e può uscire dal cadavere utilizzando 4,5 metri di
movimento, uscendo prona.
Soffio di Fulmine (Ricarica 5-6). Il behir esala fulmini in una
linea lunga 6 metri e larga 1,5 metri. Ogni creatura su quella
linea deve effettuare un tiro salvezza di Destrezza DC 16 e subire
66 (12d10) danni da fulmine se fallisce il tiro salvezza, o la metà
di questi danni se lo riesce.
Bugbear
Media umanoide (goblinoide), caotico malvagio
FORZA 15 (+2)
DESTREZZA 14 (+2)
COSTITUZIONE 13 (+1)
INTELLIGENZA 8 (-1)
SAGGEZZA 11 (+0)
Carisma 9 (-1)
Classe Armatura 16 (armatura di pelle, scudo)
\hspace*{0pt}\hfill{Punti Ferita}: 27 (5d8 + 5)
Velocità 9 m
Abilità Furtività +6, Sopravvivenza +2
Sensi scurovisione 18 m, Percezione passiva 10
Linguaggi Comune, Goblin
Sfida 1 (200 PE)
Attacco di Sorpresa. Se il bugbear sorprende una creatura e la
colpisce con un attacco durante il primo round di combattimento,
il bersaglio subisce 7 (2d6) danni aggiuntivi dall’attacco.
Bruto. Un’arma da mischia infligge un dado aggiuntivo di danno
quando il bugbear colpisce con essa (già incluso nell’attacco).
Azioni
Mazza Chiodata. Attacco con arma da mischia: +4 a colpire,
portata 1,5 m, un bersaglio.
Colpisce: 11 (2d8 + 2) danni perforanti.
Giavellotto. Attacco con arma da mischia o a Distanza: +4 a
colpire, portata 1,5 m o gittata 9/36 m, un bersaglio.
Colpisce: 9 (2d6 + 2) danni perforanti in mischia o 5 (1d6 + 2)
danni perforanti a gittata.
Bulette
Grande mostruosità, disallineato
FORZA 19 (+4)
DESTREZZA 11 (+0)
COSTITUZIONE 21 (+5)
INTELLIGENZA 2 (-4)
SAGGEZZA 10 (+0)
Carisma 5 (-3)
Classe Armatura 17 (armatura naturale)
\hspace*{0pt}\hfill{Punti Ferita}: 94 (9d10 + 45)
Velocità 12 m, scavo 12 m
Abilità Percezione +6
Sensi scurovisione 18 m, percezione tellurica 18 m, Percezione
passiva 16
Linguaggi -
Sfida 5 (1.800 PE)
Salto da Fermo. Un bulette può saltare in lungo fino a 9 metri e
in alto fino a 4,5 metri, con o senza la rincorsa.
Azioni
Morso. Attacco con arma da mischia: +7 a colpire, portata 1,5
m, un bersaglio.
Colpisce: 30 (4d12 + 4) danni perforanti.
Salto Letale. Se il bulette può saltare di almeno 4,5 metri come
parte del suo movimento, può usare poi questa azione per
atterrare in piedi in uno spazio che contiene una o più creature.
Ciascuna di queste creature deve riuscire un tiro salvezza di
Forza o Destrezza DC 16 (a scelta del bersaglio) o venire gettata
prona e subire 14 (3d6 + 4) danni da botta più 14 (3d6 + 4)
danni taglienti. Se il tiro salvezza riesce, la creatura subisce solo
la metà dei danni, non è gettata prona, e viene spinta di 1,5 metri
fuori dello spazio del bulette in uno spazio non occupato a scelta
della creatura. Se non ci sono spazi non occupati a gittata, la
creatura cade prona nello spazio del bulette.
Centauro
Grande mostruosità, neutrale buono
FORZA 18 (+4)
DESTREZZA 14 (+2)
COSTITUZIONE 14 (+2)
INTELLIGENZA 9 (-1)
SAGGEZZA 13 (+1)
Carisma 11 (+0)
Classe Armatura 12
\hspace*{0pt}\hfill{Punti Ferita}: 45 (6d10 + 12)
Velocità 15 m
Abilità Atletica +6, Percezione +3, Sopravvivenza +3
Sensi Percezione passiva 13
Linguaggi Elfico, Silvano
Sfida 2 (450 PE)
Carica. Se il centauro si muove di almeno 9 metri diretto verso il
bersaglio e colpisce con un attacco di picca durante lo stesso
turno, il bersaglio subisce 10 (3d6) danni perforanti aggiuntivi.
Azioni
Multiattacco. Il centauro effettua due attacchi: uno con la picca e
uno con gli zoccoli o due con l’arco lungo.
Picca. Attacco con arma da mischia: +6 a colpire, portata 3 m,
un bersaglio.
Colpisce: 9 (1d10 + 4) danni perforanti.
Zoccoli. Attacco con arma da mischia: +6 a colpire, portata 1,5
m, un bersaglio.
Colpisce: 11 (2d6 + 4) danni da botta.
Arco Lungo. Attacco con arma a Distanza: +4 a colpire, gittata
45/180 m, un bersaglio.
Colpisce: 6 (1d8 + 2) danni perforanti.
 
Chimera
Grande mostruosità, caotico malvagio
FORZA 19 (+4)
DESTREZZA 11 (+0)
COSTITUZIONE 19 (+4)
INTELLIGENZA 3 (-4)
SAGGEZZA 14 (+2)
Carisma 10 (+0)
Classe Armatura 14 (armatura naturale)
\hspace*{0pt}\hfill{Punti Ferita}: 114 (12d10 + 48)
Velocità 9 m, volo 18 m
Abilità Percezione +8
Sensi scurovisione 18 m, Percezione passiva 18
Linguaggi comprende il Draconico ma non può parlare
Sfida 6 (2.300 PE)
Azioni
Multiattacco. La chimera effettua tre attacchi: uno con il morso,
uno con le corna e uno con gli artigli. Quando il soffio infuocato
è disponibile, può usare il soffio al posto del morso o delle corna.
Artigli. Attacco con arma da mischia: +7 a colpire, portata 1,5
m, un bersaglio.
Colpisce: 11 (2d6 + 4) danni taglienti.
Corna. Attacco con arma da mischia: +7 a colpire, portata 1,5 m,
un bersaglio.
Colpisce: 10 (1d12 + 4) danni da botta.
Morso. Attacco con arma da mischia: +7 a colpire, portata 1,5
m, un bersaglio.
Colpisce: 11 (2d6 + 4) danni perforanti.
Soffio Infuocato (Ricarica 5-6). La testa di drago esala fuoco in
un cono di 4,5 metri. Ogni creatura in quell’area deve effettuare
un tiro salvezza di Destrezza DC 15 e subire 31 (7d8) danni da
fuoco se fallisce il tiro salvezza, o la metà di questi danni se lo
riesce.
Chuul
Grande aberrazione, caotico malvagio
FORZA 19 (+4)
DESTREZZA 10 (+0)
COSTITUZIONE 16 (+3)
INTELLIGENZA 5 (-3)
SAGGEZZA 11 (+0)
Carisma 5 (-3)
Classe Armatura 16 (armatura naturale)
\hspace*{0pt}\hfill{Punti Ferita}: 93 (11d10 + 33)
Velocità 9 m, nuoto 9 m
Abilità Percezione +4
Immunità ai Danni veleno
Immunità alle Condizioni avvelenato
Sensi scurovisione 18 m, Percezione passiva 14
Linguaggi comprende la Parlata delle Profondità ma non può
parlare
Sfida 4 (1.100 PE)
Anfibio. Il chuul può respirare aria e acqua.
Senso della Magia. Il chuul percepisce la magia entro 36 metri
da sé. Questo tratto funziona come l’incantesimo individuazione
del magico ma di per sé non è magico.
Azioni
Multiattacco. Il chuul effettua due attacchi con le chele. Se il
chuul sta afferrando una creatura, può anche usare i suoi
tentacoli una volta.
Chele. Attacco con arma da mischia: +6 a colpire, portata 3 m,
un bersaglio.
Colpisce: 11 (2d6 + 4) danni da botta. Un bersaglio è
afferrato (DC 14 per fuggire) se è di taglia Grande o inferiore e il
chuul non sta già afferrando altre due creature.
Tentacoli. Una creatura afferrata dal chuul deve riuscire un tiro
salvezza di Costituzione DC 13 o restare avvelenata per 1
minuto. Fino al termine dell’avvelenamento, il bersaglio è
paralizzato. Il bersaglio può ripetere il tiro salvezza al termine di
ciascun suo turno, terminando l’effetto per sé in caso di successo.
Coboldo
Piccola umanoide (coboldo), legale malvagio
FORZA 7 (-2)
DESTREZZA 15 (+2)
COSTITUZIONE 9 (-1)
INTELLIGENZA 8 (-1)
SAGGEZZA 7 (-2)
Carisma 8 (-1)
Classe Armatura 12
\hspace*{0pt}\hfill{Punti Ferita}: 5 (2d6 - 2)
Velocità 9 m
Sensi scurovisione 18 m, Percezione passiva 8
Linguaggi Comune, Draconico
Sfida 1/8 (25 PE)
Sensibilità alla Luce. Mentre è alla luce del sole, il coboldo ha
svantaggio ai tiri per colpire, oltre che alle prove di Saggezza
(Percezione) basate sulla vista.
Tattiche di Branco. Il coboldo ha vantaggio ai tiri per colpire
contro una creatura se almeno uno degli alleati del coboldo si
trova entro 1,5 metri dalla creatura e quell’alleato non è inabile.
Azioni
Pugnale. Attacco con arma da mischia: +4 a colpire, portata 1,5
m, un bersaglio.
Colpisce: 4 (1d4 + 2) danni perforanti.
Fionda. Attacco con arma a distanza: +4 a colpire, gittata 9/36
m, un bersaglio.
Colpisce: 4 (1d4 + 2) danni da botta.
Cockatrice
Piccola mostruosità, disallineato
FORZA 6 (-2)
DESTREZZA 12 (+1)
COSTITUZIONE 12 (+1)
INTELLIGENZA 2 (-4)
SAGGEZZA 13 (+1)
Carisma 5 (-3)
Classe Armatura 11
\hspace*{0pt}\hfill{Punti Ferita}: 27 (6d6 + 6)
Velocità 6 m, volo 12 m
Sensi scurovisione 18 m, Percezione passiva 11
Linguaggi -
Sfida 1/2 (100 PE)
Azioni
Morso. Attacco con arma da mischia: +3 a colpire, portata 1,5
m, una creatura.
Colpisce: 3 (1d4 + 1) danni perforanti, e il bersaglio deve
riuscire un tiro salvezza di Costituzione DC 11 per non essere
magicamente pietrificato. Se fallisce il tiro salvezza, la creatura
inizia a trasformarsi in pietra ed è intralciata. Al termine del
turno successivo deve ripetere il tiro salvezza. Se lo riesce,
l’effetto ha termine. Se lo fallisce, la creatura è pietrificata per 24
ore.
 
Couatl
Media celestiale, legale buono
FORZA 16 (+3)
DESTREZZA 20 (+5)
COSTITUZIONE 17 (+3)
INTELLIGENZA 18 (+4)
SAGGEZZA 20 (+5)
Carisma 18 (+4)
Classe Armatura 19 (armatura naturale)
\hspace*{0pt}\hfill{Punti Ferita}: 97 (13d8 + 39)
Velocità 9 m, volo 9 m
Tiri Salvezza Costituzione +5, Saggezza +7, Carisma +6
Resistenze al Danno radiante
Immunità al Danno psichico; da botta, perforante e
tagliente di attacchi non magici
Sensi visione del vero 36 m, Percezione passiva 15
Linguaggi tutte, telepatia 36 m
Sfida 4 (1.100 PE)
Armi Magiche. Gli attacchi con armi del couatl sono magici.
Incantesimi Innati. La caratteristica da incantatore innato del
couatl è il Carisma (DC dei tiri salvezza degli incantesimi 14). Il
couatl può lanciare questi incantesimi in maniera innata, usando
solo componenti verbali:
A volontà: individuazione del bene e del male, individuazione del
magico, individuazione dei pensieri
3/giorno ciascuno: benedizione, creare cibo e acqua, cura ferite,
protezione dai veleni, ristorare inferiore, santuario, scudo
1/giorno ciascuno: ristorare superiore, scrutare, sogno
Mente Protetta. Il couatl è immune allo scrutare e qualsiasi
effetto che percepisca le sue emozioni, legga i suoi pensieri o
individui la sua posizione.
Azioni
Morso. Attacco con arma da mischia: +8 a colpire, portata 1,5
m, una creatura.
Colpisce: 8 (1d6 + 5) danni perforanti, e il bersaglio deve
riuscire un tiro salvezza di Costituzione DC 13 o restare
avvelenato per 24 ore. Fino al termine dell’avvelenamento, il
bersaglio è privo di sensi. Un’altra creatura può effettuare
un’azione per risvegliare il bersaglio.
Stritolare. Attacco con arma da mischia: +6 a colpire, portata 3
m, una creatura di taglia Media o inferiore.
Colpisce: 10 (2d6 + 3) danni da botta, e il bersaglio è afferrato
(DC 15 per fuggire). Fino al termine dell’afferrare, il bersaglio è
intralciato, e il couatl non può stritolare un altro bersaglio.
Mutare Forma. Il couatl può trasformarsi magicamente in un
umanoide o bestia il cui grado di sfida sia pari o inferiore al proprio,
o tornare alla sua vera forma. Alla morte ritorna alla sua vera forma.
Qualsiasi equipaggiamento stia indossando o trasportando viene
assorbito o trasportato nella nuova forma (a scelta del couatl).
Nella nuova forma, il couatl mantiene le sue statistiche di gioco e la
facoltà di parlare, ma la sua Difesa, metodi di movimento, Forza,
Destrezza e altre azioni vengono rimpiazzati da quelli della nuova
forma, e ottiene qualsiasi statistica o capacità (eccetto i privilegi di
classe, azioni aggiuntive e azioni da tana) possedute dalla sua nuova
forma e non dalla sua originale. Se la nuova forma ha un attacco di
morso, il couatl può usare il proprio morso nella nuova forma.
Cumulo Strisciante
Grande pianta, disallineato
FORZA 18 (+4)
DESTREZZA 8 (-1)
COSTITUZIONE 16 (+3)
INTELLIGENZA 5 (-3)
SAGGEZZA 10 (+0)
Carisma 5 (-3)
Classe Armatura 15 (armatura naturale)
\hspace*{0pt}\hfill{Punti Ferita}: 136 (16d10 + 48)
Velocità 6 m, nuoto 6 m
Abilità Furtività +2
Resistenze al Danno freddo, fuoco
Immunità al Danno fulmine
Immunità alle Condizioni accecato, assordato, sfinimento
Sensi vista cieca 18 m (cieco oltre questo raggio), Percezione
passiva 10
Linguaggi -
Sfida 5 (1.800 PE)
Assorbimento dei Fulmini. Ogni qual volta il cumulo strisciante
subisce danni da fulmine, non subisce danni e recupera un
numero di punti ferita pari al danno da fulmine inferto.
Azioni
Multiattacco. Il cumulo strisciante effettua due attacchi di
schianto. Se entrambi gli attacchi colpiscono una creatura di
taglia Media o inferiore, il bersaglio è afferrato (DC 14 per
fuggire) e il cumulo strisciante usa Avvolgere su di esso.
Schianto. Attacco con arma da mischia: +7 a colpire, portata 1,5
m, un bersaglio.
Colpisce: 13 (2d8 + 4) danni da botta.
Avvolgere. Il cumulo strisciante avvolge una creatura di taglia
Media o inferiore che ha afferrato. Il bersaglio avvolto è
accecato, intralciato e impossibilitato a respirare, e deve riuscire
un tiro salvezza di Costituzione DC 14 all’inizio di ciascun turno
del tumulo o subire 13 (2d8 + 4) danni da botta. Se il cumulo
si muove, il bersaglio avvolto si muove con esso. Il cumulo può
avvolgere solo una creatura alla volta.
Demoni
Balor
Enorme immondo (demone), caotico malvagio
FORZA 26 (+8)
DESTREZZA 15 (+2)
COSTITUZIONE 22 (+6)
INTELLIGENZA 20 (+5)
SAGGEZZA 16 (+3)
Carisma 22 (+6)
Classe Armatura 19 (armatura naturale)
\hspace*{0pt}\hfill{Punti Ferita}: 262 (21d12 + 126)
Velocità 12 m, volo 24 m
Tiri Salvezza Forza +14, Costituzione +12, Saggezza +9,
Carisma +12
Resistenze al Danno freddo, fulmine; da botta, perforante e
tagliente di attacchi non magici
Immunità al Danno fuoco, veleno
Immunità alle Condizioni avvelenato
Sensi visione del vero 36 m, Percezione passiva 13
Linguaggi Abissale, telepatia 36 m
Sfida 19 (22.000 PE)
Armi Magiche. Gli attacchi con arma del demone sono magici.
Aura di Fuoco. All’inizio di ciascun turno del demone, ciascuna
creatura entro 1,5 metri da lui subisce 10 (3d6) danni da fuoco, e
gli oggetti infiammabili che si trovano nell’aura e che non sono
indossati o trasportati prendono fuoco. Una creatura che entri a
contatto con il demone o lo colpisca con un attacco da mischia
mentre si trova entro 1,5 metri da esso subisce 10 (3d6) danni da
fuoco.
Resistenza alla Magia. Il demone ha +1d6 ai tiri salvezza
contro incantesimi e altri effetti magici.
Spasmo Mortale. Quando il demone muore, esplode; ciascuna
creatura entro 9 metri da esso deve effettuare un tiro salvezza di
Destrezza DC 20, subendo 70 (20d6) danni da fuoco se fallisce il
tiro salvezza, o la metà di questi danni se lo riesce. L’esplosione
appicca il fuoco agli oggetti infiammabili che non sono indossati
o trasportati, e distrugge le armi del demone.
Azioni
Multiattacco. Il demone effettua due attacchi: uno con la spada
lunga e uno con la frusta.
Frusta. Attacco con arma da mischia: +14 a colpire, portata 9 m,
un bersaglio.
Colpisce: 15 (2d6 + 8) danni taglienti più 10 (3d6) danni da
fuoco, e il bersaglio deve riuscire un tiro salvezza di Forza DC
20 o venire trascinato 7,5 metri verso il demone.
Spada Lunga. Attacco con arma da mischia: +14 a colpire,
portata 3 m, un bersaglio.
Colpisce: 21 (3d8 + 8) danni taglienti più 13 (3d8) danni da
fulmine. Se il demone ottiene un colpo critico, tira il danno tre
volte, invece che due.
Teletrasporto. Il demone si teletrasporta magicamente, insieme a
tutto l’equipaggiamento che indossa o trasporta, in uno spazio
non occupato e che può vedere entro 36 metri.
Dretch
Piccola immondo (demone), caotico malvagio
FORZA 11 (+0)
DESTREZZA 11 (+0)
COSTITUZIONE 12 (+1)
INTELLIGENZA 5 (-3)
SAGGEZZA 8 (-1)
Carisma 3 (-4)
Classe Armatura 11 (armatura naturale)
\hspace*{0pt}\hfill{Punti Ferita}: 18 (4d6 + 4)
Velocità 6 m
Resistenze al Danno freddo, fulmine, fuoco
Immunità al Danno veleno
Immunità alle Condizioni avvelenato
Sensi scurovisione 18 m, Percezione passiva 9
Linguaggi Abissale, telepatia 18 m (funziona solo con le
creature che comprendono l’Abissale)
Sfida 1/4 (50 PE)
Azioni
Multiattacco. Il demone effettua due attacchi: uno con il morso e
uno con gli artigli.
Artigli. Attacco con arma da mischia: +2 a colpire, portata 1,5
m, un bersaglio.
Colpisce: 5 (2d4) danni taglienti.
Morso. Attacco con arma da mischia: +2 a colpire, portata 1,5
m, un bersaglio.
Colpisce: 3 (1d6) danni perforanti.
Nube Fetida (1/Giorno). Un disgustoso gas verde si estende in
un raggio di 3 metri dal demone. Il gas si propaga intorno agli
angoli, e la sua area è oscurata leggermente. Rimane per 1
minuto o finché non viene disperso da un forte vento. Qualsiasi
creatura che inizi il proprio round in quell’area deve riuscire un
tiro salvezza di Costituzione DC 11 o restare avvelenata fino
all’inizio del suo prossimo turno. Mentre è avvelenato in questo
modo, il bersaglio, durante il suo turno, può effettuare solo
un’azione o un’azione bonus, ma non entrambe, e non può
effettuare reazioni. 
 
Glabrezu
Grande immondo (demone), caotico malvagio
FORZA 20 (+5)
DESTREZZA 15 (+2)
COSTITUZIONE 21 (+5)
INTELLIGENZA 19 (+4)
SAGGEZZA 17 (+3)
Carisma 16 (+3)
Classe Armatura 17 (armatura naturale)
\hspace*{0pt}\hfill{Punti Ferita}: 157 (15d10 + 75)
Velocità 12 m
Tiri Salvezza Forza +9, Costituzione +9, Saggezza +7, Carisma
+7
Resistenze al Danno freddo, fulmine, fuoco; da botta,
perforante e tagliente di attacchi non magici
Immunità al Danno veleno
Immunità alle Condizioni avvelenato
Sensi visione del vero 36 m, Percezione passiva 13
Linguaggi Abissale, telepatia 36 m
Sfida 9 (5.000 PE)
Incantesimi Innati. La caratteristica da incantatore del demone è
l’Intelligenza (DC dei tiri salvezza degli incantesimi 16). Il
demone può lanciare questi incantesimi in maniera innata, senza
bisogno di componenti materiali:
A volontà: dissolvi magie, individuazione del magico, oscurità
1/giorno ciascuno: confusione, parola del potere stordire, volare
Resistenza alla Magia. Il demone ha +1d6 ai tiri salvezza
contro incantesimi e altri effetti magici.
Azioni
Multiattacco. Il demone effettua quattro attacchi: due con le
chele e due con i pugni. In alternativa, può effettuare due attacchi
con le chele e lanciare un incantesimo.
Chela. Attacco con arma da mischia: +9 a colpire, portata 3 m,
un bersaglio.
Colpisce: 16 (2d10 + 5) danni da botta. Se il bersaglio è una
creatura di taglia Media o inferiore, è afferrato (DC 15 per
fuggire). Il glabrezu possiede due chele, ciascuna delle quali può
afferrare un bersaglio.
Pugno. Attacco in mischia con arma: +9 a colpire, portata 1,5 m,
un bersaglio.
Colpisce: 7 (2d4 + 2) danni da botta.
Hezrou
Grande immondo (demone), caotico malvagio
FORZA 19 (+4)
DESTREZZA 17 (+3)
COSTITUZIONE 20 (+5)
INTELLIGENZA 5 (-2)
SAGGEZZA 12 (+1)
Carisma 13 (+1)
Classe Armatura 16 (armatura naturale)
\hspace*{0pt}\hfill{Punti Ferita}: 136 (13d10 + 65)
Velocità 9 m
Tiri Salvezza Forza +7, Costituzione +8, Saggezza +4
Resistenze al Danno freddo, fulmine, fuoco; da botta,
perforante e tagliente di attacchi non magici
Immunità al Danno veleno
Immunità alle Condizioni avvelenato
Sensi scurovisione 36 m, Percezione passiva 11
Linguaggi Abissale, telepatia 36 m
Sfida 8 (3.900 PE)
Fetore. Qualsiasi creatura che inizi il suo turno entro 3 metri dal
demone, deve riuscire un tiro salvezza di Costituzione DC 14 o
restare avvelenata fino all’inizio del proprio round. Se riesce il
tiro salvezza, la creatura è immune al fetore del demone
gracidante per 24 ore.
Resistenza alla Magia. Il demone ha +1d6 ai tiri salvezza
contro incantesimi e altri effetti magici.
Azioni
Multiattacco. Il demone effettua tre attacchi: uno con il morso e
due con gli artigli.
Artiglio. Attacco con arma da mischia: +7 a colpire, portata 1,5
m, un bersaglio.
Colpisce: 11 (2d6 + 4) danni taglienti.
Morso. Attacco con arma da mischia: +7 a colpire, portata 1,5
m, un bersaglio.
Colpisce: 15 (2d10 + 4) danni perforanti. 
Marilith
Grande immondo (demone), caotico malvagio
FORZA 18 (+4)
DESTREZZA 20 (+5)
COSTITUZIONE 20 (+5)
INTELLIGENZA 18 (+4)
SAGGEZZA 16 (+3)
Carisma 20 (+5)
Classe Armatura 18 (armatura naturale)
\hspace*{0pt}\hfill{Punti Ferita}: 189 (18d10 + 90)
Velocità 12 m
Tiri Salvezza Forza +9, Costituzione +10, Saggezza +8, Carisma
+10
Resistenze al Danno freddo, fulmine, fuoco; da botta,
perforante e tagliente di attacchi non magici
Immunità al Danno veleno
Immunità alle Condizioni avvelenato
Sensi visione del vero 36 m, Percezione passiva 13
Linguaggi Abissale, telepatia 36 m
Sfida 16 (15.000 PE)
Armi Magiche. Gli attacchi con armi del demone sono magici.
Reattivo. Il demone può effettuare una reazione durante ciascun
turno di combattimento.
Resistenza alla Magia. Il demone ha +1d6 ai tiri salvezza
contro incantesimi e altri effetti magici.
Azioni
Multiattacco. Il demone effettua sette attacchi: sei con le spade
lunghe e uno con la coda.
Coda. Attacco con arma da mischia: +9 a colpire, portata 3 m,
una creatura.
Colpisce: 15 (2d10 + 4) danni da botta. Se il bersaglio è di
taglia Media o inferiore, è afferrato (DC 19 per fuggire). Fino al
termine dell’afferrare, il bersaglio è intralciato, e il demone può
colpire automaticamente il bersaglio con la coda, ma non può
effettuare attacchi di coda contro altri bersagli.
Spada Lunga. Attacco con arma da mischia: +9 a colpire,
portata 1,5 m, un bersaglio.
Colpisce: 13 (2d8 + 4) danni taglienti.
Reazioni
Parata. Il demone somma 5 alla sua Difesa contro un attacco da
mischia che lo colpirebbe. Per farlo, il demone deve poter vedere
il suo attaccante e impugnare un’arma da mischia.
Nalfeshnee
Grande immondo (demone), caotico malvagio
FORZA 21 (+5)
DESTREZZA 10 (+0)
COSTITUZIONE 22 (+6)
INTELLIGENZA 19 (+4)
SAGGEZZA 12 (+1)
Carisma 15 (+2)
Classe Armatura 18 (armatura naturale)
\hspace*{0pt}\hfill{Punti Ferita}: 184 (16d10 + 96)
Velocità 6 m, volo 9 m
Tiri Salvezza Costituzione +11, Intelligenza +9, Saggezza +6,
Carisma +7
Resistenze al Danno freddo, fulmine, fuoco; da botta,
perforante e tagliente di attacchi non magici
Immunità al Danno veleno
Immunità alle Condizioni avvelenato
Sensi scurovisione 36 m, Percezione passiva 11
Linguaggi Abissale, telepatia 36 m
Sfida 13 (10.000 PE)
Resistenza alla Magia. Il demone ha +1d6 ai tiri salvezza
contro incantesimi e altri effetti magici.
Azioni
Multiattacco. Il demone usa, se possibile, Aureola di Orrore. Poi
effettua tre attacchi: uno con il morso e due con gli artigli.
Artiglio. Attacco con arma da mischia: +10 a colpire, portata 3
m, un bersaglio.
Colpisce: 15 (3d6 + 5) danni taglienti.
Morso. Attacco con arma da mischia: +10 a colpire, portata 1,5
m, un bersaglio.
Colpisce: 32 (5d10 + 5) danni perforanti.
Aureola di Orrore (Ricarica 5-6). Il demone emette una luce
magica multicolore e scintillante. Ogni creatura entro 4,5 metri
dal demone e che possa vedere la luce, deve riuscire un tiro
salvezza di Saggezza DC 15 o restare spaventata per 1 minuto.
Una creatura può ripetere il tiro salvezza al termine di ciascun
suo turno, terminando l’effetto per sé se lo riesce. Se il tiro
salvezza della creatura riesce o l’effetto ha termine per essa, la
creatura è immune all’Aureola di Orrore del demone gemente
per le successive 24 ore.
Teletrasporto. Il demone si teletrasporta, insieme a tutto
l’equipaggiamento che sta indossando o trasportando, in uno
spazio non occupato che possa vedere fino a 36 metri di distanza.
 
Quasit
Minuscola immondo (demone, mutaforma), caotico malvagio
FORZA 5 (-3)
DESTREZZA 17 (+3)
COSTITUZIONE 10 (+0)
INTELLIGENZA 7 (-2)
SAGGEZZA 10 (+0)
Carisma 10 (+0)
Classe Armatura 13
\hspace*{0pt}\hfill{Punti Ferita}: 7 (3d4)
Velocità 12 m (3 m, volo 12 m in forma di pipistrello; 12 m,
scalata 12 m in forma di centopiedi; 12 m, nuoto 12 m in forma
di rospo)
Abilità Furtività +5
Resistenze al Danno freddo, fulmine, fuoco; da botta,
perforante e tagliente di attacchi non magici
Immunità al Danno veleno
Immunità alle Condizioni avvelenato
Sensi scurovisione 36 m, Percezione passiva 10
Linguaggi Abissale, Comune
Sfida 1 (200 PE)
Mutaforma. Il demone può usare la sua azione per trasformarsi
in una forma bestiale da pipistrello, centopiedi o rospo, o per
tornare alla sua vera forma. Le sue statistiche sono le stesse in
tutte le forme, sebbene gli attacchi possano variare per alcune di
esse. Qualsiasi equipaggiamento stia indossando o trasportando
non viene trasformato. Alla morte ritorna alla sua vera forma.
Resistenza alla Magia. Il demone ha +1d6 ai tiri salvezza
contro incantesimi e altri effetti magici.
Azioni
Artigli (Morso in Forma di Bestia). Attacco con arma da
mischia: +4 a colpire, portata 1,5 m, un bersaglio.
Colpisce: 5 (1d4 + 3) danni perforanti. Se il bersaglio è una
creatura, deve riuscire un tiro salvezza di Costituzione DC 10 o
subire 5 (2d4) danni da veleno e restare avvelenato per 1 minuto.
La creatura può ripetere il tiro salvezza al termine di ciascun suo
turno, ponendo termine all’effetto se lo riesce.
Invisibilità. Il demone resta invisibile finché non attacca o
termina la sua concentrazione. Qualsiasi cosa che il demone stia
trasportando o indossando resta invisibile finché rimane in
contatto con il demone.
Spavento (1/Giorno). Una creatura scelta dal demone che si trovi
entro 6 metri da lui, deve riuscire un tiro salvezza di Saggezza
DC 10 o restare spaventata per 1 minuto. Il bersaglio può ripetere
il tiro salvezza al termine di ciascun suo turno, con svantaggio se
il demone è in linea di visuale, ponendo termine all’effetto
prematuramente se riesce il tiro salvezza.
Vrock
Grande immondo (demone), caotico malvagio
FORZA 17 (+3)
DESTREZZA 15 (+2)
COSTITUZIONE 18 (+4)
INTELLIGENZA 8 (-1)
SAGGEZZA 13 (+1)
Carisma 8 (-1)
Classe Armatura 15 (armatura naturale)
\hspace*{0pt}\hfill{Punti Ferita}: 104 (11d10 + 44)
Velocità 12 m, volo 18 m
Tiri Salvezza Destrezza +5, Saggezza +4, Carisma +2
Resistenze al Danno freddo, fulmine, fuoco; da botta,
perforante e tagliente di attacchi non magici
Immunità al Danno veleno
Immunità alle Condizioni avvelenato
Sensi scurovisione 36 m, Percezione passiva 11
Linguaggi Abissale, telepatia 36 m
Sfida 6 (2.300 PE)
Resistenza alla Magia. Il demone ha +1d6 ai tiri salvezza
contro incantesimi e altri effetti magici.
Azioni
Multiattacco. Il demone effettua due attacchi: uno con il becco e
uno con gli speroni.
Becco. Attacco con arma da mischia: +6 a colpire, portata 1,5 m,
un bersaglio.
Colpisce: 10 (2d6 + 3) danni perforanti.
Speroni. Attacco con arma da mischia: +6 a colpire, portata 1,5
m, un bersaglio.
Colpisce: 14 (2d10 + 3) danni taglienti.
Spore (Ricarica 6). Una nube di spore tossiche si diffonde in un
raggio di 4,5 metri intorno al demone. Le spore si propagano
intorno agli angoli. Ogni creatura in quell’area deve riuscire un
tiro salvezza di Costituzione DC 14 o restare avvelenata. Mentre
è avvelenato in questo modo, un bersaglio subisce 5 (1d10) danni
da veleno all’inizio di ciascun suo turno. Il bersaglio può ripetere
il tiro salvezza al termine di ciascun suo turno, ponendo termine
all’effetto se lo riesce. Anche svuotare una fiala di acqua sacra
sul bersaglio pone termine all’effetto.
Strillo Stordente (1/Giorno). Il demone emette uno strillo
orripilante. Ogni creatura entro 6 metri da esso e che lo possa
udire, e non sia un demone, deve riuscire un tiro salvezza di
Costituzione DC 14 o restare stordita fino al termine del
prossimo turno del demone.
Destriero da Incubo
Grande immondo, neutrale malvagio
FORZA 18 (+4)
DESTREZZA 15 (+2)
COSTITUZIONE 16 (+3)
INTELLIGENZA 10 (+0)
SAGGEZZA 13 (+1)
Carisma 15 (+2)
Classe Armatura 13 (armatura naturale)
\hspace*{0pt}\hfill{Punti Ferita}: 68 (8d10 + 24)
Velocità 18 m, volo 24 m
Immunità al Danno fuoco
Sensi Percezione passiva 11
Linguaggi comprende Abissale, Comune e Infernale ma non può
parlare
Sfida 3 (700 PE)
Conferire Resistenza al Fuoco. Il destriero da incubo può
conferire resistenza al danno da fuoco a chiunque lo cavalchi.
Illuminazione. Il destriero da incubo irradia luce intensa in un
raggio di 3 metri e luce fioca per ulteriori 3 metri.
Azioni
Zoccoli. Attacco con arma da mischia: +6 a colpire, portata 1,5
m, un bersaglio.
Colpisce: 13 (2d8 + 4) danni da botta più 7 (2d6) danni da
fuoco.
Passo Etereo. Il destriero da incubo e fino a tre creature
consenzienti entro 1,5 metri da esso possono entrare
magicamente nel Piano Etereo dal Piano Materiale e viceversa.
Diavoli
Diavolo Barbuto
Media immondo (diavolo), legale malvagio
FORZA 16 (+3)
DESTREZZA 15 (+2)
COSTITUZIONE 15 (+2)
INTELLIGENZA 9 (-1)
SAGGEZZA 11 (+0)
Carisma 11 (+0)
Classe Armatura 13 (armatura naturale)
\hspace*{0pt}\hfill{Punti Ferita}: 52 (8d8 + 16)
Velocità 9 m
Tiri Salvezza Forza +5, Costituzione +4, Saggezza +2
Resistenze al Danno freddo; da botta, perforante e tagliente
di attacchi non magici che non siano argentati
Immunità al Danno fuoco, veleno
Immunità alle Condizioni avvelenato
Sensi scurovisione 36 m, Percezione passiva 10
Linguaggi Infernale, telepatia 36 m
Sfida 3 (700 PE)
Resistenza alla Magia. Il diavolo ha +1d6 ai tiri salvezza
contro incantesimi e altri effetti magici.
Risoluto. Il diavolo non può essere spaventato finché riesce a
vedere una creatura alleata entro 9 metri da lui.
Vista del Diavolo. La scurovisione del diavolo non è limitata
dall’oscurità magica.
Azioni
Multiattacco. Il diavolo effettua due attacchi: uno con la barba e
uno con il falcione.
Barba. Attacco con arma da mischia: +5 a colpire, portata 1,5 m,
una creatura.
Colpisce: 6 (1d8 + 2) danni perforanti, e il bersaglio deve
riuscire un tiro salvezza di Costituzione DC 12 o restare
avvelenato per 1 minuto. Mentre è avvelenato in questo modo, il
bersaglio non può recuperare punti ferita. Il bersaglio può
ripetere il tiro salvezza al termine di ciascun suo turno,
terminando l’effetto se riesce il tiro salvezza.
Falcione. Attacco con arma da mischia: +5 a colpire, portata 3
m, un bersaglio.
Colpisce: 8 (1d10 + 3) danni taglienti. Se il bersaglio è una
creatura, ad esclusione di costrutti e non morti, deve riuscire un
tiro salvezza di Costituzione DC 12 o perdere 5 (1d10) punti
ferita all’inizio di ciascun suo turno a causa della ferita infernale.
Ogni volta che il diavolo colpisce il bersaglio ferito con questo
attacco, il danno inflitto dalla ferita aumenta di 5 (1d10).
Qualsiasi creatura può effettuare un’azione per bloccare la ferita
con una prova riuscita di Saggezza (Medicina) DC 12. La ferita
si richiude anche nel caso in cui il bersaglio riceva della magia
guaritrice. 
 
Diavolo delle Catene
Media immondo (diavolo), legale malvagio
FORZA 18 (+4)
DESTREZZA 15 (+2)
COSTITUZIONE 18 (+4)
INTELLIGENZA 11 (+0)
SAGGEZZA 12 (+1)
Carisma 14 (+2)
Classe Armatura 16 (armatura naturale)
\hspace*{0pt}\hfill{Punti Ferita}: 85 (10d8 + 40)
Velocità 9 m
Tiri Salvezza Costituzione +7, Saggezza +4, Carisma +5
Resistenze al Danno freddo; da botta, perforante e tagliente
di attacchi non magici che non siano argentati
Immunità al Danno fuoco, veleno
Immunità alle Condizioni avvelenato
Sensi scurovisione 36 m, Percezione passiva 11
Linguaggi Infernale, telepatia 36 m
Sfida 8 (3.900 PE)
Resistenza alla Magia. Il diavolo ha +1d6 ai tiri salvezza
contro incantesimi e altri effetti magici.
Vista del Diavolo. La scurovisione del diavolo non è limitata
dall’oscurità magica.
Azioni
Multiattacco. Il diavolo effettua due attacchi con la catena.
Catena. Attacco con arma da mischia: +8 a colpire, portata 3 m,
un bersaglio.
Colpisce: 11 (2d6 + 4) danni taglienti. Il bersaglio è afferrato
(DC 14 per fuggire) se il diavolo non sta già afferrando un’altra
creatura. Fino al termine dell’afferrare, il bersaglio è intralciato e
subisce 7 (2d6) danni perforanti all’inizio di ciascun suo turno.
Animare Catene (Ricarica dopo un Riposo Breve o Lungo).
Fino a quattro catene che il diavolo possa vedere e si trovano
entro 18 metri da lui producono dei bordi affilati e si animano
sotto il controllo del diavolo, purché quelle catene non siano né
indossate né trasportate da qualcun altro.
Ogni catena animata è un oggetto con Difesa 20, 20 punti ferita,
resistenza ai danni perforanti, e immunità ai danni psichici e da
tuono. Quando il diavolo usa Multiattacco durante il suo turno,
può usare ciascuna catena animata per effettuare un ulteriore
attacco di catena. Una catena animata può afferrare una creatura
per conto proprio ma non può effettuare attacchi mentre afferra.
Una catena animata ritorna al suo stato inanimato se viene ridotta
a 0 punti ferita o se il diavolo è reso inabile o muore.
Reazioni
Maschera Snervante. Quando una creatura che il diavolo può
vedere inizia il proprio round entro 9 metri dal diavolo, il diavolo
può creare un’illusione per assomigliare all’amore perduto o un
acerrimo rivale di quella creatura. Se la creatura può vedere il
diavolo, deve riuscire un tiro salvezza di Saggezza DC 14 o
rimanere spaventata fino al termine del suo turno.
Diavolo Cornuto
Grande immondo (diavolo), legale malvagio
FORZA 22 (+6)
DESTREZZA 17 (+3)
COSTITUZIONE 21 (+5)
INTELLIGENZA 12 (+1)
SAGGEZZA 16 (+3)
Carisma 17 (+3)
Classe Armatura 18 (armatura naturale)
\hspace*{0pt}\hfill{Punti Ferita}: 178 (17d10 + 85)
Velocità 6 m, volo 18 m
Tiri Salvezza Forza +10, Destrezza +7, Saggezza +7, Carisma
+7
Resistenze al Danno freddo; da botta, perforante e tagliente
di attacchi non magici che non siano argentati
Immunità al Danno fuoco, veleno
Immunità alle Condizioni avvelenato
Sensi scurovisione 36 m, Percezione passiva 13
Linguaggi Infernale, telepatia 36 m
Sfida 11 (7.200 PE)
Resistenza alla Magia. Il diavolo ha +1d6 ai tiri salvezza
contro incantesimi e altri effetti magici.
Vista del Diavolo. La scurovisione del diavolo non è limitata
dall’oscurità magica.
Azioni
Multiattacco. Il diavolo effettua tre attacchi da mischia: due con
il forcone e uno con la coda. Può usare Scagliare Fiamma al
posto di qualsiasi attacco da mischia.
Coda. Attacco con arma da mischia: +10 a colpire, portata 3 m,
un bersaglio.
Colpisce: 10 (1d8 + 6) danni perforanti. Se il bersaglio è una
creatura, ad esclusione di costrutti e non morti, deve riuscire un
tiro salvezza di Costituzione DC 17 o perdere 10 (3d6) punti
ferita all’inizio di ciascun suo turno a causa della ferita infernale.
Ogni volta che il diavolo ferisce il bersaglio con questo attacco,
il danno inflitto dalla ferita aumenta di 10 (3d6). Qualsiasi
creatura può effettuare un’azione per bloccare la ferita riuscendo
una prova di Saggezza (Medicina) DC 12. La ferita si richiude
anche nel caso in cui il bersaglio riceva magia guaritrice.
Forcone. Attacco con arma da mischia: +10 a colpire, portata 3
m, un bersaglio.
Colpisce: 15 (2d8 + 6) danni perforanti.
Pungiglione. Attacco con arma da mischia: +8 a colpire, portata
3 m, un bersaglio.
Colpisce: 13 (2d8 + 4) danni perforanti più 17 (5d6) danni da
veleno, e il bersaglio deve riuscire un tiro salvezza di
Costituzione DC 14, o restare avvelenato per 1 minuto. Il
bersaglio può ripetere il tiro salvezza al termine di ciascun suo
turno, terminando l’effetto se lo riesce.
Scagliare Fiamma. Attacco con incantesimo a Distanza: +7 a
colpire, gittata 45 m, un bersaglio.
Colpisce: 14 (4d6) danni da fuoco. Se il bersaglio è un oggetto
infiammabile che non sia indossato o trasportato, prende fuoco.
Diavolo della Fossa
Grande immondo (diavolo), legale malvagio
FORZA 26 (+8)
DESTREZZA 14 (+2)
COSTITUZIONE 24 (+7)
INTELLIGENZA 22 (+6)
SAGGEZZA 18 (+4)
Carisma 24 (+7)
Classe Armatura 19 (armatura naturale)
\hspace*{0pt}\hfill{Punti Ferita}: 300 (24d10 + 168)
Velocità 9 m, volo 18 m
Tiri Salvezza Destrezza +8, Costituzione +13, Saggezza +10
Resistenze al Danno freddo; da botta, perforante e tagliente
di attacchi non magici che non siano argentati
Immunità al Danno fuoco, veleno
Immunità alle Condizioni avvelenato
Sensi visione del vero 36 m, Percezione passiva 14
Linguaggi Infernale, telepatia 36 m
Sfida 20 (25.000 PE)
Arma Magica. Gli attacchi con arma del diavolo della fossa sono
magici.
Aura di Paura. Qualsiasi creatura ostile al diavolo che inizi il
suo turno entro 6 metri da esso, deve effettuare un tiro salvezza
di Saggezza DC 21, a meno che il diavolo non sia inabile. Se
fallisce il tiro salvezza, la creatura è spaventata fino all’inizio del
suo prossimo turno. Se il tiro salvezza della creatura riesce, la
creatura è immune all’Aura di Paura del diavolo per le
successive 24 ore.
Incantesimi Innati. La caratteristica da incantatore diavolo della
fossa è il Carisma (DC dei tiri salvezza degli incantesimi 21). Il
diavolo della fossa può lanciare questi incantesimi in maniera
innata, senza bisogno di componenti materiali:
A volontà: individuazione del magico, palla di fuoco
3/giorno ciascuno: blocca mostri, muro di fuoco
Resistenza alla Magia. Il diavolo ha +1d6 ai tiri salvezza
contro incantesimi e altri effetti magici.
Azioni
Multiattacco. Il diavolo effettua quattro attacchi: uno con il
morso, uno con l’artiglio, uno con la mazza e uno con la coda.
Artiglio. Attacco con arma da mischia: +14 a colpire, portata 3
m, un bersaglio.
Colpisce: 17 (2d8 + 8) danni taglienti.
Coda. Attacco con arma da mischia: +14 a colpire, portata 3 m,
un bersaglio.
Colpisce: 24 (3d10 + 8) danni da botta.
Mazza. Attacco con arma da mischia: +14 a colpire, portata 3 m,
un bersaglio.
Colpisce: 15 (2d6 + 8) danni da botta più 21 (6d6) danni da
fuoco.
Morso. Attacco con arma da mischia: +14 a colpire, portata 1,5
m, un bersaglio.
Colpisce: 22 (4d6 + 8) danni perforanti. Il bersaglio deve riuscire
un tiro salvezza di Costituzione DC 21 o restare avvelenato.
Mentre è avvelenato in questo modo, il bersaglio non può
recuperare punti ferita, e subisce 21 (6d6) danni da veleno
all’inizio di ciascun suo turno. Il bersaglio avvelenato può
ripetere il tiro salvezza al termine di ciascun suo turno,
terminando l’effetto su di sé.
Diavolo del Ghiaccio
Grande immondo (diavolo), legale malvagio
FORZA 21 (+5)
DESTREZZA 14 (+2)
COSTITUZIONE 18 (+4)
INTELLIGENZA 18 (+4)
SAGGEZZA 15 (+2)
Carisma 18 (+4)
Classe Armatura 18 (armatura naturale)
\hspace*{0pt}\hfill{Punti Ferita}: 180 (19d10 + 76)
Velocità 12 m
Tiri Salvezza Destrezza +7, Costituzione +9, Saggezza +7,
Carisma +9
Resistenze al Danno da botta, perforante e tagliente di
attacchi non magici che non siano argentate
Immunità al Danno freddo, fuoco, veleno
Immunità alle Condizioni avvelenato
Sensi vista cieca 18 m, scurovisione 36 m, Percezione passiva 12
Linguaggi Infernale, telepatia 36 m
Sfida 14 (11.500 PE)
Resistenza alla Magia. Il diavolo ha +1d6 ai tiri salvezza
contro incantesimi e altri effetti magici.
Vista del Diavolo. La scurovisione del diavolo non è limitata
dall’oscurità magica.
Azioni
Multiattacco. Il diavolo effettua tre attacchi: uno con il morso,
uno con gli artigli e uno con la coda. In alternativa effettua due
attacchi: uno con la coda e uno con lancia.
Artigli. Attacco con arma da mischia: +10 a colpire, portata 1,5
m, un bersaglio.
Colpisce: 10 (2d4 + 5) danni taglienti più 10 (3d6) danni da
freddo.
Coda. Attacco con arma da mischia: +10 a colpire, portata 3 m,
un bersaglio.
Colpisce: 12 (2d6 + 5) danni da botta più 10 (3d6) danni da
freddo.
Lancia di Ghiaccio. Attacco con arma da mischia: +10 a colpire,
portata 3 m, un bersaglio.
Colpisce: 14 (2d8 + 5) danni perforanti più 10 (3d6) danni da
freddo. Se il bersaglio è una creatura, deve riuscire un tiro
salvezza di Costituzione DC 15, o avere per 1 minuto la velocità
ridotta di 3 metri; durante ciascun suo turno può effettuare solo
un’azione o un’azione bonus, ma non entrambe; non può
effettuare reazioni. Il bersaglio può ripetere il tiro salvezza al
termine di ciascun suo turno, terminando l’effetto su di sé in caso
di successo.
Morso. Attacco con arma da mischia: +10 a colpire, portata 1,5
m, un bersaglio.
Colpisce: 12 (2d6 + 5) danni perforanti più 10 (3d6) danni da
freddo.
 
Muro di Ghiaccio (Ricarica 6). Il diavolo forma magicamente
un muro di ghiaccio opaco su di una superficie solida che possa
vedere entro 18 metri da lui. Il muro è spesso 30 centimetri e
largo fino a 9 metri per un massimo di 3 metri di altezza, oppure
è una cupola semisferica di massimo 6 metri di diametro.
Quando la parete appare, ogni creatura nel suo spazio viene
spinta fuori da esso tramite la via più breve. La creatura sceglie
su quale lato del muro finire, a meno che la creatura non sia
inabile. La creatura poi effettua un tiro salvezza di Destrezza DC
17, subendo 35 (10d6) danni da freddo se lo fallisce, o la metà di
questi danni se lo riesce.
Il muro rimane per 1 minuto o finché il diavolo non è reso
inabile o muore. Il muro può essere danneggiato e bucato; ogni
sezione di 3 metri ha Difesa 5, 30 punti ferita, vulnerabilità al danno
da fuoco, e immunità al danno da acido, freddo, necrotico,
psichico e da veleno. Se una sezione viene distrutta, lascia una
patina di aria gelida nello spazio che occupava prima il muro.
Ogni volta che una creatura finisce per muoversi attraverso
quest’aria gelida durante un turno, consenziente o meno, deve
effettuare un tiro salvezza di Costituzione DC 17, subendo 17
(5d6) danni da freddo se lo fallisce, o la metà di questi danni se
lo riesce. L’aria gelida si dissipa quando il resto del muro
svanisce.
Diavolo d’Ossa
Grande immondo (diavolo), legale malvagio
FORZA 18 (+4)
DESTREZZA 16 (+3)
COSTITUZIONE 18 (+4)
INTELLIGENZA 13 (+1)
SAGGEZZA 14 (+2)
Carisma 16 (+3)
Classe Armatura 19 (armatura naturale)
\hspace*{0pt}\hfill{Punti Ferita}: 142 (15d10 + 60)
Velocità 12 m, volo 12 m
Tiri Salvezza Intelligenza +5, Saggezza +6, Carisma +7
Abilità Inganno +7, Intuizione +6
Resistenze al Danno freddo; da botta, perforante e tagliente
di attacchi non magici che non siano argentati
Immunità al Danno fuoco, veleno
Immunità alle Condizioni avvelenato
Sensi scurovisione 36 m, Percezione passiva 12
Linguaggi Infernale, telepatia 36 m
Sfida 9 (5.000 PE)
Resistenza alla Magia. Il diavolo ha +1d6 ai tiri salvezza
contro incantesimi e altri effetti magici.
Vista del Diavolo. La scurovisione del diavolo non è limitata
dall’oscurità magica.
Azioni
Multiattacco. Il diavolo effettua tre attacchi: due con gli artigli e
uno con il pungiglione oppure uno con la sua arma inastata
uncinata e uno con il pungiglione.
Arma Inastata Uncinata. Attacco con arma da mischia: +8 a
colpire, portata 3 m, un bersaglio.
Colpisce: 17 (2d12 + 4) danni perforanti. Se il bersaglio è una
creatura di taglia Enorme o inferiore, è afferrato (DC 14 per
fuggire). Fino al termine dell’afferrare, il diavolo non può usare
la sua arma inastata su di un altro bersaglio.
Artiglio. Attacco con arma da mischia: +8 a colpire, portata 3 m,
un bersaglio.
Colpisce: 8 (1d8 + 4) danni taglienti.
Pungiglione. Attacco con arma da mischia: +8 a colpire, portata
3 m, un bersaglio.
Colpisce: 13 (2d8 + 4) danni perforanti più 17 (5d6) danni da
veleno, e il bersaglio deve riuscire un tiro salvezza di
Costituzione DC 14, o restare avvelenato per 1 minuto. Il
bersaglio può ripetere il tiro salvezza al termine di ciascun suo
turno, terminando l’effetto se lo riesce.
Diavolo Spinoso
Piccola immondo (diavolo), legale malvagio
FORZA 10 (+0)
DESTREZZA 15 (+2)
COSTITUZIONE 12 (+1)
INTELLIGENZA 11 (+0)
SAGGEZZA 14 (+2)
Carisma 8 (-1)
Classe Armatura 13 (armatura naturale)
\hspace*{0pt}\hfill{Punti Ferita}: 22 (5d6 + 5)
Velocità 6 m, volo 12 m
Resistenze al Danno freddo; da botta, perforante e tagliente
di attacchi non magici che non siano argentati
Immunità al Danno fuoco, veleno
Immunità alle Condizioni avvelenato
Sensi scurovisione 36 m, Percezione passiva 18
Linguaggi Infernale, telepatia 36 m
Sfida 2 (450 PE)
Resistenza alla Magia. Il diavolo ha +1d6 ai tiri salvezza
contro incantesimi e altri effetti magici.
Sorvolare. Il diavolo non provoca attacchi di opportunità quando
vola via dalla portata di un nemico.
Spine Limitate. Il diavolo possiede dodici spine caudali. Le
spine usate ricrescono al termine di un riposo lungo da parte del
diavolo.
Vista del Diavolo. La scurovisione del diavolo non è limitata
dall’oscurità magica.
Azioni
Multiattacco. Il diavolo effettua due attacchi: uno con il morso e
uno con il suo forcone o due con le sue spine caudali.
Forcone. Attacco con arma da mischia: +2 a colpire, portata 1,5
m, un bersaglio.
Colpisce: 3 (1d6) danni perforanti.
Morso. Attacco con arma da mischia: +2 a colpire, portata 1,5
m, un bersaglio.
Colpisce: 5 (2d4) danni taglienti.
Spina Caudale. Attacco con arma a Distanza: +4 a colpire,
gittata 6/24 m, un bersaglio.
Colpisce: 4 (1d4 + 2) danni perforanti più 3 (1d6) danni da
fuoco.
Erinni
Media immondo (diavolo), legale malvagio
FORZA 18 (+4)
DESTREZZA 16 (+3)
COSTITUZIONE 18 (+4)
INTELLIGENZA 14 (+2)
SAGGEZZA 14 (+2)
Carisma 18 (+4)
Classe Armatura 18 (armatura di piastre)
\hspace*{0pt}\hfill{Punti Ferita}: 153 (18d8 + 72)
Velocità 9 m, volo 18 m
Tiri Salvezza Destrezza +7, Costituzione +8, Saggezza +6,
Carisma +8
Resistenze al Danno freddo; da botta, perforante e tagliente
di attacchi non magici che non siano argentati
Immunità al Danno fuoco, veleno
Immunità alle Condizioni avvelenato
Sensi visione del vero 36 m, Percezione passiva 18
Linguaggi Infernale, telepatia 36 m
Sfida 12 (8.400 PE)
Armi Diaboliche. Gli attacchi con arma dell’erinni sono magici e
infliggono 13 (3d8) danni da veleno aggiuntivi quando
colpiscono (già incluso negli attacchi).
Resistenza alla Magia. L’erinni ha +1d6 ai tiri salvezza
contro incantesimi e altri effetti magici.
Azioni
Multiattacco. L’erinni effettua tre attacchi.
Spada Lunga. Attacco con arma da mischia: +8 a colpire,
portata 1,5 m, un bersaglio.
Colpisce: 8 (1d8 + 4) danni taglienti, o 9 (1d10 + 4) danni
taglienti se usata con due mani, più 13 (3d8) danni da veleno.
Arco Lungo. Attacco con arma a Distanza: +7 a colpire, gittata
45/180 m, un bersaglio.
Colpisce: 7 (1d8 + 4) danni perforanti più 13 (3d8) danni da
veleno, e il bersaglio deve riuscire un tiro salvezza di
Costituzione DC 14 o restare avvelenato. Il veleno rimane finché
non viene rimosso da un incantesimo ristorazione inferiore o
simile.
Reazioni
Parata. L’erinni somma 4 alla sua Difesa contro un attacco da
mischia che lo colpirebbe. Per farlo, l’erinni deve poter vedere il
suo attaccante e impugnare un’arma da mischia.
 
Imp
Minuscola immondo (diavolo, mutaforma), legale malvagio
FORZA 6 (-2)
DESTREZZA 17 (+3)
COSTITUZIONE 13 (+1)
INTELLIGENZA 11 (+0)
SAGGEZZA 12 (+1)
Carisma 14 (+2)
Classe Armatura 13
\hspace*{0pt}\hfill{Punti Ferita}: 10 (3d4 + 3)
Velocità 6 m, volo 12 m (6 m in forma di ratto; 6 m, volo 18 m
in forma di corvo; 6 m, scalata 6 m in forma di ragno)
Abilità Furtività +5, Inganno +4, Intuizione +3, Persuasione +4
Resistenze al Danno freddo; da botta, perforante e tagliente
di attacchi non magici che non siano argentati
Immunità al Danno fuoco, veleno
Immunità alle Condizioni avvelenato
Sensi scurovisione 36 m, Percezione passiva 11
Linguaggi Infernale, Comune
Sfida 1 (200 PE)
Mutaforma. Il diavolo può usare la sua azione per trasformarsi
in una forma bestiale da ratto, corvo o ragno, o per tornare alla
sua vera forma. Le sue statistiche sono le stesse in tutte le forme,
sebbene gli attacchi possano variare per alcune di esse. Qualsiasi
equipaggiamento stia indossando o trasportando non viene
trasformato. Alla morte ritorna alla sua vera forma.
Resistenza alla Magia. Il diavolo ha +1d6 ai tiri salvezza
contro incantesimi e altri effetti magici.
Vista del Diavolo. La scurovisione del diavolo non è limitata
dall’oscurità magica.
Azioni
Pungiglione (Morso in Forma di Bestia). Attacco con arma da
mischia: +5 a colpire, portata 1,5 m, una creatura.
Colpisce: 5 (1d4 + 3) danni perforanti, e il bersaglio deve
effettuare un tiro salvezza di Costituzione DC 11, subendo 10
(3d6) danni da veleno se lo fallisce, o la metà di questi danni se
lo riesce.
Invisibilità. Il diavolo resta invisibile finché non attacca o
termina la sua concentrazione. Qualsiasi cosa che il diavolo stia
trasportando o indossando, resta invisibile finché rimane in
contatto con il diavolo.
Lemure
Media immondo (diavolo), legale malvagio
FORZA 10 (+0)
DESTREZZA 5 (-3)
COSTITUZIONE 11 (+0)
INTELLIGENZA 1 (-5)
SAGGEZZA 11 (+0)
Carisma 3 (-4)
Classe Armatura 7
\hspace*{0pt}\hfill{Punti Ferita}: 13 (3d8)
Velocità 4,5 m
Resistenze al Danno freddo
Immunità al Danno fuoco, veleno
Immunità alle Condizioni affascinato, avvelenato, spaventato
Sensi scurovisione 36 m, Percezione passiva 10
Linguaggi comprende l’Infernale ma non può parlare
Sfida 0 (10 PE)
Rinvigorimento Diabolico. Un lemure che muore nei Nove
Inferi ritorna in vita con tutti i suoi punti ferita in 1d10 giorni a
meno che non venga ucciso da una creatura di allineamento
buono su cui sia stato eseguito l’incantesimo benedire o i suoi
resti vengano cosparsi di acqua sacra.
Vista del Diavolo. La scurovisione del diavolo non è limitata
dall’oscurità magica.
Azioni
Pugno. Attacco con arma da mischia: +3 a colpire, portata 1,5
m, un bersaglio.
Colpisce: 2 (1d4) danni da botta.
Dinosauri
Plesiosauro
Grande bestia, disallineato
FORZA 18 (+4)
DESTREZZA 15 (+2)
COSTITUZIONE 16 (+3)
INTELLIGENZA 2 (-4)
SAGGEZZA 12 (+1)
Carisma 5 (-3)
Classe Armatura 13 (armatura naturale)
\hspace*{0pt}\hfill{Punti Ferita}: 68 (8d10 + 24)
Velocità 6 m, nuoto 12 m
Abilità Furtività +4, Percezione +3
Sensi Percezione passiva 13
Linguaggi -
Sfida 2 (450 PE)
Trattenere il Fiato. Il plesiosauro può trattenere il fiato per 1
ora.
Azioni
Morso. Attacco con arma da mischia: +6 a colpire, portata 3 m,
un bersaglio.
Colpisce: 14 (3d6 + 4) danni perforanti.
Tirannosauro
Enorme bestia, disallineato
FORZA 25 (+7)
DESTREZZA 10 (+0)
COSTITUZIONE 19 (+4)
INTELLIGENZA 2 (-4)
SAGGEZZA 12 (+1)
Carisma 9 (-1)
Classe Armatura 13 (armatura naturale)
\hspace*{0pt}\hfill{Punti Ferita}: 136 (13d12 + 52)
Velocità 15 m
Abilità Percezione +4
Sensi Percezione passiva 14
Linguaggi -
Sfida 8 (3.900 PE)
Azioni
Multiattacco. Il tirannosauro effettua due attacchi: uno con il
morso e uno con la coda. Non può effettuare entrambi gli
attacchi contro lo stesso bersaglio.
Coda. Attacco con arma da mischia: +10 a colpire, portata 3 m,
un bersaglio.
Colpisce: 20 (3d8 + 7) danni da botta.
Morso. Attacco con arma da mischia: +10 a colpire, portata 3 m,
un bersaglio.
Colpisce: 33 (4d12 + 7) danni perforanti. Se il bersaglio è una
creatura di taglia Media o inferiore, è afferrato (DC 17 per
fuggire). Fino al termine dell’afferrare, il bersaglio è intralciato,
e il tirannosauro non può usare il morso contro un altro bersaglio.
Triceratopo
Enorme bestia, disallineato
FORZA 22 (+6)
DESTREZZA 9 (-1)
COSTITUZIONE 17 (+3)
INTELLIGENZA 2 (-4)
SAGGEZZA 11 (+0)
Carisma 5 (-3)
Classe Armatura 13 (armatura naturale)
\hspace*{0pt}\hfill{Punti Ferita}: 95 (10d12 + 30)
Velocità 15 m
Sensi Percezione passiva 10
Linguaggi -
Sfida 5 (1.800 PE)
Carica Travolgente. Se il triceratopo si muove di almeno 6 metri
diretto verso una creatura e la colpisce con un attacco di
incornata durante lo stesso turno, il bersaglio deve riuscire un
tiro salvezza di Forza DC 13 o cadere prono. Se il bersaglio è
prono, il triceratopo può effettuare un attacco di pestone contro
di lui come azione bonus.
Azioni
Incornata. Attacco con arma da mischia: +9 a colpire, portata
1,5 m, un bersaglio.
Colpisce: 24 (3d10 + 6) danni perforanti.
Pestone. Attacco con arma da mischia: +9 a colpire, portata 1,5
m, una creatura prona.
Colpisce: 22 (3d10 + 6) danni da botta.
 
Doppelganger
Media mostruosità (mutaforma), neutrale
FORZA 11 (+0)
DESTREZZA 18 (+4)
COSTITUZIONE 14 (+2)
INTELLIGENZA 11 (+0)
SAGGEZZA 12 (+1)
Carisma 14 (+2)
Classe Armatura 14
\hspace*{0pt}\hfill{Punti Ferita}: 52 (8d8 + 16)
Velocità 9 m
Abilità Inganno +6, Intuizione +3
Immunità alle Condizioni affascinato
Sensi scurovisione 18 m, Percezione passiva 11
Linguaggi Comune
Sfida 3 (700 PE)
Mutaforma. Il doppelganger può usare la sua azione per
cambiare la propria forma in quella di un umanoide Piccolo o
Medio che abbia visto, o per tornare alla sua vera forma. Le sue
statistiche, a parte la taglia, sono le stesse in tutte le forme.
Qualsiasi equipaggiamento stia indossando o trasportando non
viene trasformato. Alla morte ritorna alla sua vera forma.
Appostato. Nel primo round di combattimento, il doppelganger
ha vantaggio ai tiri di attacco contro qualsiasi creatura abbia
preso di sorpresa.
Attacco di Sorpresa. Se il doppelganger sorprende una creatura e
la colpisce con un attacco durante il primo round di
combattimento, il bersaglio subisce 10 (3d6) danni aggiuntivi
dall’attacco.
Azioni
Multiattacco. Il doppelganger effettua due attacchi da mischia.
Schianto. Attacco con arma da mischia: +6 a colpire, portata 1,5
m, un bersaglio.
Colpisce: 7 (1d6 + 4) danni da botta.
Leggere Pensieri. Il doppelganger legge magicamente i pensieri
di superficie di una creatura entro 18 metri da lui. L’effetto può
penetrare le barriere, ma 1 metro di legno o terra, 50 centimetri
di pietra, 5 centimetri di metallo, o un sottile foglio di piombo lo
blocca. Mentre il bersaglio è a gittata, il doppelganger può
continuare a leggerne i pensieri, purché la concentrazione del
doppelganger non venga infranta (come la concentrazione di un
incantesimo). Mentre legge la mente di un bersaglio, il
doppelganger ha vantaggio alle prove di Saggezza (Intuizione) e
Carisma (Inganno, Intimidire e Persuasione) contro il bersaglio.
Draghi Cromatici
Drago Bianco Antico
Mastodontica drago, caotico malvagio
FORZA 26 (+8)
DESTREZZA 10 (+0)
COSTITUZIONE 26 (+8)
INTELLIGENZA 10 (+0)
SAGGEZZA 13 (+1)
Carisma 14 (+2)
Classe Armatura 20 (armatura naturale)
\hspace*{0pt}\hfill{Punti Ferita}: 333 (18d20 + 144)
Velocità 12 m, nuoto 12 m, scavo 12 m, volo 24 m
Tiri Salvezza Destrezza +6, Costituzione +14, Saggezza +7,
Carisma +8
Abilità Furtività +6, Percezione +13
Immunità al Danno freddo
Sensi scurovisione 36 m, vista cieca 18 m, Percezione passiva 23
Linguaggi Comune, Draconico
Sfida 20 (25.000 PE)
Camminare sul Ghiaccio. Il drago può muoversi e arrampicarsi su
superfici ghiacciate senza bisogno di effettuare prove di
caratteristica. Inoltre, il terreno difficile composto di ghiaccio o neve
non gli costa movimento aggiuntivo.
Resistenza Leggendaria (3/Giorno). Se il drago fallisce un tiro
salvezza, può scegliere invece di riuscire.
Azioni
Multiattacco. Il drago può usare la sua Presenza Spaventosa. Poi
effettuare tre attacchi: uno con il morso e due con gli artigli.
Artiglio. Attacco con arma da mischia: +14 a colpire, portata 3
m, un bersaglio.
Colpisce: 15 (2d6 + 8) danni taglienti.
Coda. Attacco con arma da mischia: +14 a colpire, portata 6 m,
un bersaglio.
Colpisce: 17 (2d8 + 8) danni da botta.
Morso. Attacco con arma da mischia: +14 a colpire, portata 4,5
m, un bersaglio.
Colpisce: 19 (2d10 + 8) danni perforanti più 9 (2d8) danni da freddo.
Presenza Spaventosa. Ogni creatura scelta dal drago, che si trovi
entro 36 metri da esso e consapevole della sua presenza, deve
riuscire un tiro salvezza di Saggezza DC 16 o restare spaventata per
1 minuto. Una creatura può ripetere il tiro salvezza al termine di
ciascun suo turno, terminando l’effetto se lo riesce. Se il tiro salvezza
della creatura ha successo o l’effetto ha termine per essa, la creatura è
immune alla Presenza Spaventosa del drago per le successive 24 ore.
Soffio Gelido (Ricarica 5-6). Il drago esala un’esplosione di ghiaccio
in un cono di 27 metri. Ogni creatura in quell’area deve effettuare un tiro
salvezza di Costituzione DC 22 e subire 72 (16d8) danni da freddo se
fallisce il tiro salvezza, o la metà di questi danni se lo riesce.
Azioni Leggendarie
Il drago può effettuare 3 azioni aggiuntive, scelte tra le opzioni
seguenti. Può usare solo un’opzione leggendaria alla volta e solo
al termine del turno di un’altra creatura. Il drago recupera le
azioni aggiuntive spese all’inizio del proprio round.
Attacco di Ala (Costa 2 Azioni). Il drago batte le ali. Ogni creatura
entro 4,5 metri dal drago deve riuscire un tiro salvezza di Destrezza
DC 22 o subire 15 (2d6 + 8) danni da botta e venir gettato prono.
Il drago può poi volare fino a metà della sua velocità di volo.
Attacco di Coda. Il drago effettua un attacco di coda.
Individuare. Il drago effettua una prova di Saggezza (Percezione).
Drago Bianco Adulto
Enorme drago, caotico malvagio
FORZA 22 (+6)
DESTREZZA 10 (+0)
COSTITUZIONE 22 (+6)
INTELLIGENZA 8 (-1)
SAGGEZZA 12 (+1)
Carisma 12 (+1)
Classe Armatura 18 (armatura naturale)
\hspace*{0pt}\hfill{Punti Ferita}: 200 (16d12 + 96)
Velocità 12 m, nuoto 12 m, scavo 9 m, volo 24 m
Tiri Salvezza Destrezza +5, Costituzione +11, Saggezza +6,
Carisma +6
Abilità Furtività +5, Percezione +11
Immunità al Danno freddo
Sensi scurovisione 36 m, vista cieca 18 m, Percezione passiva 21
Linguaggi Comune, Draconico
Sfida 13 (10.000 PE)
Camminare sul Ghiaccio. Il drago può muoversi e arrampicarsi su
superfici ghiacciate senza bisogno di effettuare prove di
caratteristica. Inoltre, il terreno difficile composto di ghiaccio o neve
non gli costa movimento aggiuntivo.
Resistenza Leggendaria (3/Giorno). Se il drago fallisce un tiro
salvezza, può scegliere invece di riuscire.
Azioni
Multiattacco. Il drago può usare la sua Presenza Spaventosa e
poi effettuare tre attacchi: uno con il morso e due con gli artigli.
Artiglio. Attacco con arma da mischia: +11 a colpire, portata 1,5
m, un bersaglio.
Colpisce: 13 (2d6 + 6) danni taglienti.
Coda. Attacco con arma da mischia: +11 a colpire, portata 4,5
m, un bersaglio.
Colpisce: 15 (2d8 + 6) danni da botta.
Morso. Attacco con arma da mischia: +11 a colpire, portata 3 m,
un bersaglio.
Colpisce: 17 (2d10 + 6) danni perforanti più 4 (1d8) danni da freddo.
Presenza Spaventosa. Ogni creatura scelta dal drago, che si trovi
entro 36 metri da esso e consapevole della sua presenza, deve
riuscire un tiro salvezza di Saggezza DC 14 o restare spaventata per
1 minuto. Una creatura può ripetere il tiro salvezza al termine di
ciascun suo turno, terminando l’effetto se lo riesce. Se il tiro salvezza
della creatura ha successo o l’effetto ha termine per essa, la creatura è
immune alla Presenza Spaventosa del drago per le successive 24 ore.
Soffio Gelido (Ricarica 5-6). Il drago esala un’esplosione di
ghiaccio in un cono di 18 metri. Ogni creatura in quell’area deve
effettuare un tiro salvezza di Costituzione DC 19 e subire 54
(12d8) danni da freddo se fallisce il tiro salvezza, o la metà di
questi danni se lo riesce.
Azioni Leggendarie
Il drago può effettuare 3 azioni aggiuntive, scelte tra le opzioni
seguenti. Può usare solo un’opzione leggendaria alla volta e solo
al termine del turno di un’altra creatura. Il drago recupera le
azioni aggiuntive spese all’inizio del proprio round.
Attacco di Ala (Costa 2 Azioni). Il drago batte le ali. Ogni creatura
entro 3 metri dal drago deve riuscire un tiro salvezza di Destrezza
DC 19 o subire 13 (2d6 + 6) danni da botta e venir gettato prono.
Il drago può poi volare fino a metà della sua velocità di volo.
Attacco di Coda. Il drago effettua un attacco di coda.
Individuare. Il drago effettua una prova di Saggezza (Percezione).
Drago Bianco Giovane
Grande drago, caotico malvagio
FORZA 18 (+4)
DESTREZZA 10 (+0)
COSTITUZIONE 18 (+4)
INTELLIGENZA 6 (-2)
SAGGEZZA 11 (+0)
Carisma 12 (+1)
Classe Armatura 17 (armatura naturale)
\hspace*{0pt}\hfill{Punti Ferita}: 133 (14d10 + 56)
Velocità 12 m, nuoto 12 m, scavo 6 m, volo 24 m
Tiri Salvezza Destrezza +3, Costituzione +7, Saggezza +3,
Carisma +4
Abilità Furtività +3, Percezione +6
Immunità al Danno freddo
Sensi scurovisione 36 m, vista cieca 9 m, Percezione passiva 16
Linguaggi Comune, Draconico
Sfida 6 (2.300 PE)
Camminare sul Ghiaccio. Il drago può muoversi e arrampicarsi su
superfici ghiacciate senza bisogno di effettuare prove di
caratteristica. Inoltre, il terreno difficile composto di ghiaccio o neve
non gli costa movimento aggiuntivo.
Azioni
Multiattacco. Il drago può usare la sua Presenza Spaventosa. Poi
effettuare tre attacchi: uno con il morso e due con gli artigli.
Artiglio. Attacco con arma da mischia: +7 a colpire, portata 1,5
m, un bersaglio.
Colpisce: 11 (2d6 + 4) danni taglienti.
Morso. Attacco con arma da mischia: +7 a colpire, portata 3 m,
un bersaglio.
Colpisce: 15 (2d10 + 4) danni perforanti più 4 (1d8) danni da freddo.
Soffio Gelido (Ricarica 5-6). Il drago esala un’esplosione di ghiaccio
in un cono di 9 metri. Ogni creatura in quell’area deve effettuare un tiro
salvezza di Costituzione DC 15 e subire 45 (10d8) danni da freddo se
fallisce il tiro salvezza, o la metà di questi danni se lo riesce.
Drago Bianco Cucciolo
Media drago, caotico malvagio
FORZA 14 (+2)
DESTREZZA 10 (+0)
COSTITUZIONE 14 (+2)
INTELLIGENZA 5 (-3)
SAGGEZZA 10 (+0)
Carisma 11 (+0)
Classe Armatura 16 (armatura naturale)
\hspace*{0pt}\hfill{Punti Ferita}: 32 (5d8 + 10)
Velocità 9 m, nuoto 9 m, scavo 4,5 m, volo 18 m
Tiri Salvezza Destrezza +2, Costituzione +4, Saggezza +2,
Carisma +2
Abilità Furtività +2, Percezione +4
Immunità al Danno freddo
Sensi scurovisione 18 m, vista cieca 3 m, Percezione passiva 14
Linguaggi Draconico
Sfida 2 (450 PE)
Azioni
Morso. Attacco con arma da mischia: +7 a colpire, portata 3 m,
un bersaglio.
Colpisce: 15 (2d10 + 4) danni perforanti più 4 (1d8) danni da freddo.
Soffio Gelido (Ricarica 5-6). Il drago esala un’esplosione di ghiaccio
in un cono di 4,5 metri. Ogni creatura in quell’area deve effettuare un
tiro salvezza di Costituzione DC 12 e subire 22 (5d8) danni da freddo
se fallisce il tiro salvezza, o la metà di questi danni se lo riesce.
 
Drago Blu Antico
Mastodontica drago, legale malvagio
FORZA 29 (+9)
DESTREZZA 10 (+0)
COSTITUZIONE 27 (+8)
INTELLIGENZA 18 (+4)
SAGGEZZA 17 (+3)
Carisma 21 (+5)
Classe Armatura 22 (armatura naturale)
\hspace*{0pt}\hfill{Punti Ferita}: 481 (26d20 + 208)
Velocità 12 m, scavo 12 m, volo 24 m
Tiri Salvezza Destrezza +7, Costituzione +15, Saggezza +10,
Carisma +12
Abilità Furtività +7, Percezione +17
Immunità al Danno fulmine
Sensi scurovisione 36 m, vista cieca 18 m, Percezione passiva 27
Linguaggi Comune, Draconico
Sfida 23 (50.000 PE)
Resistenza Leggendaria (3/Giorno). Se il drago fallisce un tiro
salvezza, può scegliere invece di riuscire.
Azioni
Multiattacco. Il drago può usare la sua Presenza Spaventosa. Poi
effettuare tre attacchi: uno con il morso e due con gli artigli.
Artiglio. Attacco con arma da mischia: +16 a colpire, portata 3
m, un bersaglio.
Colpisce: 16 (2d6 + 9) danni taglienti.
Coda. Attacco con arma da mischia: +16 a colpire, portata 6 m,
un bersaglio.
Colpisce: 18 (2d8 + 9) danni da botta.
Morso. Attacco con arma da mischia: +16 a colpire, portata 4,5
m, un bersaglio.
Colpisce: 20 (2d10 + 9) danni perforanti più 11 (2d10) danni da
fulmine.
Presenza Spaventosa. Ogni creatura scelta dal drago, che si trovi
entro 36 metri da esso e consapevole della sua presenza, deve
riuscire un tiro salvezza di Saggezza DC 20 o restare spaventata
per 1 minuto. Una creatura può ripetere il tiro salvezza al termine
di ciascun suo turno, terminando l’effetto se lo riesce. Se il tiro
salvezza della creatura ha successo o l’effetto ha termine per
essa, la creatura è immune alla Presenza Spaventosa del drago
per le successive 24 ore.
Soffio Fulminante (Ricarica 5-6). Il drago esala fulmini in una
linea lunga 36 metri e larga 3 metri. Ogni creatura su quella linea
deve effettuare un tiro salvezza di Destrezza DC 23 e subire 88
(16d10) danni da fulmine se fallisce il tiro salvezza, o la metà di
questi danni se lo riesce.
Azioni Leggendarie
Il drago può effettuare 3 azioni aggiuntive, scelte tra le opzioni
seguenti. Può usare solo un’opzione leggendaria alla volta e solo
al termine del turno di un’altra creatura. Il drago recupera le
azioni aggiuntive spese all’inizio del proprio round.
Attacco di Ala (Costa 2 Azioni). Il drago batte le ali. Ogni
creatura entro 4,5 metri dal drago deve riuscire un tiro salvezza
di Destrezza DC 24 o subire 16 (2d6 + 9) danni da botta e
venir gettato prono. Il drago può poi volare fino a metà della sua
velocità di volo.
Attacco di Coda. Il drago effettua un attacco di coda.
Individuare. Il drago effettua una prova di Saggezza
(Percezione).
Drago Blu Adulto
Enorme drago, legale malvagio
FORZA 25 (+7)
DESTREZZA 10 (+0)
COSTITUZIONE 23 (+6)
INTELLIGENZA 16 (+3)
SAGGEZZA 15 (+2)
Carisma 19 (+4)
Classe Armatura 19 (armatura naturale)
\hspace*{0pt}\hfill{Punti Ferita}: 225 (18d12 + 108)
Velocità 12 m, scavo 12 m, volo 24 m
Tiri Salvezza Destrezza +5, Costituzione +11, Saggezza +7,
Carisma +9
Abilità Furtività +5, Percezione +12
Immunità al Danno fulmine
Sensi scurovisione 36 m, vista cieca 18 m, Percezione passiva 22
Linguaggi Comune, Draconico
Sfida 16 (15.000 PE)
Resistenza Leggendaria (3/Giorno). Se il drago fallisce un tiro
salvezza, può scegliere invece di riuscire.
Azioni
Multiattacco. Il drago può usare la sua Presenza Spaventosa. Poi
effettuare tre attacchi: uno con il morso e due con gli artigli.
Artiglio. Attacco con arma da mischia: +12 a colpire, portata 1,5
m, un bersaglio.
Colpisce: 14 (2d6 + 7) danni taglienti.
Coda. Attacco con arma da mischia: +12 a colpire, portata 4,5
m, un bersaglio.
Colpisce: 16 (2d8 + 7) danni da botta.
Morso. Attacco con arma da mischia: +12 a colpire, portata 3 m,
un bersaglio.
Colpisce: 18 (2d10 + 7) danni perforanti più 5 (1d10) danni da
fulmine.
Presenza Spaventosa. Ogni creatura scelta dal drago, che si trovi
entro 36 metri da esso e consapevole della sua presenza, deve
riuscire un tiro salvezza di Saggezza DC 17 o restare spaventata
per 1 minuto. Una creatura può ripetere il tiro salvezza al termine
di ciascun suo turno, terminando l’effetto se lo riesce. Se il tiro
salvezza della creatura ha successo o l’effetto ha termine per
essa, la creatura è immune alla Presenza Spaventosa del drago
per le successive 24 ore.
Soffio Fulminante (Ricarica 5-6). Il drago esala fulmini in una
linea lunga 27 metri e larga 1,5 metri. Ogni creatura su quella
linea deve effettuare un tiro salvezza di Destrezza DC 19 e subire
66 (12d10) danni da fulmine se fallisce il tiro salvezza, o la metà
di questi danni se lo riesce.
Azioni Leggendarie
Il drago può effettuare 3 azioni aggiuntive, scelte tra le opzioni
seguenti. Può usare solo un’opzione leggendaria alla volta e solo
al termine del turno di un’altra creatura. Il drago recupera le
azioni aggiuntive spese all’inizio del proprio round.
Attacco di Ala (Costa 2 Azioni). Il drago batte le ali. Ogni
creatura entro 3 metri dal drago deve riuscire un tiro salvezza di
Destrezza DC 20 o subire 14 (2d6 + 7) danni da botta e venir
gettato prono. Il drago può poi volare fino a metà della sua
velocità di volo.
Attacco di Coda. Il drago effettua un attacco di coda.
Individuare. Il drago effettua una prova di Saggezza
(Percezione).
Drago Blu Giovane
Enorme drago, legale malvagio
FORZA 21(+5)
DESTREZZA 10 (+0)
COSTITUZIONE 19 (+4)
INTELLIGENZA 14 (+2)
SAGGEZZA 13 (+1)
Carisma 17 (+3)
Classe Armatura 18 (armatura naturale)
\hspace*{0pt}\hfill{Punti Ferita}: 152 (16d10 + 64)
Velocità 12 m, scavo 12 m, volo 24 m
Tiri Salvezza Destrezza +4, Costituzione +8, Saggezza +5,
Carisma +7
Abilità Furtività +4, Percezione +9
Immunità al Danno fulmine
Sensi scurovisione 36 m, vista cieca 9 m, Percezione passiva 19
Linguaggi Comune, Draconico
Sfida 9 (5.000 PE)
Azioni
Multiattacco. Il drago può effettuare tre attacchi: uno con il
morso e due con gli artigli.
Artiglio. Attacco con arma da mischia: +9 a colpire, portata 1,5
m, un bersaglio.
Colpisce: 12 (2d6 + 5) danni taglienti.
Morso. Attacco con arma da mischia: +9 a colpire, portata 3 m,
un bersaglio.
Colpisce: 16 (2d10 + 5) danni perforanti più 5 (1d10) danni da
fulmine.
Soffio Fulminante (Ricarica 5-6). Il drago esala fulmini in una linea
lunga 18 metri e larga 1,5 metri. Ogni creatura su quella linea deve
effettuare un tiro salvezza di Destrezza DC 16 e subire 55 (10d10) danni
da fulmine se fallisce il tiro salvezza, o la metà di questi danni se lo riesce.
Drago Blu Cucciolo
Enorme drago, legale malvagio
FORZA 17 (+3)
DESTREZZA 10 (+0)
COSTITUZIONE 15 (+2)
INTELLIGENZA 12 (+1)
SAGGEZZA 11 (+0)
Carisma 15 (+2)
Classe Armatura 17 (armatura naturale)
\hspace*{0pt}\hfill{Punti Ferita}: 52 (8d8 + 16)
Velocità 9 m, scavo 4,5 m, volo 18 m
Tiri Salvezza Destrezza +2, Costituzione +4, Saggezza +2,
Carisma +4
Abilità Furtività +2, Percezione +4
Immunità al Danno fulmine
Sensi scurovisione 18 m, vista cieca 3 m, Percezione passiva 14
Linguaggi Draconico
Sfida 3 (700 PE)
Azioni
Morso. Attacco con arma da mischia: +5 a colpire, portata 1,5
m, un bersaglio.
Colpisce: 8 (1d10 + 3) danni perforanti più 3 (1d6) danni da
fulmine.
Soffio Fulminante (Ricarica 5-6). Il drago esala fulmini in una
linea lunga 9 metri e larga 1,5 metri. Ogni creatura su quella
linea deve effettuare un tiro salvezza di Destrezza DC 12 e subire
22 (4d10) danni da fulmine se fallisce il tiro salvezza, o la metà
di questi danni se lo riesce.
Drago Nero Antico
Mastodontica drago, caotico malvagio
FORZA 27 (+8)
DESTREZZA 14 (+2)
COSTITUZIONE 25 (+7)
INTELLIGENZA 16 (+3)
SAGGEZZA 15 (+2)
Carisma 19 (+4)
Classe Armatura 22 (armatura naturale)
\hspace*{0pt}\hfill{Punti Ferita}: 367 (21d20 + 147)
Velocità 12 m, scalata 12 m, volo 24 m
Tiri Salvezza Destrezza +9, Costituzione +14, Saggezza +9,
Carisma +11
Abilità Furtività +9, Percezione +16
Immunità al Danno acido
Sensi scurovisione 36 m, vista cieca 18 m, Percezione passiva 26
Linguaggi Comune, Draconico
Sfida 21 (33.000 PE)
Anfibio. Il drago può respirare aria e acqua.
Resistenza Leggendaria (3/Giorno). Se il drago fallisce un tiro
salvezza, può scegliere invece di riuscire.
Azioni
Multiattacco. Il drago può usare la sua Presenza Spaventosa. Poi
effettuare tre attacchi: uno con il morso e due con gli artigli.
Artiglio. Attacco con arma da mischia: +15 a colpire, portata 3
m, un bersaglio.
Colpisce: 15 (2d6 + 8) danni taglienti.
Coda. Attacco con arma da mischia: +15 a colpire, portata 6 m,
un bersaglio.
Colpisce: 17 (2d8 + 8) danni da botta.
Morso. Attacco con arma da mischia : +15 a colpire, portata 4,5
m, un bersaglio.
Colpisce: 19 (2d10 + 8) danni perforanti più 9 (4d6) danni da
acido.
Presenza Spaventosa. Ogni creatura scelta dal drago, che si trovi
entro 36 metri da esso e consapevole della sua presenza, deve
riuscire un tiro salvezza di Saggezza DC 19 o restare spaventata
per 1 minuto. Una creatura può ripetere il tiro salvezza al termine
di ciascun suo turno, terminando l’effetto se lo riesce. Se il tiro
salvezza della creatura ha successo o l’effetto ha termine per
essa, la creatura è immune alla Presenza Spaventosa del drago
per le successive 24 ore.
Soffio Acido (Ricarica 5-6). Il drago esala acido in una linea di
27 metri larga 3 metri. Ogni creatura in quell’area deve effettuare
un tiro salvezza di Destrezza DC 22 e subire 67 (15d8) danni da
acido se fallisce il tiro salvezza, o la metà di questi danni se lo
riesce.
Azioni Leggendarie
Il drago può effettuare 3 azioni aggiuntive, scelte tra le opzioni
seguenti. Può usare solo un’opzione leggendaria alla volta e solo
al termine del turno di un’altra creatura. Il drago recupera le
azioni aggiuntive spese all’inizio del proprio round.
Attacco di Ala (Costa 2 Azioni). Il drago batte le ali. Ogni
creatura entro 4,5 metri dal drago deve riuscire un tiro salvezza
di Destrezza DC 23 o subire 15 (2d6 + 8) danni da botta e
venir gettato prono. Il drago può poi volare fino a metà della sua
velocità di volo.
Attacco di Coda. Il drago effettua un attacco di coda.
Individuare. Il drago effettua una prova di Saggezza
(Percezione).
 
Drago Nero Adulto
Enorme drago, caotico malvagio
FORZA 23 (+6)
DESTREZZA 14 (+2)
COSTITUZIONE 21 (+5)
INTELLIGENZA 14 (+2)
SAGGEZZA 13 (+1)
Carisma 17 (+3)
Classe Armatura 19 (armatura naturale)
\hspace*{0pt}\hfill{Punti Ferita}: 195 (17d12 + 85)
Velocità 12 m, scalata 12 m, volo 24 m
Tiri Salvezza Destrezza +7, Costituzione +10, Saggezza +6,
Carisma +8
Abilità Furtività +7, Percezione +11
Immunità al Danno acido
Sensi scurovisione 36 m, vista cieca 18 m, Percezione passiva 21
Linguaggi Comune, Draconico
Sfida 17 (18.000 PE)
Anfibio. Il drago può respirare aria e acqua.
Resistenza Leggendaria (3/Giorno). Se il drago fallisce un tiro
salvezza, può scegliere invece di riuscire.
Azioni
Multiattacco. Il drago può usare la sua Presenza Spaventosa. Poi
effettuare tre attacchi: uno con il morso e due con gli artigli.
Artiglio. Attacco con arma da mischia: +11 a colpire, portata 1,5
m, un bersaglio.
Colpisce: 13 (2d6 + 6) danni taglienti.
Coda. Attacco con arma da mischia: +11 a colpire, portata 4,5
m, un bersaglio.
Colpisce: 15 (2d8 + 6) danni da botta.
Morso. Attacco con arma da mischia: +11 a colpire, portata 3 m,
un bersaglio.
Colpisce: 17 (2d10 + 6) danni perforanti più 4 (1d8) danni da
acido.
Presenza Spaventosa. Ogni creatura scelta dal drago, che si trovi
entro 36 metri da esso e consapevole della sua presenza, deve
riuscire un tiro salvezza di Saggezza DC 16 o restare spaventata
per 1 minuto. Una creatura può ripetere il tiro salvezza al termine
di ciascun suo turno, terminando l’effetto se lo riesce. Se il tiro
salvezza della creatura ha successo o l’effetto ha termine per
essa, la creatura è immune alla Presenza Spaventosa del drago
per le successive 24 ore.
Soffio Acido (Ricarica 5-6). Il drago esala acido in una linea di
18 metri larga 1,5 metri. Ogni creatura in quell’area deve
effettuare un tiro salvezza di Destrezza DC 18 e subire 54 (12d8)
danni da acido se fallisce il tiro salvezza, o la metà di questi
danni se lo riesce.
Azioni Leggendarie
Il drago può effettuare 3 azioni aggiuntive, scelte tra le opzioni
seguenti. Può usare solo un’opzione leggendaria alla volta e solo
al termine del turno di un’altra creatura. Il drago recupera le
azioni aggiuntive spese all’inizio del proprio round.
Attacco di Ala (Costa 2 Azioni). Il drago batte le ali. Ogni
creatura entro 3 metri dal drago deve riuscire un tiro salvezza di
Destrezza DC 19 o subire 13 (2d6 + 6) danni da botta e venir
gettato prono. Il drago può poi volare fino a metà della sua
velocità di volo.
Attacco di Coda. Il drago effettua un attacco di coda.
Individuare. Il drago effettua una prova di Saggezza
(Percezione).
Drago Nero Giovane
Grande drago, caotico malvagio
FORZA 19 (+4)
DESTREZZA 14 (+2)
COSTITUZIONE 17 (+3)
INTELLIGENZA 12 (+1)
SAGGEZZA 11 (+0)
Carisma 15 (+2)
Classe Armatura 18 (armatura naturale)
\hspace*{0pt}\hfill{Punti Ferita}: 127 (15d10 + 45)
Velocità 12 m, scalata 12 m, volo 24 m
Tiri Salvezza Destrezza +5, Costituzione +6, Saggezza +3,
Carisma +5
Abilità Furtività +5, Percezione +6
Immunità al Danno acido
Sensi scurovisione 36 m, vista cieca 9 m, Percezione passiva 16
Linguaggi Comune, Draconico
Sfida 7 (2.900 PE)
Anfibio. Il drago può respirare aria e acqua.
Azioni
Multiattacco. Il drago può effettuare tre attacchi: uno con il
morso e due con gli artigli.
Artiglio. Attacco con arma da mischia: +10 a colpire, portata 1,5
m, un bersaglio.
Colpisce: 11 (2d6 + 4) danni taglienti.
Morso. Attacco con arma da mischia: +7 a colpire, portata 3 m,
un bersaglio.
Colpisce: 11 (2d10 + 4) danni perforanti più 4 (1d8) danni da acido.
Soffio Acido (Ricarica 5-6). Il drago esala acido in una linea di 9
metri larga 1,5 metri. Ogni creatura in quell’area deve effettuare un tiro
salvezza di Destrezza DC 14 e subire 49 (11d8) danni da acido se fallisce
il tiro salvezza, o la metà di questi danni se lo riesce.
Drago Nero Cucciolo
Media drago, caotico malvagio
FORZA 15 (+2)
DESTREZZA 14 (+2)
COSTITUZIONE 13 (+1)
INTELLIGENZA 10 (+0)
SAGGEZZA 11 (+0)
Carisma 13 (+1)
Classe Armatura 17 (armatura naturale)
\hspace*{0pt}\hfill{Punti Ferita}: 33 (6d8 + 6)
Velocità 9 m, scalata 9 m, volo 18 m
Tiri Salvezza Destrezza +4, Costituzione +3, Saggezza +2,
Carisma +3
Abilità Furtività +4, Percezione +4
Immunità al Danno acido
Sensi scurovisione 18 m, vista cieca 3 m, Percezione passiva 14
Linguaggi Draconico
Sfida 2 (450 PE)
Anfibio. Il drago può respirare aria e acqua.
Azioni
Morso. Attacco con arma da mischia: +4 a colpire, portata 1,5
m, un bersaglio.
Colpisce: 7 (1d10 + 2) danni perforanti più 2 (1d4) danni da acido.
Soffio Acido (Ricarica 5-6). Il drago esala acido in una linea di
4,5 metri larga 1,5 metri. Ogni creatura in quell’area deve
effettuare un tiro salvezza di Destrezza DC 11 e subire 22 (5d8)
danni da acido se fallisce il tiro salvezza, o la metà di questi
danni se lo riesce.
Drago Rosso Antico
Mastodontica drago, caotico malvagio
FORZA 30 (+10)
DESTREZZA 10 (+0)
COSTITUZIONE 29 (+9)
INTELLIGENZA 18 (+4)
SAGGEZZA 15 (+2)
Carisma 23 (+6)
Classe Armatura 22 (armatura naturale)
\hspace*{0pt}\hfill{Punti Ferita}: 546 (28d20 + 252)
Velocità 12 m, scalata 12 m, volo 24 m
Tiri Salvezza Destrezza +7, Costituzione +16, Saggezza +9,
Carisma +13
Abilità Furtività +7, Percezione +16
Immunità al Danno fuoco
Sensi scurovisione 36 m, vista cieca 18 m, Percezione passiva 26
Linguaggi Comune, Draconico
Sfida 24 (62.000 PE)
Resistenza Leggendaria (3/Giorno). Se il drago fallisce un tiro
salvezza, può scegliere invece di riuscire.
Azioni
Multiattacco. Il drago può usare la sua Presenza Spaventosa e
poi effettuare tre attacchi: uno con il morso e due con gli artigli.
Artiglio. Attacco con arma da mischia: +17 a colpire, portata 3
m, un bersaglio.
Colpisce: 17 (2d6 + 10) danni taglienti.
Coda. Attacco con arma da mischia: +17 a colpire, portata 6 m,
un bersaglio.
Colpisce: 19 (2d8 + 10) danni da botta.
Morso. Attacco con arma da mischia: +17 a colpire, portata 4,5
m, un bersaglio.
Colpisce: 21 (2d10 + 10) danni perforanti più 14 (4d6) danni da
fuoco.
Presenza Spaventosa. Ogni creatura scelta dal drago, che si trovi
entro 36 metri da esso e consapevole della sua presenza, deve
riuscire un tiro salvezza di Saggezza DC 21 o restare spaventata
per 1 minuto. Una creatura può ripetere il tiro salvezza al termine
di ciascun suo turno, terminando l’effetto se lo riesce. Se il tiro
salvezza della creatura ha successo o l’effetto ha termine per
essa, la creatura è immune alla Presenza Spaventosa del drago
per le successive 24 ore.
Soffio Infuocato (Ricarica 5-6). Il drago esala fuoco in un cono
di 27 metri. Ogni creatura in quell’area deve effettuare un tiro
salvezza di Destrezza DC 24 e subire 91 (26d6) danni da fuoco
se fallisce il tiro salvezza, o la metà di questi danni se lo riesce.
Azioni Leggendarie
Il drago può effettuare 3 azioni aggiuntive, scelte tra le opzioni
seguenti. Può usare solo un’opzione leggendaria alla volta e solo
al termine del turno di un’altra creatura. Il drago recupera le
azioni aggiuntive spese all’inizio del proprio round.
Attacco di Ala (Costa 2 Azioni). Il drago batte le ali. Ogni
creatura entro 4,5 metri dal drago deve riuscire un tiro salvezza
di Destrezza DC 25 o subire 17 (2d6 + 10) danni da botta e
venir gettato prono. Il drago può poi volare fino a metà della sua
velocità di volo.
Attacco di Coda. Il drago effettua un attacco di coda.
Individuare. Il drago effettua una prova di Saggezza
(Percezione).
Drago Rosso Adulto
Enorme drago, caotico malvagio
FORZA 27 (+8)
DESTREZZA 10 (+0)
COSTITUZIONE 25 (+7)
INTELLIGENZA 16 (+3)
SAGGEZZA 13 (+1)
Carisma 21 (+5)
Classe Armatura 19 (armatura naturale)
\hspace*{0pt}\hfill{Punti Ferita}: 256 (19d12 + 133)
Velocità 12 m, scalata 12 m, volo 24 m
Tiri Salvezza Destrezza +6, Costituzione +13, Saggezza +7,
Carisma +11
Abilità Furtività +6, Percezione +13
Immunità al Danno fuoco
Sensi scurovisione 36 m, vista cieca 18 m, Percezione passiva 23
Linguaggi Comune, Draconico
Sfida 17 (18.000 PE)
Resistenza Leggendaria (3/Giorno). Se il drago fallisce un tiro
salvezza, può scegliere invece di riuscire.
Azioni
Multiattacco. Il drago può usare la sua Presenza Spaventosa e
poi effettuare tre attacchi: uno con il morso e due con gli artigli.
Artiglio. Attacco con arma da mischia: +14 a colpire, portata 1,5
m, un bersaglio.
Colpisce: 15 (2d6 + 8) danni taglienti.
Coda. Attacco con arma da mischia: +14 a colpire, portata 4,5
m, un bersaglio.
Colpisce: 17 (2d8 + 8) danni da botta.
Morso. Attacco con arma da mischia: +14 a colpire, portata 3 m,
un bersaglio.
Colpisce: 19 (2d10 + 8) danni perforanti più 7 (2d6) danni da
fuoco.
Presenza Spaventosa. Ogni creatura scelta dal drago, che si trovi
entro 36 metri da esso e consapevole della sua presenza, deve
riuscire un tiro salvezza di Saggezza DC 19 o restare spaventata
per 1 minuto. Una creatura può ripetere il tiro salvezza al termine
di ciascun suo turno, terminando l’effetto se lo riesce. Se il tiro
salvezza della creatura ha successo o l’effetto ha termine per
essa, la creatura è immune alla Presenza Spaventosa del drago
per le successive 24 ore.
Soffio Infuocato (Ricarica 5-6). Il drago esala fuoco in un cono
di 18 metri. Ogni creatura in quell’area deve effettuare un tiro
salvezza di Destrezza DC 21 e subire 63 (18d6) danni da fuoco
se fallisce il tiro salvezza, o la metà di questi danni se lo riesce.
Azioni Leggendarie
Il drago può effettuare 3 azioni aggiuntive, scelte tra le opzioni
seguenti. Può usare solo un’opzione leggendaria alla volta e solo
al termine del turno di un’altra creatura. Il drago recupera le
azioni aggiuntive spese all’inizio del proprio round.
Attacco di Ala (Costa 2 Azioni). Il drago batte le ali. Ogni
creatura entro 3 metri dal drago deve riuscire un tiro salvezza di
Destrezza DC 22 o subire 15 (2d6 + 8) danni da botta e venir
gettato prono. Il drago può poi volare fino a metà della sua
velocità di volo.
Attacco di Coda. Il drago effettua un attacco di coda.
Individuare. Il drago effettua una prova di Saggezza
(Percezione).
 
Drago Rosso Giovane
Grande drago, caotico malvagio
FORZA 23 (+6)
DESTREZZA 10 (+0)
COSTITUZIONE 21 (+5)
INTELLIGENZA 14 (+2)
SAGGEZZA 11 (+0)
Carisma 19 (+4)
Classe Armatura 18 (armatura naturale)
\hspace*{0pt}\hfill{Punti Ferita}: 178 (17d10 + 85)
Velocità 12 m, scalata 12 m, volo 24 m
Tiri Salvezza Destrezza +4, Costituzione +9, Saggezza +4,
Carisma +8
Abilità Furtività +4, Percezione +8
Immunità al Danno fuoco
Sensi scurovisione 36 m, vista cieca 9 m, Percezione passiva 18
Linguaggi Comune, Draconico
Sfida 10 (5.900 PE)
Azioni
Multiattacco. Il drago può effettuare tre attacchi: uno con il
morso e due con gli artigli.
Artiglio. Attacco con arma da mischia: +10 a colpire, portata 1,5
m, un bersaglio.
Colpisce: 13 (2d6 + 6) danni taglienti.
Morso. Attacco con arma da mischia: +10 a colpire, portata 3 m,
un bersaglio.
Colpisce: 17 (2d10 + 6) danni perforanti più 3 (1d6) danni da
fuoco.
Soffio Infuocato (Ricarica 5-6). Il drago esala fuoco in un cono
di 9 metri. Ogni creatura in quell’area deve effettuare un tiro
salvezza di Destrezza DC 17 e subire 56 (16d6) danni da fuoco
se fallisce il tiro salvezza, o la metà di questi danni se lo riesce.
Drago Rosso Cucciolo
Media drago, caotico malvagio
FORZA 19 (+4)
DESTREZZA 10 (+0)
COSTITUZIONE 17 (+3)
INTELLIGENZA 12 (+1)
SAGGEZZA 11 (+0)
Carisma 15 (+2)
Classe Armatura 17 (armatura naturale)
\hspace*{0pt}\hfill{Punti Ferita}: 75 (10d8 + 30)
Velocità 9 m, scalata 9 m, volo 18 m
Tiri Salvezza Destrezza +2, Costituzione +5, Saggezza +2,
Carisma +4
Abilità Furtività +2, Percezione +4
Immunità al Danno fuoco
Sensi scurovisione 18 m, vista cieca 3 m, Percezione passiva 14
Linguaggi Draconico
Sfida 4 (1.100 PE)
Azioni
Morso. Attacco con arma da mischia: +6 a colpire, portata 1,5
m, un bersaglio.
Colpisce: 9 (1d10 + 4) danni perforanti più 3 (1d6) danni da
fuoco.
Soffio Infuocato (Ricarica 5-6). Il drago esala fuoco in un cono
di 4,5 metri. Ogni creatura in quell’area deve effettuare un tiro
salvezza di Destrezza DC 13 e subire 24 (7d6) danni da fuoco se
fallisce il tiro salvezza, o la metà di questi danni se lo riesce.
Drago Verde Antico
Mastodontica drago, legale malvagio
FORZA 27 (+8)
DESTREZZA 12 (+1)
COSTITUZIONE 25 (+7)
INTELLIGENZA 20 (+5)
SAGGEZZA 17 (+3)
Carisma 19 (+4)
Classe Armatura 21 (armatura naturale)
\hspace*{0pt}\hfill{Punti Ferita}: 385 (22d20 + 154)
Velocità 12 m, nuoto 12 m, volo 24 m
Tiri Salvezza Destrezza +8, Costituzione +14, Saggezza +10,
Carisma +11
Abilità Furtività +8, Inganno +11, Intuizione +10, Percezione
+17, Persuasione +11
Immunità al Danno veleno
Immunità alle Condizioni avvelenato
Sensi scurovisione 36 m, vista cieca 18 m, Percezione passiva 27
Linguaggi Comune, Draconico
Sfida 22 (41.000 PE)
Anfibio. Il drago può respirare aria e acqua.
Resistenza Leggendaria (3/Giorno). Se il drago fallisce un tiro
salvezza, può scegliere invece di riuscire.
Azioni
Multiattacco. Il drago può usare la sua Presenza Spaventosa. Poi
effettuare tre attacchi: uno con il morso e due con gli artigli.
Artiglio. Attacco con arma da mischia: +15 a colpire, portata 3
m, un bersaglio.
Colpisce: 15 (2d6 + 8) danni taglienti.
Coda. Attacco con arma da mischia: +15 a colpire, portata 6 m,
un bersaglio.
Colpisce: 17 (2d8 + 8) danni da botta.
Morso. Attacco con arma da mischia: +15 a colpire, portata 4,5
m, un bersaglio.
Colpisce: 19 (2d10 + 8) danni perforanti più 10 (3d6) danni da veleno.
Presenza Spaventosa. Ogni creatura scelta dal drago, che si trovi
entro 36 metri da esso e consapevole della sua presenza, deve
riuscire un tiro salvezza di Saggezza DC 19 o restare spaventata per
1 minuto. Una creatura può ripetere il tiro salvezza al termine di
ciascun suo turno, terminando l’effetto se lo riesce. Se il tiro salvezza
della creatura ha successo o l’effetto ha termine per essa, la creatura è
immune alla Presenza Spaventosa del drago per le successive 24 ore.
Soffio Velenoso (Ricarica 5-6). Il drago esala gas velenosi in un
cono di 27 metri. Ogni creatura in quell’area deve effettuare un tiro
salvezza di Costituzione DC 22 e subire 77 (22d6) danni da veleno se
fallisce il tiro salvezza, o la metà di questi danni se lo riesce.
Azioni Leggendarie
Il drago può effettuare 3 azioni aggiuntive, scelte tra le opzioni
seguenti. Può usare solo un’opzione leggendaria alla volta e solo
al termine del turno di un’altra creatura. Il drago recupera le
azioni aggiuntive spese all’inizio del proprio round.
Attacco di Ala (Costa 2 Azioni). Il drago batte le ali. Ogni
creatura entro 4,5 metri dal drago deve riuscire un tiro salvezza
di Destrezza DC 23 o subire 15 (2d6 + 8) danni da botta e
venire gettato prono. Il drago può poi volare fino a metà della
sua velocità di volo.
Attacco di Coda. Il drago effettua un attacco di coda.
Individuare. Il drago effettua una prova di Saggezza
(Percezione).
Drago Verde Adulto
Enorme drago, legale malvagio
FORZA 23 (+6)
DESTREZZA 12 (+1)
COSTITUZIONE 21 (+5)
INTELLIGENZA 18 (+4)
SAGGEZZA 15 (+2)
Carisma 17 (+3)
Classe Armatura 19 (armatura naturale)
\hspace*{0pt}\hfill{Punti Ferita}: 207 (18d12 + 90)
Velocità 12 m, nuoto 12 m, volo 24 m
Tiri Salvezza Destrezza +6, Costituzione +10, Saggezza +7,
Carisma +8
Abilità Furtività +6, Inganno +8, Intuizione +7, Percezione +12,
Persuasione +8
Immunità al Danno veleno
Immunità alle Condizioni avvelenato
Sensi scurovisione 36 m, vista cieca 18 m, Percezione passiva 22
Linguaggi Comune, Draconico
Sfida 15 (13.000 PE)
Anfibio. Il drago può respirare aria e acqua.
Resistenza Leggendaria (3/Giorno). Se il drago fallisce un tiro
salvezza, può scegliere invece di riuscire.
Azioni
Multiattacco. Il drago può usare la sua Presenza Spaventosa. Poi
effettuare tre attacchi: uno con il morso e due con gli artigli.
Artiglio. Attacco con arma da mischia: +11 a colpire, portata 1,5
m, un bersaglio.
Colpisce: 13 (2d6 + 6) danni taglienti.
Coda. Attacco con arma da mischia: +11 a colpire, portata 4,5
m, un bersaglio.
Colpisce: 15 (2d8 + 6) danni da botta.
Morso. Attacco con arma da mischia: +11 a colpire, portata 3 m,
un bersaglio.
Colpisce: 17 (2d10 + 6) danni perforanti più 7 (2d6) danni da veleno.
Presenza Spaventosa. Ogni creatura scelta dal drago, che si trovi
entro 36 metri da esso e consapevole della sua presenza, deve
riuscire un tiro salvezza di Saggezza DC 16 o restare spaventata per
1 minuto. Una creatura può ripetere il tiro salvezza al termine di
ciascun suo turno, terminando l’effetto se lo riesce. Se il tiro salvezza
della creatura ha successo o l’effetto ha termine per essa, la creatura è
immune alla Presenza Spaventosa del drago per le successive 24 ore.
Soffio Velenoso (Ricarica 5-6). Il drago esala gas velenosi in un
cono di 18 metri. Ogni creatura in quell’area deve effettuare un tiro
salvezza di Costituzione DC 18 e subire 56 (16d6) danni da veleno se
fallisce il tiro salvezza, o la metà di questi danni se lo riesce.
Azioni Leggendarie
Il drago può effettuare 3 azioni aggiuntive, scelte tra le opzioni
seguenti. Può usare solo un’opzione leggendaria alla volta e solo
al termine del turno di un’altra creatura. Il drago recupera le
azioni aggiuntive spese all’inizio del proprio round.
Attacco di Ala (Costa 2 Azioni). Il drago batte le ali. Ogni
creatura entro 3 metri dal drago deve riuscire un tiro salvezza di
Destrezza DC 19 o subire 13 (2d6 + 6) danni da botta e venir
gettato prono. Il drago può poi volare fino a metà della sua
velocità di volo.
Attacco di Coda. Il drago effettua un attacco di coda.
Individuare. Il drago effettua una prova di Saggezza
(Percezione).
Drago Verde Giovane
Grande drago, legale malvagio
FORZA 19 (+4)
DESTREZZA 12 (+1)
COSTITUZIONE 17 (+3)
INTELLIGENZA 16 (+3)
SAGGEZZA 13 (+1)
Carisma 15 (+2)
Classe Armatura 18 (armatura naturale)
\hspace*{0pt}\hfill{Punti Ferita}: 136 (16d10 + 48)
Velocità 12 m, nuoto 12 m, volo 24 m
Tiri Salvezza Destrezza +4, Costituzione +6, Saggezza +4, Carisma +5
Abilità Furtività +4, Inganno +5, Percezione +7
Immunità al Danno veleno
Immunità alle Condizioni avvelenato
Sensi scurovisione 36 m, vista cieca 9 m, Percezione passiva 17
Linguaggi Comune, Draconico
Sfida 8 (3.900 PE)
Anfibio. Il drago può respirare aria e acqua.
Azioni
Multiattacco. Il drago può effettuare tre attacchi: uno con il
morso e due con gli artigli.
Artiglio. Attacco con arma da mischia: +7 a colpire, portata 1,5
m, un bersaglio.
Colpisce: 11 (2d6 + 4) danni taglienti.
Morso. Attacco con arma da mischia: +7 a colpire, portata 3 m,
un bersaglio.
Colpisce: 15 (2d10 + 4) danni perforanti più 7 (2d6) danni da veleno.
Soffio Velenoso (Ricarica 5-6). Il drago esala gas velenosi in un
cono di 9 metri. Ogni creatura in quell’area deve effettuare un tiro
salvezza di Costituzione DC 14 e subire 42 (12d6) danni da veleno se
fallisce il tiro salvezza, o la metà di questi danni se lo riesce.
 
Drago Verde Cucciolo
Media drago, legale malvagio
FORZA 15 (+2)
DESTREZZA 12 (+1)
COSTITUZIONE 13 (+1)
INTELLIGENZA 14 (+2)
SAGGEZZA 11 (+0)
Carisma 13 (+1)
Classe Armatura 17 (armatura naturale)
\hspace*{0pt}\hfill{Punti Ferita}: 38 (7d8 + 7)
Velocità 9 m, nuoto 9 m, volo 18 m
Tiri Salvezza Destrezza +3, Costituzione +3, Saggezza +2,
Carisma +4
Abilità Furtività +3, Percezione +4
Immunità al Danno veleno
Immunità alle Condizioni avvelenato
Sensi scurovisione 18 m, vista cieca 3 m, Percezione passiva 14
Linguaggi Draconico
Sfida 2 (450 PE)
Anfibio. Il drago può respirare aria e acqua.
Azioni
Morso. Attacco con arma da mischia: +4 a colpire, portata 1,5
m, un bersaglio.
Colpisce: 7 (1d10 + 2) danni perforanti più 3 (1d6) danni da veleno.
Soffio Velenoso (Ricarica 5-6). Il drago esala gas velenosi in un
cono di 4,5 metri. Ogni creatura in quell’area deve effettuare un tiro
salvezza di Costituzione DC 11 e subire 21 (6d6) danni da veleno se
fallisce il tiro salvezza, o la metà di questi danni se lo riesce.
Draghi Metallici
Drago d’Argento Antico
Mastodontica drago, legale buono
FORZA 30 (+10)
DESTREZZA 10 (+0)
COSTITUZIONE 29 (+9)
INTELLIGENZA 18 (+4)
SAGGEZZA 15 (+2)
Carisma 23 (+6)
Classe Armatura 22 (armatura naturale)
\hspace*{0pt}\hfill{Punti Ferita}: 487 (25d20 + 225)
Velocità 12 m, volo 24 m
Tiri Salvezza Destrezza +7, Costituzione +16, Saggezza +9,
Carisma +13
Abilità Arcano +11, Furtività +7, Percezione +16, Storia +11
Immunità al Danno freddo
Sensi scurovisione 36 m, vista cieca 18 m, Percezione passiva 26
Linguaggi Comune, Draconico
Sfida 23 (50.000 PE)
Resistenza Leggendaria (3/Giorno). Se il drago fallisce un tiro
salvezza, può scegliere invece di riuscire.
Azioni
Multiattacco. Il drago può usare la sua Presenza Spaventosa. Poi
effettuare tre attacchi: uno con il morso e due con gli artigli.
Artiglio. Attacco con arma da mischia: +17 a colpire, portata 3
m, un bersaglio.
Colpisce: 17 (2d6 + 10) danni taglienti.
Coda. Attacco con arma da mischia: +17 a colpire, portata 6 m,
un bersaglio.
Colpisce: 19 (2d8 + 10) danni da botta.
Morso. Attacco con arma da mischia: +17 a colpire, portata 4,5
m, un bersaglio.
Colpisce: 21 (2d10 + 10) danni perforanti.
Presenza Spaventosa. Ogni creatura scelta dal drago, che si trovi
entro 36 metri da esso e consapevole della sua presenza, deve
riuscire un tiro salvezza di Saggezza DC 21 o restare spaventata per
1 minuto. Una creatura può ripetere il tiro salvezza al termine di
ciascun suo turno, terminando l’effetto se lo riesce. Se il tiro salvezza
della creatura ha successo o l’effetto ha termine per essa, la creatura è
immune alla Presenza Spaventosa del drago per le successive 24 ore.
Arma a Soffio (Ricarica 5-6). Il drago usa una delle seguenti armi
a soffio:
Soffio Gelido. Il drago esala un’esplosione ghiacciata in un cono di
27 metri. Ogni creatura nell’area deve effettuare un tiro salvezza di
Costituzione DC 24, subendo 67 (15d8) danni da freddo se fallisce il
tiro salvezza, o la metà di questi danni se lo riesce.
Soffio Paralizzante. Il drago esala un gas paralizzante in un cono di
24 metri. Ogni creatura nell’area deve riuscire un tiro salvezza di
Costituzione DC 24 o restare paralizzata per 1 minuto. Una creatura
può ripetere il tiro salvezza al termine di ciascun suo turno,
terminando l’effetto per sé in caso di successo.
Mutare Forma. Il drago può trasformarsi magicamente in un
umanoide o bestia il cui grado di sfida sia pari o inferiore al proprio,
o tornare alla sua vera forma. Alla morte ritorna alla sua vera forma.
Qualsiasi equipaggiamento stia indossando o trasportando viene
assorbito o trasportato nella nuova forma (a scelta del drago).
Nella nuova forma, il drago mantiene il suo allineamento, punti
ferita, Dadi Vita, la facoltà di parlare, le competenze, la Resistenza
Leggendaria, le azioni da tana, e i punteggi di Intelligenza, Saggezza
e Carisma, oltre a questa azione. Le sue statistiche e capacità
vengono altrimenti rimpiazzate da quelle della nuova forma, eccetto i
privilegi di classe o azioni aggiuntive della nuova forma.
Azioni Leggendarie
Il drago può effettuare 3 azioni aggiuntive, scelte tra le opzioni
seguenti. Può usare solo un’opzione leggendaria alla volta e solo
al termine del turno di un’altra creatura. Il drago recupera le
azioni aggiuntive spese all’inizio del proprio round.
Attacco di Ala (Costa 2 Azioni). Il drago batte le ali. Ogni
creatura entro 4,5 metri dal drago deve riuscire un tiro salvezza
di Destrezza DC 25 o subire 17 (2d6 + 10) danni da botta e
venir gettato prono. Il drago può poi volare fino a metà della sua
velocità di volo.
Attacco di Coda. Il drago effettua un attacco di coda.
Individuare. Il drago effettua una prova di Saggezza
(Percezione).
 
Drago d’Argento Adulto
Enorme drago, legale buono
FORZA 27 (+8)
DESTREZZA 10 (+0)
COSTITUZIONE 25 (+7)
INTELLIGENZA 16 (+3)
SAGGEZZA 13 (+1)
Carisma 21 (+5)
Classe Armatura 19 (armatura naturale)
\hspace*{0pt}\hfill{Punti Ferita}: 243 (18d12 + 126)
Velocità 12 m, volo 24 m
Tiri Salvezza Destrezza +5, Costituzione +12, Saggezza +6,
Carisma +10
Abilità Arcano +8, Furtività +5, Percezione +11, Storia +8
Immunità al Danno freddo
Sensi scurovisione 36 m, vista cieca 18 m, Percezione passiva 21
Linguaggi Comune, Draconico
Sfida 16 (15.000 PE)
Resistenza Leggendaria (3/Giorno). Se il drago fallisce un tiro
salvezza, può scegliere invece di riuscire.
Azioni
Multiattacco. Il drago può usare la sua Presenza Spaventosa. Poi
effettuare tre attacchi: uno con il morso e due con gli artigli.
Artiglio. Attacco con arma da mischia: +13 a colpire, portata 1,5
m, un bersaglio.
Colpisce: 15 (2d6 + 8) danni taglienti.
Coda. Attacco con arma da mischia: +13 a colpire, portata 4,5
m, un bersaglio.
Colpisce: 17 (2d8 + 8) danni da botta.
Morso. Attacco con arma da mischia: +13 a colpire, portata 3 m,
un bersaglio.
Colpisce: 19 (2d10 + 8) danni perforanti.
Presenza Spaventosa. Ogni creatura scelta dal drago, che si trovi
entro 36 metri da esso e consapevole della sua presenza, deve
riuscire un tiro salvezza di Saggezza DC 18 o restare spaventata per
1 minuto. Una creatura può ripetere il tiro salvezza al termine di
ciascun suo turno, terminando l’effetto se lo riesce. Se il tiro salvezza
della creatura ha successo o l’effetto ha termine per essa, la creatura è
immune alla Presenza Spaventosa del drago per le successive 24 ore.
Arma a Soffio (Ricarica 5-6). Il drago usa una delle seguenti armi
a soffio:
Soffio Gelido. Il drago esala un’esplosione ghiacciata in un cono di
18 metri. Ogni creatura nell’area deve effettuare un tiro salvezza di
Costituzione DC 20, subendo 58 (13d8) danni da freddo se fallisce il
tiro salvezza, o la metà di questi danni se lo riesce.
Soffio Paralizzante. Il drago esala un gas paralizzante in un cono di
18 metri. Ogni creatura nell’area deve riuscire un tiro salvezza di
Costituzione DC 20 o restare paralizzata per 1 minuto. Una creatura
può ripetere il tiro salvezza al termine di ciascun suo turno,
terminando l’effetto per sé in caso di successo.
Mutare Forma. Il drago può trasformarsi magicamente in un
umanoide o bestia il cui grado di sfida sia pari o inferiore al proprio,
o tornare alla sua vera forma. Alla morte ritorna alla sua vera forma.
Qualsiasi equipaggiamento stia indossando o trasportando viene
assorbito o trasportato nella nuova forma (a scelta del drago).
Nella nuova forma, il drago mantiene il suo allineamento, punti
ferita, Dadi Vita, la facoltà di parlare, le competenze, la Resistenza
Leggendaria, le azioni da tana, e i punteggi di Intelligenza, Saggezza
e Carisma, oltre a questa azione. Le sue statistiche e capacità
vengono altrimenti rimpiazzate da quelle della nuova forma, eccetto i
privilegi di classe o azioni aggiuntive della nuova forma.
Azioni Leggendarie
Il drago può effettuare 3 azioni aggiuntive, scelte tra le opzioni
seguenti. Può usare solo un’opzione leggendaria alla volta e solo
al termine del turno di un’altra creatura. Il drago recupera le
azioni aggiuntive spese all’inizio del proprio round.
Attacco di Ala (Costa 2 Azioni). Il drago batte le ali. Ogni
creatura entro 3 metri dal drago deve riuscire un tiro salvezza di
Destrezza DC 21 o subire 15 (2d6 + 8) danni da botta e venir
gettato prono. Il drago può poi volare fino a metà della sua
velocità di volo.
Attacco di Coda. Il drago effettua un attacco di coda.
Individuare. Il drago effettua una prova di Saggezza
(Percezione).
Drago d’Argento Giovane
Grande drago, legale buono
FORZA 23 (+6)
DESTREZZA 10 (+0)
COSTITUZIONE 21 (+5)
INTELLIGENZA 14 (+2)
SAGGEZZA 11 (+0)
Carisma 19 (+4)
Classe Armatura 18 (armatura naturale)
\hspace*{0pt}\hfill{Punti Ferita}: 168 (16d10 + 80)
Velocità 12 m, volo 24 m
Tiri Salvezza Destrezza +4, Costituzione +9, Saggezza +4,
Carisma +8
Abilità Arcano +6, Furtività +4, Percezione +8, Storia +6
Immunità al Danno freddo
Sensi scurovisione 36 m, vista cieca 9 m, Percezione passiva 18
Linguaggi Comune, Draconico
Sfida 9 (5.000 PE)
Azioni
Multiattacco. Il drago può effettuare tre attacchi: uno con il
morso e due con gli artigli.
Artiglio. Attacco con arma da mischia: +10 a colpire, portata 1,5
m, un bersaglio.
Colpisce: 13 (2d6 + 6) danni taglienti.
Morso. Attacco con arma da mischia: +10 a colpire, portata 3 m,
un bersaglio.
Colpisce: 17 (2d10 + 6) danni perforanti.
Arma a Soffio (Ricarica 5-6). Il drago usa una delle seguenti armi
a soffio:
Soffio Gelido. Il drago esala un’esplosione ghiacciata in un cono di 9
metri. Ogni creatura nell’area deve effettuare un tiro salvezza di
Costituzione DC 17, subendo 54 (12d8) danni da freddo se fallisce il
tiro salvezza, o la metà di questi danni se lo riesce.
Soffio Paralizzante. Il drago esala un gas paralizzante in un cono di 9
metri. Ogni creatura nell’area deve riuscire un tiro salvezza di
Costituzione DC 17 o restare paralizzata per 1 minuto. Una creatura
può ripetere il tiro salvezza al termine di ciascun suo turno,
terminando l’effetto per sé in caso di successo.
Drago d’Argento Cucciolo
Media drago, legale buono
FORZA 19 (+4)
DESTREZZA 10 (+0)
COSTITUZIONE 17 (+3)
INTELLIGENZA 12 (+1)
SAGGEZZA 11 (+0)
Carisma 15 (+2)
Classe Armatura 17 (armatura naturale)
\hspace*{0pt}\hfill{Punti Ferita}: 45 (6d8 + 18)
Velocità 9 m, volo 18 m
Tiri Salvezza Destrezza +2, Costituzione +5, Saggezza +2,
Carisma +5
Abilità Furtività +2, Percezione +4
Immunità al Danno freddo
Sensi scurovisione 18 m, vista cieca 3 m, Percezione passiva 14
Linguaggi Draconico
Sfida 2 (450 PE)
Azioni
Morso. Attacco con arma da mischia: +6 a colpire, portata 1,5
m, un bersaglio.
Colpisce: 9 (1d10 + 4) danni perforanti.
Arma a Soffio (Ricarica 5-6). Il drago usa una delle seguenti armi
a soffio:
Soffio Gelido. Il drago esala un’esplosione ghiacciata in un cono di
4,5 metri. Ogni creatura nell’area deve effettuare un tiro salvezza di
Costituzione DC 13, subendo 18 (4d8) danni da freddo se fallisce il
tiro salvezza, o la metà di questi danni se lo riesce.
Soffio Paralizzante. Il drago esala un gas paralizzante in un cono di
4,5 metri. Ogni creatura nell’area deve riuscire un tiro salvezza di
Costituzione DC 13 o restare paralizzata per 1 minuto. Una creatura
può ripetere il tiro salvezza al termine di ciascun suo turno,
terminando l’effetto per sé in caso di successo.
 
Drago di Bronzo Antico
Mastodontica drago, caotico buono
FORZA 29 (+9)
DESTREZZA 10 (+0)
COSTITUZIONE 27 (+8)
INTELLIGENZA 18 (+4)
SAGGEZZA 17 (+3)
Carisma 21 (+5)
Classe Armatura 22 (armatura naturale)
\hspace*{0pt}\hfill{Punti Ferita}: 444 (24d20 + 192)
Velocità 12 m, nuoto 12 m, volo 24 m
Tiri Salvezza Destrezza +7, Costituzione +15, Saggezza +10,
Carisma +12
Abilità Furtività +7, Intuizione +10, Percezione +17
Immunità al Danno fulmine
Sensi scurovisione 36 m, vista cieca 18 m, Percezione passiva 27
Linguaggi Comune, Draconico
Sfida 22 (41.000 PE)
Anfibio. Il drago può respirare aria e acqua.
Resistenza Leggendaria (3/Giorno). Se il drago fallisce un tiro
salvezza, può scegliere invece di riuscire.
Azioni
Multiattacco. Il drago può usare la sua Presenza Spaventosa. Poi
effettuare tre attacchi: uno con il morso e due con gli artigli.
Artiglio. Attacco con arma da mischia: +16 a colpire, portata 3
m, un bersaglio.
Colpisce: 16 (2d6 + 9) danni taglienti.
Coda. Attacco con arma da mischia: +16 a colpire, portata 6 m,
un bersaglio.
Colpisce: 18 (2d8 + 9) danni da botta.
Morso. Attacco con arma da mischia: +16 a colpire, portata 4,5
m, un bersaglio.
Colpisce: 20 (2d10 + 9) danni perforanti.
Presenza Spaventosa. Ogni creatura scelta dal drago, che si trovi
entro 36 metri da esso e consapevole della sua presenza, deve
riuscire un tiro salvezza di Saggezza DC 20 o restare spaventata per
1 minuto. Una creatura può ripetere il tiro salvezza al termine di
ciascun suo turno, terminando l’effetto se lo riesce. Se il tiro salvezza
della creatura ha successo o l’effetto ha termine per essa, la creatura è
immune alla Presenza Spaventosa del drago per le successive 24 ore.
Arma a Soffio (Ricarica 5-6). Il drago usa una delle seguenti armi
a soffio:
Soffio Fulminante. Il drago esala fulmini in una linea lunga 36 metri
e larga 3 metri. Ogni creatura sulla linea deve effettuare un tiro
salvezza di Destrezza DC 23, subendo 88 (16d10) danni da fulmine
se fallisce il tiro salvezza, o la metà di questi danni se lo riesce.
Soffio Repulsivo. Il drago esala dell’energia repulsiva in un cono di 9
metri. Ogni creatura in quell’area deve riuscire un tiro salvezza di
Forza DC 23, altrimenti viene allontana di 18 metri dal drago.
Mutare Forma. Il drago può trasformarsi magicamente in un
umanoide o bestia il cui grado di sfida sia pari o inferiore al proprio,
o tornare alla sua vera forma. Alla morte ritorna alla sua vera forma.
Qualsiasi equipaggiamento stia indossando o trasportando viene
assorbito o trasportato nella nuova forma (a scelta del drago).
Nella nuova forma, il drago mantiene il suo allineamento, punti
ferita, Dadi Vita, la facoltà di parlare, le competenze, la Resistenza
Leggendaria, le azioni da tana, e i punteggi di Intelligenza, Saggezza
e Carisma, oltre a questa azione. Le sue statistiche e capacità
vengono altrimenti rimpiazzate da quelle della nuova forma, eccetto i
privilegi di classe o azioni aggiuntive della nuova forma.
Azioni Leggendarie
Il drago può effettuare 3 azioni aggiuntive, scelte tra le opzioni
seguenti. Può usare solo un’opzione leggendaria alla volta e solo
al termine del turno di un’altra creatura. Il drago recupera le
azioni aggiuntive spese all’inizio del proprio round.
Attacco di Ala (Costa 2 Azioni). Il drago batte le ali. Ogni
creatura entro 4,5 metri dal drago deve riuscire un tiro salvezza
di Destrezza DC 24 o subire 16 (2d6 + 9) danni da botta e
venir gettato prono. Il drago può poi volare fino a metà della sua
velocità di volo.
Attacco di Coda. Il drago effettua un attacco di coda.
Individuare. Il drago effettua una prova di Saggezza
(Percezione).
Drago di Bronzo Adulto
Enorme drago, caotico buono
FORZA 25 (+7)
DESTREZZA 10 (+0)
COSTITUZIONE 23 (+6)
INTELLIGENZA 16 (+3)
SAGGEZZA 15 (+2)
Carisma 19 (+4)
Classe Armatura 19 (armatura naturale)
\hspace*{0pt}\hfill{Punti Ferita}: 212 (17d12 + 102)
Velocità 12 m, nuoto 12 m, volo 24 m
Tiri Salvezza Destrezza +5, Costituzione +11, Saggezza +7,
Carisma +9
Abilità Furtività +5, Intuizione +7, Percezione +12
Immunità al Danno fulmine
Sensi scurovisione 36 m, vista cieca 18 m, Percezione passiva 22
Linguaggi Comune, Draconico
Sfida 15 (13.000 PE)
Anfibio. Il drago può respirare aria e acqua.
Resistenza Leggendaria (3/Giorno). Se il drago fallisce un tiro
salvezza, può scegliere invece di riuscire.
Azioni
Multiattacco. Il drago può usare la sua Presenza Spaventosa e
poi effettuare tre attacchi: uno con il morso e due con gli artigli.
Artiglio. Attacco con arma da mischia: +12 a colpire, portata 1,5
m, un bersaglio.
Colpisce: 14 (2d6 + 7) danni taglienti.
Coda. Attacco con arma da mischia: +12 a colpire, portata 4,5
m, un bersaglio.
Colpisce: 16 (2d8 + 7) danni da botta.
Morso. Attacco con arma da mischia: +12 a colpire, portata 3 m,
un bersaglio.
Colpisce: 18 (2d10 + 7) danni perforanti.
Presenza Spaventosa. Ogni creatura scelta dal drago, che si trovi
entro 36 metri da esso e consapevole della sua presenza, deve
riuscire un tiro salvezza di Saggezza DC 17 o restare spaventata per
1 minuto. Una creatura può ripetere il tiro salvezza al termine di
ciascun suo turno, terminando l’effetto se lo riesce. Se il tiro salvezza
della creatura ha successo o l’effetto ha termine per essa, la creatura è
immune alla Presenza Spaventosa del drago per le successive 24 ore.
Arma a Soffio (Ricarica 5-6). Il drago usa una delle seguenti armi
a soffio:
Soffio Fulminante. Il drago esala fulmini in una linea lunga 27 metri
e larga 1,5 metri. Ogni creatura sulla linea deve effettuare un tiro
salvezza di Destrezza DC 19, subendo 66 (12d10) danni da fulmine
se fallisce il tiro salvezza, o la metà di questi danni se lo riesce.
Soffio Repulsivo. Il drago esala dell’energia repulsiva in un cono di 9
metri. Ogni creatura in quell’area deve riuscire un tiro salvezza di
Forza DC 19, altrimenti viene allontana di 18 metri dal drago.
Mutare Forma. Il drago può trasformarsi magicamente in un
umanoide o bestia il cui grado di sfida sia pari o inferiore al proprio,
o tornare alla sua vera forma. Alla morte ritorna alla sua vera forma.
Qualsiasi equipaggiamento stia indossando o trasportando viene
assorbito o trasportato nella nuova forma (a scelta del drago).
Nella nuova forma, il drago mantiene il suo allineamento, punti
ferita, Dadi Vita, la facoltà di parlare, le competenze, la Resistenza
Leggendaria, le azioni da tana, e i punteggi di Intelligenza, Saggezza
e Carisma, oltre a questa azione. Le sue statistiche e capacità
vengono altrimenti rimpiazzate da quelle della nuova forma, eccetto i
privilegi di classe o azioni aggiuntive della nuova forma.
Azioni Leggendarie
Il drago può effettuare 3 azioni aggiuntive, scelte tra le opzioni
seguenti. Può usare solo un’opzione leggendaria alla volta e solo
al termine del turno di un’altra creatura. Il drago recupera le
azioni aggiuntive spese all’inizio del proprio round.
Attacco di Ala (Costa 2 Azioni). Il drago batte le ali. Ogni
creatura entro 3 metri dal drago deve riuscire un tiro salvezza di
Destrezza DC 20 o subire 14 (2d6 + 7) danni da botta e venir
gettato prono. Il drago può poi volare fino a metà della sua
velocità di volo.
Attacco di Coda. Il drago effettua un attacco di coda.
Individuare. Il drago effettua una prova di Saggezza
(Percezione).
 
Drago di Bronzo Giovane
Grande drago, caotico buono
FORZA 21 (+5)
DESTREZZA 10 (+0)
COSTITUZIONE 19 (+4)
INTELLIGENZA 14 (+2)
SAGGEZZA 13 (+1)
Carisma 17 (+3)
Classe Armatura 18 (armatura naturale)
\hspace*{0pt}\hfill{Punti Ferita}: 142 (15d10 + 60)
Velocità 12 m, nuoto 12 m, volo 24 m
Tiri Salvezza Destrezza +3, Costituzione +7, Saggezza +4,
Carisma +6
Abilità Furtività +3, Intuizione +4, Percezione +7
Immunità al Danno fulmine
Sensi scurovisione 36 m, vista cieca 9 m, Percezione passiva 17
Linguaggi Comune, Draconico
Sfida 8 (3.900 PE)
Anfibio. Il drago può respirare aria e acqua.
Azioni
Multiattacco. Il drago può usare effettuare tre attacchi: uno con
il morso e due con gli artigli.
Artiglio. Attacco con arma da mischia: +8 a colpire, portata 1,5
m, un bersaglio.
Colpisce: 12 (2d6 + 5) danni taglienti.
Morso. Attacco con arma da mischia: +8 a colpire, portata 3 m,
un bersaglio.
Colpisce: 16 (2d10 + 5) danni perforanti.
Arma a Soffio (Ricarica 5-6). Il drago usa una delle seguenti armi
a soffio:
Soffio Fulminante. Il drago esala fulmini in una linea lunga 18 metri
e larga 1,5 metri. Ogni creatura sulla linea deve effettuare un tiro
salvezza di Destrezza DC 15, subendo 55 (10d10) danni da fulmine
se fallisce il tiro salvezza, o la metà di questi danni se lo riesce.
Soffio Repulsivo. Il drago esala dell’energia repulsiva in un cono di 9
metri. Ogni creatura in quell’area deve riuscire un tiro salvezza di
Forza DC 15, altrimenti viene allontana di 12 metri dal drago.
Drago di Bronzo Cucciolo
Media drago, caotico buono
FORZA 17 (+3)
DESTREZZA 10 (+0)
COSTITUZIONE 15 (+2)
INTELLIGENZA 12 (+1)
SAGGEZZA 11 (+0)
Carisma 15 (+2)
Classe Armatura 17 (armatura naturale)
\hspace*{0pt}\hfill{Punti Ferita}: 32 (5d8 + 10)
Velocità 9 m, nuoto 9 m, volo 18 m
Tiri Salvezza Destrezza +2, Costituzione +4, Saggezza +2,
Carisma +4
Abilità Furtività +2, Percezione +4
Immunità al Danno fulmine
Sensi scurovisione 18 m, vista cieca 3 m, Percezione passiva 14
Linguaggi Draconico
Sfida 2 (450 PE)
Anfibio. Il drago può respirare aria e acqua.
Azioni
Morso. Attacco con arma da mischia: +5 a colpire, portata 1,5
m, un bersaglio.
Colpisce: 8 (1d10 + 3) danni perforanti.
Arma a Soffio (Ricarica 5-6). Il drago usa una delle seguenti armi
a soffio:
Soffio Fulminante. Il drago esala fulmini in una linea lunga 12 metri
e larga 1,5 metri. Ogni creatura sulla linea deve effettuare un tiro
salvezza di Destrezza DC 12, subendo 16 (3d10) danni da fulmine se
fallisce il tiro salvezza, o la metà di questi danni se lo riesce.
Soffio Repulsivo. Il drago esala dell’energia repulsiva in un cono di 9
metri. Ogni creatura in quell’area deve riuscire un tiro salvezza di
Forza DC 12, altrimenti viene allontana di 9 metri dal drago.
Drago d’Oro Antico
Mastodontica drago, legale buono
FORZA 30 (+10)
DESTREZZA 14 (+2)
COSTITUZIONE 29 (+9)
INTELLIGENZA 18 (+4)
SAGGEZZA 17 (+3)
Carisma 28 (+9)
Classe Armatura 22 (armatura naturale)
\hspace*{0pt}\hfill{Punti Ferita}: 546 (28d20 + 252)
Velocità 12 m, nuoto 12 m, volo 24 m
Tiri Salvezza Destrezza +9, Costituzione +16, Saggezza +10,
Carisma +16
Abilità Furtività +9, Intuizione +10, Percezione +17, Persuasione +16
Immunità al Danno fuoco
Sensi scurovisione 36 m, vista cieca 18 m, Percezione passiva 27
Linguaggi Comune, Draconico
Sfida 24 (62.000 PE)
Anfibio. Il drago può respirare aria e acqua.
Resistenza Leggendaria (3/Giorno). Se il drago fallisce un tiro
salvezza, può scegliere invece di riuscire.
Azioni
Multiattacco. Il drago può usare la sua Presenza Spaventosa. Poi
effettuare tre attacchi: uno con il morso e due con gli artigli.
Artiglio. Attacco con arma da mischia: +17 a colpire, portata 3
m, un bersaglio.
Colpisce: 17 (2d6 + 10) danni taglienti.
Coda. Attacco con arma da mischia: +17 a colpire, portata 6 m,
un bersaglio.
Colpisce: 19 (2d8 + 10) danni da botta.
Morso. Attacco con arma da mischia: +17 a colpire, portata 4,5
m, un bersaglio.
Colpisce: 21 (2d10 + 10) danni perforanti.
Presenza Spaventosa. Ogni creatura scelta dal drago, che si trovi
entro 36 metri da esso e consapevole della sua presenza, deve
riuscire un tiro salvezza di Saggezza DC 24 o restare spaventata per
1 minuto. Una creatura può ripetere il tiro salvezza al termine di
ciascun suo turno, terminando l’effetto se lo riesce. Se il tiro salvezza
della creatura ha successo o l’effetto ha termine per essa, la creatura è
immune alla Presenza Spaventosa del drago per le successive 24 ore.
Arma a Soffio (Ricarica 5-6). Il drago usa una delle seguenti armi
a soffio:
Soffio Infuocato. Il drago esala fuoco in un cono di 27 metri. Ogni
creatura nell’area deve effettuare un tiro salvezza di Destrezza DC
24, subendo 71(13d10) danni da fuoco se fallisce il tiro salvezza, o la
metà di questi danni se lo riesce.
Soffio Indebolente. Il drago esala del gas in un cono di 27 metri. Ogni
creatura in quell’area deve riuscire un tiro salvezza di Forza DC 24 o
avere svantaggio ai tiri di attacco basati sulla Forza, prove di Forza, e
tiri salvezza di Forza per 1 minuto. Una creatura può ripetere il tiro
salvezza al termine di ciascun suo turno, terminando l’effetto su di sé
in caso di successo.
Mutare Forma. Il drago può trasformarsi magicamente in un
umanoide o bestia il cui grado di sfida sia pari o inferiore al proprio,
o tornare alla sua vera forma. Alla morte ritorna alla sua vera forma.
Qualsiasi equipaggiamento stia indossando o trasportando viene
assorbito o trasportato nella nuova forma (a scelta del drago).
Nella nuova forma, il drago mantiene il suo allineamento, punti
ferita, Dadi Vita, la facoltà di parlare, le competenze, la Resistenza
Leggendaria, le azioni da tana, e i punteggi di Intelligenza, Saggezza
e Carisma, oltre a questa azione. Le sue statistiche e capacità
vengono altrimenti rimpiazzate da quelle della nuova forma, eccetto i
privilegi di classe o azioni aggiuntive della nuova forma.
Azioni Leggendarie
Il drago può effettuare 3 azioni aggiuntive, scelte tra le opzioni
seguenti. Può usare solo un’opzione leggendaria alla volta e solo
al termine del turno di un’altra creatura. Il drago recupera le
azioni aggiuntive spese all’inizio del proprio round.
Attacco di Ala (Costa 2 Azioni). Il drago batte le ali. Ogni
creatura entro 4,5 metri dal drago deve riuscire un tiro salvezza
di Destrezza DC 25 o subire 17 (2d6 + 10) danni da botta e
venir gettato prono. Il drago può poi volare fino a metà della sua
velocità di volo.
Attacco di Coda. Il drago effettua un attacco di coda.
Individuare. Il drago effettua una prova di Saggezza
(Percezione).
 
Drago d’Oro Adulto
Enorme drago, legale buono
FORZA 27 (+8)
DESTREZZA 14 (+2)
COSTITUZIONE 25 (+7)
INTELLIGENZA 16 (+3)
SAGGEZZA 15 (+2)
Carisma 24 (+7)
Classe Armatura 19 (armatura naturale)
\hspace*{0pt}\hfill{Punti Ferita}: 256 (19d12 + 133)
Velocità 12 m, nuoto 12 m, volo 24 m
Tiri Salvezza Destrezza +8, Costituzione +13, Saggezza +8,
Carisma +13
Abilità Furtività +8, Intuizione +8, Percezione +14, Persuasione +13
Immunità al Danno fuoco
Sensi scurovisione 36 m, vista cieca 18 m, Percezione passiva 24
Linguaggi Comune, Draconico
Sfida 17 (18.000 PE)
Anfibio. Il drago può respirare aria e acqua.
Resistenza Leggendaria (3/Giorno). Se il drago fallisce un tiro
salvezza, può scegliere invece di riuscire.
Azioni
Multiattacco. Il drago può usare la sua Presenza Spaventosa. Poi
effettuare tre attacchi: uno con il morso e due con gli artigli.
Artiglio. Attacco con arma da mischia: +14 a colpire, portata 1,5
m, un bersaglio.
Colpisce: 15 (2d6 + 8) danni taglienti.
Coda. Attacco con arma da mischia: +14 a colpire, portata 4,5
m, un bersaglio.
Colpisce: 17 (2d8 + 8) danni da botta.
Morso. Attacco con arma da mischia: +14 a colpire, portata 3 m,
un bersaglio.
Colpisce: 19 (2d10 + 8) danni perforanti.
Presenza Spaventosa. Ogni creatura scelta dal drago, che si trovi
entro 36 metri da esso e consapevole della sua presenza, deve
riuscire un tiro salvezza di Saggezza DC 21 o restare spaventata per
1 minuto. Una creatura può ripetere il tiro salvezza al termine di
ciascun suo turno, terminando l’effetto se lo riesce. Se il tiro salvezza
della creatura ha successo o l’effetto ha termine per essa, la creatura è
immune alla Presenza Spaventosa del drago per le successive 24 ore.
Arma a Soffio (Ricarica 5-6). Il drago usa una delle seguenti armi
a soffio:
Soffio Infuocato. Il drago esala fuoco in un cono di 18 metri. Ogni
creatura nell’area deve effettuare un tiro salvezza di Destrezza DC
21, subendo 66 (12d10) danni da fuoco se fallisce il tiro salvezza, o
la metà di questi danni se lo riesce.
Soffio Indebolente. Il drago esala del gas in un cono di 18 metri. Ogni
creatura in quell’area deve riuscire un tiro salvezza di Forza DC 21 o
avere svantaggio ai tiri di attacco basati sulla Forza, prove di Forza, e
tiri salvezza di Forza per 1 minuto. Una creatura può ripetere il tiro
salvezza al termine di ciascun suo turno, terminando l’effetto su di sé
in caso di successo.
Mutare Forma. Il drago può trasformarsi magicamente in un
umanoide o bestia il cui grado di sfida sia pari o inferiore al proprio,
o tornare alla sua vera forma. Alla morte ritorna alla sua vera forma.
Qualsiasi equipaggiamento stia indossando o trasportando viene
assorbito o trasportato nella nuova forma (a scelta del drago).
Nella nuova forma, il drago mantiene il suo allineamento, punti
ferita, Dadi Vita, la facoltà di parlare, le competenze, la Resistenza
Leggendaria, le azioni da tana, e i punteggi di Intelligenza, Saggezza
e Carisma, oltre a questa azione. Le sue statistiche e capacità
vengono altrimenti rimpiazzate da quelle della nuova forma, eccetto i
privilegi di classe o azioni aggiuntive della nuova forma.
Azioni Leggendarie
Il drago può effettuare 3 azioni aggiuntive, scelte tra le opzioni
seguenti. Può usare solo un’opzione leggendaria alla volta e solo
al termine del turno di un’altra creatura. Il drago recupera le
azioni aggiuntive spese all’inizio del proprio round.
Attacco di Ala (Costa 2 Azioni). Il drago batte le ali. Ogni
creatura entro 3 metri dal drago deve riuscire un tiro salvezza di
Destrezza DC 22 o subire 15 (2d6 + 8) danni da botta e venir
gettato prono. Il drago può poi volare fino a metà della sua
velocità di volo.
Attacco di Coda. Il drago effettua un attacco di coda.
Individuare. Il drago effettua una prova di Saggezza
(Percezione).
Drago d’Oro Giovane
Grande drago, legale buono
FORZA 23 (+6)
DESTREZZA 14 (+2)
COSTITUZIONE 21 (+5)
INTELLIGENZA 16 (+3)
SAGGEZZA 13 (+1)
Carisma 20 (+5)
Classe Armatura 18 (armatura naturale)
\hspace*{0pt}\hfill{Punti Ferita}: 178 (17d10 + 85)
Velocità 12 m, nuoto 12 m, volo 24 m
Tiri Salvezza Destrezza +6, Costituzione +9, Saggezza +5,
Carisma +9
Abilità Furtività +6, Intuizione +5, Percezione +9, Persuasione +9
Immunità al Danno fuoco
Sensi scurovisione 36 m, vista cieca 9 m, Percezione passiva 19
Linguaggi Comune, Draconico
Sfida 10 (5.900 PE)
Anfibio. Il drago può respirare aria e acqua.
Azioni
Multiattacco. Il drago può effettuare tre attacchi: uno con il
morso e due con gli artigli.
Artiglio. Attacco con arma da mischia: +10 a colpire, portata 1,5
m, un bersaglio.
Colpisce: 13 (2d6 + 6) danni taglienti.
Morso. Attacco con arma da mischia: +10 a colpire, portata 3 m,
un bersaglio.
Colpisce: 17 (2d10 + 6) danni perforanti.
Arma a Soffio (Ricarica 5-6). Il drago usa una delle seguenti armi
a soffio:
Soffio Infuocato. Il drago esala fuoco in un cono di 9 metri. Ogni
creatura nell’area deve effettuare un tiro salvezza di Destrezza DC
17, subendo 55 (10d10) danni da fuoco se fallisce il tiro salvezza, o
la metà di questi danni se lo riesce.
Soffio Indebolente. Il drago esala del gas in un cono di 9 metri. Ogni
creatura in quell’area deve riuscire un tiro salvezza di Forza DC 17 o
avere svantaggio ai tiri di attacco basati sulla Forza, prove di Forza, e
tiri salvezza di Forza per 1 minuto. Una creatura può ripetere il tiro
salvezza al termine di ciascun suo turno, terminando l’effetto su di sé
in caso di successo.
Drago d’Oro Cucciolo
Media drago, legale buono
FORZA 19 (+4)
DESTREZZA 14 (+2)
COSTITUZIONE 17 (+3)
INTELLIGENZA 14 (+2)
SAGGEZZA 11 (+0)
Carisma 16 (+3)
Classe Armatura 17 (armatura naturale)
\hspace*{0pt}\hfill{Punti Ferita}: 60 (8d8 + 24)
Velocità 9 m, nuoto 9 m, volo 18 m
Tiri Salvezza Destrezza +4, Costituzione +5, Saggezza +2,
Carisma +5
Abilità Furtività +4, Percezione +4
Immunità al Danno fuoco
Sensi scurovisione 18 m, vista cieca 3 m, Percezione passiva 14
Linguaggi Draconico
Sfida 3 (700 PE)
Anfibio. Il drago può respirare aria e acqua.
Azioni
Morso. Attacco con arma da mischia: +6 a colpire, portata 1,5
m, un bersaglio.
Colpisce: 9 (1d10 + 4) danni perforanti.
Arma a Soffio (Ricarica 5-6). Il drago usa una delle seguenti armi
a soffio:
Soffio Infuocato. Il drago esala fuoco in un cono di 4,5 metri. Ogni
creatura nell’area deve effettuare un tiro salvezza di Destrezza DC
13, subendo 22 (4d10) danni da fuoco se fallisce il tiro salvezza, o la
metà di questi danni se lo riesce.
Soffio Indebolente. Il drago esala del gas in un cono di 4,5 metri.
Ogni creatura in quell’area deve riuscire un tiro salvezza di Forza DC
13 o avere svantaggio ai tiri di attacco basati sulla Forza, prove di
Forza, e tiri salvezza di Forza per 1 minuto. Una creatura può ripetere
il tiro salvezza al termine di ciascun suo turno, terminando l’effetto
su di sé in caso di successo.
 
Drago d’Ottone Antico
Mastodontica drago, caotico buono
FORZA 27 (+8)
DESTREZZA 10 (+0)
COSTITUZIONE 25 (+7)
INTELLIGENZA 16 (+3)
SAGGEZZA 15 (+2)
Carisma 19 (+4)
Classe Armatura 20 (armatura naturale)
\hspace*{0pt}\hfill{Punti Ferita}: 297 (17d20 + 119)
Velocità 12 m, scavo 12 m, volo 24 m
Tiri Salvezza Destrezza +6, Costituzione +13, Saggezza +8,
Carisma +10
Abilità Furtività +6, Percezione +14, Persuasione +10, Storia +9
Immunità al Danno fuoco
Sensi scurovisione 36 m, vista cieca 18 m, Percezione passiva 24
Linguaggi Comune, Draconico
Sfida 20 (25.000 PE)
Resistenza Leggendaria (3/Giorno). Se il drago fallisce un tiro
salvezza, può scegliere invece di riuscire.
Azioni
Multiattacco. Il drago può usare la sua Presenza Spaventosa. Poi
effettuare tre attacchi: uno con il morso e due con gli artigli.
Artiglio. Attacco con arma da mischia: +14 a colpire, portata 3
m, un bersaglio.
Colpisce: 15 (2d6 + 8) danni taglienti.
Coda. Attacco con arma da mischia: +14 a colpire, portata 6 m,
un bersaglio.
Colpisce: 17 (2d8 + 8) danni da botta.
Morso. Attacco con arma da mischia: +14 a colpire, portata 4,5
m, un bersaglio.
Colpisce: 19 (2d10 + 8) danni perforanti.
Presenza Spaventosa. Ogni creatura scelta dal drago, che si trovi
entro 36 metri da esso e consapevole della sua presenza, deve
riuscire un tiro salvezza di Saggezza DC 18 o restare spaventata per
1 minuto. Una creatura può ripetere il tiro salvezza al termine di
ciascun suo turno, terminando l’effetto se lo riesce. Se il tiro salvezza
della creatura ha successo o l’effetto ha termine per essa, la creatura è
immune alla Presenza Spaventosa del drago per le successive 24 ore.
Arma a Soffio (Ricarica 5-6). Il drago usa una delle seguenti armi
a soffio:
Soffio Infuocato. Il drago esala fuoco in una linea lunga 27 metri e
larga 3 metri. Ogni creatura sulla linea deve effettuare un tiro
salvezza di Destrezza DC 21, subendo 56 (16d6) danni da fuoco se
fallisce il tiro salvezza, o la metà di questi danni se lo riesce.
Soffio Soporifero. Il drago esala del gas soporifero in un cono di 27
metri. Ogni creatura in quell’area deve riuscire un tiro salvezza di
Costituzione DC 21 o cadere svenuta per 10 minuti. Questo effetto
termina se la creatura svenuta subisce danni o qualcuno impiega
un’azione per risvegliarla.
Mutare Forma. Il drago può trasformarsi magicamente in un
umanoide o bestia il cui grado di sfida sia pari o inferiore al proprio,
o tornare alla sua vera forma. Alla morte ritorna alla sua vera forma.
Qualsiasi equipaggiamento stia indossando o trasportando viene
assorbito o trasportato nella nuova forma (a scelta del drago).
Nella nuova forma, il drago mantiene il suo allineamento, punti
ferita, Dadi Vita, la facoltà di parlare, le competenze, la Resistenza
Leggendaria, le azioni da tana, e i punteggi di Intelligenza, Saggezza
e Carisma, oltre a questa azione. Le sue statistiche e capacità
vengono altrimenti rimpiazzate da quelle della nuova forma, eccetto i
privilegi di classe o azioni aggiuntive della nuova forma.
Azioni Leggendarie
Il drago può effettuare 3 azioni aggiuntive, scelte tra le opzioni
seguenti. Può usare solo un’opzione leggendaria alla volta e solo
al termine del turno di un’altra creatura. Il drago recupera le
azioni aggiuntive spese all’inizio del proprio round.
Attacco di Ala (Costa 2 Azioni). Il drago batte le ali. Ogni
creatura entro 4,5 metri dal drago deve riuscire un tiro salvezza
di Destrezza DC 22 o subire 15 (2d6 + 8) danni da botta e
venir gettato prono. Il drago può poi volare fino a metà della sua
velocità di volo.
Attacco di Coda. Il drago effettua un attacco di coda.
Individuare. Il drago effettua una prova di Saggezza
(Percezione).
Drago d’Ottone Adulto
Enorme drago, caotico buono
FORZA 23 (+6)
DESTREZZA 10 (+0)
COSTITUZIONE 21 (+5)
INTELLIGENZA 14 (+2)
SAGGEZZA 13 (+1)
Carisma 17 (+3)
Classe Armatura 18 (armatura naturale)
\hspace*{0pt}\hfill{Punti Ferita}: 172 (15d12 + 75)
Velocità 12 m, scavo 9 m, volo 24 m
Tiri Salvezza Destrezza +5, Costituzione +10, Saggezza +6,
Carisma +8
Abilità Furtività +5, Percezione +11, Persuasione +8, Storia +7
Immunità al Danno fuoco
Sensi scurovisione 36 m, vista cieca 18 m, Percezione passiva 21
Linguaggi Comune, Draconico
Sfida 13 (10.000 PE)
Resistenza Leggendaria (3/Giorno). Se il drago fallisce un tiro
salvezza, può scegliere invece di riuscire.
Azioni
Multiattacco. Il drago può usare la sua Presenza Spaventosa. Poi
effettuare tre attacchi: uno con il morso e due con gli artigli.
Artiglio. Attacco con arma da mischia: +11 a colpire, portata 1,5
m, un bersaglio.
Colpisce: 13 (2d6 + 6) danni taglienti.
Coda. Attacco con arma da mischia: +11 a colpire, portata 4,5
m, un bersaglio.
Colpisce: 15 (2d8 + 6) danni da botta.
Morso. Attacco con arma da mischia: +11 a colpire, portata 3 m,
un bersaglio.
Colpisce: 17 (2d10 + 6) danni perforanti.
Presenza Spaventosa. Ogni creatura scelta dal drago, che si trovi
entro 36 metri da esso e consapevole della sua presenza, deve
riuscire un tiro salvezza di Saggezza DC 16 o restare spaventata per
1 minuto. Una creatura può ripetere il tiro salvezza al termine di
ciascun suo turno, terminando l’effetto se lo riesce. Se il tiro salvezza
della creatura ha successo o l’effetto ha termine per essa, la creatura è
immune alla Presenza Spaventosa del drago per le successive 24 ore.
Arma a Soffio (Ricarica 5-6). Il drago usa una delle seguenti armi
a soffio:
Soffio Infuocato. Il drago esala fuoco in una linea lunga 18 metri e
larga 1,5 metri. Ogni creatura sulla linea deve effettuare un tiro
salvezza di Destrezza DC 18, subendo 45 (13d6) danni da fuoco se
fallisce il tiro salvezza, o la metà di questi danni se lo riesce.
Soffio Soporifero. Il drago esala del gas soporifero in un cono di 18
metri. Ogni creatura in quell’area deve riuscire un tiro salvezza di
Costituzione DC 18 o cadere svenuta per 10 minuti. Questo effetto
termina se la creatura svenuta subisce danni o qualcuno impiega
un’azione per risvegliarla.
Azioni Leggendarie
Il drago può effettuare 3 azioni aggiuntive, scelte tra le opzioni
seguenti. Può usare solo un’opzione leggendaria alla volta e solo
al termine del turno di un’altra creatura. Il drago recupera le
azioni aggiuntive spese all’inizio del proprio round.
Attacco di Ala (Costa 2 Azioni). Il drago batte le ali. Ogni
creatura entro 3 metri dal drago deve riuscire un tiro salvezza di
Destrezza DC 19 o subire 13 (2d6 + 6) danni da botta e venir
gettato prono. Il drago può poi volare fino a metà della sua
velocità di volo.
Attacco di Coda. Il drago effettua un attacco di coda.
Individuare. Il drago effettua una prova di Saggezza
(Percezione).
 
Drago d’Ottone Giovane
Grande drago, caotico buono
FORZA 19 (+4)
DESTREZZA 10 (+0)
COSTITUZIONE 17 (+3)
INTELLIGENZA 12 (+1)
SAGGEZZA 11 (+0)
Carisma 15 (+2)
Classe Armatura 17 (armatura naturale)
\hspace*{0pt}\hfill{Punti Ferita}: 110 (13d10 + 39)
Velocità 12 m, scavo 6 m, volo 24 m
Tiri Salvezza Destrezza +3, Costituzione +6, Saggezza +3,
Carisma +5
Abilità Furtività +3, Percezione +6, Persuasione +5
Immunità al Danno fuoco
Sensi scurovisione 36 m, vista cieca 9 m, Percezione passiva 16
Linguaggi Comune, Draconico
Sfida 6 (2.300 PE)
Azioni
Multiattacco. Il drago può effettuare tre attacchi: uno con il
morso e due con gli artigli.
Artiglio. Attacco con arma da mischia: +7 a colpire, portata 1,5
m, un bersaglio.
Colpisce: 11 (2d6 + 4) danni taglienti.
Morso. Attacco con arma da mischia: +7 a colpire, portata 3 m,
un bersaglio.
Colpisce: 15 (2d10 + 4) danni perforanti.
Arma a Soffio (Ricarica 5-6). Il drago usa una delle seguenti armi
a soffio:
Soffio Infuocato. Il drago esala fuoco in una linea lunga 12 metri e
larga 1,5 metri. Ogni creatura sulla linea deve effettuare un tiro
salvezza di Destrezza DC 14, subendo 42 (12d6) danni da fuoco se
fallisce il tiro salvezza, o la metà di questi danni se lo riesce.
Soffio Soporifero. Il drago esala del gas soporifero in un cono di 9
metri. Ogni creatura in quell’area deve riuscire un tiro salvezza di
Costituzione DC 14 o cadere svenuta per 5 minuti. Questo effetto
termina se la creatura svenuta subisce danni o qualcuno impiega
un’azione per risvegliarla.
Drago d’Ottone Cucciolo
Media drago, caotico buono
FORZA 15 (+2)
DESTREZZA 10 (+0)
COSTITUZIONE 13 (+1)
INTELLIGENZA 10 (+0)
SAGGEZZA 11 (+0)
Carisma 13 (+1)
Classe Armatura 16 (armatura naturale)
\hspace*{0pt}\hfill{Punti Ferita}: 16 (3d8 + 3)
Velocità 9 m, scavo 4,5 m, volo 18 m
Tiri Salvezza Destrezza +2, Costituzione +3, Saggezza +2,
Carisma +3
Abilità Furtività +2, Percezione +4
Immunità al Danno fuoco
Sensi scurovisione 18 m, vista cieca 3 m, Percezione passiva 14
Linguaggi Draconico
Sfida 1 (200 PE)
Azioni
Morso. Attacco con arma da mischia: +4 a colpire, portata 1,5
m, un bersaglio.
Colpisce: 7 (1d10 + 2) danni perforanti.
Arma a Soffio (Ricarica 5-6). Il drago usa una delle seguenti armi
a soffio:
Soffio Infuocato. Il drago esala fuoco in una linea lunga 6 metri e
larga 1,5 metri. Ogni creatura sulla linea deve effettuare un tiro
salvezza di Destrezza DC 11, subendo 14 (4d6) danni da fuoco se
fallisce il tiro salvezza, o la metà di questi danni se lo riesce.
Soffio Soporifero. Il drago esala del gas soporifero in un cono di 4,5
metri. Ogni creatura in quell’area deve riuscire un tiro salvezza di
Costituzione DC 11 o cadere svenuta per 1 minuto. Questo effetto
termina se la creatura svenuta subisce danni o qualcuno impiega
un’azione per risvegliarla.
Drago di Rame Antico
Mastodontica drago, caotico buono
FORZA 27 (+8)
DESTREZZA 12 (+1)
COSTITUZIONE 25 (+7)
INTELLIGENZA 20 (+5)
SAGGEZZA 17 (+3)
Carisma 19 (+4)
Classe Armatura 21 (armatura naturale)
\hspace*{0pt}\hfill{Punti Ferita}: 350 (20d20 + 140)
Velocità 12 m, scalata 12 m, volo 24 m
Tiri Salvezza Destrezza +8, Costituzione +14, Saggezza +10,
Carisma +11
Abilità Furtività +8, Inganno +11, Percezione +17
Immunità al Danno acido
Sensi scurovisione 36 m, vista cieca 18 m, Percezione passiva 27
Linguaggi Comune, Draconico
Sfida 21 (33.000 PE)
Resistenza Leggendaria (3/Giorno). Se il drago fallisce un tiro
salvezza, può scegliere invece di riuscire.
Azioni
Multiattacco. Il drago può usare la sua Presenza Spaventosa. Poi
effettuare tre attacchi: uno con il morso e due con gli artigli.
Artiglio. Attacco con arma da mischia: +15 a colpire, portata 3
m, un bersaglio.
Colpisce: 15 (2d6 + 8) danni taglienti.
Coda. Attacco con arma da mischia: +15 a colpire, portata 6 m,
un bersaglio.
Colpisce: 17 (2d8 + 8) danni da botta.
Morso. Attacco con arma da mischia: +15 a colpire, portata 4,5
m, un bersaglio.
Colpisce: 19 (2d10 + 8) danni perforanti.
Presenza Spaventosa. Ogni creatura scelta dal drago, che si trovi
entro 36 metri da esso e consapevole della sua presenza, deve
riuscire un tiro salvezza di Saggezza DC 19 o restare spaventata per
1 minuto. Una creatura può ripetere il tiro salvezza al termine di
ciascun suo turno, terminando l’effetto se lo riesce. Se il tiro salvezza
della creatura ha successo o l’effetto ha termine per essa, la creatura è
immune alla Presenza Spaventosa del drago per le successive 24 ore.
Arma a Soffio (Ricarica 5-6). Il drago usa una delle seguenti armi
a soffio:
Soffio Acido. Il drago esala acido in una linea lunga 27 metri e larga 3
metri. Ogni creatura sulla linea deve effettuare un tiro salvezza di
Destrezza DC 22, subendo 63 (14d8) danni da acido se fallisce il tiro
salvezza, o la metà di questi danni se lo riesce.
Soffio Rallentante. Il drago esala del gas in un cono di 27 metri. Ogni
creatura in quell’area deve riuscire un tiro salvezza di Costituzione
DC 22. Se fallisce il tiro salvezza, la creatura non può usare la sua
reazione, ha la velocità dimezzata, e non può effettuare più di un
attacco durante il suo turno. Inoltre, la creatura può usare un’azione o
un’azione bonus, ma non entrambe. Questi effetti permangono 1
minuto. La creatura può ripetere il tiro salvezza al termine di ciascun
suo turno, terminando l’effetto su di sé in caso di successo.
Mutare Forma. Il drago può trasformarsi magicamente in un
umanoide o bestia il cui grado di sfida sia pari o inferiore al proprio,
o tornare alla sua vera forma. Alla morte ritorna alla sua vera forma.
Qualsiasi equipaggiamento stia indossando o trasportando viene
assorbito o trasportato nella nuova forma (a scelta del drago).
Nella nuova forma, il drago mantiene il suo allineamento, punti
ferita, Dadi Vita, la facoltà di parlare, le competenze, la Resistenza
Leggendaria, le azioni da tana, e i punteggi di Intelligenza, Saggezza
e Carisma, oltre a questa azione. Le sue statistiche e capacità
vengono altrimenti rimpiazzate da quelle della nuova forma, eccetto i
privilegi di classe o azioni aggiuntive della nuova forma.
Azioni Leggendarie
Il drago può effettuare 3 azioni aggiuntive, scelte tra le opzioni
seguenti. Può usare solo un’opzione leggendaria alla volta e solo
al termine del turno di un’altra creatura. Il drago recupera le
azioni aggiuntive spese all’inizio del proprio round.
Attacco di Ala (Costa 2 Azioni). Il drago batte le ali. Ogni
creatura entro 4,5 metri dal drago deve riuscire un tiro salvezza
di Destrezza DC 23 o subire 15 (2d6 + 8) danni da botta e
venir gettato prono. Il drago può poi volare fino a metà della sua
velocità di volo.
Attacco di Coda. Il drago effettua un attacco di coda.
Individuare. Il drago effettua una prova di Saggezza
(Percezione).
 
Drago di Rame Adulto
Enorme drago, caotico buono
FORZA 23 (+6)
DESTREZZA 12 (+1)
COSTITUZIONE 21 (+5)
INTELLIGENZA 18 (+4)
SAGGEZZA 15 (+2)
Carisma 17 (+3)
Classe Armatura 18 (armatura naturale)
\hspace*{0pt}\hfill{Punti Ferita}: 184 (16d12 + 80)
Velocità 12 m, scalata 12 m, volo 24 m
Tiri Salvezza Destrezza +6, Costituzione +10, Saggezza +7, Carisma +8
Abilità Furtività +6, Inganno +8, Percezione +12
Immunità al Danno acido
Sensi scurovisione 36 m, vista cieca 18 m, Percezione passiva 22
Linguaggi Comune, Draconico
Sfida 14 (11.500 PE)
Resistenza Leggendaria (3/Giorno). Se il drago fallisce un tiro
salvezza, può scegliere invece di riuscire.
Azioni
Multiattacco. Il drago può usare la sua Presenza Spaventosa. Poi
effettuare tre attacchi: uno con il morso e due con gli artigli.
Artiglio. Attacco con arma da mischia: +11 a colpire, portata 1,5
m, un bersaglio.
Colpisce: 13 (2d6 + 6) danni taglienti.
Coda. Attacco con arma da mischia: +11 a colpire, portata 4,5
m, un bersaglio.
Colpisce: 15 (2d8 + 6) danni da botta.
Morso. Attacco con arma da mischia: +11 a colpire, portata 3 m,
un bersaglio.
Colpisce: 17 (2d10 + 6) danni perforanti.
Presenza Spaventosa. Ogni creatura scelta dal drago, che si trovi
entro 36 metri da esso e consapevole della sua presenza, deve
riuscire un tiro salvezza di Saggezza DC 16 o restare spaventata per
1 minuto. Una creatura può ripetere il tiro salvezza al termine di
ciascun suo turno, terminando l’effetto se lo riesce. Se il tiro salvezza
della creatura ha successo o l’effetto ha termine per essa, la creatura è
immune alla Presenza Spaventosa del drago per le successive 24 ore.
Arma a Soffio (Ricarica 5-6). Il drago usa una delle seguenti armi
a soffio:
Soffio Acido. Il drago esala acido in una linea lunga 18 metri e larga
1,5 metri. Ogni creatura sulla linea deve effettuare un tiro salvezza di
Destrezza DC 18, subendo 54 (12d8) danni da acido se fallisce il tiro
salvezza, o la metà di questi danni se lo riesce.
Soffio Rallentante. Il drago esala del gas in un cono di 18 metri. Ogni
creatura in quell’area deve riuscire un tiro salvezza di Costituzione
DC 18. Se fallisce il tiro salvezza, la creatura non può usare la sua
reazione, ha la velocità dimezzata, e non può effettuare più di un
attacco durante il suo turno. Inoltre, la creatura può usare un’azione o
un’azione bonus, ma non entrambe. Questi effetti permangono 1
minuto. La creatura può ripetere il tiro salvezza al termine di ciascun
suo turno, terminando l’effetto su di sé in caso di successo.
Azioni Leggendarie
Il drago può effettuare 3 azioni aggiuntive, scelte tra le opzioni
seguenti. Può usare solo un’opzione leggendaria alla volta e solo
al termine del turno di un’altra creatura. Il drago recupera le
azioni aggiuntive spese all’inizio del proprio round.
Attacco di Ala (Costa 2 Azioni). Il drago batte le ali. Ogni creatura
entro 3 metri dal drago deve riuscire un tiro salvezza di Destrezza
DC 19 o subire 13 (2d6 + 6) danni da botta e venir gettato prono.
Il drago può poi volare fino a metà della sua velocità di volo.
Attacco di Coda. Il drago effettua un attacco di coda.
Individuare. Il drago effettua una prova di Saggezza (Percezione).
Drago di Rame Giovane
Grande drago, caotico buono
FORZA 19 (+4)
DESTREZZA 12 (+1)
COSTITUZIONE 17 (+3)
INTELLIGENZA 16 (+3)
SAGGEZZA 13 (+1)
Carisma 15 (+2)
Classe Armatura 17 (armatura naturale)
\hspace*{0pt}\hfill{Punti Ferita}: 119 (14d10 + 42)
Velocità 12 m, scalata 12 m, volo 24 m
Tiri Salvezza Destrezza +4, Costituzione +6, Saggezza +4, Carisma +5
Abilità Furtività +4, Inganno +5, Percezione +7
Immunità al Danno acido
Sensi scurovisione 36 m, vista cieca 9 m, Percezione passiva 17
Linguaggi Comune, Draconico
Sfida 7 (2.900 PE)
Azioni
Multiattacco. Il drago può effettuare tre attacchi: uno con il
morso e due con gli artigli.
Artiglio. Attacco con arma da mischia: +7 a colpire, portata 1,5
m, un bersaglio.
Colpisce: 11 (2d6 + 4) danni taglienti.
Morso. Attacco con arma da mischia: +7 a colpire, portata 3 m,
un bersaglio.
Colpisce: 15 (2d10 + 4) danni perforanti.
Arma a Soffio (Ricarica 5-6). Il drago usa una delle seguenti armi
a soffio:
Soffio Acido. Il drago esala acido in una linea lunga 12 metri e larga
1,5 metri. Ogni creatura sulla linea deve effettuare un tiro salvezza di
Destrezza DC 14, subendo 40 (9d8) danni da acido se fallisce il tiro
salvezza, o la metà di questi danni se lo riesce.
Soffio Rallentante. Il drago esala del gas in un cono di 9 metri. Ogni
creatura in quell’area deve riuscire un tiro salvezza di Costituzione
DC 14. Se fallisce il tiro salvezza, la creatura non può usare la sua
reazione, ha la velocità dimezzata, e non può effettuare più di un
attacco durante il suo turno. Inoltre, la creatura può usare un’azione o
un’azione bonus, ma non entrambe. Questi effetti permangono 1
minuto. La creatura può ripetere il tiro salvezza al termine di ciascun
suo turno, terminando l’effetto su di sé in caso di successo.
Drago di Rame Cucciolo
Media drago, caotico buono
FORZA 15 (+2)
DESTREZZA 12 (+1)
COSTITUZIONE 13 (+1)
INTELLIGENZA 14 (+2)
SAGGEZZA 11 (+0)
Carisma 13 (+1)
Classe Armatura 16 (armatura naturale)
\hspace*{0pt}\hfill{Punti Ferita}: 22 (4d8 + 4)
Velocità 9 m, scalata 9 m, volo 18 m
Tiri Salvezza Destrezza +3, Costituzione +3, Saggezza +2, Carisma +3
Abilità Furtività +3, Percezione +4
Immunità al Danno acido
Sensi scurovisione 18 m, vista cieca 3 m, Percezione passiva 14
Linguaggi Draconico
Sfida 1 (200 PE)
Azioni
Morso. Attacco con arma da mischia: +4 a colpire, portata 1,5
m, un bersaglio.
Colpisce: 7 (1d10 + 2) danni perforanti.
Arma a Soffio (Ricarica 5-6). Il drago usa una delle seguenti armi
a soffio:
Soffio Acido. Il drago esala acido in una linea lunga 6 metri e larga
1,5 metri. Ogni creatura sulla linea deve effettuare un tiro salvezza di
Destrezza DC 11, subendo 18 (4d8) danni da acido se fallisce il tiro
salvezza, o la metà di questi danni se lo riesce.
Soffio Rallentante. Il drago esala del gas in un cono di 4,5 metri.
Ogni creatura in quell’area deve riuscire un tiro salvezza di
Costituzione DC 11. Se fallisce il tiro salvezza, la creatura non può
usare la sua reazione, ha la velocità dimezzata, e non può effettuare
più di un attacco durante il suo turno. Inoltre, la creatura può usare
un’azione o un’azione bonus, ma non entrambe. Questi effetti
permangono 1 minuto. La creatura può ripetere il tiro salvezza al
termine di ciascun suo turno, terminando l’effetto su di sé in caso di
successo.
 
Drider
Grande mostruosità, caotico malvagio
FORZA 16 (+3)
DESTREZZA 16 (+3)
COSTITUZIONE 18 (+4)
INTELLIGENZA 13 (+1)
SAGGEZZA 14 (+2)
Carisma 12 (+1)
Classe Armatura 19 (armatura naturale)
\hspace*{0pt}\hfill{Punti Ferita}: 123 (13d10 + 52)
Velocità 9 m, scalata 9 m
Abilità Furtività +9, Percezione +5
Sensi scurovisione 36 m, Percezione passiva 15
Linguaggi Elfico, Sottocomune
Sfida 6 (2.300 PE)
Camminare sulla Tela. Il drider ignora le restrizioni al
movimento provocate dalle ragnatele.
Discendenza Fatata. Il drider ha +1d6 ai tiri salvezza per
non restare affascinato, e la magia non può far addormentare un
drider.
Incantesimi Innati. La caratteristica da incantatore innato del
drider è la Saggezza (DC dei tiri salvezza 13). Il drider può
lanciare in maniera innata i seguenti incantesimi, senza bisogno
di componenti materiali:
A volontà: luci danzanti
1/Giorno: luminescenza, oscurità
Scalare come Ragno. Il drider può scalare superfici difficili,
compreso lo stare a testa in giù sul soffitto, senza bisogno di
effettuare una prova di abilità.
Azioni
Multiattacco. Il drider effettua tre attacchi con la spada lunga o
con l’arco lungo. Può rimpiazzare uno di questi attacchi con un
attacco di morso.
Morso. Attacco con arma da mischia: +6 a colpire, portata 1,5
m, una creatura.
Colpisce: 2 (1d4) danni perforanti più 9 (2d8) danni da veleno.
Spada Lunga. Attacco con arma da mischia: +6 a colpire,
portata 1,5 m, un bersaglio.
Colpisce: 7 (1d8 + 3) danni taglienti, o 8 (1d8 + 3) danni
taglienti se usata con due mani.
Arco Lungo. Attacco con arma a Distanza: +6 a colpire, gittata
45/180 m, un bersaglio.
Colpisce: 7 (1d8 + 3) danni perforanti più 4 (1d8) danni da
veleno.
Driade
Media fatato, neutrale
FORZA 10 (+0)
DESTREZZA 12 (+1)
COSTITUZIONE 11 (+0)
INTELLIGENZA 14 (+2)
SAGGEZZA 15 (+2)
Carisma 18 (+4)
Classe Armatura 11 (16 con pelle di corteccia)
\hspace*{0pt}\hfill{Punti Ferita}: 22 (5d8)
Velocità 9 m
Abilità Furtività +5, Percezione +4
Sensi scurovisione 18 m, Percezione passiva 14
Linguaggi Elfico, Silvano
Sfida 1 (200 PE)
Camminata Arborea. Uno volta durante il suo turno, la driade
può usare 3 metri di movimento per entrare magicamente in un
albero vivo a sua portata ed emergere da un altro albero vivo
entro 18 metri dal primo albero, ricomparendo in uno spazio non
occupato entro 1,5 metri dal secondo albero. Entrambi gli alberi
devono essere di taglia Grande o superiore.
Incantesimi Innati. La caratteristica da incantatore innato della
driade è il Carisma (DC 14 per i tiri salvezza degli incantesimi).
La driade può lanciare in maniera innata i seguenti incantesimi,
senza aver bisogno di componenti materiali.
A volontà: arte del druido
3/giorno ciascuno: bacche benefiche, intralciare
1/giorno: passare senza tracce, pelle coriacea, randello
incantato
Parlare con Animali e Piante. La driade può comunicare con
bestie e piante come se parlassero la stessa lingua.
Resistenza alla Magia. La driade ha +1d6 ai tiri salvezza
contro incantesimi e altri effetti magici.
Azioni
Randello. Attacco con arma da mischia: +2 a colpire (+6 a
colpire con bastone), portata 1,5 m, un bersaglio.
Colpisce: 2 (1d4) danni da botta, o 8 (1d8 + 4) danni
contundenti con bastone
Fascino Fatato. La driade può prendere a bersaglio un umanoide o
bestia entro 9 metri da lei e che possa vedere. Se il bersaglio può
vedere la driade, deve riuscire un tiro salvezza di Saggezza DC 14 o
restare affascinato dalla magia. Le creature affascinate considerano la
driade un’amica fidata da ascoltare e proteggere. Sebbene il bersaglio
non sia sotto il controllo della driade, interpreterà le richieste o le
azioni della driade nel modo più favorevole possibile.
Ogni volta che la driade o i suoi alleati arrecano danno al bersaglio,
esso può ripetere il tiro salvezza, terminando l’effetto in caso di
successo. Altrimenti, l’effetto permane 24 ore o finché la driade
muore, si trova su di un piano di esistenza diverso rispetto al
bersaglio, o termina l’effetto con un’azione bonus. Se il tiro salvezza
del bersaglio riesce, il bersaglio sarà immune al Fascino Fatato della
driade per le successive 24 ore.
La driade non può tenere affascinati più di un umanoide o tre bestie
alla volta.
Duergar
Media umanoide (nano), legale malvagio
FORZA 14 (+2)
DESTREZZA 11 (+0)
COSTITUZIONE 14 (+2)
INTELLIGENZA 11 (+0)
SAGGEZZA 10 (+0)
Carisma 9 (-1)
Classe Armatura 16 (armatura di scaglie, scudo)
\hspace*{0pt}\hfill{Punti Ferita}: 26 (4d8 + 8)
Velocità 7,5 m
Resistenza al Danno veleno
Sensi scurovisione 36 m, Percezione passiva 10
Linguaggi Nanico, Sottocomune
Sfida 1 (200 PE)
Resilienza Duerga. Il duergar ha +1d6 ai tiri salvezza contro
veleni, incantesimi e illusioni, oltre al resistere al restare
affascinato o paralizzato.
Sensibilità alla Luce. Mentre è alla luce del sole, il duergar ha
svantaggio ai tiri di attacco, oltre che alle prove di Saggezza
(Percezione) basate sulla vista.
Azioni
Ingrandire (Ricarica dopo un Riposo Breve o Lungo). Per 1
minuto, il duergar aumenta magicamente di taglia, insieme a
tutto ciò che sta trasportando o indossando. Mentre è ingrandito,
il duergar è di taglia Grande, raddoppia i dadi di danno degli
attacchi con armi basate sulla Forza (già incluso negli attacchi), e
ha vantaggio alle prove di Forza e ai tiri salvezza di Forza. Se il
duergar non ha sufficiente spazio per diventare Grande, ottiene la
massima taglia concessa dallo spazio a disposizione.
Piccone da Guerra. Attacco con arma da mischia: +4 a colpire,
portata 1,5 m, un bersaglio.
Colpisce: 6 (1d8 + 2) danni perforanti, o 11 (2d8 + 2) danni
perforanti quando ingrandito.
Giavellotto. Attacco con arma da mischia o a Distanza: +4 a
colpire, portata 1,5 m o gittata 9/36 m, un bersaglio.
Colpisce: 5 (1d6 + 2) danni perforanti o 9 (2d6 + 2) danni
perforanti quando ingrandito.
Invisibilità (Ricarica dopo un Riposo Breve o Lungo). Il
duergar diventa magicamente invisibile al massimo per un’ora
(come se stesse mantenendo la concentrazione per un
incantesimo) o finché non attacca, lancia un incantesimo, usa
Ingrandire o la sua concentrazione viene spezzata. Tutto
l’equipaggiamento che il duergar indossa o trasporta diventa
invisibile assieme a lui.
Elementale dell’Acqua
Grande elementale, neutrale
FORZA 18 (+4)
DESTREZZA 14 (+2)
COSTITUZIONE 18 (+4)
INTELLIGENZA 5 (-3)
SAGGEZZA 10 (+0)
Carisma 8 (-1)
Classe Armatura 14 (armatura naturale)
\hspace*{0pt}\hfill{Punti Ferita}: 114 (12d10 + 48)
Velocità 9 m, nuoto 27 m
Resistenze al Danno acido; da botta, perforante e tagliente
di attacchi non magici
Immunità al Danno veleno
Immunità alle Condizioni afferrato, avvelenato, intralciato,
paralizzato, pietrificato, privo di sensi, prono, sfinimento
Sensi scurovisione 18 m, Percezione passiva 10
Linguaggi Aquan
Sfida 5 (1.800 PE)
Congelamento. Se l’elementale subisce danno da freddo, gela
parzialmente; la sua velocità è ridotta di 6 metri fino al termine
del suo prossimo turno.
Forma d’Acqua. L’elementale può entrare nello spazio di una
creatura ostile e fermarsi lì. Può muoversi attraverso uno spazio
stretto fino a 2,5 centimetri senza doversi stringere.
Natura Elementale. Un elementale non ha bisogno di aria, cibo,
bevande o sonno.
Azioni
Multiattacco. L’elementale effettua due attacchi di schianto.
Schianto. Attacco con arma da mischia: +7 a colpire, portata 1,5
m, un bersaglio.
Colpisce: 13 (2d8 + 4) danni da botta.
Sommergere (Ricarica 4-6). Ogni creatura nello spazio
dell’elementale deve effettuare un tiro salvezza di Forza DC 15.
Se lo fallisce, il bersaglio subisce 13 (2d8 + 4) danni
contundenti. Se è di taglia Grande o inferiore, il bersaglio è
anche afferrato (DC 14 per fuggire). Fino al termine
dell’afferrare, il bersaglio è intralciato e non può respirare a
meno che non sia in grado di respirare acqua. Se il tiro salvezza
riesce, il bersaglio viene spinto fuori dallo spazio
dell’elementale.
L’elementale può afferrare una creatura Grande o fino a due
Medie o più piccole alla volta. All’inizio di ciascun turno
dell’elementale, ogni bersaglio afferrato subisce 13 (2d8 + 4)
danni da botta. Una creatura entro 1,5 metri dall’elementale
può trascinare fuori da esso una creatura o oggetto, impiegando
un’azione per tentare di riuscire una prova di Forza DC 14.
 
Elementale dell’Aria
Grande elementale, neutrale
FORZA 14 (+2)
DESTREZZA 20 (+5)
COSTITUZIONE 14 (+2)
INTELLIGENZA 6 (-2)
SAGGEZZA 10 (+0)
Carisma 6 (-2)
Classe Armatura 15
\hspace*{0pt}\hfill{Punti Ferita}: 90 (12d10 + 24)
Velocità 0 m, volo 27 m (fluttua)
Resistenze al Danno fulmine, tuono; da botta, perforante e
tagliente di attacchi non magici
Immunità al Danno veleno
Immunità alle Condizioni afferrato, avvelenato, intralciato,
paralizzato, pietrificato, privo di sensi, prono, sfinimento
Sensi scurovisione 18 m, Percezione passiva 18
Linguaggi Auran
Sfida 5 (1.800 PE)
Forma d’Aria. L’elementale può entrare nello spazio di una
creatura ostile e fermarsi lì. Può muoversi attraverso uno spazio
stretto fino a 2,5 centimetri senza doversi stringere.
Natura Elementale. Un elementale non ha bisogno di aria, cibo,
bevande o sonno.
Azioni
Multiattacco. L’elementale effettua due attacchi di schianto.
Schianto. Attacco con arma da mischia: +8 a colpire, portata 1,5
m, un bersaglio.
Colpisce: 14 (2d8 + 5) danni da botta.
Turbine (Ricarica 4-6). Ogni creatura nello spazio
dell’elementale deve effettuare un tiro salvezza di Forza DC 13.
Se lo fallisce, il bersaglio subisce 15 (3d8 + 2) danni da botta
e viene scagliato a 6 metri di distanza dall’elementale in una
direzione casuale e cadere prono. Se un bersaglio lanciato
colpisce un oggetto, come un muro o il pavimento, subisce 3
(1d6) danni da botta per ogni 3 metri per cui è stato lanciato.
Se il bersaglio viene lanciato contro un’altra creatura, quella
creatura deve riuscire un tiro salvezza di Destrezza DC 13 o
subire lo stesso danno e cadere prona.
Se il tiro salvezza riesce, il bersaglio subisce la metà del danno
da botta e non viene scagliato via né cade prono.
Elementale del Fuoco
Grande elementale, neutrale
FORZA 10 (+0)
DESTREZZA 17 (+3)
COSTITUZIONE 16 (+3)
INTELLIGENZA 6 (-2)
SAGGEZZA 10 (+0)
Carisma 7 (-2)
Classe Armatura 13
\hspace*{0pt}\hfill{Punti Ferita}: 102 (12d10 + 36)
Velocità 15 m
Resistenze al Danno da botta, perforante e tagliente di
attacchi non magici
Immunità al Danno fuoco, veleno
Immunità alle Condizioni afferrato, avvelenato, intralciato,
paralizzato, pietrificato, prono, privo di sensi, sfinimento
Sensi scurovisione 18 m, Percezione passiva 10
Linguaggi Ignan
Sfida 5 (1.800 PE)
Forma di Fuoco. L’elementale può spostarsi attraverso uno
spazio fino a 2,5 centimetri di larghezza senza stringersi. Una
creatura che entri a contatto o colpisca l’elementale con un
attacco da mischia mentre si trova entro 1,5 metri da esso subisce
5 (1d10) danni da fuoco. Inoltre, l’elementale può entrare nello
spazio di una creatura ostile e fermarsi lì. La prima volta che
entra nello spazio di una creatura in un turno, la creatura subisce
5 (1d10) danni da fuoco e prende fuoco; finché qualcuno non
impiega un’azione per spegnere le fiamme, la creatura subirà 5
(1d10) danni da fuoco all’inizio di ciascun proprio round.
Illuminazione. L’elementale emette luce intensa in un raggio di
9 metri e luce fioca per ulteriori 9 metri.
Natura Elementale. Un elementale non ha bisogno di aria, cibo,
bevande o sonno.
Suscettibilità all’Acqua. L’elementale subisce 1 danno da freddo
per ogni 1,5 metri che si muove in acqua o per ogni 4 litri
d’acqua che gli vengono spruzzati addosso.
Azioni
Multiattacco. L’elementale effettua due attacchi di contatto.
Contatto. Attacco con arma da mischia: +6 a colpire, portata 1,5
m, un bersaglio.
Colpisce: 10 (2d8 + 5) danni da fuoco. Se il bersaglio è una
creatura o un oggetto infiammabile, prende fuoco. Finché una
creatura non impiega un’azione per spegnere le fiamme, la
creatura subirà 5 (1d10) danni da fuoco all’inizio di ciascun
proprio round.
Elementale della Terra
Grande elementale, neutrale
FORZA 20 (+5)
DESTREZZA 8 (-1)
COSTITUZIONE 20 (+5)
INTELLIGENZA 5 (-3)
SAGGEZZA 10 (+0)
Carisma 5 (-3)
Classe Armatura 17 (armatura naturale)
\hspace*{0pt}\hfill{Punti Ferita}: 126 (12d10 + 60)
Velocità 9 m, scavo 9 m
Vulnerabilità al Danno tuono
Resistenze al Danno da botta, perforante e tagliente di
attacchi non magici
Immunità al Danno veleno
Immunità alle Condizioni avvelenato, paralizzato, pietrificato,
prono, privo di sensi, sfinimento,
Sensi percezione tellurica 18 m, scurovisione 18 m, Percezione
passiva 10
Linguaggi Terran
Sfida 5 (1.800 PE)
Mostro d’Assedio. L’elementale infligge danni doppi agli oggetti
e le strutture.
Natura Elementale. Un elementale non ha bisogno di aria, cibo,
bevande o sonno.
Planata Terrestre. L’elementale può scavare attraversa la terra e
la pietra non magiche e non lavorate. Quando lo fa, l’elementale
non disturba il materiale che sposta.
Azioni
Multiattacco. L’elementale effettua due attacchi di schianto.
Schianto. Attacco con arma da mischia: +8 a colpire, portata 3
m, un bersaglio.
Colpisce: 14 (2d8 + 5) danni da botta.
Elfo, Drow
Media umanoide (elfo), neutrale malvagio
FORZA 10 (+0)
DESTREZZA 14 (+2)
COSTITUZIONE 10 (+0)
INTELLIGENZA 11 (+0)
SAGGEZZA 11 (+0)
Carisma 12 (+1)
Classe Armatura 15 (giaco di maglia)
\hspace*{0pt}\hfill{Punti Ferita}: 13 (3d8)
Velocità 9 m
Abilità Furtività +4, Percezione +2
Sensi scurovisione 36 m, Percezione passiva 12
Linguaggi Elfico, Sottocomune
Sfida 1/4 (50 PE)
Discendenza Fatata. Il drow ha +1d6 ai tiri salvezza contro
l’essere affascinato, e la magia non può mettere a dormire i
drow.
Incantesimi Innati. La caratteristica da incantatore innato del
drow è il Carisma (DC dei tiri salvezza 11). Il drow può lanciare
questi incantesimi in maniera innata, senza bisogno di
componenti:
A volontà: luci danzanti
1/giorno ciascuno: luminescenza, oscurità
Sensibilità alla Luce. Mentre è alla luce del sole, il drow ha
svantaggio ai tiri per colpire, oltre che alle prove di Saggezza
(Percezione) basate sulla vista.
Azioni
Spada Corta. Attacco con arma da mischia: +4 a colpire, portata
1,5 m, un bersaglio.
Colpisce: 5 (1d6 + 2) danni perforanti.
Balestrino. Attacco con arma a Distanza: +4 a colpire, gittata
9/36 m, un bersaglio.
Colpisce: 5 (1d6 + 2) danni perforanti, e il bersaglio deve
riuscire un tiro salvezza di Costituzione DC 13 o restare
avvelenato per 1 ora. Se il tiro salvezza fallisce di 5 o più, il
bersaglio è anche privo di sensi mentre resta avvelenato in questo
modo. Il bersaglio si risveglia se subisce danni o un’altra
creatura effettua un’azione per risvegliarlo.
 
Ettercap
Media mostruosità, neutrale malvagio
FORZA 14 (+2)
DESTREZZA 15 (+2)
COSTITUZIONE 13 (+1)
INTELLIGENZA 7 (-2)
SAGGEZZA 12 (+1)
Carisma 8 (-2)
Classe Armatura 13 (armatura naturale)
\hspace*{0pt}\hfill{Punti Ferita}: 44 (8d8 + 8)
Velocità 9 m, scalata 9 m
Abilità Furtività +4, Percezione +3, Sopravvivenza +3
Sensi scurovisione 18 m, Percezione passiva 13
Linguaggi -
Sfida 2 (450 PE)
Camminare sulla Tela. L’ettercap ignora le restrizioni al
movimento provocate dalle ragnatele.
Scalare come Ragno. L’ettercap può scalare superfici difficili,
compreso lo stare a testa in giù sul soffitto, senza bisogno di
effettuare una prova di caratteristica.
Senso della Tela. Mentre è in contatto con una ragnatela,
l’ettercap sa l’esatta posizione di qualsiasi altra creatura in
contatto con la stessa ragnatela.
Azioni
Multiattacco. L’ettercap effettua due attacchi: uno con il morso e
uno con gli artigli
Artigli. Attacco con arma da mischia: +4 a colpire, portata 1,5
m, un bersaglio.
Colpisce: 7 (2d4 + 2) danni taglienti.
Morso. Attacco con arma da mischia: +4 a colpire, portata 1,5
m, un bersaglio.
Colpisce: 6 (1d8 + 2) danni perforanti più 4 (1d8) danni da
veleno. Il bersaglio deve riuscire un tiro salvezza di Costituzione
DC 11 o restare avvelenato per 1 minuto. La creatura può
ripetere il tiro salvezza al termine di ciascun suo turno,
terminando l’effetto se riesce il tiro salvezza.
Ragnatela (Ricarica 5-6). Attacco con arma a Distanza: +4 a
colpire, gittata 9/18 m, una creatura di taglia Grande o minore.
Colpisce: La creatura è intralciata dalla ragnatela. Con
un’azione, la creatura intralciata può effettuare una prova di
Forza DC 11, liberandosi dalla tela se la riesce. L’effetto termina
se la tela è distrutta. La tela ha Difesa 10, 5 punti ferita,
vulnerabilità ai danni da fuoco, e immunità ai danni da botta,
da veleno e psichici.
Ettin
Grande gigante, caotico malvagio
FORZA 21 (+5)
DESTREZZA 8 (-1)
COSTITUZIONE 17 (+3)
INTELLIGENZA 6 (-2)
SAGGEZZA 10 (+0)
Carisma 8 (-1)
Classe Armatura 12 (armatura naturale)
\hspace*{0pt}\hfill{Punti Ferita}: 85 (10d10 + 30)
Velocità 12 m
Abilità Percezione +4
Sensi Percezione passiva 14
Linguaggi Gigante, Orco
Sfida 4 (1.100 PE)
Due Teste. L’ettin ha vantaggio alle prove di Saggezza
(Percezione) e sui tiri salvezza contro le condizioni accecato,
affascinato, assordato, privo di sensi, spaventato e stordito.
Veglia. Quando una delle due teste dell’ettin è addormentata,
l’altra è sveglia.
Azioni
Multiattacco. L’ettin effettua due attacchi: uno con l’ascia da
battaglia e uno con la mazza chiodata.
Ascia da Battaglia. Attacco con arma da mischia: +7 a colpire,
portata 1,5 m, un bersaglio.
Colpisce: 14 (2d8 + 5) danni taglienti.
Mazza Chiodata. Attacco con arma da mischia: +7 a colpire,
portata 1,5 m, un bersaglio.
Colpisce: 14 (2d8 + 5) danni perforanti.
Fantasma
Media non morto, qualsiasi allineamento
FORZA 7 (-2)
DESTREZZA 13 (+1)
COSTITUZIONE 10 (+0)
INTELLIGENZA 10 (+0)
SAGGEZZA 12 (+1)
Carisma 17 (+3)
Classe Armatura 11
\hspace*{0pt}\hfill{Punti Ferita}: 45 (10d8)
Velocità 0 m, volo 12 m (fluttua)
Resistenze al Danno acido, fulmine, fuoco, tuono; da botta,
perforante, tagliente di attacchi non magici
Immunità ai Danni freddo, necrotico, veleno
Immunità alle Condizioni affascinato, afferrato, avvelenato,
intralciato, paralizzato, pietrificato, prono, sfinimento, spaventato
Sensi scurovisione 18 m, Percezione passiva 11
Linguaggi qualsiasi lingua conosciuta in vita
Sfida 4 (1.100 PE)
Movimento Incorporeo. Il fantasma può attraversare altre creature e
oggetti come se fossero terreno difficile. Subisce 5 (1d10) danni da
forza se termina il suo turno all’interno di un oggetto.
Natura Non Morta. Il fantasma non ha bisogno di aria, cibo,
bevande o di dormire.
Vista Eterea. Il fantasma può vedere 18 metri nel Piano Etereo
quando si trova sul Piano Materiale, e vice versa.
Azioni
Tocco Avvizzente. Attacco con arma da mischia: +5 a colpire,
portata 1,5 m, un bersaglio.
Colpisce: 17 (4d6 + 3) danni necrotici.
Eterealità. Il fantasma entra nel Piano Etereo dal Piano Materiale, o vice
versa. È visibile sul Piano Materiale mentre è nel Margine Etereo, e vice
versa, ma non può interagire con nulla che si trovi sull’altro piano.
Possessione (Ricarica 6). Un umanoide, entro 1,5 metri e visibile al
fantasma, deve riuscire un tiro salvezza di Carisma DC 13 o venire
posseduto dal fantasma; il fantasma poi scompare, e il bersaglio è inabile
e perde il controllo del suo corpo. Il fantasma ora controlla il corpo ma
non priva il bersaglio della sua consapevolezza. Il fantasma non può
essere bersaglio di attacchi, incantesimi, o altri effetti, eccetto quelli che
scacciano i non morti, e mantiene il suo allineamento, Intelligenza,
Saggezza, Carisma e immunità all’essere affascinato e spaventato. Per il
resto usa altrimenti le statistiche del bersaglio posseduto, ma non accede
al sapere, i privilegi di classe o le competenze del bersaglio.
La possessione dura finché il corpo scende a 0 punti ferita, il fantasma la
termina con un’azione bonus, o il fantasma viene scacciato o espulso da un
effetto come l’incantesimo dissolvi il bene e il male. Quando la possessione
termina, il fantasma riappare in uno spazio non occupato entro 1,5 metri dal
corpo. Il bersaglio è immune alla Possessione di questo fantasma per 24 ore
dopo aver riuscito il tiro salvezza o al termine della possessione.
Viso Orripilante. Ogni creatura che non sia non morta, entro 18 metri
dal fantasma e che lo possa vedere, deve riuscire un tiro salvezza di
Saggezza DC 13 o essere spaventata per 1 minuto. Se il tiro salvezza
fallisce di 5 o più, il bersaglio invecchia anche di 1d4 x 10 anni. Un
bersaglio spaventato può ripetere il tiro salvezza al termine di ciascun
proprio round, terminando l’effetto per sé, qualora riuscisse il tiro
salvezza. Se il tiro salvezza del bersaglio riesce e per lui l’effetto ha fine, il
bersaglio è immune al Viso Orripilante del fantasma per le successive 24
ore. L’effetto di invecchiamento può essere invertito con l’incantesimo
ristorare superiore, ma solo se eseguito entro 24 dall’effetto di
invecchiamento.
Fauci Gorgoglianti
Media aberrazione, neutrale
FORZA 10 (+0)
DESTREZZA 8 (-1)
COSTITUZIONE 16 (+3)
INTELLIGENZA 3 (-4)
SAGGEZZA 10 (+0)
Carisma 6 (-2)
Classe Armatura 9
\hspace*{0pt}\hfill{Punti Ferita}: 67 (9d8 + 27)
Velocità 3 m, nuoto 3 m
Immunità alle Condizioni prono
Sensi scurovisione 18 m, Percezione passiva 10
Linguaggi -
Sfida 2 (450 PE)
Gorgoglio. Finché la fauce è in grado di vedere una creatura e
non è inabile, pronuncia frasi incoerenti. Ogni creatura che inizi
il suo turno entro 6 metri dalla fauce e può udire il suo gorgoglio
deve effettuare un tiro salvezza di Saggezza DC 10. Se lo
fallisce, la creatura non può effettuare reazioni fino all’inizio del
suo prossimo turno e tira un d8 per determinare cosa farà durante
il proprio round. Da 1 a 4, la creatura non fa nulla. Con 5 o 6, la
creatura non svolge nessun’azione o azione bonus e usa tutto il
suo movimento per muoversi in una direzione determinata
casualmente. Con 7 o 8, la creatura effettua un attacco da
mischia contro una creatura determinata a caso entro la sua
portata o non fa nulla se non è in grado di effettuare un simile
attacco.
Terreno Aberrante. Il terreno in un raggio di 3 metri intorno alla
fauce è considerato terreno difficile. Ogni creatura che inizi il
suo turno in quell’area deve riuscire un tiro salvezza di Forza DC
10 o vedere la sua velocità ridotta a 0 fino all’inizio del suo turno
successivo.
Azioni
Multiattacco. La fauce gorgogliante effettua un attacco di morso
e, se può, uno Sputo Accecante.
Morso. Attacco con arma da mischia: +2 a colpire, portata 1,5
m, una creatura.
Colpisce: 17 (5d6) danni perforanti. Se il bersaglio è di taglia
Media o inferiore, deve riuscire un tiro salvezza di Forza DC 10
o venir gettato prono. Se il bersaglio viene ucciso da questo
danno, viene assorbito dalla fauce.
Sputo Accecante (Ricarica 5-6). La fauce sputa un globo
chimico ad un punto visibile entro 4,5 metri da essa. Il globo
esplode all’impatto in un lampo accecante di luce. Ogni creatura
entro 1,5 metri dal lampo deve riuscire un tiro salvezza di
Destrezza DC 13 o restare accecata fino al termine del prossimo
turno della fauce.
 
Funghi
Fungo Stridente
Media pianta, disallineato
FORZA 1 (-5)
DESTREZZA 1 (-5)
COSTITUZIONE 10 (+0)
INTELLIGENZA 1 (-5)
SAGGEZZA 3 (-4)
Carisma 1 (-5)
Classe Armatura 5
\hspace*{0pt}\hfill{Punti Ferita}: 13 (3d8)
Velocità 0 m
Immunità alle Condizioni accecato, assordato, spaventato
Sensi vista cieca 9 m (cieco oltre questo raggio), Percezione
passiva 6
Linguaggi -
Sfida 0 (10 PE)
Falso Aspetto. Mentre il fungo stridente rimane immobile, è
indistinguibile da un normale fungo.
Azioni
Strillo. Quando una luce intensa o una creatura si trova entro 9
metri dal fungo stridente, esso emette un strillo udibile fino a 90
metri di distanza. Il fungo stridente continua a strillare finché la
fonte del disturbo non si è portata fuori gittata e per altri 1d4
turni successivi.
Fungo Violetto
Media pianta, disallineato
FORZA 3 (-4)
DESTREZZA 1 (-5)
COSTITUZIONE 10 (+0)
INTELLIGENZA 1 (-5)
SAGGEZZA 3 (-4)
Carisma 1 (-5)
Classe Armatura 5
\hspace*{0pt}\hfill{Punti Ferita}: 18 (4d8)
Velocità 1,5 m
Immunità alle Condizioni accecato, assordato, spaventato
Sensi vista cieca 9 m (cieco oltre questo raggio), Percezione
passiva 6
Linguaggi -
Sfida 1/4 (50 PE)
Falso Aspetto. Mentre il fungo violetto rimane immobile, è
indistinguibile da un normale fungo.
Azioni
Multiattacco. Il fungo effettua 1d4 attacchi con Contatto Putrido.
Contatto Putrido. Attacco con arma da mischia: +2 a colpire,
portata 3 m, un bersaglio.
Colpisce: 4 (1d8) danni necrotici.
Fuoco Fatuo
Minuscola non morto, caotico malvagio
FORZA 1 (-5)
DESTREZZA 28 (+9)
COSTITUZIONE 10 (+0)
INTELLIGENZA 13 (+1)
SAGGEZZA 14 (+2)
Carisma 11 (+0)
Classe Armatura 19
\hspace*{0pt}\hfill{Punti Ferita}: 22 (9d4)
Velocità 0 m, volo 15 m (fluttua)
Immunità ai Danni fulmine, veleno
Resistenze al Danno acido, freddo, fuoco, necrotico, tuono;
contendente, perforante e tagliente di attacchi non magici
Immunità alle Condizioni afferrato, avvelenato, intralciato,
paralizzato, privo di sensi, prono, sfinimento
Sensi scurovisione 36 m, Percezione passiva 12
Linguaggi le lingue che conosceva in vita
Sfida 2 (450 PE)
Consumare Vita. Con un’azione bonus, il fuoco fatuo può
prendere a bersaglio una creatura che può vedere entro 1,5 metri
da esso e che abbia 0 punti ferita e sia ancora in vita. Il bersaglio
deve riuscire un tiro salvezza di Costituzione DC 10 contro
questa magia o morire. Se il bersaglio muore, il fuoco fatuo
recupera 10 (3d6) punti ferita.
Effimero. Il fuoco fatuo non può indossare né trasportare nulla.
Illuminazione Variabile. Il fuoco fatuo promana luce intensa in
un raggio da 1,5 a 6 metri e luce fioca per un numero di metri
aggiuntivi pari al raggio scelto. Il fuoco fatuo può modificare
questo raggio con un’azione bonus.
Movimento Incorporeo. Il fuoco fatuo può muoversi attraverso
altre creature e oggetti come se fossero terreno difficile. Subisce
5 (1d10) danni da forza se termina il suo turno all’interno di un
oggetto.
Natura Non Morta. Il fuoco fatuo non ha bisogno di aria, cibo o
bevande.
Azioni
Scossa. Attacco con incantesimo in mischia: +4 a colpire, portata
1,5 m, una creatura.
Colpisce: 9 (2d8) danni da fulmine.
Invisibilità. Il fuoco fatuo e la sua luce diventano magicamente
invisibili finché non attacca o usa Consumare Vita, o finché la
sua concentrazione non termina (come se si stesse concentrando
su di un incantesimo).
Fustigatore
Grande mostruosità, neutrale malvagio
FORZA 18 (+4)
DESTREZZA 8 (-1)
COSTITUZIONE 17 (+3)
INTELLIGENZA 7 (-2)
SAGGEZZA 16 (+3)
Carisma 6 (-2)
Classe Armatura 20 (armatura naturale)
\hspace*{0pt}\hfill{Punti Ferita}: 93 (11d10 + 33)
Velocità 3 m, scalata 3 m
Abilità Furtività +5, Percezione +6
Sensi scurovisione 18 m, Percezione passiva 16
Linguaggi -
Sfida 5 (1.800 PE)
Falso Aspetto. Quando il fustigatore rimane immobile, è
indistinguibile da una normale formazione rocciosa, come una
stalagmite.
Scalare come Ragno. Il fustigatore può scalare superfici difficili,
compreso lo stare a testa in giù sul soffitto, senza bisogno di
effettuare una prova di abilità.
Viticci Afferranti. Il fustigatore può avere fino a sei viticci alla
volta. Ogni viticcio può essere attaccato (Difesa 20; 10 punti ferita;
immunità ai danni psichici e da veleno). Distruggere un viticcio
non infligge danni al fustigatore, che può produrre un viticcio di
rimpiazzo nel suo prossimo turno. Un viticcio può essere anche
rotto se una creatura effettua un’azione e riesce una prova di
Forza DC 15 contro di esso.
Azioni
Multiattacco. Il fustigatore può effettuare quattro attacchi con i
suoi viticci, usare avvolgere e effettuare un attacco con il morso.
Morso. Attacco con arma da mischia: +7 a colpire, portata 1,5
m, un bersaglio.
Colpisce: 22 (4d8 + 4) danni perforanti.
Viticcio. Attacco con arma da mischia: +7 a colpire, portata 15
m, una creatura.
Colpisce: Il bersaglio è afferrato (DC 15 per fuggire). Fino al
termine dell’afferrare, il bersaglio è intralciato e ha svantaggio
alle prove di Forza e ai tiri salvezza di Forza, mentre il
fustigatore non può usare lo stesso viticcio contro un altro
bersaglio.
Avvolgere. Il fustigatore trascina le creature afferrate da lui di 7,5
metri verso di lui.
Gargoyle
Media elementale, caotico malvagio
FORZA 15 (+2)
DESTREZZA 11 (+0)
COSTITUZIONE 16 (+3)
INTELLIGENZA 6 (-2)
SAGGEZZA 11 (+0)
Carisma 7 (-2)
Classe Armatura 15 (armatura naturale)
\hspace*{0pt}\hfill{Punti Ferita}: 52 (7d8 + 21)
Velocità 9 m, volo 18 m
Resistenze al Danno da botta, perforante e tagliente di
attacchi non magici che non siano di adamantio
Immunità ai Danni veleno
Immunità alle Condizioni avvelenato, pietrificato, sfinimento
Sensi scurovisione 18 m, Percezione passiva 10
Linguaggi Terran
Sfida 2 (450 PE)
Falso Aspetto. Mentre la gargoyle rimane immobile, è
indistinguibile da una statua inanimata.
Natura Elementale. Una gargoyle non ha bisogno di aria, cibo,
bevande o sonno.
Azioni
Multiattacco. La gargoyle effettua due attacchi: uno con il morso
e uno con gli artigli.
Artigli. Attacco con arma da mischia: +4 a colpire, portata 1,5
m, un bersaglio.
Colpisce: 5 (1d6 + 2) danni taglienti.
Morso. Attacco con arma da mischia: +4 a colpire, portata 1,5
m, un bersaglio.
Colpisce: 5 (1d6 + 2) danni perforanti.
 
Geni
Djinni
Grande elementale, caotico buono
FORZA 21 (+5)
DESTREZZA 15 (+2)
COSTITUZIONE 22 (+6)
INTELLIGENZA 15 (+2)
SAGGEZZA 16 (+3)
Carisma 20 (+5)
Classe Armatura 17 (armatura naturale)
\hspace*{0pt}\hfill{Punti Ferita}: 161 (14d10 + 84)
Velocità 9 m, volo 27 m
Tiri Salvezza Destrezza +6, Saggezza +7, Carisma +9
Immunità al Danno fulmine, tuono
Sensi scurovisione 36 m, Percezione passiva 13
Linguaggi Auran
Sfida 11 (7.200 PE)
Decesso Elementale. Se il djinni muore, il suo corpo si
disintegra in una brezza calda, lasciando dietro di sé solo
l’equipaggiamento che il djinni stava indossando o trasportando.
Incantesimi Innati. La caratteristica da incantatore innato del
djinni è il Carisma (DC dei tiri salvezza degli incantesimi 17, +9
a colpire con attacchi da incantesimo). Può lanciare in maniera
innata i seguenti incantesimi, senza bisogno di componenti
materiali:
A volontà: individuazione del bene e del male, individuazione del
magico, onda tonante
3/giorno ciascuno: camminare nel vento, creare cibo e acqua
(può creare vino al posto dell’acqua), linguaggi
1/giorno ciascuno: creazione, evoca elementali (solo elementale
dell’aria), forma gassosa, immagine maggiore, invisibilità,
spostamento planare
Azioni
Multiattacco. Il djinni effettua tre attacchi di scimitarra.
Scimitarra. Attacco con arma da mischia: +9 a colpire, portata
1,5 m, un bersaglio.
Colpisce: 12 (2d6 + 5) danni taglienti più 3 (1d6) danni da
fulmine o tuono (a scelta del gin).
Creare Turbine. Un cilindro d’aria turbinante di 1,5 metri di
raggio e alto 9 metri si forma magicamente in un punto visibile al
djinni entro 36 metri da esso. Il turbine resta finché il djinni
mantiene la concentrazione (come se si stesse concentrando su di
un incantesimo). Qualsiasi creatura salvo il djinni che entri nel
turbine deve riuscire un tiro salvezza di Forza DC 18 o restare
intralciata da esso. Il djinni può muovere il turbine di massimo
18 metri con un’azione, e le creature intralciate dal turbine si
muovono con esso. Il turbine termina se il djinni lo perde di
vista.
Una creatura può usare la sua azione per liberare una creatura
intralciata dal turbine, compresa se stessa, riuscendo una prova di
Forza DC 18. Se la prova riesce, la creatura non è più intralciata
e si sposta nello spazio più vicino all’esterno del turbine.
Efreeti
Grande elementale, legale malvagio
FORZA 22 (+6)
DESTREZZA 12 (+1)
COSTITUZIONE 24 (+7)
INTELLIGENZA 16 (+3)
SAGGEZZA 15 (+2)
Carisma 16 (+3)
Classe Armatura 17 (armatura naturale)
\hspace*{0pt}\hfill{Punti Ferita}: 200 (16d10 + 112)
Velocità 12 m, volo 18 m
Tiri Salvezza Intelligenza +7, Saggezza +6, Carisma +7
Immunità al Danno fuoco
Sensi scurovisione 36 m, Percezione passiva 12
Linguaggi Ignan
Sfida 11 (7.200 PE)
Decesso Elementale. Se l’efreeti muore, il suo corpo si
disintegra in un lampo di fuoco e uno sbuffo di fumo, lasciando
dietro di sé solo l’equipaggiamento che l’efreeti stava indossando
o trasportando.
Incantesimi Innati. La caratteristica da incantatore innato
dell’efreeti è il Carisma (DC dei tiri salvezza degli incantesimi
15, +7 a colpire con attacchi da incantesimo). Può lanciare in
maniera innata i seguenti incantesimi, senza bisogno di
componenti materiali:
A volontà: individuazione del magico
3/giorno ciascuno: ingrandire/ridurre, linguaggi
1/giorno ciascuno: evoca elementali (solo elementale del fuoco),
forma gassosa, immagine maggiore, invisibilità, muro di fuoco,
spostamento planare
Azioni
Multiattacco. L’efreeti effettua due attacchi di scimitarra o usa
due volte Scagliare Fiamma.
Scimitarra. Attacco con arma da mischia: +10 a colpire, portata
1,5 m, un bersaglio.
Colpisce: 13 (2d6 + 6) danni taglienti più 7 (2d6) danni da fuoco.
Scagliare Fiamma. Attacco con arma a Distanza: +7 a colpire,
gittata 36 m, un bersaglio.
Colpisce: 17 (5d6) danni da fuoco.
.
Ghoul
Ghast
Media non morto, caotico malvagio
FORZA 16 (+3)
DESTREZZA 17 (+3)
COSTITUZIONE 10 (+0)
INTELLIGENZA 11 (+0)
SAGGEZZA 10 (+0)
Carisma 8 (-1)
Classe Armatura 13
\hspace*{0pt}\hfill{Punti Ferita}: 36 (8d8)
Velocità 9 m
Resistenze al Danno necrotico
Immunità al Danno veleno
Immunità alle Condizioni affascinato, avvelenato, sfinimento
Sensi scurovisione 18 m, Percezione passiva 10
Linguaggi Comune
Sfida 2 (450 PE)
Fetore. Qualsiasi creatura che inizi il suo turno entro 1,5 metri dal
ghast deve riuscire un tiro salvezza di Costituzione DC 10 o restare
avvelenata fino all’inizio del suo prossimo turno. Se riesce il tiro
salvezza, la creatura è immune al Fetore del ghast per le successive
24 ore.
Ribellione allo Scacciare. Il ghast e tutti i ghoul entro 9 metri da
esso hanno vantaggio ai tiri salvezza contro gli effetti che
scacciano i non morti.
Azioni
Artigli. Attacco con arma da mischia: +5 a colpire, portata 1,5
m, un bersaglio.
Colpisce: 10 (2d6 + 3) danni taglienti. Se il bersaglio è una
creatura, diversa da un non morto, deve riuscire un tiro salvezza
di Costituzione DC 10 o restare paralizzata per 1 minuto. Il
bersaglio può ripetere il tiro salvezza al termine di ciascun suo
turno, terminando l’effetto se riesce il tiro salvezza.
Morso. Attacco con arma da mischia: +3 a colpire, portata 1,5
m, una creatura.
Colpisce: 12 (2d8 + 3) danni perforanti.
Ghoul
Media non morto, caotico malvagio
FORZA 13 (+1)
DESTREZZA 15 (+2)
COSTITUZIONE 10 (+0)
INTELLIGENZA 7 (-2)
SAGGEZZA 10 (+0)
Carisma 6 (-2)
Classe Armatura 12
\hspace*{0pt}\hfill{Punti Ferita}: 22 (5d8)
Velocità 9 m
Immunità al Danno veleno
Immunità alle Condizioni affascinato, avvelenato, sfinimento
Sensi scurovisione 18 m, Percezione passiva 10
Linguaggi Comune
Sfida 1 (200 PE)
Azioni
Artigli. Attacco con arma da mischia: +4 a colpire, portata 1,5
m, un bersaglio.
Colpisce: 7 (2d4 + 2) danni taglienti. Se il bersaglio è una
creatura, diversa da un elfo o un non morto, deve riuscire un tiro
salvezza di Costituzione DC 10 o restare paralizzata per 1
minuto. Il bersaglio può ripetere il tiro salvezza al termine di
ciascun suo turno, terminando l’effetto se riesce il tiro salvezza.
Morso. Attacco con arma da mischia: +2 a colpire, portata 1,5
m, una creatura.
Colpisce: 9 (2d6 + 2) danni perforanti.
 
Giganti
Gigante di Collina
Enorme gigante, caotico malvagio
FORZA 21 (+5)
DESTREZZA 8 (-1)
COSTITUZIONE 19 (+4)
INTELLIGENZA 5 (-3)
SAGGEZZA 9 (-1)
Carisma 6 (-2)
Classe Armatura 13 (armatura naturale)
\hspace*{0pt}\hfill{Punti Ferita}: 105 (10d12 + 40)
Velocità 12 m
Abilità Percezione +2
Sensi Percezione passiva 12
Linguaggi Gigante
Sfida 5 (1.800 PE)
Azioni
Multiattacco. Il gigante effettua due attacchi con il randello pesante.
Randello Pesante. Attacco con arma da mischia: +8 a colpire,
portata 3 m, un bersaglio.
Colpisce: 18 (3d8 + 5) danni da botta.
Sasso. Attacco con arma a Distanza: +8 a colpire, gittata 18/72
m, un bersaglio.
Colpisce: 21 (3d10 + 5) danni da botta.
Gigante del Fuoco
Enorme gigante, legale malvagio
FORZA 25 (+7)
DESTREZZA 9 (-1)
COSTITUZIONE 23 (+6)
INTELLIGENZA 10 (+0)
SAGGEZZA 14 (+2)
Carisma 13 (+1)
Classe Armatura 18 (armatura di piastre)
\hspace*{0pt}\hfill{Punti Ferita}: 162 (13d12 + 78)
Velocità 9 m
Tiri Salvezza Destrezza +3, Costituzione +10, Carisma +5
Abilità Atletica +11, Percezione +6
Immunità ai Danni fuoco
Sensi Percezione passiva 16
Linguaggi Gigante
Sfida 9 (5.000 PE)
Azioni
Multiattacco. Il gigante effettua due attacchi con lo spadone.
Spadone. Attacco con arma da mischia: +11 a colpire, portata 3
m, un bersaglio.
Colpisce: 28 (6d6 + 7) danni taglienti.
Sasso. Attacco con arma a Distanza: +11 a colpire, gittata 18/72
m, un bersaglio.
Colpisce: 29 (4d10 + 7) danni da botta.
Gigante del Gelo
Enorme gigante, neutrale malvagio
FORZA 23 (+6)
DESTREZZA 9 (-1)
COSTITUZIONE 21 (+5)
INTELLIGENZA 9 (-1)
SAGGEZZA 10 (+0)
Carisma 12 (+1)
Classe Armatura 15 (armatura composita)
\hspace*{0pt}\hfill{Punti Ferita}: 138 (12d12 + 60)
Velocità 12 m
Tiri Salvezza Costituzione +8, Saggezza +3, Carisma +4
Abilità Atletica +9, Percezione +3
Immunità ai Danni freddo
Sensi Percezione passiva 13
Linguaggi Gigante
Sfida 8 (3.900 PE)
Azioni
Multiattacco. Il gigante effettua due attacchi con l’ascia bipenne.
Ascia Bipenne. Attacco con arma da mischia: +9 a colpire,
portata 3 m, un bersaglio.
Colpisce: 25 (3d12 + 6) danni taglienti.
Sasso. Attacco con arma a Distanza: +9 a colpire, gittata 18/72
m, un bersaglio.
Colpisce: 28 (4d10 + 6) danni da botta.
Gigante delle Nuvole
Enorme gigante, neutrale buono (50%) o neutrale malvagio
(50%)
FORZA 27 (+8)
DESTREZZA 10 (+0)
COSTITUZIONE 22 (+6)
INTELLIGENZA 12 (+1)
SAGGEZZA 16 (+3)
Carisma 16 (+3)
Classe Armatura 14 (armatura naturale)
\hspace*{0pt}\hfill{Punti Ferita}: 200 (16d12 + 96)
Velocità 12 m
Tiri Salvezza Costituzione +10, Saggezza +7, Carisma +7
Abilità Intuizione +7, Percezione +7
Sensi Percezione passiva 17
Linguaggi Comune, Gigante
Sfida 9 (5.000 PE)
Incantesimi Innati. La caratteristica da incantatore del gigante è
il Carisma. Il gigante può lanciare questi incantesimi in maniera
innata, senza bisogno di componenti materiali:
A volontà: individuazione del magico, luce, nube di nebbia
3/giorno ciascuno: caduta morbida, passo nebbioso, telecinesi
1/giorno ciascuno: controllare tempo atmosferico, forma gassosa
Olfatto Affinato. Il gigante ha vantaggio alle prove di Saggezza
(Percezione) basate sull’olfatto.
Azioni
Multiattacco. Il gigante effettua due attacchi con la morning star.
Morning star. Attacco con arma da mischia: +12 a colpire,
portata 3 m, un bersaglio.
Colpisce: 21 (3d8 + 8) danni perforanti.
Sasso. Attacco con arma a Distanza: +12 a colpire, gittata 18/72
m, un bersaglio.
Colpisce: 30 (4d10 + 8) danni da botta. 
Gigante di Pietra
Enorme gigante, neutrale
FORZA 23 (+6)
DESTREZZA 15 (+2)
COSTITUZIONE 20 (+5)
INTELLIGENZA 10 (+0)
SAGGEZZA 12 (+1)
Carisma 9 (-1)
Classe Armatura 17 (armatura naturale)
\hspace*{0pt}\hfill{Punti Ferita}: 126 (11d12 + 55)
Velocità 12 m
Tiri Salvezza Destrezza +5, Costituzione +8, Carisma +4
Abilità Atletica +12, Percezione +4
Sensi scurovisione 18 m, Percezione passiva 14
Linguaggi Gigante
Sfida 7 (2.900 PE)
Mimetismo di Pietra. Il gigante ha vantaggio alle prove di Destrezza
(Furtività) effettuate per nascondersi su terreni rocciosi.
Azioni
Multiattacco. Il gigante effettua due attacchi con il randello pesante.
Randello Pesante. Attacco con arma da mischia: +9 a colpire,
portata 4,5 m, un bersaglio.
Colpisce: 19 (3d8 + 6) danni da botta.
Sasso. Attacco con arma a Distanza: +9 a colpire, gittata 18/72
m, un bersaglio.
Colpisce: 28 (4d10 + 6) danni da botta. Se il bersaglio è una
creatura, deve riuscire un tiro salvezza di Forza DC 17 o cadere
prona.
Reazioni
Afferrare Sassi. Se un sasso o un simile oggetto viene scagliato al
gigante, il gigante può, riuscendo un tiro salvezza di Destrezza DC
10, afferrare il proiettile e non subire danni da botta da esso.
Gigante delle Tempeste
Enorme gigante, caotico buono
FORZA 29 (+9)
DESTREZZA 14 (+2)
COSTITUZIONE 20 (+5)
INTELLIGENZA 16 (+3)
SAGGEZZA 18 (+4)
Carisma 18 (+4)
Classe Armatura 16 (armatura di scaglie)
\hspace*{0pt}\hfill{Punti Ferita}: 230 (20d12 + 100)
Velocità 15 m, nuoto 15 m
Tiri Salvezza Forza +14, Costituzione +10, Saggezza +9,
Carisma +9
Abilità Arcano +8, Atletica +14, Percezione +9, Storia +8
Resistenze al Danno freddo
Immunità al Danno fulmine, tuono
Sensi Percezione passiva 19
Linguaggi Comune, Gigante
Sfida 13 (10.000 PE)
Anfibio. Il gigante può respirare aria e acqua.
Incantesimi Innati. La caratteristica da incantatore del gigante è
il Carisma (DC dei tiri salvezza degli incantesimi 17). Il gigante
può lanciare questi incantesimi in maniera innata, senza bisogno
di componenti materiali:
A volontà: caduta controllata, individuazione del magico,
levitazione, luce
3/giorno ciascuno: controllare tempo atmosferico, respirare
sott'acqua
Azioni
Multiattacco. Il gigante effettua due attacchi con lo spadone.
Spadone. Attacco con arma da mischia: +14 a colpire, portata 3
m, un bersaglio.
Colpisce: 30 (6d6 + 9) danni taglienti.
Sasso. Attacco con arma a Distanza: +14 a colpire, gittata 18/72
m, un bersaglio.
Colpisce: 35 (4d12 + 9) danni da botta.
Colpo Fulminante (Ricarica 5-6). Il gigante scaglia una folgore
magica ad un punto visibile entro 150 metri da sé. Ogni creatura
entro 3 metri da quel punto deve effettuare un tiro salvezza di
Destrezza DC 17, subendo 54 (12d8) danni da fulmine se lo
fallisce, o la metà se lo supera.
 
Gnoll
Media umanoide (gnoll), caotico malvagio
FORZA 14 (+2)
DESTREZZA 12 (+1)
COSTITUZIONE 11 (+0)
INTELLIGENZA 6 (-2)
SAGGEZZA 10 (+0)
Carisma 7 (-2)
Classe Armatura 15 (armatura di pelle, scudo)
\hspace*{0pt}\hfill{Punti Ferita}: 22 (5d8)
Velocità 9 m
Sensi scurovisione 18 m, Percezione passiva 10
Linguaggi Gnoll
Sfida 1/2 (100 PE)
Rabbia. Quando lo gnoll riduce una creatura a 0 punti ferita con
un attacco da mischia durante il proprio round, può svolgere
un’azione bonus per muoversi fino a metà della sua velocità ed
effettuare un attacco di morso.
Azioni
Morso. Attacco con arma da mischia: +4 a colpire, portata 1,5
m, una creatura.
Colpisce: 4 (1d4 + 2) danni perforanti.
Lancia. Attacco con arma da mischia o a Distanza: +4 a colpire,
portata 1,5 m o gittata 6/18 m, un bersaglio.
Colpisce: 5 (1d6 + 2) danni perforanti o 6 (1d8 + 2) danni
perforanti se usata con due mani per effettuare un attacco da
mischia.
Arco Lungo. Attacco con arma a Distanza: +3 a colpire, gittata
45/180 m, un bersaglio.
Colpisce: 5 (1d8 + 1) danni perforanti.
Gnomo delle
Profondità (Svirfneblin)
Piccola umanoide (gnomo), neutrale buono
FORZA 15 (+2)
DESTREZZA 14 (+2)
COSTITUZIONE 14 (+2)
INTELLIGENZA 12 (+1)
SAGGEZZA 10 (+0)
Carisma 9 (-1)
Classe Armatura 15 (giaco di maglia)
\hspace*{0pt}\hfill{Punti Ferita}: 16 (3d6 + 6)
Velocità 6 m
Abilità Furtività +4, Indagare +3, Percezione +2
Sensi scurovisione 36 m, Percezione passiva 12
Linguaggi Gnomesco, Sottocomune, Terran
Sfida 1/2 (100 PE)
Astuzia Gnomesca. Lo gnomo ha +1d6 ai tiri salvezza di
Intelligenza, Saggezza e Carisma contro la magia.
Camuffamento di Pietra. Lo gnomo ha vantaggio alle prove di
Destrezza (Furtività) effettuate per nascondersi su terreni
rocciosi.
Incantesimi Innati. La caratteristica da incantatore innato dello
gnomo è l’Intelligenza (DC dei tiri salvezza 11). Lo gnomo può
lanciare questi incantesimi in maniera innata, senza bisogno di
componenti:
A volontà: anti-individuazione (personale)
1/giorno ciascuno: camuffare sé stesso, cecità/sordità, sfocatura
Azioni
Piccone da Guerra. Attacco con arma da mischia: +4 a colpire,
portata 1,5 m, un bersaglio.
Colpisce: 6 (1d8 + 2) danni perforanti.
Dardo Avvelenato. Attacco con arma a Distanza: +4 a colpire,
gittata 9/36 m, un bersaglio.
Colpisce: 4 (1d4 + 2) danni perforanti, e il bersaglio deve
riuscire un tiro salvezza di Costituzione DC 12 o restare
avvelenato per 1 minuto. Il bersaglio può ripetere il tiro salvezza
al termine di ciascun suo turno, terminando l’effetto su di sé in
caso di successo.
Golem
Golem di Argilla
Grande costrutto, disallineato
FORZA 20 (+5)
DESTREZZA 9 (-1)
COSTITUZIONE 18 (+4)
INTELLIGENZA 3 (-4)
SAGGEZZA 8 (-1)
Carisma 1 (-5)
Classe Armatura 14 (armatura naturale)
\hspace*{0pt}\hfill{Punti Ferita}: 133 (14d10 + 56)
Velocità 6 m
Immunità al Danno acido, psichico, veleno; da botta,
perforante e tagliente di attacchi non magici che non siano di
adamantio
Immunità alle Condizioni affascinato, avvelenato, paralizzato,
pietrificato, sfinimento, spaventato
Sensi scurovisione 18 m, Percezione passiva 9
Linguaggi comprende le lingue del suo creatore ma non può
parlare
Sfida 9 (5.000 PE)
Berserk. Ogni volta che il golem inizia il suo turno con 60 punti
ferita o meno, tira un d6. Se ottieni 6, il golem va in berserk.
Durante ogni suo turno mentre è in berserk, il golem attacca la
creatura più vicina che può vedere. Se non c’è nessuna creatura
abbastanza vicina da muoversi e attaccarla, il golem attacca un
oggetto, con preferenza per gli oggetti più piccoli di lui. Una
volta che il golem è andato in berserk, continuerà ad esserlo
finché non viene distrutto o recupera tutti i suoi punti ferita.
Armi Magiche. Gli attacchi con armi del golem sono magici.
Assorbimento dell’Acido. Ogni volta che il golem è vittima di
danni da acido, non subisce danni ma invece recupera un pari
numero di punti ferita.
Forma Immutabile. Il golem è immune a qualsiasi incantesimo o
effetto che altererebbe la sua forma.
Natura di Costrutto. Un golem non ha bisogno di aria, cibo,
bevande o sonno.
Resistenza alla Magia. Il golem ha +1d6 ai tiri salvezza
contro incantesimi e altri effetti magici.
Azioni
Multiattacco. Il golem effettua due attacchi di schianto.
Schianto. Attacco con arma da mischia: +8 a colpire, portata 1,5
m, un bersaglio.
Colpisce: 16 (2d10 + 5) danni da botta. Se il bersaglio è una
creatura, deve riuscire un tiro salvezza di Costituzione DC 15 o
vedere i suoi punti ferita massimi ridotti di un ammontare pari al
danno subito. Il bersaglio muore se l’attacco riduce i suoi punti
ferita massimi a 0. La riduzione resta finché non viene rimossa
dall’incantesimo ristorare superiore o altra magia.
Velocità (Ricarica 5-6). Fino al termine del suo prossimo turno,
il golem ottiene un bonus magico di +2 alla Difesa, ha vantaggio ai
tiri salvezza di Destrezza, e può usare gli attacchi di schianto
come azione bonus.
Golem di Carne
Media costrutto, neutrale
FORZA 19 (+4)
DESTREZZA 9 (-1)
COSTITUZIONE 18 (+4)
INTELLIGENZA 6 (-2)
SAGGEZZA 10 (+0)
Carisma 5 (-3)
Classe Armatura 9
\hspace*{0pt}\hfill{Punti Ferita}: 93 (11d8 + 44)
Velocità 9 m
Immunità al Danno fulmine, veleno; da botta, perforante e
tagliente di attacchi non magici che non siano di adamantio
Immunità alle Condizioni affascinato, avvelenato, paralizzato,
pietrificato, sfinimento, spaventato
Sensi scurovisione 18 m, Percezione passiva 10
Linguaggi comprende le lingue del suo creatore ma non può
parlare
Sfida 5 (1.800 PE)
Berserk. Ogni volta che il golem inizia il suo turno con 40 punti
ferita o meno, tira un d6. Se ottieni 6, il golem va in berserk.
Durante ogni suo turno mentre è in berserk, il golem attacca la
creatura più vicina che possa vedere. Se non c’è nessuna creatura
abbastanza vicina da muoversi e attaccarla, il golem attacca un
oggetto, con preferenza per gli oggetti più piccoli di lui. Una
volta che il golem è andato in berserk, continuerà ad esserlo
finché non viene distrutto o recupera tutti i suoi punti ferita.
Armi Magiche. Gli attacchi con armi del golem sono magici.
Assorbimento dei Fulmini. Ogni volta che il golem sia vittima di
un danno da fulmine, non subisce danni ma invece recupera un
pari numero di punti ferita.
Avversione al Fuoco. Se il golem subisce danni da fuoco, ha
svantaggio ai tiri di attacco e le prove di abilità fino alla fine del
suo prossimo turno.
Forma Immutabile. Il golem è immune a qualsiasi incantesimo o
effetto che altererebbe la sua forma.
Natura di Costrutto. Un golem non ha bisogno di aria, cibo,
bevande o sonno.
Resistenza alla Magia. Il golem ha +1d6 ai tiri salvezza
contro incantesimi e altri effetti magici.
Azioni
Multiattacco. Il golem effettua due attacchi di schianto.
Schianto. Attacco con arma da mischia: +7 a colpire, portata 1,5
m, un bersaglio.
Colpisce: 13 (2d8 + 4) danni da botta.
 
Golem di Ferro
Grande costrutto, disallineato
FORZA 24 (+7)
DESTREZZA 9 (-1)
COSTITUZIONE 20 (+5)
INTELLIGENZA 3 (-4)
SAGGEZZA 11 (+0)
Carisma 1 (-5)
Classe Armatura 20 (armatura naturale)
\hspace*{0pt}\hfill{Punti Ferita}: 210 (20d10 + 100)
Velocità 9 m
Immunità al Danno fuoco, psichico, veleno; da botta,
perforante e tagliente di attacchi non magici che non siano di
adamantio
Immunità alle Condizioni affascinato, avvelenato, paralizzato,
pietrificato, sfinimento, spaventato
Sensi scurovisione 36 m, Percezione passiva 10
Linguaggi comprende le lingue del suo creatore ma non può
parlare
Sfida 16 (15.000 PE)
Armi Magiche. Gli attacchi con armi del golem sono magici.
Assorbimento del Fuoco. Ogni volta che il golem sia vittima di
un danno da fuoco, non subisce danni ma invece recupera un pari
numero di punti ferita.
Forma Immutabile. Il golem è immune a qualsiasi incantesimo o
effetto che altererebbe la sua forma.
Natura di Costrutto. Un golem non ha bisogno di aria, cibo,
bevande o sonno.
Resistenza alla Magia. Il golem ha +1d6 ai tiri salvezza
contro incantesimi e altri effetti magici.
Azioni
Multiattacco. Il golem effettua due attacchi da mischia.
Schianto. Attacco con arma da mischia: +13 a colpire, portata
1,5 m, un bersaglio.
Colpisce: 20 (3d8 + 7) danni da botta.
Spada. Attacco con arma da mischia: +13 a colpire, portata 3 m,
un bersaglio.
Colpisce: 23 (3d10 + 7) danni taglienti.
Soffio Velenoso (Ricarica 6). Il golem esala un gas velenoso in
un cono di 4,5 metri. Ogni creatura in quell’area deve effettuare
un tiro salvezza di Costituzione DC 19, subendo 45 (10d8) danni
da veleno se fallisce il tiro salvezza, o la metà di questi danni se
lo riesce.
Golem di Pietra
Grande costrutto, disallineato
FORZA 22 (+6)
DESTREZZA 9 (-1)
COSTITUZIONE 20 (+5)
INTELLIGENZA 3 (-4)
SAGGEZZA 11 (+0)
Carisma 1 (-5)
Classe Armatura 17 (armatura naturale)
\hspace*{0pt}\hfill{Punti Ferita}: 178 (17d10 + 85)
Velocità 9 m
Immunità al Danno psichico, veleno; da botta, perforante e
tagliente di attacchi non magici che non siano di adamantio
Immunità alle Condizioni affascinato, avvelenato, paralizzato,
pietrificato, sfinimento, spaventato
Sensi scurovisione 36 m, Percezione passiva 10
Linguaggi comprende le lingue del suo creatore ma non può
parlare
Sfida 10 (5.900 PE)
Armi Magiche. Gli attacchi con armi del golem sono magici.
Forma Immutabile. Il golem è immune a qualsiasi incantesimo o
effetto che altererebbe la sua forma.
Natura di Costrutto. Un golem non ha bisogno di aria, cibo,
bevande o sonno.
Resistenza alla Magia. Il golem ha +1d6 ai tiri salvezza
contro incantesimi e altri effetti magici.
Azioni
Multiattacco. Il golem effettua due attacchi di schianto.
Schianto. Attacco con arma da mischia: +10 a colpire, portata
1,5 m, un bersaglio.
Colpisce: 19 (3d8 + 6) danni da botta.
Lentezza (Ricarica 5-6). Il golem prende a bersaglio una o più
creature entro 3 metri da lui e che possa vedere. Ciascun
bersaglio deve effettuare un tiro salvezza di Saggezza DC 17
contro questa magia. Se fallisce il tiro salvezza, il bersaglio non
può usare reazioni, ha la velocità dimezzata, e durante il proprio
turno non può effettuare più di un attacco. Inoltre, durante il
proprio round il bersaglio può effettuare un’azione o un’azione
bonus, ma non entrambe. Questi effetti durano per 1 minuto. Il
bersaglio può ripetere il tiro salvezza al termine di ciascun suo
turno, terminando l’effetto per sé, in caso di successo.
Gorgone
Grande mostruosità, disallineato
FORZA 20 (+5)
DESTREZZA 11 (+0)
COSTITUZIONE 18 (+4)
INTELLIGENZA 2 (-4)
SAGGEZZA 12 (+1)
Carisma 7 (-2)
Classe Armatura 19 (armatura naturale)
\hspace*{0pt}\hfill{Punti Ferita}: 114 (12d10 + 48)
Velocità 12 m
Abilità Percezione +4
Immunità alle Condizioni Pietrificato
Sensi scurovisione 18 m, Percezione passiva 14
Linguaggi -
Sfida 5 (1.800 PE)
Carica Travolgente. Se la gorgone si muove di almeno 6 metri in
linea retta verso il bersaglio e lo colpisce con un attacco di incornata
durante lo stesso turno, il bersaglio deve riuscire un tiro salvezza di
Forza DC 16 o cadere prono. Se il bersaglio è prono, la gorgone può
effettuare un attacco di zoccoli contro di lui come azione bonus.
Azioni
Incornata. Attacco con arma da mischia: +8 a colpire, portata
1,5 m, un bersaglio.
Colpisce: 18 (2d12 + 5) danni perforanti.
Zoccoli. Attacco con arma da mischia: +8 a colpire, portata 1,5
m, un bersaglio.
Colpisce: 16 (2d10 + 5) danni da botta.
Soffio Pietrificante (Ricarica 5-6). La gorgone esala un gas
pietrificante in un cono di 9 metri. Ogni creatura in quell’area
deve riuscire un tiro salvezza di Costituzione DC 13. Se il tiro
salvezza fallisce, il bersaglio inizia a trasformarsi in pietra ed è
intralciato. Il bersaglio intralciato deve ripetere il tiro salvezza al
termine del suo prossimo turno. Se lo riesce, l’effetto sul
bersaglio ha termine. Se lo fallisce, il bersaglio è pietrificato
finché non viene liberato dall’incantesimo ripristino superiore o
simile magia.
Grick
Media mostruosità, neutrale
FORZA 14 (+2)
DESTREZZA 14 (+2)
COSTITUZIONE 11 (+0)
INTELLIGENZA 3 (-4)
SAGGEZZA 14 (+2)
Carisma 5 (-3)
Classe Armatura 14 (armatura naturale)
\hspace*{0pt}\hfill{Punti Ferita}: 27 (6d8)
Velocità 9 m, scalata 9 m
Resistenza al Danno da botta, perforante e tagliente di
attacchi non magici
Sensi scurovisione 18 m, Percezione passiva 12
Linguaggi -
Sfida 2 (450 PE)
Camuffamento di Pietra. Il grick ha vantaggio alle prove di
Destrezza (Furtività) per nascondersi su terreno roccioso.
Azioni
Multiattacco. Il grick effettua un attacco con i suoi tentacoli. Se
l’attacco colpisce, il grick può effettuare un attacco di becco
contro lo stesso bersaglio.
Tentacoli. Attacco con arma da mischia: +4 a colpire, portata 1,5
m, un bersaglio.
Colpisce: 9 (2d6 + 2) danni taglienti.
Becco. Attacco con arma da mischia: +4 a colpire, portata 1,5 m,
un bersaglio.
Colpisce: 5 (1d6 + 2) danni perforanti.
 
Grifone
Grande mostruosità, disallineato
FORZA 18 (+4)
DESTREZZA 15 (+2)
COSTITUZIONE 16 (+3)
INTELLIGENZA 2 (-4)
SAGGEZZA 13 (+1)
Carisma 8 (-1)
Classe Armatura 12
\hspace*{0pt}\hfill{Punti Ferita}: 59 (7d10 + 21)
Velocità 9 m, volo 24 m
Abilità Percezione +5
Sensi scurovisione 18 m, Percezione passiva 15
Linguaggi -
Sfida 2 (450 PE)
Vista Affinata. Il grifone ha vantaggio nelle prove di Saggezza
(Percezione) basate sulla vista.
Azioni
Multiattacco. Il grifone effettua due attacchi: uno con il becco e
uno con gli artigli.
Artigli. Attacco con arma da mischia: +6 a colpire, portata 1,5
m, un bersaglio.
Colpisce: 11 (2d6 + 4) danni taglienti.
Becco. Attacco con arma da mischia: +6 a colpire, portata 1,5 m,
un bersaglio.
Colpisce: 8 (1d8 + 4) danni perforanti.
Grimlock
Media umanoide (grimlock), neutrale malvagio
FORZA 16 (+3)
DESTREZZA 12 (+1)
COSTITUZIONE 12 (+1)
INTELLIGENZA 9 (-1)
SAGGEZZA 8 (-1)
Carisma 6 (-2)
Classe Armatura 11
\hspace*{0pt}\hfill{Punti Ferita}: 11 (2d8 + 2)
Velocità 9 m
Abilità Atletica +5, Furtività +3, Percezione +3
Immunità alle Condizioni accecato
Sensi vista cieca 9 m o 3 m se assordato (cieco oltre questo
raggio), Percezione passiva 13
Linguaggi Sottocomune
Sfida 1/4 (50 PE)
Camuffamento di Pietra. Il grimlock ha vantaggio alle prove di
Destrezza (Furtività) effettuate per nascondere su terreni
rocciosi.
Sensi Ciechi. Il grimlock non può usare la vista cieca mentre è
assordato e non più fiutare.
Olfatto e Udito Affinati. Il grimlock ha vantaggio alle prove di
Saggezza (Percezione) basate su udito o olfatto.
Azioni
Randello d’Osso Appuntito. Attacco con arma da mischia: +5 a
colpire, portata 1,5 m, un bersaglio.
Colpisce: 5 (1d4 + 3) danni da botta più 2 (1d4) danni
perforanti.
Arco Lungo. Attacco con arma a Distanza: +3 a colpire, gittata
45/180 m, un bersaglio.
Colpisce: 5 (1d8 + 1) danni perforanti.
Guardiano Protettore
Grande costrutto, disallineato
FORZA 18 (+4)
DESTREZZA 8 (-1)
COSTITUZIONE 18 (+4)
INTELLIGENZA 7 (-2)
SAGGEZZA 10 (+0)
Carisma 3 (-4)
Classe Armatura 17 (armatura naturale)
\hspace*{0pt}\hfill{Punti Ferita}: 142 (15d10 + 60)
Velocità 9 m
Immunità al Danno veleno
Immunità alle Condizioni affascinato, avvelenato, paralizzato,
sfinimento, spaventato
Sensi scurovisione 18 m, vista cieca 3 m, Percezione passiva 10
Linguaggi comprende i comandi forniti in qualsiasi lingua ma
non può parlare
Sfida 7 (2.900 PE)
Accumulare Incantesimi. Un incantatore che indossi l’amuleto del
guardiano protettore può far sì che il guardiano accumuli un
incantesimo di 4° livello o più basso. Per farlo, l’incantatore deve
lanciare l’incantesimo sul guardiano. L’incantesimo non ha effetto
ma viene accumulato all’interno del guardiano. Quando gli viene
comandato di farlo da chi indossa l’amuleto o si presenta una
situazione predeterminata dall’incantatore, il guardiano lancia
l’incantesimo accumulato con tutti i parametri predisposti
dall’incantatore originale, senza bisogno di componenti. Quando
l’incantesimo viene lanciato o qualsiasi nuovo incantesimo viene
accumulato, tutti gli incantesimi precedentemente accumulati
vengono persi.
Natura di Costrutto. Il guardiano non ha bisogno di aria, cibo,
bevande o sonno.
Rigenerazione. Il guardiano protettore recupera 10 punti ferita
all’inizio del proprio round se ne possiede ancora almeno 1.
Vincolato. Il guardiano protettore è vincolato magicamente ad un
amuleto. Finché il guardiano e l’amuleto sono sullo stesso piano di
esistenza, chi indossa l’amuleto può richiamare telepaticamente il
guardiano perché lo raggiunga, e il guardiano saprà la distanza e la
direzione in cui si trova l’amuleto. Se il guardiano si trova entro 18
metri da chi indossa l’amuleto, metà dei danni subiti da chi lo
indossa (arrotondati per difetto) vengono trasferiti al guardiano. Se
l’amuleto viene distrutto, il guardiano è inabile finché non viene
creato un amuleto di rimpiazzo. L’amuleto del guardiano può essere
soggetto ad un attacco diretto qualora non sia indossato o trasportato
da nessuno. Ha Difesa 10, 10 punti ferita e immunità ai danni psichici e
da veleno. Costruire un amuleto richiede 1 settimana e costa 10.000
mo in componenti.
Azioni
Multiattacco. Il golem effettua due attacchi di pugno.
Pugno. Attacco con arma da mischia: +7 a colpire, portata 1,5
m, un bersaglio.
Colpisce: 11 (2d6 + 4) danni da botta.
Reazioni
Scudo. Quando una creatura attacca chi indossa l’amuleto del
guardiano, il guardiano conferisce un bonus di +2 alla sua Difesa, se
entro 1,5 metri dal suo controllore.
Hobgoblin
Media umanoide (goblinoide), legale malvagio
FORZA 13 (+1)
DESTREZZA 12 (+1)
COSTITUZIONE 12 (+1)
INTELLIGENZA 10 (+0)
SAGGEZZA 10 (+0)
Carisma 9 (-1)
Classe Armatura 18 (armatura di maglia, scudo)
\hspace*{0pt}\hfill{Punti Ferita}: 11 (2d8 + 2)
Velocità 9 m
Sensi scurovisione 18 m, Percezione passiva 10
Linguaggi Comune, Goblin
Sfida 1/2 (100 PE)
Vantaggio Marziale. Una volta per turno, l’hobgoblin può
infliggere 7 (2d6) danni aggiuntivi ad una creatura che colpisce
con un attacco con arma, se quella creatura si trova entro 1,5
metri da un alleato dell’hobgoblin che non sia inabile.
Azioni
Spada Lunga. Attacco con arma da mischia: +3 a colpire,
portata 1,5 m, un bersaglio.
Colpisce: 5 (1d8 + 1) danni taglienti o 6 (1d10 + 1) danni
taglienti se usata con due mani.
Arco Lungo. Attacco con arma a Distanza: +3 a colpire, gittata
45/180 m, un bersaglio.
Colpisce: 5 (1d8 + 1) danni perforanti.
 
Idra
Enorme mostruosità, disallineato
FORZA 20 (+5)
DESTREZZA 12 (+1)
COSTITUZIONE 20 (+5)
INTELLIGENZA 2 (-4)
SAGGEZZA 10 (+0)
Carisma 7 (-2)
Classe Armatura 15 (armatura naturale)
\hspace*{0pt}\hfill{Punti Ferita}: 172 (15d12 + 75)
Velocità 9 m, nuoto 9 m
Abilità Percezione +6
Sensi scurovisione 18 m, Percezione passiva 16
Linguaggi -
Sfida 8 (3.900 PE)
Teste Multiple. L’idra ha cinque teste. Finché ha più di una testa,
l’idra ha +1d6 ai tiri salvezza contro le condizioni accecata,
affascinata, assordata, spaventata, stordita o svenuta.
Ogni volta che l’idra subisce 25 o più danni in un singolo turno,
una delle sue teste muore. Se tutte le teste muoiono, anche l’idra
muore.
Al termine del suo turno, l’idra ricresce due teste per ciascuna
delle sue teste uccise dal suo ultimo turno, a meno che non abbia
subito danno da fuoco dal suo ultimo turno. L’idra recupera 10
punti ferita per ogni testa ricresciuta in questo modo.
Teste Reattive. Per ogni testa posseduta oltre la prima, l’idra
riceve una reazione extra che può essere usata solo per compiere
attacchi di opportunità.
Trattenere il Fiato. L’idra può trattenere il fiato per 1 ora.
Veglia. Mentre l’idra dorme, almeno una delle sue teste resta
sveglia.
Azioni
Multiattacco. L’idra effettua tanti attacchi di morso quante sono
le sue teste.
Morso. Attacco con arma da mischia: +8 a colpire, portata 3 m,
un bersaglio.
Colpisce: 10 (1d10 + 5) danni perforanti.
Ippogrifo
Grande mostruosità, disallineato
FORZA 17 (+3)
DESTREZZA 13 (+1)
COSTITUZIONE 13 (+1)
INTELLIGENZA 2 (-4)
SAGGEZZA 12 (+1)
Carisma 8 (-1)
Classe Armatura 11
\hspace*{0pt}\hfill{Punti Ferita}: 19 (3d10 + 3)
Velocità 12 m, volo 18 m
Abilità Percezione +5
Sensi Percezione passiva 15
Linguaggi -
Sfida 1 (200 PE)
Vista Affinata. L’ippogrifo ha vantaggio nelle prove di Saggezza
(Percezione) basate sulla vista.
Azioni
Multiattacco. L’ippogrifo effettua due attacchi: uno con il becco
e uno con gli artigli.
Artigli. Attacco con arma da mischia: +5 a colpire, portata 1,5
m, un bersaglio.
Colpisce: 10 (2d6 + 3) danni taglienti.
Becco. Attacco con arma da mischia: +5 a colpire, portata 1,5 m,
un bersaglio.
Colpisce: 8 (1d10 + 3) danni perforanti.
Kraken
Mastodontica mostruosità (titano), caotico malvagio
FORZA 30 (+10)
DESTREZZA 11 (+0)
COSTITUZIONE 25 (+7)
INTELLIGENZA 22 (+6)
SAGGEZZA 18 (+4)
Carisma 20 (+5)
Classe Armatura 18 (armatura naturale)
\hspace*{0pt}\hfill{Punti Ferita}: 472 (27d20 + 189)
Velocità 6 m, nuoto 18 m
Tiri Salvezza Forza +17, Destrezza +7, Costituzione +14,
Intelligenza +13, Saggezza +11
Immunità al Danno fulmine; da botta, perforante e tagliente
di attacchi non magici
Immunità alle Condizioni paralizzato, spaventato
Sensi visione del vero 36 m, Percezione passiva 14
Linguaggi comprende Abissale, Celestiale, Infernale e
Primordiale ma non può parlare, telepatia 36 m
Sfida 23 (50.000 PE)
Anfibio. Il kraken può respirare aria e acqua.
Libertà di Movimento. Il kraken ignora i terreni difficili, e gli
effetti magici non possono ridurne la velocità o far sì che diventi
intralciato. Può spendere 1,5 metri di movimento per liberarsi
dalle restrizioni non magiche o dall’essere afferrato.
Mostro d’Assedio. Il kraken infligge danni doppi agli oggetti e le
strutture.
Azioni
Multiattacco. Il kraken effettua tre attacchi di tentacolo, ciascuno
dei quali può essere rimpiazzato da un uso di Fiondare.
Morso. Attacco con arma da mischia: +17 a colpire, portata 1,5
m, un bersaglio.
Colpisce: 23 (3d8 + 10) danni perforanti. Se il bersaglio è una
creatura di taglia Grande o inferiore afferrato dal kraken, quella
creatura viene inghiottita, e l’afferrare ha termine. Mentre è
inghiottita, la creatura è accecata e intralciata, ha copertura totale
contro gli attacchi e altri effetti provenienti dall’esterno del
kraken, e subisce 42 (12d6) danni da acido all’inizio di ciascun
turno del kraken.
Se il kraken subisce 50 o più danni in un singolo turno da una
creatura al suo interno, il kraken deve riuscire un tiro salvezza di
Costituzione DC 25 o vomitare tutte le creature inghiottite, che
cadono prone in uno spazio entro 3 metri dal kraken. Se il kraken
muore, una creatura inghiottita non risulta più intralciata da esso
e può fuggire dal cadavere usando 4,5 metri di movimento,
uscendo prona.
Tentacolo. Attacco con arma da mischia: +17 a colpire, portata 9
m, un bersaglio.
Colpisce: 20 (3d6 + 10) danni da botta, e il bersaglio è
afferrato (DC 18 per fuggire). Fino al termine dell’afferrare, il
bersaglio è intralciato. Il kraken ha dieci tentacoli, ciascuno dei
quali può afferrare un bersaglio.
Fiondare. Un oggetto impugnato o una creatura afferrata dal
kraken, di taglia Grande o inferiore viene lanciato di 18 metri in
una direzione casuale e gettata prona. Se il bersaglio lanciato
colpisce una superficie solida, subisce 3 (1d6) danni da botta
per ogni 3 metri percorsi. Se il bersaglio viene lanciato contro
un’altra creatura, quella creatura deve riuscire un tiro salvezza di
Destrezza DC 18 o subire lo stesso danno e cadere prona.
Tempesta di Fulmini. Il kraken crea magicamente tre saette di
energia, ciascuna delle quali può colpire un bersaglio entro 36
metri e che il kraken possa vedere. Il bersaglio deve effettuare un
tiro salvezza di Destrezza DC 23, e subire 22 (4d10) danni da
fulmine se fallisce il tiro salvezza, o la metà se lo riesce.
Azioni Leggendarie
Il kraken può effettuare 3 azioni aggiuntive, scelte tra le opzioni
seguenti. Può usare solo un’opzione leggendaria alla volta e solo
al termine del turno di un’altra creatura. Il kraken recupera le
azioni aggiuntive spese all’inizio del proprio round.
Attacco di Tentacolo o Fiondare. Il kraken effettua un attacco
di tentacolo o usa Fiondare.
Nube di Inchiostro (Costa 3 Azioni). Mentre si trova
sott’acqua, il kraken espelle una nube di inchiostro con un raggio
di 18 metri. La nube si propaga intorno agli angoli, e quell’area è
oscurata pesantemente per tutte le creature tranne il kraken.
Ciascuna creatura a parte il kraken che termini il suo turno
nell’area deve riuscire un tiro salvezza di Costituzione DC 23,
subendo 16 (3d10) danni da veleno se fallisce il tiro salvezza, o
la metà se lo riesce. Una forte corrente disperde la nube, che
altrimenti svanisce al termine del prossimo turno del kraken.
Tempesta di Fulmini (Costa 2 Azioni). Il kraken usa Tempesta
di Fulmini.
 
Lamia
Grande mostruosità, caotico malvagio
FORZA 16 (+3)
DESTREZZA 13 (+1)
COSTITUZIONE 15 (+2)
INTELLIGENZA 14 (+2)
SAGGEZZA 15 (+2)
Carisma 16 (+3)
Classe Armatura 13 (armatura naturale)
\hspace*{0pt}\hfill{Punti Ferita}: 97 (13d10 + 26)
Velocità 9 m
Abilità Furtività +3, Inganno +7, Intuizione +4,
Sensi scurovisione 18 m, Percezione passiva 12
Linguaggi Abissale, Comune
Sfida 4 (1.100 PE)
Incantesimi Innati. La caratteristica da incantatore innato della
lamia è il Carisma (DC dei tiri salvezza 13). La lamia può
lanciare in maniera innata i seguenti incantesimi, senza bisogno
di componenti materiali:
A volontà: camuffare sé stesso (qualsiasi forma umanoide),
immagine maggiore
3/Giorno ciascuno: charme su persone, immagine speculare,
scrutare, suggestione
1/Giorno: restrizione
Azioni
Multiattacco. La lamia effettua due attacchi: uno con gli artigli e
uno con il pugnale o il Tocco Intossicante.
Artigli. Attacco con arma da mischia: +5 a colpire, portata 1,5
m, un bersaglio.
Colpisce: 14 (2d10 + 3) danni taglienti.
Pugnale. Attacco con arma da mischia: +5 a colpire, portata 1,5
m, un bersaglio.
Colpisce: 5 (1d4 + 3) danni perforanti.
Tocco Intossicante. Attacco con incantesimo in mischia: +5 a
colpire, portata 1,5 m, una creatura.
Colpisce: Il bersaglio è maledetto per 1 ora da questa magia.
Fino al termine della maledizione, il bersaglio ha svantaggio ai
tiri salvezza di Saggezza e a tutte le prove di abilità.
Lich
Media non morto, qualsiasi allineamento malvagio
FORZA 11 (+0)
DESTREZZA 16 (+3)
COSTITUZIONE 16 (+3)
INTELLIGENZA 20 (+5)
SAGGEZZA 14 (+2)
Carisma 16 (+3)
Classe Armatura 17 (armatura naturale)
\hspace*{0pt}\hfill{Punti Ferita}: 135 (18d8 + 54)
Velocità 9 m
Tiri Salvezza Costituzione +10, Intelligenza +12, Saggezza +9
Resistenze al Danno freddo, fulmine, necrotico
Immunità al Danno veleno; da botta, perforante e tagliente
di attacchi non magici
Immunità alle Condizioni affascinato, avvelenato, paralizzato,
sfinimento, spaventato
Sensi visione del vero 36 m, Percezione passiva 19
Linguaggi Comune più altre cinque lingue
Sfida 21 (33.000 PE)
Incantesimi. Il lich è un incantatore di 18° livello. La sua
caratteristica da incantatore è l’Intelligenza (DC dei tiri salvezza
degli incantesimi 20, +3 a colpire con attacchi da incantesimo). Il
lich ha preparati i seguenti incantesimi da mago:
Trucchetti (a volontà): mano magica, prestidigitazione, raggio
di gelo
1° livello (4 slot): dardo incantato, individuazione del magico,
onda tonante, scudo
2° livello (3 slot): freccia acida, immagine speculare,
individuazione dei pensieri, invisibilità
3° livello (3 slot): animare morti, controincantesimo, dissolvi
magie, palla di fuoco
4° livello (3 slot): inaridire, porta dimensionale
5° livello (3 slot): nube mortale, scrutare
6° livello (1 slot): disintegrazione, globo di invulnerabilità
7° livello (1 slot): dito della morte, spostamento planare
8° livello (1 slot): dominare mostri, parola del potere stordire
9° livello (1 slot): parola del potere uccidere
Natura Non Morta. Il lich non ha bisogno di aria, cibo, bevande
o sonno.
Resistenza Leggendaria (3/Giorno). Se il lich fallisce un tiro
salvezza, può scegliere invece di riuscirvi.
Resistenza allo Scacciare. Il lich ha +1d6 ai tiri salvezza
contro gli effetti che scacciano i non morti.
Rinvigorimento. Se possiede un filatterio, il lich distrutto ottiene
un nuovo corpo in 1d10 giorni, recuperando tutti i suoi punti
ferita e ritornando in attività. Il nuovo corpo compare entro 1,5
metri dal filatterio.
Sacrifici di Anime. Un lich deve periodicamente nutrire di anime
il suo filatterio per sostenere la magia che mantiene il suo corpo
e la sua coscienza. Per farlo usa l’incantesimo imprigionare.
Invece di scegliere una delle normali opzioni dell’incantesimo, il
lich lo impiega per intrappolare magicamente il corpo e l’anima
del bersaglio all’interno del filatterio. Il filatterio deve trovarsi
sullo stesso piano del lich, perché questo incantesimo funzioni. Il
filatterio di un lich può contenere solo una creatura alla volta, e
dissolvi magie lanciato come incantesimo di 9° livello sul
filatterio libera qualsiasi creatura imprigionata al suo interno.
Una creatura imprigionata nel filatterio per 24 ore viene
consumata e distrutta, dopodiché nulla salvo un intervento divino
potrà riportarla in vita.
Un lich che dimentichi o non riesca a mantenere il suo corpo con
le anime sacrificate inizia a cascare a pezzi, e potrebbe infine
trasformarsi in un semilich.
Azioni
Tocco Paralizzante. Attacco con incantesimo in mischia: +12 a
colpire, portata 1,5 m, una creatura.
Colpisce: 10 (3d6) danni da freddo. Il bersaglio deve riuscire un
tiro salvezza di Costituzione DC 18 o restare paralizzato per 1
minuto. Il bersaglio può ripetere il tiro salvezza al termine di
ciascun suo turno, terminando l’effetto su di sé in caso di
successo.
Azioni Leggendarie
Il lich può effettuare 3 azioni aggiuntive, scelte tra le opzioni
seguenti. Può usare solo un’opzione leggendaria alla volta e solo
al termine del turno di un’altra creatura. Il lich recupera le azioni
leggendarie spese all’inizio del proprio round.
Distruggere Vita (Costa 3 Azioni). Ogni creatura ad eccezione
dei non morti entro 6 metri dal lich deve effettuare un tiro
salvezza di Costituzione DC 18 contro questa magia, subendo 21
(6d6) danni necrotici se fallisce il tiro salvezza, o la metà di
questi danni se lo riesce.
Sguardo Spaventoso (Costa 2 Azioni). Il lich fissa il suo sguardo
su di una creatura visibile entro 3 metri da esso. Il bersaglio deve
riuscire un tiro salvezza di Saggezza DC 18 contro questa magia
o restare spaventato per 1 minuto. Il bersaglio spaventato può
ripetere il tiro salvezza al termine di ciascun suo turno,
terminando l’effetto su di sé in caso di successo. Se il tiro
salvezza del bersaglio è riuscito o l’effetto per lui ha termine, il
bersaglio è immune allo sguardo del lich per le successive 24
ore.
Tocco Paralizzante (Costa 2 Azioni). Il lich usa il suo Tocco
Paralizzante.
Trucchetto. Il lich lancia un trucchetto.
Lucertoloide
Media umanoide (lucertoloide), neutrale
FORZA 15 (+2)
DESTREZZA 10 (+0)
COSTITUZIONE 13 (+1)
INTELLIGENZA 7 (-2)
SAGGEZZA 12 (+1)
Carisma 7 (-2)
Classe Armatura 15 (armatura naturale, scudo)
\hspace*{0pt}\hfill{Punti Ferita}: 22 (4d8 + 4)
Velocità 9 m, nuoto 9 m
Abilità Furtività +4, Percezione +3, Sopravvivenza +5
Sensi Percezione passiva 13
Linguaggi Draconico
Sfida 1/2 (100 PE)
Trattenere il Fiato. Il lucertoloide può trattenere il fiato per 15
minuti.
Azioni
Multiattacco. Il lucertoloide effettua due attacchi in mischia,
ciascuno con un’arma diversa.
Giavellotto. Attacco con arma da mischia o a Distanza: +4 a
colpire, portata 1,5 m o gittata 9/36 m, un bersaglio.
Colpisce: 5 (1d6 + 2) danni perforanti.
Morso. Attacco con arma da mischia: +4 a colpire, portata 1,5
m, un bersaglio.
Colpisce: 5 (1d6 + 2) danni perforanti.
Randello Pesante. Attacco con arma da mischia: +4 a colpire,
portata 1,5 m, un bersaglio.
Colpisce: 5 (1d6 + 2) danni da botta.
Scudo Appuntito. Attacco con arma da mischia: +4 a colpire,
portata 1,5 m, un bersaglio.
Colpisce: 5 (1d6 + 2) danni perforanti.
 
Mannari
Cinghiale Mannaro
Media umanoide (umano, mutaforma), neutrale malvagio
FORZA 17 (+3)
DESTREZZA 10 (+0)
COSTITUZIONE 15 (+2)
INTELLIGENZA 10 (+0)
SAGGEZZA 11 (+0)
Carisma 8 (-1)
Classe Armatura 10 in forma umanoide, 11 (armatura naturale)
in forma di cinghiale o ibrida
\hspace*{0pt}\hfill{Punti Ferita}: 78 (12d8 + 24)
Velocità 9 m (12 m in forma di cinghiale)
Abilità Percezione +2
Immunità al Danno da botta, perforante e tagliente di
attacchi non magici che non siano argentati
Sensi Percezione passiva 12
Linguaggi Comune (non può parlare in forma di cinghiale)
Sfida 4 (1.100 PE)
Carica (Solo Forma di Cinghiale o Ibrida). Se il cinghiale
mannaro si muove in linea retta di almeno 4,5 metri verso un
bersaglio e poi lo colpisce con le zanne durante lo stesso turno, il
bersaglio subisce 7 (2d6) danni taglienti aggiuntivi. Se il
bersaglio è una creatura, deve riuscire un tiro salvezza di Forza
DC 13 o cadere prono.
Implacabile (Ricarica dopo un Riposo Breve o Lungo). Se il
cinghiale mannaro subisce 14 danni o meno che lo ridurrebbero a
0 punti ferita, scende invece a 1 punto ferita.
Mutaforma. Il cinghiale mannaro può usare la sua azione per
trasformarsi in un ibrido cinghiale-umanoide o in un cinghiale, o
per tornare alla sua vera forma, che è umanoide. Le sue
statistiche, a parte la Difesa, sono le stesse in tutte le forme.
Qualsiasi equipaggiamento stia indossando o trasportando non
viene trasformato. Alla morte ritorna alla sua vera forma.
Azioni
Multiattacco (Solo in Forma Umanoide o Ibrida). Il cinghiale
mannaro effettua due attacchi, di cui solo uno può essere con le
zanne.
Maglio (Soltanto in Forma Umanoide o Ibrida). Attacco con
arma da mischia: +5 a colpire, portata 1,5 m, un bersaglio.
Colpisce: 10 (2d6 + 3) danni da botta.
Zanne (Soltanto in Forma di Cinghiale o Ibrida). Attacco con
arma da mischia: +5 a colpire, portata 1,5 m, un bersaglio.
Colpisce: 10 (2d6 + 3) danni taglienti. Se il bersaglio è un
umanoide, deve riuscire un tiro salvezza di Costituzione DC 12 o
venire maledetto dalla licantropia del cinghiale mannaro.
Lupo Mannaro
Media umanoide (umano, mutaforma), caotico malvagio
FORZA 15 (+2)
DESTREZZA 13 (+1)
COSTITUZIONE 14 (+2)
INTELLIGENZA 10 (+0)
SAGGEZZA 11 (+0)
Carisma 10 (+0)
Classe Armatura 11 in forma umanoide, 12 (armatura naturale)
in forma di lupo o ibrida
\hspace*{0pt}\hfill{Punti Ferita}: 58 (9d8 + 18)
Velocità 9 m (12 m in forma di lupo)
Abilità Furtività +3, Percezione +4
Immunità al Danno da botta, perforante e tagliente di
attacchi non magici che non siano argentati
Sensi Percezione passiva 14
Linguaggi Comune (non può parlare in forma di lupo)
Sfida 3 (700 PE)
Mutaforma. Il lupo mannaro può usare la sua azione per
trasformarsi in un ibrido lupo-umanoide o in un lupo, o per
tornare alla sua vera forma, che è umanoide. Le sue statistiche, a
parte la Difesa, sono le stesse in tutte le forme. Qualsiasi
equipaggiamento stia indossando o trasportando non viene
trasformato. Alla morte ritorna alla sua vera forma.
Udito e Olfatto Affinato. Il lupo mannaro ha vantaggio nelle
prove di Saggezza (Percezione) basate su udito o olfatto.
Azioni
Multiattacco (Soltanto in Forma Umanoide o Ibrida). Il lupo
mannaro effettua due attacchi: uno con il morso e uno con gli
artigli o la lancia.
Artigli (Soltanto in Forma Ibrida). Attacco con arma da
mischia: +4 a colpire, portata 1,5 m, una creatura.
Colpisce: 7 (2d4 + 2) danni taglienti.
Lancia (Soltanto in Forma Umanoide). Attacco con arma da
mischia o a Distanza: +4 a colpire, portata 1,5 m o gittata 6/18
m, una creatura.
Colpisce: 5 (1d6 + 2) danni perforanti o 6 (1d8 + 2) danni
perforanti se usata con due mani in un attacco di mischia.
Morso (Soltanto in Forma di Lupo o Ibrida). Attacco con arma
da mischia: +4 a colpire, portata 1,5 m, un bersaglio.
Colpisce: 6 (1d8 + 2) danni perforanti. Se il bersaglio è un
umanoide, deve riuscire un tiro salvezza di Costituzione DC 12 o
venir maledetto dalla licantropia del lupo mannaro.
Orso Mannaro
Media umanoide (umano, mutaforma), neutrale buono
FORZA 19 (+4)
DESTREZZA 10 (+0)
COSTITUZIONE 17 (+3)
INTELLIGENZA 11 (+0)
SAGGEZZA 12 (+1)
Carisma 12 (+1)
Classe Armatura 10 in forma umanoide, 11 (armatura naturale)
in forma di orso o ibrida
\hspace*{0pt}\hfill{Punti Ferita}: 135 (18d8 + 54)
Velocità 9 m (12 m, scalata 9 m in forma di orso o forma ibrida)
Abilità Percezione +7
Immunità al Danno da botta, perforante e tagliente di
attacchi non magici che non siano argentati
Sensi Percezione passiva 17
Linguaggi Comune (non può parlare in forma di orso)
Sfida 5 (1.800 PE)
Mutaforma. L’orso mannaro può usare la sua azione per
trasformarsi in un ibrido orso-umanoide o in un orso, o per
tornare alla sua vera forma, che è umanoide. Le sue statistiche, a
parte la Difesa, sono le stesse in tutte le forme. Qualsiasi
equipaggiamento stia indossando o trasportando non viene
trasformato. Alla morte ritorna alla sua vera forma.
Olfatto Affinato. L’orso mannaro ha vantaggio nelle prove di
Saggezza (Percezione) basate sull’olfatto.
Azioni
Multiattacco. In forma di orso, l’orso mannaro effettua due
attacchi di artiglio. In forma umanoide, effettua due attacchi di
ascia bipenne. In forma ibrida, può attaccare come un orso o un
umanoide.
Artiglio (Soltanto in Forma di Orso o Ibrida). Attacco con arma
da mischia: +7 a colpire, portata 1,5 m, un bersaglio.
Colpisce: 13 (2d8 + 2) danni taglienti.
Ascia Bipenne (Soltanto in Forma Umanoide o Ibrida). Attacco
con arma da mischia: +7 a colpire, portata 1,5 m, un bersaglio.
Colpisce: 10 (1d12 + 4) danni taglienti.
Morso (Soltanto in Forma di Orso o Ibrida). Attacco con arma
da mischia: +7 a colpire, portata 1,5 m, un bersaglio.
Colpisce: 15 (2d10 + 4) danni perforanti. Se il bersaglio è un
umanoide, deve riuscire un tiro salvezza di Costituzione DC 14 o
venir maledetto dalla licantropia dell’orso mannaro.
Ratto Mannaro
Media umanoide (umano, mutaforma), legale malvagio
FORZA 10 (+0)
DESTREZZA 15 (+2)
COSTITUZIONE 12 (+1)
INTELLIGENZA 11 (+0)
SAGGEZZA 10 (+0)
Carisma 8 (-1)
Classe Armatura 12
\hspace*{0pt}\hfill{Punti Ferita}: 33 (6d8 + 6)
Velocità 9 m
Abilità Furtività +4, Percezione +2
Immunità al Danno da botta, perforante e tagliente di
attacchi non magici che non siano argentati
Sensi scurovisione 18 m (solo in forma di ratto), Percezione
passiva 12
Linguaggi Comune (non può parlare in forma di ratto)
Sfida 2 (450 PE)
Mutaforma. Il ratto mannaro può usare la sua azione per
trasformarsi in un ibrido ratto-umanoide o in un ratto, o per
tornare alla sua vera forma, che è umanoide. Le sue statistiche, a
parte la Difesa, sono le stesse in tutte le forme. Qualsiasi
equipaggiamento stia indossando o trasportando non viene
trasformato. Alla morte ritorna alla sua vera forma.
Olfatto Affinato. Il ratto mannaro ha vantaggio nelle prove di
Saggezza (Percezione) basate sull’olfatto.
Azioni
Multiattacco (Solo in Forma Umanoide o Ibrida). Il ratto
mannaro effettua due attacchi, di cui solo uno può essere con il
morso.
Spada Corta (Soltanto in Forma Umanoide o Ibrida). Attacco
con arma da mischia: +4 a colpire, portata 1,5 m, un bersaglio.
Colpisce: 5 (1d6 + 2) danni perforanti.
Balestra a mano (Soltanto in Forma Umanoide o Ibrida).
Attacco con arma a Distanza: +4 a colpire, gittata 9/36 m, un
bersaglio.
Colpisce: 5 (1d6 + 2) danni perforanti.
Morso (Soltanto in Forma di Ratto o Ibrida). Attacco con arma
da mischia: +4 a colpire, portata 1,5 m, un bersaglio.
Colpisce: 4 (1d4 + 2) danni perforanti. Se il bersaglio è un
umanoide, deve riuscire un tiro salvezza di Costituzione DC 11 o
venir maledetto dalla licantropia del ratto mannaro.
 
Tigre Mannara
Media umanoide (umano, mutaforma), neutrale
FORZA 17 (+3)
DESTREZZA 15 (+2)
COSTITUZIONE 16 (+3)
INTELLIGENZA 10 (+0)
SAGGEZZA 13 (+1)
Carisma 11 (+0)
Classe Armatura 12
\hspace*{0pt}\hfill{Punti Ferita}: 120 (16d8 + 48)
Velocità 9 m (12 m in forma di tigre)
Abilità Furtività +4, Percezione +5
Immunità al Danno da botta, perforante e tagliente di
attacchi non magici che non siano argentati
Sensi scurovisione 18 m, Percezione passiva 15
Linguaggi Comune (non può parlare in forma di tigre)
Sfida 4 (1.1100 PE)
Balzo. Se la tigre mannara si muove di almeno 4,5 metri in linea
retta verso una creatura e la colpisce con un attacco di artiglio
durante lo stesso turno, il bersaglio deve riuscire un tiro salvezza
di Forza DC 14 o cadere prono. Se il bersaglio è prono, la tigre
mannara può effettuare un attacco di morso contro di esso come
azione bonus.
Mutaforma. La tigre mannara può usare la sua azione per
trasformarsi in un ibrido tigre-umanoide o in una tigre, o per
tornare alla sua vera forma, che è umanoide. Le sue statistiche, a
parte la Difesa, sono le stesse in tutte le forme. Qualsiasi
equipaggiamento stia indossando o trasportando non viene
trasformato. Alla morte ritorna alla sua vera forma.
Olfatto e Udito Affinato. La tigre mannara ha vantaggio nelle
prove di Saggezza (Percezione) basate su olfatto e udito.
Azioni
Multiattacco (Solo in Forma Umanoide o Ibrida). In forma
umanoide, la tigre mannara effettua due attacchi di scimitarra o
due attacchi di arco lungo. In forma ibrida, può attaccare come
un umanoide o effettuare due attacchi di artiglio.
Artiglio (Soltanto in Forma di Tigre o Ibrida). Attacco con
arma da mischia: +5 a colpire, portata 1,5 m, un bersaglio.
Colpisce: 7 (1d8 + 3) danni taglienti.
Morso (Soltanto in Forma di Tigre o Ibrida). Attacco con arma
da mischia: +5 a colpire, portata 1,5 m, un bersaglio.
Colpisce: 8 (1d10 + 3) danni perforanti. Se il bersaglio è un
umanoide, deve riuscire un tiro salvezza di Costituzione DC 13 o
venir maledetto dalla licantropia della tigre mannara.
Scimitarra (Soltanto in Forma Umanoide o Ibrida). Attacco
con arma da mischia: +5 a colpire, portata 1,5 m, un bersaglio.
Colpisce: 6 (1d6 + 3) danni taglienti.
Arco Lungo (Soltanto in Forma Umanoide o Ibrida). Attacco
con arma a Distanza: +4 a colpire, gittata 45/180 m, un
bersaglio.
Colpisce: 6 (1d8 + 2) danni perforanti.
Manticora
Grande mostruosità, legale malvagio
FORZA 17 (+3)
DESTREZZA 16 (+3)
COSTITUZIONE 17 (+3)
INTELLIGENZA 7 (-2)
SAGGEZZA 12 (+1)
Carisma 8 (-1)
Classe Armatura 14 (armatura naturale)
\hspace*{0pt}\hfill{Punti Ferita}: 68 (8d10 + 24)
Velocità 9 m, volo 15 m
Sensi scurovisione 18 m, Percezione passiva 11
Linguaggi Comune
Sfida 3 (700 PE)
Ricrescere Spine della Coda. La manticora possiede ventiquattro
spine nella coda. Le spine usate ricrescono quando la manticora
termina un riposo lungo.
Azioni
Multiattacco. La manticora effettua tre attacchi: uno con il
morso e due con gli artigli o tre con le spine della coda.
Artiglio. Attacco con arma da mischia: +5 a colpire, portata 1,5
m, un bersaglio.
Colpisce: 6 (1d6 + 3) danni taglienti.
Morso. Attacco con arma da mischia: +5 a colpire, portata 1,5
m, un bersaglio.
Colpisce: 7 (1d8 + 3) danni perforanti.
Spine della Coda. Attacco con arma a Distanza: +5 a colpire,
gittata 30/60 m, un bersaglio.
Colpisce: 7 (1d8 + 3) danni perforanti.
Manto Assassino
Grande aberrazione, caotico neutrale
FORZA 17 (+3)
DESTREZZA 15 (+2)
COSTITUZIONE 12 (+1)
INTELLIGENZA 13 (+1)
SAGGEZZA 12 (+1)
Carisma 14 (+2)
Classe Armatura 14 (armatura naturale)
\hspace*{0pt}\hfill{Punti Ferita}: 78 (12d10 + 12)
Velocità 3 m, volo 12 m
Abilità Furtività +5
Sensi scurovisione 18 m, Percezione passiva 11
Linguaggi Parlata delle Profondità, Sottocomune
Sfida 8 (3.900 PE)
Falso Aspetto. Mentre il manto assassino resta immobile senza
esporre la parte inferiore del corpo, è indistinguibile da un manto
di pelle nera.
Sensibilità alla Luce. Mentre è alla luce del sole, il manto
assassino ha svantaggio ai tiri per colpire, oltre che alle prove di
Saggezza (Percezione) basate sulla vista.
Trasferimento di Danno. Mentre è appiccicato ad una creatura,
il manto assassino subisce solo la metà dei danni che gli sono
inferti (arrotondare per difetto), e la creatura vittima del manto
assassino subisce l’altra metà.
Azioni
Multiattacco. Il manto assassino effettua due attacchi: uno con il
morso e uno con la coda.
Morso. Attacco con arma da mischia: +6 a colpire, portata 1,5
m, una creatura.
Colpisce: 10 (2d6 + 3) danni perforanti, e se il bersaglio è di
taglia Grande o inferiore, il manto assassino vi si appiccica. Se il
manto assassino ha vantaggio contro il bersaglio, si appiccica
alla sua testa e il bersaglio è accecato e impossibilitato a
respirare finché il manto assassino vi rimane appiccicato. Mentre
è appiccicato il manto assassino può effettuare questo attacco
solo contro il bersaglio e ha vantaggio al tiro per colpire. Il
manto assassino può staccarsi spendendo 1,5 metri di
movimento. Una creatura, compreso il bersaglio, può effettuare
la sua azione per staccare il manto assassino riuscendo una prova
di Forza DC 16.
Coda. Attacco con arma da mischia: +6 a colpire, portata 3 m,
una creatura.
Colpisce: 7 (1d8 + 3) danni taglienti.
Apparizioni (Ricarica dopo un Riposo Breve o Lungo). Qualora
non si trovi sotto luce intensa, il manto assassino crea tre
duplicati illusori di sé stesso, che si muovono assieme ad esso e
ne imitano le azioni, scambiandosi di posizione per rendere
impossibile capire quale sia il reale manto assassino. Se
l'originale si trova in un’area di luce intensa, i duplicati
svaniscono.
Ogniqualvolta una creatura prenda a bersaglio il manto assassino
con un attacco o un incantesimo nocivo mentre sono ancora
presenti dei duplicati, quella creatura determina casualmente se
prende a bersaglio il manto assassino o uno dei duplicati. Una
creatura che non possa vedere o che si affida a sensi diversi dalla
vista ignora questo effetto magico.
Un duplicato possiede la Difesa e usa i tiri salvezza del manto assassino.
Se un attacco colpisce un duplicato, o se un duplicato fallisce un tiro
salvezza contro un effetto che infligge danni, svanisce.
Gemito. Ogni creatura entro 18 metri dal manto assassino, che possa
udire il suo gemito e che non sia un’aberrazione, deve riuscire un tiro
salvezza di Saggezza DC 13 o essere spaventata fino al termine del
prossimo turno del manto assassino. Se il tiro salvezza della creatura
riesce, la creatura è immune al gemito del manto assassino per le
successive 24 ore.
 
Mantoscuro
Piccola mostruosità, disallineato
FORZA 16 (+3)
DESTREZZA 12 (+1)
COSTITUZIONE 13 (+1)
INTELLIGENZA 2 (-4)
SAGGEZZA 10 (+0)
Carisma 5 (-3)
Classe Armatura 11
\hspace*{0pt}\hfill{Punti Ferita}: 22 (5d6 + 5)
Velocità 3 m, volo 9 m
Abilità Furtività +3
Sensi vista cieca 18 m, Percezione passiva 10
Linguaggi -
Sfida 1/2 (100 PE)
Ecolocazione. Il mantoscuro non può usare la vista cieca se
assordato.
Falso Aspetto. Mentre il mantoscuro rimane immobile, è
indistinguibile da una formazione rocciosa come una stalattite o
una stalagmite.
Azioni
Spaccare. Attacco con arma da mischia: +5 a colpire, portata 1,5
m, una creatura.
Colpisce: 6 (1d6 + 3) danni da botta e il mantoscuro si appiccica
alla creatura. Se il bersaglio è di taglia Media o inferiore il
mantoscuro ha vantaggio al tiro per colpire, si appiccica avvolgendo
la testa del bersaglio, che è accecato e impossibilitato a respirare
finché il mantoscuro resta appiccicato in questo modo.
Mentre è appiccicato al bersaglio, il mantoscuro non può attaccare
nessun’altra creatura salvo il bersaglio, ma ha vantaggio ai suoi tiri
per colpire. La velocità del mantoscuro diventa 0 e non può trarre
beneficio da nessun bonus alla velocità, muovendosi assieme al
bersaglio.
Una creatura può staccare il mantoscuro con un’azione e riuscendo
una prova di Forza DC 13. Durante il suo turno, il mantoscuro può
staccarsi dal bersaglio da solo usando 1,5 metri di movimento.
Aura di Oscurità (1/Giorno). Un’oscurità magica con 4,5 metri
di raggio si estende dal mantoscuro, muovendosi con esso, e
propagandosi oltre gli angoli. L’oscurità permane finché il
mantoscuro mantiene la concentrazione, massimo 10 minuti
(come se si stesse concentrando su di un incantesimo). La
scurovisione non può penetrare questa oscurità, né essa può
essere rischiarata da alcuna luce naturale. Se qualsiasi parte
dell’oscurità si sovrappone ad un’area di luce generata da un
incantesimo di 2° livello o inferiore, l’incantesimo che sta
creando la luce viene dissolto.
Medusa
Media mostruosità, legale malvagio
FORZA 10 (+0)
DESTREZZA 15 (+2)
COSTITUZIONE 16 (+3)
INTELLIGENZA 12 (+1)
SAGGEZZA 13 (+1)
Carisma 15 (+2)
Classe Armatura 15 (armatura naturale)
\hspace*{0pt}\hfill{Punti Ferita}: 127 (17d8 + 51)
Velocità 9 m
Abilità Furtività +5, Inganno +5, Intuizione +4, Percezione +4
Sensi scurovisione 18 m, Percezione passiva 14
Linguaggi Comune
Sfida 6 (2.300 PE)
Sguardo Pietrificante. Se una creatura comincia il suo turno
entro 9 metri da una medusa di cui possa vedere gli occhi, la
medusa, qualora la non sia inabile e possa vedere a sua volta la
creatura, può obbligarla ad effettuare un tiro salvezza di
Costituzione DC 14. Se la creatura fallisce il tiro salvezza di 5 o
più, viene pietrificata all’istante, altrimenti inizia magicamente a
trasformarsi in pietra ed è intralciata. La creatura intralciata deve
ripetere il tiro salvezza al termine del suo prossimo turno. Se lo
riesce, l’effetto termina. Se lo fallisce, la creatura è pietrificata
finché non viene liberata dall’incantesimo ristorare superiore o
altra magia.
Una creatura che non sia sorpresa può distogliere lo sguardo per
evitare il tiro salvezza all’inizio del proprio round. In quel caso,
non potrà vedere la medusa fino all’inizio del suo prossimo
turno, quando potrà distogliere nuovamente lo sguardo. Se nel
frattempo dovesse guardare la medusa, dovrebbe
immediatamente effettuare il tiro salvezza.
Se la medusa vede il suo riflesso su di una superficie riflettente
entro 9 metri da lei in un’area di luce intensa, a causa della
propria maledizione subirà gli effetti del suo stesso sguardo.
Azioni
Multiattacco. La medusa effettua tre attacchi – uno con i capelli
serpentini e due con la spada corta – o due attacchi a distanza
con l’arco lungo.
Capelli Serpentini. Attacco con arma da mischia: +5 a colpire,
portata 1,5 m, un bersaglio.
Colpisce: 4 (1d4 + 2) danni perforanti più 14 (4d6) danni da
veleno.
Spada Corta. Attacco con arma da mischia: +5 a colpire, portata
1,5 m, un bersaglio.
Colpisce: 5 (1d6 + 2) danni perforanti.
Arco Lungo. Attacco con arma a Distanza: +5 a colpire, gittata
45/180 m, un bersaglio.
Colpisce: 6 (1d8 + 2) danni perforanti più 7 (2d6) danni da
veleno.
Mefiti
Mefito di Ghiaccio
Piccola elementale, neutrale malvagio
FORZA 7 (-2)
DESTREZZA 13 (+1)
COSTITUZIONE 10 (+0)
INTELLIGENZA 9 (-1)
SAGGEZZA 11 (+0)
Carisma 12 (+1)
Classe Armatura 11
\hspace*{0pt}\hfill{Punti Ferita}: 21 (6d6)
Velocità 9 m, volo 9 m
Abilità Furtività +3, Percezione +2
Vulnerabilità ai danni da botta, fuoco
Immunità ai Danni freddo, veleno
Immunità alle Condizioni avvelenato
Sensi scurovisione 18 m, Percezione passiva 12
Linguaggi Aquan, Auran
Sfida 1/2 (100 PE)
Falso Aspetto. Mentre il mefito rimane immobile, è
indistinguibile da un ordinario frammento di ghiaccio.
Incantesimi Innati (1/Giorno). Il mefito può lanciare in maniera
innata nube di nebbia, senza bisogno di componenti materiali. La
sua caratteristica da incantatore innato è il Carisma.
Natura Elementale. Un mefito non ha bisogno di cibo, bevande
o sonno.
Scoppio Mortale. Quando il mefito muore, esplode in uno
scoppio di frammenti di ghiaccio. Ogni creatura entro 1,5 metri
da esso deve effettuare un tiro salvezza di Destrezza DC 10 o
subire 4 (1d8) danni taglienti in caso di fallimento, o la metà di
questi danni in caso di successo.
Azioni
Artigli. Attacco con arma da mischia: +3 a colpire, portata 1,5
m, una creatura.
Colpisce: 3 (1d4 + 1) danni taglienti più 2 (1d4) danni da freddo.
Soffio Gelido (Ricarica 6). Il mefito esala un cono di 4,5 metri di
aria fredda. Ogni creatura nell’area deve effettuare un tiro
salvezza di Destrezza DC 10, subendo 5 (2d4) danni da freddo in
caso di fallimento, o la metà di questi danni in caso di successo.
Mefito di Magma
Piccola elementale, neutrale malvagio
FORZA 8 (-1)
DESTREZZA 12 (+1)
COSTITUZIONE 12 (+1)
INTELLIGENZA 7 (-2)
SAGGEZZA 10 (+0)
Carisma 10 (+0)
Classe Armatura 11
\hspace*{0pt}\hfill{Punti Ferita}: 22 (5d6 + 5)
Velocità 9 m, volo 9 m
Abilità Furtività +3
Vulnerabilità ai Danni freddo
Immunità ai Danni fuoco, veleno
Immunità alle Condizioni avvelenato
Sensi scurovisione 18 m, Percezione passiva 10
Linguaggi Ignan, Terran
Sfida 1/2 (100 PE)
Falso Aspetto. Mentre il mefito rimane immobile, è
indistinguibile da un’ordinaria pozza di magma.
Incantesimi Innati (1/Giorno). Il mefito può lanciare in maniera
innata riscaldare metallo (DC del tiro salvezza dell’incantesimo
10), senza bisogno di componenti materiali. La sua caratteristica
da incantatore innato è il Carisma.
Natura Elementale. Un mefito non ha bisogno di cibo, bevande
o sonno.
Scoppio Mortale. Quando il mefito muore, esplode in uno
scoppio di lava. Ogni creatura entro 1,5 metri da esso deve
effettuare un tiro salvezza di Destrezza DC 11 o subire 7 (2d6)
danni da fuoco in caso di fallimento, o la metà di questi danni in
caso di successo.
Azioni
Artigli. Attacco con arma da mischia: +3 a colpire, portata 1,5
m, una creatura.
Colpisce: 3 (1d4 + 1) danni taglienti più 2 (1d4) danni da fuoco.
Soffio Infuocato (Ricarica 6). Il mefito esala un cono di 4,5
metri di fuoco. Ogni creatura nell’area deve effettuare un tiro
salvezza di Destrezza DC 11, subendo 7 (2d6) danni da fuoco in
caso di fallimento, o la metà di questi danni in caso di successo.
 
Mefito di Polvere
Piccola elementale, neutrale malvagio
FORZA 5 (-3)
DESTREZZA 14 (+2)
COSTITUZIONE 10 (+0)
INTELLIGENZA 9 (-1)
SAGGEZZA 11 (+0)
Carisma 10 (+0)
Classe Armatura 12
\hspace*{0pt}\hfill{Punti Ferita}: 17 (5d6)
Velocità 9 m, volo 9 m
Abilità Furtività +4, Percezione +2
Vulnerabilità ai Danni fuoco
Immunità ai Danni veleno
Immunità alle Condizioni avvelenato
Sensi scurovisione 18 m, Percezione passiva 12
Linguaggi Auran, Terran
Sfida 1/2 (100 PE)
Incantesimi Innati (1/Giorno). Il mefito può eseguire in maniera
innata sonno (DC del tiro salvezza dell’incantesimo 10), senza
bisogno di componenti materiali. La sua abilità da incantatore
innato è il Carisma.
Natura Elementale. Un mefito non ha bisogno di cibo, bevande
o sonno.
Scoppio Mortale. Quando il mefito muore, esplode in uno
scoppio di polvere. Ogni creatura entro 1,5 metri da esso deve
riuscire un tiro salvezza di Costituzione DC 10 o restare accecata
per 1 minuto. Una creatura accecata può ripetere il tiro salvezza
durante ciascun suo turno, terminando l’effetto su di sé in caso di
successo.
Azioni
Artigli. Attacco con arma da mischia: +4 a colpire, portata 1,5
m, una creatura.
Colpisce: 4 (1d4 + 2) danni taglienti.
Soffio Accecante (Ricarica 6). Il mefito esala un cono di 4,5
metri di polvere accecante. Ogni creatura nell’area deve riuscire
un tiro salvezza di Destrezza DC 10 o restare accecata per 1
minuto. Una creatura accecata può ripetere il tiro salvezza
durante ciascun suo turno, terminando l’effetto su di sé in caso di
successo.
Mefito di Vapore
Piccola elementale, neutrale malvagio
FORZA 5 (-3)
DESTREZZA 11 (+0)
COSTITUZIONE 10 (+0)
INTELLIGENZA 11 (+0)
SAGGEZZA 10 (+0)
Carisma 12 (+1)
Classe Armatura 10
\hspace*{0pt}\hfill{Punti Ferita}: 21 (6d6)
Velocità 9 m, volo 9 m
Immunità ai Danni fuoco, veleno
Immunità alle Condizioni avvelenato
Sensi scurovisione 18 m, Percezione passiva 10
Linguaggi Aquan, Ignan
Sfida 1/4 (50 PE)
Incantesimi Innati (1/Giorno). Il mefito può eseguire in maniera
innata sfocatura, senza bisogno di componenti materiali. La sua
abilità da incantatore innato è il Carisma.
Natura Elementale. Un mefito non ha bisogno di cibo, bevande
o sonno.
Scoppio Mortale. Quando il mefito muore, esplode in nube di
vapore. Ogni creatura entro 1,5 metri da esso deve riuscire un
tiro salvezza di Destrezza DC 10 o subire 4 (1d8) danni da fuoco.
Azioni
Artigli. Attacco con arma da mischia: +2 a colpire, portata 1,5
m, una creatura.
Colpisce: 2 (1d4) danni taglienti più 2 (1d4) danni da fuoco.
Soffio Vaporoso (Ricarica 6). Il mefito esala un cono di 4,5
metri di vapore caldo. Ogni creatura nell’area deve effettuare un
tiro salvezza di Destrezza DC 10, subendo 4 (1d8) danni da
fuoco in caso di fallimento, o la metà di questi danni in caso di
successo.
Megere
Megera Marina
Media fatato, caotico malvagio
FORZA 16 (+3)
DESTREZZA 13 (+1)
COSTITUZIONE 16 (+3)
INTELLIGENZA 12 (+1)
SAGGEZZA 12 (+1)
Carisma 13 (+1)
Classe Armatura 14 (armatura naturale)
\hspace*{0pt}\hfill{Punti Ferita}: 52 (7d8 + 21)
Velocità 9 m, nuoto 12 m
Sensi scurovisione 18 m, Percezione passiva 11
Linguaggi Aquan, Comune, Gigante
Sfida 2 (450 PE)
Anfibio. La megera può respirare aria e acqua.
Aspetto Orripilante. Qualsiasi umanoide che inizi il suo turno
entro 9 metri dalla megera e ne può vedere la vera forma deve
effettuare un tiro salvezza di Saggezza DC 11. Se fallisce il tiro
salvezza, la creatura resta spaventata per 1 minuto. Una creatura
può ripetere il tiro salvezza al termine di ciascun suo turno, con
svantaggio se la megera è in linea di visuale, e terminando
l’effetto se riesce il tiro salvezza. Se il tiro salvezza della creatura
riesce o l’effetto ha termine su di essa, la creatura è immune
all’Aspetto Orripilante per le successive 24 ore.
A meno che il bersaglio non sia sorpreso o la rivelazione della
vera forma della megera non sia improvvisa, il bersaglio può
distogliere lo sguardo e evitare di effettuare il tiro salvezza
iniziale. Fino all’inizio del suo prossimo turno, una creatura che
distolga lo sguardo ha svantaggio ai tiri di attacco contro la
megera.
Azioni
Artigli. Attacco in mischia con arma: +5 a colpire, portata 1,5 m,
un bersaglio.
Colpisce: 10 (2d6 + 3) danni taglienti.
Aspetto Illusorio. La megera ricopre se stessa e tutto quello che
sta indossando o trasportando in un’illusione magica che le dona
l’aspetto di una creatura ripugnante all’incirca della stessa taglia
e forma umanoide. L’illusione termina se la megera effettua
un’azione bonus per terminarla o se muore.
I cambiamenti apportati da questo effetto non sono in grado di
superare le ispezioni fisiche. Ad esempio, la megera potrebbe
apparire come una creatura priva di artigli, ma una persona in
contatto con le sue mani li avvertirebbe. Altrimenti, una creatura
deve effettuare un’azione per ispezionare visivamente l’illusione
e riuscire una prova di Intelligenza (Indagare) DC 16 per
comprendere che la megera si è camuffata.
Occhiata Mortale. La megera prende a bersaglio una creatura
spaventata visibile entro 9 metri da lei. Se il bersaglio può vedere
la megera, deve riuscire un tiro salvezza di Saggezza DC 11
contro questa magia o scendere a 0 punti ferita.
Megera Notturna
Media immondo, neutrale malvagio
FORZA 18 (+4)
DESTREZZA 15 (+2)
COSTITUZIONE 16 (+3)
INTELLIGENZA 16 (+3)
SAGGEZZA 14 (+2)
Carisma 16 (+3)
Classe Armatura 17 (armatura naturale)
\hspace*{0pt}\hfill{Punti Ferita}: 112 (15d8 + 45)
Velocità 9 m
Abilità Furtività +6, Inganno +7, Intuizione +6, Percezione +6,
Resistenze al Danno freddo, fuoco; da botta, perforante e
tagliente di attacchi non magici non siano argentati
Sensi scurovisione 36 m, Percezione passiva 16
Linguaggi Abissale, Comune, Infernale, Primordiale
Sfida 5 (1.800 PE)
Incantesimi Innati. La caratteristica da incantatore innato della
megera è il Carisma (DC 14 per i tiri salvezza degli incantesimi,
+6 a colpire con attacchi da incantesimo). La megera può
lanciare in maniera innata i seguenti incantesimi, senza aver
bisogno di componenti materiali.
A volontà: dardo incantato, individuazione del magico
2/giorno ciascuno: raggio di indebolimento, sonno, spostamento
planare (personale)
Resistenza alla Magia. La megera ha +1d6 ai tiri salvezza
contro incantesimi e altri effetti magici.
Azioni
Artigli (Solo in Forma di Megera). Attacco con arma da
mischia: +7 a colpire, portata 1,5 m, un bersaglio.
Colpisce: 13 (2d8 + 4) danni taglienti.
Forma Eeterea. La megera entra magicamente nel Piano Etereo
dal Piano Materiale, e viceversa. Per farlo deve essere in
possesso di un cuore di pietra.
Infestare Incubi (1/Giorno). Mentre si trova sul Piano Etereo, la
megera entra magicamente in contatto con un umanoide
addormentato che si trova sul Piano Materiale. L’incantesimo
protezione dal bene e dal male lanciato sul bersaglio previene questo
contatto, così come cerchio magico. Finché il contatto persiste, il
bersaglio soffre di orribili visioni. Se queste visioni durano per
almeno 1 ora, il bersaglio non ottiene benefici dal suo riposo, e i suoi
punti ferita massimi sono ridotti di 5 (1d10). Se questo effetto riduce
i punti ferita massimi del bersaglio a 0, il bersaglio muore, e se il
bersaglio era malvagio, la sua anima resta intrappolata nella borsa
delle anime della megera. La riduzione dei punti ferita massimi del
bersaglio rimane finché non viene rimossa dall’incantesimo ristorare
superiore o simile magia.
Mutare Forma. La megera può trasformarsi magicamente in una
femmina umanoide di taglia Piccola o Media, o tornare alla sua vera
forma. Le sue statistiche sono le stesse in qualsiasi forma. Tutto
l’equipaggiamento che stava trasportando o indossando non viene
trasformato. Alla morte, ritorna alla sua vera forma.
 
Megera Verde
Media fatato, neutrale malvagio
FORZA 18 (+4)
DESTREZZA 12 (+1)
COSTITUZIONE 16 (+3)
INTELLIGENZA 13 (+1)
SAGGEZZA 14 (+2)
Carisma 14 (+2)
Classe Armatura 17 (armatura naturale)
\hspace*{0pt}\hfill{Punti Ferita}: 82 (11d8 + 33)
Velocità 9 m
Abilità Arcano +3, Furtività +3, Inganno +4, Percezione +4
Sensi scurovisione 18 m, Percezione passiva 14
Linguaggi Comune, Draconico, Silvano
Sfida 3 (700 PE)
Anfibio. La megera può respirare aria e acqua.
Imitazione. La megera può imitare suoni animali e voci
umanoidi. Una creatura che senta questi rumori può determinare
che si tratti di un’imitazione riuscendo una prova di Saggezza
(Intuizione) DC 14.
Incantesimi Innati. La caratteristica da incantatore innato della
megera è il Carisma (DC 12 per i tiri salvezza degli incantesimi).
La megera può lanciare in maniera innata i seguenti incantesimi,
senza aver bisogno di componenti materiali.
A volontà: illusione minore, luci danzanti, beffa maligna
Azioni
Artigli. Attacco con arma da mischia: +6 a colpire, portata 1,5
m, un bersaglio.
Colpisce: 13 (2d8 + 4) danni taglienti.
Aspetto Illusorio. La megera ricopre sé stessa e tutto quello che
sta indossando o trasportando in un’illusione magica che le dona
l’aspetto di un’altra creatura all’incirca della stessa taglia e forma
umanoide. L’illusione termina se la megera effettua un’azione
bonus per terminarla o se muore.
I cambiamenti apportati da questo effetto non sono in grado di
superare le ispezioni fisiche. Ad esempio, la megera potrebbe
apparire come una creatura dalla pelle liscia, ma il contatto
rivelerebbe la sua pelle ruvida. Altrimenti, una creatura deve
effettuare un’azione per ispezionare visivamente l’illusione e
riuscire una prova di Intelligenza (Indagare) DC 20 per
comprendere che si tratta di una megera camuffata.
Passaggio Invisibile. La megera può rendersi invisibile finché
non attacca o lancia un incantesimo, o finché non termina la
concentrazione (come se si stesse concentrando su di un
incantesimo). Mentre è invisibile, non lascia traccia fisica del suo
passaggio, quindi le sue tracce possono essere seguite solo dalla
magia. Tutto l’equipaggiamento che sta trasportando o
indossando diventa invisibile assieme a lei.
Melme
Ameba Paglierina
Grande melma, disallineato
FORZA 15 (+2)
DESTREZZA 6 (-2)
COSTITUZIONE 14 (+2)
INTELLIGENZA 2 (-4)
SAGGEZZA 6 (-2)
Carisma 1 (-5)
Classe Armatura 8
\hspace*{0pt}\hfill{Punti Ferita}: 45 (6d10 + 12)
Velocità 3 m, scalata 3 m
Resistenze al Danno acido
Immunità al Danno fulmine, tagliente
Immunità alle Condizioni accecato, affascinato, assordato,
prono, sfinimento, spaventato
Sensi vista cieca 18 m (cieca oltre questo raggio), Percezione
passiva 8
Linguaggi -
Sfida 2 (450 PE)
Amorfo. L’ameba può muoversi attraverso uno spazio fino a 2,5
centimetri di larghezza senza doversi stringere.
Natura di Melma. L’ameba non necessita di dormire.
Scalare come Ragno. L’ameba può scalare superfici difficili,
compreso lo stare a testa in giù sul soffitto, senza bisogno di
effettuare una prova di abilità.
Azioni
Pseudopodo. Attacco con arma da mischia: +4 a colpire, portata
1,5 m, un bersaglio.
Colpisce: 9 (2d6 + 2) danni da botta più 3 (1d6) danni da acido.
Reazioni
Divisione. Quando un’ameba Media o più grande subisce danni da
fulmine o taglienti, si divide in due nuove amebe che hanno almeno
10 punti ferita. Ogni nuova ameba ha un numero di punti ferita pari
alla metà dell’ameba originale, arrotondati per difetto. Le nuove
amebe sono di una taglia più piccola di quella originale.
Cubo Gelatinoso
Grande melma, disallineato
FORZA 14 (+2)
DESTREZZA 3 (-4)
COSTITUZIONE 20 (+5)
INTELLIGENZA 1 (-5)
SAGGEZZA 6 (-2)
Carisma 1 (-5)
Classe Armatura 6
\hspace*{0pt}\hfill{Punti Ferita}: 84 (8d10 + 40)
Velocità 4,5 m
Immunità alle Condizioni accecato, affascinato, assordato,
prono, sfinimento, spaventato
Sensi vista cieca 18 m (cieca oltre questo raggio), Percezione
passiva 8
Linguaggi -
Sfida 2 (450 PE)
Cubo di Melma. Il cubo occupa il suo intero spazio. Le altre
creature possono entrare nello spazio, ma rimangono vittima del
Sommergere del cubo e hanno svantaggio al tiro salvezza.
Le creature all’interno del cubo sono visibili ma godono di
copertura totale.
Una creatura entro 1,5 metri dal cubo può effettuare un’azione
per tirare una creatura od oggetto fuori dal cubo. Farlo richiede la
riuscita di una prova di Forza DC 12, e la creatura che effettua il
tentativo subisce 10 (3d6) danni da acido.
Il cubo può contenere solo una creatura Grande o un massimo di
quattro creature Medie o più piccole alla volta.
Natura di Melma. Il cubo non necessita di dormire.
Trasparente. Anche quando è in piena vista, è necessario riuscire
una prova di Saggezza (Percezione) DC 15 per notare un cubo
che non si è mosso o non ha attaccato. Una creatura che cerchi di
entrare nello spazio del cubo mentre è inconsapevole della sua
presenza resta sorpresa dal cubo.
Azioni
Pseudopodo. Attacco con arma da mischia: +4 a colpire, portata
1,5 m, un bersaglio.
Colpisce: 10 (3d6) danni da acido.
Sommergere. Il cubo si muove fino al massimo della sua velocità.
Nel farlo, può entrare nello spazio di una creatura di taglia Grande o
più piccola. Ogni volta che il cubo entra nello spazio di una creatura,
la creatura deve effettuare un tiro salvezza di Destrezza DC 12.
Se il tiro salvezza riesce, la creatura può scegliere di essere spinta
indietro o di lato di 1,5 metri. Una creatura che decida di non farsi
spingere subisce le conseguenze di un tiro salvezza fallito.
Se il tiro salvezza fallisce, il cubo entra nello spazio della creatura,
che subisce 10 (3d6) danni da acido ed è sommersa. La creatura
sommersa non può respirare, è intralciata e subisce 21 (6d6) danni da
acido all’inizio del turno del cubo. Quando il cubo si muove, la
creatura sommersa si muove con esso.
Una creatura sommersa può tentare di fuggire effettuando un’azione
per compiere una prova di Forza DC 12. Se la riesce, la creatura
sfugge e ed entra nello spazio di sua scelta entro 1,5 metri dal cubo.
Melma Grigia
Media melma, disallineato
FORZA 12 (+1)
DESTREZZA 6 (-2)
COSTITUZIONE 16 (+3)
INTELLIGENZA 1 (-5)
SAGGEZZA 6 (-2)
Carisma 2 (-4)
Classe Armatura 8
\hspace*{0pt}\hfill{Punti Ferita}: 22 (3d8 + 9)
Velocità 3 m, scalata 3 m
Resistenze al Danno acido, freddo, fuoco
Immunità alle Condizioni accecato, affascinato, assordato,
prono, sfinimento, spaventato
Sensi vista cieca 18 m (cieca oltre questo raggio), Percezione
passiva 8
Linguaggi -
Sfida 1/2 (100 PE)
Amorfo. La melma può muoversi attraverso uno spazio fino a 2,5
centimetri di larghezza senza doversi stringere.
Corrodere Metallo. Qualsiasi arma non magica fatta di metallo che
colpisca la melma si corrode. Dopo aver inflitto il danno, l’arma
subisce una penalità permanente e cumulativa di -1 ai tiri di danno.
Se la penalità arriva a -5, l’arma è distrutta. Le munizioni non
magiche fatte di metallo che colpiscano la melma, si distruggono
dopo aver inflitto il danno.
La melma può divorare metallo non magico dello spessore di 5
centimetri in un 1 round.
Falso Aspetto. Quando la melma rimane immobile, è
indistinguibile da una pozza d’olio o una pietra bagnata.
Natura di Melma. La melma non necessita di dormire.
Azioni
Pseudopodo. Attacco con arma da mischia: +3 a colpire, portata
1,5 m, un bersaglio.
Colpisce: 4 (1d6 + 1) danni da botta più 7 (2d6) danni da
acido, e se il bersaglio sta indossando un’armatura di metallo,
questa viene parzialmente dissolta e subisce una penalità
permanente e cumulativa di -1 alla Difesa che offre. L’armatura è
distrutta se la penalità riduce la sua Difesa a 10.
 
Protoplasma Nero
Grande melma, disallineato
FORZA 16 (+3)
DESTREZZA 5 (-3)
COSTITUZIONE 16 (+3)
INTELLIGENZA 1 (-5)
SAGGEZZA 6 (-2)
Carisma 1 (-5)
Classe Armatura 7
\hspace*{0pt}\hfill{Punti Ferita}: 85 (10d10 + 30)
Velocità 6 m, scalata 6 m
Immunità al Danno acido, freddo, fulmine, tagliente
Immunità alle Condizioni accecato, affascinato, assordato,
prono, sfinimento, spaventato
Sensi vista cieca 18 m (cieco oltre questo raggio), Percezione
passiva 8
Linguaggi -
Sfida 4 (1.100 PE)
Amorfo. Il protoplasma nero può muoversi attraverso uno spazio
fino a 2,5 centimetri di larghezza senza doversi stringere.
Forma Corrosiva. Una creatura che entri a contatto col protoplasma
nero o lo colpisca con un attacco da mischia mentre si trova entro 1,5
metri da esso subisce 4 (1d8) danni da acido. Qualsiasi arma non
magica fatta di metallo o legno che colpisca il protoplasma nero si
corrode. Dopo aver inflitto il danno, l’arma subisce una penalità
permanente e cumulativa di -1 ai tiri di danno. Se la penalità arriva a
-5, l’arma è distrutta. Le munizioni non magiche fatte di metallo o
legno che colpiscano il protoplasma nero, si distruggono dopo aver
inflitto il danno.
Il protoplasma nero può divorare legno o metallo non magico dello
spessore di 5 centimetri in un 1 round.
Natura di Melma. Il protoplasma nero non necessita di dormire.
Scalare come Ragno. Il protoplasma nero può scalare superfici
difficili, compreso lo stare a testa in giù sul soffitto, senza
bisogno di effettuare una prova di abilità.
Azioni
Pseudopodo. Attacco con arma da mischia: +5 a colpire, portata
1,5 m, un bersaglio.
Colpisce: 6 (1d6 + 3) danni da botta più 18 (4d8) danni da
acido. Inoltre, un’armatura non magica indossata dal bersaglio
viene parzialmente dissolta e subisce una penalità permanente e
cumulativa di -1 alla Difesa che offre. L’armatura è distrutta se la
penalità riduce la sua Difesa a 10.
Reazioni
Divisione. Quando un protoplasma nero di taglia Media o più grande
subisce danni da fulmine o taglienti, si divide in due nuovi
protoplasma neri di almeno 10 punti ferita ciascuno. Ogni nuovo
protoplasma nero ha un numero di punti ferita pari alla metà del
protoplasma nero originale, arrotondati per difetto. I nuovi
protoplasmi neri sono di una taglia più piccola di quella originale.
Mezzo Drago
Modello del Mezzo Drago
Una bestia, umanoide, gigante o mostruosità può
diventare un mezzo drago. Quando una creatura
diventa un mezzo drago, mantiene tutte le sue
statistiche eccetto come indicato di seguito.
Sensi. I mezzi draghi ottengono vista cieca con un
raggio di 3 metri e scurovisione con un raggio di 18
metri.
Resistenze. Il mezzo drago ottiene resistenza ad un
tipo di danno in base al suo colore.
Colore Resistenza al Danno
Bianco o argento freddo
Blu o bronzo fulmine
Nero o rame acido
Oro, ottone o rosso fuoco
Verde veleno
Linguaggi. I mezzi draghi parlano Draconico oltre alle
lingue che già conoscono.
Gradi di Sfida. Per evitare di dover ricalcolare il grado
di sfida, applicare l’archetipo solo ad una creatura che
soddisfi i prerequisiti opzionali della tabella Arma a
Soffio seguente. Altrimenti, usare le linee guida in fondo
a questo documento per ricalcolare il grado di sfida
dopo aver applicato l’archetipo.
Nuova Azione: Soffio. Il mezzo drago possiede l’arma
a soffio della sua parte draconica. La taglia del mezzo
drago determina come funziona questa azione.
Taglia Soffio Prerequisito Opzionale
Grande o meno Come un cucciolo Sfida 2 o più
Enorme Come un drago giovane Sfida 7 o più
Mastodontica Come un drago adulto Sfida 8 o più
Mezzo Drago Rosso Veterano
Media umanoide (umano), qualsiasi allineamento
FORZA 16 (+3)
DESTREZZA 13 (+1)
COSTITUZIONE 14 (+2)
INTELLIGENZA 10 (+0)
SAGGEZZA 11 (+0)
Carisma 10 (+0)
Classe Armatura 18 (armatura di piastre)
\hspace*{0pt}\hfill{Punti Ferita}: 65 (10d8 + 20)
Velocità 9 m
Abilità Atletica +5, Percezione +2
Resistenze al Danno fuoco
Sensi scurovisione 18 m, vista cieca 3 m, Percezione passiva 12
Linguaggi Comune, Draconico
Sfida 5 (1.800 PE)
Azioni
Multiattacco. Il veterano effettua due attacchi con la spada
lunga. Se ha estratto una spada corta, può effettuare anche un
attacco con la spada corta.
Spada Lunga. Attacco con arma da mischia: +5 a colpire,
portata 1,5 m, un bersaglio.
Colpisce: 7 (1d8 + 3) danni taglienti, o 8 (1d10 + 3) danni
taglienti se usata con due mani.
Spada Corta. Attacco con arma da mischia +5 a colpire, portata
1,5 m, un bersaglio.
Colpisce: 6 (1d6 + 3) danni perforanti.
Balestra Pesante. Attacco con arma a Distanza: +3 a colpire,
gittata 30/120 m, un bersaglio.
Colpisce: 6 (1d10 + 1) danni perforanti.
Soffio Infuocato (Ricarica 5-6). Il veterano esala fuoco in un
cono di 4,5 metri. Ogni creatura in quell’area deve effettuare un
tiro salvezza di Destrezza DC 15 e subire 24 (7d6) danni da
fuoco se fallisce il tiro salvezza, o la metà di questi danni se lo
riesce.
Mimic
Media mostruosità (mutaforma), neutrale
FORZA 17 (+3)
DESTREZZA 12 (+1)
COSTITUZIONE 15 (+2)
INTELLIGENZA 5 (-3)
SAGGEZZA 13 (+1)
Carisma 8 (-1)
Classe Armatura 12 (armatura naturale)
\hspace*{0pt}\hfill{Punti Ferita}: 58 (9d8 + 18)
Velocità 4,5 m
Abilità Furtività +5
Immunità al Danno acido
Immunità alle Condizioni prono
Sensi scurovisione 18 m, Percezione passiva 11
Linguaggi -
Sfida 2 (450 PE)
Aderente (Solo Forma di Oggetto). Il mimic aderisce a qualsiasi
cosa con cui entri in contatto. Una creatura di taglia Enorme o
inferiore a cui il mimic aderisce è considerata afferrata da esso
(DC 13 per fuggire). Le prove di caratteristica effettuare per
fuggire da questo afferrare hanno svantaggio.
Afferratore. Il mimic ha vantaggio ai tiri per colpire contro una
creatura da esso afferrata.
Falso Aspetto (Solo Forma di Oggetto). Mentre il mimic rimane
immobile, è indistinguibile da un comune oggetto.
Mutaforma. Il mimic può usare la sua azione per trasformarsi in
un oggetto, o per tornare alla sua vera forma amorfa. Le sue
statistiche sono le stesse in qualsiasi forma. Qualsiasi
equipaggiamento stia indossando o trasportando non si
trasforma. Alla morte ritorna al suo vero aspetto.
Azioni
Morso. Attacco con arma da mischia: +5 a colpire, portata 1,5
m, un bersaglio.
Colpisce: 7 (1d8 + 3) danni perforanti più 4 (1d8) danni da
acido.
Pseudopodo. Attacco con arma da mischia: +5 a colpire, portata
1,5 m, un bersaglio.
Colpisce: 7 (1d8 + 3) danni da botta. Se il mimic è in forma
di oggetto, il bersaglio è vittima del tratto Aderente.
 
Minotauro
Grande mostruosità, caotico malvagio
FORZA 18 (+4)
DESTREZZA 11 (+0)
COSTITUZIONE 16 (+3)
INTELLIGENZA 6 (-2)
SAGGEZZA 16 (+3)
Carisma 9 (-1)
Classe Armatura 14 (armatura naturale)
\hspace*{0pt}\hfill{Punti Ferita}: 76 (9d10 + 27)
Velocità 12 m
Abilità Percezione +7
Sensi scurovisione 18 m, Percezione passiva 17
Linguaggi Abissale
Sfida 3 (700 PE)
Carica. Se il minotauro si muove di almeno 3 metri diretto verso
un bersaglio e lo colpisce con un attacco di incornata durante lo
stesso turno, il bersaglio subisce 9 (2d8) danni perforanti
aggiuntivi. Se il bersaglio è una creatura, deve riuscire un tiro
salvezza di Forza DC 14 o venire spinto via fino a 3 metri di
distanza e cadere prono.
Incauto. All’inizio del suo turno, il minotauro può ottenere
vantaggio su tutti i tiri per colpire con armi da mischia effettuati
durante quel turno, ma i tiri per colpire contro di esso hanno
vantaggio fino all’inizio del suo prossimo turno.
Ricordare Labirinto. Il minotauro può ricordare perfettamente
qualsiasi tragitto abbia percorso.
Azioni
Ascia Bipenne. Attacco con arma da mischia: +6 a colpire,
portata 1,5 m, un bersaglio.
Colpisce: 17 (2d12 + 4) danni taglienti.
Incornata. Attacco con arma da mischia: +6 a colpire, portata
1,5 m, un bersaglio.
Colpisce: 13 (2d8 + 4) danni perforanti.
Mummie
Mummia
Media non morto, legale malvagio
FORZA 16 (+3)
DESTREZZA 8 (-1)
COSTITUZIONE 15 (+2)
INTELLIGENZA 6 (-2)
SAGGEZZA 10 (+0)
Carisma 12 (+1)
Classe Armatura 11 (armatura naturale)
\hspace*{0pt}\hfill{Punti Ferita}: 58 (9d8 + 18)
Velocità 6 m
Tiri Salvezza Saggezza +2
Vulnerabilità al Danno fuoco
Resistenze al Danno da botta, perforante e tagliente di
attacchi non magici
Immunità al Danno necrotico, veleno
Immunità alle Condizioni affascinato, avvelenato, paralizzato,
sfinimento, spaventato
Sensi scurovisione 18 m, Percezione passiva 10
Linguaggi le lingue che conosceva in vita
Sfida 3 (700 PE)
Natura Non Morta. Un mummia non ha bisogno di aria, cibo,
bevande o sonno.
Azioni
Multiattacco. La mummia può usare la sua Occhiata Temibile ed
effettuare un attacco con il pugno putrefacente.
Pugno Putrefacente. Attacco con arma da mischia: +5 a colpire,
portata 1,5 m, un bersaglio.
Colpisce: 10 (2d6 + 3) danni da botta più 10 (3d6) danni
necrotici. Se il bersaglio è una creatura deve riuscire un tiro
salvezza di Costituzione DC 12 o venire maledetto dalla
putrefazione della mummia. Il bersaglio maledetto non può
recuperare punti ferita, e i suoi punti ferita massimi diminuiscono
di 10 (3d6) ogni 24 ore di durata della maledizione. Se la
maledizione riduce i punti ferita massimi del bersaglio a 0, il
bersaglio muore, e il suo corpo si tramuta in polvere. La
maledizione dura finché non viene rimossa dall’incantesimo
rimuovi maledizione o altra magia.
Occhiata Temibile. La mummia prende a bersaglio una creatura
che possa vedere e si trovi entro 18 metri da lei. Se il bersaglio
può vedere la mummia, deve riuscire un tiro salvezza di
Saggezza DC 11 contro questa magia o restare spaventato fino al
termine del prossimo turno della mummia. Se il bersaglio fallisce
il tiro salvezza di 5 o più, è anche paralizzato per la stessa durata.
Un bersaglio che riesca il tiro salvezza è immune all’Occhiata
Terribile di tutte le mummie (ma non delle mummie sovrane) per
le successive 24 ore.
Mummia Sovrana
Media non morto, legale malvagio
FORZA 18 (+4)
DESTREZZA 10 (+0)
COSTITUZIONE 17 (+3)
INTELLIGENZA 11 (+0)
SAGGEZZA 18 (+4)
Carisma 16 (+3)
Classe Armatura 17 (armatura naturale)
\hspace*{0pt}\hfill{Punti Ferita}: 97 (13d8 + 39)
Velocità 6 m
Tiri Salvezza Costituzione +8, Intelligenza +5, Saggezza +9,
Carisma +8
Abilità Religione +5, Storia +5
Vulnerabilità al Danno fuoco
Immunità al Danno necrotico, veleno; da botta, perforante
e tagliente di attacchi non magici
Immunità alle Condizioni affascinato, avvelenato, paralizzato,
sfinimento, spaventato
Sensi scurovisione 18 m, Percezione passiva 14
Linguaggi le lingue che conosceva in vita
Sfida 15 (13.000 PE)
Cuore della Mummia Sovrana. Come parte del rituale che crea
una mummia sovrana, il cuore e le viscere della creatura
vengono rimossi dal cadavere e piazzati all’interno di contenitori
sigillati. Questi contenitori sono di solito fatti in pietra o
ceramica, incisi o dipinti con geroglifici religiosi.
Finché il suo cuore avvizzito rimane intatto, la mummia sovrana
non può essere permanentemente distrutta. Quando scende a 0
punti ferita, la mummia sovrana si riduce in polvere e si riforma
a piena forza 24 ore più tardi, riemergendo dalla polvere in
prossimità della giara sigillata che contiene il suo cuore. Per
impedire che una mummia sovrana si riformi e distruggerla una
volta per tutte, bisogna ridurne il cuore in cenere. Per questo
motivo, la mummia sovrana di solito tiene il cuore e le viscere
nascoste all’interno di una tomba nascosta.
Il cuore della mummia sovrana ha Difesa 5, 25 punti ferita e
immunità a tutti i danni eccetto il fuoco.
Incantesimi. La mummia è un incantatore di 10° livello. La sua
caratteristica da incantatore è la Saggezza (DC dei tiri salvezza
degli incantesimi 17, +9 a colpire con attacchi da incantesimo).
La mummia ha preparati i seguenti incantesimi da chierico:
Trucchetti (a volontà): fiamma sacra, taumaturgia
1° livello (4 slot): comando, dardo tracciante, scudo della fede
2° livello (3 slot): arma spirituale, blocca persone, silenzio
3° livello (3 slot): animare morti, dissolvi magie
4° livello (3 slot): divinazione, guardiano della fede
5° livello (2 slot): contagio, piaga degli insetti
6° livello (1 slot): ferire
Natura Non Morta. Un mummia non ha bisogno di aria, cibo,
bevande o sonno.
Resistenza alla Magia. La mummia sovrana ha vantaggio ai tiri
salvezza contro incantesimi o altri effetti magici.
Rinvigorimento. Una mummia sovrana forma un nuovo corpo
entro 24 ore se il suo cuore resta intatto, recuperando tutti i punti
ferita e potendo agire nuovamente. Il nuovo corpo compare entro
1,5 metri dal cuore della mummia sovrana.
Azioni
Multiattacco. La mummia può usare la sua Occhiata Temibile ed
effettuare un attacco con il pugno putrefacente.
Pugno Putrefacente. Attacco con arma da mischia: +9 a colpire,
portata 1,5 m, un bersaglio.
Colpisce: 14 (3d6 + 4) danni da botta più 21 (6d6) danni
necrotici. Se il bersaglio è una creatura deve riuscire un tiro
salvezza di Costituzione DC 16 o venire maledetto dalla
putrefazione della mummia. Il bersaglio maledetto non può
recuperare punti ferita, e i suoi punti ferita massimi diminuiscono
di 10 (3d6) ogni 24 ore di durata della maledizione. Se la
maledizione riduce i punti ferita massimi del bersaglio a 0, il
bersaglio muore, e il suo corpo si tramuta in polvere. La
maledizione dura finché non viene rimossa dall’incantesimo
rimuovere maledizione o altra magia.
Occhiata Temibile. La mummia prende a bersaglio una creatura
che possa vedere e si trovi entro 18 metri da lei. Se il bersaglio
può vedere la mummia, deve riuscire un tiro salvezza di
Saggezza DC 16 contro questa magia o restare spaventato fino al
termine del prossimo turno della mummia. Se il bersaglio fallisce
il tiro salvezza di 5 o più, è anche paralizzato per la stessa durata.
Un bersaglio che riesca il tiro salvezza è immune all’Occhiata
Terribile di tutte le mummie (ma non delle mummie sovrane) per
le successive 24 ore.
Azioni Leggendarie
La mummia sovrana può effettuare 3 azioni aggiuntive, scelte
tra le opzioni seguenti. Può usare solo un’opzione leggendaria
alla volta e solo al termine del turno di un’altra creatura. La
mummia sovrana recupera le azioni aggiuntive spese all’inizio
del proprio round.
Attaccare. La mummia sovrana effettua un attacco con il pugno
putrefacente o usa la sua Occhiata Temibile.
Incanalare Energia Negativa (Costa 2 Azioni). La mummia
sovrana può scatenare magicamente l’energia negativa. Le
creature entro 18 metri dalla mummia sovrana, comprese quelle
dietro barriere o angoli, non possono recuperare punti ferita fino
al termine del prossimo turno della mummia sovrana.
Parola Blasfema (Costa 2 Azioni). La mummia sovrana
pronuncia una parola blasfema. Ciascuna creatura, esclusi i non
morti, entro 3 metri dalla mummia sovrana e che possa udire
questa frase magica deve riuscire un tiro salvezza di Costituzione
DC 16 o restare stordita fino al termine del prossimo turno della
mummia sovrana.
Polvere Accecante. Polvere e sabbia accecanti turbinano
magicamente intorno alla mummia sovrana. Ogni creatura entro
1,5 metri dalla mummia sovrana deve riuscire un tiro salvezza di
Costituzione DC 16 o restare accecata fino al termine del
prossimo turno della creatura.
Turbine di Sabbia (Costa 2 Azioni). La mummia sovrana può
trasformarsi magicamente in un turbine di sabbia, muovendosi di
massimo 18 metri, e tornando poi alla sua forma normale.
Mentre è in forma di turbine, la mummia sovrana è immune a
tutti i danni, e non può essere afferrata, pietrificata, gettata prona,
intralciata o stordita. L’equipaggiamento indossato o trasportato
dalla mummia sovrana rimane in suo possesso.
 
Naga
Naga Guardiano
Grande mostruosità, legale buono
FORZA 19 (+4)
DESTREZZA 18 (+4)
COSTITUZIONE 16 (+3)
INTELLIGENZA 16 (+3)
SAGGEZZA 19 (+4)
Carisma 18 (+4)
Classe Armatura 18 (armatura naturale)
\hspace*{0pt}\hfill{Punti Ferita}: 127 (15d10 + 45)
Velocità 12 m
Tiri Salvezza Destrezza +8, Costituzione +7, Intelligenza +7,
Saggezza +8, Carisma +8
Immunità ai Danni veleno
Immunità alle Condizioni affascinato, avvelenato
Sensi scurovisione 18 m, Percezione passiva 14
Linguaggi Celestiale, Comune
Sfida 10 (5.900 PE)
Incantesimi. Il naga è un incantatore di 11° livello. La sua
caratteristica da incantatore è la Saggezza (DC dei tiri salvezza
degli incantesimi 16, +8 a colpire con attacchi con incantesimo),
e ha bisogno solo delle componenti verbali per lanciare i suoi
incantesimi. Il naga prepara i seguenti incantesimi dalla lista
degli incantesimi da chierico:
Trucchetti (a volontà): fiamma sacra, riparare, taumaturgia
1° livello (4 slot): comando, cura ferite, scudo della fede
2° livello (3 slot): bloccare persone, calmare emozioni
3° livello (3 slot): chiaroveggenza, scagliare maledizione
4° livello (3 slot): esilio, libertà di movimento
5° livello (2 slot): colpo infuocato, costrizione
6° livello (1 slot): visione del vero
Rinvigorimento. Se muore, il naga ritorna in vita in 1d6 giorni e
recupera tutti i suoi punti ferita. Solo l’incantesimo desiderio può
impedire a questo tratto di funzionare.
Azioni
Morso. Attacco con arma da mischia: +8 a colpire, portata 3 m,
una creatura.
Colpisce: 8 (1d8 + 4) danni perforanti, e il bersaglio deve
effettuare un tiro salvezza di Costituzione DC 15, subendo 45
(10d8) danni da veleno se fallisce il tiro salvezza, o la metà di
questi danni se lo riesce.
Sputare Veleno. Attacco con arma a Distanza: +8 a colpire,
gittata 4,5/9 m, una creatura.
Colpisce: Il bersaglio deve effettuare un tiro salvezza di
Costituzione DC 15, subendo 45 (10d8) danni da veleno se
fallisce il tiro salvezza, o la metà di questi danni se lo riesce.
Naga Spirituale
Grande mostruosità, caotico malvagio
FORZA 18 (+4)
DESTREZZA 17 (+3)
COSTITUZIONE 14 (+2)
INTELLIGENZA 16 (+3)
SAGGEZZA 15 (+2)
Carisma 16 (+3)
Classe Armatura 15 (armatura naturale)
\hspace*{0pt}\hfill{Punti Ferita}: 75 (10d10 + 20)
Velocità 12 m
Tiri Salvezza Destrezza +6, Costituzione +5, Saggezza +5,
Carisma +6
Immunità al Danno veleno
Immunità alle Condizioni affascinato, avvelenato
Sensi scurovisione 18 m, Percezione passiva 12
Linguaggi Abissale, Comune
Sfida 8 (3.900 PE)
Incantesimi. Il naga è un incantatore di 10° livello. La sua abilità
da incantatore è l’Intelligenza (DC dei tiri salvezza degli
incantesimi 14, +6 a colpire con attacchi con incantesimo), e ha
bisogno solo delle componenti verbali per eseguire i suoi
incantesimi. Il naga prepara i seguenti incantesimi dalla lista
degli incantesimi da mago:
Trucchetti (a volontà): illusione minore, mano magica, raggio di
gelo
1° livello (4 slot): charme su persone, individuazione del magico,
sonno
2° livello (3 slot): blocca persone, individuazione dei pensieri
3° livello (3 slot): fulmine, respirare sott'acqua
4° livello (3 slot): inaridire, porta dimensionale
5° livello (2 slot): dominare persone
Rinvigorimento. Se muore, il naga ritorna in vita in 1d6 giorni e
recupera tutti i suoi punti ferita. Solo l’incantesimo desiderio può
impedire a questo tratto di funzionare.
Azioni
Morso. Attacco con arma da mischia: +7 a colpire, portata 3 m,
una creatura.
Colpisce: 7 (1d8 + 4) danni perforanti, e il bersaglio deve
effettuare un tiro salvezza di Costituzione DC 13, subendo 31
(7d8) danni da veleno se fallisce il tiro salvezza, o la metà di
questi danni se lo riesce.
Oggetti Animati
Armatura Animata
Media costrutto, disallineato
FORZA 14 (+2)
DESTREZZA 11 (+0)
COSTITUZIONE 13 (+1)
INTELLIGENZA 1 (-5)
SAGGEZZA 3 (-4)
Carisma 1 (-5)
Classe Armatura 18 (armatura naturale)
\hspace*{0pt}\hfill{Punti Ferita}: 33 (6d8 + 6)
Velocità 7,5 m
Immunità al Danno psichico, veleno
Immunità alle Condizioni accecato, affascinato, assordato,
avvelenato, paralizzato, pietrificato, sfinimento, spaventato
Sensi vista cieca 18 m (cieco oltre questo raggio), Percezione
passiva 6
Linguaggi -
Sfida 1 (200 PE)
Falso Aspetto. Mentre l’armatura rimane immobile, è
indistinguibile da una normale armatura.
Suscettibilità all’Anti Magia. L’armatura è inabile se si trova
nell’area di un campo anti-magia. Se è bersaglio di dissolvi
magie, l’armatura deve riuscire un tiro salvezza di Costituzione
contro la DC del tiro salvezza dell’incantesimo o restare svenuta
per 1 minuto.
Azioni
Multiattacco. L’armatura effettua due attacchi da mischia.
Schianto. Attacco con arma da mischia: +4 a colpire, portata 1,5
m, un bersaglio.
Colpisce: 5 (1d6 + 2) danni da botta.
Spada Volante
Piccola costrutto, disallineato
FORZA 12 (+1)
DESTREZZA 15 (+2)
COSTITUZIONE 11 (+0)
INTELLIGENZA 1 (-5)
SAGGEZZA 5 (-3)
Carisma 1 (-5)
Classe Armatura 17 (armatura naturale)
\hspace*{0pt}\hfill{Punti Ferita}: 17 (5d6)
Velocità 0 m, volo 15 m (fluttua)
Tiri Salvezza Destrezza +4
Immunità al Danno psichico, veleno
Immunità alle Condizioni accecato, affascinato, assordato,
avvelenato, paralizzato, pietrificato, spaventato
Sensi vista cieca 18 m (cieco oltre questo raggio), Percezione
passiva 7
Linguaggi -
Sfida 1/4 (50 PE)
Falso Aspetto. Mentre l’arma rimane immobile e non sta
volando, è indistinguibile da una normale spada.
Suscettibilità all’Anti Magia. La spada è inabile se si trova
nell’area di un campo anti-magia. Se è bersaglio di dissolvi
magie, la spada deve riuscire un tiro salvezza di Costituzione
contro la DC del tiro salvezza dell’incantesimo o restare svenuta
per 1 minuto.
Azioni
Spada Lunga. Attacco con arma da mischia: +3 a colpire,
portata 1,5 m, un bersaglio.
Colpisce: 5 (1d8 + 1) danni taglienti.
 
Tappeto del Soffocamento
Grande costrutto, disallineato
FORZA 17 (+3)
DESTREZZA 14 (+2)
COSTITUZIONE 10 (+0)
INTELLIGENZA 1 (-5)
SAGGEZZA 3 (-4)
Carisma 1 (-5)
Classe Armatura 12
\hspace*{0pt}\hfill{Punti Ferita}: 33 (6d10)
Velocità 3 m
Immunità al Danno psichico, veleno
Immunità alle Condizioni accecato, affascinato, assordato,
avvelenato, paralizzato, pietrificato, spaventato
Sensi vista cieca 18 m (cieco oltre questo raggio), Percezione
passiva 6
Linguaggi -
Sfida 2 (450 PE)
Falso Aspetto. Mentre il tappeto resta immobile, è indistinguibile
da un normale tappeto.
Suscettibilità all’Anti Magia. Il tappeto è inabile mentre si trova
nell’area di un campo anti-magia. Se è il bersaglio di dissolvi
magie, il tappeto deve riuscire un tiro salvezza di Costituzione
contro la DC del tiro salvezza dell’incantatore o cadere privo di
sensi per 1 minuto.
Trasferimento di Danno. Mentre afferra una creatura, il tappeto
subisce solo la metà dei danni che gli sono inferti, e la creatura
afferrata dal tappeto subisce l’altra metà.
Azioni
Soffocare. Attacco con arma da mischia: +5 a colpire, portata
1,5 m, una creatura di taglia Media o inferiore.
Colpisce: La creatura è afferrata (DC 13 per fuggire). Fino al
termine dell’afferrare, il bersaglio è intralciato, accecato e rischia
di soffocare, ma il tappeto non può soffocare un altro bersaglio.
Inoltre, all’inizio di ciascun turno del bersaglio, il bersaglio
subisce 10 (2d6 + 3) danni da botta.
Ogre
Grande gigante, caotico malvagio
FORZA 19 (+4)
DESTREZZA 8 (-1)
COSTITUZIONE 16 (+3)
INTELLIGENZA 5 (-3)
SAGGEZZA 7 (-2)
Carisma 7 (-2)
Classe Armatura 11 (armatura di pelle)
\hspace*{0pt}\hfill{Punti Ferita}: 59 (7d10 + 21)
Velocità 12 m
Sensi scurovisione 18 m, Percezione passiva 8
Linguaggi Comune, Gigante
Sfida 2 (450 PE)
Azioni
Randello Pesante. Attacco con arma da mischia: +6 a colpire,
portata 1,5 m, un bersaglio.
Colpisce: 13 (2d8 + 4) danni da botta.
Giavellotto. Attacco con arma da mischia o a Distanza: +6 a
colpire, portata 1,5 m o gittata 9/36 m, un bersaglio.
Colpisce: 11 (2d6 + 4) danni perforanti.
Ombra
Media non morto, caotico malvagio
FORZA 6 (-2)
DESTREZZA 14 (+2)
COSTITUZIONE 13 (+1)
INTELLIGENZA 6 (-2)
SAGGEZZA 10 (+0)
Carisma 8 (-1)
Classe Armatura 12
\hspace*{0pt}\hfill{Punti Ferita}: 16 (3d8 + 3)
Velocità 12 m
Abilità Furtività +4 (+6 a luce fioca o oscurità)
Vulnerabilità al Danno radiante
Resistenze al Danno acido, freddo, fulmine, fuoco, tuono;
da botta, perforante e tagliente di attacchi non magici
Immunità al Danno necrotico, veleno
Immunità alle Condizioni afferrato, avvelenato, intralciato,
paralizzato, pietrificato, prono, sfinimento, spaventato
Sensi scurovisione 18 m, Percezione passiva 10
Linguaggi -
Sfida 1/2 (100 PE)
Amorfo. L’ombra può muoversi attraverso uno spazio stretto
fino a 2,5 centimetri senza stringersi.
Debolezza alla Luce del Sole. Mentre si trova alla luce del sole,
l’ombra ha svantaggio ai tiri per colpire, le prove di abilità e i tiri
salvezza.
Furtività d’Ombra. Quando si trova a luce fioca o all’oscurità,
l’ombra può effettuare l’azione Nascondersi come azione bonus.
Natura Non Morta. Un’ombra non necessita aria, cibo, bevande
o sonno.
Azioni
Risucchio di Forza. Attacco con arma da mischia: +4 a colpire,
portata 1,5 m, una creatura.
Colpisce: 9 (2d6 + 2) danni necrotici, e il punteggio di Forza del
bersaglio viene ridotto di 1d4. Il bersaglio muore se ciò riduce la
sua Forza a 0. Altrimenti, la riduzione resta finché il bersaglio
termina un riposo breve o lungo.
Se un umanoide non malvagio muore a causa di questo attacco,
entro 1d4 ore dal suo cadavere si animerà una nuova ombra.
Omuncolo
Minuscola costrutto, neutrale
FORZA 4 (-3)
DESTREZZA 15 (+2)
COSTITUZIONE 11 (+0)
INTELLIGENZA 10 (+0)
SAGGEZZA 10 (+0)
Carisma 7 (-2)
Classe Armatura 13 (armatura naturale)
\hspace*{0pt}\hfill{Punti Ferita}: 5 (2d4)
Velocità 6 m, volo 12 m
Immunità al Danno veleno
Immunità alle Condizioni affascinato, avvelenato
Sensi scurovisione 18 m, vista cieca 3 m, Percezione passiva 10
Linguaggi comprende le lingue del suo creatore ma non può
parlare
Sfida 0 (10 PE)
Legame Telepatico. Mentre l’omuncolo si trova sullo stesso piano
di esistenza del suo padrone, può comunicare magicamente al suo
padrone quello che percepisce, e i due possono comunicare
telepaticamente.
Azioni
Morso. Attacco con arma da mischia: +4 a colpire, portata 1,5
m, una creatura.
Colpisce: 1 danno perforante, e il bersaglio deve riuscire un tiro
salvezza di Costituzione DC 10 o restare avvelenato per 1
minuto. Se il tiro salvezza viene fallito di 5 o più, il bersaglio
resta invece avvelenato per 5 (1d10) minuti e mentre è
avvelenato in questo modo è anche privo di sensi.
 
Oni
Grande gigante, legale malvagio
FORZA 19 (+4)
DESTREZZA 11 (+0)
COSTITUZIONE 16 (+3)
INTELLIGENZA 14 (+2)
SAGGEZZA 12 (+1)
Carisma 15 (+2)
Classe Armatura 16 (cotta di maglia)
\hspace*{0pt}\hfill{Punti Ferita}: 110 (13d10 + 39)
Velocità 9 m, volo 9 m
Tiri Salvezza Destrezza +3, Costituzione +6, Saggezza +4,
Carisma +5
Abilità Arcano +5, Inganno +8, Percezione +4
Sensi scurovisione 18 m, Percezione passiva 14
Linguaggi Comune, Gigante
Sfida 7 (2.900 PE)
Armi Magiche. Gli attacchi con armi dell’oni sono magici.
Incantesimi Innati. La caratteristica da incantatore dell’oni è il
Carisma (DC dei tiri salvezza degli incantesimi 13). L’oni può
lanciare questi incantesimi in maniera innata, senza bisogno di
componenti materiali:
A volontà: invisibilità, oscurità
1/giorno: charme su persone, cono di freddo, forma gassosa,
sonno
Rigenerazione. Se ha almeno 1 punto ferita, l’oni recupera 10
punti ferita all’inizio del suo turno.
Azioni
Multiattacco. L’oni effettua due attacchi, con gli artigli o con il
falcione.
Artiglio (Solo Forma di Oni). Attacco con arma da mischia: +7
a colpire, portata 1,5 m, un bersaglio.
Colpisce: 8 (1d8 + 4) danni taglienti.
Falcione. Attacco con arma da mischia: +7 a colpire, portata 3
m, un bersaglio.
Colpisce: 15 (2d10 + 4) danni taglienti, o 9 (1d10 + 4) danni
taglienti in forma Piccola o Media.
Mutare Forma. L’oni può trasformarsi magicamente in un
umanoide Piccolo o Medio, in un gigante Grande, o tornare alla sua
vera forma. A parte la taglia, le sue statistiche sono le stesse in
ciascuna forma. L’unico equipaggiamento che viene trasformato è il
falcione, che rimpicciolisce in modo da essere impugnato anche in
forma umanoide. Se l’oni muore, ritorna alla sua vera forma, e il
falcione ritorna alla sua taglia originale.
Orco
Media umanoide (orco), caotico malvagio
FORZA 16 (+3)
DESTREZZA 12 (+1)
COSTITUZIONE 16 (+3)
INTELLIGENZA 7 (-2)
SAGGEZZA 11 (+0)
Carisma 10 (+0)
Classe Armatura 13 (armatura di pelle)
\hspace*{0pt}\hfill{Punti Ferita}: 15 (2d8 + 6)
Velocità 9 m
Abilità Intimidire +2
Sensi scurovisione 18 m, Percezione passiva 10
Linguaggi Comune, Orco
Sfida 1/2 (100 PE)
Aggressivo. Come azione bonus, l’orco può muoversi fino a
metà della sua velocità verso una creatura ostile che possa
vedere.
Azioni
Ascia Bipenne. Attacco con arma da mischia: +5 a colpire,
portata 1,5 m, un bersaglio.
Colpisce: 9 (1d12 + 3) danni taglienti.
Giavellotto. Attacco con arma da mischia o a Distanza: +5 a
colpire, portata 1,5 m o gittata 9/36 m, un bersaglio.
Colpisce: 6 (1d6 + 3) danni perforanti.
Orsogufo
Grande mostruosità, disallineato
FORZA 20 (+5)
DESTREZZA 12 (+1)
COSTITUZIONE 17 (+3)
INTELLIGENZA 3 (-4)
SAGGEZZA 12 (+1)
Carisma 7 (-2)
Classe Armatura 13 (armatura naturale)
\hspace*{0pt}\hfill{Punti Ferita}: 59 (7d10 + 21)
Velocità 12 m
Abilità Percezione +3
Sensi scurovisione 18 m, Percezione passiva 13
Linguaggi -
Sfida 3 (700 PE)
Olfatto e Vista Affinati. L’orsogufo ha vantaggio nelle prove di
Saggezza (Percezione) basate su olfatto o vista.
Azioni
Multiattacco. L’orsogufo effettua due attacchi: uno con il becco
e uno con gli artigli.
Artigli. Attacco con arma da mischia: +7 a colpire, portata 1,5
m, un bersaglio.
Colpisce: 14 (2d8 + 5) danni taglienti.
Becco. Attacco con arma da mischia: +7 a colpire, portata 1,5 m,
una creatura.
Colpisce: 10 (1d10 + 5) danni perforanti.
Otyugh
Grande aberrazione, neutrale
FORZA 16 (+3)
DESTREZZA 11 (+0)
COSTITUZIONE 19 (+4)
INTELLIGENZA 6 (-2)
SAGGEZZA 13 (+1)
Carisma 6 (-2)
Classe Armatura 14 (armatura naturale)
\hspace*{0pt}\hfill{Punti Ferita}: 114 (12d10 + 48)
Velocità 9 m
Tiri Salvezza Costituzione +7
Sensi scurovisione 36 m, Percezione passiva 11
Linguaggi Otyugh
Sfida 5 (1.800 PE)
Telepatia Limitata. L’otyugh può trasmettere magicamente dei
semplici messaggi e immagini a qualsiasi creatura entro 36 metri
da esso e che possa comprendere una lingua. Questa forma di
telepatia non permette alla creatura ricevente di rispondere
telepaticamente.
Azioni
Multiattacco. L’otyugh effettua tre attacchi: uno con il morso e
due con i tentacoli.
Morso. Attacco con arma da mischia: +6 a colpire, portata 1,5
m, un bersaglio.
Colpisce: 12 (2d8 + 3) danni perforanti. Se il bersaglio è una
creatura, deve riuscire un tiro salvezza di Costituzione DC 15
contro malattia o restare avvelenato finché la malattia non viene
curata. Ogni 24 ore successive, il bersaglio deve ripetere il tiro
salvezza, riducendo il suo massimo di punti ferita di 5 (1d10) se
lo fallisce. Se il tiro salvezza riesce, la malattia è passata. Il
bersaglio muore se la malattia riduce i suoi punti ferita massimi a
0. Questa riduzione dei punti ferita massimi del personaggio,
perdura finché la malattia non viene curata.
Tentacolo. Attacco con arma da mischia: +6 a colpire, portata 3
m, un bersaglio.
Colpisce: 7 (1d8 + 3) danni da botta più 4 (1d8) danni
perforanti. Se il bersaglio è di taglia Media o inferiore, è
afferrato (DC 13 per fuggire) e intralciato fino al termine
dell’afferrare. L’otyugh ha due tentacoli, ciascun dei quali può
afferrare un bersaglio diverso.
Schianto di Tentacolo. L’otyugh schianta le creature afferrate
dai suoi tentacoli, l’una contro l’altra o sul pavimento. Ogni
creatura deve riuscire un tiro salvezza di Forza DC 14 o subire
10 (2d6 + 3) danni da botta e restare stordita fino al termine
del prossimo turno dell’otyugh. Se il tiro salvezza riesce, il
bersaglio subisce la metà dei danni da botta e non è stordito.
 
Pegaso
Grande celestiale, caotico buono
FORZA 18 (+4)
DESTREZZA 15 (+2)
COSTITUZIONE 16 (+3)
INTELLIGENZA 10 (+0)
SAGGEZZA 15 (+2)
Carisma 13 (+1)
Classe Armatura 12
\hspace*{0pt}\hfill{Punti Ferita}: 59 (7d10 + 21)
Velocità 18 m, volo 27 m
Tiri Salvezza Destrezza +4, Saggezza +4, Carisma +3
Abilità Percezione +6
Sensi Percezione passiva 16
Linguaggi comprende Celestiale, Comune, Elfico e Silvano ma
non può parlare
Sfida 2 (450 PE)
Azioni
Zoccoli. Attacco con arma da mischia: +6 a colpire, portata 1,5
m, un bersaglio.
Colpisce: 11 (2d6 + 4) danni da botta.
Persecutore Invisibile
Media elementale, neutrale
FORZA 16 (+3)
DESTREZZA 19 (+4)
COSTITUZIONE 14 (+2)
INTELLIGENZA 10 (+0)
SAGGEZZA 15 (+2)
Carisma 11 (+0)
Classe Armatura 14
\hspace*{0pt}\hfill{Punti Ferita}: 104 (16d8 + 32)
Velocità 15 m, volo 15 m (fluttua)
Abilità Furtività +10, Percezione +8
Resistenze al Danno da botta, perforante e tagliente di
attacchi non magici
Immunità ai Danni veleno
Immunità alle Condizioni afferrato, avvelenato, intralciato,
paralizzato, pietrificato, privo di sensi, prono, sfinimento
Sensi scurovisione 18 m, Percezione passiva 18
Linguaggi Auran, comprende il Comune ma non lo parla
Sfida 6 (2.300 PE)
Cacciatore Infallibile. Il convocatore assegna una preda al
persecutore. Il persecutore sa la direzione e la distanza a cui si
trova la preda finché entrambi si trovano sullo stesso piano di
esistenza. Il persecutore conosce anche la posizione del suo
convocatore.
Invisibilità. Il persecutore è invisibile.
Natura Elementale. Un persecutore invisibile non ha bisogno di
aria, cibo, bevande o sonno.
Azioni
Multiattacco. La persecutore effettua due attacchi di schianto.
Schianto. Attacco con arma da mischia: +6 a colpire, portata 1,5
m, un bersaglio.
Colpisce: 10 (2d6 + 3) danni da botta.
Pseudodrago
Minuscola drago, neutrale buono
FORZA 6 (-2)
DESTREZZA 15 (+2)
COSTITUZIONE 13 (+1)
INTELLIGENZA 10 (+0)
SAGGEZZA 12 (+1)
Carisma 10 (+0)
Classe Armatura 13 (armatura naturale)
\hspace*{0pt}\hfill{Punti Ferita}: 7 (2d4 + 2)
Velocità 4,5 m, volo 18 m
Abilità Furtività +4, Percezione +3
Sensi scurovisione 18 m, vista cieca 3 m, Percezione passiva 13
Linguaggi comprende il Comune e il Draconico ma non parla
Sfida 1/4 (50 PE)
Resistenza alla Magia. Lo pseudodrago ha vantaggio ai tiri
salvezza contro incantesimi e altri effetti magici.
Sensi Affinati. Lo pseudodrago ha vantaggio alle prove di
Saggezza (Percezione) basate su vista, udito e olfatto.
Telepatia Limitata. Lo pseudodrago può comunicare semplici
idee, emozioni e immagini telepaticamente con qualsiasi creatura
entro 30 metri da esso che può comprendere una lingua.
Azioni
Morso. Attacco con arma da mischia: +4 a colpire, portata 1,5
m, un bersaglio.
Colpisce: 4 (1d4 + 2) danni perforanti.
Pungiglione. Attacco con arma da mischia: +4 a colpire, portata
1,5 m, una creatura.
Colpisce: 4 (1d4 + 2) danni perforanti, e il bersaglio deve
riuscire un tiro salvezza di Costituzione DC 11 o restare
avvelenato per 1 ora. Se il risultato del tiro salvezza è 6 o meno,
il bersaglio cade privo di sensi per la stessa durata, o finché non
subisce danni o un’altra creatura usa un’azione per risvegliarlo.
Rakshasa
Media immondo, legale malvagio
FORZA 14 (+2)
DESTREZZA 17 (+3)
COSTITUZIONE 18 (+4)
INTELLIGENZA 13 (+1)
SAGGEZZA 16 (+3)
Carisma 20 (+5)
Classe Armatura 16 (armatura naturale)
\hspace*{0pt}\hfill{Punti Ferita}: 110 (13d8 + 52)
Velocità 12 m
Abilità Inganno +10, Intuizione +8
Vulnerabilità al Danno perforante di armi magiche impugnate
da creatura buone
Immunità al Danno da botta, perforante e tagliente di
attacchi non magici
Sensi scurovisione 18 m, Percezione passiva 13
Linguaggi Comune, Infernale
Sfida 13 (10.000 PE)
Immunità alla Magia Limitata. Il rakshasa è immune agli affetti
o all’individuazione tramite incantesimi di 6° livello o più basso
a meno che non desideri esserne soggetto. Ha vantaggio ai tiri
salvezza contro tutti gli altri incantesimi ed effetti magici.
Incantesimi Innati. La caratteristica da incantatore del rakshasa
è il Carisma (DC 18 per i tiri salvezza degli incantesimi, +10 a
colpire con attacchi con incantesimi). Il rakshasa può lanciare in
maniera innata i seguenti incantesimi, senza aver bisogno di
componenti materiali:
A volontà: camuffare sé stesso, illusione minore, individuazione
dei pensieri, mano magica
3/Giorno ciascuno: charme su persone, immagine maggiore,
individuazione del magico, invisibilità, suggestione
1/Giorno: dominare persone, spostamento planare, visione del
vero, volare
Azioni
Multiattacco. Il rakshasa può effettuare due attacchi di artiglio.
Artiglio. Attacco con arma da mischia: +7 a colpire, portata 1,5
m, un bersaglio.
Colpisce: 9 (2d6 + 2) danni taglienti, e se il bersaglio è una
creatura rimane maledetto. La maledizione magica ha effetto
ogni qualvolta il bersaglio effettua un riposo breve o lungo,
riempiendo i pensieri del bersaglio di immagini e sogni orribili.
Il bersaglio maledetto non riceve beneficio dall’aver terminato
un riposo breve o lungo. La maledizione perdura finché non
viene rimossa dall’incantesimo rimuovi maledizione o simile
magia.
 
Remorhaz
Enorme mostruosità, disallineato
FORZA 24 (+7)
DESTREZZA 13 (+1)
COSTITUZIONE 21 (+5)
INTELLIGENZA 4 (-3)
SAGGEZZA 10 (+0)
Carisma 5 (-3)
Classe Armatura 17 (armatura naturale)
\hspace*{0pt}\hfill{Punti Ferita}: 195 (17d12 + 85)
Velocità 9 m, scavo 6 m
Immunità ai Danni freddo, fuoco
Sensi scurovisione 18 m, senso tellurico 18 m, Percezione
passiva 10
Linguaggi -
Sfida 11 (7.200 PE)
Corpo Riscaldato. Una creatura che entri a contatto con il
remorhaz o lo colpisca con un attacco da mischia mentre si trova
entro 1,5 metri da esso, subisce 10 (3d6) danni da fuoco.
Azioni
Morso. Attacco in mischia con arma: +11 a colpire, portata 3 m,
un bersaglio.
Colpisce: 40 (6d10 + 7) danni perforanti più 10 (3d6) danni da
fuoco. Se il bersaglio è una creatura, è afferrato (DC 17 per
fuggire). Fino al termine dell’afferrare, il bersaglio è intralciato,
e il remorhaz non può attaccare con il morso un altro bersaglio.
Inghiottire. Il remorhaz effettua una attacco di morso contro un
bersaglio di taglia Media o inferiore che sta afferrando. Se
l’attacco colpisce, la creatura subisce il danno da morso ed è
inghiottita, e l’afferrare ha termine. Il bersaglio inghiottito è
accecato e intralciato, ha copertura totale contro gli attacchi e
altri effetti all’esterno del remorhaz, e subisce 21 (6d6) danni da
acido all’inizio di ciascun turno del remorhaz.
Se il remorhaz subisce 30 o più danni in un singolo turno da una
creatura al suo interno, il remorhaz deve riuscire un tiro salvezza
di Costituzione DC 15 al termine di quel turno o vomitare tutte le
creature inghiottite, che cadono prone in uno spazio entro 3 metri
dal remorhaz. Se il remorhaz muore, una creatura inghiottita non
è più intralciata da esso e può uscire dal cadavere utilizzando 4,5
metri di movimento, uscendo prona.
Rugginofago
Media Mostruosità, disallineato
FORZA 13 (+1)
DESTREZZA 12 (+1)
COSTITUZIONE 13 (+1)
INTELLIGENZA 2 (-4)
SAGGEZZA 13 (+1)
Carisma 6 (-2)
Classe Armatura 14 (armatura naturale)
\hspace*{0pt}\hfill{Punti Ferita}: 27 (5d8 + 5)
Velocità 12 m
Sensi scurovisione 18 m, Percezione passiva 11
Linguaggi -
Sfida 1/2 (100 PE)
Fiuto del Ferro. Il rugginofago può individuare, con l’olfatto,
l’esatta posizione di metalli ferrosi entro 9 metri.
Arrugginire Metallo. Qualsiasi arma non magica fatta di metallo
che colpisca il rugginofago si corrode. Dopo aver inflitto il
danno, l’arma subisce una penalità permanente e cumulativa di -
1 ai tiri di danno. Se la penalità scende fino a -5, l’arma è
distrutta. Le munizioni non magiche fatte di metallo e che
colpiscono il rugginofago, sono considerate distrutte dopo aver
inflitto il danno.
Azioni
Morso. Attacco con arma da mischia: +3 a colpire, portata 1,5
m, un bersaglio.
Colpisce: 5 (1d8 + 1) danni perforanti.
Antenne. Il rugginofago corrode gli oggetti di metallo ferroso
non magici che può vedere e si trovano entro 1,5 metri. Se
l’oggetto non è indossato o trasportato, il contatto col
rugginofago ne distrugge un cubo di 30 centimetri di spigolo. Se
l’oggetto è indossato o trasportato da una creatura, la creatura
può effettuare un tiro salvezza di Destrezza DC 11 per evitare il
contatto con il rugginofago.
Se l’oggetto con cui entra in contatto è un’armatura o scudo di
metallo indossati o trasportati, questi subiscono una penalità
permanente e cumulativa di -1 alla Difesa che forniscono. Le
armature ridotte a Difesa 10 o gli scudi che scendono ad un bonus di
+0 sono distrutti. Se l’oggetto con cui entra in contatto è un’arma
di metallo impugnata da qualcuno, la arrugginisce come descritto
nel tratto Arrugginire Metallo.
Sahuagin
Media umanoide (sahuagin), legale malvagio
FORZA 13 (+1)
DESTREZZA 11 (+0)
COSTITUZIONE 12 (+1)
INTELLIGENZA 12 (+1)
SAGGEZZA 13 (+1)
Carisma 9 (-1)
Classe Armatura 12 (armatura naturale)
\hspace*{0pt}\hfill{Punti Ferita}: 22 (4d8 + 4)
Velocità 9 m, nuoto 12 m
Abilità Percezione +5
Sensi scurovisione 36 m, Percezione passiva 15
Linguaggi Sahuagin
Sfida 1/2 (100 PE)
Anfibio Limitato. Il sahuagin può respirare aria e acqua, ma deve
restare sommerso almeno una volta ogni 4 ore per evitare di
soffocare.
Frenesia Sanguinaria. Il sahuagin ha vantaggio ai tiri per
colpire in mischia contro qualsiasi creatura che non sia al
massimo dei suoi punti ferita.
Telepatia con gli Squali. Il sahuagin può comandare
magicamente qualsiasi squalo entro 36 metri da sé, usando una
forma limitata di telepatia.
Azioni
Multiattacco. Il sahuagin può effettuare due attacchi da mischia:
uno con il morso e uno con gli artigli o la lancia.
Artigli. Attacco con arma da mischia: +3 a colpire, portata 1,5
m, un bersaglio.
Colpisce: 3 (1d4 + 1) danni taglienti.
Lancia. Attacco con arma da mischia o a Distanza: +3 a colpire,
portata 1,5 m o gittata 6/18 m, un bersaglio.
Colpisce: 4 (1d6 + 1) danni perforanti, o 5 (1d8 + 1) danni
perforanti se usata con due mani per effettuare un attacco da
mischia.
Morso. Attacco con arma da mischia: +3 a colpire, portata 1,5
m, un bersaglio.
Colpisce: 3 (1d4 + 1) danni perforanti.
Salamandra
Grande elementale, neutrale malvagio
FORZA 18 (+4)
DESTREZZA 14 (+2)
COSTITUZIONE 15 (+2)
INTELLIGENZA 11 (+0)
SAGGEZZA 10 (+0)
Carisma 12 (+1)
Classe Armatura 15 (armatura naturale)
\hspace*{0pt}\hfill{Punti Ferita}: 90 (12d10 + 24)
Velocità 9 m
Vulnerabilità al Danno freddo
Resistenze al Danno da botta, perforante e tagliente di
attacchi non magici
Immunità ai Danni fuoco
Sensi scurovisione 18 m, Percezione passiva 10
Linguaggi Ignan
Sfida 5 (1.800 PE)
Armi Riscaldate. Qualsiasi arma da mischia metallica che la
salamandra impugni infligge 3 (1d6) danni da fuoco aggiuntivi
per colpo (già incluso nell’attacco).
Corpo Riscaldato. Una creatura che entri a contatto con la
salamandra o la colpisce con un attacco da mischia mentre si
trova entro 1,5 metri da essa subisce 7 (2d6) danni da fuoco.
Azioni
Multiattacco. La salamandra effettua due attacchi: uno con la
lancia e uno con la coda.
Coda. Attacco con arma da mischia: +7 a colpire, portata 3 m,
un bersaglio.
Colpisce: 11 (2d6 + 4) danni da botta più 7 (2d6) danni da
fuoco, e il bersaglio è afferrato (DC 14 per fuggire). Fino al
termine dell’afferrare, il bersaglio è intralciato, la salamandra
può colpire automaticamente il bersaglio con la coda, e la
salamandra non può effettuare attacchi di coda contro altri
bersagli.
Lancia. Attacco con arma da mischia o a Distanza: +7 a colpire,
portata 1,5 m, gittata 6/18 m, un bersaglio.
Colpisce: 11 (2d6 + 4) danni perforanti, o 13 (2d8 +4) danni
perforanti se usata con due mani per effettuare un attacco da
mischia, più 3 (1d6) danni da fuoco.
 
Satiro
Media fatato, caotico neutrale
FORZA 12 (+1)
DESTREZZA 16 (+3)
COSTITUZIONE 11 (+0)
INTELLIGENZA 12 (+1)
SAGGEZZA 10 (+0)
Carisma 14 (+2)
Classe Armatura 14 (armatura di cuoio)
\hspace*{0pt}\hfill{Punti Ferita}: 31 (7d8)
Velocità 12 m
Abilità Furtività +5, Intrattenere +6, Percezione +2
Sensi Percezione passiva 12
Linguaggi Comune, Elfico, Silvano
Sfida 1/2 (100 PE)
Resistenza alla Magia. Il satiro ha +1d6 ai tiri salvezza
contro incantesimi e altri effetti magici.
Azioni
Incornata. Attacco con arma da mischia: +3 a colpire, portata
1,5 m, un bersaglio.
Colpisce: 6 (2d4 + 1) danni da botta.
Spada Corta. Attacco con arma da mischia: +5 a colpire, portata
1,5 m, un bersaglio.
Colpisce: 6 (1d6 + 3) danni perforanti.
Arco Corto. Attacco con arma a Distanza: +5 a colpire, gittata
24/96 m, un bersaglio.
Colpisce: 6 (1d6 + 3) danni perforanti.
Scheletro
Media non morto, legale malvagio
FORZA 10 (+0)
DESTREZZA 14 (+2)
COSTITUZIONE 15 (+2)
INTELLIGENZA 6 (-2)
SAGGEZZA 8 (-1)
Carisma 5 (-3)
Classe Armatura 13 (pezzi di armatura)
\hspace*{0pt}\hfill{Punti Ferita}: 13 (2d8 + 4)
Velocità 9 m
Vulnerabilità al Danno da botta
Immunità al Danno veleno
Immunità alle Condizioni avvelenato, sfinimento
Sensi scurovisione 18 m, Percezione passiva 9
Linguaggi comprende tutte le lingue che parlava in vita ma non
può parlare
Sfida 1/4 (50 PE)
Natura Non Morta. Lo scheletro non necessita aria, cibo,
bevande o sonno.
Azioni
Spada Corta. Attacco con arma da mischia: +4 a colpire, portata
1,5 m, un bersaglio.
Colpisce: 5 (1d6 + 2) danni perforanti.
Arco Corto. Attacco con arma a Distanza: +4 a colpire, gittata
24/96 m, un bersaglio.
Colpisce: 5 (1d6 + 2) danni perforanti.
Scheletro di Cavallo da Guerra
Grande non morto, legale malvagio
FORZA 18 (+4)
DESTREZZA 12 (+1)
COSTITUZIONE 15 (+2)
INTELLIGENZA 2 (-4)
SAGGEZZA 8 (-1)
Carisma 5 (-3)
Classe Armatura 13 (pezzi di bardatura)
\hspace*{0pt}\hfill{Punti Ferita}: 22 (3d10 + 6)
Velocità 18 m
Vulnerabilità al Danno da botta
Immunità al Danno veleno
Immunità alle Condizioni avvelenato, sfinimento
Sensi scurovisione 18 m, Percezione passiva 9
Linguaggi -
Sfida 1/2 (100 PE)
Natura Non Morta. Lo scheletro non necessita aria, cibo,
bevande o sonno.
Azioni
Zoccoli. Attacco con arma da mischia: +6 a colpire, portata 1,5
m, un bersaglio.
Colpisce: 11 (2d6 + 4) danni da botta.
Scheletro di Minotauro
Grande non morto, legale malvagio
FORZA 18 (+4)
DESTREZZA 11 (+0)
COSTITUZIONE 15 (+2)
INTELLIGENZA 6 (-2)
SAGGEZZA 8 (-1)
Carisma 5 (-3)
Classe Armatura 12 (armatura naturale)
\hspace*{0pt}\hfill{Punti Ferita}: 67 (9d10 + 18)
Velocità 12 m
Vulnerabilità al Danno da botta
Immunità al Danno veleno
Immunità alle Condizioni avvelenato, sfinimento
Sensi scurovisione 18 m, Percezione passiva 9
Linguaggi comprende l’Abissale ma non può parlare
Sfida 2 (450 PE)
Carica. Se lo scheletro di minotauro si muove di almeno 3 metri
in linea retta verso il bersaglio e poi lo colpisce con un attacco di
incornata durante lo stesso turno, il bersaglio subisce 9 (2d8)
danni perforanti aggiuntivi. Se il bersaglio è una creatura, deve
riuscire un tiro salvezza di Forza DC 14 o venire spinto di 3
metri indietro e cadere prono.
Natura Non Morta. Lo scheletro non necessita aria, cibo,
bevande o sonno.
Azioni
Ascia Bipenne. Attacco con arma da mischia: +6 a colpire,
portata 1,5 m, un bersaglio.
Colpisce: 17 (2d12 + 4) danni taglienti.
Incornata. Attacco con arma da mischia: +6 a colpire, portata
1,5 m, un bersaglio.
Colpisce: 13 (2d8 + 4) danni perforanti.
Segugio Infernale
Media immondo, legale malvagio
FORZA 17 (+3)
DESTREZZA 12 (+1)
COSTITUZIONE 14 (+2)
INTELLIGENZA 6 (-2)
SAGGEZZA 13 (+1)
Carisma 6 (-2)
Classe Armatura 15 (armatura naturale)
\hspace*{0pt}\hfill{Punti Ferita}: 45 (7d8 + 14)
Velocità 15 m
Abilità Percezione +5
Immunità al Danno fuoco
Sensi scurovisione 18 m, Percezione passiva 15
Linguaggi comprende l’Infernale ma non può parlare
Sfida 3 (700 PE)
Udito e Olfatto Affinato. Il segugio ha vantaggio nelle prove di
Saggezza (Percezione) basate su udito od olfatto.
Tattiche di Branco. Il segugio ha vantaggio ai tiri per colpire
contro una creatura se almeno uno degli alleati del segugio si
trova entro 1,5 metri dalla creatura e quell’alleato non è inabile.
Azioni
Morso. Attacco con arma da mischia: +5 a colpire, portata 1,5
m, un bersaglio.
Colpisce: 7 (1d6 + 3) danni perforanti più 7 (2d6) danni da
fuoco.
Soffio Infuocato (Ricarica 5-6). Il segugio esala fuoco in un
cono di 4,5 metri. Ogni creatura in quell’area deve effettuare un
tiro salvezza di Destrezza DC 12, e subire 21 (6d6) danni da
fuoco se fallisce il tiro salvezza, o la metà di questi danni se lo
riesce.
 
Sfingi
Androsfinge
Grande mostruosità, legale neutrale
FORZA 22 (+6)
DESTREZZA 10 (+0)
COSTITUZIONE 20 (+5)
INTELLIGENZA 16 (+3)
SAGGEZZA 18 (+4)
Carisma 23 (+6)
Classe Armatura 17 (armatura naturale)
\hspace*{0pt}\hfill{Punti Ferita}: 199 (19d10 + 95)
Velocità 12 m, volo 18 m
Tiri Salvezza Destrezza +6, Costituzione +11, Intelligenza +9,
Saggezza +10
Abilità Arcano +9, Percezione +10, Religione +15
Immunità al Danno psichico; da botta, perforante e
tagliente di attacchi non magici
Immunità alle Condizioni affascinato, spaventato
Sensi visione del vero 36 m, Percezione passiva 20
Linguaggi Comune, Sfinge
Sfida 17 (18.000 PE)
Armi Magiche. Gli attacchi con armi della sfinge sono magici.
Imperscrutabile. La sfinge è immune a qualsiasi effetto in grado
di percepirne le emozioni o leggerne i pensieri, oltre che a
qualsiasi incantesimo di divinazione che rifiuti. Le prove di
Saggezza (Intuizione) per discernere le intenzioni o la sincerità
della sfinge hanno svantaggio.
Incantesimi. La sfinge è un incantatore di 12° livello. La sua
caratteristica da incantatore è la Saggezza (DC del tiro salvezza
degli incantesimi 18, +10 a colpire con attacchi con
incantesimo). Non ha bisogno di componenti materiali per
lanciare i suoi incantesimi. La sfinge tiene preparati i seguenti
incantesimi da chierico:
Trucchetti (a volontà): fiamma sacra, salvare i morenti,
taumaturgia
1° livello (4 slot): comando, individuazione del magico,
individuare male e bene
2° livello (3 slot): ristorare inferiore, zona di verità
3° livello (3 slot): dissolvi magie, linguaggi
4° livello (3 slot): esilio, libertà di movimento
5° livello (2 slot): colpo infuocato, ristorare superiore
6° livello (1 slot): banchetto degli eroi
Azioni
Multiattacco. La sfinge può effettuare due attacchi di artiglio.
Artiglio. Attacco con arma da mischia: +12 a colpire, portata 1,5
m, un bersaglio.
Colpisce: 17 (2d6 + 10) danni taglienti.
Ruggito (3/Giorno). La sfinge emette un ruggito magico. Ogni
volta che ruggisce prima di terminare un riposo lungo, il ruggito
è più forte e l’effetto è diverso, come dettagliato di seguito. Ogni
creatura entro 150 metri dalla sfinge e capace di udirne il ruggito
deve effettuare un tiro salvezza.
Primo Ruggito. Ogni creatura che fallisce un tiro salvezza di
Saggezza DC 18 resta spaventata per 1 minuto. Una creatura
spaventata può ripetere il tiro salvezza al termine di ciascun suo
turno, terminandone l’effetto per sé, se lo riesce.
Secondo Ruggito. Ogni creatura che fallisce un tiro salvezza di
Saggezza DC 18 resta assordata e spaventata per 1 minuto. Una
creatura spaventata è paralizzata e può ripetere il tiro salvezza al
termine di ciascun suo turno, terminandone l’effetto per sé, se lo
riesce.
Terzo Ruggito. Ogni creatura effettua un tiro salvezza di
Costituzione DC 18. Chi fallisce il tiro salvezza subisce 44
(8d10) danni da tuono ed è gettato prono. Se il tiro salvezza
riesce, la creatura subisce la metà di questi danni e non viene
gettata prona.
Azioni Leggendarie
La sfinge può effettuare 3 azioni aggiuntive, scelte tra le
opzioni seguenti. Può usare solo un’opzione leggendaria alla
volta e solo al termine del turno di un’altra creatura. La sfinge
recupera le azioni aggiuntive spese all’inizio del proprio round.
Attacco di Artiglio. La sfinge effettua un attacco di artiglio.
Eseguire un Incantesimo (Costa 3 Azioni). La sfinge lancia un
incantesimo dalla lista degli incantesimi preparati, utilizzando
uno slot incantesimo come di norma.
Teletrasporto (Costa 2 Azioni). La sfinge si teletrasporta
magicamente, insieme a tutto l’equipaggiamento che sta
indossando o trasportando, in uno spazio non occupato che possa
vedere, fino a 36 metri di distanza.
Ginosfinge
Grande mostruosità, legale neutrale
FORZA 18 (+4)
DESTREZZA 15 (+2)
COSTITUZIONE 16 (+3)
INTELLIGENZA 18 (+4)
SAGGEZZA 18 (+4)
Carisma 18 (+4)
Classe Armatura 17 (armatura naturale)
\hspace*{0pt}\hfill{Punti Ferita}: 136 (16d10 + 48)
Velocità 12 m, volo 18 m
Abilità Arcano +14, Percezione +9, Religione +9, Storia +14
Resistenze al Danno da botta, perforante e tagliente di
attacchi non magici
Immunità al Danno psichico
Immunità alle Condizioni affascinato, spaventato
Sensi visione del vero 36 m, Percezione passiva 19
Linguaggi Comune, Sfinge
Sfida 11 (7.200 PE)
Armi Magiche. Gli attacchi con armi della sfinge sono magici.
Imperscrutabile. La sfinge è immune a qualsiasi effetto in grado
di percepirne le emozioni o leggerne i pensieri, oltre che a
qualsiasi incantesimo di divinazione che rifiuti. Le prove di
Saggezza (Intuizione) per discernere le intenzioni o la sincerità
della sfinge hanno svantaggio.
Incantesimi. La sfinge è un incantatore di 9° livello. La sua
abilità da incantatore è l’Intelligenza (DC del tiro salvezza degli
incantesimi 17, +9 a colpire con attacchi da incantesimo). Non ha
bisogno di componenti materiali per eseguire i suoi incantesimi.
La sfinge tiene preparati i seguenti incantesimi da mago:
Trucchetti (a volontà): illusione minore, mano magica,
prestidigitazione
1° livello (4 slot): identificare, individuazione del magico, scudo
2° livello (3 slot): localizza oggetto, oscurità, suggestione
3° livello (3 slot): dissolvi magie, linguaggi, rimuovi maledizione
4° livello (3 slot): esilio, invisibilità superiore
5° livello (2 slot): conoscenza delle leggende
Azioni
Multiattacco. La sfinge può effettuare due attacchi di artiglio.
Artiglio. Attacco con arma da mischia: +9 a colpire, portata 1,5
m, un bersaglio.
Colpisce: 13 (2d8 + 4) danni taglienti.
Azioni Leggendarie
La sfinge può effettuare 3 azioni aggiuntive, scelte tra le
opzioni seguenti. Può usare solo un’opzione leggendaria alla
volta e solo al termine del turno di un’altra creatura. La sfinge
recupera le azioni aggiuntive spese all’inizio del proprio round.
Attacco di Artiglio. La sfinge effettua un attacco di artiglio.
Eseguire un Incantesimo (Costa 3 Azioni). La sfinge esegue un
incantesimo dalla lista degli incantesimi preparati, utilizzando
uno slot incantesimo come di norma.
Teletrasporto (Costa 2 Azioni). La sfinge si teletrasporta
magicamente, insieme a tutto l’equipaggiamento che sta
indossando o trasportando, in uno spazio non occupato che possa
vedere, fino a 36 metri di distanza.
Spiritello
Minuscola fatato, neutrale buono
FORZA 3 (-4)
DESTREZZA 18 (+4)
COSTITUZIONE 10 (+0)
INTELLIGENZA 14 (+2)
SAGGEZZA 13 (+1)
Carisma 11 (+0)
Classe Armatura 15 (armatura di cuoio)
\hspace*{0pt}\hfill{Punti Ferita}: 2 (1d4)
Velocità 3 m, volo 12 m
Abilità Furtività +8 (la prova è fatta con svantaggio se lo
spiritello sta volando), Percezione +3
Sensi Percezione passiva 13
Linguaggi Comune, Elfico, Silvano
Sfida 1/4 (50 PE)
Azioni
Spada Lunga. Attacco con arma da mischia: +2 a colpire,
portata 1,5 m, un bersaglio.
Colpisce: 1 danno tagliente.
Arco Corto. Attacco con arma a Distanza: +6 a colpire, gittata
12 m/48 m, un bersaglio.
Colpisce: 1 danno perforante. Se il bersaglio è una creatura, deve
riuscire un tiro salvezza di Costituzione DC 10 o restare
avvelenata per 1 minuto. Se il risultato di questo tiro salvezza è 5
o meno, il bersaglio cade privo di sensi per la stessa durata, o
finché subisce danni o un’altra creatura usa un’azione per
risvegliarlo.
Invisibilità. Lo spiritello resta invisibile finché non attacca o
termina la sua concentrazione. Qualsiasi cosa che lo spiritello
stia trasportando o indossando resta invisibile finché rimane in
contatto con lo spiritello.
Vista del Cuore. Lo spiritello entra in contatto con una creatura e
ne apprende l’attuale stato emotivo. Se il bersaglio fallisce un
tiro salvezza di Costituzione DC 10, lo spiritello apprende anche
l’allineamento della creatura. Celestiali, immondi e non morti
falliscono automaticamente questo tiro salvezza.
 
Strige (Uccello Stigeo)
Minuscola bestia, disallineato
FORZA 4 (-3)
DESTREZZA 16 (+3)
COSTITUZIONE 11 (+0)
INTELLIGENZA 2 (-4)
SAGGEZZA 8 (-1)
Carisma 6 (-2)
Classe Armatura 14 (armatura naturale)
\hspace*{0pt}\hfill{Punti Ferita}: 2 (1d4)
Velocità 3 m, volo 12 m
Sensi scurovisione 18 m, Percezione passiva 9
Linguaggi -
Sfida 1/8 (25 PE)
Azioni
Risucchio di Sangue. Attacco con arma da mischia: +5 a
colpire, portata 1,5 m, una creatura.
Colpisce: 5 (1d4 + 3) danni perforanti e lo strige si attacca al
bersaglio. Mentre è attaccato, lo strige non attacca. Invece,
all’inizio di ciascun turno dello strige, il bersaglio perde 5 (1d4 +
3) punti ferita a causa della perdita di sangue.
Lo strige può staccarsi spendendo 1,5 metri di movimento. Lo fa
automaticamente dopo aver risucchiato 10 punti ferita dal
bersaglio o alla morte del bersaglio. Una creatura, compreso il
bersaglio, può usare la sua azione per staccare lo strige.
Succube/Incubo
Media immondo (mutaforma), neutrale malvagio
FORZA 8 (-1)
DESTREZZA 17 (+3)
COSTITUZIONE 13 (+1)
INTELLIGENZA 15 (+2)
SAGGEZZA 12 (+1)
Carisma 20 (+5)
Classe Armatura 15 (armatura naturale)
\hspace*{0pt}\hfill{Punti Ferita}: 66 (12d8 + 12)
Velocità 9 m, volo 18 m
Abilità Furtività 5, Intuizione +5, Percezione +5, Persuasione
+9, Raggirare +9
Resistenze al Danno freddo, fulmine, fuoco, veleno;
da botta, perforante e tagliente di attacchi non magici
Sensi scurovisione 18 m, Percezione passiva 15
Linguaggi Abissale, Comune, Infernale, telepatia 18 m
Sfida 4 (1.100 PE)
Legame Telepatico. L’immondo ignora le restrizioni di raggio di
azione della sua telepatia quando comunica con una creatura che
ha affascinato. I due non sono neppure costretti a trovarsi sullo
stesso piano di esistenza.
Mutaforma. L’immondo può usare la sua azione per trasformarsi
in un umanoide di taglia Piccola o Media, o per tornare alla sua
vera forma. Senza le ali, l’immondo perde la velocità di volo. A
parte la taglia e la velocità, le sue statistiche sono le stesse in
tutte le forme. Qualsiasi equipaggiamento stia indossando o
trasportando non viene trasformato. Alla morte ritorna alla sua
vera forma.
Azioni
Artiglio (Solo Forma Immonda). Attacco con arma da mischia:
+5 a colpire, portata 1,5 m, un bersaglio.
Colpisce: 6 (1d6 + 3) danni taglienti.
Affascinare. Un umanoide visibile all’immondo entro 9 metri da
esso deve riuscire un tiro salvezza di Saggezza DC 15 o restare
magicamente affascinato per 1 giorno. Il bersaglio affascinato
obbedisce ai comandi verbali o telepatici dell’immondo. Se il
bersaglio subisce danni o riceve un comando suicida, può
ripetere il tiro salvezza, terminando l’effetto se lo riesce. Se il
bersaglio riesce il tiro salvezza contro l’effetto, o se l’effetto
termina, il bersaglio è immune all’Affascinare dell’immondo per
le successive 24 ore.
L’immondo può tenere affascinato solo un bersaglio alla volta.
Se ne affascina un altro, l’effetto sul bersaglio precedente
termina.
Bacio Risucchiante. L’immondo bacia una creatura affascinata o
una creatura consenziente. Il bersaglio deve effettuare un tiro
salvezza di Costituzione DC 15 contro questa magia, subendo 32
(5d10 + 5) danni psichici se lo fallisce, o la metà di questi danni
se lo riesce. I punti ferita massimi del bersaglio vengono ridotti
di un ammontare pari ai danni subiti. Questa riduzione perdura
finché il bersaglio non termina un riposo lungo. Il bersaglio
muore se questo effetto riduce i suoi punti ferita massimi a 0.
Forma Eterea. L’immondo entra magicamente nel Piano Etereo
dal Piano Materiale, e viceversa.
Tarrasque
Mastodontica mostruosità (titano), disallineato
FORZA 30 (+10)
DESTREZZA 11 (+0)
COSTITUZIONE 30 (+10)
INTELLIGENZA 3 (-4)
SAGGEZZA 11 (+0)
Carisma 11 (+0)
Classe Armatura 25 (armatura naturale)
\hspace*{0pt}\hfill{Punti Ferita}: 676 (33d20 + 330)
Velocità 12 m
Tiri Salvezza Intelligenza +5, Saggezza +9, Carisma +9
Immunità al Danno fuoco, veleno; da botta, perforante e
tagliente di attacchi non magici
Immunità alle Condizioni affascinato, avvelenato, paralizzato,
spaventato
Sensi vista cieca 36 m, Percezione passiva 10
Linguaggi -
Sfida 30 (155.000 PE)
Carapace Riflettente. Ogni volta che il tarrasque è il bersaglio di
un incantesimo dardo incantato, un incantesimo a linea, o un
incantesimo che richiede un tiro di attacco a gittata, tira un d6.
Da 1 a 5, il tarrasque lo ignora. Con 6, il tarrasque lo ignora, e
l’effetto viene riflesso contro l’incantatore come se fosse
originato dal tarrasque, trasformando l’incantatore nel bersaglio.
Mostro d’Assedio. Il tarrasque infligge danni doppi agli oggetti e
le strutture.
Resistenza Leggendaria (3/Giorno). Se il tarrasque fallisce un
tiro salvezza, può scegliere invece di riuscire.
Resistenza alla Magia. Il tarrasque ha +1d6 ai tiri salvezza
contro incantesimi o altri effetti magici.
Azioni
Multiattacco. Il tarrasque può usare la sua Presenza Spaventosa.
Poi effettua cinque attacchi: uno con il morso, due con gli artigli,
uno con le corna, e uno con la coda. Al posto del morso può
usare Inghiottire.
Artiglio. Attacco con arma da mischia: +19 a colpire, portata 4,5
m, un bersaglio.
Colpisce: 28 (4d8 + 10) danni taglienti.
Coda. Attacco con arma da mischia: +19 a colpire, portata 6 m,
un bersaglio.
Colpisce: 24 (4d6 + 10) danni da botta. Se il bersaglio è una
creatura, deve riuscire un tiro salvezza di Forza DC 20 o cadere
prona.
Corna. Attacco con arma da mischia: +19 a colpire, portata 3 m,
un bersaglio.
Colpisce: 32 (4d10 + 10) danni perforanti.
Morso. Attacco con arma da mischia: +19 a colpire, portata 3 m,
un bersaglio.
Colpisce: 36 (4d12 + 10) danni perforanti. Se il bersaglio è una
creatura, è afferrata (DC 20 per fuggire). Fino al termine
dell’afferrare, il bersaglio è intralciato, e il tarrasque non può
usare il morso contro un altro bersaglio.
Inghiottire. Il tarrasque effettua una attacco di morso contro un
bersaglio di taglia Grande o inferiore che sta afferrando. Se
l’attacco colpisce, il bersaglio è inghiottito, e l’afferrare ha
termine. Il bersaglio inghiottito è accecato e intralciato, ha
copertura totale contro gli attacchi e altri effetti all’esterno del
tarrasque, e subisce 56 (16d6) danni da acido all’inizio di ciascun
turno del tarrasque.
Se il tarrasque subisce 60 o più danni in un singolo turno da una
creatura al suo interno, il tarrasque deve riuscire un tiro salvezza
di Costituzione DC 30 al termine di quel turno o vomitare tutte le
creature inghiottite, che cadono prone in uno spazio entro 3 metri
dal tarrasque. Se il tarrasque muore, una creatura inghiottita non
è più intralciata da esso e può uscire dal cadavere utilizzando 9
metri di movimento, uscendo prona.
Presenza Spaventosa. Ogni creatura scelta dal tarrasque, che si
trovi entro 36 metri da esso e consapevole della sua presenza,
deve riuscire un tiro salvezza di Saggezza DC 17 o restare
spaventata per 1 minuto. Una creatura può ripetere il tiro
salvezza al termine di ciascun suo turno, con svantaggio se il
tarrasque è in linea di visuale, terminando l’effetto per sé, se lo
riesce. Se il tiro salvezza della creatura ha successo o l’effetto ha
termine per essa, la creatura è immune alla Presenza Spaventosa
del tarrasque per le successive 24 ore.
Azioni Leggendarie
Il tarrasque può effettuare 3 azioni aggiuntive, scelte tra le
opzioni seguenti. Può usare solo un’opzione leggendaria alla
volta e solo al termine del turno di un’altra creatura. Il tarrasque
recupera le azioni aggiuntive spese all’inizio del proprio round.
Attacco. Il tarrasque effettua un attacco di artiglio o di coda.
Masticare (Costa 2 Azioni). Il tarrasque effettua un attacco di
morso o usa Inghiottire.
Muoversi. Il tarrasque si muove fino a metà della sua velocità.
 
Testuggine Dragona
Mastodontica drago, neutrale
FORZA 25 (+7)
DESTREZZA 10 (+0)
COSTITUZIONE 20 (+5)
INTELLIGENZA 10 (+0)
SAGGEZZA 12 (+1)
Carisma 12 (+1)
Classe Armatura 20 (armatura naturale)
\hspace*{0pt}\hfill{Punti Ferita}: 341 (22d20 + 110)
Velocità 6 m, nuoto 12 m
Tiri Salvezza Destrezza +6, Costituzione +11, Saggezza +7
Sensi scurovisione 18 m, Percezione passiva 11
Linguaggi Aquan, Draconico
Sfida 17 (18.000 PE)
Anfibio. La testuggine dragona può respirare aria e acqua.
Azioni
Multiattacco. Il drago può effettuare tre attacchi: uno con il
morso e due con gli artigli. Può effettuare un attacco di coda al
posto di due attacchi di artiglio.
Artiglio. Attacco con arma da mischia: +13 a colpire, portata 3
m, un bersaglio.
Colpisce: 16 (2d8 + 7) danni taglienti.
Coda. Attacco con arma da mischia: +13 a colpire, portata 4,5
m, un bersaglio.
Colpisce: 26 (3d12 + 7) danni da botta. Se il bersaglio è una
creatura, deve riuscire un tiro salvezza di Forza DC 20 o venire
spinta di 3 metri lontano dalla testuggine dragona e cadere prona.
Morso. Attacco con arma da mischia: +13 a colpire, portata 4,5
m, un bersaglio.
Colpisce: 26 (3d12 + 7) danni perforanti.
Soffio di Vapore (Ricarica 5-6). La testuggine dragona esala un
vapore caldo in un cono di 18 metri. Ogni creatura in quell’area
deve effettuare un tiro salvezza di Costituzione DC 18 e subire
52 (15d6) danni da fuoco se fallisce il tiro salvezza, o la metà di
questi danni se lo riesce. Trovarsi sott’acqua non dà resistenza
contro questo tipo di danno.
Troll
Grande gigante, caotico malvagio
FORZA 18 (+4)
DESTREZZA 13 (+1)
COSTITUZIONE 20 (+5)
INTELLIGENZA 7 (-2)
SAGGEZZA 9 (-1)
Carisma 7 (-2)
Classe Armatura 15 (armatura naturale)
\hspace*{0pt}\hfill{Punti Ferita}: 84 (8d10 + 40)
Velocità 9 m
Abilità Percezione +2
Sensi scurovisione 18 m, Percezione passiva 12
Linguaggi Gigante
Sfida 5 (1.800 PE)
Olfatto Affinato. Il troll ha vantaggio alle prove di Saggezza
(Percezione) basate sull’olfatto.
Rigenerazione. Il troll recupera 10 punti ferita all’inizio del suo
turno. Se il troll subisce danno da acido o da fuoco, questo tratto
non funziona all’inizio del prossimo turno del troll. Il troll muore
solo se inizia il suo turno a 0 punti ferita e non può rigenerarsi.
Azioni
Multiattacco. Il troll può effettuare tre attacchi: uno con il morso
e due con gli artigli.
Artiglio. Attacco con arma da mischia: +7 a colpire, portata 1,5
m, un bersaglio.
Colpisce: 11 (2d6 + 4) danni taglienti.
Morso. Attacco con arma da mischia: +7 a colpire, portata 1,5
m, un bersaglio.
Colpisce: 7 (1d6 + 4) danni perforanti.
Uomo Acquatico
Media umanoide (uomo acquatico), neutrale
FORZA 10 (+0)
DESTREZZA 13 (+1)
COSTITUZIONE 12 (+1)
INTELLIGENZA 11 (+0)
SAGGEZZA 11 (+0)
Carisma 12 (+1)
Classe Armatura 11
\hspace*{0pt}\hfill{Punti Ferita}: 11 (2d8 + 2)
Velocità 3 m, nuoto 12 m
Abilità Percezione +2
Sensi Percezione passiva 12
Linguaggi Aquan, Comune
Sfida 1/8 (25 PE)
Anfibio. L’uomo acquatico può respirare aria e acqua.
Azioni
Lancia. Attacco con arma da mischia o a Distanza: +2 a colpire,
portata 1,5 m o gittata 6/18 m, un bersaglio.
Colpisce: 3 (1d6) danni perforanti, o 4 (1d8) danni perforanti se
usata con due mani per effettuare un attacco da mischia.
Uomo Albero (Treant)
Enorme pianta, caotico buono
FORZA 23 (+6)
DESTREZZA 8 (-1)
COSTITUZIONE 21 (+5)
INTELLIGENZA 12 (+1)
SAGGEZZA 16 (+3)
Carisma 12 (+1)
Classe Armatura 16 (armatura naturale)
\hspace*{0pt}\hfill{Punti Ferita}: 138 (12d12 + 60)
Velocità 9 m
Resistenze al Danno da botta, perforante
Vulnerabilità al Danno fuoco
Sensi Percezione passiva 13
Linguaggi Comune, Druidico, Elfico, Silvano
Sfida 9 (5.000 PE)
Falso Aspetto. Mentre l’uomo albero rimane immobile, è
indistinguibile da un normale albero.
Mostro d’Assedio. L’uomo albero infligge danni doppi agli
oggetti e le strutture.
Azioni
Multiattacco. L’uomo albero effettua due attacchi di schianto.
Schianto. Attacco con arma da mischia: +10 a colpire, portata
1,5 m, un bersaglio.
Colpisce: 16 (3d6 + 6) danni da botta.
Sasso. Attacco con arma a Distanza: +10 a colpire, gittata 18/54
m, un bersaglio.
Colpisce: 28 (4d10 + 6) danni da botta.
Animare Alberi (1/Giorno). L’uomo albero anima magicamente
uno o due alberi visibili entro 18 metri da lui. Questi albeti hanno
le stesse statistiche dell’ent, eccetto che hanno punteggio di
Intelligenza e Carisma 1 (-5), non possono parlare, e hanno solo
l’opzione di attacco Schianto. Un albero animato agisce come
alleato dell’uomo albero. L’albero resta per 1 giorno o finché
muore; finché l’uomo albero muore o si trova più di 36 metri
lontano dall’albero, o finché l’uomo albero non effettua
un’azione bonus per ritrasformarlo in un albero inanimato. Poi
l’albero prenderà radici, se possibile.
 
Uomo Magma
(Magmin)
Piccola elementale, caotico neutrale
FORZA 7 (-2)
DESTREZZA 15 (+2)
COSTITUZIONE 12 (+1)
INTELLIGENZA 8 (-1)
SAGGEZZA 11 (+0)
Carisma 10 (+0)
Classe Armatura 14 (armatura naturale)
\hspace*{0pt}\hfill{Punti Ferita}: 9 (2d6 + 2)
Velocità 9 m
Resistenze al Danno da botta, perforante e tagliente di
attacchi non magici
Immunità ai Danni fuoco
Sensi scurovisione 18 m, Percezione passiva 10
Linguaggi Ignan
Sfida 1/2 (100 PE)
Illuminazione Incendiaria. Come azione bonus, l’uomo magma
può accendere o spegnere le sue fiamme. Mentre la fiamma è
accesa, l’uomo magma irradia luce intensa in un raggio di 3
metri e luce fioca per ulteriori 3 metri.
Scoppio Mortale. Quando l’uomo magma muore, esplode in uno
scoppio di fuoco e magma. Ogni creatura entro 3 metri da esso
deve effettuare un tiro salvezza di Destrezza DC 11, subendo 7
(2d6) danni da fuoco se fallisce il tiro salvezza, o la metà di
questi danni se lo riesce. Gli oggetti infiammabili che non siano
indossati o trasportati e che si trovino nell’area, prendono fuoco.
Azioni
Tocco. Attacco con arma da mischia: +4 a colpire, portata 1,5 m,
un bersaglio.
Colpisce: 7 (2d6) danni da fuoco. Se il bersaglio è una creatura o
un oggetto infiammabile, questi prende fuoco. Fino a che una
creatura effettua un’azione per estinguere la fiamma, la creatura
subisce 3 (1d6) danni da fuoco al termine di ciascun suo turno.
Unicorno
Grande celestiale, legale buono
FORZA 18 (+4)
DESTREZZA 14 (+2)
COSTITUZIONE 15 (+2)
INTELLIGENZA 11 (+0)
SAGGEZZA 17 (+3)
Carisma 16 (+3)
Classe Armatura 12
\hspace*{0pt}\hfill{Punti Ferita}: 67 (9d10 + 18)
Velocità 15 m
Immunità al Danno veleno
Immunità alle Condizioni affascinato, avvelenato, paralizzato
Sensi scurovisione 18 m, Percezione passiva 13
Linguaggi Celestiale, Elfico, Silvano, telepatia 18 m
Sfida 5 (1.800 PE)
Armi Magiche. Gli attacchi con armi dell’unicorno sono magici.
Carica. Se l’unicorno si muove di almeno 6 metri in linea retta
verso il bersaglio e lo colpisce con un attacco di corno durante lo
stesso turno, il bersaglio subisce 9 (2d8) danni perforanti
aggiuntivi. Se il bersaglio è una creatura, deve riuscire un tiro
salvezza di Forza DC 15 o cadere prono.
Incantesimi Innati. La caratteristica da incantatore innato
dell’unicorno è il Carisma (DC 14 per i tiri salvezza degli
incantesimi). L’unicorno può lanciare in maniera innata i
seguenti incantesimi, senza bisogno di componenti:
A volontà: arte del druido, individuazione del bene e male,
passare senza tracce
1/giorno ciascuno: calmare emozioni, dissolvi il bene e il male,
intralciare
Resistenza alla Magia. L’unicorno ha +1d6 ai tiri salvezza
contro incantesimi e altri effetti magici.
Azioni
Multiattacco. L’unicorno effettua due attacchi: uno con gli
zoccoli e uno con il corno.
Corno. Attacco con arma da mischia: +7 a colpire, portata 1,5 m,
un bersaglio.
Colpisce: 8 (1d8 + 4) danni perforanti.
Zoccoli. Attacco con arma da mischia: +7 a colpire, portata 1,5
m, un bersaglio.
Colpisce: 11 (2d6 + 4) danni da botta.
Telestraporto (1/Giorno). L’unicorno può teletrasportare
magicamente sé stesso e fino a tre altre creature consenzienti
visibili entro 1,5 metri da esso, insieme a tutto
l’equipaggiamento che stanno indossando o trasportando, in un
luogo familiare all’unicorno, che si trova ad un massimo di 1,5
chilometri di distanza.
Tocco Guaritore (3/Giorno). L’unicorno entra a contatto tramite
il corno con un’altra creatura. Il bersaglio recupera magicamente
11 (2d8 + 2) punti ferita. Inoltre, il contatto rimuove tutte le
malattie e neutralizza tutti i veleni che affliggono il bersaglio.
Azioni Leggendarie
L’unicorno può effettuare 3 azioni aggiuntive, scelte tra le
opzioni seguenti. Può usare solo un’opzione leggendaria alla
volta e solo al termine del turno di un’altra creatura. L’unicorno
recupera le azioni aggiuntive spese all’inizio del proprio round.
Autoguarigione (Costa 3 Azioni). L’unicorno recupera
magicamente 11 (2d8 + 2) punti ferita.
Scudo Scintillante (Costa 2 Azioni). L’unicorno crea un campo
magico scintillante che circonda lui o un’altra creatura visibile a lui
entro 18 metri. Il bersaglio ottiene un bonus di +2 alla Difesa fino al
termine del prossimo turno dell’unicorno.
Zoccoli. L’unicorno effettua un attacco con gli zoccoli.
Vampiri
Vampiro
Media non morto (mutaforma), legale malvagio
FORZA 18 (+4)
DESTREZZA 18 (+4)
COSTITUZIONE 18 (+4)
INTELLIGENZA 17 (+3)
SAGGEZZA 15 (+2)
Carisma 18 (+4)
Classe Armatura 16 (armatura naturale)
\hspace*{0pt}\hfill{Punti Ferita}: 144 (17d8 + 68)
Velocità 9 m
Tiri Salvezza Destrezza +9, Saggezza +7, Carisma +9
Abilità Furtività +9, Percezione +17
Immunità al Danno necrotico; da botta, perforante e
tagliente di attacchi non magici
Sensi scurovisione 36 m, Percezione passiva 17
Linguaggi le lingue che conosceva in vita
Sfida 13 (10.000 PE)
Mutaforma. Se il vampiro non è sotto la luce del sole o immerso
in acqua corrente, può usare la sua azione per trasformarsi in un
Minuscolo pipistrello, una nube di foschia Media, o per tornare
alla sua vera forma.
Mentre è in forma di pipistrello, il vampiro non può parlare, la
sua velocità di passeggio è 1,5 metri e ha velocità di volo 9
metri. Le sue statistiche, a parte la taglia e la velocità, sono
immutate. Qualsiasi equipaggiamento stia indossando si
trasforma con esso, ma quello che stava trasportando viene fatto
cadere a terra. Alla morte ritorna alla sua vera forma.
Mentre è in forma di foschia, il vampiro non può effettuare
azioni, parlare o manipolare oggetti. È privo di peso, ha velocità
di volo 6 metri, può fluttuare, e può entrare nello spazio di una
creatura ostile e fermarsi lì. Inoltre, se in uno spazio vi passa
dell’aria, la foschia può fare altrettanto senza stringersi, ma non
può attraversare l’acqua. ha +1d6 ai tiri salvezza di Forza,
Destrezza e Costituzione, ed è immune a tutti i danni non magici,
eccetto i danni subiti dalla luce del sole.
Debolezze del Vampiro. Il vampiro ha i seguenti difetti:
Danneggiato dall’Acqua Corrente. Il vampiro subisce 20 danni
da acido se termina il suo turno all’interno dell’acqua corrente.
Ipersensibilità alla Luce. Il vampiro subisce 20 danni da Luce
quando inizia il suo turno alla luce del sole. Mentre è alla luce
del sole, ha svantaggio ai tiri di attacco e le prove di abilità.
Paletto nel Cuore. Se un’arma perforante fatta di legno viene
conficcata nel cuore del vampiro mentre il vampiro è inabile nel
suo luogo di riposo, il vampiro resta paralizzato finché il paletto
non viene rimosso.
Proibizione. Il vampiro non può entrare in un’abitazione senza
invito da parte dei suoi occupanti.
Fuga nella Foschia. Quando scende a 0 punti ferita al di fuori
del suo luogo di riposo, il vampiro si trasforma in una nube di
foschia (come per il tratto Mutaforma) invece di cadere privo di
sensi, purché non sia esposto alla luce del sole o all’acqua
corrente. Se non può trasformarsi, viene distrutto.
Mentre si trova a 0 punti ferita in questa forma, non può tornare
alla sua forma di vampiro, e deve raggiungere il suo luogo di
riposo entro 2 ore o venire distrutto. Una volta raggiunto il suo
luogo di riposo, ritorna alla sua forma di vampiro. Resterà quindi
paralizzato finché non avrà recuperato almeno 1 punto ferita.
Dopo aver trascorso almeno 1 ora nel suo luogo di riposo a 0
punti ferita, il vampiro recupererà 1 punto ferita.
Natura Non Morta. Il vampiro non ha bisogno di aria.
Resistenza Leggendaria (3/Giorno). Se il vampiro fallisce un
tiro salvezza, può scegliere invece di riuscire.
 
Rigenerazione. Il vampiro recupera 20 punti ferita all’inizio del
suo turno se possiede almeno 1 punto ferita e non è esposto alla
luce del sole o l’acqua corrente. Se il vampiro subisce danno
radiante o danno dall’acqua sacra, questo tratto non funziona
all’inizio del prossimo turno del vampiro.
Scalare come Ragno. Il vampiro può scalare superfici difficili,
compreso lo stare a testa in giù sul soffitto, senza bisogno di
effettuare una prova di abilità.
Azioni
Multiattacco. Il vampiro può effettuare due attacchi, ma solo uno
di essi può essere un attacco con morso.
Colpo Disarmato (Solo in Forma di Vampiro). Attacco con
arma da mischia: +9 a colpire, portata 1,5 m, una creatura.
Colpisce: 8 (1d8 + 4) danni da botta. Invece di infliggere
danno, il vampiro può afferrare il bersaglio (DC per fuggire 18).
Morso (Solo in Forma di Pipistrello o Vampiro). Attacco con
arma da mischia: +9 a colpire, portata 1,5 m, una creatura
consenziente o una creatura afferrata dal vampiro, inabile o
intralciata.
Colpisce: 7 (1d6 + 4) danni perforanti più 10 (3d6) danni
necrotici. I punti ferita massimi del bersaglio sono ridotti di un
ammontare pari al danno necrotico subito, e il vampiro recupera
un numero di punti ferita pari a quell’ammontare. Questa
riduzione permane finché il bersaglio non termina un riposo
lungo. Il bersaglio muore se questo effetto riduce i suoi punti
ferita massimi a 0. Un umanoide ucciso in questo modo e poi
sepolto nel terreno si rianima la notte seguente come progenie
vampirica sotto il controllo del vampiro.
Affascinare. Il vampiro prende a bersaglio un umanoide entro 9
metri che può vedere. Se il bersaglio può vedere il vampiro, deve
effettuare un tiro salvezza di Saggezza DC 17 contro questa
magia o esserne affascinato. Il bersaglio affascinato considera il
vampiro un amico fidato da ascoltare e proteggere. Sebbene il
bersaglio non sia sotto il controllo del vampiro, prende le
richieste e le azioni del vampiro nel modo più favorevole
possibile, ed è un bersaglio consenziente dell’attacco con morso
del vampiro.
Ogni volta che il vampiro o i compagni del vampiro fanno
qualcosa di nocivo al bersaglio, questi può ripetere il tiro
salvezza, terminando l’effetto su di sé in caso di successo.
Altrimenti, l’effetto persiste 24 ore o finché il vampiro non viene
distrutto, si trova su di un piano di esistenza diverso dal
bersaglio, o effettua un’azione bonus per terminare l’effetto.
Figli della Notte (1/Giorno). Il vampiro richiama magicamente
2d4 sciami di pipistrelli o ratti, purché il sole non sia sorto.
Mentre è all’esterno, il vampiro può richiamare invece 3d6 lupi.
Le creature richiamate arrivano in 1d4 round, agendo da alleati
del vampiro e obbedendo ai suoi comandi. Le bestie restano per
1 ora, finché il vampiro non muore, o finché non le congeda con
un’azione bonus.
Azioni Leggendarie
Il vampiro può effettuare 3 azioni aggiuntive, scelte tra le
opzioni seguenti. Può usare solo un’opzione leggendaria alla
volta e solo al termine del turno di un’altra creatura. Il vampiro
recupera all’inizio del proprio round le azioni aggiuntive che ha
speso.
Colpo Disarmato. Il vampiro effettua un colpo disarmato.
Morso (Costa 2 Azioni). Il vampiro effettua un attacco con
morso.
Muoversi. Il vampiro si muove della sua velocità senza
provocare attacchi di opportunità.
Progenie Vampirica
Media non morto, neutrale malvagio
Classe Armatura 15 (armatura naturale)
\hspace*{0pt}\hfill{Punti Ferita}: 82 (11d8 + 33)
Velocità 9 m
FORZA 16 (+3)
DESTREZZA 16 (+3)
COSTITUZIONE 16 (+3)
INTELLIGENZA 11 (+0)
SAGGEZZA 10 (+0)
Carisma 12 (+1)
Tiri Salvezza Destrezza +6, Saggezza +3
Abilità Furtività +6, Percezione +3
Resistenze ai Danni necrotico; da botta, perforante e
tagliente di attacchi non magici
Sensi scurovisione 18 m, Percezione passiva 13
Linguaggi le lingue che conosceva in vita
Sfida 5 (1.800 PE)
Debolezze della Progenie Vampirica. La Progenie Vampirica ha
i seguenti difetti:
Danneggiato dall’Acqua Corrente. La Progenie Vampirica
subisce 20 danni da acido se termina il suo turno all’interno
dell’acqua corrente.
Ipersensibilità alla Luce. La Progenie Vampirica subisce 20
danni da Luce quando inizia il suo turno alla luce del sole.
Mentre è alla luce del sole, ha svantaggio ai tiri di attacco e le
prove di abilità.
Paletto nel Cuore. La Progenie Vampirica è distrutto se un’arma
perforante di legno gli viene conficcata nel cuore mentre è
inabile all’interno del suo luogo di riposo.
Proibizione. La Progenie Vampirica non può entrare in
un’abitazione senza invito da parte dei suoi occupanti.
Natura Non Morta. La Progenie Vampirica non ha bisogno di
aria.
Rigenerazione. La Progenie Vampirica recupera 10 punti ferita
all’inizio del suo turno se possiede almeno 1 punto ferita e non è
esposto alla luce del sole o l’acqua corrente. Se la Progenie
Vampirica subisce danno radiante o danno dall’acqua sacra,
questo tratto non funziona all’inizio del prossimo turno del
vampiro.
Scalare come Ragno. La Progenie Vampirica può scalare
superfici difficili, compreso lo stare a testa in giù sul soffitto,
senza bisogno di effettuare una prova di abilità.
Azioni
Multiattacco. La progenie vampirica può effettuare due attacchi,
ma solo uno di essi può essere un attacco con morso.
Artigli. Attacco con arma da mischia: +6 a colpire, portata 1,5
m, una creatura.
Colpisce: 8 (2d4 + 3) danni taglienti. Invece di infliggere danno,
il vampiro può afferrare il bersaglio (DC per fuggire 13).
Morso. Attacco con arma da mischia: +6 a colpire, portata 1,5 m,
una creatura afferrata dal vampiro, inabile o intralciata.
Colpisce: 6 (1d6 + 3) danni perforanti più 7 (2d6) danni necrotici. I
punti ferita massimi del bersaglio sono ridotti di un ammontare pari
al danno necrotico subito, e il vampiro recupera un numero di punti
ferita pari a quell’ammontare. Questa riduzione permane finché il
bersaglio non termina un riposo lungo. Il bersaglio muore se questo
effetto riduce i suoi punti ferita massimi a 0.
Verme Purpureo
Mastodontica mostruosità, disallineato
FORZA 28 (+9)
DESTREZZA 7 (-2)
COSTITUZIONE 22 (+6)
INTELLIGENZA 1 (-5)
SAGGEZZA 8 (-1)
Carisma 4 (-3)
Classe Armatura 18 (armatura naturale)
\hspace*{0pt}\hfill{Punti Ferita}: 247 (15d20 + 90)
Velocità 15 m, scavo 9 m
Tiri Salvezza Costituzione +11, Saggezza +4
Sensi vista cieca 9 m, senso tellurico 18 m, Percezione passiva 9
Linguaggi -
Sfida 15 (13.000 PE)
Scavatore di Tunnel. Il verme può scavare attraverso la roccia
solida a metà della velocità di scavare e lascia un tunnel di 3
metri di diametro dietro di sè.
Azioni
Multiattacco. Il verme effettua due attacchi: uno con il morso e
uno con il pungiglione.
Morso. Attacco con arma da mischia: +9 a colpire, portata 3 m,
un bersaglio.
Colpisce: 22 (3d8 + 9) danni perforanti. Se il bersaglio è una
creatura di taglia Grande, deve riuscire un tiro salvezza di
Destrezza DC 19 o venire inghiottita dal verme. Mentre è
inghiottita, la creatura è accecata e intralciata, ha copertura totale
contro gli attacchi e altri effetti provenienti dall’esterno del
verme, e subisce 21 (6d6) danni da acido all’inizio di ciascun
turno del verme.
Se il verme subisce 30 o più danni in un singolo turno da una
creatura al suo interno, il verme deve riuscire un tiro salvezza di
Costituzione DC 21 al termine del suo turno o vomitare tutte le
creature inghiottite, che cadono prone in uno spazio entro 3 metri
dal verme. Se il verme muore, una creatura inghiottita non risulta
più intralciata da esso e può fuggire dal cadavere usando 6 metri
di movimento, uscendo prona.
Pungiglione. Attacco con arma da mischia: +9 a colpire, portata
3 m, una creatura.
Colpisce: 19 (3d6 + 9) danni perforanti, e il bersaglio deve
effettuare un tiro salvezza di Costituzione DC 19, subendo 42
(12d6) danni da veleno se fallisce il tiro salvezza, o la metà di
questi danni se lo riesce.
Viverna
Grande drago, disallineato
FORZA 19 (+4)
DESTREZZA 10 (+0)
COSTITUZIONE 16 (+3)
INTELLIGENZA 5 (-3)
SAGGEZZA 12 (+1)
Carisma 6 (-2)
Classe Armatura 13 (armatura naturale)
\hspace*{0pt}\hfill{Punti Ferita}: 110 (13d10 + 39)
Velocità 6 m, volo 24 m
Abilità Percezione +4
Sensi scurovisione 18 m, Percezione passiva 14
Linguaggi -
Sfida 6 (2.300 PE)
Azioni
Multiattacco. La viverna può effettuare due attacchi: uno con il
morso e uno con il pungiglione. Mentre vola, può usare i suoi
artigli al posto di uno degli altri attacchi.
Artigli. Attacco con arma da mischia: +7 a colpire, portata 1,5
m, un bersaglio.
Colpisce: 13 (2d8 + 4) danni taglienti.
Morso. Attacco con arma da mischia: +7 a colpire, portata 3 m,
una creatura.
Colpisce: 11 (2d6 + 4) danni perforanti.
Pungiglione. Attacco con arma da mischia: +7 a colpire, portata
3 m, una creatura.
Colpisce: 11 (2d6 + 4) danni perforanti. Il bersaglio deve
effettuare un tiro salvezza di Costituzione DC 15, e subire 24
(7d6) danni da veleno se lo fallisce, o la metà di questi danni se
lo riesce.
 
Wight
Media non morto, neutrale malvagio
FORZA 15 (+2)
DESTREZZA 14 (+2)
COSTITUZIONE 16 (+3)
INTELLIGENZA 10 (+0)
SAGGEZZA 13 (+1)
Carisma 15 (+2)
Classe Armatura 14 (armatura borchiata)
\hspace*{0pt}\hfill{Punti Ferita}: 45 (6d8 + 18)
Velocità 9 m
Abilità Furtività +4, Percezione +3
Resistenze al Danno necrotico; da botta, perforante e
tagliente di attacchi non magici che non siano argentati
Immunità al Danno veleno
Immunità alle Condizioni avvelenato, sfinimento
Sensi scurovisione 18 m, Percezione passiva 13
Linguaggi le lingue che conosceva in vita
Sfida 3 (700 PE)
Natura Non Morta. Il wight non ha bisogno di aria, cibo,
bevande o sonno.
Sensibilità alla Luce. Mentre è alla luce del sole, il wight ha
svantaggio ai tiri di attacco, oltre che alle prove di Saggezza
(Percezione) basate sulla vista.
Azioni
Multiattacco. Il wight può effettuare due attacchi con la spada
lungha o due attacchi con l’arco lungo. Può usare Risucchiare
Vita al posto di uno dei suoi attacchi con la spada lungha.
Risucchiare Vita. Attacco con arma da mischia: +4 a colpire,
portata 1,5 m, una creatura.
Colpisce: 5 (1d6 + 2) danni necrotici. Il bersaglio deve riuscire
un tiro salvezza di Costituzione DC 13 o vedere i suoi punti
ferita massimi ridotti di un ammontare pari al danno subito.
Questa riduzione perdura finché il bersaglio non ha terminato un
riposo lungo. Il bersaglio muore se l’effetto riduce i suoi punti
ferita massimi a 0.
Un umanoide ucciso da questo attacco si rianima 24 ore più tardi
come zombi sotto il controllo del wight, a meno che l’umanoide
non venga prima riportato in vita o il corpo sia distrutto. Il wight
non può controllare più di dodici zombi alla volta.
Spada Lunga. Attacco con arma da mischia: +4 a colpire,
portata 1,5 m, un bersaglio.
Colpisce: 6 (1d8 + 2) danni taglienti o 7 (1d10 + 2) danni
taglienti se usata con due mani.
Arco Lungo. Attacco con arma a Distanza: +4 a colpire, gittata
45/180 m, un bersaglio.
Colpisce: 6 (1d8 + 2) danni perforanti.
Wraith
Media non morto, neutrale malvagio
FORZA 6 (-2)
DESTREZZA 16 (+3)
COSTITUZIONE 16 (+3)
INTELLIGENZA 12 (+1)
SAGGEZZA 14 (+2)
Carisma 15 (+2)
Classe Armatura 13
\hspace*{0pt}\hfill{Punti Ferita}: 67 (9d8 + 27)
Velocità 0 m, volo 18 m (fluttua)
Resistenze al Danno acido, freddo, fulmine, fuoco, tuono;
da botta, perforante e tagliente di attacchi non magici che
non siano argentati
Immunità al Danno necrotico, veleno
Immunità alle Condizioni affascinato, afferrato, avvelenato,
intralciato, paralizzato, pietrificato, prono, sfinimento
Sensi scurovisione 18 m, Percezione passiva 12
Linguaggi le lingue che conosceva in vita
Sfida 5 (1.800 PE)
Movimento Incorporeo. Il wraith può attraversare creature e
oggetti come fossero terreno difficile. Subisce 5 (1d10) danni da
forza se termina il proprio round all’interno di un oggetto.
Natura Non Morta. Il wraith non ha bisogno di aria, cibo,
bevande o sonno.
Sensibilità alla Luce. Mentre è alla luce del sole, il wraith ha
svantaggio ai tiri di attacco, oltre che alle prove di Saggezza
(Percezione) basate sulla vista.
Azioni
Risucchiare Vita. Attacco con arma da mischia: +6 a colpire,
portata 1,5 m, una creatura.
Colpisce: 21 (4d8 + 3) danni necrotici. Il bersaglio deve riuscire
un tiro salvezza di Costituzione DC 14 o vedere i suoi punti
ferita massimi ridotti di un ammontare pari al danno subito.
Questa riduzione perdura finché il bersaglio non ha terminato un
riposo lungo. Il bersaglio muore se l’effetto riduce i suoi punti
ferita massimi a 0.
Creare Spettro. Il wraith prende a bersaglio un umanoide entro 3
metri da esso e che sia morto da non più di 1 minuto e per cause
violente. Lo spirito del bersaglio si anima come spettro nello
spazio del suo cadavere e nello spazio più vicino non occupato.
Lo spettro è sotto il controllo del wraith. Il wraith non può tenere
più di sette spettri alla volta sotto il suo controllo.
Xorn
Media elementale, neutrale
FORZA 17 (+3)
DESTREZZA 10 (+0)
COSTITUZIONE 22 (+6)
INTELLIGENZA 11 (+0)
SAGGEZZA 10 (+0)
Carisma 11 (+0)
Classe Armatura 19 (armatura naturale)
\hspace*{0pt}\hfill{Punti Ferita}: 73 (7d8 + 42)
Velocità 6 m, scavo 6 m
Abilità Furtività +3, Percezione +6
Resistenze al Danno perforante e tagliente di attacchi non
magici che non siano di adamantio
Sensi scurovisione 18 m, senso tellurico 18 m, Percezione
passiva 18
Linguaggi Terran
Sfida 5 (1.800 PE)
Mimetismo di Pietra. Lo xorn ha vantaggio alle prove di
Destrezza (Furtività) effettuate per nascondersi su terreno
roccioso.
Scorrere sulla Terra. Lo xorn può scavare attraversa la terra e la
pietra non magiche e non lavorate. Quando lo fa, lo xorn non
disturba il materiale che sposta.
Senso del Tesoro. Lo xorn può individuare precisamente, con
l’olfatto, la posizione di metalli e pietre preziose, come monete e
gemme, entro 18 metri da esso.
Azioni
Multiattacco. Lo xorn effettua tre attacchi di artiglio e un attacco
di morso.
Artiglio. Attacco con arma da mischia: +6 a colpire, portata 1,5
m, un bersaglio.
Colpisce: 6 (1d6 + 3) danni taglienti.
Morso. Attacco con arma da mischia: +6 a colpire, portata 1,5
m, un bersaglio.
Colpisce: 13 (3d6 + 3) danni perforanti.
Zombi
Media non morto, neutrale malvagio
FORZA 13 (+1)
DESTREZZA 6 (-2)
COSTITUZIONE 16 (+3)
INTELLIGENZA 3 (-4)
SAGGEZZA 6 (-2)
Carisma 5 (-3)
Classe Armatura 8
\hspace*{0pt}\hfill{Punti Ferita}: 22 (3d8 + 9)
Velocità 6 m
Tiri Salvezza Saggezza +0
Immunità al Danno veleno
Immunità alle Condizioni avvelenato
Sensi scurovisione 18 m, Percezione passiva 8
Linguaggi comprende tutte le lingue che parlava in vita ma non
può parlare
Sfida 1/4 (50 PE)
Natura Non Morta. Lo zombi non ha bisogno di aria, cibo,
bevande o sonno.
Tempra dei Non Morti. Se il danno riduce lo zombi a 0 punti
ferita, lo zombi deve effettuare un tiro salvezza di Costituzione
DC 5 + il danno subito, a meno che il danno non sia radiante o
un colpo critico. Se riesce, lo zombi scende invece a 1 punto
ferita.
Azioni
Schianto. Attacco con arma da mischia: +3 a colpire, portata 1,5
m, un bersaglio.
Colpisce: 4 (1d6 + 1) danni da botta.
Zombi Ogre
Grande non morto, neutrale malvagio
FORZA 19 (+4)
DESTREZZA 6 (-2)
COSTITUZIONE 18 (+4)
INTELLIGENZA 3 (-4)
SAGGEZZA 6 (-2)
Carisma 5 (-3)
Classe Armatura 8
\hspace*{0pt}\hfill{Punti Ferita}: 85 (9d10 + 36)
Velocità 9 m
Tiri Salvezza Saggezza +0
Immunità al Danno veleno
Immunità alle Condizioni avvelenato
Sensi scurovisione 18 m, Percezione passiva 8
Linguaggi comprende Comune e Gigante ma non può parlare
Sfida 2 (450 PE)
Natura Non Morta. Lo zombi non ha bisogno di aria, cibo,
bevande o sonno.
Tempra dei Non Morti. Se il danno riduce lo zombi a 0 punti
ferita, lo zombi deve effettuare un tiro salvezza di Costituzione
DC 5 + il danno subito, a meno che il danno non sia radiante o
un colpo critico. Se riesce, lo zombi scende invece a 1 punto
ferita.
Azioni
Mazza Chiodata. Attacco con arma da mischia: +6 a colpire,
portata 1,5 m, un bersaglio.
Colpisce: 13 (2d8 + 4) danni da botta.
 
Appendice A: Creature
Varie
Questa appendice contiene le statistiche di vari animali,
parassiti e altre creature. Le statistiche sono
organizzate in ordine alfabetico.
Albero Risvegliato
L’albero risvegliato è un normale albero fornito dalla
magia di capacità senziente e mobilità.
Enorme pianta, disallineato
FORZA 19 (+4)
DESTREZZA 6 (-2)
COSTITUZIONE 15 (+2)
INTELLIGENZA 10 (+0)
SAGGEZZA 10 (+0)
Carisma 7 (-2)
Classe Armatura 13 (armatura naturale)
\hspace*{0pt}\hfill{Punti Ferita}: 59 (7d12 + 14)
Velocità 6 m
Vulnerabilità al Danno fuoco
Resistenze al Danno da botta, perforante
Sensi Percezione passiva 10
Lingue una lingua conosciuta dal suo creatore
Sfida 2 (450 PE)
Falso Aspetto. Mentre l’albero rimane immobile, è
indistinguibile da un normale albero.
Azioni
Schianto. Attacco con Arma da Mischia: +6 a colpire, portata 3
m, un bersaglio.
Colpisce: 14 (3d6 + 4) danni da botta.
Alce
Grande bestia, disallineato
FORZA 16 (+3)
DESTREZZA 10 (+0)
COSTITUZIONE 12 (+1)
INTELLIGENZA 2 (-4)
SAGGEZZA 10 (+0)
Carisma 6 (-2)
Classe Armatura 10
\hspace*{0pt}\hfill{Punti Ferita}: 13 (2d10 + 2)
Velocità 15 m
Sensi Percezione passiva 10
Lingue -
Sfida 1/4 (50 PE)
Carica. Se l’alce si muove di almeno 6 metri diretto verso il
bersaglio e lo colpisce con un attacco di rostro durante lo stesso
turno, il bersaglio subisce 7 (2d6) danni da botta aggiuntivi.
Se il bersaglio è una creatura, deve riuscire un tiro salvezza di
Forza DC 13 o cadere prono.
Azioni
Rostro. Attacco con Arma da Mischia: +5 a colpire, portata 1,5
m, un bersaglio.
Colpisce: 6 (1d6 + 3) danni da botta.
Zoccoli. Attacco con Arma da Mischia: +5 a colpire, portata 1,5
m, una creatura prona.
Colpisce: 8 (2d4 + 3) danni da botta.
Alce Gigante
Enorme bestia, disallineato
FORZA 19 (+4)
DESTREZZA 16 (+3)
COSTITUZIONE 14 (+2)
INTELLIGENZA 7 (-2)
SAGGEZZA 14 (+2)
Carisma 10 (+0)
Classe Armatura 14 (armatura naturale)
\hspace*{0pt}\hfill{Punti Ferita}: 42 (5d12 + 10)
Velocità 18 m
Abilità Percezione +4
Sensi Percezione passiva 14
Lingue Alce Gigante, comprende il Comune, l’Elfico e il
Silvano ma non può parlarli
Sfida 2 (450 PE)
Carica. Se l’alce si muove di almeno 6 metri diretto verso il
bersaglio e lo colpisce con un attacco di rostro durante lo stesso
turno, il bersaglio subisce 7 (2d6) danni da botta aggiuntivi.
Se il bersaglio è una creatura, deve riuscire un tiro salvezza di
Forza DC 14 o cadere prono.
Azioni
Rostro. Attacco con Arma da Mischia: +6 a colpire, portata 3 m,
un bersaglio.
Colpisce: 11 (2d6 + 4) danni perforanti.
Zoccoli. Attacco con Arma da Mischia: +6 a colpire, portata 1,5
m, una creatura prona.
Colpisce: 22 (4d4 + 4) danni da botta.
Aquila
Piccola bestia, disallineato
FORZA 6 (-2)
DESTREZZA 15 (+2)
COSTITUZIONE 10 (+0)
INTELLIGENZA 2 (-4)
SAGGEZZA 14 (+2)
Carisma 7 (-2)
Classe Armatura 12
\hspace*{0pt}\hfill{Punti Ferita}: 3 (1d6)
Velocità 3 m, volo 18 m
Abilità Percezione +4
Sensi Percezione passiva 14
Lingue -
Sfida 0 (10 PE)
Vista Affinata. L’aquila ha vantaggio nelle prove di Saggezza
(Percezione) basate sulla vista.
Azioni
Speroni. Attacco con Arma da Mischia: +4 a colpire, portata 1,5
m, un bersaglio.
Colpisce: 4 (1d4 + 2) danni taglienti.
Aquila Gigante
L’aquila gigante è una nobile creatura che parla la
propria lingua e comprende quella di altre razze.
Grande bestia, neutrale buono
FORZA 16 (+3)
DESTREZZA 17 (+3)
COSTITUZIONE 13 (+1)
INTELLIGENZA 8 (-1)
SAGGEZZA 14 (+2)
Carisma 10 (+0)
Classe Armatura 13
\hspace*{0pt}\hfill{Punti Ferita}: 26 (4d10 + 4)
Velocità 3 m, volo 24 m
Abilità Percezione +4
Sensi Percezione passiva 14
Lingue Aquila Gigante, comprende il Comune e l’Aereo ma non
può parlarli
Sfida 1 (200 PE)
Vista Affinata. L’aquila ha vantaggio nelle prove di Saggezza
(Percezione) basate sulla vista.
Azioni
Multiattacco. L’aquila effettua due attacchi: uno con il becco e
uno con gli speroni.
Becco. Attacco con Arma da Mischia: +5 a colpire, portata 1,5
m, un bersaglio.
Colpisce: 6 (1d6 + 3) danni perforanti.
Speroni. Attacco con Arma da Mischia: +5 a colpire, portata 1,5
m, un bersaglio.
Colpisce: 10 (2d6 + 3) danni taglienti.
Avvoltoio
Media bestia, disallineato
FORZA 7 (-2)
DESTREZZA 10 (+0)
COSTITUZIONE 13 (+1)
INTELLIGENZA 2 (-4)
SAGGEZZA 12 (+1)
Carisma 4 (-3)
Classe Armatura 10
\hspace*{0pt}\hfill{Punti Ferita}: 5 (1d8 + 1)
Velocità 3 m, volo 15 m
Abilità Percezione +3
Sensi Percezione passiva 13
Lingue -
Sfida 0 (10 PE)
Olfatto e Vista Affinati. L’avvoltoio ha vantaggio nelle prove di
Saggezza (Percezione) basate su olfatto o vista.
Tattiche di Branco. L’avvoltoio ha vantaggio al tiro di attacco
contro una creatura se almeno uno degli alleati dell’avvoltoio si
trova entro 1,5 metri dalla creatura e quell’alleato non è inabile.
Azioni
Becco. Attacco con Arma da Mischia: +2 a colpire, portata 1,5
m, un bersaglio.
Colpisce: 2 (1d4) danni perforanti.
Avvoltoio Gigante
L’avvoltoio gigante possiede un’intelligenza superiore e
un’attitudine maligna.
Grande bestia, neutrale malvagio
FORZA 15 (+2)
DESTREZZA 10 (+0)
COSTITUZIONE 15 (+2)
INTELLIGENZA 6 (-2)
SAGGEZZA 12 (+1)
Carisma 7 (-2)
Classe Armatura 10
\hspace*{0pt}\hfill{Punti Ferita}: 22 (3d10 + 6)
Velocità 3 m, volo 18 m
Abilità Percezione +3
Sensi Percezione passiva 13
Lingue comprende il Comune ma non può parlare
Sfida 1 (200 PE)
Olfatto e Vista Affinati. L’avvoltoio ha vantaggio nelle prove di
Saggezza (Percezione) basate su olfatto o vista.
Tattiche di Branco. L’avvoltoio ha vantaggio al tiro di attacco
contro una creatura se almeno uno degli alleati dell’avvoltoio si
trova entro 1,5 metri dalla creatura e quell’alleato non è inabile.
Azioni
Multiattacco. L’avvoltoio effettua due attacchi: uno con il becco
e uno con gli speroni.
Becco. Attacco con Arma da Mischia: +4 a colpire, portata 1,5
m, un bersaglio.
Colpisce: 7 (2d4 + 2) danni perforanti.
Speroni. Attacco con Arma da Mischia: +4 a colpire, portata 1,5
m, un bersaglio.
Colpisce: 9 (2d6 + 2) danni taglienti.
Babbuino
Piccola bestia, disallineato
FORZA 8 (-1)
DESTREZZA 14 (+2)
COSTITUZIONE 11 (+0)
INTELLIGENZA 4 (-3)
SAGGEZZA 12 (+1)
Carisma 6 (-2)
Classe Armatura 12
\hspace*{0pt}\hfill{Punti Ferita}: 3 (1d6)
Velocità 9 m, scalata 9 m
Sensi Percezione passiva 11
Lingue -
Sfida 0 (10 PE)
Tattiche di Branco. Il babbuino ha vantaggio al tiro di attacco
contro una creatura se almeno uno degli alleati del babbuino si
trova entro 1,5 metri dalla creatura e quell’alleato non è inabile.
Azioni
Morso. Attacco con Arma da Mischia: +1 a colpire, portata 1,5
m, un bersaglio.
Colpisce: 1 (1d4 - 1) danni perforanti.
 
Balena Assassina
(Orca)
Enorme bestia, disallineato
FORZA 19 (+4)
DESTREZZA 10 (+0)
COSTITUZIONE 13 (+1)
INTELLIGENZA 3 (-4)
SAGGEZZA 12 (+1)
Carisma 7 (-2)
Classe Armatura 12 (armatura naturale)
\hspace*{0pt}\hfill{Punti Ferita}: 90 (12d12 + 12)
Velocità 0 m, nuoto 18 m
Abilità Percezione +3
Sensi vista cieca 36 m, Percezione passiva 13
Lingue -
Sfida 3 (700 PE)
Ecolocazione. La balena non può usare la vista cieca se
assordata.
Trattenere il Fiato. La balena può trattenere il fiato per 30
minuti
Udito Affinato. La balena ha vantaggio alle prove di Saggezza
(Percezione) basate sull’udito.
Azioni
Morso. Attacco con Arma da Mischia: +6 a colpire, portata 1,5
m, un bersaglio.
Colpisce: 21 (5d6 + 4) danni perforanti.
Becco d’Ascia
Il becco d’ascia è un grosso e slanciato volatile privo di
ali ma con potenti gambe, un becco a cuneo, e un
pessimo carattere.
Grande bestia, disallineato
FORZA 14 (+2)
DESTREZZA 12 (+1)
COSTITUZIONE 12 (+1)
INTELLIGENZA 2 (-4)
SAGGEZZA 10 (+0)
Carisma 5 (-3)
Classe Armatura 11
\hspace*{0pt}\hfill{Punti Ferita}: 19 (3d10 + 3)
Velocità 15 m
Sensi Percezione passiva 10
Lingue -
Sfida 1/4 (50 PE)
Azioni
Becco. Attacco con Arma da Mischia: +4 a colpire, portata 1,5
m, un bersaglio.
Colpisce: 6 (1d8 + 2) danni taglienti.
Cammello
Grande bestia, disallineato
FORZA 16 (+3)
DESTREZZA 8 (-1)
COSTITUZIONE 14 (+2)
INTELLIGENZA 2 (-4)
SAGGEZZA 8 (-1)
Carisma 5 (-3)
Classe Armatura 9
\hspace*{0pt}\hfill{Punti Ferita}: 15 (2d10 + 4)
Velocità 15 m
Sensi Percezione passiva 9
Lingue -
Sfida 1/8 (25 PE)
Azioni
Morso. Attacco con Arma da Mischia: +5 a colpire, portata 1,5
m, un bersaglio.
Colpisce: 2 (1d4) danni da botta.
Cane della Morte
Il cane della morte è un orribile segugio a due teste che
si aggira per pianure, deserti e sotterranei.
Media mostruosità, neutrale malvagio
FORZA 15 (+2)
DESTREZZA 14 (+2)
COSTITUZIONE 14 (+2)
INTELLIGENZA 3 (-4)
SAGGEZZA 13 (+1)
Carisma 6 (-2)
Classe Armatura 12
\hspace*{0pt}\hfill{Punti Ferita}: 39 (6d8 + 12)
Velocità 12 m
Abilità Furtività +4, Percezione +5
Sensi visione al buio 36 m, Percezione passiva 15
Lingue -
Sfida 1 (200 PE)
Bicefalo. Il cane ha vantaggio nelle prove di Saggezza
(Percezione) e nei tiri salvezza contro le condizioni accecato,
affascinato, assordato, spaventato, stordito o svenuto.
Azioni
Multiattacco. Il cane effettua due attacchi di morso.
Morso. Attacco con Arma da Mischia: +4 a colpire, portata 1,5
m, un bersaglio.
Colpisce: 5 (1d6 + 2) danni perforanti. Se il bersaglio è una
creatura, deve riuscire un tiro salvezza di Costituzione DC 12
contro la malattia o restare avvelenato finché la malattia non
viene curata. Dopo ogni 24 ore, la creatura deve ripetere il tiro
salvezza, riducendo i suoi punti ferita massimi di 5 (1d10) in
caso di fallimento. Questa riduzione perdura finché la malattia
non viene curata. La creatura muore se la malattia riduce i suoi
punti ferita massimi a 0.
Cane Intermittente
Il cane intermittente deriva il nome dalla sua abilità di
entrare e uscire dalla realtà, un talento che usa per
attaccare ed evitare di essere attaccato.
Media fatato, legale buono
FORZA 12 (+1)
DESTREZZA 17 (+3)
COSTITUZIONE 12 (+1)
INTELLIGENZA 10 (+0)
SAGGEZZA 13 (+1)
Carisma 11 (+0)
Classe Armatura 13
\hspace*{0pt}\hfill{Punti Ferita}: 22 (4d8 + 4)
Velocità 12 m
Abilità Furtività +5, Percezione +3
Sensi Percezione passiva 13
Lingue Cane Intermittente, comprende il Silvano ma non può
parlarlo
Sfida 1/4 (50 PE)
Udito e Olfatto Affinato. Il cane ha vantaggio nelle prove di
Saggezza (Percezione) basate su udito o olfatto.
Azioni
Morso. Attacco con Arma da Mischia: +3 a colpire, portata 1,5
m, un bersaglio.
Colpisce: 4 (1d6 + 1) danni perforanti.
Teletrasporto (Ricarica 4-6). Il cane si teletrasporta
magicamente, insieme a qualsiasi cosa stia indossando o
trasportando, fino a 12 metri in uno spazio non occupato che
possa vedere. Prima o dopo il teletrasporto, il cane può effettuare
un attacco di morso.
Caprone
Media bestia, disallineato
FORZA 12 (+1)
DESTREZZA 10 (+0)
COSTITUZIONE 11 (+0)
INTELLIGENZA 2 (-4)
SAGGEZZA 10 (+0)
Carisma 5 (-3)
Classe Armatura 10
\hspace*{0pt}\hfill{Punti Ferita}: 4 (1d8)
Velocità 12 m
Sensi Percezione passiva 10
Lingue -
Sfida 0 (10 PE)
Carica. Se il caprone si muove di almeno 6 metri diretto verso il
bersaglio e colpisce con un attacco di rostro durante lo stesso
turno, il bersaglio subisce 2 (1d4) danni da botta aggiuntivi.
Se il bersaglio è una creatura, deve riuscire un tiro salvezza di
Forza DC 10 o cadere prona.
Piedi Saldi. Il caprone ha +1d6 ai tiri salvezza di Forza e
Destrezza effettuati contro effetti che lo farebbero cadere prono.
Azioni
Rostro. Attacco con Arma da Mischia: +3 a colpire, portata 1,5
m, un bersaglio.
Colpisce: 3 (1d4 + 1) danni da botta.
Caprone Gigante
Grande bestia, disallineato
FORZA 17 (+3)
DESTREZZA 11 (+0)
COSTITUZIONE 12 (+1)
INTELLIGENZA 3 (-4)
SAGGEZZA 12 (+1)
Carisma 6 (-2)
Classe Armatura 11 (armatura naturale)
\hspace*{0pt}\hfill{Punti Ferita}: 19 (3d10 + 3)
Velocità 12 m
Sensi Percezione passiva 11
Lingue -
Sfida 1/2 (100 PE)
Carica. Se il caprone si muove di almeno 6 metri diretto verso il
bersaglio e colpisce con un attacco di rostro durante lo stesso
turno, il bersaglio subisce 5 (2d4) danni da botta aggiuntivi.
Se il bersaglio è una creatura, deve riuscire un tiro salvezza di
Forza DC 13 o cadere prona.
Piedi Saldi. Il caprone ha +1d6 ai tiri salvezza di Forza e
Destrezza effettuati contro effetti che lo farebbero cadere prono.
Azioni
Rostro. Attacco con Arma da Mischia: +5 a colpire, portata 1,5
m, un bersaglio.
Colpisce: 8 (2d4 + 3) danni da botta.
Cavallo da Corsa
Grande bestia, disallineato
FORZA 16 (+3)
DESTREZZA 10 (+0)
COSTITUZIONE 12 (+1)
INTELLIGENZA 2 (-4)
SAGGEZZA 11 (+0)
Carisma 7 (-2)
Classe Armatura 10
\hspace*{0pt}\hfill{Punti Ferita}: 13 (2d10 + 2)
Velocità 18 m
Sensi Percezione passiva 10
Lingue -
Sfida 1/4 (50 PE)
Azioni
Zoccoli. Attacco con Arma da Mischia: +5 a colpire, portata 1,5
m, un bersaglio.
Colpisce: 8 (2d4 + 3) danni da botta. 
 
Cavallo da Guerra
Grande bestia, disallineato
FORZA 18 (+4)
DESTREZZA 12 (+1)
COSTITUZIONE 13 (+1)
INTELLIGENZA 2 (-4)
SAGGEZZA 12 (+1)
Carisma 7 (-2)
Classe Armatura 11 (più possibile bardatura)
\hspace*{0pt}\hfill{Punti Ferita}: 19 (3d10 + 3)
Velocità 18 m
Sensi Percezione passiva 11
Lingue -
Sfida 1/2 (100 PE)
Carica Travolgente. Se il cavallo si muove di almeno 6 metri diretto
verso il bersaglio e lo colpisce con un attacco di zoccoli durante lo
stesso turno, il bersaglio deve riuscire un tiro salvezza di Forza DC
14 o cadere prono. Se il bersaglio è prono, il cavallo può effettuare
un altro attacco di zoccoli contro di lui come azione bonus.
Azioni
Zoccoli. Attacco con Arma da Mischia: +6 a colpire, portata 1,5
m, un bersaglio.
Colpisce: 11 (2d6 + 4) danni da botta.
Cavallo da Tiro
Grande bestia, disallineato
FORZA 18 (+4)
DESTREZZA 10 (+0)
COSTITUZIONE 12 (+1)
INTELLIGENZA 2 (-4)
SAGGEZZA 11 (+0)
Carisma 7 (-2)
Classe Armatura 10
\hspace*{0pt}\hfill{Punti Ferita}: 19 (3d10 + 3)
Velocità 12 m
Sensi Percezione passiva 10
Lingue -
Sfida 1/4 (50 PE)
Azioni
Zoccoli. Attacco con Arma da Mischia: +6 a colpire, portata 1,5
m, un bersaglio.
Colpisce: 9 (2d4 + 4) danni da botta.
Cavallo Marino
Minuscola bestia, disallineato
FORZA 1 (-5)
DESTREZZA 12 (+1)
COSTITUZIONE 8 (-1)
INTELLIGENZA 1 (-5)
SAGGEZZA 10 (+0)
Carisma 2 (-4)
Classe Armatura 11
\hspace*{0pt}\hfill{Punti Ferita}: 1 (1d4 - 1)
Velocità 0 m, nuoto 6 m
Sensi Percezione passiva 10
Lingue -
Sfida 0 (0 PE)
Respirare Acqua. Il cavallo marino può respirare solo sottacqua.
Cavallo Marino Gigante
Il cavallo marino gigante viene spesso impiegato come
cavalcatura dagli umanoidi acquatici.
Grande bestia, disallineato
FORZA 12 (+1)
DESTREZZA 15 (+2)
COSTITUZIONE 11 (+0)
INTELLIGENZA 2 (-4)
SAGGEZZA 12 (+1)
Carisma 5 (-3)
Classe Armatura 13 (armatura naturale)
\hspace*{0pt}\hfill{Punti Ferita}: 16 (3d10)
Velocità 0 m, nuoto 12 m
Sensi Percezione passiva 11
Lingue -
Sfida 1/2 (100 PE)
Carica. Se il cavallo marino si muove di almeno 6 metri diretto
verso il bersaglio e colpisce con un attacco di rostro durante lo
stesso turno, il bersaglio subisce 7 (2d6) danni da botta
aggiuntivi. Se il bersaglio è una creatura, deve riuscire un tiro
salvezza di Forza DC 11 o cadere prona.
Respirare Acqua. Il cavallo marino può respirare solo sottacqua.
Azioni
Rostro. Attacco con Arma da Mischia: +3 a colpire, portata 1,5
m, un bersaglio.
Colpisce: 4 (1d6 + 1) danni da botta. 
Centopiedi Gigante
Piccola bestia, disallineato
FORZA 5 (-3)
DESTREZZA 14 (+2)
COSTITUZIONE 12 (+1)
INTELLIGENZA 1 (-5)
SAGGEZZA 7 (-2)
Carisma 3 (-4)
Classe Armatura 13 (armatura naturale)
\hspace*{0pt}\hfill{Punti Ferita}: 4 (1d6 + 1)
Velocità 9 m, scalata 9 m
Sensi vista cieca 9 m, Percezione passiva 8
Lingue -
Sfida 1/4 (50 PE)
Azioni
Morso. Attacco con Arma da Mischia: +4 a colpire, portata 1,5
m, una creatura.
Colpisce: 4 (1d4 + 2) danni perforanti e il bersaglio deve riuscire
un tiro salvezza di Costituzione DC 11 o subire 10 (3d6) danni
da veleno. Se il danno da veleno riduce il bersaglio a 0 punti
ferita, il bersaglio è stabile ma resta avvelenato per 1 ora, anche
dopo aver recuperato i punti ferita, e mentre è avvelenato in
questo modo resta paralizzato.
Cervo
Media bestia, disallineato
FORZA 11 (+0)
DESTREZZA 16 (+3)
COSTITUZIONE 11 (+0)
INTELLIGENZA 2 (-4)
SAGGEZZA 14 (+2)
Carisma 5 (-3)
Classe Armatura 13
\hspace*{0pt}\hfill{Punti Ferita}: 4 (1d8)
Velocità 15 m
Sensi Percezione passiva 12
Lingue -
Sfida 0 (10 PE)
Azioni
Morso. Attacco con Arma da Mischia: +2 a colpire, portata 1,5
m, un bersaglio.
Colpisce: 2 (1d4) danni perforanti.
Cinghiale
Media bestia, disallineato
FORZA 13 (+1)
DESTREZZA 11 (+0)
COSTITUZIONE 12 (+1)
INTELLIGENZA 2 (-4)
SAGGEZZA 9 (-1)
Carisma 5 (-3)
Classe Armatura 11 (armatura naturale)
\hspace*{0pt}\hfill{Punti Ferita}: 11 (2d8 + 2)
Velocità 12 m
Sensi Percezione passiva 9
Lingue -
Sfida 1/4 (50 PE)
Carica. Se il cinghiale si muove di almeno 6 metri diretto verso
il bersaglio e colpisce con un attacco di zanna durante lo stesso
turno, il bersaglio subisce 3 (1d6) danni taglienti aggiuntivi. Se il
bersaglio è una creatura, deve riuscire un tiro salvezza di Forza
DC 11 o cadere prono.
Implacabile (Ricarica dopo un Riposo Breve o Lungo). Se il
cinghiale subisce 7 danni o meno che lo ridurrebbero a 0 punti
ferita, scende invece a 1 punto ferita.
Azioni
Zanna. Attacco con Arma da Mischia: +3 a colpire, portata 1,5
m, un bersaglio.
Colpisce: 4 (1d6 + 1) danni taglienti.
Cinghiale Gigante
Grande bestia, disallineato
FORZA 17 (+3)
DESTREZZA 10 (+0)
COSTITUZIONE 16 (+3)
INTELLIGENZA 2 (-4)
SAGGEZZA 7 (-2)
Carisma 5 (-3)
Classe Armatura 12 (armatura naturale)
\hspace*{0pt}\hfill{Punti Ferita}: 42 (5d10 + 15)
Velocità 12 m
Sensi Percezione passiva 8
Lingue -
Sfida 2 (450 PE)
Carica. Se il cinghiale si muove di almeno 6 metri diretto verso
il bersaglio e colpisce con un attacco di zanna durante lo stesso
turno, il bersaglio subisce 7 (2d6) danni taglienti aggiuntivi. Se il
bersaglio è una creatura, deve riuscire un tiro salvezza di Forza
DC 13 o cadere prono.
Implacabile (Ricarica dopo un Riposo Breve o Lungo). Se il
cinghiale subisce 10 danni o meno che lo ridurrebbero a 0 punti
ferita, scende invece a 1 punto ferita.
Azioni
Zanna. Attacco con Arma da Mischia: +5 a colpire, portata 1,5
m, un bersaglio.
Colpisce: 10 (2d6 + 3) danni taglienti.
 
Coccodrillo
Grande bestia, disallineato
FORZA 15 (+2)
DESTREZZA 10 (+0)
COSTITUZIONE 13 (+1)
INTELLIGENZA 2 (-4)
SAGGEZZA 10 (+0)
Carisma 5 (-3)
Classe Armatura 12 (armatura naturale)
\hspace*{0pt}\hfill{Punti Ferita}: 19 (3d10 + 3)
Velocità 6 m, nuoto 9 m
Abilità Furtività +2
Sensi Percezione passiva 10
Lingue -
Sfida 1/2 (100 PE)
Trattenere il Fiato. Il coccodrillo può trattenere il fiato per 15
minuti.
Azioni
Morso. Attacco con Arma da Mischia: +4 a colpire, portata 1,5
m, una creatura.
Colpisce: 7 (1d10 + 2) danni perforanti, e il bersaglio è afferrato
(DC 12 per fuggire). Fino al termine dell’afferrare, il bersaglio è
intralciato, e il coccodrillo non può usare il morso contro un altro
bersaglio.
Coccodrillo Gigante
Enorme bestia, disallineato
FORZA 21 (+5)
DESTREZZA 9 (-1)
COSTITUZIONE 17 (+3)
INTELLIGENZA 2 (-4)
SAGGEZZA 10 (+0)
Carisma 7 (-2)
Classe Armatura 14 (armatura naturale)
\hspace*{0pt}\hfill{Punti Ferita}: 85 (9d12 + 27)
Velocità 9 m, nuoto 15 m
Abilità Furtività +5
Sensi Percezione passiva 10
Lingue -
Sfida 5 (1.800 PE)
Trattenere il Fiato. Il coccodrillo può trattenere il fiato per 30
minuti.
Azioni
Multiattacco. Il coccodrillo effettua due attacchi: uno con il
morso e uno con la coda.
Coda. Attacco con Arma da Mischia: +8 a colpire, portata 3 m,
un bersaglio non afferrato dal coccodrillo.
Colpisce: 14 (2d8 + 5) danni da botta. Se il bersaglio è una
creatura, deve riuscire un tiro salvezza di Forza DC 16 o cadere
prono.
Morso. Attacco con Arma da Mischia: +8 a colpire, portata 1,5
m, un bersaglio.
Colpisce: 21 (3d10 + 5) danni perforanti, e il bersaglio è
afferrato (DC 16 per fuggire). Fino al termine dell’afferrare, il
bersaglio è intralciato, e il coccodrillo non può usare il morso
contro un altro bersaglio.
Corvo
Minuscola bestia, disallineato
FORZA 2 (-4)
DESTREZZA 14 (+2)
COSTITUZIONE 8 (-1)
INTELLIGENZA 2 (-4)
SAGGEZZA 12 (+1)
Carisma 6 (-2)
Classe Armatura 12
\hspace*{0pt}\hfill{Punti Ferita}: 1 (1d4 - 1)
Velocità 3 m, volo 15 m
Abilità Percezione +3
Sensi Percezione passiva 13
Lingue -
Sfida 0 (10 PE)
Imitazione. Il corvo può imitare dei semplici suoni che ha udito,
come il sussurro di una persona, il pianto di un bambino o il
verso di un animale. Una creatura che ode il suono può
identificarlo come imitazione riuscendo una prova di Saggezza
(Intuizione) DC 10.
Azioni
Becco. Attacco con Arma da Mischia: +4 a colpire, portata 1,5
m, un bersaglio.
Colpisce: 1 danno perforante.
Donnola
Minuscola bestia, disallineato
FORZA 3 (-4)
DESTREZZA 16 (+3)
COSTITUZIONE 8 (-1)
INTELLIGENZA 2 (-4)
SAGGEZZA 12 (+1)
Carisma 3 (-4)
Classe Armatura 13
\hspace*{0pt}\hfill{Punti Ferita}: 1 (1d4 - 1)
Velocità 9 m
Abilità Furtività +5, Percezione +3
Sensi Percezione passiva 13
Lingue -
Sfida 0 (10 PE)
Udito e Olfatto Affinati. La donnola ha vantaggio nelle prove di
Saggezza (Percezione) basate su udito o olfatto.
Azioni
Morso. Attacco con Arma da Mischia: +5 a colpire, portata 1,5
m, un bersaglio.
Colpisce: 1 danno perforante.
Donnola Gigante
Media bestia, disallineato
FORZA 11 (+0)
DESTREZZA 16 (+3)
COSTITUZIONE 10 (+0)
INTELLIGENZA 4 (-3)
SAGGEZZA 12 (+1)
Carisma 5 (-3)
Classe Armatura 13
\hspace*{0pt}\hfill{Punti Ferita}: 9 (2d8)
Velocità 12 m
Abilità Furtività +5, Percezione +3
Sensi visione al buio 18 m, Percezione passiva 13
Lingue -
Sfida 1/8 (25 PE)
Udito e Olfatto Affinati. La donnola ha vantaggio nelle prove di
Saggezza (Percezione) basate su udito o olfatto.
Azioni
Morso. Attacco con Arma da Mischia: +5 a colpire, portata 1,5
m, un bersaglio.
Colpisce: 5 (1d4 + 3) danni perforanti.
Elefante
Enorme bestia, disallineato
FORZA 22 (+6)
DESTREZZA 9 (-1)
COSTITUZIONE 17 (+3)
INTELLIGENZA 3 (-4)
SAGGEZZA 11 (+0)
Carisma 6 (-2)
Classe Armatura 12 (armatura naturale)
\hspace*{0pt}\hfill{Punti Ferita}: 76 (8d12 + 24)
Velocità 12 m
Sensi Percezione passiva 10
Lingue -
Sfida 4 (1.000 PE)
Carica Travolgente. Se l’elefante si muove di almeno 6 metri
diretto verso una creatura e la colpisce con un attacco di
incornata durante lo stesso turno, il bersaglio deve riuscire un
tiro salvezza di Forza DC 12 o cadere prono. Se il bersaglio è
prono, l’elefante può effettuare un attacco di pestone contro di
lui come azione bonus.
Azioni
Incornata. Attacco con Arma da Mischia: +8 a colpire, portata
1,5 m, un bersaglio.
Colpisce: 19 (3d8 + 6) danni perforanti.
Pestone. Attacco con Arma da Mischia: +8 a colpire, portata 1,5
m, un bersaglio prono.
Colpisce: 22 (3d10 + 6) danni da botta.
Falco
Minuscola bestia, disallineato
FORZA 5 (-3)
DESTREZZA 16 (+3)
COSTITUZIONE 8 (-1)
INTELLIGENZA 2 (-4)
SAGGEZZA 14 (+2)
Carisma 6 (-2)
Classe Armatura 13
\hspace*{0pt}\hfill{Punti Ferita}: 1 (1d4 - 1)
Velocità 3 m, volo 18 m
Abilità Percezione +4
Sensi Percezione passiva 14
Lingue -
Sfida 0 (10 PE)
Vista Affinata. Il falco ha vantaggio alle prove di Saggezza
(Percezione) basate sulla vista.
Azioni
Speroni. Attacco con Arma da Mischia: +5 a colpire, portata 1,5
m, un bersaglio.
Colpisce: 1 danno tagliente.
Falco di Sangue
Dovendo il suo nome alle sue piume cremisi e alla sua
natura aggressiva, il falco di sangue attacca senza
timore usando il suo becco appuntito.
Piccola bestia, disallineato
FORZA 6 (-2)
DESTREZZA 14 (+2)
COSTITUZIONE 10 (+0)
INTELLIGENZA 3 (-4)
SAGGEZZA 14 (+2)
Carisma 5 (-3)
Classe Armatura 12
\hspace*{0pt}\hfill{Punti Ferita}: 7 (2d6)
Velocità 3 m, volo 18 m
Abilità Percezione +4
Sensi Percezione passiva 14
Lingue -
Sfida 1/8 (25 PE)
Tattiche di Branco. Il falco ha vantaggio ai tiri di attacco contro
una creatura se almeno uno degli alleati del falco si trova entro
1,5 metri dalla creatura e quell’alleato non è inabile.
Vista Affinata. Il falco ha vantaggio alle prove di Saggezza
(Percezione) basate sulla vista.
Azioni
Becco. Attacco con Arma da Mischia: +4 a colpire, portata 1,5
m, un bersaglio.
Colpisce: 4 (1d4 + 2) danni perforanti.
 
Frizzo
Il frizzo è un pesce carnivoro dai denti affilati.
Minuscola bestia, disallineato
FORZA 2 (-4)
DESTREZZA 16 (+3)
COSTITUZIONE 9 (-1)
INTELLIGENZA 1 (-5)
SAGGEZZA 7 (-2)
Carisma 2 (-4)
Classe Armatura 13
\hspace*{0pt}\hfill{Punti Ferita}: 1 (1d4 - 1)
Velocità 0 m, nuoto 12 m
Sensi visione al buio 18 m, Percezione passiva 8
Lingue -
Sfida 0 (10 PE)
Frenesia Sanguinaria. Il frizzo ha vantaggio ai tiri di attacco in
mischia contro qualsiasi creatura che non sia al massimo dei
punti ferita.
Respirare Acqua. Il frizzo può respirare solo sottacqua.
Azioni
Morso. Attacco con Arma da Mischia: +5 a colpire, portata 1,5
m, un bersaglio.
Colpisce: 1 danno perforante.
Gatto
Minuscola bestia, disallineato
FORZA 3 (-4)
DESTREZZA 15 (+2)
COSTITUZIONE 10 (+0)
INTELLIGENZA 3 (-4)
SAGGEZZA 12 (+1)
Carisma 7 (-2)
Classe Armatura 12
\hspace*{0pt}\hfill{Punti Ferita}: 2 (1d4)
Velocità 12 m, scalata 9 m
Abilità Furtività +4, Percezione +3
Sensi Percezione passiva 11
Lingue -
Sfida 0 (10 PE)
Olfatto Affinato. Il gatto ha vantaggio alle prove di Saggezza
(Percezione) basate sull’olfatto.
Azioni
Artigli. Attacco con Arma da Mischia: +0 a colpire, portata 1,5
m, un bersaglio.
Colpisce: 1 danno tagliente.
Granchio
Minuscola bestia, disallineato
FORZA 2 (-4)
DESTREZZA 11 (+0)
COSTITUZIONE 10 (+0)
INTELLIGENZA 1 (-5)
SAGGEZZA 8 (-1)
Carisma 2 (-4)
Classe Armatura 11 (armatura naturale)
\hspace*{0pt}\hfill{Punti Ferita}: 2 (1d4)
Velocità 6 m, nuoto 6 m
Abilità Furtività +2
Sensi vista cieca 9 m, Percezione passiva 9
Lingue -
Sfida 0 (10 PE)
Anfibio. Il granchio può respirare aria e acqua.
Azioni
Artiglio (Chela). Attacco con Arma da Mischia: +0 a colpire,
portata 1,5 m, un bersaglio.
Colpisce: 1 danno da botta.
Granchio Gigante
Media bestia, disallineato
FORZA 13 (+1)
DESTREZZA 15 (+2)
COSTITUZIONE 11 (+0)
INTELLIGENZA 1 (-5)
SAGGEZZA 9 (-1)
Carisma 3 (-4)
Classe Armatura 15 (armatura naturale)
\hspace*{0pt}\hfill{Punti Ferita}: 13 (3d8)
Velocità 9 m, nuoto 9 m
Abilità Furtività +4
Sensi vista cieca 9 m, Percezione passiva 9
Lingue -
Sfida 1/8 (25 PE)
Anfibio. Il granchio può respirare aria e acqua.
Azioni
Artiglio (Chela). Attacco con Arma da Mischia: +3 a colpire,
portata 1,5 m, un bersaglio.
Colpisce: 4 (1d6 + 1) danni da botta e il bersaglio è afferrato
(DC 11 per fuggire). Il granchio ha due chele, ciascuna delle
quali può afferrare un solo bersaglio.
Gufo
Minuscola bestia, disallineato
FORZA 3 (-4)
DESTREZZA 13 (+1)
COSTITUZIONE 8 (-1)
INTELLIGENZA 2 (-4)
SAGGEZZA 12 (+1)
Carisma 7 (-2)
Classe Armatura 11
\hspace*{0pt}\hfill{Punti Ferita}: 1 (1d4 - 1)
Velocità 1,5 m, volo 18 m
Abilità Furtività +3, Percezione +3
Sensi visione al buio 36 m, Percezione passiva 13
Lingue -
Sfida 0 (10 PE)
Sorvolare. Il gufo non provoca attacchi di opportunità quando
vola via dalla portata di un nemico.
Udito e Vista Affinati. Il gufo ha vantaggio nelle prove di
Saggezza (Percezione) basate su udito o vista.
Azioni
Speroni. Attacco con Arma da Mischia: +3 a colpire, portata 1,5
m, un bersaglio.
Colpisce: 1 danno tagliente.
Gufo Gigante
I gufi giganti sono creature intelligenti che proteggono i
regni silvani.
Grande bestia, neutrale
FORZA 13 (+1)
DESTREZZA 15 (+2)
COSTITUZIONE 12 (+1)
INTELLIGENZA 8 (-1)
SAGGEZZA 13 (+1)
Carisma 10 (+0)
Classe Armatura 12
\hspace*{0pt}\hfill{Punti Ferita}: 19 (3d10 + 3)
Velocità 1,5 m, volo 18 m
Abilità Furtività +4, Percezione +5
Sensi visione al buio 36 m, Percezione passiva 15
Lingue Gufo Gigante, comprende Comune, Elfico e Silvano ma
non può parlarli
Sfida 1/4 (50 PE)
Sorvolare. Il gufo non provoca attacchi di opportunità quando
vola via dalla portata di un nemico.
Udito e Vista Affinati. Il gufo ha vantaggio nelle prove di
Saggezza (Percezione) basate su udito o vista.
Azioni
Speroni. Attacco con Arma da Mischia: +3 a colpire, portata 1,5
m, un bersaglio.
Colpisce: 8 (2d6 + 1) danni perforanti.
Iena
Media bestia, disallineato
FORZA 11 (+0)
DESTREZZA 13 (+1)
COSTITUZIONE 12 (+1)
INTELLIGENZA 2 (-4)
SAGGEZZA 12 (+1)
Carisma 5 (-3)
Classe Armatura 11
\hspace*{0pt}\hfill{Punti Ferita}: 5 (1d8 + 1)
Velocità 15 m
Abilità Percezione +3
Sensi Percezione passiva 13
Lingue -
Sfida 0 (10 PE)
Tattiche di Branco. La iena ha vantaggio ai tiri di attacco contro
una creatura se almeno uno degli alleati della iena si trova entro
1,5 metri dalla creatura e quell’alleato non è inabile.
Azioni
Morso. Attacco con Arma da Mischia: +2 a colpire, portata 1,5
m, un bersaglio.
Colpisce: 3 (1d6) danni perforanti.
Iena Gigante
Grande bestia, disallineato
FORZA 16 (+3)
DESTREZZA 14 (+2)
COSTITUZIONE 14 (+2)
INTELLIGENZA 2 (-4)
SAGGEZZA 12 (+1)
Carisma 7 (-2)
Classe Armatura 12
\hspace*{0pt}\hfill{Punti Ferita}: 45 (6d10 + 12)
Velocità 15 m
Abilità Percezione +3
Sensi Percezione passiva 13
Lingue -
Sfida 1 (200 PE)
Rabbia. Quando la iena riduce una creatura a 0 punti ferita con
un attacco di mischia durante il proprio round, la iena può
svolgere un’azione bonus per muoversi fino a metà della sua
velocità ed effettuare un attacco di morso.
Azioni
Morso. Attacco con Arma da Mischia: +5 a colpire, portata 1,5
m, un bersaglio.
Colpisce: 10 (2d6 + 3) danni perforanti.
 
Leone
Grande bestia, disallineato
FORZA 17 (+3)
DESTREZZA 15 (+2)
COSTITUZIONE 13 (+1)
INTELLIGENZA 3 (-4)
SAGGEZZA 12 (+1)
Carisma 8 (-1)
Classe Armatura 12
\hspace*{0pt}\hfill{Punti Ferita}: 26 (4d10 + 4)
Velocità 15 m
Abilità Furtività +6, Percezione +3
Sensi Percezione passiva 13
Lingue -
Sfida 1 (200 PE)
Balzo. Se il leone si muove di almeno 6 metri diretto verso una
creatura e la colpisce con un attacco di artiglio durante lo stesso
turno, il bersaglio deve riuscire un tiro salvezza di Forza DC 13 o
cadere prono. Se il bersaglio è prono, il leone può effettuare un
attacco di morso come azione bonus.
Olfatto Affinato. Il leone ha vantaggio alle prove di Saggezza
(Percezione) basate sull’olfatto.
Salto con Rincorsa. Con 3 metri di rincorsa, il leone può saltare
in lungo fino a 7,5 metri.
Tattiche di Branco. Il leone ha vantaggio ai tiri di attacco contro
una creatura se almeno uno degli alleati del leone si trova entro
1,5 metri dalla creatura e quell’alleato non è inabile.
Azioni
Artiglio. Attacco con Arma da Mischia: +5 a colpire, portata 1,5
m, un bersaglio.
Colpisce: 6 (1d6 + 3) danni taglienti.
Morso. Attacco con Arma da Mischia: +5 a colpire, portata 1,5
m, un bersaglio.
Colpisce: 7 (1d8 + 3) danni perforanti.
Lucertola
Minuscola bestia, disallineato
FORZA 2 (-4)
DESTREZZA 11 (+0)
COSTITUZIONE 10 (+0)
INTELLIGENZA 1 (-5)
SAGGEZZA 8 (-1)
Carisma 3 (-4)
Classe Armatura 10
\hspace*{0pt}\hfill{Punti Ferita}: 2 (1d4)
Velocità 6 m, scalata 6 m
Sensi visione al buio 9 m, Percezione passiva 9
Lingue -
Sfida 0 (10 PE)
Scalare come Ragno. La lucertola può scalare superfici difficili,
compreso lo stare a testa in giù sul soffitto, senza bisogno di
effettuare una prova di abilità.
Azioni
Morso. Attacco con Arma da Mischia: +0 a colpire, portata 1,5
m, un bersaglio.
Colpisce: 1 danno perforante.
Lucertola Gigante
Le lucertole giganti sono temibili predatori e spesso
vengono impiegate come cavalcature o animali da tiro
da umanoidi rettiloidi e residenti del sottosuolo.
Grande bestia, disallineato
FORZA 15 (+2)
DESTREZZA 12 (+1)
COSTITUZIONE 13 (+1)
INTELLIGENZA 2 (-4)
SAGGEZZA 10 (+0)
Carisma 5 (-3)
Classe Armatura 12 (armatura naturale)
\hspace*{0pt}\hfill{Punti Ferita}: 19 (3d10 + 3)
Velocità 9 m, scalata 9 m
Sensi visione al buio 9 m, Percezione passiva 10
Lingue -
Sfida 1/4 (50 PE)
Azioni
Morso. Attacco con Arma da Mischia: +4 a colpire, portata 1,5
m, un bersaglio.
Colpisce: 6 (1d8 + 2) danni perforanti.
VARIANTE
Alcune lucertole giganti possiedono uno o entrambi i seguenti
tratti.
Scalare come Ragno. La lucertola può scalare superfici difficili,
compreso lo stare a testa in giù sul soffitto, senza bisogno di
effettuare una prova di abilità.
Trattenere il Fiato. La lucertola può trattenere il fiato per 15
minuti. (Una lucertola con questo tratto possiede anche velocità
di nuoto 9 metri).
Lupo
Media bestia, disallineato
FORZA 12 (+1)
DESTREZZA 15 (+2)
COSTITUZIONE 12 (+1)
INTELLIGENZA 3 (-4)
SAGGEZZA 12 (+1)
Carisma 6 (-2)
Classe Armatura 13 (armatura naturale)
\hspace*{0pt}\hfill{Punti Ferita}: 11 (2d8 + 2)
Velocità 12 m
Abilità Furtività +4, Percezione +3
Sensi Percezione passiva 13
Lingue -
Sfida 1/4 (50 PE)
Udito e Olfatto Affinato. Il lupo ha vantaggio nelle prove di
Saggezza (Percezione) basate su udito o olfatto.
Tattiche di Branco. Il lupo ha vantaggio ai tiri di attacco contro
una creatura se almeno uno degli alleati del lupo si trova entro
1,5 metri dalla creatura e quell’alleato non è inabile.
Azioni
Morso. Attacco con Arma da Mischia: +4 a colpire, portata 1,5
m, un bersaglio.
Colpisce: 7 (2d4 + 2) danni perforanti. Se il bersaglio è una
creatura, deve riuscire un tiro salvezza di Forza DC 11 o cadere
prona.
Dinolupo (Metalupo)
Grande bestia, disallineato
FORZA 17 (+3)
DESTREZZA 15 (+2)
COSTITUZIONE 15 (+2)
INTELLIGENZA 3 (-4)
SAGGEZZA 12 (+1)
Carisma 7 (-2)
Classe Armatura 14 (armatura naturale)
\hspace*{0pt}\hfill{Punti Ferita}: 37 (5d10 + 10)
Velocità 15 m
Abilità Furtività +4, Percezione +3
Sensi Percezione passiva 13
Lingue -
Sfida 1 (200 PE)
Udito e Olfatto Affinato. Il lupo ha vantaggio nelle prove di
Saggezza (Percezione) basate su udito o olfatto.
Tattiche di Branco. Il lupo ha vantaggio ai tiri di attacco contro
una creatura se almeno uno degli alleati del lupo si trova entro
1,5 metri dalla creatura e quell’alleato non è inabile.
Azioni
Morso. Attacco con Arma da Mischia: +5 a colpire, portata 1,5
m, un bersaglio.
Colpisce: 10 (2d6 + 3) danni perforanti. Se il bersaglio è una
creatura, deve riuscire un tiro salvezza di Forza DC 13 o cadere
prona.
Lupo Invernale
I lupi invernali abitano nelle regioni artiche e sono
creature malvagie e intelligenti dal manto bianco come
la neve e gli occhi color del ghiaccio.
Grande mostruosità, neutrale malvagio
FORZA 18 (+4)
DESTREZZA 13 (+1)
COSTITUZIONE 14 (+2)
INTELLIGENZA 7 (-2)
SAGGEZZA 12 (+1)
Carisma 8 (-1)
Classe Armatura 13 (armatura naturale)
\hspace*{0pt}\hfill{Punti Ferita}: 75 (10d10 + 20)
Velocità 15 m
Abilità Furtività +3, Percezione +5
Immunità al Danno freddo
Sensi Percezione passiva 15
Lingue Comune, Gigante, Lupo Invernale
Sfida 3 (700 PE)
Camuffamento di Neve. Il lupo ha vantaggio alle prove di
Destrezza (Furtività) per nascondersi su terreno innevato.
Udito e Olfatto Affinato. Il lupo ha vantaggio nelle prove di
Saggezza (Percezione) basate su udito o olfatto.
Tattiche di Branco. Il lupo ha vantaggio ai tiri di attacco contro
una creatura se almeno uno degli alleati del lupo si trova entro
1,5 metri dalla creatura e quell’alleato non è inabile.
Azioni
Morso. Attacco con Arma da Mischia: +6 a colpire, portata 1,5
m, un bersaglio.
Colpisce: 11 (2d6 + 4) danni perforanti. Se il bersaglio è una
creatura, deve riuscire un tiro salvezza di Forza DC 14 o cadere
prona.
Soffio Gelido (Ricarica 5-6). Il lupo esala un’esplosione di vento
gelido in un cono di 4,5 metri. Ogni creatura in quell’area deve
effettuare un tiro salvezza di Destrezza DC 12, e subire 18 (4d8)
danni da freddo se fallisce il tiro salvezza, o la metà di questi
danni se lo riesce.
Mammut
Il mammut è una creatura simile all’elefante dalla folta
pelliccia e lunghe zanne.
Enorme bestia, disallineato
FORZA 24 (+7)
DESTREZZA 9 (-1)
COSTITUZIONE 21 (+5)
INTELLIGENZA 3 (-4)
SAGGEZZA 11 (+0)
Carisma 6 (-2)
Classe Armatura 13 (armatura naturale)
\hspace*{0pt}\hfill{Punti Ferita}: 126 (11d12 + 55)
Velocità 12 m
Sensi Percezione passiva 10
Lingue -
Sfida 6 (2.300 PE)
Carica Travolgente. Se il mammut si muove di almeno 6 metri
diretto verso una creatura e la colpisce con un attacco di
incornata durante lo stesso turno, il bersaglio deve riuscire un
tiro salvezza di Forza DC 18 o cadere prono. Se il bersaglio è 
 
prono, il mammut può effettuare un attacco di pestone contro di
lui come azione bonus.
Azioni
Incornata. Attacco con Arma da Mischia: +10 a colpire, portata
3 m, un bersaglio.
Colpisce: 25 (4d8 + 7) danni perforanti.
Pestone. Attacco con Arma da Mischia: +10 a colpire, portata
1,5 m, una creatura prona.
Colpisce: 29 (4d10 + 7) danni da botta.
Mastino
I mastini sono impressionanti segugi apprezzati dagli
umanoidi per la loro realtà e sensi affinati.
Media bestia, disallineato
FORZA 13 (+1)
DESTREZZA 14 (+2)
COSTITUZIONE 12 (+1)
INTELLIGENZA 3 (-4)
SAGGEZZA 12 (+1)
Carisma 7 (-2)
Classe Armatura 12
\hspace*{0pt}\hfill{Punti Ferita}: 5 (1d8 + 1)
Velocità 12 m
Abilità Percezione +3
Sensi Percezione passiva 13
Lingue -
Sfida 1/8 (25 PE)
Udito e Olfatto Affinato. Il mastino ha vantaggio nelle prove di
Saggezza (Percezione) basate su udito o olfatto.
Azioni
Morso. Attacco con Arma da Mischia: +3 a colpire, portata 1,5
m, un bersaglio.
Colpisce: 4 (1d6 + 1) danni perforanti. Se il bersaglio è una creatura,
deve riuscire un tiro salvezza di Forza DC 11 o cadere prono.
Mulo
Media bestia, disallineato
FORZA 14 (+2)
DESTREZZA 10 (+0)
COSTITUZIONE 13 (+1)
INTELLIGENZA 2 (-4)
SAGGEZZA 10 (+0)
Carisma 5 (-3)
Classe Armatura 10
\hspace*{0pt}\hfill{Punti Ferita}: 11 (2d8 + 2)
Velocità 12 m
Sensi Percezione passiva 10
Lingue -
Sfida 1/8 (25 PE)
Bestia da Soma. Il mulo è considerato un animale Grande al fine
di determinare la sua capacità di carico.
Piedi Saldi. Il mulo ha +1d6 ai tiri salvezza di Forza e
Destrezza effettuati contro effetti che lo farebbero cadere prono.
Azioni
Zoccoli. Attacco con Arma da Mischia: +2 a colpire, portata 1,5
m, un bersaglio.
Colpisce: 4 (1d4 + 2) danni da botta.
Orso Bruno
Grande bestia, disallineato
FORZA 19 (+4)
DESTREZZA 10 (+0)
COSTITUZIONE 16 (+3)
INTELLIGENZA 2 (-4)
SAGGEZZA 13 (+1)
Carisma 7 (-2)
Classe Armatura 11 (armatura naturale)
\hspace*{0pt}\hfill{Punti Ferita}: 34 (4d10 + 12)
Velocità 12 m, scalata 9 m
Abilità Percezione +3
Sensi Percezione passiva 13
Lingue -
Sfida 1 (200 PE)
Olfatto Affinato. L’orso ha vantaggio alle prove di Saggezza
(Percezione) basate sull’olfatto.
Azioni
Multiattacco. L’orso effettua due attacchi: uno con il morso e
uno con gli artigli.
Artigli. Attacco con Arma da Mischia: +5 a colpire, portata 1,5
m, un bersaglio.
Colpisce: 11 (2d6 + 4) danni taglienti.
Morso. Attacco con Arma da Mischia: +5 a colpire, portata 1,5
m, un bersaglio.
Colpisce: 8 (1d8 + 4) danni perforanti.
Orso Nero
Media bestia, disallineato
FORZA 15 (+2)
DESTREZZA 10 (+0)
COSTITUZIONE 14 (+2)
INTELLIGENZA 2 (-4)
SAGGEZZA 12 (+1)
Carisma 7 (-2)
Classe Armatura 11 (armatura naturale)
\hspace*{0pt}\hfill{Punti Ferita}: 19 (3d8 + 6)
Velocità 12 m, scalata 9 m
Abilità Percezione +3
Sensi Percezione passiva 13
Lingue -
Sfida 1/2 (100 PE)
Olfatto Affinato. L’orso ha vantaggio alle prove di Saggezza
(Percezione) basate sull’olfatto.
Azioni
Multiattacco. L’orso nero effettua due attacchi: uno con il morso
e uno con gli artigli.
Artigli. Attacco con Arma da Mischia: +3 a colpire, portata 1,5
m, un bersaglio.
Colpisce: 7 (2d4 + 2) danni taglienti.
Morso. Attacco con Arma da Mischia: +3 a colpire, portata 1,5
m, un bersaglio.
Colpisce: 5 (1d6 + 2) danni perforanti.
Orso Polare
Grande bestia, disallineato
FORZA 20 (+5)
DESTREZZA 10 (+0)
COSTITUZIONE 16 (+3)
INTELLIGENZA 2 (-4)
SAGGEZZA 13 (+1)
Carisma 7 (-2)
Classe Armatura 12 (armatura naturale)
\hspace*{0pt}\hfill{Punti Ferita}: 42 (5d10 + 15)
Velocità 12 m, nuoto 9 m
Abilità Percezione +3
Sensi Percezione passiva 13
Lingue -
Sfida 2 (450 PE)
Olfatto Affinato. L’orso ha vantaggio alle prove di Saggezza
(Percezione) basate sull’olfatto.
Azioni
Multiattacco. L’orso effettua due attacchi: uno con il morso e
uno con gli artigli.
Artigli. Attacco con Arma da Mischia: +7 a colpire, portata 1,5
m, un bersaglio.
Colpisce: 12 (2d6 + 5) danni taglienti.
Morso. Attacco con Arma da Mischia: +7 a colpire, portata 1,5
m, un bersaglio.
Colpisce: 9 (1d8 + 5) danni perforanti.
VARIANTE: ORSO DELLE DifesaVERNE
Alcuni orsi si sono adattati alla vita sottoterra. Costoro hanno le
stesse statistiche degli orsi polari, ma con visione al buio 18 m.
Pantera
Media bestia, disallineato
FORZA 14 (+2)
DESTREZZA 15 (+2)
COSTITUZIONE 10 (+0)
INTELLIGENZA 3 (-4)
SAGGEZZA 14 (+2)
Carisma 7 (-2)
Classe Armatura 12
\hspace*{0pt}\hfill{Punti Ferita}: 13 (3d8)
Velocità 15 m, scalata 12 m
Abilità Furtività +6, Percezione +4
Sensi Percezione passiva 14
Lingue -
Sfida 1/4 (50 PE)
Balzo. Se la pantera si muove di almeno 6 metri diretta verso una
creatura e la colpisce con un attacco di artiglio durante lo stesso
turno, il bersaglio deve riuscire un tiro salvezza di Forza DC 12 o
cadere prono. Se il bersaglio è prono, la pantera può effettuare un
attacco di morso contro di esso come azione bonus.
Olfatto Affinato. La pantera ha vantaggio alle prove di Saggezza
(Percezione) basate sull’olfatto.
Azioni
Artiglio. Attacco con Arma da Mischia: +4 a colpire, portata 1,5
m, un bersaglio.
Colpisce: 4 (1d4 + 2) danni taglienti.
Morso. Attacco con Arma da Mischia: +4 a colpire, portata 1,5
m, un bersaglio.
Colpisce: 5 (1d6 + 2) danni perforanti.
 
Piovra
Piccola bestia, disallineato
FORZA 4 (-3)
DESTREZZA 15 (+2)
COSTITUZIONE 11 (+0)
INTELLIGENZA 3 (-4)
SAGGEZZA 10 (+0)
Carisma 4 (-3)
Classe Armatura 12
\hspace*{0pt}\hfill{Punti Ferita}: 3 (1d6)
Velocità 1,5 m, nuoto 9 m
Abilità Furtività +4, Percezione +2
Sensi visione al buio 9 m, Percezione passiva 12
Lingue -
Sfida 0 (10 PE)
Camuffamento Subacqueo. La piovra ha vantaggio alle prove di
Destrezza (Furtività) effettuate sottacqua.
Respirare Acqua. La piovra può respirare solo sottacqua.
Trattenere il Fiato. Mentre è fuori dell’acqua, la piovra può
trattenere il fiato per 30 minuti.
Azioni
Tentacoli. Attacco con Arma da Mischia: +4 a colpire, portata
4,5 m, un bersaglio.
Colpisce: 1 danno da botta e il bersaglio è afferrato (DC 10
per fuggire). Fino al termine dell’afferrare, la piovra non può
usare i suoi tentacoli contro un altro bersaglio.
Nube di Inchiostro (Ricarica dopo un Riposo Breve o Lungo).
Una nube di inchiostro del raggio di 1,5 metri si estende
tutt’intorno alla piovra quando si trova sottacqua. L’aria è
oscurata pesantemente per 1 minuto, sebbene una forte corrente
possa dissolvere la nube più rapidamente. Dopo aver rilasciato
l’inchiostro, la piovra può usare l’azione Scattare come azione
bonus.
Piovra Gigante
Grande bestia, disallineato
FORZA 17 (+3)
DESTREZZA 13 (+1)
COSTITUZIONE 13 (+1)
INTELLIGENZA 4 (-3)
SAGGEZZA 10 (+0)
Carisma 4 (-3)
Classe Armatura 11
\hspace*{0pt}\hfill{Punti Ferita}: 52 (8d10 + 8)
Velocità 3 m, nuoto 18 m
Abilità Furtività +5, Percezione +4
Sensi visione al buio 18 m, Percezione passiva 14
Lingue -
Sfida 1 (200 PE)
Camuffamento Subacqueo. La piovra ha vantaggio alle prove di
Destrezza (Furtività) effettuate sottacqua.
Respirare Acqua. La piovra può respirare solo sottacqua.
Trattenere il Fiato. Mentre è fuori dell’acqua, la piovra può
trattenere il fiato per 1 ora.
Azioni
Tentacoli. Attacco con Arma da Mischia: +5 a colpire, portata
4,5 m, un bersaglio.
Colpisce: 10 (2d6 + 3) danni da botta. Se il bersaglio è una
creatura, è afferrata (DC 16 per fuggire). Fino al termine
dell’afferrare, il bersaglio è intralciato, e la piovra non può usare
i suoi tentacoli contro un altro bersaglio.
Nube di Inchiostro (Ricarica dopo un Riposo Breve o Lungo).
Una nube di inchiostro del raggio di 6 metri si estende
tutt’intorno alla piovra quando si trova sottacqua. L’aria è
oscurata pesantemente per 1 minuto, sebbene una forte corrente
possa dissolvere la nube più rapidamente. Dopo aver rilasciato
l’inchiostro, la piovra può usare l’azione Scattare come azione
bonus.
Pipistrello
Minuscola bestia, disallineato
FORZA 2 (-4)
DESTREZZA 15 (+2)
COSTITUZIONE 8 (-1)
INTELLIGENZA 2 (-4)
SAGGEZZA 12 (+1)
Carisma 4 (-3)
Classe Armatura 12
\hspace*{0pt}\hfill{Punti Ferita}: 1 (1d4 - 1)
Velocità 1,5 m, volo 9 m
Sensi vista cieca 18 m, Percezione passiva 11
Lingue -
Sfida 0 (10 PE)
Ecolocazione. Il pipistrello non può usare la vista cieca se
assordato.
Udito Affinato. Il pipistrello ha vantaggio alle prove di Saggezza
(Percezione) basate sull’udito.
Azioni
Morso. Attacco con Arma da Mischia: +0 a colpire, portata 1,5
m, una creatura.
Colpisce: 1 danno perforante.
Pipistrello Gigante
Grande bestia, disallineato
FORZA 15 (+2)
DESTREZZA 16 (+3)
COSTITUZIONE 11 (+0)
INTELLIGENZA 2 (-4)
SAGGEZZA 12 (+1)
Carisma 6 (-2)
Classe Armatura 13
\hspace*{0pt}\hfill{Punti Ferita}: 22 (4d10)
Velocità 3 m, volo 18 m
Sensi vista cieca 18 m, Percezione passiva 11
Lingue -
Sfida 1/4 (50 PE)
Ecolocazione. Il pipistrello non può usare la vista cieca se
assordato.
Udito Affinato. Il pipistrello ha vantaggio alle prove di Saggezza
(Percezione) basate sull’udito.
Azioni
Morso. Attacco con Arma da Mischia: +4 a colpire, portata 1,5
m, una creatura.
Colpisce: 5 (1d6 + 2) danni perforanti.
Pony
Media bestia, disallineato
FORZA 15 (+2)
DESTREZZA 10 (+0)
COSTITUZIONE 13 (+1)
INTELLIGENZA 2 (-4)
SAGGEZZA 11 (+0)
Carisma 7 (-2)
Classe Armatura 10
\hspace*{0pt}\hfill{Punti Ferita}: 11 (2d8 + 2)
Velocità 12 m
Sensi Percezione passiva 10
Lingue -
Sfida 1/8 (25 PE)
Azioni
Zoccoli. Attacco con Arma da Mischia: +4 a colpire, portata 1,5
m, un bersaglio.
Colpisce: 7 (2d4 + 2) danni da botta.
Ragno
Minuscola bestia, disallineato
FORZA 2 (-5)
DESTREZZA 14 (+2)
COSTITUZIONE 8 (-1)
INTELLIGENZA 1 (-5)
SAGGEZZA 10 (+0)
Carisma 2 (-4)
Classe Armatura 12
\hspace*{0pt}\hfill{Punti Ferita}: 1 (1d4 - 1)
Velocità 6 m, scalata 6 m
Abilità Furtività +4
Sensi visione al buio 9 m, Percezione passiva 10
Lingue -
Sfida 0 (10 PE)
Camminare sulla Tela. Il ragno ignora le restrizioni al
movimento provocate dalle ragnatele.
Scalare come Ragno. Il ragno può scalare superfici difficili,
compreso lo stare a testa in giù sul soffitto, senza bisogno di
effettuare una prova di abilità.
Senso della Tela. Mentre è in contatto con una ragnatela, il
ragno sa l’esatta posizione di qualsiasi altra creatura in contatto
con la stessa ragnatela.
Azioni
Morso. Attacco con Arma da Mischia: +4 a colpire, portata 1,5
m, una creatura.
Colpisce: 1 danno perforante e il bersaglio deve riuscire un tiro
salvezza di Costituzione DC 9 o subire 2 (1d4) danni da veleno.
 
Ragno Fase
Il ragno fase possiede l’abilità magica di entrare ed
uscire dal Piano Etereo. Sembra apparire dal nulla e
scompare rapidamente dopo aver attaccato.
Grande mostruosità, disallineato
FORZA 15 (+2)
DESTREZZA 15 (+2)
COSTITUZIONE 12 (+1)
INTELLIGENZA 6 (-2)
SAGGEZZA 10 (+0)
Carisma 6 (-2)
Classe Armatura 13 (armatura naturale)
\hspace*{0pt}\hfill{Punti Ferita}: 32 (5d10 + 5)
Velocità 9 m, scalata 9 m
Abilità Furtività +6
Sensi visione al buio 18 m, Percezione passiva 10
Lingue -
Sfida 3 (700 PE)
Camminare sulla Tela. Il ragno ignora le restrizioni al
movimento provocate dalle ragnatele.
Gita Eterea. Come azione bonus, il ragno può magicamente
spostarsi dal Piano Materiale al Piano Etereo, o viceversa.
Scalare come Ragno. Il ragno può scalare superfici difficili,
compreso lo stare a testa in giù sul soffitto, senza bisogno di
effettuare una prova di abilità.
Azioni
Morso. Attacco con Arma da Mischia: +4 a colpire, portata 1,5
m, una creatura.
Colpisce: 7 (1d10 + 2) danni perforanti e il bersaglio deve
effettuare un tiro salvezza di Costituzione DC 11, e subire 18
(4d8) danni da veleno se fallisce il tiro salvezza, o la metà di
questo danno se lo riesce. Se il danno da veleno riduce il
bersaglio a 0 punti ferita, il bersaglio è stabile ma avvelenato per
1 ora, anche dopo aver recuperato i punti ferita, e mentre è
avvelenato in questo modo resta paralizzato.
Ragno Gigante
Grande bestia, disallineato
FORZA 14 (+2)
DESTREZZA 16 (+3)
COSTITUZIONE 12 (+1)
INTELLIGENZA 2 (-4)
SAGGEZZA 11 (+0)
Carisma 4 (-3)
Classe Armatura 14 (armatura naturale)
\hspace*{0pt}\hfill{Punti Ferita}: 26 (4d10 + 4)
Velocità 9 m, scalata 9 m
Abilità Furtività +7
Sensi vista cieca 3 m, visione al buio 18 m, Percezione passiva
10
Lingue -
Sfida 1 (200 PE)
Camminare sulla Tela. Il ragno ignora le restrizioni al
movimento provocate dalle ragnatele.
Scalare come Ragno. Il ragno può scalare superfici difficili,
compreso lo stare a testa in giù sul soffitto, senza bisogno di
effettuare una prova di abilità.
Senso della Tela. Mentre è in contatto con una ragnatela, il
ragno sa l’esatta posizione di qualsiasi altra creatura in contatto
con la stessa ragnatela.
Azioni
Morso. Attacco con Arma da Mischia: +5 a colpire, portata 1,5
m, una creatura.
Colpisce: 7 (1d8 + 3) danni perforanti e il bersaglio deve
effettuare un tiro salvezza di Costituzione DC 11, e subire 9
(2d8) danni da veleno se fallisce il tiro salvezza, o la metà di
questi danni se lo riesce. Se il danno da veleno riduce il bersaglio
a 0 punti ferita, il bersaglio è stabile ma avvelenato per 1 ora,
anche dopo aver recuperato i punti ferita, e mentre è avvelenato
in questo modo resta paralizzato.
Ragnatela (Ricarica 5-6). Attacco con Arma a Gittata: +5 a
colpire, gittata 9/18 m, una creatura.
Colpisce: Il bersaglio è intralciato dalla ragnatela. Con
un’azione, il bersaglio intralciato può effettuare una prova di
Forza DC 12 e, in caso di successo, spezzare la tela. La ragnatela
può essere anche attaccata e distrutta (Difesa 10; pf 5; vulnerabilità
al danno da fuoco; immunità ai danni da botta, psichici e da
veleno).
Ragno Lupo Gigante
Un ragno lupo gigante caccia le prede su terreno aperto
o si nasconde in tane o fessure del terreno per tendere
imboscate.
Media bestia, disallineato
FORZA 12 (+1)
DESTREZZA 16 (+3)
COSTITUZIONE 13 (+1)
INTELLIGENZA 3 (-4)
SAGGEZZA 12 (+1)
Carisma 4 (-3)
Classe Armatura 13
\hspace*{0pt}\hfill{Punti Ferita}: 11 (2d8 + 2)
Velocità 12 m, scalata 12 m
Abilità Furtività +7, Percezione +3
Sensi vista cieca 3 m, visione al buio 18 m, Percezione passiva
13
Lingue -
Sfida 1/4 (50 PE)
Camminare sulla Tela. Il ragno ignora le restrizioni al
movimento provocate dalle ragnatele.
Scalare come Ragno. Il ragno può scalare superfici difficili,
compreso lo stare a testa in giù sul soffitto, senza bisogno di
effettuare una prova di abilità.
Senso della Tela. Mentre è in contatto con una ragnatela, il
ragno sa l’esatta posizione di qualsiasi altra creatura in contatto
con la stessa ragnatela.
Azioni
Morso. Attacco con Arma da Mischia: +3 a colpire, portata 1,5
m, una creatura.
Colpisce: 4 (1d6 + 1) danni perforanti e il bersaglio deve
effettuare un tiro salvezza di Costituzione DC 11, e subire 7
(2d6) danni da veleno se fallisce il tiro salvezza, o la metà di
questi danni se lo riesce. Se il danno da veleno riduce il bersaglio
a 0 punti ferita, il bersaglio è stabile ma avvelenato per 1 ora,
anche dopo aver recuperato i punti ferita, e mentre è avvelenato
in questo modo resta paralizzato.
Rana
Minuscola bestia, disallineato
FORZA 1 (-5)
DESTREZZA 13 (+1)
COSTITUZIONE 8 (-1)
INTELLIGENZA 1 (-5)
SAGGEZZA 8 (-1)
Carisma 3 (-4)
Classe Armatura 11
\hspace*{0pt}\hfill{Punti Ferita}: 1 (1d4 - 1)
Velocità 6 m, nuoto 6 m
Abilità Furtività +3, Percezione +1
Sensi visione al buio 9 m, Percezione passiva 11
Lingue -
Sfida 0 (0 PE)
Anfibio. La rana può respirare aria e acqua.
Salto da Fermo. Una rana può saltare in lungo fino a 3 metri e in
alto fino a 1,5 metri, con o senza la rincorsa.
Una rana è sprovvista di attacchi. Si nutre di piccoli
insetti e di solito vive in prossimità di acquitrini, dentro
gli alberi o sottoterra.
Rana Gigante
Media bestia, disallineato
FORZA 12 (+1)
DESTREZZA 13 (+1)
COSTITUZIONE 11 (+0)
INTELLIGENZA 2 (-4)
SAGGEZZA 10 (+0)
Carisma 3 (-4)
Classe Armatura 11
\hspace*{0pt}\hfill{Punti Ferita}: 18 (4d8)
Velocità 9 m, nuoto 9 m
Abilità Furtività +3, Percezione +2
Sensi visione al buio 9 m, Percezione passiva 12
Lingue -
Sfida 1/4 (50 PE)
Anfibio. La rana può respirare aria e acqua.
Salto da Fermo. Una rana può saltare in lungo fino a 6 metri e in
alto fino a 3 metri, con o senza la rincorsa.
Azioni
Morso. Attacco con Arma da Mischia: +3 a colpire, portata 1,5
m, un bersaglio.
Colpisce: 4 (1d6 + 1) danni perforanti e il bersaglio è afferrato
(DC 11 per fuggire). Fino al termine dell’afferrare, il bersaglio è
intralciato, e la rana non può usare il morso contro un altro
bersaglio.
Inghiottire. La rana effettua una attacco di morso contro un
bersaglio di taglia Piccola o inferiore che sta afferrando. Se
l’attacco colpisce, il bersaglio è inghiottito, e l’afferrare ha
termine. Il bersaglio inghiottito è accecato e intralciato, ha
copertura totale contro gli attacchi e altri effetti all’esterno della
rana, e subisce 5 (2d4) danni da acido all’inizio di ciascun turno
della rana. La rana può inghiottire solo un bersaglio alla volta.
Se la rana muore, una creatura inghiottita non è più intralciata da
essa e può uscire dal cadavere utilizzando 1,5 metri di
movimento, uscendo prona.
 
Ratto
Minuscola bestia, disallineato
FORZA 2 (-4)
DESTREZZA 11 (+0)
COSTITUZIONE 9 (-1)
INTELLIGENZA 2 (-4)
SAGGEZZA 10 (+0)
Carisma 4 (-3)
Classe Armatura 10
\hspace*{0pt}\hfill{Punti Ferita}: 1 (1d4 - 1)
Velocità 6 m
Sensi visione al buio 9 m, Percezione passiva 10
Lingue -
Sfida 0 (10 PE)
Olfatto Affinato. Il ratto ha vantaggio alle prove di Saggezza
(Percezione) basate sull’olfatto.
Azioni
Morso. Attacco con Arma da Mischia: +0 a colpire, portata 1,5
m, un bersaglio.
Colpisce: 1 danno perforante.
Ratto Gigante
Piccola bestia, disallineato
FORZA 7 (-2)
DESTREZZA 15 (+2)
COSTITUZIONE 11 (+0)
INTELLIGENZA 2 (-4)
SAGGEZZA 10 (+0)
Carisma 4 (-3)
Classe Armatura 12
\hspace*{0pt}\hfill{Punti Ferita}: 7 (2d6)
Velocità 9 m
Sensi visione al buio 18 m, Percezione passiva 10
Lingue -
Sfida 1/8 (25 PE)
Olfatto Affinato. Il ratto ha vantaggio alle prove di Saggezza
(Percezione) basate sull’olfatto.
Tattiche di Branco. Il ratto ha vantaggio al tiro di attacco contro
una creatura se almeno uno degli alleati del ratto si trova entro
1,5 metri dalla creatura e quell’alleato non è inabile.
Azioni
Morso. Attacco con Arma da Mischia: +4 a colpire, portata 1,5
m, un bersaglio.
Colpisce: 4 (1d4 + 2) danni perforanti.
VARIANTE: RATTO GIGANTE AMMALATO
Alcuni ratti giganti recano una terribile malattia che diffondono
tramite il morso. Un ratto gigante ammalato ha grado di sfida 1/8
(25 PE) e la seguente azione invece del suo normale attacco di
morso.
Morso. Attacco con Arma da Mischia: +4 a colpire, portata 1,5
m, un bersaglio.
Colpisce: 4 (1d4 + 2) danni perforanti. Se il bersaglio è una
creatura, deve riuscire un tiro salvezza di Costituzione DC 10 o
contrarre una malattia. Fino a che la malattia non viene curata, il
bersaglio non può recuperare punti ferita eccetto tramite metodi
magici, e i punti ferita massimi del bersaglio diminuiscono di 3
(1d6) ogni 24 ore. Se i punti ferita massimi del bersaglio
scendono a 0 come risultato della malattia, il bersaglio muore.
Rinoceronte
Grande bestia, disallineato
FORZA 21 (+5)
DESTREZZA 8 (-1)
COSTITUZIONE 15 (+2)
INTELLIGENZA 2 (-4)
SAGGEZZA 12 (+1)
Carisma 6 (-2)
Classe Armatura 11 (armatura naturale)
\hspace*{0pt}\hfill{Punti Ferita}: 45 (6d10 + 12)
Velocità 12 m
Sensi Percezione passiva 11
Lingue -
Sfida 2 (450 PE)
Carica. Se il rinoceronte si muove di almeno 6 metri diretto
verso un bersaglio e lo colpisce con un attacco di incornata
durante lo stesso turno, il bersaglio subisce 9 (2d8) danni
contundenti aggiuntivi. Se il bersaglio è una creatura, deve
riuscire un tiro salvezza di Forza DC 15 o cadere prono.
Azioni
Incornata. Attacco con Arma da Mischia: +7 a colpire, portata
1,5 m, un bersaglio.
Colpisce: 14 (2d8 + 5) danni da botta.
Rospo Gigante
Grande bestia, disallineato
FORZA 15 (+2)
DESTREZZA 13 (+1)
COSTITUZIONE 13 (+1)
INTELLIGENZA 2 (-4)
SAGGEZZA 10 (+0)
Carisma 3 (-4)
Classe Armatura 11
\hspace*{0pt}\hfill{Punti Ferita}: 39 (6d10 + 6)
Velocità 6 m, nuoto 12 m
Sensi visione al buio 9 m, Percezione passiva 10
Lingue -
Sfida 1 (200 PE)
Anfibio. Il rospo può respirare aria e acqua.
Salto da Fermo. Un rospo può saltare in lungo fino a 6 metri e in
alto fino a 3 metri, con o senza la rincorsa.
Azioni
Morso. Attacco con Arma da Mischia: +4 a colpire, portata 1,5
m, un bersaglio.
Colpisce: 7 (1d10 + 2) danni perforanti più 5 (1d10) danni da
veleno, e il bersaglio è afferrato (DC 13 per fuggire). Fino al
termine dell’afferrare, il bersaglio è intralciato, e il rospo non
può usare il morso contro un altro bersaglio.
Inghiottire. Il rospo effettua una attacco di morso contro un
bersaglio di taglia Media o inferiore che sta afferrando. Se
l’attacco colpisce, il bersaglio è inghiottito, e l’afferrare ha
termine. Il bersaglio inghiottito è accecato e intralciato, ha
copertura totale contro gli attacchi e altri effetti all’esterno della
rana, e subisce 10 (3d6) danni da acido all’inizio di ciascun turno
del rospo. Il rospo può inghiottire solo un bersaglio alla volta.
Se il rospo muore, una creatura inghiottita non è più intralciata
da esso e può uscire dal cadavere utilizzando 1,5 metri di
movimento, uscendo prono.
Scarabeo di Fuoco
Gigante
Uno scarabeo di fuoco gigante è una creatura notturna
che possiede una coppia di ghiandole luminose capaci
di emettere luce per 1d6 giorni dopo la morte dello
scarabeo.
Piccola bestia, disallineato
FORZA 8 (-1)
DESTREZZA 10 (+0)
COSTITUZIONE 12 (+1)
INTELLIGENZA 1 (-5)
SAGGEZZA 7 (-2)
Carisma 3 (-4)
Classe Armatura 13 (armatura naturale)
\hspace*{0pt}\hfill{Punti Ferita}: 4 (1d6 + 1)
Velocità 9 m
Sensi vista cieca 9 m, Percezione passiva 8
Lingue -
Sfida 0 (10 PE)
Illuminazione. Lo scarabeo irradia luce intensa in un raggio di 3
metri e luce fioca per ulteriori 3 metri.
Azioni
Morso. Attacco con Arma da Mischia: +1 a colpire, portata 1,5
m, un bersaglio.
Colpisce: 2 (1d6 - 1) danni taglienti.
Sciacallo
Piccola bestia, disallineato
FORZA 8 (-1)
DESTREZZA 15 (+2)
COSTITUZIONE 11 (+0)
INTELLIGENZA 3 (-4)
SAGGEZZA 12 (+1)
Carisma 6 (-2)
Classe Armatura 12
\hspace*{0pt}\hfill{Punti Ferita}: 3 (1d6)
Velocità 12 m
Abilità Percezione +3
Sensi Percezione passiva 13
Lingue -
Sfida 0 (10 PE)
Tattiche di Branco. Lo sciacallo ha vantaggio ai tiri di attacco
contro una creatura se almeno uno degli alleati dello sciacallo si
trova entro 1,5 metri dalla creatura e quell’alleato non è inabile.
Udito e Olfatto Affinato. Lo sciacallo ha vantaggio nelle prove
di Saggezza (Percezione) basate su udito o olfatto.
Azioni
Morso. Attacco con Arma da Mischia: +1 a colpire, portata 1,5
m, un bersaglio.
Colpisce: 1 (1d4 - 1) danni perforanti.
Sciami
Gli sciami presentati qui di seguito non sono dei normali
o benigni raduni di piccole creature. Si formano invece
come risultato di un’influenza esterna, spesso maligna.
Anche i druidi non sono in grado di affascinare questi
sciami, e la loro aggressività è quasi innaturale.
Sciame di Centopiedi
Medio sciame di Minuscole bestie, disallineato
FORZA 3 (-4)
DESTREZZA 13 (+1)
COSTITUZIONE 10 (+0)
INTELLIGENZA 1 (-5)
SAGGEZZA 7 (-2)
Carisma 1 (-5)
Classe Armatura 12 (armatura naturale)
\hspace*{0pt}\hfill{Punti Ferita}: 22 (5d8)
Velocità 6 m, scalata 6 m
Resistenze al Danno da botta, perforante, tagliente
Immunità alle Condizioni affascinato, afferrato, intralciato,
paralizzato, pietrificato, prono, spaventato, stordito
Sensi vista cieca 3 m, Percezione passiva 8
Lingue -
Sfida 1/2 (100 PE)
Sciame. Lo sciame può occupare lo spazio di un’altra creatura e
viceversa, e lo sciame può muoversi attraverso qualsiasi apertura
grande abbastanza per un Minuscolo insetto. Lo sciame non può
recuperare punti ferita né ottenere punti ferita temporanei.
Azioni
Morsi. Attacco con Arma da Mischia: +3 a colpire, portata 0 m,
un bersaglio nello spazio dello sciame.
Colpisce: 10 (4d4) danni perforanti, o 5 (2d4) danni perforanti se
lo sciame è ha metà o meno dei suoi punti ferita. Una creatura
ridotta a 0 punti ferita da uno sciame di centopiedi e stabile resta
avvelenata per 1 ora, anche dopo aver recuperato i punti ferita, e
rimane paralizzata dal veleno durante questo periodo.
Sciame di Corvi
Medio sciame di Minuscole bestie, disallineato
FORZA 6 (-2)
DESTREZZA 14 (+2)
COSTITUZIONE 8 (-1)
INTELLIGENZA 3 (-4)
SAGGEZZA 12 (+1)
Carisma 6 (-2)
Classe Armatura 12
\hspace*{0pt}\hfill{Punti Ferita}: 24 (7d8 – 7)
Velocità 3 m, volo 15 m
Abilità Percezione +5
Resistenze al Danno da botta, perforante, tagliente
Immunità alle Condizioni affascinato, afferrato, intralciato,
paralizzato, pietrificato, prono, spaventato, stordito
Sensi Percezione passiva 15
Lingue -
Sfida 1/4 (50 PE)
Sciame. Lo sciame può occupare lo spazio di un’altra creatura e
viceversa, e lo sciame può muoversi attraverso qualsiasi apertura
grande abbastanza per un Minuscolo corvo. Lo sciame non può
recuperare punti ferita né ottenere punti ferita temporanei.
Azioni
Becchi. Attacco con Arma da Mischia: +4 a colpire, portata 1,5
m, un bersaglio nello spazio dello sciame.
Colpisce: 7 (2d6) danni perforanti, o 3 (1d6) danni perforanti se
lo sciame è ha metà o meno dei suoi punti ferita.
 
Sciame di Frizzi
Medio sciame di Minuscole bestie, disallineato
FORZA 13 (+1)
DESTREZZA 16 (+3)
COSTITUZIONE 9 (-1)
INTELLIGENZA 1 (-5)
SAGGEZZA 7 (-2)
Carisma 2 (-4)
Classe Armatura 13
\hspace*{0pt}\hfill{Punti Ferita}: 28 (8d8 – 8)
Velocità 0 m, nuoto 12 m
Resistenze al Danno da botta, perforante, tagliente
Immunità alle Condizioni affascinato, afferrato, intralciato,
paralizzato, pietrificato, prono, spaventato, stordito
Sensi visione al buio 18 m, Percezione passiva 8
Lingue -
Sfida 1 (200 PE)
Frenesia Sanguinaria. Lo sciame ha vantaggio ai tiri di attacco
in mischia contro qualsiasi creatura che non sia al massimo dei
punti ferita.
Respirare Acqua. Lo sciame può respirare solo sottacqua.
Sciame. Lo sciame può occupare lo spazio di un’altra creatura e
viceversa, e lo sciame può muoversi attraverso qualsiasi apertura
grande abbastanza per un Minuscolo frizzo. Lo sciame non può
recuperare punti ferita né ottenere punti ferita temporanei.
Azioni
Morsi. Attacco con Arma da Mischia: +5 a colpire, portata 0 m,
una creatura nello spazio dello sciame.
Colpisce: 14 (4d6) danni perforanti, o 7 (2d6) danni perforanti se
lo sciame è ha metà o meno dei suoi punti ferita.
Sciame di Insetti
Medio sciame di Minuscole bestie, disallineato
FORZA 3 (-4)
DESTREZZA 13 (+1)
COSTITUZIONE 10 (+0)
INTELLIGENZA 1 (-5)
SAGGEZZA 7 (-2)
Carisma 1 (-5)
Classe Armatura 12 (armatura naturale)
\hspace*{0pt}\hfill{Punti Ferita}: 22 (5d8)
Velocità 6 m, scalata 6 m
Resistenze al Danno da botta, perforante, tagliente
Immunità alle Condizioni affascinato, afferrato, intralciato,
paralizzato, pietrificato, prono, spaventato, stordito
Sensi vista cieca 3 m, Percezione passiva 8
Lingue -
Sfida 1/2 (100 PE)
Sciame. Lo sciame può occupare lo spazio di un’altra creatura e
viceversa, e lo sciame può muoversi attraverso qualsiasi apertura
grande abbastanza per un Minuscolo insetto. Lo sciame non può
recuperare punti ferita né ottenere punti ferita temporanei.
Azioni
Morsi. Attacco con Arma da Mischia: +3 a colpire, portata 0 m,
un bersaglio nello spazio dello sciame.
Colpisce: 10 (4d4) danni perforanti, o 5 (2d4) danni perforanti se
lo sciame è ha metà o meno dei suoi punti ferita.
Sciame di Pipistrelli
Medio sciame di Minuscole bestie, disallineato
FORZA 5 (-3)
DESTREZZA 15 (+2)
COSTITUZIONE 10 (+0)
INTELLIGENZA 2 (-4)
SAGGEZZA 12 (+1)
Carisma 4 (-3)
Classe Armatura 12
\hspace*{0pt}\hfill{Punti Ferita}: 22 (5d8)
Velocità 0 m, volo 9 m
Resistenze al Danno da botta, perforante, tagliente
Immunità alle Condizioni affascinato, afferrato, intralciato,
paralizzato, pietrificato, prono, spaventato, stordito
Sensi vista cieca 18 m, Percezione passiva 11
Lingue -
Sfida 1/4 (50 PE)
Ecolocazione. Lo sciame non può usare la vista cieca se
assordato.
Sciame. Lo sciame può occupare lo spazio di un’altra creatura e
viceversa, e lo sciame può muoversi attraverso qualsiasi apertura
grande abbastanza per un Minuscolo pipistrello. Lo sciame non
può recuperare punti ferita né ottenere punti ferita temporanei.
Udito Affinato. Lo sciame ha vantaggio alle prove di Saggezza
(Percezione) basate sull’udito.
Azioni
Morsi. Attacco con Arma da Mischia: +4 a colpire, portata 0 m,
una creatura nello spazio dello sciame.
Colpisce: 5 (2d4) danni perforanti, o 2 (1d4) danni perforanti se
lo sciame è ha metà o meno dei suoi punti ferita.
Sciame di Ragni
Medio sciame di Minuscole bestie, disallineato
FORZA 3 (-4)
DESTREZZA 13 (+1)
COSTITUZIONE 10 (+0)
INTELLIGENZA 1 (-5)
SAGGEZZA 7 (-2)
Carisma 1 (-5)
Classe Armatura 12 (armatura naturale)
\hspace*{0pt}\hfill{Punti Ferita}: 22 (5d8)
Velocità 6 m, scalata 6 m
Resistenze al Danno da botta, perforante, tagliente
Immunità alle Condizioni affascinato, afferrato, intralciato,
paralizzato, pietrificato, prono, spaventato, stordito
Sensi vista cieca 3 m, Percezione passiva 8
Lingue -
Sfida 1/2 (100 PE)
Camminare sulla Tela. Lo sciame ignora le restrizioni al
movimento provocate dalle ragnatele.
Scalare come Ragno. Lo sciame può scalare superfici difficili,
compreso lo stare a testa in giù sul soffitto, senza bisogno di
effettuare una prova di abilità.
Senso della Tela. Mentre è in contatto con una ragnatela, lo
sciame sa l’esatta posizione di qualsiasi altra creatura in contatto
con la stessa ragnatela.
Sciame. Lo sciame può occupare lo spazio di un’altra creatura e
viceversa, e lo sciame può muoversi attraverso qualsiasi apertura
grande abbastanza per un Minuscolo insetto. Lo sciame non può
recuperare punti ferita né ottenere punti ferita temporanei.
Azioni
Morsi. Attacco con Arma da Mischia: +3 a colpire, portata 0 m,
un bersaglio nello spazio dello sciame.
Colpisce: 10 (4d4) danni perforanti, o 5 (2d4) danni perforanti se
lo sciame è ha metà o meno dei suoi punti ferita.
Sciame di Ratti
Medio sciame di Minuscole bestie, disallineato
FORZA 9 (-1)
DESTREZZA 11 (+0)
COSTITUZIONE 9 (-1)
INTELLIGENZA 2 (-4)
SAGGEZZA 10 (+0)
Carisma 3 (-4)
Classe Armatura 10
\hspace*{0pt}\hfill{Punti Ferita}: 24 (7d8 - 7)
Velocità 9 m
Resistenze al Danno da botta, perforante, tagliente
Immunità alle Condizioni affascinato, afferrato, intralciato,
paralizzato, pietrificato, prono, spaventato, stordito
Sensi visione al buio 9 m, Percezione passiva 10
Lingue -
Sfida 1/4 (50 PE)
Olfatto Affinato. Lo sciame ha vantaggio alle prove di Saggezza
(Percezione) basate sull’olfatto.
Sciame. Lo sciame può occupare lo spazio di un’altra creatura e
viceversa, e lo sciame può muoversi attraverso qualsiasi apertura
grande abbastanza per un Minuscolo ratto. Lo sciame non può
recuperare punti ferita né ottenere punti ferita temporanei.
Azioni
Morsi. Attacco con Arma da Mischia: +2 a colpire, portata 0 m,
un bersaglio nello spazio dello sciame.
Colpisce: 7 (2d6) danni perforanti, o 3 (1d6) danni perforanti se
lo sciame è ha metà o meno dei suoi punti ferita.
Sciame di Scarabei
Medio sciame di Minuscole bestie, disallineato
FORZA 3 (-4)
DESTREZZA 13 (+1)
COSTITUZIONE 10 (+0)
INTELLIGENZA 1 (-5)
SAGGEZZA 7 (-2)
Carisma 1 (-5)
Classe Armatura 12 (armatura naturale)
\hspace*{0pt}\hfill{Punti Ferita}: 22 (5d8)
Velocità 6 m, scalata 6 m, scavo 6 m
Resistenze al Danno da botta, perforante, tagliente
Immunità alle Condizioni affascinato, afferrato, intralciato,
paralizzato, pietrificato, prono, spaventato, stordito
Sensi vista cieca 3 m, Percezione passiva 8
Lingue -
Sfida 1/2 (100 PE)
Sciame. Lo sciame può occupare lo spazio di un’altra creatura e
viceversa, e lo sciame può muoversi attraverso qualsiasi apertura
grande abbastanza per un Minuscolo insetto. Lo sciame non può
recuperare punti ferita né ottenere punti ferita temporanei.
Azioni
Morsi. Attacco con Arma da Mischia: +3 a colpire, portata 0 m,
un bersaglio nello spazio dello sciame.
Colpisce: 10 (4d4) danni perforanti, o 5 (2d4) danni perforanti se
lo sciame è ha metà o meno dei suoi punti ferita.
 
Sciame di Serpenti Velenosi
Medio sciame di Minuscole bestie, disallineato
FORZA 8 (-1)
DESTREZZA 18 (+4)
COSTITUZIONE 11 (+0)
INTELLIGENZA 1 (-5)
SAGGEZZA 10 (+0)
Carisma 3 (-4)
Classe Armatura 14
\hspace*{0pt}\hfill{Punti Ferita}: 36 (8d8)
Velocità 9 m, nuoto 9 m
Resistenze al Danno da botta, perforante, tagliente
Immunità alle Condizioni affascinato, afferrato, intralciato,
paralizzato, pietrificato, prono, spaventato, stordito
Sensi vista cieca 3 m, Percezione passiva 10
Lingue -
Sfida 2 (450 PE)
Sciame. Lo sciame può occupare lo spazio di un’altra creatura e
viceversa, e lo sciame può muoversi attraverso qualsiasi apertura
grande abbastanza per un Minuscolo serpente. Lo sciame non
può recuperare punti ferita né ottenere punti ferita temporanei.
Azioni
Morsi. Attacco con Arma da Mischia: +6 a colpire, portata 0 m,
una creatura nello spazio dello sciame.
Colpisce: 7 (2d6) danni perforanti, o 3 (1d6) danni perforanti se
lo sciame è ha metà o meno dei suoi punti ferita, e il bersaglio
deve effettuare un tiro salvezza di Costituzione DC 10, e subire
14 (4d6) danni da veleno se fallisce il tiro salvezza, o la metà di
questi danni se lo riesce.
Sciame di Vespe
Medio sciame di Minuscole bestie, disallineato
FORZA 3 (-4)
DESTREZZA 13 (+1)
COSTITUZIONE 10 (+0)
INTELLIGENZA 1 (-5)
SAGGEZZA 7 (-2)
Carisma 1 (-5)
Classe Armatura 12 (armatura naturale)
\hspace*{0pt}\hfill{Punti Ferita}: 22 (5d8)
Velocità 1,5 m, volo 9 m
Resistenze al Danno da botta, perforante, tagliente
Immunità alle Condizioni affascinato, afferrato, intralciato,
paralizzato, pietrificato, prono, spaventato, stordito
Sensi vista cieca 3 m, Percezione passiva 8
Lingue -
Sfida 1/2 (100 PE)
Sciame. Lo sciame può occupare lo spazio di un’altra creatura e
viceversa, e lo sciame può muoversi attraverso qualsiasi apertura
grande abbastanza per un Minuscolo insetto. Lo sciame non può
recuperare punti ferita né ottenere punti ferita temporanei.
Azioni
Morsi. Attacco con Arma da Mischia: +3 a colpire, portata 0 m,
un bersaglio nello spazio dello sciame.
Colpisce: 10 (4d4) danni perforanti, o 5 (2d4) danni perforanti se
lo sciame è ha metà o meno dei suoi punti ferita.
Scimmione
Media bestia, disallineato
FORZA 16 (+3)
DESTREZZA 14 (+2)
COSTITUZIONE 14 (+2)
INTELLIGENZA 6 (-2)
SAGGEZZA 12 (+1)
Carisma 7 (-2)
Classe Armatura 12
\hspace*{0pt}\hfill{Punti Ferita}: 19 (3d8 + 6)
Velocità 9 m, scalata 9 m
Abilità Atletica +5, Percezione +3
Sensi Percezione passiva 13
Lingue -
Sfida 1/2 (100 PE)
Azioni
Multiattacco. Lo scimmione effettua due attacchi di pugno.
Pugno. Attacco con Arma da Mischia: +5 a colpire, portata 1,5
m, un bersaglio.
Colpisce: 6 (1d6 + 3) danni da botta.
Sasso. Attacco con Arma a Gittata: +5 a colpire, gittata 7,5/15
m, un bersaglio.
Colpisce: 6 (1d6 + 3) danni da botta.
Scimmione Gigante
Enorme bestia, disallineato
FORZA 23 (+6)
DESTREZZA 14 (+2)
COSTITUZIONE 18 (+4)
INTELLIGENZA 7 (-2)
SAGGEZZA 12 (+1)
Carisma 7 (-2)
Classe Armatura 12
\hspace*{0pt}\hfill{Punti Ferita}: 157 (15d12 + 60)
Velocità 12 m, scalata 12 m
Abilità Atletica +9, Percezione +4
Sensi Percezione passiva 14
Lingue -
Sfida 7 (2.900 PE)
Azioni
Multiattacco. Lo scimmione effettua due attacchi di pugno.
Pugno. Attacco con Arma da Mischia: +9 a colpire, portata 3 m,
un bersaglio.
Colpisce: 22 (3d10 + 6) danni da botta.
Sasso. Attacco con Arma a Gittata: +9 a colpire, gittata 15/30 m,
un bersaglio.
Colpisce: 30 (7d6 + 6) danni da botta.
Scorpione
Minuscola bestia, disallineato
FORZA 2 (-4)
DESTREZZA 11 (+0)
COSTITUZIONE 8 (-1)
INTELLIGENZA 1 (-5)
SAGGEZZA 8 (-1)
Carisma 2 (-4)
Classe Armatura 11 (armatura naturale)
\hspace*{0pt}\hfill{Punti Ferita}: 1 (1d4 - 1)
Velocità 3 m
Sensi vista cieca 3 m, Percezione passiva 9
Lingue -
Sfida 0 (10 PE)
Azioni
Pungiglione. Attacco con Arma da Mischia: +2 a colpire, portata
1,5 m, una creatura.
Colpisce: 1 danno perforante e il bersaglio deve effettuare un tiro
salvezza di Costituzione DC 9, e subire 4 (1d8) danni da veleno
se fallisce il tiro salvezza, o la metà di questi danni se lo riesce.
Scorpione Gigante
Grande bestia, disallineato
FORZA 15 (+2)
DESTREZZA 13 (+1)
COSTITUZIONE 15 (+2)
INTELLIGENZA 1 (-5)
SAGGEZZA 9 (-1)
Carisma 3 (-4)
Classe Armatura 15 (armatura naturale)
\hspace*{0pt}\hfill{Punti Ferita}: 52 (7d10 + 14)
Velocità 12 m
Sensi vista cieca 18 m, Percezione passiva 9
Lingue -
Sfida 3 (700 PE)
Azioni
Multiattacco. Lo scorpione effettua tre attacchi: due con gli
artigli e uno con il pungiglione.
Artiglio. Attacco con Arma da Mischia: +4 a colpire, portata 1,5
m, un bersaglio.
Colpisce: 6 (1d8 + 2) danni da botta e il bersaglio è afferrato
(DC 12 per fuggire). Lo scorpione ha due artigli, ciascuno dei
quali può afferrare solo un bersaglio.
Pungiglione. Attacco con Arma da Mischia: +4 a colpire, portata
1,5 m, una creatura.
Colpisce: 7 (1d10 + 2) danni perforanti e il bersaglio deve
effettuare un tiro salvezza di Costituzione DC 12, e subire 22
(4d10) danni da veleno se fallisce il tiro salvezza, o la metà di
questi danni se lo riesce.
Serpente Costrittore
Grande bestia, disallineato
FORZA 15 (+2)
DESTREZZA 14 (+2)
COSTITUZIONE 12 (+1)
INTELLIGENZA 1 (-5)
SAGGEZZA 10 (+0)
Carisma 3 (-4)
Classe Armatura 12
\hspace*{0pt}\hfill{Punti Ferita}: 13 (2d10 + 2)
Velocità 9 m, nuoto 9 m
Sensi vista cieca 3 m, Percezione passiva 10
Lingue -
Sfida 1/4 (50 PE)
Azioni
Morso. Attacco con Arma da Mischia: +4 a colpire, portata 1,5
m, una creatura.
Colpisce: 5 (1d6 + 2) danni perforanti.
Stritolare. Attacco con Arma da Mischia: +4 a colpire, portata
1,5 m, una creatura.
Colpisce: 6 (1d8 + 2) danni da botta, e il bersaglio è afferrato
(DC 14 per fuggire). Fino al termine dell’afferrare, la creatura è
intralciata, e il serpente non può stritolare un altro bersaglio.
Serpente Costrittore
Gigante
Enorme bestia, disallineato
FORZA 19 (+4)
DESTREZZA 14 (+2)
COSTITUZIONE 12 (+1)
INTELLIGENZA 1 (-5)
SAGGEZZA 10 (+0)
Carisma 3 (-4)
Classe Armatura 12
\hspace*{0pt}\hfill{Punti Ferita}: 60 (8d12 + 8)
Velocità 9 m, nuoto 9 m
Abilità Percezione +2
Sensi vista cieca 3 m, Percezione passiva 12
Lingue -
Sfida 2 (450 PE)
Azioni
Morso. Attacco con Arma da Mischia: +6 a colpire, portata 3 m,
una creatura.
Colpisce: 11 (2d6 + 4) danni perforanti.
Stritolare. Attacco con Arma da Mischia: +6 a colpire, portata
1,5 m, una creatura.
Colpisce: 13 (2d8 + 4) danni da botta, e il bersaglio è
afferrato (DC 16 per fuggire). Fino al termine dell’afferrare, la
creatura è intralciata, e il serpente non può stritolare un altro
bersaglio.
 
Serpente Velenoso
Minuscola bestia, disallineato
FORZA 2 (-4)
DESTREZZA 16 (+3)
COSTITUZIONE 11 (+0)
INTELLIGENZA 1 (-5)
SAGGEZZA 10 (+0)
Carisma 3 (-4)
Classe Armatura 13
\hspace*{0pt}\hfill{Punti Ferita}: 2 (1d4)
Velocità 9 m, nuoto 9 m
Sensi vista cieca 3 m, Percezione passiva 10
Lingue -
Sfida 1/8 (25 PE)
Azioni
Morso. Attacco con Arma da Mischia: +5 a colpire, portata 1,5
m, un bersaglio.
Colpisce: 1 danno perforante e il bersaglio deve effettuare un tiro
salvezza di Costituzione DC 10, e subire 5 (2d4) danni da veleno
se fallisce il tiro salvezza, o la metà di questi danni se lo riesce.
Serpente Velenoso
Gigante
Media bestia, disallineato
FORZA 10 (+0)
DESTREZZA 18 (+4)
COSTITUZIONE 13 (+1)
INTELLIGENZA 2 (-4)
SAGGEZZA 10 (+0)
Carisma 3 (-4)
Classe Armatura 14
\hspace*{0pt}\hfill{Punti Ferita}: 11 (2d8 + 2)
Velocità 9 m, nuoto 9 m
Abilità Percezione +2
Sensi vista cieca 3 m, Percezione passiva 12
Lingue -
Sfida 1/4 (50 PE)
Azioni
Morso. Attacco con Arma da Mischia: +6 a colpire, portata 3 m,
un bersaglio.
Colpisce: 6 (1d4 + 4) danni perforanti e il bersaglio deve
effettuare un tiro salvezza di Costituzione DC 11, e subire 10
(3d6) danni da veleno se fallisce il tiro salvezza, o la metà di
questi danni se lo riesce.
Serpente Volante
Un serpente volante è una serpe alata, dai colori
intensi, rinvenuta in giungle remote.
Minuscola bestia, disallineato
FORZA 4 (-3)
DESTREZZA 18 (+4)
COSTITUZIONE 11 (+0)
INTELLIGENZA 2 (-4)
SAGGEZZA 12 (+1)
Carisma 5 (-3)
Classe Armatura 14
\hspace*{0pt}\hfill{Punti Ferita}: 5 (2d4)
Velocità 9 m, nuoto 9 m, volo 18 m
Sensi vista cieca 3 m, Percezione passiva 11
Lingue -
Sfida 1/8 (25 PE)
Sorvolare. Il serpente non provoca attacchi di opportunità
quando vola via dalla portata di un nemico.
Azioni
Morso. Attacco con Arma da Mischia: +6 a colpire, portata 1,5
m, un bersaglio.
Colpisce: 1 danno perforante più 7 (3d4) danni da veleno.
Squalo Cacciatore
Uno squalo cacciatore è lungo da 4,5 a 6 metri e di
solito caccia in solitario nelle acque più profonde.
Grande bestia, disallineato
FORZA 18 (+4)
DESTREZZA 13 (+1)
COSTITUZIONE 15 (+2)
INTELLIGENZA 1 (-5)
SAGGEZZA 10 (+0)
Carisma 4 (-3)
Classe Armatura 12 (armatura naturale)
\hspace*{0pt}\hfill{Punti Ferita}: 45 (6d10 + 12)
Velocità 0 m, nuoto 12 m
Abilità Percezione +2
Sensi vista cieca 9 m, Percezione passiva 12
Lingue -
Sfida 2 (450 PE)
Frenesia Sanguinaria. Lo squalo ha vantaggio ai tiri di attacco
in mischia contro qualsiasi creatura che non sia al massimo dei
punti ferita.
Respirare Acqua. Lo squalo può respirare solo sottacqua.
Azioni
Morso. Attacco con Arma da Mischia: +6 a colpire, portata 1,5
m, un bersaglio.
Colpisce: 13 (2d8 + 4) danni perforanti.
Squalo Corallino
Gli squali corallini sono lunghi da 2 a 3 metri e vivono
nelle acque meno profonde e lungo le barriere coralline.
Media bestia, disallineato
FORZA 14 (+2)
DESTREZZA 13 (+1)
COSTITUZIONE 13 (+1)
INTELLIGENZA 1 (-5)
SAGGEZZA 10 (+0)
Carisma 4 (-3)
Classe Armatura 12 (armatura naturale)
\hspace*{0pt}\hfill{Punti Ferita}: 22 (4d8 + 4)
Velocità 0 m, nuoto 12 m
Abilità Percezione +2
Sensi vista cieca 9 m, Percezione passiva 12
Lingue -
Sfida 1/2 (100 PE)
Respirare Acqua. Lo squalo può respirare solo sottacqua.
Tattiche di Branco. Lo squalo ha vantaggio al tiro di attacco
contro una creatura se almeno uno degli alleati dello squalo si
trova entro 1,5 metri dalla creatura e quell’alleato non è inabile.
Azioni
Morso. Attacco con Arma da Mischia: +4 a colpire, portata 1,5
m, un bersaglio.
Colpisce: 6 (1d8 + 2) danni perforanti.
Squalo Gigante
Lo squalo gigante è lungo 9 metri e lo si incontra
normalmente solo negli oceani più profondi.
Enorme bestia, disallineato
FORZA 23 (+6)
DESTREZZA 11 (+0)
COSTITUZIONE 21 (+5)
INTELLIGENZA 1 (-5)
SAGGEZZA 10 (+0)
Carisma 5 (-3)
Classe Armatura 13 (armatura naturale)
\hspace*{0pt}\hfill{Punti Ferita}: 126 (11d12 + 55)
Velocità 0 m, nuoto 15 m
Abilità Percezione +3
Sensi vista cieca 18 m, Percezione passiva 13
Lingue -
Sfida 5 (1.800 PE)
Frenesia Sanguinaria. Lo squalo ha vantaggio ai tiri di attacco
in mischia contro qualsiasi creatura che non sia al massimo dei
punti ferita.
Respirare Acqua. Lo squalo può respirare solo sottacqua.
Azioni
Morso. Attacco con Arma da Mischia: +9 a colpire, portata 1,5
m, un bersaglio.
Colpisce: 22 (3d10 + 6) danni perforanti.
 
Strige
Questo orrendo mostro sembra un incrocio tra un
grosso pipistrello e una zanzara sovradimensionata. Le
sue zampe terminano in lunghe pinze, e la sua lunga
proboscide, simile ad un ago, fende l’aria mentre cerca
di nutrirsi del sangue delle creature viventi.
Minuscola bestia, disallineato
FORZA 4 (-3)
DESTREZZA 16 (+3)
COSTITUZIONE 11 (+0)
INTELLIGENZA 2 (-4)
SAGGEZZA 8 (-1)
Carisma 6 (-2)
Classe Armatura 14 (armatura naturale)
\hspace*{0pt}\hfill{Punti Ferita}: 2 (1d4)
Velocità 3 m, volo 12 m
Sensi visione al buio 18 m, Percezione passiva 9
Lingue -
Sfida 1/8 (25 PE)
Azioni
Risucchio di Sangue. Attacco con Arma da Mischia: +5 a
colpire, portata 1,5 m, una creatura.
Colpisce: 5 (1d4 + 3) danni perforanti e lo strige si attacca al
bersaglio. Mentre è attaccato, lo strige non attacca. Invece,
all’inizio di ciascun turno dello strige, il bersaglio perde 5 (1d4 +
3) punti ferita a causa della perdita di sangue.
Lo strige può staccarsi spendendo 1,5 metri di movimento. Lo fa
automaticamente dopo aver risucchiato 10 punti ferita dal
bersaglio o alla morte del bersaglio. Una creatura, compreso il
bersaglio, può usare la sua azione per staccare lo strige.
Tasso
Minuscola bestia, disallineato
FORZA 4 (-3)
DESTREZZA 11 (+0)
COSTITUZIONE 12 (+1)
INTELLIGENZA 2 (-4)
SAGGEZZA 12 (+1)
Carisma 5 (-3)
Classe Armatura 10
\hspace*{0pt}\hfill{Punti Ferita}: 3 (1d4 + 1)
Velocità 6 m, scavo 1,5 m
Sensi visione al buio 9 m, Percezione passiva 11
Lingue -
Sfida 0 (10 PE)
Olfatto Affinato. Il tasso ha vantaggio alle prove di Saggezza
(Percezione) basate sull’olfatto.
Azioni
Morso. Attacco con Arma da Mischia: +2 a colpire, portata 1,5
m, un bersaglio.
Colpisce: 1 danno perforante.
Tasso Gigante
Media bestia, disallineato
FORZA 13 (+1)
DESTREZZA 10 (+0)
COSTITUZIONE 15 (+2)
INTELLIGENZA 2 (-4)
SAGGEZZA 12 (+1)
Carisma 5 (-3)
Classe Armatura 10
\hspace*{0pt}\hfill{Punti Ferita}: 13 (2d8 + 4)
Velocità 9 m, scavo 3 m
Sensi visione al buio 9 m, Percezione passiva 11
Lingue -
Sfida 1/4 (50 PE)
Olfatto Affinato. Il tasso ha vantaggio alle prove di Saggezza
(Percezione) basate sull’olfatto.
Azioni
Multiattacco. Il tasso effettua due attacchi: uno con il morso e
uno con gli artigli.
Artigli. Attacco con Arma da Mischia: +3 a colpire, portata 1,5
m, un bersaglio.
Colpisce: 6 (2d4 + 1) danni taglienti.
Morso. Attacco con Arma da Mischia: +3 a colpire, portata 1,5
m, un bersaglio.
Colpisce: 4 (1d6 + 1) danni perforanti.
Tigre
Grande bestia, disallineato
FORZA 17 (+3)
DESTREZZA 15 (+2)
COSTITUZIONE 14 (+2)
INTELLIGENZA 3 (-4)
SAGGEZZA 12 (+1)
Carisma 8 (-1)
Classe Armatura 12
\hspace*{0pt}\hfill{Punti Ferita}: 37 (5d10 + 10)
Velocità 12 m
Abilità Furtività +6, Percezione +3
Sensi visione al buio 18 m, Percezione passiva 13
Lingue -
Sfida 1 (200 PE)
Balzo. Se la tigre si muove di almeno 6 metri diretta verso una
creatura e la colpisce con un attacco di artiglio durante lo stesso
turno, il bersaglio deve riuscire un tiro salvezza di Forza DC 13 o
cadere prono. Se il bersaglio è prono, la tigre può effettuare un
attacco di morso contro di esso come azione bonus.
Olfatto Affinato. La tigre ha vantaggio alle prove di Saggezza
(Percezione) basate sull’olfatto.
Azioni
Artiglio. Attacco con Arma da Mischia: +5 a colpire, portata 1,5
m, un bersaglio.
Colpisce: 7 (1d8 + 3) danni taglienti.
Morso. Attacco con Arma da Mischia: +5 a colpire, portata 1,5
m, un bersaglio.
Colpisce: 8 (1d10 + 3) danni perforanti.
Tigre dai Denti a
Sciabola
Grande bestia, disallineato
FORZA 18 (+4)
DESTREZZA 14 (+2)
COSTITUZIONE 15 (+2)
INTELLIGENZA 3 (-4)
SAGGEZZA 12 (+1)
Carisma 8 (-1)
Classe Armatura 12
\hspace*{0pt}\hfill{Punti Ferita}: 52 (7d10 + 14)
Velocità 12 m
Abilità Furtività +6, Percezione +3
Sensi Percezione passiva 13
Lingue -
Sfida 2 (450 PE)
Balzo. Se la tigre si muove di almeno 6 metri diretta verso una
creatura e la colpisce con un attacco di artiglio durante lo stesso
turno, il bersaglio deve riuscire un tiro salvezza di Forza DC 14 o
cadere prono. Se il bersaglio è prono, la tigre può effettuare un
attacco di morso contro di esso come azione bonus.
Olfatto Affinato. La tigre ha vantaggio alle prove di Saggezza
(Percezione) basate sull’olfatto.
Azioni
Artiglio. Attacco con Arma da Mischia: +6 a colpire, portata 1,5
m, un bersaglio.
Colpisce: 12 (2d6 + 5) danni taglienti.
Morso. Attacco con Arma da Mischia: +6 a colpire, portata 1,5
m, un bersaglio.
Colpisce: 10 (1d10 + 5) danni perforanti.
Vespa Gigante
Media bestia, disallineato
FORZA 10 (+0)
DESTREZZA 14 (+2)
COSTITUZIONE 10 (+0)
INTELLIGENZA 1 (-5)
SAGGEZZA 10 (+0)
Carisma 3 (-4)
Classe Armatura 12
\hspace*{0pt}\hfill{Punti Ferita}: 13 (3d8)
Velocità 3 m, volo 15 m
Sensi Percezione passiva 10
Lingue -
Sfida 1/2 (100 PE)
Azioni
Pungiglione. Attacco con Arma da Mischia: +4 a colpire, portata
1,5 m, una creatura.
Colpisce: 5 (1d6 + 2) danni perforanti e il bersaglio deve
effettuare un tiro salvezza di Costituzione DC 11, e subire 10
(3d6) danni da veleno se fallisce il tiro salvezza, o la metà di
questi danni se lo riesce. Se il danno da veleno riduce il bersaglio
a 0 punti ferita, il bersaglio è stabile ma avvelenato per 1 ora,
anche dopo aver recuperato i punti ferita, e mentre è avvelenato
in questo modo resta paralizzato.
Worg
I worg sono mostruosi predatori dall’aspetto simile ad
un lupo che amano cacciare e divorare le creature più
deboli di loro.
Grande mostruosità, neutrale malvagio
FORZA 16 (+3)
DESTREZZA 13 (+1)
COSTITUZIONE 13 (+1)
INTELLIGENZA 7 (-2)
SAGGEZZA 11 (+0)
Carisma 8 (-1)
Classe Armatura 13 (armatura naturale)
\hspace*{0pt}\hfill{Punti Ferita}: 26 (4d10 + 4)
Velocità 15 m
Abilità Percezione +4
Sensi visione al buio 18 m, Percezione passiva 14
Lingue Goblin, Worg
Sfida 1/2 (100 PE)
Udito e Olfatto Affinato. Il worg ha vantaggio nelle prove di
Saggezza (Percezione) basate su udito o olfatto.
Azioni
Morso. Attacco con Arma da Mischia: +5 a colpire, portata 1,5
m, un bersaglio.
Colpisce: 10 (2d6 + 3) danni perforanti. Se il bersaglio è una
creatura, deve riuscire un tiro salvezza di Forza DC 13 o cadere
prona.
 
Appendice B:
Personaggi Non
Giocanti
Questa appendice contiene le statistiche di vari
personaggi non giocanti (PNG) umanoidi che gli
avventurieri possono incontrare nel corso di una
campagna, da infimi popolani a potenti arcimaghi.
Queste statistiche possono essere utilizzate per
rappresentare PNG umani e non.
Personalizzare i PNG
Esistono molti semplici modi di personalizzare i PNG di
questa appendice per l’uso nella tua campagna
casalinga.
Tratti Razziali. Puoi aggiungere tratti razziali ad un
PNG. Ad esempio, un mezzuomo sacerdote avrebbe
velocità 7,5 metri e il tratto Fortunato. Aggiungere i tratti
razziali ad un PNG non ne modifica il grado di sfida.
Cambiare Incantesimi. Un modo per personalizzare
un PNG incantatore è quello di rimpiazzare uno o più
dei suoi incantesimi. Puoi sostituire qualsiasi
incantesimo della lista di incantesimi del PNG con un
diverso incantesimo dello stesso livello dalla stessa
lista di incantesimi. Cambiare incantesimi in questo
modo non modifica il grado di sfida del PNG.
Cambiare Armi e Armatura. Puoi migliorare o
peggiorare l’armatura del PNG o aggiungere o
cambiare armi. Le modifiche alla Classe Armatura e ai
danni possono modificare il grado di sfida del PNG.
Oggetti Magici. Più potente è un PNG, maggiori le
probabilità che possieda uno o più oggetti magici. Un
mago, ad esempio, potrebbe avere una bacchetta o un
bastone magico, oltre ad una o più pozioni e
pergamene. Fornire un PNG di un potente oggetto
magico capace di infliggere danni potrebbe modificarne
il grado di sfida.
Alcuni oggetti magici di esempio sono descritti più
avanti in questo documento.
Combattenti
I combattenti sono individui che si guadagnano da
vivere mettendo la loro spada al servizio di un individuo
o un ideale.
Guardia
Le guardie comprendono membri della ronda cittadina,
sentinelle di una cittadella o città fortificata e le guardie
del corpo di nobili e mercanti.
Media umanoide (qualsiasi razza), qualsiasi allineamento
FORZA 13 (+1)
DESTREZZA 12 (+1)
COSTITUZIONE 12 (+1)
INTELLIGENZA 10 (+0)
SAGGEZZA 11 (+0)
Carisma 10 (+0)
Classe Armatura 16 (giaco di maglia, scudo)
\hspace*{0pt}\hfill{Punti Ferita}: 11 (2d8 + 2)
Velocità 9 m
Abilità Percezione +2
Sensi Percezione passiva 12
Lingue una qualsiasi lingua (di solito il Comune)
Sfida 1/8 (25 PE)
Azioni
Lancia. Attacco con Arma da Mischia o a Gittata: +3 a colpire,
portata 1,5 m o gittata 6/18 m, un bersaglio.
Colpisce: 4 (1d6 + 1) danni perforanti o 5 (1d8 + 1) danni
perforanti se impiegata con due mani per effettuare un attacco da
mischia.
Veterano
Guerrieri sopravvissuti a lungo, guadagnandosi una
grande fama di esperti e abili combattenti.
Media umanoide (qualsiasi razza), qualsiasi allineamento
FORZA 16 (+3)
DESTREZZA 13 (+1)
COSTITUZIONE 14 (+2)
INTELLIGENZA 10 (+0)
SAGGEZZA 11 (+0)
Carisma 10 (+0)
Classe Armatura 17 (armatura di strisce)
\hspace*{0pt}\hfill{Punti Ferita}: 58 (9d8 + 18)
Velocità 9 m
Abilità Atletica +5, Percezione +2
Sensi Percezione passiva 12
Lingue una lingua qualsiasi (di solito il Comune)
Sfida 3 (700 PE)
Azioni
Multiattacco. Il veterano effettua due attacchi con la spada
lunga. Se ha estratto una spada corta, può effettuare anche un
attacco con la spada corta.
Spada Lunga. Attacco con Arma da Mischia: +5 a colpire,
portata 1,5 m, un bersaglio.
Colpisce: 7 (1d8 + 3) danni taglienti, o 8 (1d10 + 3) danni
taglienti se usata con due mani.
Spada Corta. Attacco con Arma da Mischia: +5 a colpire, portata
1,5 m, un bersaglio.
Colpisce: 6 (1d6 + 3) danni perforanti.
Balestra Pesante. Attacco con Arma a Gittata: +3 a colpire,
gittata 30/120 m, un bersaglio.
Colpisce: 6 (1d10 + 1) danni perforanti.
Cavaliere
I cavalieri sono combattenti che giurano fedeltà a
sovrani, ordini religiosi, e nobili cause. L’allineamento
del cavaliere determina fino a che punto è disposto ad
onorare il suo giuramento.
Media umanoide (qualsiasi razza), qualsiasi allineamento
FORZA 16 (+3)
DESTREZZA 11 (+0)
COSTITUZIONE 14 (+2)
INTELLIGENZA 11 (+0)
SAGGEZZA 11 (+0)
Carisma 15 (+2)
Classe Armatura 18 (armatura di piastre)
\hspace*{0pt}\hfill{Punti Ferita}: 52 (8d8 + 16)
Velocità 9 m
Tiri Salvezza Costituzione +4, Saggezza +2
Sensi Percezione passiva 10
Lingue una qualsiasi lingua (di solito il Comune)
Sfida 3 (700 PE)
Coraggioso. Il cavaliere ha +1d6 ai tiri salvezza contro
l’essere spaventato.
Azioni
Multiattacco. Il cavaliere effettua due attacchi da mischia.
Spada Grossa. Attacco con Arma da Mischia: +5 a colpire,
portata 1,5 m, un bersaglio.
Colpisce: 10 (2d6 + 3) danni taglienti.
Balestra Pesante. Attacco con Arma a Gittata: +2 a colpire,
gittata 30/120 m, un bersaglio.
Colpisce: 5 (1d10) perforanti.
Autorità (Ricarica dopo un Riposo Breve o Lungo). Per 1
minuto, il cavaliere può pronunciare un comando speciale o
avvertimento ogni qualvolta una creatura non ostile entro 9 metri
da lui, e che possa vedere, effettua un tiro di attacco o tiro
salvezza. La creatura può sommare un d4 al suo tiro purchè
possa udire e comprendere il cavaliere. Una creatura può
beneficiare di un solo dado Autorità alla volta. Questo effetto
termina se il cavaliere è inabile.
Reazioni
Parata. Il cavaliere può aggiungere 2 alla sua Difesa contro un
attacco da mischia che lo colpirebbe. Per farlo, il cavaliere deve
vedere l’attaccante e star impugnando un’arma da mischia.
 
Gladiatore
Addestrati per intrattenere le folle, sono tra i
combattenti più pericolosi in circolazione.
Media umanoide (qualsiasi razza), qualsiasi allineamento
FORZA 18 (+4)
DESTREZZA 15 (+2)
COSTITUZIONE 16 (+3)
INTELLIGENZA 10 (+0)
SAGGEZZA 12 (+1)
Carisma 15 (+2)
Classe Armatura 16 (armatura di cuoio borchiato, scudo)
\hspace*{0pt}\hfill{Punti Ferita}: 112 (15d8 + 45)
Velocità 9 m
Tiri Salvezza Forza +7, Destrezza +5, Costituzione +6
Abilità Atletica +10, Intimidazione +5
Sensi Percezione passiva 11
Lingue una lingua qualsiasi (di solito il Comune)
Sfida 5 (1.800 PE)
Bruto. Un’arma da mischia infligge un dado aggiuntivo di danno
quando un gladiatore colpisce con essa (già incluso nell’attacco).
Coraggioso. Il gladiatore ha +1d6 ai tiri salvezza contro
l’essere spaventato.
Azioni
Multiattacco. Il gladiatore effettua tre attacchi da mischia o due
attacchi a gittata.
Lancia. Attacco con Arma da Mischia o a Gittata: +7 a colpire,
portata 1,5 m o gittata 6/18 m, un bersaglio.
Colpisce: 11 (2d6 + 4) danni perforanti, o 13 (2d8 + 4) danni
taglienti se usata con due mani.
Botta di Scudo. Attacco con Arma da Mischia: +7 a colpire,
portata 1,5 m, un bersaglio.
Colpisce: 9 (2d4 + 4) danni da botta. Se il bersaglio è una
creatura di taglia Media o inferiore, deve riuscire un tiro salvezza
di Forza DC 15 o cadere prono.
Reazioni
Parata. Il gladiatore somma 3 alla sua Difesa contro un attacco da
mischia che lo colpirebbe. Per farlo, il gladiatore deve vedere
l’attaccante e impugnare un’arma da mischia.
Cittadini
In questa categoria rientrano quegli individui che si
occupano di mandare avanti il mondo, svolgendo le
mansioni necessarie affinché i campi vengano coltivati,
le città amministrate, il cibo coltivato e nuovi territori
esplorati.
Nobile
I nobili comandano sulla popolazione, in virtù di un
diritto di nascita o per le ricchezze accumulate. Tra
costoro si annoverano anche i cortigiani che affollano le
corti dei ricchi e dei potenti.
Media umanoide (qualsiasi razza), qualsiasi allineamento
FORZA 11 (+0)
DESTREZZA 12 (+1)
COSTITUZIONE 11 (+0)
INTELLIGENZA 12 (+1)
SAGGEZZA 14 (+2)
Carisma 16 (+3)
Classe Armatura 15 (pettorale)
\hspace*{0pt}\hfill{Punti Ferita}: 9 (2d8)
Velocità 9 m
Abilità Intuizione +4, Persuasione +5, Raggiro +5
Sensi Percezione passiva 12
Lingue due lingue qualsiasi
Sfida 1/8 (25 PE)
Azioni
Stocco. Attacco con Arma da Mischia: +3 a colpire, portata 1,5
m, un bersaglio.
Colpisce: 5 (1d8 + 1) danni perforanti.
Reazioni
Parata. Il nobile somma 2 alla sua Difesa contro un attacco da
mischia che lo colpirebbe. Per farlo, il nobile deve vedere
l’attaccante e impugnare un’arma da mischia.
Popolano
I popolani comprendono contadini, servi, schiavi,
servitori, pellegrini, mercanti, artigiani ed eremiti.
Media umanoide (qualsiasi razza), qualsiasi allineamento
FORZA 10 (+0)
DESTREZZA 10 (+0)
COSTITUZIONE 10 (+0)
INTELLIGENZA 10 (+0)
SAGGEZZA 10 (+0)
Carisma 10 (+0)
Classe Armatura 10
\hspace*{0pt}\hfill{Punti Ferita}: 4 (1d8)
Velocità 9 m
Sensi Percezione passiva 10
Lingue una qualsiasi lingua (di solito il Comune)
Sfida 0 (10 PE)
Azioni
Randello. Attacco con Arma da Mischia: +2 a colpire, portata
1,5 m, un bersaglio.
Colpisce: 2 (1d4) danni da botta.
Criminali
I criminali sono individui che vivono al margine della
legalità, procurandosi il pane svolgendo attività spesso
considerate illecite e immorali.
Picchiatore
I picchiatori sono criminali spietati abili nell’intimidire e
perpetrare atti di violenza. Lavorano per soldi e si fanno
pochi scrupoli.
Media umanoide (qualsiasi razza), qualsiasi allineamento non
buono
FORZA 15 (+2)
DESTREZZA 11 (+0)
COSTITUZIONE 14 (+2)
INTELLIGENZA 10 (+0)
SAGGEZZA 10 (+0)
Carisma 11 (+0)
Classe Armatura 11 (armatura di cuoio)
\hspace*{0pt}\hfill{Punti Ferita}: 32 (5d8 + 10)
Velocità 9 m
Abilità Intimidazione +2
Sensi Percezione passiva 10
Lingue una lingua qualsiasi (di solito il Comune)
Sfida 1/2 (100 PE)
Tattiche di Branco. Il picchiatore ha vantaggio ai tiri di attacco
contro una creatura se almeno uno degli alleati del picchiatore si
trova entro 1,5 metri dalla creatura e quell’alleato non è inabile.
Azioni
Multiattacco. Il picchiatore effettua due attacchi da mischia.
Mazza. Attacco con Arma da Mischia: +4 a colpire, portata 1,5
m, una creatura.
Colpisce: 5 (1d6 + 2) danni da botta.
Balestra Pesante. Attacco con Arma a Gittata: +2 a colpire,
gittata 30/120 m, un bersaglio.
Colpisce: 5 (1d10) danni perforanti.
Bandito/Pirata
Che siano uomini di strada o di mare (pirati) costoro
guadagnano da vivere depredando il prossimo.
Media umanoide (qualsiasi razza), qualsiasi allineamento non
legale
FORZA 11 (+0)
DESTREZZA 12 (+1)
COSTITUZIONE 12 (+1)
INTELLIGENZA 10 (+0)
SAGGEZZA 10 (+0)
Carisma 10 (+0)
Classe Armatura 12 (armatura di cuoio)
\hspace*{0pt}\hfill{Punti Ferita}: 11 (2d8 + 2)
Velocità 9 m
Sensi Percezione passiva 10
Lingue una qualsiasi lingua (di solito il Comune)
Sfida 1/8 (25 PE)
Azioni
Scimitarra. Attacco con Arma da Mischia: +3 a colpire, portata
1,5 m, un bersaglio.
Colpisce: 4 (1d6 + 1) danni taglienti.
Balestra Leggera. Attacco con Arma a Gittata: +3 a colpire,
gittata 24/96 metri, un bersaglio.
Colpisce: 5 (1d8 + 1) danni taglienti.
Spia
Una spia è un individuo addestramento nel reperire
segreti per conto di qualcuno, o a volte per rivenderli al
miglior offerente.
Media umanoide (qualsiasi razza), qualsiasi allineamento
FORZA 10 (+0)
DESTREZZA 15 (+2)
COSTITUZIONE 10 (+0)
INTELLIGENZA 12 (+1)
SAGGEZZA 14 (+2)
Carisma 16 (+3)
Classe Armatura 12
\hspace*{0pt}\hfill{Punti Ferita}: 27 (6d8)
Velocità 9 m
Abilità Furtività +4, Intuizione +4, Investigazione +5,
Percezione +6, Persuasione +5, Raggiro +5, Rapidità di Mano +4
Sensi Percezione passiva 16
Lingue due lingue qualsiasi
Sfida 1 (200 PE)
Attacco Furtivo (1/Turno). La spia infligge 7 (2d6) danni aggiuntivi
quando colpisce un bersaglio con un attacco con arma e ha vantaggio
al tiro di attacco, o quando il bersaglio è entro 1,5 metri da un alleato
dell’assassino che non è inabile e l’assassino non ha svantaggio al
tiro di attacco.
Azione Astuta. Durante ciascun suo turno, la spia può usare un’azione
bonus per effettuare l’azione Disimpegnarsi, Nascondersi o Scattare.
Azioni
Multiattacco. La spia effettua due attacchi da mischia.
Spada Corta. Attacco con Arma da Mischia: +4 a colpire, portata
1,5 m, un bersaglio.
Colpisce: 5 (1d6 + 2) danni perforanti.
Balestrino. Attacco con Arma a Gittata: +4 a colpire, gittata 9/36
m, un bersaglio. Colpisce: 5 (1d6 + 2) danni perforanti.
 
Capitano dei Banditi/Pirata
Che viva in terra o in mare, è un individuo munito di una
grande personalità che riesce a tenere in riga la
marmaglia che risponde ai suoi ordini.
Media umanoide (qualsiasi razza), qualsiasi allineamento non
legale
FORZA 15 (+2)
DESTREZZA 16 (+3)
COSTITUZIONE 14 (+2)
INTELLIGENZA 14 (+2)
SAGGEZZA 11 (+0)
Carisma 14 (+2)
Classe Armatura 15 (armatura di cuoio borchiato)
\hspace*{0pt}\hfill{Punti Ferita}: 65 (10d8 + 8)
Velocità 9 m
Tiri Salvezza Forza +4, Destrezza +5, Saggezza +2
Abilità Atletica +4, Raggiro +4
Sensi Percezione passiva 10
Lingue due lingue qualsiasi
Sfida 2 (450 PE)
Azioni
Multiattacco. Il capitano effettua tre attacchi da mischia: due con
la scimitarra e uno con il pugnale. Oppure il capitano effettua
due attacchi a gittata con i pugnali.
Scimitarra. Attacco con Arma da Mischia: +5 a colpire, portata
1,5 m, un bersaglio.
Colpisce: 6 (1d6 + 3) danni taglienti.
Pugnale. Attacco con Arma da Mischia o a Gittata: +5 a colpire,
portata 1,5 m o gittata 6/18 metri, un bersaglio.
Colpisce: 5 (1d4 + 3) danni perforanti.
Reazioni
Parata. Il capitano somma 2 alla sua Difesa contro un attacco da
mischia che lo colpirebbe. Per farlo, il capitano deve vedere
l’attaccante e impugnare un’arma da mischia.
Assassino
Solitari o membri di una gilda, gli assassini sono pagati
per eliminare, spesso in modo silenzioso e discreto,
rivali e nemici dei loro datori di lavoro.
Media umanoide (qualsiasi razza), qualsiasi allineamento non
buono
FORZA 11 (+0)
DESTREZZA 16 (+3)
COSTITUZIONE 14 (+2)
INTELLIGENZA 13 (+1)
SAGGEZZA 11 (+0)
Carisma 10 (+0)
Classe Armatura 15 (armatura di cuoio borchiato)
\hspace*{0pt}\hfill{Punti Ferita}: 78 (12d8 + 24)
Velocità 9 m
Tiri Salvezza Destrezza +6, Intelligenza +4
Abilità Acrobazia +6, Furtività +9, Percezione +3, Raggiro +3
Sensi Percezione passiva 13
Lingue Gergo dei Ladri più due altre lingue
Sfida 8 (3.900 PE)
Assassinare. Durante il suo primo turno, l’assassino ha
vantaggio ai tiri di attacco contro le creature che non hanno
ancora svolto nessun turno. Qualsiasi colpo che l’assassino
mandi a segno contro una creatura sorpresa, è un colpo critico.
Attacco Furtivo (1/Turno). L’assassino infligge 14 (4d6) danni
aggiuntivi quando colpisce un bersaglio con un attacco con arma
e ha vantaggio al tiro di attacco, o quando il bersaglio è entro 1,5
metri da un alleato dell’assassino che non è inabile e l’assassino
non ha svantaggio al tiro di attacco.
Evasione. Se l’assassino è vittima di un effetto che permette di
effettuare un tiro salvezza di Destrezza per dimezzare i danni,
l’assassino non prende danni se riesce il tiro salvezza, e solo la
metà se lo fallisce.
Azioni
Multiattacco. L’assassino effettua due attacchi con le spade corte.
Spada Corta. Attacco con Arma da Mischia: +6 a colpire, portata
1,5 m, un bersaglio.
Colpisce: 6 (1d6 + 3) danni perforanti, e il bersaglio deve effettuare un
tiro salvezza di Costituzione DC 15, subendo 24 (7d6) danni da veleno se
fallisce il tiro salvezza, o la metà di questi danni se lo riesce.
Balestra Leggera. Attacco con Arma a Gittata: +6 a colpire,
gittata 24/96 metri, un bersaglio.
Colpisce: 7 (1d8 + 3) danni perforanti, e il bersaglio deve effettuare un
tiro salvezza di Costituzione DC 15, subendo 24 (7d6) danni da veleno se
fallisce il tiro salvezza, o la metà di questi danni se lo riesce.
Magi
I magi trascorrono la vita nello studio e la pratica della
magia.
VARIANTE: FAMIGLI
Qualsiasi incantatore che possa eseguire l’incantesimo trovare
famiglio (come un magio) è probabile che abbia un famiglio. Il
famiglio può essere una delle creature descritte nell’incantesimo
(vedi le Regole Base) o qualche altro mostro Minuscolo, come
un artiglio strisciante, un diavoletto, uno pseudodrago o un
demonietto.
Magio Avventuriero
Un magio novizio, che ha superato con successo le sue
prime avventure e ha iniziato a stabilire una
reputazione come nobile o famigerato avventuriero.
Media umanoide (qualsiasi razza), qualsiasi malvagio
FORZA 9 (-1)
DESTREZZA 14 (+2)
COSTITUZIONE 11 (+0)
INTELLIGENZA 17 (+3)
SAGGEZZA 12 (+1)
Carisma 11 (+0)
Classe Armatura 12
\hspace*{0pt}\hfill{Punti Ferita}: 22 (5d8)
Velocità 9 m
Tiri Salvezza Intelligenza +5, Saggezza +3
Abilità Arcano +5, Storia +5
Sensi Percezione passiva 11
Lingue quattro lingue qualsiasi
Sfida 1 (200 PE)
Incantesimi. Il magio è un incantatore di 4° livello. La sua
abilità da incantatore è l’Intelligenza (DC dei tiri salvezza degli
incantesimi 13, +5 al colpire con attacchi con incantesimo). Il
magio ha preparato i seguenti incantesimi da mago:
Trucchetti (a volontà): luce, mano magica, stretta folgorante
1° livello (4 slot): charme su persone, dardo incantato
2° livello (3 slot): bloccare persona, passo velato
Azioni
Bastone. Attacco con Arma da Mischia: +1 a colpire, portata 1,5
m, un bersaglio.
Colpisce: 3 (1d8 - 1) danni da botta.
Grande Magio
Un magio che ha stabilito una discreta fama nel
territorio e che attira intorno a sé studenti da ogni dove.
Media umanoide (qualsiasi razza), qualsiasi allineamento
FORZA 9 (-1)
DESTREZZA 14 (+2)
COSTITUZIONE 11 (+0)
INTELLIGENZA 17 (+3)
SAGGEZZA 12 (+1)
Carisma 11 (+0)
Classe Armatura 12 (15 con armatura del magio)
\hspace*{0pt}\hfill{Punti Ferita}: 40 (9d8)
Velocità 9 m
Tiri Salvezza Intelligenza +6, Saggezza +4
Abilità Arcano +6, Storia +6
Sensi Percezione passiva 11
Lingue quattro lingue qualsiasi
Sfida 6 (2.300 PE)
Incantesimi. Il magio è un incantatore di 9° livello. La sua
abilità da incantatore è l’Intelligenza (DC dei tiri salvezza degli
incantesimi 14, +6 al colpire con attacchi con incantesimo). Il
magio ha preparato i seguenti incantesimi da mago:
Trucchetti (a volontà): dardo infuocato, luce, mano magica,
prestidigitazione
1° livello (4 slot): armatura del magio, dardo incantato,
individuare magia, scudo
2° livello (3 slot): passo velato, suggestione
3° livello (3 slot): controincantesimo, palla di fuoco, volare
4° livello (3 slot): invisibilità superiore, tempesta di ghiaccio
5° livello (1 slot): cono di freddo
Azioni
Pugnale. Attacco con Arma da Mischia o a Gittata: +5 a colpire,
portata 1,5 m o gittata 6/18 m, un bersaglio.
Colpisce: 4 (1d4 + 2) danni perforanti.
 
Arcimagio
Un mago molto potente (e anche molto anziano) che
studia i segreti del multiverso.
Media umanoide (qualsiasi razza), qualsiasi allineamento
FORZA 10 (+0)
DESTREZZA 14 (+2)
COSTITUZIONE 12 (+1)
INTELLIGENZA 20 (+5)
SAGGEZZA 15 (+2)
Carisma 16 (+3)
Classe Armatura 12 (15 con armatura del magio)
\hspace*{0pt}\hfill{Punti Ferita}: 99 (18d8 + 18)
Velocità 9 m
Tiri Salvezza Intelligenza +9, Saggezza +6
Abilità Arcano +13, Storia +13
Resistenze al Danno danno degli incantesimi; da botta,
perforante e tagliente non magico (da pelle di pietra)
Sensi Percezione passiva 12
Lingue sei lingue qualsiasi
Sfida 12 (8.400 PE)
Incantesimi. Il magio è un incantatore di 18° livello. La sua
abilità da incantatore è l’Intelligenza (DC dei tiri salvezza degli
incantesimi 17, +9 al colpire con attacchi con incantesimo).
L’arcimagio può eseguire camuffare sé stesso e invisibilità a
volontà e ha preparato i seguenti incantesimi da mago:
Trucchetti (a volontà): dardo infuocato, luce, mano magica,
prestidigitazione, stretta folgorante
1° livello (4 slot): armatura magica*, dardo incantato,
identificare, individuare magia
2° livello (3 slot): immagine speculare, individuazione dei
pensieri, passo velato
3° livello (3 slot): controincantesimo, fulmine
4° livello (3 slot): esilio, pelle di pietra*, scudo di fuoco
5° livello (3 slot): cono di freddo, muro di forza, scrutare
6° livello (1 slot): globo di invulnerabilità
7° livello (1 slot): teletrasporto
8° livello (1 slot): vuoto mentale*
9° livello (1 slot): fermare il tempo
* L’arcimagio esegue questi incantesimi su di sé prima del
combattimento.
Azioni
Pugnale. Attacco con Arma da Mischia o a Gittata: +6 a colpire,
portata 1,5 m o gittata 6/18 m, un bersaglio.
Colpisce: 4 (1d4 + 2) danni perforanti.
Sacerdoti
I sacerdoti sono devoti di una divinità o una fede che si
prendono cura di impartire gli insegnamenti divini al loro
gregge.
Cultista
I cultisti giurano fedeltà ai poteri oscuri, e nelle loro
credenze e pratiche mostrano spesso segni di follia.
Media umanoide (qualsiasi razza), qualsiasi allineamento non
buono
FORZA 11 (+0)
DESTREZZA 12 (+1)
COSTITUZIONE 10 (+0)
INTELLIGENZA 10 (+0)
SAGGEZZA 11 (+0)
Carisma 10 (+0)
Classe Armatura 12 (armatura di cuoio)
\hspace*{0pt}\hfill{Punti Ferita}: 9 (2d8)
Velocità 9 m
Abilità Raggiro +2, Religione +2
Sensi Percezione passiva 10
Lingue una qualsiasi lingua (di solito il Comune)
Sfida 1/8 (25 PE)
Oscura Devozione. Il cultista ha vantaggio sui tiri salvezza
contro l’essere affascinato o spaventato.
Azioni
Scimitarra. Attacco con Arma da Mischia: +3 a colpire, portata
1,5 m, una creatura.
Colpisce: 4 (1d6 + 1) danni taglienti.
Accolito
Gli accoliti sono membri di grado minore del clero, e di
solito rispondono ad un sacerdote di rango superiore.
Svolgono diverse funzioni in un tempio e gli viene
conferita dalla loro divinità l’abilità di eseguire
incantesimi minori.
Media umanoide (qualsiasi razza), qualsiasi allineamento
FORZA 10 (+0)
DESTREZZA 10 (+0)
COSTITUZIONE 10 (+0)
INTELLIGENZA 10 (+0)
SAGGEZZA 14 (+2)
Carisma 11 (+0)
Classe Armatura 10
\hspace*{0pt}\hfill{Punti Ferita}: 9 (2d8)
Velocità 9 m
Abilità Medicina +4, Religione +2
Sensi Percezione passiva 12
Lingue una qualsiasi lingua (di solito il Comune)
Sfida 1/4 (50 PE)
Incantesimi. L’accolito è un incantatore di 1° livello. La sua
abilità da incantatore è la Saggezza (DC dei tiri salvezza degli
incantesimi 12, +4 al colpire con attacchi con incantesimo).
L’accolito ha preparato i seguenti incantesimi da chierico:
Trucchetti (a volontà): fiamma sacra, luce, taumaturgia
1° livello (3 slot): benedizione, cura ferite, santuario
Azioni
Randello. Attacco con Arma da Mischia: +2 a colpire, portata
1,5 m, un bersaglio.
Colpisce: 2 (1d4) danni da botta.
Fanatico del Culto
Sono i capi di un culto, che usano il proprio carisma e i
propri dogmi per influenzare i deboli di volontà.
Media umanoide (qualsiasi razza), qualsiasi allineamento non
buono
FORZA 11 (+0)
DESTREZZA 14 (+2)
COSTITUZIONE 12 (+1)
INTELLIGENZA 10 (+0)
SAGGEZZA 13 (+1)
Carisma 14 (+2)
Classe Armatura 13 (armatura di cuoio)
\hspace*{0pt}\hfill{Punti Ferita}: 33 (6d8 + 6)
Velocità 9 m
Abilità Persuasione +4, Raggiro +4, Religione +2
Sensi Percezione passiva 11
Lingue una qualsiasi lingua (di solito il Comune)
Sfida 2 (450 PE)
Incantesimi. Il sacerdote è un incantatore di 4° livello. La sua
abilità da incantatore è la Saggezza (DC dei tiri salvezza degli
incantesimi 11, +3 al colpire con attacchi con incantesimo). Il
sacerdote ha preparato i seguenti incantesimi da chierico:
Trucchetti (a volontà): fiamma sacra, luce, taumaturgia
1° livello (4 slot): comando, infliggi ferite, scudo della fede
2° livello (3 slot): arma spirituale, blocca persona
Oscura Devozione. Il cultista ha vantaggio sui tiri salvezza
contro l’essere affascinato o spaventato.
Azioni
Multiattacco. Il fanatico effettua due attacchi da mischia.
Pugnale. Attacco con Arma da Mischia o a Gittata: +4 a colpire,
portata 1,5 m o gittata 6/18 m, una creatura.
Colpisce: 4 (1d4 + 2) danni perforanti.
Gran Sacerdote
Sono individui al comando di un tempio o altro luogo
sacro e che hanno a loro disposizione diversi accoliti.
Media umanoide (qualsiasi razza), qualsiasi allineamento
FORZA 10 (+0)
DESTREZZA 10 (+0)
COSTITUZIONE 12 (+1)
INTELLIGENZA 13 (+1)
SAGGEZZA 16 (+3)
Carisma 13 (+1)
Classe Armatura 13 (giaco di maglia)
\hspace*{0pt}\hfill{Punti Ferita}: 27 (5d8 + 5)
Velocità 7,5 m
Abilità Medicina +7, Persuasione +3, Religione +4
Sensi Percezione passiva 13
Lingue due lingue qualsiasi
Sfida 2 (450 PE)
Eminenza Divina. Come azione bonus, il sacerdote può spendere
uno slot incantesimo per far sì che il suo attacco con arma da
mischia infligge 10 (3d6) danni da Luce aggiuntivi. Il beneficio
dura fino al termine del turno. Se il sacerdote spende uno slot di
2° livello o più alto, il danno aggiuntivo aumenta di 1d6 per ogni
livello sopra il 1°.
Incantesimi. Il sacerdote è un incantatore di 5° livello. La sua
abilità da incantatore è la Saggezza (DC dei tiri salvezza degli
incantesimi 13, +5 al colpire con attacchi con incantesimo). Il
sacerdote ha preparato i seguenti incantesimi da chierico:
Trucchetti (a volontà): fiamma sacra, luce, taumaturgia
1° livello (4 slot): cura ferite, dardo tracciante, santuario
2° livello (3 slot): arma spirituale, ristorare inferiore
3° livello (2 slot): dissolvi magie, guardiani spirituali
Azioni
Mazza. Attacco con Arma da Mischia: +2 a colpire, portata 1,5
m, un bersaglio.
Colpisce: 3 (1d6) danni da botta.
 
Selvaggi
Questi individui vivono ai margini della civiltà, a volte
entrandovi raramente in contatto. A disagio tra le mura
e nelle terre civilizzate, si trovano nel loro ambiente
quando possono muoversi tra le terre selvagge.
Berserker
Provenienti da terre selvagge, gli imprevedibili
berserker si radunano in compagnie di guerra e sono
sempre alla ricerca di conflitti in cui combattere.
Media umanoide (qualsiasi razza), qualsiasi allineamento
caotico
FORZA 16 (+3)
DESTREZZA 12 (+1)
COSTITUZIONE 17 (+3)
INTELLIGENZA 9 (-1)
SAGGEZZA 11 (+0)
Carisma 9 (-1)
Classe Armatura 13 (armatura di pelle)
\hspace*{0pt}\hfill{Punti Ferita}: 67 (9d8 + 27)
Velocità 9 m
Sensi Percezione passiva 10
Lingue una qualsiasi lingua (di solito il Comune)
Sfida 2 (450 PE)
Incauto. All’inizio del suo turno, il berserker può ottenere
vantaggio su tutti i tiri di attacco con armi da mischia effettuati
durante quel turno, ma i tiri di attacco contro di esso hanno
vantaggio fino all’inizio del suo prossimo turno.
Azioni
Ascia Grossa. Attacco con Arma da Mischia: +5 a colpire,
portata 1,5 m, un bersaglio.
Colpisce: 9 (1d12 + 3) danni taglienti.
Combattente Tribale
Sono i difensori delle tribù che vivono ai margini della
civiltà.
Media umanoide (qualsiasi razza), qualsiasi allineamento
FORZA 13 (+1)
DESTREZZA 11 (+0)
COSTITUZIONE 12 (+1)
INTELLIGENZA 8 (-1)
SAGGEZZA 11 (+0)
Carisma 8 (-1)
Classe Armatura 12 (armatura di pelle)
\hspace*{0pt}\hfill{Punti Ferita}: 11 (2d8 + 2)
Velocità 9 m
Sensi Percezione passiva 10
Lingue una qualsiasi lingua
Sfida 1/8 (25 PE)
Tattiche di Branco. Il combattente tribale ha vantaggio ai tiri di
attacco contro una creatura se almeno uno degli alleati del
picchiatore si trova entro 1,5 metri dalla creatura e quell’alleato
non è inabile.
Azioni
Lancia. Attacco con Arma da Mischia o a Gittata: +3 a colpire,
portata 1,5 m o gittata 6/18 m, un bersaglio.
Colpisce: 4 (1d6 + 1) danni perforanti, o 5 (1d8 + 1) danni
perforanti se usata con due mani per effettuare un attacco da
mischia.
Druido
I druidi proteggono il mondo naturale dai mostri e
dall’avanzare della civiltà. Alcuni sono sciamani tribali
che curano i malati, pregano agli spiriti animali e
forniscono consigli spirituali.
Media umanoide (qualsiasi razza), qualsiasi allineamento
FORZA 10 (+0)
DESTREZZA 12 (+1)
COSTITUZIONE 13 (+1)
INTELLIGENZA 12 (+1)
SAGGEZZA 15 (+2)
Carisma 11 (+0)
Classe Armatura 11 (16 con pelle di corteccia*)
\hspace*{0pt}\hfill{Punti Ferita}: 27 (5d8 + 5)
Velocità 9 m
Abilità Medicina +4, Natura +3, Percezione +4
Sensi Percezione passiva 14
Lingue Druidico più due altre lingue
Sfida 2 (450 PE)
Incantesimi. Il sacerdote è un incantatore di 4° livello. La sua
abilità da incantatore è la Saggezza (DC dei tiri salvezza degli
incantesimi 12, +4 al colpire con attacchi con incantesimo). Il
sacerdote ha preparato i seguenti incantesimi da druido:
Trucchetti (a volontà): arte druidica, bastone, produrre fiamma
1° livello (4 slot): intralciare, onda tonante, parlare con gli
animali, passo veloce
2° livello (3 slot): animale messaggero, pelle di corteccia
Azioni
Bastone da Combattimento. Attacco con Arma da Mischia: +2 a
colpire (+4 a colpire con bastone*), portata 1,5 m o gittata 6/18
m, un bersaglio.
Colpisce: 3 (1d6) danni da botta, o 6 (1d8 + 2) danni
contundenti con bastone o se impugnato con due mani.
Esploratore
Abili cacciatori e battitori di piste.
Media umanoide (qualsiasi razza), qualsiasi allineamento
FORZA 11 (+0)
DESTREZZA 14 (+2)
COSTITUZIONE 12 (+1)
INTELLIGENZA 11 (+0)
SAGGEZZA 13 (+1)
Carisma 11 (+0)
Classe Armatura 13 (armatura di cuoio)
\hspace*{0pt}\hfill{Punti Ferita}: 16 (3d8 + 3)
Velocità 9 m
Abilità Furtività +6, Natura +4, Percezione +5, Sopravvivenza
+5
Sensi Percezione passiva 15
Lingue una qualsiasi lingua (di solito Comune)
Sfida 1/2 (100 PE)
Olfatto e Vista Affinati. L’esploratore ha vantaggio nelle prove
di Saggezza (Percezione) basate su olfatto o vista.
Azioni
Multiattacco. L’esploratore effettua due attacchi da mischia o
due attacchi a gittata.
Spada Corta. Attacco con Arma da Mischia: +4 a colpire, portata
1,5 m, un bersaglio.
Colpisce: 5 (1d6 + 2) danni perforanti.
Arco Lungo. Attacco con Arma da Mischia: +4 a colpire, gittata
45/180 m, un bersaglio.
Colpisce: 6 (1d8 + 2) danni perforanti.
 
OPEN GAME LICENSE Version 1.0a
The following text is the property of Wizards of
the Coast, Inc. and is Copyright 2000 Wizards of
the Coast, Inc ("Wizards"). All Rights Reserved.
1. Definitions: (a)"Contributors" means the
copyright and/or trademark owners who have
contributed Open Game Content; (b)"Derivative
Material" means copyrighted material including
derivative works and translations (including into
other computer languages), potation,
modification, correction, addition, extension,
upgrade, improvement, compilation, abridgment
or other form in which an existing work may be
recast, transformed or adapted; (c) "Distribute"
means to reproduce, license, rent, lease, sell,
broadcast, publicly display, transmit or otherwise
distribute; (d)"Open Game Content" means the
game mechanic and includes the methods,
procedures, processes and routines to the extent
such content does not embody the Product
Identity and is an enhancement over the prior art
and any additional content clearly identified as
Open Game Content by the Contributor, and
means any work covered by this License,
including translations and derivative works under
copyright law, but specifically excludes Product
Identity. (e) "Product Identity" means product
and product line names, logos and identifying
marks including trade dress; artifacts; creatures
characters; stories, storylines, plots, thematic
elements, dialogue, incidents, language, artwork,
symbols, designs, depictions, likenesses, formats,
poses, concepts, themes and graphic,
photographic and other visual or audio
representations; names and descriptions of
characters, spells, enchantments, personalities,
teams, personas, likenesses and special abilities;
places, locations, environments, creatures,
equipment, magical or supernatural abilities or
effects, logos, symbols, or graphic designs; and
any other trademark or registered trademark
clearly identified as Product identity by the
owner of the Product Identity, and which
specifically excludes the Open Game Content; (f)
"Trademark" means the logos, names, mark, sign,
motto, designs that are used by a Contributor to
identify itself or its products or the associated
products contributed to the Open Game License
by the Contributor (g) "Use", "Used" or "Using"
means to use, Distribute, copy, edit, format,
modify, translate and otherwise create Derivative
Material of Open Game Content. (h) "You" or
"Your" means the licensee in terms of this
agreement.
2. The License: This License applies to any Open
Game Content that contains a notice indicating
that the Open Game Content may only be Used
under and in terms of this License. You must affix
such a notice to any Open Game Content that
you Use. No terms may be added to or
subtracted from this License except as described
by the License itself. No other terms or conditions
may be applied to any Open Game Content
distributed using this License.
3.Offer and Acceptance: By Using the Open Game
Content You indicate Your acceptance of the
terms of this License.
4. Grant and Consideration: In consideration for
agreeing to use this License, the Contributors
grant You a perpetual, worldwide, royalty-free,
non-exclusive license with the exact terms of this
License to Use, the Open Game Content.
5.Representation of Authority to Contribute: If
You are contributing original material as Open
Game Content, You represent that Your
Contributions are Your original creation and/or
You have sufficient rights to grant the rights
conveyed by this License.
6.Notice of License Copyright: You must update
the COPYRIGHT NOTICE portion of this License to
include the exact text of the COPYRIGHT NOTICE
of any Open Game Content You are copying,
modifying or distributing, and You must add the
title, the copyright date, and the copyright 
holder's name to the COPYRIGHT NOTICE of any
original Open Game Content you Distribute.
7. Use of Product Identity: You agree not to Use
any Product Identity, including as an indication as
to compatibility, except as expressly licensed in
another, independent Agreement with the owner
of each element of that Product Identity. You
agree not to indicate compatibility or coadaptability with any Trademark or Registered
Trademark in conjunction with a work containing
Open Game Content except as expressly licensed
in another, independent Agreement with the
owner of such Trademark or Registered
Trademark. The use of any Product Identity in
Open Game Content does not constitute a
challenge to the ownership of that Product
Identity. The owner of any Product Identity used
in Open Game Content shall retain all rights, title
and interest in and to that Product Identity.
8. Identification: If you distribute Open Game
Content You must clearly indicate which portions
of the work that you are distributing are Open
Game Content.
9. Updating the License: Wizards or its designated
Agents may publish updated versions of this
License. You may use any authorized version of
this License to copy, modify and distribute any
Open Game Content originally distributed under
any version of this License.
10 Copy of this License: You MUST include a copy
of this License with every copy of the Open Game
Content You Distribute.
11. Use of Contributor Credits: You may not
market or advertise the Open Game Content
using the name of any Contributor unless You
have written permission from the Contributor to
do so.
12 Inability to Comply: If it is impossible for You
to comply with any of the terms of this License
with respect to some or all of the Open Game
Content due to statute, judicial order, or
governmental regulation then You may not Use
any Open Game Material so affected.
13 Termination: This License will terminate
automatically if You fail to comply with all terms
herein and fail to cure such breach within 30 days
of becoming aware of the breach. All sublicenses
shall survive the termination of this License.
14 Reformation: If any provision of this License is
held to be unenforceable, such provision shall be
reformed only to the extent necessary to make it
enforceable.
15 COPYRIGHT NOTICE
Open Game License v 1.0 Copyright 2000,
Wizards of the Coast, Inc.
System Reference Document 5.0 Copyright 2016,
Wizards of the Coast, Inc.; Authors Mike Mearls,
Jeremy Crawford, Chris Perkins, Rodney Thompson,
Peter Lee, James Wyatt, Robert J. Schwalb, Bruce R.
Cordell, Chris Sims, and Steve Townshend, based on
original material by E. Gary Gygax And Dave
Arneson.
 3.02 Copyright 2018, Michele
Bonelli di Salci, Federico Lorenzo Gavrioli
VERSIONE 2.1
Aggiornato il glossario alla quinta edizione ufficiale
italiana.
VERSIONE 3.0
Aggiornata Appendice A e B
VERSIONE 3.01
Aggiunta la specifica “Orca” alla Balena Assassina
VERSIONE 3.02
Ulteriori aggiornamenti del glossario alla quinta
edizione ufficiale.

\end{multicols}