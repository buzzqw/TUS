\documentclass[a4paper,11pt,twoside,openany]{book}
%%\documentclass[a4paper,11pt,twoside,openany]{dndbook}
\usepackage{quoting}
%\usepackage[many]{tcolorbox}
\usepackage{tcolorbox}
\usepackage{tikz}
\usetikzlibrary{shadows}
\usepackage{multicol}
\usepackage{tocloft}
\usepackage{lmodern}
\usepackage{caption}
\usepackage[utf8]{inputenc}
\usepackage[T1]{fontenc}
\usepackage{setspace}
\usepackage[a4paper]{geometry}
\geometry{verbose,tmargin=2cm,bmargin=2.5cm,lmargin=1.5cm,rmargin=2cm}
\setcounter{secnumdepth}{-1}
\usepackage{booktabs}
\usepackage{url}
\usepackage[italian]{babel}
\usepackage{setspace}
\usepackage{graphicx}
\PassOptionsToPackage{normalem}{ulem}
\usepackage{ulem}
\usepackage{makeidx}
\usepackage{multirow}
\usepackage{titlesec}
\usepackage{textcomp}
\usepackage{background}
\usepackage[unicode=true,
bookmarks=true,
pdftitle={TUS - The Untitled System},pdfauthor={Andres Zanzani},
breaklinks=false,pdfborder={0 0 1},backref=section,colorlinks=false]
{hyperref}
\hypersetup{hidelinks,pdfcreator={LaTeX}}
\usepackage{bookmark}
%%\usepackage{DejaVuSans}
%%\usepackage{libertine}
\usepackage{palatino}
\usepackage{wrapfig}
\graphicspath{ {./img/} }
\usepackage{fancyhdr}
\usepackage{tcolorbox}
\tcbuselibrary{skins}
\tcbset{colback=brown!10, fonttitle=\scshape}
\usepackage{imakeidx}
\makeindex[columns=3, title=Indice, intoc=true]
\usepackage{lettrine}

\fancyhf{} % clear all header and footers
\renewcommand{\headrulewidth}{0pt} % remove the header rule
\fancyfoot[LE,RO]{\thepage} % Left side on Even pages; Right side on Odd pages
\pagestyle{fancy}
\fancypagestyle{plain}{%
	\fancyhf{}%
	\renewcommand{\headrulewidth}{0pt}%
	\fancyhf[lef,rof]{\thepage}%
}
\usepackage{tabularx}
\usepackage{longtable}
\usepackage{pdfpages}
\usepackage{hyperref}
%%\makeatletter
%%\makeatother
\raggedbottom



\usepackage{array}
\newcolumntype{L}[1]{>{\raggedright\let\newline\\\arraybackslash\hspace{0pt}}m{#1}}
\newcolumntype{k}[1]{>{\centering\let\newline\\\arraybackslash\hspace{0pt}}m{#1}}
\newcolumntype{R}[1]{>{\raggedleft\let\newline\\\arraybackslash\hspace{0pt}}m{#1}}
\newcolumntype{D}[1]{>{\centering}m{#1}}

\titleformat{\section}{\huge\bfseries}{\thesection}{1em}\textsc{}
\titleformat{\subsection}{\large\bfseries}{\thesubsection}{1em}\textsc{}
\titleformat{\subsubsection}{\normalsize\bfseries}{\thesubsubsection}{1em}\textsc{}

\setcounter{tocdepth}{3}

\newtcolorbox{note}{
	enhanced, % enable advanced settings
	left = 10mm, % pushes text away from the left edge by 10mm
	sharp corners, % disables rounded corners
	rounded corners = southeast, % "round" the bottom right corner
	arc is angular, % make the "round" corner an angle
	arc = 3mm, % controls corner cut
	boxrule=0.6pt, % sets box line thickness
	underlay={%
		%\path[fill=tcbcolback!80!black] ([yshift=3mm]interior.south east)--++(-0.4,-0.1)--++(0.1,-0.2); % triangle
		%\path[draw=tcbcolframe,shorten <=-0.05mm,shorten >=-0.05mm] ([yshift=3mm]interior.south east)--++(-0.4,-0.1)--++(0.1,-0.2); % triangle edge
		%\path[fill=gray!50!black,draw=none] (interior.south west) rectangle node[brown!10]{\Huge\bfseries ?!} ([xshift=8mm]interior.north west);
		
     	\path[fill=black] ([yshift=3mm]interior.south east)--++(-0.4,-0.1)--++(0.1,-0.2); % triangle
		\path[draw=black,shorten <=-0.05mm,shorten >=-0.05mm] ([yshift=3mm]interior.south east)--++(-0.4,-0.1)--++(0.1,-0.2); % triangle edge
		\path[fill=gray!50!black,draw=none] (interior.south west) rectangle node[brown!10]{\Huge\bfseries ?!} ([xshift=8mm]interior.north west);
		
	},
	drop fuzzy shadow % adds drop shadow
}

\newtcolorbox{example}{
	enhanced,
	title = Example,
	before upper={\parindent15pt\noindent} % add paragraph indentation
	
}

%\backgroundsetup{
%	scale=1,
%	color=black,
%	opacity=0.4,
%	angle=0,
%	contents={%
%		\includegraphics[width=\paperwidth,height=\paperheight]{paper.jpg}
%	}%
%}

\backgroundsetup{contents={}}

\begin{document}

\title{The Untitled System (TUS)\\Core Rules}
\date{\today\\\textbf{Gauss Edition} v3.1.0\\
\includegraphics[bb=0 0 1148 636,width=6.85139in,height=3.79514in]{copertina.png}}
\author{Andres Zanzani}
\maketitle
\thispagestyle{empty}

\newpage~\newpage~

%%\normalsize

%%\linespread{1.5}

Dedicato all'unica Donna mai amata, colei che ogni giorno mi accompagna nei sogni

Mai rinunciare ai tuoi sogni.
\thispagestyle{empty}

\newpage~\thispagestyle{empty}%%\newpage~\thispagestyle{empty}

\setcounter{page}{0}

\begin{multicols}{2}
	\tableofcontents{}
\end{multicols}

\pagebreak{}

\section{Introduzione}

\pagestyle{plain}

\begin{tcolorbox}[enhanced,arc=5pt,boxrule=0.3pt]{Si può scoprire di più su una persona in un'ora di gioco che in un anno di conversazione. (Platone)}\end{tcolorbox}\medskip

%	\begin{note}
%		Notes are displayed like this, and often convey important information to the %player or Narratore regarding a certain mechanic.
%	\end{note}
%	\begin{example}
%		Examples are displayed like this, and show a situation in which something is %used.
%	\end{example}

%\begin{tcolorbox}[title = Ranged Attack Example] 
%	Will, the Ranger is out hunting for his party. 
%	Will must roll $Longbow-6$ to succeed.
%\end{tcolorbox}


Per chi ha già esperienza con i giochi di ruolo: TUS è un rpg ispirato a D20 OGL che usa 3d6 al posto del d20 per una distribuzione normalizzata. E' un sistema classless dove la magia è freeform ed è ispirata a Ars Magica e Talislanta. Il sistema è basato su prove che possono esplodere (anche il danno) ed un sistema di svantaggi e vantaggi alla Gurps.

Per tutti gli altri...

TUS è un gioco di ruolo cooperativo e narrativo nel quale i giocatori creano personaggi che vivranno fantastiche e strabilianti avventure. Il Narratore si preoccuperà di dipanare la storia e fare partecipare i personaggi. Come in un gioco di narrazione ogni personaggio contribuirà attivamente alla storia con le sue scelte, le sue decisioni e le sue azioni.

Se sei un giocatore allora preparati a creare un personaggio che ti accompagnerà in terribili caverne e oscuri villaggi e fin sulle vette delle montagne più alte. Sarai tu a decidere tutto, dall'aspetto,al nome, alle sue capacità e ciò che possiede. Sarà un pirata rubacuori o un cavaliere timido.. un barbaro delle steppe o uno stregone ? Oro o Gloria ? Onore o Tirannia ? La scelta dipende sempre e solo da te.

Se sei il Narratore invece tu governi il mondo, la storia, l'avventura. Il tuo ruolo è di illustrare lo scenario in cui i giocatori si muovono e prendono decisioni. Li condurrai nelle profondità della terra alla ricerca del Tomo dimenticato di Atmos oppure a sfidare i grandi Draghi per la corona dell'Onniscienza?

Il tuo compito non è facile, usa fantasia, buon senso e la regola principale: divertititi. Quando sei in difficoltà non cercare la regola precisa, usa la tua più grande alleata: l'immaginazione, unisci un pizzico di assennatezza e cerca di stupire i giocatori. Lo scopo è sempre e solo uno, divertirsi insieme e crescere, come giocatori, come personaggi, come amici.

Oltre questo manuale avrai bisogno anche di un pò di dadi, i classici usati nei giochi di ruolo.
Chiamati solitamente d4, d6, d8, d10, d12, d20 stanno ad indicare un dado a 4 facce, il dado a 6 facce (di questi devi averne 3 o 4 almeno!), il dato a 8 facce, quello a 10 (solitamente vengono venduti in coppia, per poter ottenere il d100), il solitario dado a 12 facce e il sempiterno dado a 20 facce.
Ogni qual volta ti verrà chiesto di tirare un dado questo sarà scritto con la notazione XdZ, ovvero tiro X volte un dado con Z facce. Es. 4d6 indica di tirare 4 volte il dado a 6 facce.

Anche qualche miniatura potrebbe essere necessaria, altrimenti anche gli omaggi di merendine o degli ovetti al cioccolato possono essere sufficienti.

All'interno di questo manuale troverai tutto il necessario, come regole, per giocare, a te (voi) servirà fantasia, amicizia, dadi, qualche foglio di carta e divertimento (sorry, patatine e bibite non sono incluse nel manuale!)

Disegna ed usa una mappa ogni qual volta la descrizione e la situazione necessitano di una descrizione accurata ed un posizionamento precisa.

Crea e gioca il personaggio che più ti aggrada, che più senti tuo e ti fa divertire, non cercare le combinazioni di Abilità e capacità che ti danno più potere altrimenti prima o poi il personaggio ti verrà a noia.

Più giocherai, più il tuo personaggio guadagnerà esperienza ed anche tu lo interpreterai meglio. Il personaggio acquisirà oggetti meravigliosi, armature lucenti, armi volanti, oro, preziosi e gioielli e chissà cos'altro.

Il Narratore si preoccuperà di dirti quanta esperienza il tuo personaggio ha maturato, in base a come hai giocato, come hai collaborato nel gruppo, quanto hai aiutato il gruppo di giocatori a divertirsi. Ti terrà impegnato in scontri pericolosi, forse mortali, metterà a dura prova il tuo personaggio e come gruppo riuscirete, forse non sempre, a risolvere le intricate situazioni che il Narratore ha preparato. Ricorda che il Narratore ha sempre l'ultima parola in ogni discussione.

In questo libro troverai molte regole, eppure molte situazioni dovranno essere gestite utilizzando la prima regola, divertirsi. Buon senso, esperienza e fiducia nel Narratore risolveranno ogni situazione.

In questo manuale troverai anche un Mostruario pronto all'uso e l'indicazione per usare ben due sistemi di magia e due sistemi per gestire la morale.


Sia per la Magia che per la Morale un e' piu' complesso, ma libero e assoluto,  basato sul principio di declamare la magia (Essenze) in base a cio' che ci vuole fare e su cosa (Verbo-Nome) e su principio che ogni personaggio e' pennellato da piu' tratti caratteriali, l'altro, riconoscibile dalla scritta OGL nel titolo e' piu' conforme agli standard ruolistici classici con una divisione basato sulla matrice Bene/Neutralità/Male e Legge/Neutralità/Chaos.

I sistemi sono interscambiali e, con l'adeguata esperienza potete anche mescolare entrambi i sistemi nel party.

Come se non bastasse troverete anche due sistemi per gestire le divinita', anche qui la scritta OGL vi aiutera' a gestire le scelte.

Sia che tu decida di essere il Narratore sia che tu decida di interpretare un personaggio è necessario che tu legga con attenzione i capitoli che seguono.
è importante che tu abbia una buona conoscenza delle regole base e che, soprattutto, sappia dove cercare cosa quando ti servira'!

Buona lettura e buon divertimento!
\vspace*{\fill}

\begin{note}
Nota riguardo ai generi: il TUS il giocatore viene indicato come maschile. Non è una questione di sessismo, semplicemente in italiano la forma neutra (e quindi applicabile a giocatori e giocatrici) è maschile, le giocatrici sanno giocare meglio dei compagni maschi!
\end{note}

%\begin{wrapfigure}{l}{0.2\textwidth}
%	\centering
%	\includegraphics[width=4cm]{Dragon_by_Henry_Justice_Ford.jpg}
%\end{wrapfigure}


%\begin{figure}[h]\centering	\includegraphics[width=18cm]{Dragon_by_Henry_Justice_Ford.jpg}\end{figure}

\pagebreak{}

\subsection{Termini Comuni}

Ti elenco un pò di termini\index{termini comuni} che troverai ripetuti più volte nel libro.

\textbf{Arrotondamenti}: \index{Arrotondamenti}sempre per difetto se non esplicitato diversamente. Es. 7/2 = 3, 9/4=2.

\textbf{Abilita'}: \index{Abilita'}sono le capacità particolari che il personaggio ha imparato ad usare. Sono spesso al limite del magico, permettono azioni particolari ed anche di sovvertire le regole. Sono rare e si prendono ai passaggi di livello.

\textbf{Azione}: \index{Azione}è ciò che si fa in un intervallo di tempo. Ogni cosa che viene fatta dal personaggio si misura in Azioni. Combattere, lanciare Essenze, scassinare, bere pozioni, lo spostarsi... in ogni round si possono fare 3 Azioni.

\textbf{Bonus}: \index{Bonus}qualsiasi modifica dovuta a fattori esterni, ambientali, magici, di circostanza o che decida il Narratore è un bonus o malus da applicare al tiro di dado o difficoltà nella prova. Se ci sono più bonus dello stesso tipo si tiene solo il valore maggiore.

\textbf{Classe}: In TUS non ci sono classi. Ogni personaggio è costruito in base a ciò che sa fare. Quindi non troverete la parola Classe nel manuale.

\textbf{Check/Prova}: \index{Check}\index{Prova}un check (o prova) è il tiro di 3d6 più un punteggio indicato dalla Caratteristica e Competenza coinvolta.

\textbf{Prova di Concentrazione}:\index{Prova di Concentrazione} quando un incantatore vuole usare una Essenza ma è disturbato, attaccato, ferito o comunque distratto durante il lancio deve effettuare una prova di concentrazione per capire se riesce a lanciare la magia.
Se e' una distrazione allora la prova di magia questa deve riuscire di almeno 15.
Se il mago e' colpito prima di lanciare una Essenza la prova stessa deve riuscire almeno 10 + danno subito altrimenti l'incantesimo non riesce.

\textbf{Classe di Difficoltà (DC)}:\index{Classe di Difficolta'} \index{DC}indica quanto è difficile riuscire in una prova. può essere usato per verificare le competenze (nuotare..) come le conoscenze (veleni..). Indica nelle magie la difficoltà a resistere ad una magia. Indica a che valore arrivare per superare e riuscire nel nella prova.

\textbf{Competenza} \index{Competenza}(skill)\index{Skill}: la competenza indica il grado di conoscenza di una singola capacità. Possa essere lo studio di una lingua, l'arrampicarsi, il notare piccole cose.

\textbf{Competenza con le Armi (da mischia o distanza) (CA):} \index{Competenza con le Armi}\index{CA}è la tua capacità di saper colpire l'avversario con armi da mischia (spade, mazze..) o da tiro/distanza (pugnali da lancio, archi, balestre..)

\textbf{Competenza Magica (CM):} \index{Competenza Magica}\index{CM}è la tua capacità di attingere dalle Essenze, più è alto questo valore più le tue magie saranno efficaci.

\textbf{Creatura}\index{Creatura}: qualsiasi essere presente e partecipante nell'avventura viene identificata come creatura.

\textbf{Dadi Vita}\index{Dadi Vita}: per dadi vita si intendono i livelli di una creatura. Di base servono ad indicare quanti punti ferita e livello ha. Se non indicato una creatura ha 6 PF per Dado Vita.

\textbf{Difesa:} \index{Difesa}per Difesa si intende il valore totale ottenuto da 10 + Scudo + Armatura + Agilità + vari ed eventuali bonus.

\textbf{+1d6 oppure -1d6}: è un bonus o malus ad una prova. Aggiungi o sottrai un tiro di dado a 6 alla prova, oppure un +4/-4

\textbf{Distanza}:\index{Distanza} la distanza, per quando riguarda il combattimento è misurato in quadretti da 1 metro.

\textbf{Devoto}\index{Devoto}: un usufruitore di Essenze che si e’ legato ad un Patrono e’ ha almeno 3 tratti in comune.
Puo’ scegliere le Essenze del suo Patrono con i vantaggi e svantaggi che concedono. Vedi anche Seguace

\textbf{Esplosione del 6}:\index{Esplosione del 6} quando, esegui il Tiro per Colpire, Tiro Salvezza, prova di Magia (leggi le specifiche nel capitolo dedicato) o comunque ogni volta che viene indicato che vale l'esplosione del 6 significa che per ogni dado tirato che ha fatto 6 va segnato il risultato e ritirato il dado. Il risultato del nuovo tiro va anche lui sommato e se si fa un 6 nuovamente si somma e si continua a ritirare finché si continua a fare 6.

\textbf{Essenza:} \index{Essenza}indica un potere magico, una magia o incantesimo. Le Essenze si suddividono in varie tipologie che governano aspetti diversi della magia.

\textbf{Iniziativa}: \index{Iniziativa}è una prova di Agilità oppure Intelletto. Stabilisce l'ordine delle azioni in combattimento. Chi ha il punteggio più alto agisce per primo.

\textbf{Livello}:\index{Livello} il livello indica la competenza e potere raggiunto dal personaggio. Può indicare quando è forte il nemico/personaggio.

\textbf{LP}: \index{LP}Livello Potere indica a che forza si manifesta una Essenza, possa essere tramite un oggetto o tramite un potere magico di un mostro.

\textbf{Incantatore:} \index{Incantatore}indica un qualsiasi usufruitore di magia/Essenze a qualsiasi titolo.

\textbf{Mischia}: \index{Mischia}con mischia si intende il combattimento di contatto,spada a spada, ovvero quando il tuo personaggio combatte con una spada o comunque con un'arma che non è da tiro (arco/balestre..) contro un avversario.
Si considera in mischia qualsiasi creatura che il personaggio possa raggiungere con la sua arma non da tiro. Un nemico di grandi dimensioni (o con un arma lunga) potrebbe essere in mischia con il personaggio ma non viceversa.

\textbf{Movimento}: \index{Movimento}il movimento rappresenta la capacità di spostarsi. Una Azione di movimento rappresenta lo spostarsi del personaggio. Più è alto il valore di movimento più una creatura può muoversi lontano.

\textbf{Narratore:}\index{Narratore} il Narratore è la persona che conduce l'avventura, stabilisce le regole e controlla gli elementi della storia. Il dovere di ogni Narratore è fare divertire ed essere corretto.

\textbf{Patrono}:\index{Patrono} o divinità. Il Patrono è un essere superiore che può concedere poteri e garantire vantaggi. Il Patrono concede l'uso delle Essenze.

\textbf{Penalita'} \index{Penalita'}: come il bonus le penalità o malus sono valori, numeri, che indicano le circostanze sfavorevoli, magie penalizzanti o quant'altro renda più difficile la prova. Purtroppo a differenza dei Bonus le penalità, se non specificato diversamente, si sommano sempre fra loro.

\textbf{PG, Personaggio}: \index{Personaggio}è la creatura che viene guidata, gestita, ruolata dal giocatore.

\textbf{PNG}: \index{PNG}personaggio non giocante. Sono personaggi particolari, importanti o meno che il Narratore tiene per condurre l'avventura.

\textbf{Punti Esperienza}: \index{Punti Esperienz} \index{PX} ogni qual volta si risolvano difficoltà, mostri, indovinelli, si giochi bene il personaggio e ci si diverta si guadagna esperienza. Questi punti accumulati nel tempo stabiliscono il livello del personaggio.

\textbf{Punteggi caratteristica}: \index{Punteggi caratteristica} \index{Statistiche} abbreviati anche in caratteristica o statistiche. Ogni personaggio ha 5 Caratteristiche: Potenza (POT), Agilità (AGI), Intelletto (INT), Volontà (VOL) e Magnetismo (MAG). Più è alto il valore maggiore è la valenza o capacità del personaggio in quello specifico ambito.

\textbf{Punti Fato}:\index{Punti Fato} \index{Fortuna del Principiante}o Fortuna del Principiante sono dei punti a disposizione che il giocatore puo’ trasformare in d6 da aggiungere ai Tiri Salvezza o Tiri per Colpire o Tiri Competenze. Vengono chiamati Fortuna dei Principianti perche’ il loro numero e’ diminuisce all’aumentare di livello.

\textbf{Punti ferita (PF)}:\index{Punti ferita} \index{PF}indicano l’energia vitale della creatura. Finche’ la creatura ha 1 punto ferita combattera’ al suo meglio , senza problemi (certo.. potrebbe anche decidere di scappare piuttosto che morire..).
Ad ogni passaggio di livello si guadagna un certo numero di punti ferita, stabilito da certe regole. Ogni ferita si sottrae da questa somma di energie e quando si raggiungono gli 0 (zero) punti ferita si sviene, incapaci di agire. Se si viene ulteriormente feriti o comunque i propri punti ferita scendono fino 10+valore triplo della Potenza allora si e’ morti.

\textbf{Resistenza alla Magia (RM)}:\index{Resistenza alla Magia} \index{RM}Una creatura potrebbe avere una naturale resistenza alle Essenze, in ogni forma si presenti. Ogni qual volta la creature è influenzata direttamente da una Essenza deve effettuare una prova di RM, ovvero tirare 3d6 sommare il valore di RM e se è superiore alla prova di magia effettuata dall'incantatore l'Essenza non ha effetto.

\textbf{Riduzione del Danno (DR)}: \index{Riduzione del Danno} \index{DR} alcune creature hanno una resistenza innata al danno e ferite. Questa resistenza si denota come DR.
E’ solitamente indicata come Valore/Particolare, dove il valore indica quanto si e’ resistenti e il Particolare indica da che cosa si e’ danneggiati. Es “10/cold iron” indica che si ha una resistenza a tutti i danni di 10 tranne quelli causati da ferro freddo.
Se il particolare e’ indicato da un trattino “-” allora questa resistenza non e’ oltrepassabile e si applica a tutti i tipi di danno.

\textbf{Resistenza al Danno (RD)} \index{Resistenza al Danno}\index{RD} : una creatura potrebbe avere una resistenza ad una tipologia di danno. In questo caso si considera che dimezzi automaticamente il danno subito.
In caso di Essenza se il TS riesce il danno è ridotto a 0.

\textbf{Round}:\index{Round} il combattimento o azioni sono divise in round. Un round rappresenta una unità temporale di circa 6 secondi. Durante il round ogni creatura ha la possibilità di agire in base alla sua iniziativa ed eseguire fino a 3 Azioni.

\textbf{Sucesso Critico/Fallimento Critico}\index{Sucesso Critico} \index{Fallimento Critico}: nel caso in cui il giocatore passi la prova di un ampio magine otterrà benefici ( o malus). Controllate nelle competenze e nella magia.

\textbf{Tiro per colpire}:\index{Tiro per colpire} \index{TC}è una prova di CA (Competenza Armi) contro Difesa (armatura + scudo + abilità + magia...). Il Tiro per Colpire può essere da mischia (ovvero per le creature prossime alla tua arma, a distanza di mischia) oppure da distanza (per archi, balestre, ma anche pugnali..).. Leggi bene il capitolo del combattimento.

\textbf{Tiro Salvezza (TS)}:\index{Tiro Salvezza} \index{TS}quando una creatura è sottoposta ad un effetto particolare spesso viene concesso un Tiro Salvezza per mitigare o annullare gli effetti. Il Tiro Salvezza è un'azione che non occupa tempo o azioni.

I Tiro Salvezza riguardano i riflessi e lo schivare (Agilità), resistere a veleni/malattie o cambiamenti del corpo (Potenza) oppure per resistere ad attacchi mentali ed effetti che agiscano sull'arbitrio (Volonta').

\textbf{Tratto}: \index{Tratto}indica una componente del carattere. Ogni personaggio sceglie 4 tratti per comporre e costruire la sua personalità.

\textbf{Turno}: \index{Turno}sono 10 minuti, ovvero 100 round

\textbf{Uno porta male}: \index{Uno porta male}se tiri un 1 con il dato togli 1 dal risultato totale. Non per questo un 6 tirato diventa un 5, l’esplosione del 6 rimane.. solo che togli 1 al risultato finale. Detta diversamente 1 vale 0.


%%\begin{figure}[h]
%%	\centering
%%	\includegraphics[width=1cm]{dungeons-and-dragons-dungeons-dragons-d-d-dice-game-d20}
%%\end{figure}

\pagebreak

\section{Razze}\index{razze}
\begin{tcolorbox}[enhanced,arc=5pt,boxrule=0.3pt]{Il vero viaggio di scoperta non consiste nel trovare nuovi territori, ma nel possedere altri occhi, vedere l'universo attraverso gli occhi di un altro, di centinaia d'altri: di osservare il centinaio di universi che ciascuno di loro osserva, che ciascuno di loro e'. (Marcel Proust)}\end{tcolorbox}\medskip


\begin{tcolorbox}[enhanced,arc=5pt,boxrule=0.3pt]{Non è la specie più forte o la più intelligente a sopravvivere, ma quella che si adatta meglio al cambiamento. (Leon C. Megginson)}\end{tcolorbox}\medskip

\subsection{Umani}\index{Umani}

Gli uomini con il loro desiderio di scoperte, potere, gloria e violenza sono la razza dominatrice.

Le caratteristiche fisiche degli umani sono varie quanto i climi del mondo. Il colore della pelle, l'abbigliamento, le tradizioni culturali ed alimentari, gli stili di vita possono essere i piu' disparati ed originali e tutto rende solo piu' umano il personaggio.

Lasciate fuori il razzismo da Yeru, ci sono gia' abbastanza guerre per crearne di nuove solo perche' quelli usano le asce al posto delle spade.

Gli umani sono stati la razza creata da Ljust e Calicante insieme perché con la loro spinta caotica, mutevole e vitale potessero fare e disfare ricominciando continuamente da capo e migliorando di continuo.

\textbf{Modificatori razziali:} +1 ad una caratteristica a piacere

\textbf{Caratteristiche fisiche}: altezza 150-185 cm, 50-130 kg, aspettativa di vita 70 anni

\textbf{Dimensioni:} Medie

\textbf{Velocita'}: 9m

\textbf{Linguaggi}: Comune

\textbf{Vantaggio}: +1 abilità al primo livello

\subsection{Elfi}\index{Elfi}

\label{elfi}

Gli elfi sono la razza creata direttamente da Ljust perché guidasse il mondo con l'eleganza, l'intelligenza e la lungimiranza di una razza immortale.

Dopo millenni di pace e vita nell'intero mondo, dopo che bellezze naturali ed architettoniche si erano diffuse in armonia il mondo, la creazione delle nuove razze e la loro spinta espansionistica hanno portato gli elfi a diventare insofferenti, ad essere infastiditi dagli altri.
Sono diventati progressivamente xenofobi ed hanno incominciato a stravolgere l'impianto originale del loro mandato.

Se erano stati creati come guida etica, morale e culturale di tutto il creato adesso molte fazioni vedono come necessaria una pulizia etnica per portare a compimento la purezza originaria del piano divino.

Molti hanno preso a conquistare, soggiogare e sterminare le razze inferiori.. qualsiasi creatura che non sia elfica, un una spirale di violenza ed espansione senza eguali.

Altri hanno preso a ritirarsi sempre più lontano, sempre più all'interno del loro regno, rimanendo custodi solitari della purezza del creato.

Gli elfi hanno rappresentato l'idea originale del creato e questo spesso li ha portati ad essere più affini agli gli dei originari e con Kyriel che con le successive divinità.

Gli elfi sono generalmente più alti e snelli degli umani. Gli occhi sono sempre grigi, con riflessi metallici, le gambe agili.

Gli elfi apprezzano la parola scritta e la magia. Sono una razza istintiva, guidata da mente acuta e sensi eccellenti, da passione e amore per le scoperte e conoscenze.

Gli piace la ricerca magica ma spesso questa è più frutto di ispirazione che scrupolosi studi. La magia è funzionale ed artistica fondendosi in una vera arte simile al canto ed alla poesia, al ballo se non alla pittura.
La magia è arte e amore per il mondo elfico.


\textbf{Modificatori razziali:} +1 Intelletto, +1 Agilità, -1 Potenza

\textbf{Caratteristiche fisiche}: altezza 165-195 cm, 50-110 kg, aspettativa
di vita 1000+ anni

\textbf{Dimensioni:} Medie

\textbf{Velocita'}: 9m

\textbf{Linguaggi}: Elfico

\textbf{Vantaggio} visione crepuscolare di 18 metri

\subsection{Nani}\index{Nani}

\label{nani}

I nani sono una razza stoica e severa abituata al comunismo più puro, senza un vero concetto di proprietà ma di pura comunanza di beni secondo l'idea che ogni nano lavora per la comunità e non per se stesso.

I nani sono una razza bassa e piazzata, raggiungono un'altezza massima di circa 140 cm con una corporatura robusta e compatta che dà loro un aspetto massiccio. Sia i maschi che le femmine portano orgogliosamente i capelli lunghi e gli uomini decorano spesso le barbe con vari generi di fermagli e trecce intricate, altresì vero che nani pelati sono frequenti, ma non senza barba. Le donne nane non hanno barba ne peluria in eccesso. Il sesso è libero e socialista.

I nani sono guidati da onore e tradizione e comunismo. Sono spesso visti come burberi, ma hanno un forte sentimento di amicizia e giustizia e rispetto per chi lavora sodo e si impegna per la comunità.

I nani sono la razza creata da Erondil con l'aiuto di Atmos.

Giudicano gli Elfi con severità perché non hanno saputo portare a termine il dettato della Creazione e quindi si sentono il compito, l'onere e l'onore di forgiare il creato e nel creato la bellezza e la maestosità di Erondil.

\textbf{Modificatori razziali:} +1 Potenza, +1 Volontà, -1 Magnetismo

\textbf{Caratteristiche fisiche}: altezza 100-140 cm, 45-90 kg, aspettativa
di vita 450 anni

\textbf{Dimensioni:} Medie

\textbf{Velocita'}: 6m

\textbf{Linguaggi}: Nanico

\textbf{Speciale:} Professione: Architetto o Fabbro ha un +1

\textbf{Vantaggio}: scurovisione di 18 metri

\textbf{Svantaggio:} Pessimo carattere

\subsection{Mezzelfo}\index{Mezzelfo}

\label{mezzelfo}

Per un elfo non c'è nulla di più impuro di un mezz'elfo. Nessun mezz'elfo nasce per volontà di un Elfo. Ogni mezz'elfo è figlio di violenza. Questo è almeno quello che continuano a dire gli elfi.

Ci sono anche rari mezz'elfi nati da rapporti romantici. Benché solitamente di breve durata, anche per gli standard umani, questi incontri segreti portano di solito alla nascita dei mezzelfi, una razza che discende da due culture, ma non è erede di nessuna. I mezzelfi possono riprodursi tra loro, ma persino questi mezzelfi "di sangue puro" sono visti come bastardi dagli elfi.

I mezzelfi sono più alti degli umani ma più bassi degli elfi. Ereditano la corporatura slanciata e i lineamenti attraenti del loro lignaggio elfico, ma il colore della loro pelle è normalmente dettato dalla loro parte umana. I loro occhi tendono ad essere simili a quelli degli umani nella forma, ma presentano un'esotica gamma di colori dall'ambra al viola fino al verde smeraldo e al blu scuro, sempre con riflessi metallici.

I mezzelfi comprendono la solitudine e sanno che il carattere spesso è più un prodotto dell'esperienza di vita che della razza di appartenenza.

\textbf{Modificatori razziali:} +1 ad una Caratteristica a propria
scelta

\textbf{Caratteristiche fisiche}: altezza 160-185 cm, 50-100 kg, aspettativa di vita 210 anni

\textbf{Dimensioni:} Medie

\textbf{Velocita'}: 9m

\textbf{Linguaggi}: Comune oppure Elfico

\textbf{Vantaggio}: visione crepuscolare di 18 metri

\index{Mezzorco}

\subsection{Mezzorco}

\label{mezzorco}

Agli occhi delle razze civilizzate, i mezzorchi sono delle mostruosità, il risultato di perversione e violenza e raramente sono il risultato di unioni amorose, come tali solitamente sono costretti a crescere velocemente e duramente, lottando continuamente per proteggersi o farsi un nome. Alcuni mezzorchi trascorrono le loro intere vite a dimostrare agli orchi purosangue che sono feroci quanto loro.

I mezzorchi sono alti in media 1.9 metri, con fisico potente e pelle verdastra o grigia. I loro canini crescono spesso piuttosto lunghi fino a sporgere dalle loro bocche e queste "zanne", unite ad una fronte ampia e le orecchie un pò a punta, danno loro quel noto aspetto "bestiale". A dispetto di questi ovvi tratti orcheschi, i mezzorchi sono tanto variegati quanto i loro genitori umani.

Se all'interno delle tribù orchesche devono guadagnarsi continuamente il rispetto dei "purosangue", nella società umana non va meglio. Derisi, sbeffeggiati, esclusi ed abbandonati i mezzorchi spesso trovano rifugio nella criminalità.

Gli orchi sono stati creati direttamente da Cattalm con l'aiuto di Calicante. Molto della tendenza caotica e distruttrice del loro creatore rimane nella natura dei mezzorchi.

I mezzorchi sono spesso vittime di pregiudizi.

\textbf{Modificatori razziali:} +2 Potenza -1 Magnetismo

\textbf{Caratteristiche fisiche}: altezza 160-210 cm, 60 - 140 kg,
aspettativa di vita 70 anni

\textbf{Dimensioni:} Medie

\textbf{Velocita'}: 9m

\textbf{Linguaggi}: Comune oppure Orchesco

\textbf{Vantaggio} visione crepuscolare di 18 metri

\textbf{Svantaggio:} Seguire il Chaos

\subsection{Drow}\index{Drow}

\label{drow}

La genesi e storia dei Drow si divide un due grandi tronconi storici, strettamente legati alla storia Elfica.

In principio Shayalia, gelosa degli Elfi plasmo i Drow a loro immagine e poi li rese cupi, oscuri, freddi come la notte perché fossero l'ombra nera degli elfi.

Conosciuti anche come elfi scuri, dimorano nelle profondità del sottosuolo in complesse città plasmate nella roccia dalle Essenze.

I drow hanno una fisicità simile a quella degli uomini, mediamente sono più bassi degli elfi ma condividono con loro i lineamenti e lo slancio, comprese le caratteristiche lunghe orecchie a punta. Il colore della pelle dei drow varia dal nero carbone al viola scuro. I loro capelli sono solitamente bianchi o argentei, sebbene non siano insolite altre varianti.

La società drow è per tradizione matriarcale e suddivisa in classi. I maschi drow solitamente adempiono ai ruoli militari, difendendo la famiglia dai pericoli esterni, mentre le femmine drow assumono ruoli di comando e autorità.

A rafforzare questi ruoli, un drow ogni cinquanta nasce con capacità eccezionali e viene quindi considerato un nobile, e la maggioranza di questi drow speciali sono femmine. Le casate nobili determinano la politica drow, e ciascuna di esse è governata da una nobile matriarca e composta di famiglie di rango inferiore, imprese commerciali e compagnie militari.

I drow sono fortemente motivati dall'interesse e dalla crescita personale, che plasmano la loro cultura con ribollenti intrighi e conflitti politici, mentre i drow comuni fanno del loro meglio per ottenere il favore della nobiltà, e quest'ultima si eleva al potere per mezzo di una miscela di omicidi, seduzioni e tradimenti.

I drow hanno un forte senso di superiorità razziale e suddividevano le altre razze in due gruppi distinti: gli schiavi e coloro che non sono ancora schiavi.

L'odio dei drow verso gli elfi separa questi esseri da tutte le altre razze, e gli elfi scuri non desiderano nulla di più al mondo che distruggere tutto quello che ha a che vedere con i loro cugini di superficie.

I drow danno grande importanza al potere e alla sopravvivenza e non provano alcun rimorso a causa delle scelte spregevoli che potrebbero essere costretti a fare per assicurare la propria sopravvivenza. Non sanno cosa farsene della compassione e sono spietati nei confronti dei loro nemici, antichi o attuali che siano.

Poi gli Elfi diventarono più reclusi, indifferenti, xenofobi e nazisti, pari passo che Shayalia riusciva a manipolare la creazione di Ljust. E mentre Shayalia era distratta con gli Elfi, Sumkjr si faceva largo nei drow. Una volta conosciuti come l'anima nera del mondo adesso sono tra i maggiori portatori di speranza, vita, saggezza e cultura.

Riuscendo ad adattare il proprio regime sociale le matriarche Drow sono diventate le filantrope che si interessano dei poveri, degli emarginati, degli svantaggiati, dell'ambiente e cultura promuovendo una nuova consapevolezza universale.

Chiaro che, purtroppo, non tutti hanno accettato questa conversione e si possono trovare, a pari degli Elfi, soggetti che perseguono le vecchie abitudini.

\textbf{Modificatori razziali:} +1 Agilità, +1 Intelletto, -1 Potenza

\textbf{Caratteristiche fisiche}: altezza 140-170 cm, 40 - 100 kg,
aspettativa di vita 1000+ anni

\textbf{Dimensioni:} Medie

\textbf{Velocita'}: 9m

\textbf{Linguaggi}: Drow

\textbf{Vantaggio}: scurovisione di 36 metri

\textbf{Svantaggio:} \sout{razzisti}

\subsection{Nibali}\index{Nibali}

\label{nibali}

I Nibali sono una razza creata magicamente per essere schiava ai grandi maghi del nord.

La leggenda dice che i terribili maghi del nord, partendo da una coppia di umani (dopo che a migliaia erano morti atrocemente nei precedenti esperimenti) riuscì a creare manipolando con la magia, un razza più robusta, più forte, più intelligente ed allo stesso tempo più docile e disciplinata con pregio che ogni figlio generato sarebbe stato assolutamente identico fisicamente al padre o alla madre.

Queste cose accadevano ormai più di 2000 anni or sono ed il regno del male eterno crollò sotto la sua stessa incapacità di evolversi e percepire i nuovi problemi.

I Nibali hanno continuato a prosperare ed usufruendo di quanto il regno del ghiaccio gli aveva lasciato hanno creato una tra le civiltà più moderne, democratiche e civili del mondo.

Per molti l'estrema efficienza e dedizione dei Nibali è odiosa, un giogo che non lascia spazio alle libertà personali, per i Nibali è solo un modo efficiente di progredire.

Tutti i Nibali sono uguali tra loro a parità di sesso ma il fatto che non possano avere figli con altre razze non li rende un popolo chiuso o razzista, anzi l'assorbire il meglio di ogni cultura li rende migliori ed anche ottimi diplomatici.

Ciò che veramente distingue un Nibali da un altro è l'acconciatura, i tatuaggi, il vestiario... L'estrema libertà personale, legata indissolubilmente alla libertà di gruppo, permette ad un nibali di esprimersi come meglio crede nell'aspetto esteriore.

\textbf{Modificatori razziali:} +1 Potenza, +1 Intelletto, - 1 Volonta'

\textbf{Caratteristiche fisiche}: altezza 183cm maschi, 172 cm femmine, 50 - 120 kg, aspettativa di vita 130 anni

\textbf{Dimensioni:} Medie

\textbf{Velocita'}: 9m

\textbf{Linguaggi}: Comune

\textbf{Svantaggio}: Seguire la Legge

\subsection{Diversi}\index{Diversi}

\label{diversi}

Benedetti o maledetti i Diversi non sono come noi. Non sono gli amici che ti aspetti. Un Diverso è frutto di una unione non voluta. Se i Patroni non possono agire direttamente nel mondo, o almeno questo è quello che cerca di evitare Gradh, sovente invece usano i loro poteri per creare una stirpe a loro fedele.

Un Diverso è fedele al suo Patrono e non può fare diversamente. Per fortuna sono sterili con gli umani, altrimenti avrebbero già dominato il mondo.

Un Diverso è più forte e più intelligente e può meglio nell'oscurita. Purtroppo per loro la loro vità frenetica è segnata da una breve durata. Solitamente un Diverso non supera i 40 anni di vita.

Un Diverso è segnato, da qualche parte sul suo corpo c'è il simbolo del suo Patrono.

\textbf{Modificatori razziali:} +1 Potenza, +1 Intelletto

\textbf{Caratteristiche fisiche}: altezza 155-185 cm, 50-110 kg, aspettativa di vita 45 anni (40+1d10 anni)

\textbf{Dimensioni:} Medie

\textbf{Velocita'}: 9m

\textbf{Linguaggi}: Comune

\textbf{Speciale} visione crepuscolare di 18 metri, deve individuare un Patrono ed avere almeno 3 tratti comuni.

\subsection{Altri}\index{Altri}

\label{altri}

in un mondo dominato dal chaos chi ha provato a scappare nell'oscurità delle caverne e della notte ha subito la punizione di Gradh per non aver tentato di migliorare il mondo.

Questi esseri insolitamente gracili hanno una forte intelligenza e agilità, la loro carnagione è diventata chiara, quasi madreperlacea. Ormai sono passati duemila anni da quando il primo Altro nacque e a seguito ogni madre per generazioni partorì solo Altri, finché non ci fu nessun umano, finché tutti ebbero pagato il peccato di non volere migliorare il mondo.

La maggior parte degli Altri si è votata a Calicante ed ai Patroni Oscuri, alle arti magiche più malvagie e corruttive. Pochi, reietti, sentono la colpa e abbracciano la Luce e vengono in superficie.

Un Altro può essere riconosciuto da una voglia naturale, come un tatuaggio, che disegna tre anelli dorati attorno al polso.

Trattati come mostri o malvagi senza neanche una domanda, un Altro non ha mai la vita facile, per fortuna la loro naturale agilità e la capacità innata di vedere nell'oscurità gli permette di vivere, anche se spesso solo di notte, lontano dalle luci e dagli affetti che vorrebbero provare.

\textbf{Modificatori razziali:} +1 Intelletto, +2 Agilità , -2 Potenza

\textbf{Caratteristiche fisiche}: altezza 155-185 cm, 50-110 kg, aspettativa
di vita 100 anni

\textbf{Dimensioni:} Medie

\textbf{Velocita'}: 12m

\textbf{Linguaggi}: Comune

\textbf{Vantaggio}: scurovisione di 36 metri

\bigskip

\textbf{Nota sulle Razze}\index{Razze}\index{Razza}: Nessuna descrizione di una razza potra' mai imbrigliare e sottomettere un personaggio. Ogni giocatore e' libero di creare il personaggio della razza preferita (concessa dal Narratore) e descriverlo, inquadrarlo, sentirlo, renderlo vivo come piu' gli piace.
Non limitatevi alle descrizioni qui proposte, vogliono essere solo spunti, non sentitevi limitati nelle scelte perche' la vostra razza dice questo o quello.
Fate nascere i piu' belli ed completi personaggi possibili.
Ogni personaggio e' vivo ed e' una persona e come tale sara' sempre diverso l'uno dall'altro, ognuno fantastico in maniera diversa.

\textbf{Nota sugli Svantaggi}: il giocatore, in accordo con il Narratore, può scegliere uno svantaggio diverso da quello indicato purché sia coerente con la storia del personaggio.

\pagebreak

\section{Caratteristiche Speciali}

\label{caratteristiche-speciali}

\begin{tcolorbox}[enhanced,arc=5pt,boxrule=0.3pt]{Non basta avere gli occhi per vedere (anonimo)}\end{tcolorbox}\medskip


\subsection{Visione Crepuscolare}\index{Visione Crepuscolare}

Quello che per molti e’ oscurità per chi ha visione crepuscolare e’ vedere bene purche’ ci sia una fonte minima di luce.

La visione crepuscolare è una visione a colori.
Un incantatore dotato di visione crepuscolare può leggere una Pergamena fino a quando ha accanto come fonte di luce anche la più smorta delle candele.

I personaggi dotati di visione crepuscolare possono vedere all’esterno nelle notti illuminate dalla luna come se si trovassero alla luce del giorno.

Nella assoluta mancanza di luce la visione crepuscolare non aiuta, rimane buio pesto impenetrabile.

\subsection{Scurovisione}\index{Scurovisione}

La Scurovisione è la capacità straordinaria di vedere senza fonti di luce, fino ad una distanza massima indicata per ogni creatura. 

La Scurovisione è solo in bianco e nero (non consente al personaggio di distinguere i colori). Non permette ai personaggi di vedere nulla che non possano altrimenti vedere: gli oggetti Invisibili sono ancora Invisibili, e le Illusioni sono ancora visibili per quello che sembrano essere.

Alla stessa maniera, la Scurovisione rende una creatura soggetta agli attacchi con lo sguardo normalmente. La presenza di luce non altera la Scurovisione.

\subsection{Fiuto}\index{Fiuto}

Questa qualità speciale permette ad una creatura di sfruttare l'olfatto per individuare i nemici nascosti o in avvicinamento e di seguire le tracce. Le creature dotate di fiuto possono identificare con l'olfatto gli odori familiari come gli umani fanno con quello che vedono.

La creatura può individuare le creature entro 6 metri di distanza con l'olfatto. Se l'avversario è sottovento, il raggio aumenta a 18 metri; se è sopravento, il raggio diminuisce a distanza di mischia.
Gli odori più forti, come il fumo, spazzatura o corpi in decomposizione, possono essere individuati al doppio del raggio sopra indicato.

Quando una creatura individua un odore, non viene rivelata l'esatta posizione della sua fonte, ma solo la sua presenza entro il raggio d'azione. La creatura può utilizzare un'Azione per individuare la direzione da cui proviene l'odore. Quando si trova a distanza di mischia dalla fonte, ne individua la posizione.

Una creatura dotata di fiuto può seguire tracce utilizzando l'olfatto, effettuando una prova di Sopravvivenza per trovare e seguire una traccia. La tipica DC di una traccia fresca è 10 (a prescindere dalla superficie su cui si trova la traccia). La DC aumenta o diminuisce a seconda dell'intensità della traccia, del numero di creature che la lasciano e del tempo trascorso da quando è stata lasciata. Per ogni ora trascorsa la DC aumenta di 2.

Per il resto, questa capacità segue le regole dell'abilità Sopravvivenza. Le creature che seguono tracce con il fiuto ignorano gli effetti delle superfici su cui si trova la traccia e della scarsa visibilità.

Una creatura con la capacità Fiuto identifica gli odori familiari così come un umano potrebbe identificare un luogo familiare. L'acqua, e in particolare l'acqua corrente, nega la capacità di seguire tracce delle creature.

Alcuni forti odori possono facilmente mascherarne altri. La presenza di un odore simile rende impossibile individuare o identificare esattamente una creatura mediante il Fiuto; la DC base dell'abilità Sopravvivenza per seguire tracce in presenza di odori coprenti passa da 10 a 20.


\subsection{Vista Cieca (aka “Daredevil”)}\index{Vista Cieca}

Utilizzando sensi diversi dalla vista, come la percezione delle vibrazioni, un fiuto sensibile, un udito acuto o un sonar, una creatura dotata di vista cieca si muove e combatte bene quanto una creatura dotata della vista.
Utilizzando sensi diversi dalla vista, come la percezione delle vibrazioni, un fiuto sensibile, un udito acuto o un sonar, una creatura dotata di vista cieca si muove e combatte bene quanto una creatura dotata della vista.

Invisibilità, buio e la maggior parte delle forme di copertura sono inutili, anche se la creatura dotata di vista cieca deve avere una linea di effetto per notare una determinata creatura o oggetto.

Il raggio della capacità è indicato nella descrizione della creatura. La creatura, in genere, non deve effettuare prove di Consapevolezza per notare creature entro il raggio della sua vista cieca.

Il raggio della capacità è indicato nella descrizione della creatura. La creatura, in genere, non deve effettuare prove di Consapevolezza per notare creature entro il raggio della sua vista cieca.

A meno che non sia diversamente indicato, la vista cieca è sempre attiva e la creatura non deve compiere azioni per attivarla. Alcune forme di vista cieca devono essere attivate come azione immediata. In questo caso, viene indicato nella descrizione della creatura.

Se una creatura deve attivare la vista cieca, ne ottiene i benefici solo durante il proprio turno.

\subsection{Tremorsense}\index{Tremorsense}
Una creatura dotata di Tremorsense è sensibile alle vibrazioni del suolo, e può automaticamente individuare qualsiasi cosa sia in contatto con il terreno entro il raggio specificato dal tremorsense.

Le Creature Acquatiche dotate di tremorsense (ecolocalizzazione) possono percepire la posizione di creature in contatto con l’acqua.

Il raggio della capacità è specificato nel testo descrittivo della creatura

\pagebreak

\section{Le Caratteristiche}\index{Caratteristiche}

\label{le-caratteristiche}

\begin{tcolorbox}[enhanced,arc=5pt,boxrule=0.3pt]{Vivere non è respirare: è agire, è fare uso degli organi, dei sensi, delle facoltà, di tutte quelle parti di noi stessi per cui abbiamo il sentimento di esistere. (Jean-Jacques Rousseau)}\end{tcolorbox}\medskip

Ogni personaggio ha 5 caratteristiche (chiamate anche statistiche) che rappresentano i suoi attributi base e costituiscono il suo potenziale talento e capacità innata.

Anche se raramente un personaggio effettua una prova usando soltanto una sua Caratteristica, i punteggi di Caratteristica influiscono praticamente su ogni aspetto delle capacità e delle abilità del personaggio.

Le 5 caratteristiche sono:

\textbf{Potenza}\index{Potenza}: indica la forza fisica ma anche la resistenza agli sforzi del personaggio, Un personaggio con un punteggio di Potenza pari a -5 è morto.

\textbf{Agilità}\index{Agilita}: indica la capacità di coordinamento, riflessi ed agilità del personaggio, Un personaggio con un punteggio di Agilità pari a -5 è incapace di muoversi ed è completamente immobile (ma non privo di sensi).

\textbf{Intelletto}\index{Intelletto}: indica la componente razionale, logica, cognitiva del personaggio. Un personaggio con un punteggio di Intelletto pari a -5 è in stato di coma.

\textbf{Volonta'}\index{Volonta'}: indica la forza di volontà, il buon senso, la perspicacia e l'intuito del personaggio. Un personaggio con un punteggio di Volontà pari a -5 è incapace di pensiero razionale ed è privo di sensi.

\textbf{Magnetismo}\index{Magnetismo}: misura la forza della personalità, la capacità di persuasione, il magnetismo personale, la predisposizione al comando e il fascino di un personaggio. Un personaggio con un punteggio di Magnetismo pari a -5 è privo di sensi.

\smallskip

Ogni punteggio di Caratteristica in genere va da 0 a 3, anche se i bonus e le penalità razziali possano alterarli; un punteggio di Caratteristica buona è 1, 2 ottima, 0 è "normale", 3 è giudicato "eccezionale".

Un punteggio di -1 e giudicato debole, un -2 subnormale, un -3 severamente problematico, un -4 porta quasi ad un non utilizzo della caratteristica, un -5 è opportuno che stia nel letto e basta (se non è già in una bara).

Ogni giocatore distribuisce 6 punti tra le 5 Caratteristiche, ogni Caratteristica deve avere come minimo un punteggio di -1 e come massimo 2 prima dei modificatori razziali.

Ogni quattro livelli (4, 8, 12, 16, 20..) si può aumentare di un punto una caratteristica, fino a raggiungere un massimo di valore 5. Per aumentare oltre 5 sono necessario oggetti magici o Essenze.

Il punteggio delle caratteristiche non e' tutto in un personaggio ne tanto meno in un mostro.\\
I mostri piu' "istintivi" ed aggressivi avranno sicuramente punteggi negativi di Intelletto e Magnetismo, ma non per questo sono "stupidi", semplicemente agiscono in base ai loro schemi naturali.

\section{Punti Ferita}\index{Punti Ferita}

\begin{tcolorbox}[enhanced,arc=5pt,boxrule=0.3pt]{Chi non stima la vita, non la merita. (Leonardo da Vinci)}\end{tcolorbox}\medskip


Punti ferita rappresentano l’energia vitale del personaggio e finche’ il personaggio/avversario ha almeno 1 punto ferita combattera’ e lottera’ al meglio delle sue capacita'’.

Ogni personaggio parte con 4 punti ferita al primo livello + il punteggio della Potenza.
Ad ogni livello, oltre il primo, guadagna 1d4 Punti Ferita + il punteggio della Potenza.

Ogni punto preso in Competenza Armi aumenta i punti ferita presi di 3. Ulteriori abilita’ possono alzare questo punteggio.

Segna nella scheda i PF (Punti Ferita) totali che hai e indica il valore attuale di volta in volta che per vari motivi di gioco ne perdi o riprendi.

Segna sulla scheda sempre qual e’ il totale di punti ferita attuale, dopo ogni colpo o danno.
I punti ferita si recuperano in diversi modi:

\begin{itemize}
	\item
	      per ogni notte di riposo (almeno 8 ore) il proprio valore di Potenza + CA (con un minimo di 1)
	\item
	      Tramite l'Essenza della Cura (magie , pozioni.. o altri oggetti magici)
	\item
	      Competenza Sopravvivenza (Guarire), tramite trattamenti più o meno lunghi
\end{itemize}

I Punti Ferita possono essere anche temporanei ovvero aggiunti temporaneamente ai tuoi attuali. I PF Temporanei vanno tolti per primi quando si viene feriti.
Se non esplicitato diversamente i punti ferita temporanei scompaiono dopo un ora da quando si sono acquisiti.

\section{Punti Fato (Fortuna del Principiante)}\index{Punti Fato}
\begin{tcolorbox}[enhanced,arc=5pt,boxrule=0.3pt]{Se il destino è contro di noi, peggio per lui. (motto del 1º Reggimento Carabinieri Paracadutisti "Tuscania")}\end{tcolorbox}\medskip

In un mondo non facile la Fortuna del Principiante aiuta’ chi non ha esperienza.
Ogni personaggio ha un numero di Punti Fato pari a (20 - Livello)/4, con un minimo di 1. I Punti Fato si conteggiano per sessione di gioco.

Ad ogni sessione si azzerano e si ricalcolano, ne consegue che non si accumulano Punti Fato tra una sessione di gioco e l’altra.

Esempio di calcolo:
Un personaggio di livello 5 ha: 20-5 = 15/4 = 4 (arrotondi per eccesso) Punti Fato da usare nella sessione se non li userai tutti non potrai cumularli per la sessione successiva.

Un Punto Fato si usa come azione di reazione ed ogni Punto Fato utilizzato concede un bonus di +1d6 alla prova in corso.

Il Punto Fato puo'’ essere utilizzato per avere un dado in più nel Tiro Salvezza oppure nei Tiri per Colpire, con potenziale esplosione del dado, oppure una prova di competenza o per aumentare la propria Difesa per quel round.

I Punti Fato si devono dichiarare prima del tiro o dopo il tiro ma prima di sapere se la prova (Tiro Salvezza, Tiro per Colpire, prova... propria o dell’avversario) ha avuto successo.

Una volta dichiarato l’ammontare di Punti Fato che si vogliono utilizzare non e’ possibile utilizzarne di più o di meno.

\pagebreak

\section{L'Etica}\index{Etica}

In questo capitolo troverete i due approcci di TUS alla gestione dell'Etica o piu' comunemente chiamato Allineamento.\\

Vi invito a leggere la gestione tramite Tratti e Allineamento (OGL) e scegliere quella che meglio si addice all'avventura che andrete a fare, quella che meglio si adatta al vostro stile di gioco.\\

E' una scelta importante, parlatene tutti insieme, Narratore e Giocatori.\\
Potenzialmente potete anche fare un sistema ibrido ovvero parte dei giocatori con i Tratti e parte con l'Allineamento, la gestione e' sicuramente interessante e spinge a tante interpretazioni ma risultera' un po' piu' difficile da gestire per il Narratore.

Oltretutto potete semplicemente decidere di lasciare fuori dal gioco l'aspetto etico e morale ovvero non andare a definire degli schemi nel quale inquadrare il personaggi, lasciare che i giocatori interpretino i personaggi come meglio credono.

In quest'ultimo caso suggerisco comunque di stabilire prima di giocare le linee guida per la morale/etica dei personaggi, pena lo sfaldamento del gruppo.

\subsection{I Tratti}

\label{tratti}
\begin{tcolorbox}[enhanced,arc=5pt,boxrule=0.3pt]{Chi dunque sa fare il bene e non lo compie, commette peccato. (Giacomo il Giusto 4.17, Lettera di Giacomo. NdA riferendosi ai Tratti scelti)\\\\
E' un diritto naturale saziarsi l'anima con la vendetta. (Attila)\\\\
Est Sularus Oth Mithas. (“Il mio onore è la mia vita”, Giuramento dei Cavalieri di Solamnia.)}\end{tcolorbox}\medskip
\index{Tratti}
In TUS non c'è una netta distinzione tra bene e male, legge e caos, tra ciò che è giusto e ciò che è sbagliato.

In TUS esistono i Tratti, aspetti e sfumature caratteriali che contribuiscono al background del personaggio, aiutano il giocatore a ruolare meglio e gli possono fornire quelle linee guida per interpretare in maniera più corretta il personaggio che ha voluto creare.

Un Tratto è un dettaglio che aiuta meglio a inquadrare il personaggio, ne delinea i 'trattì principali concedendogli sfumature diverse.

\textbf{Ogni giocatore sceglie 4 Tratti per il proprio personaggio alla creazione del giocatore.} Questi saranno le "bussole morali, etiche e comportamentali" che guideranno il personaggio nell'agire e nelle scelte.

I Tratti non sono il personaggio, non lo bloccano ne lo fissano eterno nel tempo. Un personaggio è sempre in costante evoluzione e così il suo carattere, morale, comportamento e desideri. Non essere rigido ma usa i Tratti per darti delle linee guida entro cui muoverti.

\textbf{Dei Tratti scelti al primo livello individuane uno, questo partirà, sempre al primo livello, con valore 1, gli altri partiranno a valore 0.}

Col passare del tempo e delle avventure potranno essere guadagnati o sostituiti (in concerto tra Narratore e giocatore in base a come giocato) da altri Tratti.

Potranno essere anche enfatizzati certi Tratti, ovvero il Narratore a seguito di particolari scene e ruolate potrà fare aumentare di un punto un Tratto del personaggio.

Ad esempio a seguito di una particolare scelta e climax di avventura il Narratore potrebbe concedere a tutti o qualcuno Tratto Coraggioso o dare un +1 a Coraggioso a chi ha già questo Tratto. Per i Tratti non presi si considera il valore base in punti di -1. ovvero il primo punto serve per prendere il Tratto ed i successivi per enfatizzarli.

Mentre è "relativamente" facile acquisire nuovi Tratti è estremamente difficile cambiare quelli già presente. Parlane con il Narratore, saprà preparare situazioni ed avventure che ti aiuteranno a comprendere come evolvere il personaggio ed eventualmente a cambiare i Tratti scelti.

Ogni azione particolarmente importante dove il personaggio abbia seguito un Tratto porta il personaggio ad avvicinarsi al Patrono competente per quel tratto.

Nella scheda troverai dei check da mettere vicino ai tratti, questi vengono segnati a seguito di azioni idonee ad accrescere il valore del tratto; raggiunti i 10 punti il Tratto aumenterà di 1 punto e si ricomincierà a segnare una nuova decina.

Sarà il Narratore durante l'avventura a dirti quando segnare, o cancellare, dei punti parziali. In linea di massima si presume che un personaggio acquisti almeno un punto in in Tratto a livello.

All'aumentare del valore del Tratto il personaggio potrà acquisire dei poteri, indipendentemente sia un credente di quella divinità (Patrono) o meno.


- A \textbf{5} punti si può incominciare a sentire la presenza di un Patrono legato ad un Tratto

- A \textbf{10} punti si sente la vicinanza di un Patrono legato ad un Tratto

- A \textbf{15}  punti si è legati ad Patrono da un Tratto

- A \textbf{20} punti si è un Campione del Patrono legato ad un Tratto.


Non è necessario credere in un Patrono per sentirne la vicinanza, semplicemente è la propria natura (i propri Tratti) che è affine al Patrono, che lo si voglia o meno.
Dato che lo scopo di un Patrono è fare che i propri tratti siano dominanti sugli altri, avere persone di alto livello e potere che siano così affini a lui tornerà utile nel giudizio dei 1000 anni.

Chiunque voglia diventare un incantatore deve avere individuare un Patrono ed avere almeno due tratti in comune con questo ed altri due tratti a piacimento.

\smallskip

Il valore dei tratti solitamente aumenta con il passare del livello e dell'esperienza. Il Narratore può sempre decidere in base ad azioni particolarmente ispirate o comportamenti idonei di aumentare anche nel mezzo di un livello il valore di un tratto.

Il Narratore è libero di inserire nuovi Tratti a suo piacere o richiesti dai giocatori, ci si deve però ricordare di attribuire questi tratti anche ai Patroni.

\bigskip

\textbf{Tabella dei Tratti}\index{Tratti}

\bigskip

\begin{tabular}{lllll}
	\toprule
	Accumulatore    & Aggressivo     & Allegro     & Anarchico       & Arrogante\\
	Caritatevole    & Altruista      & Altezzoso   & Aperto          & Avventato\\
	Combattivo      & Attento        & Avaro       & Calmo           & Calcolatore\\
	Crudele         & Bugiardo       & Buono       & Caritatevole    & Casto\\
	Disordinato     & Cattivo        & Clemente    & Codardo         & Controllato\\
	Egoista         & Corretto       & Cortese     & Creativo        & Coraggioso\\
	Freddo          & Determinato    & Diretto     & Disciplinato    & Curioso\\
	Impacciato      & Distaccato     & Distruttivo & Doppiogiochista & Disponibile\\
	Incostante      & Equilibrato    & Esuberante  & Fiducioso       & Empatico\\
	Indipendente    & Gentile        & Giusto      & Immaturo        & Generoso\\
	Introverso      & Impetuoso      & Implacabile & Incontentabile  & Imparziale\\
	Istintivo       & Ingenuo        & Indomito    & Indifferente    & Indisciplinato\\
	Logorroico      & Iracondo       & Innovativo  & Integerrimo     & Industrioso\\
	Onesto          & Morigerato     & Ironico     & Irrazionale     & Leale\\
	Ordinato        & Pessimista     & Lussurioso  & Libero          & Meticoloso\\
	Perfezionista   & Prudente       & Narcisista  & Negligente      & Mite\\
	Pio             & Riflessivo     & Paranoico   & Osservatore     & Permaloso\\
	Saccente        & Sadomasochista & Pratico     & Passionale      & Pianificatore\\
	Socievole       & Serio          & Rigido      & Riservato       & Protettivo\\
	Studioso        & Solitario      & Sarcastico  & Semplice        & Razionale\\
	Tenace          & Superficiale   & Scontroso   & Sincero         & Sadico\\
	Tradizionalista & Tranquillo     & Sicuro      & Sprovveduto     & Semplice\\
	Valoroso        & Truffatore     & Silenzioso  & Tollerante      & Sognatore\\
	Vanitoso        & Vendicativo    & Sospettoso  & Volubile        & Superbo\\
\end{tabular}

\pagebreak

\subsection{L'Allineamento (OGL)}

\begin{note}
Questo sistema e' alternativo ai Tratti, usato questo o l'altro.\\Questo sistema e' conforme allo standard OGL.
\end{note}


Le proprie convinzioni morali e il comportamento personale sono rappresentati dall'allineamento: legale buono, neutrale buono, caotico buono, legale neutrale, neutrale, caotico neutrale, legale malvagio, neutrale malvagio e caotico malvagio.\\

L'allineamento è uno strumento per definire l'identità del personaggio, non qualcosa che lo debba limitare. Ogni allineamento rappresenta un'ampia gamma di personalità o filosofie personali, in modo tale che due personaggi dello stesso allineamento siano comunque diversi l'uno dall'altro. Inoltre, sono poche le persone perfettamente coerenti con se stesse.\\

Tutte le creature hanno un allineamento. L'allineamento determina l'efficacia di alcuni incantesimi e oggetti magici.\\

Gli animali e le altre creature incapaci di azioni dettate dalla morale sono neutrali. Anche le vipere più letali o le tigri mangiatrici di uomini sono da ritenersi neutrali, in quanto mancano della capacità di distinguere un comportamento giusto da uno sbagliato sul piano morale. I cani potranno anche essere obbedienti e i gatti spiriti liberi, ma mancano delle capacità per essere veramente legali o caotici.\\

L'allineamento è una riassume la filosofia e la morale di un soggetto, ma due personaggi con lo stesso allineamento non sono esattamente identici. In ogni caso, l'allineamento dice molto riguardo l'animo di un personaggio e il modo in cui questi si rapporta agli altri.\\

Ciascun allineamento dispone di un elenco di filosofie o dottrine che i personaggi potrebbero seguire, accompagnato da una lista di concetti base che occorre tenere presente quando si gioca un personaggio di quel determinato allineamento. Si potrebbe decidere che uno di questi termini sia particolarmente determinante per un personaggio, sia che si tratti di giustizia, avidità o egoismo. Risulterà rapidamente evidente che alcuni di questi termini sono presenti in diversi allineamenti. Per una persona il termine "libertà" può voler dire libertà per sé e per gli altri, mentre per un'altra potrebbe voler dire la libertà di fare ciò che più le aggrada.\\

Quando si pensa agli allineamenti, si utilizzi questo semplice test: come reagirebbe il personaggio di fronte a uno sconosciuto in pericolo? Un soggetto caotico buono, che vede uno sconosciuto venire derubato, correrebbe in suo aiuto: una persona in pericolo ha bisogno di aiuto. Un personaggio legale buono agirebbe in modo da gestire la situazione, per far sì che giustizia sia fatta. Un personaggio neutrale potrebbe restarsene in disparte e osservare l'evolversi della situazione, agendo come gli sembra più opportuno in questo caso, per poi comportarsi in maniera totalmente opposta in un'altra situazione analoga. Un personaggio caotico malvagio prenderebbe parte allo scontro, nel probabile tentativo di derubare sia la vittima che l'assalitore. Un personaggio legale malvagio rimarrebbe in disparte fino alla fine del lo scontro, e quindi approfitterebbe della situazione per il proprio vantaggio, oppure per quello della sua divinità o del suo culto.\\


\subsubsection{Bene Contro Male}

I personaggi e le creature buone lottano per proteggere la vita degli innocenti. I personaggi e le creature malvagie, invece, tendono a disprezzare o a distruggere le vite innocenti, sia per soddisfazione che per guadagno personale.\\

Il bene comprende l'altruismo, il rispetto per la vita e per la dignità di tutti gli esseri senzienti. I personaggi buoni si sacrificano per aiutare gli altri.\\

Il male comporta invece il ferire, l'opprimere e l'uccidere il prossimo. Per alcune creature malvagie è praticamente impossibile provare compassione per gli altri: uccidono senza pensarci due volte, se ciò si rivela conveniente. Altri invece perseguono il male attivamente, uccidendo per gusto personale o al servizio di qualche padrone o divinità malvagia.\\\

Quelli che si pongono in una posizione di neutralità nei confronti del bene e del male hanno degli scrupoli nell'uccidere gli innocenti ma non si sentono in obbligo di fare sacrifici personali per proteggerli o aiutarli.\\

\subsubsection{Legge Contro Caos}

I personaggi legali dicono la verità, mantengono la parola data, rispettano l'autorità, onorano le tradizioni e si ergono a giudici di chi non mantiene i propri impegni. I personaggi caotici seguono solo la propria coscienza, rifiutano che venga loro imposto cosa fare, preferiscono le nuove idee alla tradizione e rispettano le proprie promesse in base all'umore del momento.\\

La legge implica l'onore, l'essere degni di fiducia, l'obbedienza all'autorità e l'affidabilità. La legalità però può anche voler dire chiusura mentale, attaccamento reazionario alla tradizione, tendenza a giudicare e mancanza di adattabilità. Quanti promuovono la legalità sostengono che solo un comportamento legale è in grado di creare una società in cui le persone possono fidarsi l'una dell'altra e prendere le decisioni giuste, sicuri in tutto e per tutto che gli altri agiranno come dovrebbero.\\

Il caos comporta libertà, adattabilità e flessibilità, ma può voler dire anche avventatezza, disprezzo per l'autorità, arbitrarietà nelle decisioni e irresponsabilità. Quelli che hanno consciamente un comportamento caotico sostengono che solo la libertà definitiva da ogni costrizione permette all'individuo di esprimersi pienamente, e prediligono il potenziale che ogni individuo ha dentro di sé rispetto ai benefici che la società comporta.\\

Le persone neutrali fra la legge e il caos mantengono un certo rispetto per le autorità e non sentono né il bisogno di obbedire né quello di ribellarsi. Pur essendo fondamentalmente oneste, sono spesso tentate di mentire o di ingannare gli altri.\\

\subsubsection{Gradi di Allineamento}

Spesso le regole fanno riferimento a “gradi” quando trattano dell'allineamento. In questi casi “gradi” si riferisce al numero di caselle di allineamento tra due allineamenti, come mostrato nel diagramma seguente. Notate che un “grado” in diagonale conta come due gradi. Per esempio, un personaggio legale neutrale è ad un grado dall'allineamento legale buono e a tre gradi da caotico malvagio. L'allineamento di un Devoto deve essere entro un grado da quello della sua divinità.\\

\medskip

\begin{tabular}{llll}
\toprule
					&\textbf{Legale}	&\textbf{Neutrale}	&\textbf{Caotico}\\
\textbf{Buono}		&Legale Buono		&Neutrale Buono		&Caotico Buono\\	
\textbf{Neutrale}	&Legale Neutrale	&Neutrale			&Caotico Neutrale\\
\textbf{Malvagio}	&Legale Malvagio	&Neutrale Malvagio	&Caotico Malvagio\\		
\end{tabular}

\medskip

\subsubsection{I Nove Allineamenti}
Nove diversi allineamenti descrivono tutte le combinazioni di legge-caos, bene-male. Ogni descrizione illustra il personaggio tipico appartenente a quell'allineamento. Non si dimentichi che i singoli individui possono variare rispetto alla norma, e che di giorno in giorno un personaggio può agire più o meno in accordo con il suo allineamento. È quindi meglio usare queste descrizioni come guida, non come un copione.\\

I primi sei allineamenti, da legale buono a caotico neutrale, sono gli allineamenti standard dei personaggi. I tre allineamenti malvagi appartengono invece ai mostri e ai nemici. Con il permesso del GM, un giocatore può assegnare un allineamento malvagio al suo personaggio, ma tali personaggi sono spesso fonti di distruzione e conflitto con i membri del gruppo buoni e neutrali. Il Narratore è incoraggiato a considerare attentamente come un PG malvagio potrebbe influire sulla campagna prima di ammetterlo.\\

\subsubsection{Legale Buono}
La giustizia è tutto. L'onore è la mia corazza. Colui che commette un crimine ne pagherà le conseguenze. Senza la verità e la legge esiste solo il caos. Io sono la luce. Io sono la spada della rettitudine. Il mio nemico finirà per pagare. La correttezza è forza. La mia anima è pura. La mia parola è verità.\\

\textit{Concetti Base}: Correttezza, dignità, dovere, onestà, onore, qualità, responsabilità, verità, virtù.
Un personaggio legale buono crede nell'onore. Un codice o un credo nel quale crede ciecamente plausibilmente guida i suoi passi. Preferirebbe morire piuttosto che tradire questo credo e i seguaci più estremi di questo allineamento sono disposti (a volte persino felici) di divenire martiri.\\

Un personaggio legale buono, all'estremo della scala leg­ge-­caos, può sembrare spietato. Potrebbe diventare ossessionato dal dispensare la giustizia, non pensare a nient'altro che a dedicarsi a dare la caccia a un Drago malvagio attraverso il mondo intero, oppure inseguire un Diavolo fino all'Inferno. Potrebbe essere percepito come un vero negriero, determinato a perseguire i propri fini senza deviare dalla sua strada, e potrebbe considerare tutti i soggetti meno determinati come dei deboli. Sebbene possa sembrare arcigno, persino duro, è sempre coerente, agendo per la sua dottrina o il suo credo. Il suo è un mondo di ordine, quindi obbedisce ai suoi superiori e trova quasi impossibile poter credere che ci sia del male in essi. Potrebbe essere più facilmente ingannato da questi impostori, ma alla fine vedrà trionfare la giustizia, per sua stessa mano, se necessario.\\

Un personaggio legale buono si comporta esattamente come ci si aspetta che agisca una persona di indole buona. Unisce un forte sentimento di opposizione nei confronti del male alla disciplina necessaria per combatterlo senza tregua. Dice la verità, mantiene la parola data, aiuta i bisognosi e si scaglia contro ogni ingiustizia. Un personaggio legale buono odia vedere i colpevoli farla franca.\\

Legale buono è l’allineamento migliore per chi vuole mettere assieme onore e compassione.\\

\subsubsection{Neutrale Buono}
Faccio sempre del mio meglio. Vedo il bene in ognuno. Aiuto gli altri. Mi prodigo per il bene comune. La mia anima è buona, indipendentemente dal mio aspetto. Non si giudica mai un libro dalla sua copertina. La devozione al bene in vita non richiede approvazione. La carità comincia a casa propria. Sii gentile.\\

\textit{Concetti Base}: Benevolenza, bontà, carità, correttezza, gentilezza, ragione, sollecitudine, umanità.
Un personaggio neutrale buono appartiene al bene, ma non è vincolato all'ordine. Vede il bene in tutto ciò che guarda, ma sa che il male può esistere persino nel luogo più ordinato.\\

Un personaggio neutrale buono fa tutto il possibile, e collabora con chiunque, per raggiungere il bene comune. Un personaggio con queste caratteristiche è determinato a essere buono e fa qualsiasi cosa in suo potere per raggiungere questo scopo. Potrebbe perdonare una persona malvagia, se crede che si sia pentita, ed è convinto che vi sia un po' di bene in ogni individuo.\\

Un personaggio neutrale buono fa sempre del suo meglio, nei limiti delle sue possibilità. Sente l’impulso di aiutare gli altri, lavora volentieri con re e magistrati, ma non sente alcun Legame verso di loro.\\

Neutrale buono è l’allineamento migliore per chi vuol fare del bene senza avere pregiudizi a favore o contro l’autorità.\\

\subsubsection{Caotico Buono}
La mia anima è buona, ma libera. La legge non ha coscienza. Il cieco ordine promuove il disordine. La bontà non può essere appresa semplicemente da un libro di preghiere. La compassione non indossa alcuna uniforme. Il più piccolo atto di gentilezza non è mai sprecato. Ripaga la gentilezza con la stessa moneta. Sii gentile con coloro che si trovano nei guai: il giorno dopo potresti essere tu ad aver bisogno di gentilezza.\\

\textit{Concetti Base}: Benevolenza, calore, carità, gentilezza, gioia, indipendenza, pietà.
Un personaggio caotico buono apprezza l'indipendenza e avere il diritto di seguire la propria strada. Potrebbe avere etiche e filosofie personali, ma non ne è rigidamente vincolato. Potrebbe cercare di fare del bene ogni giorno, magari mostrandosi gentile con uno straniero oppure elargendo denaro ai meno fortunati, ma compie questi atti spinto semplicemente dalla gioia che questi gli fanno provare. Un personaggio di questo tipo sviluppa una personale concezione di cosa è bene e cosa è male, generalmente basandosi su verità e fatti, ma non si illude che le azioni malvagie possano essere buone. La sua bontà è generosa, magari a volte cicca, ma sempre con buoni propositi.\\

Un personaggio caotico buono potrebbe sembrare imprevedibile, facendo l'elemosina a uno sfortunato fuori da una chiesa, per poi rifiutarsi di offrire un obolo, una volta al suo interno. Fa affidamento ai suoi istinti, e potrebbe avere più fiducia nelle parole di un mendicante dagli occhi gentili, che negli insegnamenti di un vescovo dall'aspetto severo. Potrebbe rubare ai ricchi per donare ai poveri, oppure spendere ingenti somme per la propria felicità, e per quella dei suoi amici. In circostanze estreme, un personaggio caotico buono potrebbe sembrare avventato nella sua generosità.\\

Un personaggio caotico buono segue soltanto la propria coscienza, senza preoccuparsi di ciò che gli altri si aspettano da lui. Segue la sua strada, ma riesce comunque ad essere gentile e bendisposto. Crede nel bene e nella giustizia, ma non ha molto rispetto per leggi e regole. Odia quando qualcuno cerca di intimorire gli altri e di imporre loro cosa fare. Segue la sua morale che, anche se positiva, potrebbe non andare d’accordo con quella della società.\\

Caotico buono è l’allineamento migliore per chi vuole conciliare lo spirito libero con la bontà di cuore.\\

\subsubsection{Legale Neutrale}

L'ordine genera ordine. La mia parola è vincolante. Il caos distruggerà il mondo. Rispetta la gerarchia. Io vivo seguendo il mio codice d'onore, e morirò seguendolo. La tradizione deve continuare. L'ordine è il fondamento di ogni cultura. Io sono il giudice di me stesso.\\

\textit{Concetti Base}: Armonia, gerarchia, giuramento, lealtà, ordine, organizzazione, regola, sistema, tradizione.\\
Un personaggio legale neutrale ammira l'ordine e la tradizione, oppure tenta di vivere attenendosi a un codice d'onore. Potrebbe temere il caos e il disordine, e magari avere buoni motivi per farlo, in ragione di passate esperienze. Un individuo legale neutrale non si preoccupa di chi lo governa, ma piuttosto di quanto siano al sicuro lui e i suoi compagni, e trova grande conforto nella normalità della società. Questi personaggi possono ammirare i comandanti e le punizioni più severe, a patto che servano a mantenere l'ordine e potrebbero essere favorevoli a guerre contro altre nazioni, persino quando il loro paese si trasforma in un brutale invasore: l'unica cosa che gli importa è che l'azione militare sia giustificata.\\

Un personaggio legale neutrale, che segue un proprio codice d'onore, non lo tradirà mai intenzionalmente e potrebbe diventare un martire pur di difenderlo.\\

Un personaggio legale neutrale agisce come gli suggeriscono le leggi, le tradizioni o il suo codice d’onore personale. L’ordine e l’organizzazione sono di importanza vitale per lui. Può credere nell'ordine per se stesso e vivere quindi secondo un codice preciso, oppure credere nell'ordine per tutti e quindi essere dalla parte di un governo forte e organizzato.\\

Legale neutrale è l’allineamento migliore per chi vuole essere affidabile e onorevole senza diventare un fanatico.\\

\subsubsection{Neutrale}

I nostri capricci e desideri sono irrilevanti, se paragonati al ciclo vitale del mondo. Io sono ciò che sono. Non fidarti di nessuno tranne che dei tuoi amici e dela tua famiglia. La ruota gira a dispetto di ciò me facciamo. I governi vanno e vengono. Tutti gli imperi svaniscono. Il tempo guarisce tutto. Le stagioni sono immutabili. Il sole non si preoccupa di ciò che illumina al suo passaggio.\\

\textit{Concetti Base}: Armonia, ciclicità, equilibrio, equità, imparzialità, ineluttabilità, natura, stagioni\\.
Un personaggio neutrale è insolito, in quanto potrebbe seguire una tra due filosofie ben distinte: potrebbe essere neutrale in quanto apatico o diffidente nei confronti degli altri, oppu­re desiderare di riservare un atteggiamento totalmente neutrale verso il mondo e rifiutare l'estremismo.\\

Un personaggio neutrale potrebbe sembrare egoista o disinteressato. Potrebbe essere spinto principalmente dalla sua accettazione del fato, e gli appartenenti a questo allineamento più estremisti diventano eremiti, nascondendosi dai fanatici del mondo. Alcuni personaggi neutrali, al contrario, si battono apertamente per la neutralità, e rifuggono qualsiasi azione che viri in maniera troppo decisa verso un qualsivoglia allineamento. Questo tipo di personaggio neutrale è orgoglioso della propria abilità di agire tra caos e legge, bene e male. Potrebbe avere un punto di vista fatalistico nei confronti della natura e dei poteri fondamentali della notte e del giorno.\\

Un personaggio neutrale fa sempre ciò che gli sembra essere una buona idea. Quando si tratta del male e del bene, della legge e del caos non ha particolari tendenze dall'una o dall'altra parte. Nella maggior parte dei casi si tratta più di una mancanza di convinzioni o di inclinazioni morali che di una reale Devozione alla neutralità. Un personaggio con questo allineamento considera il bene migliore del male (tutto sommato, preferisce avere compagni e regnanti buoni piuttosto che malvagi), ma non si sente obbligato a perseguire il bene in modo teorico e universale.\\

D’altra parte alcuni personaggi perseguono la neutralità da un punto di vista filosofico: vedono il bene, il male, la legge e il caos come degli estremi pericolosi e quindi sostengono la causa della via di mezzo neutrale ritenendola la strada migliore e più equilibrata a lungo andare.\\

Neutrale è l’allineamento migliore per chi vuole agire spontaneamente, senza pregiudizi né costrizioni.\\

\subsubsection{Caotico Neutrale}

Il muschio non cresce su una pietra che rotola. Non esiste che il presente. Sii come il vento e lasciati condurre dove il fato ti porta. Colui che si batte contro il fato corteggia la follia. Si vive una volta sola. Potere a coloro che non desiderano il potere. Evita qualsiasi cosa che porti un'uniforme. Sfida il vecchio ordine.\\

\textit{Concetti Base}: Fato, imprevedibilità, indipendenza, individualismo, libertà, padronanza di sé, volubilità.\\
Un personaggio caotico neutrale considera irrinunciabile la sua indipendenza e la possibilità di fare scelte. Evita ogni autorità e non ha paura di distinguersi dagli altri o di sembrare diverso. In casi estremi, potrebbe intraprendere uno stile di vita a lui completamente confacente, come vivere in una caverna poco lontana da una città, diventare un artista, oppure sfidare in qualche altro modo le tradizioni. Non accetta mai nulla considerando solo le apparenze, ma sviluppa una propria opinione, piuttosto che accettare ciecamente ciò che gli altri gli dicono di pensare o di fare.\\

Un personaggio caotico neutrale segue esclusivamente il suo arbitrio. È l’individualista per eccellenza. Mette la sua libertà sopra ogni altra cosa ma non lotta per difendere quella degli altri, respinge l’autorità, odia le costrizioni e si scaglia contro le tradizioni. Un personaggio caotico neutrale non tenta di distruggere l’ordine intenzionalmente, come parte di una campagna tesa all'anarchia. Per fare questo dovrebbe essere motivato dal bene (guidato da un desiderio di liberare gli altri) o dal male (spinto da un desiderio di far soffrire chi è diverso da sé). Si tenga presente che un personaggio caotico neutrale può essere imprevedibile, ma che le sue azioni non sono dettate dal caso. Difficilmente salterà giù da un ponte quando può invece attraversarlo.\\

Caotico neutrale è l’allineamento migliore per chi cerca la completa libertà sia dalle restrizioni della società che dallo zelo dei benefattori.\\

\subsubsection{Legale Malvagio}
Un giorno sarò io a regnare. Un forte leader è ammirato, un leader debole deposto. Io ho dei principi e sono nel giusto. Il caos porta la morte. In questo mondo esistono solo l'ordine o l'oblio. Le gerarchie devono essere rispettate e temute. I deboli seguiranno leader determinati. Peccare dà soddisfazione. Ognuno ha i propri vizi.\\

\textit{Concetti Base}: Calcolatore, disciplina, malevolenza, potenza, punizione, razionalità, sottomissione, terrore.\\
Un personaggio legale malvagio persegue i propri scopi, motivato solamente dal suo interesse personale, sapendo che, in ultimo, l'ordine lo proteggerà. Cerca di raggiungere i propri obiettivi, grazie all'ordine e non al caos. Anche quando è accecato dalla rabbia è più probabile che si fermi ad architettare un accurato piano di vendetta, piuttosto che rischiare di morire in un'azione frettolosa. Talvolta una simile vendetta richiede anni prima di venire messa in atto, ma questo è ritenuto accettabile.\\

Un personaggio legale malvagio, agli estremi dello spettro, è zelante nel perseguire i suoi scopi e sarà disposto a qualsiasi sa­crificio per raggiungerli. La sua contorta filosofia potrebbe renderlo paranoico, anche nei confronti dei suoi seguaci più fidati, e persino verso familiari e ­amici. Non si fermerà davanti a nulla pur di ottenere il controllo, dal momento che è solo per mezzo del controllo che si può ottenere la pace. Eppure, anche l'organizzazione più potente e ordinata ha dei nemici e, per un personaggio legale malvagio, solo la distruzione di questi nemici può portare all'appagamento.\\

L'ordine è tutto, costi quel che costi.\\

Un individuo legale malvagio prende ciò che vuole, seguendo un suo codice morale, ma senza alcun riguardo per chi ferisce nel farlo. È attaccato alla tradizione, alla lealtà e all'ordine ma non rispetta la libertà, la dignità o la vita. Gioca secondo le regole, ma senza alcuna pietà o compassione. Si trova a suo agio all'interno di una gerarchia e ambisce a dominare, ma è comunque disposto a servire. Giudica gli altri non per le loro azioni, ma secondo la razza, la religione, la terra di origine e la posizione sociale. È restio a infrangere la legge o le promesse fatte, una riluttanza causata in parte dalla sua natura e in parte dal fatto che considera quest’ordine necessario per proteggersi da quanti si oppongono a lui sul piano morale.\\

Alcuni individui legali malvagi hanno particolari principi morali, come non uccidere mai a sangue freddo (lasciando eventualmente che siano i propri sottoposti a farlo), o non fare del male ai bambini (se si può evitare). Credono che questo atteggiamento li ponga al di sopra dei criminali senza scrupoli.\\

Alcuni legali malvagi sono votati al male con uno zelo paragonabile a quello con cui un crociato è votato al bene. Oltre ad essere disposti a far male al prossimo per i loro scopi, si compiacciono nel diffondere il male come fine a se stesso. Potrebbero anche considerare il male che causano come parte del loro dovere verso un padrone o una divinità malvagia.\\

Legale malvagio potrebbe essere l’allineamento più pericoloso poiché rappresenta una malvagità metodica, intenzionale ed organizzata.\\

\subsubsection{Neutrale Malvagio}

Io sono la cosa più importante del creato. Fai ciò che vuoi, ma senza farti mai scoprire. La coscienza è per gli angeli. Il male è fine a se stesso. Il vizio è di per sé la ricompensa. I peccatori si godono la vita. Male è solo una parola. Gli altri invidiano la mia libertà e la mia vita, non offuscate dalla coscienza.\\

\textit{Concetti Base}: Abiettezza, crudeltà, desiderio, egoismo, immoralità, necessità, peccato, perversità, vizio.\\
Motivato dalle proprie necessità e brame, un personaggio neutrale malvagio è privo di coscienza e agisce solo per il proprio interesse. Potrebbe avvalersi dei legami del culto e della malvagità, ma solamente perché questi lo portano più vicino al peccato e alla perversione. Mentre una persona legale malvagia è portata al compromesso, e una caotica malvagia a reagire violentemente, un individuo neutrale malvagio è incline a perseguire il proprio interesse. Sotto molti aspetti è l'epitome del male, dal momento che non è realmente leale a null'altro che al proprio mero interesse.\\

Un personaggio neutrale malvagio estremista tende a essere un solitario, dal momento che ha tradito, o ucciso, tutti coloro che si sono avvicinati a lui quanto basta da conoscerlo.\\

Un individuo neutrale malvagio è disposto a fare di tutto, finché gli è consentito di cavarsela e pensa solo a se stesso. Non versa lacrime per quelli che uccide, non importa se per guadagno, gusto personale o convenienza; non ha una particolare propensione per l’ordine, né si illude che seguire le leggi, le tradizioni e i codici morali lo rendano migliore o più nobile. D’altra parte, non è perennemente insoddisfatto di natura e non ha lo stesso amore per il conflitto che un caotico malvagio potrebbe avere.\\

Alcuni neutrali malvagi sono votati al male come ideale e lo praticano come fine a se stesso. Spesso questo genere di persone venera divinità malvagie o appartiene a società segrete.\\

Neutrale malvagio potrebbe essere l’allineamento più pericoloso poiché rappresenta il male allo stato puro, senza onore e senza volubilità.\\

\subsubsection{Caotico Malvagio}

Se voglio una cosa, me la prendo. Vige la regola del più forte. I forti dominano i deboli. Rispettami o subiscine le conseguenze. Esiste solo l'oggi, e oggi prendo ciò che mi serve. La rabbia fa emergere la mia parte migliore. Io sono il più forte.\\

\textit{Concetti Base}: Amoralità, anarchia, brutalità, caos, depravazione, empietà, indipendenza, rabbia, violenza.\\
Un personaggio caotico malvagio è guidato solamente dalla propria rabbia e dalle proprie necessità. È assolutamente avventato nelle sue azioni e agisce d'impulso, non considerando la sofferenza che questo causa agli altri. Sotto molti aspetti, un personaggio caotico malvagio è vincolato dalla sua stessa natura a essere imprevedibile. È come un fuoco che avvampa, una tempesta che si avvicina, la lama di una spada che non è mai stata utilizzata. Un personaggio caotico malvagio estremista tende ad associarsi a individui a lui simili, non perché necessiti di compagnia, ma piuttosto perché trova familiarità in questo caos, e gioisce dell'opportunità di essere fedele alla propria natura, in compagnia di altri che condividono questa medesima gioia.\\

Un personaggio caotico malvagio fa solo ciò che la sua ingordigia, il suo odio e il suo desiderio di distruzione lo spingono a fare. È facile all'ira, immorale, ingiustificatamente violento ed imprevedibile. Quando cerca di impossessarsi di ciò che può avere, si comporta in modo brutale e senza scrupoli. Se invece è votato alla diffusione del male e del caos, è anche peggio. Fortunatamente i suoi piani tendono ad essere assurdi, e ogni gruppo che formi o a cui decida di unirsi è assai poco organizzato. Solitamente, i caotici malvagi possono essere costretti a lavorare assieme solo con la forza, e chi li comanda si mantiene al potere solo finché riesce a contrastare i loro tentativi di sostituirlo o assassinarlo.\\

Caotico malvagio potrebbe essere l’allineamento più pericoloso poiché rappresenta la nemesi non solo della vita e della bellezza, ma anche dell’ordine dal quale la vita e la bellezza dipendono.\\

\subsubsection{Cambiamento Forzato dell'Allineamento}

Quando un cambiamento di allineamento forzato è puramente arbitrario (come quello causato da una maledizione o da un Oggetto Magico), alcuni giocatori considerano questo cambiamento come un'opportunità per far agire il personaggio in modo totalmente differente, ma la maggior parte preferisce la propria concezione originale del personaggio e desidera farlo tornare quanto prima alla normalità. \\

I Narratore dovrebbero evitare di abusare dei cambiamenti di allineamento forzati, oppure renderli solo temporanei (per esempio in una situazione nella quale i personaggi vengono posseduti da entità malvagie, per essere poi liberati quando queste hanno portato a termine un particolare fine).\\


\begin{note}
L'Allineamento non e' una regola immutabile, solo perche' sei un Elfo non devi essere malvagio, o solo perche' sei un Altro non devi seguire gli dei oscuri.\\
Anche Diavoli o Angeli possono essere piu' completi, mutevoli di quanto la loro natura inumana preveda.\\
Il giocatore plasma il persoaggio come piu' gli aggrada e nel tempo questo evolve, come noi tutti, facendosi opinioni, carattere.. abitudini.\\
Non siate rigidi, fate crescere il personaggio!	
\end{note}



\pagebreak

\section{Competenze}\index{Competenze}

\label{competenze}
\begin{tcolorbox}[enhanced,arc=5pt,boxrule=0.3pt]{
Chi dice che una cosa è impossibile, non dovrebbe disturbare chi la sta facendo.\\
Non hai veramente capito qualcosa fino a quando non sei in grado di spiegarlo a tua nonna. (Albert Einstein)}\end{tcolorbox}\medskip


Ogni personaggio ha conoscenze e/o sa fare qualcosa e questo qualcosa è una competenza. A seconda dei background ed avventure giocate i personaggi valorizzano e imparano determinate competenze.

Alla creazione del personaggio attribuire 1 punto libero in Artigianato o Professione o Intrattenere o Cultura per giustificare competenze di background.

Se non specificato diversamente per tutte le prove di competenza (Base, Attive) valgono tre regole base \index{Regole Base} (Golden Rules):\index{Golden Rules}

\begin{itemize}
	\item
	      I \textbf{6 esplodono}, ovvero se nella prova dei 3d6 un dato fa sei, somma il risultato e ritira, e se fa 6 nuovamente sommi il risultato e ritiri ancora e ancora..
	\item
	      Gli \textbf{1 portano male}, se fai 1 con il dado togli 1 alla somma dei dadi tirati (e quindi il dado che ha fatto 1 conta zero)
	\item
	      \textbf{Affidarsi alla sorte}. Per ogni 4 punti di competenza che rinunci a sommare nella prova tira un dado a 6 in piu', che rispetta le regole base.
\end{itemize}

\textbf{Regola Bonus}: quando una penalità è indicata come "-1d6" ( o peggio..) significa che si toglie 1 dado alla prova, se non è una prova si toglie 4 al risultato finale per ogni dado di penalità. Il risultato minimo per il tiro dei dadi è 0.


\subsection{Competenze di Base}\index{Competenze di Base}

\label{competenze-di-base}

\begin{tcolorbox}[enhanced,arc=5pt,boxrule=0.3pt]{
		Anche se indubbiamente il desiderio di conoscere è naturale per tutti gli uomini, la voglia di imparare non è cosa da tutti; la maggior parte, anzi, assaggiato quanto lo studio sia fatica e provata la stanchezza sulla propria pelle, butta alla leggera la noce ancor prima di aver rotto il guscio per gustarne il gheriglio. (Richard de Bury)\\\\
		Lo studio è per i perdenti! (Lobo)
	}\end{tcolorbox}\medskip

Ogni personaggio al primo livello sceglie delle Competenze di Base, su queste distribuisce 4 punti, con un massimo di 2 punti per Competenza al primo livello.

Ad ogni livello successivo distribuisce un numero di punti pari la metà del punteggio di Intelletto +2 ((Int+2)/2)) con un minimo di 1 punto tra le competenze già conosciute perfezionate nell'avventura o apprese ex novo. Nessuna competenza di base può avere un  punteggio superiore al livello del personaggio+1.

Un personaggio puo' apprendere una nuova competenza con uno studio/pratica di 4 ore al giorno per almeno 4 mesi. Dopo questo lasso di tempo il giocatore puo' assegnare un punto alla competenza di base per cui si e' applicato.

\bigskip

Queste le competenze ed i loro ambiti di utilizzo:

\textbf{Acrobatica}(Agilità)\textbf{:}\index{Acrobatica} arrampicarsi, equilibrio, saltare, acrobazia

\textbf{Arcano}(Intelletto):\index{Arcano} Conoscenza arcana, piani, occulta, riconoscere essenze, creature magiche

\textbf{Consapevolezza} (Volonta'):\index{Consapevolezza} percezione, percepire inganni, nascondersi nelle ombre

\textbf{Cultura} (Intelletto): \index{Cultura}geografia, natura, erboristeria, storia, religione, tradizioni e storia locale, dungeon, ingegneria, falsificare, lingue

\textbf{Criminalita'} (Agilità):\index{Criminalita} travestirsi, disattivare congegni, artista della fuga, mani di fata, muoversi silenziosamente

\textbf{Faccia tosta} (Magnetismo): \index{Faccia tosta}valutare, intimidire, diplomazia (arguzia), raggirare, persuadere

\textbf{Intrattenere} (Magnetismo): \index{Intrattenere}cantare, suonare, recitare, travestirsi, diplomazia (oratoria)

\textbf{Lavoro} (Volonta'):\index{Lavoro} artigianato, professione (es.: allevatore, architetto, azzeccagarbugli, barcaiolo, cacciatore, mercante, sarto, fabbro, apicoltore, conciatore..)

\textbf{Resistenza} (Potenza): \index{Resistenza}nuotare, correre, saltare, scalare

\textbf{Sopravvivenza} (Volonta'):\index{Sopravvivenza} seguire tracce, sopravvivenza, gestire animali, cavalcare, usare una corda, curare/pronto soccorso, creature naturali

\bigskip

Il Narratore potrebbe quindi chiederti genericamente una prova di Criminalità oppure dirti di fare una prova da mani di fata (borseggiare), il risultato non cambia effettui comunque una prova di Criminalità, ovvero 3d6 + punteggio in Criminalità + Agilità + eventuali bonus o malus.

Ricordati sempre delle 3 regole base: esplosione del 6, gli uno portano male ed affidarsi alla sorte.

\bigskip

Potranno esserci situazioni od oggetti che concedono un bonus specifico ad una competenza ovvero non tanto un bonus a Criminalità ma solo a mani di fata, in quel caso il bonus si applica non a tutte le prove di Criminalità ma solo a quelle specifiche di mani di fata.


\subsection{Competenze Attive}\index{Competenze Attive}

\label{competenze-attive}
\begin{tcolorbox}[enhanced,arc=5pt,boxrule=0.3pt]{C'è solo un modo per allenarsi: quello giusto. (Carl Lewis)\\\\
		Wang Chi: Sei pronto?\\
		Jack Burton: Io sono nato pronto! (Grosso guaio a Chinatown, Film 1986)
	}\end{tcolorbox}\medskip

Ogni personaggio prende 3 punti da distribuire nelle Competenze Attive a livello. Si può assegnare massimo 1 punto a livello in una singola Competenza Attiva.

Le Competenze Attive sono: Competenza Magica\index{Competenza Magica}, Competenza Armi\index{Competenza Armi}, Tiro Salvezza su Tempra,\index{Tiro Salvezza su Tempra} Tiro Salvezza su Arbitrio \index{Tiro Salvezza su Arbitrio}e Tiro Salvezza su Riflessi\index{Tiro Salvezza su Riflessi}.

\textbf{Competenza Magica (CM)} \index{CM}(varie): indica la capacità e competenza nel lanciare una Essenza.

\textbf{Competenza Armi (CA)} \index{CA}(Potenza o Agilità): è la capacità e bravura di combattere con un'arma da mischia o da tiro/distanza.

Il \textbf{Tiro Salvezza su Tempra} indica quanto si è in grado di sopportare le sofferenze fisiche o attacchi contro la propria vitalità e salute. Ai Tiri Salvezza su Tempra si aggiunge il valore della Potenza.

Il \textbf{Tiro Salvezza su Arbitrio} indica la resistenza contro l'influenza mentale ed altri effetti magici, ciò che vuole modificare il tuo libero arbitrio nelle scelte e nell'agire. Ai Tiri Salvezza su Arbitrio si aggiunge il valore di Volontà.

Il \textbf{Tiro Salvezza su Riflessi} indica quanto si è agili e pronti per evitare ostacoli o magie. Ai Tiri Salvezza su Riflessi si aggiunge il valore di Agilità

\bigskip

\begin{tcolorbox}[title = un personaggio di quarto livello distribuisce le competenze attive in questa maniera] 

1 livello: +1 Competenza Armi , +1 Tiro Salvezza Tempra, +1 Tiro Salvezza Riflessi

2 livello: +1 Competenza Armi, +1 Competenza Magica, +1 Tiro Salvezza Riflessi

3 livello: +1 Competenza Magica, +1 Tiro Salvezza Tempra, +1 Tiro Salvezza Arbitrio

4 livello:+ 1 Competenza Armi, +1 Competenza Magica, +1 Tiro Salvezza Riflessi

per un totale di +3 CA, +3 CM, +3 TS Riflessi, +2 TS Tempra, +1 TS Arbitrio

\end{tcolorbox}


\bigskip

Ogni punto attribuito (CA, CM, TS) permette di usufruire di +1 nella prova (o Tiro Salvezza) relativa.

La \textbf{Competenza Armi} (abbreviata in \textbf{CA}) indica la capacità e bravura nel colpire l'avversario con armi.

Il \textbf{Tiro per colpire}\index{armi da mischia}, per le armi da mischia si risolve con una prova di Competenza Armi (\textbf{CA}) + Potenza + eventuali capacità e bonus magici contrapposto alla Difesa dell'avversario (Agilità + armatura/scudi/bonus).

Il \textbf{Tiro per Colpire con armi da distanza} \index{Armi da distanza}(archi, pugnali da lancio, sassi..) si risolve con una prova di Competenza Armi (\textbf{CA}) + Agilità + eventuali capacità e bonus magici contrapposto alla Difesa dell'avversario (Agilità (schivare) + armatura/scudi/bonus).

Quando si assegna un punto ad \textbf{CA} va sempre precisata su quale gruppo di arma si prende, se non si dichiara allora è come averlo preso nel gruppo Armi Semplici.
Controllare l'elenco Armi per Tipologia Omogenea.\index{Tipologia Omogenea}

Il personaggio può decidere di assegnare il suo punto ad una tipologia che già conosce, migliorando così la sua capacità e competenza nell'uso od apprendere un altra tipologia di arma.
Il giocatore deve considerare che migliore è la sua capacità con una tipologia di arma più facilmente può usufruire di vantaggi nella stessa, ma conoscerà meno armi.

Se il giocatore non ha assegnato alcun punto nella \textbf{CA} può utilizzare, senza penalità al colpire, solo le armi raggruppate come Armi Semplici.

Le \textbf{armi semplici} sono: Pugnale, Mazza Leggera, Randello, Morningstar,
Lancia corta da fante, Bastone, Balestra (Leggera), Giavellotto\index{Armi Semplici}

Usare un'\textbf{Arma senza l'adeguata competenza} nel gruppo di appartenenza impone un -2d6 al Tiro per Colpire.\index{Arma senza competenza}

Per poter utilizzare \textbf{Armature Leggere} è necessario avere almeno un punto in Competenza Armi.\index{Armature Leggere}

Per poter utilizzare \textbf{Armature Medie} e \textbf{Scudi Leggeri} o \textbf{Medi} è necessario avere almeno 2 punto in Competenza Armi.\index{Armature Medie}

Con almeno 3 punti in CA ed 1 in Potenza si possono usare senza penalità \textbf{Armature Pesanti} e \textbf{Scudi Pesanti}.\index{Armature Pesanti}\index{Scudi Pesanti}

Usare un'\textbf{Armatura senza l'adeguata competenza} impedisce di usare il valore di Agilità in Difesa ed il bonus conferito dall'armatura alla Difesa si riduce di 1.\index{Armatura senza competenza}

Usare uno \textbf{Scudo senza l'adeguata competenza} peggiora il Tiro per Colpire di 1 e lo scudo conferisce un bonus massimo a Difesa di 1.\index{Scudo senza competenza}

La \textbf{Competenza Magica} (abbreviata in \textbf{CM}) permette al personaggio di poter lanciare più magie , più potenti, più efficaci e più facilmente.

Un personaggio con alta \textbf{CM} sa manipolare più Essenze e con risultati migliori.

I tiri salvezza si eseguono confrontando la DC (prova di difficolta') che conosce il Narratore con la prova relativa.

Se ti chiedono un Tiro Salvezza Riflessi per evitare un fulmine (Essenza Attacco, Elettricità), farai un Tiro Salvezza Riflessi (3d6 + valore TS Riflessi + bonus Agilità +- varie ed eventuali) questo valore lo comunicherai al Narratore che lo confronterà con la DC del Tiro Salvezza.

Il Narratore non ti dirà di fare un Tiro Salvezza a difficoltà 18, è lui che confronta il tuo tiro con la difficoltà, potrà dirti che la prova è complessa, difficile o facile...


\pagebreak

\section{Costruiamo il Personaggio}\index{Personaggio}

\label{costruiamo-il-personaggio}
\begin{tcolorbox}[enhanced,arc=5pt,boxrule=0.3pt]{
		"Mai dimenticare chi sei, perché di certo il mondo non lo dimenticherà.Trasforma chi sei nella tua forza, così non potrà mai essere la tua debolezza. Fanne un'armatura, e non potrà mai essere usata contro di te." (Tyrion Lannister)
	}\end{tcolorbox}\medskip

Come prima cosa prepara davanti a te la scheda ed un foglio dove prendere note ed appunti.

Parti immaginando, visualizzando l'aspetto del tuo personaggio. Come te lo immagini ? In possente barbaro delle steppe ghiacciate od un mago scavezzacollo alle prime esperienze ?

Individua il nome e pensa a ciò che conosce, quali esperienze ha avuto e quali lo hanno segnato.

Come si comporta con gli altri? è un tipo ordinato, ha delle fisse, ha qualche tic o abitudine ?

è cresciuto in famiglia, in un clan, vagabondo, per strada.. cosa l'ha portato e che scelte ha fatto per arrivare fino ad adesso ?

Quale è il suo stile di combattimento e strategia tipo ? Magia, Spada, dalle retrovie.. incitare i compagni.. scappare...

Per incominciare leggi il capito sulle Razze ed individua quella del tuo personaggio.

Hai 6 punti caratteristica, 0 è un valore medio, -1 debole, +1 buono, al massimo una caratteristica può avere 2 (buono) come valore. Messi i punteggi applica i modificatori razziali se presenti.

Se hai Intelletto pari o superiore a 2 scegli un altra lingua parlata/scritta oltre al comune.

Competenze Attive, qui ha 3 punti da distribuire tra Competenza Armi, Competenza Magica e Tiri Salvezza.

La competenza armi ti aiuta nel colpire meglio. La competenza magica è l'unica cosa che ti permette di usare la magia.

Ricorda anche che i punti in Competenze Armi vanno dichiarati a quale lista di armi sono stati applicati.

Se non hai punti in Competenza Armi puoi usare solo le armi semplici senza incorrere in penalità al Tiro per Colpire.

Puoi assegnare ai Tiri Salvezza un punto per tipo (non puoi mettere due punti in un solo Tiro Salvezza al passaggio di un livello). Questi valori determinano la tua capacità di sopravvivenza e di resistere a traumi e magie

Competenze di Base, al primo livello distribuisci 4 punti con un massimo di 2 punti per Competenza.

Assegna anche 1 punto libero a Professione / Artigianato / Intrattenere / Cultura giustificando il background del personaggio.

Ad ogni livello successivo distribuisci un numero di punti pari a metà del punteggio di Intelletto+2 tra le competenze già conosciute e perfezionate nell'avventura o apprese ex novo.

A questo punto scegli i Tratti. Fallo con attenzione e cura, stai costruendo il tuo personaggio ed i tratti delineano a forti pennellate il carattere. Ricordati che saranno fondamentali per le Essenze.

Scegli lo svantaggio di ruolo e se vuoi anche svantaggi e vantaggi. Ricorda di giocarlo, altrimenti non è divertente.

Se hai messo dei punti in Competenza Magica a questo punto devi scegliere (dopo aver parlato con il Narratore su quale sistema magico è in uso) quali Essenze conosci. Ricorda che per ogni punto in Competenza Magica puoi decidere di apprendere due nuove Essenze o specializzarti in un Essenza già nota dandogli un bonus di +1 alle prove.

Passa alle Abilità, al primo livello ne scegli due, stai attento ai prerequisiti ed anche ad eventuali abilità che ti concede la tua razza. Ogni livello dispari prenderai un altra abilità.

Scegli l'equipaggiamento, armatura, armi, zaino, due torce, qualche razione di cibo.. un peluche.. quello che ti sembra indispensabile per l'avventura.

Aggiorna poi la parte di scheda relativa alla Difesa segnando che bonus ti da l'armatura e scudo indossata.

Infine entra nella parte, concediti di giocare questo straordinario personaggio. Se mai ti stufassi di giocarlo e volessi provare qualcosa di diverso parlane con il Narratore, saprà consigliarti e suggerirti la strada migliore.
Oltretutto hai il vantaggio che in TUS le classi non esistono, il personaggio cresce evolve ed impara in base a ciò che fai e sperimenti.

\pagebreak

\section{Regole per le competenze}\index{Regole per le competenze}\index{Competenze}

\label{regole-per-le-competenze}
\begin{tcolorbox}[enhanced,arc=5pt,boxrule=0.3pt]{
		Occorre che la legge sia breve, perché più facilmente i mal pratici la ricordino. (Seneca)}\end{tcolorbox}\medskip

\textbf{Le prove (i check) per le competenze si eseguono tirando 3d6, al risultato dei dadi si somma il punteggio della competenza (di base, attiva) e della caratteristica collegata ed eventuali bonus magici e di circostanza o Abilità, il risultato ottenuto deve essere comunicato al Narratore, il quale lo confronterà con il DC della prova.}

Quando dovete stabilire una difficoltà partite pensando che la prova deve essere rapportata da una persona "normale". Non pensate "se la dovessi fare io allora la prova sarebbe impossibile", "se la prova la fa Arsenio Lupin la prova è facilissima". Partite dal presupposto che la difficoltà deve racchiudere in se tutti gli elementi circostanziali.

Pensate se piove, c'è poca luce, il personaggio sta correndo, è ferito, fa le cose di fretta ed anche alla complessità della cosa che deve fare, saltare un fosso di 3 metri non è come uno di 3 metri al buio, senza scarpe, sotto la pioggia ed inseguiti e con le tasche strapiene di monete...

Decifrare uno scritto antico potrà essere una passeggiata per un linguista esperto, ma per una "persona normale" che non ha idea di cosa può avere davanti la prova è semplicemente impossibile.

E non spaventatevi se i personaggi falliscono le prove, renderà l'avventura più interessante, e vi permetterà a voi Narratore di introdurre fatti, suggerimenti ed indizi.

\bigskip

\textbf{Quando devi fare una prova per una competenza di base in cui non sei preparato, ovvero non hai punti devi tirare solo 1d6 + punteggio della caratteristica collegata.}
Quando si scrive -1d6 significa che si tira un dado in meno (o due se è -2d6), parimente se c'è scritto +1d6 si tira un dado a 6 in più e si somma.

\bigskip

La tabella qui sotto serve a rapportare la difficoltà alla abilità minima necessaria per riuscire la prova con un tiro medio (un punteggio di 10 lanciando 3d6). Usate queste indicazione per avere una idea delle scale di difficoltà.

Il Narratore non ti dirà fammi una prova a difficoltà 10, ma dirà che la prova non presenta elementi di particolare difficoltà.

\bigskip

\textbf{Classe di Difficolta'}\index{Classe di Difficolta'}
\medskip

\begin{tabular}{lll}
	\toprule
	\textbf{Classe di Difficoltà (DC)} & \textbf{Descrizione difficolta'} & \textbf{ Competenza necessaria}\\
	DC 5               & Estremamente facile              & Mediocre\\
	DC 10              & Facile           & Normale\\
	DC 15              & Normale          & Buona\\
	DC 20              & Difficile        & Ottimo\\
	DC 25              & Molto difficile  & Eccellente\\
	DC 30              & Estremamente difficile           & Stupefacente\\
	DC 35              & Quasi impossibile& Fenomenale\\
	DC 40              & Leggendaria      & Oltre l'umano\\
\end{tabular}

\bigskip

Se devi fare una prova su una Caratteristica devi tirare 3d6 e sommare il punteggio della Caratteristica e altri bonus inerenti. Comunica questo risultato al Narratore che la confronterà con la difficoltà (DC).

\subsection{Superare o Fallire la prova di tanto...}\index{Superare o Fallire la prova di tanto}

Ogni qual volta la prova è superata brillantemente (di 10 o più rispetto alla difficoltà necessaria) il Narratore può decidere di dare maggiori informazioni, concedere bonus alle azioni successive (+1/+2).. qualsiasi cosa possa valorizzare quanto agevolmente la prova è stata superata.

Viceversa se la prova fallisce di 10 o più rispetto al valore necessario il Narratore potrebbe descrivere come miseramente la prova è fallita e come il risultato pessimo influenzi l'azione o quelle successive.

Questi modificatori aggiuntivi non si applicano alla prova del Tiro per Colpire.

\subsection{Prove opposte}\index{Prove opposte}

Ci sono situazioni in cui il personaggio deve effettuare una prova in contrapposizione con un avversario, ad esempio muoversi silenziosamente alle spalle di una guardia, rubare dalle tasche del mercante, intimidire l'orchetto per farsi dare indicazioni..

In questi casi il personaggio ed il Narratore effettuano una prova, chi ottiene il valore più alto vince, in caso di parità vince chi ha il valore più alto nella competenza, poi nella caratteristica ed infine l'eventuale "avversario".

\bigskip

\textbf{Alcuni esempi di prova contrapposta}

\begin{itemize}
	\item Ingannare qualcuno: Faccia Tosta Vs Consapevolezza
	\item Travestirsi per sembrare qualcun'altro: Intrattenere Vs Consapevolezza
	\item Creare una mappa falsa: Cultura Vs Cultura
	\item Nascondersi: Consapevolezza Vs Consapevolezza
	\item Indimidire: Faccia tosta Vs TS Volonta'
	\item Rubare: Criminalità Vs Consapevolezza
	\item Slegarsi da delle corde: Criminalità Vs Sopravvivenza
\end{itemize}

\begin{note}
Ogni qual volta la prova opposta riguarda una \textbf{caratteristica correlata ai Tiri Salvezza} (Potenza, Agilità, Volonta') per tutti gli interessati, fate fare un Tiro Salvezza come valore contrapposto. Es. una prova di braccio di ferro è un Tiro Salvezza Tempra contrapposto. Chi riesce a fare più alto con il Tiro Salvezza vince.
\end{note}


\bigskip

\subsection{Vantaggi e Svantaggi}

\begin{tcolorbox}[enhanced,arc=5pt,boxrule=0.3pt]{ Audentes fortuna iuvat ("La fortuna aiuta gli audaci", Virgilio) }\end{tcolorbox}

Il Narratore a seconda delle circostanze può concedervi un bonus od un svantaggio giudicando la prova.

\bigskip

\begin{tabular}{lll}
	\toprule
	\textbf{Vantaggio/Svantaggio} & \textbf{Valore in prove dinamiche} & \textbf{Valore in prove fisse}\\
	Bonus leggero & +1 & +1\\
	Bonus normale & +2 & +2\\
	Bonus forte   & +1d6               & +4\\
	Bonus molto forte             & +2d6               & +8\\
	Svantaggio leggero            & -1 & -1\\
	Svantaggio normale            & -2 & -2\\
	Svantaggio forte              & -1d6               & -4\\
	Svantaggio molto forte        & -2d6               & -8\\
\end{tabular}

\bigskip

Il valore nelle prove dinamiche è da usarsi quando la prova viene fatta tirando i 3d6, in questo caso si potranno sommare bonus (+2) o addirittura tirare dadi in più (+2d6) o se in svantaggio dadi in meno, fino a tirare solo 1d6 (con 2d6 di penalita')

Si intendo prove a valore fisso quando non è necessario tirare dei dadi (es. Difesa), in questo caso il punteggio si alza/abbassa del punteggio indicato.

Cercate di rimanere sempre tra questi valori di vantaggio e svantaggio, altrimenti potete dire che la prova è direttamente riuscita o fallità.
Il giocatore può comunque richiedere di effettuare la prova anche se il risultato è certo.

\textbf{Se un personaggio non è in difficoltà o pressione} nell'effettuare la prova può prendere il 10 (+ competenze + abilita..), ovvero non tirare i dadi e considerare che abbia tirato 10 con i dadi. L'azione impiega 10 round.

\textbf{Se il personaggio non ha impellenti limiti di tempo}, ovvero può dedicare almeno 10 minuti per lavorarci (100 round) può considerare di prendere 15. Ovvero come se avesse fatto la prova e tirato 15 con i 3d6.

\textbf{Se il tempo diventa un fattore da non considerare}, ovvero il personaggio ha almeno 1 ora per pensare e lavorare considerare di avere tirato 18 (ma non c'è nessuna esplosione di dadi anche se il totale è 18...)

Se vuoi prendere questi valori chiedilo al Narratore, sarà lui che ti permette o meno di usare questi punteggi, in base alla situazione, urgenza, pericolosità di ciò che ti circonda. Mettersi a scassinare una porta in un dungeon chiedendo il 10 richiede un estremo sangue freddo o incoscienza.


Prendere il 18 è fattibile solo se il personaggio non ha penalità nell'effettuare la prova.

\textbf{Aiutare un altro}:\index{Aiutare un altro} si può aiutare un amico in una prova dandogli supporto e suggerimenti. L'aiutante deve effettuare una prova a due gradini di difficoltà inferiore (-10) (esempio se il personaggio impegnato deve fare una prova a difficoltà 25 l'aiutante la fa a difficoltà 15, se ci riesce da un +2 alla prova del compagno.


Più personaggi possono aiutare lo stesso amico; i bonus di questo tipo sono cumulabili fino ad un bonus pari alla metà della difficoltà da battere (es +12 nel caso di difficoltà 25).

Il Narratore valuterà la possibilità che più di un personaggio fornisca aiuto considerando spazi, modi e tempi (non è facile aiutare qualcuno ad infilare un filo nella cruna di un ago).

\textbf{Esplosione del 6}: \index{Esplosione del 6}anche nelle prove di delle competenze base c'è l'esplosione del 6. Se con un dado fai 6 lo sommi e ritiri e continui così se fai ancora 6.

\textbf{Tirare un 1 nei check}: \index{Tirare un 1 nei check}porta male anche nei check competenze, ovvero non si somma.
Tirare un 3 nella prova di competenza (tutti 1 nei 3d6) non è un fallimento automatico è solo un tiro molto basso (zero + valore di competenza)

\subsubsection{Successo Parziale - Opzionale}\index{Successo Parziale}\index{Prova con Rischio}

Il Narratore può anche decidere di valutare una prova non riuscita come un successo parziale.

Se la prova fallisce di 1 potrà considerarsi riuscita anche se con un problema leggero, se è fallita di 2 si porta dietro un problema serio se è fallita di 4 è riuscita con un problema critico.

Ad esempio se Tups vuole slegarsi i polsi, con un fallimento di 1 avrà fatto rumore forse svegliando la guardia, con un fallimento di 2 si è slogato il pollice con un -2 alle prove di CA e prove con quella mano per un giorno.

Con un fallimento di 4 è si riuscito a liberarsi a costo di essersi slogato il polso!


\begin{note}
In questo sistema può essere anche il giocatore a richiedere una \textbf{"Prova con Rischio}" in situazioni di particolare tensione ed urgenza. Questa richiesta va fatta prima di tirare i dadi.
\end{note}

\pagebreak

\subsection{Descrizione delle Competenze di Base}

\label{descrizione-delle-competenze-di-base}
\begin{tcolorbox}[enhanced,arc=5pt,boxrule=0.3pt]{
		E quando Alessandro vide l'ampiezza dei suoi domini pianse, perché non c'erano più mondi da conquistare. Sono i vantaggi di un'istruzione classica. (Hans, Trappola di Cristallo (film), 1988)
	}\end{tcolorbox}\medskip

\textbf{Criminalità} (Agilità):\index{Criminalità}

questa competenza riguarda molte delle capacità spesso usate dai ladri. Ogni qual volta c'è da scassinare un lucchetto, liberarsi da dei legacci o manette, rubare dalle tasche di qualcuno, oppure travestirsi, mascherarsi, truccarsi per sembrare qualcun'altro, è necessario una prova di Criminalità.

Solitamente ad una prova di Criminalità è contrapposta alla prova di Consapevolezza dell'avversario.

\textbf{Sopravvivenza} (Volonta'):\index{Sopravvivenza}

questa competenza riguarda tutta una serie di capacità e conoscenze legate dall'attenzione e dalla vita selvaggia.

Ogni qual volta si debba legare un nemico, le tracce, sopravvivere nei boschi recuperando cibo ed un giaciglio,  si debba riconoscere o calmare ed addestrare animali entra in gioco la Sopravvivenza.

Anche l'abilità di pronto soccorso, per determinare malattie, prestare le prime cure o anche a lungo termine, ricade in sopravvivenza.

\textbf{Faccia Tosta} (Magnetismo):\index{Faccia Tosta}

ogni qual volta il personaggio deve interagire con un altro convincendolo, persuadendolo anche in maniera aggressiva oppure per ingannarlo si usa la competenza Faccia Tosta.

Se il personaggio vuole ingannare un avversario è una prova opposta di Faccia Tosta contro Consapevolezza (percepire inganni).

Intimidire l'avversario è una prova contrapposta a difficoltà 3d6 + livello (dadi vita) dell'avversario + Magnetismo dell'avversario.

\textbf{Acrobatica} (Agilità):\index{Acrobatica}

se devi trovare l'equilibrio su un sottile cornicione, atterrare sui piedi dopo un salto, saltare su un terrazzo, arrampicarsi su una parete sono tutte prove di Acrobatica.

Ogni qual volta agilità, precisione ed equilibrio del personaggio sono messe a prova viene richiesta una prova di Acrobatica.

\textbf{Consapevolezza}(Volonta'):\index{Consapevolezza}

la Consapevolezza è la capacità di verificare le piccole cose, di accorgersi dei piccoli mutamenti. Consapevolezza è la capacità di calarsi nell'ambiente e farlo proprio.

Ogni qual volta dovete verificare, vedere, udire, percepire qualcosa, sentire nella voce dell'avversario una incrinatura perché si sta raccontando menzogne oppure voi dovete fondervi nell'ambiente muovendosi silenziosamente o in maniera furtiva sono effettuate una prova di Consapevolezza.

\textbf{Cultura} (Intelletto):\index{Cultura}

tutto ciò che è sapere non magico è cultura. Possa essere la storia e gli avvenimenti di un continente, le tradizioni ed abitudini di una nazione. Sapere come si prepara un decotto o legge una mappa, ogni ora passata su un libro è conoscenza.

Sapere se un edificio è stato costruito bene, e come e quando può essere una prova di ingegneria e quindi Cultura.

Conoscere le feste tipiche di una divinità è una prova di religione. Conoscere i mostri tipici di un dungeon e le loro abitudini alimentari è una competenza di sopravvivenza.

Falsificare uno scritto o mappa in maniera credibile richiede Cultura

\textbf{Arcano} (Intelletto):\index{Arcano}

ricordare i dettagli su un antico oggetto magico, riconoscere uno scritto magico in una pergamena, comprendere le Essenze lanciate da altri o riconoscerne gli effetti.

Sapere la geografia astrale dei piani, sapere usare un oggetto magico intuendone le funzioni, riconoscere creauture magiche (DC è il CR della creatura modificato dalla rarità della stessa). Riconoscere o sapere formulare rituali non standard.

Ogni qual volta si ha a che fare con la magia in qualsiasi sua forma è una prova di Arcano.

\textbf{Lavoro} (Volonta'):\index{Lavoro}

qualsiasi mestiere che costruisca (Artigianato, come sarto, fabbro, apicoltore, barcaiolo.. cacciatore.. conciatore..) o fornisca servizi (Professione come architetto, avvocato.. mercante.. ) è assimilabile ad un Lavoro.

La competenza in lavoro permette di guadagnarsi da vivere e conoscere un mestiere che potrebbe rendersi utile.

Lavoro è spesso una competenza di background del personaggio, ma rimane sempre utile in tante e diverse situazioni. Se ad esempio è necessario lavorare del legno, creare una corda, esaminare un palazzo, ogni qual volta una professione può essere pertinente con una prova si applica il bonus di competenza.

\textbf{Intrattenere} (Magnetismo):\index{Intrattenere}

un personaggio che voglia cantare, suonare, recitare, travestirsi deve effettuare una prova di Intrattenere.

Maggiore è la competenza di Intrattenere maggiore sarà noto il nome del personaggio, maggiore sarà l'affluenza alle sue esibizioni.

Una prova di diplomazia fatta su Intrattenere sarà basata più sulla fisicità e sensualità che sulla capacità di argomentare.

\textbf{Resistenza} (Potenza):\index{Resistenza}

Nuotare in acque placide o burrascose, correre per chilometri , saltare un crepaccio o scalare una dirupo sono tutte prove di Resistenza.

Ogni qual volta si deve mettere a prova forza e resistenza fisica entra in campo la prova di resistenza.
Un salto che richieda potenza (per la distanza) lo si fa con una prova di Resistenza, mentre un salto di precisione o in disequilibrio si farà su Acrobatica.

Valutate oltre alla pura "forza/resistenza" anche i fattori esterni in quanto possono influire facilmente con modificatori anche significativi.

\bigskip

Come sempre ogni volta che si stabilisce una difficoltà cercate sempre di essere obiettivi e lineari, valutate ogni interferenza e fattore esterno che possa essere applicato, considerate i bonus e malus della situazione e siate sempre corretti.

\subsubsection{Esempi Prove Competenza}

\label{esempi-prove-competenza}

\begin{itemize}
	\item Una prova riuscita (base DC 15) in Sopravvivenza (pronto soccorso) può fare recuperare 1d4 PF dopo uno scontro o concedere un +2 ad un tiro salvezza su tempra per resistere ad un veleno. Costo 3 Azioni
	
	\item Una prova riuscita (base DC 15) in Sopravvivenza riduce di 1 i danni da Sanguinamento. Per ogni valore di Sanguinamento sopra 1 la difficoltà aumenta di 3, e la prova la riduce comunque di 1 alla volta. Costo 3 Azioni
\end{itemize}

\begin{itemize}
	\item Una prova riuscita (base DC 13) in Sopravvivenza (pronto soccorso) per prendersi cura per 8 ore di un paziente fa recuperare a questo il doppio dei punti ferita ((CA+Potenza)*2 con un minimo di 4) e concede un nuovo Tiro Salvezza su Tempra per resistere a Malattie o Veleni già in corso.
\end{itemize}

\begin{itemize}
	\item Saltare
\end{itemize}

\index{Saltare}\index{Salto}

\begin{tabular}{lllll}
	\cmidrule(l){1-2}
	\textbf{Salto in Lungo (Distanza)} & DC\\
	1.5 m                              & 5\\
	3 m                                & 10\\
	5 m                                & 15\\
	7 m                                & 20\\
	+1,5 m                             & +5\\
\end{tabular}
\bigskip


\begin{tabular}{ll}
	\toprule
	\textbf{	Salto in Alto (Altezza)} & DC\\
	0.02 m                           & 4\\
	0.5 m                            & 8\\
	1 m                              & 12\\
	1.5 m                            & 16\\
	+0.5 m                           & +4\\
\end{tabular}

\bigskip

\subsubsection{Linguaggi}\index{Linguaggi}

\label{linguaggi}

In Yeru ogni razza è custode di una propria lingua parlata e scritta. Ogni personaggio che abbia almeno Intelletto -1 parla il linguaggio della propria razza, con 0 lo scrive anche.
Per ogni punto superiore o pari a 2 parla e scrive un altra lingua che sarà scelta alla creazione del personaggio.
Per ogni punto in Cultura che dedica specificatamente ai linguaggi, parla e scrive un altra lingua. Alcune lingue non possono essere parlate se non da creature appartenenti a quella razza, ad esempio le lingue degli Elementali.

Un appartenente ad una razza puo' benissimo avere come prima lingua non quella della propria razza se il background lo giustifica (es Nano cresciuti presso Goblin).

\bigskip

\textbf{Tabella delle Lingue}

\medskip

\begin{tabular}{lll}
	\toprule
	\textbf{Razza}            & \textbf{Linguaggio Parlato} & \textbf{Scritto}\\
	Nanica                    & Nanico			            & Nanico\\
	Umani                     & Comune                      & Comune\\
	Elfica                    & Elfico                      & Elfico\\
	Drow                      & Elfico			            & Elfico\\
	Orco, Goblin, Gnoll       & Goblinoide                  & Goblinoide\\
	Giganti                   & Gigante                     & Gigante\\
	Uccelli senzienti         & Auran                       & Elfico\\
	Abitanti marini senzienti & Acquan                      & Elfico\\
	Abitanti dei boschi       & Silvano                     & \\
	Draghi                    & Draconico                   & Draconico\\
	Celestiale                & Celestiale                  & Celestiale\\
	Infernale                 & Infernale                   & Infernale\\
	Abissale                  & Abissale                    & Abissale\\
	Elementali del Fuoco      & Ignam                       & \\
	Elementali della Terra    & Terran                      & \\
	Elementali dell'Acqua     & Acquan                      & \\
	Elementali dell'Aria	  & Auran                       & \\
	Linguaggi Speciali*		  & Linguaggio dei Segni        & \\ 
	Linguaggi Speciali*		  & Linguaggio Druidico			& Druidico\\ 
	Linguaggi Speciali*		  & Parlata delle Profondità    & Nanico\\ 

\end{tabular}

\bigskip

\begin{note}
I \textbf{Linguaggi Speciali}* possono essere presi sono dietro autorizzazione del Narratore a seguito di background o Abilita' del personaggio.
\end{note}

La \textbf{Telepatia}\index{Telepatia} non e' propriamente una lingua ma un mezzo per parlare con qualsiasi creatura senziente (Intelletto maggiore di -3). Non c'e' il vincolo del linguaggio, la telepatia funge anche da traduttore universale.

\subsubsection{Volare}

\label{volare}

Volare è una competenza che si impara con sofferenza ed impegno. Se si vuole imparare a governare il volo è necessario dedicare specifici punti di competenza a Volare e sapere volare.
La prova di Volare si fa su Agilità.

\pagebreak

\section{Combattimento}\index{Combattimento}

\label{combattimento}
\begin{tcolorbox}[enhanced,arc=5pt,boxrule=0.3pt]{
		Si vis pacem, para bellum ("se vuoi la pace, prepara la guerra", anonimo)\\\\
		Non conta come cadi, ma se e come ti rialzi (anonimo)\\\\
		Non sono un eroe. No e non lo sarò mai. Sono solo un cattivo che viene pagato per pestare tipi peggiori di lui. (Deadpool)\\\\
		Occhio per occhio... e il mondo diventa cieco (Mahatma Gandhi, NdA i suoi Tratti aborrivano la violenza!)
	}\end{tcolorbox}\medskip

Il combattimento è tra le fasi principali di un avventura ed è quando i prodi o codardi danno sfoggio della loro maestria con le armi o Essenze.

\bigskip

Il combattimento è diviso in 2 fasi:\index{combattimento}
\begin{itemize}
	\item verifica dell'iniziativa
	\item risoluzione delle azioni (movimento, attacco, azione varie..)
\end{itemize}

\subsection{L'Iniziativa}\index{Iniziativa}

\label{liniziativa}

L'iniziativa è una prova (3d6+) di Agilità o Intelletto ed Abilità inerenti che potete avere.

Il giocatore sceglie il valore che preferisce. Se viene scelta Agilità saranno i riflessi a determinare la reazione del personaggio, mentre Intelletto guiderà la capacità di cogliere le tattiche dell'avversario ed anticiparle.

Chi ha l'iniziativa tra giocatori e nemici più alta incomincia per primo e successivamente agiscono gli altri in ordine decrescente, dichiarando le Azioni ed eseguendole.

In caso di Iniziativa di pari punteggio agisce per primo chi ha il punteggio caratteristica più alto, altrimenti lo scontro sarà in contemporanea.

L'iniziativa vale per l'intero scontro e si ritira al cambio dell'avversario.

\subsubsection{Risoluzione delle Azioni}\index{Risoluzione delle Azioni}

\begin{tcolorbox}[enhanced,arc=5pt,boxrule=0.3pt]{
...il passato è il prologo e il futuro sta nelle vostre mani e nelle mie. (Antonio, La Tempesta, Shakespeare)}
s\end{tcolorbox}\medskip


\label{risoluzione-delle-azioni}

Dal più veloce al più lento c'è la risoluzione delle azioni.

Il Narratore chiederà al più veloce di dichiarare la sua azione e agire, proseguirà poi a chiedere e fare agire gli altri giocatori e nemici.

In questo modo la scelta dell'azione avviene quando è il turno del giocatore che potrà agire anche in base alle azioni e risoluzioni già avvenute.

E' possibile ritardare la propria Azione per aspettare una determinata situazione. Il personaggio che ritarda la propria Azione agisce per primo tra i soggetti che agiscono in quel valore o successivo di iniziativa .

Se un personaggio dichiara di fare una certa Azione in conseguenza di un'altra vuol dire che ritarda la propria azione, ciò gli sarà possibile solo se ha ancora Azioni da spendere nel round ed è più veloce di colui che deve compiere l'azione che scatena l'Azione (Es. se il mago incomincia a formulare una Essenza gli tiro una freccia)

\subsection{Azioni nel Round}\index{Azioni nel Round}\index{Azioni}

\label{azioni-nel-round}

Un personaggio può eseguire 3 Azioni per round.

Queste azione possono essere eseguite nell'ordine preferito.

Nella tabella sottostante sono indicate le Azioni principali che un personaggio può eseguire, sono linee guida da seguire. Nel capitolo dedicato al combattimento vengono elencate altre Azioni ed i loro costi, in azioni, relative.

Una Azione non può essere interrotta da un altra Azione, ma può essere seguita da una Reazione o da una Azione Immediata, se nel proprio round.

Se un personaggio vuole fare più attacchi spostandosi nel campo di battaglia puo', ad esempio, usare una Azione per eseguire un attacco, usare una Azione di movimento per spostarsi fino a tutto il suo movimento a disposizione, ed usare un ultima azione di attacco per eseguire un ultimo singolo attacco, questo secondo attacco (e singolo ) conta comunque come attacco multiplo con le dovute penalità.

\medskip

\textbf{Tabella Azioni per Round}

\medskip

\begin{tabular}{ll}
	\toprule
	\textbf{Cosa si fa}                                & \textbf{Costo Azioni}\\
	Eseguire un unico attacco con armi in mischia      & 1\\
	Eseguire due o più attacchi con armi in mischia    & 2\\
	Scoccare una freccia/dardo                         & 1\\
	Scoccare due o più frecce/dardo                    & 2\\
	Lanciare una Essenza                               & 2\\
	Eseguire una Azione di Movimento (ci si sposta fino a tutto
	il proprio movimento)                              & 1\\
	Alzarsi da prono                                   & 1 \\
	Scambiare un discorso con qualcuno                 & 1\\
	Cercare qualcosa nello zaino di pronto             & 2\\
	Usare qualcosa di appena preso dallo zaino/cintura & 1\\
	Bere una pozione tenuta alla cintura               & 1\\
	Estrarre l'arma (poi rimane estratta)              & 1\\
	Imbracciare lo scudo (poi rimane imbracciato)      & 1\\
	Usare un anello/bacchetta/verga/bastone magico     & 1\\
	Scambiare poche battute con qualcuno               & 0\\
	Eseguire una prova su una competenza               & 2\\
	Nascondersi, Scattare(doppio movimento)			   & 2\\
	Mantenere la concentrazione di una Essenza         & 1\\
	Azione Immediata                                   & {*}\\
	Azione Reazione                                    & {*}\\
\end{tabular}

\smallskip

Questo elenco non è completo, prendetelo come linee guida per stabilire il peso delle decisioni dei giocatori.

\bigskip

L'ordine con cui si eseguono le Azioni non è importante se non per correlazione logica e fisica. L'Azione di Movimento può essere spezzata tra altre Azioni (movimento parziale, attacco/essenza altra azione, movimento parziale).

Un personaggio potrebbe attaccare, muoversi ed ancora attaccare, questo secondo attacco avrebbe le penalità descritte negli attacchi multipli.
\smallskip

Una Azione "\textbf{Reazione}" \index{Azione Reazione}può essere eseguita liberamente anche fuori dal proprio round. Questa Azione è solitamente dovuta ad Abilità particolari. Se non indicato diversamente una Reazione costa zero Azioni e accade immediatamente dopo la causa che la scatena.

\smallskip

Una Azione "\textbf{Immediata}" \index{Azione Immediata}può essere eseguita liberamente nel proprio round, primo o dopo la propria Azione. Se non indicato diversamente una Azione Immediata costa zero Azioni.

E' possibile se non descritto specificatamente nell'Abilità eseguire solo una Azione Immediata ed una Azione di Reazione per round.

\begin{note}
Una creatura che ha una distanza di mischia (portata) superiore all'avversario si considera che abbia un bonus di \textbf{+4 all'iniziativa} per il primo round, ovvero come se usasse un'arma lunga. Il round successivo la sua iniziativa in corpo a corpo non avrà questo vantaggio a meno che abbia mantenuto la distanza non facendosi raggiungere.
Questo bonus non si applica con le armi da lancio (gittata). 
\end{note}



\subsubsection{Il Tempo (Round, Minuti e Turni)}\index{Round}

\label{il-tempo-round-minuti-e-turni}
\begin{tcolorbox}[enhanced,arc=5pt,boxrule=0.3pt]{
"L'esitazione è la morte del vantaggio" (Magic, V.E. Schwab)}\end{tcolorbox}\medskip

Un Round dura 6 secondi circa, è un lasso di tempo normale per agire, correre, parlare.. combattere. Un Minuto sono quindi 10 round, ed un Turno dura 10 Minuti (o 100 round).

I round si usano nelle azioni di combattimento o dove la tensione deve rimanere costantemente alta ed a ogni azione corrisponde un evolversi della situazione.

\pagebreak

\subsection{Movimento}\index{Movimento}

\label{movimento}

\begin{tcolorbox}[enhanced,arc=5pt,boxrule=0.3pt]{"Un mobile più lento non può essere raggiunto da uno più rapido; giacché quello che segue deve arrivare al punto che occupava quello che è seguito e dove questo non è più (quando il secondo arriva); in tal modo il primo conserva sempre un vantaggio sul secondo" (Paradosso di Zenone)}
\end{tcolorbox}\medskip

Il movimento di un personaggio è dato dalla sua taglia e razza e da ciò che porta, dai pesi, ingombri ma anche magie ed oggetti magici.

Il movimento indica quanti metri per Azione (di Movimento) il personaggio può fare.

Una creatura o personaggio potrebbe anche decidere di spostarsi più velocemente del solito ovvero correndo o andando veloci.

In caso di \textbf{Corsa} \index{Corsa}si raddoppiano i metri percorsi (2x9 metri) per Azione (di Movimento). Per un umano (Movimento 9) significa fare 18 metri in una Azione.

\begin{itemize}
\item Chi attacca il personaggio che \textbf{corre/ha corso} ha un bonus di 1d6 al Tiro per colpire. Il personaggio che va veloce ha un malus di 2d6 nel Tiro per Colpire nel round in cui si corre.
\end{itemize}

Non è possibile spostarsi anche solo di 1 metro se non si spendono Azioni di Movimento.

Queste precisazioni hanno senso e vanno usate quando si tratta di combattere ed il dislocamento è fondamentale, durante gli spostamenti normali, mentre si cavalca o cammina liberi senza pericoli, si usa la normale gestione del movimento orario.

Nel caso di spostamento diagonale si conta una distanza di 1,5 metri per quadretto.

Per \textbf{distanza di Tocco} \index{Distanza di Tocco} \index{Tocco}si intende una distanza che permette il toccare l'avversario, quindi non più di un metro per creature di taglia media. La distanza di tocco e' anche la \textbf{distanza di mischia}.

Se non indicata nell'avversario/mostro la distanza di tocco aumenta di 0.5 metri per ogni taglia oltre la media.

Per \textbf{distanza di Mischia} \index{Dstanza di Mischia} \index{Mischia}si intende una distanza che permette il combattimento corpo a corpo (1 metro attorno al personaggio). Nei mostri questa distanza è indicata dalla portata, per le armi da lancio è chiamata gittata.

\begin{tcolorbox}[title = Esempi di Distanza in Combattimento] 
Es. per un umano armato di lancia, la distanza di mischia è 2 metri perché l'arma è lunga.

Per un nano armato di martello la distanza di mischia è 1 metro.
Per un gigante delle colline la distanza di mischia è 2 metri. Per un grande drago la distanza di mischia per il soffio (portata) è 18 metri dato che può attaccare da quella distanza facilmente, mentre per gli artigli è 3 metri.

\end{tcolorbox}

Quando si parla di "\textbf{quadretto}" \index{Quadretto}per indicare una distanza od una influenza si intende un quadretto di mappa di 1 metro x 1 metro.
Se un movimento termina in "mezzo" quadretto si arrotonda per difetto.\\

\textbf{Se ci si sposta in terreno "difficile", un umano copre 4 metri per Azione di movimento (la meta' del proprio movimento).}

L'Azione (di movimento) può avvenire prima e dopo l'Azione (di Attacco).

A distanza di mischia una creatura di dimensioni medie può avere al massimo 8 creature medie.


\subsubsection{Creature Grandi e Piccole in Combattimento {*} (Opzionale)}

\label{creature-grandi-e-piccole-in-combattimento-opzionale}

Le creature molto piccole possono stare più di una ad una distanza di mischia, mentre creature grandi tenderanno ad occupare tutto lo spazio di mischia.

\medskip

\textbf{Tabella: Taglia e Scala delle Creature}

\medskip
\begin{tabular}{ll}
	\toprule
	\textbf{Taglia della Creatura} & \textbf{Creature in distanza di mischia}\\
	Piccolissima                   & 100\\
	Minuta                         & 64\\
	Minuscola                      & 32\\
	Piccola                        & 16\\
	Media                          & 8\\
\end{tabular}

\smallskip
Questi sono i valori tipici delle creature per la taglia indicata.
Sono frequenti eccezioni.
\bigskip

\pagebreak

\subsection{Vita e Morte}

\label{vita-e-morte}
\begin{tcolorbox}[enhanced,arc=5pt,boxrule=0.3pt]{Chi non conosce la morte, non conosce la vita. (Grand Hotel, film 1932)}\end{tcolorbox}\medskip

Quando un personaggio raggiunge i 0 (zero) Punti Ferita si considera svenuto, ovvero inabile a fare qualsiasi cosa. Una Cura (Essenza, Pozione) lo portera’ cosciente ed ai punti ferita curati. Una prova di pronto soccorso (Sopravvivenza) a DC 15 lo portera’ ad 1 punto ferita.

Un personaggio morente ha Punti Ferita negativi (-1 o meno) ed è privo di sensi e prossimo alla morte. Continuera’ a perdere un punto ferita a round fiche il valore non raggiungera’ il triplo della Potenza+10 ed il personaggio morira’, se non viene curato.
Una Essenza di Cura, di qualsiasi livello di potere lo portera’ a 1 Punti Ferita successive cure funzioneranno normalmente.

Una prova di Sopravvivenza (pronto soccorso, 3 Azioni) a difficolta’ 11 piu’il valore dei punti ferita negativi portera’ il personaggio a 0 punti ferita.\\

\begin{tcolorbox}[title = Tups sta morendo] 
Es. Tups e’ gravemente ferito ed ha attualmente -6 punti ferita, Jade decide di provare a curarlo (dopo averlo spostato in un posto piu’ sicuro). Jade tenta una prova di pronto soccorso per almeno stabilizzare il compagno, la sua difficolta’ alla prova e’ 11+6 ovvero deve superare con Sopravvivenza DC 17 per riportarlo a 0 PF (svenuto)

Una successiva prova di pronto soccorso a DC 15 potra' portarlo a 1 PF.
\end{tcolorbox}


\textbf{Quando un personaggio arriva a punti ferita negativi pari 10+triplo del suo punteggio di Potenza e’ morto (-10-(Pot*3))}.

Es. Se ha Potenza 2 morira’ a -10-6=-17 PF, se ha Potenza 0 morira’ a -10 PF, se ha Potenza -2 morira’ a -10+6=-4 PF. In caso di valori di Potenza pari od inferiore a -3 il personaggio muore a -3 punti ferita.

\bigskip

Un personaggio morto non puo’ beneficiare delle cure normali o magiche, e non puo'’ essere riportato in vita da una Essenza. Solo un Patrono ha sufficiente potere per riportare l’anima nel corpo e riportare in vita la creatura. L’Essenza di Distruzione può rianimare un corpo, ma come non morto.

\subsubsection{Recupero da 0 PF*} \index{Recupero} \index{Svenuto}

\textbf{Nel caso vogliate un sistema meno letale potete applicare la regole opzionali.}

Ogni round successivo ad essere andato a 0 punti ferita, quindi svenuto, deve effettuare un Tiro Salvezza Tempra a difficolta’ 15, se riesce riprende coscienza e va ad 1 punto ferita.

Se fallisce la prova puo’ effettuarne un altra a DC +1 rispetto alla precedente il round successivo. Quanto la difficolta’ raggiunge 18 (ovvero 3 prove fallite di seguito) il personaggio incomincia a morire, va a -1 punti ferita e diventa morente.
Appena la prova riesce (entro i 3 fallimenti) il personaggio torna ad 1 punto ferita.

Eventuali punti caratteristica persi si recuperano al ritmo di 1 punto totale al giorno, se non indicati come perdita permanente.

\pagebreak

\subsection{Tiro per Colpire e Difesa}\index{Tiro per Colpire}\index{Difesa}

\label{tiro-per-colpire}
\begin{tcolorbox}[enhanced,arc=5pt,boxrule=0.3pt]{Applica sempre la giusta forza, mai troppa mai troppo poca. (Kano Jigoro)}\end{tcolorbox}\medskip

Il Tiro per Colpire è una prova contrapposta alla Difesa dell'avversario.

Se l'attaccante usa:

\begin{itemize}
	\item \textbf{Armi da Mischia o Contatto}: l'attaccante deve effettuare un \textbf{Tiro per colpire (TC)}= 3d6 + Competenza Armi + Potenza ed eventuali bonus di Abilita' o magici dell'arma o fattori circostanziali (ambiente, maledizioni..)
	
	\item
	\textbf{Armi da Distanza o Versatili}: l'attaccante deve effettuare un Tiro per Colpire (TC) = 3d6+ Competenza Armi + Agilità + Bonus vari (archi, balestre, pugnali, scimitarre...)
	
	\item
	\textbf{Essenze}: l'attaccante deve effettuare un Tiro per Colpire (TC) = 3d6+ Agilità	
\end{itemize}

Se la Distanza è tocco userà il valore della Difesa a tocco.

Chi si difende ha una \textbf{Difesa} pari a: 10 + Agilità + Scudo + Armatura + eventuali bonus magici ed Abilità e bonus circostanziali.
Il giocatore può decidere di rinunciare a del bonus dato dalla Competenza con Armi per avere un migliore punteggio di Difesa. Questi punti non saranno a disposizione nell'attacco successivo.

Ogni qual volta di parla di Bonus Difesa si intende un valore da sommare al valore Difesa ottenuto con il calcolo di cui sopra.

\subsection{La Difesa}

\label{la-difesa}
\begin{tcolorbox}[enhanced,arc=5pt,boxrule=0.3pt]{La difesa è sempre legittima (anonima vittima)}\end{tcolorbox}\medskip

Ogni Tiro per Colpire (3d6 + Competenza con Armi + Potenza o Agilità + eventuali bonus/malus) si raffronta la Difesa ovvero un valore pari a 10 + Agilità + Scudo + Armatura + eventuali bonus/malus.

Se il Tiro per Colpire è pari o superiore al valore della Difesa l'avversario è stato colpito e si stabilirà il danno della ferita, dato dall'arma + punteggio Potenza ed altri fattori quali bonus magici e di abilità (se in mischia).

Se il TC (Tiro per Colpire) e' piu' basso della Difesa allora l'avversario avrà parato, schivato, evitato.. La scelta la si lascia al giocatore (o Narratore), evitato l'attacco non subiscono ferite.

Altre situazione possono avvantaggiare la Difesa quali coperture, nascondigli, come fosse, porte, compagni di taglia molto più grande della propria. Consultate i paragrafi relativi per capire il vantaggio che possono dare.

Ci sono occasioni in cui non è importante penetrare la difesa e sferrare un colpo ma semplicemente basta toccare l'avversario.

Altre volte l'avversario è sorpreso e non può difendersi completamente.

Se il personaggio usa una Essenza, in distanza di Mischia, se non opera tramite un arma, ovvero "consegna" la magia con una spadata, si considera che sia un incantesimo a tocco.

Se è \textbf{sufficiente toccare l'avversario} la Difesa sarà 10 + Agilità + bonus magici, senza bonus Scudo e Armatura.

Se \textbf{l'avversario è sorpreso} ovvero non si aspetta l'attacco la Difesa sarà 10 + bonus magici + Armatura, senza bonus di Scudo e Agilità.

\textbf{Anche per il Tiro per Colpire valgono le Golden Rules. I d6 esplodono in caso si tiri 6 con il dado, fare 1 porta male ed affidarsi alla sorte.}

Se il Tiro per colpire (TC) è superiore od uguale alla Difesa allora hai colpito, se è inferiore hai mancato.

Se i modificatori e circostanze portano il danno inflitto ad essere negativo comunque farai 1 di danno.

Questa regola si applica ai modificatori del danno dell'arma che appunto non possono portare il danno totale ad essere inferiore a 1, se ci sono protezioni magiche o riduzioni del danno questo può diventare zero e quindi non ferirai l'avversario (ma se diventa negativo non lo curi!).

Come prima cosa, come spiegato poco prima, ricorda che per ogni 6 tirato (nei 3d6 del Tiro per Colpire) devi tirarne un altro e continuate a tirare finché continui a fare 6 con il dado.

Se colpisci, \textbf{ogni due 6 tirati} (contando quelli del Tiro per Colpire e quelli successivi scaturiti dal fatto di aver tirato 6), l'arma fa del danno in più ovvero un critico. Tira nuovamente il danno dell'arma , senza magia o Potenza o Abilità particolari ogni due 6 tirati nel Tiro per Colpire.

Puoi \textbf{togliere 4} o multipli al tuo attacco per tirare un d6 in piu'. La scelta è da fare nelle situazioni più disperate dove solo la fortuna può risolvere il duello. Il valore lo togli dal valore di CA e non di Potenza o Agilita'.

In caso si tiri un 1 nel Tiro per Colpire questo abbassa di 1 (quindi 1 non conta) il valore totale ma non influisce sul fatto di aver fatto critico o meno.

\textbf{Il fatto di tirare un critico non è garanzia di aver colpito, bisogna sempre superare la Difesa}.

Anche per il Tiro per Colpire valgono le regole base delle Competenze.
La Difesa è un valore fisso e come tale non è soggetta a modifiche
causate dalle regole base delle Competenze.

\subsection{Tirare 3 volte 1}\index{Tirare 3 volte 1}

Se nei primi 3 tiri per colpire fai tre volte 1 mancherai l'avversario (indipendentemente dal risultato finale del Tiro per Colpire) ed il Narratore potrebbe decidere brutte cose sul tuo attacco (ti cade l'arma, colpisci un amico, ti si rompe l'arma, ti ferisci, cadi, appare un demone delle fosse per caso...)

\subsection{Tirare 3 volte 6}\index{Tirare 3 volte 6}

Se nei primi 3 tiri per colpire fai tre volte 6 prenderai l'avversario indipendentemente dal risultato finale del Tiro per Colpire. Oltre ad avere la certezza di aver fatto un critico (vedi sotto) il Narratore potrebbe decidere di applicare qualche effetto descrittivo (o effettivo) ulteriore.

\subsection{Tiro Critico}\index{Tiro Critico}

Ogni qual volta hai colpito, tiri un danno aggiuntivo di arma (senza bonus magici o di Abilità o Potenza, solo arma) per ogni due volte che hai tirato 6 nel tiro per colpire.

\medskip

\begin{tcolorbox}[title = Esempio Tiro Critico] 
Es tiro 6 4 5, tiro in aggiunta 6, tiro in aggiunta un 6, tiro in aggiunta 4: come danno tiri 2 volte il danno dell'arma, una perché ho colpito una perché hai tirato tre volte 6 (se avessi tirato un ulteriore 6 sarebbero stati Arma + Potenza + bonus/abilita + 2{*}Arma).
\end{tcolorbox}

\subsection{Esplosione del Danno}\index{Esplosione del Danno}

Ogni qual volta dal tiro del dado del danno ottieni il valore massimo (nel classico d8 per la spada ad esempio fai 8 ed è quindi il valore massimo del dado), ritiri il dado e sommi ancora il valore (del solo dado).

In caso di armi con più dadi (esempio 2d4, il valore massimo deve essere ottenuto come somma dei due dadi, ovvero 8). Non c'è esplosione del danno per le armi con danno massimo inferiore o uguale a 6.

Alcune armi hanno una esplosione del danno diversa. Nella tabella delle armi dove è segnato EDX (es ED9), il valore X sta per il valore minimo sufficiente per tirare un'altra volta il danno, quindi in caso di ED9 puoi fare il critico con 9 o 10.

Questa è una caratteristica di alcune armi estremamente letali.

L'esplosione del danno non esplode a sua volta, anche se fai il massimo del dado con il dado aggiunto questo non esplode nuovamente.

Il danno aggiunto da critico, ottenuto lanciando almeno due 6, non ha il vantaggio dell'esplosione del danno. Anche se il dado dell'arma in piu' tirato grazie al Tiro Critico fa il massimo non ritiri ed aggiungi il danno.

\subsection{Tiro Critico Variante Opzionale*}\index{Tiro Critico Variante}

Il Narratore potrebbe prediligere meno il caso e gestire i critici in base alla "bravura" del personaggio di usare l'arma.
Un metodo alternativo e' quello di concedere un danno di arma aggiuntivo (solo il danno dell'arma, senza altri bonus) quando il Tiro per Colpire e' superiore di 5 (e multipli) rispetto alla Difesa dell'avversario.

\medskip

\begin{tcolorbox}[title = Tiro Critico Variante] 
Es. TC 21, la Difesa dell'avversario e' 13. Lo colpisco con un margine di 8, ovvero aggiungo 1 danno di arma in piu'
Se il TC fosse stato 26 si sarebbero aggiunti 2 danni Critici.
\end{tcolorbox}

In questa variante non si ha Tiro Critico (ottenere piu' 6 nel TC) ne l'esplosione del danno.


\subsection{Attacchi multipli in mischia}\index{Attacchi multipli}

Con una Azione il personaggio può eseguire un singolo attacco in mischia.

Con due Azioni il personaggio può effettuare più tiri per colpire consecutivi.

La prima azione di attacco non ha malus mentre la seconda azione di attacco ha -5 al colpire cumulativo per attacco.

Se ho CA 5 e Potenza 2, il primo Tiro per Colpire sara' 3d6+7, il secondo sara' 3d6+2. Non e' possibile effettuare un terzo attacco in quanto il bonus al colpire diventerebbe negativo.
Si conta solo CA e Potenza/Agilita' e bonus dati da Abilita' ma nessun bonus magico dell'arma si conta per capire quanti attacchi e' possibile fare.

Se il malus al colpire cumulativo diventa -1 o peggio non e' piu' possibile fare ulteriori attacchi.

Nel caso il personaggio voglia eseguire attacchi multipli, deve dichiarare se fare gli attacchi su un solo avversario o su più.

Se dopo il primo attacco il target muore (in caso di azione di multiattacco) non puoi dirigere gli attacchi rimanenti su altri bersagli tranne se ha l'abilità Proseguire e il successivo avversario è già in mischia con te.

Diversamente può dichiarare di effettuare il primo attacco su una creatura ed il secondo (o successivi) su altro, purché in mischia con il personaggio.

\subsection{Armi da Tiro - Archi - Balestre (Arco / Balestre / Pugnali..)}

Il numero di attacchi massimi con armi da lancio è 2 usando due Azioni, il secondo lancio si considera come un secondo attacco con un -5 al colpire.

Solo particolari Abilità permettono di effettuare ulteriori attacchi.

Il bonus al danno dato da Potenza si applica in automatico per le fionde, Pugnali..ovvero con tutte le armi che vengono lanciate "a mano", gli archi applicano questo bonus solo se sono di tipo composito, le balestre non lo applicano mai.

\subsection{Attacchi con armi a spargimento (olio incendiato/acqua benedetta..)} \index{Armi a spargimento}

In caso l'attacco manchi tirare un d8 e consultare questo schemino per capire dove la boccia è caduta:

1 2 3

4 \textbf{X} 5

6 7 8

\textbf{X} si considera il bersaglio del tiro.

Se il tiro manca di 5 o più tirare un 2d6 per determinare lungo la direzione indicata dal d8 precedente a quanti metri è caduto distante dal bersaglio, ovvero contate i metri dal target.

Ad esempio con il tiro del d8 faccio 5 e poi tirando 2d6 faccio 4, significa che la boccetta è caduta a destra del bersaglio a 4 metri.

è anche possibile che ci si sia tirati sui piedi la boccetta (es faccio 7 e poi 6.. potrei averla tirata addosso ad un compagno o dietro di me!).

\subsection{Impreparato -- Colti di Sorpresa}\index{Impreparato}\index{Sorpresa}

Se i personaggi vengono colti di sorpresa, ovvero non si aspettano di essere attaccati, si deve considerare questo primo round come round di sorpresa. Quando si è sorpresi non si può usare la propria Agilità o Scudo in Difesa.

Per quel round e per quell'attacco ti difenderai solo con la Armatura (senza scudo), non potrai reagire; dal round successivo potrai dichiarare l'iniziativa ed agire normalmente. Le medesime considerazioni valgono per gli avversari.

Per valutare se un personaggio è sorpreso effettuate un tiro salvezza su Riflessi, confrontandolo con la bravura nel nascondersi degli avversari, se la prova è fallita il personaggio è sorpreso.

Quando personaggi e nemici sono colti entrambi di sorpresa per valutare chi effettivamente è sorpreso effettuate un Tiro Salvezza su Riflessi, chi ottiene il risultato più basso è sorpreso e per quel round non potrà agire.

\subsection{Modificatori di attacco o difesa per situazioni particolari} \index{Situazioni particolari}

Il migliore suggerimento che si può dare nel gestire le situazioni di combattimento più caotiche è pensare a queste come ad un film, valutate la cinematicità della situazione.

Non è una questione di miniature, spazi, quadretti.. E' una questione di divertimento e visualizzazione della scena. Soluzioni non ortodosse per situazioni non ortodosse.

Concedete un d6 di bonus o malus (ovvero togliete un d6 dal numero di d6 da tirare) ogni qual volta il giocatore abbia un vantaggio o svantaggio ed allo stesso modo all'avversario. Se il bonus o malus è sulla Difesa allora usate +-2 (o +-4 se il bonus è maggiore) di bonus al posto del d6.

\bigskip

\textbf{Esempi in situazione di Attacco (bonus o malus al Tiro per Colpire)}

\begin{itemize}
	\item Situazione con +2 bonus: più di uno ad attaccare un avversario

	\item Situazioni con 1d6 bonus: posizione sopraelevata, carica, invisibile

	\item Situazione con 1d6 di svantaggio: sei abbagliato, sei intralciato, sei prono, sei ristretto nei movimenti, sei spaventato o scosso, usare un arma da lancio contro un avversario in mischia, attaccare con arma lunga in distanza da mischia
\end{itemize}

\textbf{Esempi in situazione di Difesa:}

\begin{itemize}
	\item Situazioni con +2/+4 bonus (bonus alla Difesa): hai copertura (vedi sotto),

	\item Situazioni con -4 di svantaggio (malus alla Difesa): sei accecato, immobilizzato, sei in ginocchio o seduto, sei prono, sei ristretto in uno spazio, sei stordito, lanci una Essenza
\end{itemize}

\textbf{Quando si scrive -1d6 significa che si tira un dado in meno (o due se è -2d6), parimente se c'è scritto +1d6 si tira un dado a 6 in più e si somma}.

\textbf{Quando il malus è alla Difesa considerare ogni -1d6 come un -4 alla Difesa}.

Se non si vuole tirare il dado di bonus/malus considerare allora ogni d6 come valore +-4 (a seconda che sia un bonus od un malus).

\textbf{In linea di principio in combattimento un bonus leggero è un +2, un bonus medio è +1d6 (o +4), un bonus molto alto è +2d6 (o +8), viceversa per per i malus}.

\bigskip
Ricordate sempre lo scopo è divertirsi, a scapito (per il Narratore) di qualche mostro, non siate rigidi ma dinamici e adattatevi alle situazioni.

\subsection{Azioni particolari in combattimento:} \label{sec:Azioni particolari in combattimento}

\subsubsection{Attacco a mani nude} \index{Pugno}\index{Calci} \index{Fare a botte} due armi che non mancheranno mai a nessuno sono i propri pugni e calci.
Se non si ha preso la lista d'armi "Pugno nudo" un pugno o calcio fara' 1d2 + Potenza di danno non letale.
Solo con la lista d'armi si diventa artisti marziali e le "armi" diventano Versatili (si puo' applicare come danno il valore di Potenza o Agilità) ed il danno diventa letale.

\subsubsection{Attacco di Opportunita'} \index{Opportunità}se un soggetto usando una Azione di movimento esce o attraversa, comunque termina il suo movimento fuori dalla zona di mischia del personaggio, al personaggio è concesso un singolo attacco. Questo attacco è una Reazione che costa una Azione. Stessa cosa vale per gli avversari.

\subsubsection{Carica} \index{Carica}l'avversario deve essere ad una distanza entro 2 Azione di movimento (18 o 12 metri). Si deve correre fino ad essere a distanza di mischia.

Si ottiene un +1d6 a Tiro per Colpire, -4 alla Difesa, l'attacco successivo al primo ha un -15 al colpire (questo per valutare se è possibile farlo o meno). La carica e attacco costa 3 Azioni. Non si considerano altri malus per avere corso oltre quelli indicati.

\subsubsection{Controcarica}\index{Controcarica} un'arma con il talento controcarica se usata contro un avversario/cavalcatura in carica infligge il doppio del danno dell'arma e colpisce per prima, tranne in cui l'avversario abbia un arma lunga e sia in carica in questo caso l'attacco è contemporaneo. Preparare un'arma per la controcarica è una Reazione che costa una Azione.

Se l'avversario attacca con un arma lunga senza caricare non si usa l'azione di controcarica, ma si tirano le rispettive iniziative.

\subsubsection{Carica con Arma da Controcarica} \index{Controcarica}se usi un arma con il talento controcarica per caricare un avversario la tua arma fa il doppio del danno dell'arma, ovvero tiri due volte il dado dell'arma, senza contare nel secondo tiro i bonus magici eventuali. Se il difensore non ha un arma lunga allora valgono anche le considerazioni di Arma Lunga.

\subsubsection{Aiutare un altro}\index{Aiutare} Si può aiutare un amico ad attaccare o a difendersi negli scontri in mischia, distraendo o interferendo con un avversario.

si può portare un attacco in mischia (1 Azione) contro un avversario che ha già ingaggiato battaglia con un proprio alleato.

Si effettua un Tiro per Colpire contro Difesa dell'avversario con 1d6 di bonus. Se l'attacco va a segno, non si fa danno, l'amico ottiene bonus di +1d6 al Tiro per Colpire con il prossimo attacco (entro la fine del successivo round) verso quell'avversario o un bonus di +4 alla Difesa contro il prossimo attacco di quell'avversario (a propria scelta).

Più personaggi possono aiutare lo stesso alleato; i bonus di questo tipo sono cumulabili (massimo 4 su taglia media), purché l'avversario sia circondato.

\subsubsection{Colpo di Grazia} \index{Colpo di Grazia}costa 3 Azioni, si può utilizzare un'arma da mischia per infliggere un colpo di grazia ad un target indifeso. Si può anche usare un arco o una balestra, l'importante è che si sia adiacente al bersaglio.

L'attaccante colpisce automaticamente ed infligge due colpi critici (due volte in più il danno dell'arma e Potenza). Se il difensore sopravvive al danno, deve superare un Tiro Salvezza su Tempra DC pari ai danni inflitti o muore.

Le creature immuni ai colpi critici, non subiscono danni critici, né devono superare un Tiro Salvezza su Tempra per evitare di essere uccisi da un Colpo di Grazia.

\subsubsection{Danno non letale}\index{Danno non letale} il danno non letale è una forma di danno causato da armi particolari o quando volutamente lo scopo è fare svenire il nemico e non ucciderlo.

Il danno non letale si tratta come il danno normale ma va segnato a parte nella scheda.

\subsubsection{Senza Competenza}\index{Senza Competenza} (Arma) usare un'Arma senza l'adeguata competenza impone un -2d6 al Tiro per Colpire (quindi il TC diventa 1d6+Potenza+abilità..)

\subsubsection{Armi Leggere} \index{Armi Leggere}Il giocatore può usare queste armi come armi secondarie senza subire penalità.

\subsubsection{Armi Versatili} \index{Armi Versatili}Il giocatore può liberamente usare la Agilità invece della Potenza sui tiri per colpire e danno con armi versatili.

\subsubsection{Lanciare armi} \index{Lanciare armi}una spada o comunque un arma non fatta per essere lanciata può comunque essere scagliata contro l'avversario.
Il Tiro per Colpire prende un -1d6 e l'arma fa una categoria di danno inferiore (la spada lunga fa 1d6, una spada corta 1d4..). La gittata di lancio è 3 metri.

\subsubsection{Colpi Potenti}\index{Colpi Potenti} il giocatore può liberamente aggiungere un +2 al danno togliendo 1 al Competenza Armi (requisito Competenza Armi +1). Non si può togliere più di CA/4 al Tiro per Colpire.

\subsubsection{Maestria del combattimento} \index{Maestria del combattimento}il giocatore può liberamente aggiungere +4 alla Difesa per ogni -1d6 al Competenza Armi. Il bonus è applicabile solo per gli attacchi in mischia.

Viceversa può prendere un -4 Difesa per alzare di +1d6 il Tiro per Colpire e quindi migliorare l'attacco. Non è possibile assegnare in questa maniera più di 2d6.

\subsubsection{Danno non letale con arma non idonea} \index{Danno non letale con arma non idonea}se vuoi fare danno non letale con un'arma non predisposta al danno non letale hai un -1d6 al Tiro per Colpire.

\subsubsection{Fiancheggiare} \index{Fiancheggiare}se due personaggi sono attorno allo stesso bersaglio ma non sono a fianco entrambi prendono +2 al Tiro per Colpire o alla Difesa (a loro scelta quale bonus prendere).

Al massimo ci possono essere 4 personaggi attorno ad una creatura di taglia media che prendono il bonus di fiancheggiare. Il tipo di bonus si sceglie round per round, se non dichiarato vale come +2 al Tiro per Colpire.

Se tirando una ipotetica riga che collega i due personaggi questa attraversa in pieno il quadretto dell'avversario allora c'è la situazione di fiancheggiamento.

\bigskip

Esempio di fiancheggiamento

\medskip

\begin{tabularx}{0.95\textwidth}{XXX}
	\toprule
	Personaggio A & Personaggio G & Personaggio D\\
	Personaggio B & Avversario    & Personaggio E\\
	Personaggio C & Personaggio H & Personaggio F\\
\end{tabularx}

\bigskip

In questo schema il fiancheggiamento è preso dalle coppie: A-F, B-E, C-D, G-H

\bigskip

Se la creatura può fronteggiare più creature contemporaneamente queste non godranno del bonus di fiancheggiamento.

\subsubsection{Arma Doppia} \index{Arma Doppia}un'arma doppia è un'arma che è pericolosa da entrambe le estremità. può essere usata come arma singola, oppure, incorrendo nelle penalità del combattimento con due armi, come appunto due armi. Se non specificato un'arma doppia usata per combattimento con due armi equivale ad usare due armi medie.

\subsubsection{Arma Lunga} \index{Arma Lunga}l'arma lunga da diritto a colpire più lontano ovvero entro 2 metri. Concede un bonus all'iniziativa pari a +4. Questo bonus rimane valido finché l'avversario non entra in distanza di mischia (ovvero entro 1 metro).

Nel caso in cui anche l'avversario abbia un arma lunga non considerare il bonus di 4 all'iniziativa (avendolo entrambi si annulla a vicenda).

\begin{tcolorbox}[title = Esempio di Combattimento con Arma Lunga] 
Es. Tups armato di spada lunga affronta uno brigante armato di lancia lunga. Tups ha iniziativa 5, il brigante 2 ma ha un arma lunga e quindi la sua iniziativa è 6.

Il brigante attacca Tups mentre questo si avvicina, il valore di iniziativa mi "mostra" come il brigante sfruttando la sua arma lunga riesca ad agire prima di Tups. Una volta che Tups si è avvicinato in mischia sarà più veloce del brigante.

Se il brigante avesse avuto iniziativa 0 Tups avrebbe attaccato per primo, praticamente il brigante non sarebbe riuscito a sfruttare il vantaggio dato dell'arma lunga (5 contro 4 di iniziativa).

Se il brigante avesse dichiarato di attaccare e poi allontanarsi avrebbe costretto Tups ad usare due azioni di movimento per raggiungerlo.

Il brigante avrebbe attaccato da distanza di 2 metri ed usato una Azione per andare in distanza di 11 metri (2+9 di velocita'). Tups avanzando normalmente non avrebbe raggiunto il brigante solo facendo una doppia Azione riesce ad andare in mischia

Tups potrebbe correre, usando quindi solo un'Azione (9mx3=27 metri) e poi attaccare ma avrebbe un -2d6 al Tiro per Colpire.

Il brigante una volta raggiunto da Tups butta l'arma lunga a terra per non avere -1d6 di penalità al colpire ed estrae un pugnale.
\end{tcolorbox}

\subsubsection{Arma lunga a breve distanza} \index{Arma lunga a breve distanza}è possibile usare un'arma lunga a distanza di 1 metro (mischia) con un -1d6 al Tiro per Colpire.

\subsubsection{Magia in combattimento}\index{Magia in combattimento} l'incantatore che lancia una Essenza mentre è in combattimento prende un -4 alla Difesa. Se viene colpito prima di formulare l'Essenza deve fare una prova di concentrazione per mantenere l'incantesimo.

\subsubsection{Prova di Concentrazione} \index{Prova di Concentrazione}quando un incantatore vuole usare una Essenza ma è severamente disturbato o ferito durante il lancio deve effettuare una prova di concentrazione per capire se riesce a lanciare la magia. Vedi capitolo Magia.

\subsubsection{Uscire da una zona minacciata} \index{Uscire da una zona minacciata}Il nemico potrebbe colpirti mentre la attraversi se ha ancora azioni disponibili. Uscire dal combattimento ed entrare in zona di mischia costa un movimento.

\subsubsection{Preparare una arma lunga contro una carica} \index{Preparare una arma lunga contro una carica}è una Reazione che costa una Azione.

\subsubsection{Prendere la Mira (cecchino)} \index{Prendere la Mira (cecchino)}per ogni round in cui prendi solo la mira, 2 Azioni, guadagni un +1 al Tiro per Colpire, fino ad un massimo di +3 alla fine del terzo round, quando puoi scagliare la freccia (o dardo o pugnale..), oppure, tirato l'iniziativa, nel round 4 con un bonus sempre di +3.

\subsubsection{Alzarsi da prono}\index{Alzarsi da prono} costa due Azione. Prendi un -4 alla Difesa ed un -4 Iniziativa. Una prova di Acrobatica con difficoltà 15 ti permette di dimezzare questi malus e costa una sola Azione alzarsi. Con difficoltà 20 annulli i malus e costa 1 Azione alzarsi.

\subsubsection{Combattimento con due armi}\index{Combattimento con due armi} ll combattimento con due armi è possibile solo se l'arma secondaria è leggera o si usa un arma doppia.

Non serve usare Azioni per attaccare con l'arma secondaria, sull'arma secondaria non applichi il danno dato dalla Potenza.

I tiri per colpire di entrambe le armi hanno un -4.

Con l'arma secondaria se non si hanno abilità specifiche si fa solo 1 attacco.

\subsubsection{Usare un'arma da lancio sotto minaccia} \index{Usare un'arma da lancio sotto minaccia}usare un'arma da lancio come arco, balestra o pugnale (che si vuole lanciare) mentre si combatte in mischia impone la negazione del bonus della Agilità alla Difesa ed il Tiro per Colpire ha un -4.

\subsubsection{Usare un'arma da lancio mirando ad un avversario impegnato
in combattimento}\index{Usare un'arma da lancio mirando ad un avversario impegnato in combattimento} non è facile prendere la mira corretta e non prendere il proprio compagno.
Hai un -1d6 al Tiro per Colpire. Il bonus si annulla se c'è una differenza di 2 o più taglie tra avversario e compagno.

\subsubsection{Usare un'arma con due mani} \index{Usare un'arma con due mani}un'arma non leggera se usata a due mani permette di applicare una volta e mezza il danno dovuto dalla Potenza.

\subsubsection{Difesa totale} \index{Difesa totale}costa 2 Azioni. Non fai nessun attacco o magia, puoi fare solo una Azione e guadagni un +8 in Difesa. Non causi Attacchi di Opportunità se attraversi la zona di mischia di un avversario.

\subsubsection{Disingaggiare} \index{Disingaggiare}costa 2 Azioni e ti sposti di 3 metri. Un'avversario ti può colpire se ha una iniziativa migliore della tua o ti insegue (ed è veloce quanto te). Non causi Attacchi di Opportunità se attraversi la zona di mischia di un avversario. Puo' essere usata per uscire dalla mischia e poi muoversi od effettuare un attacco utilizzando l'Azione rimasta.
Se per disingaggiare l'avversario e' necessario spostarsi di piu' di 3 metri, a causa della grande portata dell'avversario, e' necessario usare anche un'Azione di Movimento.

\subsubsection{Mettersi sulla difensiva} \index{Mettersi sulla difensiva}prendi un bonus di +4 alla Difesa, il tuo Tiro per Colpire ha una penalità di -1d6

\subsubsection{Disarmare*}\index{Disarmare} fai una prova contrapposta Competenza Armi + Agilità (chi disarma) contro Competenza Armi + Potenza (chi viene disarmato)

Un'arma a due mani concede un bonus di +4, un'arma leggera un malus di -2 a chi deve essere disarmato. Se si fallisce di 5 o più hai disarmato te stesso e non l'avversario. Costa 1 Azione.

\subsubsection{Finta*} \index{Finta}fai una prova contrapposta di Competenza Armi + Faccia Tosta (chi fa la finta) contro Competenza Armi + Consapevolezza (chi subisce la finta). Se la prova riesce l'avversario perde il bonus della Agilità alla Difesa fino alla fine del round successivo.

Se fallisci di 5 o più perdi tu il round prossimo il bonus di Agilità.
Costa 1 Azione.

\subsubsection{Spingere un avversario*} \index{Spingere un avversario}è una prova contrapposta di Potenza (TS Tempra contrapposto).

Se vinci spingi l'avversario fino a 0.5 metri nella direzioni che vuoi per successo nella prova (al massimo del tuo movimento), altrimenti l'avversario ti spinge nella direzione che vuole fino a 0.5 metri per successo ottenuto (se vinci la prova di 7 sposti l'avversario fino a 3.5 metri). La massima distanza e' di 5 metri.
Costa due azioni.

\subsubsection{Afferrare un avversario*}\index{Afferrare un avversario} è una prova contrapposta di Potenza (TS Tempra). Chi ha una taglia maggiore guadagna un bonus di +2 per taglia di differenza.

Costa 2 Azioni fare e mantenere e liberarsi dalla presa. Si considera che chi afferra (o è afferrato) abbia almeno una mano occupata nell'afferrare.

I due contendenti perdono il bonus di Agilità alla Difesa.

Finche' l'avversario e' afferato il terreno si considera difficile.

Puoi attaccare l'avversario afferrato con un arma corta.

L'avversario afferrato puo' usare 1 azione per cercare di liberarsi. E' sempre una prova di Potenza contrapposta.

\subsubsection{Fare cadere un avversario} \index{Fare cadere un avversari}è una prova contrapposta di Potenza o Agilità, ogni contendente sceglie quella che preferisce.

Ognuno fa un Tiro Salvezza su Tempra o Riflessi e si confrontano i risultati, se la prova e' inferiore, rispetto all'avversario, di 5 o più è chi voleva fare cadere che cade.

Per ogni gamba/zampa oltre la seconda che ha l'avversario questo ha un bonus di +2 alla prova. 

Costa 2 Azioni. L'avversario se fallisce la prova diventa prono.

\subsubsection{Modificare le proprie dimensioni*}\index{Modificare le proprie dimensioni}

Nel caso il personaggio \index{Modificare le dimensioni} la sua Difesa cambia di conseguenza

\bigskip

\begin{tabular}{ll}
	\toprule
	\textbf{Nuova Taglia} & \textbf{Modificatore alla Difesa}\\
	Piccolissima          & +8\\
	Minuta                & +4\\
	Minuscola             & +2\\
	Piccola               & +1\\
	Media                 & +0\\
	Grande                & -1\\
	Enorme                & -2\\
	Mastodontica          & -4\\
	Colossale             & -8\\
\end{tabular}

\bigskip

\textbf{{*} Le azioni marcate con {*} sono opzionali e concesse a
	discrezione del Narratore}.
\pagebreak

\section{Nascondigli e coperture}\index{Nascondigli}\index{Coperture}

\label{nascondigli-e-coperture}
Non sempre l'avversario si palesa davanti a noi, spesso questo puo' essere nascosto se non addirittura invisibile.
Potrebbe essere nascosto dietro un muretto o dei barili, se non dietro un grosso e gigantesco famiglio.
E se fosse alle nostre spalle e neanche l'abbiamo visto ?

L'obiettivo in questo caso si dice che abbia copertura o invisibilità. Questa copertura può essere leggera, media e completa.

Se l'obiettivo ha più della metà (ma non totale) della superficie "esposta" allora la copertura si definisce \textbf{leggera}, ovvero ha +2 alla Difesa.

Se l'obiettivo ha meno della metà (ma non completamente) della superficie "esposta" allora la copertura si definisce \textbf{media}, ovvero ha +4 alla Difesa.

Se l'obiettivo si sa dove è ma si nasconde completamente affacciandosi solo per controllare i personaggi o tirare una freccia ogni tanto, dietro ad un muro, finestra, porta, una creatura più grande di lui (almeno 2 taglie).. allora la copertura si definisce \textbf{completa}, ovvero ha +8 alla Difesa.

Metà del Bonus di copertura si applica anche ai Tiri Salvezza contro Essenze che abbiano un effetto ad area (es. Fuoco Palle che esplodono intorno..).

Se un avversario è invisibile allora si seguono le regole della Invisibilità.

\subsection{Invisibilita'}\index{Invisibilita'}

\label{invisibilita}

Anche se si e' invisibili non e' detto che non si possa essere percepiti diversamente attraverso altri sensi, come l'olfatto, l'udito o il tatto.

L'invisibilità rende una creatura non individuabile tramite la vista ma non rende di per sé una creatura immune ai Tiri Critici o Esplosioni del Danno.

Una creatura accecata, o che combatte contro una creatura invisibile, può effettuare una prova di Consapevolezza a difficoltà 20 (oppure 10 + prova di Consapevolezza dell'avversario se questo di nasconde attivamente) per individuare la creatura purché questa sia entro un raggio di 3 metri dal personaggio.

L'osservatore ha la sensazione che "ci sia qualcosa" ma non può vederlo o prenderlo di mira in modo accurato con un attacco.

E' praticamente impossibile (DC 30) determinare la posizione esatta di una creatura invisibile con una prova di Consapevolezza.

Una creatura invisibile oggetto di un attacco specifico nel "suo quadretto", ovvero il giocatore decide di colpire un quadretto a caso, se non prima individuata ha un vantaggio alla Difesa come se avesse \textbf{Copertura media} (+4 Difesa).

Una creatura invisibile ha un bonus di +1d6 al colpire contro creature che non lo vedono, anche se la sua posizione viene determinata (prova di Consapevolezza DC 30 riuscita).

Ci sono molti modificatori che possono essere applicati a questa DC ad esempio se la creatura invisibile si sta muovendo o sta compiendo un'attività rumorosa.

\bigskip

\textbf{Tabella Modificatori Consapevolezza per Rilevare Creature Invisibili}

\medskip

\begin{tabular}{ll}
	\toprule
	\textbf{La Creatura Invisibile sta...} & \textbf{Consapevolezza}\\
	Muovendosi a velocità dimezzata        & -5\\
	Muovendosi a piena velocità            & -10\\
	Correndo o caricando                   & -20\\
	Usando Muoversi Silenziosamente        & Prova di Furtività (Consapevolezza) +10\\
	Ferma                                  & +20\\
	A qualche metro di distanza (3 metri)  & +1. +2 per ogni 3 metri oltre\\
	Dietro un ostacolo (porta)             & +5\\
	Dietro un ostacolo (parete di pietra)  & +20\\
\end{tabular}

\bigskip

Una creatura particolarmente grossa e lenta potrebbe godere di una probabilità inferiore di essere mancata.

Se un personaggio invisibile raccoglie un oggetto visibile, l'oggetto resta visibile. Una creatura invisibile può raccogliere un piccolo oggetto visibile e nasconderselo addosso (mettendolo in una tasca o sotto il mantello, chiudendolo nel pugno) e renderlo effettivamente invisibile.

Uno potrebbe spargere su un oggetto invisibile della farina per tenere traccia almeno della sua posizione (finché la farina non cade del tutto o viene soffiata via).

Le creature invisibili lasciano impronte. Le loro tracce possono essere seguite senza problemi. Impronte su sabbia, fango o altre superfici soffici possono dare ai nemici indicazioni sulla posizione della creatura invisibile, riducendo il loro bonus di Difesa a +4 (come se fossero stati individuati).

Una creatura invisibile nell'acqua muove il liquido, rivelando la propria posizione. La creatura invisibile rimane comunque difficile da vedere e gode dei benefici di una copertura leggera (+2 alla Difesa).
Una torcia accesa invisibile emana comunque luce (così come un oggetto invisibile soggetto ad una Essenza di Illusione di luce o un altra Essenza simile).

Le creature invisibili non possono utilizzare gli attacchi con lo sguardo. L'invisibilità non influisce sulla Essenza di Rivelazione.

\pagebreak

\section{Lista Armi per Tipologia Omogenea}\index{Lista Armi}\index{Tipologia Omogenea}

\label{lista-armi-per-tipologia-omogenea}
\begin{tcolorbox}[enhanced,arc=5pt,boxrule=0.3pt]{La forza non risiede in una Spada, ma nelle braccia di un valoroso. (The Legend of Zelda: Twilight Princess)}\end{tcolorbox}\medskip

Ogni qual volta si assegna un punto a Competenza Armi si può decidere se continuare a perfezionarsi in una Lista di Armi o prenderne una nuova.

Nella scheda segnatevi quando assegnate un punto in Competenza Armi come questo viene usato.
Per riassegnare un punto di CA sono necessari almeno 4 ore di allenamento per 4 mesi.

Ricordo che usare un arma senza l'adeguata competenza impone un -2d6 al Tiro per Colpire.
I punti assegnati in una Lista d'Arma non si sommano al TC, bisogna verificare il punteggio nella lista d'arma con gli eventuali bonus che la stessa lista elenca.

\subsection{Armi Leggere}\index{Armi Leggere}Pugnale, Spada Corta, Mazza leggera, Martello Leggero, Stocco, Scimitarra

Puoi usare Agilità al posto di Potenza con queste armi per il Tiro per Colpire.

\begin{itemize}
	\item 4 punti: Aumenti di un grado il dado di danno dell'arma (d4 - d6 - d8 - d10 - 2d6 - 2d8 - 2d10 - 3d6..)

	\item 8 punti: +2 Tiro per Colpire

	\item 12 punti: la tua arma acquista EDX anche con 6 di danno massimo
\end{itemize}

\subsection{Asce}\index{Asce} Ascia ad una mano, Ascia da battaglia, Ascia Martello, Grande Ascia Doppia

\begin{itemize}

	\item 4 punti: La furia dei tuoi attacchi è tale che guadagni un +2 al danno

	\item 8 punti: Le ferite che provochi sono cosi profonde che provochi sanguinamento. Il primo attacco del round se andato a segna causa 1d4 di danno extra da sanguinamento. Il danno non si applica il round successivo.

	\item 12 punti: Le asce nelle tue mani abbattono nemici cosi come abbattono degli arbusti. Puoi sacrificare 5 al Tiro per Colpire ma aumentare il bonus alla forza a x2 per le armi a 1 mano e a x3 per quelle a due. Il bonus non si cumula con usare armi a due mani, è alternativo.

\end{itemize}

\subsection{Rompi Cranio} \index{Rompi Cranio}Randello, Mazza Leggera, Mazza Pesante, Morningstar, Martello Leggero, Flagello, Martello da guerra, Grosso randello, Flagello Pesante

\begin{itemize}
	\item 4 punti: Sei diventato cosi abile che puoi controllare la forza dei tuo colpi,puoi fare danno non letale senza malus al colpire (altrimenti -4). Puoi scegliere di ridurre di 4 il Tiro per Colpire per aumentare il danno di 4

	\item 8 punti: I tuo colpi frastornano il nemico. Se anche solo un tuo attacco va a segno l'avversario deve fare un Tiro Salvezza Tempra (DC 10+CA) se fallisce subirà -2 Iniziativa ed -2 Difesa per la prossima azione

	\item 12 punti: Aumenti di un grado il dado di danno dell'arma (d4 - d6 - d8 - d10 - 2d6 -2d8 - 2d10)
\end{itemize}

\subsection{Archi} \index{Archi}Fionda, Arco Lungo, Arco Corto, Arco Lungo Composito, Arco Corto Composito

\begin{itemize}

	\item 4 punti: Aggiungi il valore di Agilità al danno, indipendentemente dal tipo di arco. Questo bonus non e' cumulabile ed e' alternativo al danno causato da Potenza per gli archi compositi.

	\item 8 punti: La tua maestria nell'utilizzo dell'arco in combattimento è tale che non subisci nessuna penalità nel lanciare frecce a nemici in mischia o con copertura pari o minore di leggera.

	\item 12 punti: Scagli una freccia in piu' con un malus di -5 al TC

\end{itemize}

L'aggiungere il bonus di Agilità al danno non si somma se si applica un danno da Potenza ( in caso di archi compositi), devi scegliere che bonus applicare

\subsection{Balestre}\index{Balestre}Balestra leggera, Balestra pesante, Balestra ad una mano, Balestra leggera a ripetizione, Balestra pesante a ripetizione

\begin{itemize}

	\item 4 punti: Guadagni l'abilità Ricarica rapida

	\item 8 punti: La tua maestria nell'utilizzo delle balestre in combattimento è tale che non subisci nessuna penalità nel lanciare frecce a nemici in mischia o con copertura pari o minore di leggera

	\item 12 punti: La tua Potenza della tua arma unita alla tua esperienza e precisione sono armi mortali. Puoi decidere di prendere la mira su un nemico per un round, se quel nemico è ancora colpibile nel turno successivo scagli la freccia che se colpisce il bersaglio farà il cinque volte il danno.

\end{itemize}

\subsection{Armi doppie} \index{Armi doppie}Bastone, Grande Ascia Doppia, Flagello Doppio, Spada a due lame, Urgrosh

\begin{itemize}
	\item 4 punti: La tua competenza nell'uso di queste armi ti rende estremamente versatile dandoti la possibilità a inizio del tuo turno di scegliere se essere difensivo o offensivo aumentando di 2 il Tiro per Colpire o la Difesa. Non costa azioni.

	\item 8 punti: La tua tecnica è imprevedibile per l'avversario puoi scegliere se avere un +1d6 di danno con tutti i tuoi attacchi o eseguire un attacco extra questo turno

	\item 12 punti: La tua maestria è tale che l'avversario vede 3 lame. Per ogni attacco con la mano primaria puoi eseguire due attacchi extra senza bonus di danno alcuno.
\end{itemize}

\subsection{Armi da carceriere} \index{Armi da carceriere} Flagello, Flagello Pesante, Flagello Doppio, Frusta

\begin{itemize}
	\item 4 punti: Nati come attrezzi da contadino e strumenti di lavoro nelle tue mani dispensano morte e sofferenza, guadagni +2 danno

	\item 8 punti: La tua capacità di infliggere dolore con le tue armi è terrificante. Dopo una tua azione di attacco se hai fatto un critico, tutti i nemici che ti possono vedere devono superare un Tiro Salvezza Arbitrio DC pari a 10+CA se falliscono subiscono -3 ai tiri per colpire per quel round.

	\item 12 punti: I tuoi colpi sono letali. L'EDX dell'arma, se presente, diminuisce di 1 (ad esempio su un flagello che è 8, su 1d8 diventa 7)
\end{itemize}

\subsection{Palle rotanti} Flagello, Flagello Pesante, Catena chiodata, Frusta

\begin{itemize}
	\item 4 punti: Ignori il bonus di protezione dato dallo scudo.

	\item 8 punti: L'impatto dei tuo colpi è tale da frantumare le ossa. L'avversario deve fare un tiro salvezza su Tempra (DC 10+CA) su fallimento subisce -2 Agilità per 1 minuto. Una creatura non può essere influenzata da questo effetto più di due volte

	\item 12 punti: La velocità e forza dei tuo colpi è tale da distruggere le difese del nemico, l'avversario prende -5 alla Difesa contro di te
\end{itemize}

\subsection{Armi aggraziate}\index{Armi aggraziate} Stocco, Scimitarra, Falcione

puoi decidere di usare l'Agilità per determinare il bonus al colpire ed al danno.

\begin{itemize}
	\item 4 punti: Puoi eseguire un critico, anche su creature normalmente immuni
	      ai critici

	\item 8 punti: L'EDX dell'arma, se presente, diminuisce di 1

	\item 12 punti: Per ogni -1 al danno che prendi la tua iniziativa aumenta di 2, fino ad un massimo di +6
\end{itemize}

\subsection{Armi della morte}\index{Armi della morte} Picca Leggera, Picca Pesante, Falce, Falcetto

\begin{itemize}
	\item 4 punti: Puoi eseguire un Colpo di Grazie con il costo di 1 Azione

	\item 8 punti: Aumenti di un grado il dado di danno (d4 - d6 - d8 - d10 - 2d6 - 2d8 - 2d10 - 3d6...)

	\item 12 punti: Aumenti di un grado il dado di danno (d6 - d8 - d10 - 2d6 - 2d8 - 2d10 - 3d6...)

\end{itemize}

\subsection{Armi da stordimento}\index{Armi da stordimento} Pugno nudo, Manganello, Guanto chiodato

+\begin{itemize}
	\item 4 punti: Un avversario inconsapevole se colpito con queste armi (durante il round di sorpresa) deve dare un Tiro Salvezza Tempra con DC pari al danno (del round causato da Armi da Stordimento) o rimanere stordito per 1d6 round

	\item 8 punti: Raddoppi il tuo bonus di danno dato dalla Potenza

	\item 12 punti: La tua arma da stordimento fa 1d6 di danno in piu'
\end{itemize}

\subsection{Lance} \index{Lance}Alabarda, Tridente, Urgrosh, Lancia da fante, Naginata, Falcione in asta, Lancia, Brandistocco, Tridente

\begin{itemize}
	\item 4 punti: Puoi usarla anche contro avversari a distanza di mischia senza malus, perdi il bonus di controcarica mentre sei in mischia

	\item 8 punti: Usata contro una carica fai il quadruplo del danno

	\item 12 punti: Se non sei in mischia con un avversario puoi usare la tecnica della Colpo Perforante (questa azione richiede tutto 3 Azioni) puoi caricare un avversario tra 6 e 18 metri: puoi sacrificare 1 punto CA e guadagnare 5 al danno (massimo 10 CA/50 danno) poi esegui un attacco solo col arma. Questo colpo ti porta in mischia con l'avversario e ti lascia scoperto per quel turno, hai un -4 alla Difesa

\end{itemize}

\subsection{Armi letali} Pugnale, Machete\index{Armi letali}

\begin{itemize}

	\item 4 punti: Contro avversari sorpresi aggiungi al danno il tuo CA

	      \textbf 8 punti: La tua arma fa più danno. Guadagni una categoria di danno (d4 - d6 - d8..)

	      \textbf 12 punti: Guadagni EDX. Lo si applica solo facendo il danno massimo con il dado, se l'arma ha già un edx (perché con il bonus precedente è arrivata ad 1d8 di danno) questo diminuisce di 1

\end{itemize}

\subsection{Aste} \index{Aste}Giavellotto, Lancia da fante, Tridente, Alabarda

\begin{itemize}

	\item 4 punti: In caso di critico puoi lasciare l'arma nel corpo dell'avversario,
	      penalizzandolo con un -2 Agilità. L'arma quando rimossa fa il suo dado di danno (senza Potenza o bonus magici)

	\item 8 punti: Raddoppi la gittata

	\item 12 punti: Guadagni un +2 all'iniziativa
\end{itemize}

\subsection{Spade}\index{Spade} Spada Corta, Spada Lunga, Spadone a due mani, Spada bastarda, Spada a due lame, Katana

\begin{itemize}

	\item 4 punti: La tua maestria nella tecnica della spada ti conferisce +1 a danno, Tiro per Colpire e Difesa

	\item 8 punti: La tua abilità con la spada ora ti permette di disarmare l'avversario. Questa abilità consuma una azione e l'avversario deve superare un Tiro Salvezza su Agilità (DC 10+CA) per evitare di essere disarmato. Valgono le considerazioni di Disarmare

	\item 12 punti: Hai raggiunto l'apice della maestria con la spada i tuo colpi sono precisi e difficili da prevedere ottieni +5 a danno, Tiro per Colpire e Difesa, oltretutto hai rimosso ogni movimento superfluo dalla tua tecnica
\end{itemize}

\subsection{Scudi}\index{Scudi} Leggeri, Medi, Pesanti

sei un maestro nell'uso degli scudi, anche come arma.

Puoi usare lo scudo come arma, uno scudo piccolo fa 1d4 di danno (B/S), uno scudo medio fa 1d6 di danno (B/S), uno scudo pesante fa 1d8 di danno (B/S).

Attaccare con lo scudo e con l'arma è considerato attacco a due mani.

\begin{itemize}
	\item 1 punto: Sei competente in tutte le tipologie di scudo. Non hai il vincolo del limite di Potenza 1 sugli Scudi Pesanti

	\item 2 punti: Il bonus di Difesa aumenta di 1 e ogni 4 volte che prendi la competenza

	\item 3 punti: La penalità CM diminuisce di 1 e di 1 ogni 4 volte che prendi la competenza

	\item 4 punti: la penalità CA diminuisce di 1 e di 1 ogni 4 volte che prendi la competenza

	\item 5 punti: Aumenta di 1 la categoria di danno dello scudo (1d4 - 1d6 - 1d8 - 1d10 - 2d6 - 2d8) ed ogni 4 punti ulteriori (9,13,17..)

	\item 8 punti: Abituato a prevedere e parare gli attacchi nemici ora riesci a difendere anche gli alleati adiacenti a te, ogni alleato adiacente a te ha un +1 Difesa. Se desideri puoi subire il danno di un attacco diretto ad un alleato entro 1 metro (al tuo fianco). Usare questa abiltià è una Reazione che non costa Azioni.

\end{itemize}

\subsection{Bloccanti} Bolas, Net\index{Bloccanti}

\begin{itemize}
	\item 4 punti: Una creature avvolta dalla tua rete o bolas è intralciato e non può muoversi

\end{itemize}

\subsection{Armi da tiro} Pugnale, Lancia corta da fante, Martello Leggero, Ascia ad una mano, Tridente\index{Armi da tiro}

Hai accesso a due abilita':

Tiro Devastante: puoi lanciare una delle tue armi con tale forza da triplicarne il danno (arma e Potenza) ma la precisione ne risente -8 al colpire. Costa 2 Azioni.

Ventaglio di lame: Puoi scagliare le tue armi per un massimo di 6 alla volta lancia un solo Tiro per Colpire con una penalità di -6. Le tue armi colpiscono a caso (scelti dal Narratore) chiunque sia intorno a te nel raggio della tua portata di tiro. Ricordati di raccogliere tutte le armi dopo averle lanciate. Costa 3 Azioni.

\begin{itemize}
	\item 4 punti: Se diventato estremamente preciso nel lancio della tua arma hai un +2 al colpire e un +1 ai danni

	\item 8 punti: La tua abilità ti permette di non avere tempi morti dopo il lancio di un arma puoi istantaneamente estrarne un altra senza consumare azioni.

	\item 12 punti: Raddoppi la Gittata dell'arma
\end{itemize}

\subsection{Pugno nudo} pugni/calci\index{Pugno nudo}

\begin{itemize}
	\item 1 punto :tuoi pugni fanno danno letale (1d4) e diventano Versatili
\end{itemize}

\textbf{Pugno Nudo}: Ogni volta che prendi questa competenza il danno aumenta seguendo questa progressione: 1d6 (presa la lista 2 volte), 1d8 (5), 2d6 (7), 2d8 (9), 2d10 (11), 3d6 (13), 3d8 (15), 3d10 (17), 4d6 (19), 4d8 (21)...

Il giocatore puo' anche decidere di fare danno non letale non incorrendo in alcuna penalita', al danno puo' applicare a proprio piacere il valore di Potenza o Agilita'.

\subsection{Armi Semplici} Pugnale, Mazza Leggera, Randello, Morningstar, Lancia corta da fante, Bastone, Balestra (Leggera), Giavellotto\index{Armi Semplici}

Questa suddivisione è sceglibile anche da chi ha Competenza Armi a zero

\pagebreak

\section{Abilita'}\index{Abilita'}

\label{abilita}
\begin{tcolorbox}[enhanced,arc=5pt,boxrule=0.3pt]{Il martirio è l'unica maniera per un uomo di diventare famoso se non ha abilità (George Bernard Shaw, The Devil's Disciple)} \end{tcolorbox}\medskip

Le Abilità sono capacità peculiari, frutto di allenamento o doti particolari. Le Abilità si ottengono hanno sempre un effetto pratico.

\textbf{Al primo livello si prendono due Abilità.} Ogni 2 livelli (e quindi al 3, 5, 7, 9... ) si prende un'altra Abilità che può essere la stessa già presa oppure una nuova Abilità appresa durante le avventure.

E' possibile che siano indicati dei Prerequisiti sotto il nome dell'Abilità, in questo caso vanno rispettati per prendere l'Abilità in questione.
Eventuali prerequisiti successivi vengono indicati volta per volta.

Non prendete le Abilità in base al potere, forza, combinazione che hanno ma perché il linea con la storia del personaggio.
Scegliere un accozzaglia di abilità solo perché forti non rende un personaggio forte ma sbilanciato, non fate il powerplayer ad ogni costo.

\medskip
\textbf{Le Abilità devono essere prese in base al percorso evolutivo del personaggio, in base a quanto vissuto ed appreso durante le avventure.}
\medskip

E' possibile cambiare una Abilità scelta, rispettando comunque i requisiti, al 5, 7, 11 e 17 livello, purché ci sia una giustificazione e si sia in accordo con il Narratore.

\subsection{Animalia}\index{Animalia}

Prerequisiti: Essenza Trasformazione, Competenza Magica 2.

Si acquisisce la capacità di trasformarsi in un animale. Costo 2 Azioni

La \textbf{prima volta} che si prende questa Abilità ci si può trasformare in animali non magici di taglia piccola o media, per 10 minuti per punteggio in Competenza Magica. Ci si può trasformare 1 sola volta al giorno.

La \textbf{seconda volta}, Competenza Magica 6, che si prende questa abilità si acquisisce la capacità di trasformarsi in animali minuscoli o grandi e ci si può trasformare due volte in più al giorno.

La \textbf{terza volta} che si prende questa Abilità, Competenza Magica 10, si acquisisce la capacità di trasformarsi in animali di taglia minuta o enorme e ci si può trasformare altre tre volte al giorno. Il tempo minimo di trasformazione giornaliera è di 16 ore

La \textbf{quarta volta}, Competenza Magica 16, si acquisisce di trasformarsi in animali di taglia piccolissima o mastodontica ed anche magici (sempre nel limite della taglia). Il tempo minimo di trasformazione è di 24 ore al giorno, e può trasformarsi quante volte vuole al giorno.

\subsection{Armatura del Devoto}\index{Armatura del Devoto}

Requisito: Tratti in comune 1 (somma dei tratti in comune con il Patrono)

Il costante allenamento con la tua armatura ti permette di indossare armature leggere senza rischio di sbagliare il lancio di Essenze.

La \textbf{seconda volta} che si prende questa Abilità, Tratti in comune 6, puoi lanciare Essenze senza rischio di fallire con armature medie.

La \textbf{quarta volta} che si prende l'abilita, Tratti in comune con il Patrono 12, puoi portare Armature Pesanti senza penalità.

\subsection{Arciere a cavallo}\index{Arciere a cavallo}

Il malus di tirare frecce da cavallo diminuisce di 2 ogni volta che prendi questa Abilità.

Le penalità standard sono -4 e -6 a seconda che si trotti (movimento x2) o galoppi (movimento x3)

\subsection{Arma Focalizzata}\index{Arma Focalizzata}

Ottieni un +1 a Iniziativa e Tiro per Colpire quando usi un arma specifica di cui hai competenza.

\subsection{Attacco turbinante}\index{Attacco turbinante}

Requisito: Competenza Armi 12

Usando 3 Azioni puoi eseguire un singolo attacco (con un malus di 5 al Tiro per Colpire) contro tutti gli avversari in mischia attorno a te.

\subsection{Colpi poderosi}\index{Colpi poderosi}

Requisito: Competenza Armi 1

Il tuo stile enfatizza colpi poderosi.

Guadagni un +1 al danno con una lista d'arma.

\subsection{Colpo furtivo (Attacco alle spalle)}\index{Attacco alle spalle}\index{Colpo furtivo}

Requisito: Competenza Armi 3

Ogni qual volta l'avversario viene attaccato in mischia di sorpresa, per ogni volta che hai preso questa Abilità (prerequisito Competenza Armi +2 rispetto alla volta precedente) il danno aumenta di +2d6.

\subsection{Colpo Indebolente}\index{Colpo Indebolente}

Requisito: Colpo furtivo 6d6, Competenza Armi 12

Colpo Indebolente è una forma avanzata di colpo furtivo. Ogni colpo Indebolente abbassa o Potenza o Agilità (scelta giocatore) di 1 punto.

All'avversario è concesso un Tiro Salvezza Riflessi con DC 10 + 1/2CA.

\subsection{Colpo Mortale}\index{Colpo Mortale}

Requisito: Competenza Armi 5

Esegui il Tiro per Colpire come sempre, ma ne dimezzi il risultato ottenuto. Se colpisci il danno causato dall'attacco viene raddoppiato ad esclusione dei bonus magici. Colpo mortale è una Azione Immediata da dichiararsi prima che il Narratore ti informi se si è colpito o meno.

\subsection{Colpo Paralizzante}\index{Colpo Paralizzante}

Requisito: Colpo Indebolente, Colpo furtivo 8d6, Competenza Armi 18

L'obiettivo dopo che è stato studiato per 10 round (2 Azioni a round) con il prossimo tuo colpo andato a segno in mischia, entro 10 round dal termine studio, deve effettuare un Tiro Salvezza Tempra con DC pari al doppio del danno inflitto o rimanere paralizzato per 3d6 round.

\subsection{Combattere alla Cieca}\index{Combattere alla Cieca}

è la capacità di attaccare gli avversari che non sono chiaramente percepibili.

Requisito: Consapevolezza 2

Un avversario con copertura leggera non ottiene bonus alla Difesa, con copertura media ha un +2 alla Difesa, con copertura totale ha un +4 alla Difesa.

Un attaccante invisibile, non ottiene alcun vantaggio al colpire il personaggio in mischia. I bonus dell'attaccante Invisibile si applicano lo stesso solo per gli attacchi da distanza.

Non c'è bisogno di effettuare prove di Acrobatica per muoversi a piena velocità mentre si è Accecati.

La \textbf{seconda volta} che prendi l'Abilità (Consapevolezza a 6), gli attacchi in mischia riducono di ulteriore due il bonus alla Difesa.

\textit{"Livello Zatoichi"}, la \textbf{terza volta} che prendi l'Abilità (Consapevolezza a 12), una creatura invisibile non ha alcun vantaggio contro di te ne tu hai malus contro di lui.

\subsection{Combattimento con due armi}\index{Combattimento con due armi}\index{Due armi}

Requisito: Agilità 2, Potenza 2 , Competenza Armi 2

La \textbf{prima volta} che prendi questa abilità puoi eseguire un attacco con l'arma secondaria, che deve essere leggera o a due lame, e non applichi il danno dato dalla Potenza. Entrambi i tiri per colpire hanno un -2 dato dall'attaccare a due mani.

Requisito Agilità 3, Competenza Armi 12

La \textbf{seconda volta} che prendi questa abilità puoi fare con l'arma secondaria leggera o a due mani, fino a 2 attacchi e non applichi il danno dato dalla Potenza. L'arma secondaria ha un -2 ai TC.

Requisito Agilità 3, Competenza Armi 18

La \textbf{terza volta} puoi usare un arma media come arma secondaria. Applichi il danno dato dalla Potenza con la mano secondaria. Non hai malus ai TC con la mano secondaria.

I TC dell'arma secondaria si non si considerano attacchi multipli. Il loro TC viene modificato solo dalle penalita' qui indicate.

\subsection{Creare Oggetti Magici}\index{Creare Oggetti Magici}

Requisito: Competenza Magica 6

tramite questa Abilità l'incantatore è in grado di infondere una Essenza fino al Livello di Potere 13 in un oggetto magico. Se infonde più Essenze deve possederle e la somma non può essere superiore a 26.

\subsection{Creare Oggetti Magici Superiori}\index{Creare Oggetti Magici Superiori}

Requisito: Creare Oggetti Magici, Competenza Magica 12

tramite questa Abilità l'incantatore è in grado di infondere un'Essenza fino al Livello di Potere 21 in un oggetto magico. Se infonde più Essenze deve possederle e la somma non può essere superiore a 42.

\subsection{Creare Oggetti Magici Meravigliosi}\index{Creare Oggetti Magici Meravigliosi}

Requisito: Creare Oggetti Magici Superiori, Competenza Magica 16

tramite questa Abilità l'incantatore è in grado di infondere un'Essenza fino al Livello di Potere 29 in un oggetto magico. Se infonde più Essenze deve possederle e la somma non può essere superiore a 56

\subsection{Decifrare scritti magici}\index{Decifrare scritti magici}

Requisito: Competenza Magica 1

Saper leggere le scritte magiche. Puoi leggere una pergamena contenente la descrizione di una Essenza con un bonus di +5

\subsection{Difendere Cavalcatura}\index{Difendere Cavalcatura}

Ogni qual volta la cavalcatura viene colpita, puoi effettuare una prova di cavalcare per negare il colpo. La tua prova di Sopravvivenza deve essere maggiore del Tiro per colpire dell'avversario

L'Abilità è utilizzabile solo una volta per round, per un solo attacco, costa 1 Azione.

\subsection{Distillare pozioni}\index{Distillare pozioni}

Requisito: Competenza Magica 1

Competenza nel distillare pozioni.

Acquisti un bonus di +4 su Conoscenze Erboristeria (Cultura) per distillare e creare pozioni e veleni naturali.

\subsection{Doppia porzione}\index{Doppia porzione}

Requisito: Combattimento con due armi, Competenza Armi 4

Il costante allenamento con due armi ti permette di applicare il bonus al danno dovuto alla Potenza in maniera piena anche all'arma secondaria.

\subsection{Energia Psichica}\index{Energia Psichica}

Requisito: Potenza 1, Volontà 2, Competenza Armi 1, Competenza Magia 1

Dopo anni di allenamento, meditazione e stage a Nanda Parbat sei in grado di raccogliere la tua Energia Chi.

Ogni giorno dopo almeno 6 ore di riposo e 2 ore di meditazione/allenamento riempi il tuo corpo di energia Chi pari a (CA+CM)/2+Volonta'/2

Requisito: Potenza 1, Volontà 2, Competenza Armi 4, Competenza Magia 4

Recuperi 2 punto Chi per ogni ora di riposo.

\subsection{Colpo Psichico}\index{Colpo Psichico}

Requisito: Energia Psichica, Agilità 2

Costo 1 Azione di Attacco, Tiro per Colpire di contatto, e fare 1d6 di danno per punto Chi speso. Non puoi usare un numero di punti Psichici superiore alla Volonta'

Requisito: Colpo Psichico, Volontà 3, Competenza Armi 7

Puoi utilizzare fino a doppio del tuo punteggio in Volontà per potenziare il Raggio Psichico

\subsection{Raggio Psichico}\index{Raggio Psichico}

Requisito: Colpo Psichico, Volontà 3, Competenza Armi 5

Puoi effettuare un attacco a distanza entro 9 metri usando l'Energia Psichica.
Il colpo, Tiro per Colpire a tocco, causa 1d6 di danno per punto Psichico speso. E' possibile usare uno o più punti Psichici per aumentare la distanza ogni volta di 9 metri.
Non puoi usare un numero di punti Chi totali (per distanza e e danno) superiore alla Volontà. Costo 1 Azione di Attacco.

Requisito: Colpo Psichico, Volontà 3, Competenza Armi 9

Puoi utilizzare fino a doppio del tuo punteggio in Volontà per potenziare il Raggio Psichico

\subsection{Essenza Psichica}\index{Essenza Psichica}

Requisito Energia Psichica, Volontà 3, Competenza Armi 8, Competenza Magia 3

Puoi usare la tua Energia Psichica per eseguire prestazioni oltre l'umano.

Sei in grado di simulare le Essenze di Cura e Movimento (Spostare solo se stesso) e Alterazione (solo se stesso) usando l'energia Psichica. Costo 2 Azioni

\bigskip

\begin{tabular}{ll}
	\toprule
	\textbf{Punti Chi consumati} & \textbf{Livello di Potere ottenuto}\\
	1                            & 11\\
	2                            & 13\\
	3                            & 16\\
	5                            & 19\\
	8                            & 22\\
	13                           & 25\\
\end{tabular}

\bigskip

La durata di Alterazione e Movimento è 1 round. E' possibile fare durare 10 round in piu per ogni punto Psichico speso.

Non è possibile usare le Essenze con Competenza Magica per 1 ora dopo aver usato Essenza Psichica.

\subsection{Eludere}\index{Eludere}

Requisito: Agilità 2, Competenza Armi 4

Il costante allenamento ad evitare essenze e trappole ti permette di annullare il danno nel caso ci sia un Tiro Salvezza su Riflessi per dimezzare, ed il TS ti riesca.


La \textbf{seconda volta} che si prende: Requisito Agilità 3, Competenza Armi 7

Il costante allenamento ad evitare essenze e trappole ti permette, nel caso il Tiro Salvezza su Riflessi permetta di dimezzare, di annullare totalmente il danno e se fallisci il Tiro Salvezza su Riflessi di dimezzare il danno.

\subsection{Esperto}\index{Esperto}

Prerequisito: Caratteristica collegata almeno a 1

Sei un esperto in un argomento. Ogni qual volta prendi questa abilità guadagni un +1 alle prove su una competenza a tua scelta.

\subsection{Animaletto / Famiglio}\index{Famiglio}

Guadagni un animale naturale. Questo animaletto ha al massimo un numero di dadi vita pari alla tua Volontà. Puoi insegnare azioni di base al tuo animale e fargli fare dei compiti semplici.

Requisito: Competenza Magica 1, se prendi due volte questa Abilità guadagni un Famiglio (vedi argomento specifico).

\subsection{Fare Infuriare}\index{Fare Infuriare}

Le tue abilità dialettiche sono incredibili

Prerequisito: Competenza Armi 2 e Magnetismo 3

Impieghi 2 Azioni ad infamare ed inveire contro un avversario. Il target deve fare un Tiro Salvezza Arbitrio a DC 10+1/2CA + Magnetismo oppure perdere il bonus di Agilità (al Tiro per Colpire e Difesa) fino alla fine del round successivo.

L'avversario può non comprendere la tua lingua ma deve avere Intelletto maggiore di -2.

\subsection{Ferocia}\index{Fare Infuriare}

Prerequisito; Competenza Armi 1

La tua rabbia è tale da sconfiggere, temporaneamente, la morte.

Quando scendi sotto lo 0 punti ferita non svieni ed incominci a perdere 1 punto ferita a round.
Una creatura dotata di ferocia sviene quando ha un punteggio di punti ferita negativo pari al doppio dei punti di Potenza e muore comunque quando i suoi punti ferita scendono al punteggio negativo pari al suo quadruplo del punteggio di Potenza+5 (POT*4+5)

\subsection{Finta Morte}\index{Finta Morte}

Sei in grado di simulare la morte, rallentando il cuore.

Come Reazione sei in grado di cadere a terra (stramazzare!) morto. Solo una prova di Sopravvivenza (Cura) DC 20 può rivelare che sei vivo.

L'effetto dura al massimo 2 minuti. La capacità non è ripetibile in intervalli inferiori ai 10 minuti.

\subsection{Flagello Danzante}\index{Flagello Danzante}

Requisito: Competenza Armi 1

quando usi il tuo Flagello hai un bonus di +1 alla Difesa e +1 iniziativa.

\subsection{Forgiato nella furia}\index{Forgiato nella furia}

Requisito: Competenza Armi 5

Quando effettui un critico, ovvero hai tirato almeno 2 volte 6, si considera che tu abbia tirato un 6 in più per il conteggio totale del numero di critici

\subsection{Freccia chiamata, freccia consegnata}\index{Freccia chiamata, freccia consegnata}

Requisito: Competenza Armi 2

puoi tirare 2 frecce, una volta al giorno, come azione immediata.

\subsection{Furia}\index{Furia}

Requisito: Competenza Armi 1

il tuo stile di combattimento è rappresentato dalla cieca furia omicida. Aggiungi +1d6 al danno ad ogni Tiro per Colpire che fai ed i tuoi avversare guadagnano +1d6 al colpire verso di te.

\subsection{Giocoliere}\index{Giocoliere}

Requisito: Agilità 2

hai un talento naturale per maneggiare gli oggetti.

Qualsiasi prova di Atletica che coinvolga il maneggiare oggetti o l'equilibrio ha un +2 di Bonus.

Puoi lanciare un secondo pugnale come azione immediata all'azione di attacco di lancio pugnale con un -3 al Tiro per Colpire. Un eventuale terzo pugnale lanciato ha il normale malus di -5 (e -10.. e così via).

\subsection{Guerriero dell'Essenza}\index{Guerriero dell'Essenza}

Non segui solo la via della magie e neanche quella della spada, il tuo stile fonde entrambi in un fendente di pura magia

La \textbf{prima volta} che prendi questa abilità, Competenza Armi 2, Competenza Magia 2: sei in grado di scaricare un incantesimo a distanza di mischia con la tua arma. Effettui un Tiro per Colpire normale (CA+Potenza+...) e se colpisci oltre al danno dell'attacco scarichi anche l'Essenza. L'attaccare e lanciare l'essenza costa 3 Azioni.

La \textbf{seconda volta} che prendi questa abilità Competenza Armi 6, Competenza Magia 3: sei in grado di attaccare con l'arma e poi riattaccare scaricando l'Essenza con l'arma. Consumi 3 Azioni, fai un attacco, un altro attacco (con penalità per attacchi multipli) dove scarichi l'Essenza.

\subsection{Ho detto CADI!}\index{Ho detto CADI!}

Requisito: Competenza Armi 4

Se colpisci 3 volte consecutivamente (fino a 3 round distinti ma consecutivi) un avversario questo deve fare una Tiro Salvezza su Tempra DC 10+1/2CA + Potenza o cadere prono.

\subsection{Incantare in Combattimento}\index{Incantare in Combattimento}

Ogni volta che prendi questa Abilità il bonus alla prova di concentrazione aumenta di 4.

\subsection{Incantatore Prudente}\index{Incantatore Prudente}

La prima volta che prendi questa abilità il malus alla Difesa mentre lanci una Essenza sotto minaccia diminuisce di 2 (da -4 a -2).

La \textbf{seconda volta} che prendi questa abilità, CA minimo 3, il malus alla Difesa diminuisce di 1 (e va a -1)

La \textbf{terza volta} che prendi questa abilità, CA minimo 6, il malus alla Difesa diventa 0.

In ogni caso se si viene colpito bisogna fare la prova di concentrazione.

\subsection{Immunità ai veleni}\index{Immunità ai veleni}

Il corpo si abitua ai veleni, il personaggio guadagna un +2 TS.

La seconda volta che prendi l'Abilità divieni immune ai veleni naturali.
Non riesci più ad ubriacarti normalmente.

La terza volta hai un +4 ai TS ai veleni magici, e non ti puoi più ubriacare o subire gli effetti di fumi tossici (ma puoi sempre soffocare).

\subsection{Imposizione delle mani (energia negativa o positiva a seconda dei tratti)}\index{Imposizione delle mani}

Requisito: Competenza Magica 3, Tratti comuni 3

Se i tuoi tratti sono in comune con un Patrono positivo puoi convogliare energia positiva (cura), se sono in comune con un Patrono neutrale o malvagio puoi convogliare energia negativa. Usabile un numero di volte pari al valore di Volontà. Effetto curativo/dannoso pari a 1d6+Volonta'

La \textbf{seconda volta}, requisito Competenza Magica 6, che prendi questa Abilità aumenti di 2d6 l'effetto e di 1 volte l'uso.

La \textbf{terza volta}, requisito Competenza Magica 12, che prendi questa Abilità aumenti di 3d6 l'effetto e di 1 volte l'uso.

La \textbf{quarta volta}, requisito Competenza Magica 18: che prendi questa Abilità aumenti di 4d6 l'effetto e di 1 volte l'uso.

L'energia proviene dalle mani (non conta se ci sono guanti) e si applica solo a tocco. Usa 2 Azioni.

\subsection{Incanalare energia (energia negativa o positiva a seconda dei tratti)}\index{Incanalare energia}

Competenza Magica 1, Tratti comuni 3

Sei in grado di incanalare l'energia magica.

Se i tuoi tratti sono in comune con un Patrono positivo puoi convogliare energia positiva (cura), se sono in comune con un Patrono neutrale o malvagio puoi convogliare energia negativa. Usabile un numero di volte pari punteggio di Volontà. Effetto curativo/dannoso pari a 1d6+Volonta'. Influenzi 1 creatura.

La \textbf{seconda volta}, Competenza Magica 6, che prendi questa Abilità aumenti di 2d6 l'effetto e di 1 volta l'uso. Influenzi fino a 2 creature.

La \textbf{terza volta}, Competenza Magica 12, che prendi questa Abilità aumenti di 3d6 l'effetto e di 1 volta l'uso. Influenzi fino a 4 creature.

La \textbf{quarta volta}. Competenza Magica 18, che prendi questa Abilità aumenti di 4d6 l'effetto e di 1 volta l'uso. Influenzi fino a 6 creature.

L'energia proviene dalle mani (non conta se ci sono guanti) ed influenza una o più creature entro un tre metri da te. Usa 2 Azioni.

\subsection{Incanalare energia a distanza}\index{Incanalare energia a distanza}

Requisito: Incanalare energia

Puoi lanciare l'energia, come da Incanalare Energia, fino a 9 metri, influenza un raggio di 3 metri. Usa 2 azioni.

La \textbf{seconda volta} che prendi questa Abilità l'energia arriva fino a 18 metri.

La \textbf{terza volta} che prendi questa Abilità il l'energia arriva fino a 50 metri.

\subsection{Incanalare energia concentrata}\index{Incanalare energia concentrata}

Requisito: Incanalare energia

Puoi lanciare l'energia fino a distanza 18 metri. Singolo obiettivo. Usa 2 azioni.

Ogni volta che prendi questa competenza aggiungi un obiettivo entro i 18 metri sul quale dividere a piacimento i dadi disponibili dell'incanalare energia.

L'abilità non è cumulabile con "Incanalare energia a distanza".

\subsection{Iniziativa migliorata}\index{Iniziativa migliorata}

Aumenti l'iniziativa di +1. L'Abilità può essere presa più volte ed il bonus si cumula.

\subsection{Iaijutsu}\index{Iaijutsu}

Per ogni -5 al Tiro per Colpire guadagni un +10 all'Iniziativa e vice versa.
Il bonus deve essere usato entro la fine del round successivo. La dichiarazione va eseguita ogni round che si intende usare al momento del controllo delle iniziative.

\subsection{La mia morte la tua morte}\index{La mia morte la tua morte}

Per ogni singolo avversario di combattimento puoi fare che il primo colpo a segno dello scontro causi un danno aggiuntivo pari al doppio di Competenza Armi. L'avversario guadagna un bonus al Tiro per Colpire ed al danno pari al valore della tua Competenza Armi e attacca prima di te quando dichiari di usare questa Abilità.

\subsection{La mia Testa è più Dura}\index{La mia Testa è più Dura}

Requisiti: Competenza Armi 1

La tua Arma Rompi Cranio fa +2 danni

\subsection{Lo scudo è mio amico}\index{Lo scudo è mio amico}

Requisiti: Competenza Armi 1

La penalità alla Competenza Magica diminuisce di 1

La \textbf{seconda volta} che si prende questa Abilità, Competenza Armi 3, la penalità al CA diminusce di 1, la penalità CM diminuisce ulteriormente 2.

La \textbf{terza volta} che si prende questa Abilità, Competenza Armi 5, la penalità al CA diminuisce di 3, la penalità CM diminuisce di ulteriore 2.

\subsection{Magie efficaci}\index{Magie efficaci}

Requisiti: Competenza Magica 5

Le tue magie sono straordinariamente efficaci.

Scegli una Essenza, i DC per resistere alle magie di questa Essenza aumentano di 1. L'Abilità può essere presa più volte ma il totale deve essere inferiore a CM/4 ed il bonus si somma o si applica ad altra Essenza.

\subsection{Montagna umana}\index{Montagna umana}

Forse una volta eri gracile e debole, adesso sei una montagna di muscoli.

Quando prendi questa Abilità aumenti di 1 i punti ferita presi per livello.

La \textbf{seconda volta} che prendi questa Abilità aumenti di 1 i punti ferita presi per livello.

La \textbf{terza volta} che prendi questa Abilità aumenti il dado per tirare i punti ferita (da d4 a d6).
I bonus sono cumulativi e retroattivi ai livelli precedenti, tranne che l’aumento di dado vita.

La \textbf{quarta volta} che prendi questa Abilità aumenti di una taglia (S > M > L > H).

\subsection{Occhio Clinico}\index{Occhio Clinico}

Requisiti: Competenza Armi 2

Sei in grado di fare critici a creature normalmente immuni ai critici.

\subsection{Occhio di Falco}\index{Occhio di Falco}

Requisito: Competenza Armi 3

La penalità per i tiri oltre il primo incremento di range diminuisce di 1d6.

La \textbf{seconda volta} che prendi questa Abilità, la penalità per i tiri fino al secondo incremento di range diminuisce di 2d6.

La \textbf{terza volta} che prendi questa abilità sei in grado di estendere ancora di più il tuo tiro e portarlo ad un quarto incremento con un -1d6 di penalità al colpire. Tirare nei primi due incrementi non ha penalità.

\subsection{Opportunista}\index{Opportunista}

Requisiti: Competenza Armi 2

Puoi tentare di colpire un avversario (un attacco di opportunita') che esce da un area che tu minacci. L'abilità è usabile una volta per round come Reazione a costo 0 Azioni.

\subsection{Passo Veloce}\index{Passo Veloce}

Il tuo passo e' naturalmente rapido. aumenti il tuo movimento di un terzo.
Se hai movimento 6m passi a movimento 8m, se hai movimento 9m passi a movimento 12m.

L'Abilita' presa altre due volte, per un totale di 3 volte, porta un ulteriore velocizzazione del passo.
Da movimento 8m passi a 12m, da movimento 12m passi a 18m per Azione di Movimento.

\subsection{Passo sicuro}\index{Passo sicuro}

E' la capacità di non essere rallentati in un ambiente ostile. E' necessario dichiarare su quale ambiente si prende l'abilità. In questi ambienti il terreno non è difficile per te.

\bigskip

\begin{tabular}{ll}
	\toprule
	\textbf{Ambiente}                                  & \textbf{Ambiente}\\
	Acquatico (sopra e sotto la superficie dell'acqua) & Giungla\\
	Foresta (conifere e decidue)                       & Pianura\\
	Deserto (terre brulle e deserto sabbioso)          & Montagna (compreso colline)\\
	Freddo (ghiacciai, ghiaccio, neve e tundra)        & Palude\\
	Sotterraneo (caverne, dungeon)                     & Urbano (edifici, strade, fogle)\\
\end{tabular}

\bigskip

Ogni qual volta si prende nuovamente questa abilità si sceglie un ambiente diverso e si aggiunge al precedente.

\subsection{Passo tattico}\index{Passo tattico}

Requisiti: Competenza Armi 3, Agilità 2

Guadagni una Azione di Movimento per round. L'Abilità può essere presa al massimo 1 volta. Questa Azione può essere fatta solo nel tuo round come Azione bonus.

\subsection{Percettivo}\index{Percettivo}

La tua Consapevolezza e attenzione ai particolari è sopra la media.
Prendi un bonus di +2 alla prove di Consapevolezza. L'Abilità puo' essere presa più volte, il bonus aggiuntivo oltre la prima volta diventa +1.

\subsection{Persona veramente malvagia}\index{Persona veramente malvagia}

Requisiti: Competenza Armi 1

Due volte al giorno aggiungi il tuo valore di CA al colpire ed al danno, in mischia ad un avversario che vedi. L'Abilità può essere dichiarata come Reazione a costa 1 Azione, dopo il Tiro per Colpire ma prima che il Narratore dica se il colpo è andato a segno o meno.

\subsection{Più sono grossi più fanno rumore quando cadono}\index{Più sono grossi più fanno rumore quando cadono}

Requisiti: Competenza Armi 1

Quando attacchi una creatura di almeno 2 taglie più grosse di te fai +1 danno aggiuntivo ogni 2 punti CA. Se è solo una taglia superiore aggiungi 1 danno in più ogni 3 punti CA.

\subsection{Proseguire}\index{Proseguire}

La \textbf{prima volta} che prendi questa abilità requisiti: Competenza Armi 1

Se uccidi l'avversario con il tuo ultimo colpo, in mischia, puoi effettuare una reazione di attacco con 3d6 + Competenza Armi + Potenza + Abilità (senza contare bonus dovuti alla magia dell'arma) ed attaccare l'avversario successivo entro 1m con un -2 al colpire -1 al danno, se uccidi questa creature con un colpo non puoi effettuare altri attacchi ad altre creature.

La \textbf{seconda volta} che prendi questa abilità Requisiti: Proseguire, Competenza Armi 6

Se uccidi la creatura con il tuo ultimo colpo, in mischia, puoi effettuare una reazione di attacco con l'arma senza contare bonus dovuti alla magia dell'arma e attaccare la creatura successiva in distanza di 1 metro con un -2 al colpire -1 al danno, se la uccidi puoi proseguire con la reazione di attacco (e ti sposti entro 2 metri) con la creature successiva e così via, ogni volta hai un -2 al colpire ed un -1 al danno cumulativo.

Questa abilità permette di usare più reazioni per round.

\subsection{Questo è il mio pugnale}\index{Questo è il mio pugnale}

Requisiti: Competenza Armi 1

Ogni qual volta fai un critico con il tuo pugnale sommi la tua CA al danno. L'Abilità è usabile 1 volta per avversario nelle 24 ore e si applica automaticamente al primo critico effettuato.

\subsection{Questa è la mia arma!}\index{Questa è la mia arma!}

Requisiti: Competenza Armi 1

Ogni volta che colpisci il medesimo avversario fai un danno aggiuntivo (Max +1 per round di combattimento, anche se lo colpisci più nel round). Fino ad un massimo +5. La prima volta che non colpisci nel round l'avversario il bonus torna a +0.

\subsection{Radici magiche}\index{Radici magiche}

Requisiti: Competenza Magica 1

Finché sei influenzato da una Essenza, utilizzando un'Azione la tua arma guadagna un +1 al colpire ed al danno e si considera un'arma magica. Per ogni Essenza che ti influenza nel round, oltre la prima (non da oggetti magici) il bonus aumenta di +1/+1 fino ad un massimo di +3/+3.

\subsection{Rappresaglia}\index{Rappresaglia}

Vedere i tuoi amici feriti ti riempie di rabbia.
Quanto un compagno (o te stesso) scende sotto meta'’ dei punti ferita guadagni un +1 a Difesa e Tiro per Colpire e Tiri Salvezza. La durata massima dell’effetto e’ 1 minuto (10 round) al giorno e deve essere consecutiva. Il giocatore sceglie se attivare o meno l’abilita’.
Puoi prendere questa Abilità fino a 3 volte, ogni volta il bonus massimo sale di 1.

\subsection{Resistenza della pietra}\index{Resistenza della pietra}

Nel tempo hai allenato la tua Potenza a reggere gli urti, trasformazioni, veleni e quant'altro volesse modificare il tuo corpo. Ogni qual volta prendi questa Abilità ottieni un bonus di +2 al Tiro Salvezza su Tempra. Il bonus è cumulativo, +2 la prima volta, +1 la seconda, +1 la terza ed ultima volta possibile

\subsection{Rilevare il Magico}\index{Rilevare il Magico}

Competenza Magica 1

Se lo puoi vedere sai anche se è magico. Costa una Azione attivare la vista magica

\subsection{Ricarica rapida (Balestra)}\index{Ricarica rapida}

Agilità 2, Tiro preciso

Come Abilità Tiro Rapido, solo per balestre

\subsection{Riflessi fulminei}\index{Riflessi fulminei}

Nel tempo hai allenato i tuoi riflessi a schivare e prevedere qualsiasi ostacolo. La prima volta prendi questa Abilità ottieni un bonus di +2 ai Tiri Salvezza su Riflessi. Il bonus è cumulativo, +2 la prima volta, +1 la seconda, +1 la terza ed ultima volta possibile.

\subsection{Scacciare i non morti}

Concentrandoti sulla potenza del tuo Patrono convogli l'energie positiva e allontani o distruggi i non-morti.

Effettua una prova tirando 1d6, per ogni volta che hai preso questa abilita' + la somma dei tuoi punti Tratto. Confronta il risultato con questa tabella per capire gli effetti ottenuti.

\medskip
\textbf{Tabella Scacciare Non Morti}
\medskip

\begin{tabular}{lllllllllll}
\textbf{Non Morto} & \textbf{Prova}\\
o \textbf{CR}            	&2-4 & 5-8 &9-11 &12-15&16-18&19-22&23-25&26-29&30-32&33-36\\
\textbf{Scheletro}       	& - & T   & T   & T   & D   & D   & D   & D*  & D*  & D*  \\
\textbf{Zombie} 			& - & -   & T   & T   & T   & D   & D   & D   & D*  & D*  \\
\textbf{Ghoul}				& - & -   & -   & T   & T   & T   & D   & D   & D   & D*  \\
\textbf{Ghast}				& - & -   & -   & -   & T   & T   & T   & D   & D   & D   \\
\textbf{Wraith}				& - & -   &-    & -   & -   & T   & T   & T   & D   & D   \\
\textbf{Mummia}				& - & -   &-    &-    & -   & -   & T   & T   & T   & D   \\
\textbf{Spettro}			&-  & -   &-    &-    &-    & -   & -   & T   & T   & T   \\
\textbf{Vampiro}			&-  & -   & -   &-    &-    &-    & -   & -   & T   & T   \\
\textbf{Fantasma}			&-  & -   &-    & -   &-    &-    &-    & -   & -   & T   \\
\textbf{Lich}				&-  & -   &-    & -   &-    &-    &-    & -   & -   & T   \\
\end{tabular}

\medskip

\flushleft \textit{Legenda}:
\flushleft T: 1d4 creature scappano per 1 minuto il piu' lontano possibile. Se aggredite rispondono all'attacco.\\
D: 1d4 creature vengono distrutte\\
D*: 2d4 creature vengono distrutte\\
\medskip
L'abilita' e' usabile quanto volte si vuole ma un non morto puo' essere influenzato solo una volta al giorno dall'effetto.

\subsection{Schivare trappole}\index{Schivare trappole}

La \textbf{prima volta} che prendi l'abilità requisiti Agilità 3 ottieni un bonus di +4 ai TS per evitare l'effetto delle trappole.

La \textbf{seconda volta} che prendi l'abilità requisiti Schivare trappole, Competenza Armi 5

Anche se la trappola non concede TS la tua naturale propensione ad evitare i danni ti concede un TS per dimezzare i danni.

E' anche possibile usare questa Abilità per evitare Attacco furtivo (TS Riflessi superiore a Tiro Colpire avversario)

La \textbf{terza volta} che prendi l'abilità requisiti Schivare trappole, Competenza Armi 9

Il TS se riuscito ti permette di evitare qualsiasi effetto della trappola, se fisicamente possibile.

\subsection{Schivata prodigiosa}\index{Schivata prodigiosa}

come Reazione ad una Azione di attacco puoi aggiungere +2 alla tua Difesa. Puoi applicare il bonus dopo il Tiro per Colpire dell'avversario ma prima di sapere se ti ha colpito o meno.

\subsection{Seconda pelle}\index{Seconda pelle}

Requisito: Competenza Armi 1

Il costante allenamento con la tua armatura ti permette di indossarle senza grosse penalità.

Il malus alle prove di Agilità diminuisce di 1.

La \textbf{seconda volta} che si prende questa Abilità, Competenza Armi 6, il malus alle prove di Agilità diminuisce di ulteriori 2.
Il malus alle penalità al movimento diminuisce di 1.

Puoi dormire in armature medie senza essere affaticato la mattina.

La \textbf{terza volta} che si prende questa Abilità, Competenza Armi 11, il malus alle prove di Agilità diminuisce di ulteriori 2. Il malus alle penalità al movimento diminuisce di un ulteriore 1.

Puoi dormire in armature pesanti senza essere affaticato la mattina.

\subsection{Segugio}\index{Segugio}

Requisito: Intelletto 1, Volontà 1, CA 1

Hai un talento naturale per seguire le persone

Con due Azioni ti focalizzi su un target che puoi vedere e finché lo vedi rimani focalizzato. Tutte le tue Azioni che coinvolgono quel target hanno un +1 di bonus. Rimanere focalizzato costa 1 Azione per round.

La \textbf{seconda} volta che prendi questa abilità, Competenza Armi 6, il bonus sale a +2.

La \textbf{terza} volta che prendi questa abilità, Competenza Armi 12, il bonus sale a +3.

Il bonus può essere usato al TC, TS causati dall'avversatio, prove di competenza.. ma non al danno.

\subsection{Senso Trappola}\index{Senso Trappola}

Requisiti: Intelletto 2, Agilità 3

Hai un senso innato per trovare le trappole. Ti viene concesso una prova di consapevolezza (reazione) nel passare entro 1 metro da una trappola (che farà il Narratore) .

La \textbf{seconda volta} che prendi l'Abilità il raggio aumenta fino a 3 metri e prendi un +2 alla prova. La \textbf{terza volta} che prendi l'Abilità il raggio aumenta a 9 metri.

\subsection{Senza Traccia}\index{Senza Traccia}

Requisiti: Passo sicuro

la capacità di non lasciare impronte nell'ambiente scelto. Ogni volta che prendi questa Abilità puoi scegliere un ambiente diverso (vedi Abilità Passo Sicuro) di cui hai competenza. La prova di sopravvivenza per inseguirti ha una difficoltà aumentata di 10.

\subsection{Stai giu'!}\index{Stai giu'!}

Quando esegui un critico su un avversario la forza del tuo colpo è tale da metterlo prono. L'avversario deve fare un Tiro Salvezza Tempra DC 10+1/2CA+Potenza o cadere prono. L'Abilità funziona su creature di taglia pari o inferiore a quella del personaggio.

La \textbf{seconda volta} che prendi l'Abilità puoi influenzare anche creature di una taglia superiore.

La \textbf{terza volta} che prendi l'Abilità puoi influenzare anche creature di due taglie superiori.

\subsection{Tiro preciso}\index{Tiro preciso}

Requisiti: Agilità 3, Competenza Armi 1

guadagni un +1 colpire e +1 al danno per i tiri, con armi da tiro o archi, entro 9 metri.

\subsection{Tiro rapido}\index{Tiro rapido}

Requisiti: Agilità 3, Tiro Preciso, Competenza Armi 2

Puoi effettuare un tiro in più con Arco o Pugnale lanciato e le penalita' per l'attacco multiplo sono inferiori.

Ogni proiettile lanciato oltre al primo prende un -4 al Tiro per Colpire. Per poter usufruire dell'attacco in più devi usare l'Azione di attacco multiplo.
La prima freccia ha un TC normale, la seconda un -4, la terza un -8

\subsection{Toccata e fuga}\index{Toccata e fuga}

Prendendo -5 al Tiro per Colpire all'Azione di Attacco, puoi effettuare un'Azione di 1 movimento in più (oltre le 3 Azioni standard).

\subsection{Tocco pietoso}\index{Tocco pietoso}

Requisiti: Patrono buono, Imposizione delle mani, Competenza Magica 3

Il tuo tocco lenisce non solo le ferite ma anche le sofferenze e dolori. Ogni qual volta usi l'Abilità Imposizione delle mani puoi aggiungere anche questa Abilità come Azione Immediata.

Usando l'Imposizione delle mani puoi, rinunciando ad un numero di d6 curativi indicati, rimuovere le seguenti afflizioni.

\textbf{2d6} Tratti in comune 3

Affaticato: il soggetto non è più affaticato

Scosso: il soggetto non è più scosso.

Infermo: il soggetto non è più infermo.

Frastornato: il soggetto non è più frastornato

\textbf{3d6} Tratti in comune 6

Malato: funziona come l'Essenza di Cura, fatto da un incantatore di pari livello.

Stordito: il soggetto non è più stordito

Confuso: il soggetto non è più confuso

Nauseato: il soggetto non è più nauseato

\textbf{4d6} Tratti in comune 9

Maledetto: funziona come Essenza di Protezione rimuovere maledizione, usando il livello del incantatore come livello.

Impaurito: il soggetto non è più impaurito

Avvelenato: funziona come Essenza di Cura rimuovi la condizione di avvelenato usando il livello del incantatore come livello di potere.

Ristorativo: il soggetto recupera 1d4 punti in una caratteristica

\textbf{5d6} Tratti in comune 11

Rigenerante: il tocco dell'incantatore può fare rigenerare arti tagliati, se il soggetto è ancora vivo.

Accecato: il soggetto non è più cieco

Sordo: il soggetto non è più sordo

Paralizzato: il soggetto non è più paralizzato

Pietrificato: come Essenza di Trasformazione pietra in carne. Il soggetto non è più pietrificato

\subsection{Vampiro}\index{Vampiro}

Requisiti: Odore del sangue

La tua sete di sangue diventa cura. Il bonus di sete di sangue può aumentare fino a +5.

Se il bonus aumenta da +3 a +4 o +5 puoi, ingurgitando il sangue avversario, curarti di 1d6 impiegando un 2 azioni

\subsection{Volontà Ferrea}\index{Volontà Ferrea}

Nel tempo hai allenato la tua volontà per resistere a qualsiasi debolezza e paura. La prima volta prendi questa Abilità ottieni un bonus di +2 ai Tiri Salvezza su Arbitrio. Il bonus è cumulativo, +2 la prima volta, +1 la seconda, +1 la terza ed ultima volta possibile

\pagebreak

\section{Famiglio}\index{Famiglio}

\label{famiglio}
\begin{tcolorbox}[enhanced,arc=5pt,boxrule=0.3pt]{
Abbiamo imparato a volare come gli uccelli, a nuotare come i pesci, tuttavia non abbiamo imparato l'arte di vivere come fratelli. (Martin Luther King )}\end{tcolorbox}\medskip

I famigli sono animali scelti dal personaggio, tramite l'Abilità Famiglio, perché gli siano d'aiuto nelle avventure e per compagnia. Un famiglio ha un legame speciale con il suo padrone.

Un famiglio è un normale animale che mantiene aspetto, Dadi Vita, Competenza Armi, bonus ai Tiri Salvezza Base, Abilità e Talenti del normale animale che era, ma viene trattato come creatura magico al fine di determinare qualsiasi effetto che dipenda dal suo tipo.

Solo un normale animale, non modificato, può diventare un famiglio.

Un famiglio conferisce delle Capacità Speciali al suo padrone, come indicato nella tabella sotto. Queste Capacità Speciali si applicano solo quando il padrone e il famiglio sono entro 100 m l'uno dall'altro.

Se un famiglio viene congedato, perso oppure muore, può essere sostituito una settimana dopo con uno speciale rituale che costa 2 punti di Potenza temporanea del personaggio. Per completare il rituale occorrono 8 ore.

\bigskip

\begin{tabularx}{0.95\textwidth}{lX}
	\toprule
	\textbf{Famiglio}                & \textbf{Capacità speciale}\\
	Lucertola / Capra    & Il padrone guadagna bonus +2 alle prove di Sopravvivenza\\
	Corvo                & Il padrone guadagna bonus +2 alle prove di Faccia Tosta\\
	Donnola / Volpe      & Il padrone guadagna bonus +1 al Tiro Salvezza su Riflessi\\
	Falco                & Il padrone guadagna bonus +2 su Consapevolezza sulla vista\\
	Gatto                & Il padrone guadagna bonus +2 alle prove di Criminalita'\\
	Gufo                 & Il padrone guadagna bonus +2 su Consapevolezza sulla vista\\
	Lontra / Ornitorinco & Il padrone guadagna bonus +2 alle prove di Resistenza\\
	Pipistrello          & Il padrone guadagna bonus +2 alle prove di Acrobatica\\
	Riccio               & Il padrone guadagna bonus +1 al Tiro Salvezza su Volonta'\\
	Scimmia              & Il padrone guadagna bonus +2 alle prove di Criminalita'\\
	Topo                 & Il padrone guadagna bonus +1 al Tiro Salvezza su Tempra\\
\end{tabularx}

\bigskip

Utilizzare le statistiche base di una creatura della specie del famiglio, apportando i seguenti cambiamenti.

\bigskip

\textbf{Dadi Vita}: Ai fini degli effetti legati al numero dei Dadi Vita, utilizzare il punteggio di CM del personaggio del padrone o il normale totale di DV del famiglio, quale dei due sia più alto.

\textbf{Attacchi}: Utilizzare la CA del padrone. Utilizzare il modificatore di Agilità o Potenza del famiglio, quale dei due sia più alto per calcolare il bonus di attacco del famiglio con gli Attacchi Naturali. Il danno è uguale a quello di una normale creatura della specie del famiglio.

\textbf{Difesa}: il famiglio ha una Difesa pari a quello dell'animale standard piu' bonus dovuto alla CM del padrone. Vedi tabella sotto.

\textbf{Tiro Salvezza}: Per ogni Tiro Salvezza, utilizzare i bonus al Tiro Salvezza Base del famiglio (Tempra +2, Riflessi +2, Volontà +0) o quelli del padrone quali siano i migliori. Il famiglio applica i suoi valori di caratteristica come bonus ai Tiri Salvezza e non condivide nessuno dei bonus che il suo padrone potrebbe ricevere ai propri Tiri Salvezza.

\bigskip

\textbf{Descrizione delle Capacità del Famiglio}

Tutti i famigli possiedono Capacità Speciali (oppure le attribuiscono ai loro padroni) a seconda dei livelli combinati del padrone nelle classi che concedono i famigli, come indicato nella tabella seguente. Le capacità elencate nella tabella sono cumulative.

\bigskip

\begin{tabularx}{0.95\textwidth}{lllX}
	\toprule
	\textbf{CM del Padron}e & \textbf{Mod. Difesa Fam.} & \textbf{Intelletto Fam.} & \textbf{Speciale}\\
	1-2      & +1       & -2    & Allerta, Condividere Essenza, \\
	         &  		&  		& Eludere Migliorato, Legame Empatico\\
	3-4      & +2       & -1    & Trasmettere Essenza a contatto\\
	5-6      & +3       & 0     & Parlare con il Padrone\\
	7-8      & +4       & 0     & Parlare con gli Animali della Sua Specie\\
	9-10     & +5       & 1     & Vedere attraverso Famiglio\\
	11-12    & +6       & 1     & -\\
	13-14    & +7       & 2     & -\\
	15-16    & +8       & 2     & -\\
	17-18    & +9       & 3     & -\\
	19-20    & +10      & 3     & -\\
\end{tabularx}

\bigskip

\textbf{CM del Padrone}: Il numero indicato qui è il valore di CM del padrone del famiglio, articolato in fasce.

\textbf{Modificatore armatura}: Il numero indicato qui è in aggiunta al Bonus di Armatura Naturale esistente del famiglio.

\textbf{Intelletto}: Il punteggio di Intelletto del famiglio. Si tiene questo valore o quello del famiglio a seconda di quale sia più alto.

\textbf{Speciale}: Le capacità speciali acquisite dal famiglio (e/o dal padrone).

\textbf{Allerta}: Quando il famiglio è a portata di braccio dal padrone, questi guadagna +1 alle prove di Consapevolezza

\textbf{Condividere Essenze}: A propria discrezione, il padrone può lanciare qualsiasi Essenza che abbia effetto su di "sé" sul suo famiglio (come una Essenza a contatto), al posto di se stesso.

Il padrone può lanciare sul suo famiglio Essenze anche se queste normalmente non hanno effetto su creature del tipo del famiglio (creature magiche).

\textbf{Eludere Migliorato}: Se il famiglio è soggetto a un attacco che normalmente permette un Tiro Salvezza su Riflessi per dimezzare i danni, il famiglio non subisce danni se supera il Tiro Salvezza e solo la metà dei danni se fallisce il Tiro Salvezza.

\textbf{Legame Empatico}: Il padrone ha un legame empatico con il suo famiglio fino a una distanza di 1 km. Il padrone non può vedere attraverso gli occhi del famiglio, ma può comunicare telepaticamente con esso. A causa della natura limitata del legame, si possono comunicare solo emozioni generiche.

\textbf{Trasmettere Essenze a Contatto}: Se il padrone ha Competenza Magica 3 o superiore, il famiglio può trasmettere Essenza a contatto per lui. Se il padrone e il famiglio sono entro 9 metri quando il padrone lancia un'Essenza a contatto, egli può designare il suo famiglio come "colui che consegna l'Essenza" (se a tocco).

Il famiglio può trasmettere l'Essenza proprio come il padrone. E' necessario che l'attacco del famiglio sia nello stesso round, ma successivamente come azione del lancio dell'Essenza.

\textbf{Parlare col Padrone}: Se il padrone ha Competenza Magica 5 o superiore, il famiglio e il padrone possono comunicare verbalmente, come se utilizzassero un linguaggio comune. Le altre creature o animali non sono in grado di comprendere la loro conversazione, se non utilizzando ausili magici. La capacita' funziona entro i 50m e devono sentirsi.

\textbf{Parlare con Animali della Sua Specie}: Se il padrone ha Competenza Magica 7 o superiore, il famiglio è in grado di comunicare con animali della sua specie generica: pipistrelli con pipistrelli, topi con roditori, gatti con felini, falchi e gufi e corvi con uccelli, serpenti e lucertole con rettili, rospi con anfibi, scimmie con altri primati, donnole con ermellini e mustelidi... La comunicazione è limitata dalla Intelligenza delle creature con cui il famiglio comunica.

\textbf{Vedere attraverso Famiglio}: Se il padrone ha Competenza Magica 9 o superiore, può vedere attraverso il famiglio. Attivare questa abilità costa 1 azione immediata. Il famiglio deve essere entro 30 metri.

\pagebreak

\section{Altre Abilità speciali}

\label{altre-abilita-speciali}

\subsection{Etereo}\index{Etereo}

\label{etereo}

Una creatura diventata Eterea è situata nel Piano Etereo che è sovrapposto a quello Materiale.

Una creatura eterea è Invisibile, senza sostanza e capace di muoversi in qualsiasi direzione, persino su e giu', ma solo a velocità dimezzata. Una creatura eterea può muoversi attraverso oggetti solidi, incluse altre creature viventi. Una creatura eterea può vedere e udire ciò che accade sul Piano Materiale, ma ogni cosa appare grigia ed effimera. La vista e l'udito di una creatura eterea che si trova sul Piano Materiale sono limitati a una distanza di 9 metri.

Le Essenze se non opportunamente formulate e modificate non agiscono su creature eteree. Una creatura eterea ha Resistenza al Danno verso Luce o Vuoto, ed ignora tutte le altre forme di Energia.

Una creatura eterea non può attaccare una creatura materiale ed Essenze lanciate mentre ci si trova in condizione di etereo possono influenzare solo elementi eterei. Alcune creature o oggetti materiali hanno attacchi o effetti speciali che funzionano anche sul Piano Etereo. Una creatura eterea considera tutte le altre creature eteree come se tutti fossero materiali.

\subsection{Resistenza al Danno}\index{Resistenza al Danno}

Determinate creature o protezioni conferiscono la capacita' di Resistere ad una tipologia di Danno.

Essere Resistenti al Danno significa automaticamente dimezzare il danno ricevuto prima di applicare qualsiasi altra resistenza o Tiro Salvezza.

La Resistenza al Danno puo' assumere anche dei valori. Quando viene scritto Resistenza al Danno: Fulmine, il soggetto dimezza automaticamente i danni, se scritto Resistenza al Danno: Filmine 10, significa che riduce il danno da elettricità di 10 punti prima di applicare il tiro salvezza o altri bonus.

Un creatura con una Resistenza al Fuoco dimezza (riduce) tutto il danno che riceve dalla fiamme, magiche o meno. Possono esistere abilita' o incantesimi che ignorano questa Resistenza.

\subsection{Riduzione del Danno}\index{Riduzione del Danno}

Determinate creature o Abilità conferiscono la capacità soprannaturale di resistere al danno di certe tipologie di armi o fino ad un certo ammontare (per attacco).

Solitamente assume il valore di XX/ZZ ovvero quanto danno (XX) è ignorato se non si è attaccati con (ZZ). Ignorare il danno significa anche che effetti connessi all'attacco non funzionano, come veleni sull'arma.

E' applicabile un'unica DR in caso ce ne siano di più di una contemporanea, la scelta va fatta ad inizio scontro e rimane la stessa finche non è finito lo scontro.

Determinate armi, particolarmente magiche possono ignorare la DR \index{Ignorare la DR}

\bigskip

\begin{tabular}{lll}
	\toprule
	\textbf{DR da superare} & \textbf{Incantamento sull'arma} & \textbf{Attacco Naturale}\\
	Incantamento +1         & +1              & Livello 3\\
	Incantamento +2         & +2              & Livello 6\\
	Ferro Freddo / Argento  & +3              & Livello 9\\
	Adamantio               & +4              & Livello 12\\
\end{tabular}

Proiettili (frecce, dardi, sassi) tirati da armi magiche sono considerate
magiche al pari dell'arma che li lancia.


\subsection{Vulnerabilità al Danno}\index{Vulnerabilità al Danno}

Determinate creature o magie rendono piu' efficaci alcuni effetti causando maggiore danno al soggetto vulnerabile.

Essere Vulnerabili ad un tipo specifico di Danno significa automaticamente raddoppiare il danno ricevuto prima di applicare qualsiasi altra resistenza o Tiro Salvezza.

Un creatura con una Vulnerabilità al Fuoco raddoppia tutto il danno subito poi se possibile effettua il TS indicato dall'incantesimo o effetto.

\subsection{Resistenza alla Magia}\index{Resistenza alla Magia}

Una creatura potrebbe avere una naturale resistenza alle Essenze.

Il valore di RM (Resistenza Magia) indica tale resistenza e più è alta più la creatura è immune alle Essenze, che lo voglia o meno.

Ogni qual volta la creatura è influenzata direttamente da una Essenza deve effettuare una prova di RM, ovvero tirare 3d6 sommare il valore di RM e se è superiore alla prova di magia effettuata dall'incantatore l'Essenza non ha effetto.

In caso di essenze scaturite da oggetti (anelli, bastoni, pozioni) la prova di RM deve superare 6+LP (livello di potere) della Essenza generata per annullarne gli effetti.

\subsection{Paura}\index{Paura}

\label{paura}

Essenze, Oggetti Magici e certe creature possono influenzare i personaggi con paura. In molti casi, il personaggio deve effettuare un Tiro Salvezza su Volontà per resistere agli effetti, e un tiro fallito indica che il personaggio è scosso, spaventato o in preda al panico.

\textbf{Scosso}\index{Scosso}

I personaggi che sono scossi subiscono penalità di -2 ai Tiri per Colpire, ai Tiri Salvezza e alle prove.

\textbf{Spaventato}\index{Spaventato}

I personaggi spaventati sono anche scossi, e inoltre fuggono dalla fonte della loro paura il più velocemente possibile, anche se possono scegliere la direzione di fuga. A parte cio', una volta che sono fuori vista (o udito) dalla fonte della loro paura, possono agire normalmente. Se la durata della paura non è ancora arrivata al termine, qualora dovessero incontrare di nuovo la fonte della loro paura, cercherebbero nuovamente di fuggire. I personaggi che non sono in grado di fuggire possono combattere (anche se continuano ad essere scossi).

\textbf{In Preda al Panico}\index{In Preda al Panico}

I personaggi in preda al panico sono scossi e, inoltre, hanno una probabilità del 50\% di far cadere a terra qualsiasi cosa stanno tenendo in mano, e di fuggire dalla fonte del loro terrore il più in fretta possibile seguendo un percorso di fuga completamente casuale. I personaggi in preda al panico fuggono davanti a qualsiasi altro pericolo che possano trovarsi di fronte.

A parte cio', una volta che sono fuori vista (o udito) dalla fonte della loro paura, possono agire normalmente. I personaggi in preda al panico prendono anche la condizione Accovacciato se non possono fuggire.

\textbf{Terrore Crescente}\index{Terrore Crescente}

Gli effetti della paura sono cumulativi. Un personaggio scosso che viene nuovamente scosso diventa spaventato, mentre invece un personaggio scosso cheviene spaventato cade in preda al panico. Un personaggio spaventato che viene scosso o spaventato cade in preda al panico.


\subsection{Paralizzato}\index{Paralizzato}

\label{paralizzato}

Un personaggio paralizzato è bloccato sul posto ed è incapace di muoversi od agire

Ha punteggi effettivi di Potenza e Agilità pari a -4 (-4 alla Difesa oltre ad non avere bonus di Agilità), è Indifeso e può compiere azioni esclusivamente mentali. Una creatura alata in volo, nel momento in cui viene paralizzata non può più battere le ali e precipita. Un nuotatore paralizzato non può più Nuotare e potrebbe annegare.

\pagebreak

\section{La Magia}\index{Magia}\index{Essenza}


\label{la-magia}
\begin{tcolorbox}[enhanced,arc=5pt,boxrule=0.3pt]{
"Le parole sono, nella mia NON modesta opinione, la nostra massima ed inesauribile fonte di magia. In grado sia di infliggere dolore che di alleviarlo" (Albus Silente)\\\\
Non lascerai vivere colei che pratica la magia. (Libro dell'Esodo)} \end{tcolorbox} \medskip

\begin{note}
	In TUS troverete due sistemi di Magia da utilizzare, il primo basato sulle Essenze il secondo basato su liste di incantesimi.

Le \textit{Essenze} sono la magia declinabile tramite l'associazione Verbo - Nome, ovvero la possibilita' di creare effetti magici associando un verbo (creare, distruggere, muovere..) ad un nome (persona, oggetto, elemento..). Questo sistema e' meno intuitivo nei primi utilizzi eppure una volta entrati nel processo creativo vi renderete conto di avere la possibilita' di fare quasi tutto.\\

Il sencondo sistema si basa su \textit{liste di incantesimi} prefatte dove il personaggio conosce degli incantesimi e lancia gli stessi, in un approccio molto consueto nei giochi di ruolo.\\
C'e' chiaramente qualche \emph{twist} per rendere la cosa piu' interessante e meno piatta.

\end{note}

\subsection{Le Essenze}

Si intende incantatore o mago qualsiasi usufruitore di Essenze a qualsiasi titolo ed uso.

La Magia ci circonda ed è accessibile, ma non tutti sanno dominarla e chi non la sa dominare ne viene dominato.

Sono considerazioni false, ma difficilmente contraddicibili al popolino.

Tra una levatrice che segue Atherim, un cavaliere di Sumkjr ed un negromante di Sixiser o una spia di Shayalia ci sono notevoli differenze di pensiero, comportamento ed azione.

Non sempre chi si dice di essere fedele ad un Patrono lo è o peggio lo è di qualcun' altro.

Bisogna sempre prestare attenzione ad un Devoto, i suoi comportamenti seguono interessi non sempre lineari od ovvi.

Solo essendo un Devoto si hanno i bonus concessi dal Patrono. \index{Devoto}

Le essenze scelte possono essere solo quelle offerte dal Patrono prescelto.

Se hai scelto di essere un Devoto,\index{Devoto} quindi hai almeno 3 tratti in comune, dovrà scegliere le Essenze del suo Patrono con le limitazioni e vantaggi indicati.

Se hai scelto di essere un Seguace \index{Seguace}allora hai almeno 2 tratti in comune, puoi scegliere tutte le Essenze che preferisci e non hai ne svantaggi ne vantaggi.

\subsubsection{Competenza Magica ed Essenza}\index{Competenza Magica}\index{Essenza}

\label{competenza-magica-ed-essenza}

Ogni qual volta il personaggio attribuisce un punto alla Competenza Magica può decidere se aggiungere due Essenze a quelle da lui conosciute, oppure attribuire un +1 ai check di magia ad una Essenza già conosciuta (bonus di specializzazione).

Quando deve fare una prova magia su una Essenza appresa ma non come specialista tirerà 3d6 + punteggio di Competenza Magica + caratteristica collegata + vari ed eventuali.

Quando deve fare una prova magia su una Essenza in cui ha dedicato una competenza aggiuntiva tirerà 3d6 + Competenza Magica + Bonus di specializzazione + caratteristica collegata + vari ed eventuali.

\bigskip

\textbf{Es. Un incantatore di 6 livello ha un punteggio di 6 in Competenza Magica} ed ha attribuito i suoi punti in questa maniera

Essenza Alterare (non ha assegnato punti aggiuntivi, ha preso solo l'Essenza)

Essenza Attacco +1 +1 +1

Essenza Rivelazione

Essenza Cura +1

Se deve usare una Essenza di Alterare o Rivelazione potrà fare una prova di CM a +6 (più caratteristica collegata), se deve fare una prova magia su Attacco il suo punteggio di CM è 6+1+1+1 (più la caratteristica collegata), mentre su Cura ha 6+1 (più la caratteristica collegata).

Questo bonus di specializzazione si somma anche nelle prove di Concentrazione che riguardino questa essenza.

\bigskip

\textbf{Un incantatore di 8 livello invece ha diviso 4 punti di Competenza
	Magica in questa maniera}:

Essenza Cura +1

Essenza Creazione

Essenza Protezione

Essenza Difesa +1

L'Essenza di Creazione e Protezione userà un CM a +4, per la Cura e Difesa a +5 e caratteristica collegata.

\bigskip

\textbf{Il punteggio di specializzazione che una Essenza può avere deve essere inferiore o pari a metà del valore di Competenza Magica}. Es. se hai CM a 4 il bonus di specializzazione massimo ad una singola Essenza può essere +2

\subsubsection{Le regole delle Essenze}\index{regole delle Essenze}

\label{le-regole-delle-essenze}

Ci sono dei punti fermi, delle regole che sovrintendono la magia e queste sono:
\begin{itemize}
	\item Non è permesso riportare in vita i morti. Solo un Patrono può restituire l'anima ad un corpo.

	\item Non è permesso creare vita

	\item Declama la tua magia o non funzionerà

\end{itemize}

\subsubsection{Creature ed Elementi}\index{Creature ed Elementi}

\label{creature-ed-elementi}

Ogni Essenza che si va a formulare ha un ambito di applicazione che riguarda \textbf{Creature} (a loro volta divise in \textbf{Creature Naturali}, \textbf{Creature Magiche}), \textbf{Elementi}, \textbf{Energia, Concetto o Virtu'}.\index{Creature Naturali}
\index{Creature}\index{Magiche}\index{Elementi}\index{Energia}\index{Concetto}\index{Virtu}
\bigskip

Le \textbf{Creature Naturali} sono Insetti, Rettili, Bestie, Umanoidi, Piante, Creature acquatiche, Monstrusita', Melme.

Le \textbf{Creature Magiche} sono: Immondi, Fatati, Spiriti, Non morti, Giganti, Celestiali, Costrutti, Aberrazioni (tutto ciò che e' alieno o innaturale) e Draghi.

Se una Creatura Naturale ha poteri magici allora si considera anche come Creatura Magica.

Una descrizione piu' completa di questi "mostri" la trovate nel Capito delle Mostruosità.

Gli \textbf{Elementi} sono: Acqua, Terra, Aria, Metallo, Legno, Ghiaccio, Nebbia

I \textbf{Concetti} sono: Spazio, Tempo, Essenza

\textbf{Energia} comprende: Fuoco, Luce, Suono, Elettricità, Energia Positiva, Energia Negativa, Freddo, Vuoto.

Le \textbf{Virtù} comprendono i Tratti


\bigskip

\textbf{Tabella Raggruppamenti Elementi e Creature}

\medskip
\begin{tabular}{llllll}
\toprule
	\multicolumn{2}{c}{\textbf{Creature}} &\multirow{2}{*}{\textbf{Energia}}  &\multirow{2}{*}{\textbf{Elementi}}  
	&\multirow{2}{*}{\textbf{Concetto}} &\multirow{2}{*}{\textbf{Virtù}}\\
	\textit{Naturali}& \textit{Magiche} \\
	\hline
	\\
Acquatiche  & Immondo   	& Fuoco  			& Acqua 	& Spazio    & Tratti\\
Piante      & Fatati   		& Suono  			& Aria   	& Tempo    	& \\
Rettili     & Spiriti   	& Elettricità      	& Terra     & Essenza   & \\
Umanoidi    & Non Morti 	& Energia Positiva 	& Legno     & 			& \\
Bestie   	& Aberrazioni   & Energia Negativa 	& Metallo   &  			& \\
Insetti		& Draghi		& Luce				& Ghiaccio 	&			& \\
Mostruosità & Giganti     	& Vuoto  			& Nebbia 	&           & \\
Melme		& Celestiali    & Freddo 			&		    &           & \\
			& Costrutti     &        			&		    &           & \\
\end{tabular}

\bigskip

Nelle specifiche delle Essenze troverete se queste lavorano su Elementi, Creature Naturali o Magiche, Energia, Concetti o Virtù o solo specifiche componenti di queste.

Il danno causato da \textbf{Luce} e' per meta' da fuoco e per meta' da energia positiva, ovvero una resistenza al fuoco od all'energia positiva si applica solo su meta' del danno causato dall'attacco.

Il danno causato da \textbf{Vuoto} e' per meta' da freddo e per meta' da energia negativa, eventuali protezioni si applicano alle rispettive meta' del danno.

\subsubsection{Caratteristiche base delle Essenze}\index{Caratteristiche base delle Essenze}

\label{caratteristiche-base-delle-essenze}

Ogni magia che si va a creare ha queste caratteristiche di base:

\smallskip

\textbf{Tempo di lancio}: due Azioni\index{Tempo di lancio}

\textbf{Durata}: istantanea\index{Durata}

\textbf{Distanza}: distanza di mischia (a tocco)\index{Distanza}

\textbf{Area di Effetto}: 1 creatura \index{Area di Effetto}

\textbf{Obiettivi}: quando lanci una Essenza determina se l'obiettivo è una Creatura o Elemento oppure un punto nello spazio entro la distanza stabilita.\index{Obiettivi}

\textbf{Obiettivi Speciali}:\index{Obiettivi Speciali} puoi anche lanciare una Essenza su un oggetto e la prima creatura che toccherà l'oggetto diventerà l'obiettivo della magia. La durata rimane limitata ad un minuto ed a costo 3.

\textbf{Potenziamenti}: l'incantatore decide di potenziare una Essenza come preferisce, aumentando la difficoltà di esecuzione della stessa. Consultare l'elenco per i dettagli\index{Potenziamenti}

\textbf{Sommando le varie caratteristiche di base della magia si determina la difficoltà totale da superare con una prova su CM + Punteggio Caratteristica correlata all'Essenza + Bonus.}

\textbf{Solo in caso di superamento si riesce a lanciare la magia e si verifica che livello di potere di potere si è raggiunto (una volta sottratta la difficolta')}

Di base una formulazione magica sara': applico l'Essenza X alla/e Creatura o Elemento Z o area che si trova Y distante.

\subsubsection{Tiro per Colpire ed Essenze}\index{Tiro per Colpire ed Essenze}

Quando la l'Area di Effetto e' una creatura ovvero la magia deve colpire una sola creatura e' necessario un Tiro per Colpire.\\
Questo Tiro per Colpire e' contro la Difesa a Tocco sia che l'Essenza venga consegnata tramite Tocco (ovvero la distanza e' mischia. TC su Potenza) oppure se la Distanza e' oltre alla mischia (TC su Agilita').\\
Se il TC va a segno allora la creatura fara' il Tiro Salvezza del caso.
Quando l'Area di Effetto e' data da piu' singoli soggetti (selezionati) devo fare un TC per ogni avversario.

Quando l'Area di Effetto e' ad area non e' necessario effettuare un TC se non per difficili e specificate aree, ovvero si mira in una area ben circoscritta.


\subsubsection{Recitare l'Essenza}\index{Recitare l'Essenza}\index{Recitare}

\label{recitare-lessenza}

Può sembrare sciocco o inutile ma se un giocatore non recita la sua Essenza questa non funzionerà.

In TUS la magia è libera e freeform basata sul Verbo-Nome, ovvero non ci sono liste di incantesimi, ogni giocatore si inventa gli effetti che vuole, prendendo ispirazione (e limiti) dalle linee guida della Essenza.

Il giocatore declamera la sua Essenza dichiarando che Essenza e a cosa la applica e formulera' la magia "Crea - Fuoco. Possa questa piana ardere come il Deserto di Fiamma di Daruk-Yum" ed in base alla prova effettuata vedrà se va a fuoco veramente oppure è poco più di una candela.

Il giocatore come visto sopra, e dettagliato successivamente, stabilisce il Verbo (l'Essenza) e il nome (su cosa applicarla) e poi sarà la prova di magia a stabilire quanto viene influenzato (l'effetto, ovvero il Livello di Potere) l'obiettivo.

Il Narratore deve preoccuparsi di fare declamare sempre l'Essenza, questo perché aiuta a comprendere cosa si vuole ottenere dall'Essenza, cosa che i fattori numerici (distanza, obiettivo, durata... ) non descrivono accuratamente.

\subsubsection{Potenziamenti delle Caratteristiche dell'Essenza}\index{Potenziamenti delle Caratteristiche dell'Essenza}

\label{potenziamenti-delle-caratteristiche-dellessenza}

I potenziamenti definiscono e migliorano la magia che si va a lanciare; questi possono riguardare Distanza, Area di effetto, Contingenza, Selezione, Durata.

La tabella va usata per determinare per ogni singolo fattore (Durata, Distanza, Obiettivo - Area di Effetto (AoE)) e si sommano le relative Difficolta' trovate.
A questo valore si sottrae l'eventuale riduzione dato dal Tempo di Lancio


\bigskip

\begin{tabularx}{0.95\textwidth}{lXXXXX}
	\hline
	\textbf{Difficoltà} &\textbf{Durata} &\textbf{Distanza} &\textbf{Obiettivo/AoE} & \textbf{Volume/Massa} &\textbf{Tempo di Lancio (riduzione)} \\
	\hline
	0	& Istantanea		& Tocco	& Se Stesso&& 2 Azioni\\
\hline
	+1	& Concentrazione - 1r*CM	& 3 metri& 1 obiettivo&& 1 round\\
\hline
	+2	& 1 minuto	& entro 10 metri&2 obiettivo - 3m/r&& 3 round\\
\hline
	+3	&	10 minuti& entro 50 metri& 3 obiettivi - 3m/r&& 5 round\\
\hline
	+4	& 20 minuti	& entro 100 metri&4 obiettivi - 6m/r &&1 minuto\\
\hline
	+5&30 minuti&entro 250 metri&5 obiettivi - 6 m/r&&5 minuti\\
\hline
	+6&45 minuti&&6 target - 8 m/r&&1 turno (10 minuti)\\
	\hline
	+7&1 ora&entro 500 metri&7 obiettivi - 8m/r&&1 ora\\
	\hline
	+8&4 ore&entro 700 metri&8 obiettivi - 10m/r&&3 ore\\
	\hline
	+9&6 ore&entro 1000 metri&9 obiettivi - 10m/r&&6 ore\\
	\hline	
	+11&12 ore&entro 3 km&11 obiettivi - 12m/r& 10cm\^{}3/100gr&1 giorno\\
	\hline	
	+13&1 giorno&entro 5 km&&20cm\^{}3/500gr&1 settimana\\
	\hline	
	+16&1 settimana&entro 10 Km&13 obiettivi - 14m/r&50cm\^{}3/3kg&1 mese\\
	\hline	
	+19&10 giorni&entro 20 Km&16 obiettivi - 18m/r&1m\^{}3/25kg&\\
	\hline	
	+20&2 settimane&entro 50 Km&20 obiettivi - 22 m/r&&-\\
	\hline	
	+22&3 settimane&entro 70 Km&50 obiettivi - 18m/r&2CB/100kg&\\
	\hline	
	+25&1 mese&entro 100 Km&una piccola citta'&4CB/200kg&\\
	\hline	
	+28&3 mesi&entro 150 Km&una citta'&8CB/400kg&\\
	\hline	
	+30&6 mesi&entro 250 Km&una grande citta'&&-\\
	\hline	
	+31&8 mesi&entro 300 Km&una regione'&16CB/800kg&-\\
	\hline	
	+34&9 mesi&entro 350 Km&una regione grande&32CB/1.6ton&-\\
	\hline	
	+37&11 mesi&entro 450 Km&una regione grande&64CB/3.2ton&-\\
	\hline	
	+40&1 anno&entro 500 Km&un continente&&-\\
	\hline	
	+43&5 anni&entro 1000 Km&piu' continenti&128CB/6.4ton&-\\
	\hline	
	+60&permanente&il pianeta&il pianeta&&-\\
	\hline	
\end{tabularx}
\bigskip

\textbf{Durata} \index{Durata}: per Durata di una Essenza si intende sia quanto dura l'effetto sia quanto lo si può trattenere sia la possibilita' di attivarlo a posteriori prima che debba manifestarsi. Un incantatore può trattenere un numero di round pari al suo valore in Competenza Magica + Intelletto.

L'Essenza di Cura e Attacco hanno sempre durata Istantanea ovvero producono gli effetti e cessano di essere attivi e possono essere attivati (contingenza) o manifestati (concentrazione) in base alla Durata.

La Distruzione di materia è \textbf{sempre permanente} come durata ed istantanea come effetto ed ha costo di +8.

In caso di contingenza ovvero di lancio posticipato di una Essenza a seguito di un evento scatenante concordato, la difficolta' della Durata e' pari alla meta' della difficolta' data della Durata stessa.

Es. se voglio che una Cura mi si scateni entro il giorno (24 ore) appena scendo sotto i 10 PF la difficolta' aumenta di 7 (13/2).


\textbf{Distanza} \index{Distanza}: Per Distanza si intende a che misura si deve manifestare l'Essenze.
Qualsiasi distanza oltre se stessi o tocco, quindi in mischia, aumenta la difficolta'.


\textbf{Obiettivo - Area di Effetto} \index{Target}\index{Area di Effetto}: 1 soggetto (+1): per ogni soggetto influenzato. Se il l'obiettivo e' di taglia superiore alla media la difficoltà diventà +2 per obiettivo.

I soggetti influenzati dalla medesima essenza devono essere entro 3 metri dal primo obiettivo oppure è necessario operare tramite un area di effetto circolare (es in 3 metri di raggio.)

Ogni +2 l'Area di Effetto aumenta di 3 metri.

L'Area di Effetto solitamente si usa su Creature.

\textbf{Volume/Massa}: le Essenze di Trasformazione, Creazione, Movimento, Distruzione agiscono su dei Volumi o Masse predefinite, non hanno un risultato varibile in base alla prova di magia effettuata a differenza di Attacco o Cura ad esempio.

Per queste Essenze una volta calcolate le Difficolta' base e' sufficiente con la prova di magia superare di 13 le Difficolta'.

Un risultato superiore a 13 nella prova di magia non otterà risultati concreti o oggettivi superiori ma il Narratore potrebbe decidere e farvi descrivere quanto bene e' venuto il risultato finale.

La dicitura CB sta per Cubo Base, ovvero un cubo con spigolo 1m*1m*1m.

\textbf{Deselezione} (+1)\index{deselezione}: con questo potenziamento escludi una creatura od oggetto dall'area di  effetto. Ogni +1 toglie una persona dagli effetti della magia (se Area di Effetto a Raggio).

Es. voglio tirare una Fuoco Palla toroidale attorno a me. Pago +2 per i tre metri di raggio e +1 di Deselezione (mi escludo dall'esplosione).

Es. voglio tirare una Fuoco Palla ai miei nemici intorno a me. Pago +4 (perché scelgo 4 soggetti) nella Area di Effetto e su ognuno di loro "cadrà" una Fuoco Palla di che interesserà solo loro singolarmente. I soggetti influenzati devono essere tutti nell'Area di Effetto. In questo caso l'area di effetto e' sempre singola per obiettivo.


\textbf{Tempo di Lancio} \index{Tempo di lancio} \index{Riduzione costi}
Aumentando significativamente il tempo di lancio  si puo' diminuire la difficolta' totale.
Il valore di difficoltà ottenuto aumentando il tempo di lancio si somma alla prova di CM fatta.

\subsubsection{Aree di effetto diverse}\index{Aree di effetto diverse}

\label{aree-di-effetto-diverse}

\textbf{L'Area di Effetto può essere non solo sferica, ma anche una linea od un cono.}

L'usufruitore di magia potrà restringere l'area di effetto, fino ad essere uno spicchio (il cono) della circonferenza iniziale oppure una linea.

La lunghezza dell'effetto in linea e' pari al doppio dell'Area di Effetto. Quindi un Essenza di Attacco con Area di Effetto 12m/r diventa una linea larga un metro (fisso) con lunghezza dal punto di origine (Distanza) di 24 metri.

In caso di effetto a cono il punto di origine e' sempre il mago a la distanza raggiunta e' quella pari all'Area di Effetto mentre la parte finale e' pari alla meta' dell'Area di Effetto.

Es. Tramite l'Essenza di Creazione voglio creare uno sbuffo di vento a forma di cono.
Con un Area di Effetto di 10m/r (Difficolta' 9) posso creare un cono che parte dalla mia mano e lungo 10 metri con una parte finale larga 5 metri.

Tenete a disposizione dei segnalini per "disegnare" l'area di effetto.

\subsubsection{Influenzati da più Essenze}\index{Influenzati da più Essenze}

\label{influenzati-da-piu-essenze}

Quando un personaggio è influenzato da \textbf{due o più effetti temporanei creati da Essenze} che danno lo stesso tipo di bonus, malus o danno (protezione verso fuoco, bonus alla Difesa o TS... , multiple palle di acido), si tiene conto solo di quella dal livello di potere maggiore.

\subsubsection{Scegliere l'effetto dell'Essenza}\index{Scegliere l'effetto dell'Essenza}

\label{scegliere-leffetto-dellessenza}

Nella descrizione delle Essenze quando trovate per un livello di potere elencati più possibilità, dovete sceglierne uno solo.

Esempio:

\medskip

\begin{tabularx}{0.95\textwidth}{lX}
	\toprule
	<11 & Rimuovi la condizione abbagliato\\
	    & Curi 1d6 pf
\end{tabularx}

oppure se sono separate da una {/}
 
Esempio: \textit{19 - Attribuisci la condizione di: Malato / Accecato / Assordato / Esausto / Nauseato}

\subsubsection{Altre regole}

\label{altre-regole}

\subsubsection{Attacco con Essenze non di Attacco}\index{Attacco con Essenze non di Attacco}

Alcune Essenze implicano un danno anche se non sono Essenze di Attacco, come riportato negli esempi per Alterazione, ma concettualmente valido anche per altre Essenze

Se l'incantatore acquisisce la capacità di un attacco (solitamente magico e non naturale) tramite una Trasformazione od una Alterazione, ma anche Creazione (vedi pioggia di fuoco..) potrà usare questi poteri dal round successivo facendo un danno di due categoria di Livello Potere immediatamente inferiore a quello ottenuto se fosse stato nella Essenza Attacco, se questo è di forma magica.

Es. Creo un Muro di Fuoco, la magia ha successo e creo un muro con LP 25 (4 cubi base). Il danno per chi' attraversa o ci e' in contatto e' di 5d6, come se fosse una Essenza di Attacco a LP 19.

Si deve considerare che la manifestazione dell'Essenza si completi e sia usabile dal round successivo. La manifestazione fara' effetto il round successivo nella sequenza di iniziativa pari a quella del round precedente.

Se acquisisce una forma di attacco naturale potrà comunque usare l'attacco dal round successivo con un danno coerente alla forma di attacco acquisito (morso, artiglio..).

\subsubsection{Usare due Essenze concatenate}\index{Usare due Essenze concatenate}

Ci sono situazioni in cui diviene necessario usare due Essenze una dietro l'altra, in questo caso il tempo di lancio aumenta in modo significativo.

Partendo dal presupposto che si devono conteggiare i potenziamenti base (distanza, obiettivo, durata..) per ogni Essenza usata si deve fare un solo check di competenza magica con la difficoltà più alta. Il tempo di lancio aumenta di 2 round.

Se quindi lanciare una magia di norma costa due Azioni, lanciare due Essenze porta il tempo di lancio a 3 round. Il Lancio di tre Essenze viene terminato alla fine dei 5 round. Il livello di potere raggiunto sarà il medesimo (essendo unica la difficoltà, quella maggiore) per tutte le magie accodate.

\subsubsection{Essenze Cumulate}\index{Essenze Cumulate}

Il giocatore potrebbe volere sommare in un unico lancio di magia piu' essenze.

Ad esempio potrebbe declamare una magia di Attacco che insegua (Movimento) l'obiettivo, oppure una Illusione (Illusione) che effettivamente scaldi (Creazione fuoco).


In questo caso si devono conteggiare un unica volta potenziamenti, e quindi la magia formulata anche se e' generata da due Essenze ha una sola comune distanza, obiettivo, durata.., il successo ottenuto (Prova di Magia - potenziamenti) va dimezzato e confrontato con i Livelli di Potere delle Essenze utilizzate.


\subsubsection{Alterare le Essenze}\index{Alterare le Essenze}

\label{alterare-le-essenze}

Il mago può modificare a piacimento la difficoltà della magia che va a formulare tramite le proprie energie vitali.

\begin{itemize}
	\item
	      \textbf{Magia efficace}
	      Magia efficace: sacrificando PF puo’ aumentare la difficolta’ a resistere alla magia
	      \begin{itemize}
      \item Sacrificando 4 punti ferita la difficolta’ del Tiro Salvezza aumenta di 1
      \item Sacrificando 8 punti ferita la difficolta’ del TS aumenta di 2

      \item Sacrificando 16 punti ferita la difficolta’ del TS aumenta di 3
	      \end{itemize}
\end{itemize}
%
\begin{itemize}
	\item
	      \textbf{Magia eterea}: aumentando di 2 la difficoltà di lancio (ovvero tolgo 2 al risultato della prova di competenza magica) le proprie magie hanno pieno effetto su creature eteree o incorporee
\end{itemize}
%
\begin{itemize}
	\item
	      \textbf{Magia pietosa}: aumentando di 3 la difficoltà di lancio (ovvero tolgo 3 al risultato della prova di competenza magica) le magie infliggono danni temporanei. Le magie che infliggono danni di un tipo particolare (come da fuoco) infliggono danni temporanei dello stesso tipo.
\end{itemize}

\subsubsection{Esecuzione di Essenze in maniera collaborativa}\index{Esecuzione di Essenze in maniera collaborativa}\index{Collaborativa}

Nel caso in cui si voglia usare un Essenza con una difficoltà totale non raggiungibile è possibile, sotto certi limiti, fare in modo che un gruppo di incantatori riesca nell'impresa.

Si divide la difficoltà totale (DC) della Essenza da lanciare tra i vari incantatori (non più di 7 incantatori possono partecipare) ed ogni incantatore deve superare una prova pari al doppio della difficoltà ottenuta.

Il tempo di lancio è di 1 round per ogni incantatore impegnato nella formulazione.

Es. 5 incantatori vogliono lanciare una Protezione estesa e duratura, per un a difficoltà totale minima 32. In questa situazione ogni incantatore deve fare una prova di CM (32/5)x2= 14, ovvero ogni incantatore deve superare una prova di CM a difficoltà 14. Se anche un solo incantatore sbaglia la prova al termine del lancio dell'Essenza, dopo 5 round, questa fallirà e l'Essenza non sarà lanciata.

\subsubsection{Tentare Essenza con impedimenti}\index{Tentare Essenza con impedimenti}\index{impedimenti}

se mani e bocca sono bloccati l'incantatore non può formulare Essenze. Per usare una Essenza è necessario avere entrambe le mani e la bocca liberi.

Aumentando di 5 la difficoltà puoi non usare le mani, se aumenta di 10 la difficoltà puoi non usare la bocca. Quindi se l'incantatore è legato ed imbavagliato può lanciare una Essenza con la sola forza del pensiero con una difficoltà aumentata di 15, ovvero la difficoltà base aumenta di 15.

Una Essenza lanciata con impedimenti se non supera 11 come valore non ottiene l'effetto minimo dell'essenza (e si considera comunque formulata).

\subsubsection{Riuscire e Fallire nella prova di Magia}\index{Riuscire e Fallire nella prova di Magia}

\label{riuscire-e-fallire-nella-prova-di-magia}

Per capire se si riesce nella Magia si deve innanzitutto superare, con una prova di Competenza Magica (3d6 + CM + Punteggio Caratteristica correlata all'Essenza + Bonus) il valore dato dalla somma ottenuta da Tempo di Lancio, Durata, Distanza, Area di Effetto/Obiettivi ai relativi valori scelti.

Si sottrae al valore ottenuto nella prova di Competenza Magica la somma delle difficoltà base (Tempo di Lancio, Durata, Distanza, Area di Effetto/Obiettivi, Potenziamenti...) e si controlla il risultato nella colonna Livello di Potere dell'Essenza usata per verificarne l'effetto ottenuto.

Nella tabella delle Essenza il primo livello di potere è indicato come "< xx", ovvero se si riesce a superare la difficoltà impostata dai fattori base ed il valore eccedente è inferiore a xx, si usa quel effetto.

Se non si riesce a superare la difficoltà base l'Essenza non avrà effetto e manifestazione e sara' contata tra le Essenze usate nel giorno.

In sintesi l'incantatore deve superare, con il suo check su Competenza Magica, la difficoltà data dai parametri base (Area di Effetto, Distanza, Durata, impedimenti..) se supera questo valore ha di sicuro ottenuto il valore minimo di effetto, controllando di quanto ha superato il valore capira' il livello di potere ottenuto.

Il sistema privilegia un uso accorto e ragionato delle Essenze, il giocatore e' spinto a calcolare ed usare sempre il minimo sufficiente di Area di Effetto, Durata, Distanza proprio per ottenere il massimo effetto (piu' sono alti i fattori di base piu' e' probabile che il livello di potere ottenuto sia basso)
Tentare una cura planetaria puo' essere divertente, ma e' solo uno spreco di Essenza, a meno di non essere un Patrono o quasi.

\bigskip

\textbf{Un incantatore può sempre scegliere un Livello di potere inferiore rispetto a quello ottenuto.}

Ad esempio voglio shockare con l'elettricità un avversario:

Distanza: nasce dal palmo della mano, ovvero ha come distanza massima mischia (tocco), difficoltà +0

Area di Effetto: un solo obiettivo, difficoltà +1 (non e' 0 un quanto la magia agisce non su me stesso)

Durata: 0, istantanea, difficolta'+0

La difficoltà base è quindi 1, Se con la prova di magia ottengo 8 (o comunque 1 o piu') avrò il minimo effetto, ovvero uno shock da 1d6 di danno.

\bigskip

\textbf{Un incantatore può formulare nel giorno un numero di Essenze pari a (CM/2)+3.} \index{Magie al giorno}

\textbf{Se nel lancio di una Essenza ottiene almeno un critico (esplosione di magia) non si computa questa Essenza per il numero di Essenze lanciabili al giorno.}

Come si evince nessun incantatore ha il perfetto controllo delle Essenze dato che non può controllarne a pieno la forza.

Portare un armatura senza le dovute competenze ed Abilità rende più difficile la prova di Competenza Magia. Vedere il capitolo armature per le penalità relative.

\subsubsection{L'esplosione del 6 nella Magia}\index{esplosione del 6 nella Magia}

\label{lesplosione-del-6-nella-magia}

Anche nella prova di Competenza Magica i 6 esplodono, ma in maniera diversa.

I 6 tirati nella prova di CM vengono ritirati, e ritirati ancora nel caso, ma qualsiasi 6 successivo ai primi tre tiri, anche se ritirati, non si somma per determinare il totale della prova di magia.

Ogni due 6 tirati si aumenta di uno il livello di potere ottenuto, si scale al livello immediatamente successivo.

Es. Tups vuole incenerire l'orchetto che lo sta caricando. La sua prova di Competenza Magica è data da 3d6 + 7. Tira con i dadi 6, 4, 3. Quindi la sua prova ha un totale di 20.

Ritira poi il 6 ed ottiene un altro 6, ritira anche questo e ottiene un altro 6! La situazione è decisamente esplosiva!!! Ritira ancora e ottiene un 2.

Con la prova di CM a 20 (si contano solo i primi 3d6 tirati piu' bonus di Intelletto e CM), data la distanza entro 10 metri (costo 2), la selezione (1 soggetto, costo 1) il danno è 5d6 ma avendo fatto ben tre 6 nel tiro il livello di potere aumenta di 1, arrivando il danno a ben 7d6. L'orchetto è incenerito a dovere!

Per le prove di Competenza Magica l'uno non viene conteggiato, conta 0.

\subsubsection{Tentare la sorte con la Magia}\index{Tentare la sorte con la Magia}

\label{tentare-la-sorte-con-la-magia}

Anche nella prova di competenza magica puoi Tentare la Sorte, ovvero rinunci ad un +4 di bonus (da CM, Intelletto, non da oggetti magici...) e aggiungi un d6 in più nel tiro della prova.

\subsubsection{Resistere all'Essenza (Tiro Salvezza)}\index{Resistere all'Essenza}\index{Tiro Salvezza}

\label{resistere-allessenza-tiro-salvezza}

Una volta che la prova di magia è superata e quindi l'Essenza liberata, anche in base alla descrizione e note dell'Essenza, è possibile dimezzare o annullare l'effetto dell'Essenza.

Il tiro salvezza richiesto, in base a quanto indicato nell'Essenza, ha difficoltà pari alla stessa prova superata dal incantatore con +3 per ogni due 6 ottenuti nella prova.

Se il Tiro Salvezza riesce o fallisce di più di 10 (\textbf{successo critico}\index{Successo Critico} o \textbf{fallimento critico}\index{Fallimento Critico}) il Narratore potrà decidere di applicare svantaggi o vantaggi al risultato finale.\index{Più di 10}.

Nella descrizione delle Essenze è indicato cosa succede in caso di riuscita o fallimento del Tiro Salvezza ed anche se e' possibile un successo o fallimento critico.

\subsubsection{Più Essenze nello stesso round}\index{Più Essenze nello stesso round}

Ad alti livelli un incantatore può usare i Livelli di Poteri inferiore con estrema facilità fino a poter usare più Essenze nello stesso round.

L'incantatore può lanciare più Essenze nello stesso round purché la somma dei Livelli di Potere usati non superi il suo punteggio in CM+Intelletto.

Questa capacità non è usufruibile prima di avere CM a 22

\subsubsection{Mantenere la Concentrazione}\index{Mantenere di Concentrazione}\index{Concentrazione}

Una Essenza formulata puo' essere conservata nel mago per 1 round per CM aumentando la Difficolta' di 1.
Il mago non puo' pero' formulare altre Essenze finche' mantiene la concentrazione attiva.

Quando vuole rilasciare l'Essenza formulata dovra' tirare l'iniziativa e rilasciarla al momento stabilito.

La Durata in si intende di conservazione prima dell'effetto non di Durata dell'effetto.

\subsubsection{Check di Concentrazione}\index{Check di Concentrazione}

Se il mago viene è severamente distratto, impedito, disturbato, sotto attacco, mentre effettua una magia la prova di magia questa deve riuscire di almeno 15 altrimenti la "distrazione" e' stata tale da impedire il buon esito della magia.

Se il mago e' colpito prima di lanciare una Essenza la prova stessa deve riuscire almeno 10 + danno subito altrimenti l'incantesimo non riesce.

\subsubsection{Un ultimo suggerimento}

L'ultimo consiglio è infine rivolto specificatamente ai Narratore, lasciate che i giocatori si esprimano inventando nuove magie e manifestazioni curiose e poco ortodosse. Cercate di valutarne la correttezza ricordando che le Essenza e come sono descritte vogliono essere degli esempi. Lo scopo finale è sempre e solo divertirsi.

\subsubsection{Lista delle Essenze}\index{Lista delle Essenze}\index{Essenze}

\begin{itemize}

	\item
	      \textbf{Alterare} (Intelletto): la capacità di alterare il corpo per dargli capacità o aspetto diverse o superiori\index{Alterare}
	\item
	      \textbf{Attacco} (Intelletto): la capacità di utilizzare la magia per attaccare e fare danno\index{Attacco}
	\item
	      \textbf{Charme} (Magnetismo): la capacità di controllare pensieri
	      ed emozioni di altre creature\index{Charme}
	\item
	      \textbf{Convocazione} (Intelletto): la capacità di chiamare l'archetipo
	      della creatura.\index{Convocazione}
	\item
	      \textbf{Creazione} (Volonta): la capacità di creare oggetti, materiali o elementi liberi\index{Creazione}
	\item
	      \textbf{Cura} (Volonta): la capacità di curare o ripare esseri viventi o oggetti\index{Cura}
	\item
	      \textbf{Difesa} (Magnetismo): la capacità di proteggersi contro il danno, magico o normale\index{Difesa}
	\item
	      \textbf{Distruzione} (Volonta): la capacità di distruggere oggetti, materiali, elementi liberi o creature o anche equilibri organici\index{Distruzione}
	\item
	      \textbf{Illusione} (Magnetismo): la capacità di produrre illusioni più o meno reali e complesse
	\item
	      \textbf{Movimento} (Agilita)': la capacità di influenzare qualsiasi tipo di movimento quali il volo, levitazione, movimento del corpo o muovere oggetti.\index{Movimento}
	\item
	      \textbf{Protezione} (Potenza): la capacità di proteggere da veleni, malattie, controllo del pensiero, elementi, dall'ambiente..\index{Protezione}
	\item
	      \textbf{Rivelazione} (Magnetismo): la capacità di aumentare la consapevolezza nel proprio ambiente e utilizzare la magia per osservazione e divinazione\index{Rivelazione}
	\item
	      \textbf{Trasformazione} (Potenza): la capacità di trasformare un elemento o creatura in un altro elemento e /o creature\index{Trasformazione}

\end{itemize}

\bigskip

Suggerisco di segnarsi nella scheda le Essenze e formulazioni più usate, quasi a creare un proprio libro di magia così che sia più facile calcolare i costi degli incantesimi tipici.

\pagebreak

\subsubsection{Essenza Alterare}\index{Essenza Alterare}

\textbf{Caratteristica}: Intelletto\\
\textbf{Verbo}: Alterare\\
\textbf{Nome}: Creature, Elementi\\

\label{essenza-alterare---intelletto}

\textbf{Alterare} è la capacità di \textbf{donare capacità non possedute} ad una creatura o elemento.
Alterare permette di modificare una Creatura o Elemento con le qualita' di una Creatura oppure un Elemento o della Energia.
Tramite un Essenza di Alterare puoi conferire la resistenza della pietra o delle branchie per   respirare sott'acqua.
Ma anche delle ali da un pegaso, o le mani roventi oppure il potente soffio di un drago.

\bigskip

\textbf{Essenza Alterare}
\begin{itemize}
	\item
	      Al target viene concesso un Tiro Salvezza su Arbitrio per negare gli effetti
	\item
	      Per modifiche minori si intende: aspetto (occhi, bocca, naso, capelli), respirazione, forme di attacco naturali
	\item
	      Per modifica maggiori si intende: razza, sesso, movimento
	\item
	      Per modifica superiori si intendono capacità magiche (movimento/attacco..)
\end{itemize}

\bigskip

\begin{tabularx}{0.95\textwidth}{lX}
	\toprule
	\textbf{Livello di Potere} & \textbf{Creature Naturali}\\
	<=11   & Concedi al target la Visione Crepuscolare a distanza di 18 metri\\
	13     & Concedi al target un +1 in una caratteristica\\
	       & Concedi al target la Visione Crepuscolare a distanza di 36 metri\\
	16     & Concedi al target un +2 in una caratteristica\\
	19     & Respiri sott'acqua\\
	       & Concede al target una modifica minore del corpo.\\
	       & Concedi al target delle ali. +1 Azione Movimento, manovrabilita’ bassa\\
	22     & Concedi al target di potersi adattare ad un ambiente di fuoco.\\
	       & Concede al target due modifiche minore del corpo.\\
	       & Concedi al target delle ali. +2 Azioni Movimento, manovrabilita’ media\\
	25     & Concedi al target di adattarsi ad un elemento dell’Essenza Attacco. \\
	       & Concede al target una modifica maggiore del corpo ed una minore\\
	       & Concedi al target delle ali. +4 Azioni Movimento , manovrabilita’ alta\\
	28     & Concede al target due modifiche maggiori del corpo e due minori\\
	31     & Concede al target una modifica superiore e due maggiori\\
\end{tabularx}

\bigskip

Esempi:
\begin{itemize}
	\item
	      "Con il potere della natura. Questo ragno mi concederà la sua tela"
	\item
	      "Per il soffio del grande Gurthok. Possa io soffiare fiamme" .

	      Solo dal round successivo potrò soffiare fiamme e potrà sfruttare l'Alterazione come forma di Attacco con un livello di potere di due gradini inferiori a quello ottenuto per l'Alterazione.
\end{itemize}

\pagebreak

\subsubsection{Essenza Attacco}\index{Essenza Attacco}

\textbf{Caratteristica}: Intelletto\\
\textbf{Verbo}: Attaccare\\
\textbf{Nome}: Creature, Elementi\\

\label{essenza-attacco---intelletto}
\begin{itemize}
	\item
     Essenza \textbf{Attacco significa generare Energia come attacco contro l'avversario.}\\
Va sempre specificato la forma di energia con cui si attacca ed eventualmente verificato con gli elementi di Attacco del Patrono.
	\item
Al target viene concesso un Tiro Salvezza su Riflessi per dimezzare il danno. In caso di \textbf{successo critico} si dimezza ulteriormente. In caso di fallimento critico si raddoppiano i danni.
	\item
La Durata massima di un Essenza di Attacco è sempre istantanea, non puoi affogare un soggetto semplicemente riempiendo la stanza con un attacco ad acqua, ne puoi creare una Fuocopalla Ardente che brucia per un minuto.
Nota: dovresti usare Creazione in entrambi i casi.
	\item
Se si Attacca con energia negativa un non morto lo si cura, con energia positiva lo si danneggia
	\item
Se si Attacca con energia positiva un vivente non gli si fa nulla (non lo si cura), con energia negativa lo si danneggia
	\item
Se si attacca con una forma di Energia che ha il proprio Patrono o energia neutrale (-) il danno è quello riportato in tabella, altrimenti si ottiene il Livello di Potere inferiore a quello determinato.
\end{itemize}

\bigskip

\begin{tabularx}{0.95\textwidth}{lX}
	\toprule
	\textbf{Livello di Potere} & \textbf{Luce(P), Energia Positiva(P), Fuoco(-) Elettricità(-), Freddo (-), Suono(-), Energia Negativa(N), Vuoto(N)}\\
	<=11                       & 1d6\\
	13                         & 2d6\\
	16                         & 3d6\\
	19                         & 5d6\\
	22                         & 7d6\\
	25                         & 10d6\\
	28                         & 13d6\\
	31                         & 15d6\\
	34                         & 18d6\\
	37                         & 20d6\\
	43                         & 25d6\\
\end{tabularx}

P = Energia Positiva, - = Energia neutra, N = Energia Negativa
\bigskip

Esempi:
\begin{itemize}
	\item
	      "Mani brucianti di Alac Zalzir"
	\item
	      "Invoco i demoni dei ghiacci che infilzino i miei avversari nelle loro gelide lance"
	\item
	      "Canto i segreti riti di Zungur e rompo il dito secco della pettegola perché la mia voce tramortisca i miei avversari"
	\item
	      "Traccio nell'aria le antiche rune di Boz Dan Don e tre lame di acciaio trafiggano i nemici"
	\item
	      "Per tutte le battaglie: Palla di Fuoco!"
	\item
	      "IT'S OVER 9000!"
\end{itemize}
\bigskip

Un attacco di Luce causa danno suddiviso equamente da calore (assimilabile a fuoco) e da energia positiva.

Esempio pratico:

\textbf{Mani brucianti di Alac Zalzir}

Distanza: esce dal palmo della mano, ovvero ha come distanza massima è mischia, difficoltà +0

Area di Effetto: un solo obiettivo, difficoltà +1

Durata: 0, istantanea, difficolta'+0

Quindi fatte le somme (Distanza + AoE + Durata) questa versione di Mani Brucianti ha difficoltà 1.

Se con la prova faccio 15 ottengo un livello di potere pari a 14 (15-1) che significa che le mie Mani Brucianti fanno 2d6 di danno

\textbf{Fuocopalla Ardente}

Distanza: entro i 10 metri, difficolta +2

Area di Effetto: distanza 3 metri radius, difficolta +2

Durata: istantanea, difficoltà 0

I costi base sono 4, se con la prova faccio 24, avrò un livello di potere pari a 20, sufficiente per fare 5d6 di danno!

\pagebreak

\subsubsection{Essenza Charme}\index{Essenza Charme}

\textbf{Caratteristica}: Magnetismo\\
\textbf{Verbo}: Charme\\
\textbf{Nome}: Creature\\

\label{essenza-charme---magnetismo}

\begin{itemize}
	\item
	      L'Essenza Charme \textbf{agisce sull'attitudine della Creatura}. Il soggetto deve essere senziente e con Volontà ed Intelletto maggiori o uguali a -2
	\item
	      L'Essenza di Charme permette anche di comunicare non verbalmente. Attenzione alla difficoltà data da Durata e Obiettivi e Distanza
	\item
	      Non si può usare l'Essenza di Charme su creature con 3 CR superiori alla propria CM se l'obiettivo non e' consenziente.
	\item
	      I CR indicati si riferiscono alla sommatoria di creature influenzate.
	\item
	      Al target viene concesso un Tiro Salvezza su Arbitrio per negare gli effetti. In caso di fallimento critico la durata viene raddoppiata.
	\item
Puoi anche influenzare emotivamente l'obiettivo rendendolo piu' coraggioso, impaurito, attento, impavido, brutale...      
\end{itemize}

\medskip

\begin{tabularx}{0.95\textwidth}{lX}
	\toprule
	\textbf{Livello di Potere} & \textbf{Creature Naturali o Magiche}\\
	<=11  & Influenzi obiettivi fino a CR 1/3\\
	13    & Influenzi obiettivi fino a 1 CR  \\
	      & Comunichi con una creatura non più di 144 caratteri telepaticamente che capisca la tua lingua. \\
	16    & Influenzi obiettivi fino a 2  CR\\
          & Comunichi con una creatura telepaticamente che capisca la tua lingua.\\
	19    & Influenzi obiettivi fino a 3 CR  \\
          & Comunichi con una creatura telepaticamente che non capisca la tua lingua ed abbia Intelletto 2 o piu' \\
	22    & Influenzi obiettivi fino a 5 CR \\
	      & Comunichi con una creatura telepaticamente che non capisca la tua lingua e abbia Intelletto 1 o più.  \\
	25    & Influenzi obiettivi fino a 7 CR \\
	28    & Influenzi obiettivi fino a 9 CR \\
	      & Comunichi con un obiettivo telepaticamente che non comunichi verbalmente. \\
	31    & Influenzi obiettivi fino a 15 CR \\
	34    & Influenzi obiettivi fino a 11 CR \\
	37    & Influenzi obiettivi fino a 13 CR \\
	43    & Influenzi obiettivi fino a 15 CR \\
\end{tabularx}

Una Creatura Impaurita ha un -2 TC, Coraggiosa +2 TC, Prudente +2 TS, Eroica +1 Difesa +1 TC... queste sono solo linee guida al giocatore e Narratore la possibilita' di creare innumerevoli frasi motivazionali o demotivazionali.

\medskip

Un incantatore può utilizzare un Essenza di Charme per influenzare e quindi rendere Amichevoli od Impaurire, a seconda della differenza tra la CM dell'incantatore e CR dell'obiettivo si possono avere effetti diversi.

\medskip

Se la creatura ha:

\begin{tabular}{L{2.5cm} L{7cm} L{7cm}}
	\toprule
  & \textbf{Rendere Amichevole}         & \textbf{Impaurire}\\
   CM-CR è 1 o più   & fallimento di 2 o meno la creatura è amichevole     & se il TS fallisce di 2 o meno la creatura è scossa\\
   & fallimento di 3 la creatura è affascinata     & fallimento di 3 la creatura è spaventata\\
   & fallimento di 4 la creatura è charmata        & fallimento di 4 la creatura è in preda al panico\\
  & se il TS fallisce di 5 o più la creatura è dominata         & \\
 CM-CR tra 0 e -1  & fallimento di 3 o meno la creatura è amichevole   & fallimento di 3 o meno la creatura è scossa\\
 & fallimento di 4 la creatura è affascinata  & fallimento di 4 la creatura è spaventata\\
 & fallimento di 5 la creatura è charmata     & fallimento di 5 o più la creatura è in preda al panico\\
 & se il TS fallisc di 6 o più la creatura è dominata          & \\
 CM-CR tra -2 e -3 & 4 o meno la creatura è amichevole  & fallimento di 4 o meno la creatura è scossa\\
& fallimento di 5 la creatura è affascinata      & fallimento di 5 la creatura è spaventata\\
& fallimento di 6 la creatura è charmata         & fallimento di 6 o più la creatura è in panico\\
 & se il TS fallisce di 7 o più la creatura è dominata & \\
\end{tabular}


\bigskip

Esempi:
\begin{itemize}
	\item
	      "Per il potere di Garya. Possa il mio tocco renderti docile"
	\item
	      "Racconto la storia della stupenda Aralda Hucnoss e fisso gli occhi dell'orco. Ora sei mio, dolce amore"
	\item
	      "Sembra talco ma non e', serve a darti l'allegria. Se lo mangi o lo respiri ti da subito l'allegria!"
\end{itemize}

\pagebreak

\subsubsection{Essenza Convocazione}\index{Essenza Convocazione}

\textbf{Caratteristica}: Intelletto\\
\textbf{Verbo}: Convocazione\\
\textbf{Nome}: Creature\\

\label{essenza-convocazione---intelletto}

L'Essenza Convocazione è la \textbf{capacità di richiamare l'archetipo di una Creatura per farla agire al tuo fianco.}
\begin{itemize}
	\item
      Il CR indicato si riferisce alla sommatoria dei CR totali convocati di creature naturali
	\item
      Un incantatore non può evocare creature con più di 3 CR rispetto al suo valore di CM
	\item
      Evocare una creatura magica o Elementale costa a parità di CR il livello di potere immediatamente successivo. Con LP 22 evochi una creatura magica con CR 3
	\item
      Evocare un Drago a parità di CR è più difficile di 2 livelli. Con LP 31 evochi un drago CR 7
	\item
      CR 13 è il massimo CR di creature convocabili, oltre sono sempre sommatorie di più creature convocate.
\end{itemize}

\bigskip

\begin{tabular}{ll}
	\toprule
	\textbf{Livello di Potere} & \textbf{Creature}\\
	\textless=11               & Convochi fino a 1/3 CR\\
	13                         & Convochi fino a 1/2 CR\\
	16                         & Convochi fino a 1 CR\\
	19                         & Convochi fino a 3 CR\\
	22                         & Convochi fino a 5 CR\\
	25                         & Convochi fino a 7 CR\\
	28                         & Convochi fino a 9 CR\\
	31                         & Convochi fino a 11 CR, max CR 10\\
	34                         & Convochi fino a 13 CR, max CR 11\\
	37                         & Convochi fino a 15 CR, max CR 12\\
	43                         & Convochi fino a 17 CR, max CR 13\\
\end{tabular}

\bigskip

Esempi:
\begin{itemize}
	\item
	      "O sommi antenati concedetemi la sapienza di richiamare il mammuth lanoso"
	\item
	      "Dalle cime delle montagne più alte urlo il richiamo di Ferlin Caf. A me venga il Drago di bronzo"
	\item
	      "Lime, Rum, Ghiaccio, Menta e Zucchero grezzo. Agito e offro. Io convoco il Pirata Verdemarcio"
	\item
	      "Conchiglie, grasso di pecora e nessun nome. Qui voglio il Ciclope"
\end{itemize}

\pagebreak

\subsubsection{Essenza Creazione}\index{Essenza Creazione}

\textbf{Caratteristica}: Volontà\\
\textbf{Verbo}: Creare\\
\textbf{Nome}: Elementi, Energia\\


\label{essenza-creazione---volonta}

Creare è l'\textbf{atto di plasmare la magia per creare Elementi e manifestare dal nulla un elemento od oggetto}.

Non si possono creare Creature (naturali o magiche) secondo la regola che non si può creare Vita. Quando si usa l'Essenza Creare per richiamare un Elementale in realtà si deve usare l'Essenza della Convocazione.

\begin{itemize}
	\item Creare un elemento non è congiurare un Elementale. Crei un Elemento entro le dimensioni e pesi limite stabiliti.
\end{itemize}

\begin{itemize}
	\item Se si vuole creare un Unione di Elementi (cibo, ottone, lava..) la quantità e volumi prodotti sono inversamente proporzionali alla complessità dell'elemento. Più è complesso l'elemento creato meno ne puoi creare. In base alla complessità e precisione dell'oggetto da creare diminuire massa e volumi.
	\item E' possibile creare più oggetti contemporaneamente. Calcolata la somma dei volumi/masse si prende la difficoltà immediatamente superiore.
	\item Non è possibile creare qualcosa all'interno di creature vive.
	\item Non è possibile creare qualcosa dove non vedi.
	\item Un muro di ghiaccio (ad esempio) non potrà fare danno immediatamente ma solo dal round successivo. Si deve considerare che il muro si manifesti in accrescimento nel round che viene creato. Il danno è pari all'Essenza di Attacco di due Livelli di Potere inferiori.
	\item In caso di oggetti che cadono, il danno è quella dell'Essenza di Attacco di due Livelli di Potere inferiori. L'Area di Effetto è quella stabilita dai costi base.
    \item La Creazione e' permanente (nei limiti fisici dell'oggetto, es. si scioglie, disperde, si consuma...) se si crea un Elemento. La Difficolta' come Durata e' 8.
	\item Se si crea qualcosa di intangibile (es. Luce) e che non fa danno, la Durata ha costo dimezzato.
	\item Non e' possibile creare qualcosa di magico
	\item Se si crea materia solida attorno all'obiettibo ed entro distanza di 3 metri (o il doppio della sua portata se è maggiore) dallo stesso, viene concesso un Tiro Salvezza su Riflessi per uscire dalla creazione prima che questa sia completa.
	\item La massa creata si distribuisce in blocchi connessi tra loro.
	\item La massa creata obbedisce alle leggi delle fisica quando possibile. Un muro d'acqua cade il round successivo alla creazione.
	\item Un cubo base è un cubo di lato 1 metro
\end{itemize}

\bigskip


La Difficolta' della Massa/Volume cambia in funzione del materiale che si va a creare.

\bigskip

Consultare la \textbf{Tabella Modificatore Creazione Elementi} per verificare il moltiplicatore al Volume/Massa necessario in base al materiale che si vuole creare.

\bigskip

\textbf{Tabella Modificatore Creazione Elementi}

\medskip
\begin{tabular}{lll}
	\toprule
	\textbf{Moltiplicatore} & \textbf{Durezza}       & \textbf{Esempio}\\
	0.5                   & Estremamente facile    & Sabbia / Luce \\
	0.7                   & Facile                 & Vetro / Acqua\\
	1                     & Normale                & Legno / Terriccio\\
	1.1                   & Difficile              & Ceramica / Pietra / Terra dura\\
	1.3                   & Molto difficile        & Ferro / Mattone\\
	1.5                   & Estremamente difficile & Acciaio/ Mithral\\
	1.7                   & Quasi impossibile      & Acciaio Nanico, Argento\\
	2                     & Inumana                & Adamantio, Oro\\
	2.5                   & Ultraterrena           & Acciaio Nanico Runico, Platino\\
	5                     & Divina                 & Artefatti, Gemme\\
\end{tabular}


Esempi:
\begin{itemize}
	\item "Evoco il grande spirito di Lunzac sommo artigiano reale perché crei un comodino di perfetta fattura"
	\item "Chiedo alle possenti maree del mare del sud di riempire d'acqua il campo di battaglia"
	\item "Per tutti le braci infernali che questo fuoco risplenda nella notte"
\end{itemize}

\bigskip

Esempi pratici
\begin{itemize}
	\item Per creare una Luce equivalente ad una torcia è sufficiente la Difficoltà 11 ( raggio 3 metri), con Difficoltà 13 si crea l'equivalente di una lanterna (raggio 6 metri di luce) sempre come Difficolta' di Volume/Massa.
	\item Mentre l'Essenza Distruzione crea oscurità perché distrugge la luce, tramite l'Essenza Creazione non e' possibile creare oscurità o luce magica, solo luce naturale. E' pero' possibile creare una nebbia molto fitta che impedisca la vista.
	\item Con Difficoltà 20 puoi creare Cibo per 1 persona per 1 giorno (comprensivo di costi base)
	\item Creo un muro di ghiaccio fatto di 4 cubi base (2m lunghezza{*}1m larghezza{*}2m altezza. Livello potere 25) farà danno e gli cade sopra pari a livello di potere 19 in Essenza Attacco (5d6).
	\item Suggerisco di tenere a disposizione dei cubetti 2x2 Lego per costruire "visivamente", come cubi base, le proprio creazioni
\end{itemize}

\pagebreak

\subsubsection{Essenza Cura}\index{Essenza Cura}

\label{essenza-cura---volonta}


\textbf{Caratteristica}: Volontà\\
\textbf{Verbo}: Curare - Riparare\\
\textbf{Nome}: Creature, Elementi\\

L'Essenza della Cura è la \textbf{capacità di riempire il vuoto causato da una Distruzione} o \textbf{sanare le ferite di un Attacco} o riparare un oggetto. L'Essenza di \textbf{Cura} agisce su \textbf{Creature} o \textbf{Elementi}. Se usata su Creatura ripristina le energie vitali (punti ferita), se usata su Elementi, oggetti, puo' risaldare rotture di pezzi.


\begin{itemize}
	\item
	      La Durata di un Essenza di Cura è sempre e solo istantanea, tranne se specificato diversamente
	\item
	      Una Essenza di Cura usata su un non morto equivale a causargli un danno pari all'ammontare che l'Essenza di Attacco avrebbe causato. Un Tiro Salvezza su Tempra può dimezzare i danni. Un successo critico li dimezza ulteriormente. Un fallimento Critico li raddoppia.
	\item
	      L'Essenza di Cura ha un Area di Effetto sempre a target e mai a raggio. Se sono indicati più target questi devono essere entro un metro l'uno dall'altro.
	\item      
	     L'Essenza di Cura usata su un oggetto ne sistema i meccanismi rotti ma non puo' crearne pezzi mancanti 
\end{itemize}


Esempi:
\begin{itemize}
	\item
	"Grande Ljust protettrice di ciò che è vivo concedimi di lenire le sofferenze di questo bravo uomo."
	\item
	"Che la mano del guaritore curi le tue malattie"
	\item
	"Possa questo bacio purificarti"
	\item
	"Con l'aiuto degli antichi sacerdoti il tuo spirito sia ripristinato"
	\item
	"Possa la spada dei grandi guerrieri infonderti l'energia che queste empie creature ti hanno tolto"
	\item "Un lieve tocco e questa corda tornerà unita"
\end{itemize}

\bigskip


\begin{tabularx}{0.95\textwidth}{lX}
	\toprule
	\textbf{Livello di Potere} & \textbf{Concetto (solo Vita)}\\
	<=11   & Rimuovi la condizione abbagliato. \\
	       & Curi 1d6 PF \\
	13     & Rimuovi la condizione Frastornato \\
	       & Curi 2d6 pf \\
	       & Crei un link vitale tra te ed un obiettivo. \\
           & Puoi condividere i tuoi punti ferita con le creature collegate. \\
           & Durata 10 minuti. Costa 1 azione mantenere ed usare questa condivisione. \\
	16     & Rimuovi la condizione Affaticato / Scosso / Infermo / Nauseato \\
           & Curi 4d6 pf, puoi dividere la cura fino a 2 obiettivi \\
           & Crei un link vitale tra te e fino a 3 obiettivi. Puoi condividere i tuoi punti ferita con le creature collegate. Durata 1 ora. Costa 1 azione mantenere ed usare questa condivisione. \\
	19     & Rimuovi la condizione di Malato / Accecato / Assordato / Confuso / Esausto \\
           & Ristori 2 punti ad una Caratteristica\\
           & Curi 6d6 pf, puoi dividere la cura fino a 3 obiettivi\\
           & Crei un link vitale tra te e fino a 4 obiettivi\\
           & Puoi condividere i tuoi punti ferita con le creature collegate. Durata 1 ora. Costa 1 azione mantenere ed usare questa condivisione. \\
	22     & Rimuovi la condizione di Avvelenato\\
           & Ristori 3 punti divisi su più Caratteristiche \\
           & Curi 9d6 pf, puoi dividere la cura fino a 6 obiettivi\\
           & Crei un link vitale tra te e fino a 5 obiettivi. Puoi condividere i tuoi punti ferita con le creature collegate. Durata 1 ora. Costa 1 azione mantenere ed usare questa condivisione \\
	25     & Rinsaldi fratture\\
           & Recuperi tutti i punti caratteristica \\
           & Curi 12d6 pf, puoi dividere la cura fino a 9 obiettivi \\
           & Crei un link vitale tra te e fino a 7 target.Puoi condividere i tuoi punti ferita con le creature collegate. Durata 1 ora. Costa 1 azione mantenere ed usare questa condivisione.  \\
	28     & Curi fino a 60 pf e tutte le malattie. \\
           & Ristori un livello temporaneo perso \\
	31     & Rigeneri i tessuti ed arti\\
           & Ristori un livello permanente perso\\
           & Curi 16d6 pf, puoi dividere la cura fino a 10 obiettivi\\
	34     & Ringiovanisci il target di 3d6 anni\\
           & Curi 20d6 pf, puoi dividere la cura fino a 16 obiettivi\\
           & Curi completamente tutte le ferite e condizioni di un obiettivo\\
	37     & Curi l'obiettivi di tutte le condizioni, punti caratteristica, livelli e punti ferita, ringiovanisci il target di 3d6 anni \\
	40     & Sacrifichi la tua vita per portare in vita un’altra creatura.\\
\end{tabularx}
\bigskip

Esempio Pratico\\

\textbf{Mano calda di Ljust}

Distanza: mischia, costo 0

Area di Effetto: 1 target, costo 0

Durata: Istantenea, costo 0

E’ necessario una prova di CM pari a 13 per ottenere una Mano calda di Ljust che curi su un target
2d6 punti ferita. La difficoltà e’ 0+0+0+13 = 13\\

\textbf{Sfera curativa di Ljust}

Distanza: 10 metri, difficoltà +2

Area di Effetto: 5 creature da includere, +5

Ricordiamoci che la Cura non può essere usata come area di effetto sferica, il costo è per poter selezionare le persone da curare.

Durata: istantanea, difficoltà +0

Essenza Cura: 19, cura 9d6, puoi dividere la cura fino a 6 target

Fatte tutte le somme una Sfera curativa di Ljust da 9d6 ha difficoltà +2+5+0+19 = 26\\

\textbf{Benedizione della Fenice}

Distanza: 0, tocco

Area di Effetto: 0, se stessi

Durata: 1 giorno, contingenza, +6

Cura: 4d6 PF, +16

Tempo di lancio: 10 minuti -6 

La Benedizione della Fenice e' una contingenza che dura un giorno, se si viene feriti mortalmente (i pf scendono sotto la meta') viene lanciata in automatico una cura da 4d6 (riuscendo in una prova di magia a difficolta' 16, altrimenti curera' di meno)

\pagebreak

\subsubsection{Essenza Difesa}\index{Essenza Difesa}

\textbf{Caratteristica}: Magnetismo\\
\textbf{Verbo}: Difendo\\
\textbf{Nome}: Creature, Elementi\\

\label{essenza-difesa---magnetismo}

l'Essenza di Difesa \textbf{permette di creare delle barriere/scudi/armature che possono proteggere una creatura od oggetto dal danno o da un elemento definito}. L'Essenza di Difesa si applica su Creature o Elementi. Nella tabella viene indicato il massimo della protezione concessa o a round, la Durata della Difesa e' sempre da determinare.
\begin{itemize}
	\item
	      Al target viene concesso un Tiro Salvezza su Tempra per negare gli effetti.
	\item
	      Essenza di Difesa può essere usata come controincantesimo per l'Essenza di Attacco. E' necessario superare con una prova di competenza magica la prova di competenza magica dell'avversario. Si annulla solo l'effetto su un obiettivo (se stesso o altro). Si consuma un utilizzo di Essenze.
	\item
	      Se ci sono più Essenze di Difesa attive non si sommano i tipi di bonus equivalenti, si tiene solo quello che fornisce il bonus più alto.
\end{itemize}

\bigskip
\begin{tabularx}{0.95\textwidth}{lX}
	\toprule
	\textbf{Livello di Potere} & \textbf{Creature, Presenza}\\
	<=11     & +1 Difesa  \\
	         & +1 Tiri Salvezza \\
	13       & Crei una difesa che protegge per 4 Punti Ferita in tutto \\
	16       & Crei una difesa da un elemento per 3 Punti Ferita a round\\
             & +2 ad un Tiro Salvezza\\
             & +3 Difesa\\
	19       & Crei una difesa da un elemento per 5 Punti Ferita a round\\
             & +2 a tutti i Tiri Salvezza   \\
             & +4 Difesa, +2 ad un Tiro Salvezza \\
	22       & Crei una difesa per 6 Punti Ferita a round \\
             & Crei una difesa da un elemento specifico \\
             & anche magico per 60 Punti Ferita in tutto \\
             & Barriera verso gli insetti normali \\
	25       & Crei una difesa per 8 Punti Ferita a round \\
             & Immune ad un elemento non magico \\
             & +6 Difesa \\
	28       & Resistenza ad un Energia da Essenza Attacco o naturale \\
             & Barriera verso le piante normali\\
             & +6 ad un Tiro Salvezza\\
             & Immune a livello di potenza 11 di Attacco\\
	31       & +6 a tutti i Tiro Salvezza\\
             & Barriera verso gli animali normali\\
             & Immune a livello di potenza 16 di Attacco / Trasformazione\\
             & Crei una barriera intorno a te, che ti scherma\\
             & da tutti gli attacchi fisici non magici\\
	34       & Barriera verso gli animali magici\\
	         & Crei una barriera intorno a te che ti scherma da tutti gli attacchi fisici anche magici\\
             & Immune a livello di potenza 22 di Attacco / Trasformazione / Distruzione \\
	37       & Immune a livello di potenza 28 di Attacco / Trasformazione / Distruzione \\
	43       & Immune a livello di potenza 34 di Attacco / Trasformazione / Distruzione \\
\end{tabularx}

\bigskip

Esempi:
\begin{itemize}
	\item
	      "Chiamo la magia del grande cristallo perché mi protegga contro i miei avversari"
	\item
	      "Canto le invocazioni di morte. Possano le osse dei miei antenati proteggermi"
	\item
	      "Incido il mio petto con le sacre rune di Qizdo!"
\end{itemize}

\pagebreak

\subsubsection{Essenza Distruzione}\index{Essenza Distruzione}

\textbf{Caratteristica}: Volontà\\
\textbf{Verbo}: Distruggo\\
\textbf{Nome}: Elementi, Creature, Energia\\

\label{essenza-distruzione---volonta}

\textbf{Essenza Distruzione -- Elementi}

La Distruzione di Elementi è la \textbf{distruzione di Elementi}
\begin{itemize}
	\item
	      La distruzione di Elementi è diretta, senza Tiro Salvezza.
	\item
	      La distruzione si intente di singolo e specifico oggetto non di volumi di più oggetti, tranne se omogenei e contigui(es un tavolo di legno, la serratura di metallo..).
	\item
	      La distruzione di materia è sempre permanente come durata. La durata ha difficoltà 8 + eventuale contingenza.
\end{itemize}

\bigskip

La Difficolta' data da Massa/Volume cambia anche in funzione del materiale che si va a distruggere.

\bigskip

Consultare la \textbf{Tabella Modificatore Distruzione Elementi} per verificare il moltiplicatore al Volume/Massa necessario in base al materiale che si vuole distruggere.

\bigskip

\textbf{Tabella Modificatore Distruzione Elementi}

\medskip
\begin{tabular}{lll}
	\toprule
	\textbf{Moltiplicatore} & \textbf{Durezza}       & \textbf{Esempio}\\
	0.5                   & Estremamente facile    & Sabbia / Luce \\
0.7                   & Facile                 & Vetro / Acqua\\
1                     & Normale                & Legno / Terriccio\\
1.1                   & Difficile              & Ceramica / Pietra / Terra dura\\
1.3                   & Molto difficile        & Ferro / Mattone\\
1.5                   & Estremamente difficile & Acciaio/ Mithral\\
1.7                   & Quasi impossibile      & Acciaio Nanico, Argento\\
2                     & Inumana                & Adamantio, Oro\\
2.5                   & Ultraterrena           & Acciaio Nanico Runico, Platino, Gemme\\
5                     & Divina                 & Artefatti \\

\end{tabular}

\bigskip

Se quindi si vuole dimostrare la propria potenza distruggendo una serratura di metallo semplicemente toccandola la difficoltà è 11 (livello di potere) {*} 1.3 (ferro) = 14.3 = 14 di difficoltà più i fattori di base (8 di durata, distanza, contingenza...)

L'arrotondamento si fa per eccesso.

Esempi:
\begin{itemize}
	\item
	      "Per tutte le torce consumate. Buio!"
	\item
	      "Chiamo a me i terremoti passati. Il tuo castello crolli come la sabbia
	\item
	      "Getto a terra il sangue di un Ragnroll. Che una fossa ti colga!"
\end{itemize}


\textbf{Essenza Distruzione -- Creature}
\begin{itemize}
	\item
	      La distruzione di Creature causa la distruzione dell'equilibrio metabolico e mentale.
	\item
	      Una Essenza di Distruzione non può mai causare danno diretto, si deve usare l'Essenza Attacco.
	\item
	      Al target viene concesso un Tiro Salvezza su Tempra per negare gli effetti. La perdita di punti caratteristica e' temporanea e si recupera 1 punto al giorno.
	\item
	      Non è possibile distruggere "parti" di esseri viventi
	\item
	      Su molte creature magiche le condizioni indicate non hanno effetto. Tenere sempre conto della durata degli effetti.
\end{itemize}

\bigskip

\begin{tabularx}{0.95\textwidth}{lX}
	\toprule
	\textbf{Livello di Potere} & \textbf{Creature}\\
	<=11  & Attribuisci la condizione abbagliato.  \\
	13    & Attribuisci la condizione: Frastornato    \\
	16    & Attribuisci la condizione: Affaticato / Scosso / Infermo  \\
	19    & Attribuisci la condizione di: Malato / Accecato / Assordato / Esausto / Nauseato\\
	      & Diminuisci di 1 punto una caratteristica   \\
	22    & Attribuisci la condizione di Avvelenato    \\
	      & Diminuisci di 2 punti una caratteristica   \\
	25    & Distruggi l’armonia spirituale della creatura (-1 al colpire, al danno, ai Tiri Salvezza)\\
	      & 3 AGI oppure -3 POT  \\
	28    & Causi dolori lancianti -3 POT e AGI  \\
	      & Distruggi temporaneamente le esperienze, il target perde un livello di esperienza \\
	31    & Distruggi i tessuti ed arti, causi la distruzione del corpo (TS o morte). \\
	      & Distruggi permanentemente le esperienze, il target perde un livello di esperienza \\
	34    & Invecchi il target di 3d6 anni \\
\end{tabularx}

\bigskip

Esempi
\begin{itemize}
	\item
	      "Evoco lo stregone Adbul Aziz. Mi conceda di fare marcire le tue interiora!"
	\item
	      "Grande martello, Grande martello, affonda la tua testa nella sua"
	\item
	      "Osserva la spirale di Oman Gur Tha. Non sei mai stato così stanco"
	\item
	      "Oh Padrone Shayalia ti offro lo spirito del mio nemico"
\end{itemize}

\bigskip

\textbf{Essenza Distruzione - animazione dei morti}

Molti incantatori seguaci di Sixiser utilizzano i non morti per portare caos e distruzione nel creato

\begin{itemize}
	\item
	      L'Essenza di Distruzione (animazione) su Creature non può concedere più dadi vita di quanti posseduti dalla creatura originale. Le caratteristiche di Volontà e Intelletto e Magnetismo vengono ridotte ad un terzo, a meno di aumentare il livello di potere a quello superiore.
	\item
	      Il CR indicato si riferisce alla sommatoria dei CR totali animati di creature naturali
	\item
	      Un incantatore non può animare creature naturali con più di 3 CR rispetto al suo valore di CM
	\item
	      Se si vuole animare una creatura magica la difficoltà passa a quella successiva e si somma con il costo del mantenimento delle caratteristiche mentali
	\item
	      La difficoltà della Durata dell'animazione è 8. Ed è permanente finché la creatura animata non viene distrutta o dismessa dall'incantatore.
\end{itemize}

\bigskip

\begin{tabularx}{0.95\textwidth}{lX}
	\toprule
	\textbf{Livello di Potere} & \textbf{Creature}\\
	\textless=11               & Animi fino a 1/3 CR\\
	13                         & Animi fino a 1/2 CR\\
	16                         & Animi fino a 1 CR\\
	19                         & Animi fino a 3 CR\\
	22                         & Animi fino a 5 CR\\
	25                         & Animi fino a 7 CR\\
	28                         & Animi fino a 9 CR\\
	31                         & Animi fino a 11 CR, max CR 9\\
	34                         & Animi fino a 13 CR, max CR 9\\
	37                         & Animi fino a 15 CR, max CR 10\\
	43                         & Animi fino a 17 CR, max CR 10\\
\end{tabularx}

\bigskip

Esempi: d'uso:
Creare una zona di Oscurità equivale ad usare l'Essenza Distruzione sulla Luce. Considerate la grandezza dell'ambiente ed il fatto che l'effetto e' immediato, ovvero permanente per la luce che era presente ma nulla vieta ad altra luce di prendere il suo posto.
\pagebreak

\subsubsection{Essenza Illusione}\index{Essenza Illusione}

\label{essenza-illusione---magnetismo}

\textbf{Caratteristica}: Magnetismo\\
\textbf{Verbo}: Illudo\\
\textbf{Nome}: --\\

l'Essenza Illusione è la \textbf{capacità di creare immagini, suoni,
	odori, profumi di cose che non esistono}.
\begin{itemize}
	\item
	      Una Illusione non può mai essere usata per ferire direttamente un avversario.
	\item
	      Una illusione non offre mai resistenza fisica.
	\item
	      In caso di creazione di proprie immagine una Essenza che causi danno ad area le distruggerà tutte.
	\item
	      Un Tiro salvezza su Arbitrio permette di discernere l'illusione.
	\item
	      E' sempre e solo sensoriale (uditiva, visiva, olfattiva), e non influenza mai il tocco.
	\item
	      In base alla complessità e precisione dell'oggetto diminuire quantità e volumi.
\end{itemize}

\bigskip

\begin{tabularx}{0.95\textwidth}{lX}
	\toprule
	\textbf{Livello di Potere} & \textbf{Dimensione}\\
	<=11    & Un cubo di lato fino a 20 cm                                          \\
	13      & Un cubo di lato fino a 50 cm    \\
	        & Crei 1d4 immagini illusorie di te.      \\
	        & Ogni attacco a segno elimina prima una immagine.    \\
	16      & Un cubo di lato fino a 1 metro    \\
	        & Crei una illusione che ti rende sfocato, +3 CA    \\
	        & Crei 2d4 immagini illusorie di te.   \\
	        & Ogni attacco a segno elimina prima una immagine.       \\
	        & Crei un Allarme sonoro che si attiva al passaggio    \\
	19      & Un cubo di lato fino a 2 metri    \\
	        & Crei una illusione che ti rende invisibile. Attaccare rende visibile. \\
	        & Crei un Allarme sonoro che si attiva quando visto   \\
	22      & Un cubo di lato fino a 3 metri    \\
	        & Crei una illusione di te a 2 metri, +4 Difesa                         \\
	        & Crei un Allarme sonoro che si attiva al passaggio o suono o se visto. \\
	        & Crei un Allarme sonoro e visivo che si attiva al passaggio o se visto \\
	25      & Un cubo di lato fino a 4 metri    \\
            & Crei una illusione che rende invisibile te e altre 3 creature di taglia media. Attaccare rende visibile.         \\
	        & Crei un Allarme sonoro e visivo che si attiva se ci sono creature anche invisibili\\
	28      & Un cubo di lato fino a 6 metri\\
	31      & Un cubo di lato fino a 10 metri\\
	34      & Un cubo di lato fino a 15 metri\\
	37      & Un cubo di lato fino a 20 metri\\
	43      & Un cubo di lato fino a 50 metri\\
\end{tabularx}

\bigskip

Esempi:
\begin{itemize}
	\item
	      "Dal Deserto di Darnhub chiamo il miraggio di una bellissima donna"
	\item
	      "Lasciami proteggere la porta. Creo questa campanellina perché suoni ad ogni passaggio"
	\item
	      "Incido la runa di Abildan. Chiunque passi da qui farà suonare le trombe di Torricelli"
\end{itemize}

\pagebreak

\subsubsection{Essenza Movimento}\index{Essenza Movimento}

\label{essenza-movimento---agilita}

\textbf{Caratteristica}: Agilità\\
\textbf{Verbo}: Muovo\\
\textbf{Nome}: Creature, Elementi\\


L'Essenza di Movimento significa \textbf{potersi teletrasportare e spostarsi di piani nonché alterare la velocità e spostare le cose}.

\textbf{Essenza Movimento (Spostare) -- Creature}

\begin{itemize}
	\item
	      La Creatura Naturale o Magica che viene influenzata dell'Essenza di Movimento deve avere taglia media o inferiore.
	\item
	      Per ogni creatura oltre la prima, per taglia superiore alla media, la difficoltà aumenta di 2. In caso di taglia superiore alla media e creatura oltre la prima i costi si sommano. Quindi spostare 3 creature grandi a distanza di kilometri ha difficoltà 19+2*2 (2 creature oltre la prima) +2*3 (3 creature di taglia oltre la media) = 29.
	\item
	      Al target viene concesso un TS su Potenza per resistere agli effetti.
	\item
	      La durata è sempre istantanea (e' un teletrasporto).
\end{itemize}

\bigskip

\begin{tabularx}{0.95\textwidth}{lX}
	\toprule
	\textbf{Livello di Potere} & \textbf{Creature}\\
	<=11           & Ci si puo'’ spostare entro raggio di 3 metri       \\
	               & Caduta piuma                                         \\
	13             & Ci si puo'’ spostare entro raggio di 12 metri        \\
	               & Levitazione                                          \\
	16             & Ci si puo'’ spostare entro raggio di 100 metri       \\
	               & Si diventa eterei                                    \\
	               & Volare                                               \\
	19             & Ci si puo'’ spostare entro 5 km                      \\
	               & Si puo'’ passare da un piano all’altro               \\
	22             & Ci si può spostare entro 20 km\\
	25             & Ci si può spostare entro 100 km\\
	28             & Si può coprire una distanza di 200 km\\
\end{tabularx}

\bigskip

Esempi:
\begin{itemize}
	\item
	      "Lucido la punta degli stivali e sbatto i tacchi. E sono dove voglio"
	\item
	      "Per i grandi rapaci possa io arrivare in cima alla montagna"
	\item
	      "Banchetti e monete, specchietti e cadute. Portatemi a palazzo Dornean"
\end{itemize}

\bigskip

\textbf{Essenza Movimento - Creatura}

Alterare il Movimento è alterare la velocità di azione di una creatura.
Questo concede di muoversi più velocemente o attaccare più volte o riuscire ad agire più volte nel round.

Al target viene concesso un Tiro Salvezza su Arbitrio per negare gli effetti.

\bigskip

\begin{tabularx}{0.95\textwidth}{lX}
	\toprule
	\textbf{Livello di Potere} & \textbf{Concetto}\\
	\textless=11         & Il target ottiene una Azione di movimento bonus\\
	13       & Il target ottiene due Azioni di movimento bonus\\
	16       & Il target ottiene un bonus di una Azione\\
	19       & Il target ottiene un bonus di una Azione ed una Azione di movimento\\
	22       & Il target ottiene due Azioni in piu'\\
	25       & Il target ottiene tre Azioni in piu'\\
	28       & Il target può usare due abilità a disposizione (ma non può lanciare 2 Essenze)\\
\end{tabularx}

\bigskip

Esempi:
\begin{itemize}
	\item
	      "Possa lo spirito dei grandi corridori guidarti i passi"
	\item
	      "Evoco a me i riti del mago Ratl Go Nw. Ora il tuo corpo è veloce come l'argento"
	\item
	      "Un passo a destra, uno a sinistra, incrocia le braccia. La velocità del Lupo è nei tuoi piedi"
\end{itemize}
Esempio pratico:
\begin{itemize}
	\item
	      con Essenza Movimento a difficoltà 28 posso usare Incanalare energia due volte oppure Imposizione delle mani ed Incanalare energia (non posso usare due Essenze in un round)
\end{itemize}

\bigskip

\textbf{Essenza Movimento (Blocco/Spostamento) - Elementi/Creature}

Tramite l'Essenza Movimento è possibile inibire lo spostamento di creature.

Al target viene concesso un Tiro Salvezza su Tempra per negare gli effetti.
\begin{itemize}
	\item
	      Il valore massimo è riferito al numero massimo di CR influenzati.
	\item
	      il target non può essere di 3 CR superiore al valore di CM dell'incantatore
	\item
	      Il numero massimo di creature influenzabili è pari alla metà dei dadi vita influenzati, con un minimo di 1
\end{itemize}

\bigskip

\begin{tabularx}{0.95\textwidth}{lX}
	\toprule
	\textbf{Livello di Potere} & \textbf{Creature}\\
	\textless=11   & Il target inciampa, -1 azione\\
	13       & Il target è rallentato, viene dimezzato la velocità di movimento\\
	16       & Il target è fermo nella sua posizione (immobilizzato), massimo 2	CR\\
	19       & I target rimangono fermi nella loro posizione, totale 5 CR non piu'di 3 CR a target\\
	22       & Il target è fermo nella sua posizione, massimo 7 CR\\
	25       & I target rimangono fermi nella loro posizione, totale 12. Non piu'di 4 CR a target\\
	28      & Il target è fermo nella sua posizione, massimo 9 CR\\
	31      & I target rimangono fermi nella loro posizione, totale 16 CR. Non più di 5 CR a target\\
	34       & Il target è fermo nella sua posizione, massimo 12 CR non puo' teletrasportarsi o cambiare di piano.\\
	37       & I target rimangono fermi nella loro posizione, totale 24 CR .Non più di 7 CR a target\\
	43      & Il target rimane fermo nella sua posizione, massimo 18 CR\\
\end{tabularx}

\bigskip

Usare la \textbf{Difficoltà Massa/Volume per le Creature/Elementi} da muovere e non Obiettivo/Area di Effetto, basandosi sul Peso totale delle Creature/Elementi.

La velocita' con cui si muove la Creatura o Elemento e' di 6 m/r. Per ogni successo critico la velocita' aumenta di 3 m/r.
Se si sposta una Creatura questa ha diritto ad un TS su Tempra ogni round di durata.
\bigskip\

\textbf{Chiarimenti}
\begin{itemize}
	\item
	      Tramite l'Essenza Movimento è possibile spostare un oggetto, sollevandolo e muovendolo.
	\item
	      Se si muovono più target considerare la somma di volumi e pesi per determinare il costo.
	\item
	      Il movimento ottenuto è di 10 metri a round. Se serve più velocità aumentare la difficoltà (+5 = raddoppio velocita')
	\item
	      Al target se vivente è concesso un Tiro Salvezza su Tempra per resistere e non farsi spostare
	\item
	      Il target viene spostato, nel corso dei round, fino a Distanza calcolata nelle difficoltà.
	\item
	      E' possibile usare l'Essenza di Movimento per scagliare proiettili. Usato per scagliare proiettili (sassi, dardi, frecce.. oggetti fino a taglia media) il danno causato è pari a quello di due livelli di potere inferiore causato dall'Essenza di Attacco.
	\item
	      E' possibile usare la Essenza di Movimento per fare cadere una massa su singolo target o più target a seconda della dimensione della massa, vedi indicazione Area. Il danno causato è di 1d6 per cubo base che colpisce l'avversario. Quindi una colonna alta 8 cubi e larga 1 fa 8d6 di danni. Massimo 20d6, Tiro Salvezza su Riflessi per dimezzare. In caso di fallimento critico i danni si raddoppiano, in caso di successo critico i danni si dimezzano ulteriormente.
	\item
	      Le aree influenzate sono indicative, ricordarsi di distribuire i cubi secondo numero e forma volute.
	      
	\item E' possibile usare l'Essenza di Movimento per tenere bloccata una porta (vuoi spingendo la porta o spostando il chiavistello...)
	
\end{itemize}
\bigskip


Esempio:
\begin{itemize}
	\item 	"Chiamo a me le ossa dei ladri. Fermate la creatura"
	\item 	"Potenze dell'aria, rallentate la fuga della creatura"
	\textbf{Essenza Movimento -- Elementi, Creature}
	\item "Chiamo i venti possenti che distrussero Orton Gal No. Sollevate questo carro!"
	\item "Per il passo leggero degli Orunkes, io cammino nell'aria"
	\item "Spiriti delle tempeste, scagliate la vostra rabbia contro chi osa sfidarci!"
\end{itemize}

\pagebreak

\subsubsection{Essenza Protezione}\index{Essenza Protezione}

\label{essenza-protezione---potenza}

\textbf{Caratteristica}: Potenza\\
\textbf{Verbo}: Proteggo\\
\textbf{Nome}: Creature\\

l'Essenza di Protezione si applica su Creature \textbf{permette di schermare o annullare gli effetti magici e non che altererebbero il nostro corpo}.

\begin{itemize}
	\item
	      Al target viene concesso un Tiro Salvezza su Arbitrio per negare gli effetti
	\item
	      è possibile usare l'Essenza di Protezione come controincantesimo verso l'Essenza di Trasformazione o Alterazione o Charme o Movimento o Rivelazione. E' necessario superare con il proprio check di magia il valore di difficoltà della prova che ha generato l'Essenza che si vuole controbattere.
\end{itemize}

\bigskip

\begin{tabularx}{0.95\textwidth}{lX}
	\toprule
	\textbf{Livello di Potere} & \textbf{Creature, Presenza}\\
	\textless=11& Proteggi dalla condizione di Abbagliato.\\
	13& Proteggi dalla condizione di Frastornato / Scosso\\
	16& Proteggi dalla condizione di Affaticato / Infermo / Spaventato\\
	19& Proteggi dalla condizione di Malato / Esausto / Nauseato / Confuso\\
	22& Proteggi dalla condizione di Avvelenato / Charmato / In preda al panico\\
	25& Proteggi dalla condizione di Posseduto / Accecato / Assordato \\
	  & Proteggi dalla distruzione fino a 2 livello di esperienza   \\
	28& Proteggi dalla condizione di Dominato / Maledetto   \\
	  & Proteggi dalla distruzione fino a 4 livelli di esperienza   \\
	31& Proteggi dalla distruzione di esperienza fino a durata   \\
	  & Proteggi da tutti i condizionamenti mentali fino a durata\\
	34& Proteggi dall'Essenza di Charm e Rivelazione fino al livello 22\\
	37& Proteggi dall'Essenza di Charm e Rivelazione fino al livello 25\\
	43& Proteggi dall'Essenza di Charm e Rivelazione fino al livello 28\\
\end{tabularx}

\bigskip


Esempi:
\begin{itemize}
	\item "
	      "Possa la saggezza dei miei antenati proteggermi
	\item
	      "Spirito e Bontà. Possa la Massima Ljust proteggermi dalle Ombre" (protezione esperienza)
	\item
	      "Traccio il circolo dei Sacerdoti Gurla. Nessuno potrà osservarmi"
	\item
	      "Sbatto il piede e pronuncio la parola di potere Shrak. Non mi trasformerai in un rospo!" (controincantesimo )
\end{itemize}

\pagebreak

\subsubsection{Essenza Rivelazione}\index{Essenza Rivelazione}

\label{essenza-rivelazione---magnetismo}

\textbf{Caratteristica}: Magnetismo\\
\textbf{Verbo}: Rivelo\\
\textbf{Nome}: Creature, Elementi, Virtu'\\

La Rivelazione si applica a Creature (Naturali o Magiche), Elementi, Concetti e Virtù. \textbf{Permette di capire i tratti fisici principali e lo stato di "salute". Permette di espandere la propria coscienza per accedere alla comprensione di eventi od oggetti del passato}.
\begin{itemize}
	\item
	      Alla Creatura oggetto della Rivelazione viene concesso un Tiro Salvezza
	      su Arbitrio per annullare gli effetti.
	\item
	      Non è possibile usare l'Essenza Rivelazione per determinare un evento futuro.
	\item
	      L'Essenza Rivelazione serve per divinare o percepire in maniera più approfondita. Viene lasciato al giocatore ed al Narratore l'utilizzo creativo di questa Essenza. Quelli qui sotto proposti sono esempi di utilizzo, linee guida.
\end{itemize}

\bigskip

\begin{tabularx}{0.95\textwidth}{lX}
	\toprule
	\textbf{Livello di Potere} & \textbf{Creature, Elementi, Virtu'}\\
	<=11   & Comprendi se un oggetto e’ magico  \\
	  & Sei in grado di leggere il magico  \\
	13& Comprendi eta’, peso e dimensioni. \\
	  & Comprendi se e’ presente una protezione \\
	  & Sei in grado di leggere una pergamena magica senza difficolta’   \\
	16& Comprendi lo stato del corpo se influenzato da qualche Essenza   \\
	  & Comprendi quale tipo di protezione e’ presente    \\
	  & Comprendi i tratti del soggetto.   \\
	  & Conferisci ai tuoi occhi la visione della magia   \\
	  & Conferisci ai tuoi occhi la visione crepuscolare 18m   \\
	19& Comprendi lo stato originario del target.    \\
	  & Comprendi la natura e specifiche di un oggetto magico  \\
	  & Conferisci ai tuoi occhi di vedere nell’oscurita’ \\
	22& La tua Consapevolezza e’ oltre le illusioni di pari potenza.\\
	  & Conferisci ai tuoi occhi di vedere nell’oscurita’ magica    \\
	25& Puoi scrutare luoghi fino ad un 1 km di distanza  \\
	28& Puoi scrutare persone fino ad 5 km di distanza.   \\
	  & Toccando un oggetto ti permette di conoscere la sua storia. \\
	  & La tua Consapevolezza di permette di vedere il vero delle cose e persone   \\
	31& Puoi scrutare luoghi e persone fino a 10 km di distanza.    \\
	  & Concentrandoti puoi conoscere la storia di un logo/oggetto fino a 5 km di distanza   \\
	34& La tua comprensione ti permette di conoscere nei dettagli la storia e
	leggende \\
	  & di qualsiasi manufatto purche’ tu ne abbia una vaga idea di come e’ fatto  \\
	37& La tua comprensione e’ leggendaria. Puoi conoscere fatti ed accadimenti  di qualsiasi era \\
	43& La tua comprensione e’ tale come se tu fossi stato presente e partecipe    \\
\end{tabularx}

\bigskip

Esempi:
\begin{itemize}
	\item
	      "Grande Atmos, invoco la tua benedizione. Aiutami a comprendere questa bacchetta"
	\item
	      "Per i tomi della biblioteca del tempo. Chi costruì il castello di Hul Barton?"
	\item
	      "Benedetto servitore di Atmos concedimi di scrutare nella biblioteca segreta. Raccontami la storia di Rozanda Durand"
\end{itemize}
\pagebreak


\subsubsection{Essenza Trasformazione}\index{Essenza Trasformazione}

\textbf{Caratteristica}: Potenza
\textbf{Verbo}: Trasformo
\textbf{Nome}: Creature, Elementi, Virtu'

\label{essenza-trasformazione---potenza}

l'Essenza della Trasformazione \textbf{altera la forma e sostanza di Creature, Elementi, Energia}.

\bigskip

\textbf{Essenza Trasformazione -- Elementi, Energia}

L'Essenza di Trasformazione prende la materia già pronta e concede all'incantatore di Trasformare le forme e sostanze come più gli aggrada. La forma trasformata obbedisce alle leggi della fisica es. l'acqua non può volare se non è retta da qualcosa.

\begin{itemize}
	\item
	      L'Essenza di Trasformazione può essere applicata ad una Essenza di Attacco. La prova di competenza magica deve essere superiore alla prova effettuata dall'avversario (getti aria possono farti volare via o danneggiare lo stesso se non si è protetti). E' una Azione di Reazione
	\item
	      E' anche possibile trasformare l'acqua in fuoco, ma per il principio che una Essenza deve fare una sola cosa, il danno da fuoco potrà esserci solo dal round successivo, se ancora esiste
	\item
	      Se si vuole trasformare la materia in un insieme di materia (fango, lava, pane.. si devono considerare le difficolà di ogni singolo elemento che lo compone, acqua e terra, terra e fuoco)
	\item
	      Se un soggetto è influenzato da una Essenza di Trasformazione viene concesso un Tiro Salvezza su Tempra per annullarne gli effetti.
	\item	
		  La Trasformazione se applicata per ottenere un Elemento/Energia Magico ha una Difficolta' aggiuntiva di +6.
	\item	
		  La trasformazione di Elementi o Energia e' permanente ed ha Difficolta' come Durata 8      
		  
\end{itemize}



Esempi:
\begin{itemize}
	\item
	      "Acqua e Terra, Fuoco e Acqua. Le mie mani non hanno fine"
	\item
	      "Per i riti degli antichi alchimisti ciò che era adesso non è più lui"
	      Esempio pratico:

	      Trasformando in acqua un cubo di 3{*}3{*}3 (27 cubi base) di terra si mette in seria difficoltà l'avversario (Tiro salvezza su riflessi concesso per evitare di cadere)
	\item
		  Si puo' trasformare un pezzo di un lucchetto/serratura per bloccarlo od aprirlo. La prova va confrontata con la resistenza del Elemento che compone il lucchetto.      
\end{itemize}

\bigskip

\textbf{Essenza Trasformazione -- Creature}

\begin{itemize}
	\item
	      L'Essenza di Trasformazione può trasformare Creature Naturali, Creature Magiche ed Elementi tra loro.
	\item
	      L'Essenza di Trasformazione se effettuata su una creatura senziente costa il livello superiore di potere. Per trasformare in pietra una creatura normale di 4 CR (in caso di PG il Livello è il CR) la prova ha difficoltà 28
	\item
	      Non si può influenzare o trasformare in una creatura con più di 3 CR superiore alla CM dell'incantatore
   	\item 
   		  Si puo' trasformare solo una Creatura alla volta. Il costo per l'Obiettivo e' +1
	\item
	      Se si vuole trasformare in una creatura magica la Difficolta' passa la Livello di Potere superiore (e si somma con il costo della senziente)
	\item
	      I CR indicati si riferiscono alla somma dei CR/Livelli influenzati
	\item
	      Al target viene concesso un Tiro Salvezza su Arbitrio per negare gli effetti
	\item
	      Il target trasformato mantiene le caratteristiche mentali (Intelletto, Volontà) precedenti ma prende quelle fisiche della creatura (Potenza, Agilità, Magnetismo)

    \item
          La trasformazione di Creature e' permanente ed ha difficolta' come Durata 8.      
	      
\end{itemize}

\bigskip

\textbf{Tabella Trasformazione Creature}
\medskip

\begin{tabularx}{0.95\textwidth}{lX}
	\toprule
	\textbf{Livello di Potere} & \textbf{Creature}\\
	<=11	& 1/3 CR    \\
	13		& 1/2 CR   \\
	16		& 1 CR \\
	19		& 2 CR \\
	22		& 3 CR \\
	25		& 5 CR \\
	28		& 7 CR \\
	31		& 9 CR \\
	34		& 11 CR\\
	37		& Fino a 13 CR, max singolo CR 7\\
	43		& Fino a 15 CR, max singolo CR 9\\
\end{tabularx}

\bigskip


Esempi:
\begin{itemize}
	\item
	 "Chiedo l'aiuto di tutte le streghe. Trasformate il mio nemico in un rospo!"
	\item
	 "Bava di Lumaca e sterco di vacca. Rumina nel prato"
\end{itemize}

\pagebreak

\subsubsection{Esempi di formulazioni Essenze}\index{Esempi Essenze}

Questi sono alcuni esempi di Essenze formulate.

Vengono presentate cosi' che sia piu' facile compredere il sistema magico e fornire una base di partenza per le vostre creazioni


Questo il template di base


\flushleft \textbf{Nome Magia}: \\ \index{Esempio Magia}
\textbf{Verbo}: \\
\textbf{Nome}: \\
\textbf{Tempo di Lancio} (): \\
\textbf{Distanza} (): \\
\textbf{Area di Effetto}/\textbf{Massa/Volume} (): \\
\textbf{Durata} (): \\
\textbf{Difficolta' base}: \\
\textbf{Descrizione}: \\


\flushleft \textbf{Nome Magia}: Acqua benedetta\\  \index{Acqua benedetta}
\textbf{Verbo}: Trasformazione\\
\textbf{Nome}: Elementi\\
\textbf{Tempo di Lancio} (0): 2 Azioni\\
\textbf{Distanza} (0): tocco\\
\textbf{Massa/Volume} (2+6): tre boccette di acqua, per un totale di circa 500ml. Il +6 e' dato dal fatto che si vuole trasformare l'acqua in un elemento magico\\
\textbf{Durata}: permanente (8)\\
\textbf{Difficolta' base}: 16\\
\textbf{Descrizione}: Con questa formulazione trasformi fino a mezzo litro d'acqua in acqua benedetta o maledetta (a seconda del Patrono). Riuscire nella prova di Competenza Magica con un punteggio oltre 16 non cambia il risultato finale.\\


\flushleft \textbf{Nome Magia}: Trasforma Pietra in Carne \\ \index{Esempio Magia}
\textbf{Verbo}: Trasformare\\
\textbf{Nome}: Creature\\
\textbf{Tempo di Lancio} (0): 2 Azioni\\
\textbf{Area di Effetto} (1): una creatura\\
\textbf{Distanza} (3): entro 50m\\
\textbf{Durata} (8): permamente\\
\textbf{Difficolta' base}: 12 \\
\textbf{Descrizione}: In base al risultato ottenuto verificare se si e' raggiunto un Livello di Potere sufficiente a trasformare l'obiettivo in Pietra.
Ricordarsi che in caso di creatura magica si passa al LP successivo.\\


\flushleft \textbf{Nome Magia}: \\ \index{Esempio Magia}
\textbf{Verbo}: \\
\textbf{Nome}: \\
\textbf{Tempo di Lancio} (): \\
\textbf{Distanza} (): \\
\textbf{Area di Effetto} (): \\
\textbf{Massa/Volume} (): \\
\textbf{Durata} (): \\
\textbf{Difficolta' base}: \\
\textbf{Descrizione}: \\


\pagebreak

\subsection{La Magia (Semplificata - Opzionale)}\index{La Magia Semplificata}

\textit{Si, c'e' un terzo sistema..}\\

Questo sistema di Magia è più "classico" e riprende gli standard della 3ed e Pathfinder.

Per ogni punto in Competenza Magica l'usufruitore di magia sceglie due Incantesimi. Il livello massimo sceglibile e lanciabile è indicato nella Tabella Punti Magia Posseduti.

Ogni volta che si prende un punto in CM è possibile dimenticare un incantesimo e sostituirlo con un altro.

Un usufruitore di magia ha a disposizione un numero di Punti Magia dato dal suo punteggio di CM, vedi tabella Punti Magia Posseduti.

Un incantesimo costa un numero di Punti Magia pari al doppio del suo livello.

Usate come lista di incantesimi quella di Pathfinder compresa nel Core Book, oppure online su http://aonprd.com/SpellsCustom.aspx\ e selezionare PFS Legal

La DC per resistere all'incantesimo è 10+CM+Intelletto.

Ogni volta che l'incantesimo fa riferimento al livello dell'usufruitore di magia considerate invece il punteggio di CM, gli altri fattori dell'incantesimo rimangono inalterati.

\bigskip

\textbf{Tabella Punti Magia posseduti}

\bigskip

\begin{tabularx}{0.95\textwidth}{XXX}
	\toprule
	\textbf{Valore Competenza Magia} & \textbf{Punti Magia Posseduti} & \textbf{Max livello incantesimo lanciabile e sceglibile} \\
	1  & 3& 1\\
	2  & 5& 1\\
	3  & 7& 2\\
	4  & 11    & 2\\
	5  & 13    & 3\\
	6  & 17    & 3\\
	7  & 19    & 3\\
	8  & 23    & 4\\
	9  & 29    & 4\\
	10 & 31    & 5\\
	11 & 37    & 5\\
	12 & 41    & 6\\
	13 & 43    & 6\\
	14 & 47    & 7\\
	15 & 53    & 7\\
	16 & 59    & 7\\
	17 & 61    & 8\\
	18 & 67    & 8\\
	19 & 71    & 9\\
	20 & 73    & 9\\
\end{tabularx}

\pagebreak

\section{Vantaggi}\index{Vantaggi}

\label{vantaggi}
\begin{tcolorbox}[enhanced,arc=5pt,boxrule=0.3pt]{Adoro fare il supereroe! L'orario di lavoro è pessimo, la paga è inesistente... ma almeno non corro il rischio di venire licenziato! (PK)
	}\end{tcolorbox}\medskip

\bigskip

Ogni personaggio può avere, e non è obbligatorio averne, dei vantaggi. Questi devono essere interessanti, piacevoli, divertenti e soprattutto giocabili.

Ogni vantaggio ha un costo, da pagare ad ogni livello. Come detto non deve essere obbligatorio prendere un vantaggio, né tanto meno si devono prendere vantaggi solo perché fanno essere forti. Lo scopo di un vantaggio è stupire e divertirsi.

Avere un vantaggio significa essere diverso, essere un freak, avere quel particolare che ti rende diverso ed unico, ma non per questo sempre il più forte, potente o invincibile. Un vantaggio non è solo una capacità, è un'occasione di gioco di ruolo. Il giocatore è invitato ad essere creativo nella scelta dei vantaggi ed anche nella creazione di nuovi, il costo poi si decide con il Narratore.

Diversi vantaggi non hanno un effetto pratico concreto ed immediato ma sono di arricchimento al background, alla storia del personaggio, introducono occasioni di gioco e divertimento. Quando si scelgono i vantaggi, e di conseguenza gli svantaggi, non è come andare a fare scorta di poteri super e straordinarie abilità, ma di peculiarità, manie, specialità che il personaggio possiede e che ancora una volta lo rendono diverso, unico, solo tuo.

Pertanto vantaggi e svantaggi vanno anche e soprattutto giocati ed interpretati.

I Vantaggi con {*} e tutte quelle con costo 20 o superiore sono a discrezione del Narratore.

I vantaggi si scelgono al primo livello, ogni vantaggio preso a livelli successivi va concordato con il Narratore.

I punti di costo di un Vantaggio si pagano con i punti guadagnati dagli Svantaggi.

I bonus dati alle competenze si intendono specifiche sulla prova quando indicato tra parentesi.

\bigskip

\textbf{Ali della provvidenza} \index{Ali della provvidenza}20 : hai delle ali, a te la scelta di forma e colore, solitamente stanno sulle scapole e ti fanno volare (volare buono)

\textbf{Ambidestro}\label{Ambidestro}\index{Ambidestro} 10: puoi usare indifferentemente le mani. I malus alle prove dove si usano due mani diminuiscono di 2

\textbf{Amico degli animali}\index{Amico degli animali} 5: +2 alle prove per gestire gli animali (anche selvaggi)

\textbf{Anfibio}\index{Anfibio} 20: puoi respirare sia sott'acqua che l'aria

\textbf{Arcobaleno}\index{Arcobaleno} 10: sei un artista. Le tue dita spontaneamente
producono colore

\textbf{Aura di coraggio}\index{Aura di coraggio} 15: intorno a te, in distanza entro 3 metri infondi coraggio. +2 TS vs Essenza Charme

\textbf{Artigli}\index{Artigli} 5: ogni tanto ricordati di spuntare gli unghiotti. 1d4 di danno per attacco. Gli attacchi con la seconda mano non hanno le penalità delle due armi.

\textbf{Bere fa bene}\index{Bere fa bene} 5: Prerequisito: Il fegato non conta. Il tuo corpo metabolizza l'alcool in maniera molto efficace. Un litro di birra ti fa recuperare 1d4 PF, un bottiglia di liquore 1d8 PF. Se di pessima qualità no..

\textbf{Caduta gatto}\index{Caduta gatto} 5: +2 alle prove di Agilità sulle cadute

\textbf{Camaleonte}\index{Camaleonte} 10-20: la tua pelle può cambiare colore. Tempo necessario 1 minuto/1round

\textbf{Cambiaforma}\index{Cambiaforma} 40: come Essenza Alterazione, Livello Potere 18

\textbf{Camminare sull'aria} \index{Camminare sull'aria}30: non troppo controllato. Qualsiasi cosa che non sia camminare richiede una prova di Agilità

\textbf{Camminare sulle acque} \index{Camminare sulle acque} 30: ma non darti delle arie..

\textbf{Magnetico} \index{Magnetico}5-10: sprigioni luce quando vuoi. per fortuna non letteralmente. +2 alle prove al Magnetismo

\textbf{Consumi ridotti} \index{Consumi ridotti}5: bevi e mangi la metà di un uomo normale. Sei sotto peso.

\textbf{Controllo del metabolismo} \index{Controllo del metabolismo} 10: solo il nome è fantastico! Annulli il danno da Sanguinamento.

Recuperi i punti ferita come se avessi il doppio del punteggio di Potenza.

\textbf{Cure efficaci} \index{Cure efficaci}10: +1d6 ogni volta che una Essenza
ti cura

\textbf{Daredevil} \index{Daredevil}10: ti piace buttarti nelle mischia, specialmente se si corrono pericoli. +2 Tiri per Colpire / Difesa finché sei circondato da tre o più avversari

\textbf{Denti} \index{Denti}5: il tuo morso fa male, 1d4, lavati i denti ogni tanto..

\textbf{Digestione universale} \index{Digestione universale}5: purché non faccia male si mangia, +2 TS su Tempra vs Veleni

\textbf{Direzione Assoluta} \index{Direzione Assoluta}5: sai sempre dove è il nord magnetico. Hai un +4 alle prove di orientamento (Sopravvivenza)

\textbf{Duro da soggiogare} \index{Duro da soggiogare}5: +2 TS su Arbitrio su Essenze Charme

\textbf{Duro da uccidere} \index{Duro da uccidere}5: non svieni a 0 PF, ma a -LV/2 in PF. Muori a 15+Potenza x 3 PF

\textbf{Empatia con le piante} \index{Empatia con le piante}10: io comprendo la sofferenza dell'erba pestata

\textbf{Empatia} 5: +2 alle prove di percepire inganni (Consapevolezza)

\textbf{Empatia Animale} \index{Empatia Animale}10: +4 alle prove per gestire gli animali (anche selvaggi)

\textbf{Empatia spirituale} \index{Empatia spirituale}5: non parli con gli spiriti, ma ne senti le emozioni

\textbf{Ermafrodito} \index{Ermafrodito}10: lgbtE!

\textbf{Forgiato nell'acciaio} \index{Forgiato nell'acciaio}5: Tramite dolorose operazioni la tua pelle è stata rivestita con placche di metallo. +3 alla Difesa

\textbf{Forma d'ombra} \index{Forma d'ombra}30: considera il poterti trasformare in un ombra 1 ora per livello. Non puoi andare in spazi assolati e senza ombra.

\textbf{Fortunato} \index{Fortunato}5-10: 3 volte al giorno puoi ritirare un 1 sul dado a 6, da dichiarare prima che il Narratore ti abbia detto se la prova è riuscita o meno.

\textbf{Guarigione accelerata}\index{Guarigione accelerata}: 5 ogni mattina recuperi il doppio dei Punti Ferita che normalmente recupereresti. Si cumula con Controllo Metabolismo. \index{Controllo Metabolismo.}

\textbf{Guaritore}\index{Guaritore} 5: sai dove mettere le mani. +4 alle prove di Sopravvivenza (pronto soccorso)

\textbf{Il fegato non si conta} \index{Il fegato non si conta}10: puoi bere tanto e non ti ubriachi

\textbf{Illuminato} \index{Illuminato}10-20: fai luce.. letteralmente. Emetti luce in un raggio di 3/6 metri. Puoi controllare (20) l'emissione o meno (10)

\textbf{Immune}\index{Immune} 5-20: a cosa ?

\textbf{Invisibile} \index{Invisibile}40: il tuo corpo è invisibile. Sempre. E non è magia...

\textbf{Ira} \index{Ira}5: sei capace di infuriarti. +2 POT -1 Difesa. Ogni altri 5 punti +2 POT -1 a Difesa Max 20 punti. Durata 4 (anche non consecutivi)round ogni 5 punti. Si attiva come Azione a costo 1.

\textbf{La mia ombra è mia amica} \index{La mia ombra è mia amica}10: Riesci a posizionare la tua ombra dove vuoi. Si considera tu possa lanciare Essenze a tocco tramite la tua ombra (che deve essere presente) entro raggio 3 metri.

\textbf{Legami di furia} 15\index{Legami di furia} : Puoi evocare lacci eterei che minacciano i tuoi nemici. Per 3 volte al giorno con il costo di 1 Azione tutti gli avversari in raggio entro 9 metri attorno a te sono intralciati per un round. TS vs Riflessi DC 10+½ lv + Magnetismo) per liberarsi.

\textbf{Lento e Fermo} 5: \index{Lento e Fermo}Sei eccezionalmente stabile sui tuoi piedi. Non puoi essere mosso se non da una creatura di 2 taglie superiori.

\textbf{Lingua universale} \index{Lingua universale}10. Le tue capacità linguistiche sono impressionanti. Dopo due giorni a contatto con una nuova lingua sei in grado di parlarla correttamente. Dopo 3g di lontananza dall'ambiente dimentichi la lingua. Guadagni un +2 ai check basati sulla lingua

\textbf{Magia esplosiva} \index{Magia esplosiva}10: le tue Essenze di Attacco hanno un dado in più di danno (quando c'è da tirare un dado..)

\textbf{Mani di Fata} \index{Mani di Fata}10: +4 prova di Criminalità che coinvolgano le mani. Puoi prendere 16 come prendessi un 10 nelle prove relative.

\textbf{Mano Piede palmata} \index{Mano Piede palmata}5: +4 alle prove di nuotare

\textbf{Mattiniero} \index{Mattiniero}5-10-15: ti basta dormire 6/5/4 ore per notte
per essere riposato completamente

\textbf{Medium} \index{Medium}10-20: alcune volte lo vuoi tu, altre volte ti cercano loro

\textbf{Memoria fotografica}\index{Memoria fotografica} 20-50: per fortuna non è permanente (50). +8 alle prove per ricordare dettagli (Cultura e Consapevolezza)

\textbf{Naso peloso} \index{Naso peloso}5: le tue narici filtrano le tossine presenti nell'aria che respiri. +2 alle prove relative. Il tuo naso è di dimensioni.. non piccole.

\textbf{Non dormi}\index{Non dormi} 20{*}: e non so come fai..

\textbf{Non invecchi}\index{Non invecchi} 20{*}: non invecchi (ma possono ucciderti lo stesso)

\textbf{Non mangi bevi} \index{Non mangi bevi}20: e non so come fai..

\textbf{Non respiri} \index{Non respiri}20: e non so come fai..

\textbf{L'Odore del sangue} \index{L'Odore del sangue}10: L'odore di sangue è una droga potente
Prerequisiti: non puoi avere "Il fegato non conta" . Guadagni un +1 a Tiro per Colpire ed un +1 al danno per ogni nemico che hai ucciso con la tua arma nel round. Questo bonus non può superare il +4/+4. Il bonus rimane attivo fino al round successivo all'ultima uccisione fatta. Creature con meno di 3 lv di te non contano.

\textbf{Oracolo} \index{Oracolo}20: per qualcuno è una maledizione

\textbf{Ottima vista} \index{Ottima vista}5: hai un ottima vista (12/10). +2 alle prove relative che usano la vista.

\textbf{Ottimo olfatto e gusto} \index{Ottimo olfatto e gusto}5: hai un ottimo gusto ed olfatto. +2 alle prove relative che usano olfatto o gusto.

Con un prova su Intelletto a DC 15 puoi capire cosa è una pozione.

\textbf{Ottimo tatto} \index{Ottimo tatto}5: hai un ottimo tatto. sai leggere con le dita. Sei in grado di trovare una porta nascosta toccando la parete.

\textbf{Ottimo udito}\index{Ottimo udito} 5: hai un ottimo udito. +2 alle prove che coinvolgono l'udito

\textbf{Parlare con gli animali}\index{Parlare con gli animali} 20: scegli una famiglia (ovini, marsupiali, caviette..)

\textbf{Parlare con le piante} \index{Parlare con le piante}20: ho sempre voluto parlare con le zucchine..

\textbf{Percezione Cieca}(vista cieca):\index{Percezione Cieca} \index{Vista Cieca}30: riesci a percepire qualsiasi cosa con i tuoi sensi, dall’odore, al calore. Riesci a “vedere” attraverso e fino 18 metri, 10 cm di pietra, 20 cm di legno, 0.5 cm di metallo

\textbf{Perfetto equilibrio} 5:\index{Perfetto equilibrio} +2 alle prove relative di Acrobatica

\textbf{Piedi veloci} 10: il tuo movimento aumenta di 3 metri

\textbf{Pollice verde} 5: +4 alle prove di Lavoro (Erboristeria, Professione Giardiniere..)

\textbf{Polmoni di ferro} 5: puoi trattenere il respiro 20*POT round (minimo 20 round)

\textbf{Precognizione} 30{*}: Puoi usare l'Essenza Rivelazione. In automatico conosci l'Essenza ed hai un valore in Competenza Magica pari al livello per questa Essenza (oppure un +6 al check se hai già l'Essenza)

\textbf{Recupero} 10: il tuo corpo produce spontaneamente caffeina.  Ignori la condizione affaticato

\textbf{Resistenza} 5-10: +1/+2 TS a Riflessi o Tempra o Arbitrio

\textbf{Resistenza al danno} 10: -1 danno. -1 danno aggiuntivo ogni 5 punti aggiuntivi

\textbf{Resistenza al magico} 20: Hai una RM 3, ogni qual volta sei influenzato da una Essenza tira 3d6+RM se è superiore alla prova di Competenza Magica di chi ti lancia l'Essenza o 6+Livello Potere (in caso di oggetti/pozioni..) l'Essenza non ha effetto.

\textbf{Resistenza al fuoco/freddo/Elettricità} 5-10: ignori i primi 3/6 punti di danno per round

\textbf{Ricostruzione} 30: perdere una mano non è mai stato un problema..

\textbf{Rigenerazione} 30: +1PF {*}T (non rigeneri arti)

\textbf{Rigenerazione} \textbf{veloce} 40: +1PF per round (non rigeneri arti). Muori se distruggono il tuo corpo (o non rimane che cenere).

\textbf{Rimpicciolimento} 30: puoi diminuire fino a due taglie. Durata fino a 8 ore

\textbf{Rinoceronte} 10 : La tua carica è distruttiva. Si considera che niente sotto la robustezza di sbarre di ferro (durezza 15) possa fermare la tua carica. Dietro di te lasci una scia di distruzione. +2 ai Tiri per Colpire in Carica.

\textbf{Scudo Mentale} \index{Scudo Mentale}10: +2 TS su controlli ed influenze mentali

\textbf{Sensi protetti}\index{Sensi protetti} 5: +2 TS contro suoni/luci/vapori o Essenza di Distruzione che agisca sui tuoi sensi

\textbf{Senso comune} \index{Senso comune}5: se stai per fare una brutta figura un campanellino ti avvisa

\textbf{Senso della moda}\index{Senso della moda} 5: sai sempre come vestirti bene. anche solo con uno straccetto

\textbf{Senso delle vibrazioni} \index{Senso delle vibrazioni} \index{tremorsense} (tremorsense) 30: tutto emette vibrazioni, o quasi, raggio di 18 metri intorno a te

\textbf{Senso del tempo} \index{Senso del tempo}5: sai sempre che ore sono, giorno o notte.

\textbf{Senso ragno}\index{Senso ragno} 15: no non ti ha morso un uomo radioattivo,ma sei estremamente sensibile ai pericoli. +2 iniziativa, non puoi essere sorpreso

\textbf{Senza paura} \index{Senza paura}10: sei immune alla paura, magica o meno.

\textbf{Silenzioso} \index{Silenzioso}5: +4 alle prove di Consapevolezza (muoversi silenziosamente)

\textbf{Spine} \index{Spine}5: e sei pure brutto. 1d4 di danno

\textbf{Super piastrine} \index{Super piastrine}5 Riduci il danno da Sanguinamento di 1 a fine di ogni round.

\textbf{Talento per le lingue}\index{Talento per le lingue} 5: impari due lingue investendo 1 punto in Conoscenza Linguistica

\textbf{Talento selvaggio}: \index{Talento selvaggio}parliamone

\textbf{Tocco gelido} \index{Tocco gelido}10: toccando un morto (entro 1 giorno per livello) puoi vedere e sentire cosa è successo nel suo ultimo round di vita.

\textbf{Troll} \index{Troll}50: rigeneri 5 pf a round anche se i PF sono negativi. Rigeneri anche arti. Puoi essere "ucciso" solo da fuoco o acido. Una condizione potrebbe comunque tenerti a punti ferita negativi (es. immerso sott'acqua).

\textbf{Udito subsonico}\index{Udito subsonico} 10: senti le frequenze inudibili per gli umani (come un cane)

\textbf{Vedere l'invisibile} \index{Vedere l'invisibile}15: meglio la vista a raggiX.. sbav..

\textbf{Comprensione del vero}\index{Comprensione del vero} 10: la verità ha un suono tutto suo. +4 alla prove di percepire inganni

\textbf{Visione oscura} \index{Visione oscura}15: vedi nell'oscurità più totale, anche magica, fino a 18 metri

\textbf{Visione Perimetrale} \index{Visione Perimetrale}5: sogliola ? +2 alle prove di Consapevolezza da lato

\textbf{Visione Telescopica}\index{Visione Telescopica} 10: +4 alle prove di Consapevolezza e visione solo da lontano

\textbf{Voce suadente} \index{Voce suadente}5: +2 alle prove di Magnetismo che usano la voce

\textbf{Voce subsonica}\index{Voce subsonica} 10: emetti suoni non udibili dagli umani. I cani ti odiano

\pagebreak

\section{Svantaggi}\index{Svantaggi}

\label{svantaggi}
\begin{tcolorbox}[enhanced,arc=5pt,boxrule=0.3pt]{Se devi essere storpio, meglio essere uno storpio ricco. (Tyrion Lannister)}\end{tcolorbox}\medskip

Uno svantaggio caratterizza il personaggio, ne definisce limiti e paure. Ogni personaggio deve avere almeno 1 svantaggio di ruolo e questo non gli da punti bonus.

I punti presi con gli Svantaggi psico/fisici servono a coprire i punti spesi con i Vantaggi. Ovviamente l'Evil Narratore gradisce anche più svantaggi...

\textbf{Ogni giocatore deve giocare i suoi svantaggi altrimenti non acquisisce punti esperienza e gli sarà negato l'uso dei Vantaggi.}

Uno svantaggio può essere "annullato" nel corso della storia del personaggio e deve esserci una avventura che giustifichi il tutto. Come sempre il Narratore ha l'ultima parola su ogni scelta di vantaggi e svantaggi.

\bigskip

Suggerimenti
\begin{itemize}
	\item
	      prendi degli svantaggi che siano divertenti da giocare, anche se ti metteranno nei guai.
	\item
	      prendi degli svantaggi che siano interessanti da giocare con gli altri giocatori anche se metteranno loro nei guai
	\item
	      prendi degli svantaggi che c'entrino con il personaggio
	\item
	      prendi degli svantaggi di cui non andrai a pentirti
\end{itemize}

\textbf{Fai attenzione}:

\begin{itemize}
	\item
	      evita gli svantaggi che sono difficili da giocare o perché completamente avulsi dal sistema o totalmente inutili o severamente dannosi per gli altri. Se vuoi essere un pacifista estremo, valuta bene il personaggio ed il gruppo..
	\item
	      non prendere svantaggi che ti possa vergognare a recitare
	\item
	      non prendere svantaggi che non c'entrano con il personaggio (in perfetta contraddizione con quanto già detto)
	\item
	      non prendere svantaggi insulsi (tipo la paura di girare a destra , degli ascensori..)
	\item
	      se prendi uno svantaggio severo, recitalo bene, il Narratore saprà ricompensarti
\end{itemize}

Gli svantaggi si dividono in due categorie, \textbf{Svantaggi di Ruolo} e \textbf{Svantaggi psico/fisici}.

Gli \textbf{Svantaggi di Ruolo} sono dei piccoli difetti, tic, problemi grandi e piccoli che servono a dare uno spessore più "umano" al personaggio. Hanno una descrizione volutamente ambigua e scherzosa, sceglili con attenzione e discuti con il Narratore come intendi interpretare questo svantaggio.

Il giocatore è invitato a creare nuovi svantaggi di ruolo. Questi svantaggi non concedono un bonus o malus ne danno punti per prendere vantaggi.

\bigskip

Gli \textbf{Svantaggi psico/fisici} sono invece più impattanti nel gioco, nella quotidianità dando concreti svantaggi. Questi svantaggi forniscono i punti con i quali "pagare" i vantaggi. In fondo trovate un elenco di Fobie

\pagebreak

\subsubsection{Svantaggi di Ruolo}\index{Svantaggi di Ruolo}

\bigskip

\textbf{Alcolismo}:\index{Alcolismo} ti piace bere, e tanto.. ma quando smetti ?

\textbf{Alla moda}\index{Alla moda}: tua probabilmente, anche con vestiti nuovi non ti vesti mai bene. L'accostamento di colori è sempre un pugno nell'occhio.

\textbf{Amico degli animali}:\index{Amico degli animali} intesi come pulci, zecche, pidocchi, cimici.. mosche. Hai uno zoo su di te.

\textbf{Attira animali}: \index{Attira animali}non sai il perché ma sei sempre circondata da gatti, cani, coniglietti, coccatrici..

\textbf{Attira guai}\index{Attira guai}: non è colpa mia se il drago ha deviato per venire a fare la popò qui..

\textbf{Banana}: \index{Banana}quella che provi a farti nei capelli, ma non riesci.
I tuoi capelli non vanno d'accordo con te

\textbf{Bassa soglia del dolore}: \index{Bassa soglia del dolore}mi ha graffiato, aiuto! sto morendo!!!

\textbf{Brufoli}: \index{Brufoli}pieno, hai la faccia butterata e continuano a formarsi questi disgustosi brufoli gialli

\textbf{Ciuccione:} \index{Ciuccione}non lo fai spesso, ma nei momenti in cui sei più nervoso tiri fuori il vecchio ciuccio di legno.. (o in mancanza va sempre bene il proprio pollice)

\textbf{Codardo}:\index{Codardo} è meglio scappare, pardon, raccogliamo prima tutte le informazioni

\textbf{Cogito ergo sum}: \index{Cogito ergo sum}hai la tendenza a parlare tra te e te, ma ad alta voce anche se ci sono persone intorno e pure se non sono amichevoli

\textbf{Credulone}: \index{Credulone}ma dai ? davvero ? e a quale altezza volava l'asino ?

\textbf{Criceto}: \index{Criceto}intesa come memoria. Non riesci ad associare nomi a volti.

\textbf{Denti marci}: \index{Denti marci}probabilmente lo spazzolino che usi non ha setole di vero cinghiale...

\textbf{Dita nel naso}:\index{Dita nel naso} spero che siano almeno buone

\textbf{Diva}: \index{Diva}o almeno tu credi di esserlo. Non perdi occasione per dare sfoggio delle tue inesistenti capacità canore, comiche, estetiche... con grosse risate di tutti

\textbf{Faccia comune}: \index{Faccia comune}come ti chiami ? mi sembra di averti già visto...

\textbf{Galante}: \index{Galante}al limite del maniacale, in ogni tuo gesto sei formale, appropriato e cordiale

\textbf{Killer}:\index{Killer} no, non sei un assassino. Hai però sempre le mani ed i piedi freddi.

\textbf{Impaurisci animali}: \index{Impaurisci animali}può essere anche comodo, se non fosse per i cavalli che scappano e gli orsi che attaccano...

\textbf{Incapace di divertirsi}: \index{Incapace di divertirsi}quindi ? è un problema tuo, non mio

\textbf{Inglese:} \index{Inglese}inteso come umorismo. Nessuno mai capisce le tue battute

\textbf{Mangione}: \index{Mangione}CIOMP!. Mai lesinare, potrebbe essere l'ultimo pasto!

\textbf{Meteora}:\index{Meteora} soffri di meteorismo compulsivo e rumoroso, per non parlare dell'odore sgradevole

\textbf{Megalomane}:\index{Megalomane} coinvolgiamo gli eserciti dei sette regni e penetriamo nel dungeon!

\textbf{Mentina:} \index{Mentina}se mangiassi solo aglio e cipolla il tuo alito sarebbe meno puzzolente

\textbf{Musichiere}: \index{Musichiere}con la bocca. Fischi di continuo, in ogni occasione che sei sovrappensiero o molto teso.. ti metti a fischiettare

\textbf{Non empatico}: \index{Non empatico}perché piange il bambino a cui ho appena dato a fuoco l'orsetto ?

\textbf{Ossessione}:\index{Ossessione} ancora, ancora, ancora. Un'altra crema per la pelle!!

\textbf{Pacco}: \index{Pacco}il tuo. Hai sempre una mano laggiu. Forse i pantaloni sono stretti ? e no, non ti stringo la mano.

\textbf{Pessimo carattere}: \index{Pessimo carattere}va bene essere burbero.. ma devi sempre renderlo palese ?

\textbf{Pezzata}: \index{Pezzata}no, non la mucca o la tua cavalla ma la tua ascella. Sudi copiosamente, che sia per caldo o freddo.. o nervoso.

\textbf{Rigidezza mentale}:\index{Rigidezza mentale} no, non capisco, la mappa dice di andare a destra. Non mi importa se non c'è una destra.

\textbf{Saccente}\index{Saccente}: la risposta giusta è solo la tua. Non c'è dubbio.. per te.

\textbf{Sangue dal naso}: \index{Sangue dal naso}capita, e sempre appena vedi una donna/uomo (a seconda dei gusti) che ti piace

\textbf{Sciarpina}: \index{Sciarpina}devi sempre avere addosso e visibile un capo di un certo tipo, altrimenti non esci di caverna.

\textbf{Segreto}: \index{Segreto}ho un segreto, talmente tanto segreto che non so se lo so neanche io...

\textbf{Seguire il Chaos}: \index{Seguire il Chaos}è più forte di te, non riesci mai ad ubbidire a qualsiasi legge o autorità preposta.

\textbf{Seguire la Legge}: \index{Seguire la Legge}è più forte di te, non importa che legge sia, tu non la violi.

\textbf{Tatuato:} \index{Tatuato}il tatuaggio è il modo di vivere. Hai almeno il 30\% del corpo già tatuato e non perdi occasioni per farti nuovi tatuaggi.

\textbf{Topi}:\index{Topi} sei una TOPI!

\textbf{Unghie}:\index{Unghie} sei un divoratore compulsivo di unghie, la punta delle dita ti sanguina a volte

\textbf{Ultima parola}\index{Ultima parola}: è più forte di te, devi avere l'ultima parola in ogni discorso,

\pagebreak

\subsection{Svantaggi psico/fisici}\index{Svantaggi psico/fisici}

\label{svantaggi-psicofisici}

\textbf{Albino}\index{Albino}

Bianco, quasi latte. Non ti abbronzi e non sopporti la luce, la tua pelle è delicata.

\textbf{13}: Oltre ad essere estremamente riconoscibile hai i seguenti svantaggi: Miopia e Fotosensibilità e Pelle Sensibile.

\textbf{Allergia}\index{Allergia}

Hai una qualche forma allergica. Spero non grave. Assicurati di avere sempre con te una pozione di Essenza rimuovi veleno.

\textbf{5:} In presenza di un allergene specifico il personaggio starnutisce sonoramente finché l'allergene non viene allontanato, -1 a tutti i Check. (es. Allergico alla Birra)

\textbf{10}: Il personaggio soffre di attacchi di tosse, iperlacrimazione, giramenti di testa, -2 a tutte le prove. Tiro Salvezza su Tempra DC 10 per non soffocare. Il tiro va ripetuto ogni 20 round finché non ti sei allontanato dall'allergene.

\textbf{15:} Il personaggio soffre di violenti attacchi di tosse, nausea, sudori freddi, palpitazione. -5 a tutte le prove, è necessario un Tiro Salvezza su Tempra DC 15 o perdere i sensi. I tiri vanno ripetuti ogni 5 round finché l'allergene non è allontanato.

\textbf{20}: Il personaggio cade in preda di una crisi respiratoria, ed è incapace di compiere qualsiasi azione che non sia vomitare , annaspare e tossire sangue. Fallendo un Tiro Salvezza su Tempra DC 25 il personaggio muore annegando nel suo stesso vomito. Il tiro va ripetuto ogni round fino a che l'allergene non è allontanato.

\textbf{Allucinazioni}\index{Allucinazioni}

c'è qualcosa che non va nella tua testa, ogni tanto si innesca una scintilla.

\textbf{10}: Il personaggio vede e sente cose che non ci sono. Ogni giorno tiri un dado a sei facce.
Se escono 1 o 2, non succede nulla.
Con 3,4 o 5 si verificheranno uno o due episodi allucinatori con modalità e tempi a discrezione del Narratore.
Con 6 il personaggio sarà vittima di visioni orrende e disgustose con durata di 1d4 ore.

\textbf{Amnesia}\index{Amnesia}

\textbf{10}: Hai dimenticato il tuo passato e con quello il ricordo di amici, nemici, obiettivi. Non c'è modo di recuperare i ricordi perduti.

\textbf{Asceta}\index{Asceta}

10, lo dice la regola. Non porterai con te più di 10 oggetti.

\textbf{20}: non puoi avere più di 10 oggetti con te, magici o normali o monete o armi. Per fortuna i vestiti non contano.

\textbf{Balbuziente}\index{Balbuziente}

Sai parlare, ma male.

\textbf{5:} Hai una fastidiosa tendenza a balbettare proprio quando hai qualcosa da dire di importante. In queste situazioni critiche dalle tue labbra escono solo suoni abbozzati.

\textbf{Pessimo Carattere}\index{Pessimo Carattere}

Le buone maniere non sono mai una opzione.

\textbf{5}: Non hai mai imparato l'arte della diplomazia, e detesti essere contraddetto o insultato. Questo non significa che passi alle vie di fatto, ma che di fronte ad un insulto o ad una critica schietta tendi a zittire il proprio interlocutore con espressioni davvero poco simpatiche. Hai un -2 alle prove basate sul Magnetismo

\textbf{Spendaccione}\index{Spendaccione}

\textbf{10}: devi spendere metà dei tuoi guadagni di missione in piaceri futili (mangiare cibi costosi, bere vino e liquori pregiati, vestiti lussuosi, no armi od oggetti magici)

\textbf{15}: devi spendere tutti i tuoi guadagni di missione in piaceri futili (mangiare cibi costosi, bere vino e liquori pregiati, vestiti lussuosi, no armi od oggetti magici)

\textbf{Caritatevole}\index{Caritatevole}

\textbf{10}: devi donare metà dei tuoi guadagni di missione in beneficenza

\textbf{15}: non può tenere più di 10 mo in contanti

\textbf{Cecita'}\index{Cecita'}

\textbf{10}: Sei orbo, visione laterale compromessa, problemi nel capire la distanza delle cose.
Le competenze quali Sopravvivenza e i Check CA per colpire con armi da lancio hanno
un -4. La Difesa peggiora di 2

\textbf{20}: sei cieco. Non vedi. tutti i nemici sono Invisibili.

\textbf{Cleptomania}\index{Cleptomania}

\textbf{5}: Senti il bisogno irresistibile di appropriarti di oggetti “interessanti”, di tanto in tanto. Se in un giorno non hai rubato almeno un oggetto non potrai usare Punti Fato per quel giorno.

\textbf{Codice Etico/Voto}\index{Codice Etico}\index{Voto}

Hai fatto un voto, una promessa, un giuramento che condiziona il tuo agire.

5-10 : stabilisci bene le regole, nero su bianco, e si chiaro con il Narratore

\textbf{Compulsivo}\index{Compulsivo}

Ci sono certi comportamenti, per te necessari, dei quali non puoi fare assolutamente a meno (es: camminare evitando le macchie sul terreno o passando solo su quelle, sfilare l'arma solo in un certo modo, ecc)
Questi comportamenti vanno dichiarati ed esplicitati al momento della scelta dello svantaggio.

\textbf{5-10}: quando sei preda del comportamento compulsivo hai un -2 alle prove di Consapevolezza / sei sempre l'ultimo ad agire indipendentemente dall'iniziativa tirata o dall'ordine di marcia / altro

\textbf{Daltonismo}\index{Daltonismo}

Sei cieco ai colori, un tramonto sarà qualcosa di triste visto in grigio

\textbf{5}: non hai la consapevolezza dei colori (acromatopsia). Vedi tutto in scala di grigi.

\textbf{Deformita'}\index{Deformita'}

Non tutti nascono belli o dritti. c'è anche chi nasce storto e brutto.

\textbf{5}: Malformazione minore, incide a scelta tra Potenza o Agilità. Togli 1 punto a questa statistica.

\textbf{10}: Due caratteristiche a tua scelta non possono superare i 2 punti se non magicamente. Hai movimento dimezzato.

\textbf{20}: Malformazione grave. Tre caratteristiche a tua scelta non possono superare 1 punto. Hai movimento dimezzato

\textbf{Depressione}\index{Depressione}

Ogni giorno è un pessimo giorno e nulla lo farà migliorare

\textbf{8}: Adori il Blues ma purtroppo hai perso la gioia di vivere, l'entusiasmo, la speranza.

Nulla sembra avere importanza, non fai che trascinarti stancamente da un giorno all'altro. -2 ad ogni prova di competenza

\textbf{Dipendenza}\index{Dipendenza}

\textbf{10}: Hai una dipendenza, possa essere alcool, droga, donne...Se non ne consumi ogni giorno una congrua dose (il Narratore ti saprà dire quanto basta) prendi un -2 a tutti i Tiri Salvezza. Dopo 3 giorni di astinenza divieni anche Depresso

\textbf{Dislessia}\index{Dislessia}

jk j0j zo mdbbdfd

\textbf{10}: Non sei in grado di leggere e scrivere. Non sei capace di associare correttamente suoni a lettere e forme a suoni

\textbf{Disonestà Compulsiva}\index{Disonestà Compulsiva}

Menti, è più forte di te.

\textbf{5}: Il personaggio è portato dalla propria insicurezza a mentire sempre e comunque. Ogni volta che il personaggio è costretto ad ammettere le proprie responsabilità o comunque a parlare contro il proprio interesse, o in qualunque situazione in cui si senta "esaminato", egli si inventerà storielle piuttosto fantasiose anche mettendo in pericolo amici e parenti.

\textbf{Dolore Cronico}\index{Dolore Cronico}

oh che male. Incatatore usi un'Essenza di cura su di me anche oggi ?

\textbf{10}: non recuperi Punti Ferita se non magicamente

\textbf{Emofilia}\index{Emofilia}

tendi a sanguinare sempre, anche nei momenti meno opportuni

\textbf{8}: CEROTTO!!! (ogni attacco che subisci automaticamente cumula Sanguinamento +1)

\textbf{Epilessia}\index{Epilessia}

sempre e solo nei momenti meno opportuni

\textbf{15}: ogni qual volta fai un 3 con un Tiro Salvezza o un Tiro per Colpire, cadi a terra per 1d6 round in preda alle convulsioni, si considera che il Tiro per Colpire o salvezza sia fallito. Sei considerato indifeso.

\textbf{Feticismo}\index{Feticismo}

Se non annusi un piede di donna diventi depresso.

\textbf{5}: Il personaggio è irresistibilmente attratto da un oggetto, corpo, categoria... Ogni giorno in cui egli si trova lontano dalla sua fonte di piacere, si consideri caduto in Depressione.

\textbf{Ricordi}\index{Ricordi}

ehi.. ci sei? perché ti sei paralizzato ? e queste cose quando le hai imparate ?

\textbf{5}: ad ogni check di competenza tira un d4. Con 1-2 fai la prova normale, con 3 fai la prova con un -2, con 4 fai la prova con un +2

\textbf{Fobie}\index{Fobie}

\textbf{Varie}: Il personaggio è terrorizzato da un oggetto, da una categoria di persone o di esseri viventi, da una situazione. In presenza della causa scatenante, il personaggio cade in preda ad un attacco di panico: l'unico suo desiderio è quello di fuggire il più possibile lontano dalla fonte del suo terrore, con ogni mezzo; chiunque gli sbarri il cammino è da considerarsi un nemico. Se il personaggio si trova nell'impossibilità di fuggire, egli cade in uno stato catatonico finché la causa scatenante non viene eliminata. Vedere in fondo tabella possibili fobie

\textbf{Fotosensibilita'}\index{Fotosensibilita}

La luce anche se leggera ti da fastidio.

\textbf{5}: Il personaggio ha un -1 in ogni tiro in cui la luminosità è almeno quella diurna

\textbf{10}: Il personaggio ha un -2 in ogni tiro in cui la luminosità è almeno quella di una lanterna

\textbf{20}: Il personaggio ha un -3 in ogni tiro in cui la luminosità è almeno quella di una torcia. Il personaggio è così sensibile che è per lui impossibile muoversi liberamente in luoghi direttamente o meno illuminati, preferirà muoversi e viaggiare di notte.

\textbf{Ghiro}\index{Ghiro}

ti piace dormire e tanto. Ronf

\textbf{5}: +2 per ogni 2 ore oltre le 8, altrimenti sei affaticato.

\textbf{Goffaggine}\index{Goffaggine}

\textbf{10}:Il punteggio della Agilità non può superare 2. Hai un -2 a tutte le prove che richiedano Agilità (disattivare congegni, svuotare tasche, arrampicarsi, iniziativa....)

\textbf{Igenista}\index{Igenista}

ho finito il sapone. HO FINITO IL SAPONE! .. non tocco quella spada, anche se brilla di luce sacra e vola a mezz'aria finché non sarà disinfettata!

\textbf{5}: hai l'impulso a pulirti di continuo e pulire tutto ciò che dovrai toccare.

\textbf{Incoscienza}\index{Incoscienza}

\textbf{5}: Non hai paura di nulla. Letteralmente. Se devi fare una cosa il piano più diretto ed immediato è la scelta migliore. Non riesci a studiare piani che durino più di un minuto. Prendi un +1 all'Iniziativa ed un -2 al Competenza con Armi

\textbf{Indeciso}\index{Indeciso}

Non facciamolo, aspettiamo domani..magari è meglio!

\textbf{10}: non agisci mai per primo. -4 alle prove di iniziativa

\textbf{Incubi Ricorrenti}\index{Incubi Ricorrenti}

\textbf{8}: Il personaggio non riesce a dormire bene. Ogni notte tira un 1d4. Con 1 il personaggio dorme normalmente, 2 o 3 il personaggio dorme un sonno agitato e si sveglia affaticato, con 4 ti svegli in piena notte urlando, la mattina sei esausto.

\textbf{Libro Aperto}\index{Libro Aperto}

si, lo so, posso stare zitto, tanto avete già capito tutto.

\textbf{5}: non è che non sei in grado di mentire è che hai un -4 alle prove di Ingannare

\textbf{Emicrania}\index{Emicrania}

Non è mai un buon giorno. Soffri di continui e feroci mal di testa.

\textbf{15:} Il personaggio soffre di violenti mal di testa.. Ogni giorno il personaggio tira un d4: con 1 il personaggio non lamenta alcun effetto, con 2 o 3 subisce una penalità di -1 a tutti lle prove, con 4 la penalità diventa -2.

\textbf{Maledetto}\index{Maledetto}

Sei Maledetto. Un oscuro destino ha macchiato la tua anima

\textbf{5-10}: porti una maledizione. Discutine con il Narratore

\textbf{Miopia}\index{Miopia}

Spera di trovare degli occhiali

\textbf{5}: Ci vedi poco. Hai un -2 a tutti i check competenza con armi da colpire da lontano e prove di Consapevolezza oltre i 12 metri.

\textbf{15}: Ci vedi molto poco. Hai una prova -4 competenza con armi da colpire da lontano e prove di Consapevolezza oltre i 9 metri.

\textbf{Muto}\index{Muto}

Non puoi parlare e cosa peggiore non riesci neanche ad infamare il
tizio che ti sta pestando il piede

\textbf{10}: Non sei in grado di emettere suoni. Non parli o meglio nessuno ti sente. Prendi un -4 ai check basati su Magnetismo e competenze orali

\textbf{Discalculia}\index{Discalculia}

1+1= ?

\textbf{10}: il personaggio ha un disturbo che gli impedisce di padroneggiare il concetto di numerazione. Non solo non è in grado di svolgere le operazioni più semplici, non è neanche in grado di comprendere i concetti di maggiore/minore, o informazioni quantitative di qualunque tipo.
Attenzione al resto che ti danno...

\textbf{Obesita'}\index{Obesita'}

Sei decisamente fuori forma, e di tanto.

\textbf{10}: Agilità non può essere sopra 2. Hai un -4 alla prove di Agilità ed ai tiri salvezza su riflessi. Guadagni un +2 ai Tiri Salvezza su Tempra

\textbf{Olfatto/Gusto Difettoso}\index{Olfatto/Gusto Difettoso}

Naso, palato, lingua bruciata, abuso di peperoncino o wasabi.. possono essere tante le cause

\textbf{5}: -2 due alle prove che usano gusto od olfatto. Non senti sapori e odori se non estremi.

\textbf{Onestà Compulsiva}\index{Onestà Compulsiva}

\textbf{7}: Non sai mentire, la sola idea di dire una menzogna ti rende nervoso. Prendi un -4 a Faccia Tosta. Se messo alle strette, il personaggio confesserà tutto a prescindere dall'importanza delle informazioni in suo possesso.

\textbf{Ossa di Cristallo}\index{Ossa di Cristallo}

Si chiamerebbe osteogenesi imperfetta ma per te sono solo dolori continui.

\textbf{5}: Il personaggio ha le ossa fragili. Ogni danno causato da arma da botta causa 2 PF in più di danno

\textbf{10}: Il personaggio ha le ossa fragili. Ogni danno causato da arma da botta causa 5 PF in più di danno

\textbf{Paranoioso}\index{Paranoioso}

Sei paranoico e noioso.

\textbf{5}: Ti comporti sempre in modo furtivo, anche senza che ce ne sia effettivo bisogno, destando così sospetti nelle persone che hai attorno.

Ogni prova di Consapevolezza ha una difficoltà di -5 aggiuntiva ed un fallimento critico indica che il target ha qualcosa di vitale da nascondere.

\textbf{Pelle Sensibile}\index{Pelle Sensibile}

Non ami il Sole, o almeno la tua pelle non lo ama.

\textbf{5}: Il tuo personaggio si scotta facilmente, un'esposizione prolungata senza le adeguate protezioni comporta dolorose e antiestetiche bruciature e disagi.

\textbf{10}: Sei oltremodo sensibile agli ultravioletti. Ogni danno da fuoco causa 2 danni aggiunti.

\textbf{Pigro}\index{Pigro}

sei lento e svogliato

\textbf{5}: -2 all'iniziativa

\textbf{Rumoroso}\index{Rumoroso}

Non lo fai apposta, ma c'è sempre un qualche rumore intorno a te. Una spada che sbattocchia, uno sbadiglio, un rutto, una scarpa rumorosa..

\textbf{5}: hai un -4 alle prove di muoversi silenziosamente

\textbf{Sangue Debole}\index{Sangue Debole}

\textbf{10}: Il sistema immunitario del personaggio fa decisamente pena. -2 ai Tiri Salvezza su Tempra

\textbf{Sbadataggine}\index{Sbadataggine}

Ops..non me ne ero accorta!

\textbf{7}: Tendi a non fare caso a quello che succede intorno a te, meno che tu non abbia ottimi motivi per stare all'erta, o non stia cercando attivamente qualcosa prendi un -4 a Consapevolezza

\textbf{Schizofrenia}\index{Schizofrenia}

Non sono stato io, ma l'altro!

\textbf{4}: Hai più personalità, o forse ne è convinto l'altro.

Il personaggio ha almeno una seconda personalità (max 6).
Ogni Personalità in più da gestire, oltre la prima, concede un +1 al costo.
Quindi avere 3 personalità portà lo svantaggio a 6 punti

Ogni giorno viene tirato 1d6. Con 6, durante il giorno la seconda personalità viene alla luce.

\textbf{Sfortunato}\index{Sfortunato}

le cose non capitano e basta, bisogna saperle anche cercare

\textbf{5}: ignori il primo critico che fai (TC o TS) nella giornata

\textbf{7}: ignori i primi tre critici che fai (TC o TS) nella giornata

\textbf{Sindrome Maniaco Depressiva}\index{Sindrome Maniaco Depressiva}\index{Incoscienza}\index{Depressione}

Oggi è venerdi'!!! è Venerdi!!!

\textbf{7}: Il personaggio alterna stati di euforia a momenti di cupa disperazione. Ogni giorno viene tirato 1d4. Con 1 il personaggio ha un umore "normale". Con 2 o 3 si consideri in Depressione , con 4 è in uno stato di gioiosa esaltazione (vedi Incoscienza ) e spavalderia.

\textbf{Soggezione}\index{Soggezione}

chiedo scusa

\textbf{10}: Il personaggio è molto insicuro e tende a fidarsi ciecamente degli altri, specie se carismatici . Prendi un -2 alle prove di Criminalità e Faccia Tosta.
Prendi un -2 ai Tiri Salvezza su Essenza Charme

\textbf{Sonno Leggero}\index{Sonno Leggero}\index{affaticato}

ogni rumore di disturba, non riesci mai a dormire bene

\textbf{5}: Se dormi in una zona con rumori naturali / umani (bosco/citta') non riesci a riposare bene. La mattina sei Affaticato. Puoi evitare il problema usando tappi per le orecchie, che ti impongono un -4 alle prove di Consapevolezza su udito per svegliarti.

\textbf{Sordita'}\index{Sordita'}

Il silenzio ha un suono tutto suo dice chi ci sente, per te è solo uno straziante urlo muto.

\textbf{10}: Non ci senti. Non puoi fare prova di Consapevolezza che richiedano l'uso dell'udito. Non puoi ascoltare le persone che parlano. Ma puoi leggere le labbra se sai farlo.

\textbf{Vertigini}\index{Vertigini}

I disagi si manifestano nel momento in cui il personaggio è conscio dell'altezza. Solo per il fatto di camminare in montagna non ha penalita'

\textbf{5}: Ad altezze superiori i 20 metri tendi a bloccarti. Prendi un -2 a tutti i check

\textbf{7}: Ad altezze superiori i 10 metri tendi a bloccarti. Prendi un -2 a tutti i check

\textbf{10}: Ad altezze superiori i 6 metri tendi a bloccarti. Prendi un -2 a tutti i check

\textbf{Visione notturna ridotta}\index{Visione notturna ridotta}

I tuoi occhi non lavorano bene con luminosità ridotta.

\textbf{5}: Quando la luminosità è pari o inferiore a quella di una torcia il personaggio ha un -2 ai Tiro per Colpire.

\textbf{Timidezza}\index{Timidezza}

\textbf{5}: Sei timido e riservato.

Hai un -2 alle prove basate su Magnetismo

\textbf{Zoppo}\index{Zoppo}

sei claudicante

\textbf{5}: Hai movimento dimezzato

\textbf{10}: sei significativamente storpio. -2 alle prove che richiedono Agilità, il tuo movimento è dimezzato

\bigskip

\textbf{Tabella Fobie (5-15 punti)}\index{Fobie}

\begin{tabularx}{0.95\textwidth}{lX}
	\toprule
	\textbf{Nome Fobia} & \textbf{Descrizione}\\
	Blennofobia         & Paura Delle Cose Viscide\\
	Keraunofobia        & Paura Dei Tuoni\\
	Ipocondria          & Paura Delle Malattie\\
	Claustrofobia       & Paura Dei Luoghi Chiusi\\
	Coimetrofobia       & Paura Del Cimitero\\
	Edonofobia          & Paura Di Poter Provare Piacere Fisico\\
	Eisoptrofobia       & Paura Degli Specchi\\
	Glossofobia         & Paura Di Parlare In Pubblico\\
	Monofobia           & Paura Di Rimanere Solo\\
	Necrofobia          & Paura Dei Corpi Morti\\
	Nictofobia          & Paura Del Buio\\
	Acrofobia           & Paura Delle Altezze\\
	Agorafobia          & Paura Degli Spazi Aperti\\
	Rupofobia           & Paura Dello Sporco E Non Igienico. Senti il bisogno di pulire\\
	Afefobia            & Paura Del Contatto E Di Essere Toccati\\
	Asimmetrofobia      & Paura Delle Cose Non Simmetriche\\
	Gimnofobia          & Paura Della Nudita'\\
	Emofobico           & Paura Del Sangue\\
	Traumatofobia       & Paura Di Ferirsi\\
	Sciofobia           & Paura Delle Ombre\\
\end{tabularx}

\pagebreak

\section{Cosmologia}\index{Cosmologia}

\label{cosmologia}
\begin{tcolorbox}[enhanced,arc=5pt,boxrule=0.3pt]{
E' più facile dominare su chi non crede in niente (La Storia Infinita, Kmorf)\\
Tu credi che c'è un Dio solo? Fai bene; anche i demoni lo credono e tremano! (Giacomo Il Giusto 2, 19. NdA Riferendosi al proprio Patrono...)}\end{tcolorbox}\medskip

In TUS le divinità sono leggermente diverse dalle tradizionali divinità dei giochi di ruolo.

In TUS le divinità amano sporcarsi le mani, partecipare nelle faccende delle creature che le adorano, per loro è una sfida continua ad avere più credenti, adepti e persone più simili, per tratti, a loro.

I Patroni sono stati creati come parossismo dell'animo umano, dove tutto è un eccesso. Come spiriti liberati dal vaso di Pandora hanno il solo scopo di portare i loro Tratti al dominio rendendoli i più comuni e presenti tra le creature.

\noindent\rule{\textwidth}{1pt}

\bigskip

In principio era il nulla che in sé conteneva il tutto.

Energia derivante dalle più primordiali pulsioni esplodeva in tutta la sua potenza e senza alcun controllo.

Amore, odio, paura, dolore, gioia, serenità...tutto era aggrovigliato in una fitta ed infinita matassa il cui bandolo era nascosto, attorcigliato, introvabile o...probabilmente ancora inesistente.

Nei millenni a seguire dette energie e pulsioni hanno iniziato a scindersi, trovando una propria personale connotazione e si sono venute a creare tre entita': Atmos, colui che per suo principio controllava l'andamento del tempo e dello spazio, il testimone, lo scriba; Ljust l'energia positiva, il calore, la luce e la vita; Calicante, energia negativa, gelido odio, distruzione e morte.

Atmos è una sorta di spettatore imparziale mentre Ljust e Calicante rappresentano le due lingua di fiamma di un unica energia creatrice.

\textbf{Ljust} \index{Cosmologia}è la rappresentazione di ciò che luce ed amore portano sempre con se'. Rappresenta la purezza del sentimento d'amore, la protezione della vita, il rispetto per l'altro, la curiosità per il nuovo, la voglia di migliorarsi sempre, la forza di combattere con coraggio e valore per il proprio credo.

\textbf{Calicante}\index{Calicante} è la rappresentazione del buio, dell'odio e della rabbia. Lui è vendetta e fredda distruzione. Lui non protegge alcuna forma di vita, le usa, le sfrutta e solo in tali casi ne subisce la presenza. Lui ama sadicamente la sofferenza.

\textbf{Atmos} \index{Atmos}è lo storico, colui che segna il passaggio del tempo e trascrive ogni accadimento di Yeru.

è il testimone del groviglio divino che sono Ljust e Calicante, due creature unite da un unica energia.

Insieme i due Patroni della Genesi hanno dato vita a tutto ciò che conosciamo. Calicante ha creato Tiya\index{Tiya} ed Ljust ha generato Curyan\index{Curyan}, i due regni che compongono il nostro mondo, Yeru. Hanno giocato con forme ed energie creando due regni fra loro speculari ma contrapposti e distinti. Tiya e Curyan, come Calicante ed Ljust, sono parte di un tutto ma, esattamente come i Patroni delle Genesi, sono anche profondamente diversi e magicamente divisi. Esiste infatti una barriera sia fisica formata da catene montuose quasi invalicabili, tempeste marine perenni e mortali ma anche magiche, che ne delimitano i confini e che li tiene nettamente divisi.

Ma proprio come i loro due creatori che divisi e distanti totalmente non possono stare, non possono esistere, così Tiya e Curyan sono sì divisi ma anche in contatto fra loro tramite i Portali. Portali che si generano autonomamente, senza alcun controllo e previsione, a causa dell'entropia magica che preme, spinge e si autoalimenta nel "non luogo" al confine dei due regni e che è generata dalle continue sfide fra i due Patroni. Sono queste magiche vie che consentono di spostarsi tra Tiya e Curyan e viaggiare nel "non luogo" ovvero ciò che è al di fuori di Yeru.

Ljust e Calicante decisero, stranamente di comune accordo, di generare un Patrono che sovrintendesse a queste fratture, che fosse capace di percepire, aprire e bloccare questi Portali. Così venne creato \textbf{Lynx}, il Guardiano dei Portali.

In molti cercano di passare da Tiya a Curyan per cercare la pace, la serenità...altri cercano di valicare il confine inverso alla ricerca di avventura, potere\ldots .alcuni ci provano per vie normali, altri attraversando i Portali, molti si sono persi per sempre nel "non luogo".

Lynx \index{Lynx}sovrintende allo spazio, ai portali che con l'avvicendarsi di caos e ordine, di bene e male, di luce e tenebra stanno sempre più creando fratture al confine esistente fra i due regni. Lynx li percepisce, li "sente", sa dove si stanno generando o riassorbendo, con il passare del tempo infatti alcuni di questi Portali sono divenuti stabili e definitivi mentre altri sono stati riassorbiti dalla naturale entropia, altri invece si generano casualmente e sempre in modo totalmente casuale rimangono attivi o si esauriscono. Viaggiando di continuo nel non luogo Lynx chiude i portali più grandi ma per uno che ne chiude un altro si apre.

Proprio nello svolgimento di questo suo importante ruolo Lynx si scontrò con una strana creatura, rettiloide, gigantesca, alata, potente, forte, sapiente e magica.

Un Drago rosso,\index{Ta'hil} Ta'hil. Quest'ultimo si muoveva nel "non luogo" con la massima libertà, senza alcuna difficoltà e si avvicinò a Lynx. I diari di Atmos narrano di come Lynx cercò di fermarlo e di parlarci, di come venne ferocemente attaccato, delle urla del Patrono Custode che si sentirono echeggiare in entrambi regni, del suono quasi simile ad un gutturale ruggito che squarciò il silenzio nei regni di Tiya e Curyan. Dell'intervento di Ljust e Calicante. La prima a salvare Lynx ed il secondo a scoprire, conoscere questa nuova affascinante "arma".

Lynx si salvo'. Ljust gli infuse le sue essenze di cura e lo aiutò a rigenerarsi. Si lasciò però sfregiato a memoria dell'incontro.

I Draghi avendo scoperto il nostro mondo e mossi dalla loro sete di conoscenza e di potere si sono avventurati nel "non luogo" e sono usciti dai Portali presenti su Tiya e Curyan.

Un'orda di draghi di tutti i colori ha oscurato i cieli. Da quel momento saccheggi, razzie e violenza furono perpetrati indifferentemente nei due regni. Erano molto intelligenti e furbe. Potenti oltre l'immaginabile, manipolavano una magia per alcuni versi diversa e slegata dalle Essenze.

Avevano una robustezza fuori dall'ordinario. Ma soprattutto, non temevano i Patroni. Non si sottomisero a loro.

Atmos incanalò le energie primordiali e divine dei Patroni della Genesi andando a creare delle divinità che potessero rivaleggiare con i draghi e potessero difendere Yeru.

Il primo creato da Atmos, con l'aiuto di Ljust fu \textbf{Gradh}\index{Gradh}, Patrono dell'Umanità (e di tutte le razze senzienti), colui che avrebbe difeso il creato dai draghi e dagli altri Patroni.

Gradh racchiude in sé il dualismo dei due Patroni della Genesi, l'istinto innato alla protezione, alla difesa ed alla cura propri di Ljust e l'istinto di vendetta, violenza e furia di Calicante.

Si getta con coraggio nelle battaglie, attacca il nemico senza paura, protegge il più debole, difende la vita ma non teme di percorrere la strada della vendetta più distruttiva verso chi sfrutta e distrugge vite senza motivo.

Gradh ama "calarsi" fra la gente e vivere con loro, come loro.
Non si sente totalmente a suo agio nè nel Pantheon con gli altri Patroni nè fra la gente comune, lui è Umano tra i Patroni e Patrono tra gli Umani. Passionale e gentile è il Patrono che maggiormente ha a cuore le sorti di Yeru e delle sue razze.

Le lingue di energie divine erano troppo intense, chaotiche e pure perché Atmos potesse governarle per plasmare da solo gli ulteriori Patroni con la stessa naturalezza con cui aveva creato Gradh. Ed ecco che questi altri Patroni risultano meno perfetti e divini, più imperfetti ed "umani" in quanto originati da pulsioni violente e incontrollabili, vere e non filtrate in alcun modo.

Ogni Patrono ha un accesso limitato o negato ad alcune Essenze, non sono più liberi e totali manipolatori dell'energia arcana e divina legata ad un definito reame magico ma al contrario hanno un accesso limitato e vincolato alla magia.
I Patroni plasmano le volontà, fondano regni, comandano nell'ombra come pedine le creature che osano chiedere i loro favori.

Gradh percepì sin da subito che i Draghi rappresentavano un elemento di ulteriore chaos, di ulteriore sofferenze e guerra. Come Patrono di Yeru e delle sue creature sentiva i Draghi come creature aliene, non originarie, non facenti parte del piano della Genesi.

Diffidente per natura Gradh decise di proporre ai Patroni della Genesi di fare un patto con i Draghi.

Ecco che poco più di 300 anni fa, il 15 Prineva del 65 del sesto ciclo, sull'isola di Atilantis che divide Tiya e Curyan si trovarono Atmos, la fiamma di Ljust e Calicante e Gradh da una parte mentre Ta'hil, il drago rosso malvagio e immortale e Dyenos\index{Dyenos}, il drago d'argento sapiente e buono dall'altra.

Gradh cercò di imporre la cacciata dei Draghi e la chiusura dei Portali, Atmos rimase in silenzio a trascrivere la discussione. Ljust cercò di mediare capendo che non tutti i Draghi erano malvagi e che avrebbero potuto dare tanto a Yeru.

Calicante finse di dare ragione a Ljust con il solo scopo di portare maggior caos e distruzione attraverso i Draghi.

Capito che l'esito dell'incontro era già deciso Gradh abbandonò la Piana della Solitudine lasciando ai Draghi ed ai Patroni della Genesi di formalizzare la spartizione di Yeru. Per lui era stata una sonora sconfitta e da allora fu ancora di più è diffidente, se non prevenuto, verso tutti i draghi.

Calicante e Ta'hil rimasero strettamente in contatto e si stanziarono a Tiya mentre Dyenos giurò fedeltà e fiducia a Ljust e decisero di governare insieme Curyan, grazie allo loro volontà di calarsi tra gli umani.

Se un Patrono agisce in prima persona o in modo indiscriminato sa che scatenerà la reazione di Gradh o l'intervento di Atmos che gli impediranno un uso incontrollato e massivo dei suoi poteri direttamente sul mondo. Questo però non sempre li ferma e la stessa natura creature e piante, vengono spesso influenzate dal volere dei Patroni.

A Tiya, ma a volte anche a Curyan, nascono sempre più spesso aberrazioni, malattie sempre nuove, terre maledette dove non può crescere nulla, per non parlare di pazzie che spesso coinvolgono chi invece dovrebbe proteggere i comuni cittadini.

è una dura vita quella dell'uomo comune che continuamente deve affrontare siccità o alluvioni, morie di animali ed un meteo irregolare se non assurdo. Ad ogni passo deve guardarsi intorno perché non puoi mai sapere chi ha venduto l'anima per vivere un giorno in più.

A Curyan si vede svilupparsi l'armonia e la quasi perfetta convivenza fra natura e razze superiori. Esiste il dolore, esiste la malattia e la morte ma il tutto come naturale ciclo della vita come parte della stessa che viene protetta, guidata, aiutata.

I nemici principali sono i Draghi che spesso fanno incursioni per portare distruzione e morte e seminare paura e pazzia.

Non è sempre tutto idilliaco, vaste regioni di Curyan stanno diventando incubatrici di razze oscure e malvagie, legioni di non morti guidate da potenti negromanti si ammassano sui confini, i Draghi addestrano i loro adepti corrotti, e scure spire nere nel cielo promettono tempesta.

\pagebreak

\subsection{Patroni}\index{Patroni}

\label{patroni-dei}
\begin{tcolorbox}[enhanced,arc=5pt,boxrule=0.3pt]{
Conan: A quali dei preghi?\\
Subotai: Io prego ai quattro venti e tu?\\
Conan: Io prego Crom, ma solo raramente... lui non ascolta. (Conan il Barbaro, film 1982)\\\\
Infatti, come il corpo senza lo spirito è morto, così anche la fede senza le opere è morta. (Giacomo Il Giusto 2, 26. NdA Riferendosi ai punteggi dei Tratti collegati al Patrono...)}\end{tcolorbox}\medskip

Le creature tutte, anche chi non usa le essenze possono sentire l'influenza di questi Poteri, di questi Patroni.

Se un personaggio per il suo modo di essere (giocare) e comportarsi ha almeno un tratto in comune con un Patrono ed anzi matura e potenzia queste convinzioni, anche se non ha giurato fedeltà ad un Patrono potrebbe comunque sentire l'influenza del Patrono e ricevere dei doni da lui.

Un Patrono è ben contento se qualcuno segue i suoi dettami, Tratti, senza che sia un usufruitore di magia e dona a coloro che lo fanno dei piccoli poteri come riconoscimento per la fedeltà a lui riservata, volutamente o meno. I poteri indicati sotto "Tratti in Comune" sono cumulativi.

Ad ogni Patrono troverete associate le Essenze (privilegiate, normali, limitate e negate), troverete i Tratti che li caratterizza ed anche due forme di Energia, che rappresentano la loro forma di attacco tipico.

Ricordatevi di usare questi Elementi quando effettuate un'Essenza di Attacco, o un incanalare energia, altrimenti non raggiungerete il Livello di Potere prefissato, tranne se userete un elemento neutrale.

Le forme di Energia vengono distinte tra fonti positive, neutrali e negative, vi servono anche per inquadrare meglio il vostro Padrone pardon il Patrono che servite.

Fate la somma degli elementi, se positiva il Patrono si può considerare buono, se a valore zero il Patrono è neutrale, se a valore negativo il Patrono è malvagio.

Nella descrizione del Patrono troverete anche la sua manifestazione, ovvero cosa accade quando un personaggio agisce in maniera particolarmente e significativamente consona ai tratti seguiti dal Patrono. L'effetto è puramente ambientale e di circostanza ma lascia sempre colpito chiunque lo possa osservare.

Un incantatore che si affida ad un Patrono, almeno 3 Tratti in comune, diventando quindi un Devoto, \index{Devoto}segue le normali regole delle Essenza tenendo però conto dei vantaggi (essenze favorite) e svantaggi (essenze limitate e negate) che comporta.

\bigskip

\textbf{Tabella Elementi}
\medskip

\begin{tabular}{lll}
	\toprule
	\textbf{Positivi} (+1) & \textbf{Neutrali} (0) & \textbf{Negativi} (-1)\\
	Energia Positiva       & Fuoco                 & Energia Negativa\\
	Luce                   & Freddo                & Vuoto\\
	                       & Suono                 & \\
	                       & Elettricita'          & \\
\end{tabular}
\bigskip

Ogni Patrono concede delle Essenza privilegiate, neutrali, limitate o negate.

Sulla \textbf{Essenza privilegiata} \index{Essenza privilegiata}l'incantatore prende un +2 alle prove di CM

Sulla \textbf{Essenza normale}\index{Essenza normale} non ha bonus/malus

Sulla \textbf{Essenza limitata}\index{Essenza limitata} ha un -2 alle prove di CM

Le \textbf{Essenze negate} \index{Essenze negate}non sono accessibili/sceglibili, ovvero non può investire punti competenza magica per prendere queste essenze e non potrà mai lanciare magie da pergamena da queste essenze.

Utilizzare i poteri concessi dai Tratti costa 2 Azioni se non specificato diversamente.

In caso di utilizzo della Magia OGL i Patroni concedono solo i poteri dovuti alla comunanza di Tratti.

I Patroni sono:

\subsubsection{Ljust}

\label{ljust}\index{Ljust}

la Dama della Luce, colei che irradia calore e amore. Generatrice delle pulsioni d'amore, protezione, gentilezza, gioia e perdono. Racchiude in sé l'aspetto protettivo di una madre, la forza e l'audacia di una combattente, la passionalità di una giovane amante e l'allegria, la ricerca del nuovo, la fantasia di una bambina. Ljust incarna la bellezza della vita ed ogni creatura che la contempla vede quella che per lei è la massima armonia e cade prona al suo fascino.

Ljust può essere scelta solo da un personaggio con 4 tratti in comune con lei, fondamentalmente si nasce per essere Devoti di Ljust. Nel corso dei decenni Ljust decise di selezionare, scegliere e premiare le donne che più mostravano in modo innato e profondo amore per la vita, curiosità per il nuovo, forza incrollabile, dedizione, fiducia, rispetto e cura degli altri donando loro i poteri e la possibilità di studiare e crescere come Allieve della Luce. Queste Allieve devono seguire la regola degli 8 Passi.

\textbf{Simbolo:} una stella circondata da raggi solari

\textbf{Tratti}: Coraggioso, Generoso, Empatico, Protettivo

\textbf{Manifestazione}: luce dorata inonda l'incantatore.

\bigskip

Tratto in comune a 5 punti: un oggetto che tocchi diventa luminoso come una torcia (3 metri di raggio) per 1 ora. Due volte al giorno

Tratto in comune a 10 punti: guadagni un +2 ai Tiri Salvezza contro Distruzione

Tratto in comune a 15 punti: una armatura di luce di protegge, guadagni un +2 a tutti i Tiri Salvezza

Tratto in comune a 20 punti: puoi emettere una luce che causa 10d6 di danno. Distanza 54 metri, TS Riflessi DC 25 per dimezzare, un target

\bigskip

Elementi: Energia Positiva, Luce

\bigskip

Essenze Privilegiate: Cura, Protezione (+2 alle prove di CM)\\
Essenze Normali: Difesa, Charme, Attacco (+0 alle prove di CM)\\
Essenze Limitate: Creazione, Rivelazione, Alterazione, Movimento (-2 alle prove di CM)
Essenze Negate: Convocazione, Illusione, Distruzione, Trasformazione (Essenze non accessibili)\\

\medskip

Accesso alla Scuole di Magia:\\
Scuole Privilegiate: Universale, Invocazione, Abiurazione (+2 alle prove di CM)\\
Scuole Normali: Ammaliamento, Necromanzia (+0 alle prove di CM)\\
Scuole Limitate: Trasmutazione, Divinazione, Evocazione, Illusione(-2 alle prove di CM)\\
Scuole Negate: - (scuole non accessibili)\\

\paragraph{Gli 8 Passi delle Allieve}\index{8 Passi delle Allieve}\index{Allieve}

\label{gli-8-passi-delle-allieve}

Le Allieve della Luce sono una gruppo segreto di Devote che per totale affinità con Ljust hanno intrapreso il duro percorso del bene e dell'amore. E' tra i gruppi più antichi fondati a Yeru.

Le Allieve, 99 come numero massimo, ma purtroppo spesso meno numerose, sono Devote di Ljust e devono seguire gli 8 Passi della Luce

\begin{enumerate}
	\item Ama e proteggi con tutta te stessa, con totale e sincera dedizione chi hai attorno a te.

	\item Non lasciare che la tua inazione generi sofferenza.

	\item Si un punto di paragone. Fai che la tua Luce elevi le persone che hai intorno e possano vedere in Tu sei speranza, serenità, calma, protezione e sicurezza.

	\item Usa l'intelligenza, la furbizia e l'arguzia. Si lungimirante e risoluta nell'azione.

	\item La tua opera è per il bene comune. Fa che la tua Luce sia sempre alta ed intensa.

	\item Non cercare altra Luce se non la tua e quella delle tue sorelle.

	\item Sii luminosa ma non accecare chi è intorno a te.

	\item Sii la differenza tra la disperazione e la speranza.
\end{enumerate}

Le Allieve hanno costruito un ballo armonioso trasformando in danza i passi della loro Regola.

\subsubsection{Calicante}\index{Calicante}

\label{calicante}

è oscuro, gelido e arrabbiato. Racchiude in sé odio, violenza, distruzione, vendetta e perenne insoddisfazione. Raccoglie la personalità capricciosa e scontenta di un bambino, la noia violenta e sadica di un giovane uomo e la forza distruttiva di un uragano e la rabbia di un combattente che non ha più nulla da perdere. Calicante solo con la presenza mette a disagio, ti fa sentire in pericolo, affascina ma con le armi della paura e della incostanza.

Calicante può essere scelto solo dai personaggi che hanno 4 tratti in comune con lui. I suoi Devoti sono i migliori assassini, sua professione più affine. Coloro che mostrano il maggiore sprezzo del pericolo e della vita altrui. I suoi prediletti sono coloro che sono temuti, odiati\ldots coloro che sono violenti e crudeli ma mortalmente efficienti e decisivi in ogni situazione di combattimento.

\textbf{Simbolo}: un turbine nero

\textbf{Tratti}: Egoista, Vendicativo, Superbo, Iracondo

\textbf{Manifestazione}: spada grondante sangue nero

\bigskip

Tratto in comune a 5 punti: Puoi creare una zona di oscurità. Raggio 1 metro, entro 9 metri, durata 10 minuti. Una volta al giorno

Tratto in comune a 10 punti: La tua lama si ammanta di ombra. Guadagni un +2 al Tiro per Colpire e +1d4 di danno per 2d6 round

Tratto in comune a 15 punti: Crei 4 dardi di energia negativa. Ogni dardo fa 2d6 di danno, colpisce automaticamente entro 18 metri. Una volta al giorno

Tratto in comune a 20 punti: Crei una zona di energia negativa attorno a te nel raggio di 3 metri, dimezzi tutto il danno che ricevi, non puoi curarti nel mentre. Durata 10 minuti consecutivi, 1 volta al giorno.

\bigskip

Elementi: Energia Negativa, Vuoto

\bigskip

Essenze Privilegiate: Distruzione, Attacco\\
Essenze Normali: Illusione, Charme, Trasformazione\\
Essenze Limitate: Creazione, Difesa, Convocazione, Movimento\\
Essenze Negate: Protezione, Alterazione, Cura, Creazione\\

\medskip

Accesso alla Scuole di Magia:\\
Scuole Privilegiate: Universale, Invocazione, Necromanzia (+2 alle prove di CM)\\
Scuole Normali: Ammaliamento, Evocazione (+0 alle prove di CM)\\
Scuole Limitate: Trasmutazione, Divinazione, Abiurazione, Illusione (-2 alle prove di CM)\\
Scuole Negate: - (scuole non accessibili)\\


\subsubsection{Atmos}\index{Atmos}

\label{atmos}

il custode del Tempo e della Torre dell'Orologio, come ha avviato il tempo e la creazione dei nuovi Patroni così fermerà la sfida fra loro, i Patroni sopravvissuti saranno giudicati, le loro opere valutate e Ljust o Calicante ne trarranno giovamento. Come una sfida da una singola moneta di rame nuovi Patroni, nuovi ideali saranno creati e noi, piccole creature vedremo nascere nuove civiltà e regni fiorenti. La storia è poco nota, solo i pochi Devoti di Atmos, scribi e studiosi della biblioteca del Tempo, conoscono il segreto e lo scorrere del tempo e della gara, gli altri, ignoranti, vivranno il loro tempo con un padrone sicuramente guidato da un Patrono.

Atmos, il Patrono del Tempo è il custode della storia e del tempo, è colui che tiene traccia dei mille e più mondi che sono stati creati.

Ha il compito di avviare ed interrompere il tempo. Atmos ha il potere unico e riservato solo a lui di poter bandire dal creato un Patrono qualora questo diventasse troppo forte e minacciasse Calicante e Ljust. Atmos ha già usato questo potere in passato. Atmos sia per la sua natura totalmente neutrale sia per il suo ruolo non si è mai schierato. Tutti i Patroni temono Atmos per il suo potere, il più terribile per loro, ovvero il loro alienamento, l'oblio, la dimenticanza l'essere distolti dal tempo e dalla sfida.

Per essere un Devoto di Atmos al momento del rito è necessario che il futuro Devoto possieda almeno quattro tratti in comune con lui, e amare la storia e la conoscenza.

Vestito di un morbido saio marrone e calzari di cuoio si muove tra gli infiniti scaffali della Biblioteca del Sapere con sempre uno strano misuratore del tempo appeso alla vita.

\textbf{Simbolo:} un libro bianco con un orologio da taschino appoggiato
sopra

\textbf{Tratti}: Osservatore, Distaccato, Studioso, Riflessivo

\textbf{Manifestazione}: l'Essenza si sviluppa come a rallentatore, in realtà è solo un effetto illusorio

\bigskip

Tratto in comune a 5 punti: Conosci sempre la data esatta e l'ora.

Tratto in comune a 10 punti: Hai una intuizione innata per la conoscenza. Hai +1d6 alle prove di Cultura. Arcano prende un bonus +2

Tratto in comune a 15 punti: Puoi creare 8 tue immagini speculari per trarre in inganno i tuoi avversari. Una volta al giorno, durata 1 ora o finché colpite.

Tratto in comune a 20 punti: Ogni qual volta che devi fare una provadi Cultura o Arcano puoi prendere il 16 come prendessi 10

\bigskip

Elementi: Suono, Freddo

\bigskip

Essenze Privilegiate: Rivelazione, Creazione\\
Essenze Normali: Illusione, Cura, Attacco\\
Essenze Limitate: Charme, Difesa, Convocazione, Distruzione\\
Essenze Negate: Protezione, Alterazione, Trasformazione, Movimento\\

\medskip

Accesso alla Scuole di Magia:\\
Scuole Privilegiate: Universale, Invocazione, Divinazione (+2 alle prove di CM)\\
Scuole Normali: Ammaliamento, Evocazione (+0 alle prove di CM)\\
Scuole Limitate: Necromanzia, Illusione, Abiurazione (-2 alle prove di CM)\\
Scuole Negate: Trasmutazione (scuole non accessibili)\\

\subsubsection{Lynx}\index{Lynx}

\label{lynx}

Patrono dei Portali, è sceglibile solo da personaggi che abbiano almeno 3 tratti in comune. E' il primo Patrono generato da Ljust e Calicante, creato per proteggere Yeru.

Serio, sguardo gelido di un azzurro chiarissimo è il Custode dei Portali e di ciò che è Oltre. Letale guardiano per chi cerca di passarli senza permesso, guida attenta per chi chiede il suo aiuto ed il suo permesso. Si fa scudo delle sue cicatrici per allontanare tutti.
Solitario controllore del mondo.

I suoi Devoti sono i viaggiatori per eccellenza, coloro che presidiano e proteggono Yeru da ciò che è alieno, da ciò che potrebbe disturbare la creazione.

\textbf{Simbolo}: un portale

\textbf{Tratti}: Solitario, Serio, Rigido, Controllato

\textbf{Manifestazione}: come se il panorama non avesse più orizzonte

\bigskip

Tratto in comune a 5 punti: Una volta al giorno puoi eseguire una Azione di movimento in piu'

Tratto in comune a 10 punti: Acquisisci una Azione di movimento in più a round

Tratto in comune a 15 punti: Puoi toccare una creatura extraplanare e costringerla a tornare sul suo piano. TS su Arbitrio DC 30. Una volta al giorno

Tratto in comune a 20 punti: Puoi teletrasportarti per 500km al giorno (anche più teletrasporti purché la somma totale non superi 500km)

\bigskip

Elementi: Fuoco, Elettricità

\bigskip

Essenze Privilegiate: Movimento\\
Essenze Normali: Protezione, Convocazione, Rivelazione\\
Essenze Limitate: Cura, Difesa, Attacco, Illusione\\
Essenze Negate: Charme, Alterazione, Trasformazione, Creazione, Distruzione\\

\medskip

Accesso alla Scuole di Magia:\\
Scuole Privilegiate: Universale, Evocazione (+2 alle prove di CM)\\
Scuole Normali: Abiurazione, Trasmutazione, Invocazione, Divinazione (+0 alle prove di CM)\\
Scuole Limitate: Ammaliamento, Necromanzia, ,  (-2 alle prove di CM)\\
Scuole Negate: Illusione (scuole non accessibili)\\


\subsubsection{Gradh}\index{Gradh}

\label{gradh}

Il primo Patrono creato da Atmos sotto la guida di Ljust e l'influenza di Calicante.

Gradh racchiude in sé l'istinto innato alla protezione, alla difesa ed alla cura propri di Ljust. Gradh è quanto di più simile e profondamente legato a Ljust sia stato generato. Lui è equilibrio, razionalità ed empatia.

Dove vi è difesa, cura e protezione e creazione vi è Gradh.

Gradh non ama sfidare apertamente Cattalm perché sa che farebbe esattamente il suo gioco, ecco che con astuzia cerca di attirarlo nel suo terreno di gioco, dove nessuna vita sarà in pericolo e lì da sfoggio a della sua superiorità strategica e di combattimento.

Ma Calicante non poteva permettere la creazione di un Patrono totalmente votato ad Ljust e così infuse in Gradh la freddezza della vendetta e la furia della rabbia. Ecco che allora Gradh nell'atto di difendere l'umanità, spesso la deve in primis proteggere da sé stesso.

Passionale e freddo è forse il Patrono più umano del Pantheon attuale. Il suo sguardo caldo e carismatico che quando ama e protegge è di un rassicurante color cioccolato, può divenire freddo e tagliente con le sfumature della fredda terra ghiacciata quando è preda della furia della battaglia o della vendetta. Gradh ama studiare il mondo attorno a sé e passare inosservato. Spesso si nasconde fra la gente e "vive" la sua vita umana. Ma non si lascia avvicinare veramente da nessuno.

Gradh attira a sé con la stessa facilità con cui allontana da se'.

Il Devoto di Gradh è fiero ed orgoglioso, indomito e protettivo, ed addolorato, perché per quanto si sforzi di punire il male questo continua sempre a prosperare.

\textbf{Simbolo}: uno scudo con incise sopra due spirali intrecciate.

\textbf{Tratti}: Indomito, Protettivo, Vendicativo, Coraggioso

\textbf{Manifestazione}: due spire una nera come ombra ed una lucente come scintilla circondano la sua arma intrecciandosi

\bigskip

Tratto in comune a 5 punti: Il tuo tocco cura 3d6 PF, ma ti causa 1d6 di danno. 2 volte al giorno

Tratto in comune a 10 punti: Per 10 minuti hai un bonus di +4 Ts su Riflessi e Tempra. Una volta al giorno

Tratto in comune a 15 punti: Emani un aura che concede a tutti i tuoi compagni entro raggio 3 metri un +2 TS. Una volta al giorno, per 30 minuti consecutivi

Tratto in comune a 20 punti: Esplodi la tua ira in una palla di energia negativa. 10d6 di danno, raggio 6 metri entro 36 metri. Una volta al giorno

\bigskip

Elementi: Energia Positiva - Energia Negativa

\bigskip

Essenze Privilegiate: Protezione, Attacco\\
Essenze Normali: Cura, Difesa, Creazione\\
Essenze Limitate: Alterazione, Trasformazione, Movimento\\
Essenze Negate: Charme, Illusione, Distruzione,Rivelazione, Convocazione\\


\medskip

Accesso alla Scuole di Magia:\\
Scuole Privilegiate: Universale, Invocazione,  (+2 alle prove di CM)\\
Scuole Normali: Abiurazione, Trasmutazione, Divinazione, Necromanzia (+0 alle prove di CM)\\
Scuole Limitate: Illusione, Evocazione  (-2 alle prove di CM)\\
Scuole Negate: Ammaliamento (scuole non accessibili)\\


\subsubsection{Atherim}\index{Atherim}

\label{atherim}

il Patrono custode. Molti vedono nel seno generoso di Atherim un segno di voluttà e passione. Si lasciano incantare dalla sua procace bellezza e non vedono gli occhi di cristallo che incutono timore a chi osa anche solo pensare di avvicinarla.

Atherim è la custode dei sogni e delle speranze, colei alla quale affidare, come ad una madre, i desideri. E' il Patrono dei Bambini, dei Segreti e delle Levatrici.

Dal sorriso allegro e dall'animo buono sarà sempre pronta ad aiutarti a realizzare i tuoi sogni. E come una madre Atherim protegge e custodisce i segreti e le passioni. Atherim è muta. E' colei che custodisce per sempre, dentro il suo animo i segreti di Yeru.

Il Devoto di Atherim si prende a cuore coloro che hanno fatto una promessa, punisce chi le infrange e chi svela i segreti. Molti Devoti di Atherim sono diplomatici, notai e levatrici.

\textbf{Simbolo:} una mano di donna guantata che tiene un'ampolla ricca di flussi

\textbf{Tratti}: Allegro, Calmo, Industrioso, Buono

\textbf{Manifestazione}: un silenzio sereno e tranquillizzante cala attorno all'incantatore

\bigskip

Tratto in comune a 5 punti: Puoi aggiungere 1d6 ad un Tiro salvezza dopo averlo tirato ma prima di sapere se ha avuto successo o meno. Una volta al giorno, come Reazione.

Tratto in comune a 10 punti: Guadagni 30 pf temporanei. Durata 1 ora, una volta al giorno, come azione immediata.

Tratto in comune a 20 punti: Ogni pozione che bevi fa il doppio di durata o effetto se immediata.

\bigskip

Elementi: Energia Positiva, Elettricità

\bigskip

Essenze Privilegiate: Difesa\\
Essenze Normali: Cura, Movimento, Protezione\\
Essenze Limitate: Alterazione, Attacco, Illusione, Creazione\\
Essenze Negate: Distruzione, Rivelazione, Trasformazione, Convocazione, Charme\\

\medskip

Accesso alla Scuole di Magia:\\
Scuole Privilegiate: Universale, Abiurazione  (+2 alle prove di CM)\\
Scuole Normali: Invocazione, Trasmutazione, Divinazione, Necromanzia (+0 alle prove di CM)\\
Scuole Limitate: Illusione,Ammaliamento (-2 alle prove di CM)\\
Scuole Negate: Evocazione (scuole non accessibili)\\


\subsubsection{Belevon}\index{Belevon}

\label{belevon}

è il Patrono che meglio incarna la bugia e la finzione al fine di un proprio tornaconto. Lui ama solo se stesso. E' un narcisista che si circonda solo di persone che lo assecondano e lo adulano. Aborrisce la solitudine ma allo stesso tempo odia essere toccato da qualcuno.

è sempre alla ricerca di nuove cose, di oggetti meravigliosi che scambia e ricambia con altri oggetti. Gli piace discutere e mercanteggiare, controbattere e portare fino allo stremo la vendita.

Dall'aspetto di un giovane ragazzo incarna perfettamente una pericolosa canaglia.

Il Devoto di Belevon è ben descritto dal mercante ricco e curioso che mai si lascia perdere una occasione di trattare merci nuove. Non è spinto dalla cupidigia o dall'accumulo bensì dall'Arte del commercio e dello scambio.

\textbf{Simbolo}: una gabbia dorata

\textbf{Tratti}: Bugiardo, Narcisista, Casto, Doppiogiochista

\textbf{Manifestazione}: come se le sbarre dorate di una gabbia si intrecciassero attorno all'incantatore

\bigskip

Tratto in comune a 5 punti: Puoi creare un suono immaginario. Durata 10 secondi, entro 9 metri , 3 volte al giorno. Azione Reazione.

Tratto in comune a 10 punti: Acquisisci la capacità di respirare sott'acqua per 10 minuti. Una volta al giorno. Azione Immediata.

Tratto in comune a 15 punti: La creatura che tocchi si placa e diventa indifferente a quello che succedo. TS Arbitrio DC 30. 3 volte al giorno. Costa 2 Azioni.

Tratto in comune a 20 punti: Toccando un oggetto vieni a conoscenza per sommi capi della storia di chi l'ha creato. Una volta al giorno. Costa 3 Azioni.

\bigskip

Elementi: Fuoco, Suono

\bigskip

Essenze Privilegiate: Illusione\\
Essenze Normali: Charme, Alterazione, Rivelazione\\
Essenze Limitate: Movimento, Protezione, Distruzione, Attacco\\
Essenze Negate: Creazione, Trasformazione, Cura, Convocazione, Difesa\\

\medskip

Accesso alla Scuole di Magia:\\
Scuole Privilegiate: Universale, Illusione (+2 alle prove di CM)\\
Scuole Normali: Invocazione, Ammaliamento, Trasmutazione, Divinazione (+0 alle prove di CM)\\
Scuole Limitate: Evocazione, Necromanzia (-2 alle prove di CM)\\
Scuole Negate: Abiurazione (scuole non accessibili)\\

\subsubsection{Cattalm}\index{Cattalm}

\label{cattalm}

Generato direttamente da Calicante, come risposta alla creazione di Gradh da parte di Ljust, è pura distruzione, chaos ed entropia. Cattalm si prefigura il solo scopo di distruggere, portare chaos e malattie, terremoti ed alluvioni.

Cattalm è tra i pochi Patroni che osa sfidare apertamente Gradh e lo fa con gioia perché sa che la loro battaglia altro non farà che portare ulteriore distruzione. Cattalm accetta ed invita ad essere suo Devoto ogni creatura capace di odio, capace di distruggere e ferire. Molti suoi Devoti sono creature mostruose o aberrazioni.

Cattalm invece è tra i Patroni più meravigliosi, con una candida pelle lucente, ali di piuma argentee ed una leggera armatura argentata. Per quanto i lineamenti delicati ne facciano un essere bellissimo per quanto ambisca alla distruzione.

Cattalm adora il chaos che manifesta nei modi più violenti con terremoti, alluvioni, maremoti, malattie se non direttamente piogge infuocate. Non agisce quasi mai uccidendo le persone piuttosto infettandole e spargendo piaghe, carestie e piaghe per ottenere il massimo risultato.

Ljust non poteva non intervenire nella creazione di un Patrono così esplicitamente malvagio e, di nascosto da Calicante, instillò in Cattalm l'amore e protezione per i bambini. Cattalm distrugge, avvelena, indebolisce ma non i bambini, neanche indirettamente, piuttosto si attiva lui stesso per annullare i malefici causati dalla sua natura.
E' già capitato che interi villaggi venissero inondati e fossero trovati sui tetti in legno a modo di chiatte tutti i piccoli.

Ogni qual volta succede una calamità si suole dire che "Cattalm ha battuto il piede"

\textbf{Simbolo}: un'onda gigante che sovrasta la costa

\textbf{Tratti}: Distruttivo, Anarchico, Meticoloso, Sadico

\textbf{Manifestazione}: il rumore del tuono

\bigskip

Tratto in comune a 5 punti: Attraverso le tue armi indebolisci l'avversario designato. -2 Potenza per 1 minuto dopo un attacco andato a segno. Una volta al giorno.

Tratto in comune a 10 punti: Il tuo tocco imputridisce cibo (fino a 50kg) e acqua (50m/r). Una volta al giorno

Tratto in comune a 15 punti: Il tuo sguardo riempie di collera. TS Arbitrio DC 30 o il target attacca un soggetto a caso. Due volta al giorno

Tratto in comune a 20 punti: Generi un cono d'ombra che danneggia i tuoi avversari. Influenzi un cono che al termine è largo 6 metri e lungo 27 metri, 10d6 di danno. Una volta al giorno

\bigskip

Elementi: Energia Negativa - Vuoto

\bigskip

Essenze Privilegiate: Distruzione, Trasformazione\\
Essenze Normali:, Charme, Alterazione, Attacco\\
Essenze Limitate: Illusione, Convocazione, Movimento\\
Essenze Negate: Cura, Creazione, Protezione, Difesa, Rivelazione\\

\medskip

Accesso alla Scuole di Magia:\\
Scuole Privilegiate: Universale, Invocazione, Necromanzia (+2 alle prove di CM)\\
Scuole Normali: Ammaliamento, Trasmutazione, (+0 alle prove di CM)\\
Scuole Limitate: Evocazione,  Illusione, Divinazione (-2 alle prove di CM)\\
Scuole Negate: Abiurazione (scuole non accessibili)\\

\subsubsection{Efrem}\index{Efrem}

\label{efrem}

è il Patrono di chi fa della natura la propria casa. Incarna in sé gli aspetti più puri della natura stessa, aggressivo come solo i felini più letali sanno essere; ma anche selvaggio come le radure più nascoste e rigorosa come solo la natura può essere.

Efrem si prefigge di difendere la Natura dalla contaminazione dell'uomo, da questa specie infestante che distrugge tutto ciò che incontra.

I Devoti di Efrem sono legati maggiormente all'elemento naturale. Manipolano le essenze principalmente elementali e si difendono o attaccano usando anche animali e creature naturali. In rari casi costringendo anche i Draghi alla ubbidienza.

I Devoti di Efrem hanno l'obiettivo supremo di proteggere gli animali e le piante, i luoghi e tutto ciò che è naturale e non artificiale. Solitamente solitario e scontroso non riesce a capire il perché dell'odio che, dal suo punto di vista, l'uomo scarica su Yeru.

Un Devoto di Efrem rispetta la vita come la morte, nel processo naturale che è l'evoluzione ed il ciclo vitale. A volte decide di stabilirsi in un certo ambiente e lo elegge come suo territorio e come la sua casa lo protegge. Altre volte decide di essere ramingo ed intervenire in tutta Yeru per proteggere i suoi amati fiori e cuccioli.

\textbf{Simbolo}: una staffa con un rampicante attorcigliato attorno

\textbf{Tratti}: Indifferente, Leale, Fiducioso, Pratico

\textbf{Manifestazione}: spire di foglie avvolgono la spada

\bigskip

Tratto in comune a 5 punti: Il tuo tocco rende docili gli animali non magici. TS Arbitrio 20 per resistere. 3 volte al giorno. Costo 2 Azioni.

Tratto in comune a 10 punti: Guadagni un +4 a tutte le prove di Sopravvivenza che si effettuano in un all'ambiente naturale. Costo 2 Azioni.

Tratto in comune a 15 punti: Puoi benedire delle bacche affinché queste siano nutrienti e curative. Puoi incantare 1d6 bacche al giorno. Ogni bacca, max 1 al giorno, cura 1d6 PF e rimuove le malattie o veleni. Costa 3 Azioni.

Tratto in comune a 20 punti: Il tuo tocco è quello del padrone. Puoi ammansire creature anche magiche che tocchi. TS Arbitrio DC 30. Una volta al giorno. Costo 2 Azioni

\bigskip

Elementi: Elettricità, Suono

\bigskip

Essenze Privilegiate: Creazione\\
Essenze Normali: Convocazione, Trasformazione, Alterazione\\
Essenze Limitate: Cura, Distruzione, Attacco\\
Essenze Negate: Protezione, Difesa, Illusione, Charme, Rivelazione, Movimento\\

\medskip

Accesso alla Scuole di Magia:\\
Scuole Privilegiate: Universale, Evocazione (+2 alle prove di CM)\\
Scuole Normali: Invocazione, Trasmutazione, Necromanzia, Ammaliamento (+0 alle prove di CM)\\
Scuole Limitate: Abiurazione, Divinazione (-2 alle prove di CM)\\
Scuole Negate: Illusione (scuole non accessibili)\\

\subsubsection{Erondil}\index{Erondil}

\label{erondil}

Patrono di Terra e Aria, Erondil è il Signore degli elementi più concreti e razionali. Colui che dotato di infinito potere e razionalità dona ai suoi Devoti il potere della manipolazione della terra. Il dono di creare con semplice "fango" costruzioni gigantesche e di millenaria forza. Conclude le sue opere con attenzione e precisione.
Pur con fatica perché se il risultato finale non lo soddisfa scatena i suoi fulmini per distruggerlo all'istante. Perfezionista ed incontentabile, difficilmente qualcosa è esattamente come lui se la immaginava.

Ordinato ed esuberante è il signore delle tempeste, dei tuoni e dei fulmini,dei terremoti e distruzioni. Ama circondarsi del fragore del tuono, del rombo della terra che si sgretola. Sa essere distruttivo verso coloro che non rispettano Yeru.
Ha braccia e petto ricoperti da tatuaggi quasi argentei che narrano le leggende di Terra e Aria. Erondil, il signore di tuoni e terremoti.


I Devoti di Erondil sono gli ingegneri dell'impossibile, ogni qual volta si deve sfidare la materia, la gravità e la stessa ragione un Devoto di Erondil troverà pane per i suoi denti, troverà la sfida adatta ad un Costruttore dell'Impossibile.

\textbf{Simbolo:} un castello di sabbia con un fulmine sopra

\textbf{Tratti}: Perfezionista, Incontentabile, Sognatore, Esuberante

\textbf{Manifestazione}: suono di tempesta e rombo di frana

\bigskip

Tratto in u'omune a 5 punti: Non temi più le cadute. Ogni volta che cadi da più di 1 metro un soffio d'aria ti sostiene facendoti atterrare dolcemente.

Tratto in comune a 10 punti: Il tuo tocco plasma la pietra, Puoi aprire un passaggio (3m/h{*}1m/l{*}3m/p) in una parete di pietra. Una volta al giorno. Costo 2 Azioni.

Tratto in comune a 15 punti: Puoi scagliare un fulmine dalle tue mani. 10D6 di danno, fino a 3 target. TS Riflessi DC 30 per dimezzare. Costo 2 Azioni.

Tratto in comune a 20 punti: Sei in grado di creare una fossa profondissima (1km e oltre) sotto il tuo avversario (taglia fino a grande). TS Riflessi 35 o cadere. Una volta al giorno. Dopo 1 minuto la fossa si chiude con chi c'è dentro. Costo 2 Azioni.

\bigskip

Elementi: Fuoco, Elettricità

\bigskip

Essenze Privilegiate: Creazione\\
Essenze Normali: Trasformazione, Distruzione\\
Essenze Limitate: Difesa, Attacco, Protezione\\
Essenze Negate: Convocazione, Movimento, Charme, Protezione, Alterazione,Illusione, Cura, Rivelazione\\

\medskip

Accesso alla Scuole di Magia:\\
Scuole Privilegiate: Universale, Trasmutazione (+2 alle prove di CM)\\
Scuole Normali: Invocazione, Evocazione, Abiurazione, Ammaliamento (+0 alle prove di CM)\\
Scuole Limitate: Illusione, Divinazione (-2 alle prove di CM)\\
Scuole Negate: Necromanzia (scuole non accessibili)\\


\subsubsection{Gaya}\index{Gaya}

\label{gaya}

Patrono di Acqua e Fuoco, nelle profondità della terra, della nebbia più fitta, Gaya si diverte a dipingere. Adora circondarsi dei flussi di fuoco e acqua quasi a creare una danza in mezzo a loro. Adora i suoni della natura, l'infrangersi delle onde sugli scogli, il cadere delle gocce di pioggia sull'acciottolato, il borbottare di un fuoco scoppiettante.

Dipinge mescolando il caldo ed il freddo. L'acqua cristallina ed impetuosa al fuoco intrigante ed ardente. Gelosa del bello e delle arti tiene tutte le sue opere al sicuro in un ordine quasi maniacale e protette. Innovativa ma essenziale, utilizza elementi semplici per far risplendere le meraviglie della natura. Gaya è la pittrice di tramonti e tempeste.

I Devoti di Gaya sono artisti volubili e sopra le righe. Sono coloro che ricreano la magia dell'alba o del tramonto o del mare in tempesta nelle loro opere. sono coloro che mettono poesia e follia nella normalità.

\textbf{Simbolo:} un pennello sul cielo

\textbf{Tratti}: Innovativo, Ordinato, Istintivo, Prudente

\textbf{Manifestazione}: spire di fuoco e acqua avvolgono all'incantatore

\bigskip

Tratto in comune a 5 punti: Puoi creare fino a 5 litri di acqua o 1 litro di liquore di buona qualità. Una volta al giorno. Costo 2 Azioni.

Tratto in comune a 10 punti: Il tuo metabolismo non teme il freddo. Resisti al Danno magico da freddo e sei immune a quello naturale.

Tratto in comune a 15 punti: Puoi respirare sott'acqua come respiri l'aria.

Tratto in comune a 20 punti: Generi una pioggia di fuoco. In 6 metri di circonferenza, 10d6 di danno, intorno a te. Resisti il danno da fuoco, anche magico, per 5pf/round. Costo 2 Azioni.

\bigskip

Elementi: Elettricità, Fuoco

\bigskip

Essenze Privilegiate: Creazione\\
Essenze Normali: Trasformazione, Distruzione\\
Essenze Limitate: Difesa, Attacco, Protezione\\
Essenze Negate: Convocazione, Movimento, Charme, Protezione, Alterazione, Illusione, Cura, Rivelazione\\

\medskip

Accesso alla Scuole di Magia:\\
Scuole Privilegiate: Universale, Trasmutazione (+2 alle prove di CM)\\
Scuole Normali: Invocazione, Evocazione, Abiurazione, Ammaliamento (+0 alle prove di CM)\\
Scuole Limitate: Illusione, Divinazione (-2 alle prove di CM)\\
Scuole Negate: Necromanzia (scuole non accessibili)\\


\bigskip

\textbf{Gaia} ed \textbf{Erondil} sono come le due facce della stessa medaglia e sovraintendono agli elementi, Gaia acqua e fuoco e Erondil Aria e Terra; agiscono come espressione diretta dei Patroni maggiori, sono piccole manifestazione del loro immane potere.

\subsubsection{Krondal}\index{Krondal}

\label{krondal}

è un Patrono potente ma schivo e riservato. Si tiene in disparte, fuori dai giochi finché non percepisce la privazione della libertà.

Non può vedere nel futuro, non può conoscere le persone ma il suo formidabile istinto lo fa diventare il combattente più temibile che si può incontrare. Coraggioso fin quasi all'avventato, agisce in battaglia senza paura. Dallo spirito buono, Krondal entra in campo nei momenti più importanti, quando non è una situazione a decidersi ma il futuro della vita, della propria libertà personale.

Krondal nutre un profondo rispetto per la libertà ed è profondamente contrario ad ogni schiavismo, razzismo o dittatura.

Un Devoto di Krondal è tipicamente una guardia del corpo, un protettore, lo sceriffo che sa e deve decidere per il bene del suo paese, costi quello che costi.

Un Devoto di Krondal non giudica le persone o i fatti bensi si attiene alla sua etica di protezione e libertà.

Sotto vestiti dimessi e lisi, ma sempre puliti nasconde un fisico da combattente.

\textbf{Simbolo}: una spada tenuta verticalmente davanti a se

\textbf{Tratti}: Avventato, Pio, Corretto, Libero

\textbf{Manifestazione}: il mantello o veste del Devoto diventa sporca di terra e sangue

\bigskip

Tratto in comune a 5 punti: Maledici il tuo avversario , dandogli un -2 TC e Difesa, per 1 minuto. TS Arbitrio DC 20 per resistere. Costo 2 Azione.

Tratto in comune a 10 punti: Non puoi essere legato o ammanettato. Ad un tuo gesto i nodi si sciolgono e le manette si aprono. Costo 1 Azione Immediata

Tratto in comune a 15 punti: La tua presenza porta speranza. Ogni compagno entro 3 metri da te guadagna 1d6 da usare entro 3 round. Una volta al giorno. Costo 1 Azione.

Tratto in comune a 20 punti: La tua arma è più efficace contro i nemici. Ogni creatura colpita deve fare un Tiro Salvezza Arbitrio DC 25 o rimanere paralizzata per 3 round. Una volta che la creatura riesce nel TS non può essere più influenzata nelle successive 24 ore.

\bigskip

Elementi: Energia Positiva, Fuoco

\bigskip

Essenze Privilegiate: Distruzione\\
Essenze Normali: Cura, Difesa, Attacco\\
Essenze Limitate: Convocazione, Trasformazione, Charme, Movimento\\
Essenze Negate: Illusione, Rivelazione, Protezione, Creazione, Alterazione\\

\medskip

Accesso alla Scuole di Magia:\\
Scuole Privilegiate: Universale, Invocazione (+2 alle prove di CM)\\
Scuole Normali: Trasmutazione, Evocazione, Abiurazione, Necromanzia (+0 alle prove di CM)\\
Scuole Limitate: Ammaliamento, Illusione (-2 alle prove di CM)\\
Scuole Negate: Divinazione (scuole non accessibili)\\

\subsubsection{Ledyal}\index{Ledyal}

\label{ledyal}

è il Patrono senza un volto preciso, senza una voce se non un canto. Mutevole di corpo e senza una definizione chiara del suo essere. Si manifesta con un lungo mantello color rosso fuoco dal tessuto fatto da mille farfalle. Il suo tocco è vita e pace, protegge chi necessita dei suoi favori indipendentemente dal fatto che li chieda o meno. Desidera un mondo senza sofferenza, con solo felicità ed armonia. Sospettoso/a e profondamente introverso non crede a coloro che gli danno ragione. Ha il cuore pieno di vita e di bontà ma non puo'/non sa amare.

Ledyal ha anche una sorella gemella, o forse un'altra personalità. O forse sono lo stesso Patrono, nessuno le ha mai viste insieme. La "gemella" \textbf{Laydel} non tollera la sofferenza, disprezza chi causa dolore, uccide senza timore qualunque creatura abbia peccato contro un innocente, chiunque abbia causato sofferenza.

I Devoti di Ledyal al momento del Rito possono prendere Cura come Essenza privilegiata oppure Attacco.

\textbf{Simbolo}: una farfalla viola/rosso sangue che vola

\textbf{Tratti}: Caritatevole, Sospettoso. Introverso/Integerrimo, Clemente/Implacabile

\textbf{Manifestazione}: come se un mantello di farfalle avvolgesse il Devoto

\bigskip

Tratto in comune a 5 punti: Il tuo tocco è vita/attacco. 3 volte al giorno puoi toccare una creatura vivente e curarla/causa di 1d6 PF. Costo 2 Azioni.

Tratto in comune a 10 punti: Il tuo tocco è pace. La creatura toccata deve riuscire in un Tiro Salvezza Arbitrio DC 25 o cadere addormentata. Non puoi attaccare/danneggiare questa creatura. Una volta al giorno. Costo 2 Azioni (comprende anche l'Azione di tocco)

Tratto in comune a 15 punti: La tua aura protegge i tuoi compagni. Entro raggio 3 metri i tuoi compagni hanno un +4 alla Difesa ed un +2 ai Tiri Salvezza. Durata 10 minuti consecutivi, una volta al giorno. Costo 2 Azioni.

Tratto in comune a 20 punti: Irradi una sfera curativa/attacco intorno a te. Ogni creatura nel raggio di 6 metri viene curata di 60PF. Una volta al giorno. In caso di Laydel l'effetto è opposto. Costo 2 Azioni.

\bigskip

Elementi: Energia Positiva, Elettricità

\bigskip

Essenze Privilegiate: Cura/Attacco\\
Essenze Normali: Alterazione, Protezione, Difesa\\
Essenze Limitate: Movimento, Convocazione, Illusione, Rivelazione\\
Essenze Negate: Distruzione, Attacco/Cura, Creazione, Trasformazione,Charm\\

\medskip

Accesso alla Scuole di Magia:\\
Scuole Privilegiate: Universale, Invocazione/Abiurazione (+2 alle prove di CM)\\
Scuole Normali: Trasmutazione, Evocazione, Illusione, Necromanzia (+0 alle prove di CM)\\
Scuole Limitate: Ammaliamento, Divinazione  (-2 alle prove di CM)\\
Scuole Negate: Invocazione/Abiurazione (scuole non accessibili)\\


\subsubsection{Nethergal}\index{Nethergal}

\label{nethergal}

Patrono Messaggero. Sulla piuma di un'oca vola la lettera di Nethergal. Rapida, impetuosa, diretta, Nethergal è la messaggera, colei alla quale affidare pensieri e scritti. Sarcastica e logorroica curioserà sui tuoi scopi, ti chiederà informazioni sugli scritti affidatole con esplicita franchezza ed avrà sempre qualcosa da ridire sul messaggio da portare ma sarà anche altrettanto diretta e precisa nel consegnarlo.

Nethergal non è solo chiacchiere e pettegolezzi, qualsiasi testo venga scritto lei lo conosce, non esiste codice o segreto scritto che lei non conosca.

Il Devoto di Nethergal è un fine linguista, un esperto di indovinelli e rebus, un Devoto che a differenza di Atmos non si limita a custodire gli scritti ma ne diffonde la conoscenza.

Un Devoto di Nethergal è un maestro, un professore di lingue di un Collegio, un dotto esperto di mille argomenti.

\textbf{Simbolo:} una piuma bianca cangiante

\textbf{Tratti}: Sarcastico, Impetuoso, Immaturo, Logorroico

\textbf{Manifestazione}: cascata di piume, un oca in volo

\bigskip

Tratto in comune a 5 punti: Puoi inviare un messaggio di massimo 144 caratteri ad un soggetto entro 50 metri senza essere udito/visto. Una volta all'ora. Costo 1 Azione.

Tratto in comune a 10 punti: Mettendo la mano su un tomo ne apprendi il contenuto come se lo avessi letto. Un tomo a settimana. Perdi le conoscenze così acquisite dopo una settimana. Costo 3 Azioni.

Tratto in comune a 15 punti: Puoi volare, 1 ora al giorno. manovrabilità buona. Costo 1 Reazione.

Tratto in comune a 20 punti: Puoi costringere una creatura a rivelarti le informazioni che ha. TS su Arbitrio DC 30 per resistere. Una volta al giorno. Costo 2 Azioni.

\bigskip

Elementi: Elettricità, Suono

\bigskip

Essenze Privilegiate: Movimento\\
Essenze Normali: Illusione, Trasformazione, Charme\\
Essenze Limitate: Alterazione, Distruzione, Rivelazione, Attacco\\
Essenze Negate: Protezione, Convocazione, Cura, Creazione, Difesa\\

\medskip

Accesso alla Scuole di Magia:\\
Scuole Privilegiate: Universale, Trasmutazione (+2 alle prove di CM)\\
Scuole Normali: Invocazione, Ammaliamento, Illusione, Divinazione  (+0 alle prove di CM)\\
Scuole Limitate: Necromanzia, Evocazione (-2 alle prove di CM)\\
Scuole Negate: Abiurazione (scuole non accessibili)\\

\subsubsection{Nedraf}\index{Nedraf}

\label{nedraf}

Il Patrono Sopravvissuto, il vecchio lupo ormai stanco che ha attraversato e combattuto mille battaglie. La sua carne è ferita, il suo corpo ricoperto di cicatrici di guerra e lividi ma nulla lo farà crollare. Tenacia, passione, esperienza e tanta rabbia rendono Nedraf non solo un combattente eccellente in qualsiasi occasione ma un conoscitore dell'ambiente attorno a se'. Grazie al suo impeccabile allenamento sa sfruttare al meglio le risorse a disposizione. Sa spronare con passione gli uomini a suoi ordini.
Nedraf rappresenta colui che vorresti sempre accanto in ogni battaglia.

Molti capitani di ventura e ufficiali al comando sono Devoti di Nedraf. Il Devoto di Nedraf non si arrende, non rinuncia, non abbandona i compagni ma non per questo è avventato o irrazionale nelle scelte.

\textbf{Simbolo:} una mano forte, avvolta in una benda sporca di sangue
che brandisce una spada

\textbf{Tratti}: Disciplinato, Combattivo, Tenace, Aggressivo

\textbf{Manifestazione}: si spande nell'aria odore di sangue e metallo

\bigskip

Tratto in comune a 5 punti: Puoi portare armature leggere senza penalità alla CM

Tratto in comune a 10 punti: Acquisisci un punto bonus su una Lista armi. Può essere nota o meno

Tratto in comune a 15 punti: Puoi portare armature medie senza penalità alla CM ed Agilità

Tratto in comune a 20 punti: Puoi portare armature pesanti senza penalità alla CM ed Agilità

\bigskip

Elementi: Energia positiva, Suono

\bigskip

Essenze Privilegiate: Attacco\\
Essenze Normali: Convocazione, Alterazione, Protezione\\
Essenze Limitate: Difesa, Movimento, Creazione, Distruzione\\
Essenze Negate: Distruzione, Cura, Illusione, Charme, Trasformazione\\

\medskip

Accesso alla Scuole di Magia:\\
Scuole Privilegiate: Universale, Invocazione, (+2 alle prove di CM)\\
Scuole Normali: , Abiurazione, Trasmutazione, Evocazione, Divinazione  (+0 alle prove di CM)\\
Scuole Limitate: Necromanzia, Illusione (-2 alle prove di CM)\\
Scuole Negate: Ammaliamento (scuole non accessibili)\\


\subsubsection{Nihar}\index{Nihar}

\label{nihar}

è il Patrono degli eroi per caso. Ponderato e tranquillo è anche amante del buon vino e del gozzovigliare. E' colui che non sceglieresti mai come compagno d'armi a causa del suo aspetto "comune" e del suo atteggiamento goliardico. Ma poi al momento di esserci, di combattere, di far la differenza ecco che strabilia tutti e "risolve" la partita.

Ha le sembianze di un piccolo uomo, dai vestiti sfarzosi e ricercati e dall'espressione guardinga ed allegra. Si protegge sempre e a qualunque costo, mostrando al mondo esattamente ciò che l mondo vuole vedere. Controlla attentamente la realtà attorno a sé e anche se è sempre più facile vederlo con un calice in mano, se non ci si lascia ingannare dalle apparenze si noterà come i suoi occhi non perdano mai di vista il pericolo, il problema. Sta attento, non si fida di nulla e di nessuno. Ha fatto dei suoi difetti i suoi punti di forza.

\textbf{Simbolo}: una daga appoggiata vicino ad un calice di vino

\textbf{Tratti}: Altruista, Determinato, Cortese, Attento

\textbf{Manifestazione}: il suono di un brindisi

\bigskip

Tratto in comune a 5 punti: Puoi trasformare l'acqua in vino. Un litro al giorno. Costo 2 Azioni.

Tratto in comune a 10 punti: Costo una Azione immediata, ottieni un bonus di +2d6 ad una azione in quel round.

Tratto in comune a 15 punti: Il tuo pugnale causa 1d4 di danno aggiuntivo.

Tratto in comune a 20 punti: I manicaretti che prepari sono buonissimi. Chiunque si sazi con una pietanza da te preparata recupera 2d6 PF e viene curato dai veleni. Max 6 persone al giorno.

\bigskip

Elementi: Energia Positiva, Fuoco

\bigskip

Essenze Privilegiate: Trasformazione\\
Essenze Normali: Creazione, Protezione, Difesa\\
Essenze Limitate: Movimento, Alterazione, Cura, Convocazione\\
Essenze Negate: Illusione, Rivelazione, Charme, Distruzione, Attacco\\

\medskip

Accesso alla Scuole di Magia:\\
Scuole Privilegiate: Universale, Trasmutazione (+2 alle prove di CM)\\
Scuole Normali: Abiurazione, Evocazione, Invocazione, Divinazione  (+0 alle prove di CM)\\
Scuole Limitate: Illusione, Ammaliamento (-2 alle prove di CM)\\
Scuole Negate: Necromanzia (scuole non accessibili)\\


\subsubsection{Orudjs}\index{Orudjs}

\label{orudjs}

ovvero il Patrono della illusione e della finzione. Colui che solo con il dono della parola, il gesticolare delle mani, la voce carismatica e lo sguardo intrigante riesce a vendere ogni sua parola come verità assoluta. Adora il teatro per ciò che per lui e', la rappresentazione della falsità umana, l'essere tante persone ed in realtà nessuna. Adora la politica ed i suoi intrighi. Finge di ascoltare chi gli sta vicino ma in realtà non è interessato alle storie altrui perché le sue sono sempre le migliori.

è un codardo senza limiti e le poche verità che dice, e sono veramente rare, sono da lui dette solo per salvarsi.

Dall'aspetto piuttosto ordinario e quasi scontato appena apre bocca ed inizia i suoi racconti riesce a calamitare l'attenzione dell'intera sala. Possiede infatti una voce calda e suadente che accompagnata alla buonissima dialettica da lui posseduta, incanta ogni orecchio in ascolto.

I suoi Devoti sono abili attori ed intrattenitori, spie sotto copertura, diplomatici o politicanti.

\textbf{Simbolo}: una maschera teatrale con solo la bocca aperta e
gli occhi

\textbf{Tratti}: Ironico, Codardo, Saccente, Socievole

\textbf{Manifestazione}: il suono di una risata profonda e contagiosa

\bigskip

Tratto in comune a 5 punti: Il tuo eloquio è già leggendario. +2 alle prove di Intrattenere.

Tratto in comune a 10 punti: Sei in grado di creare fino a 4 suoni/rumori distanti 6 metri l'uno dall'altro. Durata 1 minuto. Tre volte al giorno. Costo 1 Azione Immediata

Tratto in comune a 15 punti: Il tuo eloquio è già leggendario. +4 aggiuntivo alle prove di Intrattenere.

Tratto in comune a 20 punti: La tua voce è suadente. Chiunque ti ascolti per più di un minuto deve fare un Tiro Salvezza Arbitrio DC 30 oppure considerarsi un tuo amico. Una volta al giorno

\bigskip

Elementi: Elettricità, Fuoco

\bigskip

Essenze Privilegiate: Charme\\
Essenze Normali: Illusione, Rivelazione, Alterazione\\
Essenze Limitate: Distruzione, Difesa, Cura, Movimento\\
Essenze Negate: Convocazione, Trasformazione, Protezione, Creazione, Attacco\\


\medskip

Accesso alla Scuole di Magia:\\
Scuole Privilegiate: Universale, Ammaliamento (+2 alle prove di CM)\\
Scuole Normali: Illusione, Abiurazione, Necromanzia, Divinazione  (+0 alle prove di CM)\\
Scuole Limitate: Invocazione, Trasmutazione(-2 alle prove di CM)\\
Scuole Negate: Evocazione (scuole non accessibili)\\


\subsubsection{Orlaith}\index{Orlaith}

\label{orlaith}

ovvero il Patrono della giustizia e della Vendetta. Lui segue pedissequamente le leggi e pretende che i suoi sottoposti eseguano senza alcuna discussione gli ordini impartiti. E' mosso da uno spirito gentile e buono che però tiene ben nascosto dietro le sue azioni dirette ed incisive, spudorate, vendicative e mortali. Orlaith è vendetta che si fa legge. Agisce per senso di giustizia con i suoi metodi. Di lui attirano il portamento e lo sguardo fiero.

I Devoti di Orlaith spesso sono giudici e giustizieri, persone che hanno deciso di portare la giustizia ovunque, perché Orlaith non può stare fermo, c'è sempre qualcuno da giudicare e punire.

\textbf{Simbolo}: la bilancia

\textbf{Tratti}: Imparziale, Giusto, Gentile, Valoroso

\textbf{Manifestazione}: l'immagine di una stadera, sbilanciata.

\bigskip

Tratto in comune a 5 punti: Richiami a te 1 mastino (normale) che obbedisce ai tuoi comandi. Durata 1 minuto. Una volta al giorno. Costo 2 Azioni.

Tratto in comune a 10 punti: Un paio di manette si manifesta attorno ai polsi della creatura (massimo taglia grande). TS Riflessi DC 25 per annullare. Costo 2 Azioni.

Tratto in comune a 15 punti: La creatura toccata deve dire la verità alle tue domande. Durata 10 minuti. Una volta al giorno. Costo 2 Azioni.

Tratto in comune a 20 punti: Crei raggio di luce lungo 27 metri e largo pochi centimetri. Ogni creatura attraversata subisce 5d6 di danno. Una volta al giorno. Costo 2 Azioni.

\bigskip

Elementi: Luce, Suono

\bigskip

Essenze Privilegiate: Convocazione\\
Essenze Normali: Alterazione, Attacco, Cura\\
Essenze Limitate: Distruzione, Trasformazione, Protezione, Difesa\\
Essenze Negate: Creazione, Rivelazione, Movimento, Illusione, Charme\\

\medskip

Accesso alla Scuole di Magia:\\
Scuole Privilegiate: Universale, Evocazione (+2 alle prove di CM)\\
Scuole Normali: Trasmutazione, Abiurazione, Necromanzia, Divinazione  (+0 alle prove di CM)\\
Scuole Limitate: Illusione, Invocazione (-2 alle prove di CM)\\
Scuole Negate: Ammaliamento (scuole non accessibili)\\

\subsubsection{Rezh}\index{Rezh}

\label{rezh}

il Patrono che disprezza tutto. Rezh ama, vuole, tocca, rimira solo le sue monete lucide e brillanti. Non sono mai abbastanza, nessuna ricchezza è mai abbastanza per lei. Rezh, l'avara tiene tutto per se', non conosce compassione, non conosce carità, non conosce condivisione. La sua fame di denaro, di ricchezze la rende prona a qualsiasi bassezza. Disprezza tutto e tutti e giudica tutto e tutti seguendo solo il suo personale metro di giudizio. In ogni moneta c'è un pò di Rezh. Nella ossidatura di ogni moneta si può vedere l'impronta di Rezh.

Se il denaro compra la libertà Rezh deve accumularne ancora e ancora se mai sarà abbastanza.

I Devoti di Rezh solitamente sono scelti da lei tra le fila dei più avidi e ricchi. Il loro scopo è di accumulare ricchezze, sempre di più.

Spesso i Devoti di Rezh diventano esploratori, tombaroli, persone sempre alla ricerca di un tesoro e di una moneta in più.

\textbf{Simbolo:} una pila di monete con un ratto vicino

\textbf{Tratti}: Avaro, Arrogante, Cattivo, Freddo

\textbf{Manifestazione}: un rumore di monete che cadono avvolge l'incantatore

\bigskip

Tratto in comune a 5 punti: Sei un esperto di monete e gemme, nessun falsario può ingannarti. +4 alle prove di Consapevolezza e Cultura relative.

Tratto in comune a 10 punti: Puoi incantare una gemma (valore minimo 10mo) e usarla per proiettare una illusione fino a 20m{*}10m{*}10m. La gemma viene poi distrutta. Durata illusione 1 ora ogni 10 mo di valore della gemma. Costo 2 Azioni.

Tratto in comune a 15 punti: Puoi tirare fuori dalle tasche 1 moneta d'oro ogni volta che vuoi. Max 10 mo al giorno. Costo 1 Azione.

Tratto in comune a 20 punti: La tua armatura viene coperta da monete d'oro. Guadagni +4 alla Difesa e +4 TS Tempra per 1 ora. Costo 1 Reazione

\bigskip

Elementi: Vuoto, Elettricità

\bigskip

Essenze Privilegiate: Protezione\\
Essenze Normali: Charme, Illusione, Convocazione\\
Essenze Limitate: Movimento, Alterazione, Trasformazione, Difesa\\
Essenze Negate: Distruzione, Cura, Attacco, Creazione, Rivelazione\\

\medskip

Accesso alla Scuole di Magia:\\
Scuole Privilegiate: Universale, Abiurazione (+2 alle prove di CM)\\
Scuole Normali: Illusione. Evocazione, Ammaliamento, Necromanzia (+0 alle prove di CM)\\
Scuole Limitate: Trasmutazione, Invocazione (-2 alle prove di CM)\\
Scuole Negate: Divinazione  (scuole non accessibili)\\


\subsubsection{Sumkjr}\index{Sumkjr}

\label{sumkjr}

Patrono dell'Arcano di Luce. Sumkjr è bontà, correttezza, lealtà, giustizia, protezione.

Sumkjr è il cavaliere che protegge gli innocenti, è la spada "di Ljust" nella battaglia finale. Difende i deboli e lenisce le ferite.

Sumkjr porta la Luce di Ljust ovunque, nessun pericolo potrà mai fermare Sumkjr dalla sua continua, infinita, cerca del bene.

Un Devoto di Sumkjr agisce lealmente e con onore, sempre perseguendo il bene ultimo, il suo essere non può essere piegato al male, all'ingiustizia, al disonore.

Con coraggio e determinazione il Devoto affronta ogni sfida ma non solo per senso del dovere, ma perché profondamente votato al suo destino. Sumkjr sa che poche persone reggono tale standard perché a differenza dei Devoti della Patrona delle Genesi i suoi Devoti non nascono per essere tali, ma lo diventano grazie alla loro profonda e determinata forza di volontà.

Per questo motivo Ljust interviene in loro favore con l'elaborato Rito del Rinnovo, grazie al quale ogni anno al Devoto meritevole e pentito di aver perso anche solo per poco la giusta direzione, la Luce, viene fatto recuperare ogni punto Tratto perso perché agito fuori dalle 7 Regole Luminose.

Sumkjr è un soldato valoroso, il migliore amico del giusto.

Calicante, preso dall'orrore alla vista di un Patrono così fatto, lo privò della capacità di amare e provare veri sentimenti d'affetto. Portare il bene per un Devoto di Sumkjr è un qualcosa di normale come è normale non riuscire ad essere empatico con chi soffre. Il Devoto sa cosa deve fare e perché, ma non riesce a commuoversi od amare di fronte alle sofferenze od alle carezze di una donna/uomo.

\textbf{Simbolo:} tre gocce di sangue che cadono una dietro l'altra

\textbf{Tratti}: Giusto, Curioso, Buono, Valoroso

\textbf{Manifestazione}: il Devoto è avvolto da un mantello di broccato dorato

\bigskip

Tratto in comune a 5 punti: Il tocco della tua spada e’ vita. Una creatura toccata con la tua arma recupera 3d6 punti ferita. Una volta al giorno. Costo 2 Azioni.

Tratto in comune a 10 punti: La tua Volonta’ e’ piu’ forte del metallo. Guadagni un +2 ai Tiri Salvezza su Arbitrio

Tratto in comune a 15 punti: Concentri l’energia del tuo Patrono in un cono di assordante. Il cono e’ lungo 18 metri e largo al termine 3 metri, chiunque sia preso nell’area subisce 10d6 di danno.
Una volta al giorno. Costo 2 Azioni.

Tratto in comune a 20 punti: Sacrifichi la tua vita per portare in vita una creatura morta da non piu’ di 1 giorno. Una volta. Costo 2 Azioni.

\bigskip

Elementi: Energia Positiva, Elettricità

\bigskip

Essenze Privilegiate: Cura\\
Essenze Normali: Protezione, Difesa, Creazione\\
Essenze Limitate: Convocazione, Trasformazione,Charme\\
Essenze Negate: Attacco, Movimento, Illusione, Alterazione, Rivelazione, Distruzione\\

\medskip

Accesso alla Scuole di Magia:\\
Scuole Privilegiate: Universale, Invocazione (+2 alle prove di CM)\\
Scuole Normali: Abiurazione, Trasmutazione, Illusione. Evocazione, Ammaliamento (+0 alle prove di CM)\\
Scuole Limitate: Divinazione (-2 alle prove di CM)\\
Scuole Negate: Necromanzia  (scuole non accessibili)\\


\bigskip

\paragraph{Le 7 Regole Luminose}\index{7 Regole Luminose}

\label{le-7-regole-luminose}

Le Sette regole Luminose sono un insieme di norme e comportamenti tenuti, a vario titolo, dai Devoti che vogliono seguire la strada della Luce di Ljust.

I Devoti di Sumkjr devono seguirle tutte e 7 pena la perdita di potere (punti Tratto), altri Devoti di altri Patroni, sempre positivi od almeno neutrali, seguono solo alcune di questi dettami, come regola per non cadere nelle braccia di Calicante


\begin{enumerate}
	\item Proteggi i deboli e chi non sa difendersi dai soprusi
	\item Ama la vita e proteggila. L'Amore deve vincere sopra ogni cosa
	\item Combatti contro le ingiustizie e chi porta sofferenze e dolore
	\item Lenisci le ferite ed i dolori. Placa gli animi e favorisci la pace
	      ed armonia
	\item Onestà e Lealtà sono le tua fondamenta
	\item Sei un maestro di virtù. Fa che gli altri possano prendere ispirazione
	      dalle tue gesta
	\item Sii luminoso ma non accecare gli altri
\end{enumerate}

\subsubsection{Shayalia}\index{Shayalia}

\label{shayalia}

Patrono dell'Arcano di Tenebra. Shayalia è l'anima oscura della perdizione, del tradimento, della lussuria più sordida e peccaminosa. Adora i bordelli. Le piace l'odore acre del sudore, la pelle lucida di oli e profumi. Le passioni, le vendette che li si consumano, la distruzione fisica e morale che in quei luoghi viene perpetrata è la sua vita.

Shayalia è una donna che gode di se stessa, che vive dei piaceri più sordidi. Vive di vendette lungamente e ben dettagliatamente progettate. Vendicativa ed amorale, non giudica con metro di giudizio umano, il suo godere non è neppure lontanamente comprensibile. Shayalia è quanto di più vicino a Calicante sia stato creato. Sono le passioni, le pulsioni, i liquidi umorali che la fanno inebriare.

Shayalia è la concubina che ti ammalia e ti distrugge, goccia dopo goccia. I veleni sono le sue armi, le debolezze umane il suo campo.

I Devoti di Shayalia sono spie, figli bastardi, amanti di potenti signori che agiscono all'ombra.

Ljust disgustata dalla visione di un Patrono del genere instillò in Shayalia l'amore per la natura, piante ed animali. E così molti dei più famosi botanici, erboristi e zoologi sono Devoti di Shayalia, forse le uniche cose che Shayalia veramente può amare.

\textbf{Simbolo:} un cuscino stropicciato e sporco di sangue

\textbf{Tratti}: Lussurioso, Volubile, Pessimista, Sadomasochista

\textbf{Manifestazione}: il Devoto è avvolto da un mantello di velluto nero

\bigskip

Tratto in comune a 5 punti: I tempi per preparare una pozione sono dimezzati.

Tratto in comune a 10 punti: I costi per preparare una pozione sono dimezzati.

Tratto in comune a 15 punti: Dal tuo palmo secerni veleno. Il tuo tocco, o tramite arma in mischia veicola il veleno. TS Tempra DC 25 o -2 a Volontà e Agilità per 10 minuti. Costo 1 Azione.

Tratto in comune a 20 punti: Il tuo tocco è vita per la natura. Puoi curare animali e piante anche magiche con la Essenza di Distruzione, gli effetti sono quelli della Cura.

\bigskip

Elementi: Vuoto, Elettricità

\bigskip

Essenze Privilegiate: Distruzione\\
Essenze Normali: Charme, Alterazione, Illusione\\
Essenze Limitate: Attacco, Convocazione, Trasformazione\\
Essenze Negate: Movimento, Protezione, Creazione, Cura, Rivelazione,Difesa\\

\medskip

Accesso alla Scuole di Magia:\\
Scuole Privilegiate: Universale, Invocazione (+2 alle prove di CM)\\
Scuole Normali: Necromanzia, Illusione, Evocazione, Ammaliamento (+0 alle prove di CM)\\
Scuole Limitate: Divinazione, Trasmutazione(-2 alle prove di CM)\\
Scuole Negate: Abiurazione (scuole non accessibili)\\



\textbf{Sumkjr} e \textbf{Shayalla} sono complementari nel tenere in mano le file sfuggenti e pericolose della magia. Agiscono come espressione diretta di dei Patroni maggiori

\subsubsection{Sixiser}\index{Sixiser}

il Patrono che è indifferente al presente in quanto totalmente, compulsivamente ossessionato dal futuro e dal suo destino. Negli angoli più remoti dei mondi conosciuti si narra che Sixiser accumuli di tutto, indifferente a tutto e tutti.

Terrorizzato dal futuro che vede, da una ipotetica fine di sé e del tutto vive una vita di ritiro, spirituale e fisico. Si priva volontariamente di tutto il necessario. Ma allo stesso accumula qualunque oggetto incroci la sua strada nella speranza di un ritorno.

è paranoico e non si fida di nessuno. Usa i suoi poteri di divinazione per conoscere e scrutare tutti.

I Devoti di Sixiser sono spesso negromanti circondati da non morti ed altre creature silenziose ed ubbidienti. Chi si rifugia alla ricerca della solitudine e dello studio chi invece mira ad espandere e governare intere città e nazioni al fine di sentirsi più sicuro e' devoto a Sixiser.

\textbf{Simbolo}: un forziere straripante di ogni cosa che non si può chiudere

\textbf{Tratti}: Riservato, Morigerato, Accumulatore, Paranoico

\textbf{manifestazione}: due mani uncinate che circondano, come a nascondere, la testa dell'incantatore

\bigskip

Tratto in comune a 5 punti: acquisisci la fino 18 metri, o 36 metri se già presente.

Tratto in comune a 10 punti: vedi nell'oscurità anche magica. Vedi le trappole nel raggio di mischia intorno a te.

Tratto in comune a 15 punti: Toccando un oggetto sei in grado di capirne tutte le proprietà magiche e non.

Tratto in comune a 20 punti: Sei in grado di animare una creatura morta da non più di un giorno come non morto da 1 CR (tipo zombi/scheletro a secondo dello stato). Una volta al giorno. Costo 2 Azioni.

\bigskip

Elementi: Elettricità, Energia Negativa

\bigskip

Essenze Privilegiate: Rivelazione\\
Essenze Normali: Distruzione, Illusione, Difesa\\
Essenze Limitate: Movimento, Attacco, Convocazione, Charme\\
Essenze Negate: Creazione, Trasformazione, Difesa, Cura, Alterazione\\

\medskip

Accesso alla Scuole di Magia:\\
Scuole Privilegiate: Universale, Divinazione (+2 alle prove di CM)\\
Scuole Normali: Necromanzia, Evocazione, Illusione, Ammaliamento (+0 alle prove di CM)\\
Scuole Limitate: Invocazione, Trasmutazione (-2 alle prove di CM)\\
Scuole Negate: Abiurazione  (scuole non accessibili)\\


\subsubsection{Tazher}\index{Tazher}

\label{tazher}

il Patrono delle Ombre; colui che silenzioso, ti uccide. Non saprai mai il perché. Non conoscerai mai il suo aspetto ma, se improvvisamente hai una sensazione di gelo, Tazher è dietro di te pronto a prendere la tua vita.
Doppiogiochista dall'animo cattivo, chiedi il suo aiuto solo se sei disposto a pagarne il prezzo che lui e lui solo deciderà.

Vive di notte, vive la notte. Le ombre sono le sue amiche e la tenebra il suo mantello. Profondamente individualista con un carattere scontroso e permaloso, non ha amici, non intrattiene relazioni di alcun tipo.

Il Devoto di Tazher è il ladro, l'assassino, il bandito, chiunque viva per l'oscurità ed il proprio tornaconto. Un Devoto di Tazher è estremamente pericoloso in combattimento.

\textbf{Simbolo}: lo scintillio della lama nel buio

\textbf{Tratti}: Scontroso, Calcolatore, Perfezionista, Cattivo

\textbf{Manifestazione}: l'ombra del Devoto prende vita muovendo l'arma

\bigskip

Tratto in comune a 5 punti: Guadagni +4 alle prove di criminalità.

Tratto in comune a 10 punti: Una volta al giorno fai un attacco in più. Una Azione Immediata.

Tratto in comune a 15 punti: finché cammini sopra delle ombre o al buio sei invisibile. Puoi essere comunque rilevato con la luce o Essenza di Rivelazione.

Tratto in comune a 20 punti: Una volta al giorno su tutti gli attacchi andati a segno in quel round fai il doppio del danno. Costo 1 Azione.

\bigskip

Elementi: Vuoto, Fuoco

\bigskip

Essenze Privilegiate: Attacco\\
Essenze Normali: Convocazione, Trasformazione, Charme\\
Essenze Limitate: Illusione, Alterazione, Distruzione, Movimento\\
Essenze Negate: Rivelazione, Protezione, Creazione, Cura, Difesa\\

\medskip

Accesso alla Scuole di Magia:\\
Scuole Privilegiate: Universale, Necromanzia (+2 alle prove di CM)\\
Scuole Normali: Evocazione, Illusione, Ammaliamento (+0 alle prove di CM)\\
Scuole Limitate: Divinazione, Invocazione, Trasmutazione (-2 alle prove di CM)\\
Scuole Negate: Abiurazione  (scuole non accessibili)\\

\subsubsection{Thaft}\index{Thaft}

\label{thaft}

il Patrono che accompagna nella nascita e nella morte. Silenzioso, resta in disparte e osserva lo scorrere della vita degli uomini. Quasi umile nella sua semplicità, Thaft è ovunque. Testimone silenzioso della vita umana; nel momento in cui una vita scivola via, Thaft assiste, nell'attimo in cui una vita nasce, Thaft è presente.

Thaft sa anche che non si può essere sempre e solo osservatori. Attraverso il suo taccuino sacro e magico può decidere e giudicare della vita degli uomini. perché se una spada ferisce, è solo Thaft che ne decide la morte.

I Devoti di Thaft sono i sacerdoti dell'ultimo viaggio, coloro che proteggono e vegliano sulle anime e corpi dei morti. Profondamente contrari all'utilizzo dei non-morti ne perseguono la distruzione.

Un Devoto di Thaft rispetta la vita come la morte e non teme di arrecare distruzione per un equilibrio maggiore.

Thaft è stato plasmato da Atmos.

\textbf{Simbolo}: un libro aperto con un teschio sopra

\textbf{Tratti}: Semplice, Silenzioso, Mite, Sicuro

\textbf{Manifestazione}: si sente il pianto di un bambino appena nato o il sospiro della morte

\bigskip

Tratto in comune a 5 punti: il tuo tocco è letale per i non morti. Un tuo tocco infligge 2d6 di danno ad un non morto. Costo 2 Azioni.

Tratto in comune a 10 punti: Il tuo tocco lenisce. Una volta al giorno puoi rimuovere Cecità o Sordità. Costo 2 Azioni.

Tratto in comune a 15 punti: Un non morto deve effettuare un Tiro Salvezza Tempra DC 30 o essere distrutto se toccato dalla tua mano. Costo 2 Azioni.

Tratto in comune a 20 punti: Uccidi la creatura toccata. TS Arbitrio DC 35 o morte. Una volta alla settimana. Costo 2 Azioni.

\bigskip

Elementi: Suono, Elettricità

\bigskip

Essenze Privilegiate: Distruzione\\
Essenze Normali: Cura, Movimento, Convocazione\\
Essenze Limitate: Protezione, Difesa, Trasformazione, Charme\\
Essenze Negate: Alterazione, Rivelazione, Illusione, Creazione, Attacco\\

\medskip
Accesso alla Scuole di Magia:\\
Scuole Privilegiate: Universale, Necromanzia (+2 alle prove di CM)\\
Scuole Normali: Invocazione, Abiurazione, Ammaliamento (+0 alle prove di CM)\\
Scuole Limitate: Divinazione, Illusione, Trasmutazione (-2 alle prove di CM)\\
Scuole Negate: Evocazione (scuole non accessibili)\\

\subsubsection{Torbiorn}\index{Torbiorn}

\label{torbiorn}

il Patrono che meglio incarna il concetto "non è mai abbastanza".

Alto, bello come un quadro ma, proprio come quest'ultimo, senza calore e vita, Torbiorn rasenta la perfezione maniacale nel vestirsi, nell'atteggiarsi.

Nulla è mai abbastanza per lui. Nessuno è mai alla sua altezza. Ed eccolo che con arroganza e ironia va a modificare tutto il modificabile per poter placare questa profonda insoddisfazione. Qualora il risultato finale raggiunto non lo soddisfi, e accade molto spesso, ecco che prende il sopravvento il suo cinismo e distrugge tutto senza curarsi della sofferenza che sta arrecando a chi gli sta attorno.

Il Devoto di Torbiorn è il tipico aristocratico ricco e svogliato,colui che cerca sempre la strada più facile e meno rischiosa.

Incurante degli altri si diverte nello sfruttare i lavori altrui e trarne giovamento.

\textbf{Simbolo}: uno specchio opaco

\textbf{Tratti}: Altezzoso, Indifferente, Vanitoso, Permaloso

\textbf{Manifestazione}: schegge di specchio rotto tutto intorno al Devoto come un turbine

\bigskip

Tratto in comune a 5 punti: Con un gesto puoi rinfrescare i tuoi vestiti rendendoli puliti e profumati. Costo 1 Azione.

Tratto in comune a 10 punti: Il tuo sputo è velenoso. TS Tempra DC 20 oppure -2 Potenza. Durata 1 minuti. Tre volte al giorno. Costo 1 Azione.

Tratto in comune a 15 punti: Fissando l'obiettivo negli occhi lo costringi a fermarsi. Il target non può più muovere le gambe. TS Arbitrio DC 30. Una volta al giorno. Costo 2 Azioni.

Tratto in comune a 20 punti: Dalle tue dita partono dei viticci che pungono fino a 10 avversari.

Ogni viticcio causa 2d6 di danno, TS Tempra DC 25 per dimezzare. Costo 2 Azioni.

\bigskip

Elementi: Fuoco, Suono

\bigskip

Essenze Privilegiate: Trasformazione\\
Essenze Normali: Distruzione, Movimento, Attacco\\
Essenze Limitate: Charme, Rivelazione, Convocazione, Alterazione\\
Essenze Negate: Illusione, Protezione, Cura, Difesa, Creazione\\

\medskip

Accesso alla Scuole di Magia:\\
Scuole Privilegiate: Universale, Trasmutazione, (+2 alle prove di CM)\\
Scuole Normali: Necromanzia, Invocazione, Divinazione, Ammaliamento (+0 alle prove di CM)\\
Scuole Limitate: Illusione, Abiurazione (-2 alle prove di CM)\\
Scuole Negate: Evocazione (scuole non accessibili)\\

\subsubsection{Tabella collegamento Patrono - Tratto}\index{Tabella collegamento Patrono - Tratto}

\label{tabella-collegamento-patrono---tratto}
\medskip
\begin{tabular}{llllll}
	\toprule
	\textbf{Nome Patrono} & \textbf{Tratto} & \textbf{Tratto} & \textbf{Tratto}   & \textbf{Tratto}\\
	Atherim& Allegro    & Calmo & Industrioso  & Buono\\
	Atmos  & Osservatore& Distaccato & Studioso& Riflessivo\\
	Belevon& Bugiardo   & Narcisista & Casto   & Doppiogiochista\\
	Calicante   & Egoista    & Vendicativo& Superbo & Iracondo\\
	Cattalm& Distruttivo& Anarchico  & Meticoloso   & Sadico\\
	Efrem  & Indifferente    & Leale & Fiducioso    & Pratico\\
	Erondil& Perfezionista   & Incontentabile  & Sognatore    & Esuberante\\
	Gaya   & Innovativo & Ordinato   & Istintivo    & Prudente\\
	Gradh  & Indomito   & Protettivo & Vendicativo  & Coraggioso\\
	Krondal& Avventato  & Pio   & Corretto& Libero\\
	Ledyal & Caritatevole    & Sospettoso & Introverso/integerrimo & Clemente/implacabile\\
	Ljust  & Generoso   & Empatico   & Coraggioso   & Protettivo\\
	Lynx   & Solitario  & Rigido& Serio   & Controllato\\
	Nedraf & Disciplinato    & Combattivo & Tenace  & Aggressivo\\
	Nethergal   & Sarcastico & Impetuoso  & Immaturo& Logorroico\\
	Nihar  & Altruista  & Determinato& Cortese & Attento\\
	Orudjs & Ironico    & Codardo    & Saccente& Socievole\\
	Orlaith& Imparziale & Giusto& Gentile & Valoroso\\
	Rezh   & Avaro & Arrogante  & Cattivo & Freddo\\
	Shayalia    & Lussurioso & Volubile   & Pessimista   & Sadomasochista\\
	Sixiser& Riservato  & Morigerato & Accumulatore & Paranoico\\
	Sumkjr & Giusto& Curioso    & Buono   & Valoroso\\
	Tazher & Scontroso  & Calcolatore& Perfezionista& Cattivo\\
	Thaft  & Semplice   & Silenzioso & Mite    & Sicuro\\
	Torbiorn    & Altezzoso  & Indifferente    & Vanitoso& Permaloso\\
\end{tabular}

\pagebreak

\subsection*{Dei Antichi - OGL}\index{Dei Antichi}

Ci fu un tempo, millenni orsono, in cui i Patroni, pardon Dei, non erano capricciose rappresentazioni delle virtu' e difetti umani, bensì potenze che dall'alto del loro scranno guidavano le coscienze ed impartivano la fede.

Non si mescolavano con le creature mortali, non agivano personalmente, non erano mosse da curiosita' o volonta' di ripicca su altri dei.

Le prime divinità create erano potenze ultraterrene che, venerate o meno, comunque sarebbero esistite e non si curerebbero di avere pochi fedeli.

Gli dei della Genesi crearono "Il Libro degli Dei" un potentissimo artefatto che illustrava e spiegava le divinità, descrivendole nei dogmi, regole, feste e dettagli.
Un libro talmente potente che cio' che veniva li scritto diveniva realta'.

Queste divinità non piacquero agli dei della Genesi, non creavano abbastanza chaos e mutamento per i loro voleri, non riuscivano ad innalzare il creato ad uno stato superiore.
Cosi' furono abbandonati, li fecero dimenticare e li ritirarono altrove.

Il tomo venne disperso, forse distrutto forse ridotto a brandelli; poche pagine vennero ritrovate ma ancora qualche saggio e culto antico cercano di recuperare e ricostruire l'antico artefatto.

\subsubsection{Utilizzo delle divinità Antiche}

Gli Dei Antichi vengono qui presentati come alternativa ai Patroni attuali per un approccio ad un pantheon fantasy piu' tradizionale, meno legato ai tratti e piu' conforme alla tradizione.

Il Narratore potra' decidere di usare i Patroni o gli Dei Antichi, oppure a seguito di un buon background autorizzare entrambi gli dei.

Molti maghi fanno affidamento alla fede per potenziare le loro competenze e capacità magiche, l'essere infatti fedele e obbedire e seguire i dogmi di una divinità concede dei bonus peculiari possano essere Abilita' o Vantaggi.

Ogni divinità concede una o piu' essenze come favorite, in questa essenza il fedele ha un +2 alla prova di Competenza Magica, allo stesso tempo potrebbe avere delle Essenze limitate in quanto contrarie al proprio credo, su queste ha un -2 alla prova di Competenza Magica.

Nel caso di utilizzo della Magia OGL viene indicata una Scuola di Magia favorita (+2 alla prova di competenza magica), non ci sono scuole di magia negate o sfavorita.

Non c'e' bisogno di avere tratti specifici o allineamento particolare per essere fedeli di una Divinità bensì e' sufficiente dichiarare la propria appartenenza e cercare di seguire i suoi dogmi.

\subsubsection{Le divinità Antiche}\index{divinità Antiche}

I pochi dogmi presentati sono quanto e' stato ricordato dalla lettura de \textit{Il libro degli Dei} dal primo, e unico, devoto che lo lesse milenni addietro.

Essendo basati su ricordi e tradizioni orali e non su documentazione oggettiva non si ha la certezza che siano tutti corretti o completi, ma la fede, come la terra, e' eterna e quindi finché l'arcano tomo non verra' ritrovato e i nomi di tutti gli dei conosciuti, fino ad allora queste saranno le uniche e vere tavole della fede.

\bigskip

\begin{tabularx}{0.95\textwidth}{XXXX}
\textbf{Bene} & \textbf{Male} & \textbf{Chaos} & \textbf{Legge}\\
Koira - La Carità & Daraka - L'Oscurita' & 	Sefryn - Gli Elementi&	Talo - La Conoscenza\\
Daern - Il Creatore & Zarkor - La Sofferenza & Zxcvbnm - I'Inganno & Moraim - La Giustizia\\
Solimi - La Luce &	Eramide - I Morti & Jaskara - La Magia & Ediath - La Morte\\
Seeyek - L'Avventura & 	Averim - La Bellezza & Xabax - Il Destino \\
&& Isar - La Fortuna \\
\textbf{Generico Buono} & \textbf{Generico Malvagio} & \textbf{Generico Chaotico} & \textbf{Generico Legge}\\
\textbf{Divinità Neutrale}: &Kyriel \\

\end{tabularx}

\pagebreak

\subsubsection{Koira}

Aspetto: La Vita\index{Koira}
\bigskip


I fedeli della dea Koira sono i curatori per eccellenza, padroni della vita e portatrici di salute.

I fedeli di Koira sono non violenti e curatori di tutti gli ammalati e bisognosi, l'unica arma loro concessa e' il bastone, usato solo come autodifesa.

Questi i loro dogmi:

\begin{itemize}
	\item Mai rifiutare una cura a chi ne' ha bisogno
	\item Mai curare se stessi se gli altri ne hanno maggiore bisogno
	\item Dare morte pietosa a chi non puo' essere curato
	\item Colpire solo in autodifesa e mai uccidere se non necessario
	\item Impedire uccisioni inutili e indiscriminate
	\item La vita e' sacra ed inviolabile
\end{itemize}


I fedeli di Koira sono spesso curatori vagabondi che girano per villaggi a portare aiuto e assistenza.\\


\textbf{Vantaggi}: Cure Efficaci e Controllo del Metabolismo oppure Imposizione delle Mani \\
\textbf{Essenza Favorita}: Cura\\
\textbf{Essenza Sfavorita}: Distruzione\\

\medskip

Accesso alla Scuole di Magia:\\
Scuole Privilegiate: Universale, Necromanzia (+2 alle prove di CM)\\
Scuole Normali: Illusione, Abiurazione, Invocazione, Trasmutazione, Divinazione, Evocazione, Ammaliamento (+0 alle prove di CM)\\


\bigskip

\subsubsection{Daern}

Aspetto: La Creazione\index{Daern}
\bigskip


I fedeli di Daern sono i custodi della creazione e della purezza originaria.

Essi sostengono di essere i seguaci del dio generatore di tutto e tutti.

Spesso artisti o ingegneri, i maggiori prelati si trovano tra i nani che si reputano la prima razza dal dio creata.

\begin{itemize}
	\item Tutto il creato e' sacro, ogni cosa ha un costo nell'universo, ogni atto di distruzione non necessario e' un sacrilegio.
	\item Daern e' il primo ed ultimo
	\item Colui che plasma e' in grazia al dio
\end{itemize}


I fedeli di Daern hanno una organizzazione abbastanza capillare in tutto il territorio, e' il culto con piu' seguaci.\\

\textbf{Vantaggi}: Radici Magiche\\
\textbf{Essenza Favorita}: Creazione\\
\textbf{Essenza Sfavorita}: Illusione\\

\medskip

Accesso alla Scuole di Magia:\\
Scuole Privilegiate: Universale, Evocazione (+2 alle prove di CM)\\
Scuole Normali: Illusione, Abiurazione, Invocazione, Necromanzia, Trasmutazione, Divinazione, Ammaliamento (+0 alle prove di CM)\\


\bigskip
\subsubsection{Moraim}

Aspetto: La Giustizia\index{Moraim}
\bigskip


I fedeli del dio Moraim sono giudici e giustizieri per scelta e vocazione. essi si considerano gli unici portatori della giustizia divina in terra.


Culto molto guerriero e' diffuso presso gli alti ranghi guerriero-aristocratici

\begin{itemize}
	\item Lotta eterna contro il male e il chaos: tutto cio' che e' male o puo' divenirlo deve essere distrutto
	\item Chi combatte per la verita' non muore, chi la disprezza e' maledetto e condannato in eterno
\end{itemize}

Meno diffusi di Daern, i fedeli di Moraim appartengono spesso a piccole elite guerriere, spesso partecipano a guerre sacre o in opere di colonizzazione\\

\textbf{Vantaggi}: Colpi Poderosi\\
\textbf{Essenza Favorita}: Protezione\\
\textbf{Essenza Sfavorita}: Rivelazione\\

\medskip

Accesso alla Scuole di Magia:\\
Scuole Privilegiate: Universale, Abiurazione (+2 alle prove di CM)\\
Scuole Normali: Illusione, Abiurazione, Invocazione, Necromanzia, Trasmutazione, Divinazione, Evocazione, Ammaliamento (+0 alle prove di CM)\\


\bigskip
\subsubsection{Solimi}

Aspetto: Il Sole\index{Solimi}
\bigskip

I fedeli della dea Solimi sono portatori di luce e verita', spesso fungono da oracoli. Sempre ben abbronzati, potrebbero fissare il sole tutto il tempo.

Di morale positiva sono tra i fedeli piu' diffusi

\begin{itemize}
	\item La luce e' vita e conoscenza
	\item Il sole sorge dal grembo di Solimi ed e' portatore di vita
	\item Mai spegnere una fonte di luce
\end{itemize}

Fedeli tranquilli e pacifici sanno pero' usare fuoco e fiamme per difendere i loro protetti.\\

\textbf{Vantaggi}: Illuminato oppure Direzione Assoluta\\
\textbf{Essenza Favorita}: Attacco\\
\textbf{Essenza Sfavorita}: Trasformazione\\

\medskip

Accesso alla Scuole di Magia:\\
Scuole Privilegiate: Universale, Invocazione, (+2 alle prove di CM)\\
Scuole Normali: Illusione, Abiurazione, Necromanzia, Trasmutazione, Divinazione, Evocazione, Ammaliamento (+0 alle prove di CM)\\

\bigskip

\subsubsection{Isar}

Aspetto: La Fortuna\index{Isar}
\bigskip

Com'e' il bicchiere ? mezzo pieno?, no quasi colmo. questi sono i fedeli del dio Isar ,ottimisti e spensierati, ma non per questo ingenui o sciocchi. Sanno di essere fortunati e di cio' ne approfittano

\begin{itemize}
	\item Mai disperarsi, non siamo soli 
	\item Se qualcosa puo' andare male, il fedele si riterrà fortunato perche' sà che poteva andare peggio
	\item Se c'e' un problema c'e' anche una soluzione
\end{itemize}


Culto diffusissimo, noto ovunque e discretamente praticato.\\


\textbf{Vantaggi}: Fortunato\\
\textbf{Essenza Favorita}: Movimento\\
\textbf{Essenza Sfavorita}: Convocazione\\

\medskip

Accesso alla Scuole di Magia:\\
Scuole Privilegiate: Universale, Trasmutazione (+2 alle prove di CM)\\
Scuole Normali: Illusione, Abiurazione, Necromanzia, Invocazione, Divinazione, Evocazione, Ammaliamento (+0 alle prove di CM)\\

\bigskip

\subsubsection{Talo}

Aspetto: La Conoscenza\index{Talo}
\bigskip

I fedeli di Talo sono gli studiosi e i custodi della conoscenza universale. Spesso stantii nelle grandi biblioteche i fedeli di Talo possono sapere qualsiasi cosa


\begin{itemize}
	\item Scopo nella vita e' accumulare sapienza. non e' importante l'uso che di essa e ne fa'
	\item La conoscenza va raccolta e diffusa, chi mantiene la conoscenza solo per se' compie peccato contro Talo
	\item Chi fa si che la conoscenza venga distrutta compie un sacrilegio
	\item La conoscenza pratica e quella teorica sono di eguale rispetto
\end{itemize}

I fedeli di Talo sono tra i principali creatori di magie e molte delle piu' interessanti creazioni e combinazioni sono loro.\\


\textbf{Vantaggi}: Lingua Universale\\
\textbf{Essenza Favorita}: Rivelazione\\
\textbf{Essenza Sfavorita}: Attacco\\


\medskip

Accesso alla Scuole di Magia:\\
Scuole Privilegiate: Universale, Divinazione (+2 alle prove di CM)\\
Scuole Normali: Illusione, Abiurazione, Necromanzia, Invocazione, Trasmutazione, Evocazione, Ammaliamento (+0 alle prove di CM)\\


\bigskip


\subsubsection{Sefryn}

Aspetto: Gli Elementi\index{Sefryn}
\bigskip


I fedeli della dea Sefryn sono credenti nei quattro elementi generatori del mondo. Spesso di atteggiamento imprevedibile sanno ottimamente manipolare l'ambiente intorno a loro.


\begin{itemize}
	\item Tutto e' nato da Sefryn e tutto a lei torna
	\item Gli elementi sono purificatori ed ognuno e' sacro
\end{itemize}

Purtroppo solo questi due dogmi sono stati tramandati dal racconto del sacro libro, questo fa si che i fedeli di Sefryn siano tra i piu' liberi da vincoli etici.\\


\textbf{Vantaggi}: Resistenza ad un elemento, sceglilo. Ignori i primi 3 punti ferita di danno per round

\textbf{Essenza Favorita}: Difesa\\
\textbf{Essenza Sfavorita:} Illusione\\

\medskip

Accesso alla Scuole di Magia:\\
Scuole Privilegiate: Universale, Evocazione (+2 alle prove di CM)\\
Scuole Normali: Illusione, Abiurazione, Necromanzia, Invocazione, Trasmutazione, Divinazione, Ammaliamento (+0 alle prove di CM)\\


\bigskip

\subsubsection{Jaskara}

Aspetto: La Magia\index{Jaskara}
\bigskip


I fedeli della dea Jaskara si considerano gli unici veri maghi perche' considerano la loro dea la creatrice di tutte le Essenze

\begin{itemize}
	\item La forma piu' alta di potere e' la magia
	\item La creazione magica e' il sacrificio che Jaskara predilige
	\item Il mago ringrazi la Dea al plenilunio 
	\item La ricerca e' il cammino verso la magica jaskara
\end{itemize}


\textbf{Vantaggi}: Rilevare il magico\\
\textbf{Essenza Favorita}: una a scelta\\
\textbf{Essenza Sfavorita}: una a scelta\\


\medskip

Accesso alla Scuole di Magia:\\
Scuole Privilegiate: Universale, una a scelta (+2 alle prove di CM)\\
Scuole Normali: le rimanenti scuole di magia (+0 alle prove di CM)\\


\bigskip

\subsubsection{Ediath}

Aspetto: La Morte\index{Ediath}
\bigskip


I fedeli della dea della morte sono persone che e' meglio non incontrare se si e' anziani o se si hanno bevuto troppe pozioni di giovinezza. Dotati di un particolare codice di comportamento sono a volte spietati assasini o santi curatori, mah..., valli a capire...

\begin{itemize}
	\item La morte e' eterna e senza riposo
	\item Ediath coglie tutti
	\item La morte giunge al suo momento
	\item I non morti sono aberrazioni
\end{itemize}


Culto poco diffuso e poco pubblicizzato, presente, comunque in ogni maggiore citta'.\\


\textbf{Vantaggi}: Tocco gelido\\
\textbf{Essenza Favorita}: Distruzione\\
\textbf{Essenza Sfavorita:} Illusione\\

\medskip

Accesso alla Scuole di Magia:\\
Scuole Privilegiate: Universale, Necromanzia (+2 alle prove di CM)\\
Scuole Normali: Illusione, Abiurazione, Evocazione, Invocazione, Trasmutazione, Divinazione, Ammaliamento (+0 alle prove di CM)\\

\bigskip


\subsubsection{Xabax}

Aspetto: Il Chaos\index{Xabax}
\bigskip

I fedeli del dio Xabax sono persone molto particolari e difficilmente capibili.

Forse pessimisti o ottimisti per natura, predicano l'agnosticismo universale, correndo in contro, se cosi' gli va', a qualsiasi destino.


\begin{itemize}
	\item Nulla e' eterno ed immutabile
	\item La vita e' affidata alla sorte, di cui Xabax e' padrone, Il fedele confida nel Dio
	\item Xabax e' mutevole, cosi' il suo favore, il vero fedele sa di non sapere niente e prende la vita cosi' come viene
	\item Non puoi guardare oltre , perche' li' e' Xabax
\end{itemize}

Culto molto piccolo ma molto conosciuto per le sue particolari idee della vita.

Le feste degli accoliti di Xabax sono le piu' famose e gettonate e si protraggono per giorni e giorni.\\


\textbf{Vantaggi}: Senza paura\\
\textbf{Essenza Favorita}: Convocazione\\
\textbf{Essenza Sfavorita:} Rivelazione\\

\medskip

Accesso alla Scuole di Magia:\\
Scuole Privilegiate: Universale, Evocazione (+2 alle prove di CM)\\
Scuole Normali: Illusione, Abiurazione, Necromanzia, Invocazione, Trasmutazione, Divinazione, Ammaliamento (+0 alle prove di CM)\\


\bigskip

\subsubsection{Seeyek}

Aspetto: L'Avventura\index{Seeyek}
\bigskip

I fedeli di Seeyek sono avventurieri, cercatori di gloria e fama, amiconi e protettori del gruppo.

Ed al gruppo molto e' legato il suo credo e la loro forza.


\begin{itemize}
	\item Il vero fedele ama l'avventura e rispetta i compagni, poiche' grazie a loro potra' aumentare la sua esperienza
	\item Una torcia, una mazza e la fede, solo grazie a Seeyek si aggiunge esperienza
	\item iIprovvisare adattarsi e raggiungere lo scopo, cosi' il dio impone
\end{itemize}

Autentici self-made man. i fedeli di Seeyek sono fondamentalmente buoni e sanno essere validi compagni, fedeli ed onesti.\\


\textbf{Vantaggi}: Scudo mentale\\
\textbf{Essenza Favorita}: una a selta\\
\textbf{Essenza Sfavorita}: una a scelta\\

\medskip

Accesso alla Scuole di Magia:\\
Scuole Privilegiate: Universale, una a scelta (+2 alle prove di CM)\\
Scuole Normali: le rimanenti scuole di magia (+0 alle prove di CM)\\

\bigskip



\subsubsection{Daraka}

Aspetto: L'Oscurita\index{Daraka}
\bigskip

se Daern e' il culto positivo principale, quello della dea Daraka e' il malvagio piu' noto.

I fedeli di Daraka non fanno nascondimento del loro credo e con le loro scure vesti girano per le strade a cercare accoliti o ad eseguire le loro oscure trame.


I fedeli della Dea Daraka bramano il potere, con qualsiasi mezzo possa essere ottenuto

\begin{itemize}
	\item Diffondere il culto imponendo il terrore del Dio
	\item Potenziare il culto assurgendo a cariche di potere
	\item Solo nell'oscurita' la verita' esiste
	\item Nella notte senza stelle il sorriso di Daraka benedice i suoi discepoli
	\item Nella confusione Daraka appare, e tutto ora e' chiaro
\end{itemize}


E' l'unico culto, malvagio, che abbia pubbliche chiese nelle maggiori citta'.\\


\textbf{Vantaggi}: La mia ombra è mia amica\\
\textbf{Essenza Favorita}: Convocazione\\
\textbf{Essenza Sfavorita}: Trasformazione\\

\medskip

Accesso alla Scuole di Magia:\\
Scuole Privilegiate: Universale, Evocazione (+2 alle prove di CM)\\
Scuole Normali: Illusione, Abiurazione, Necromanzia, Invocazione, Trasmutazione, Divinazione, Ammaliamento (+0 alle prove di CM)\\

\bigskip

\subsubsection{Zxcvbnm}

Aspetto: L'Inganno\index{Zxcvbnm}
\bigskip


I fedeli della dea dell'inganno, non appaiono mai come tali: travestitismo, sotterfugio, inganno sono la loro vita e passione. abilissimi ladri e dotati di limitate capacita' illusionistiche, i fedeli di sua Signora degli Inganni sono i principali componenti e fondatori delle gilde dei ladri. Culto non strettamente malvagio e' presente nei bassifondi di tutte le citta'.

\begin{itemize}
	\item La realta' e illusione, l'illusione e' la realta', chi e' padrone delle illusioni e' padrone della realta'
	\item Un furto spettacolare o una truffa sono l'offerta' piu' gradita alla Dea
	\item Inganno, diplomazia, negoziato, spada: solo in quest'ordine agisce il fedele
\end{itemize}

Quindi quando incontrate un ladruncolo, state attenti potrebbe essere un fedele di SSDI, e potreste attirare l'attenzione di tutti gli altri devoti.\\

\textbf{Vantaggi}: Sensi protetti\\
\textbf{Essenza Favorita}: Illusione\\
\textbf{Essenza Sfavorita}: Difesa\\

\medskip

Accesso alla Scuole di Magia:\\
Scuole Privilegiate: Universale, Illusione (+2 alle prove di CM)\\
Scuole Normali: Evocazione, Abiurazione, Necromanzia, Invocazione, Trasmutazione, Divinazione, Ammaliamento (+0 alle prove di CM)\\

\bigskip

\bigskip


\subsubsection{Zarkor}

Aspetto: La Sofferenza\index{Zarkor}
\bigskip


I fedeli della dea Zarkor, sono dei veri e propri untori, si divertono a martoriare la gente, o diffondere malattie e sofferenze.

\begin{itemize}
	\item La dea gioisce della sofferenza di tutti gli esseri: infliggendo dolore il devoto si magnifica di fronte alla dea
	\item La corruzione della carne e della mente e' il normale stato della natura: portare corruzione, malattia e degrado e' la missione del fedele
	\item Nella malattia e nel dolore la mente si libera e si avvicina alla dea, infliggersi dolore e' grande offerta
\end{itemize}

L'unica cosa divertente di questo culto e vedere come siano cacciati da tutti gli altri fedeli, compresi i Koirani.\\


\textbf{Vantaggi}: Denti oppure Artigli\\
\textbf{Essenza Favorita}: Alterazione\\
\textbf{Essenza Sfavorita} Cura\\

\medskip

Accesso alla Scuole di Magia:\\
Scuole Privilegiate: Universale, Necromanzia (+2 alle prove di CM)\\
Scuole Normali: Evocazione, Abiurazione,  Illusione, Invocazione, Trasmutazione, Divinazione, Ammaliamento (+0 alle prove di CM)\\


\bigskip

\subsubsection{Eramide}

Aspetto: I Non Morti\index{Eramide}
\bigskip

I fedeli del dio Eramide sono i cultori della non vita, artisti nel riportare dall'oltretomba alla loro vita. dio molto amato da alcuni necromanti e' relativamente poco conosciuto e pregato.


\begin{itemize}
	\item Solo Eramide custodisce la chiave per l'ultimo volo
	\item Lui e' padrone della non vita, e abbraccia chi lui ama
	\item I non-morti sono le braccia del Dio, attraverso essi si raggiungera' la pace totale
	\item Lui e' portatore della vera vita, il fedele la anela
\end{itemize}

Fedeli dal tocco non-mortale ( che bella battuta...) sono pericolosi perche' spesso accompagnati dalle loro creature. \\


\textbf{Vantaggi}: Scudo Mentale\\
\textbf{Essenza Favorita}: Distruzione\\
\textbf{Essenza Sfavorita}: Creazione\\

\medskip

Accesso alla Scuole di Magia:\\
Scuole Privilegiate: Universale, Necromanzia (+2 alle prove di CM)\\
Scuole Normali: Evocazione, Abiurazione,  Illusione, Invocazione, Trasmutazione, Divinazione, Ammaliamento (+0 alle prove di CM)\\


\bigskip

\subsubsection{Averim}

Aspetto: La Bellezza\index{Averim}
\bigskip


I fedeli della dea Averim sono i belli, felici, puliti, puri...culto potente perché ammalia le menti delle persone, per fortuna poco diffuso e noto.

\begin{itemize}
	\item La bellezza e' potere
	\item Il fedele sa usare il potere per piegare al proprio volere chi dal potere e' soggiogato
	\item Amore e bellezza sono facce opposte di due medaglie diverse
	\item Il vero fedele sa usare ogni potere lui concesso per il piacere della Dea
\end{itemize}

Veri incantatori, meglio non guardarli negli occhi se non si vuole rimanere stregati e ridotti in schiavi.\\


\textbf{Vantaggi}: Senso della moda, Voce suadente\\
\textbf{Essenza Favorita}: Charme\\
\textbf{Essenza Sfavorita}: Convocazione\\

\medskip

Accesso alla Scuole di Magia:\\
Scuole Privilegiate: Universale, Ammaliamento (+2 alle prove di CM)\\
Scuole Normali: Evocazione, Abiurazione,  Illusione, Invocazione, Trasmutazione, Divinazione, Necromanzia (+0 alle prove di CM)\\


\bigskip

\subsubsection{Kyriel} 

Aspetto: La Natura\index{Kyriel}
\bigskip

I fedeli della dea Kyriel sono i protettori della Natura e di cio' che e' di Yeru. Molto del suo lavoro viene fatto direttamente nell'ambiente proteggendo la flora e la fauna.

\begin{itemize}
	\item La natura non e' crudele e' indifferente
	\item La natura non fa niente senza scopo
	\item Proteggi la natura e proteggerai te stesso
	\item Custodisci e proteggi la flora e la fauna
\end{itemize}

I fedeli sono per qualcuno selvaggi, per altri maestri di vita, per altri protettori di cio' che va protetto. I devoti di Kyriel possono essere stanziali e dedicati ad un luogo o girovaghi e dedicare la propria vita alla protezione della Natura.\\


\textbf{Vantaggi}: Animalia\\
\textbf{Essenza Favorita}: Trasformazione\\
\textbf{Essenza Sfavorita}: Distruzione\\

\medskip

Accesso alla Scuole di Magia:\\
Scuole Privilegiate: Universale, Evocazione (+2 alle prove di CM)\\
Scuole Normali: , Abiurazione, Illusione, Invocazione, Trasmutazione, Divinazione, Ammaliamento, Necromanzia (+0 alle prove di CM)\\

\bigskip

\subsubsection{Gli aspetti generici}\index{Divita' generici}

\medskip

Vengono anche presentati le divinità generiche, ovvero la possibilita' di dichiarare la propria fedelta' ad un aspetto generico, quale il Bene, il Male, la Legge, il Chaos.

Come per i Patroni anche gli aspetti generici concedono dei vantaggi e svantaggi alle Essenze, ogni Aspetto concede una Essenza privilegiata (+2 CM alle prove) e nega una Essenza (non utilizzabile).

Non e' necessario specificare la divinità adorata bensì solo se si e' fedele al Bene, Male, Chaos o Legge.

\medskip

\begin{tabular}{llll}
\toprule
\textbf{Divinta' Generica}	& \textbf{Vantaggio/Abilita' Concessa} & \textbf{Essenza Favorita} & \textbf{Essenza Negata} \\
Buona	& Incanalare Energia & Cura & Distruzione \\
Malvagia	&  Forgiato nella Furia &Distruzione  & Cura \\
Chaotica	&  Colpo Furtivo&Movimento  & Creazione \\
Legale & Colpi Poderosi & Creazione &  Movimento\\
\end{tabular}

\bigskip

In accordo con il Narratore, ed adeguatamente motivato, e' possibile cambiare Abilita' ed Essenze.

\medskip

Accesso alla Scuole di Magia:\\
Scuole Privilegiate: Universale, una a scelta (+2 alle prove di CM)\\
Scuole Normali: le rimanenti scuole di magia (+0 alle prove di CM)\\


\includepdf[pages={1},scale=0.95]{simboli-dei-antichi1.pdf}

\includepdf[pages={1},scale=0.95]{simboli-dei-antichi2.pdf}

\pagebreak

\section{Equipaggiamento}

\label{equipaggiamento}

\subsection{Ricchezza e Denaro}\index{Ricchezza e Denaro}


\begin{tcolorbox}[enhanced,arc=5pt,boxrule=0.3pt]{
- Doc... c'è soltanto bisogno di un pochino di plutonio.\\
- Ah, sono certo che nell'85 il plutonio si compra nella drogheria sotto casa, ma nel '55 la faccenda è molto più complicata! (Ritorno al futuro, Film 1985)}\end{tcolorbox}\medskip


\label{ricchezza-e-denaro}

Un personaggio che inizia a giocare generalmente ha monete d'oro sufficienti per acquistare gli elementi di base: qualche arma, un'armatura di seconda mano (quella meno costosa) ed un pò di attrezzatura varia. Man mano che il personaggio intraprende avventure e accumula bottino può permettersi un equipaggiamento migliore ed oggetti magici. Al primo livello i personaggi hanno monete ed equipaggiamento per un totale di 100 mo.

Inoltre, ogni personaggio inizia il gioco con un abito del valore di 10 mo o meno. Per personaggi di livello superiore al 1°, vedi \textbf{Tabella: Ricchezza dei personaggio per Livello}.

\medskip

\textbf{Vendere il Bottino}\index{Vendere il Bottino}: In generale, è possibile vendere qualsiasi cosa alla metà del prezzo indicato, comprese armi, armature, equipaggiamento, oggetti magici e oggetti creati dai personaggi. Le merci di scambio costituiscono l'eccezione alla regola del metà prezzo.

Una merce di scambio, in questo senso, è un bene di valore che può essere facilmente scambiato quasi come fosse equivalente ai contanti.

\subsubsection{Monete}\index{Monete}

La moneta più comune è la moneta d'oro (mo). Una moneta d'oro vale 10 monete d'argento (ma). Ogni moneta d'argento vale 10 monete di rame (mr). Oltre a monete di rame, argento e oro ci sono anche le monete di platino (mp), che valgono ognuna 10 monete d'oro.

\begin{tabular}{L{3cm} L{3cm} L{3cm} L{3cm} L{3cm}}
	\toprule
	\textbf{Valore di Cambio} & \textbf{Moneta Rame (mr)} & \textbf{Moneta Argento (ma)} & \textbf{Moneta Oro (mo)} & \textbf{Moneta Platino (mp)}\\
	Moneta Rame& 1& 1/10& 1/100& 1/1000\\
	Moneta Argento  & 10    & 1   & 1/10 & 1/100\\
	Moneta Oro & 100   & 10  & 1    & 1/10\\
	Moneta Platino  & 1000  & 100 & 10   & 1\\
\end{tabular}

\subsubsection{Altre Ricchezze}\index{Altre Ricchezze}

I mercanti di solito scambiano merci anche senza l'uso di monete.
Per farsi un'idea delle transazioni commerciali, alcune merci di scambio sono descritte nella tabella.

\medskip

\begin{tabular}{ll}
	\toprule
	\textbf{Costo} & \textbf{Oggetto}\\
	1 mr & Frumento (0.5 kg)\\
	2 mr & Farina (0.5 kg) o pollo (1)\\
	1 ma & Ferro (0.5 kg)\\
	5 ma & Tabacco o rame (0.5 kg)\\
	1 mo & Cannella (0.5 kg) o capra (1)\\
	2 mo & Zenzero o pepe (0.5 kg) o pecora (1)\\
	3 mo & Maiale (1)\\
	4 mo & Lino (1 m\textsuperscript{2})\\
	5 mo & Sale o argento (0.5 kg)\\
	10 mo& Seta (1 m) o mucca (1)\\
	15 mo& Zafferano o chiodi di garofano (0.5 kg) o bue (1)\\
\end{tabular}

\medskip

Consultate anche il capitolo sull'Ingombro in Movimento e Trasporto.

\pagebreak

\section{Equipaggiamento - Armi}\index{Equipaggiamento}\index{Armi}

\label{equipaggiamento---armi}
\begin{tcolorbox}[enhanced,arc=5pt,boxrule=0.3pt]{
Questo è il mio fucile. Ce ne sono tanti come lui, ma questo è il mio. Il mio fucile è il mio migliore amico, è la mia vita. Io debbo dominarlo come domino la mia vita. Senza di me il mio fucile non è niente; senza il mio fucile io sono niente. Debbo saper colpire il bersaglio, debbo sparare meglio del mio nemico che cerca di ammazzare me, debbo sparare io prima che lui spari a me e lo faro'. Al cospetto di Dio giuro su questo credo: il mio fucile e me stesso siamo i difensori della patria, siamo i dominatori dei nostri nemici, siamo i salvatori della nostra vita e così sia, finché non ci sarà più nemico ma solo pace, amen.
\\
(Full Metal Jacket, Film, 1987)}\end{tcolorbox}\medskip

Ricordo che usare un'Arma senza l'adeguata competenza impone un -2d6 al colpire

La tabella presenta il nome dell'arma, il suo costo in monete d'oro, il danno ed il tipo di danno (se da Taglio, Botta o Punta), la gittata, la Lista d'Arma appartenente e le caratteristiche speciali che puo' avere. \hyperref[sec:Azioni particolari in combattimento]{Vedi sezione Azioni Particolari in Combattimento}, vedi anche \hyperref[sec:capacita-di-carico-e-trasporto-ingombro]{Capacita' di Carico e Trasporto.}

\bigskip

\begin{tabularx}{1\textwidth}{lccXc}
	\textbf{Nome Lista Arma}& \textbf{Costo} & \textbf{Danno/Tipo} & \textbf{Gittata,Speciale} & Ingomb.\\
	
Alabarda& 10 & 1d10 P/T& \textbf{Lance}, \textbf{Aste}, Controcarica, Arma lunga, ED9 & 2\\
Arco Corto& 30 & 1d6 P& 15 metri, \textbf{Arco}, da tiro& 1\\
Arco Corto Composito& 75 & Frecce& 20 metri, \textbf{Arco}, da tiro& 1\\
Arco Lungo& 75 & Frecce& 20 metri, \textbf{Arco}, da tiro& 2\\
Arco Lungo Composito& 110& Frecce& 36 metri, \textbf{Arco}, da tiro& 2\\
Ascia ad una mano& 6  & 1d6 T& 6 metri, \textbf{Asce}, \textbf{Armi da Tiro}, Versatile& 1\\
Ascia da battaglia& 10 & 1d10 T&\textbf{Asce}& 1\\
Ascia Martello& 16 & 1d6/1d8 T/B& \textbf{Asce}& 1\\
Balestra ad una mano& 100& Dardi& 12 metri, \textbf{Balestre}, da tiro& 1\\
Balestra leggera& 35 & Dardi& 15 metri, \textbf{Balestre}, \textbf{Armi Semplici}, da tiro& 1\\
Balestra leggera (Ric.) & 250& Dardi & 6 metri, \textbf{Balestre}, da tiro, 6 cariche& 1\\
Balestra pesante& 50 & Dardi& 20 metri \textbf{Balestre}, da tiro& 2\\
Balestra pesante (Ric.) & 400& Dardi& 12 metri, \textbf{Balestre}, da tiro, 8 cariche& 2\\
Bastone& 3& 1d6 B& \textbf{Armi doppie}, \textbf{Armi Semplici}, Arma lunga, Versatile& 1\\
Bolas& 4& 1d3 B&\textbf{Intralcianti}, intralciato& L\\
Brandistocco& 10 & 2d4 P/T& \textbf{Lance}, Controcarica, Arma lunga& 2\\
Catena chiodata& 25 & 2d4 P& \textbf{Palle rotanti}, Arma lunga& 2\\
Falce& 18 & 2d4 P/T& \textbf{Armi della Morte}, Arma lunga& 2\\
Falcetto& 6& 1d6 T& \textbf{Armi della Morte} & L\\
Falcione& 75 & 2d4 T& \textbf{Armi Aggraziate}, \textbf{Lance}, ED7& 1\\
Falcione in asta& 12 & 1d10 P/T& \textbf{Lance}, Controcarica, Arma lunga, ED9& 2\\
Fionda& -& 1d4 B& 10 metri, \textbf{Archi}, da tiro& L\\
Flagello& 8& 1d8 B& \textbf{Armi da Carceriere}, \textbf{Rompi Cranio}& 1\\
Flagello Doppio& 90 & 1d10 B& \textbf{Armi doppie}, \textbf{Armi da Carceriere}& 2\\
Flagello Pesante& 15 & 1d10 B& \textbf{Armi Doppie}, \textbf{Armi da Carceriere}& 2\\
Frusta& 1& 1d3 T& \textbf{Armi da Carceriere}, \textbf{Palle Rotanti}, Arma lunga& 2\\
Giavellotto& 1& 1d6P& 12 metri, \textbf{Aste}, \textbf{Armi Semplici}& L\\
Grande Ascia Doppia& 25 & 1d12 T& \textbf{Asce}, \textbf{Armi doppie}, Arma lunga& 2\\


\end{tabularx}

\bigskip

\begin{tabularx}{0.95\textwidth}{lccXc}
	\textbf{Nome Lista Arma}& \textbf{Costo} & \textbf{Danno/Tipo} & \textbf{Gittata,Speciale} & Ingomb.\\
	Grosso randello& 2& 1d8 B&\textbf{Rompi Cranio}& 2\\
	Guanto chiodato& 5& 1d4 P&\textbf{Armi da Stordimento}& 1\\
	Katana& 300& 1d10 T& \textbf{Spade}, ED9, Versatile& 1\\
	Lancia& 10 & 1d8 P&\textbf{Lance}, Arma lunga, Controcarica& 2\\
	Lancia corta da fante& 1& 1d6 P& 6 metri, \textbf{Armi da tiro},\textbf{ Armi Semplici},Versatile & 1\\
	Lancia da fante& 2& 1d8 P&6 metri, \textbf{Lance}, \textbf{Aste}, Arma lunga, Controcarica& 1 \\
	Machete& 10 & 1d6 T&\textbf{Armi letali} & 1\\
	Manganello& 1& 1d6 B& \textbf{Armi da stordimento}, non letale& 1\\
	Martello da guerra& 5& 1d8 B/P& 6 metri, \textbf{Rompi Cranio}& 1\\
	Martello Leggero& 3& 1d6 B/P& 6 metri, \textbf{Rompi Cranio}, \textbf{Armi da tiro}, \textbf{Armi Leggere}, Versatile & 1\\
	Mazza Leggera& 3& 1d6 B/T& \textbf{Armi Leggere}, \textbf{Rompi Cranio}, \textbf{Armi Semplici}, Versatile&1\\
	Mazza Pesante& 5& 1d8 B/T& \textbf{Rompi Cranio}& 1\\
	Morningstar& 6& 1d8 B/P&\textbf{Rompi Cranio},\textbf{ Armi Semplici}& 1\\
	Naginata& 8& 1d10 T&\textbf{Lance}, Arma lunga, ED9& 2\\
	Picca Leggera& 4& 1d4 P&\textbf{Armi della morte}& 1\\
	Picca Pesante& 8& 1d6 P&\textbf{Armi della morte}, Arma lunga& 2\\
	Pugnale& 2& 1d4 P& 6 metri, \textbf{Armi leggere}, \textbf{Armi da tiro}, \textbf{Armi letali}, \textbf{Armi Semplici}, Versatile& L\\
	Pugno/Calcio nudo& 0& 1d4* B&Versatile& -\\
	Randello& 1& 1d6 B&\textbf{Rompi Cranio}, \textbf{Armi Semplici}& 1\\
	Rete& 8& -&\textbf{Bloccanti}, intralciato& 1\\
	Scimitarra& 15 & 1d6 T&\textbf{Armi Leggere}, \textbf{Armi Aggraziate}, Versatile& 1\\
	Spada a due lame& 100& 1d8 T& \textbf{Armi Doppie}, \textbf{Spade}, Arma lunga, arma doppia& 2\\
	Spada bastarda& 35 & 1d10 T&\textbf{Spade}& 2\\
	Spada Corta& 10 & 1d 6P&\textbf{Armi Leggere}, \textbf{Spade}, Versatile& L\\
	Spada Lunga& 15 & 1d8 T&\textbf{Spade}& 1\\
	Spadone a due mani& 50 & 2d6 T&\textbf{Spade}& 2\\
	Stocco& 20 & 1d6 P& \textbf{Armi Leggere}, \textbf{Armi Aggraziate}, Versatile& 1\\
	Tridente& 15 & 1d8 P/T& 3 metri, \textbf{Lance}, \textbf{Aste}, \textbf{Armi da tiro}, Arma Lunga, Controcarica& 2\\
	Urgrosh& 18 & 1d6/1d8 T/P& \textbf{Armi Doppie}, \textbf{Lance}, Controcarica, Arma lunga & 2\\
\end{tabularx}

\bigskip

\begin{tabularx}{0.95\textwidth}{lccXc}	
	Biglie di Marmo (fionde)& 15/1 mo & 1d4 B & & L\\
	Dardi da balestra leggeri & 10/1 mo & 1d6 P & & L\\
	Dardi da balestra pesanti & 5/1 mo & 1d8 P & & L\\
	Dardi per balestra pesante & 3/1 mo & 1d10 P & & L\\
	Dardi per balestra leggera & 5/1 mo & 1d4 P & & L\\
	Frecce da caccia& 20/1 mo & 1d6 P& &L\\
	Frecce da guerra& 10/1 mo & 1d8 P& &L\\
	Sasso (fionde)& -& 1d2 B& &-\\
\end{tabularx}

\medskip

\textbf{Pugno Nudo}: Ogni volta che prendi questa competenza il danno aumenta seguendo questa progressione: 1d6 (presa la lista 2 volte), 1d8 (5), 2d6 (7), 2d8 (9), 2d10 (11), 3d6 (13), 3d8 (15), 3d10 (17), 4d6 (19)...

\bigskip

\textbf{Gittata}\index{Gittata}
La distanza indicata è quello a pieno Tiro per Colpire. Ogni arma a distanza può colpire  fino a due volte la distanza indicata. Se il target è entro la distanza indicata non si hanno malus al colpire, se il target è tra il primo e secondo incremento il malus al colpire è -1d6. Se il target è tra il secondo è terzo incremento il malus al colpire è di -2d6.

Un giavellotto tirato entro 12 metri non ha malus, ma tirato entro 24 metri ha un -1d6 al colpire, a distanza di 36 metri un -2d6 al colpire.

\medskip


Una \textbf{Freccia o Dardo che colpisce si considera distrutta}, se manca si considera che abbia un 50\% (4-5-6 su un d6) di probabilita' che sia ancora integra.

\medskip

Una \textbf{arma di taglia superiore} \index{Arma di taglia superiore} come ad esempio una Spada Lunga forgiata per un Ogre aumenta di una categoria il suo dado di danno (1d4-1d6-1d8-1d10-2d6-2d8-2d10..)

\medskip

Attaccare con un'\textbf{Arma troppo grande} \index{Arma troppo grande}rispetto alla propria taglia impone un -1d6 al Tiro per Colpire per ogni taglia di differenza tra arma e personaggio.

\medskip

Le Armi hanno tutte segnato una \textbf{ipologia di danno}\index{Tipologia di danno}, ovvero T/B/P.
Queste lettere stanno ad indicare se il danno è di tipo Taglio, Botta o da Penetrazione. Questa caratteristica può essere importante perché 	determinate creature possono essere immuni o subire meno danno da un particolare tipo di ferita (es uno scheletro contro un'arma da penetrazione o un cubo gelatinoso contro un arma da taglio..)

\medskip

\textbf{Armi Improvvisate}\index{Armi Improvvisate}

Talvolta oggetti che non sono stati creati per essere armi possono avere una certa efficacia in combattimento. Dal momento che non si tratta di oggetti pensati per questo utilizzo, la creatura che attacca con uno di essi subisce una penalità -1d6 al Tiro per Colpire. Un'arma improvvisata di piccole dimensioni (bottiglia) fa 1d3 di danno, di medie dimensioni (la gamba di una sedia) da 1d6, di grandi dimensioni (la gamba di un tavolo) fa 1d8 di danno.

Un'arma da lancio improvvisata ha una gittata 3 metri.

\medskip

\textbf{Lanciare armi}\index{Lanciare armi}

Una spada o comunque un arma non fatta per essere lanciata può comunque essere scagliata contro l'avversario. Il Tiro per Colpire prende un -1d6 e l'arma fa una categoria di danno inferiore (la spada lunga fa 1d6, una spada corta 1d4..). La gittata è 3 metri.

\medskip

\textbf{Usare un'Arma senza l'adeguata competenza se non e' un Arma Semplice}
Impone un -2d6 al Tiro per Colpire.

\medskip

\pagebreak

\subsection{Equipaggiamento - Armature e Scudi}\index{Armature}\index{Scudi}

\label{equipaggiamento---armature-e-scudi}

Le armature aiutano ad essere non colpiti (alzano la Difesa), riducono il danno subito{*} (Bonus protezione) e penalizzano la prova di check magia e le prove di competenza basate su Agilità.

Allo stesso tempo le prove di Agilità saranno fatte con una penalità di 3 ed il movimento diminuirà di due metri per Azione, fino ad un minimo di 0.

Quando una armatura si danneggia (è a 0 di resistenza), il suo Bonus di Protezione al colpo ed il Bonus di Difesa diminuiscono di 3 (con un minimo di 0), quando raggiunge un valore negativo pari a resistenza totale (es -20 per un armatura leggera) è a brandelli e non può più proteggere (ne essere riparata).

Armature diverse, specifiche o magiche hanno punteggio diversi, questa tabella serve come linea guida per il Narratore.

Quasi tutte le armature, ad eccezione della Imbottita penalizzano l'uso di certe competenze basate su Agilità.

La Prove Agilità è la penalità che si applica alle prove di competenze di Agilità mentre si indossa un quel tipo di armatura

\subsubsection{Tabella Armature}
\medskip

\label{tabella-armature}
%\begin{tabular}{llllllllll}
\begin{tabularx}{0.95\textwidth}{lXXllXXXXX}
	\toprule
	\textbf{Armatura} & \textbf{Costo} & \textbf{Difesa} & \textbf{Protezione} & \textbf{Resistenza} & \textbf{Prove Agilità} & \textbf{Prove CM} & \textbf{Tipo} & \textbf{Mov.} & \textbf{Ingombro}\\
	Imbottita    & 5    & 1& 0 & 10   & 0   & 0  & L   & 0   & 1\\
	Cuoio   & 10   & 2& 1 & 20   & 0   & -1 & L   & 0   & 1\\
	Cuoio rinforzato  & 25   & 3& 1 & 25   & 0   & -2 & L   & 0   & 2\\
	Giaco di Maglia   & 15   & 4& 2 & 30   & -1  & -3 & M   & 0   & 2\\
	Scaglie & 50   & 5& 2 & 40   & -1  & -4 & M   & 0   & 1\\
	Anelli  & 150  & 6& 3 & 35   & -1  & -5 & M   & 0   & 2\\
	Pettorale    & 200  & 6& 3 & 40   & -2  & -5 & M   & 0   & 2\\
	Bande   & 250  & 7& 4 & 50   & -2  & -6 & P   & 0   & 2\\
	Mezza armatura    & 1200 & 8& 4 & 55   & -2  & -7 & P   & 1   & 3\\
	da Campo& 1400 & 9& 5 & 60   & -3  & -7 & P   & 2   & 3\\
	Completa& 1500 & 10    & 6 & 70   & -4  & -8 & P   & 3   & 4\\
\end{tabularx}

\smallskip

\textbf{Costo}: e' espresso in Monete d'Oro.

\textbf{Difesa}: e' il bonus data alla Difesa

\textbf{Protezione e Resistenza}: sono opzionali. Il Narratore può decidere di non fare tenere traccia dei danni subiti dall'armatura o quanto assorbe dei colpi.

\textbf{Prove Agilita'}: e' il malus dato alle prove di Agilità dato dal peso ed ingombro dell'armatura.

\textbf{Prove CM}: e' il malus dato alle prove di Competenza Magica.

\textbf{Tipo}: indica se l'armature a' \textbf{L}eggera, \textbf{Me}dia oppure \textbf{P}esante

\textbf{Movimento}: e' la riduzione di movimento da applicare per Azione di Movimento.

\textbf{Ingombro}: e' l'ingombro dell'armatura da conteggiare

\bigskip

\textbf{Usare un'Armatura senza l'adeguata competenza} impedisce di usare il bonus di Agilità

\textbf{Usare uno Scudo senza l'adeguata competenza} peggiora il Tiro per Colpire di 1 e diminuisce di 1 il Bonus Difesa concesso.

\textbf{Dormire in Armatura}: se si dorme in un'armatura media o pesante, il giorno seguente si è automaticamente Affaticati. Si subisce penalità -1 a Potenza e Agilità e non si può Caricare o Correre.

Dormire in un'armatura leggera non provoca Affaticamento.

La \textbf{capacità di movimento} del personaggio rimarrà la medesima fino all'armatura a bande poi calerà progressivamente. Il valore indicato nella colonna Mov. sono i metri in meno che il personaggio fa per Azione di Movimento.

Ad Esempio un umano in armatura completa ha movimento 6 metri, un nano 3 metri.

\textbf{Peso}: il peso indicato si riferisce alla versione per personaggi di taglia Media. Le armature adattate per personaggi di taglia Piccola pesano la metà, mentre per quelli di taglia Grande pesano il doppio.

\pagebreak

Gli \textbf{Scudi} \index{Scudi}permettono di aumentare la propria Difesa, più lo scudo è imponente e pesante più protegge, più aumentano le penalità alle prove di competenza magica, meno rende facile combattere (penalità Tiro per Colpire).

Gli Scudi possono essere di tipo Leggero, Medio, Pesante.

\subsubsection{Tabella Scudi}

\label{tabella-scudi}
\medskip
\begin{tabular}{lllllll}
	\toprule
	\textbf{Scudi} & \textbf{Costo (MO)} & \textbf{Bonus Difesa} & \textbf{Penalità TC} & \textbf{Penalità CM} & \textbf{Ingombro} & \textbf{Tipo}\\
	Brocchiero& 5    & 1 & 0& 1& 1  & L\\
	Scudo leggero di legno   & 3    & 2 & 0& 2& 1  & L\\
	Scudo leggero di metallo & 9    & 2 & 0& 3& 2  & L\\
	Scudo medio legno   & 5    & 3 & -1    & 4& 2  & M\\
	Scudo medio metallo & 12   & 3 & -1    & 5& 2  & M\\
	Scudo pesante di legno   & 7    & 4 & -2    & 6& 3  & P\\
	Scudo pesante di metallo & 20   & 4 & -2    & 8& 3  & P\\
\end{tabular}

Uno scudo ha Resistenza totale pari a 20 volte il suo bonus di Difesa (Opzionale).

\bigskip

Un Armatura o Scudo magico aggiunge un +20 alla resistenza totale
per ogni +1 magico posseduto, oltre ad eventuali bonus aggiuntivi
alla Difesa

\bigskip
\textbf{Indossare e Togliere Armature}
\bigskip

Indossare e togliere armature è una operazione che richiede tempo ed attenzione, farlo in fretta non aiuta ed anzi tende a peggiorare la protezione data dall'armatura.

\bigskip

\begin{tabular}{llll}
	\toprule
	\textbf{Tipo di Armatura}& \textbf{Indossare} & \textbf{Indossare in fretta} & \textbf{Togliere}\\
	Scudo& 1 azione & -   & 1 azione\\
	Imbottita, Cuoio, Cuoio rinforzata   \\
	Giaco di Maglia& 1 minuto & 5 round  & 5 round\\
	Scaglie, Anelli, Pettorale, Bande  & 4 minuti & 1 minuto{*}   & 1 minuto\\
	Mezza armatura, da Campo, Completa & 4 minuti{*}{*}& 4 minuti{*}   & 1d4+1 minuti\\
\end{tabular}

\bigskip

{*} Se qualcuno aiuta, il tempo si dimezza. Un singolo personaggio che non sta facendo altro può aiutare uno o due personaggi adiacenti a lui. Due personaggi non possono aiutarsi l'un l'altro a indossare un'armatura contemporaneamente.

	{*}{*} Bisogna essere aiutati per indossare questa armatura. Senza aiuto è possibile indossarla solo in fretta.

Indossare un'armatura in fretta implica un malus di -1 al Bonus di protezione, Difesa ed un malus aggiuntivo di +1 alle prove di Agilità

\pagebreak

\section{Merci e Servizi}\index{Merci}\index{Servizi}

\label{merci-e-servizi}

Oltre ad armi e armature, un personaggio può avere una notevole varietà di attrezzature a disposizione, dalle razioni da viaggio, alle corde (che possono essere utili in molte circostanze).


\subsection{Equipaggiamento d'Avventura}\label{Equipaggiamento}\index{Equipaggiamento}

\label{equipaggiamento-davventura}



\begin{tabularx}{1\textwidth}{XllXll}
%	\tabcolsep=0.11cm
	\textbf{Oggetto}    & \textbf{Costo} & \textbf{Ingom.} & \textbf{Oggetto}    & \textbf{Costo}  & \textbf{Ingom.} \\
	Acciarino e pietra focaia   & 1 mo& —   & Ago da cucito & 5 ma& —   \\
	Amo da pesca  & 1 ma& —   & Ampolla (vuota)& 3 mr& L   \\
	Anello con sigillo  & 5 mo& —   & Ariete portatile    & 10 mo& 2   \\
	Asta (3 m)    & 5 mr& 1   & Barile (vuoto)& 2 mo& 1   \\
	Boccale di ceramica & 2 mr& L   & Boccetta di inchiostro o pozione  & 1 mo& —   \\
	Borsa da cintura (vuota)    & 1 mo& L   & Bottiglia di vetro  & 2 mo& L   \\
	Brocca di ceramica  & 3 mr& 1   & Campanella    & 1 mo& —   \\
	Candela & 1 mr& —   & Cannocchiale  & 1.000 mo  & L   \\
	Caraffa di ceramica & 2 mr& L   & Carrucola e paranco & 5 mo& L   \\
	Carta (foglio)& 4 ma& —   & Cassa (vuota) & 2 mo& 1   \\
	Catena (3 m)  & 30 mo    & 1   & Ceralacca& 1 mo& -   \\
	Cesto (vuoto) & 4 ma& 1   & Chiodo da rocciatore& 1 ma& L   \\
	Clessidra& 25 mo    & L   & Coperta invernale   & 5 ma& 1   \\
	Corda di canapa (15 m)& 1 mo& 1   & Corda di canapa grossa (15 m)& 2 mo& 1   \\
	Corda di seta (15 m)& 10 mo    & L   & Cote per affilare   & 2 mr& L   \\
	Custodia per mappe o pergamene    & 1 mo& L   & Fischietto    & 8 ma& —   \\
	Gessetto, (1 pezzo) & 1 mr& —   & Giaciglio& 1 ma& 1   \\
	Inchiostro (boccetta da 30 g)& 8 mo& —   & Lampada comune& 1 ma& L   \\
	Lanterna a lente sporgente  & 12 mo    & L   & Lanterna schermabile& 7 mo& L   \\
	Legna da ardere (per giorno)& 1 mr& 2   & Maglio& 1 mo& 2   \\
	Manette & 15 mo    & L   & Manette perfette    & 50 mo& L   \\
	Martello& 5 ma& 1   & Olio (ampolla da 0,5 l)& 1 ma& L   \\
	Orologio ad acqua   & 1.000 mo & 10  & Otre  & 1 mo& 1   \\
	Pennino & 1 ma& —   & Pentola di ferro    & 8 ma& L   \\
	Pala o badile & 2 mo& 1   & Pergamena (Foglio)  & 2 ma& —   \\
	Piccone da minatore & 3 mo& 1   & Piede di porco& 2 mo& 1   \\
	Rampino & 1 mo& L   & Razioni da viaggio (al giorno)    & 5 ma& L   \\
	Rete da pesca (2,25 m)& 4 mo& L   & Sacco (vuoto) & 1 ma& L   \\
	Sapone (per 0,5 kg) & 5 ma& L   & Scala a pioli (3 m) & 2 ma& 2   \\
	Secchio (vuoto)& 5 ma& L   & \\
	Serratura/lucchetto Semplice& 20 mo    & L   & Serratura/lucchetto Media & 40 mo& L   \\
	Serratura/lucchetto Buona & 80 mo    & L   & Serratura/lucchetto Superiore& 150 mo    & L   \\
	Specchio piccolo di metallo & 10 mo    & L   & Tela (m2)& 1 ma& L   \\
	Tenda & 10 mo    & 1   & Torcia& 1 mr& L   \\
	Tribolo & - mo& L   & Zaino & 2 mo& L   \\
\end{tabularx}
\bigskip

\begin{tabularx}{0.95\textwidth}{XllXll}
	\textbf{Oggetti e sostanze speciali}&    & & \textbf{Oggetti e sostanze speciali}&& \\
	Acido (ampolla)   & 10 mo    & L& Acqua santa (ampolla)   & 25 mo& L\\
	Antitossina (boccetta)  & 50 mo    & —& Bastone di fumo   & 20 mo& L\\
	Borsa dell’impedimento  & 50 mo    & L& Fuoco dell’alchimista (ampolla)& 20 mo& L\\
	Pietra del tuono  & 30 mo    & L& Tizzone ardente   & 1 mo& —\\
	Torcia inestinguibile   & 110 mo   & L& Verga del sole    & 2 mo& L\\
	\textbf{Attrezzi di classe e di Abilita'} &    & & \textbf{Attrezzi di classe e di Abilita'} && \\
	Agrifoglio e vischio    & —  & —& Arnesi da artigiano& 5 mo& L\\
	Arnesi da artigiano perfetti  & 55 mo    & 1& Arnesi da scasso  & 30 mo& L\\
	Arnesi da scasso perfetti& 100 mo   & 1& Attrezzi da scalatore   & 80 mo& L\\
	Attrezzi perfetti & 50 mo    & L& Bilancia da mercante    & 2 mo& L\\
	Borsa del guaritore& 50 mo    & L& Laboratorio da alchimista& 200 mo    & 2\\
	Lente d’ingrandimento   & 100 mo   & —& Simbolo sacro d’argento & 25 mo& L\\
	Simbolo sacro di legno  & 1 mo& L& Strumento musicale comune& 5 mo& 1\\
	Strumento musicale perfetto   & 100 mo   & 1& Trucchi per il camuffamento   & 50 mo& L\\
	\textbf{Vestiario}&    & & \textbf{Vestiario}&& \\
	Abito da artigiano& 1 mo& 1& Abito da contadino& 1 ma& L\\
	Abito da cortigiano& 30 mo    & 1& Abito da esploratore    & 10 mo& 1\\
	Abito da intrattenitore & 3 mo& 1& Abito da Monaco   & 5 mo& 1\\
	Abito da nobile   & 75 mo    & 1& Abito da studioso & 5 mo& 1\\
	Abito da viaggiatore    & 1 mo& 1& Abito invernale   & 8 mo& 1\\
	Abito reale & 200 mo   & 1& Veste da Chierico & 5 mo& 1\\
	\textbf{Vitto e alloggio}&    & & \textbf{Vitto e alloggio}&& \\
	Banchetto (a persona)   & 10 mo    & —& Birra Boccale& 4 mr& L\\
	Birra Caraffa& 2 ma& 1& Carne (1 pezzo)   & 3 ma& L\\
	Formaggio (1 pezzo)& 1 ma& L&   Pane (a pagnotta) & 2 mr& L\\
	\textbf{Locanda (al giorno)} &&&\textbf{Locanda (al giorno)}&&\\
	Buona & 2 mo& —& Normale& 5 ma& —\\
	Scadente    & 2 ma& —&&&\\
	\textbf{Pasti (al giorno)}    &    & & \textbf{Pasti (al giorno)}    && \\
	Buono & 5 ma& —& Normale& 3 ma& —\\
	Scadente    & 1 ma& —& Vino  && \\
	Comune (caraffa)  & 2 ma& 1& Buono (bottiglia) & 10 mo& L\\
	\textbf{Cavalcature e relativo Equipaggiamento} &    & & \textbf{Cavalcature e relativo Equipaggiamento} && \\
	Asino o mulo& 8 mo& -& Bardatura   && \\
	Cane da galoppo   & 150 mo   & —& Cane da guardia   & 25 mo& —\\
	Cavallo leggero   & 75 mo    & —& Cavallo leggero (addestrato)  & 110 mo    & —\\
	Cavallo pesante   & 200 mo   & —& Cavallo pesante (addestrato)  & 300 mo    & —\\
	Pony  & 30 mo    & —& Pony (addestrato) & 45 mo& —\\
	Morso e briglie   & 2 mo& L& Nutrimento (al giorno)  & 5 mr& 1\\
	Sacche da sella   & 4 mo& 1& && \\
	Sella Da carico   & 5 mo& 1& Sella Da galoppo  & 10 mo& 1\\
	Sella Militare    & 50 mo    & 2& Sella esotica& 40 mo& 2\\
	Sella Da carico   & 15 mo    & 2& Sella Da galoppo  & 30 mo& 2\\
	Stallaggio (al giorno)  & 5 ma& —& && \\
	\textbf{Trasporti}&    & & \textbf{Trasporti}&& \\
	Barca a remi& 50 mo    & 3& Barcone& 3.000 mo  & —\\
	Carretto    & 15 mo    & 4& Carro & 35 mo& 6\\
	Carrozza    & 100 mo   & 7& Galea & 30k mo    & —\\
	Nave a vela & 10k mo   & —& Nave da guerra    & 25k mo    & —\\
	Nave lunga  & 10k mo   & —& Remo  & 2 mo& 1\\
	Slitta& 20 mo    & 1& && \\
\end{tabularx}
\bigskip

\begin{tabularx}{0.95\textwidth}{XllXll}
	\textbf{Servizio} & \textbf{Costo} & & \textbf{Servizio} & \textbf{Costo}  & \\
	Diligenza pubblica& 3 mr/1,5 Km    & & Messaggero  & 5 mr per 1,5 Km & \\
	Mercenario esperto& 8 mo al giorno & & Mercenario normale& 1 mo al giorno  & \\
	Pedaggio stradale o d’ingresso& 1 ma& & Passaggio in nave & 1 ma per 1,5 Km & \\
\end{tabularx}


\bigskip


\textbf{Acciarino e Pietra Focaia}: 1 mo, Accendere una torcia con acciarino e pietra focaia costa 3 Azioni e accendere qualsiasi altro fuoco in questo modo richiede almeno altrettanto tempo.

\textbf{Ago da Cucito}: 5 ma, arnese in acciaio usato per cucire, filiforme, appuntito ad un'estremità e munito all'altra di un forellino ovale (cruna), nel quale si fa passare il filo

\textbf{Alveare da Viaggio}: 10 mo, questi cesti di paglia forniscono una casa portatile alle api. Sono a forma di cupola, con un buco sulla sommità coperto da un cestino con maglie più strette, a mò di tappo. Questo buco permette di raccogliere piccole quantità di miele senza distruggere l'intero alveare. Alcuni agricoltori pagano gli apicoltori per viaggiare fino alle loro fattorie con delle api in modo che queste ultime possano impollinare le loro colture.

Distruggere un alveare da viaggio fa sciamare le api in una nube di 3 metri. Una creatura resta Accecata fintanto che rimane nella nube, e deve superare un Tiro Salvezza su Tempra con DC 12 o diventa Inferma per 1 minuto. La condizione Infermo è un effetto di Veleno.

\textbf{Amo da Pesca}: 1 ma,-, piccolo uncino metallico con due punte divergenti, sul quale si infila l'esca per far abboccare il pesce alla lenza.

\textbf{Ampolla (vuota)}: 3 mr, L, piccola anfora in vetro o ceramica con una sola ansa e collo sottile terminante in un beccuccio.

\textbf{Anello con Sigillo:} 5 mo,-, cerchietto di metallo, generalmente pregiato, con un'incisione atta ad imprimere sigilli su ceralacca.

\textbf{Anello per Veleno:} +20 mo rispetto a costo anello, questo anello ha un piccolo scompartimento sotto la gemma, di solito utilizzato per contenere del veleno. Aprirlo e chiuderlo richiede un'azione; farlo senza essere notati richiede una prova di Rapidità di Mano con DC 20.

\textbf{Ariete Portatile:}10 mo, 1, questa trave di legno rivestita di metallo fornisce bonus +2 alle prove di Potenza per sfondare porte, ma permette a una seconda persona di Aiutare senza dover effettuare alcun tiro, aggiungendo un altro +2 alla prova.

\textbf{Barile (vuoto):} 2 mo, 2/4 (contiene circa 115 lt), recipiente a forma di cilindro allargato al centro, fatto di doghe di legno tenute insieme da cerchi di ferro.

\textbf{Biglie}: 1 ma, L, come i triboli, le biglie possono rallentare gli avversari. Un sacco di biglie pesante 1 kg può coprire 1 metro quadro. Una creatura che entri nella zona piena di biglie deve effettuare un Tiro Salvezza su Riflessi con DC 10 o cade Prona. Una creatura che si muove a metà della sua velocità o più lentamente può attraversare una zona piena di biglie senza problemi.

\textbf{Boa Comune:} 5 ma, 1, una boa viene utilizzata per segnare un determinato punto nell'acqua, permettendo di ritornarvi successivamente. E' formata da un galleggiante (una vescica piena d'aria o una zucca sigillata), una cima copre una lunghezza di 60 metri ed una pietra di 20 kg come ancora. Il galleggiante è solitamente dipinto con colori sgargianti ed ha una bandierina per attirare l'attenzione. Anche se le boe resistono alle correnti ed al tempo, offrono ben poca opposizione alle creature intelligenti che desiderano sabotarle.

\textbf{Boa Superiore:} 10 mo, 2, questa boa ha un galleggiante tondo o ovoidale, solitamente di rame, una catena invece che una cima ed un'ancora di metallo invece di un peso. Per il resto è come una boa normale.

\textbf{Boccale di Ceramica}: 2 mr, L, bicchiere alto e largo con manico e, in alcuni casi, con beccuccio.

\textbf{Boccetta o Fiala}, 1 mo, L, un contenitore di vetro o metallo che contiene 30 grammi di liquido (genericamente puo' contenere 1 dose)

\textbf{Borsa da Cintura (vuota):} 1 mo, L, custodia, a forma di sacchetto di varie fogge, in pelle, in stoffa ecc., in cui si trasportano denaro, cose personali, oggetti vari.

\textbf{Bottiglia di Vetro:} 2 mo, L, recipiente per liquidi in vetro con corpo generalmente cilindrico e collo di diametro notevolmente più piccolo, che può essere chiuso da un tappo.

\textbf{Brocca di Ceramica:} 3 mr, L, una semplice brocca di ceramica chiusa con un tappo che contiene 4,5 litri di liquido.

\textbf{Calumet:} 20 mo, 1, un calumet è una pipa cerimoniale in due pezzi con un fornello fatto di pietra o argilla e un cannello di legno con intricate incisioni decorato con feticci appesi. La pipa viene solitamente trasportata in una speciale sacca di cuoio abbellita con perle, decorazioni e ciondoli. La pipa viene usata per fumare varie misture a base di erbe richieste per certi rituali. Fumare collettivamente un calumet a volte rientra negli incontri diplomatici come segno di solidarietà tra i diversi gruppi. Si ottiene Bonus +1 alle prove di Diplomazia (Faccia Tosta) effettuate contro chiunque con cui si abbia fumato assieme il calumet in questo modo.

\textbf{Campanella:} 1 mo,-, piccola campana con batacchio interno, suonata tirando una fune o scuotendola con la mano.

\textbf{Candela:} 1 mr, -, una candela Illumina con luce tenue un'area di mishcia. Una candela non può aumentare il livello della luce oltre quella normale. Una candela brucia per 1 ora.

\textbf{Cannocchiale:} 1.000 mo, L, gli oggetti visti attraverso un cannocchiale sono ingranditi al doppio della loro misura.

\textbf{Caraffa di Ceramica:} 2 mr, L, vaso con corpo e bocca larghi, collo stretto e un solo manico.

\textbf{Carrucola e Paranco:} 5 mo, 1, dispositivo per il sollevamento manuale di pesi, formato da una staffa che sorregge una ruota scanalata in cui scorre una fune. Il paranco è un sistema meccanico usato per il sollevamento di carichi pesanti, costituito da due o più carrucole collegate da un cavo.

\textbf{Carta (foglio):} 4 ma, -,un foglio di carta misura normalmente 20 per 15 centimetri e non è adatto per le pergamene magiche. Ha durezza 0, 1 punto ferita ed una DC per romperlo di 5.

\textbf{Carta di Riso (foglio):} 5 mr, -,questa carta è fatta di riso. Ha Durezza 0, 1 Punto Ferita ed una DC per romperla di 2.

\textbf{Cassa (vuota):} 2 mo, 1, contenitore in legno o metallo a forma di parallelepipedo, usato per imballaggio e trasporto o come mobile per conservare oggetti, abiti.

\textbf{Catena (3 m):}30 mo, 1, la catena ha Durezza 10 e 5 Punti Ferita. Può essere spezzata con una prova di Potenza con DC 26.

\textbf{Caviglia per Impiombature:} 8 ma, L, questi chiodi di metallo lucido possono aiutare a usare le corde in molti modi, incluso intrecciare e disfare nodi, districare corde e cordicelle, raccordarle e metterle in tensione. Di solito, un chiodo è lungo da 10 a 30 centimetri, ha un corpo sottile quasi simile a un ago, ed è smussato a entrambe le estremità. I chiodi più piccoli vengono legati al collo tramite dei cordini, mentre quelli più grandi vengono tenuti in dei tubini. Una caviglia per impiombature fornisce Bonus +2 alle Prove di Competenza che coinvolgono l'uso di una corda.

\textbf{Ceralacca}: 1 mo, -, Miscuglio di resine naturali, minerali e coloranti che si rammollisce col calore per poi indurirsi nuovamente, adatto per sigillature.

\textbf{Cesto (vuoto):} 4 ma, 1, Contenitore quadrangolare o ovale, con sponde alte e manico o manici per afferrarlo.

\textbf{Chiodo da Rocciatore:} 1 ma, L, i chiodi da roccia sono degli ancoraggi artificiali utilizzati dagli arrampicatori e dagli alpinisti allo scopo di proteggersi, in caso di caduta, oppure per autoassicurarsi in caso di sosta. Possono anche essere utilizzati per fissare la corda per le calate o per la progressione in arrampicata artificiale. Si tratta, in genere, di lame o sottili cunei di metallo la cui forma ne consente l'infissione nelle fessure della roccia, grazie all'utilizzo di un apposito martello. La parte terminale del chiodo da roccia è sempre costituita da un occhiello, o foro, che consente l'inserimento di un moschettone o di un cordino.

\textbf{Clessidra (1 minuto):} 20 mo, L, una normale clessidra richiede 1 ora per riempire di sabbia la camera inferiore; esistono clessidre più grandi e più piccole, che arrivano a durare appena 6
secondi.

\textbf{Clessidra (1 ora):} 25 mo, L

\textbf{Clessidra (6 secondi):} 15 mo, L

\textbf{Colonia di Scarafaggi Necrofagi}: 3 mo, L, questa giara di vetro contiene scarafaggi necrofagi carnivori. Gli scarafaggi devono essere nutriti con almeno 125 grammi di carne al giorno oppure muoiono. Quando rilasciati su un organismo morto, ne divorano le carni in 1d4 giorni, lasciando solo le ossa. Gli scarafaggi necrofagi mangiano soltanto la carne morta e non possono danneggiare le creature viventi. Una volta rilasciati, gli scarafaggi non possono essere rimessi nella giara.

\textbf{Coperta:} 2 ma, L, questa coperta calda ha delle cinghie che permettono di legarla una volta arrotolata.

\textbf{Coperta Invernale:} 5 ma, 1, panno pesante che si stende sul letto o giaciglio. Di solito sono pellicce ricucite assieme.

\textbf{Corda di Canapa}: (15 m) 1 mo,1, Questa corda ha 2 Punti Ferita e può essere spezzata con un prova di Potenza con DC 23.

\textbf{Corda di Ragnatela} (15m),100 mo, 1, tanto rara da essere sconosciuta in superficie, la corda di ragnatela viene tessuta nelle profondita’ a partire dalla tela secreta dai ragni giganti.

La corda di ragnatela ha 6 Punti Ferita e può essere rotta con una prova di Potenza con DC 25

\textbf{Corda di Seta} (15 m): 10 mo, L, questa corda ha 4 Punti Ferita e può essere spezzata con una prova di Potenza con DC 24

\textbf{Corda per Armi}: 1 ma,-, le corde per armi sono legacci di pelle lunghi 60 centimetri che si allacciano all’elsa dell’arma e al polso. Se si lascia andare l’arma o si viene disarmati si può recuperarla con un’azione veloce ed essa non si allontana dall’area di mischia. Non è possibile però cambiare arma finché non si è slegata la prima (3 Azioni) o si taglia la corda (Azione di attacco, Durezza 0, 0 Punti Ferita). A differenza di quanto avviene con un guanto d’arme con sicura, è possibile utilizzare una mano cui è legata una corda per armi, anche se l’arma penzolante può interferire con le azioni più delicate.

\textbf{Corno da Segnalazione:}1 mo, L, suonare un corno richiede una prova di Intrattenere (strumenti a fiato) con DC 10 e può comunicare concetti come "All'attacco!", "Aiuto!", "Avanzare!", "Ritirata!", "Fuoco!" e "Allarme!". Il suono di un corno può essere udito chiaramente a 500 metri di distanza. Per ogni successivo incremento di 250 metri, la prova di Consapevolezza per sentirlo subisce penalità -1.

\textbf{Cote per Affilare:} 2 mr, L, pietra di forma tondeggiante di smeriglio o altro materiale abrasivo che, ruotando, serve ad affilare o a levigare.

\textbf{Custodia per Pergamene:} 1 mo, L, una custodia di legno o pelle può tenere quattro Pergamene; è possibile inserirne di piu', ma recuperarne una diviene difficile e costa 2 Azioni, mentre recuperarne la sola presente costa 1 Azione. Per danneggiarne il contenuto è necessario distruggere il contenitore (Durezza 2 per la pelle e 5 per il legno, 2 Punti Ferita, Rompere DC 15). Una custodia per pergamene non è impermeabile.

\textbf{Fischietto da Segnalazione:} (o silenzioso) 8 ma (9ma), -, con una prova di Intrattenere (strumenti a fiato) con DC 5 si riescono comunicare concetti come "All'attacco!", "Aiuto!", "Avanzare!", "Ritirata!", "Fuoco!" e "Allarme!".

Il suono del fischietto si può udire chiaramente (Consapevolezza DC 0) fino a 250 metri di distanza. Per ogni 250 metri successivi la prova di Consapevolezza subisce penalità -2. I fischietti silenziosi possono essere uditi solo dagli animali o dalle creature con l'udito particolarmente fino.

\textbf{Fondina da Manica:} 100 mo,L, quando indossata all'interno di maniche voluminose, questa fondina di cuoio permette di estrarre una balestra a mano o una pistola da giacca nascosta come Azione. L'arma è posta su delle corsie e viene estratta direttamente nella propria mano.
A differenza di un fodero da polso, una fondina da manica è abbastanza voluminosa da essere evidente a un'ispezione ravvicinata, sebbene se portata al di sotto di abiti sufficientemente larghi potrebbe non suscitare una prova di Consapevolezza per notarla. Una singola fondina da manica può contenere una balestra a mano o una pistola da giacca, non entrambe.

\textbf{Forziere piccolo:}

\textbf{Forziere medio} 5 mo 25 kg

\textbf{Forziere grande} 10 mo 50 kg

\textbf{Forziere enorme} 25 mo 125 kg

\textbf{Gessetto} (1 pezzo) 1 mr, -,Barretta di gesso bianco o colorato utilizzata per scrivere

\textbf{Gesso per Impronte:} 2 mo, L, Questo gesso a presa rapida è perfetto per preservare una serie di impronte al fine di esaminarle in un secondo momento. Spendendo 1 minuto a disporre il gesso e aspettare che si asciughi, si possono copiare le impronte, permettendo ad altri di esaminarle senza viaggiare fino al luogo in cui sono state rilevate, ed evitando che la DC della prova di Sopravvivenza per analizzarle aumenti per via del tempo trascorso o delle condizioni atmosferiche.

\textbf{Giaciglio:} 1 ma, 1, letto misero, per lo più fatto di paglia o di cenci.

\textbf{Giardino da Viaggio:} 200 mo, 5, questo kit per carri pesanti comprende scatole e vasi predispo­sti per la crescita di un'ampia varietà di vegetali, in aggiunta allo spazio per una coppia di animali, come delle capre, e il loro nutrimento. Un giardino da viaggio fornisce cibo ed erbe curative. Funziona in modo simile a una Borsa del Guaritore, fornendo fino a 5 usi al giorno, e non si consuma mai. Inoltre, coloro che ingeriscono quotidianamente una certa varietà di ­erbe e verdure fresche del giardino ottengono bonus +1 ai Tiri Salvezza contro Malattia.

\textbf{Gilè di Sughero:} 25 mo, L, questo gilè di tessuto contiene tasche piene di sughero, che forniscono a chi lo indossa una maggiore capacità di galleggiamento. Inizialmente indossato da pescatori e marinai, protegge dall'annegamento. Mentre si indossa un gilè di sughero si subisce penalità -2 alle prove di Agilità e Nuotare, ma anziché finire sott'acqua dopo aver fallito la prova di 5 o piu', questo accade solo fallendo di 10 o più.

Inoltre, si ottiene bonus +4 alle prove di Nuotare (Resistenza) per evitare danni da Affaticamento. Un gilè di sughero può essere indossato sotto l'armatura.

\textbf{Inchiostro} (boccetta da 30 g): 8 mo, -, un inchiostro diverso da nero costa il doppio.

\textbf{Kit da Cacciatore:} 15 mo,2, il kit comprende acciarino e pietra focaia, una borsa da cin­tura, una borsa per componenti di incantesimi, una corda, un giaciglio, un kit da rancio, un otre, una pentola di ferro, razioni da viaggio (5 giorni) e uno zaino.

\textbf{Kit da Cortigiana:} 10 mo , 1, il kit comprende oggetti che aiutano una cortigiana a dare sollievo al corpo e allo spirito. Per il corpo, il kit contiene un rasoio, oli profumati e balsami, profumi, una pentola per tenere in caldo e una certa varietà di vestiti accattivanti. Libri di poesia, letteratura e teatro, spesso riguardanti argomenti salaci e pieni di doppi sensi, servono invece per intrattenere la mente.

\textbf{Kit da Intrepido:} 9 mo, 2, il kit comprende acciarino e pietra focaia, una borsa da cin­tura, una corda, un giaciglio, un kit da rancio, un otre, una pentola di ferro, razioni da viaggio (5 giorni), sapone, torce (10) e uno zaino.

\textbf{Kit da Investigatore:} 40 mo, 2, il kit comprende acciarino e pietra focaia, una borsa da cin­tura, una corda, un giaciglio, inchiostro, un kit da rancio, un kit per creazioni alchemiche, un otre, un pennino, una pentola di ferro, razioni da viaggio (5 giorni), sapone, torce (10) e uno zaino.

\textbf{Laccio per Libri:} 3 ma, L, questa cordicella intrecciata di metallo possiede un fermaglio che si fissa al lucchetto di un normale libro. L'altra estremità della cordicella viene attaccata a una cintura o a un anello da cintura. La cordicella è lunga 3 metri e ritraibile. Se si fa cadere il proprio libro mentre è attaccato al laccio, lo si può recuperare con il costo di 1 Azione.

Mentre è attaccato al personaggio, il libro non può mai trovarsi oltre a 1 metro di distanza dal personaggio. Slacciare il libro richiede un'Azione; in alternativa, la cordicella (Durezza 5, 10 Punti Ferita) può essere tagliata per liberare il libro.

\textbf{Lampada Comune:} 3 ma, L, una lampada Illumina con luce normale un'area piccola di 4,5 metri di raggio. Una lampada non può aumentare il livello della luce oltre quella normale o intensa. Una lampada brucia per 6 ore con 0.5 litri d'olio. E' dotata di una piccola schermatura per impedire alla luce di uscire. E' possibile trasportarla con una mano.

\textbf{Lavagna:} 1 mo, 1, questa piastra di pietra nera (ardesia) levigata grande come un libro è circondata da una cornice di legno. Strofinandone la superficie con un panno bagnato si cancellano le scritte lasciate con il gesso.

\textbf{Legna da Ardere} (per giorno): 1 mr, 1, insieme di pezzi di rami o di tronchi d'albero da ardere.

\textbf{Lente del Cacciatore:} 100 mo, -, questa complessa lente viene posta su un occhio e occupa lo slot occhi quando è in uso. Quando la si utilizza con un attacco a distanza, si riduce qualsiasi penalità di gittata ai propri attacchi di 2. Tuttavia, gli oggetti entro 9 metri diventano difficili da vedere, e si subisce penalità -2 alle prove di Consapevolezza basate sulla vista mentre si porta una lente del cacciatore.

\textbf{Magnete}: 5 ma, L, i magneti più piccoli sono piuttosto deboli ed utilizzati principalmente per individuare o attirare a brevi distanze ferro, mithral o adamantio. Questo magnete a ferro di cavallo può sollevare fino a 1 kg di metallo.

\textbf{Manette:} Molte manette hanno serrature; aggiungere il costo della serratura desiderata al costo delle manette.

Allo stesso prezzo si possono comprare manette per creature di taglia Piccola. Per creature di taglia Grande le manette costano 10 volte il costo indicato e per creature di taglia Enorme 100 volte il costo indicato. Le creature di taglia Mastodontica, Colossale, Minuscola, Minuta e Piccolissima possono essere trattenute solo con manette costruite appositamente che costano 100 volte il costo indicato.

\textbf{Manette Perfette:} 50 mo, L

\textbf{Martello:} 5 ma, 1, Se viene usato in combattimento, il martello viene considerato un'arma improvvisata ad una mano che infligge danni contundenti come fosse un guanto d'arme chiodato della stessa taglia.

\textbf{Olio:} (ampolla da 0.5 l) 1 ma, L, in una lanterna 0.5 litri d'olio bruciano per 6 ore. E' possibile utilizzare un'ampolla d'olio come arma a spargimento, e occorrono 3 Azioni per preparare un'ampolla con una miccia. Una volta lanciata, c'è solo una probabilità del 75\% che l'ampolla prenda fuoco. Si consideri l'attacco come un attacco da contatto a distanza con portata di 6 metri.

Il colpo diretto provoca 1d6 danni da fuoco. Tutte le creature entro raggio di mischia dal punto in cui è caduta l'ampolla subiscono 1 danno da fuoco come effetto dello spargimento. Nel round successivo al colpo diretto la vittima subisce 1d6 danni aggiuntivi.

La vittima può sfruttare 2 Azioni per tentare di spegnere le fiamme prima di subire questi danni aggiuntivi. Occorre superare un Tiro Salvezza su Riflessi con DC 15 per spegnere le fiamme. Rotolarsi per terra dà al personaggio bonus +2 al Tiro Salvezza.
Tuffarsi in un lago o smorzare le fiamme con mezzi magici spegne automaticamente le fiamme.

\textbf{Orologio ad Acqua:} 1.000 mo, 3, questo grande congegno ingombrante fornisce l'ora esatta con lo scarto di mezz'ora per giorno da quando è stato regolato l'ultima volta. Richiede una fonte d'acqua e deve essere tenuto immobile poiché segna il tempo con il flusso regolare delle gocce d'acqua.

\textbf{Ospedale Mobile:} 1.000 mo, 4, questo kit per carri include tutto l'equipaggiamento necessario a prendersi cura di un massimo di 10 malati o feriti contemporaneamente. Comprende due tende grandi, 10 giacigli con coperte, un tavolo robusto, un kit da cerusico e cinque borse del guaritore. Fornisce a chiunque lo usi bonus +2 alle prove di Guarire per prestare pronto soccorso, può essere usato per trattare ferite mortali con un singolo uso della borsa del guaritore anziché due, e raddoppia il ritmo di recupero dei pazienti durante le cure a lungo termine.

\textbf{Otre:} 1 mo, L, recipiente di pelle animale, utilizzato per trasportare e conservare i liquidi.

\textbf{Pala o Badile:} 2 mo, 1, se una pala è usata in combattimento, viene considerata come un'arma improvvisata ad una mano che infligge danni contundenti pari a quelli di un randello della stessa taglia.

\textbf{Pennino;} 1 ma, -, piccola lamina in acciaio innestata sul cannello della penna per scrivere; l'inchiostro la raggiunge attraverso l'immersione nel calamaio, oppure direttamente da un serbatoio inserito nel cannello.

\textbf{Pentola di Ferro:} 8 ma, L, recipiente da cucina munito di due manici, utilizzato per cuocere le vivande.

\textbf{Pergamena} (Foglio), 2 ma, -, pelle ovina o caprina, lavata, depilata e sbiancata, usata per scriverci.

\textbf{Piccone da Minatore:} 3 mo, 1, se un piccone da minatore è usato in combattimento, viene considerato come un'arma improvvisata ad una mano che infligge danni perforanti pari a quelli di un piccone pesante della stessa taglia.

\textbf{Piede di Porco:} 2 mo, L, un piede di porco fornisce Bonus +2 alle prove di Potenza effettuate per forzare una porta o uno scrigno. Se usato in combattimento, il piede di porco viene considerato un'arma improvvisata ad una mano che infligge danni contundenti pari a quelli di un randello della stessa taglia.

\textbf{Polveri:} 1 mr, L, il gesso polverizzato, la farina ed altri materiali simili sono popolari fra gli avventurieri per notare le creature invisibili. Lanciare un sacco di polveri in una zona di mischia richiede un Tiro per Colpire verso Difesa 5 e rivela momentaneamente se vi si trova una creatura Invisibile. Un metodo più efficace è quello di spargere le polveri su una superficie (3 Azioni) e cercare le tracce.

\textbf{Pompa Antincendio} 200 mo, 5 , questo kit per carri pesanti comprende un serbatoio d'acqua, una pompa e un erogatore rotante. Se l'operatore supera una prova di Potenza con DC 20, la pompa antincendio rilascia un flusso d'acqua che raggiunge fino i 18 metri di distanza. Ciascuna persona che aiuta con la pompa fa diminuire a DC di 5. Operare o aiutare costa 3 Azioni. La pompa estingue un area di mischia di fuoco non magico per round. Il serbatoio contiene acqua a sufficienza per 10 round di pompaggio e ci vogliono 10 minuti per ricaricarlo da un corso d'acqua, uno stagno, un lago o un altro corpo idrico.

\textbf{Prigione Portatile} 200 mo, 3, questo kit per carri comprende una gabbia di sbarre di metallo con una porta su un lato. Sebbene le prigioni portatili siano state originariamente ideate dai viaggiatori per contenere animali feroci, le guardie cittadine le usano abitualmente per radunare i criminali, e alcuni cacciatori di taglie le impiegano per trasportare grandi gruppi di prigionieri. La maggior parte di queste prigioni è dotata di lucchetti: aggiungere il costo del lucchetto desiderato al costo della prigione portatile. Una gabbia pensata per le persone include panche e corrimano a cui vengono attaccate delle manette. Una prigione portatile pensata per gli animali include un trogolo per l'acqua e una porta più piccola per fornire cibo.

\textbf{Punta di Metallo} 5 mr, L, questa punta di metallo lunga 30 centimetri si usa per tenere le porte aperte o per assicurarvi corde per scalare. Sentire una punta di metallo che viene martellata in posizione richiede una prova di Consapevolezza con DC 5.

\textbf{Rampino} 1 mo, L, lanciare efficacemente il rampino da scalata richiede un attacco a distanza, considerandolo un'arma da lancio con gittata di 3 metri. Gli oggetti con ampio spazio per ricevere l'aggancio di un rampino hanno Difesa 5.

\textbf{Razioni da Viaggio} (al giorno) 5 ma, L, quantità di cibo, bevande ecc. che consuma giornalmente un'avventuriero o un viaggiatore.

\textbf{Rete da Pesca} (2,25 m) 4 mo, L, attrezzo costituito da un intreccio, a maglie più o meno fitte, di fili di fibre naturali o artificiali, usato per pescare.

\textbf{Rete per Farfalle} 5 mo L, una delle estremità di quest'asta di 2 metri ha una rete sottile. E' possibile utilizzarla per setacciare materiali abbastanza sottili da passare attraverso la stretta retina, come la sabbia o l'acqua. E' possibile anche utilizzarla per catturare creature Minute o Piccolissime come si trattasse di una rete (l'arma), anche se non è necessario ripiegare la rete se manca il bersaglio, ed il manico della retina si utilizza come la corda di una rete.

\textbf{Richiamo per Animali} 1 ma,-, Questi fischietti di canna o bambù imitano i versi di vari animali selvatici. Ciascun fischietto è legato a uno specifico tipo di animale e un verso specifico (che solitamente segnala la disponibilità di cibo o di un compagno per attirare l'animale). Un richiamo fornisce bonus +2 alle prove di Sopravvivenza per seguire le tracce di animali del tipo a cui è legato o per cavarsela in territori selvaggi.

\textbf{Sacco (vuoto)} 1 ma, L, Involucro di tela ruvida, carta, tela o altri materiali che si prestano all'uso, di forma allungata e aperto in alto, in cui si conservano o si trasportano materiali o oggetti

\textbf{Sapone} (per 0.5 kg) 5,L, prodotto comunemente usato per detergere persone, abiti, oggetti.

\textbf{Scala a Pioli} (3 m) 2 ma, 1, struttura fissa a gradini che permette di salire o di scendere da un livello all'altro in edifici o in luoghi aperti. Una semplice scala a pioli di legno.

\textbf{Scendicorde} 50 mo, 1, si può collocare questo macchinario di metallo su una sezione di una corda tesa connessa da un punto alto a uno più basso, permettendo così di scendere lungo la corda verso il basso con facilità. Usare un scendi corde richiede una sola mano, lasciando l'altra libera durante la discesa. Fissare lo scendi corde a una corda è un Azione. Iniziare la discesa è un'azione veloce. La corda viene discesa al ritmo di 18 metri per round (2 movimenti per round). Farlo non richiede alcun'azione, ma ci si deve muovere lungo la corda verso il basso. Recuperare lo scendi corde una volta che si è terminata la discesa e raggiunta l'estremità della corda è un'Azione.
Si può lasciar andare lo scendi corde come azione immediata.

\textbf{Secchio} (vuoto) 5 ma, L, recipiente piuttosto capace, di forma cilindrica o troncoconica, in legno o in metallo, dotato di manico semicircolare, usato per contenere liquidi o altri materiali

\textbf{Serratura o Lucchetto} La DC per aprire una serratura (o un lucchetto) con Disattivare Congegni (Criminalita') dipende dalla qualità della serratura (o del lucchetto); molto semplice (DC 20), media (DC 25), buona (DC 30) e superiore (DC 40).

\textbf{Semplice} 20 mo, L

\textbf{Media} 40 mo, L

\textbf{Buona} 80 mo, L

\textbf{Superiore} 150 mo, L

\textbf{Spago} (15 m) 1 mr, L, venduto in gomitoli di 15 metri, spaghi e lana sono utili per creare gli interruttori delle trappole e sono necessari per le frecce e gli uncini da scalata. Spaghi e lana hanno Durezza 0, 1 Punto Ferita e una DC per romperli di 14.

\textbf{Specchio Piccolo di Metallo}: 10 mo, L, lastra levigata di vetro, metallizzata su una faccia, che riflette la luce e le immagini.

\textbf{Tabacchiera} (Stagno o Legno) 5 mo, -, Il coperchio a cerniera di questa piccolissima scatoletta ornata aderisce alla guarnizione formando una tenuta stagna. La scatoletta è usata per contenere vari tabacchi, polveri e sostanze simili. La scatoletta può essere fatta di qualsiasi tipo di materiale, dal legno all'avorio fino ai metalli preziosi incastonati di gemme.

\textbf{Tabacchiera} (Osso o Guscio di Tartaruga) 25 mo

\textbf{Tabacchiera} (Avorio o Metallo Prezioso) 300 mo

\textbf{Tappi per Orecchie} 3 mr, -, Fatti di cotone o sughero cerato, i tappi per orecchie concedono Bonus +2 al Tiro Salvezza contro gli effetti che richiedono l'udito ma infliggono penalità -5 alle prove di Consapevolezza basate sull'udito.

\textbf{Tatuaggio} 1 mr - 20 mo,-, il costo di un tatuaggio dipende dalla sua qualità, dalla dimensione e dal numero di colori utilizzati. Un tatuaggio grande come una moneta, di colore blu che scolorirà nel giro di dieci anni può costare 1 mr, uno grande come una mano in inchiostro nero che non scolorisce ma ed uno che copre l'intera schiena e che richiede più sessioni costa 10 mo.
Ogni colore aggiuntivo costa come un singolo tatuaggio della stessa taglia.

\textbf{Tela} (m2) 1 ma, L, uno dei tre tipi, con la saia e il raso, di armatura dei tessuti, in cui i fili della trama passano alternativamente sopra e sotto i fili dell'ordito, costituendo un tessuto compatto senza rovescio.

\textbf{Tenda piccola} 10 mo,1, le tende hanno diverse dimensioni e possono ospitare tra 1 e 10 persone. Un tenda piccola ospita una creatura Media e richiede 20 minuti per essere montata, una tenda media ospita 2 creature e richiede 30 minuti per essere montata, una grande ospita 4 creature e richiede 45 minuti ed un padiglione ospita 10 creature e richiede 90 minuti (due creature Piccole contano come una Media ed una Grande conta come due Medie). Le tende a padiglione sono abbastanza grandi da consentire di accendere un piccolo fuoco al centro. Smontare una tenda richiede la metà del suo tempo di montaggio.

\textbf{Tenda media} 15 mo, 2

\textbf{Tenda grande} 30 mo, 2

\textbf{Tenda enorme} (padiglione) 100 mo, 3

\textbf{Tomo delle Imprse Epiche} 50 mo,L, questo corposo libro è rilegato in tela cerata e decorato con scene di gloriosi combattimenti tra antichi eroi e feroci mostri.
Contiene svariati racconti di valore, sconfitta e vittoria, tutti accompagnati da illustrazioni dai colori vivaci.
Dopo aver consultato il libro per 1 ora, per le 24 ore successive si guadagna Bonus +2 alle prove di Intrattenere (canto) e Intrattenere (oratoria) e Bonus +2 alle prove di Conoscenze (nobilta') riguardanti i lignaggi eroici.

\textbf{Torcia} 1 mr, L, una torcia brucia per 1 ora ed Illumina con luce normale un'area di 3 metri. Se usata in combattimento, la torcia viene considerata un'arma improvvisata ad una mano che infligge danni contundenti 1d4 più 1 punto ferita da fuoco.

\textbf{Trampolino Pieghevole} 50 mo, 2, questo compatto trampolino si smonta e rimonta come una tenda, permettendo un agevole trasporto. Montare o smontare il trampolino richiede 1 minuto. Quando utilizzato da due creature, un trampolino pieghevole fornisce bonus +5 a tutte le prove di Acrobatica effettuate per Saltare.

Se una creatura in caduta cade sul trampolino, ignora i primi danni dati dalla distanza di caduta.

\textbf{Tribolo} 1 mo, -, i triboli sono chiodi di ferro a quattro punte costruiti in modo da avere sempre una punta rivolta verso l'alto. Si spargono sul terreno nella speranza che i nemici ci camminino sopra o almeno rallentino per evitarli. Una borsa contenente 1 kg di triboli copre un quadretto.

Ogni volta che ci si muove in un'area coperta con i triboli (o si passa un round combattendo mentre ci si trova nell'area), si rischia di pestarne uno. I triboli effettuano un Tiro per Colpire senza alcun bonus contro la creatura. Per questo attacco lo scudo, l'armatura della creatura non contano. Se la creatura indossa le scarpe o qualche altra copertura per i piedi, ha un bonus alla Difesa di +2. Se i triboli riescono a colpire, la creatura ne ha pestato uno. Il tribolo infligge 1 punto ferita e la velocità della creatura è dimezzata a causa del piede ferito. Questa penalità al movimento dura 24 ore, fino a quando la creatura non viene curata con successo con una prova di Sopravvivenza con DC 15 oppure fino a quando non riceve almeno 1 punto di cure magiche.
Una creatura alla Carica o che sta correndo deve fermarsi immediatamente se pesta un tribolo. Qualsiasi creatura che si muove a velocità dimezzata o più lentamente può camminare attraverso una distesa di triboli senza problemi.

I triboli potrebbero essere inefficaci contro avversari insoliti.

\textbf{Veste da Apicoltore} 20 mo, 1, questi pesanti strati di vestiti, uniti ad un cappello ampio e dotato di una rete, rendono impossibile a creature Minute e Piccolissime di entrare in contatto con il corpo. Indossare una veste da apicoltore dimezza la velocità ma concede RD 10/- contro gli sciami di creature Piccolissime e RD 5/--- contro sciami di creature Minute.

\textbf{Veste Uncinata} 10 mo, 1, piccole coperture di pelle impediscono alle centinaia di piccoli aghi uncinati che ricoprono la superficie di questo abito di ferire chi lo indossa. Qualsiasi creatura che ferisca chi lo porta con un attacco naturale o senz'armi deve superare un Tiro Salvezza su Riflessi con DC 15 o subisce 1 danno.

Se una creatura ingoia chi lo indossa, subisce 1 danno per round finché non lo sputa o chi lo indossa non fugge o muore (in questo caso la veste ha subito troppi danni per essere una minaccia). La veste può essere indossata soltanto se non si indossano armature o se ne indossa una di tipo leggero.

\textbf{Zaino} 2 mo, L, sacco di grossa tela o di altro materiale molto resistente, che si porta appeso alle spalle, può contenere 0,05 metri cubi (50 litri) di materiali nella tasca principale.

\textbf{Zaino Perfetto} 50 mo, 1 questo zaino ha numerose tasche, utili per conservare gli oggetti necessari per andare in avventura. Ci sono ganci per attaccare oggetti come borracce, borse e coperte arrotolate.
Ha fasce imbottite che si tirano sul petto e sulle spalle per distribuire meglio il peso. Come un normale zaino può contenere 0,05 metri cubi (50 litri) di materiali nella tasca principale. Indossando uno zaino perfetto, il punteggio di Potenza ai fini di determinare l'Ingombro è considerato maggiore di +1.

\subsection{Oggetti e Sostanze Speciali}\index{Sostanze Speciali}

\label{oggetti-e-sostanze-speciali}

Tutti gli oggetti inclusi nella lista, fatta eccezione per la torcia inestinguibile e l'acquasanta, possono essere fabbricati da un personaggio con la competenza Lavoro (alchimia). La DC per creare gli aggetti è indicata come: Creazione DC XX

\textbf{Acido} (ampolla) 10 mo, L, è possibile lanciare un'ampolla d'acido come arma a spargimento. Si consideri l'attacco come un attacco di contatto con gittata 3 metri. Il colpo diretto provoca 1d6 danni da acido. Tutte le creature entro 1,5 metri dal punto in cui è caduta l'ampolla subiscono 1 danno da acido come effetto dello spargimento.
Creazione DC 15

\textbf{Acquasanta} (ampolla) 25mo, L, l'acquasanta infligge danni ai Non Morti e agli Esterni malvagi quasi come se fosse acido. Un'ampolla di acquasanta può essere lanciata come arma a spargimento.

Si consideri l'attacco come un attacco di contatto con gittata di 3 metri. Un'ampolla si rompe se scagliata contro il corpo di una creatura corporea, ma contro una creatura incorporea l'ampolla deve essere aperta e l'acquasanta versata sulla creatura. Di conseguenza, si può spruzzare una creatura incorporea con l'acquasanta solo se si è adiacenti ad essa.

Il colpo diretto di un'ampolla di acquasanta provoca 2d4 danni ai Non Morti e agli Esterni malvagi. Tutte le creature di questo tipo entro raggio di mischia da dove è caduta l'ampolla subiscono 1 danno come effetto dello spargimento.

I templi dei Dei buoni vendono acquasanta a prezzo di costo (senza guadagno). L'acquasanta si ottiene usando l'Essenza di Creazione, acqua, a livello potere 18 per 5 fialette.

\textbf{Antiemetico} 25 mo, L, questo liquido verde dolce e saporito crea un senso di calore e conforto. Lo sciroppo copre lo stomaco e lo rende più resistente. Per 1 ora dopo averlo bevuto si ottiene Bonus +5 ai Tiri Salvezza per resistere agli effetti che rendono Nauseati o Infermi. Monodose. Creazione DC 18

\textbf{Antibiotico} (fiala) 50 mo, -,  bevendo una fiala di questo liquido bianco latte dal pessimo sapore si ottiene Bonus +5 ai Tiri Salvezza contro le Malattie, effettuati nell'ora successiva. Se già infetti, si possono effettuare due Tiro Salvezza per resistere alla Malattia in quella determinata giornata (senza il bonus +5) e tenere il risultato migliore. Monodose. Creazione DC 18

\textbf{Antitossina} (boccetta) 50 mo, -, se si beve l'antitossina, si ottiene Bonus +5 a tutti i Tiri Salvezza su Tempra contro Veleni per 1 ora. Monodose. Creazione DC 18

\textbf{Bastone del Fumo} 20, L, questo bastone di legno trattato con procedimento alchemico crea istantaneamente un denso fumo opaco quando viene infiammato. Il fumo riempie un cubo con spigolo di 3 metri (distanza di mischia) (come per l'Essenza Creazione), tranne che il fumo viene dissipato in 1 round da un vento moderato o più intenso. Il bastone si consuma in 1 round e il fumo si dissolve
poi naturalmente. Creazione DC 18

\textbf{Benedizione dell'Alchimista} 1 mo, -, molto amata dai giovani libertini, si tratta di una polvere cristallina simile al sale. Mischiata con l'acqua crea una bevanda frizzante che cura gli effetti della sbornia. Monodose. Creazione DC 15

\textbf{Borsa dell'Impedimento} 50 mo. L, questa borsa di cuoio rotonda è piena di melassa, resina o altra sostanza appiccicosa. Quando si scaglia la borsa contro una creatura (come attacco di contatto a distanza con gittata 3 metri), la borsa si apre e la sostanza contenuta invischia ed intralcia la vittima, diventando resistente ed elastica con l'esposizione all'aria.

Una creatura Intralciata subisce penalità -2 al Tiro per Colpire e penalità -2 alla Agilità, e inoltre deve effettuare un Tiro Salvezza su Riflessi con DC 15 o resta appiccicata al suolo, incapace di muoversi. Anche con un Tiro Salvezza riuscito, può solo muoversi con una penalità dl movimento di 1.

La sostanza non agisce su creature di taglia Enorme o superiore. Una creatura volante non viene appiccicata al suolo, ma deve effettuare un Tiro Salvezza su Riflessi con DC 15 o perde la capacità di Volare (sempre che usi le ali per farlo), cadendo a terra. La borsa dell'impedimento non funziona sott'acqua.

Una creatura appiccicata al suolo (o impossibilitata a Volare) può liberarsi con una prova di Potenza riuscita con DC 17 oppure infliggendo 15 danni alla sostanza con un'arma tagliente. Una creatura che tenta di sfregare via la sostanza da sé o da un'altra creatura che assiste non ha bisogno di effettuare un Tiro per Colpire; colpire la sostanza è automatico, poi la creatura che colpisce effettua un tiro per i danni per vedere quanta sostanza è riuscita a sfregare via. Una volta libera, la creatura si muove a velocità dimezzata, anche volando.

Una creatura invischiata dalla sostanza può lanciare Essenze ma deve superare una prova di Concentrazione con DC 20. La sostanza diventa fragile dopo 2d4 round, staccandosi da sola e perdendo ogni effetto. Un'applicazione di solvente universale su una creatura appiccicata dissolve la sostanza alchemica immediatamente. Creazione DC 18

\textbf{Fermasangue} 25 mo,-, questa sostanza rosa e appiccicosa aiuta a curare le ferite. Utilizzarne una dose concede Bonus +4 alle prove di Sopravvivenza quando si effettua pronto soccorso, si guariscono le ferite da tribolo ed oggetti simili o si trattano ferite mortali.

Una dose di fermasangue pone termine ad un effetto di Sanguinamento come se si fosse superata una prova di Sopravvivenza con DC 15. Quando si trattano le ferite mortali, utilizzare una dose di fermasangue conta come un utilizzo della borsa del guaritore (e si ottiene bonus +4). La confezione contiene 3 dosi. Creazione DC 18

\textbf{Fiasco Alcalino} 15 mo, L, questo fiasco di liquidi caustici reagisce con gli acidi naturali delle melme. E' possibile lanciare un fiasco alcalino come arma a spargimento con gittata 3 metri. Contro le creature non melme un fiasco alcalino funziona come un'Ampolla d'acido. Contro le melme e altre creature acide il fiasco alcalino infligge i danni raddoppiati indicati da Ampolla d'Acido. Creazione DC 18

\textbf{Fumogeno} 25 mo, -, questa piccola sfera di argilla contiene due sostanze alchemiche separate da una sottile barriera. Quando si rompe la sfera, le sostanze si uniscono e riempiono un area di mischia con una nuvola di fumo nerastro e innocuo. Il fumogeno funziona come un bastone del fumo, ma il fumo rimane per 1 round prima di disperdersi. E' possibile lanciare un fumogeno come attacco di contatto con gittata 3 metri. Creazione DC 18

\textbf{Fuoco dell'Alchimista} 20 mo, L, si può lanciare un'ampolla di fuoco dell'alchimista come arma a spargimento. Si consideri l'attacco come un attacco di contatto a distanza, con gittata 3 metri.

Il colpo diretto provoca 1d6 danni da fuoco. Tutte le creature entro raggio di mischia dal punto in cui è caduta l'ampolla subiscono 1 danno da fuoco come effetto dello spargimento. Nel round successivo al colpo diretto la vittima subisce 1d6 danni da fuoco aggiuntivi. La vittima può sfruttare 2 Azioni per tentare di spegnere le fiamme prima di subire questi danni aggiuntivi.

Occorre superare un Tiro Salvezza su Riflessi con DC 15 per spegnere le fiamme. Rotolarsi per terra (1 Azione) dà al personaggio bonus +2 al Tiro Salvezza. Tuffarsi in un lago o smorzare le fiamme con mezzi magici spegne automaticamente le fiamme. Creazione DC 18

\textbf{Gesso per Calchi:} 5 ma, L, questa polvere bianca e secca, mischiata con l’acqua, si addensa nel giro di un’ora per creare un materiale solido. Può essere utilizzato per creare un calco di un’orma o di un bassorilievo, riempire buchi o crepe nei muri o (se applicato ad una copertura di stoffa) per fermare un osso rotto. Il gesso indurito ha Durezza 1 e 5 Punti Ferita ogni 2.5 centimetri di spessore. Un vaso di 2 kg di gesso può coprire un raggio di mischia per la profondità di 2.5 centimetri, creare cinque ingessature per l’avambraccio o il polpaccio di una creatura di taglia Media o due ingessature complete per braccio o gamba. Monodose. Creazione DC 18

\textbf{Ghiaccio Liquido} (fiala) 40 mo, L, detto anche "ghiaccio dell'alchimista", questo fluido blu cristallino inizia ad evaporare appena tolto dal contenitore. Nei successivi 1d6 round è possibile utilizzarlo per congelare un liquido o coprire un oggetto con un sottile strato di ghiaccio. E' possibile anche lanciare il ghiaccio liquido come arma a spargimento. Un colpo diretto infligge 1d6 danni da freddo, mentre le creature entro raggio di mischia subiscono 1 danno da freddo per lo spargimento. La confezione contiene 3 dosi. Creazione DC 18

\textbf{Grasso Alchemico} 5 mo, L, ogni vaso di questa sostanza nerastra può coprire una creatura Media o due Piccole. Coprendosi di grasso alchemico si ottiene Bonus +5 alle prove di Criminalità e per sfuggire alle prese. L'effetto dura 4 ore o finché si lava via il grasso. Creazione DC 18

\textbf{Individua Luce} 1 mo, -, questa piastra di metallo grande quanto una mano è coperta da una crema trasparente sensibile alla luce. Se esposta alla luce, la crema si scurisce e diviene opaca a seconda di quanta luce sia presente. La luce intensa la fa scurire in 1 round, quella normale in 3 round, quella fioca in 10 round.

Viene spesso utilizzata da creature dotate di Visione Crepuscolare per capire se sono passate di recente creature che per vedere utilizzano la luce. La piastra viene venduta avvolta in un panno pesante per evitare esposizioni accidentali. Creazione DC 18

\textbf{Pietra del Tuono} 30 mo, L, si può scagliare questa pietra con un attacco a distanza con gittata 6 metri. Quando colpisce una superficie dura (o è colpita con Potenza), crea un rumore assordante che equivale a un attacco sonoro. Le creature presenti entro una distanza di 3 metri devono effettuare un Tiro Salvezza su Tempra con DC 15 o restano Assordate per 1 ora.

Le creature Assordate, oltre alle ovvie conseguenze, subiscono penalità -4 all'Iniziativa e una probabilità del 20\% di sbagliare a lanciare e perdere qualsiasi Essenza con una componente verbale che cercano di lanciare. Monouso. Creazione DC 18

Dal momento che non è necessario colpire uno specifico bersaglio, si può mirare su un determinato area di mischia. Si consideri la zona come se avesse Difesa 5. Creazione DC 18

\textbf{Polvere Lampo} 50 mo, L, questa polvere grigia brucia ed esplode quasi istantaneamente se esposta al fuoco, frizionandola o lanciandola con Potenza contro una superficie (1 Azione). Le creature entro raggio 3 metri sono Accecate per 1 round (Tempra DC 13 nega). La confezione contiene 3 dosi. Creazione DC 18

\textbf{Polvere per Starnuti} (borsa) 60 mo, L, questa polvere giallo-rossa è un'arma a spargimento che causa starnuti incontrollabili per 1d4+1 round. Chiunque si trovi nella zona di mischia dell'impatto deve superare un Tiro Salvezza su Tempra con DC 12 per resistere alla polvere, mentre per chi si trova nella zona di 3 metri adiacente la DC è 8.

Le creature che lo falliscono devono superare un Tiro Salvezza con DC 10 in ogni round di effetto o sono affaticate fino al loro turno successivo. La confezione contiene 3 dosi. Creazione DC 18

\textbf{Proteggilama} 40 mo,-, questa resina trasparente protegge un'arma dagli attacchi di Melme, Rugginofagi ed effetti che corrodono o sciolgono le armi, rendendola immune a tali attacchi per 24 ore. Un vasetto può coprire un'arma a due mani, due armi ad una mano o leggere o 50 munizioni. Applicarla richiede 2 Azioni. La confezione contiene 3 dosi. Creazione DC 18

\textbf{Sali} 25 mo, -,  questi cristalli grigi dall'odore pungente fanno riprendere conoscenza a chi li inala. I sali concedono un nuovo Tiro Salvezza per resistere ad Essenze o effetti che rendono Privi di Sensi.
Un contenitore di sali può essere usato una dozzina di volte se tappato dopo ogni utilizzo, ma si dissolve in poche ore se lasciato aperto. Creazione DC 18

\textbf{Solvente Universale} (fiala) 20 mo, L. questa gelatina viola ribollente divora gli adesivi. Ogni fiala può coprire un raggio di mischia. Distrugge i normali adesivi (come la pece, la resina o la colla) in 1 round, ma richiede 1d4+1 round per dissolvere adesivi più potenti (borse dell'impedimento, ragnatele, ecc.). Non ha effetti sugli adesivi magici. Creazione DC 18

\textbf{Tizzone Ardente} 1 mo, -, la sostanza alchemica sulla punta di questo piccolo bastone di legno si infiamma quando viene sfregata contro una superficie ruvida. Creare una fiamma con un tizzone ardente è molto più rapido che crearla con acciarino, pietra focaia (o lente d'ingrandimento) e esca. Accendere una torcia con un tizzone ardente costa 2 Azione (invece che 3 Azioni) e per accendere qualsiasi altro fuoco occorre almeno 2 Azioni. Creazione DC 18

\textbf{Verga del Sole} 2 mo, L, questa verga di ferro lunga 30 cm e con la punta dorata risplende vivacemente quando viene percossa (2 Azione). Illumina con luce normale un'area di 3 metri di raggio. Una verga del sole brilla per 6 ore dopodiché la punta dorata si consuma e diventa inutile. Creazione DC 18

\pagebreak

\subsection{Armi Alchemiche}\index{Armi Alchemiche}

\label{armi-alchemiche}

Le armi alchemiche sono ideate per ferire gli altri, sebbene possano avere anche altri utilizzi. Ciascuna di queste sostanze può essere prodotta superando una prova di abilità di Lavoro (alchimia).

\textbf{Fiala di Polvere di Diamante} (1 dose) 25 mo, - , ciascuna di queste fiale è riempita di cristalli minerali finemente macinati. Quando si infrange una fiala con un proprio pugno, il bersaglio colpito deve superare un Tiro Salvezza su Riflessi con DC 20 per proteggersi gli occhi o resta Accecato per 1 round. Creazione DC 18

\textbf{Tirapugni Spargi Polveri} 50 mo, L, questo guanto di cuoio senza dita include quattro piccole borsette lungo le nocche in cui si possono inserire minuscole fiale di vetro. Si possono riempire le fiale di veleno o minerali macinati.
Quando si sferra un pugno a qualcuno, le fiale si infrangono, rilasciando il loro contenuto sulla faccia e sugli occhi del bersaglio. Insieme, le quattro fiale contengono una dose di veleno o minerali macinati; non hanno alcun effetto a meno che tutte e quattro non siano piene. Creazione DC 18

\textbf{Unguento dell'Arma Sacra} 30 mo, L, quest' unguento violetto è conservato in un piccolo vasetto di ceramica. Quando applicato su un'arma (2 Azioni), forma un rivestimento trasparente. Le armi ricoperte da questo unguento infliggono 2d4 danni addizionali ai Non Morti e agli Esterni malvagi.
Una creatura influenzata dal balsamo deve superare un Tiro Salvezza su Riflessi con DC 10 o subisce 1d4 danni addizionali il round successivo. Qualsiasi arma non magica ricoperta con questo unguento influenza i Non Morti o gli Esterni malvagi come se fosse un'arma magica. Qualsiasi arma magica ricoperta con questo unguento influenza i Non Morti o gli Esterni malvagi come se l'arma avesse la capacità speciale Tocco Fantasma. L'unguento rimane attivo finché non si mette a segno un attacco con l'arma o passa 1 minuto, quale dei due eventi si verifichi prima. Ciascuna dose di unguento può ricoprire un'arma o 10 munizioni. Creazione DC 18

\subsection{Attrezzature Alchemiche}

\label{attrezzature-alchemiche}

Le attrezzature alchemiche sono oggetti da avventurieri che possono rivelarsi estremamente utili in varie situazioni, compresa la battaglia, l'esplorazione di dungeon o la fabbricazione di altri oggetti alchemici. Queste attrezzature possono essere realizzate da chi possiede l'abilità Artigianato (alchimia).

\textbf{Capsula del Vomito} 12 mo, -, queste piccole capsule sono fatte da un mix concentrato di erbe che provocano la nausea. Per usare una capsula, la si morde e se ne ingeriscono i contenuti, che causano quasi immediatamente il vomito. L'attacco di vomito dura per tutto il round durante il quale non si possono compiere altre azioni. I round seguenti si recupera pienamente, e non si soffrono altri effetti negativi.

Queste capsule vengono molto spesso usate dai Ladri che lavorano in squadra per creare diversivi e distrazioni in modo da attirare o sviare l'attenzione della gente dalle loro attività, così come da coloro che sono interessati a fingersi malati, come i pugili che truccano gli incontri o i criminali che cercano di seminare il caos durante un arresto. Creazione DC 18

\textbf{Carta Reagente} 1 mo, -, questo pezzo di carta può aiutare a identificare i liquidi. Il suo colore cambia a seconda di tratti come acidità, salinità e magia. Consumare un foglio conferisce Bonus +2 alle prove di Lavoro (alchimia) o Arcano per identificare Pozioni o altri liquidi. Creazione DC 18

\textbf{Corda di Liana di Sangue}: 200 mo, L, questa robusta e leggera corda lunga 15 metri è ricavata da una liana di sangue trattata alchemicamente, una rara liana di colore scarlatto che cresce solo nelle giungle calde.
Apprezzata dagli scalatori per la sua resistenza, la liana di sangue può anche essere usata per legare le creature. Una corda di liana di sangue ha Durezza 5 e 10 Punti Ferita, e può essere rotta superando una prova di Potenza con DC 30.
Una creatura legata con una corda di liana di sangue può liberarsi superando una prova di Criminalita’ (Artista della Fuga) con DC 35 o una prova di Potenza con DC 30. Creazione DC 18

\textbf{Flagranza Mascherante} (Animale) 25 mo, -, quest'oggetto è disponibile in una varietà di fragranze (che corrispondono a qualsiasi singolo Animale, Umanoide o Bestia Magica). Una fiala applicata su una creatura Media ne cambia l'odore rendendolo uguale a quello della creatura della fragranza mascherante per 8 ore. Creazione DC 18

\textbf{Flagranza Mascherante} (Umanoide) 50 mo, -, Creazione DC 21

\textbf{Flagranza Mascherante} (Bestia Magica) 100 mo, -, Creazione DC 24

\textbf{Inchiostro Luce di Fuoco} (fiala) 40 mo, -, questo inchiostro infuso alchemicamente aiuta ad assicurarsi che un messaggio segreto venga distrutto dopo essere stato letto. Se la luce colpisce l'inchiostro dopo che quest'ultimo si è asciugato, le sostanze chimiche lo fanno bruciare spontaneamente nel giro di 1 minuto
Questa combustione è di piccole dimensioni: non è abbastanza significativa da dar fuoco ad altro che alla carta. L'inchiostro usato su altri materiali come pietra o legno semplicemente svanisce, non lasciando alcuna traccia della scrittura
Una fiala di questo inchiostro ne contiene abbastanza da scrivere 10 brevi messaggi di non più di 50 parole ciascuno. Creazione DC 18

\textbf{Liquido dell'Aderenza} 20 mo, -, questa bottiglia di vetro è piena di una sostanza appiccicosa apprezzata dai marinai per l'aderenza che fornisce sui ponti delle navi. Quando applicato sulle suole delle calzature e fatto asciugare per 1 ora, il liquido dell'aderenza fornisce Bonus +2 alle prove di Acrobatica per mantenere l'equilibrio.
Il liquido dell'aderenza non ha alcun effetto quando viene in contatto con superfici scivolose o molto scivolose come ghiaccio o Unto. Creazione DC 18

\textbf{Olio dei Maestri} 50 mo, -, quest'olio dorato profuma di truciolato di legno. Quando lo si applica sulle corde di uno strumento a corda o sul corpo di uno strumento di legno, ne migliora la qualità del suono. Per 1 ora, chiunque suoni lo strumento ottiene Bonus +2 alla prova di Intrattenere appropriata. Creazione DC 18

\textbf{Pastiglia dell'Usignolo} 50 mo, -, questa caramella ricoperta di miele è fatta di reagenti calmanti. Se mangiata, ha bisogno di 1 round per iniziare ad avere effetto, dopodiché conferisce Bonus +2 alle prove di Intrattenere (canto) per 1 ora. Creazione DC 18

\textbf{Pietre di Via} 50 mo, -, questi piccoli sassolini bianchi sono trattati alchemicamente in modo che emanino una luce soffusa quando attivati sfregandoli gli uni contro gli altri. La luminescenza è fioca, appena sufficiente a illuminare la pietra.
Sebbene non siano abbastanza luminose da fungere da effettiva fonte di illuminazione, possono essere disposte secondo degli schemi in modo da creare messaggi o disposte su un sentiero, segnalandolo in modo che altri possano seguirlo. Creazione DC 18

\textbf{Polvere Tracciante} 30 mo, -, quando sparsa per terra, questa sottilissima polvere blu chiaro rivela le tracce di qualsiasi creatura o individuo che sia passato nell'area nelle ultime 48 ore.
La polvere fornisce anche Bonus +10 alle prove di Sopravvivenza per seguire tracce o, se non si ha addestramento in Sopravvivenza, permette invece di seguire le tracce delle creature le cui impronte sono state rivelate fino a 1,5 chilometri di distanza usando Consapevolezza anziché Sopravvivenza. Una singola applicazione può coprire un'area di 3 metri.
La polvere tracciante viene venduta in piccole borse di cuoio che contengono 10 applicazioni ciascuna. Creazione DC 18

\textbf{Tabacco del Battipista} 200 mo, -, quando inalato, questo tabacco finemente macinato e trattato alchemicamente potenzia significativamente i propri sensi, specialmente l'olfatto. Fornisce la capacità Fiuto e Bonus +2 alle prove di Consapevolezza per 1 ora. Una volta che l'effetto svanisce, il proprio corpo è scosso da terribili dolori mentre le proprie articolazioni si irrigidiscono e si bloccano, e si subiscono 1d2 danni a Agilità. Creazione DC 18

\textbf{Tonico Rauco} 50 mo, - ,questo tonico è fangoso, e il suo odore assomiglia al sentore di trucioli di ferro. Bere un tonico rauco rende la voce più profonda e roca per 1 ora, fornendo Bonus +5 alle prove di Intimidire. Creazione DC 18

\subsection{Rimedi Alchemici}\index{Rimedi Alchemici}

\label{rimedi-alchemici}

I rimedi alchemici sono sostanze usate per superare condizioni avverse o proteggersi da tipi specifici di attacchi. La maggior parte dei rimedi si utilizza per ingestione o applicandoli sulla propria pelle o sui vestiti. Queste sostanze possono essere realizzate da chi possiede l'abilità Lavoro (alchimia).

\textbf{Aiuto Gassato} 25 mo, -, questo pacchetto è pieno di foglie dai bordi spinosi e ha un odore pungente quasi abbastanza forte da far lacrimare gli occhi. Mentre si masticano le foglie, si ignorano gli effetti della fatica. Le foglie durano per 10 round, dopodiché ne rimane solo un mucchietto di poltiglia.
Quando l'effetto dell'aiuto dell'iracondo si esaurisce, si diventa invece Esausti. Un pacchetto basta per 1 sola volta. Creazione DC 18

\textbf{Balsamo Anti-veleno} 15 mo, -, questo balsamo alle erbe può essere applicato direttamente sulla pelle per prevenire gli effetti dei Veleni a contatto. Se una creatura tocca un veleno a contatto, ma applica su di sé il balsamo entro 1 round dal contatto, effettua il Tiro Salvezza due volte e tiene il risultato migliore. Monouso. Creazione DC 18

\textbf{Balsamo Coagulante} 30 mo, -, applicare questo balsamo alle erbe su una ferita sanguinante cura 1 danno e impedisce ulteriori danni da Sanguinamento per 1 ora per applicazione. Dopo un'ora, se l'effetto di Sanguinamento non è stato appropriatamente trattato,
La ferita riprende a sanguinare e deve essere applicato altro balsamo. Benché il balsamo coagulante possa essere applicato successivamente alla stessa ferita, applicarne più dosi non guarisce danni addizionali. La confezione è per 3 usi. Creazione DC 18

\textbf{Intruglio Fortificante} 20 mo, L, questo liquido genera una piacevole sensazione di calore quando ingerito. Per l'ora successiva, si ottiene Bonus Morale +2 ai Tiri Salvezza contro Paura. Usare più dosi nell'arco delle stesse 24 ore rende Nauseati per 1 ora. La confezione è per 3 usi. Creazione DC 18

\textbf{Tabacco Antiemetico}: 50 mo, -, questo tabacco da fiuto può essere usato per liberarsi dagli effetti della Nausea. Se lo si assume prima di entrare in contatto con un effetto che renderebbe Nauseati e che permetterebbe un Tiro Salvezza, si effettuano due Tiro Salvezza contro quell'effetto e si tiene il risultato migliore. Una singola dose fornisce questo beneficio per 1 ora. La confezione e’ per 3 usi. Creazione DC 18

\pagebreak

\subsection{Attrezzi per professioni ed artigiani}\index{Attrezzi}\index{professioni}\index{Artigiani}

\label{attrezzi-per-professioni-ed-artigiani}

Questo Equipaggiamento è particolarmente utile se si possiedono certe competenze ed abilità.

\textbf{Abaco} 2 mo, L, Questo oggetto aiuta nei calcoli matematici.

\textbf{Arnesi da Artigiano} 5 mo, L, questo è un set di arnesi speciali necessari per qualsiasi lavoro artigianale. Senza questi arnesi bisogna usare attrezzi improvvisati (penalità -2 alle prove di Artigianato) se si è costretti a realizzare comunque il lavoro.

\textbf{Arnesi da Artigiano Perfetti} 55 mo,1, come gli arnesi da artigiano questi sono gli arnesi perfetti per il lavoro, quindi forniscono Bonus +2 alle prove di Artigianato effettuate usandoli.

\textbf{Arnesi da Scasso} 30 mo, L, il set include grimaldelli e altri attrezzi da impiegare quando si usa Disattivare Congegni. Senza questi arnesi si devono utilizzare attrezzi improvvisati e si subisce penalità di circostanza -2 alle prove di Disattivare Congegni.

\textbf{Arnesi da Scasso Perfetti} 100 mo, 1, questo set contiene arnesi di fattura migliore che conferiscono Bonus +2 alle prove di Disattivare Congegni (Criminalita')

\textbf{Asta da Equilibrista} 8 ma, 1, queste aste flessibili sono lunghe da 4,5 a 9 metri e, se usate in modo appropriato, aiutano a restare in equilibrio quando si attraversa una superficie stretta. Utilizzare un'asta da equilibrista concede Bonus +2 alle prove di Acrobatica per attraversare una superficie stretta.

\textbf{Astrolabio} 100 mo,1,  questo oggetto è un disco piatto su cui sono montati altri due dischi. I dischi possono ruotare su un'asse centrale, che permette loro di muoversi con il passare dei giorni. Il disco piatto rappresenta la latitudine di chi lo utilizza, il disco superiore il cielo, pieno di indicazioni astronomiche.
Chiunque può imparare ad utilizzare l'astrolabio per conoscere data ed ora durante la notte (in 1 minuto). Un astrolabio concede Bonus +2 alle prove di Conoscenze (geografia) e Sopravvivenza per muoversi nelle zone selvagge (e alle prove di Professione (marinaio) effettuate in navigazione).

\textbf{Attrezzi da Alchimista} 25 mo, 1,un Alchimista con gli attrezzi da alchimista ha tutte le componenti materiali necessarie a creare i suoi estratti, i Mutageni e le Bombe, eccetto per quelle componenti materiali dal costo specifico. Gli attrezzi da alchimista non concedono bonus alle prove di Artigianato (alchimia).

\textbf{Attrezzi da Armaiolo} 15 mo, L, questo piccolo kit contiene tutti gli strumenti di cui una persona ha bisogno per creare, riparare e ripristinare le Armi da Fuoco, fatta eccezione per le materie prime necessarie. In mancanza di tale kit, non si può correttamente costruire o manutenzionare le Armi da Fuoco.

\textbf{Attrezzi da Cartografo} 10 mo,L, al suo interno si trovano una piccola lavagnetta con una griglia incisa sopra e diversi gessi colorati. Utilizzandoli per disegnare una mappa in viaggio si ottiene Bonus +2 alle prove di Sopravvivenza per evitare di perdersi.

\textbf{Attrezzi da Scalatore} 80 mo, 1, uesti chiodi, corde e ramponi conferiscono Bonus +2 alle prove di Scalare.

\textbf{Attrezzo Perfetto} 50 mo,L, questo oggetto di ottima fattura è l'attrezzo ideale per il lavoro richiesto e \textbf{aggiunge Bonus +2 alla relativa Prova di Competenza} (se necessaria). I bonus forniti da molteplici oggetti perfetti utilizzati per la stessa Prova di Competenza non si sommano.

\textbf{Bilancia da Mercante} 2 mo, L, una bilancia da mercante conferisce Bonus +2 alle prove di Valutare per gli oggetti la cui stima avviene in base al peso, compresa qualsiasi cosa fatta di metallo prezioso.

\textbf{Borsa del Guaritore} 50 mo, L, questa borsa piena di erbe, pomate e bende conferisce Bonus +2 alle prove di curare. Viene consumata dopo dieci utilizzi.

\textbf{Bussola} 10 mo, -, una normale bussola che punta al nord concede Bonus +2 alle prove di Sopravvivenza per evitare di perdersi. Può essere utilizzata sottoterra allo stesso scopo con le prove di Conoscenze (dungeon).

\textbf{Calderone} 1 mo ,1, questo pentolone di metallo ha un uncino per appenderlo sul fuoco. Quelli da viaggio hanno tre o quattro piedi che li tengono sollevati. Può contenere circa 3,5 litri e può essere utilizzato per cucinare, creare Pozioni e così via.

\textbf{Carrucola} 2 mo,1, questa semplice puleggia, quando fissata, aggiunge Bonus +5 alle prove di Potenza per sollevare oggetti pesanti. assicurare una puleggia richiede 1 minuto.

\textbf{Finti Sintomi} 25 mo,L, questa piccola scatola di legno ha diversi piccoli scompartimenti che ospitano oggetti utili per fingere una malattia, oltre ad un manuale che descrive i sintomi delle malattie più gravi. La scatola include false pustole, pillole che creano la schiuma alla bocca e misture di erbe che causano febbre e vomito.
Utilizzare i finti sintomi concede Bonus +5 alle prove di Camuffare per fingersi malati. Vengono consumati dopo 10 utilizzi.

\textbf{Incudine} 5 mo,1-2, anche se la taglia delle incudini varia a seconda della fucina dove viene usata, tutte hanno la stessa forma e costruzione. Le incudini da fabbro sono di solito più grandi e pesanti (45 kg) delle incudini da maniscalco (4,5 kg).
Senza un'incudine, la maggior parte dei lavori di metallurgia è impossibile.

\textbf{Laboratorio da Alchimista} 200 mo,2,  questa è l'attrezzatura perfetta per creare oggetti alchemici e conferisce Bonus +2 a qualsiasi prova di Artigianato (alchimia), ma non ha peso sui costi legati alla competenza Artigianato (alchimia).
Senza questo laboratorio, un personaggio con la competenza Artigianato (alchimia) ha comunque abbastanza attrezzi per utilizzare l'abilità, ma non per avere il bonus +2 fornito dal laboratorio

\textbf{Laboratorio da Alchimista Portatile} 75 mo,1, questa versione compatta di un laboratorio da alchimista concede Bonus +1 alle prove di Artigianato (alchimia).

\textbf{Lente d'Ingrandimento} 100 mo, -, questa semplice lente consente di osservare oggetti piccoli. E' utile come sostituto di acciarino, pietra focaia ed esca quando si accendono fuochi.
Per appiccare un fuoco con una lente d'ingrandimento ci vuole una fonte forte di luce, come la luce del sole diretta per focalizzare, esca da infiammare e tutto il round. Conferisce Bonus +2 alle prove di Valutare qualsiasi oggetto che sia piccolo o molto dettagliato, come una pietra preziosa

\textbf{Libro delle Impronte} 50 mo, L,  questo libro di 50 pagine contiene disegni accurati di tutte le impronte di animali, umanoidi e mostri, oltre che informazioni sulla lunghezza del passo, la profondità dell'impronta e altre informazioni simili. Il libro concede Bonus +2 alle prove per identificare una creatura dalle sue tracce, anche se l'uso di scarpe rende difficile o impossibile identificare gli umanoidi.
Anche se il libro non permette di riconoscere gli individui, permette di distinguere un'impronta di Troll da quella di un Ogre, o quella di un Orso da quella di un Orsogufo. Libri venduti in regioni diverse possono contenere impronte diverse, a seconda delle creature più comuni nella zona.

\textbf{Libro per Ritratti} 10 mo, L,  Questo libro di 100 pagine contiene disegni di tutte le razze presenti. Scegliendo il disegno appropriato ed aggiungendo capelli, barba ed altre caratteristiche come nei e cicatrici è possibile, anche per un pessimo disegnatore, ricostruire l'aspetto di una persona.

\textbf{Mantice} 1 mo, 1, i mantici sono utili per accendere un fuoco, e concedono Bonus +1 a simili prove di Sopravvivenza.

\textbf{Mazzo da Cartomante Comune} 1 mo, L, questo mazzo di carte illustrate è utilizzato da chi è in sintonia con il mondo degli spiriti e predice il futuro o dai ciarlatani che truffano le persone ingenue o disperate. Un mazzo comune ha semplici disegni su pergamene o semplici tavolette di legno.
Un mazzo da cartomante di qualità è di legno con immagini raffinate; può fare da focus per l'Essenza Rivelazione e concede bonus +1 alle prove di Professione (cartomante), Professione (medium) e altre prove simili.
Un mazzo da cartomante perfetto può essere di legno, avorio o anche metallo, con immagini dipinte o incise e spesso abbellite da intarsi d'oro e gemme incastonate; hatutti i benefici di un mazzo di qualità, ma concede Bonus +2 alle prove sopra menzionate.

\textbf{Mazzo da Cartomante di Qualita'} 25 mo, L

\textbf{Mazzo da Cartomante Perfetto} 50 m, L

\textbf{Sega} 4 mr, L, è possibile inserire una sega fra una porta ed il suo telaio per tagliare barre o chiavistelli di legno, infliggendo 5 danni più il bonus di Potenza, ed impiegando tutto il round.
Per sentire una sega che viene usata è necessaria una prova di Consapevolezza con DC 10. Le seghe utilizzate per tagliare il ghiaccio sui fiumi hanno una punta per spaccarlo prima di segare.

\textbf{Sestante} 500 mo, L, un sestante serve a misurare la latitudine. Concede Bonus +4 alle prove di Sopravvivenza per orientarsi in superficie.

\textbf{Strumenti per Forgiare Armi da Fuoco} 15 mo, L, questa piccola serie di strumenti contiene tutto il necessario per creare, riparare e rimettere in funzione le armi da fuoco, tranne le materie prime necessarie. Senza, non è possibile costruire o provvedere adeguatamente alla manutenzione di armi da fuoco.

\textbf{Strumento Musicale Comune} 5 mo, 1, uno strumento perfetto conferisce Bonus +2 alle prove di Intrattenere in cui viene utilizzato

\textbf{Strumento Musicale Perfetto} 100 mo, 1

\textbf{Trappola per Orsi} 2 mo. 1, ache se sono create per intrappolare grandi animali, queste trappole funzionano bene anche su umanoidi o mostri. Le fauci taglienti di queste trappole sono agganciate ad una catena, di solito assicurata al suolo così che la vittima non possa trascinarsi via. Aprire le fauci della trappola o staccarla da suolo richiede una prova di Potenza con DC 20.

\textbf{TAGLIOLA CR 1}

Tipo meccanico; Consapevolezza DC 15; Disattivare Congegni (Criminalita') DC 20

Funzionamento:  Attivatore posizione; Ripristino manuale

Effetti Tiro per Colpire +10 mischia , danni 2d6+3; fauci si chiudono attorno alla caviglia della creatura e dimezzano la velocità base della creatura (o tengono immobile la creatura se la trappola è legata ad un oggetto solido); la creatura può fuggire con una prova di Criminalità con DC 22 o una prova di Potenza con DC 26.

\textbf{Trapano} 5 ma, L, un trapano può creare un buco di 2.5 centimetri di diametro nella roccia, nel legno e nel metallo (2 Azioni). Il materiale più resistente usura o rompe il trapano più in fretta. Sentire il rumore di un trapano richiede una prova di Consapevolezza con DC 15.

\textbf{Trucchi per il Camuffamento} 50 mo, 1, questa è l'attrezzatura perfetta per camuffarsi e conferisce Bonus +2 alle prove di Camuffare. Viene consumata dopo dieci utilizzi.

\textbf{Vaso di sanguisughe} 5 mo, 1, questo resistente vaso di ceramica ha un coperchio forato che permette il passaggio dell'aria. Di norma è pieno a metà di acqua e contiene quattro sanguisughe adulte, lunghe circa 9 centimetri.
Un vaso di sanguisughe concede Bonus +2 alle prove di Guarire per trattare i Veleni. Utilizzate per i salassi medici, le sanguisughe sopravvivono per sei mesi fra un pasto e l'altro.

\pagebreak

\subsection{Cavalcature e Relativo Equipaggiamento}\index{Cavalcature}

\label{cavalcature-e-relativo-equipaggiamento}

Queste sono le cavalcature comuni che si possono trovare nelle città. Alcune città potrebbero avere delle cavalcature in piu', come cammelli o perfino grifoni, in base alla zona in cui si trovano. Queste scelte addizionali sono a discrezione del Narratore.

\subsubsection{Accessori e Varie}

\label{accessori-e-varie}

\textbf{Bardatura per Creatura Media} 30 mo La bardatura è semplicemente un tipo di armatura che copre la testa, il collo, l'addome, il corpo e possibilmente le zampe di un cavallo o di un'altra cavalcatura. Più pesante è la bardatura, migliore è la protezione e minore la velocità. Le bardature sono realizzate con ogni tipo di Armatura.

Come per qualsiasi creatura non umanoide di taglia Grande, un'armatura per un Cavallo costa quattro volte il costo di quella di un umano (cioè di una creatura umanoide di taglia Media) e pesa anche il doppio. Se la bardatura è per un Pony, o per un'altra creatura di taglia Media, il costo è solo il doppio e il peso è lo stesso di un'armatura Media indossata da un umanoide. Le bardature medie o pesanti rallentano le cavalcature come mostrato nella tabella sotto.

Le cavalcature volanti non possono Volare con bardature medie o pesanti.

Per mettere e togliere la bardatura occorre cinque volte il tempo indicato per una normale armatura. Gli animali bardati non possono essere usati per trasportare carichi che non siano il cavaliere e le normali sacche da sella.

Un cavallo bardato perde il 30\% della sua velocità.

\textbf{Finimenti per Animali} 2 mo. L, queste imbracature in pelle o canapa permettono di bloccare e controllare gli animali domestici. Finimenti preconfezionati per gli animali addomesticati più comuni, come cani, gatti, cavalli e buoi si trovano in tutti i mercati, ma possono essere creati per qualsiasi animale.

\textbf{Gabbia, Piccolissima o Minuta} 10 mo. L, queste gabbie portatili e sicure servono a contenere creature, in genere animali, ma quelle più grandi possono contenere di tutto. Le gabbie sono fatte di ferro, legno o bambu', a seconda del luogo e del mercante che le vende.

\textbf{Gabbia, Minuscola} 2 mo, L

\textbf{Gabbia, Piccola} o Media 15 mo, 1

\textbf{Gabbia, Grande} 30 mo, 3

\textbf{Gabbia, Enorme} 60 mo 6

\textbf{Morso e Briglie} 2 mo, L, una briglia è parte dell'attrezzatura usata per guidare una cavalcatura. La briglia include la testiera e il morso, che va collocato nella bocca del cavallo. A quest'ultimo sono attaccate le redini.

\textbf{Nutrimento} (al giorno) 5 mr,  1, cavalli, asini, muli e pony possono pascolare per nutrirsi, ma è molto meglio procurare loro il cibo. Se si possiede un cane da galoppo, bisogna nutrirlo almeno con un pò di carne.

\textbf{Sacche da Sella} 4 mo, 1, queste robuste borse a tenuta stagna sono appese ad una sella per incrementare la capacità di trasporto.
Ogni lato delle sacche da sella può in genere trasportare 10 kg di oggetti che possono essere contenuti in una borsa.
Tali sacche non aumentano l'ammontare di peso che una cavalcatura può trasportare, offrono semplicemente un luogo dove stivare dell'attrezzatura.

\textbf{Sella da Carico} 5 mo, 1, una sella da carico porta equipaggiamento e provviste, non un cavaliere. Una sella da carico tiene tanto equipaggiamento quanto la cavalcatura può trasportare.

\textbf{Sella da Galoppo} 10 mo, 1,se si viene colpiti e si perdono i sensi mentre si è su una sella da galoppo, si ha una probabilità del 50\% di rimanere in sella.

\textbf{Sella Militare} 20 mo, 2, una sella militare cinge il cavaliere aggiungendo Bonus +2 alle prove di Cavalcare per rimanere in sella. Se si viene colpiti e si perdono i sensi mentre si è su una sella militare, si ha una probabilità del 75\% di rimanere in sella.

\textbf{Slitta per Cani} 20 mo, 3, questa slitta è lunga un paio di metri ed è creata per essere trascinata sulla neve da una muta di cani. La maggior parte delle slitte ha una piattaforma sul fondo su cui si appoggia il cocchiere.
Una slitta per cani ha una capacità di trasporto pari a quella sommata di tutti i cani che la tirano.

\textbf{Stallaggio} (al giorno) 5 ma

\textbf{Asino o Mulo} 8 mo , 5, l'asino e il mulo sono imperturbabili di fronte al pericolo, coraggiosi, dal piede fermo e capaci di trasportare carichi pesanti per grandi distanze. Diversamente dai cavalli, sono disposti (ma non sono impazienti) ad entrare nei dungeon o in altri posti strani o minacciosi.

\textbf{Cane da Galoppo} 150 mo, 3, questo cane di taglia Media è addestrato in modo particolare per trasportare un cavaliere umanoide Piccolo. E' coraggioso in combattimento come un cavallo da guerra. Data la statura, non si subiscono danni quando si cade da un cane da galoppo.

\textbf{Cane da Guardia} 25 mo, 2, questo cane di taglia Piccola è stato addestrato alla battaglia. Ha una buona Potenza, un corpo spesso e un basso centro di massa. I cani da guardia sono venduti presso molte grandi città, ed in alcune culture sono utilizzati come combattenti per sport o impiegati in speciali unità di fanteria.

\textbf{Cavallo Leggero} 75 mo, 5, un cavallo è adatto come cavalcatura per Umani. Un pony è più piccolo di un cavallo standard ed è una cavalcatura adatta per umani piccoli. I cavalli da guerra e i pony da guerra possono essere cavalcati facilmente incombattimento.
Vedi Addestrare Animali per una lista di comandi che cavalli e pony possono conoscere se addestrati per il combattimento.

\textbf{Cavallo leggero Addestrato al Combattimento} 110 mo

\textbf{Cavallo Pesante} 200 mo, 6

\textbf{Cavallo Pesante Addestrato al Combattimento} 300 mo

\textbf{Pony} 30 mo, 5

\textbf{Pony Addestrato al Combattimento} 45 mo

\pagebreak

\subsection{Vestiario}\index{Vestiario}

\label{vestiario}

Si presuppone che un personaggio inizi il gioco con un abito del valore di 10 mo o meno. Abiti addizionali possono essere comprati normalmente.

\textbf{Abito da Artigiano} 1 mo, L, una camicia con bottoni, una gonna o pantaloni con i lacci, scarpe e forse un cappello o un berretto. Quest' abito può includere anche una cintura o un grembiule di pelle o di stoffa per tenere gli attrezzi.

\textbf{Abito da Contadino} 1 ma, L, un'ampia camicia e calzoni sformati di stoffa oppure un'ampia camicia e una gonna o sopravveste. Fasce di stoffa usate come scarpe.

\textbf{Abito da Cortigiano} 30 mo, L, eleganti abiti di sartoria in qualsiasi moda o qualunque sia lo stile diffuso nelle corti dei nobili. Chiunque tenti di influenzare nobili o cortigiani, mentre indossa abiti da strada, incontrerà notevoli difficoltà (penalità -2 alle prove basate sul Magnetismo per esercitare influenza su queste persone). Senza gioielli (che costano circa 50 mo aggiuntive) si ha l'apparenza di una persona comune fuori posto.

\textbf{Abito da Esploratore} 10 mo, 1 questo è un corredo completo di abiti per qualcuno che non sa mai cosa lo aspetta. Comprende stivali robusti, calzoni o gonna di pelle, una cintura, una camicia (magari con un panciotto o una giubba), guanti e un mantello.
Piuttosto che una gonna di pelle, si può indossare una sopravveste di pelle sopra la gonna di stoffa. Gli abiti hanno parecchie tasche (soprattutto il mantello). Il corredo include anche qualsiasi accessorio che possa essere utile, come una sciarpa o un cappello a tesa larga.

\textbf{Abito da Intrattenitore} 3 mo. L,  un corredo di abiti vistosi e forse anche appariscenti per fare spettacolo. Anche se gli abiti sembrano stravaganti, il loro taglio decisamente pratico permette di compiere acrobazie, ballare, camminare sulla corda o anche solo Correre (se il pubblico diventa minaccioso).

\textbf{Abito da Monaco} 5 mo, L, questi semplici abiti comprendono sandali, calzoni larghi e una camicia ampia, tenuti insieme da fasce. Questi abiti sono ideati per dare massima mobilità e sono fatti con stoffa di alta qualità. Si possono nascondere piccole armi nelle tasche celate nelle pieghe e le fasce sono abbastanza resistenti da servire come corde corte.

\textbf{Abito da Nobile} 75 mo, 1, questo corredo di abiti è disegnato specificamente per essere costoso e per essere esibito. Metalli e pietre preziose sono lavorati nella stoffa. Per inserirsi in un ambiente nobiliare, ogni aspirante nobile ha bisogno anche di un anello con sigillo e di gioielli (del valore di almeno 100 mo).

\textbf{Abito da Studioso} 5 mo. L, un abito lungo, una cintura, un cappello, scarpe morbide e possibilmente un mantello, sono adatti perfettamente per chi studia

\textbf{Abito da Viaggiatore} 1 mo, L, stivali, una gonna o pantaloni di lana, una robusta cintura, una camicia (magari con un panciotto o una giubba) e un ampio mantello con cappuccio

\textbf{Abito Invernale} 8 mo, 1, un soprabito di lana, camicia di lino, cappello di lana, mantello pesante, pantaloni o gonna pesanti e stivali. Quando si indossano abiti invernali, si aggiunge Bonus +5 ai Tiri Salvezza su Tempra contro l'esposizione al freddo.

\textbf{Abito Regale} 200 mo, 1,  questi sono solo gli abiti, non lo scettro, la corona, l'anello e altri oggetti regali. Gli abiti regali sono ostentati, con pietre preziose, oro, seta e pelliccia in abbondanza.

\textbf{Pelliccia} 12 mo, L, la forma più basilare di difesa dal freddo, le pellicce tengono caldo chi le indossa. Coprirsi con una pelliccia concede bonus +2 ai Tiri Salvezza su Tempra per resistere agli ambienti freddi e ai loro effetti. Non si somma ai bonus ottenuti dall'abilità Sopravvivenza.

\textbf{Racchette da Neve} 5 mo, L, eeti di corda o tendini in tensione all'interno di cornici di legno permettono di distribuire meglio il peso sulla neve, permettendo di camminarvi con maggiore facilità. Riducono le penalità dovute a camminare sulla neve, muoversi sulla neve costa 1 movimento e con le racchette costa 0 azioni movimento di penalita'

\textbf{Ramponi} 5 mo, L, utili sui terreni dove è difficile avere trazione, i ramponi sono punte o uncini che si aggiungono alla suola della scarpa. Riducono le penalità dovute al camminare su una superficie liscia, camminare sul ghiaccio è terreno difficile, ma con i ramponi no. I ramponi causano danni alle superfici delicate.

\textbf{Vesti per Ambienti Caldi} 8 mo, L, vestirsi con questi abiti leggeri e traspiranti tiene molto più fresco di quanto non accada restando nudi. Di solito comprendono una veste ampia di lino ed un turbante o velo. Questi vestiti concedono bonus +2 ai Tiri Salvezza su Tempra per resistere al caldo ed ai suoi effetti.

Questi oggetti pesano un quarto del valore se vengono fatti per personaggi di taglia Piccola, ma costano la stessa cifra.

\pagebreak

\subsection{Vitto e Alloggio}\index{Alloggio}\index{Vitto}

\label{vitto-e-alloggio}

Questi prezzi sono per vitto e alloggio nei locali commerciali di una città di media grandezza.

\textbf{Banchetto} (a persona) 10 mo --- Grande pranzo con molti invitati.

\textbf{Birra Boccale} 4 mr, 0.5Lt bevanda alcolica ottenuta dalla fermentazione del malto, dell'orzo o di altri cereali, con aggiunta aromatizzante di luppolo e altri "cose" che l'oste non di dirà mai...

\textbf{Birra Caraffa} 2 ma 4 kg, evanda alcolica ottenuta dalla fermentazione del malto, dell'orzo o di altri cereali, con aggiunta aromatizzante di luppolo e altri "cose" che l'oste non di dirà mai...

\textbf{Carne} (1 pezzo) 3 ma 0.25 kg Alimento costituito dalla parte commestibile degli animali macellati.

\textbf{Formaggio} (1 pezzo) 1 ma 0.25 kg Prodotto che si ricava dal latte per coagulazione

\textbf{Locanda Buona} (al giorno) 2 mo Un alloggio scadente in una locanda consta di un posto sul pavimento vicino al camino. Un alloggio normale è un posto su un pavimento sollevato e riscaldato, con una coperta e un cuscino. Un buon alloggio è una piccola stanza privata con un letto, qualche comodità e un vaso da notte coperto in un angolo.

\textbf{Locanda Normale} (al giorno) 5 ma

\textbf{Locanda Scadente} (al giorno) 2 ma

\textbf{Pane} (a pagnotta) 2 mr 0.25 kg

\textbf{Pasti Buono} (al giorno) 5 ma --- Un pasto scadente può essere composto da pane, rape cotte, cipolle e acqua. Un pasto normale può comprendere pane, stufato di pollo, carote e birra o vino annacquati. Un buon pasto può essere composto da pane e dolci, manzo, piselli e birra o vino.

\textbf{Pasti Normale} (al giorno) 3 ma ---

\textbf{Pasti Scadente} (al giorno) 1 ma ---

\textbf{Vino Comune} (caraffa) 2 ma 1 lt Bevanda alcolica ottenuta dal mosto d'uva fatto fermentare.

\textbf{Vino Buono} (bottiglia) 10 mo 1 lt Bevanda alcolica ottenuta dal mosto d'uva fatto fermentare.

\pagebreak

\subsection{Trasporti}\index{Trasporti}

\label{trasporti}

I prezzi indicati sono per comprare il veicolo, escluso ciurme o animali.

\textbf{Barca a remi} 50 mo, 2, una barca lunga tra i 2,4 e i 3,6 metri, a due remi, per due o tre persone di taglia Media. Si muove alla velocità di 2,25 km/h.

\textbf{Barcone} 3.000 mo, 6, una barca lunga tra i 15 e i 22.5 metri e larga tra i 4,5 e i 6 metri. Dotata di pochi remi per integrare il suo unico albero con vela quadrata, ha un equipaggio variabile dalle 8 alle 15 unità. Può trasportare dalle 40 alle 50 tonnellate di carico oppure 100 soldati. Può sia compiere traversate per mare che viaggiare lungo il corso dei fiumi (ha la chiglia piatta). Viaggia a una velocità di 1,5 km/h.

\textbf{Carretto} 15 mo, 3, un veicolo a due ruote trainato da un solo cavallo (o altro animale da soma). Comprende anche i finimenti.

\textbf{Carro} 35 mo, 4, questo è un veicolo aperto a quattro ruote per trasportare carichi pesanti. In genere, lo tirano due cavalli (o altre bestie da soma). Comprende anche i finimenti necessari per tirarlo.

\textbf{Carrozza} 100 mo, 6, questo veicolo a quattro ruote può trasportare fino a quattro persone in una cabina chiusa più i due conducenti. In genere, lo tirano due cavalli (o altre bestie da soma). Comprende anche i finimenti necessari per tirarlo.

\textbf{Galea} 30.000 mo, una nave a tre alberi con 70 remi su ciascun lato e un equipaggio totale di 200 unità. Ha una lunghezza di 39 metri e una larghezza di 6 metri.
Può trasportare fino a 150 tonnellate di carico o 250 soldati. Con l'aggiunta di 8.000 mo può essere dotata di sperone e castelli con piattaforme di lancio a prua, a poppa e a metà scafo. Questa nave non può affrontare viaggi in mare aperto e si mantiene vicina alla costa.
Viaggia alla velocità di circa 6 km/h, se si impiegano i remi o le vele.

\textbf{Nave a vela} 10.000 mo , questa nave ha una lunghezza compresa tra i 22.5 e i 27 metri e una larghezza di 6 metri, un equipaggio di 20 unità e capacità di trasporto fino a 150 tonnellate. Dotata di due alberi con vele quadrate, può compiere traversate per mare. Raggiunge una velocità di circa 3 km/h.

\textbf{Nave da guerra} 25.000 mo, una nave della lunghezza di 30 metri, ad albero unico, e possibilità di usare i remi. Ha un equipaggio variabile dai 60 agli 80 rematori. Può trasportare fino 160 soldati, ma non per lunghe distanze, poiché non vi è lo spazio sufficiente a stivare le provviste che sarebbero necessarie a un tal numero di soldati. Non può compiere traversate per mare e resta vicina alla costa. Non viene usata per il trasporto merci. Viaggia alla velocità di 3,75 km/h se si usano i remi o la vela.

\textbf{Nave Lunga} 10.000 mo, una nave della lunghezza di 22.5 metri, con 50 remi e un equipaggio totale di 50 unità. Ha un unico albero con una vela quadrata. Può trasportare fino a 50 tonnellate di carico oppure 120 soldati. Può compiere traversate in mare aperto. Viaggia alla velocità di 4,5 km/h se si impiegano i remi o la vela.

\textbf{Remo} 2 mo, 1, un remo di 2 metri per una barca

\textbf{Slitta} 20 mo, 4, si tratta di un carro su pattini adatto per muoversi sulla neve e sul ghiaccio. Di solito, la tirano due cavalli (o altre bestie da soma). Comprende anche dei finimenti necessari per trascinarla.

\pagebreak

\subsection{Magie e Servizi}\index{Magie e Servizi}\index{Servizi}

\label{magie-e-servizi}

Talvolta la migliore soluzione a un problema è affidarsi a qualcun altro che lo possa risolvere.

\textbf{Diligenza Pubblica} 3 mr per 1,5 Km Il prezzo indicato vale per una corsa su una diligenza che trasporta persone (e bagagli leggeri) tra due città. Su una diligenza che trasporti persone entro la medesima città, una corsa costa 1 mr, e permette solitamente di arrivare ovunque si voglia.

\textbf{Livello di Potere} \texttimes 50 mo (per lP) Questo è il costo per avere un incantatore che manipola le Essenze. Questo costo presuppone che si possa andare dall'incantatore e chiedergli di manipolare una certa Essenza a proprio piacimento (solitamente gli servono almeno 8 ore per prepararsi). Se si vuole portare da qualche parte l'incantatore per fargli usare l'Essenza è necessario negoziare con lui, e la risposta di base è "no".

Se l'Essenza ha conseguenze pericolose, l'incantatore deve ricevere delle prove certe che il personaggio ha la possibilità di pagare e che non mancherà di farlo nel caso queste conseguenze si verifichino (sempre che accetti di manipolare l'Essenza richiesta, cosa nient'affatto sicura). Quando si tratta di Essenze che trasportano il personaggio e l'incantatore lungo una distanza, è necessario pagare l'Essenza due volte anche se il personaggio non desidera tornare indietro con l'incantatore.

Non tutti i villaggi e i paesi hanno un incantatore abbastanza capace a manipolare essenze. Come regola generale, è necessario spostarsi almeno in un piccolo paese per essere abbastanza sicuri di trovare un incantatore in grado di lanciare il Livello di Potere richiesto. In un piccolo paese si potrebbe trovare un incantatore in grado di lanciare Essenze di Punti Potere 11, in un grande paese per quelli di Punti Potere 13, in una piccola città per quelli di Punti Potere 15, in una grande città per quelli di livello potere 15-24, in una metropoli per quelli di livello potere 25+. Nemmeno in una metropoli si è certi di trovare un incantatore capaci di lanciare magie con Punti Potere 34+.

\textbf{Mercenario Esperto} 3 ma al giorno Il prezzo indicato è la paga giornaliera di artigiani, carrettieri, muratori, scrivani, soldati di ventura e altri aiutanti abili in un mestiere. Il valore rappresenta la paga minima, perché alcuni aiutanti addestrati chiedono molto di più.

\textbf{Mercenario Normale} 1 ma al giorno Il prezzo indicato è la paga giornaliera di camerieri, cuochi, facchini, operai e altri semplici lavoratori.

\textbf{Messaggero} 2 mr per 1,5 Km Questo termine include sia i messaggeri a cavallo che quelli a piedi. Se accettano di consegnare un messaggio perché il destinatario si trova in un luogo dove erano comunque diretti, potrebbero chiedere metà della somma indicata.

\textbf{Pedaggio Stradale o d'Ingresso} 5 mr Può capitare di dover pagare una tassa per transitare su una strada molto frequentata, ben sorvegliata e tenuta in buone condizioni, per il pattugliamento e la manutenzione. Occasionalmente le grandi città fortificate richiedono il pagamento di un pedaggio all'ingresso o all'uscita della città stessa (a volte solo all'ingresso).

\textbf{Passaggio in Nave} 1 ma per 1,5 Km La maggioranza delle navi non è attrezzata per il trasporto di passeggeri, ma molte hanno la capienza per imbarcarne alcuni a bordo mentre trasportano le merci. Raddoppiare il costo indicato per creature di taglia superiore a quella Media o che sono difficili da stivare nella nave.

\subsection{Oggetti da Intrattenimento}\index{Intrattenimento}

\label{oggetti-da-intrattenimento}

\textbf{Carte Segnate} 1 mo,-, che siano piegate, colorate o graffiate, le carte segnate permettono a chi ne fa uso di riconoscere la carta a seconda dei segni fatti sul suo retro. Accorgersi di carte segnate richiede una prova di Consapevolezza con DC 25.

\textbf{Dadi Truccati Normali} 10 mo, -, la maggior parte dei dadi truccati è appesantita da una sostanza più pesante situata all'opposto del numero che si desidera. E' possibile accorgersi di un dado truccato con una prova di Consapevolezza o Valutare con DC 15. I dadi di qualità superiore (ad esempio, dadi di legno intagliati attorno ad una occlusione più pesante) hanno DC maggiori che possono andare da 20 a 30.

\pagebreak

\subsection{Materiali Speciali}\index{Materiali Speciali}

Le armature e le armi si possono costruire con materiali che possiedono delle innate qualità speciali. Se si costruisce un'armatura o arma con più di un materiale speciale, si ricevono i benefici solo del materiale prevalente. Si può però costruire un'arma doppia con ogni testa fatta di un materiale speciale diverso.

\subsubsection{Acciaio Forgiato a Caldo}\index{Acciaio Forgiato a Caldo}

\label{acciaio-forgiato-a-caldo}

\begin{tabular}{ll}
	\toprule
	\textbf{Tipo di oggetto in Acciaio Forgiato a Caldo} & \textbf{Modificatore al costo}\\
	Munizione			& +15 mo per munizione\\
	Arma				& +600 mo\\
	Armatura leggera    & +1000 mo\\
	Armatura media      & +2500 mo\\
	Armatura pesante    & +3000 mo\\
\end{tabular}

I grandi fabbri si sono imbattuti nel segreto della lavorazione dell'acciaio
forgiato a caldo, nel tentativo di creare strumenti facilmente utilizzabili
in fucina. Non ci volle molto tempo per adattare le sue proprieta'
uniche ad armi e armature. L'acciaio forgiato a caldo incanala il
calore in una sola direzione per proteggere chi lo indossa o chi lo
impugna.

\subsubsection{Acciaio Forgiato a Freddo}\index{Acciaio Forgiato a Freddo}

\label{acciaio-forgiato-a-freddo}

\begin{tabular}{ll}
	\toprule
	\textbf{Tipo di oggetto in Acciaio Forgiato a Freddo} & \textbf{Modificatore al costo}\\
	Munizione               & +15 mo per munizione\\
	Arma                    & +600 mo\\
	Armatura leggera        & +1000 mo\\
	Armatura media          & +2500 mo\\
	Armatura pesante        & +3000 mo\\
\end{tabular}

\subsubsection{Acciaio Vivente}\index{Acciaio Vivente}

\label{acciaio-vivente}

\begin{tabular}{ll}
	\toprule
	\textbf{Tipo di oggetto in Acciaio Vivente} & \textbf{Modificatore al costo}\\
	Munizione                         & +10 mo per munizione\\
	Arma                              & +600 mo\\
	Armatura leggera                  & +600 mo\\
	Armatura media                    & +1000 mo\\
	Armatura pesante                  & +1500 mo\\
	Scudo                             & +100 mo\\
	Altri oggetti                     & 500 mo/kg\\
\end{tabular}

\subsubsection{Adamantio}\index{Adamantio}

\label{adamantio}

\begin{tabular}{ll}
	\toprule
	\textbf{Tipo di oggetto in Adamantio} & \textbf{Modificatore al costo}\\
	Munizione                   & +60 mo per munizione\\
	Arma                        & +1500 mo\\
	Armatura leggera            & +5000 mo\\
	Armatura media              & +10000 mo\\
	Armatura pesante            & +15000 mo\\
	Scudo                       & +1000 mo\\
	Altri oggetti               & 5000 mo/kg\\
\end{tabular}

Questo metallo durissimo si trova solo nei meteoriti e contribuisce alla qualità di un'arma o di un'armatura. Le armi in adamantio sono estremamente resistenti ed hanno Durezza 25

Quindi le armi e le munizioni in adamantio hanno Bonus di +1 ai Tiri per Colpire, e la penalità date dall'armatura (agilità e CM) viene diminuita di 1 rispetto ad una normale armatura del suo stesso tipo. Gli oggetti senza parti metalliche non possono essere costruiti con l'adamantio. Una freccia può essere in adamantio, ma un bastone ferrato
no.

\subsubsection{Argento Alchemico}\index{Argento Alchemico}

\label{argento-alchemico}

\begin{tabular}{ll}
	\toprule
	\textbf{Tipo di Oggetto in Argento Alchemico} & \textbf{Modificatore al costo}\\
	Munizione                      & +2 mo per munizione\\
	Arma leggera                   & +20 mo\\
	Arma media                     & +90 mo\\
	Arma pesante                   & +180 mo\\
	Scudo                         & +100 mo\\
\end{tabular}

Il processo di argentatura alchemica non può essere applicato alle armi non metalliche, e non funziona sui metalli speciali come l'adamantio, il ferro freddo e il mithral.

\subsubsection{Ferro Freddo}\index{Ferro Freddo}

\label{ferro-freddo}

Questo ferro viene estratto nelle profondità del sottosuolo ed è noto per la sua efficacia contro demoni e folletti. Viene forgiato ad una temperatura inferiore per conservare le sue delicate proprietà. Costruire armi fatte di ferro freddo costa il doppio rispetto alle loro normali controparti. Inoltre qualsiasi potenziamento magico costa 2.000 mo addizionali. Questo aumento viene applicato la prima volta che l'oggetto viene potenziato, non una volta per qualità aggiunta.

Gli oggetti senza parti di metallo non possono essere costruiti in ferro freddo. Una freccia potrebbe essere fatta di ferro freddo ma un randello no. Un'arma doppia che è fatta solo per metà di ferro freddo aumenta il suo costo del 50\%.

\subsubsection{Mithral}\index{Mithral}

\label{mithral}

\begin{tabular}{ll}
	\toprule
	\textbf{Tipo di Oggetto in Mithral} & \textbf{Modificatore al costo}\\
	Armatura leggera                    & +1000 mo\\
	Armatura media                      & +4000 mo\\
	Armatura pesante                    & +9000 mo\\
	Scudo                               & +1000 mo\\
	Altri oggetti                       & +1000 mo/kg\\
\end{tabular}

\bigskip

Il mithral è un metallo molto raro, luccicante, simile all'argento, più leggero del ferro ma altrettanto duro. Quando viene lavorato come l'acciaio, diventa un meraviglioso materiale con cui creare armature, e occasionalmente viene usato anche per altri oggetti. La maggior parte delle armature in mithral è più leggera di una categoria del normale, ed è più agevole per il movimento e le altre limitazioni. Le armature pesanti sono trattate come armature medie, e le armature medie sono trattate come leggere, ma le armature leggere restano leggere.

Questa diminuzione non si applica alla competenza necessaria per indossare l'armatura in questione (per indossare un'armatura pesante di mithral occorre avere CA 3, anche se questa viene considerata come media per altri fattori). Occorre essere competenti nel tipo di armatura appropriato, altrimenti si incorre nelle relative penalità come di norma.

Le probabilità di fallimento di una Essenza per armature e scudi in mithral diminuiscono di 5 punti e la penalità ad agilità diminuiscono di 3 (fino a un minimo di 0), le penalità al movimento diminuiscono di 1 metro.

\subsubsection{Pelle di Drago}\index{Pelle di Drago}

\label{pelle-di-drago}

I fabbricanti di armature possono lavorare le pelli dei draghi per produrre armature o scudi.
Un drago fornisce pelle sufficiente per una singola armatura di pelle per una creatura di una taglia più piccola del drago. Selezionando solo le scaglie e le parti di pelle migliori, un fabbricante di armature può produrre una corazza di bande per una creatura di due taglie più piccola, una mezza armatura per una creatura tre taglie più piccola e una corazza di piastre o un'armatura completa per una creatura di quattro taglie più piccola.

In ogni caso, c'è sempre pelle sufficiente per produrre uno scudo leggero o pesante in aggiunta all'armatura, purché il drago sia Grande o maggiore.
e la pelle di drago proviene da un Drago che ha immunità ad un tipo di energia, anche l'armatura è immune a quel tipo di energia, sebbene non conferisca alcuna protezione a chi la indossa. Se allo scudo o all'armatura viene conferita in seguito la capacità di proteggere chi la indossa da un tipo di energia specifico, il costo di questo potenziamento viene ridotto del 25\%.

Le armature di pelle di drago costano il doppio di un'armatura di quel tipo, ma non richiedono più tempo per essere costruite (si raddoppino tutti i risultati di Artigianato).

\pagebreak

\section{Sfondare ed Entrare}\index{Sfondare}\index{Entrare}

\label{sfondare-ed-entrare}

Quando si tenta di spaccare un oggetto le scelte sono due: colpirlo con un'arma o romperlo con la forza bruta.

\textbf{Tabella: Taglia e Difesa degli Oggetti - Colpire un Oggetto}
\medskip

\begin{tabular}{ll}
	\toprule
	\textbf{Taglia e Difesa degli Oggetti} & \textbf{Modificatore Difesa}\\
	Colossale                              & -8\\
	Mastodontica                           & -6\\
	Enorme                                 & -4\\
	Grande                                 & -2\\
	Media                                  & +0\\
	Piccola                                & +2\\
	Minuscola                              & +4\\
	Minuta                                 & +6\\
	Piccolissima                           & +8\\
\end{tabular}

\bigskip

\textbf{Modificatore Difesa}: Gli oggetti sono più facili da colpire delle creature poiché di solito non si muovono, ma molti sono abbastanza resistenti da scrollarsi di dosso qualche danno ad ogni colpo. La Difesa di un oggetto è pari a 10 + il suo modificatore di Taglia (vedi Tabella: Colpire un Oggetto) + il suo modificatore di Agilità (caso mai ne avesse uno).

Se si usano 3 Azioni per prendere la mira, si colpisce automaticamente con un'arma da mischia e si ottiene bonus +1d6 al colpire con un'arma a distanza.

\textbf{Durezza}

Questi conteggi non vengono applicati ad Armature e Scudi che seguono le loro tabelle di resistenza e durezza.

\bigskip

\begin{tabular}{lll}
	\toprule
	\textbf{Sostanza} & \textbf{Durezza} & \textbf{Punti Ferita} \\
	Vetro             & 1                & 1 ogni 2,5 cm di spessore\\
	Carta o stoffa    & 0                & 2 ogni 2,5 cm di spessore\\
	Corda             & 0                & 2 ogni 2,5 cm di spessore\\
	Ghiaccio          & 0                & 3 ogni 2,5 cm di spessore\\
	Cuoio o pelle     & 2                & 5 ogni 2,5 cm di spessore\\
	Legno             & 5                & 10 ogni 2,5 cm di spessore\\
	Pietra            & 15               & 15 ogni 2,5 cm di spessore\\
	Ferro o acciaio   & 10               & 10 ogni 2,5 cm di spessore\\
	Mithral           & 15               & 30 ogni 2,5 cm di spessore\\
	Adamantio         & 20               & 40 ogni 2,5 cm di spessore\\
\end{tabular}

\bigskip

\textbf{Attacchi di Energia}: Gli attacchi di energia (fuoco, elettricità..) infliggono metà danno alla maggior parte degli oggetti; dividere per 2 i danni prima di applicare la Durezza. Alcuni tipi di energia possono essere particolarmente efficaci contro certi oggetti, a discrezione del Narratore.
Per esempio, il fuoco potrebbe infliggere danno pieno a pergamene, stoffa e altri oggetti che bruciano facilmente. Un attacco sonoro potrebbe causare danno pieno (massimo valore dei dadi) ad oggetti di vetro e cristallo o ceramiche.

\textbf{Danni da Armi a Distanza}: Gli oggetti subiscono la metà dei danni da un'arma a distanza (tranne che per le Macchine d'Assedio e simili). Dividere per 2 i danni prima di applicare la Durezza dell'oggetto.

\textbf{Armi Inefficaci}: Certe armi semplicemente non possono infliggere danni a certi oggetti. Per esempio, un'arma contundente non è in grado di tagliare una corda.
Allo stesso modo è decisamente difficile abbattere una porta o un muro di pietra con la maggior parte delle armi da mischia, a meno che non siano specificamente ideate per farlo, come picconi e martelli.

\textbf{Immunita'}: Gli oggetti inanimati sono immuni ai Danni Non Letali e ai Colpi Critici. Anche gli oggetti animati, se non considerati come delle creature, hanno queste immunità.

\textbf{Armi, Armature e Scudi Magici}: Ogni +1 di Bonus aggiunge anche 2 alla Durezza ad armi, armature e scudi, e +20 alla resistenza dell'oggetto.

\textbf{Vulnerabilità a Certi Attacchi}: Certi attacchi possono essere particolarmente efficaci contro alcuni oggetti. In questi casi gli attacchi infliggono danni raddoppiati e possono ignorare la Durezza dell'oggetto.

\textbf{Oggetti Danneggiati}: Un oggetto danneggiato rimane pienamente funzionale con la condizione Rotto fino a quando i Punti Ferita non arrivano a 0, e a quel punto è considerato distrutto. Gli oggetti danneggiati (ma non quelli distrutti) possono essere riparati con la Competenza Artigianato e alcune Essenze (vedi la condizione Rotto per maggiori dettagli).

\textbf{Tiro Salvezza}: Gli oggetti non magici incustoditi non effettuano mai Tiro Salvezza. Si considera che abbiano fallito i loro Tiro Salvezza, e siano quindi sempre soggetti ad Essenze ed altri attacchi che ammettono un Tiro Salvezza per resistere o negare l'effetto.

Un oggetto custodito da un personaggio (che lo tenga in mano, lo tocchi o lo indossi) ottiene un Tiro Salvezza proprio come se lo stesse effettuando il personaggio (cioè usando il suo bonus al Tiro Salvezza).

\textbf{Gli Oggetti Magici hanno sempre Tiro Salvezza}. Il bonus ai Tiri Salvezza su Tempra, Riflessi o Volontà di un Oggetti Magico sono pari a 2 + metà Livello dell'incantatore che li ha creati.

Gli Oggetti Magici custoditi effettuano i Tiri Salvezza come il loro possessore oppure usano i loro Tiri Salvezza, quali che siano i migliori.

\bigskip

\begin{tabular}{llll}
	\toprule
	\textbf{Oggetto}& \textbf{Durezza} & \textbf{Punti Ferita} & \textbf{DC per Romperlo}\\
	Corda (2,5 cm di diametro)     & 0      & 2           & 23\\
	Porta di legno semplice        & 5      & 10          & 13\\
	Cassa piccola                  & 5      & 1           & 17\\
	Porta di legno buona           & 5      & 15          & 15\\
	Cassa del tesoro               & 5      & 15          & 23\\
	Porta di legno robusta         & 5      & 20          & 18\\
	Muro di pietra (spesso 30 cm)  & 8      & 90          & 35\\
	Pietra tagliata (spessa 90 cm) & 8      & 540         & 50\\
	Catena                         & 10     & 5           & 26\\
	Manette                        & 10     & 10          & 26\\
	Manette perfette               & 10     & 10          & 28\\
	Porta di ferro (spessa 5 cm)   & 10     & 60          & 28\\
\end{tabular}

\bigskip

Oggetti animati: Gli oggetti animati contano come creature per determinarne la Difesa (non sono considerati oggetti inanimati).

\subsection{Rompere Oggetti}\index{Rompere Oggetti}

\label{rompere-oggetti}

Quando si tenta di rompere qualcosa con Potenza improvvisa piuttosto che infliggendo danni regolari, bisogna effettuare una prova di Potenza (invece di un Tiro per Colpire e per il danno, come per Spezzare) per vedere se ci si riesce.

Poiché la Durezza non influisce sulla DC per rompere l'oggetto, questo valore dipende più dal modo in cui è costruito l'oggetto che non dal materiale. Vedi Tabella: DC per Rompere o forzare Oggetti per una lista delle DC più comuni relative al rompere gli oggetti.

Creature di Taglia superiore o inferiore a quella Media hanno bonus o penalità di taglia sulla prova di Potenza per sfondare una porta:

\bigskip

\begin{tabular}{ll}
	\toprule
	\textbf{Taglia} & \textbf{Modificatore per sfondare porta}\\
	Piccolissima    & -16\\
	Minuta          & -12\\
	Minuscola       & -8\\
	Piccola         & -4\\
	Normale         & +0\\
	Grande          & +4\\
	Enorme          & +8\\
	Mastodontica    & +12\\
	Colossale       & +16\\
\end{tabular}

\bigskip

Un piede di porco o un ariete portatile aumentano la probabilità del personaggio di sfondare una porta (+1d6)

\subsubsection{Tabella: DC per Rompere o Forzare oggetti - Prova di Potenza}

\label{tabella-dc-per-rompere-o-forzare-oggetti---prova-di-potenza}
\bigskip

\begin{tabular}{ll}
	\toprule
	\textbf{Cosa abbatti}        & \textbf{DC}\\
	Abbattere una porta semplice & 13\\
	Abbattere una porta buona    & 15\\
	Abbattere una porta robusta  & 18\\
	Forzare corde legate         & 23\\
	Piegare sbarre di ferro      & 24\\
	Abbattere una porta robusta  & 25\\
	Forzare catene legate        & 26\\
	Abbattere una porta di ferro & 28\\
\end{tabular}
\bigskip

\pagebreak

\section{Ambiente}\index{Ambiente}

\label{ambiente}
\begin{tcolorbox}[enhanced,arc=5pt,boxrule=0.3pt]{La natura non è crudele, è solo spietatamente indifferente. Questa è una delle più dure lezioni che un essere umano debba imparare. (Richard Dawkins)\\

L'antidoto principale contro un cattivo ambiente consiste, naturalmente, nel sostituirlo con uno buono. (Robert Baden-Powell)}\end{tcolorbox}\medskip

Dai deserti senza vita ai dungeon zeppi di trappole, l'ambiente aiuta a definire il mondo. Renderlo vivo, dinamico e ricco consente di creare un'esperienza di gioco emozionante e coinvolgente. Questo capitolo contiene le regole per aiutare il Narratore a definire il mondo di gioco, come dungeon, trappole, terreni e pericoli ambientali.

\subsection{Regole Ambientali}

\label{regole-ambientali}

I pericoli relativi a un tipo di terreno specifico sono descritti in Avventure nelle Terre Selvagge. I rischi ambientali comuni a più di un terreno sono invece descritti di seguito.

\subsubsection{Visione e Luce}\index{Visione}\index{Luce}

\label{sec:visione-e-luce}

In un ambiente naturale l'illuminazione può assumere diverse gradazioni e queste gradazioni aiutano a capire fino a che distanza una creatura può vedere.

Le gradazione di luce possono essere:
\begin{itemize}
	\item
	      \textbf{Oscurita}': buio pesto, può essere naturale o magico
	\item
	      \textbf{Penombra}: la poca illuminazione permette di riconoscere le sagome
	\item
	      \textbf{Luce}: una luce brillante, coprente, assolata.
\end{itemize}

\medskip

\textbf{Tabella delle fonti di luce}

\medskip

\index{Luce estesa}\index{Penombra}

\begin{tabular}{llll}
	\toprule
	\textbf{Oggetto} & \textbf{Luce Normale (raggio)} & \textbf{Luce estesa/Penombra} & \textbf{Durata}\\
	Candela          & 1 metro         & -              & 1 ora\\
	Torcia           & 3 metri         & 6 metri        & 1 ora\\
	Lanterna         & 6 metri         & 12 metri       & 6 ore/boccetta\\
\end{tabular}

\bigskip

In linea di massima una fonte di luce crea luce estesa in un raggio doppio rispetto alla luce normale.\\

Stare alla luce delle stelle con luna quasi piena è essere in penombra (+2 Difesa)\\

Una creatura con \textbf{Visione Normale} \index{Visione Normale}vede fino alla distanza, come raggio circolare intorno alla fonte di luce, indicato in Luce Normale. Oltre e' Penombra e oltre ancora Oscurita'.\\

Una creatura con \textbf{Visione Crepuscolare} \index{Visione Crepuscolare}vede fino alla distanza, come raggio circolare intorno alla fonte di luce, indicato in Luce estesa, o indicato dalla razza se minore, oltre è oscurità.\\

Una creatura con \textbf{Scurovisione} \index{Scurovisione} vede fino alla distanza indicata dalla sua capacita' di Scurovisione, indipendentemente che ci sia luce o meno, oltre non puo' vedere.
La Scurovisione e' una visione in bianco e nero.\\
\smallskip
\textbf{Oscurita'}\index{Oscurita'}: è il buio più completo senza alcuna fonte di luce.

Per creature con visione normale l'oscurità è ciò che c'è oltre la Luce Estesa.

Il \textbf{personaggio cieco}\index{Cieco} ha -4 alla Consapevolezza visiva e tutti gli avversari sono invisibili.\\

\textbf{Penombra}: può essere una notte stellata oppure luce estesa.\index{Penombra}

Per una creatura con visione normale la penombra concede agli avversari un Copertura parziale, ovvero +2 alla Difesa.

\textbf{Luce}

Per una creatura con visione normale è come essere al centro del raggio di illuminazione, oppure sotto il sole.

\subsubsection{Buio}\index{Buio}

\label{buio}

Le torce e le lanterne possono essere spente all'improvviso da una folata di vento, le fonti di luce magiche possono essere dissolte o contrastate ed alcune trappole magiche possono creare aree di buio impenetrabile.

In certi casi, alcuni personaggi o mostri potrebbero essere in grado di vedere mentre gli altri sono Accecati. Ai fini delle regole che seguono, una creatura Accecata è semplicemente una creatura che non è in grado di vedere nelle tenebre circostanti.

\subsubsection{Accecato}\index{Accecato}

\label{accecato}

Le creature Accecate perdono la loro capacità di infliggere danni extra causati ad esempio dall'Abilità di pugnalare alle spalle.

Le creature accecate devono effettuare una prova di Acrobatica con DC 10 per muoversi più velocemente della loro velocità dimezzata. Se la prova fallisce cadono a terra prone. Le creature accecate non possono correre o caricare.

Tutti gli avversari di una creatura accecata sono invisibili nei suoi confronti ovvero hanno un +8 alla Difesa.

Una creatura accecata, o che combatte contro una creatura invisibile,\index{invisibile} può effettuare una prova di Consapevolezza a difficoltà 20 (oppure 15+Criminalità dell'avversario se questo non vuole farsi trovare) per individuare la creatura purché questa sia entro area di mischia dal personaggio.

Se la prova riesce è possibile tentare l'attacco a -1d6 Tiro per colpire, la creatura gode sono di un +4 alla Difesa, tranne l'attacco avviene con Essenze/attacco ad area.

Una creatura Accecata \index{Accecata}subisce penalità -4 alle prove di Consapevolezza e alla maggior parte delle prove basate su Potenza e Agilità e fallisce automaticamente qualsiasi check di Competenze dipendenti dalla vista.

Inoltre, una creatura accecata dal buio non può usare Essenze che prevedano l'uso dello sguardo ed è immune alle Essenze che prevedono lo sguardo.

Se una creatura Accecata viene colpita da un nemico non visto, riesce a individuare la posizione attuale della creatura che lo ha colpito (finché la creatura non si muove, naturalmente). L'unica eccezione avviene se la creatura usa un attacco a distanza (nel qual caso il personaggio Accecato sa la direzione generica del nemico, ma non la sua posizione precisa).

Se viene quindi colpito in mischia si considera come se la prova di Consapevolezza per determinare l'avversario sia riuscito (-1d6 al Tiro per Colpire, +4 Difesa)

\subsubsection{Cadute}\index{Cadute}

\label{cadute}

Le creature che cadono subiscono 1d6 danno per cadute da altezze di 3 metri, più 1d6 ogni 3 metri oltre i 3. Dividi l'altezza in metri per 3, arrotonda per difetto, il numero che risulta sono i d6 di danno subiti. Es 16 metri di caduta sono 16/3=5d6 di danno. I danni da caduta non possono superare i 20d6 di danno (anche se la caduta è da kilometri di altezza).

Le creature che subiscono danni letali da una caduta, atterrano in posizione prona.

Una prova di Acrobatica riuscita con DC 15 permette al personaggio di dimezzare il danno quando cade da meno di 20 metri.

Cadute su superfici morbide (terreno morbido, fango ecc.) convertono i primi 1d6 danni in Danni Non Letali. Questa riduzione è cumulativa con la diminuzione del danno per l'uso della competenza Acrobatica.

Un personaggio non può utilizzare Essenze magiche mentre cade, a meno che la caduta non sia superiore a 150 metri o l'Essenza magica sia rapida. Utilizzare un'Essenza magica mentre si cade richiede una prova di Concentrazione con DC pari a 20.

\textbf{Cadere in Acqua}\index{Cadere in Acqua}

Le cadute in acqua sono gestite in modo leggermente diverso. Fino a quando l'acqua ha una profondità di almeno di 3 metri ed il tuffo è da una altezza entro 12 metri non si subiscono danni.

Si subiscono 2d6 di danni da una caduta oltre i 15 metri e 5d6 per cadute oltre i 15 metri.

I personaggi che si tuffano volontariamente in acqua non subiscono danni se superano una prova di Acrobatica o di Nuotare (Resistenza) con DC 15 se l'acqua è profonda almeno 6 metri. La DC della prova aumenta di 5 ogni 5 metri oltre i 15.

\subsubsection{Effetti dell'Acido}\index{Acido}

\label{effetti-dellacido}

Gli acidi corrosivi infliggono 1d6 danni per round di esposizione, tranne nel caso di totale immersione (come in una vasca d'acido), che infligge 10d6 danni per round. Un attacco con l'acido, come quello di una boccetta lanciata o la saliva/soffio di un mostro, deve essere considerato come un round di esposizione.

I vapori prodotti dalla maggior parte degli acidi sono equivalenti a veleni inalati. Coloro che si avvicinano molto ad un grosso ammasso di acido devono effettuare un Tiro Salvezza su Tempra con DC 13 o subiranno 1 danno alla Potenza a round. Questo veleno non ha frequenza, pertanto una creatura è salva se si allontana dall'acido.

Le creature immuni alle proprietà caustiche dell'acido potrebbero comunque annegare se vi vengono totalmente immerse (vedi Annegamento).

\subsubsection{Effetti del Fumo}\index{Fumo}

\label{effetti-del-fumo}

Un personaggio costretto a respirare del fumo denso deve superare un Tiro Salvezza su Tempra ogni round (DC 15, +1 per ogni prova precedente) oppure passa il round a tossire e soffocare. Un personaggio che continua a soffocare per 2 round consecutivi subisce 1d6 Danni Non Letali. Il fumo oscura la vista, fornendo Copertura (+2 Difesa) ai personaggi che si trovano al suo interno.

\subsubsection{Fame e Sete}\index{Fame}\index{Sete}

\label{fame-e-sete}

I personaggi potrebbero trovarsi senz'acqua o cibo e privi dei mezzi per procurarsene. Nei climi normali, i personaggi di taglia Media hanno bisogno di almeno 2 litri di liquidi e 0.5 kg di cibo decente al giorno per evitare la fame. (I personaggi di taglia Piccola necessitano della meta'). Nei climi molto caldi, i personaggi possono aver bisogno di due o tre volte quella quantità d'acqua per evitare la disidratazione.

\subsubsection{Oggetti Cadenti}\index{Oggetti Cadenti}

\label{oggetti-cadenti}

Proprio come i personaggi subiscono danni dalle cadute superiori a distanze di mischia, allo stesso modo subiscono danni se vengono colpiti da oggetti cadenti.

Gli oggetti che cadono addosso ai personaggi infliggono danni a seconda del loro peso e della distanza da cui sono caduti.

La \textbf{Tabella: Danno da Oggetti Cadenti} determina la quantità di danni inflitti da un oggetto in base alla sua taglia. Si presume che l'oggetto sia fattodi un materiale denso e pesante, come la pietra.
Gli oggetti fatti di materiali più leggeri potrebbero infliggere la metà o meno del danno indicato, a discrezione del Narratore. Per esempio un masso Enorme che colpisce un personaggio infligge 6d6 danni, mentre un carro di legno potrebbe infliggerne solo 3d6.

Inoltre, se l'oggetto cade da una distanza inferiore ai 3 metri, infligge la metà dei danni indicati. Se un oggetto cade da una distanza superiore ai 20 metri, infligge danni raddoppiati. L'oggetto che cade subisce la stessa quantità di danni che infligge.

\bigskip

\textbf{Tabella: Danno da Oggetti Cadenti}

\begin{tabular}{ll}
	\toprule
	\textbf{Taglia dell'Oggetto} & \textbf{Danno}\\
	Minuscola o Più Piccola      & 1d6\\
	Piccola                      & 2d6\\
	Media                        & 3d6\\
	Grande                       & 4d6\\
	Enorme                       & 6d6\\
	Mastodontica                 & 8d6\\
	Colossale                    & 10d6\\
\end{tabular}

\bigskip

Lasciar cadere addosso ad una creatura un oggetto richiede un attacco di contatto a distanza (Difesa a tocco contro attacco basato su Agilità). Questi attacchi hanno di solito una gittata di 3 metri. Se un oggetto cade su una creatura (invece di venire lanciato), quella creatura deve effettuare un Tiro Salvezza su Riflessi con DC 15 per dimezzare il danno se è consapevole dell'oggetto che sta cadendo. Gli oggetti cadenti che sono parte di una trappola usano le regole relative alle trappole invece che quelle qui descritte.

\subsubsection{Pericoli dell'Acqua}\index{Pericoli dell'Acqua}\index{Acqua}

\label{pericoli-dellacqua}

Qualsiasi personaggio può attraversare acque relativamente calme che non abbiano una profondità maggiore alla sua altezza, senza bisogno di prove. Allo stesso modo, Nuotare (Resistenza) in acque calme richiede una prova di Resistenza con DC 10. I nuotatori addestrati possono prendere 10. Si ricordi che l'armatura o l'equipaggiamento pesante rendono qualsiasi tentativo di nuotare più difficile

D'altra parte, le acque rapide sono molto più pericolose.

Con una prova riuscita di Resistenza o una prova di Potenza con DC 15, i personaggi non rischiano di finire sott'acqua. Se falliscono, subiscono 1d3 Danni Non Letali per round (1d6 danni letali se le acque scorrono sopra rocce e avvallamenti).

L'acqua molto profonda non è solo nera come la pece, rendendo molto pericolosa la navigazione, ma infligge danni ancora peggiori a causa della pressione dell'acqua nell'ordine di 1d6 danni al minuto ogni 30 metri che separano il personaggio dalla superficie. Un Tiro Salvezza su Tempra superato con successo (DC 15, +1 per ogni prova precedente) indica che il personaggio immerso non subisce danni in quel minuto. L'acqua molto fredda infligge 1d6 Danni Non Letali per minuto di esposizione a causa dell'ipotermia.

\textbf{Annegamento}\index{Annegamento}

Qualsiasi personaggio può trattenere il fiato per un numero di round pari 8 volte il suo punteggio di Potenza, con un minimo di 6 round. Se il personaggio compie almeno 2 Azioni, la durata restante per cui può trattenere il fiato è ridotta di 1 round. Trascorso questo periodo di tempo, il personaggio deve effettuare una prova di Potenza con DC 10 ad ogni round per continuare a trattenere il fiato. Ogni round, la DC aumenta di 1.

Si può annegare in sostanze diverse dall'acqua, come la sabbia, le sabbie mobili, la polvere molto fine o un silos pieno di farro o semplicemente trattenendo il respiro.

\subsubsection{Pericoli del Caldo}\index{Caldo}

\label{pericoli-del-caldo}

Una creatura sottoposta a temperature molto elevate (sopra i 40° C) deve superare un Tiro Salvezza su Tempra ogni ora (DC 15, +1 per ogni prova precedente) oppure subisce 1d4 Danni Non Letali. Se indossa abiti pesanti o qualsiasi tipo di armatura, subisce penalità -4 a questi Tiri Salvezza. Un personaggio con la competenza Sopravvivenza può ricevere un bonus a questo Tiri Salvezza ed essere in grado di applicarlo anche ad altri personaggi (vedi la descrizione dell'Abilita'). I personaggi Privi di Sensi iniziano a subire danni letali (1d4 danni all'ora).

In situazioni di caldo estremo (sopra i 40° C), un personaggio deve effettuare un Tiro Salvezza su Tempra ogni 10 minuti (DC 15, +1 per ogni prova precedente) oppure subisce 1d4 danni non letali. I personaggi che indossano abiti pesanti o qualsiasi tipo di armatura, subiscono penalità -4 a questi Tiri Salvezza. Un personaggio con la competenza Sopravvivenza può ricevere un bonus a questo Tiro Salvezza ed essere in grado di applicarlo anche ad altri personaggi. I personaggi Privi di Sensi iniziano a subire danni letali (1d4 danni ogni 10 minuti).

Un personaggio che subisce Danni Non Letali a causa dell'esposizione al caldo, è soggetto ad un colpo di calore ed è Affaticato. Queste penalità terminano quando il personaggio recupera dai Danni Non Letali subiti a causa del caldo.

Il caldo infernale (temperatura dell'aria sopra i 60° C, fuoco, acqua che bolle, lava) infligge danni letali. Respirare l'aria con queste temperature infligge 1d6 danni da fuoco al minuto (senza Tiro Salvezza).
Inoltre, il personaggio deve superare un Tiro Salvezza su Tempra ogni 5 minuti (DC 15, +1 per ogni prova precedente) oppure subisce 1d4 Danni Non Letali. Coloro che indossano abiti pesanti o qualsiasi tipo di armatura, subiscono penalità -4 a questi Tiro Salvezza.

L'acqua bollente infligge 1d6 danni da scottatura, a meno che il personaggio non vi venga completamente immerso, nel qual caso subirebbe 10d6 danni per round di esposizione.

\subsubsection{Prendere Fuoco}\index{Prendere Fuoco}\index{Fuoco}

\label{prendere-fuoco}

I personaggi esposti ad olio bollente, fuochi da campo, e fuochi magici non istantanei possono vedere i loro abiti, capelli o equipaggiamento prendere fuoco. Le Essenze magiche con durata istantanea non sono in grado di appiccare il fuoco, in quanto il calore e la fiammata appaiono e scompaiono in un attimo.

I personaggi che rischiano di prendere fuoco possono effettuare un Tiro Salvezza su Riflessi con DC 15 per evitare questo destino. Se i vestiti o i capelli di un personaggio prendono fuoco, egli subisce immediatamente 1d6 danni. Per ogni round successivo il personaggio in fiamme deve effettuare un altro Tiro Salvezza su Riflessi. Il fallimento indica che subisce altri 1d6 danni in quel round. Il successo indica che il fuoco si è estinto (ovvero, una volta che supera il Tiro Salvezza, non sta più andando a fuoco).

Un personaggio che va a fuoco può estinguere automaticamente le fiamme saltando dentro a dell'acqua sufficiente a spegnerle. Se non ci sono grosse quantità d'acqua a disposizione, rotolarsi sul terreno o smorzare la fiamma con mantelli o simili può concedere al personaggio un altro Tiro Salvezza con bonus +4.

Coloro che sono talmente sfortunati dal vedere il loro Equipaggiamento o vestiti prendere fuoco devono superare un Tiro Salvezza su Riflessi (DC 15) per ogni oggetto. Gli oggetti infiammabili che falliscono il tiro, subiscono la stessa quantità di danni del personaggio.

\medskip

\textbf{Effetti della Lava}\index{Lava}

La lava o il magma infliggono 2d6 danni per round di esposizione, tranne in caso di totale immersione (come quando un personaggio cade nel cratere di un vulcano attivo), che infligge 20d6 danni per round (più eventuali danni da caduta).

I danni provocati dal magma continuano per 1d3 round dopo il termine dell'esposizione, ma questi danni addizionali sono solo la metà di quelli inflitti durante l'effettivo contatto (cioe', 1d6 o 10d6 per round). Un'Immunità o una Resistenza al fuoco servono anche come resistenza o resistenza alla lava o al magma. Tuttavia, le creature Immuni o Resistenti al Fuoco potrebbero annegare se immerse nella lava (vedi Annegamento).


\subsubsection{Pericoli del Freddo}\index{Freddo}

\label{pericoli-del-freddo}

I personaggi non ben vestiti in climi freddi (sotto i 5° C) devono superare un Tiro Salvezza su Tempra ogni ora (DC 15, +1 per ogni prova precedente) oppure subiscono 1d6 Danni Non Letali. Un personaggio con la Competenza Sopravvivenza può ricevere un bonus a questo Tiro Salvezza ed essere in grado di applicarlo anche ad altri personaggi.

In condizioni di freddo estremo o di esposizione sotto i -17° C, un personaggio scoperto deve effettuare un Tiro Salvezza su Tempra ogni 10 minuti (DC 15, +1 per ogni prova precedente), subendo 1d6 Danni Letali per ogni Tiro Salvezza fallito. Un personaggio con la Competenza Sopravvivenza può ricevere un bonus a questo Tiro Salvezza ed essere in grado di applicarlo anche ad altri personaggi. I personaggi che indossano abiti invernali hanno bisogno di effettuare la prova per il freddo e l'esposizione solo una volta all'ora.

Un personaggio che subisce Danni Non Letali a causa del freddo o dell'esposizione, è soggetto ai geloni o all'ipotermia (considerarlo come Affaticato). Queste penalità terminano quando il personaggio recupera dai Danni Non Letali subiti a causa del freddo e dell'esposizione.

Le condizioni di freddo intollerabile o di esposizione (sotto i -28° C) infliggono ai personaggi 1d6 danni letali per minuto (senza alcun Tiro Salvezza) se non specificatamente protetti.

\subsubsection{Effetti del Ghiaccio}\index{Ghiaccio}

I personaggi che camminano sul ghiaccio è come se fossero su terreno difficile. Il movimento è dimezzato, eventuali prove di Acrobatica hanno un aumento di difficoltà +5. I personaggi che sono per lungo tempo a contatto con il ghiaccio potrebbero subire dei danni da freddo estremo.

\subsubsection{Soffocamento Lento}\index{Soffocamento}

Un personaggio di taglia Media può respirare tranquillamente per circa 6 ore in una camera sigillata che misura 3 metri di lato. Dopo questo tempo, subisce 1d6 Danni Non Letali ogni 15 minuti. Ogni ulteriore personaggio di taglia Media oppure ogni fuoco significativo (una torcia, per esempio) riducono proporzionalmente la durata dell'aria respirabile. Una volta privi di sensi per l'accumulo di Danni Non Letali, i personaggi iniziano a subire Danni Letali allo stesso ritmo. I personaggi di taglia Piccola consumano metà dell'aria dei personaggi di taglia Media.

\pagebreak

\subsection{Tempo Atmosferico - Meteo}\index{Meteo}

\label{tempo-atmosferico---meteo}

A volte il tempo atmosferico può giocare un ruolo importante nel corso di un'avventura. La Tabella: Tempo Atmosferico Casuale è una tabella generica che può essere utilizzata per stabilire le condizioni atmosferiche locali. I termini della tabella sono definiti qui di seguito:

\bigskip

\textbf{Tabella: Tempo Atmosferico Casuale}

\begin{tabularx}{0.95\textwidth}{lXXXX}
	\toprule
	\textbf{d\%} & \textbf{Tempo Atmosferico} & \textbf{Clima Freddo}& \textbf{Clima Temperato {*}}   & \textbf{Deserto}\\
	01-70   & Normale& Freddo, calmo   & Normale per la stagione {*}{*} & Torrido,calmo\\
	71-80   & Anormale    & Ondata di Caldo (01-30)   \\
	   &   & Ondata di Freddo (31-100) & Ondata di Caldo (01-50)\\
	   &   &  & Ondata di Freddo (51-100) & Torrido,ventilato\\
	81-90   & Inclemente  & Precipitazioni (neve)& Precipitazioni    \\
	   &   &  & (normali per la stagione) & Torrido, ventilato\\
	91-99   & Tempesta    & Tempesta di neve& Tempesta di fulmini    \\
	   &   &  & tempesta di neve& Tempesta di polvere\\
	100& Tempesta violenta& Tormenta   & Bufera,tormenta   \\
	   &   &  & uragano,tornado & Acquazzone\\
\end{tabularx}

* Temperato comprende foreste, colline, paludi, montagne, pianure
e zone marine calde.

** L'inverno è freddo, l'estate è calda, l'autunno e la primavera sono moderati. Le paludi sono sempre leggermente più calde d'inverno.

\bigskip

\textbf{Acquazzone}: Considerarlo come pioggia (vedi Precipitazioni sotto), ma offre copertura come la nebbia. Può provocare inondazioni e dura di solito 2d4 ore.

\textbf{Caldo}: La temperatura è tra 15° e 30° C di giorno, e tra 6 e 11 gradi in meno di notte.

\textbf{Calmo}: Vento leggero (tra 0 e 15 km/h).

\textbf{Freddo}: Temperatura tra -17° e 5° C durante il giorno, e tra 6 a 11 gradi in meno di notte.

\textbf{Moderato}: Temperatura tra i 5° e i 15° C durante il giorno, e tra 6 e 11 gradi in meno di notte.

\textbf{Ondata Caldo}: Fa aumentare la temperatura di 6° C.

\textbf{Ondata Freddo}: Abbassa la temperatura di 6° C.

\textbf{Precipitazioni}: Tirare un d100 per determinare se la precipitazione è nebbia (01-30), pioggia/neve (31-90), o nevischio/ grandine (91-00). La neve e il nevischio si verificano solo quando la temperatura è di 0° C o inferiore. La maggior parte delle precipitazioni dura 2d4 ore. La grandine, invece, dura solo 1d20 minuti ma di solito è accompagnata da 1d4 ore di pioggia.

\textbf{Tempesta} (di Fulmini/di Neve/di Polvere): Il vento è molto forte (da 45 a 75 km/h) e la visibilità è ridotta di tre quarti. Le tempeste durano 2d4-1 ore. Vedi Tempeste, sotto, per ulteriori dettagli.

\textbf{Tempesta} (Bufera/Tormenta/Uragano/Tornado): La velocità del vento è superiore ai 75 km/h (vedi Tabella: Effetti del Vento). Inoltre, le tormente sono accompagnate da pesanti nevicate (1d3 \texttimes{} 30 cm), e gli uragani sono accompagnati da acquazzoni. Le bufere durano 1d6 ore, le tormente 1d3 giorni. Gli uragani possono durare fino a una settimana, ma l'impatto maggiore per i personaggi avverrà in un periodo di tempo tra le 24 e le 48 ore, mentre il centro della tempesta si sposta nella loro zona. I tornado durano molto poco (1d6 \texttimes{} 10 minuti), e di solito si formano come parte di una tempesta di fulmini.

\textbf{Torrido}: Temperatura tra i 30° e i 43° C durante il giorno e tra 6 e 11 gradi in meno di notte.

\textbf{Ventilato}: La velocità del vento va da moderata a forte (da 15 a 45 km/h); vedi Tabella: Effetti del Vento.

\textbf{Pioggia, Neve, Nevischio e Grandine}

La brutta stagione frequentemente rallenta o blocca i trasporti via terra e rende praticamente impossibile la navigazione. acquazzoni torrenziali e bufere oscurano la visuale tanto quanto lo farebbe una nebbia densa.

La maggior parte delle precipitazioni si manifesta come pioggia, ma nei climi freddi possono manifestarsi anche come neve, nevischio o grandine. Le precipitazioni di qualsiasi tipo, seguite da un calo della temperatura da sopra a sotto gli 0° C possono produrre ghiaccio.

\textbf{Pioggia}\index{Pioggia}

La pioggia dimezza la visibilità, e impone penalità -4 alle prove di Consapevolezza. Ha lo stesso effetto di un vento molto forte sulle fiamme, sugli attacchi con armi a distanza e sulle prove di Consapevolezza come vento molto forte.

\textbf{Neve}\index{Neve}

Mentre cade, la neve ha gli stessi effetti della pioggia su visibilità, attacchi con armi a distanza e prove di Consapevolezza ed il movimento costa 1 azione in più. Una nevicata della durata di un giorno lascia al suolo 1d6 \texttimes{} 2.5 centimetri di neve.

\textbf{Neve Fitta}

Una fitta nevicata ha gli stessi effetti di una nevicata normale, ma oscura la visibilità come la nebbia (vedi Nebbia). Un giorno di neve fitta lascia sul terreno 1d4 x 30 centimetri di neve ed entrare in zona di mischia così alta costa 2 azioni di movimento. Una fitta nevicata accompagnata da venti forti o molto forti può dare origine a cumuli di neve profondi 1d4 x 1,5 metri, specialmente sopra e intorno ad oggetti abbastanza grandi da deflettere il vento (una capanna o una grande tenda, per esempio).
c'è una probabilità del 10\% che una nevicata fitta sia accompagnata da fulmini (vedi Tempesta di Fulmini). La neve ha gli stessi effetti del vento moderato sulle fiamme. La neve rende il terreno difficile.

\textbf{Nevischio}

Si tratta fondamentalmente di pioggia congelata, che ha gli stessi effetti della pioggia quando cade (eccetto che la probabilità di estinguere fiamme protette è del 75\%) e quelli della neve una volta depositatasi.

\textbf{Grandine}

La grandine non riduce la visibilità, ma il suono della grandine che cade rende più difficili le prove di Consapevolezza basate sull'udito (penalità -4). A volte (probabilità del 5\%) la grandine può essere talmente grossa da infliggere 1 danno letale (per tempesta) a qualsiasi cosa si trovi all'aperto. Una volta depositata, la grandine ha lo stesso effetto della neve sul movimento.

\subsubsection{Tempeste}\index{Tempeste}

\label{tempeste}

Gli effetti combinati delle precipitazioni (o della polvere) e del vento, che accompagnano tutte le tempeste, riducono la visibilità di tre quarti, imponendo penalità -8 a tutte le prove di Consapevolezza. Le tempeste rendono impossibili gli attacchi con le armi a distanza, tranne che con le armi da assedio, che subiscono penalità -4 i Tiri per Colpire.
Estinguono automaticamente le candele, le torce o simili fiamme non protette. Le fiamme protette, come quelle delle lanterne, vengono agitate violentemente e hanno una probabilità del 50\% di estinguersi. Vedi Tabella: Effetti del Vento per le possibili conseguenze sulle creature sorprese all'esterno senza ripari.

Le tempeste sono di tre tipi.

\textbf{Tempesta di Polvere (CR 3)}

queste tempeste desertiche si differenziano dalle altre tempeste in quanto non hanno precipitazioni. Al contrario, le tempeste di polvere trasportano granelli di sabbia che oscurano la vista, soffocano le fiamme non protette e possono addirittura spegnere quelle protette (probabilità del 50\%). Molte tempeste di polvere sono accompagnate da venti molto forti e si lasciano alle spalle un deposito di 1d6 \texttimes{} 2.5 centimetri di sabbia.
Esiste anche una probabilità del 10\% di incontrare grandi tempeste di polvere con bufere di vento (vedi Tabella: Effetti del Vento). Queste violente tempeste di polvere infliggono 1d3 danni non letali per round a chiunque venga sorpreso all'aperto senza riparo e pongono anche il rischio del soffocamento (vedi Annegamento, eccetto che un personaggio con una sciarpa o simile protezione sulla bocca e il naso, non inizia a soffocare se non dopo un numero di round pari 10 \texttimes{} il suo punteggio di Potenza). Le grandi tempeste di polvere si depositano alle spalle (2d3-1) x 30 centimetri di sabbia.

\textbf{Tempesta di Neve}

oltre ai venti e alle precipitazioni comuni alle altre tempeste, le tempeste di neve depositano 1d6 \texttimes{} 2.5 centimetri di neve sul terreno.

\textbf{Tempesta di Fulmini}

oltre ai venti e alle precipitazioni (di solito pioggia, ma a volte anche grandine), le tempeste di fulmini sono accompagnate da scariche elettriche che rappresentano un pericolo per i personaggi che si trovano all'aperto senza riparo (specialmente se indossano armature metalliche). Come regola generale, si può considerare un fulmine al minuto per un periodo di un'ora nel cuore della tempesta. Ogni fulmine infligge danni da elettricità tra 4d8 e 10d8. Una tempesta di fulmini su dieci viene accompagnata da un tornado.

\textbf{Tempeste Violente}

Venti molto forti e precipitazioni torrenziali riducono la visibilità a zero, e rendono impossibile effettuare prove di Consapevolezza e compiere attacchi con armi a distanza. Le fiamme non protette vengono automaticamente spente, e c'è una probabilità del 75\% che ciò si verifichi anche per quelle protette. Le creature sorprese in queste zone devono effettuare un Tiro Salvezza su Tempra o devono affrontare effetti a seconda della propria taglia (vedi Tabella: Effetti del Vento). Le tempeste violente sono suddivise nei seguenti quattro tipi.

\textbf{Bufera}: Sebbene abbiano poche o nessuna precipitazione, le bufere possono provocare danni ingenti a causa della Potenza del vento.

\textbf{Tormenta}: La combinazione di forti venti, neve fitta (di solito 1d3 \texttimes{} 30 cm) e freddo intenso rende le tormente letali per chiunque non vi sia preparato.

\textbf{Uragano}: Oltre ai venti molto forti e alla pioggia intensa, gli uragani sono seguiti da inondazioni. Molte attività in un'avventura sono impossibili in queste condizioni.

\textbf{Tornado}: Oltre ai venti molto forti, i tornado possono ferire gravemente ed uccidere quelli che vengono catturati al suo interno.

\subsubsection{Nebbia}\index{Nebbia}

\label{nebbia}

Sia nella forma di una nube a bassa altitudine che di una foschia che sale dal terreno, la nebbia ostacola la visuale oltre la distanza di mischia. Le creature più lontane di mischia godono di Copertura leggera (+2 Difesa).

La nebbia rende il terreno difficile.


\subsubsection{Venti}\index{Venti}

\label{venti}

I venti possono creare turbini di sabbia o polvere, alimentare grossi incendi, rovesciare piccole imbarcazioni e disperdere gas o vapori. Se sono forti a sufficienza possono addirittura buttare a terra i personaggi (vedi Tabella: Effetti del Vento), interferire con gli attacchi a distanza, o imporre penalità ad alcune Prove di Competenze.

\textbf{Tabella: Effetti del Vento Potenza del Vento}

\medskip

\begin{tabular}{lll}
	\toprule
	\textbf{Potenza del Vento} & \textbf{Velocità del Vento}   & \textbf{Attacchi a Distanza} \\
	Leggero     & 0-15km \\
	Moderato    & 16,5-30 km/h  \\
	Forte       & 31.5-45        & -2 \\
	Molto forte & 45.5-75km/h    & -4 \\
	Bufera      & 76.5-111km/h   & impossibile  \\
	Uragano     & 12-261km/h     & impossibile \\
	Tornado     & 262-450km/h    & impossibile\\
\end{tabular}

\bigskip

\textbf{Vento Leggero}

Una brezza gentile, che non ha effetti pratici sul gioco.

\textbf{Vento Moderato}

Un vento sostenuto, che ha una probabilità del 50\% di estinguere qualsiasi piccola fiamma non protetta, come quella di una candela.

\textbf{Vento forte:} Folate che spengono automaticamente le fiamme non protette (candele, torce e simili). Queste folate impongono penalità -2 ai tiri per colpire a distanza ed alle prove di Consapevolezza.

\textbf{Vento Molto Forte}

Oltre a spegnere automaticamente le fiamme non protette, i venti di questa intensità agitano violentemente le fiamme protette (come quelle di una lanterna) e hanno una probabilità del 50\% di estinguerle. Gli attacchi con le armi a distanza e le prove di Consapevolezza subiscono penalità -4. Questa è la stessa velocità del vento prodotta da un Essenza di Creazione Aria a livello 5.

\textbf{Bufera}\index{Bufera}

Abbastanza forti da abbattere i rami o addirittura interi alberi, le bufere estinguono automaticamente le fiamme non protette e hanno una probabilità del 75\% di estinguere quelle protette, come quelle delle lanterne. Gli attacchi con le armi a distanza sono impossibili, e anche le armi da assedio subiscono penalità -4 ai Tiri per Colpire. Le prove di Consapevolezza basate sull'udito subiscono penalità -8 per l'ululare del vento.

\textbf{Uragano}\index{Uragano}

Estingue tutte le fiamme. Gli attacchi a distanza sono impossibili (eccetto con le armi da assedio che subiscono penalità -8 ai tiri per colpire). Anche le prove di Consapevolezza basate sull'udito sono impossibili e tutto ciò che i personaggi possono udire è l'ululare del vento. Gli uragani spesso sono in grado di abbattere gli alberi.

\textbf{Tornado (CR 10)}\index{Tornado}

Estingue tutte le fiamme. Tutti gli attacchi a distanza sono impossibili (compresi quelli con le armi da assedio), così come le prove di Consapevolezza basate sull'udito. Invece di essere portati via (vedi Tabella: Effetti del Vento), i personaggi che si trovano nelle immediate vicinanze di un tornado e che falliscono un Tiro Salvezza su Tempra vengono risucchiati dentro il tornado.

Coloro che entrano in contatto con la nube conica vengono sollevati da terra e sbatacchiati per 1d10 round, subendo 6d6 danni per round, prima di venirne espulsi violentemente (con l'applicazione dei danni da caduta).

Sebbene la velocità rotatoria di un tornado possa raggiungere i 450 km/h, il cono stesso si muove in avanti ad una media di 45 km/h (circa 75 metri per ogni round). Un tornado è in grado di sradicare alberi, distruggere edifici e provocare altre forme di simile devastazione.

\pagebreak

\section{Avventure in Acqua}\index{Avventure in Acqua}

\label{avventure-in-acqua}
\begin{tcolorbox}[enhanced,arc=5pt,boxrule=0.3pt]{Guardò il mare e capì fino a che punto era solo, adesso. (Il vecchio e il mare, Ernest Hemingway)}\end{tcolorbox}\medskip

L'acqua permette alle società di esistere, ma può anche distruggerle. La vita non potrebbe esistere senza di essa. Il commercio ed il viaggio sono agevolati dalla sua presenza. Eppure, l'acqua può anche uccidere, sia annegando le persone, sia generando alluvioni e tsunami su larga scala. La vita terrestre è dipendente dall'acqua ma allo stesso tempo la teme.

\textbf{Avventure Acquatiche}

Un'avventura acquatica può aver luogo ovunque l'acqua rappresenti l'elemento principale del territorio: come paludi, fiumi, laghi, stagni, oceani, il Piano dell'Acqua e simili. Le avventure Acquatiche, comunque, non richiedono che i personaggi abbiano la capacità di respirare sott'acqua; l'introduzione di sfide Acquatiche per avventurieri di basso livello apportano ad un'avventura un bel pò di tensione e sensazione di pericolo.

\textbf{Adattarsi agli Ambienti Acquatici}

Le regole per il combattimento sott'acqua si applicano alle creature che non sono native di questo pericoloso ambiente, come la maggior parte dei PG. Per avventure Acquatiche prolungate ed esplorazioni particolarmente in profondità, i personaggi necessiteranno dell'uso della magia per proseguire le proprie avventure. Essenze di Alterazione per respirare Sott'Acqua è di ovvia utilità, mentre Essenze di Difesa per Resistere all'Energia può aiutare con la temperatura.

Il danno da pressione può essere totalmente evitato tramite effetti di Essenze di Movimento. Le Essenze di Trasformazione sono forse le più utili in acqua, purché la forma assunta sia di natura acquatica.

\textbf{Adattamento Naturale:} Qualsiasi creatura del Sottotipo Acquatico può respirare sott'acqua facilmente e non viene influenzata dalle temperature estreme riscontrabili nel proprio ambiente tipico. Le creature Acquatiche e quelle con la capacità di trattenere il respiro sono molto più resistenti al danno da pressione: non subiscono questo tipo di danno a meno che non vengano spostate istantaneamente da una profondità ad un'altra (in tal caso si adattano al cambio di pressione dopo aver superato con successo cinque Tiro Salvezza su Tempra contro gli effetti della pressione).

\textbf{Avventure Nautiche}

L'acqua può fornire l'ambientazione per un'esperienza di gioco diversa ed unica: l'avventura nautica. In un simile scenario, gli effetti e i pericoli delle avventure subacquee sono sostituiti dalle sfide di superficie, dal momento che i personaggi e i loro avversari utilizzano navi e barche per spostarsi in tale ambiente. Di solito, le avventure nautiche si risolvono normalmente, con un combattimento a bordo di una nave simile ad uno terrestre. Se il combattimento avviene durante una tempesta o in mari agitati, considerate il ponte della nave come terreno difficile. Ricordatevi di considerare gli effetti sulle prove di Concentrazione per il tempo atmosferico o il rollio.

\textbf{Combattimento Rapido in Mare}

Quando sono le navi a combattere, le cose cambiano un po'. Le regole seguenti non hanno lo scopo di simulare accuratamente tutti gli aspetti di un combattimento navale, ma solo fornirvi rapide e semplici regole per sbrogliare tali situazioni quando si tramutano in un'avventura nautica, che sia una battaglia tra due navi o tra una nave ed un mostro marino.

	{Preparazione:} Stabilite quali tipi di navi sono coinvolte nel combattimento (vedi Tabella: Statistiche delle Navi). Utilizzate una griglia da battaglia ampia e vuota per rappresentare le acque in cui ha luogo la battaglia. Un singolo quadretto corrisponde a 9 metri di distanza. Raffigurate ogni nave piazzando dei segnalini che occupino l'appropriato numero di quadretti (le navi giocattolo sono ottimi segnalini e potete reperirle nei negozi di modellismo).{}

	{Cominciare il Combattimento:} Quando il combattimento inizia, lasciate che i personaggi (ed importanti PNG alleati) tirino l'Iniziativa normalmente; la nave si muove ed attacca sulla base del risultato di iniziativa del capitano. Se una delle navi in battaglia usa le vele per spostarsi, determinate casualmente in quale direzione sta soffiando il vento tirando 1d8 e seguendo le linee guida per le Armi a Spargimento che mancano il bersaglio.{}

	{Movimento:} Sulla base del punteggio di Iniziativa del capitano, la nave può muoversi alla propria velocità base in un singolo round come se l'Azione corrispondesse a quella del capitano stesso (o al doppio della sua velocità come unica azione del round), finché ha il proprio equipaggio minimo al completo. La nave può incrementare o diminuire la propria velocità di 9 metri per round, fino al raggiungimento della velocità massima. In alternativa, il capitano può cambiare direzione (al massimo un lato di un quadretto alla volta) (2 Azioni). Una nave può cambiare direzione solo all'inizio del turno.{}

	{Attacchi:} I membri in eccesso rispetto al requisito minimo di equipaggio di una nave possono essere collocati a manovrare le Macchine d'Assedio. Le Macchine d'Assedio attaccano sulla base del punteggio di Iniziativa del capitano.{}

Una nave può anche tentare di speronare un bersaglio se ospita l'equipaggio minimo. Per speronare un bersaglio, la nave deve muoversi di almeno 9 metri e finire con la prua in un quadretto adiacente ad esso.
Quindi, il capitano della nave effettua una prova di Professione (marinaio): se il risultato è pari o superiore alla Difesa del bersaglio, la nave colpisce il suo obiettivo, infliggendogli danni come indicato nella Tabella: Statistiche delle Navi e allo stesso tempo subendo il danno minimo. Una nave equipaggiata con uno sperone infligge al bersaglio 3d6 danni addizionali (l'imbarcazione attaccante non subisce danno addizionale).

\textbf{Affondamento}\index{Affondamento}

Una nave ottiene la condizione in affondamento quando i suoi Punti Ferita scendono a 0 o meno. Una nave in affondamento non può muoversi o attaccare e dopo 10 round si considera affondata. Ogni 25 danni subiti da una nave che affonda si riduce l'affondamento di 1 round. L’Essenza di Creazione consente di riparare una nave che affonda se i Punti Ferita della stessa sono riportati sopra lo 0, caso in cui la nave perde la condizione in affondamento. In genere, le riparazioni non magiche richiedono troppo tempo per salvare una nave dall'affondamento una volta che questa inizia a sprofondare.

\textbf{Statistiche di una Nave}

Nel mondo reale esiste una grande varietà di barche e navi, dalle piccole zattere agli imponenti galeoni. A rappresentanza di cio', la Tabella: Statistiche delle Navi classifica sette dimensioni standard di nave e le rispettive statistiche. Così come le culture del mondo reale hanno creato ed adattato differenti tipi di imbarcazioni, così le razze di mondi fantasy potrebbero creare le proprie bizzarre navi.
I Narratore potrebbero utilizzare o modificare queste statistiche per soddisfare le esigenze delle loro creazioni e, comunque, descrivere tali mezzo di trasporto a proprio piacimento. Tutte le navi presentano i seguenti tratti.

\textbf{Tipo}: Si tratta di una categoria generale che elenca la tipologia base di nave.

\textbf{Difesa}: La Difesa della nave. Per calcolare la Difesa effettiva di una nave, aggiungete il punteggio di Professione (marinaio) del capitano alla Difesa base della stessa. Gli attacchi di contatto contro una nave ignorano il modificatore del capitano. Una nave non è mai Impreparata.

\textbf{TS Base}: Il modificatore dei Tiri Salvezza Base di una nave (Tempra, Riflessi e Volonta') hanno lo stesso valore. Per determinare gli effettivi modificatori dei Tiro Salvezza di una nave, aggiungete il modificatore di Professione (marinaio) del capitano a questo valore.

Velocità Massima: La velocità massima di una nave in combattimento. Un asterisco indica che la nave ha delle vele e può spostarsi a velocità raddoppiata se si muove nella stessa direzione del vento. Una nave che abbia solo delle vele può spostarsi solo in presenza di vento.

\textbf{Armamenti}: Il numero di Macchine d'Assedio che possono essere equipaggiate sulla nave. Uno sperone utilizza uno di questi slot e una nave può essere equipaggiata soltanto con uno sperone.

\textbf{Speronamento}: L'ammontare di danni che infligge una nave con un attacco di speronamento riuscito (senza uno sperone).

\textbf{Quadretti}: Il numero di quadretti che la nave occupa sulla griglia di combattimento. Una nave si considera sempre della larghezza di un quadretto.

\textbf{Equipaggio}: Il primo numero indica l'equipaggio minimo di cui la nave ha bisogno per funzionare normalmente, ad esclusione degli addetti alle armi. Il secondo indica il numero massimo della ciurma più i soldati o passeggeri aggiuntivi. Una nave senza il suo equipaggio minimo può soltanto muoversi, cambiare velocità, cambiare direzione, o speronare se il suo capitano supera una prova di Professione (marinaio) con DC 20.
Un equipaggio che eccede il numero minimo non influenza il movimento, ma i suoi componenti possono sostituire i membri caduti o manovrare armi aggiuntive.

\bigskip

\textbf{Tabella: Statistiche delle Navi}

\medskip

\begin{tabularx}{0.95\textwidth}{lXlllllXll}
	\toprule
	\textbf{Tipo}  & \textbf{Difesa} & \textbf{PF} & \textbf{TS base} & \textbf{Vel. (m/s)} & \textbf{Arma} & \textbf{Speronamento} & \textbf{Quad}. & \textbf{Equipaggio}\\
	Zattera   & 9     & 10& +0& 4.5  & 0   & 1d6    & 1    & 1/4\\
	Barca a Remi   & 9& 20& +2& 9    & 0   & 2d6+6  & 1    & 1/3\\
	Battello  & 8& 60& +4& 9    & 1   & 2d6+6  & 2    & 4/15+100\\
	Nave Lunga& 6& 75& +5& 18   & 1   & 4d6+18 & 3    & 50/75+100\\
	Barca a Vela   & 6& 125    & +6& 18   & 2   & 3d6+12 & 3    & 20/50+120\\
	Nave da Guerra & 2& 175    & +7& 18   & 3   & 3d6+12 & 4    & 60/80+160\\
	Galea& 2& 200    & +8& 27   & +4  & 6d6+24 & 4    & 200/250+200\\
\end{tabularx}

\pagebreak

\section{Avventure in Citta'}\index{Citta'}

\label{avventure-in-citta}
\begin{tcolorbox}[enhanced,arc=5pt,boxrule=0.3pt]{Dio creò la campagna, e l'uomo creò la città. (William Cowper)}\end{tcolorbox}\medskip

A prima vista, una città è molto simile a un dungeon, in quanto è composta da pareti, porte, stanze e corridoi. Le avventure ambientate in città differiscono da quelle ambientate nei dungeon per due motivi principali. I personaggi hanno accesso a un maggior numero di risorse e devono tenere conto della presenza delle forze dell'ordine.

\textbf{Accesso alle Risorse}: A differenza dei dungeon e delle terre selvagge, i personaggi possono comprare e vendere Equipaggiamento molto rapidamente in città. Una città grande o una metropoli probabilmente dispongono di PNG ed esperti di alto livello specializzati nei settori più oscuri della conoscenza, in grado di offrire aiuto e di interpretare gli indizi. E quando i personaggi sono malconci e contusi, possono sempre fare ritorno alle comodità delle loro camere nella locanda.

La libertà di effettuare una ritirata e l'accesso alle merci del mercato significa che i giocatori dispongono di un maggior controllo sui ritmi di gioco di un'avventura in città.

\textbf{Forze dell'Ordine}: L'altro elemento di distinzione tra andare all'avventura in una città ed esplorare un dungeon sta nel fatto che il dungeon e', quasi per definizione, un luogo senza regole dove la sola legge è quella dellagiungla: uccidere o essere uccisi.

Una città, d'altro canto, è sorretta da un codice di leggi, molte delle quali sono state ideate esplicitamente per prevenire quel genere di comportamento nel quale gli avventurieri indulgono spesso e volentieri: uccidere e saccheggiare. Tuttavia, le leggi cittadine riconoscono la gravità della minaccia che i mostri costituiscono alla stabilità cittadina, ed è assai raro che la proibizione di uccidere valga anche per mostri come le aberrazioni o gli esterni malvagi.

La maggior parte degli umanoidi malvagi, tuttavia, solitamente gode della stessa protezione riservata a tutti gli altri cittadini. Appartenere a un allineamento malvagio non è un crimine (tranne forse in quelle città dove vige una severa teocrazia, col potere magico necessario per far valere la legge); soltanto gli atti malvagi vengono considerati un'infrazione alla legge.

Anche quando gli avventurieri incontrano un malfattore impegnato a commettere i crimini più orribili nei confronti della popolazione cittadina, la legge vede comunque di cattivo occhio chi si fa giustizia da solo uccidendo il malfattore o impedendo in altri modi che venga condotto davanti a un tribunale per essere processato.

\textbf{Limitazioni alle Armi e alle Essenze}

Ogni città ha le sue leggi riguardo alle armi che è possibile portare con sé circolando in pubblico e alle limitazioni alle Essenze.

Le leggi cittadine potrebbero non influenzare tutti i personaggi in egual modo. Un uomo di fede che si muove con un'arma al seguito non viene ostacolato in alcun modo dalla legge che impone di legare con un laccio le armi, ma un incantatore subisce una riduzione considerevole del suo potere se il suo Scrigno viene confiscato alle porte della città.

\textbf{Elementi Urbani}

Pareti, porte, illuminazione scarsa e terreno sconnesso: sotto molti aspetti, una città è simile a un dungeon. Di seguito vengono descritti nuovi elementi adatti a un'ambientazione cittadina.

\textbf{Mura e Cancelli}

Molte città sono difese da un cerchio di mura. Delle normali mura cittadine sono in pietra rinforzata, spesse 1,5 metri e alte 6 metri. Un muro simile è piuttosto liscio ed è necessaria una prova di Scalare (Resistenza) con DC 30 per potervisi arrampicare. Le mura dispongono di piccoli merli su un lato per fornire un parapetto alle guardie in cima, e lo spazio per camminare sulle mura è a malapena sufficiente per una guardia.

\textbf{Le mura}

A differenza delle città più piccole, le metropoli spesso sono dotate anche di mura interne, a volte delle vecchie mura erette quando la città era più piccola, oppure mura che separano i vari quartieri gli uni dagli altri. A volte queste mura sono alte e larghe come quelle esterne, ma molto più spesso hanno le dimensioni di quelle di una città grande o piccola.

\textbf{Torri di Guardia}: Alcune mura cittadine sono dotate di torri di guardia che spuntano a intervalli regolari. Sono poche le città che hanno guardie a sufficienza da collocare su ogni torre di guardia, a meno che la città non si aspetti un attacco dall'esterno. Le torri offrono una visuale elevata della campagna circostante oltre a un baluardo di difesa contro gli invasori nemici.

Le torri di guardia solitamente sono più alte di 3 metri rispetto al muro di cui fanno parte, e il loro diametro è pari a 5 volte lo spessore delle mura. Delle feritoie per gli arcieri si aprono ai piani alti della torre, e la cima è merlata allo stesso modo delle mura circostanti. Nelle torri più piccole (del diametro di circa 7,5 metri, lungo un muro spesso 1,5 metri) una semplice scala a pioli collegal'interno della torre al tetto. Nelle torri più grandi si trovano vere e proprie scale.

L'accesso alla torre è protetto da pesanti porte in legno, con rinforzi in ferro e serrature buone (Disattivare Congegni (Criminalita') DC 30). Normalmente è il capitano delle guardie a custodire la chiave d'accesso alla torre, e una seconda copia viene conservata nella fortezza interna o nella caserma cittadina.

\textbf{Cancelli}: Un tipico cancello d'accesso alla città è composto da una guardiola con due saracinesche e delle feritoie nello spazio tra di esse. Nei paesi e nelle città piccole, l'entrata principale è protetta da doppie porte di ferro incastrate nelle mura cittadine.

I cancelli rimangono solitamente aperti durante il giorno e chiusi a chiave o sbarrati di notte. Generalmente, soltanto un cancello lascia entrare i viaggiatori dopo il tramonto, ed è sorvegliato da guardie che apriranno le porte solo per qualcuno che abbia un aspetto onesto, presenti i documenti appropriati, o le corrompa con una cifra sufficiente (in base al tipo di città e di guardie).

\textbf{Guardie e Soldati}

Una città solitamente è dotata di personale militare di servizio a tempo pieno pari all'1\% della sua popolazione adulta, in aggiunta ai soldati di turno o di leva pari al 5\% della popolazione. I soldati a tempo pieno sono guardie cittadine responsabili del mantenimento dell'ordine in città, con un ruolo simile a quello della polizia moderna, e (in misura assai minore) della difesa della città dagli assalti esterni. I soldati in leva forzata vengono chiamati alle armi in caso di un attacco in città.

Un tipico schieramento di guardie cittadine si distribuisce in tre turni diservizio da otto ore ciascuno, col 30\% delle sue forze in servizio di giorno (dalle 8 alle 16), il 35\% in servizio di sera (dalla 16 alle 24) e il 35\% di servizio nel turno di notte (dalle 24 alle 8). In qualsiasi momento, l'80\% delleguardie in servizio è di pattuglia per le strade, mentre il 20\% rimanente è assegnato a varie postazioni per la città, pronti a reagire ad eventuali allarmi. Una postazione di guardia simile è presente almeno in ogni vicinato cittadino (un vicinato è composto da vari quartieri).

La maggioranza delle guardie cittadine è composta da combattenti, quasi tutti di 1° livello. Gli ufficiali sono combattenti di livello più alto, e forse anche qualche incantatore.

\textbf{Macchine d'Assedio}\index{Macchine d'Assedio}

Le macchine d'assedio sono grosse armi, strutture temporanee o meccanismi tradizionalmente usati per assediare un castello o una fortezza.

\textbf{Catapulta Pesante}: \index{Catapulta}Una catapulta pesante è una gigantesca macchina d'assedio in grado di scagliare macigni o altri oggetti pesanti con grande forza.Dal momento che l'arco di lancio della catapulta è molto alto, il marchingegno è in grado di colpire anche aree al di fuori della sua linea di visuale. Per fare fuoco con una catapulta pesante, il capo degli operatori del macchinario effettua una prova speciale con DC 15 usando solo il suo valore di Competenza di Difesa, il suo modificatore di Intelletto, la penalità per la gittata e il modificatore relativo alla sezione inferiore della Tabella: Macchine d'Assedio.

Se la prova ha successo, il macigno della catapulta colpisce la zona di mischia a cui la catapulta aveva mirato, infliggendo i danni indicati a qualsiasi oggetto o personaggio nella zona. I personaggi che superano con successo un Tiro Salvezza su Riflessi con DC 15 subiscono danni dimezzati. Una volta che il macigno ha colpito la zona, i tiri successivi colpiranno la stessa zona, a meno che la catapulta non venga ridirezione o il vento non cambi direzione o velocità.

Se il macigno di una catapulta manca il bersaglio, si tira 1d8 per determinare dove atterra. Il risultato indica la direzione in cui il colpo devia, dove 1 indica verso la catapulta stessa e i valori da 2 a 8 le direzioni successive in senso orario attorno alla zona bersaglio. La distanza coperta è pari a 1d4x10 metri.

Per caricare una catapulta è necessaria una serie di azioni che portano via tutto il round. Occorre una prova di Potenza con DC 15 per abbassare il braccio della catapulta; la maggior parte delle catapulte hanno ruote che permettono fino a due operatori di usare l'azione di Aiutare un Altro per assistere l'operatore principale della carrucola.

Una prova di Professione (ingegnere d'assedio) con DC 15 consente di agganciare il braccio in posizione, e poi un'altra prova di Professione (ingegnere d'assedio) con DC 15 servirà per caricare il proiettile sulla catapulta. Sono necessarie quattro round per ricaricare una catapulta pesante (vari operatori della catapulta possono compiere queste azioni nello stesso round, quindi quattro persone possono ricaricare una catapulta nel giro di 1 solo round).

Una catapulta pesante occupa uno spazio di 4,5 metri.

\textbf{Catapulta Leggera}: Questa è una versione più piccola e più leggera della catapulta pesante. Funziona essenzialmente come una catapulta pesante, con la differenza che è necessaria una prova di Potenza con DC 10 per agganciare il braccio al suo posto, e soltanto 2 round per ridirezionare la catapulta.

Una catapulta leggera occupa uno spazio di 3 metri.

\textbf{Balista}: \index{Balista}Una balista è in pratica una balestra pesante enorme fissa. La sua taglia rende difficile il suo utilizzo per la maggior parte delle creature.Quindi, una creatura media subisce penalità --4 ai Tiri per Colpire quando usa una balista, e una creatura piccola subisce penalità --6. Per una creatura di taglia inferiore alla grande sono necessari 2 round per ricaricare la balista dopo aver fatto fuoco.

Una balista occupa uno spazio di 1,5 metri.

\textbf{Ariete}:\index{Ariete} Questo tronco massiccio a volte è legato e sospeso a un traliccio mobile che consente a coloro che lo manovrano di farlo oscillare con Potenza sempre crescente contro un bersaglio. Come unica azione del round, il personaggiopiù vicino alla punta dell'ariete effettua un CA per Colpire contro la Difesa della costruzione, applicando penalità --4 per la mancanza di competenza (non è possibile avere competenza nell'uso di questo macchinario). Oltre ai danni indicati nella Tabella: Macchine d'Assedio, fino a nove altri personaggi possono spingere l'ariete e aggiungere i loro modificatori di Potenza al danno dell'ariete, se riservano un'azione di attacco per farlo. E' necessaria almeno una creatura Enorme o di taglia superiore, 2 creature Grandi, 4 creature Medie oppure 8 creature Piccole per manovrare un ariete (le creature Minuscole o di taglia inferiore non possono usare un ariete).

Un ariete solitamente è lungo 9 metri. In una battaglia, le creature che manovrano un ariete devono disporsi in due file adiacenti di eguale lunghezza con l'ariete sorretto tra le due file.

\textbf{Torre da Assedio}\index{Torre da Assedio}: Questo macchinario è un'enorme torre di legno montata su ruote o cilindri che può essere spinta contro un muro per consentire agliassedianti di scalare la torre e quindi arrivare in cima alle mura beneficiando di Copertura. Le pareti in legno della torre di solito sono spesse circa 30 cm.

Una torre da assedio tipica occupa uno spazio di 4,5 metri. Le creature al suo interno la spingono a una velocità di 3 metri (una torre da assedio non può correre). Le otto creature che spingono la torre al pian terreno godono di Copertura totale, quelle ai piani superiori godono di Copertura migliorata e possono tirare attraverso le feritoie per gli arcieri.

\bigskip

\textbf{Tabella: Modificatori di Attacco delle Catapulte}

\begin{tabularx}{0.95\textwidth}{XX}
	\toprule
	\textbf{Circostanza}      & \textbf{Modificatore}\\
	La linea di visuale non giunge fino alla zona bersaglio & -6\\
	Tiro consecutivo (gli operatori riescono a vedere dove sono caduti i colpi andati a vuoto più recenti )   & +2 cumulativo per per colpo  mancato precedente (max +10)\\
	Tiro consecutivo (gli operatori non riescono a vedere dove sono caduti i colpi andati a vuoto più recenti ma un osservatore fornisce indicazioni) & +1 cumulativo per per colpo mancato precedente (max +5))\\
\end{tabularx}

\bigskip

\textbf{Tabella: Macchine d'Assedio}

\medskip

\begin{tabularx}{0.95\textwidth}{XXXXX}
	\toprule
	\textbf{Macchina} & \textbf{Costo (mo)} & \textbf{Danno} & \textbf{Gittata (metri)} & \textbf{Operatori}\\
	Catapulta pesante & 800  & 6d6  & 60   & 4\\
	Catapulta leggera & 5500 & 4d6  & 45   & 2\\
	Ballista& 500  & 3d8  & 36   & 1\\
	Ariete  & 1000 & 3d6  & -    & 10\\
	Torri da Assedio  & 2000 & -    & -    & 20\\
\end{tabularx}

\bigskip

\textbf{Strade Cittadine}\index{Strade Cittadine}

Le tipiche strade di una città sono strette e tortuose. La maggior parte delle vie cittadine è larga dai 4,5 ai 6 metri, mentre i vicoli vanno da una larghezza di 3 metri a una di soltanto 1,5 metri. Se il pavimento lastricato è in buone condizioni, è possibile muoversi normalmente, mentre le strade in brutte condizioni e gravemente rovinate vengono considerate equivalenti a detriti sparsi, e aumentano la DC delle prove di Acrobatica di 2.

Alcune città non hanno grandi viali d'accesso, specialmente quelle che sono cresciute gradualmente partendo come piccoli insediamenti. Le città che sono state progettate a tavolino, o che forse sono state consumate da un grave incendio che ha consentito alle autorità di costruire nuove strade su quelle che un tempo erano aree abitate, potrebbero disporre di alcune strade più grandi che le attraversano. Queste strade principali sono ampie 7,5 metri, e consentono ai carri di passare l'uno di fianco all'altro, con marciapiedi di 1,5 metri su entrambi i lati.

\textbf{Folla}: Le strade cittadine sono gremite di gente che va e viene, impegnata nelle varie faccende giornaliere. Nella maggior parte dei casi non è necessario includere ogni popolano di 1° livello sulla mappa quando si giunge a un combattimento sul viale principale della città.

è sufficiente invece indicare quali zone sulla mappa sono occupati dalla folla. Se la folla vede qualcosa di pericoloso, si allontanerà alla velocità di 9 metri per round a conteggio di Iniziativa 0. Per entrare in contatto con la folla bisogna avere una distanza di mischia. La folla fornisce Copertura a chiunque riesca a entrarvi, consentendo unaprova di Furtività (Consapevolezza) e fornendo un bonus alla Difesa e ai Tiri Salvezza su Riflessi.

\textbf{Dirigere la Folla}: è necessaria una prova di Diplomazia (Faccia Tosta) con DC 15 o di Intimidire (Faccia Tosta) con DC 20 per convincere una folla aspostarsi in una certa direzione, e la folla deve essere in grado di sentire o vedere il personaggio che effettua il tentativo. E' necessaria tutto un round per effettuare la prova di Diplomazia, mentre serve solo un'Azione per effettuare la prova di Intimidire.

Se due o più personaggi tentano di spingere la folla in due direzioni diverse, effettuano prove di Diplomazia (Faccia Tosta) o di Intimidire contrapposte per determinare a chi la folla darà ascolto. La folla ignorerà entrambi, se tutti e due i risultati delle prove dovessero essere inferiori alle DC sopra indicate.

\textbf{Sopra e Sotto le Strade}

\textbf{Tetti}: Per arrampicarsi su un tetto di solito è necessario scalare un muro (vedi la sezione Pareti), a meno che un personaggio non possa raggiungere un tetto saltando giù da una finestra, un balcone o un ponte più alto. I tetti piatti sono comuni solo nelle zone a clima caldo (la neve, accumulandosi, può far crollare un tetto piatto) e sono facili da percorrere correndo. Spostarsi sulla cima di un tetto richiede una prova di Acrobatica con DC 20. Spostarsi orizzontalmente su un tetto inclinato (muovendosi in parallelo alla sua cima, in pratica) richiede una prova di Acrobatica con DC 15. Spostarsi su e giù lungo un tetto inclinato richiede una prova di Acrobatica con DC 10.

Prima o poi un personaggio giungerà alla fine del tetto, e dovrà effettuare un lungo salto per passare al tetto successivo o per scendere a terra. La distanza che separa un tetto dal successivo di solito è di 3 metri, ma il tetto dall'altra parte potrebbe essere più alto o più basso di 1,5 metri, o alla stessa altezza. Si usano le indicazioni date per Acrobatica (il picco d'altezza in un salto in lungo è pari ad un quarto della distanza orizzontale) per determinare se il personaggio è in grado di effettuare un salto.

\textbf{Fognature}: Per entrare nelle fognature, i personaggi solitamente devono aprire una grata (1 round) e saltare in basso per 3 metri. Le fognature sono costruite esattamente come dei dungeon, con la differenza che il pavimento è scivoloso o ricoperto d'acqua. Le fognature sono anche simili ai dungeon per quello che riguarda le creature che è possibile incontrare al loro interno. Alcune città sono state costruite sulle rovine di civiltà più antiche, quindi le fognature potrebbero anche condurre a tesori e pericoli appartenenti a un'era passata.

\textbf{Edifici Cittadini}

La maggior parte degli edifici cittadini è divisa in tre categorie. Molti edifici in una città sono alti da due a cinque piani e sono costruiti l'uno di fianco all'altro per formare lunghe file, interrotte soltanto dalle vie principali o secondarie. Questi edifici a schiera solitamente ospitano un negozio a pianterreno, con uffici o appartamenti ai piani superiori.

Le locande, le imprese commerciali più ricche e i magazzini più grandi (oltre a eventuali mulini, concerie e altre attività che richiedano molto spazio) in generesono grossi edifici indipendenti alti fino a cinque piani.

Infine, le abitazioni, i negozi, i magazzini e i depositi più piccoli sono dei semplici edifici di legno a un piano, specialmente nei quartieri più poveri.

\textbf{Illuminazione Cittadina}

Se una città possiede grandi viali d'accesso, questi saranno illuminati da lanterne appese a un'altezza di circa 2 metri sui lati degli edifici. Queste lanterne sono poste a una distanza di 9 metri l'una dall'altra, quindi l'illuminazione in queste strade è praticamente continua. Le strade secondarie e i vicoli non sono illuminati; è consuetudine per i cittadini pagare un lanternaio che li accompagni, se devono uscire di notte.

I vicoli possono essere luoghi bui anche di giorno, grazie alle ombre degli edifici più alti circostanti. Un vicolo buio di giorno non è buio a sufficienza da poter conferire copertura completa ma leggera.

\pagebreak

\section{Avventure e Disastri}\index{Avventure}\index{Disastri}

\label{avventure-e-disastri}
\begin{tcolorbox}[enhanced,arc=5pt,boxrule=0.3pt]{Per prima cosa, nessuno rimane indietro. (anonimo)}\end{tcolorbox}\medskip
I disastri naturali sono pericoli ambientali terrificanti che portano morte e devastazione. Quelli soprannaturali posso­no essere anche più distruttivi, poiché possono sfigurare per sempre un mondo. Un disastro è più simile ad un'avventura che ad un incontro, e non ha uno specifico Grado di Sfida. Piuttosto, ogni parte del disastro dovrebbe essere trattata come un incontro separato ideato con un CR adeguato ai PG.

Sotto vengono presentate le regole per gestire gli effetti di tre diversi tipi di disastri, sia naturali che soprannaturali. Alcuni disastri si verificano rapidamente, come terremoti e tsunami, mentre altri procedono attraverso numerose fasi, come gli incendi forestali, i vulcani e le sollevazioni di non morti. Aggiustate lo schema dell'avventura per adattarlo al disastro, per permettere agli eventi di svolgersi nel corso di pochi minuti o molti giorni a seconda di quello che vi serve.

\textbf{Vulcani}\index{Vulcani}

Quando la crosta terrestre si rompe ed espelle il suo cuore fuso ha luogo uno dei disastri naturali più drammatici: l'eruzione di un vulcano. Le eruzioni vulcaniche offrono una vasta gamma di opzioni al Narratore, inclusi lava, bombe laviche, gas venefici e colate piroclastiche. I Narratore potrebbero anche considerare l'idea di far presagire una drammatica eruzione vulcanica (o draghi vulcanici) con pericoli preesistenti, come valanghe e terremoti minori.

\textbf{Lava}\index{Lava}

I flussi lavici generalmente sono associati alle eruzioni non esplosive e possono essere un elemento permanente dei vulcani attivi. Le colate laviche sono per lo più lente e si muovono a 4,5 metri per round (penalità azione movimento 1), ma quelle più calde sono rapide, e raggiungono i 12 metri per round (nessuna penalità azione movimento). La lava incanalata, come in un tubo lavico, è molto pericolosa, poiché si muove alla velocità di 36 metri per round (4 azioni di movimento a round) (un pericolo con CR 6). Le creature raggiunte da una colata lavica devono superare un Tiro Salvezza su Riflessi con DC 20 o sono sommerse dalla lava. Il successo indica che sono a contatto con la Lava ma non Immerse.

\textbf{Bombe Laviche} (CR 2 o 8)\index{Bombe Laviche}

Agglomerati di pietra fusa possono essere scagliati a molti chilometri da un vulcano che erutta, raffreddandosi in solida pietra prima di raggiungere il terreno. Una tipica bomba lavica colpisce un punto designato dal Narratore ed esplode in un raggio medio. Tutte le creature nell'area devono superare un Tiro Salvezza su Riflessi con DC 15 o subiscono 4d6 danni. Le creature che hanno Copertura o sono in grado di coprirsi (come con uno­scudo) ottengono bonus +2 a questo tiro. A volte si formano bombe laviche molto grandi che infliggono 12d6 danni. Le bombe laviche normali hanno CR 2, quelle grandi CR 5.

\textbf{Gas Venefici} (CR 5)\index{Gas Venefici}

Una delle minacce più insidiose di un vulcano è il gas tossico, spesso non notato tra il fuoco e la distruzione. Diversi tipi di vapori venefici scaturiscono da un'eruzione vulcanica, alcuni visibili, altri no. I gas venefici infliggono 1d6 danni alla Potenza per round se inalati (Tempra DC 15 nega, la DC aumenta di 1 per ogni Tiro Salvezza precedente), e quelli visibili funzionano anche come Fumo Denso. Le nubi di gas venefici fluiscono verso il basso, e generalmente arrivano ad una altezza di 6 metri. Forti venti possono deviare le nubi di gas, così come alte barriere, a condizione che il gas abbia un altro posto dove andare.

\textbf{Colate Piroplastiche} (CR 10)

Alcune eruzioni vulcaniche creano una devastante ondata di cenere ardente, gas bollenti e detriti vulcanici chiamata colata piroclastica che può viaggiare per chilometri. Una colata piroclastica viene trattata come una Valanga che viaggia a 150 metri per round, combinata con gli effetti dei gas venefici indicati sopra. Il contatto con i detriti roventi della colata infligge 2d6 danni da fuoco per round, mentre qualsiasi creatura seppellita dalla colata subisce 10d6 danni per round.

\textbf{Tsunami}\index{Tsunami}

Gli tsunami, talvolta attribuiti ad onde di marea, sono tremende ondate d'acqua causate da terremoti sottomarini, esplosioni vulcaniche, smottamenti o impatti di asteroidi. Gli tsunami non si possono individuare finché non raggiungono l'acqua poco profonda, quando la massa d'acqua forma una grande onda. A seconda dalle dimensioni dello tsunami e del­la pendenza della costa, l'onda può coprire qualsiasi distanza, dal centinaio di metri fino ad oltre un chilometro sulla terra ferma, lasciandosi dietro una scia di distruzione. L'acqua poi si ritira, trascinando via ogni sorta di detriti e creature fino in alto mare.

L'esatta devastazione causata è soggetta alla discrezione del Narratore, ma un tipico tsunami abbatte o sradica tutte le strutture temporanee o mal costruite sul suo percorso, distrugge circa il 25\% degli edifici ben costruiti (causando danni significativi a quelli che restano) e lascia le fortificazioni solide leggermente danneggiate. Almeno 1/4 della popolazione che vive nell'area (inclusi animali e mostri) muore nel disastro, trascinato in mare, affogato sulla spiaggia o seppellito sotto le macerie.

Una creatura può evitare di essere portata via dal mare con una prova di Nuotare (Resistenza) con DC 25; altrimenti viene trascinata a 6d6 x 3 metri dalla riva. Le acque dopo uno tsunami sono sempre considerate agitate o tempestose, salvo influenze magiche. Una creatura coinvolta nel cedimento di un edificio subisce 6d6 danni (TS su Riflessi DC 15 dimezza), o la metà se la struttura è particolarmente piccola. c'è una probabilità del 50\% che la creatura venga sepolta (come per un Crollo), o che lo tsunami possa distruggere l'edificio, liberando la creatura dalle macerie.

\textbf{Sollevazione di Non Morti}\index{Non Morti}

Frutto di un'antica maledizione o di atti necromantici, uno dei disastri soprannaturali più terrificanti è la sollevazione di Non Morti: il morto che emerge dal­la tomba per reclamare il vivo. Questo disastro può colpire qualsiasi area dove sono stati sepolti dei morti, non solo paesi e città. Più di un campo di battaglia ha visto sorgere una legione di rinsecchiti combattenti Non Morti. Le sollevazioni di Non Morti si svolgono ad ondate, con la tempistica che varia secondo le forze principali in gioco. Gli eventi possono succedersi nel corso di pochi giorni, con la devastazione di una città, o protrarsi per settimane con la popolazione terrorizzata che si rannicchia dietro porte sprangate e lotta per sopravvivere. Durante il giorno, spesso la vita ritorna ad una parvenza di normalità, poiché la luce del giorno sopprime temporaneamente il potere della non morte.

\textbf{I Morti Inquieti}

Nelle prime notti di una sollevazione di Non Morti, i morti recenti si rianimano come zombi. Quelli sepolti in terra consacrata non si rianimano, ma i corpi lasciati insepolti o in fosse comuni barcollano fuori per le strade, portando scompiglio. Inizialmente, solo alcuni cadaveri sono capaci di liberarsi dal­le loro bare e tombe, ma ogni sera, il numero di cadaveri vivi aumenta. Quando giunge l'alba, i morti cercano la sicurezza nelle loro tombe o di altri luoghi nascosti. Chiunque venga colto dalla luce del giorno si agita Confuso finché non viene distrutto o raggiunge un rifugio. A discrezione del Narratore, cadaveri di non umanoidi possono risorgere come Non Morti nelle notti seguenti.

\textbf{Il Risveglio degli Scheletri}

Con l'avanzare della sollevazione, cadaveri sempre più vecchi si uniscono alle schiere dei Non Morti. Scheletri che recano tracce di vesti funebri marcite da tempo scavano con gli artigli una via d'uscita da cimiteri e cripte, ed agiscono con una malevolenza ed organizzazione raramente riscontrate tra i loro simili. I Non Morti rimangono privi di Intelletto, ma il potere magico dietro all'incursione dona loro l'efficienza e l'acume tattico di un esercito di viventi. Gli Scheletri scovano armi e corazze con cui equipaggiarsi per la battaglia. L'élite degli Scheletri campioni guida le truppe, utilizzando Oggetti Magici trafugati da tombe abbandonate. Infine, anche Ghoul e Wight vagano in cerca di preda per le strade durante il buio, insieme ad altri Non Morti minori dotati di libero arbitrio

\textbf{Anime Perse}

Mentre la sollevazione raduna le forze, si risvegliano anche le anime inquiete di cadaveri da tempo ridotti in polvere. Fantasmi, Ombre, Wraith e persino Spettri sorgono per dare la caccia ai vivi. Alcuni Fantasmi potrebbero liberarsi dalla malevola influenza della sollevazione e dei personaggi intraprendenti potrebbero raccogliere preziose informazioni da questi spiriti inquieti.

L'infusione di energia negativa fortifica i Non Morti all'interno dell'area dell'incursione, concedendo i benefici di una Essenza di Protezione (+2 Difesa/+2 Tiri Salvezza). Le aree una volta consacrate sono ora trattate come terreno normale, e possono fungere da nuove fonti di cadaveri per le armate Non Morte; il terreno santificato rimane inviolato.

Quando i Non Morti diventano più forti, l'ondata crescente di energia negativa avvicina il Piano delle Ombre, stingendo o ingrigendo i colori tranne durante le ore più brillanti del giorno. Anche i Non Morti più vulnerabili alla luce possonomuoversi impunemente dal tardo pomeriggio alla mezza mattinata.

\textbf{Necropoli}

Il flusso di energia negativa è irreversibile, l'oscurità infine reclama l'area, coprendola con un'ombra perpetua.Il terreno santificato resta un raro santuario, ma solo finché non viene distrutto dalle forze malevoli forze esterne.

Gli eroi morti negli scontri ritornano come spaventosi generali Non Morti. I pochi superstiti viventi vengono assoggettati come schiavi. L'area diviene una città della morte o ne viene cominciata la costruzione se non esisteva o non è sopravvissuta alcuna città. I Non Morti dotati di libero arbitrio si radunano in questo nuovo santuario e solo gli eroi più grandi riescono a tornare da quest'area ormai avvizzita al mondo dei vivi.

\pagebreak

\section{Avventure nei Dungeon}\index{Dungeon}

\begin{tcolorbox}[enhanced,arc=5pt,boxrule=0.3pt]{
Linux is user friendly. It's just very picky about who its friends are (anonimo) (Full Metal Jacket, Film, 1987)}\end{tcolorbox}
\medskip


\label{avventure-nei-dungeon}
\begin{tcolorbox}[enhanced,arc=5pt,boxrule=0.3pt]{Il dungeon è inclinato. Le creature sono infuriate perché non riescono a giocare a biglie (Dungeon Keeper 2,Videogioco, 1999)}\end{tcolorbox}\medskip

Di tutti i luoghi strani che un avventuriero può esplorare, nessuno è più letale di un dungeon. Questi labirinti, pieni di trappole mortali, mostri affamati etesori meravigliosi, provano ogni Abilità dei personaggi. Queste regole si possono applicare a qualsiasi tipo di dungeon, dal relitto di una nave ad un vasto complesso di grotte sotterranee.

\textbf{Tipi di Dungeon}

I quattro tipi base di dungeon sono definiti dal loro stato attuale. Molti dungeon sono varianti di questi tipi base o combinazioni di più tipi. Occasionalmente, antichi dungeon vengono usati ripetutamente da nuovi abitanti per scopi diversi.

\textbf{Struttura in Rovina}: Un tempo abitato, questo luogo è ora abbandonato (completamente o in parte) dai suoi creatori originari ed è occupato da altre creature. Molte creature sotterranee vanno alla ricerca di costruzioni sotterraneeabbandonate in cui stabilire le loro tane. Qualsiasi trappola che possa essere esistita è stata probabilmente già rimossa o attivata, ma è possibile trovare bestie erranti.

\textbf{Struttura Occupata}: Questo dungeon viene ancora utilizzato. Delle creature (di solito intelligenti) ancora lo abitano, anche se potrebbero non essere i creatori del dungeon. Una struttura occupata potrebbe essere una casa, una fortezza, un tempio, una miniera attiva, una prigione, un quartier generale.

Questo tipo di dungeon è meno probabile che abbia trappole o bestie erranti, e più probabilmentedispone di guardie organizzate, sia stazionarie che di pattuglia. Le trappole e le bestie erranti che si possono incontrare sono spesso sotto il controllo degli occupanti. Le strutture occupate dispongono di arredo adatto agli abitanti, così come decorazioni, riserve di cibo, e la possibilità per gli abitanti di muoversi.

Gli abitanti possono disporre anche di un sistema di comunicazione, e quasi sempre
controllano almeno un accesso verso l'esterno.

Alcuni dungeon sono parzialmente occupati e parzialmente vuoti o in rovina. In questi casi, gli occupanti di solito non sono gli originari costruttori del luogo, ma bensì un gruppo di creature intelligenti che hanno stabilito la loro base, tana o fortificazione all'interno del dungeon abbandonato.

\textbf{Riparo Sicuro}: Quando qualcuno vuole proteggere una cosa, spesso la seppellisce sottoterra. Che l'oggetto che vuole proteggere sia un favoloso tesoro, un artefatto proibito o il cadavere di un uomo importante, questi oggetti di valore vengono posti all'interno di un dungeon e circondati da barriere, trappole e guardiani.

Il dungeon del tipo riparo sicuro è quello che avrà più trappole e meno bestie erranti. E' normalmente costruito in base alla funzionalità piuttosto che all'aspetto, anche se a volte viene decorato con statue e pareti dipinte, specie per le tombe di personaggi importanti.

A volte, pero', una sala del tesoro o una cripta vengono costruite in modo da ospitare guardiani viventi. Il problema con questa strategia è che occorre tenere in vita le creature tra un tentativo di intrusione e un altro. La magia è di solito la soluzione migliore per rifornire di cibo e acqua queste creature. I costruttori di tombe e sepolcri, di solito, pongono non morti e costrutti, che non hanno bisogno di sostentamento o di riposo, a protezione dei loro dungeon. Le trappole magiche possono attaccare gli intrusi convocando mostri nel dungeon che scompaiono quando terminano il loro compito.

\textbf{Complesso di Caverne Naturali}: Le caverne sotterranee offrono riparo a qualsiasi tipo di creatura delle profondità. Create naturalmente e collegate da un sistema di passaggi labirintici, queste caverne mancano di qualsiasi parvenza di ordine, logica o decorazioni. Senza alcuna potenza intelligente che lo abbia costruito, questo tipo di dungeon è quello che ha minori probabilità di presentare trappole o porte.

Molteplici varietà di funghi vivono nelle caverne, a volte crescendo fino a formare enormi foreste di funghi e vesce, dove si aggirano predatori sotterranei si aggirano a caccia di chi si nutre di questi vegetali. Alcune varietà di funghi producono un bagliore fosforescente in grado di fornire al complesso di caverne naturali una propria limitata fonte di illuminazione. In altre zone, l'uso di Essenza di Creazione può garantire luce sufficiente per la crescita di piante verdi.

Spesso, un complesso di caverne naturali è collegato ad altri tipi di dungeon, essendo stato scoperto quando è stato costruito il dungeon artificiale. Un complesso di caverne può collegare due dungeon indipendenti, producendo a volte uno strano ambiente misto. Un complesso di caverne naturali unito a un altro dungeon, spesso, offre un percorso che le creature sotterranee possono usare per raggiungere un dungeon artificiale e popolarlo.

\textbf{Terreno del Dungeon}

Le regole seguenti riguardano i terreni di base che si possono trovare in un dungeon.

\textbf{Pareti}

A volte, pareti in mattoni (pietre accatastate una sopra l'altra solitamente, ma non sempre, tenute insieme con la calce) dividono i dungeon in corridoi e stanze. Le pareti dei dungeon possono anche essere scolpite nella nuda roccia, ottenendo così un aspetto scalpellato, oppure possono essere composte di pietra liscia e semplice come si trova nelle caverne naturali. Le pareti dei dungeon sono difficili da danneggiare o da sfondare, ma di solito sono facilmente scalabili.

\bigskip

\textbf{Tabella: Pareti}
\medskip

\begin{tabular}{llllll}
	\toprule
	\textbf{Tipo di Parete} & \textbf{Spessore Tipico} & \textbf{DC per Sfondare} & \textbf{Durezza} & \textbf{Punti Ferita} & \textbf{DC per Scalare}\\
	Mattoni  & 30 cm& 35   & 8 & 90& 20\\
	Mattoni superiori  & 30 cm& 35   & 8 & 120    & 25\\
	Mattoni rinforzati & 30   & 45   & 8 & 180    & 20\\
	Pietra Scolpita    & 90   & 50   & 8 & 540    & 25\\
	Pietra grezza & 150 cm    & 65   & 8 & 900    & 25\\
	Ferro    & 7.5 cm    & 30   & 10& 90& 25\\
	Carta    & variabile & 1    & --& 1 & 30\\
	Legno    & 15 cm& 20   & 5 & 60& 21\\
\end{tabular}

\bigskip

\textbf{Pareti in Mattoni}: Il tipo più comune di parete per un dungeon, le pareti in mattoni di solito sono spesse almeno 30 centimetri. Spesso queste antiche pareti presentano fori e fessure, all'interno dei quali possono annidarsi fanghiglie e piccole creature, che aspettano lì le loro prede. Le pareti di mattone sono in grado di bloccare tutti i rumori, tranne quelli più forti. E' necessaria una prova di Scalare (Resistenza) con DC 20 per muoversi lungo una parete in mattoni.

\textbf{Pareti in Mattoni di Qualità Superiore}: A volte le pareti in mattoni sono costruite meglio (più lisce, con pietre meglio incastrate e meno danneggiate) e occasionalmente queste pareti di qualità superiore sono coperte da calcina o stucco. Queste pareti sono spesso abbellite da dipinti, bassorilievi o altre decorazioni. Le pareti in mattoni di qualità superiore non sono più difficili da danneggiare delle normali pareti in mattoni, ma sono più difficili da Scalare (Resistenza) (DC 25).

\textbf{Pareti rinforzate} Queste sono pareti in mattoni con sbarre di ferro su uno o entrambi i lati, o inserite all’interno della parete stessa per rinforzarla. La Durezza della parete rinforzata resta la stessa, ma i Punti Ferita vengono raddoppiati e la DC per la prova di Potenza per sfondarla viene incrementata di 10.

\textbf{Pareti di Pietra Scolpita}: Queste pareti generalmente si trovano in stanze o passaggi scavati nella nuda roccia. La ruvida superficie di una parete scolpita presenta minuscole sporgenze su cui possono crescere funghi e crepe all'interno delle quali possono vivere parassiti, pipistrelli o serpi sotterranee.

Quando una parete di questo tipo ha un "altro lato" (la parete separa due stanze in un dungeon), la parete è spessa almeno 90 centimetri; se fosse più sottile rischierebbe di far crollare tutto perché non sarebbe in grado di sostenere il peso della volta di pietra. E' necessaria una prova di Scalare (Resistenza) con DC 25 per scalare una parete di pietra scolpita.

\textbf{Pareti di Pietra Grezza}: Queste superfici sono irregolari e raramente piatte. Sono lisce al tocco ma piene di minuscoli buchi, alcove nascoste e sporgenze a varie altezze. Di solito sono bagnate o perlomeno umide, in quanto le caverne naturali sono in genere il prodotto di infiltrazioni d'acqua. Quando una parete di questo tipo da un "altro lato", la parete è di solito spessa almeno 150 centimetri.

è necessaria una prova di Scalare con DC 15 per muoversi lungo una parete di pietra grezza.

\textbf{Pareti di Ferro}: Queste pareti sono poste all'interno dei dungeon intorno a luoghi importanti come le sale del tesoro.

\textbf{Pareti di Carta}: Le pareti di carta sono l'opposto di quelle di ferro, utilizzate come schermi per impedire la vista ma nulla più.

\textbf{Pareti di Legno}: Le pareti di legno si trovano spesso come recenti aggiunte a dungeon più antichi, utilizzate per creare recinti per animali, depositi, o anche solo per dividere in una serie di stanze più piccole una più grande.

\textbf{Pareti Trattate Magicamente}: Queste pareti sono più forti della media, con una Durezza maggiore, con più Punti Ferita e per sfondarle bisogna superare una DC maggiore. La magia può di solito raddoppiare la Durezza e i Punti Ferita della parete e aggiungere fino a +20 alla sua DC per sfondarla. Una parete trattata magicamente ottiene anche un Tiro Salvezza contro Essenze che potrebbero avere effetto su di essa, con il bonus al Tiro Salvezza pari a 2 + metà del livello dell'incantatore della magia che rinforza la parete. Creare una parete magica richiede il talento Creare Oggetti Meravigliosi e la spesa di 1.500 mo per ogni sezione di 3 per 3 metri.

\textbf{Pareti con Feritoie}: Le pareti con feritoie possono essere costruite con qualsiasi materiale resistente, ma sono di solito fatte in mattoni, pietra scolpita o legno. Permettono ai difensori di scagliare frecce o quadrelli da balestra contro gli intrusi restando dietro la relativa protezione di un muro. Gli arcieri dietro alle feritoie godono di una Copertura superiore che fornisce loro bonus +8 alla Difesa, bonus +4 ai Tiri Salvezza su Riflessi.

\textbf{Pavimenti}

Così come per le pareti, esistono molti tipi di pavimenti per dungeon.

\textbf{Lastricato}: Come le pareti in mattoni, i pavimenti possono essere composti da pietre incastrate tra loro. Sono di solito piene di fessure e solitamente appena livellate. Fanghiglie e muffe crescono all'interno di queste fessure. In certi casi l'acqua scorre in piccoli scoli attraverso le pietre o forma pozze stagnanti. Il lastricato è il tipo di pavimento più comune nei dungeon.

\textbf{Lastricato Irregolare}: Col passare del tempo, alcuni pavimenti possono diventare talmente irregolari da richiedere una prova di Acrobatica con DC 10 per correre o Caricare sulla loro superficie. Coloro che falliscono la prova non possono muoversi durante quel round. Pavimenti così pericolosi dovrebbero essere in realtà l'eccezione e non la regola.

\textbf{Pavimento di Pietra Scolpita}: Ruvidi e irregolari, i pavimenti scolpiti nella pietra sono di solito coperti da pietre smosse, ghiaia, polvere e altri detriti. Una prova di Acrobatica con DC 10 è necessaria per correre o Caricare su un simile pavimento. Un fallimento significa che il personaggio può ancora agire, ma non può correre o Caricare in quel round.

\textbf{Pietrisco Scarso}: Piccoli e sparuti detriti sono presenti a terra. Un pavimento su cui sia presente del pietrisco scarso aggiunge 2 alla DC delle prove di Acrobatica.

\textbf{Pietrisco Denso}: Il terreno è ricoperto di detriti di tutte le dimensioni. Entrare in una zona di mischia ricoperto di pietrisco denso costa 2 azioni di movimento. Un pavimento cosparso di pietrisco denso aggiunge 5 alla DC delle prove di Acrobatica, e aggiunge 2 alla DC delle prove di Consapevolezza (Muoversi Silenziosamente)

\textbf{Pavimento di Pietra Liscia}: Pavimenti lisci, perfetti e a volte anche levigati si trovano solo nei dungeon creati da costruttori capaci e attenti.

\textbf{Pavimento di Pietra Naturale}: Il pavimento di una caverna naturale è irregolare quanto le pareti. E' difficile che queste caverne presentino ampie superfici piane; è più probabile che i loro pavimenti siano disposti su più livelli.

Alcune superfici adiacenti potrebbero variare in elevazione di appena 30 centimetri, cosicché lo spostamento da un punto all'altro non sia più difficile del salire un gradino di una scala, ma in certi punti il pavimento potrebbe scendere o salire di diverse decine di centimetri, obbligando il personaggio a una prova di Resistenza (Scalare) per spostarsi da una superficie a un'altra.

A meno che non ci sia un percorso scavato dal tempo o ben battuto il terreno è considerato difficile e quindi il movimento è dimezzato, la DC delle prove diAcrobatica è aumentata di 5. La Carica e la corsa in questi ambienti sono impossibili, tranne che sui percorsi in questione.

\textbf{Scivoloso}: Acqua, ghiaccio, melma o sangue possono rendere qualunque pavimento descritto in questa sezione più insidioso. I pavimenti scivolosi aumentano la DC delle prove di Acrobatica di 5.

\textbf{Grata}: Una grata spesso copre una fossa o una zona al di sotto del pavimento principale. Le grate sono di solito costruite in ferro, ma quelle più grosse potrebbero essere anche fatte di tronchi d’albero rinforzati. Molte grate hanno cardini che permettono l’accesso alla zona sottostante (queste grate possono essere chiuse a chiave come una porta), mentre altre sono fisse e create per non poter essere spostate. Una tipica grata di ferro spessa 2,5 centimetri ha 25 Punti Ferita, Durezza 10, e DC 27 per sfondarla o smuoverla.

\textbf{Sporgenze}: Le sporgenze permettono alle creature di camminare al di sopra di un'area sottostante. Spesso sono disposte intorno a fosse, lungo il corso di fiumi sotterranei, come balconate che circondano un'ampia stanza oppure forniscono una posizione dalla quale gli arcieri possono appostarsi per attaccare i nemici dall'alto.

Le sporgenze strette (di ampiezza inferiore a 30 centimetri) richiedono a coloro che vi si muovono sopra delle prove di Acrobatica. Un fallimento implica che il personaggio che si stava muovendo cade dalla sporgenza.

A volte le sporgenze hanno una ringhiera. In questi casi i personaggi ottengono Bonus +5 alle prove di Acrobatica per muoversi lungo la sporgenza. Un personaggio vicino alla ringhiera ha Bonus +2 alla propria prova contrapposta di Potenza per evitare di essere spinto giù dalla sporgenza.

Le sporgenze a volte possono anche essere delimitate da balaustre alte 60-90 centimetri. Simili muri forniscono Copertura da aggressori entro distanza 3 metri dall'altro lato del muro, ammesso che il bersaglio sia più vicino alla balaustra di chi attacca.

Pavimenti Trasparenti: I pavimenti trasparenti, fatti di vetro rinforzato o di materiali magici (o addirittura dall'Essenza di Creazione, Muro), permettono di osservare un ambiente pericoloso dall'alto. I pavimenti trasparenti sono di solito posti al di sopra di pozze di lava, arene, tane di mostri e stanze di tortura.Possono essere usati dai difensori
per sorvegliare un'area.

\textbf{Pavimenti Scorrevoli}: Un pavimento scorrevole è un tipo di botola, creato per essere spostato e rivelare qualcosa che si trova al di sotto. In genere un pavimento scorrevole si muove tanto lentamente che chiunque vi si trovi sopra può evitare di cadere nell'apertura, purché abbia spazio per spostarsi. Se un pavimento di questo tipo scorre tanto velocemente che c'è la possibilità che un personaggio cada in quello che si trova sotto di esso (lance acuminate, una vasca con olio bollente, o una pozza infestata da squali) allora è una trappola.

\textbf{Pavimenti Trappola}: Questi pavimenti sono stati progettati per diventare di colpo pericolosi. Con l'applicazione della giusta quantità di peso o l'azionamento di una leva nelle vicinanze, spuntoni sbucano dal pavimento, fiammate o sbuffi di vapore partono da fori nascosti, o l'intero pavimento si muove. Questi strani pavimenti si trovano di solito dentro alle arene, progettati per rendere i combattimenti più appassionanti e letali. Questo tipo di pavimento è costruito nello stesso modo di una trappola.

\textbf{Porte} \index{Porte}Le porte all'interno dei dungeon sono ben più che semplici entrate o uscite. Spesso possono essere dei veri e propri incontri. Le porte dei dungeon si presentano in tre tipi basilari: di legno, di pietra e di ferro.

\bigskip

\textbf{Tabella: Porte}

\bigskip

\begin{tabularx}{0.95\textwidth}{llllll}
	\toprule
	\textbf{Tipo di porta} & \textbf{Spessore tipico (cm)} & \textbf{Durezza} & \textbf{Punti Ferita} & \textbf{DC per sfondare} \\
    &&   &   & Bloccata  & Chiusa a chiave\\
	Legno semplice    & 2.5  & 5 & 10& 13   & 15\\
	Legno buono  & 3.75 & 5 & 15& 16   & 18\\
	Legno robusto& 5    & 5 & 20& 23   & 25\\
	Pietra  & 10   & 8 & 60& 28   & 28\\
	Ferro   & 5    & 10& 60& 28   & 28\\
	Saracinesca di legno   & 7.5  & 5 & 30& 25{*}& 25{*}\\
	Saracinesca di ferro   & 5    & 10& 60& 25{*}& 25{*}\\
	Serratura    & -    & 15& 30& -    & -\\
	Cardini & -    & 10& 30& -    & -\\
\end{tabularx}

{*} DC per sollevare. Usate la voce appropriata di porta per sfondare.

\bigskip

\textbf{Porte di Legno}: Costruite con spesse assi inchiodate, a volte rinforzate con sbarre di ferro (poste anche per impedire le deformazioni prodotte dall'umidità dei dungeon), quelle di legno sono il tipo più comune di porta. Le porte di legno variano per durezza: possono essere semplici, buone o robuste. Le porte semplici (DC 15 per sfondarle) non sono progettate per tenere alla larga assalitori motivati.

Le porte di buona fattura (DC 18 per sfondarle), sebbene forti e resistenti, non sono comunque progettate per subire una grande quantità di danni. Le porte robuste (DC 25 per sfondarle) sono rivestite in ferro e sono delle barriere discretamente resistenti contro coloro che cerchino di oltrepassarle. Cardini di ferro sorreggono la porta, e di solito un anello circolare posto al centro serve ad aprirla.A volte, al posto di un anello, una porta dispone di una sbarra di ferro su uno o entrambi i lati che funziona come maniglia.

Nei dungeon abitati queste porte sono di solito ben tenute (non bloccate) e non chiuse a chiave, anche se le zone importanti probabilmente saranno chiuse a chiave.

\textbf{Porte di Pietra}: Costruite da blocchi di pietra solida, queste porte pesanti e poco maneggevoli sono spesso pensate in modo da ruotare su se stesse quando vengono aperte, anche se i nani e altri abili artigiani sono in grado di costruire cardini forti abbastanza da sostenere il peso di una porta di pietra.

Le porte segrete nascoste lungo una parete di pietra sono solitamente di pietra. Altrimenti, le porte di questo tipo sono studiate per diventare resistenti barriere che proteggono qualsiasi cosa si trovi al di là di esse. Di conseguenza si trovano spesso chiuse a chiave o sbarrate.

\textbf{Porte di Ferro}: Arrugginite ma resistenti, le porte di ferro in un dungeon sono dotate di cardini come quelle di legno. Queste porte sono le porte più resistenti del tipo non magico. Sono di solito chiuse a chiave o sbarrate.

\textbf{Sfondare}: Le porte dei dungeon possono essere chiuse a chiave, munite di trappole, rinforzate, sbarrate, sigillate magicamente o, a volte, semplicemente bloccate.

Tutti, ad eccezione dei personaggi più deboli, riusciranno a buttar giù una porta con un pesante attrezzo come un maglio, e numerosi Essenze ed oggetti magici possono offrire ai personaggi un modo facile per superare una porta chiusa.

\textbf{DC 10 o inferiore}: Una porta che chiunque può sfondare.

\textbf{DC 11--15}: Una porta che una persona forte dovrebbe sfondare con un solo tentativo, e che una persona di potenza media potrebbe avere qualche speranza di abbattere in un solo colpo.

\textbf{DC 16--20}: Una porta che praticamente chiunque potrebbe sfondare, avendo a disposizione il tempo necessario.

\textbf{DC 21--25}: Una porta che solo una persona forte o molto forte ha una speranza di sfondare, e probabilmente non al primo tentativo.

\textbf{DC 26 o superiore}: Una porta che solo una persona dotata di una potenza eccezionale può avere una qualche speranza di sfondare.

\textbf{Serrature}: Le porte dei dungeon sono spesso chiuse a chiave e così torna utile l'Abilità Disattivare Congegni. Le serrature sono di solito costruite sulle porte, sul bordo opposto ai cardini o dritte nel centro della porta. Le serrature costruite dentro le porte di solito controllano una sbarra di ferro che si estende dalla porta dentro il muro che la sostiene, o una sbarra di ferro o di legno massiccio scorrevole che si prolunga dietro tutta la porta.

Al contrario, i lucchetti non sono costruiti dentro la porta ma di solito scorrono tra due anelli, uno sulla porta e uno sul muro. Serrature più complesse, come quelle a combinazione o quelle ad enigma, sono di solito costruite dentro la porta stessa.

Siccome queste serrature senza chiave sono più grandi e complesse, di solito si trovano solo sulle porte resistenti (robuste porte di legno, di pietra o di ferro).
La DC per scassinare una serratura con una prova di Disattivare Congegni (Criminalita') spesso ricade tra 20 e 30, anche se esistono serrature con DC maggiori o inferiori. Una porta può disporre di più di una serratura, ognuna delle quali da aprire separatamente.

Le serrature sono spesso dotate di trappole, di solito aghi avvelenati che scattano all'infuori per pungere le dita del ladro.

\textbf{Spaccare una serratura}

Una porta speciale potrebbe avere una serratura senza chiave, ma che richiede che venga indovinata la giusta combinazione delle leve vicine o vengano premuti nell'ordine corretto i simboli su un pannello per riuscire ad aprirla.

\textbf{Porte Bloccate}: I dungeon sono spesso luoghi umidi, e in alcuni casi le porte rimangono bloccate, in modo particolare se sono fatte di legno. Di solito si suppone che all'incirca il 10\% delle porte di legno e il 5\% delle altre porte siano bloccate. Questi valori possono essere raddoppiati (al 20\% e 10\% rispettivamente) nel caso di dungeon da tempo abbandonati o trascurati.

\textbf{Porte Sbarrate}: Quando un personaggio cerca di sfondare una porta sbarrata, è la qualità della sbarra che fa la differenza, non il materiale della porta in sé. Sfondare una porta chiusa da una sbarra di legno richiede una prova di Potenza con DC 25, e la DC sale a 30 nel caso di una sbarra metallica.

I personaggi possono attaccare la porta e distruggerla, lasciando la sbarra appesa nel passaggio sgombro.

\textbf{Sigilli Magici}: Essenze di Attacco messe su una porta possono rendere ostico l'attraversamento di una porta.

Una porta su cui è stato lanciato un blocco magico si considera chiusa anche se non ha fisicamente una serratura. E' necessario una Essenza che scassina o Distruggi magie oppure una prova riuscita di Potenza per oltrepassare una porta chiusa in questo modo.

\textbf{Cardini}: La maggior parte delle porte è dotata di cardini. Ovviamente le porte scorrevoli non lo sono (queste sono piuttosto dotate di solchi sul pavimento, che permettono loro di scorrere a lato con facilita').

\textbf{Cardini Standard}: Questi cardini sono di metallo e tengono unita la porta al suo sostegno o alla parete. Ricordarsi che la porta si apre verso il lato dove si trovano i cardini (quindi se i cardini sono dal lato dei PG, la porta si aprirà verso di loro; altrimenti si aprirà verso l’altra direzione).

Gli avventurieri possono rimuovere i cardini uno alla volta superando varie prove di Disattivare Congegni (Criminalita') (solo se, naturalmente, sono davanti al lato della porta su cui si trovano i cardini). Una simile azione ha una DC di 20, in quanto molti dei cardini sono arrugginiti o bloccati.

Spaccare un cardine è difficile. La maggior parte ha Durezza 10 e 30 Punti Ferita. La DC per spaccare un cardine è la stessa che serve per abbattere la porta

\textbf{Cardini a Inserimento}: Questi cardini sono molto più complessi e si trovano solo in zone di eccellente costruzione. Questi cardini sono costruiti dentro la parete e permettono alla porta di aprirsi in entrambe le direzioni. I personaggi non possono raggiungere i cardini per rimuoverli a meno che non sfondino il sostegno della porta o la parete. I cardini a inserimento si trovano di solito sulle porte di pietra, ma a volte si vedono anche su porte di legno o di ferro.

\textbf{Perni}: I perni non sono veri cardini, ma semplici pioli che si protendono dal lato superiore e inferiore della porta e si infilano dentro i buchi nel suo sostegno, permettendole di girare. I vantaggi dei perni è che non possono essere rimossi come i cardini e che sono facili da realizzare. Lo svantaggio è che siccome la porta gira sul suo centro di gravità (di solito nel mezzo), nulla più grosso di metà dell'ampiezza della porta vi può passare attraverso.

Le porte dotate di perni sono di solito di pietra e spesso anche abbastanza larghe per ovviare allo svantaggio. Un'altra soluzione è quella di piazzare il perno verso un'estremità e fare la porta più spessa da quella parte e più sottile dall'altra, in modo che si apra più o meno come una porta normale.

Le porte segrete all'interno di muri spesso ruotano, in quanto la mancanza di cardini rende più facile occultare la presenza della porta. I perni permettono anche a oggetti come una libreria di essere usati come porte segrete.

\textbf{Porte Segrete}: Camuffata da comune porzione di muro (o di pavimento o di soffitto), da libreria, da focolare, da fontana, una porta segreta porta ad un passaggio segreto oppure ad una stanza.

Qualcuno che stia esaminando la zona può trovare una porta segreta (se ne esiste una) con una prova riuscita di Consapevolezza (con DC 20 per una porta segreta comune e DC 30 per una porta molto ben nascosta).

Molte porte segrete richiedono un metodo speciale per essere aperte, come un bottone nascosto o una piastra a pressione. Le porte segrete possono aprirsi come porte comuni, girare su un perno, scorrere, sprofondare, sollevarsi o anche calare come un ponte levatoio.

Un costruttore potrebbe piazzare una porta segreta molto bassa vicino al pavimento oppure molto in alto su un muro, in modo da rendere più difficile sia il rinvenimento che l'utilizzo della porta.

\textbf{Porte Magiche} Incantata dal costruttore originario, una porta può apostrofare gli esploratori invitandoli a non proseguire. Potrebbe essere protetta dai danni, con una Durezza maggiore o un numero maggiore di Punti Ferita, oltre che un bonus al Tiro Salvezza migliorato contro Essenze di Distruzione ed effetti simili. Una porta magica potrebbe non condurre allo spazio che si trova dietro di essa, ma essere in realtà un portale verso un luogo molto distante o addirittura verso un altro piano di esistenza. Altre porte magiche potrebbero aver bisogno di una parola d'ordine o di chiavi speciali per aprirsi.

\textbf{Saracinesche}: Queste porte speciali sono fatte con aste di ferro o di spesso legno rinforzato che calano da un recesso nella parte superiore di un arco. A volte una saracinesca dispone di barre orizzontali a formare una griglia, altre volte no. Sollevate di solito con un argano o simile macchinario, le saracinesche possono esser fatte scendere in fretta, e le sbarre terminano in punte per scoraggiare chiunque dal passarci sotto (o dal tentare di attraversarle in corsa mentre calano). Una volta scesa, una saracinesca si chiude, a meno che non sia così grande che nessuna persona normale sarebbe in grado di sollevarla. In ogni caso, sollevare una tipica saracinesca richiede una prova di Potenza con DC 25.

\textbf{Pareti, Porte ed azioni di Individuazione}

Le pareti di pietra, di ferro e le porte di ferro sono generalmente sufficientemente spessi da bloccare la maggior parte delle Essenze di Rivelazione. Le pareti di legno, le porte di legno e di pietra in genere non sono sufficientemente spesse da fare altrettanto. Tuttavia, una porta segreta di pietra costruita in un muro e spessa come il muro stesso (almeno 30 centimetri) bloccherà la maggior parte di queste Azioni.

\textbf{Scale} Il metodo più tradizionale per collegare differenti livelli di un dungeon è attraverso le scale. Un personaggio può salire o scendere una scala come parte del suo movimento senza penalità ma non può correre. Aumentate la DC di qualsiasi prova di Acrobatica effettuate su una scala di 4. Alcune scale, particolarmente ripide, vengono trattate come terreno difficile.

\subsubsection{Pericoli nei Dungeon}

Nei dungeon e nelle caverne oltre ai mostri ci sono anche altri pericoli tra crolli, muffe, funghi e altro.

\textbf{Crolli e Cedimenti (CR 8)}

I crolli e i cedimenti nei tunnel sono estremamente pericolosi. Non solo gli esploratori di dungeon corrono il rischio di essere schiacciati da tonnellate di pietra, ma anche, qualora dovessero sopravvivere, di rimanere bloccati sotto un mucchio di detriti o di essere impossibilitati a raggiungere un'uscita.

Un crollo seppellisce chiunque si trovi nel mezzo della zona sepolta, e quindi i detriti che rotolano via infliggeranno danni a tutti coloro che si trovano nelle zone periferiche alla zona sepolta. Un tipico corridoio soggetto a un crollo potrebbe disporre di una zona sepolta con raggio 3 metri e una zona di scorrimento con raggio di mischia all'estremità di quella sepolta.

Un soffitto pericolante può essere identificato con una prova di Conoscenze (ingegneria) con DC 20 o Artigianato (lavori in muratura) con DC 20. Da non dimenticare che le prove di Artigianato possono essere effettuate senza addestramento come prove di Intelligenza. Un Nano può effettuare questa prova semplicemente passando entro 3 metri di distanza da un soffitto pericolante.

Un soffitto pericolante può crollare sotto l'impatto di una grossa potenza. Un personaggio può provocare un crollo distruggendo la metà dei pilastri che reggono il soffitto.

I personaggi che si trovano nella zona sepolta subiscono 8d6 danni, o danni dimezzati se superano un Tiro Salvezza su Riflessi con DC 15. A quel punto sono sepolti. I personaggi nella zona di scorrimento subiscono 3d6 danni, o nessun danno se superano un Tiro Salvezza su Riflessi con DC 15. I personaggi che si trovano nella zona di scorrimento, sono anch'essi sepolti, se falliscono il Tiro Salvezza.

I personaggi sepolti subiscono 1d6 danni non letali per ogni minuto che rimangono sotto le macerie. Se un personaggio in queste condizioni cade privo di sensi, deve effettuare una prova di Potenza con DC 15. Se il personaggio fallisce la prova, inizia a subire 1d6 danni letali al minuto fino a quando non viene liberato o muore.

I personaggi che non sono stati sepolti possono estrarre i loro compagni da sotto le macerie. In 1 minuto, usando solo le mani, un personaggio può spostare una quantità di roccia e detriti pari a cinque volte il proprio limite di carico pesante. La quantità di roccia smossa che riempie un'area di mischia pesa all'incirca 1 tonnellata (1.000 kg). Equipaggiato con gli strumenti adatti, come un piccone, un piede di porco, o una pala, uno scavatore può impiegare la metà del tempo che impiegherebbe facendolo a mano. Si potrebbe anche concedere a un personaggio sepolto di liberarsi da solo superando una prova di Potenza con DC 25.

\textbf{Fanghiglie, Muffe e Funghi}

Negli umidi e oscuri recessi dei dungeon, le muffe e i funghi prosperano. Per quanto riguarda Essenze e altri effetti speciali, tutte le fanghiglie, le muffe e i funghi sono considerati vegetali. Come le trappole, le fanghiglie e le muffe pericolose sono dotate di un CR, e i personaggi guadagnano Punti Esperienza per averle incontrate.

Una lucida melma organica ricopre qualsiasi cosa che rimanga per troppo tempo immersa nell'oscurità e nell'umidità dei dungeon. Questo tipo di fanghiglia, benché possa essere repellente, non è pericoloso. Le muffe e i funghi abbondano nei luoghi bui, freddi e umidi. Sebbene alcuni siano innocui quanto le normali fanghiglie dei dungeon, altri sono alquanto pericolosi. Funghi commestibili, vesce, lieviti, muffe e altri tipi di funghi fibrosi, bulbosi o intere distese di spore fungine possono essere rinvenuti nella maggior parte dei dungeon. Di solito sono innocui e spesso sono anche commestibili (anche se la maggior parte è poco invitante o ha uno strano sapore).

\textbf{Boleto Stridente}\index{Boleto Stridente}: Questi funghi viola di grandezza umana emettono un suono penetrante che dura 1d3 round ogni volta che c'è un movimento o una sorgente di luce entro raggio 3 metri. Questo grido rende impossibile sentire altri suoni o rumori entro raggio di mischia. Il suono attira le creature nelle vicinanze che sono disposte ad investigare. Alcune creature che vivono vicino ai boleti stridenti hanno imparato che il rumore significa molto spesso cibo.

\textbf{Fanghiglia Verde}\index{Fanghiglia Verde} (CR 4): Questo pericolo dei dungeon è una varietà insidiosa della normale fanghiglia. La fanghiglia verde divora la carne e i materiali organici che vi entrano in contatto, ed è addirittura capace di dissolvere i metalli. Di un verde splendente, bagnata e appiccicosa, si distribuisce a chiazze su pareti, pavimenti e soffitti e si riproduce consumando materiale organico. Si lascia cadere dalle pareti e dai soffitti quando individua del movimento (e possibile nutrimento) sotto di sé.

La fanghiglia verde infligge 1d3 danni alla Potenza per ogni round in cui divora la carne. Al primo round di contatto, la fanghiglia può essere asportata da una creatura (con la probabile distruzione dell'oggetto utilizzato per asportarla), ma dopo il primo round deve essere congelata, bruciata o tagliata (infliggendo danni anche alla sua vittima) per essere rimossa. Tutto ciò che infligge danni da fuoco o da freddo, la luce solare o una Essenza di Cura rimuovi malattia distruggono una chiazza di fanghiglia verde. Nel caso di legno o metallo, la fanghiglia verde infligge 2d6 danni per round, ignorando la Durezza del metallo ma non quella del legno. Non danneggia la pietra.

\textbf{Fungo Fosforescente}\index{Fungo Fosforescente}: Questo strano fungo sotterraneo emana una debole luminescenza violacea che illumina le caverne e i passaggi sotterranei come una candela. Rare macchie di questo fungo illuminano come una torcia.

\textbf{Muffa Gialla} \index{Muffa Gialla}(CR 6): Se disturbata, nel raggio di 3 metri rilascia una nube di spore velenose. Tutti coloro entro raggio di 3 metri dalla muffa devono superare un Tiro Salvezza su Tempra con DC 15 o subiscono 1d3 danni a Potenza. Un altro Tiro Salvezza su Tempra con DC 15 è necessario una volta per round per i successivi 5 round o per evitare di subire altri 1d3 danni a Potenza. Un Tiro Salvezza riuscito blocca questo effetto. Il fuoco distrugge la muffa gialla, mentre la luce solare la rende inerte.

\textbf{Muffa Marrone} \index{Muffa Marrone}(CR 2): La muffa marrone si nutre di calore, estraendolo da tutto ciò che la circonda. Di solito si presenta in chiazze con diametro di dimensione mischia e la temperatura attorno alla muffa risulta sempre fredda in un raggio di 3 metri. Le creature viventi entro una distanza di mischia da essa subiscono 3d6 danni non letali da freddo. Se viene portata una fonte di fuoco entro mischia dalla muffa questa raddoppia immediatamente le proprie dimensioni. I danni da freddo, come quelli inflitti da un cono di freddo, la distruggono all'istante.

\pagebreak

\section{Pericoli in Avventura}\index{Pericoli in Avventura}


\begin{tcolorbox}[enhanced,arc=5pt,boxrule=0.3pt]{Un'avventura è un risultato ragionevole. Due sono meglio, tre meritano di essere tramandate, e quattro... nessuno potrà mai contestare quattro avventure. (John Steinbeck)}\end{tcolorbox}\medskip


\label{pericoli-in-avventura}
\begin{tcolorbox}[enhanced,arc=5pt,boxrule=0.3pt]{Corre meno pericoli colui che, anche se è al sicuro, sta in guardia. (Publilio Siro)}\end{tcolorbox}\medskip
Il mondo è pieno dì pericoli oltre che di draghi ed immondi famelici. I pericoli sono minacce basate sulle peculiarità della zona che hanno molto in comune con le trappole, ma che di solito fanno parte del posto anziché venir costruite. I pericoli si dividono in tre categorie principali: ambientali, viventi e magici.

I pericoli ambientali includono frane, incendi e simili. I pericoli viventi includono creature che pur non essendo considerate mostri, rappresentano una minaccia per gli av­venturieri incauti, come fanghiglie, funghi e muschi. I pericoli magici sono i più imprevedibili e possono essere residui di esperimenti arcani, strane radiazioni sotterraneo o antichi Essenze fallite.

\textbf{Antidweomer (CR 6)}\index{Antidweomer}

Zona di entropia magica che distrugge le Essenze, gli antidweomer si formano sui siti di grandi duelli magici, attraverso la distruzione di potenti artefatti o da vortici di energia mistica ai margini delle zone di antimagia. Le dimensioni variano da piccole bolle di appena pochi metri fino a grandi aree delle dimensioni di una città.

Una prova riuscita di Sapienza Magica con DC 20 rivela la vicinanza di un antidweomer con un formicolio nell'aria. Una magia attiva portata in un antidweomer potrebbe venir dissolta, e qualsiasi Essenza lanciata al suo interno è soggetta ad un controincantesimo immediato (l'incantatore deve fare un Tiro Salvezza su Arbitrio a difficoltà 20). Il conseguente rilascio di energia magica infligge 1d6 danni Comptenza Magia in un'esplosione a raggio mischia centrata su chi ha portato la magia nell'area o chi ne ha lanciato una nuova al suo interno (TS su Riflessi con DC 15 dimezza).

Se più scoppi sovrapposti colpiscono lo stesso bersaglio, si applica solo quello più dannoso. Una magia che ha resistito ad un tentativo di dissoluzione, non viene influenzato nuovamente a meno che non esca e rientri nell'antidweomer.

Gli antidweomer più potenti sono ancora più distruttivi. Ogni +1 di incremento del CR aumenta il LI delle prove di dissoluzione di 2 e la DC del Tiro Salvezza per i danni dell'esplosione di 1.

\textbf{Aria Viziata (CR 1 o 4)}\index{Aria Viziata}

Un pericolo invisibile, le sacche di gas sono un rischio per minatori, speleologi e avventurieri che investigano nelle caverne. I gas ininfiammabili come il diossido di carbonio o l'azoto hanno CR 1 e richiedono una prova di Sopravvivenza con DC 25 per essere notati.

Le creature che respirano quell'aria devono superare un Tiro Salvezza su Tempra (DC 15 +1 per ogni tiro precedente) ogni ora o diventano Affaticate. Una volta Affaticate, iniziano a Soffocare Lentamente. Le creature che trattengono il fiato possono evitare questi effetti.

I vapori infiammabili come il gas di carbone sono molto più pericolosi (CR 4). Questo gas sostituisce l'aria respirabile nei polmoni, provocando affaticamento: inoltre, qualsiasi fiamma aperta o scintilla causa un'esplosione che infligge 6d6 danni (TS su Riflessi con DC 15 dimezza) a chi è nella caverna o entro distanza mischia da un ingresso. Il fuoco brucia l'ossigeno nell'aria, rendendola irrespirabile per 2d4 minuti. Dopo un'esplosione,il gas infiammabile generalmente impiega molti giorni per ritornare a livelli pericolosi.

\textbf{Parassiti}\index{Parassiti}

Parassiti come cercaorecchie o larve necrofaghe provocano parassitosi, un tipo di Afflizione simile alle Malattie. Le parassitosi possono essere guarite solo attraverso trattamenti specifici; indipendentemente da quanti Tiri Salvezza si effettuano, la parassitosi continua ad affliggere il bersaglio. Anche se una Essenza di Cura per Rimuovi Malattia (o un effetto simile) blocca immediatamente una parassitosi, l'immunità alle Malattie non offre protezione, dato che è causata da parassiti.

\textbf{Cercaorecchie (CR 5)}\index{Cercaorecchie}

I cercaorecchie sono minuscoli vermi bianchi che vivono nel legno marcio o altri detriti organici. Si possono notare con una prova di Consapevolezza (DC 15). Altrimenti, una creatura vivente che frughi nella loro tana si trasferisce inavvertitamente addosso uno o più cercaorecchie, i quali poi cercano una zona calda sul corpo della creatura, prediligendo il condotto uditivo, e li depongono 2d8 uova prima di morire.

Le uova si schiudono 4d6 ore dopo e le larve divorano la carne intorno. Alla morte del loro ospite, i vermetti strisciano fuori e ne cercano uno nuovo.

Rimuovi Malattia (Cura LP 19) uccide tutti i cercaorecchie o le uova non ancora schiuse su un ospite. Alcuni cercaorecchie preferiscono vivere nel legno corrotto, spesso nascondendosi nelle porte dei sotterranei. I piccoli fori lasciati da questa variante sono molto difficili da notare (Consapevolezza DC 20).

\textbf{Cercaorecchie}

Tipo: Parassitosi

TS: Tempra DC 15

Insorgenza: 4d6 ore

Frequenza: 1/ora

Effetti: 1d3 a Potenza

\textbf{Cristalli Mnemonici (CR 3)}\index{Cristalli Mnemonici}

I cristalli mnemonici sono grandi (3-12 metri d'altezza) grappoli di cristalli di quarzo viola che irradiano un'aura di Distruzione forte. Per identificarli occorre una prova di Conoscenze (arcane) con DC 25.

I cristalli mnemonici cumulano energia magica per crescere e difendersi, risucchiando gli incantesimi preparati degli incantatori che devono effettuare un Tiro Salvezza su Volontà con DC 22 ogni round mentre sono entro raggio di 3 metri dai cristalli.

Se il tiro fallisce, perdono il 10\% dei Punti Potere a disposizione. Danneggiando o rompendo i cristalli, le magie assorbite vengono espulsi con un'esplosione di energia mentale che infligge 1d3 danni alla Saggezza a tutti coloro che si trovano entro raggio di mischia.

I cristalli mnemonici sono molto fragili (Durezza 0, 1 Punto Ferita).
In aree ricche di cristalli, le creature che vi passano attraverso devono superare una prova di Acrobatica con DC 10 per evitare di camminarci sopra o sfiorarli rompendoli.

\textbf{Larve Necrofaghe (CR 4)}\index{Larve Necrofaghe}

Una volta occupato un corpo vivente, le larve scavano verso il cuore, il cervello e altri organi interni chiave dell'ospite, provocandone infine la morte.

Nel primo round di parassitosi, applicando del fuoco nel foro di ingresso si possono uccidere le larve e salvare l'ospite, ma questo subisce 1d6 danni da fuoco.

Anche estrarle funziona, ma più a lungo le larve restano nell'ospite, più danni provoca questo metodo. Per estrarre le larve occorre un'arma tagliente ed una prova di Guarire con DC 20, infliggendo 1d6 danni per ogni round che l'ospite è stato afflitto da parassitosi. Se la prova di Guarire riesce una larva viene rimossa. Rimuovi Malattia (Cura LP 19) uccide tutte le larve necrofaghe presenti in un ospite.

\textbf{Larve Necrofaghe}

Tipo: Parassitosi

TS: Tempra DC 17

Insorgenza: immediata

Frequenza: 1/round

Effetti: 1d2 danni a Potenza per larva

\textbf{Minerale Magnetizzato (CR 2)}\index{Minerale Magnetizzato}

Le strane energie del mondo sotterraneo possono caricare pietre e vene di minerali con potenti campi magnetici, creando un pericolo per chi porta o indossa metalli ferrosi. Tutte le cose di ferro o acciaio portate entro raggio di 3 metri dal minerale sono trascinate verso di esso.

Le creature Piccole vengono trascinate anche con 7,5 kg di metallo, quelle Grandi solo con 30 kg. Per creature di altre taglie, il peso cambia in base alle regole della Capacità di Trasporto. Le creature con indosso armature metalliche subiscono una penalità, chi è colpito è trascinato fino a 9 metri, subisce 2d6 danni per l'impatto con la roccia ed è considerato afferrato. Liberare un oggetto colpito richiede una prova di Potenza DC 20-25

\textbf{Polla Maledetta (CR 3)}\index{Polla Maledetta}

I prolungati effetti di antiche maledizioni o l'energia nociva che si propaga da un oggetto magico maledetto sommerso possono trasformare una semplice polla d'acqua in un rischioso pericolo magico. Una polla maledetta attira i passanti nelle sue profondità attraverso l'illusione (TS su Volontà con DC 16 per dubitare) di uno sfavillante tesoro sul fondo profondo 3 metri. Qualsiasi creatura che giunga al tesoro attiva la maledizione.

Una creatura all'interno della polladeve superare un Tiro Salvezza su Volontà con DC 16 o è colpita dalla maledizione, che distorce la sua percezione della polla. L'acqua sembra addensarsi in un viscoso sapropelite, mentre la polla sembra raggiungere una profondità di 12 metri.

Le prove di Nuotare (Resistenza) nella polla subiscono penalità -10, la velocità viene ridotta alla metà del normale a causa di questi effetti e lanciare Essenze al suo interno richiede una prova di Concentrazione con DC pari a 20.

Una polla maledetta irradia una forte magia, e può essere distrutta da Distruzione Magie o da Protezione Rimuovi Maledizione (prova di livello dell'incantatore
con LP 15).

\textbf{Quercia Velenosa (CR 1 o 3)}\index{Quercia Velenosa}

Il contatto con una quercia velenosa (CR 1) causa una dolorosa eruzione cutanea pruriginosa che rende la vittima Inferma finché i danni non guariscono. Un pieno contatto col corpo o l'inalazione del fumo di una quercia velenosa che brucia potrebbero essere fatali (CR 3). Una prova di Conoscenze (natura) con DC 15 rivela i pericoli insiti nella pianta all'apparenza innocua. Questo pericolo può essereusato anche per piante nocive simili (edera velenosa, sommaco velenoso od ortiche pungenti, ma quest'ultime non sono pericolose quando bruciano).

\textbf{Quercia Velenosa}

Tipo: Veleno, contatto

TS: Tempra DC 13

Insorgenza: 1 ora

Effetti: 1d4 danni a Agilità, la creatura è Inferma finché i danni
non guariscono

Cura: 1 TS

\pagebreak

\subsection{Avventure e Trappole}\index{Trappole}

\label{avventure-e-trappole}
\begin{tcolorbox}[enhanced,arc=5pt,boxrule=0.3pt]{Chi pone la trappola sempre allo stesso posto non prenderà alcun'iguana. (proverbio africano)}\end{tcolorbox}\medskip

Le trappole sono un pericolo comune nei dungeon. Da sbuffi di vapore bollente a raffiche di dardi avvelenati, le trappole possono servire a proteggere tesori o ad impedire agli intrusi di procedere.

\textbf{Elementi di una Trappola}

Tutte le trappole, meccaniche o magiche, sono definite da queste peculiarita': CR, tipo, DC di Consapevolezza, DC di Disattivare Congegni, attivatore, ripristino ed effetti. Alcune trappole potrebbero anche includere elementi opzionali, quali i veleni o un tipo di aggiramento. Queste caratteristiche sono descritte sotto.

\textbf{Tipo}

Una trappola può essere di natura meccanica o magica.

\textbf{Meccaniche}: I dungeon sono spesso dotati di letali trappole meccaniche (non magiche). Una trappola viene di solito definita dalla sua posizione e dal meccanismo di attivazione, quanto è difficile notarla prima che venga attivata, quanti danni è in grado di infliggere, e dal fatto che gli eroi possano compiere o meno un Tiro Salvezza per mitigarne gli effetti. Le trappole che utilizzano frecce, lame affilate e altre armi, effettuano normali Tiri per Colpire, con un bonus di attacco specifico che dipende dal tipo di trappola. Si può costruire una trappola meccanica utilizzando con successo l'abilità Artigianato (costruire trappole). (Vedi Progettare una Trappola più avanti e la descrizione della Competenza).

Le creature che superano una prova di Consapevolezza possono individuare una trappola meccanica prima che venga attivata. La DC della prova dipende dalla trappola stessa. In genere il successo indica che la creatura ha individuato il meccanismo di attivazione della trappola, come piastre a pressione, meccanismi collegati a porte e altri tipi di attivazioni insolite. Superare la prova di 5 punti o più fornisce anche alcune indicazioni su quello che la trappola è predisposta a fare.

\textbf{Magica}: Ci sono molte Essenze che possono essere utilizzate per realizzare trappole pericolose. A meno che la descrizione dell'Essenza o dell'oggetto non specifichi altrimenti, è consigliabile tenere conto dei seguenti punti.

Una prova riuscita di Consapevolezza (DC 28) permette di individuare una trappola magica prima che scatti.

Le trappole magiche concedono un Tiro Salvezza per evitarne gli effetti (DC 15).

Le trappole magiche possono essere disarmate da un personaggio con Scoprire Trappole che superi una prova di Disattivare Congegni (Criminalita') (DC 28). Gli altri personaggi non hanno possibilità di disarmare una trappola magica.

Le trappole magiche sono a loro volta suddivise in trappole a Essenza e trappole a congegno magico. Le trappole a congegno magico sprigionano degli effetti magici una volta attivate, proprio come le bacchette, le verghe, gli anelli e gli altri oggetti magici. Per creare una trappola a congegno magico è necessario il talento Creare Oggetti Meravigliosi.

Le trappole a Essenza non sono altro che Essenze utilizzate come trappole. Per creare una trappola a Essenza sono necessari i servigi di un personaggio che sia in grado di lanciare l'Essenza richiesta, che normalmente è il personaggio stesso che crea la trappola, oppure un PNG incantatore assunto a tale scopo.

\textbf{DC di Consapevolezza e Disattivare Congegni (Criminalita')}

Il costruttore stabilisce le DC delle prove di Consapevolezza e di Disattivare Congegni (Criminalita') per le trappole meccaniche. Per le trappole magiche, i valori delle DC dipendono dall'incantatore che ha creato la trappola.

\textbf{Trappola meccanica}: \index{Trappola meccanica}Tutte le prove di Consapevolezza e di Disattivare Congegni (Criminalita') hanno una DC base di 20. Aumentare o diminuire una o entrambe le DC modifica il CR della trappola (Tabella: Modificatori al CR delle Trappole Meccaniche).

\textbf{Trappola Magica}:\index{Trappola Magica} Tutte le prove di Consapevolezza e di Disattivare Congegni (Criminalita') hanno una DC base di 28. Soltanto i personaggi addestrati su Senso Trappola possono effettuare una prova di Disattivare Congegni (Criminalita') su una trappola magica.

\pagebreak

\textbf{Tabella: Modificatori al CR delle Trappole Meccaniche}


\begin{tabular}{ll}
	\toprule
	Elemento   & Modifica al CR\\
	\textbf{DC Consapevolezza}& \\
	15 o meno  & -1   \\
	16-20 & -    \\
	21-25 & +1   \\
	26-29 & +2   \\
	30+   & +3   \\
	\textbf{DC Disattivare congegni}
	 & \\
	15 o meno  & -1   \\
	16-20 & -    \\
	21-25 & +1   \\
	26-29 & +2   \\
	30+   & +3   \\
	\textbf{Modificatori Tiri Salvezza} & \\
	15 o meno  & -1   \\
	16-20 & -    \\
	21-25 & +1   \\
	26-29 & +2   \\
	30+   & +3   \\
	\textbf{Competenza Armi}  & \\
	+0    & -2   \\
	+1/+5 & -1   \\
	+6/+10& -    \\
	+11/+15    & +1   \\
	+16/+20    & +2   \\

	Ogni 10 punti di danno medio   & +1   \\
	Ripristino automatico& +1   \\
	Attivatore Visivo o di Prossimita’  & +1   \\
	Veleno& da +1/+10 \\
\end{tabular}

\bigskip

\textbf{Attivatore}

L'attivatore è il meccanismo che definisce le condizioni che fanno scattare la trappola.

\textbf{Posizione}: Un meccanismo basato sulla posizione fa scattare la trappola quando qualcuno si trova in una zona di mischia predefinita.

\textbf{Prossimita'}: Questo meccanismo fa scattare la trappola quando una creatura si avvicina ad una distanza prestabilita. L'attivatore di prossimità si differenzia da quello di posizione poiché non è necessario che la creatura si trovi nella zona di mischia predefinita. Le creature in volo possono far scattare una trappola di prossimità ma non quelle con un meccanismo di posizione. Gli attivatori di prossimità meccanici sono estremamente sensibili al minimo spostamento d'aria. Pertanto, le trappole di prossimità sono particolarmente indicate in quei luoghi come le cripte, dove l'aria è solitamente stagnante.

L'attivatore di prossimità usato più spesso nelle trappole a congegno magico è l'Essenza Rivelazione.

\textbf{Sonoro}: Questo attivatore magico fa scattare la trappola quando viene individuato un suono. L'attivatore sonoro funziona come un orecchio dotato di bonus +15 alle prove di Consapevolezza. E' bene notare che questo tipo di attivatore viene ingannato da prove riuscite di Muoversi Silenziosamente, o lanciando un'Essenza di Illusione o di distruzione per ricreare una sorta di silenzio magico o altri effetti che bloccano l'udito. Una trappola con attivatore sonoro richiede l'uso dell'Essenza di Rivelazione costruzione.

\textbf{Visivo}: Questo attivatore magico funziona come un occhio, facendo scattare la trappola quando "vede" qualcosa. Il raggio visivo e il bonus alle prove di Consapevolezza dipendono dal potere della Rivelazione usata.

\textbf{Contatto}: In genere, l'attivatore a contatto, che fa scattare la trappola quando viene toccata, è quello più facile da costruire. Questo attivatore può essere o meno integrato con il dispositivo che infligge il danno. Si può creare un attivatore a contatto magico aggiungendo un'Essenza di Illusione che attivi una sorta di allarme alla trappola e riducendo l'area di effetto fino a selezionare solo il punto di attivazione.

\textbf{A Tempo}: Questo attivatore fa scattare la trappola ad intervalli di tempo prestabiliti.

\textbf{Magia}: Tutte le trappole ad Essenza sono dotate di questo tipo di attivatore. La descrizione delle Essenze spiegano le modalità di attivazione delle trappole ad attivazione di Essenza.

\textbf{Durata}
A meno che non sia indicato diversamente, la maggior parte delle trappole ha durata istantanea; una volta attivate, non ci sono altri effetti e terminano di funzionare. Alcune trappole hanno una durata misurata in round. Alcune trappole continuano ad avere gli effetti indicati ad ogni round all'inizio dell'ordine di Iniziativa (o quando sono state attivate, se questo è avvenuto durante un combattimento).

\textbf{Ripristino}
Il ripristino di una trappola è l'insieme di condizioni per cui una trappola viene riattivata, pronta a scattare di nuovo. Solitamente per ripristinare una trappolaoccorre un minuto. Per una trappola con un metodo di ripristino più complicato, il tempo ed il lavoro richiesti potrebbero aumentare.

\textbf{Irripristinabile}: A meno di ricostruire la trappola, non c'è modo di farla scattare più di una volta. Le trappole ad Essenza non permettono alcun tipo di ripristino.

\textbf{Riparabile:} La trappola può funzionare di nuovo, ma deve essere riparata. Riparare una trappola meccanica richiede una prova di Artigianato (costruire trappole) con una DC pari a quella necessaria per costruirla. Il costo deimateriali grezzi è un quinto del prezzo di mercato della trappola. Per calcolare il tempo necessario a riparare una trappola si deve calcolare il tempo necessario per costruirla, ma utilizzare il costo delle materie prime invece del prezzo di mercato della trappola.

\textbf{Manuale}: Per risistemare la trappola è necessario che qualcuno rimetta le parti al loro posto. E' il meccanismo di ripristino più comune tra le trappole meccaniche.

\textbf{Automatico:} La trappola si ripristina da sé dopo essere scattata ad un intervallo di tempo prestabilito.

\textbf{Aggiramento} (Elemento Opzionale)

Se un personaggio prevede di dover passare nei pressi della trappola che ha costruito o piazzato, è buona norma costruire un meccanismo di aggiramento che permetta di disarmare temporaneamente la trappola. Gli aggiramenti, in genere, sono abbinati alle trappole meccaniche; le trappole ad Essenza, invece, consentono di specificare delle condizioni intrinseche che permettono all'incantatore diaggirarle.

\textbf{Serratura}: Una serratura di aggiramento può essere aperta con una prova di Disattivare Congegni (Criminalita') con DC 30.

\textbf{Leva Nascosta}: Una leva nascosta può essere trovata con una prova di Consapevolezza con DC 25.

\textbf{Serratura Nascosta}: Una serratura di aggiramento nascosta combina le peculiarità dei precedenti elementi: può essere trovata con una prova diConsapevolezza con DC 25 e aperta con una prova di Disattivare Congegni (Criminalita') con DC 30.

\textbf{Effetto}
Gli effetti di una trappola sono ciò che accade a chi la fa scattare. In genere, la trappola infligge danni o sprigiona gli effetti di una Essenza, ma alcune trappole hanno effetti speciali. Una trappola, di norma, effettua un Tiro per Colpire o dà diritto ad un Tiro Salvezza per essere evitata. A volte una trappola utilizza entrambe queste opzioni, altre volte nessuna (vedi Infallibile).

\textbf{Fosse}: Le fosse (coperte o scoperte) sono delle buche all'interno delle quali possono cadere i personaggi e subire danni da caduta. Una fossa non deve effettuare un Tiro per Colpire, ma superare un Tiro Salvezza su Riflessi (DC prestabilita dal costruttore) consente di non caderci dentro. Anche le altre trappole meccaniche che danno diritto ad un Tiro Salvezza rientrano in questa categoria. Le creature che cadono subiscono 1d6 danno per cadute entro 3 metri +1d6 ogni 3 metri di caduta. Le creature che subiscono danni letali da una caduta,
atterranno in posizione prona.

Una prova di Acrobatica riuscita con DC 15 permette al personaggio di dimezzare il danno se si cade da meno di 20 metri.

Cadute su superfici morbide (terreno morbido, fango ecc.) convertono i primi 1d6 danni in Danni Non Letali. Questa riduzione è cumulativa con la diminuzione del danno per l'uso della competenza Acrobatica.

Le fosse presenti nei dungeon possono essere ripartite in tre categorie diverse: scoperte, coperte e baratri. Si possono oltrepassare fosse e baratri con un uso attento di Acrobatica o attraverso vari metodi magici.

Le fosse scoperte servono principalmente a impedire agli intrusi di avanzare verso una direzione, anche se possono provocare guai seri a quegli avventurieri che avanzano al buio, e possono rendere un combattimento in mischia nelle vicinanze assai più complicato.

Le fosse coperte sono assai più pericolose. Possono essere individuate con una prova di Consapevolezza con DC 20, ma soltanto se i personaggi esaminano attentamente l'area prima di attraversarla. Un personaggio che non riesce a individuare una fossa coperta ha diritto a un Tiro Salvezza su Riflessi con DC 20 per evitare di caderci dentro. Tuttavia, se stava correndo o se camminava senzaguardare, non ha diritto ad alcun Tiro Salvezza, e cade nella fossa automaticamente.

Una trappola può essere coperta semplicemente da un cumulo di oggetti (paglia, foglie, rametti, detriti), da un tappeto, oppure da una botola vera e propria costruita per apparire come una normale parte del pavimento.Tale botola solitamente si apre quando su di essa viene esercitato un peso sufficiente a farla scattare (di solito tra i 25 e i 40 kg). I costruttori di trappole più infidi a volte costruiscono botole che si richiudono subito dopo essere state aperte, per essere pronte a scattare su una nuova vittima. La botola potrebbe richiudersi achiave una volta scattata, lasciando il personaggio intrappolato incolume, maprigioniero a tutti gli effetti. Aprire una botola simile ha una difficoltà simile a quella richiesta per aprire una porta normale (sempre che il personaggio in questione riesca a raggiungerla) ed è necessaria una prova di Potenza con DC 13 per tenere aperta una porta che si chiude a scatto.

Le fosse spesso contengono qualcosa di più pericoloso del duro pavimento sul fondo. Un costruttore di trappole potrebbe collocarvi spuntoni, mostri, pozze d'acido o di lava, o perfino dell'acqua (considerato che anche una vittima in grado di Nuotare prima o poi si stancherà e affogherà, se intrappolata a lungo). Per spuntoni e altri elementi vedi Altre Peculiarità delle Trappole.

A volte nelle fosse vivono dei mostri. Qualsiasi mostro in grado di entrare nella fossa potrebbe essere stato collocato là dentro dall'ideatore del dungeon, o potrebbe semplicemente esservi caduto per caso senza riuscire ad arrampicarsi fuori.

Una trappola secondaria, meccanica o magica, all'interno di una fossa, può rivelarsi particolarmente letale. Se attivata da una vittima caduta nella fossa, la trappola secondaria attacca il personaggio già ferito quando meno se lo aspetta.

\textbf{Trappole con Attacco a Distanza}: Queste trappole scagliano dardi, frecce, lance e armi simili contro chiunque le abbia fatte scattare. Il costruttore prestabilisce il Bonus di Attacco della trappola. Una trappola con attacco a distanza può essere preparata per simulare gli effetti di un arco composito con un alto valore di potenza; che fornisce alla trappola un bonus ai danni pari al suo punteggio di Potenza. Queste trappole infliggono il danno a seconda del tipo di munizione impiegata. Se una trappola è costruita con un alto punteggio di Potenza, avrà il corrispondente bonus ai danni.

\textbf{Trappole con Attacco in Mischia}: Queste trappole comprendono lame falcianti che spuntano dalle pareti e blocchi di pietra in caduta dal soffitto. Anche in questo caso, il costruttore prestabilisce il bonus di attacco della trappola. Queste trappole infliggono gli stessi danni delle armi da mischia "impiegate". Nel caso di un blocco di pietra in caduta, il Narratore può prestabilire un danno contundente a piacere; tuttavia, è bene ricordare che perrimettere il blocco al suo posto, qualcuno dovrà essere in grado di sollevarlo.

Si può costruire una trappola con attacco in mischia con incorporato un bonus ai tiri per i danni, come se la trappola stessa disponesse di un alto punteggio di Potenza.

\textbf{Trappole ad Essenza}: Le trappole ad Essenza producono gli effetti dell'Essenza caricata. Come tutte le Essenze, per ogni trappola ad Essenza che consente un Tiro Salvezza, la DC è pari a 5+LP caricata nella trappola.

\textbf{Trappole a Congegno Magico}: Queste trappole producono gli effetti ditutte le Essenze che sono state lanciate su di esse, secondo le rispettive descrizioni. Se l'Essenza lanciata su un congegno magico consente un Tiro Salvezza, la DC (in base al tipo di Essenza) del tiro è 20.

\textbf{Speciale}: Alcune trappole sono dotate di peculiarità che producono effetti speciali, quali l'annegamento in una fossa piena d'acqua o i danni alle caratteristiche dei veleni. A seconda dei casi, i Tiro Salvezza e i danni dipendono dal tipo di veleno o vengono prestabiliti dal costruttore.

\subsubsection{Altre Peculiarità delle Trappole}

Alcune trappole sono dotate di peculiarità opzionali che le rendono decisamente più letali. Le peculiarità più comuni sono descritte di seguito:

\textbf{Attacco di Contatto}: Questa peculiarità si applica alle trappole che colpiscono con un semplice attacco di contatto (in mischia o a distanza) riuscito.

\textbf{Bersagli Multipli}: Le trappole con questa peculiarità possono aver effetto contemporaneamente su più bersagli.

\textbf{Composto alchemico}: Le trappole meccaniche possono incorporare alcuni composti alchemici o altre sostanze e oggetti speciali, quali Borse dell'Impedimento, Fuoco dell'Alchimista, pietre del tuono, e così via. Alcuni di questi oggetti imitano gli effetti di una Essenza. Se l'oggetto riproduce l'effetto di una Essenza, il CR viene modificato come indicato nella Tabella: Modificatori al CR delle Trappole Meccaniche.

\textbf{Danni Ritardat}i: I danni ritardati sono quei danni che vengono inflitti solo dopo che è trascorso un certo lasso di tempo da quando la trappola è scattata. Una trappola infallibile infligge danni ritardati.

\textbf{Fondo della Fossa}: Se in fondo alla fossa c'è qualcosa di diverso dagli spuntoni, è più semplice trattare questa insidia come una trappola separata (vedi Trappole Multiple) con un attivatore di posizione ad impatto, come nel caso di un personaggio in caduta.

\textbf{Gas}: I Veleni ad inalazione sono il principale pericolo di una trappola a gas. Le trappole a gas, in genere, hanno le peculiarità infallibile e danni ritardati.

\textbf{Infallibile}: Quando l'intero dungeon crolla sui personaggi e li seppellisce, neanche i riflessi più rapidi possono servire a qualcosa, poiché la mira delle pareti è infallibile. Una trappola di questo tipo non ha un Bonus di Attacco né dà diritto ad un Tiro Salvezza per essere evitata, ma può infliggere danni ritardati. Anche molte trappole di liquido o gas sono infallibili.

\textbf{Liquido}: Tutte le trappole che prevedono un pericolo di annegamento ricadono in questa categoria. Le trappole che sfruttano un elemento liquido di solito sono infallibili e infliggono danni ritardati.

\textbf{Spuntoni}: Gli spuntoni sul fondo di una fossa sono considerati pugnali, ciascuno con bonus di attacco +10. Il bonus ai danni per ogni spuntone è +1 ogni per caduta da 1 metro, +2 entro 3 metri, +5 per cadute entro 9 metri, +7 per cadute entro i 12 metri , +10 per cadute oltre i 12 metri. Per cadute oltre i 3 metri considerare anche il danno da caduta.

Ogni personaggio che cade nella fossa è attaccato da 1d4 spuntoni. Questo danno va aggiunto a quello inferto dalla caduta stessa, e le statistiche presentate sopra sono solo indicative delle trappole più comuni: alcune infatti potrebbero avere degli spuntoni più pericolosi sul fondo. Gli spuntoni non vengono sommati al danno medio della trappola (vedi Danno Medio, più avanti).

\textbf{Veleno}: Le trappole che impiegano Veleno sono molto più letali delle rispettive versioni senza veleno, pertanto hanno CR superiori. Per calcolare il modificatore di CR di un Veleno, vedi la Tabella: Modificatori al CR delle Trappole Meccaniche. Soltanto i Veleni che agiscono per contatto, ferimento e inalazione possono essere impiegati per una trappola; quelli ad ingestione no. Alcune trappole infliggono solo i danni da avvelenamento. Altre infliggono anche danni con attacchi a distanza o in mischia.

\subsubsection{Progettare una Trappola}

Progettare una trappola è semplice. Iniziate col decidere che tipo di trappola volete creare.

\textbf{Trappole Meccaniche}: Selezionate gli elementi di cui si vuole dotare la trappola e aggiungete i modificatori al CR della trappola che tali elementi comportano (vedi Tabella: Modificatori al CR delle Trappole Meccaniche) per ottenere il CR finale di una trappola. Dal CR deriva la DC della prova di Artigianato (costruire trappole) per costruire la trappola (vedi più avanti).

\textbf{Trappole Magiche}: Come nel caso delle trappole meccaniche non serve altro che sapere quali elementi andranno a determinare il CR della trappola risultante. Se un personaggio vuole progettare e costruire una trappola magica, deve avere il talento Creare Oggetti Meravigliosi. Inoltre, deve essere in grado di lanciare l'Essenza o Essenze richieste dalla trappola (o, nel caso non sia in grado di farlo, di assoldare un PNG incantatore che lanci l'Essenza per lui).

\textbf{Danno Medio}: Se una trappola (meccanica o magica che sia) infligge danni in punti ferita, si calcola il danno medio di un colpo andato a segno e si arrotonda quel valore al multiplo di 10 più vicino. Se la trappola è ideata per colpire più di un bersaglio, si deve moltiplicare questo valore per 2. Se la trappola è ideata per infliggere danni nel corso di più round, si deve moltiplicare questo valore per il numero di round in cui la trappola resta attivata (o la media di essi, se il numero di round è variabile). Si usa tale valore per modificare il CR della trappola, come indicato nella Tabella: Modificatori al CR delle Trappole Meccaniche. Eventuali danni dai veleni non contano ai fini di determinare tale valore, mentre i danni inferti da spuntoni e attacchi multipli vengono calcolati.

Nel caso di una trappola magica, viene applicato solo un modificatore al CR.

\textbf{Trappole Multiple}: Se una trappola in realtà è composta da due o più trappole collegate tra loro che agiscono più o meno sulla stessa area, si determina il CR di ogni trappola separatamente.

\textbf{Trappole Multiple Dipendent}i: Se una trappola dipende dal successo di un'altra (cioè un personaggio evita direttamente la seconda rappola se riesce asfuggire alla prima), allora i personaggi guadagnano PX per entrambe le trappole superando solo la prima, anche se fanno scattare la seconda.

\textbf{Trappole Multiple Indipendenti}: Se due o più trappole agiscono indipendentemente (cioè nessuna dipende dal successo di un'altra per essere attivata), allora i personaggi guadagnano PX solo per le trappole
che superano.

\textbf{Costo delle Trappole Meccaniche}

Il costo base delle trappole meccaniche è 1.000 mo \texttimes{} il CR della trappola. Se la trappola usa Essenze per il suo attivatore o ripristino, occorre calcolare questi costi separatamente. Se la trappola non può essere ripristinata, bisogna dimezzare il costo. Se ha un ripristino automatico, si aumenta il costo della metà (+50\%). Le trappole molto semplici, come le fosse, potrebbero costare molto meno, a discrezione del Narratore. Tali trappole non dovrebbero costare più di 150 mo \texttimes{} il CR della trappola.

Dopo aver determinato il costo base per il Grado di Sfida, viene aggiunto il prezzo di eventuali composti alchemici o veleni incorporati nella trappola. Se la trappola utilizza uno di questi elementi e dispone di ripristino automatico, il costo del veleno o del composto alchemico viene moltiplicato per 20 al fine di fornire un numero adeguato di dosi.

\textbf{Trappole multiple}: Se una trappola è composta in realtà da due o più trappole collegate, va determinato il costo finale di ogni trappola separatamente, e poi vengono sommati i valori. Questo vale sia per le trappole multiple dipendenti che per quelle indipendenti.

Mediamente il costo di una trappola è di 50mo per CR

\pagebreak

\textbf{Esempi di Trappole}

Le seguenti trappole sono solo alcuni esempi delle possibilità offerte
dalle trappole per sfidare i personaggi.

\begin{multicols}{2}

	\textbf{Dardo Avvelenato}\\
	CR: 1 \\
	Tipo: meccanico \\
	DC Consapevolezza: 20 \\
	DC Disattivare Congegni: 20 \\
	Attivatore: contatto \\
	Ripristino: nessuno \\
	Effetto: Attacco a distanza 12 metri +10 (1d3 più Bava fermentata di Lucos)\\

	\textbf{Freccia}\\
	CR: 1 \\
	Tipo: meccanico \\
	DC Consapevolezza: 20 \\
	DC Disattivare Congegni: 20 \\
	Attivatore: contatto \\
	Ripristino: nessuno \\
	Effetto: Attacco a distanza 12 metri +15 (1d8+1/×3)\\

	\textbf{Fossa}\\
	CR: 1 \\
	Tipo: meccanico \\
	DC Consapevolezza: 20 \\
	DC Disattivare Congegni: 20 \\
	Attivatore: posizione \\
	Ripristino: manuale \\
	Effetto: fossa profonda 3 metri (2d6 danni da caduta) \\
	TS: Riflessi DC 20 evita \\
	Bersaglio: bersagli multipli (tutti i bersagli raggio di 3 metri)\\

	\textbf{Lama Falciant}e\\
	CR: 1 \\
	Tipo: meccanico \\
	DC Consapevolezza: 20 \\
	DC Disattivare Congegni: 20 \\
	Attivatore: posizione \\
	Ripristino: manuale \\
	Effetto: Attacco in mischia +10 (1d8+1/×3) \\
	Bersaglio: bersagli multipli (tutti i bersagli in una linea entro 3 metri)\\

	\textbf{Fossa con Spuntoni}\\
	CR: 2 \\
	Tipo: meccanico \\
	DC Consapevolezza: 20 \\
	DC Disattivare Congegni: 20 \\
	Attivatore: posizione \\
	Ripristino: manuale \\
	Effetto: fossa profonda 3 metri m (1d6 danni da caduta) + spuntoni (Attacco in mischia +10, 1d4 spuntoni per bersaglio per 1d4+2 danni ciascuno) \\
	TS: Riflessi DC 20 evita \\
	Bersaglio: bersagli multipli (tutti i bersagli in un quadrato di 3 metri di lato)\\

	\textbf{Mani Brucianti}\\
	CR: 2 \\
	Tipo: magico \\
	DC Consapevolezza: 26 \\
	DC Disattivare Congegni: 26 \\
	Attivatore: prossimità (Allarme) \\
	Ripristino: nessuno \\
	Effetto: Essenza Attacco (2d4 danni da fuoco) \\
	TS: Riflessi DC 11 dimezza \\
	Bersaglio: bersagli multipli (tutti i bersagli in un cono di 6 metri di lunghezza e 3 metri di finale)\\

	\textbf{Giavellotto}\\
	CR: 2 \\
	Tipo: meccanico \\
	DC Consapevolezza: 20 \\
	DC Disattivare Congegni: 20 \\
	Attivatore: posizione \\
	Ripristino: nessuno \\
	Effetto: Attacco a distanza 12 metri +15 (1d6+6), entro raggio 6 metri\\

	\textbf{Freccia Acida}\\
	CR: 3 \\
	Tipo: magico \\
	DC Consapevolezza: 27 \\
	DC Disattivare Congegni: 27 \\
	Attivatore: prossimità (Allarme) \\
	Ripristino: nessuno \\
	Effetto: Essenza Attacco a distanza di 16 metri (2d4 danni da acido per 4 round)\\

	\textbf{Fossa Celata}\\
	CR: 3 \\
	Tipo: meccanico \\
	DC Consapevolezza: 25 \\
	DC Disattivare Congegni: 20
	Attivatore: posizione \\
	Ripristino: manuale \\
	Effetto: fossa profonda media (3d6 danni da caduta) \\
	TS: Riflessi DC 20 evita \\
	Bersaglio: bersagli multipli (tutti i bersagli in un quadrato di 3 metri di lato)\\

	\textbf{Arco Elettrico}\\
	CR: 4 \\
	Tipo: meccanico \\
	DC Consapevolezza: 25 \\
	DC Disattivare Congegni: 20 \\
	Attivatore: contatto \\
	Ripristino: nessuno \\
	Effetto: Essenza Attacco (Arco elettrico, 4d6 danni da elettricità )\\
	TS: Riflessi DC 20 dimezza \\
	Bersaglio: bersagli multipli (tutti i bersagli in una linea a distanza 6 metri)\\

	\textbf{Falce a Parete}\\
	CR: 4 \\
	Tipo: meccanico \\
	DC Consapevolezza: 20 \\
	DC Disattivare Congegni: 20 \\
	Attivatore: posizione \\
	Ripristino: automatico \\
	Effetto: Attacco in mischia +20 (2d4+6/×4)\\

	\textbf{Blocco in Caduta}\\
	CR: 5 \\
	Tipo: meccanico \\
	DC Consapevolezza: 20 \\
	DC Disattivare Congegni: 20 \\
	Attivatore: posizione \\
	Ripristino: manuale \\
	Effetto: Attacco in mischia +15 (6d6) \\
	Bersaglio: bersagli multipli (tutti i bersagli in un quadrato di 3 metri di lato)\\

	\textbf{Aria infuocata}\\
	CR: 5 \\
	Tipo: magico \\
	DC Consapevolezza: 28 \\
	DC Disattivare Congegni: 28 \\
	Attivatore: prossimità (Allarme) \\
	Ripristino: nessuno \\
	Effetto: Essenza Attacco (6d6 danni da fuoco, distanza media)\\
	TS: Riflessi DC 14 dimezza \\
	Bersaglio: bersagli multipli (tutti i bersagli in un’esplosione di raggio 3 metri)\\

	\textbf{Colpo Infuocato}\\
	CR: 6 \\
	Tipo: magico \\
	DC Consapevolezza: 30 \\
	DC Disattivare Congegni: 30 \\
	Attivatore: prossimità (Allarme) \\
	Ripristino: nessuno \\
	Effetto: Essenza Attacco (8d6 danni da fuoco, distanza media)\\
	TS: Riflessi DC 17 dimezza \\
	Bersaglio: bersagli multipli (tutti i bersagli in un cilindro di raggio 3 metri)\\

	\textbf{Freccia Avvelenata}\\
	CR: 6 \\
	Tipo: meccanico \\
	DC Consapevolezza: 20 \\
	DC Disattivare Congegni: 20 \\
	Attivatore: posizione \\
	Ripristino: nessuno \\
	Effetto: Attacco a distanza 18 metri +15 (1d6 più Veleno ×3)\\

	\textbf{Zanne Gelide}\\
	CR: 7 \\
	Tipo: meccanico \\
	DC Consapevolezza: 25 \\
	DC Disattivare Congegni: 20 \\
	Attivatore: posizione \\
	Durata: 3 round \\
	Ripristino: nessuno \\
	Effetto: Essenza Attacco distanza 3 metri (spruzzo di acqua gelata, 3d6 danni da freddo) \\
	TS: Riflessi DC 20 dimezza \\
	Bersaglio: bersagli multipli (tutti i bersagli in una stanza di 3x3x3 metri)\\

	\textbf{Trappola a Gas}\\
	CR: 8 \\
	Tipo: meccanico \\
	DC Consapevolezza: 25 \\
	DC Disattivare Congegni: 20 \\
	Attivatore: posizione \\
	Ripristino: riparabile \\
	Effetto: Gas velenoso \\
	Bersaglio: bersagli multipli (tutti i bersagli che si trovano in una stanza 3x3x3 metri)\\

	\textbf{Raffica di Frecce}\\
	CR: 9 \\
	Tipo: meccanico \\
	DC Consapevolezza: 25 \\
	DC Disattivare Congegni: 25 \\
	Attivatore: visivo ( Occhio Arcano) \\
	Ripristino: riparabile \\
	Effetto: Attacco a distanza +20 (6d6) \\
	Bersaglio: bersagli multipli (tutti i bersagli in una linea di 6 metri)\\

	\textbf{Fossa Celata con Spuntoni}\\
	CR: 8 \\
	Tipo: meccanico \\
	DC Consapevolezza: 25 \\
	DC Disattivare Congegni: 20 \\
	Attivatore: posizione \\
	Ripristino: manuale \\
	Effetto: Fossa profonda 15 m (5d6 danni da caduta) + spuntoni (Attacco in mischia +15, 1d4 spuntoni per bersaglio per 1d6+5 danni ciascuno) \\
	TS: Riflessi DC 20 evita \\
	Bersaglio: bersagli multipli (tutti i bersagli in un cubo con lato 3x3x3 metri)\\

	\textbf{Pavimento Folgorante}\\
	CR: 9 \\
	Tipo: magico \\
	DC Consapevolezza: 26 \\
	DC Disattivare Congegni: 26 \\
	Attivatore: prossimità (Allarme) \\
	Durata: 1d6 round \\
	Ripristino: nessuno \\
	Effetto: Essenza Attacco (Attacco di contatto in mischia +9, 4d6 danni da Elettricità)
	Bersaglio: bersagli multipli (tutti i bersagli in una stanza di 6x6x3 metri)\\

	\textbf{Risucchio di Energia}\\
	CR: 10 \\
	Tipo: magico \\
	DC Consapevolezza: 34 \\
	DC Disattivare Congegni: 34 \\
	Attivatore: visivo (Visione del Vero) \\
	Ripristino: nessuno \\
	Effetto: Essenza Distruzione (Attacco di contatto a distanza 18 metri +10, 2d4 Livelli Negativi Temporanei) \\
	TS: Tempra DC 23 nega dopo 24 ore\\

	\textbf{Stanza di Lame}\\
	CR: 10 \\
	Tipo: meccanico \\
	DC Consapevolezza: 25 \\
	DC Disattivare Congegni: 20 \\
	Attivatore: posizione \\
	Durata: 1d4 round \\
	Ripristino: riparabile \\
	Effetto: Attacco in mischia +20 (3d8+3) \\
	Bersaglio: bersagli multipli (tutti i bersagli che si trovano in una stanza di 3x3x3 metri)\\

	\textbf{Cono di Schegge di Ghiaccio}\\
	CR: 11 \\
	Tipo: magico \\
	DC Consapevolezza: 30 \\
	DC Disattivare Congegni: 30 \\
	Attivatore: prossimità (Allarme) \\
	Ripristino: nessuno \\
	Effetto: Essenza Attacco (cono di lance di ghiaccio, 15d6 danni da freddo) \\
	TS: Riflessi DC 17 dimezza \\
	Bersaglio: bersagli multipli (tutti i bersagli in un cono di 18 metri di lunghezza e 6 metri finali)\\

	\textbf{Lancia Mortale}\\
	CR: 18 \\
	Tipo: meccanico \\
	DC Consapevolezza: 30 \\
	DC Disattivare Congegni: 30 \\
	Attivatore: visivo\\
	Ripristino: manuale \\
	Effetto: Attacco a distanza 36 metri +20 (1d8+6 più veleno)\\

	\textbf{Inferno di fuoco}\\
	CR: 13 \\
	Tipo: magico \\
	DC Consapevolezza: 31 \\
	DC Disattivare Congegni: 31 \\
	Attivatore: prossimità (Allarme) \\
	Ripristino: nessuno \\
	Effetto: Essenza Attacco (60 danni da fuoco) \\
	TS: Riflessi DC 14 dimezza \\
	Bersaglio: bersagli multipli (tutti i bersagli in un’esplosione di 6 metri di raggio)\\

	\textbf{Masso Schiacciante}\\
	CR: 15 \\
	Tipo: meccanico \\
	DC Consapevolezza: 30 \\
	DC Disattivare Congegni: 20 \\
	Attivatore: posizione \\
	Ripristino: manuale \\
	Effetto: Attacco in mischia +15 (16d6) \\
	Bersaglio: bersagli multipli (tutti i bersagli in un quadrato di 3 metri di lato)\\

	\textbf{Attacco Potenziato}\\
	CR: 16 \\
	Tipo: magico \\
	DC Consapevolezza: 33 \\
	DC Disattivare Congegni: 33 \\
	Attivatore: visivo (Visione del Vero) \\
	Ripristino: nessuno \\
	Effetto: Essenza Attacco (+9 contatto a distanza 18 metri, 30d6 danni, TS: Tempra DC 19 riduce a 5d6 danni)\\

	\textbf{Ferimento}\\
	CR: 14 \\
	Tipo: magico \\
	DC Consapevolezza: 31 \\
	DC Disattivare Congegni: 31 \\
	Attivatore: contatto \\
	Ripristino: nessuno \\
	Effetto: Essenza Distruzione (Sanguinamento 6, attacco di contatto in mischia +6)\\
	TS: Volontà DC 19 annulla\\

	\textbf{Galleria di Fulmini}\\
	CR: 17 \\
	Tipo: magico \\
	DC Consapevolezza: 29 \\
	DC Disattivare Congegni: 29 \\
	Attivatore: prossimità (Allarme) \\
	Durata: 1d6 round \\
	Ripristino: nessuno \\
	Effetto: Essenza Attacco (8d6 danni da Elettricità) \\
	TS: Riflessi DC 16 dimezza \\
	Bersaglio: tutti i bersagli in una stanza di 12x3x3 metri\\

	\textbf{Fossa Avvelenata}\\
	CR: 12 \\
	Tipo: meccanico \\
	DC Consapevolezza: 25 \\
	DC Disattivare Congegni: 20 \\
	Attivatore: posizione \\
	Ripristino: manuale \\
	Effetto: Fossa profonda 15 m (5d6 danni da caduta) + spuntoni (attacco in mischia +15, 1d4 spuntoni per bersaglio per 1d6+5 danni ciascuno più veleno)\\
	TS: Riflessi DC 25 evita \\
	Bersaglio: bersagli multipli (tutti i bersagli in un quadrato di 3x3 metri)\\

	\textbf{Sciame di Meteore}\\
	CR: 19 \\
	Tipo: magico \\
	DC Consapevolezza: 34 \\
	DC Disattivare Congegni: 34 \\
	Attivatore: visivo\\
	Ripristino: nessuno \\
	Effetto: Essenza Attacco (4 meteore a bersagli separati, +9 contatto a distanza 27 metri, 2d6 da impatto più 6d6 danni da fuoco)\\
	TS: Riflessi DC 23 dimezza danni da fuoco\\
	Bersaglio: bersagli multipli (quattro bersagli, due dei quali non possono trovarsi ad una distanza superiore ai 12m l’uno dall’altro)\\

	\textbf{Distruzione}\\
	CR: 20 \\
	Tipo magico \\
	DC Consapevolezza: 34 \\
	DC Disattivare Congegni: 34 \\
	Attivatore: prossimità (Allarme) \\
	Ripristino: nessuno \\
	Effetto: Essenza Distruzione (TS Morte)\\
	TS: Tempra DC 23 riduce a 5d12 danni altrimenti 10d12\\

\end{multicols}

\pagebreak


\section{Veleni e Pozioni}\index{Veleni}\index{Pozioni}

\label{veleni-e-pozioni}


\begin{tcolorbox}[enhanced,arc=5pt,boxrule=0.3pt]{Un giorno, un uomo fu colpito da una freccia avvelenata. Gli amici e i parenti, in ansia, chiamarono un medico. Quando gli si avvicinarono per prendere la freccia, l'uomo disse loro: "Prima di farlo, vorrei sapere chi mi ha trafitto con questa freccia... Era uno schiavo, un re, o un bramino? Era grande? Piccolo? Di che colore era la sua pelle? Dove viveva? E la freccia com'è stata costruita? Quale veleno è stato impiegato? ..." Mentre si stava ponendo tutte queste domande... il veleno fece il suo effetto e l'uomo ferito finì per morire. (Budda)}\end{tcolorbox}\medskip


Dal morso di una vipera alla lama avvelenata di un assassino, il veleno è una costante minaccia. I veleni possono essere curati con Tiro Salvezza su Tempra ed Essenze di Cura

\bigskip

\textbf{Tipo di Veleno e Pozione}

\textbf{Contatto}: sono contratti nel momento in cui qualcuno tocca il veleno con la pelle nuda. Tali veleni possono essere usati come veleni da ferimento. I veleni a contatto hanno solitamente un tempo di insorgenza di 1 round. Un veleno a contatto può essere un unguento, balsamo, liquido di qualsiasi densità o anche polvere se specifica per contatto e non inalazione.

\textbf{Ingestione}: si attivano quando una creatura li mangia o li beve. I veleni ad ingestione hanno solitamente un tempo di insorgenza di 10 minuti.

\textbf{Ferimento}: vengono trasferiti soprattutto con gli attacchi di alcune creature e tramite armi cosparse di veleno. I veleni a ferimento non hanno solitamente un tempo di insorgenza.

\textbf{Inalazione}: si attivano nel momento in cui una creatura entra in un'area che contiene tali veleni. Molti veleni ad inalazione riempiono un volume pari ad un cubo con spigolo di 3x3x3 metri per dose. Le creature possono tentare di trattenere il fiato mentre si trovano all'interno dell'area per evitare di inalare la tossina. Le creature che trattengono il fiato hanno devono fare una prova su Potenza (3d6+bonus Potenza) a difficoltà 10 ogni round per non inalare il gas. Ogni round in cui si trattiene il fiato la prova di difficoltà aumenta di 1.

Vedi anche le regole per trattenere il fiato e soffocare in Ambiente.

\textbf{Insorgenza ed Effetto}

Per insorgenza si intende quanto tempo ci mette il veleno o pozione a fare effetto. Se il tempo di insorgenza è 1 turno significa che per gli effetti del veleno/pozione il Tiro Salvezza lo si effettua dopo 10 minuti. Se nella tabella del veleno/pozione insorgenza non è specificata significa che l'effetto è immediato dopo l'entrata in contatto con il veleno.

L'effetto di un veleno/pozione è immediato dopo l'insorgenza. Verificare la descrizione del veleno per capirne l'effetto. Se il Tiro Salvezza riesce il veleno non ha fatto effetto e si può ritenere neutralizzato.

\bigskip

\textbf{Avvelenati}\index{Avvelenati}

\textbf{Prima dose}: Quando si viene esposti a un veleno per la prima volta (durante la propria azione o quella di qualcun altro), è necessario effettuare un Tiro Salvezza per evitare di venire avvelenati.

\textbf{Successo}: Si resiste al veleno. Non si subiscono effetti negativi e non sono necessari ulteriori Tiri Salvezza.

\textbf{Fallimento}: Siete stati avvelenati e si subisce subito l'effetto elencato.

\textbf{Più dosi}: Se si vieni esposti a più dosi dello stesso veleno nello stesso round la difficoltà del TS aumenta di 1 per dose aggiuntiva.

\textbf{In tempi diversi}: se si viene esposti al veleno in tempi diversi, ogni volta ci sarà un nuovo Tiro Salvezza e si subiranno gli eventuali effetti nei tempi previsti.

\bigskip

\textbf{Applicare il Veleno}\index{Applicare il Veleno}

Applicare il veleno ad un'arma o ad una munizione richiede 3 azioni.

Ogni volta che un personaggio applica o prepara un veleno per l'uso deve tirare 3d6+Intelletto e se ottiene come somma 3 o 4 è entrato in contatto con il veleno e deve effettuare un Tiro Salvezza contro il veleno come di norma. Ciò non consumala dose di veleno.

Ogni volta che un personaggio attacca con un'arma avvelenata, se ottiene un 3 o 4 naturale col Tiro per Colpire, si espone agli effetti del veleno. Ciò consuma il veleno sull'arma.

Un pozione di veleno è sufficiente per coprire di veleno un arma media oppure 3 frecce. Il veleno viene così consumato e rimane attivo sull'arma finché questa non colpisce.

Una creatura che sotto gli effetti di un veleno, che si siano gia' scatenati o meno, ha la condizione di Avvelenata.

\medskip

\textbf{Creare Veleni Naturali}\index{Creare Veleni Naturali}

I veleni possono essere realizzati usando Artigianato (alchimia) o Conoscenze (Erboristeria). La DC per preparare un veleno è uguale alla DC del Tiro Salvezza su Tempra che richiede -5. Il costo per preparare un veleno è pari alla metà del costo di vendita.

Ottenendo un 3 o 4 naturale con la prova di Artigianato o Erboristeria ci si espone al veleno durante la sua preparazione. Il tempo necessario per preparare i veleni è pari alla DC in ore.

Gli esempi seguenti rappresentano solo alcuni dei possibili veleni.

\subsubsection{Veleni Naturali}

\textbf{Tabella Veleni}

\medskip

\begin{tabularx}{0.95\textwidth}{XlllXl}
\toprule
	\textbf{Nome Veleno}  & \textbf{Uso} & \textbf{Tempra} & \textbf{Ins.} & \textbf{Effetto (danno)} & \textbf{Costo (mo)}\\
	Mistura Rossa \index{Mistura Rossa}   & F  & 13    & -  & -1d6 TC/TS per 10 minuti   & 10\\
	Nocciolo di Dennar \index{Nocciolo di Dennar}  & I  & 13    & 1 turno  & -1d2 Potenza, per 3gg    & 15\\
	Succo di Ythis\index{Succo di Ythis} & I  & 14    & 1 turno  & -1d2 Intelletto, per 1g  & 20\\
	Sangue di Thrun \index{Sangue di Thrun}   & C  & 26    & -   & -1d3 Potenza   & 80\\
	Erba puntuta rosa \index{Erba puntuta rosa}    & I  & 22    & 1 turno  & -1d6 Agilità, per 1 ora  & 60\\
	Dita di Daraka\index{Dita di Daraka} & F  & 17    & -   & -1d6 Potenza, per 1 ora   & 35\\
	Polline di Rosa di Omro\index{Polline di Rosa di Omro}   & I  & 15    & -   & -1d3 Potenza e Agilità, per 1 ora   & 25\\
	Fumi di Curna\index{Fumi di Curna}   & R  & 18    & -   & -1d3 Volontà   & 40\\
	Olio di Nabar \index{Olio di Nabar}  & R-F& 20    & -   & Confuso per 2d6 round    & 50\\
	Bacche Azzurre di fosso \index{Bacche Azzurre di fosso}  & I  & 21    & 1 turno  & -1d3 Intelletto e Volontà per 6 ore& 55\\
	Pelle di Rospo Azzurro \index{Pelle di Rospo Azzurro}    & C  & 22    & 1 minuto & Paralizzato per 1d6 turni& 60\\
	Cenere di Corteccia Gialla \index{Cenere di Corteccia Gialla} & F  & 15    & 6 round  & Privo di sensi per 1d3 ore    & 25\\
	Fiocco bianco di Mucot \index{Fiocco bianco di Mucot}    & C  & 20    & -   & Dorme per 2d12 ore  & 20\\
	Bava fermentata di Lucos \index{Bava fermentata di Lucos}& F  & 15    & -   & 1d8 PF    & 25\\
	Bacca Viola di Barsar\index{Bacca Viola di Barsar}  & I  & 18    & 1 turno  & Incapace di eseguire azioni violente per 3d8 ore  & 40   \\
	Lingua di Kreex \index{Lingua di Kreex}   & F  & 20    & -   & La ferita sanguina. +1 danno da sanguinamento per round per 2 minuti Max +5 sanguinamento & 50   \\
	Fegato di Toporagno Viola \index{Fegato di Toporagno Viola}   & I  & 25    & 1 ora    & 2d6 di danno a Volontà e Intelletto. Permanente   & 75   \\
	Muschio Giallo \index{Muschio Giallo}& I  & 20    & 1 round  & la creatura guadagna una taglia. Sovradosaggi sono possibili. Durata 10 minuti  & 50\\
	Veleno di Serpe del Sangue \index{Veleno di Serpe del Sangue} & F  & 25    & -   & Paralisi per 1d6 ore -1d4 punti Potenza per 7 giorni   & 75   \\
	Profumo di Ragmor \index{Profumo di Ragmor}    & R  & 16    & -   & -1d3 Magnetismo, per 1 giorno & 30\\
	Grasso di Toporagno Viola \index{Grasso di Toporagno Viola}   & C  & 13    & 1 round  & 2d12 PF & 15\\
	Veleno di Ottalm\index{Veleno di Ottalm}  & F  & 20    & -   & Morte o -1d2 Potenza permanente  & 50\\
\end{tabularx}

Applicazione: \textbf{I}(ngestione), \textbf{F}(erimento),  \textbf{C}(ontatto), \textbf{R}(espirazione)
\bigskip

I punti caratteristica persi si recuperano al ritmo di 1 al giorno se non indicato diversamente.

\subsection{Pozioni naturali}\index{Pozioni}

Il tempo per preparare queste pozioni/droghe è pari alla DC/2 in ore, mentre la difficoltà è pari alla DC -5. Se gli ingredienti si comprano il costo per preparare la pozione è metà del costo di vendita indicato, se si cercano in natura il costo per produzione scende ad un quarto.

Se la prova di DC (Cultura, Erboristeria) ha successo se ne preparano 1d2 pozioni (da 1 dose).

Non si puo' beneficiare di piu' di una dose di pozioni naturali (per tipo) al giorno, a differenza di quelle magiche.

\bigskip

\begin{tabularx}{0.95\textwidth}{llllXll}

	\textbf{Nome}  & \textbf{Uso} & \textbf{Ins.} & \textbf{DC} & \textbf{Effetto}& \textbf{Loc.} & \textbf{Costo} \\
	Arlandas\index{Arlandas} & R  & 1 ora& 24& Rinsalda le fratture & CF5 & 100  \\
	Burthelas \index{Burthelas}   & I  & 1 turno   & 32& Rigenera le mani& HD7 & 410  \\
	Musekiss\index{Musekiss} & C  & 1 ora& 30& Rigenera arti inferiori   & TH9 & 550  \\
	Bacche di Ljust \index{Bacche di Ljust} & I  & 1 round   & 16& Preso la sera recuperi il doppio dei PF (minimo 4) & AZ6 & 30   \\
	Culcoa\index{Culcoa}& C  & 1 round   & 16& Recuperi 2d6 da danno da fuoco & TS7 & 30   \\
	Jojopo\index{Jojopo}& C  & 1 round   & 15& Recuperi 2d6 da danno da ghiaccio   & FM6 & 25   \\
	Kelventare\index{Kelventare}  & I  & 1d4 round & 28& Recuperi 2d6    & TT7 & 90   \\
	Harfy \index{Harfy} & C  & -    & 12& Interrompe il sanguinamento    & SS6 & 10   \\
	Arlan\index{Arlan}  & C  & -    & 15& Cura 1d6+3 PF   & TT5 & 25   \\
	Darsurion\index{Darsurion}    & C  & 1 round   & 25& Cura 1d4 PF& CM4 & 75   \\
	Draaf \index{Draaf} & C  & 1 round   & 20& Cura 1d8 PF& SO6 & 50   \\
	Garioe\index{Garioe}& I  & 1 round   & 25& Cura 2d6 PF& AZ7 & 75   \\
	Geffnull \index{Geffnull}& I  & 5 round   & 28& Cura 3d8+3 PF   & EV8 & 90   \\
	Mirenna\index{Mirenna}   & I  & 1 round   & 20& Cura 5 PF  & CM6 & 50   \\
	Rewky\index{Rewky}  & I  & -    & 25& Cura 2d8 PF& TD6 & 75   \\
	Wickalim\index{Wickalim} & I  & -    & 15& Cura 2 PF  & TD4 & 25   \\
	Lingua Rossa di Xabax\index{Lingua Rossa di Xabax}& C  & 1 turno   & 20& Cura 2d6 PF ma se c'è malattia o veleno la rimuove ma causando 2d6 di danno & TA7 & 50   \\
	Yaveth\index{Yaveth}& I  & -    & 20& Cura 2d8 PF& MO5 & 50   \\
	Bacio di Ljust\index{Bacio di Ljust}    & C  & 1 round   & 35& Cura 100 PF& HO8 & 125  \\
	Polline di Rosa Verde\index{Polline di Rosa Verde}& R  & 3 turni   & 25& Recuperi 2d4 danni Intelletto e Volontà  & FA8 & 75   \\
	Arkasun\index{Arkasun}   & C  & -    & 25& Cura 1d6 PF a turno per 3 turni& MT7 & 75   \\
Attarna\index{Attarna}   & I  & 1 turno   & 20& Concede un nuovo TS per Malattie con un +4    & TF7 & 50   \\
Delrean\index{Delrean}   & C  & 1 round   & 15& Allontana insetti per 1 giorno & CC6 & 2    \\
Delrean Plus\index{Delrean Plus}   & I  & 1 round   & 18& Allontana insetti per 3 giorni & CC6 & 5    \\
Melandrir\index{Melandrir}    & I  & 1 round   & 15& Concede un nuovo TS per Malattie con +5  & CF7 & 25   \\
Uovo di Urk\index{Uovo di Urk}& I  & 1 turno   & 12& 1 giorno di cibo& FH7 & 2    \\
Barannie\index{Barannie} & I  & -    & 15& Rimuove nausea   & MD6 & 10   \\
Eldrin'tail\index{Eldrin'tail}& I  & -    & 15& Concede un nuovo TS su Veleni  & FH7 & 25   \\
Harlindar\index{Harlindar}    & I  & 1 turno   & 15& Fa abortire& SS7 & 25   \\
\end{tabularx}

\bigskip

\begin{tabularx}{0.95\textwidth}{XlllXll}
\textbf{Nome}  & \textbf{Uso} & \textbf{Ins.} & \textbf{DC} & \textbf{Effetto}& \textbf{Loc.} & \textbf{Costo} \\
Klandor\index{Klandor}   & I  & -    & 15& Rimuove paralisi& HB6 & 25   \\
Klynkyx\index{Klynkyx}   & C  & 6 turno   & 15& Fa cadere tutti i capelli per 1d6+4 gg   & MO6 & 8    \\
Arduuar\index{Arduuar}   & I  & 1 round   & 25& Rimuove Veleni  & SZ7 & 75   \\
Nazamuse \index{Nazamuse}& I  & -    & 30& Rimuove Veleni e Malattie & EW9 & 100  \\
Nelthalion \index{Nelthalion} & I  & -    & 15& Fa vomitare& SR3 & 1    \\
Uscaboo \index{Uscaboo}  & R  & 1 turno   & 25& Rimuove cecità  & MO7 & 75   \\
Ucsaboo \index{Ucsaboo}  & C  & 1 turno   & 30& Rigenera occhi  & MO8 & 200  \\
Febfendi \index{Febfendi}& C  & 1 turno   & 25& Rigenera orecchie    & CF7 & 75   \\
Siranmuse\index{Siranmuse}    & I  & 1 giorno  & 30& Rigenera organi interni   & SS8 & 350  \\
Klagul\index{Klagul}& C  & 1 turno   & 20& Pulisce i denti & SS4 & 30   \\
Corteccia di Aklent\index{Corteccia di Aklent}    & I  & 1 turno   & 10& La corteccia masticata per almeno 10 round concede per le 24 ore successive un +1 TS vs Veleno  & MT6 & 5    \\
Petali di Lisbeth \index{Petali di Lisbeth}  & I  & 1 turno   & 15& Cura tosse e raffreddore  & MC6 & 20   \\
Estratto di radice Gisenosa\index{Estratto di radice Gisenosa}   & I  & -    & 15& +2 Intelletto, -2 Agilità per 10 minuti  & MT6 & 5    \\
Gylvert\index{Gylvert}   & I  & -    & 25& Concede respirare sott'acqua per 4 ore   & MO7 & 75 \\
Gusterbloon \index{Gusterbloon}    & C  & 1 round   & 20& La pelle diventa piu' scura concedendo un +4 alla prove di Nascondersi & CM5 & 40  \\
Lievito di Muschio Bianco \index{Lievito di Muschio Bianco} & I  & -    & 12& I prodotti da forno che usano questo lievito causano meteorismo incontrollabile ed incredibilmente puzzolente per 12 ore & CA3 & 1    \\
Estratto di Bacca Illa bruciata\index{Estratto di Bacca Illa bruciata}& I  & - & 15& +2 Iniziativa, +2 Agilità , -4 TS Arbitrio, per 10 minuti    & MS6 & 5    \\
Corteccia polverizzata di Dagmather\index{Polvere di corteccia di Dagmather}    & R  & 1 round   & 25& Rimuove condizione esausto e affaticato  & SS5 & 2    \\
Radice secca di Kathaus\index{Radice secca di Kathaus} & R  & -    & 20& +2 Potenza e Agilità per 1 ora & FW6 & 25   \\
\end{tabularx}

\bigskip

\subsection{Droghe}\index{Droga}

\begin{tabularx}{0.95\textwidth}{XlllXll}
\textbf{Nome}  & \textbf{Uso} & \textbf{Ins.} & \textbf{DC} & \textbf{Effetto}& \textbf{Loc.} & \textbf{Costo} \\
Foglie fermentate di Luside*\index{Foglie fermentate di Luside}  & I  & 1 turno  & 17& Allucinazioni sensoriali per 2d4 ore. +2 Magnetismo ed Intelletto & SF7 & 5    \\
Ferpillon{*} \index{Ferpillon}& I  & 1 round   & 30& Fa dormire per 24 ore& SC5 & 50   \\
Unto Grigio{*} \index{Unto Grigio} & I  & 1 round   & 24& Rimuove condizionamenti mentali fino a LP 21  & AH9 & 80   \\
Cenere di Arpasur{*} \index{Cenere di Arpasur}    & R  & 1 round   & 20& Rimuove condizione di affaticato    & FT6 & 5    \\
Carne secca di Ragno Viola* \index{Carne secca di Ragno Viola}   & I  & 1 round   & 24& +4 Potenza -4 Intelletto per 1 turno& SH7 & 30   \\
Estratto alcolico di Melzaa*\index{Estratto alcolico di Melzaa}  & I  & -    & 20& +1d4 Potenza , +1d4 Agilità . -4 TS su Arbitrio. Per 3 ore   & AF6 & 25   \\
Essenza profumata di Inut*\index{Essenza profumata do Inut} & R  & -    & 15& +2 Intelletto, per 1d8 ore& HB6 & 15   \\
Polline di Julnnaus\index{Polline di Julnnaus}{*} & R  & -    & 20& +3 Potenza per 2 ore & FO6 & 25   \\
Miele polverizzato del fiore di Erain* \index{Miele polverizzato del fiore di Erain} & R  & -    & 20& +2 Potenza e Intelletto e Agilità. +3d6 PF temporanei, per 1 ora  & FT7 & 75   \\
\end{tabularx}

\medskip

Le droghe danno dipendenza. Terminato l'effetto effettuare un Tiro Salvezza su Arbitrio a difficoltà 15 o prenderne un altra dose, il successivo Tiro Salvezza avrà difficoltà +1 e così via.

Ogni qual volta si prende una nuova dose entro 2 settimane dalla prima il Tiro Salvezza per non diventare dipendenti aumenta di 1.

\bigskip

\textbf{Tabella decodifica codici localita'}
\smallskip

Es: Gusterbloon FT5

La prima Lettera indica il CLIMA, la Seconda indica l'AMBIENTE, la
Terza indica la RARITA'

La rarità indica la possibilità, su un d10, di trovare l'erba/pianta
ricercata. Tirare 1d10 e fare meno del numero indicato, chiaramente
se c'è corrispondenza di clima ed ambiente.
\bigskip

\begin{tabularx}{0.95\textwidth}{XXXX}
	\toprule
	\textbf{Prima Lettera} & \textbf{Clima}   & \textbf{Seconda Lettera} & \textbf{Ambiente}\\
	A       & Arido 			 & A         & Alpino\\
	C       & Freddo			 & B         & Gole\\
	E       & Ghiacci perenni	 & C         & Foresta di Conifere\\
	F       & Freddo severo   	 & D         & Foresta Decidua\\
	H       & Umido e caldo  	 & F         & Argini fiumi e torrenti\\
	M       & Temperato          & G         & Campi ghiacciati\\
	S       & Semi arido         & H         & Campi secchi\\
	T       & Temperato fresco   & J         & Giungla, Foreste piovose\\
	X       & Sconosciuto        & M         & Montagna\\
	        &                    & N         & Oceano, distese salate\\
	        &                    & S         & Erba bassa\\
	        &                    & T         & Erba alta\\
	        &                    & U         & Caverne e underground\\
	        &                    & V         & Vulcanica\\
 	        &                    & W         & Discariche/Rifiui\\
	        &                    & Z         & Deserto\\
 	        &                    & X         & Sconosciuto\\
\end{tabularx}

\subsection{Pozioni generiche}\index{pozioni generiche}\index{pozioni}

Il Narratore e' libero di usare tutte le pozioni e veleni indicate sopra oppure usare delle pozioni generiche pronte all'uso, comprabili in quasi tutti i negozi di erboristeria o di pozioni.

\medskip

Nella tabella i costi ed effetti di queste pozioni, l'insorgenza e' sempre immediata, la durata per le cure e' immediata, per le altre e' 1 ora.
\bigskip

\begin{tabularx}{0.95\textwidth}{lXll}
	\textbf{Nome Pozione}&  \textbf{Effetto}&  \textbf{Costo (mo)}& \textbf{Applicazione}\\ 
	\toprule
	Cura					& recuperi 1d8+1 PF 							& 50 & Ingestione\\ 
	Cura potenziata			& recuperi 3d8+3 PF 							& 250  & Ingestione\\ 
	Indebolente				& -1d6 TC. TS DC 15 Tempra 						& 25 & Ingestione\\ 
	Indebolente potenziata	& -1d6 TC. TS DC 18 Tempra						& 50 & Ferimento \\ 
	Veleno					& subisci 2d4+2 di danno. TS DC 15 Tempra 		& 25 & Ingestione \\ 
	Veleno potenziata		& subisci 2d4+2 di danno. TS DC 18 Tempra 		& 50 & Ferimento \\ 
	Rimuovi Veleno			& annulla l'insorgenza di un veleno se presa entro l'attivazione, oppure concede un nuovo TS con +4 & 75 & Ingestione\\
\end{tabularx} 



\pagebreak

\section{Movimento e Trasporto}\index{Trasporto}\index{Movimento}

\label{movimento-e-trasporto}

\begin{tcolorbox}[enhanced,arc=5pt,boxrule=0.3pt]{
	- E ti puoi trovare un'altra moglie!\\
	- Ah, questo sì. ma il guaio è che mi ha portato via il fucile e il cavallo! Peccato, era così bella, io mi ci ero affezionato. Le davo qualche frustata, ma lei non ci faceva caso.\\
	- Chi, tua moglie?\\
	- No, la mia cavalla. A trovare un'altra moglie si fa presto, ma una cavalla come quella non la ritrovo più. (Ombre rosse, film 1939)}\end{tcolorbox}\medskip


Molte delle regole qui riportate sono opzionali, il Narratore può tenere conto solo di ciò che ritiene più opportuno.

Vi sono tre scale di movimento nel gioco:
\begin{itemize}
	\item Tattico, per il combattimento, si usano le distanze Mischia e i quadretti
	      di 1 metro di lato
	\item Locale, per esplorare una zona, misurato in metri al minuto.
	\item Via Terra, per muoversi da un posto all'altro, misurato in km all'ora o al giorno.
\end{itemize}
\textbf{Tipi di Movimento}

Quando si muovono nelle differenti scale di movimento, le creature generalmente camminano o vanno veloci o corrono.

\textbf{Camminare}:\index{Camminare} Camminare rappresenta un movimento non affrettato ma deciso di circa 4 km all'ora per un umano senza Ingombro.

\textbf{Correre}\index{Correre}: Significa muoversi di circa 13 km all'ora per un
umano in armatura completa.

Chi attacca il personaggio che va veloce ha un bonus di 1d6 al Tiro per colpire. Il personaggio che va veloce ha un malus di 2d6 nel Tiro per Colpire nel round in cui si corre.

Correre come azione di movimento raddoppia la velocità di movimento e non la triplica. Solo in situazioni di non combattimento la corsa triplica il movimento.

\subsection{Tabella: Movimento e Distanza e Velocità : a Piedi}\index{a Piedi}

\medskip

\begin{tabular}{lccc}
	\toprule
	\multirow{2}*{Tipo di movimento} &
	\multicolumn{3}{c}{Movimento Velocità (metri)}        \\
	\cmidrule(lr){2-4}     & 6m
	                    & 9m
	                                 & 12m                \\
	\midrule
	Camminare                        & 6m   & 9m   & 12m  \\
	Correre (x3)                     & 18m  & 18m  & 24m  \\
	\textbf{Un minuto (locale)}                           \\
	Camminare                        & 60m  & 90m  & 120m \\
	Correre (x3)                     & 180m & 270m & 360m \\
	\textbf{Un’ora (via terra)}                           \\
	Camminare                        & 3km  & 4km  & 6km  \\
	Correre (x3)                     & 9km  & 12km & 18km \\
	\textbf{Un giorno (via terra})                        \\
	Camminare                        & 24km & 32km & 54km \\
\end{tabular}

\subsection{Movimento Tattico}\index{Movimento Tattico}

Durante un combattimento si utilizza la velocità tattica e la distanza.

\textbf{Movimento Ostacolato}

Terreno difficile, ostacoli o scarsa visibilità possono impedire i movimenti. Quando il movimento è ostacolato si va a metà della velocità. Quindi sono necessari 2 Azioni per coprire la propria distanza di 9 metri (se si è umano senza ingombro..). Oppure con una Azione di movimento si copre solo 4 metri.

Se esiste più di una condizione particolare, aggiungere tra loro tutti i costi aggiuntivi applicabili.

In alcune situazioni il movimento è talmente ostacolato che la distanza percorribile per Azione è minima.. in tal caso si possono utilizzare tutte e 3 le Azioni per muoversi di una Azione di movimento (9/6 metri) in qualsiasi direzione.

Non applicare questa regola per attraversare terreni impraticabili o per muoversi quando non è possibile farlo in alcun modo.

Non si può Correre o Caricare agevolmente (Atletica DC 20) attraverso un percorso che ostacola il movimento.

\subsubsection{Movimento Locale}\index{Movimento Locale}

I personaggi che esplorano una zona usano il movimento locale, misurato in metri al minuto.
\begin{itemize}
	\item
	      Camminare: Un personaggio può camminare senza problemi in scala locale.
	\item
	      Andare Veloci: Un personaggio può andare veloce senza problemi in scala locale. Vedi Movimento via terra, sotto, per il movimento in km all'ora.
	\item
	      Correre: Un personaggio può Correre per un numero di round pari al triplo del proprio punteggio di Potenza su scala locale senza bisogno di riposarsi (minimo un round). Vedi la relativa sezione in Combattimento per le regole riguardanti la Corsa per Periodi Più Prolungati.
\end{itemize}

\subsubsection{Movimento Via Terra}\index{Movimento Via Terra}

I personaggi che percorrono lunghe distanze usano il movimento via terra. Il movimento via terra è misurato in ore o giorni. Un giorno rappresenta 8 ore di tempo di viaggio reale. Per imbarcazioni a remi, un giorno significa remare per 10 ore. Per navi a vela, rappresenta 24 ore.

\textbf{Camminare}\index{Camminare}

Si può camminare per 8 ore in un giorno di viaggio senza problemi.

Camminare più a lungo può sfinire (vedi Marcia forzata, sotto).

\textbf{Andare Veloci}\index{Andare Veloci}

Si può andare veloci per 1 ora senza problemi. Andare veloci per una seconda ora compresa tra due cicli di sonno provoca 1 Danno Non Letale, e ogni ora aggiuntiva provoca il doppio dei danni subiti nell'ora precedente. Un personaggio che subisce Danni Non Letali da andatura veloce è considerato Affaticato.

Un personaggio Affaticato non può Correre o Caricare e subisce penalità -1 a Potenza e Agilita

\textbf{Correre}\index{Correre}

Non è possibile Correre per un lungo periodo di tempo. Tentativi di Correre e riposarsi a cicli funzionano come andare veloci.

\textbf{Terreno}\index{Terreno}

Il terreno su cui si viaggia influenza quale distanza viene percorsa in un'ora o in un giorno (vedi Tabella: Terreno e Movimento Via Terra). Una strada maestra è una strada principale, dritta e lastricata. Una strada comune è solitamente un cammino impervio. Un sentiero è come una strada comune tranne per il fatto che permette di viaggiare solo in fila indiana e non avvantaggia un gruppo che viaggia con veicoli. Un terreno libero è una zona selvaggia senza sentieri segnati.

\bigskip

\textbf{Tabella: Terreno e Movimento Via Terra (Opzionale)}

Nella tabella sono indicati i moltiplicatori per la distanza percorsa.

\medskip

\begin{tabular}{llll}
	\toprule
	\textbf{Terreno}  & \textbf{Strada maestra} & \textbf{Strada comune} & \textbf{Sentiero non battuto}\\
	Brughiera         & x1                      & x1                     & x3/4\\
	Collina           & x1                      & x3/4                   & x1/2\\
	Deserto Sabbioso  & x1                      & x1/2                   & x1/2\\
	Foresta           & x1                      & x3/4                   & x1/2\\
	Giungla           & x1                      & x3/4                   & x1/4\\
	Montagna          & x3/4                    & x3/4                   & x1/2\\
	Palude            & x1                      & \texttimes 3/4         & \texttimes 1/2\\
	Pianura           & x1                      & \texttimes 3/4         & \texttimes 1/2\\
	Tundra Ghiacciata & x1                      & \texttimes 3/4         & \texttimes 3/4\\
\end{tabular}

\bigskip

\textbf{Marcia Forzata}\index{Marcia Forzata}

In un giorno di cammino normale, si può camminare per 8 ore. Il resto del giorno viene sfruttato per fare e disfare il campo, riposarsi e mangiare.

\textbf{Movimento in sella}\index{Movimento in sella}

Una cavalcatura che porta un cavaliere può muoversi con andatura veloce. Tuttavia, i danni che subisce sono danni normali invece che non letali. Può anche essere costretta a una marcia forzata, ma le sue prove di Potenza falliscono automaticamente e di nuovo i danni che subisce sono danni normali. Anche le cavalcature sono considerate Affaticate quando subiscono danni da andatura veloce o marcia forzata.

\subsection{Tabella: Cavalcature e Veicoli}\index{Cavalcature}\index{Veicoli}

\medskip

\label{tabella-cavalcature-e-veicoli}\index{Cane}\index{Pony}\index{Carretto}\index{Zattera}\index{Barca}\index{Nave}

\begin{tabular}{lll}
	\toprule
	\textbf{Cavalcatura o Veicolo (carico trasportato)} & \textbf{All'ora} & \textbf{Al giorno}\\
	Cane da Galoppo     & 6km              & 48km              \\
	Cane da Galoppo (50.5-150 kg)*      & 4.5km            & 36km              \\
	Cavallo Leggero     & 7.5km            & 60km              \\
	Cavallo Leggero (115,5-345 kg)*     & 5.25km           & 42km              \\
	Cavallo Pesante     & 7.5km            & 60km              \\
	Cavallo Pesante (150.5-450 kg)*     & 5.52km           & 42km              \\
	Pony& 6km              & 48km              \\
	Pony (75,5-225 kg)* & 4.5km            & 36km              \\
	Carretto o Carro    & 3km              & 24km              \\
	\textbf{Imbarcazione}               &  &   \\
	Zattera o Chiatta (pertica o rimorchio)             & 0.75km           & 7.5km             \\
	Barcone (a Remi)**  & 1.5km            & 15km              \\
	Barca a Remi**      & 2.25km           & 22.5km            \\
	Nave a Vela (vele)  & 3km              & 72km              \\
	Nave da Guerra (vele e remi)        & 3.75km           & 90km              \\
	Nave Lunga (vele e remi)            & 4.5km            & 108km             \\
	Galea (remi e vele) & 6km              & 144km             \\
\end{tabular}

*I quadrupedi, come i cavalli, possono portare carichi superiori rispetto ai personaggi. Vedi Capacità di Trasporto per maggiori informazioni.

**Zattere, chiatte e barconi sono usati su laghi e fiumi. Se seguono la corrente, sommare la velocità della corrente (di solito 4,5 km/h) alla velocità dell'imbarcazione. Oltre a essere spinta con i remi per 10 ore, l'imbarcazione può anche essere trasportata dalla corrente per altre 14 ore, se qualcuno è in grado di guidarla, e quindi si aggiungono altri 63 km alla distanza giornaliera percorsa. Queste imbarcazioni non possono essere spinte a remi contro una corrente molto forte, ma possono essere tirate controcorrente da animali da soma sulla riva.

\subsection{Fuga e Inseguimento}\index{Fuga}\index{Inseguimento}

Nel movimento round per round è impossibile per un personaggio lento sfuggire ad un personaggio veloce senza qualche tipo di aiuto. Allo stesso modo, non è un problema per un personaggio veloce sfuggire ad uno più lento.

Quando la velocità dei due personaggi coinvolti è uguale, c'è un metodo abbastanza semplice per risolvere un inseguimento: se una creatura sta inseguendo un'altra ed entrambe si muovono alla stessa velocità, e l'inseguimento prosegue almeno per alcuni round, occorre effettuare prove contrapposte di Agilità per vedere chi si muove più in fretta in questi round.

Se la creatura inseguita vince si allontana di 9 metri se questa distanza diventa superiore ai 100 metri ha seminato l'inseguitore. Se è l'inseguitore a vincere, accorcia di 9 metri la distanza e quando la distanza è mischia ha catturato il fuggitivo.

A volte un inseguimento che si svolge via terra potrebbe durare per lungo tempo (10 prove di Agilità non risolutive) , con entrambe le parti che riescono solo a scorgersi a distanza.

Nel caso di un lungo inseguimento, una prova contrapposta di Resistenza determina quale delle due parti può mantenere più a lungo il ritmo. Se la creatura inseguita ottiene il risultato più alto, riesce a fuggire, altrimenti è l'inseguitore che riesce a raggiungerla.

\subsection{Capacità di Carico e Trasporto: Ingombro}\index{Capacità di Carico}\index{Ingombro}

\label{sec:capacita-di-carico-e-trasporto-ingombro}

\subsubsection{Ingombro}\index{Ingombro}

Portare tesori, pezzi di drago, armature complete per non parlare di armi sproporzionate rendono difficile il movimento.

Ogni oggetto ha un valore di Ingombro ovvero quando è pesante e scomodo portarlo.

Ci possono essere oggetti leggeri ma estremamente ingombranti (tronchi cavi, tappeti di seta) oppure piccoli ma pesantissimi (sfere di mercurio, vestiti intessuti d'oro), per tutti questi oggetti il valore di Ingombro sarà significativo.

Viceversa dei gessetti, biglie, stracci saranno sia leggeri che poco pesanti.

I valori di Ingombro degli oggetti si sommano tra di loro per dare l'Ingombro totale portato.

\textbf{Il valore totale di Ingombro che si può portare senza penalità è pari al valore della Potenza+2}.
Se hai un Ingombro superiore a questo valore sei appunto Ingombrato. In ogni caso non puoi portare su di te oggetti con valore di ingombro superiore a 10+Potenza.

\bigskip

\textbf{Tabella valori e penalità di Ingombro}

\medskip

\begin{tabular}{lll}
	\hline
	\textbf{Ingombro} & \textbf{ Penalità Movimento}  & \textbf{Penalità Prova Agilità} \\
	Entro Potenza +2  & Movimento pieno               & Nessuna penalità\\
	Entro Potenza +5  & Velocita' Movimento dimezzato & -3              \\
	Entro Potenza +7  & 1 metro a round               & -6              \\
\end{tabular}

\subsubsection{Oggetti e valori di Ingombro}

Ogni oggetto ha un valore di Ingombro che può assumere un valore numerico, oppure indicato come 'L' (leggero) oppure non significativo '-'.

Ogni 10 oggetti Leggeri si conta 1 Ingombro (si arrotonda per difetto).

Oggetti con ingombro non significativo non si valutano nel computo totale dell'ingombro tranne se portati in grosse quantità.

Ad esempio una armatura completa ha un ingombro di 4, una spada lunga 2, un pugnale o pergamena ha un ingombro Leggero.

Il Narratore stabilisce eventuali ingombri speciali e particolari.

\subsubsection{Come stimare l'Ingombro}

Come regola generale un oggetto che pesa dai 5 ai 10 kg ha Ingombro 1, un oggetto che pesa circa di 500gr è leggero, se pesa meno di 10gr allora non è significativo.

Un corpo di una creatura media ha ingombro di 1 ogni 20 kg di peso, se trascinato la metà.

\subsubsection{Ingombro delle Monete}

E' un ingombro piacevole quando si tratta di monete di Platino, lo sappiamo tutti...
Ogni 1000 Monete contate 1 di Ingombro, arrotondando per difetto.


\subsubsection{Trascinare}

In certe situazione è più facile spingere o trascinare che portare su di se.
In questi casi si considera un Ingombro con valore dimezzato.

\begin{itemize}
	\item Nel caso in cui l'Ingombro sia inferiore a Potenza +2 il personaggio avrà movimento pieno.

	\item Se è pari a Potenza ma non superiore a Potenza +5 il movimento è dimezzato.

	\item Se è superiore a Potenza +5 ed inferiore a Potenza +7, può spostarlo di 1 metro a round.

	\item Se è superiore a Potenza +7 non puoi spostare o spingere.
\end{itemize}

In caso piu' creature spingano "l'ingombro" considerate la Potenza piena della creatura piu' forte piu' meta' della Potenza delle creature con forza minore (minimo 1). Valutate chiaramente anche quante persone possono riuscire contemporaneamente a spingere data la dimensione dell'oggetto da trasportare.

\subsubsection{Creature Più Grandi e Più Piccole}

\textbf{Creature bipedi più grandi} possono trasportare più Ingombro in base alla categoria di taglia:

\begin{itemize}
	\item Piccolissima: Ingombro = Potenza /16
	\item Minuta: Ingombro = Potenza /8
	\item Minuscola: Ingombro = Potenza /4
	\item Piccola: Ingombro = Potenza /2 +3
	\item Grandi: Ingombro = Potenza x2 +5
	\item Enormi: Ingombro = Potenza x4 +5
	\item Mastodontica: Ingombro = Potenza x8 +5
	\item Colossale: Ingombro = Potenza x16 +5
\end{itemize}

\bigskip


\textbf{Le creature quadrupedi possono trasportare pesi superiori. }


\begin{itemize}
	\item Piccolissima: Ingombro = Potenza *1/4
	\item Minuta: Ingombro = Potenza x1/2
	\item Minuscola: Ingombro = Potenza x3/4
	\item Piccola: Ingombro = Potenza 1
	\item \textbf{media}: Ingombro = Potenza x1.5 +5
	\item Grandi: Ingombro = Potenza x3 +5
	\item Enormi: Ingombro = Potenza x6 +5
	\item Mastodontica: Ingombro = Potenza x12 +5
	\item Colossale: Ingombro = Potenza x24 +5
\end{itemize}

\subsection{Altri Tipi di Movimento}

\label{altri-tipi-di-movimento}

Le informazioni qui di seguito sono raccolte da varie sezioni e messe qui per vostra comodità.

\textbf{Nuotare}\index{Nuotare}

Una creatura con una velocità di Nuotare può muoversi attraverso l'acqua alla sua velocità indicata senza fare prove di Resistenza. Si guadagna un bonus di +8 su qualsiasi prova di Resistenza per eseguire un'azione particolare o evitare un pericolo. La creatura può sempre scegliere di prendere 10 su una prova di Nuotare, anche se distratti o in pericolo quando si nuota. Una tale creatura può utilizzare l'azione di correre mentre nuota, a condizione che nuoti in linea retta.\\
Se non si il tipo di movimento Nuotare si considera come "terreno" difficile, e quindi ci si muovo a meta' della velocita' indicata da movimento.

\textbf{Scalare}\index{Scalare}

Una creatura con una velocità di Scalare ha un bonus di +8 su tutti le prove di Resistenza. La creatura deve fare una prova di Resistenza per arrampicarsi su qualsiasi parete o pendenza con una DC superiore a 0, ma può sempre scegliere di prendere 10, anche se di fretta o minacciata durante la salita.

Se una creatura con una velocità di Scalare tenta una scalata rapida (vedi sopra), guadagna un 2 punti movimento e fa una singola prova di Scalare (Resistenza) con una penalità di -5.

Una creatura mantiene il suo bonus di Agilità alla Difesa (se presente) durante la salita, e gli avversari non ottengono bonus speciale per i loro attacchi contro di esso. Non puo', tuttavia, utilizzare l'azione correre mentre si arrampica.

Se non si il tipo di movimento Scalare si considera come "terreno" difficile, e quindi ci si muovo a meta' della velocita' indicata da movimento.

\textbf{Scavare}\index{Scavare}

Una creatura con una velocità di Scavare può scavare tunnel attraverso la terra, ma non attraverso la roccia a meno che il testo descrittivo non dica il contrario. Le creature non possono caricare o correre mentre scavano.

La maggior parte delle creature scavatrici non lascia tunnel che altre creature possono utilizzare (sia perché il materiale attraverso cui scavano riempie il tunnel dietro di loro o perché in realtà non spostano materiale quando scavano), vedere la descrizione della singola creatura per i dettagli.

\textbf{Velocità Su Terreno}

La Velocità sul terreno é la normale velocità per personaggi che non scalano, nuotano o volano.

\textbf{Volare}\index{Volare}

Una creatura con una velocità di Volare riceve gratuitamente l'abilità Volare come competenza.

\textbf{Volo e Manovrabilita'}

\medskip

Una creatura con una velocità di volare naturale riceve bonus (o penalita') alle prove di Volare in base alla propria manovrabilita':

\begin{tabular}{llll}
	\hline
	\textbf{Manovrabilità} & \textbf{ Malus Manovrabilita'} & \textbf{Manovrabilità} & \textbf{ Bonus Manovrabilita'} \\
	Maldestra              & -8             & Perfetta               & +8             \\
	Scarsa & -4             & Buona  & +4             \\
	Media  & +0       \\
\end{tabular}


Le creature che non hanno una specifica manovrabilità (es. un umano con una Essenza di Volo), si presume abbiano manovrabilità scarsa.

\textbf{Volo e Taglia}

Una creatura più grande o più piccola della Taglia Media ha bonus o penalità di taglia alle prove di Volare in base alla sua categoria di taglia:

\medskip

\begin{tabular}{llll}
	\hline
	\textbf{Taglia} & \textbf{ Bonus Manovrabilita'} & \textbf{Taglia} & \textbf{ Malus Manovrabilita'} \\
	Piccolissima    & +8             & Colossale       & -8             \\
	Minuta          & +6             & Mastodontica    & -6             \\
	Minuscola       & +4             & Enorme          & -4             \\
	Piccola         & +2             & Grande          & -2             \\
\end{tabular}

Se nella creatura è indicata una classe di manovrabilità si intende già compresa di questi fattori. Vedere l'abilità Volare per ulteriori dettagli.

\pagebreak

\section{Masterizzare}\index{Masterizzare}\index{Narratore}

\label{masterizzare}


\subsection{Il Narratore}

\begin{tcolorbox}[enhanced,arc=5pt,boxrule=0.3pt]{Chi comanda al racconto non è la voce: è l'orecchio. (Italo Calvino)}\end{tcolorbox}\medskip


\label{il-narratore}

Mentre il giocatore interpreta un personaggio in un'avventura, Il Narratore è colui che la gestisce. Ha certamente molto più lavoro, ma ricreare un mondo intero affinché i propri amici lo esplorino, può dare molte soddisfazioni.

Il ruolo del Narratore non è facile ma concede enormi privilegi. Vedere i propri amici giocare, divertirsi, "ammattirsi" dietro dubbi, indovinelli e situazioni da te create da tantissima soddisfazione e divertimento.

Il tuo ruolo è quello del grande orchestratore, pianificatore o anche paesaggista se preferisci, con poche semplici pennellate delinei la struttura e saranno poi i giocatori ad aggiungere dettagli e situazioni.



\subsection{Punti Esperienza}\index{Punti Esperienza}\index{PX}

\label{punti-esperienza}

In TUS il passaggio di livello non è vincolato da un numero di mostri affrontati o dai tesori ottenuti, bensì dal fattore di difficoltà degli incontri e da come i giocatori hanno giocato.

Il consiglio e suggerimento primario è "passate di livello ogni qualvolta lo ritenete necessario al buon gioco ed all'avventura".

Se questo consiglio può sembrare un pò troppo scarno propongo un altro approccio, semplice ma ancora più efficace e stimolante.

Prendete questa tabella dei punti esperienza

\subsubsection{Tabella punti Esperienza / Livello}

\label{tabella-punti-esperienza-livello}

\begin{tabular}{ll}
	\toprule
	\textbf{Livello} & \textbf{Punti Esperienza}\\
	2& 15 xp\\
	3& 25 xp\\
	4 - 10           & 35 xp per livello\\
	11 - 20          & 25 xp per livello\\
\end{tabular}

\bigskip

Ovvero sono necessari 15 punti esperienza per passare dal primo al secondo livello ed altrettanti per passare dal secondo al terzo livello.

Per passare dal terzo al quarto servono 25 punti esperienza, e per passare ogni livello dal 4 al 10 livello ne servono 35 di punti esperienza.

Dall'undicesimo al ventesimo servono 25 punti esperienza ad ogni livello.
\medskip
\begin{itemize}
\item
\textbf{Per ogni incontro designato per sfidare il gruppo in maniera media o difficile assegnate 1 punto esperienza.}
\item
\textbf{Per ogni incontro designato per essere potenzialmente mortale assegnate 2 punti esperienza.}
\item
\textbf{Per ogni incontro finale, il climax dell'avventura, assegnate 3 punti esperienza, questi punti più che per lo scontro "con il Boss finale" vanno assegnati come merito per aver portato a termine una lunga avventura.}
\end{itemize}

Questi punti saranno assegnati al gruppo e quindi a tutti i giocatori, purché abbiano almeno cercato di partecipare agli scontri/sfide.

Se il gruppo per propria "incapacità" o per "sfortuna" trasforma un incontro facile (da 0 punti esperienza) in un incontro mortale, non dovete dare 2 punti esperienza. Cercate al più di premiare lo spirito di gruppo, le energie spese e se possibile la creatività nell'uscirne vivi  nonostante tutto.

Quando dico "incontro" non pensate al semplice scontro con i mostri, per incontro si intende qualsiasi evento di ruolo che sfidi e metta alla prova i giocatori. Questa sfida può essere una arguta discussione con il nobile che non li vuole pagare al termine di una missione, alla sfida di un indovinello, rebus, delle trappole ben piazzate.

\bigskip

Ogni qual volta il giocatore o il gruppo:\index{Bonus PX}\index{Bonus Esperienza}
\begin{itemize}
	\item
	      \textbf{raggiunga gli obiettivi prefissati};
	\item
	      \textbf{faccia un ottimo gioco di ruolo};
	\item
	      \textbf{ottimizzi l'uso delle proprie Abilità e capacità (senza cadere nel powerplayer)};
	\item
	      \textbf{risolva i problemi in maniera creativa e fantasiosa e funzionale};
	\item
	      \textbf{abbia buona collaborazione ed interpretazione dei diversi ruoli all'interno del gruppo e come gruppo verso i "terzi"};
	\item
	      \textbf{scopra o avvii indizi di avventura e creazione di nuovi plot};
	\item
	      \textbf{raccolga 10000 monete d'oro (o tesoro equivalente)};
\end{itemize}

\bigskip

Premiate il giocatore/giocatori con 1 punto esperienza. Questi punti vanno dati per sessione di gioco.

In questo sistema sono necessarie circa 10 sessioni per passare di livello, potenzialmente anche molte meno se i giocatori si dimostrano bravi ed interpretano personaggi e situazioni in maniera brillante.

Fate in modo che ogni sessione possa assegnare 1-3 punti almeno. Costruite la sessione perché tutti i giocatori possano essere partecipanti e nessuno si senta escuso.

Include situazioni stimolanti che possano coinvolgere e sfidare il gruppo.
Nel limite del possibile ogni sessione dovrebbe includere una parte di ruolo, una parte di esplorazione, una parte di combattimento, una parte di riposo.

\bigskip

Una nota riguarda i \textbf{premi per monete d'oro}: può sembrare anacronistico quando c'è già in sviluppo la sesta edizione del più famoso gioco di ruolo tornare a premiare i giocatori in base all'oro preso ai mostri. Posso però garantirvi che qualora il vostro gruppo sia particolarmente "povero" di giocate di ruolo o semplicemente preferisca uno stile più combattivo, sapere che l'oro raccolto equivale ad esperienza può rendere molto più dinamico ed avvincente l'andare in avventura.

\subsection{Incontri}\index{Incontri}


\begin{tcolorbox}[enhanced,arc=5pt,boxrule=0.3pt]{Che è la vita senza speranza? Una gittata di dadi fra le tenebre, fra i deliri. (Ambrogio Bazzero)}\end{tcolorbox}\medskip

\label{incontri}

Un incontro è un momento di tensione e speranza, paura e sfida. E' l'occasione di mostrare e manifestare le proprie capacità e di lavorare come gruppo.

Un incontro non è l'occasione per fare sfoggio del proprio potere assoluto, sia come Narratore, che come Giocatore. Il Narratore saprà punire il giocatore che vuole essere oltre il gruppo e non parte di esso.

Troverete nelle pagine seguenti le istruzioni per creare delle sfide facili (0 punti esperienza), medie e alte (1 punto esperienza), straordinarie (2 punti esperienza) ed epiche (3 punti esperienza).

Sarete comunque sempre voi, il Narratore, a stabilire e sapere se una sfida è provante o meno, se è sfidante e critica per i giocatori e quindi volutarne sia l'impatto come punti esperienza che come difficoltà.

Un incontro è un evento che mette i personaggi di fronte ad un problema specifico che devono risolvere. Molti sono combattimenti con i mostri o i PNG ostili, ma ce ne sono altri tipi: un corridoio irto di trappole, un'interazione politica con un re sospettoso, un passaggio pericoloso sopra un ponticello di corda traballante, un argomento scomodo con un PNG amichevole che ritiene che un personaggio lo abbia tradito, o qualsiasi cosa che aggiunga un pò di drammaticità al gioco.

Rompicapi, sfide interpretative e prove di abilità sono i metodi classici per la risoluzione degli incontri, ma gli incontri più complessi da costruire sono i più comuni incontri di combattimento.

Nel progettare un incontro di combattimento, in primo luogo decidete che livello di sfida volete far fronteggiare ai PG, quindi seguite i punti descritti qui di seguito.

\textbf{Determinare APL}: \index{APL}Determinate il livello medio dei personaggi: questo è il Livello Medio del Gruppo (APL in breve, Average Party Level). Dovreste arrotondate questo valore al numero intero più vicino (questa è una delle poche eccezioni alla regola dell'arrotondamento per difetto).

Si noti che questa guida di riferimento della creazione di un incontro presuppone un gruppo di quattro o cinque PG. Se il vostro gruppo ha sei o più giocatori, aggiungete uno al loro livello medio. Se il vostro gruppo contiene tre o meno giocatori, sottraete uno dal loro livello medio. Per esempio, se il vostro gruppo consiste di sei giocatori, due di 4° livello e quattro di 5° livello, il APL è il 6° (28 livelli totali, diviso per sei giocatori, arrotondando per eccesso e aggiungendo uno al risultato finale).

\textbf{Determinare il CR}: Il Grado di Sfida (o CR) è un numero di convenienza usato per indicare i rischi relativi presentati da un mostro, una trappola, un pericolo o un altro incontro: più il CR è alto, più pericoloso è l'incontro. Riferitevi alla Tabella: Determinare gli Incontri per determinare il Grado di Sfida che il vostro gruppo dovrebbe affrontare, in base alla difficoltà della sfida che volete e al APL.

\bigskip

\textbf{Tabella: Determinare gli Incontri}

\begin{tabular}{ll}
	\toprule
	\textbf{Difficolta'} & \textbf{Grado di Sfida (CR)}\\
	Facile               & APL -1\\
	Media& APL\\
	Alta & APL +1\\
	straordinaria        & APL +2\\
	Epica& APL +3\\
\end{tabular}

\bigskip

\textbf{Costruire l'Incontro}: per costruire un incontro come prima cosa calcolate il valore dell' APL.

Per sviluppare il vostro incontro, aggiungete le creature, le trappole ed i pericoli finché non arrivate a vostro APL programmato.

Parti calcolando le sfide con CR più alto dell'incontro, completando il resto con sfide minori.

Per esempio, volete che il vostro gruppo di sei personaggi di 8° livello affronti alcuni Gargoyle (CR 4 ciascuno) e il loro capo, un Gigante delle Rocce (CR 8). I personaggi hanno APL 9 e la Tabella: Determinare gli Incontri stabilisce che unasfida Alta per un APL 9 è un incontro di CR 10.

Partendo da un CR stabilito (10) seguite questa tabella per stabilire quanti "mostri inserire nello scontro".

\bigskip

\begin{tabular}{lll}
	\toprule
	\textbf{CR obiettivo} & \textbf{CR creatura rispetto ad obiettivo APL} & \textbf{"Peso" per singola creatura}\\
	CR                    & -6                                             & 10\\
	                      & -5                                             & 15\\
	                      & -4                                             & 20\\
	                      & -3                                             & 30\\
	                      & -2                                             & 50\\
	                      & -1                                             & 75\\
	                      & 0                                              & 100\\
\end{tabular}

\bigskip

\textbf{Per raggiungere l'obiettivo dobbiamo sommare "i pesi"
	fino a raggiungere 100, ovvero il 100\% della sfida.}

Nel nostro esempio un Gigante delle Rocce ha CR 8, ovvero un CR -2 rispetto al nostro obiettivo di difficoltà CR 10, ed i Gargoyle hanno CR 4 ovvero -6 rispetto al CR 10.

Un nemico con CR -2 ha peso 50, un CR -6 ha un peso di 10, per raggiungere l'obiettivo di un CR 10 metterò 1 CR -2 ( ovvero UNO gigante delle rocce) e 5 CR -6 (ovvero CINQUE gargoyle) perché 50 + 10{*}5 = 100

Se avessi avuto come obiettivo un CR 8, avrei potuto mettere 3 CR 5 (CR -3 = 30) ed un CR 2 (CR -6 = 10), oppure 5 CR 4 (CR +4 = 20)

Avversari con CR inferiore a 7 rispetto al APL si contano, "pesano", solo se sono superiori a 20 come unità.

\textbf{Aggiungere i PNG}

Una creatura che possiede livelli, Abilità , competenze, che potrebbe essere un personaggio si considera un PNG. Queste creature possono svolgere un ruolo molto importante e non vanno usate come semplici "mostri". Dategli uno spessore e creerete dei personaggi indimenticabili.

\textbf{Modifiche ad Hoc del CR}

Mentre potete modificare il CR specifico del mostro avanzandolo, applicando modifiche o livelli, potete anche aggiustare la difficoltà dell'incontro applicando modifiche ad hoc all'incontro o alla creatura in sé.

Qui descritti ci sono tre modi aggiuntivi con cui potete alterare la difficoltà dell'incontro.

\textbf{Terreno Favorevole ai PG}

Un incontro contro un mostro che non è nel suo elemento preferito (come uno Yeti incontrato in una caverna piena di lava, o un Drago enorme incontrato in una stanza molto piccola) da ai personaggi un vantaggio. Sviluppate l'incontronormalmente considerate l'incontro come se avesse un CR più basso del suo CR reale.

\textbf{Terreno Sfavorevole ai PG}

I mostri sono progettati con il presupposto che siano incontrati nel loro terreno preferito: incontrare un Aboleth sott’acqua non aumenta il CR dell'incontro, anche se nessun personaggio è in grado di respirare sott'acqua.

Se, d’altra parte, il terreno ha un impatto più significativo sull'incontro (come un incontro contro una creatura con Vista Cieca in una zona che sopprime ogni fonte di luce), si possono, nel caso, aumentare il CR dell'incontro fosse di un grado più alto.


\textbf{Modifiche all'Equipaggiamento dei PNG}

Potete aumentare o diminuire la difficoltà data dai PNG modificandone l'Equipaggiamento. Il valore combinato dell'Equipaggiamento di un PNG è dato in Creare i PNG alla Tabella: Equipaggiamento dei PNG. Un PNG incontrato senza Equipaggiamento dovrebbe avere un CR ridotto di 1 (a condizione che la perdita di Equipaggiamento sia realmente controproducente per il PNG), mentre un PNG che ha un Equipaggiamento equivalente a quello di un personaggio (come indicato sulla Tabella: Ricchezza dei PNG per Livello) ha un CR superiore di 1 al suo CR reale.

Occorre prestare attenzione ad assegnare ai PNG questo equipaggiamento supplementare, specie ai livelli più alti, in cui potete consumare l'intero tesoro della vostra avventura in un colpo solo!

\textbf{Assegnare i PX}

I personaggi avanzano di livello sconfiggendo mostri, superando sfide e completando l'avventura: nel farlo guadagnano i Punti Esperienza (PX in breve). Potete assegnare Punti Esperienza non appena una sfida viene superata, ma ciò potrebbero interrompere il flusso del gioco. E' più facile assegnare i punti esperienza alla fine di una sessione di gioco che permetta ai personaggi di riflettere su quanto accaduto. Può usare il tempo a disposizione fra le sessioni di gioco per aggiornare la scheda.

\textbf{Disporre Tesori}

Mentre i personaggi avanzano di livello, anche la quantità di tesori che trasportano ed usano aumenta. In TUS si suppone che tutti i personaggi di pari livello abbiano più o meno la stessa quantità di tesoro e oggetti magici. Poiché il reddito primario per un personaggio deriva dai tesori e dai bottini ricavati dalle avventure, è importante moderare la ricchezza e i tesori nelle vostre avventure. Per aiutarvi nel disporre i tesori, la quantità di oggetti magici e di bottino che i personaggi ricevono per le loro avventure è legata al CR degliincontri che affrontano: più è alto il CR dell'incontro, maggiore sarà il tesoro assegnato.

\textbf{La Tabella: Ricchezza dei PNG} per Livello indica la quantità di tesoro che ogni personaggio dovrebbe avere ad un livello specifico. Si noti che questa tabella si basa su un modello standard di gioco.

I giochi con magia rara potrebbero assegnare soltanto la metà di questo valore, mentre i giochi più epici potrebbero raddoppiarlo. Si presume che parte del tesoro sia consumato nel corso di un'avventura (come pozioni e pergamene) e che alcuni degli oggetti meno utilizzati siano venduti per metà del loro valore per acquistare un Equipaggiamento più utile.

La Tabella: Ricchezza dei PNG per Livello può anche essere usata per stanziare l'Equipaggiamento per i personaggi che cominciano dopo il 1° livello, come un nuovo personaggio creato per sostituirne uno morto. I personaggi non dovrebbero spendere più della metà della loro ricchezza totale su un singolo oggetto.

Per un metodo equilibrato, i personaggi che vengono creati dopo il 1° livello dovrebbero spendere il 25\% della loro ricchezza per le armi, il 25\% per armatura e oggetti di protezione, il 25\% per altri oggetti magici, il 15\% per oggetti che si consumano come bacchette, pergamene e pozioni e il 10\% per un Equipaggiamento normale e monete. Tipi di personaggio differenti potrebbero spendere diversamente la loro ricchezza rispetto a come suggerito; ad esempio, gli incantatori arcani potrebbero spendere di più per oggetti magici e a consumo che per le armi.

\textbf{La Tabella: Valori del Tesoro per Incontro} elenca la quantità di tesoro che ogni incontro dovrebbe assegnare in base al livello medio dei personaggi e alla velocità di progressione di PX della campagna. Gli incontri facili dovrebbero assegnare un tesoro di un livello più basso rispetto al livello medio dei PG. Gli incontri più pericolosi, difficili ed eroici dovrebbero assegnare rispettivamente un tesoro di uno, due o tre livelli superiore al livello medio dei PG. Se nel gioco la magia è rara, dimezzate questi valori. Se il gioco è più epico, raddoppiate questi valori.

\bigskip

\textbf{Tabella: Valori del Tesoro per Incontro}\index{Tesoro}

\begin{tabular}{llll}
	\toprule
	\textbf{Livello Medio Gruppo} & \textbf{per Incontro (mo)} & \textbf{Livello Medio Gruppo} & \textbf{per Incontro (mo)}\\
	1             & 130        & 11            & 3500\\
	2             & 250        & 12            & 4500\\
	3             & 400        & 13            & 5500\\
	4             & 5500       & 14            & 7500\\
	5             & 750        & 15            & 9000\\
	6             & 1000       & 16            & 12000\\
	7             & 1300       & 17            & 16000\\
	8             & 1700       & 18            & 20000\\
	9             & 2100       & 19            & 25000\\
	10            & 2500       & 20            & 32000\\
\end{tabular}
\bigskip

\textbf{Tabella: Ricchezza dei Personaggi per Livello}

\bigskip

\begin{tabular}{llll}
	\toprule
	\textbf{Livello Personaggio} & \textbf{Richezza} & \textbf{ Livello Personaggio} & \textbf{Richezza}\\
	1            & 100               & 11            & 82000\\
	2            & 500               & 12            & 55000\\
	3            & 1500              & 13            & 75000\\
	4            & 3000              & 14            & 100000\\
	5            & 5000              & 15            & 120000\\
	6            & 8000              & 16            & 160000\\
	7            & 12000             & 17            & 210000\\
	8            & 17000             & 18            & 270000\\
	9            & 25000             & 19            & 350000\\
	10           & 35000             & 20            & 500000\\
\end{tabular}

\bigskip

Gli incontri contro dei PNG solitamente ricompensano con un tesoro
tre volte superiore a quello con un mostro, grazie all'Equipaggiamento
del PNG. Per compensare, assicuratevi che i personaggi affrontino
un paio di incontri supplementari che assegnano poco in fatto di tesori.

Animali, Vegetali, Costrutti, Non Morti non intelligenti, Melme e trappole sono ottimi "incontri con poco tesoro". In alternativa, se i personaggi affrontano un certo numero di creature con poco o nessun tesoro, dovrebbero avere l'occasione di ottenere un certo numero di oggetti di valore più significativo nell'immediato futuro per compensare lo squilibrio. Come regola generale, i personaggi non dovrebbero possedere alcun oggetto magico di valore superiore alla metà della ricchezza totale del personaggio, pertanto controllate bene prima di ricompensare i personaggi con oggetti molti costosi.

\subsubsection{Costruire un Bottino}\index{Costruire un Bottino}

Spesso è sufficiente dire ai vostri giocatori che hanno trovato 5.000
mo in gemme e 10.000 mo in gioielli. Ma a volte, è più interessante
fornire dei particolari. Dare a un tesoro una personalità può non
solo aiutare la verosimiglianza del gioco, ma può a volte innescare
nuove avventure.

Le informazioni nelle pagine seguenti possono aiutarvi per determinare tipi di tesori in modo casuale: per molti degli oggetti sono stati dati dei valori, ma potete assegnarli come ritenete meglio. E' più facile collocare gli oggetti più costosi prima: se volete potete anche determinare gli oggetti magici in modo casuale usando le tabelle in Oggetti Magici, per stabilire quali oggetti siano presenti nel tesoro.

Una volta che avete consumato una parte considerevole del valore del tesoro, il resto può semplicemente essere composto da monete sparse e oggetti non magici con valori definiti in base alle vostre esigenze.

\textbf{Monete}: Le monete in un tesoro possono essere di rame, argento, oro e platino: quelle d'argento e d'oro sono le più comuni, ma potete decidere diversamente. Per le monete ed il loro valore di cambio andate all'Equipaggiamento.

\textbf{Gemme}: Anche se potete assegnare qualsiasi valore ad una gemma, alcune possono valere di più delle altre. Utilizzate le categorie di valore qui sotto (e le pietre preziose associate) come guida di riferimento quando assegnate i valori alle pietre preziose.

\textbf{Gemme di Bassa Qualita'} (10 mo): agata; azzurrite; quarzo blu; ematite; lapislazzuli; malachite; ossidiana; rodocrosite; occhio di tigre; turchese; perla di fiume (irregolare).

\textbf{Gemme Semi Preziose} (50 mo): eliotropio, corniola; calcedonio; crisoprasio; citrino; diaspro; lunaria; onice; crisolito; cristallo di roccia (quarzo chiaro); sardonica; sardonice; quarzo rosato, affumicato o rosa di stella; zircone.

\textbf{Pietre Preziose di Media Qualita'} (100 mo): ambra; ametista;
crisoberillo; corallo; granato rosso o verde-marrone; giada; giaietto;
perla bianca, dorata, rosa o argentata; spinello rosso, marrone-rosso
o verde scuro; tormalina.

\textbf{Pietre Preziose di Alta Qualita'} (500 mo): alessandrite; acquamarina; granato viola; perla nera; spinello blu scuro; topazio giallo oro.

\textbf{Gioielli} (1.000 mo): smeraldo; opale bianco, nero, o di fuoco; zaffiro blu; corindone giallo fuoco o vermiglio; zaffiro a stella blu o nero.

\textbf{Gioielli Eccezional}i (5.000 mo o piu'): smeraldo verde brillante cristallino, diamante, giacinto, rubino.

\textbf{Tesori non Magici} Questa categoria include monili, abiti raffinati, merci, oggetti alchemici, oggetti perfetti e altri.

Diversamente delle gemme, molti di questi oggetti hanno valori stabiliti, ma potete sempre aumentare il valore dell'oggetto decorandolo con pietre preziose o con fatture particolarmente artistiche.

Questo aumento di costo non conferisce capacità aggiuntive: una scimitarra di Ferro Freddo impreziosita da gemme del valore di 40.000 mo funziona come una normale scimitarra di Ferro Freddo da 330 mo. Qui di seguito trovate numerosi esempi di tesori non magici, con i valori tipici.

\textbf{Oggetti d'Arte Raffinati} (100 mo o piu'): Anche se alcuni oggetti d'arte sono composti di materiali preziosi, il valore della maggior parte di pitture, sculture, opere letterarie, abiti raffinati, e simili consiste nella fattura con cui sono realizzati e nella bravura di chi li ha realizzati. Gli oggetti d'arte sono spesso ingombranti o difficili da spostare, e fragili, rendendone il recupero ed il trasporto un'avventura a sé.

\textbf{Monili Minori} (50 mo): Questa categoria comprende monili realizzati con materiali come ottone, bronzo, rame, avorio, o legni esotici, a volte impreziositi con gemme di bassa qualità molto piccole o difettate. I monili minori includono anelli, braccialetti e orecchini.

\textbf{Monili Normali} (100--500 mo): La maggior parte dei monili è realizzata con argento, oro, giada, o corallo, e decorata spesso con gemme semi preziose o pietre preziose di qualità media. I monili normali comprendono tutti i tipi di monili minori più bracciali, collane e spille.

\textbf{Monili Preziosi} (500 mo o piu'): I monili preziosi sono realizzati in oro, mithral, platino, o simili metalli rari. Tali oggetti comprendono i tipi di monili normali più scettri, pendenti ed altri grandi oggetti.

\textbf{Attrezzi perfetti} (100--300 mo): Questa categoria include attrezzi d'Abilita': vedi Equipaggiamento per i dettagli e i costi di questi oggetti.

\textbf{Oggetti Comun}i (fino a 1.000 mo): Ci sono molti oggetti di valore di natura alchemica o comune che possono essere utilizzati come tesoro. La maggior parte degli oggetti alchemici sono oggetti portabili e stimabili, ma anche altri come serrature, simboli sacri, cannocchiali, vini prelibati o abiti raffinati possono costituire parti interessanti di un tesoro. Anche le merci commerciali possono servire da tesoro: 5 kg di zafferano, per esempio, valgono 150 mo.

\textbf{Mappe del Tesoro e Oggetti d'Informazione} (variabili): Gli oggetti come mappe del tesoro, documenti legali di navi e case, liste di informatori o dei turni di guardia, parole d'accesso, e simili possono essere divertenti oggetti da trovare in un tesoro: potete stabilire il valore di questi oggetti come volete e possono essere di doppia utilità in quanto possono generare idee per nuove avventure.

\textbf{Oggetti Magici}

Naturalmente, la scoperta di un Oggetto Magico è il vero premio per qualsiasi avventuriero. Fate attenzione a collocare gli Oggetti Magici in un tesoro: è molto più soddisfacente per molti giocatori trovare un oggetto magico piuttosto che comprarlo, così non è sbagliato mettere degli oggetti che poi verranno usati dai giocatori!

Anche se in genere dovreste collocare gli oggetti con attenta riflessione sui loro probabili effetti sulla vostra campagna, può essere divertente generare gli oggetti magici in un tesoro a caso. Fate attenzione, comunque! è facile, con un pò di fortuna (o sfortuna) dei dadi gonfiare il vostro gioco con troppo tesoro o privarlo dello stesso. Il collocamento di oggetti magici casuali dovrebbe essere temperato sempre dal buon senso del Narratore.

\subsection{Recitare}\index{Recitare}

\label{recitare}

Un gioco di ruolo non è un semplice tirare dadi, è un incontro di pensieri, opinioni, sfide, lotte. E' un gioco catartico, liberatorio.

è giusto che ci sia combattimento, lotta, sangue paura ed azione, allo stesso modo deve esserci la possibilità di giocare i propri personaggi con i loro svantaggi storie.

Il giocatore deve sempre impersonare il personaggio, immedesimarsi e partecipare attivamente.

Ci possono essere situazioni di contorno, gestite velocemente, che vengono fatte in terza persona, eppure ogni volta che si rende necessario giocare questo deve essere vero, fatto dal giocatore calandosi appieno nel personaggio.

Quando un giocatore interpreta bene e descrive l'azione che va a svolgere in maniera partecipativa, coinvolgente, ispirata, dategli un premio, concedete un bonus di +1 all'azione che sta svolgendo.

Fatelo presente al giocatore che grazie alla sua interpretazione ha quel bonus.

\pagebreak

\section{Creare Oggetti Magici}\index{Creare Oggetti Magic}

\label{creare-oggetti-magici}

Per Creare Oggetti Magici è necessario avere le Abiltà Creazione oggetti magici.

Si possono Creare Oggetti Magici di vario tipo come:

\bigskip

\textbf{Creare Anelli Magici}\index{Anelli Magici}

Per creare un anello magico, un personaggio ha bisogno di una fonte di calore. Ha anche bisogno di una provvista di materiali, di cui il più ovvio è un anello o pezzi di anello da assemblare. Il costo dei materiali è compreso nel costo della creazione dell'anello.

\bigskip

\begin{tabular}{ll}
	\toprule
	\textbf{Livello di Potere inserito nell'anello} & \textbf{Costo dell'anello (mo)}\\
	\textless=11    & 1000\\
	13              & 2000\\
	16              & 3700\\
	19              & 25000\\
	22              & 50000\\
\end{tabular}

\bigskip

Un anello permette di fissare un Essenza in un anello per rendere l'effetto sempre attivo.

è anche possibile inserire un Essenza ad attivazione, in questo caso consultare i costi delle Verghe.

Il livello massimo di potere di un anello magico preparabile da un incantatore è 12+CM+bonus all'Essenza+caratteristica correlata all'Essenza.

Forgiare un anello richiede 1 giorno per ogni 1.000 mo del prezzo base.

Talento di creazione oggetto richiesto: Creare Oggetti Magici Superiori

Competenza usata nella creazione: Cultura

\subsection{Creare Armature Magiche}\index{Creare Armature Magiche}

Per creare un'armatura magica, un personaggio ha bisogno di una fonte di calore e di alcuni attrezzi per lavorare il ferro, il legno o il cuoio. Ha anche bisogno di una provvista di materiali, di cui il più ovvio è l'armatura stessa o i pezzi di armatura da assemblare. Un'armatura che va incantata deve essere di qualita'.

Se i prerequisiti per la creazione dell'armatura comprendono delle Essenze, l'incantatore deve conoscere dette Essenze.

Creare armature magiche richiede un giorno per ogni 1.000 mo del valore
del prezzo base.

Talento di creazione oggetto richiesto: Creare Oggetti Magici

Competenza usata nella creazione: Sapienza Magica o Artigianato (fabbricare armature).

\subsection{Creare Armi Magiche}\index{Creare Armi Magiche}

Per creare un'arma magica, un personaggio ha bisogno di una fonte di calore e alcuni attrezzi per lavorare il ferro, il legno o il cuoio. Ha anche bisogno di una provvista di materiali, di cui il più ovvio è l'arma stessa o i pezzi di arma da assemblare. Solo un'arma di qualita' può essere incantata per diventare un'arma magica, e il suo costo va aggiunto al costo totale di incantamento per determinare il valore finale di mercato.

Un'arma magica deve avere almeno bonus di potenziamento +1 per avere una qualsiasi delle Capacità Speciali delle armi da mischia o da distanza.

Se i prerequisiti per la creazione dell'arma comprendono delle Essenze, l'incantatore deve conoscere dette Essenze.

Nel momento della creazione, l'incantatore deve decidere se l'arma emana luce o meno, come effetto secondario della magia infusa nell'arma. Questa decisione non influenza il prezzo o il tempo di creazione, ma una volta che l'oggetto è completato, la decisione è definitiva.

Creare armi doppie viene considerato analogo a creare due armi per quanto riguarda il costo, il tempo e le Capacità Speciali.

Creare un'arma magica richiede una giornata per ogni 1.000 mo del valore del prezzo base.

Talento di creazione oggetto richiesto: Creare Oggetti Magici

Competenza usata nella creazione: Arcana, Artigianato (costruire archi) (per archi e frecce magici) o Artigianato (fabbricare armi) (per tutte le altre armi).

\subsection{Creare Bacchette}\index{Creare Bacchette}

\bigskip

\textbf{Costi Base delle Bacchette}

\begin{tabular}{ll}
	\toprule
	\textbf{Livello di Potere inserito nella bacchetta} & \textbf{Costo della bacchetta (mo)}\\
	\textless=11        & 375\\
	13  & 750\\
	16  & 4500\\
	19  & 11250\\
	22  & 21000\\
\end{tabular}

\bigskip

Una bacchetta è un oggetto magico che conserva in se una Essenza caricata in precedenza.

Per ricaricare una bacchetta un incantatore deve infondere parte della sua magia nello stesso, la bacchetta recupera una carica ma l'incantatore oltre ad avere usato una magia ha per le successive 24 ore un -4 a tutti i CM.

Per creare una bacchetta, un personaggio ha bisogno di una provvista di materiali, di cui il più ovvio è una bacchetta o i pezzi di una bacchetta da assemblare. Le bacchette sono sempre pienamente cariche (25 cariche) all'atto della creazione.

L'incantatore deve conoscere l'Essenza che inserisce nella Bacchetta.

Il livello massimo di potere di una bacchetta preparabile da un incantatore è 14+CM+bonus all'Essenza+caratteristica correlata all'Essenza.

Creare una bacchetta richiede 1 giorno per ogni 1.000 mo del valore del prezzo base.

Talento di creazione oggetto richiesto: Creare Oggetti Magici.

Competenza usata nella creazione: Sapienza Magica, Artigianato (oreficeria),
Artigianato (scultura) o Professione (taglialegna).

\subsection{Creare Bastoni}\index{Creare Bastoni}

\textbf{Costi Base dei Bastoni}

\bigskip

\begin{tabular}{ll}
	\toprule
	\textbf{Livello di Potere inserito nel bastone} & \textbf{Costo del bastone (mo)}\\
	\textless=11    & 375\\
	13              & 750\\
	16              & 4500\\
	19              & 11250\\
	22              & 21000\\
	25              & 40000\\
	28              & 65000\\
	31              & 120000\\
\end{tabular}

\bigskip

Un Bastone è un oggetto magico dove si carica una o più Essenze il cui livello totale non può essere superiore a quello indicato come Livello di Potere inserito nel bastone.

Quando un bastone viene attivato è possibile usare un'Essenza alla volta.

Per creare un bastone, un personaggio ha bisogno di una provvista di materiali, di cui il più ovvio è un bastone o i pezzi di un bastone da assemblare.

I bastoni sono sempre pienamente carichi (10 volte il Livello di Potere indicato) all'atto della creazione.

Il livello massimo di potere di un bastone preparabile da un incantatore è 14+CM+bonus all'Essenza+caratteristica correlata all'Essenza.

Per ricaricare un Bastone un incantatore deve infondere parte della sua magia nello stesso, il bastone recupera una carica ma l'incantatore oltre ad avere usato una magia ha per le successive 24 ore un -4 a tutti i CM.

l'incantatore deve conoscere le Essenze che inserisce nel bastone.

Creare un bastone richiede 1 giorno per ogni 1.000 mo del prezzo base.

Talento di creazione oggetto richiesto: Creare Oggetti Magici Superiori

Competenza usata nella creazione: Arcana, Artigianato (oreficeria), Artigianato (scultura) o Professione (taglialegna).

\pagebreak

\subsection{Creare Oggetti Magici}\index{Creare Oggetti Magici}

Per creare un oggetto meraviglioso, un personaggio di solito ha bisogno di un determinato Equipaggiamento o attrezzi particolari. Inoltre ha bisogno di una provvista di materiali, di cui il più ovvio è l'oggetto stesso o i pezzi dell'oggetto da assemblare.

Il costo dei materiali è compreso nel costo della creazione dell'oggetto. I costi degli oggetti meravigliosi sono difficili da calcolare. Riferitevi alla Tabella: Calcolare il Valore dell'Oggetto Magico in Monete d'Oro e usate i prezzi deglioggetti nelle descrizioni come punto di riferimento. Creare un oggetto costa la metà del prezzo di mercato indicato.

l'incantatore deve conoscere le Essenze che inserisce nell'oggetto.

Creare un oggetto meraviglioso richiede 1 giorno per ogni 1.000 mo del valore del prezzo base.

Talento di creazione oggetto richiesto: Creare Oggetti superiori.

Competenza usata nella creazione: Sapienza Magica o un'Abilità adeguata
di Artigianato o Professione.

\subsection{Creare Pergamene}\index{Creare Pergamene}\index{Pergamene}

\begin{tabular}{ll}
	\toprule
	\textbf{Livello di Potere della pergamena} & \textbf{Costo della pergamena (mo)}\\
	\textless=11               & 100\\
	13         & 200\\
	16         & 400\\
	19         & 800\\
	22         & 1600\\
	25         & 3200\\
	28         & 6400\\
	31         & 12800\\
	34         & 25600\\
	37         & 52000\\
\end{tabular}

\bigskip

Se una pergamena include più Essenze il costo è pari alla somma dei livelli di potere delle varie Essenze.

L'incantatore deve conoscere le Essenze che inserisce nella pergamena.

Per creare una pergamena c'è un costo fisso di materiali pari a 100 mo per livello di potere da caricare. Per preparare una pergamena è necessario 1 ora di lavoro per livello di potere.

Il livello massimo di potere che una pergamena puo' avere è 16+CM+bonus all'Essenza+caratteristica correlata all'Essenza.

Per leggere una pergamena è necessario:
- per comprendere il contenuto e' sufficiente una prova di Arcana con DC pari a meta' del LP contenuto nella pergamena
- per comprendere anche il livello di potere della pergamena la difficolta' e' pari al LP della pergamena -5
- per poter leggere e lanciare l'Essenza della pergamena o si riesce in un prova di Essenza di Rivelazione a DC 13 oppure una prova di Arcana con DC pari al livello della pergamena

Talento di creazione dell'oggetto richiesto: Creare Oggetti Magici

Competenza usata nella creazione: Arcana, Artigianato (calligrafia) o Professione (scrivano).

\subsection{Creare Pozioni}\index{Creare Pozioni}\index{Pozioni}

Una pozione contiene l'infuso di una Essenza, ogni pozione è quindi monouso

\bigskip

\begin{tabular}{ll}
	\toprule
	\textbf{Livello di potere} & \textbf{Costo}\\
	\textless=11               & 50\\
	13         & 100\\
	16         & 200\\
	19         & 400\\
	22         & 800\\
\end{tabular}

\bigskip

Il costo per creare la pozione è metà del prezzo base. Per creare una pozione, un personaggio ha bisogno di un piano di lavoro orizzontale e alcuni contenitori per mescolare i liquidi, oltre a una fonte di calore per bollire l'infuso.

Il livello massimo di potere di una pozione preparabile da un incantatore è 14+CM+bonus all'Essenza+caratteristica correlata all'Essenza.

Tutti gli ingredienti e i materiali per mescere una pozione devono essere freschi e mai usati.

L'incantatore deve conoscere l'Essenza che inserisce nella pozione.

Talento di creazione oggetto richiesto: Distillare Pozioni

Competenza usata nella creazione: Sapienza Magica o Artigianato (alchimia).

\subsection{Creare Verghe}\index{Creare Verghe}\index{Verghe}

Una verga è una bacchetta speciale che è capace di rigenerare le proprie cariche. Sono oggetti preziosi e molto costosi.

Per creare una verga, un personaggio ha bisogno di una provvista di materiali, di cui il più ovvio è una verga o i pezzi di una verga da assemblare.

\medskip

\begin{tabular}{ll}
	\toprule
	\textbf{Livello di Potere inserito nella verga} & \textbf{Costo della verga (mo)}\\
	\textless=11    & 750\\
	13              & 1500\\
	16              & 3000\\
	19              & 20000\\
	22              & 40000\\
\end{tabular}

\bigskip

Una verga è in grado di lanciare 1 volta al giorno la propria Essenza. Moltiplicare il costo per 4 se è in grado di lanciarla 2 volte, moltiplicare per 8 se è in grado di lanciarla 3 volte al giorno.

Si può anche lanciare una volta in più nel giorno l'Essenza contenuta nella verga, dopo di che la verga si distrugge.

Il livello massimo di potere di una verga preparabile da un incantatore è 10+CM+bonus all'Essenza+caratteristica correlata all'Essenza.

l'incantatore deve conoscere le Essenze che inserisce nella pozione.

Creare una verga richiede 1 giorno per ogni 1.000 mo del prezzo base.

Talento di creazione oggetto richiesto: Creare Oggetti Magici Superiori.

Competenza usata nella creazione: Arcana, Artigianato (fabbricare armi), Artigianato (oreficeria) o Artigianato (scultura).

\subsection{Aggiungere Nuove Capacita'}\index{Aggiungere Nuove Capacita}

A volte, la mancanza di fondi o tempo rende impossibile realizzare l'oggetto magico voluto, ma fortunatamente è possibile potenziare o modificare un oggetto magico creato. Solo il tempo, l'oro ed i vari prerequisiti richiesti dalla nuova capacità che si vuole aggiungere all'oggetto magico pongono delle restrizione sul tipo di poteri addizionali che uno può infondere.

Il costo per aggiungere capacità addizionali ad un oggetto è lo stesso che se l'oggetto non fosse magico, meno il valore dell'oggetto originale. Quindi, una spada lunga +1 può diventare una spada lunga vorpal +2, e il costo della creazione è uguale a quello di una spada lunga vorpal +2 meno il costo di una spada lunga +1.

Quando si determina il prezzo di un oggetto magico inventato bisogna considerare molti fattori. Il modo più semplice per decidere il prezzo è confrontare il nuovo oggetto a un oggetto in questo capitolo che ha già un prezzo, e usare tale prezzo come guida.

Altrimenti, si possono usare le indicazioni riassunte nella Tabella: Calcolare il Valore dell'Oggetto Magico in Monete d'Oro.

\bigskip

\textbf{Calcolare il Valore dell'Oggetto Magico in Monete d'Oro}\index{Valore}

\begin{tabular}{lll}
	\toprule
	\textbf{Effetto}        & \textbf{Prezzo base}         & \textbf{Esempio}\\
	Bonus di caratteristica & Bonus al quadrato x 1.000 mo & Cintura dell'agilità +2\\
	Bonus di armatura       & Bonus al quadrato x 1.000 mo & Cotta di maglia+1\\
	Essenza                 & Vedi costi pergamena {*} 8   & Perla del potere\\
	Bonus alla CA           & Bonus al quadrato x 2.000 mo & Anello di protezione+3\\
	Bonus ai Tiri Salvezza  & Bonus al quadrato x 1.000 mo & Mantello della resistenza +5\\
	Bonus di competenza     & Bonus al quadrato x 100 mo   & Mantello del passo gatto\\
	Bonus dell'arma         & Bonus al quadrato x 2.000 mo & Spada lunga+1\\
\end{tabular}

\pagebreak

\section{Oggetti Magici}\index{Oggetti Magici}

\label{oggetti-magici}
\begin{itemize}
	\item
	      Un personaggio può portare innumerevoli oggetti magici su di sé ma per determinare il bonus alla Difesa non si possono sommare più di 2 oggetti (es. 1 anello magico ed un braccialetto).L'armatura non si considera in questo conteggio, e deve essere una sola.
	\item
	      Lo stesso principio vale per il bonus ai Tiri Salvezza, puoi sommare solo i bonus provenienti da due oggetti.
	\item
	      Se il bonus è alle caratteristiche si conta solo quello con il bonus maggiore.
	\item
	      Un personaggio non può portare più di due anelli magici altrimenti
	      entrano in risonanza causando 1d6 di danno a round per ogni anello oltre il secondo.
\end{itemize}



\subsection{Tabella: Generazione Casuale degli Oggetti Magici}\index{Generazione Casuale}

\label{tabella-generazione-casuale-degli-oggetti-magici}

\begin{tabular}{llll}
	\toprule
	\textbf{Oggetto magico Minore} & \textbf{ Oggetto magico Medio} & \textbf{Oggetto magico Maggiore} & \textbf{Oggetto}    \\
	01--04         & 01--10         & 01--10           & Armature e scudi\\
	05--09         & 11--20         & 11--20           & Armi\\
	10--44         & 21--30         & 21--25           & Pozioni\\
	45--46         & 31--40         & 26--35           & Anelli\\
	---            & 41--50         & 36--45           & Verghe\\
	47--81         & 51--65         & 46--55           & Pergamene\\
	---            & 66--68         & 56--75           & Bastoni\\
	82--91         & 69--83         & 76--80           & Bacchette\\
	92--100        & 84--100        & 81--100          & Oggetti meravigliosi\\
\end{tabular}


\subsection{Taglia e Oggetti Magici}

\label{taglia-e-oggetti-magici}

Quando un capo di vestiario o un gioiello magici vengono scoperti, il più delle volte la taglia non è un problema: molti vestiti magici sono di facile utilizzo per tutti oppure si adattano magicamente a chi li indossa. Di regola, la taglia non dovrebbe impedire ai personaggi di varia tipologia fisica l'utilizzo di un oggetto magico.

Ci possono essere delle rare eccezioni, specie con gli oggetti realizzati per una razza specifica.

Le armi e le armature rinvenute casualmente hanno una probabilità del 30\% di essere Piccole (01--30), del 60\% di essere Medie (31--90), e del 10\% di essere di un'altra taglia.

\subsection{Oggetti Magici sul Corpo}\index{Oggetti Magici sul Corpo}

\label{oggetti-magici-sul-corpo}

Molti oggetti magici devono essere indossati da un personaggio che voglia usarli o beneficiare delle loro capacità. Per una creatura di forma umanoide è possibile indossare fino a 15 oggetti magici alla volta. Ognuno di questi oggetti deve essere indossato sopra una parte specifica del corpo denominata "slot".

Un corpo di forma umanoide può portare addosso Equipaggiamento magico consistente di un oggetto per ognuno dei gruppi seguenti, legato alla parte del corpo sulla quale viene indossato l'oggetto.

\textbf{Anello} (due al massimo): anelli.

\textbf{Armatura}: corazze e armature.

\textbf{Cintura}: cinture.

\textbf{Collo}: amuleti, collane, medaglioni, scarabei, spille e talismani.

\textbf{Corpo}: tuniche e vesti.

\textbf{Fronte}: corone, fasce e filatteri.

\textbf{Mani}: guanti e guanti d'arme.

\textbf{Occhi}: occhi, occhiali e lenti.

\textbf{Piedi}: scarpe, stivali e pantofole.

\textbf{Polso}: braccialetti e bracciali.

\textbf{Scudo}: scudi.

\textbf{Spalle}: cappe e mantelli.

\textbf{Testa}: cappelli, diademi, elmi e maschere.

\textbf{Torace}: camicie, giubbe, maglie e manti.

Naturalmente, un personaggio può possedere quanti oggetti vuole di uno stesso tipo. Ma oggetti magici dello stesso tipo addizionali, oltre a quelli previsti negli slot, non funzioneranno.

Alcuni oggetti possono essere indossati o trasportati senza occupare spazio sul corpo del personaggio. La descrizione di un oggetto indica quando l'oggetto possiede questa proprietà.


\subsection{Tiri Salvezza Contro i Poteri degli Oggetti Magici}\index{Tiri Salvezza}

\label{tiri-salvezza-contro-i-poteri-degli-oggetti-magici}

Gli oggetti magici normalmente riproducono Essenze o altri effetti magici. Per un Tiro Salvezza contro la magia o un effetto magico generato da un oggetto magico, la DC è sempre il livello potere dell'Essenza generata.

Le descrizioni di molti oggetti riportano le DC dei Tiri Salvezza relativi ai vari effetti, in modo particolare quando l'effetto non è descritto da un incantesimo equivalente (che rende difficile determinare rapidamente il suo livello di potere).


\subsection{Danneggiare gli Oggetti Magici}\index{Danneggiare gli Oggetti Magici}

\label{danneggiare-gli-oggetti-magici}

Un oggetto magico non deve compiere un Tiro Salvezza a meno che non sia incustodito, sia il bersaglio specifico dell'effetto, o il suo possessore ottenga un 3 naturale al suo Tiro Salvezza.

Gli oggetti magici hanno sempre diritto a un tiro salvezza contro qualcosa che potrebbe danneggiarli, anche quando un oggetto normale dello stesso tipo non avrebbe alcuna possibilità di effettuare un tiro salvezza. Gli oggetti magici usano sempre lo stesso bonus ai Tiri Salvezza, indipendentemente dal tipo (Tempra, Riflessi o Arbitrio). Il bonus ai Tiri Salvezza di un oggetto magico è pari a 2 + 1/2 del livello potere massimo usabile. Le sole eccezioni a questa regola sono gli oggetti magici intelligenti, che effettuano i tiri salvezza su Arbitrio basandosi sul loro punteggio di Volontà.


\subsection{Riparare gli Oggetti Magici}\index{Riparare gli Oggetti Magici}
\label{riparare-gli-oggetti-magici}

Per riparare un oggetto magico occorrono materiali e tempo, pari alla metà del tempo e del costo per crearlo. L'Essenza di Creazione con livello di potere pari al livello di potere massimo dell'oggetto magico ripara gli oggetti magici danneggiati.


\subsection{Cariche, Dosi e Usi Multipli}\index{Cariche}\index{Dosi}\index{Usi Multipli}

\label{cariche-dosi-e-usi-multipli}

Molti oggetti, e in modo particolare le bacchette e i bastoni, hanno un potere limitato al numero di cariche che contengono. Normalmente gli oggetti dotati di cariche non superano mai il massimo di 25 cariche (10 per i bastoni). Se oggetti simili vengono trovati come parte casuale di un tesoro, si tira un 5d6 e si divide per 2 per determinare il numero delle cariche rimaste (arrotondando per difetto, minimo 1). Se un oggetto ha un numero massimo di cariche diverso da 25, si tira casualmente per stabilire quante cariche sono rimaste.

I prezzi dati si riferiscono sempre agli oggetti al massimo delle loro cariche (quando un oggetto viene creato, ha sempre il massimo delle cariche). Se un oggetto perde di valore perché non ha più cariche (il che è valido per quasi tutti gli oggetti a cariche), il valore dell'oggetto parzialmente usato è pari al numero di cariche rimaste. Nel caso di oggetti che invece potrebbero avere un'utilita'anche se privi di cariche, soltanto parte del valore dell'oggetto sarà basato sul numero di cariche rimaste.


\subsection{Acquisire Oggetti Magici}\index{Acquisire Oggetti Magici}

\label{acquisire-oggetti-magici}

\bigskip

\begin{tabular}{lllll}
	\toprule
	\textbf{Dimensioni Comunita'} & \textbf{Valore Base} & \textbf{Minore} & \textbf{Medio} & \textbf{Maggiore}\\
	Insediamento  & 50mo & 1d4 oggetti     && \\
	Borgo         & 200mo& 1d6 oggetti     && \\
	Villaggio     & 500mo& 2d4 oggetti     & 1d4 oggetti    & \\
	Piccolo paese & 1000mo               & 3d4 oggetti     & 1d6 oggetti    & \\
	Grande paese  & 2000mo               & 3d4 oggetti     & 2d4 oggetti    & 1d4 oggetti\\
	Piccola città & 4000mo               & 4d4 oggetti     & 3d4 oggetti    & 1d6 oggetti\\
	Grande città  & 8000mo               & d4 oggetti      & 3d4 oggetti    & 2d4 oggetti\\
	Metropoli     & 16000mo              & {*}             & 4d4 oggetti    & 3d4 oggetti\\
\end{tabular}

{*} In una metropoli si trovano quasi tutti gli oggetti magici minori.

\bigskip

Gli oggetti magici sono preziosi e la maggior parte delle grandi città ha almeno uno o due fornitori di oggetti magici, dal semplice venditore di pozioni ad un fabbro specializzato nel forgiare spade magiche. Naturalmente, non ogni oggetto in questo manuale è disponibile in ogni città.

Le linee guida seguenti aiutano i Narratori a determinare quali oggetti sono disponibili in una specifica comunità. Esse presuppongono una campagna con un livello medio di magia. Alcune città potrebbero deviare di molto da questa linea di base a discrezione del Narratore. Il Narratore dovrebbe tenere una lista deglioggetti disponibili da ogni mercante e dovrebbe rimpinguare occasionalmente le scorte con nuove acquisizioni.

Il numero ed i tipi di oggetti magici disponibili in una comunità dipendono dalla sua dimensione. Ogni comunità ha un valore base legato ad essa (vedi Tabella: Oggetti Magici Disponibili).

c'è una probabilità del 75\% che qualsiasi oggetto di quel valore o inferiore si possa trovare in vendita facilmente in quella comunità. Inoltre, la comunità ha un certo numero di altri oggetti in vendita. Questi oggetti sono determinati a caso e sono ripartiti in categorie (minore, medio o maggiore).

Dopo aver determinato il numero di oggetti disponibili in ogni categoria, consultate la Tabella: Generazione Casuale degli Oggetti Magici per determinare il tipo di ogni oggetto (pozione, pergamena, anello, arma,ecc.) prima di passare alle tabelle specifiche per stabilire l'oggetto esatto. Ritirate ogni volta che gli oggetti non si adeguano al valore base della comunità.

Se l'uso della magia nella campagna in cui si gioca è raro, occorre dimezzare il valore base e il numero di oggetti in ogni comunità. Nelle campagne con magia estremamente rara o senza magia potrebbero non esserci affatto oggetti magici invendita. I Narratori che conducono questo tipo di campagne dovrebbe prevedere delle modifiche alle sfideaffrontate dai personaggi data la mancanza di oggetti magici.

Le campagne con abbondanti oggetti magici potrebbero avere comunità con il doppio del valore base stabilito e degli oggetti magici casuali disponibili. In alternativa, si potrebbe stabilire che tutte le comunità siano di una categoria di dimensione maggiore allo scopo di stabilire gli oggetti magici disponibili. In una campagna con magia molto comune, tutti gli oggetti magici si possonoacquistare in una metropoli.

Oggetti e attrezzi non magici sono in genere disponibili in una comunità di qualsiasi dimensione a meno che l'oggetto non sia molto costoso, come un'armatura completa, o fatto di un materiale insolito, come una spada lunga in adamantio. Questi oggetti dovrebbero seguire la linea guida del valore base per determinare la loro disponibilità, a discrezione del Narratore.

\pagebreak

\section{Armature Magiche}\index{Armature Magiche}

\label{armature-magiche}

Normalmente le armature magiche proteggono chi le indossa meglio di quanto potrebbe fare una qualsiasi armatura normale, pari modo gli scudi.

Una armatura magica ha un valore di Resistenza al colpo più alto e una Resistenza Totale migliore, ed è molto probabile che abbia pure una penalità alle prove di Agilità e CM più bassa.

Una armatura può avere un valore bonus da +1 a +5 più eventuali altre capacità magiche. Ogni +1 di una armatura alza di 1 il valore di Resistenza al colpo e raddoppia (o triplica, quadruplica\ldots ) il valore di Resistenza Totale.

Ogni +2 si abbassa di 1 la penalità di Prove Agilità, ed ogni +1 si abbassa di 1 la penalistà alle prove di Competenza Magica.

Un'armatura viene sempre costruita in modo che, anche se dotata di stivali, elmo oguanti d'arme, questi pezzi possano essere sostituiti con altri stivali, elmo o guanti d'arme magici.


\subsection{Tabella: Generazione casuale di Armature e Scudi Magici}

\label{tabella-generazione-casuale-di-armature-e-scudi-magici}

\begin{tabularx}{0.95\textwidth}{XXXXX}
	\toprule
	\textbf{Minore} & \textbf{Medio} & \textbf{Maggiore} & \textbf{Capacità Speciale}        & \textbf{Prezzo{*}}\\
	01--60          & 01--05         & ---               & scudo+1           & 1.000 mo\\
	61--80          & 06--10         & ---               & armatura+1        & 1.000 mo\\
	81--85          & 11--20         & ---               & scudo+2           & 4.000 mo\\
	86--87          & 21--30         & ---               & armatura+2        & 4.000 mo\\
	---             & 31--40         & 01--08            & scudo+3           & 9.000 mo\\
	---             & 41--50         & 09--16            & armatura+3        & 9.000 mo\\
	---             & 51--55         & 17--27            & scudo+4           & 16.000 mo\\
	---             & 56--57         & 28--38            & armatura+4        & 16.000 mo\\
	---             & ---            & 39--49            & scudo+5           & 25.000 mo\\
	---             & ---            & 50--57            & armatura+5        & 25.000 mo\\
	---             & ---            & ---               & armatura/scudo+6{*}               & 36.000 mo\\
	---             & ---            & ---               & armatura/scudo+7{*}               & 49.000 mo\\
	---             & ---            & ---               & armatura/scudo+8{*}               & 64.000 mo\\
	---             & ---            & ---               & armatura/scudo+9{*}               & 81.000 mo\\
	---             & ---            & ---               & armatura/scudo+10{*}              & 100.000 mo\\
	88--89          & 58--60         & 58--60            & Armatura specifica{*}{*}          & -\\
	90--91          & 61--63         & 61--63            & Scudo specifico{*}{*}{*}          & -\\
	92--100         & 64--100        & 64--100           & Capacità speciale e tirate ancora{*}{*},{*}{*}{*} & -\\
\end{tabularx}

{*} Armature e scudi non possono avere bonus di potenziamento superiori a +5. Usate queste indicazioni per determinarne il prezzo quando vengono aggiunte delle Capacità Speciali.

	{*}{*} Tirate sulla Tabella: Armature Specifiche.

	{*}{*} Tirate sulla Tabella: Scudi Specifici.



\subsection{Tabella Generazione Capacità Speciali delle Armature}

\label{tabella-generazione-capacita-speciali-delle-armature}

\begin{tabularx}{0.95\textwidth}{lllXX}

\toprule
	\textbf{Minore} & \textbf{Medio} & \textbf{Maggiore} & \textbf{Capacità Speciale}        & \textbf{ Modificatore Prezzo Base}\\
	01--25          & 01--05         & 01--03            & Felpa             & +2.700 mo\\
	26--32          & 06--08         & 04& Fortificazione Leggera            & bonus +1{*}\\
	33--52          & 09--11         & ---               & Scivolosa         & +3.750 mo\\
	53--92          & 12--17         & ---               & Ombra             & +3.750 mo\\
	93--96          & 18--19         & ---               & Mimetica          & bonus +1\\
	97              & 20--29         & 05--07            & Scivolosa Migliorata              & +15.000 mo\\
	98--99          & 30--49         & 08--13            & Ombra Migliorata  & +15.000 mo\\
	---             & 50--74         & 14--28            & Resistenza all'Energia            & +18.000 mo\\
	---             & 75--79         & 29--33            & Tocco Fantasma    & bonus +3\\
	---             & 80--84         & 34--35            & Invulnerabilità   & bonus +3\\
	---             & 85--89         & 36--40            & Fortificazione Moderata           & bonus +3\\
	---             & 90--94         & 41--42            & Della forma animale               & bonus +3\\
	---             & ---            & 44--48            & Scivolosa Superiore               & +33.750 mo\\
	---             & ---            & 49--58            & Ombra Superiore   & +33.750 mo\\
	---             & ---            & 59--83            & Resistenza all'Energia Migliorata & +42.000 mo\\
	---             & ---            & 84--88            & Ritira o scegli   & bonus +4{*}\\
	---             & ---            & 89& Forma Eterea      & +49.000 mo\\
	---             & ---            & 90& Controllo dei Non Morti           & +49.000 mo\\
	---             & ---            & 91--92            & Fortificazione Pesante            & bonus +5{*}\\
	---             & ---            & 93--94            & Ritira o scegli   & bonus +5{*}\\
	---             & ---            & 95--99            & Resistenza all'Energia Superiore  & +66.000 mo\\
	100             & 100            & 100               & Tirare ancora due volte{*}{*}     & ---\\
\end{tabularx}



\subsection{Tabella Generazione Capacità Speciali degli Scudi}

\label{tabella-generazione-capacita-speciali-degli-scudi}


\begin{tabularx}{0.95\textwidth}{lllXX}
	\toprule
	\textbf{Minore} & \textbf{Medio} & \textbf{Maggiore} & \textbf{Capacità Speciale}        & \textbf{Modificatore Prezzo Base{*}}\\
	01--20          & 01--10         & 01--05            & Attirare Frecce   & Bonus +1{*}\\
	21--40          & 11--20         & 06--08            & Sfondamento       & bonus +1{*}\\
	41--50          & 21--25         & 09--10            & Accecante         & bonus +1{*}\\
	51--75          & 26--40         & 11--15            & Fortificazione Leggera            & bonus +1{*}\\
	76--92          & 41--50         & 16--20            & Deviazione delle Frecce           & bonus +2{*}\\
	93--97          & 51--57         & 21--25            & Animato           & bonus +2{*}\\
	98--99          & 58--59         & - & Della forma animale               & bonus +2\\
	---             & 60--79         & 26--41            & Resistenza all'Energia            & +18.000 mo\\
	---             & 80--85         & 42--46            & Tocco Fantasma    & bonus +3{*}\\
	---             & 86--95         & 47--56            & Fortificazione Moderata           & bonus +3{*}\\
	---             & 96--98         & 57--58            & Ritira o scegli   & bonus +3{*}\\
	---             & 99             & 59& Selvatica         & bonus +3{*}\\
	---             & ---            & 60--84            & Resistenza all'Energia Migliorata & +42.000 mo\\
	---             & ---            & 85--86            & Ritira o scegli   & bonus +4{*}\\
	---             & ---            & 87& Controllo dei Non Morti           & +49.000 mo\\
	---             & ---            & 88--91            & Fortificazione Pesante            & bonus +5{*}\\
	---             & ---            & 92--93            & Riflettente       & bonus +5{*}\\
	---             & ---            & 94& Ritira o scegli   & bonus +5{*}\\
	---             & ---            & 95--99            & Resistenza all'Energia Superiore  & +66.000 mo\\
	100             & 100            & 100               & Tirare ancora due volte2          & ---\\
\end{tabularx}

{*} Da aggiungere al Bonus di Potenziamento sulla Tabella: Armature
e Scudi per determinare il prezzo di mercato totale.


\subsection{Tabella: Armature e Scudi Magici: Capacità Speciale}

\label{tabella-armature-e-scudi-magici-capacita-speciale}

\begin{tabular}{ll}
	\toprule
	\textbf{Armatura} / \textbf{Scudo} & \textbf{Costo}\\
	Scudo+1 / Armatura +1              & 1000\\
	Scudo +2 / Armatura+2              & 4000\\
	Scudo +4 / Armatura+4              & 16000\\
	Scudo +5 / Armatura+5              & 25000\\
	Scudo +6 / Armatura+6 {*}          & 36000\\
	Scudo +7 / Armatura+7 {*}          & 49000\\
	Scudo +8 / Armatura+8 {*}          & 64000\\
	Scudo +9 / Armatura+9 {*}          & 81000\\
	Scudo +10 / Armatura+10 {*}        & 100000\\
\end{tabular}

{*} Armature e scudi non possono avere bonus alla Difesa superiore
a +5. Usate queste indicazioni per determinarne il prezzo quando vengono
aggiunte delle Capacità Speciali.

\subsection{Scudi Magici}\index{Scudi Magici}

\label{scudi-magici}

I bonus di difesa degli scudi sono cumulativi con l'Agilità per determinare la difesa. I bonus magici degli scudi non vengono calcolati come bonus di attacco o ai danni quando uno scudo viene usato per attaccare. La capacità specialeSfondamento, tuttavia, conferisce bonus +1 agli attacchi (CA) e ai danni.

Si può costruire uno scudo che funzioni anche come un'arma magica, ma il costo del bonus magico offensivo deve essere sommato al costo del bonus difensivo alla Difesa dello scudo.

\textbf{Attivazione}

Normalmente un personaggio trae beneficio da un'armatura magica o da uno scudo magico esattamente allo stesso modo in cui lo trae da un'armatura o da uno scudo normale: indossandoli. Se un'armatura o uno scudo sono dotati di una capacità speciale che necessita di essere attivata da chi li indossa, allora chi li utilizza di solito deve pronunciare la parola di comando (2 Azioni).

\textbf{Armature per Creature Insolite}

Il costo dell'armatura per le creature non umanoidi, così come per le creature che non sono né Medie né Piccole, varia. Il costo di qualsiasi potenziamento magico rimane lo stesso.



\subsection{Capacità Speciali delle Armature Magiche e degli Scudi Magici}

\label{capacita-speciali-delle-armature-magiche-e-degli-scudi-magici}

Oltre alla resistenza od alla difesa l'armatura o lo scudo potrebbe avere delle capacità speciali. Le capacità speciali contano come bonus aggiuntivi per determinare il prezzo di mercato di un oggetto. Un'armatura o uno scudo non può avere un bonus effettivo (bonus di Difesa, bonus di resistenza più i bonus equivalenti delle capacità speciali, inclusi quelli derivanti da capacità ed Essenze) superiore a +10. Un'armatura o uno scudo dotata di una capacità speciale deve avere almeno bonus di +1.

\bigskip

\textbf{Capacità Speciali degli Scudi}

\medskip

\begin{tabular}{ll}
	\toprule
	\textbf{Capacità Scudo}       & \textbf{Capacità Scudo}\\
	Attirare Frecce: bonus +1{*}  & Sfondamento: bonus +1{*}\\
	Accecante: bonus +1{*}        & Fortificazione Leggera: bonus +1{*}\\
	Guardia: bonus +1{*}          & Determinazione: +30.000 mo\\
	Guardia Superiore: bonus +2{*}& Animato: bonus +2{*}\\
	Resistenza all'Energia: +18.000 mo            & Tocco Fantasma: bonus +3{*}\\
	Fortificazione Moderata: bonus +3{*}          & Riflettente: bonus +5{*}\\
	Controllo dei Non Morti: +49.000 mo           & Selvatica: bonus +3{*}\\
	Resistenza all'Energia Migliorata: +42.000 mo & Resistenza all'Energia Superiore: +66.000 mo\\
\end{tabular}

\bigskip

\textbf{Capacità Speciali delle Armature}

\begin{tabular}{ll}
	\toprule
	\textbf{Capacità Armatura}             & \textbf{Capacità Armatura}\\
	Amorfa:\index{Amorfa} +4500 mo         & Invulnerabilita'\index{Invulnerabilita'}: bonus +3{*}\\
	Giusto/Oscuro:\index{Giusto/Oscuro:} +27000 mo         & Determinazione\index{Determinazione}: +30000 mo\\
	Mimetica: \index{Mimetica}+2.700 mo    & Felpa\index{Felpa}: +2\\
	Denegante:\index{Denegante} bonus +4{*}& Fortificazione Leggera\index{Fortificazione Leggera}: bonus +1{*}\\
	Irrintracciabile:\index{Irrintracciabile} + 7500 mo    & Scivolosa\index{Scivolosa}: +3750 mo\\
	Ombra: \index{Ombra}+3750 mo           & Scivolosa Migliorata\index{Scivolosa Migliorata}: +15000 mo\\
	Ombra Migliorata:\index{Ombra Migliorata} +15000 mo    & Resistenza all'Energia: \index{Resistenza all'Energia}+18000 mo\\
	Fortificazione Moderata: \index{Fortificazione Moderata}bonus +3{*}    & Della Forma Animale\index{Della Forma Animale}: bonus +3{*}\\
	Scivolosa Superiore:\index{Scivolosa Superiore} +33750 mo              & Ombra Superiore: \index{Ombra Superiore}+33750 mo\\
	Resistenza all'Energia Migliorata: \index{Resistenza all'Energia Migliorata}:+42000 mo & Forma Eterea\index{Forma Eterea}: +49000 mo\\
	Controllo dei Non Morti:\index{Controllo dei Non Morti} +49000 mo      & Fortificazione Pesante:\index{Fortificazione Pesante} bonus +5{*}\\
	Resistenza all'Energia Superiore\index{Resistenza all'Energia Superiore}: +66000 mo    & Nube Esplosiva\index{Nube Esplosiva}: +5000 mo\\
\end{tabular}

{*} Da aggiungere al Bonus di Potenziamento sulla Tabella: Armature
e Scudi per determinare il prezzo di mercato totale.



\subsection{Descrizione delle Capacità Speciali delle Armature e Scudi Magici}

\label{descrizione-delle-capacita-speciali-delle-armature-e-scudi-magici}

Gran parte delle armature e degli scudi magici hanno solo bonus di potenziamento, ma certi possiedono alcune delle capacità speciali descritte qui sotto. Un'armatura o uno scudo dotati di Capacità Speciali devono avere almeno bonus di potenziamento +1.

Il valore di CM indicato è il livello minimo di competenza magica che si deve avere per creare l'oggetto.

\textbf{Accecante}\index{Accecante}

Uno scudo dotato di questo incantamento emana una luce accecante per un massimo di due volte al giorno su comando di chi lo impugna. Tutti coloro che si trovano entro 3 metri dallo scudo, eccetto chi lo impugna, devono effettuare un TiroSalvezza su Riflessi con DC 14 o restano Accecati per 1d4 round.

Essenza Creazione, Creare Oggetti Magici, CM 7, Prezzo bonus +1.

\textbf{Amorfa}\index{Amorfa}

Le armature con questa capacità speciale forniscono a chi le indossa bonus di +4 alle prove di Artista della Fuga. In aggiunta, una volta al giorno a comando, chi indossa l'armatura (insieme a qualsiasi equipaggiamento indossi) può assumere la forma di un liquido viscoso che è in grado di passare attraverso qualsiasi spazio nel quale potrebbe ragionevolmente scorrere del fango denso.

Mentre si usa questa capacità, la propria velocità viene ridotta della metà e si possono effettuare solo azioni di movimento.

Si può assumere questa forma per 1 minuto o finché non si spende un'azione per tornare alla propria forma naturale.

Un'armatura amorfa deve essere fatta principalmente di cuoio, stoffa o altro materiale organico e flessibile.

Essenza Trasformazione; CM 8; Creare Oggetti Magici Superiori, Prezzo
+4.500 mo.

\textbf{Animato}\index{Animato}

Come Azione, uno scudo animato può essere lasciato da solo a difendere il suo possessore. Per i 4 round successivi, lo scudo conferisce il suo bonus a chi lo ha lasciato e poi cade.

Mentre è animato, lo scudo conferisce il suo bonus di scudo e i bonus dati da qualsiasi altra capacità speciale abbia, ma non può intraprendere azioni di sua volontà, come quelle conferite dalle capacità accecante e sfondamento.

Mentre è animato, lo scudo condivide lo stesso spazio con il personaggio che lo ha attivato e lo accompagna, anche se il personaggio si muove tramite mezzi magici. Un personaggio con uno scudo animato continua a subire le penalità associate all'uso dello scudo sulla competenza magica.

Se chi lo lascia ha una mano libera, può afferrarlo quando gli effetti magici svaniscono come Azione reazione. Una volta che uno scudo è stato ripreso, non può essere animato nuovamente se non dopo almeno 4 round.

Essenza Movimento; CM 12, Creare Oggetti Magici Superiori, Prezzo bonus +2.

\textbf{Attirare Frecce}\index{Attirare Frecce}

Uno scudo dotato di questa capacità attrae su di sé le armi a distanza. E' dotato di un bonus di Difesa +1 contro le armi a distanza in quanto le armi da lancio o da tiro virano verso di esso. Inoltre ogni arma da lancio o da tiro diretta verso un bersaglio che si trova entro distanza di mischia da chi impugna lo scudo devia dal suo bersaglio per dirigersi invece verso il portatore dello scudo.

Se chi impugna lo scudo gode di copertura totale rispetto all'attaccante, l'arma da lancio o da tiro non viene deviata. Inoltre, coloro che attaccano chi impugna lo scudo con armi a distanza ignorano le probabilità di mancare che verrebbero normalmente applicate.

Le armi da lancio e da tiro che hanno un bonus magico superiore al bonus magico dello scudo non vengono deviate verso chi impugna lo scudo. Chi impugna lo scudo attiva e disattiva questa capacità con una parola di comando.

Essenza Movimento, CM 8, Creare Oggetti Magici, Prezzo bonus +1.

\textbf{Controllo dei Non Morti}\index{Controllo dei Non Morti}

Le armature e gli scudi del controllo dei Non Morti sono decorati da macabri ornamenti e orpelli. Chi indossa un'armatura o uno scudo con questa proprietà può controllare fino a 13 CR di non morti al giorno, come creati dall'Essenza Distruzione. All'alba di ogni giorno, colui che indossa l'armatura perde il controllo su qualsiasi non morto ancora ai suoi ordini. Le armature e gli scudi con questa capacità sembrano fatti d'ossa; questa peculiarità è puramente decorativa e non ha nessun altro effetto sull'armatura.

Essenza Distruzione, CM 13, Creare Oggetti Magici Superiori, Prezzo +49.000 mo.

\textbf{Denegante}\index{Denegante}

Una volta al giorno, quando colui che indossa l'armatura è bersaglio di un colpo critico o di un Attacco alle spalle effettuato con un'arma da mischia, può automaticamente negare questo critico o questo Attacco alle spalle e renderlo un attacco normale. Questa capacità può essere applicata solo alle armature pesanti.

Essenza Difesa; Creare Oggetti Magici Meravigliosi, CM 13, Prezzo bonus +4.

\textbf{Determinazione}\index{Determinazione}

Essenza Cura, Creare Oggetti Magici Superiori, CM 10, Prezzo +30.000
mo.

\textbf{Felpa}\index{Felpa}

Le armature dotate della capacità Felpa sono solitamente armature medie o pesanti. Una armatura Felpa non ha malus all'Agilità e riduce di 5 le penalità alla Competenza Magica.

Essenza Alterazione, Creare Oggetti Magici Superiori, CM 6, Prezzo
+2

\textbf{Forma Eterea}\index{Forma Eterea}

A comando, questo incantamento permette a chi indossa l'armatura di diventare Etereo (come per l'Essenza di Movimento) una volta al giorno. Il personaggio può rimanere Etereo per quanto tempo desidera ma, una volta tornato alla normalità, per quel giorno non può più diventare Etereo.

Essenza Movimento; CM 13; Creare Oggetti Magici Meravigliosi, Prezzo +49.000 mo.

\textbf{Fortificazione}\index{Fortificazione}

Questo scudo o armatura genera una forza magica che protegge con più efficacia le parti vitali di chi li indossa. Quando un colpo critico o un Attacco alle spalle vanno a segno su chi li indossa, c'è una probabilità che vengano negati e che il danno venga invece tirato normalmente.

Fortificazione Leggera: 25\%: bonus +1

Fortificazione Moderata: 50\%: bonus +3

Fortificazione Pesante: 75\%: bonus +5

Essenza Difesa, CM 13; Creare Oggetti Magici Superiori, Prezzo variabile (vedi sopra)

\textbf{Guardia}\index{Guardia}

Un scudo da guardia consente a chi lo brandisce di trasferire, in parte o per intero, il Bonus di Difesa ad una creatura adiacente (questo si somma a qualsiasi altro bonus).

Come azione immediata, all'inizio del suo turno e prima di utilizzare una qualsiasi delle altre capacità dello scudo, chi indossa lo scudo può scegliere un bersaglio adiacente e decidere in che misura conferire un bonus di Difesa andrà allocato su di esso all'inizio del suo turno.

Il suo bonus alla difesa del bersaglio dura fino al turno successivo di colui che indossa lo scudo, oppure finché quest'ultimo e il bersaglio non si trovano a più di distanza di mischia tra loro, a quel punto il bonus sul bersaglio termina e il Bonus di Difesa dello scudo riprende a funzionare normalmente per il suo portatore.

Questa capacità ha effetto solamente sul Bonus di Difesa conferito dallo scudo, e non al suo Bonus ai Tiri per Colpire (se presente) né su qualsiasi altra capacità dello scudo.

Essenza Difesa, CM 8, Creare Oggetti Magici, Prezzo bonus +1.

\textbf{Guardia Superiore}\index{Guardia Superiore}

Identica alla proprietà da guardia, eccetto che, come Azione immediata, colui che indossa lo scudo può scegliere un qualsiasi numero di alleati a lui adiacenti perché ricevano del bonus dello scudo.

Tutti gli alleati selezionati ricevono il medesimo bonus. Se un bersaglio degli effetti dello scudo si trova a più distanza di mischia dal portatore, gli effetti terminano per quello specifico bersaglio, ma non per gli altri eventuali bersagli.

Essenza Difesa, Livello 12; Creare Oggetti Magici Meravigliosi, Prezzo bonus +2.

\textbf{Giusto}\index{Giusto}

Un'armatura dotata di questa capacità spesso reca dei simboli arcani di Ljust o Simkjr istoriati o smaltati su di essa. A comando, una volta al giorno chi la indossa può invocare gli effetti dell'Essenza Cura ed Attacco. Questa capacità conferisce una competenza magica di 5 usabili tra le Essenze Cura ed Attacco al giorno. Non è possibile invocare il potere più di 3 volte al giorno.

Una armatura del giusto è sempre allineata verso il bene (energia positiva), al fine di determinare gli effetti dell'Essenza. Un'armatura del giusto fornisce un livello negativo permanente a qualsiasi creatura malvagia che tenti di indossarla. Questo livello negativo permane fintanto che l'armatura è indossata e svanisce non appena questa viene rimossa.

Questo livello negativo non può essere eliminato in alcun modo (nemmeno per effetto dell'Essenza Cura) fintanto che la creatura indossa l'armatura.

Essenza Cura ed Attacco, CM 10, CreareOggetti Magici Superiori, Costo +27.000 mo.

\textbf{Oscuro}\index{Oscuro}

Questa armatura reca spesso cesellati su di essa simboli arcani di Calicante o Cattalm. A comando, una volta al giorno chi la indossa può invocare gli effetti dell'Essenza Distruzione ed Attacco. Questa capacità conferisce una competenza magica di 5 usabili tra le Essenze Distruzione ed Attacco al giorno. Non è possibile invocare il potere più di 3 volte al giorno.

Una armatura dell'Oscuro è sempre allineata verso il male (energia negativa), al fine di determinare gli effetti dell'Essenza. Un'armatura dell'oscuro fornisce un livello negativo permanente a qualsiasi creatura buona che tenti di indossarla. Questo livello negativo permane fintanto che l'armatura è indossata e svanisce non appena questa viene rimossa. Questo livello negativo non può essere eliminato inalcun modo (nemmeno per effetto dell'Essenza Cura) fintanto che la creatura indossa l'armatura.

Essenza Distruzione ed Attacco, CM 10, Creare Oggetti Magici Superiori,
Costo +27.000 mo.

\textbf{Invulnerabilita'}\index{Invulnerabilita'}

Essenza Difesa, CM 18, Creare Oggetti Magici Meravigliosi, Prezzo bonus +3.

\textbf{Irrintracciabile}\index{Irrintracciabile}

Un'armatura irrintracciabile alleggerisce i passi di chi la indossa e ne camuffa l'aspetto. Le prove di Sopravvivenza per seguire le tracce del portatore subiscono penalità -5, e chi indossa l'armatura ottiene Bonus di Competenza +5 alle prove di Muoversi Silenziosamente. Soltanto le armature di cuoio o di pelle possono essere irrintracciabili.

Essenza Trasformazione, CM 5, Creare Oggetti Magici, Prezzo +7.500
mo.

\textbf{Mimetica}\index{Mimetica}

A comando, un'armatura di questo tipo muta la sua forma e appare come un normale set di vestiti. L'armatura conserva tutte le sue proprietà (compreso il peso) anche quando è mascherata. Solo Essenza di Rivelazione di livello potere 13 o più rivelano la reale natura dell'armatura trasformata.

Essenza Illusione, CM 10, Creare Oggetti Magici Superiori, Prezzo +2.700 mo.

\textbf{Nube Esplosiva}\index{Nube Esplosiva}

Questa armatura è abitualmente decorata con incisioni in rilievo di nubi tempestose e fulmini. Se l'avversario colpisce chi la indossa e infligge almeno 10 danni da elettricità, l'armatura diventa visibilmente caricata di energia per 1 round. Come azione immediata nel suo turno successivo chi la indossa può utilizzare un'Essenza di Attacco a tocco infliggendo 1d6 danni da elettricità per ogni 10 danni subiti dal portatore dal termine del suo turno precedente (massimo 5d6 per 50 o più danni da elettricità subiti).

Essenza Protezione e Attacco, CM 5, Creare Oggetti Magici, Prezzo
+ 5.000 mo.

\textbf{Ombra}\index{Ombra}

Questo tipo di armatura rende chi la indossa sfuocato ogni volta che tenta di nascondersi, fornendo un bonus di competenza +1d6 alle sue prove di Criminalità (Muoversi Silenziosamente). La penalità di armatura alla prova basate su Agilità si applica normalmente.

Essenza Illusione, CM 5, Creare Oggetti Magici, Prezzo +3.750 mo.

\textbf{Ombra Migliorata}\index{Ombra Migliorata}

Questo tipo di armatura rende chi la indossa sfuocato ogni volta che tenta di nascondersi, fornendo un bonus di competenza +2d6 alle sue prove di Muoversi Silenziosamente. La penalità di armatura alla prova basate su Agilità si applica normalmente.

Essenza Illusione, CM 10, Creare Oggetti Magici superiore, Prezzo
+15.000 mo.

\textbf{Ombra Superiore}\index{Ombra Superiore}

Questo tipo di armatura rende chi la indossa sfuocato ogni volta che
tenta di nascondersi, fornendo un bonus di competenza +3d6 alle sue
prove di Muoversi Silenziosamente. La penalità di armatura alla prova
basate su Agilità si applica normalmente.

Essenza Illusione, CM 15, Creare Oggetti Magici Meravigliosi, Prezzo
+33.750 mo.

\textbf{Resistenza all'Energia}\index{Resistenza all'Energia}

Questo tipo di armatura o scudo protegge contro un tipo di energia (acido, freddo, elettricità, fuoco o suono) ed è decorata da disegni che raffigurano l'elemento dal quale protegge. L'armatura o lo scudo assorbono i primi 10 danni di energia per attacco che verrebbero subiti normalmente da chi li indossa.

Essenza Protezione, CM 3, Creare Oggetti Magici, Prezzo +18.000 mo.

\textbf{Resistenza all'Energia Migliorata}\index{Resistenza all'Energia Migliorata}

Questo tipo di armatura o scudo protegge contro un tipo di energia
(acido, freddo, elettricità, fuoco o suono) ed è decorata da disegni
che raffigurano l'elemento dal quale protegge. L'armatura o lo scudo
assorbono i primi 20 danni di energia per attacco che verrebbero subiti
normalmente da chi li indossa

Essenza Protezione, CM 7, Creare Oggetti Magici superiore , Prezzo
+42.000 mo.

\textbf{Resistenza all'Energia Superiore}\index{Resistenza all'Energia Superiore}

Questo tipo di armatura o scudo protegge contro un tipo di energia (acido, freddo, elettricità, fuoco o suono) ed è decorata da disegni che raffigurano l'elemento dal quale protegge. L'armatura o lo scudo assorbono i primi 30 danni di energiaper attacco che verrebbero subiti normalmente da chi li indossa (come per l'Essenza di Protezione).

Essenza Protezione, CM 11, Creare Oggetti Magici Meravigliosi, Prezzo +66.000 mo.

\textbf{Riflettente}\index{Riflettente}

Questo scudo è simile a uno specchio. La sua superficie riflette perfettamente le immagini. Una volta al giorno può essere usato per riflettere una magia contro l'incantatore che l'ha lanciato. La prova di magia dell'Essenza riflessa non può essere superiore a 18.

Essenza Protezione, CM 14, Creare Oggetti Magici Meravigliosi, Prezzo
bonus +5.

\textbf{Scivolosa}\index{Scivolosa}

Un'armatura scivolosa sembra perennemente rivestita da una sottile patina di grasso. Fornisce un bonus di competenza +1d6 alle prove di liberarsi da prese e manette di chi la indossa. La penalità di armatura alla prova si applica normalmente.

Essenza Alterazione, CM 4, Creare Oggetti Magici, Prezzo +3.750 mo.

\textbf{Scivolosa Migliorata}

Come scivolosa, ma fornisce un bonus di competenza +2d6 alle prove di Criminalita'-

Essenza Alterazione, CM 10, Creare Oggetti Magici Superiori, Prezzo +10000 mo.

\textbf{Scivolosa Perfetta}

Come scivolosa, ma fornisce un bonus di competenza +3d6 alle prove di Criminalità.

Essenza Alterazione, CM 15, Creare Oggetti Magici meravigliosa, Prezzo +15750 mo.

\textbf{Della Forma Animale}\index{Della Forma Animale}

Chi indossa un'armatura con questa capacità conserva il bonus di protezione anche sotto un'Essenza di Trasformazione in animale. Le armature con questa capacità sembrano di solito ricoperti di foglie. Mentre sei trasformato l'armatura non è visibile.

Essenza Difesa e Trasformazione, CM 9, Creare Oggetti Magici Superiori, Prezzo bonus +3.

\textbf{Selvatica}\index{Selvatica}

Chi indossa uno scudo con questa capacità conserva il bonus di protezione anche sotto un'Essenza di Trasformazione in animale. Gli scudi con questa capacità sembrano di solito ricoperti di foglie. Mentre sei trasformato l'armatura non è visibile.

Essenza Difesa e Trasformazione, CM 9, Creare Oggetti Magici Superiori, Prezzo bonus +3.

\textbf{Sfondamento}\index{Sfondamento}

Questo scudo è fatto per effettuare un attacco con lo scudo. Uno scudo da sfondamento infligge danni come se fosse un'arma di due categorie di taglia più grande (uno scudo leggero quindi infliggerebbe 1d6 danni e uno scudo pesante infliggerebbe 1d8 danni). Lo scudo agisce come un'arma +1 quando viene usato per compiere attacchi con lo scudo. (Solo gli scudi leggeri e pesanti possono acquisire questa capacita').

Essenza Attacco, CM 8, Creare Oggetti Magici superiore, Prezzo bonus +1.

\textbf{Tocco Fantasma}\index{Tocco Fantasma}

Questa armatura o scudo sembrano quasi trasparenti. Il loro bonus magico si applica a pieno contro creature incorporee. L'armatura o lo scudo possono essere raccolti, spostati e indossati in qualsiasi momento dalle creature corporee ed incorporee. Le creature incorporee ottengono il bonus di difesa contro attacchi corporei ed incorporei, e mantengono comunque la capacità di passare attraverso gli oggetti solidi.

Essenza Movimento, CM 14, Creare Oggetti Magici Superiore, Prezzo bonus +2

\pagebreak

\section{Armi Magiche e Speciali}\index{Armi Magiche e Speciali}

\label{armi-magiche-e-speciali}

Le armi magiche sono armi potenziate per colpire più facilmente ed infliggere danni maggiori. Le armi magiche hanno dei bonus di potenziamento che variano da +1 a +5 e che si applicano sia ai tiri per colpire che ai tiri per i danni quando vengono usate in combattimento. 

Le armi magiche si suddividono in due categorie principali: da mischia e a distanza. Alcune di quelle elencate come armi da mischia (come il pugnale) possono venire usate anche come armi a distanza. In questo caso, il loro bonus di potenziamento viene applicato ad entrambi i tipi di attacco.

Alcune armi magiche possono essere dotate di capacità speciali. Le capacità speciali contano come bonus addizionali per determinare il prezzo di mercato dell'oggetto, ma non modificano i bonus di attacco o ai danni (tranne quando specificamente indicato).

Una sola arma non può possedere un bonus effettivo (il bonus di potenziamento più i bonus equivalenti delle capacità speciali, inclusi quelli derivanti da capacità ed Essenze del personaggio) superiore a +10.

Un'arma con una capacità speciale deve avere almeno bonus di potenziamento +1. Le armi non possono avere la stessa capacità speciale più di una volta.



\subsection{Tabella: Armi Magiche}\index{Armi Magiche}

\label{tabella-armi-magiche}

\begin{tabular}{ll}
	\toprule
	\textbf{Bonus Magico} & \textbf{Costo (mo)}\\
	+1{*} & 2000\\
	+2{*} & 8000\\
	+3{*} & 18000\\
	+4{*} & 32000\\
	+5{*} & 50000\\
	+6{*}{*}              & 72000\\
	+7{*}{*}              & 98000\\
	+8{*}{*}              & 128000\\
	+9{*}{*}              & 162000\\
	+10{*}{*}             & 200000\\
\end{tabular}

{*} Per le munizioni, questo prezzo vale per 50 frecce, quadrelli o proiettili per fionda.

	{*}{*} Un'arma non può avere bonus di potenziamento maggiore di +5. Usate queste linee guida per determinarne il prezzo quando vengono aggiunte delle Capacità Speciali.

\bigskip

\textbf{Armi a Distanza e Munizioni}

I bonus di delle armi a distanza e i bonus di delle munizioni non si cumulano tra loro. Si applica solo il più alto dei due bonus.

\textbf{Munizioni Magiche e Rottura}

Quando una freccia, un quadrello da balestra o un proiettile da fionda magici mancano il bersaglio, c'è una probabilità del 50\% che si rompano o che siano resi inutilizzabili. Una freccia, un quadrello o un proiettile magici che vanno a segno si distruggono automaticamente dopo aver inflitto il danno.

\textbf{Emanazione di Luce}

Il 30\% delle armi magiche emanano una luce intensa quanto quella emanata dall'Essenza Creazione Luce Livello Potere 11 (torcia). Queste armi luminose sono visibilmente magiche, non possono essere occultate quando vengono estratte e la loro luce non può essere spenta. Alcune delle armi descritte più avanti brillano sempre o mai, come specificato nella descrizione.

\textbf{Danneggiare le Armi Magiche}

un'arma magica può essere danneggiata solo da un'arma magica di pari o superiore grado.

\textbf{Attivazione}

Normalmente un personaggio sfrutta un'arma magica nella stessa maniera in cui sfrutta un'arma comune, vale a dire usandola per attaccare. Se un'arma ha una capacità speciale che necessita di essere attivata da chi la usa, allora è necessario che questi pronunci l'apposita parola di comando (2 Azioni). Un personaggio può attivare le Capacità Speciali di 50 munizioni nello stesso momento, sempre che ogni munizione abbia la medesima capacità.

\textbf{Armi Magiche e Colpi Critici}

Alcune qualità delle armi, e alcune armi specifiche hanno un ulteriore effetto sui colpi critici. Questo effetto speciale agisce anche contro creature che ignorano i colpi critici. Con un tiro critico riuscito, applicate l'effetto speciale ma non il danno aggiuntivo.

\textbf{Armi per Creature Insolite}

Il costo delle armi per le creature che non sono né Medie né Piccole varia. Il costo della di qualsiasi potenziamento magico rimane lo stesso.

\bigskip

\textbf{Capacità Speciali}

\bigskip

\begin{tabular}{ll}
	\toprule
	\textbf{Capacità speciale}    & \textbf{Costo (mo)}\\
	Conduttiva \index{Conduttiva} & bonus +1\\
	Corrosiva\index{Corrosiva}    & bonus +1\\
	Astuzia \index{Astuzia}       & bonus +1\\
	Furiosa \index{Furiosa}       & bonus +1\\
	Fiammagrigia \index{Fiammagrigia}             & bonus +1\\
	Cacciatore \index{Cacciatore} & bonus +1\\
	Giurista \index{Giurista}     & bonus +1\\
	rasformante \index{Trasformante}              & +10000\\
	Munizione Fantasma \index{Munizione Fantasma} & bonus +2000 mo\\
	Prensile \index{Prensile}     & bonus +2.000 mo\\
	Volante \index{Volante}       & bonus +5\\
	Anatema\index{Anatema}        & bonus +1\\
	Difensiva\index{Difensiva}    & bonus +1\\
	Infuocata \index{Infuocata}   & bonus +1\\
	Gelida \index{Gelida}         & bonus +1\\
	Folgorante \index{Folgorante} & bonus +1\\
	Tocco Fantasma \index{Tocco Fantasma}         & bonus +1\\
	Pietosa \index{Pietosa}       & bonus +1\\
	Accumula Magie \index{Accumula Magie}         & bonus +1\\
	Tonante \index{Tonante}       & bonus +1\\
	Distruzione \index{Distruzione}               & bonus +3\\
	Ferimento \index{Ferimento}   & bonus +2\\
	Velocità \index{Velocita'}    & bonus +3\\
	Energia Luminosa \index{Energia Luminosa}     & bonus +4\\
	Danzante \index{Danzante}     & bonus +4\\
	Vorpal** \index{Vorpal}       & bonus +5\\
	Incendiaria \index{Incendiaria}               & bonus +2\\
	Nata dalla Furia \index{Nata dalla Furia}     & bonus +2\\
\end{tabular}

\subsection{Tabella: Capacità Speciali delle Armi a Distanza}

\label{tabella-capacita-speciali-delle-armi-a-distanza}

\begin{tabular}{ll}
	\toprule
	\textbf{Capacita'}       & \textbf{Costo (mo)}\\
	Conduttiva \index{Conduttiva}            & bonus +1\\
	Corrosiva\index{Corrosiva}               & bonus +1\\
	Astuzia \index{Astuzia}  & bonus +1\\
	Giurista\index{Giurista} & bonus +1\\
	Anatema \index{Anatema}  & bonus +1\\
	Distanza\index{Distanza} & bonus +1\\
	Infuocata\index{Infuocata}               & bonus +1\\
	Gelida \index{Gelida}    & bonus +1\\
	Pietosa \index{Pietosa}  & bonus +1\\
	Ritornante \index{Ritornante}            & bonus +1\\
	Folgorante\index{Folgorante}             & bonus +1\\
	Ricercante \index{Ricercante}            & bonus +1\\
	Tonante \index{Tonante}  & bonus +1\\
	Velocità \index{Velocita'}               & bonus +3\\
	Energia Luminosa\index{Energia Luminosa} & bonus +4\\
\end{tabular}

\bigskip

Da aggiungere al bonus di potenziamento della Tabella: Armi, per determinare
il prezzo di mercato totale. Il bonus di potenziamento di un'arma
e i bonus equivalenti delle capacità speciali non possono superare
il totale di +10.

\subsection{Descrizione delle Capacità Speciali delle Armi Magiche}

\label{descrizione-delle-capacita-speciali-delle-armi-magiche}

Un'arma con una capacità speciale deve avere almeno bonus di potenziamento +1.

\textbf{Pericolosa}\index{Pericolosa}

Questa capacità migliora l'EDX dell'arma di 1. Se un'arma prima esplodeva il suo danno con 8 adesso migliora la possibilità portandolo a 7.

Essenza Attacco, CM 10, Creare Oggetti Magici Superiori, Prezzo bonus +2.

\textbf{Anatema}\index{Anatema}

Un'arma anatema eccelle nell'attaccare un determinato tipo o sottotipo di creatura. Contro il nemico giurato, il suo bonus di potenziamento effettivo aumenta di +2 rispetto al normale. L'arma, inoltre, infligge +2d6 danni addizionali contro questo tipo di nemico.

Essenza Attacco, CM 8, Creare Oggetti Magici superiore, Prezzo bonus +1.

\textbf{Conduttiva}\index{Conduttiva}

Un'arma conduttiva è in grado di trasferire una Essenza spontanea posseduta dal personaggio attraverso la sua lama (esempio Incanalare Energia). Questa capacità speciale dell'arma può essere utilizzata solamente una volta per round.

Essenza Movimento, CM 8, Creare Oggetti Magici, Prezzo bonus +1.

\textbf{Corrosiva}\index{Corrosiva}

A comando, un'arma corrosiva si ricopre di uno strato di acido che infligge 1d6 danni aggiuntivi da acido quando colpisce il bersaglio. L'acido non danneggia chi la impugna. L'effetto permane fino a quando non viene impartito un nuovo comando.

Essenza Attacco, CM 10, Creare Oggetti Magici, Prezzo bonus +1.

\textbf{Danzante}\index{Danzante}

Con 2 Azioni, un'arma danzante può essere lasciata libera in modo che combatta da sola. L'arma combatte per 4 round usando il CA di colui che l'ha lasciata libera e poi cade a terra.

Mentre danza la persona che l'ha rilasciata non è considerata armata con quell'arma, ma in tutti gli altri casi viene considerata impugnata o custodita dalla creatura per determinare tutte le manovre e gli effetti mirati contro un oggetto.

Mentre danza, occupa lo stesso spazio del personaggio che l'ha attivata e può attaccare i nemici adiacenti (le armi con portata possono attaccare gli avversari fino a distanza 3 metri).

Rimane sempre accanto alla persona che l'ha liberata, anche se si sposta con mezzi fisici o magici. Se colui che l'ha lasciata libera ha una mano libera può riprendere l'arma che sta attaccando da sola, come Azione reazione, ma una volta ripresa, la spada non potrà più danzare (attaccare da sola) prima di 4 round.

Essenza Movimento, Livello 15, Creare Oggetti Magici superiore, Prezzobonus +4.

\textbf{Difensiva}\index{Difensiva}

Un'arma difensiva permette a chi la impugna di trasferire una parte o tutto il bonus magico alla Difesa come un bonus cumulabile ad eventuali altri bonus. Come Azione reazione, chi la impugna può scegliere come disporre del bonus di potenziamento dell'arma all'inizio del round.

Essenza Difesa, CM 8, Creare Oggetti Magici, Prezzo bonus +1.

\textbf{Distanza}\index{Distanza}

Questa capacità speciale può essere messa solo su armi a distanza, aumentando di 6 metri l'incremento di gittata.

Essenza Movimento, CM 6, Creare Oggetti Magici, Prezzo bonus +1.

\textbf{Distruzione}\index{Distruzione}

Un'arma della distruzione è la rovina di tutti i Non Morti. Ogni creatura non morta colpita in combattimento deve superare un Tiro Salvezza su Volontà con DC 14 o viene distrutta. Se il Tiro Salvezza riesce l'arma fa doppio danno (arma + bonus magici dell'arma). Un'arma della distruzione deve essere un'arma da mischia contundente (tipo B).

Essenza Cura o Attacco, CM 14, Creare Oggetti Magici superiore, Prezzo bonus +3

\textbf{Energia Luminosa}\index{Energia Luminosa}

Un'arma di energia luminosa si trasforma per la maggior parte in luce, anche se ciò non influisce sul suo peso. Fornisce sempre luce come una torcia (distanza 3 metri).

Un'arma di energia luminosa ignora la materia non vivente. Le armature e scudi contano solo per il bonus magico che hanno perché contro di loro l'arma passa attraverso (Agilità, deviazione, schivare e altri bonus simili si applicano normalmente.) U

n'arma di energia luminosa non può ferire Costrutti ed oggetti. Un non-morto subisce danno massimo dal dado dell'arma.

Questa proprietà si può applicare sono ad armi da mischia, da lancio e munizioni.

Essenza Trasformazione, CM 16, Creare Oggetti Magici Meravigliosi, Prezzo bonus +4.

\textbf{Ferimento}\index{Ferimento}

Un'arma da ferimento infligge 1 danno da Sanguinamento quando colpisce una creatura. Danni multipli di quest'arma aumentano il danno da Sanguinamento. Le creature sanguinanti subiscono il danno da sanguinamento all'inizio del loro turno.

Il Sanguinamento può essere fermato con una prova di Guarire con DC 15 o con un'Essenza di Cura qualsiasi che curi le ferite. Un colpo critico non aggiunge ulteriore danno da Sanguinamento.

Le creature immuni ai colpi critici sono immuni ai danni da Sanguinamento inflitti da quest'arma.

Essenza Distruzione, CM 10, Creare Oggetti Magici, Prezzo bonus +2.

\textbf{Fiammagrigia}\index{Fiammagrigia}

Quest’arma risponde all'energia positiva o negativa incanalata. Quando chi la impugna usa incanalare energia nell'arma, questa si accende di una strana fiamma grigia che illumina come una torcia, aumenta di +1 il bonus di potenziamento dell'arma, ed infligge +1d6 danni alle creature da essa colpite.

Questa fiamma dura 1 round per ogni d6 danni o cure che incanalare normalmente fornisce.

Quando viene caricata con energia positiva, la fiamma è di colore grigio argenteo, le creature buone sono immuni al danno addizionale inflitto dall'arma e l’arma viene considerata come buona e di argento ai fini di superare la riduzione del danno.

Quando viene caricata con energia negativa, la fiamma è di color grigio cenere, le creature malvagie sono immuni al danno addizionale inflitto dall'arma, e l'arma viene considerata come malvagia e di ferro freddo ai fini di superare la riduzione del danno.

Essenza Cura o Distruzione, CM 6, Creare Oggetti Magici, Prezzo bonus +1.

\textbf{Folgorante}\index{Folgorante}

A comando, un'arma folgorante si ricopre di energia elettrica, che non danneggia chi la impugna ed ogni suo colpo andato a segno infligge 1d6 danni addizionali da elettricità. L'effetto rimane attivo finché non viene disattivato con un altro comando.

Essenza Attacco, CM 8, Creare Oggetti Magici superiore, Prezzo bonus +1

\textbf{Gelida}\index{Gelida}

A comando, un'arma gelida emana un freddo glaciale. Il freddo non danneggia chi impugna l'arma e ogni colpo andato a segno infligge 1d6 danni addizionali da freddo. L'effetto rimane attivo finché non viene disattivato con un altro comando.

Essenza Attacco, CM 8, Creare Oggetti Magici superiore, Prezzo bonus +1

\textbf{Incendiaria}\index{Incendiaria}

Un'arma incendiaria funziona come arma Infuocata che fa anche Prendere Fuoco al bersaglio colpito da un colpo critico. Il bersaglio non ottiene un Tiro Salvezza su Agilità per evitare di Prendere Fuoco, ma può effettuare un Tiro Salvezza ogni round nel suo turno per spegnere le fiamme. La capacità Infuocata deve essere attiva affinché l'arma incendi i nemici.

Essenza Attacco, CM 12, Creare Oggetti Magici superiore, Prezzo bonus +2.

\textbf{Infuocata}\index{Infuocata}

A comando un'arma infuocata prende fuoco. Le fiamme non danneggiano chi impugna l'arma e infliggono 1d6 danni addizionali da fuoco per ogni colpo andato a segno. L'effetto rimane attivo finché non viene disattivato con un altro comando.

Essenza Attacco, CM 8, Creare Oggetti Magici, Prezzo bonus +1.

\textbf{Lancio}\index{Lancio}

Questa capacità può essere posta solo su armi da mischia. Un'arma da mischia incantata con questa capacità acquisisce una gittata di lancio di 3 metri e può essere lanciata senza malus se competente nel suo uso normale.

Essenza Movimento, CM 5, Creare Oggetti Magici, Prezzo bonus +1.

\textbf{Munizione Fantasma}\index{Munizione Fantasma}

Questa capacità può essere conferita solo alle munizioni (frecce o dardi). Una munizione con questa capacità speciale delle armi si dissolve 1 round dopo essere stata scagliata. In aggiunta, se il proiettile colpisce un bersaglio, la ferita causata si richiude non appena la munizione si disintegra. Il proiettile infligge danni normalmente, ma non lascia alcuna traccia visibile di violenza.

Il prezzo si riferisce a 50 munizioni fantasma.

Essenza Trasformazione, CM 7, Creare Oggetti Magici, Prezzo +1.000.

\textbf{Nata dalla Furia}\index{Nata dalla Furia}

Un'arma nata dalla furia attinge potere dalla rabbia e dalla frustrazione provate da colui chi la impugna quando si batte contro un avversario che si rifiuta di morire.

Ogni volta che chi la impugna infligge danni ad un avversario con quest'arma, il suo Bonus aumenta di +1 quando effettua attacchi contro quel nemico (Bonus totale massimo di +5).

Questo Bonus addizionale svanisce se l'avversario muore, se colui che impugna l'arma la utilizza per colpire un altro avversario (e ricomincia il ciclo), oppure quando è trascorsa 1 ora. Solo le armi da mischia possono essere dotate della capacità nata dalla furia.

Essenza Charme e Attacco, Livello 7, Creare oggetti magici superiore, Prezzo bonus +2.

\textbf{Pietosa}\index{Pietosa}

L'arma infligge 1d6 danni addizionali ma tutto il danno è non letale. A comando, l'arma disattiva questa capacità fino a quando non le viene ordinato di riattivarla (permettendole di infliggere danni letali, ma senza i danni addizionali derivanti dalla capacita').

Essenza Attacco, CM 5, Creare Oggetti Magici, Prezzo bonus +1.

\textbf{Prensile}\index{Prensile}

Questa capacità può essere conferita solo alle fruste. Una frusta prensile puo', come azione veloce, aggrapparsi a un oggetto come se fosse un rampino. La frusta può poi essere usata per scalare superfici o dondolare attraverso una stanza o qualsiasi area all'aperto.

Essenza Movimento, CM 7, Creare Oggetti Magici, Prezzo +2.500.

\textbf{Ricercante}\index{Ricercante}

Solo le armi a distanza possiedono la capacità ricercante. L'arma vira verso il suo bersaglio, negando qualsiasi malus che si potrebbe applicare, come quella dovuta alla copertura. Il possessore deve comunque mirare l'arma nella zona di mischia giusta. Le frecce sparate per errore in uno spazio vuoto, per esempio, non virano per colpire gli avversari Invisibili, se ce n'è qualcuno nelle vicinanze.

Essenza Rivelazione, CM 12, Creazione oggetti magici superiore, Prezzo bonus +1.

\textbf{Ritornante}\index{Ritornante}

Questo incantamento può essere posto solo su armi che possono essere
lanciate.

Un'arma ritornante con questa capacità può ritornare indietro a chi l'ha lanciata fluttuando nell'aria. Ritorna appena prima che inizi il turno successivo di chi l'ha lanciata, e in questo modo è pronta per essere usata di nuovo in quel turno.

Riprendere un'arma ritornante mentre torna indietro è un'Azione immediata.

Se il personaggio non può afferrarla, o se il personaggio si è spostato dopo averla lanciata, l'arma cade a terra nella zona di mischia dalla quale è stata lanciata.

Essenza Movimento, CM 7, Creare Oggetti Magici, Prezzo bonus +1.

\textbf{Tocco Fantasma}\index{Tocco Fantasma}

Un arma dotata del talento Tocco Fantasma è in grado di colpire creature eteree infliggendo pieno danno.

\textbf{Tonante}\index{Tonante}

Un'arma tonante crea un tremendo frastuono simile a quello di un tuono, quando mette a segno un colpo critico. L'energia sonora non danneggia chi tiene in mano l'arma e infligge 1d8 danni sonori addizionali in caso di critico. Chi è soggetto ad un colpo critico da un'arma tonante deve effettuare un Tiro Salvezza su Tempra con DC 14 o resta sordo in modo permanente.

Essenza Distruzione, CM 5, Creare Oggetti Magici, Prezzo bonus +1.

\textbf{Trasformante}\index{Trasformante}

Questa capacità può essere conferita solo ad un'arma da mischia. Un'arma trasformante altera la sua forma a comando di chi la impugna, diventando una qualsiasi altra arma da mischia dotata della medesima forma generica e durezza dell'originale.

Ad esempio, una spada lunga trasformante Media può assumere la forma di una qualsiasi altra arma di mischia ad una mano Media, come una scimitarra, un mazzafrusto od un tridente, ma non un'arma da mischia leggera o a due mani Media.

Se lasciata incustodita, l'arma ritorna alla sua forma originaria.

Essenza Trasformazione, CM 10, Creare Oggetti Magici superiore, Prezzo +10.000 mo.

\textbf{Velocita'}\index{Velocita'}

Quando compie usa due azioni per l'attacco, il possessore di un'arma di velocità può compiere un attacco addizionale con l'arma. L'attacco usa il CA pieno di chi la impugna, più qualsiasi modificatore appropriato alla situazione. Questo beneficio non è cumulabile con Essenze di Movimento simili

Essenza Movimento, CM 7, Creare Oggetti Magici, Prezzo bonus +3.

\textbf{Volante}\index{Volante}

Questa capacità speciale può essere conferita solo alle armi da mischia. Un'arma volante funziona come un'arma danzante, ma mentre danza può essere direzionata in modo che attacchi nemici a distanza di 3 metri.

In aggiunta, in qualsiasi momento (persino quando l'arma non sta danzando), l'ultimo ad aver estratto l'arma può farla ritornare a sé come azione veloce.

L'arma vola fino a un massimo di 150 metri a round per tornare dal suo proprietario, compiendo un tentativo di Spezzare a round per penetrare qualsiasi barriera che non possa aggirare o contro qualsiasi creatura che provi a trattenerla o bloccarla (liberandosi in caso di successo).

Quando ritorna dal suo proprietario, l'arma vola in una mano libera o, se non ce l'ha, cade davanti ai suoi piedi. Se l'arma non riesce a tornare entro 4 round, cade inerte.

Essenza Movimento, CM 16, Creare Oggetti Magici Meravigliosi, Prezzo bonus +5.

\textbf{Vorpal}\index{Vorpal}

Questa temuta e potente capacità permette all'arma di tagliare la testa di coloro che colpisce. Dopo aver ottenuto un 17 o più naturale con i primi 3d6 del check di arma, l'arma stacca la testa dell'avversario (se ne ha una) dal corpo. Alcune creature, come molte Aberrazioni o tutte le Melme, non hanno testa. Altre, come i Costrutti o i Non Morti (a parte i Vampiri), non sono influenzate dalla perdita della testa. La maggior parte delle altre creature, invece, muore quando la testa viene tagliata. Un'arma vorpal deve essere un'arma da mischia tagliente.

Essenza Attacco, CM 18, Creazione oggetti magici meravigliosi, Prezzo bonus +5.

\pagebreak

\section{Anelli Magici Speciali}\index{Anelli Magici Speciali}

\label{anelli-magici-speciali}

Un anello concede poteri magici, pochi usano delle cariche e molto spesso il potere è permanente.

Gli anelli si adattano alla grandezza delle dita da piccolissimi a giganteschi, ma non per questo hanno meno o più poteri. Tutti possono portare fino a 2 anelli, oltre i due anelli si subiscono 1d6 di danno per anello oltre il secondo a round e l'anello aggiunto non funziona.

Solitamente un anello è un oggetto senza un particolare peso o aspetto tranne quando descritto diversamente.

Un anello ha LP 25 o più.

Un anello viene attivato tramite un comando vocale se non descritto diversamente.

Per gli anelli che hanno delle Essenze troverete direttamente il livello di potere massimo usabile, che vale anche come TS per resistere agli effetti.

Se devi generare casualmente un anello tira un d100, confronta il risultato con la "Tabella livello Anelli" in base al tipo di anello che trovi ritira 1d100 sulla Tabella degli Anelli.

\bigskip

\textbf{Tabelle livello di Anelli}

\medskip
\begin{tabular}{ll}
	\toprule
	\textbf{d100} & \textbf{Risultato}\\
	1             & Speciale\\
	2-85          & Normale\\
	86-95         & Superiore\\
	96-99         & Maggiore\\
	100           & Maledetto\\
\end{tabular}

\bigskip

Un anello Speciale è intelligente e tira 1 volta su Maggiore ed 1 volta su Superiore Maggiore.

Un anello con cariche non può essere speciale

\pagebreak

\textbf{Tabella degli Anelli}

\medskip

\begin{tabular}{lllll}
	\toprule
	\textbf{Normale} & \textbf{Superiore} & \textbf{Maggiore} & \textbf{Nome Anello}    & \textbf{Costo}\\
	0-18             & -  & - & Anello protezione \index{Anello protezione}+1           & 2000\\
	19-28            & -  & - & Anello protezione \index{Anello protezione}             & 2500\\
	29-36            & -  & - & Anello del sostentamento\index{Anello del sostentamento}& 2500\\
	37-44            & -  & - & Anello dello Scalare \index{Anello dello Scalare}       & 2500\\
	45-52            & -  & - & Anello del Saltare \index{Anello del Saltare}           & 2500\\
	53-60            & -  & - & Anello del Nuotare \index{Anello del Nuotare}           & 2500\\
	61-70            & 01-05              & - & Anello dello scudo mentale \index{Anello dello scudo mentale}           & 8000\\
	71-75            & 06-08              & - & Anello protezione +2    & 8000\\
	81-85            & 19-23              & - & Anello dello scudo di forza\index{Anello dello scudo di forza}          & 8500\\
	86-90            & 24-28              & - & Anello dell'Ariete      & 8600\\
	-& 29-34              & - & Anello dello scalare superiore          & 10000\\
	-& 35-40              & - & Anello del saltare migliorato           & 10000\\
	-& 41-46              & - & Anello del nuotare migliorato           & 10000\\
	91-93            & 47-50              & - & Anello dell'amicizia con gli animali \index{Anello dell'amicizia con gli animali}       & 10800\\
	94-96            & 51-56              & 01-02             & Anello resistenza energia minore \index{Anello resistenza energia}      & 12000\\
	97-98            & 57-61              & - & Anello del potere del camaleonte \index{Anello del potere del camaleonte}               & 12700\\
	99-100           & 62-66              & - & Anello di camminare sull'acqua\index{nello di camminare sull'acqua}     & 18000\\
	-& 67-71              & 03-07             & Anello protezione +3    & 18000\\
	-& 72-76              & 08-10             & A scelta del Narratore  & 18000\\
	-& 77-81              & 11-15             & Anello dell'Invisibilità \index{nello dell'Invisibilita'}               & 20000\\
	-& 82-85              & 16-19             & Anello della Essenza I \index{Anello della Essenza}     & 20000\\
	-& 86-90              & 20-25             & Anello dell'Eludere \index{Anello dell'Eludere}         & 25000\\
	-& 91-93              & 26-28             & Anello visione raggi x\index{Anello visione raggi x}    & 25000\\
	-& 94-97              & 29-32             & Anello dell'intermittenza \index{Anello dell'intermittenza}             & 27000\\
	-& 98-100             & 33-39             & Anello della resistenza all'energia maggiore            & 28000\\
	-& -  & 40-49             & Anello protezione +4    & 32000\\
	-& -  & 50-55             & Anello della Essenza II & 40000\\
	-& -  & 56-60             & Anello della libertà di movimento \index{Anello della libertà di movimento}             & 40000\\
	-& -  & 61-63             & Abello di resistenza all'energia superiore              & 44000\\
	-& -  & 64-65             & Anello di protezione +5 & 50000\\
	-& -  & 66-70             & Anello delle stelle cadenti \index{nello delle stelle cadenti}          & 50000\\
	-& -  & 71-74             & Tira due volte su Superiore             & 50000\\
	-& -  & 75-79             & Anello della Essenza III& 70000\\
	 & -  & 80-83             & Anello della Telecinesi \index{Anello della Telecinesi} & 75000\\
	-& -  & 84-86             & Anello della rigenerazione \index{Anello della rigenerazione}           & 90000\\
	 &    & 87-88             & Anello di Riflettere Essenza\index{Anello di Riflettere Essenza}        & 100000\\
	 &    & 89-91             & Anello della Essenza IV & 100000\\
	 &    & 92-93             & Anello dei tre desideri \index{Anello dei tre desideri} & 120000\\
	 &    & 94& Anello richiama del Djinni \index{Anello richiama del Djinni}           & 125000\\
	 &    & 95& Anello del Comando degli Elementari della Terra \index{Anello del Comando degli Elementari della Terra} & 200000\\
	 &    & 96& Anello del Comando degli Elementari della Aria\index{Anello del Comando degli Elementari della Aria}    & 200000\\
	 &    & 98& Anello del Comando degli Elementari della Fuoco \index{Anello del Comando degli Elementari della Fuoco} & 200000\\
	 &    & 99& Anello del Comando degli Elementari della Acqua \index{Anello del Comando degli Elementari della Acqua} & 200000\\
	 &    & 100               & Tira 2 volte su normale e 2 volte su superiore          & 250000\\
\end{tabular}

\pagebreak

\textbf{Anello Anti Vomito}\index{Anello Anti Vomito}

Essenza Cura 13, costo 200 mo

L'anello concede a chi lo indossa +1d6 sulle prove contro vomito e offre una straordinaria resistenza all'alcool.

\textbf{Anello di Protezione}\index{Anello di Protezione}

Essenza Difesa 18, 2000 mo

Prezzo: 2.000 mo (+1), 8.000 mo (+2), 18.000 mo (+3), 32.000 mo (+4),
50.000 mo (+5)

Questo Anello offre costantemente una protezione Magica sotto forma di bonus alla difesa che varia da +1 a +5.

\textbf{Anello di Caduta Morbida}\index{Anello di Caduta Morbida}

Essenza Movimento 10, 2200 mo

Questo anello è decorato sul suo bordo con una serie di piume. Conferisce gli stessi effetti di un incantesimo di Caduta Piuma (Movimento LP11), che si attivano immediatamente se chi lo indossa cade per più di 1,5 m.

\textbf{Anello del Sostentamento}\index{Anello del Sostentamento}

Essenza Creazione 15, 2500 mo

Questo anello fornisce costantemente a chi lo indossa il necessario nutrimento per vivere. L'anello è anche in grado di rinfrancare il suo corpo e la sua mente in modo che a chi lo indossa siano necessarie solo 2 ore di sonno al giorno per ottenere i benefici che normalmente otterrebbe con 8 ore di riposo. L'anello dev'essere indossato per un'intera settimana prima che cominci a funzionare. Se viene rimosso per qualsiasi motivo, il possessore deve di nuovo indossarlo per una settimana prima che ricominci a funzionare.

\textbf{Anello di Scalare}\index{Anello di Scalare}

Essenza Alterazione 15, 2500 mo

Questo è un semplice anello di corda che si lega al dito. Conferisce a chi lo indossa bonus di competenza +1d6 alle prove di Scalare.

\textbf{Anello del Saltare}\index{Anello del Saltare}

Essenza Alterazione 15, 2500 mo

Questo anello di gomma permette a chi lo indossa di saltare meglio, concedendo bonus di competenza +1d6 a tutte le prove di Acrobazia effettuate per saltare in alto e in lungo.

\textbf{Anello di Nuotare}\index{Anello di Nuotare}

Essenza Alterazione 15, 2500 mo

Questo anello d'argento è decorato con i disegni di creature marine lungo tutto il bordo. Conferisce continuamente a chi lo indossa bonus di competenza +1d6 alle prove di Nuotare.

\textbf{Anello di Scudo Mentale}\index{Anello di Scudo Mentale}

Essenza Protezion 18, 8000 mo

Questo anello,di solito finemente lavorato e realizzato in oro puro, rende chi lo indossa costantemente immune ad incantesimi di Charme di livello potere 13 o meno

\textbf{Anello dello Scudo di Forza}\index{Anello dello Scudo di Forza}

Essenza Creazione 18, 8500 mo

Questo anello, forgiato come una semplice banda di ferro, genera uno scudo di forza delle dimensioni e della forma di uno scudo, che rimane legato all'anello e può essere impugnato da chi lo indossa come se fosse uno scudo pesante (Difesa +3). Questa speciale creazione, dato che non pesa e non ingombra, non ha penalità di armatura alla prova, né probabilità di fallimento degli incantesimi. Può essere attivata e disattivata a piacere come Azione reazione.

\textbf{Anello dell'Ariete}\index{Anello dell'Ariete}

Essenza Attacco 21, 8600 mo

L'anello dell'ariete è un anello decorato e forgiato in ferro o in una lega di ferro che porta come decorazione la piccola testa di un ariete. Chi lo indossa può ordinare all'anello di far partire una forza pari alla carica di un ariete, che si manifesta in una forma a malapena discernibile simile a una testa d'ariete o di capra.

Questa forza colpisce un unico bersaglio, infliggendo 1d6 danni se viene consumata una carica, 2d6 danni se ne vengono consumate 2 o 3d6 se ne vengono consumate 3 (il massimo).

L'attacco va considerato come un attacco a distanza, con una gittata massima di 9 metri e nessuna penalità per la distanza. La forza del colpo è notevole, e i soggetti colpiti dall'anello vengono considerati colpiti da una spinta (TS su Tempra DC 22 o indietreggiare a distanza di 3 metri), se si trovano a distanza media o meno da chi lo indossa. L'ariete è di taglia Grande e ha Potenza 8.

Oltre alla sua modalità di attacco, l'anello dell'ariete ha anche il potere di aprire porte come se fosse un personaggio con Potenza 1. Se si spendono 2 cariche, l'effetto è equivalente a quello di un personaggio di Potenza 2, e se se ne spendono 3, a un personaggio di Potenza 3. Un anello appena creato è dotato di 50 cariche. Una volta esaurite tutte e 50, l'anello diventa un normale oggetto non magico.

\textbf{Anello di Scalare, Nuotare, Saltare Migliorato}

Essenza Alterazione 18, 10000 mo

Dall'aspetto assolutamente identico agli anelli con potere base, queste
versioni concedono un bonus di 2d6 alle rispettive prove.

\textbf{Anello di Amicizia con gli Animali}\index{Anello di Amicizia con gli Animali}

Essenza Charme 15, 10800 mo

Un anello dell'amicizia con gli animali è forgiato con decori di tipo animale. A comando, questo anello influenza un animale come se chi lo indossa avesse lanciato su di lui Essenza Charme LP 15.

\textbf{Anello di Resistenza all'Energia}\index{Anello di Resistenza all'Energia}

Essenza Protezione 18-21-24, 12.000 mo (minore), 28.000 mo (maggiore),
44.000 mo (superiore)

Questo anello protegge costantemente chi lo indossa dai danni di un tipo specifico di energia: acido, elettricità, freddo, fuoco o suono (a scelta del creatore dell'oggetto; determinarlo a caso se fa parte di un tesoro ritrovato). Quando chi lo indossa subirebbe tali danni, bisogna sottrarre il valore di resistenza dell'anello ai danni inflitti.

Un anello di resistenza all'energia minore conferisce 10 punti di resistenza. Un anello di resistenza all'energia maggiore conferisce 20 punti di resistenza. Un anello di resistenza all'energia superiore conferisce 30 punti di resistenza.

\textbf{Anello di Potere del Camaleonte}\index{Anello di Potere del Camaleonte}

Essenza Illusione 15, 12700 mo

Come Azione immediata, chi indossa questo anello ottiene la capacità di confondersi magicamente con l'ambiente circostante, guadagnando un bonus di competenza +2d6 alle prove di Furtività. Con 2 Azioni inoltre, può usare l'Essenza di Illusione per cambiare il suo aspetto rimanendo nella taglia)

\textbf{Anello di Camminare sull'Acqua}\index{Anello di Camminare sull'Acqua}

Essenza Movimento 18, 15000 mo

Questo anello è spesso ricavato da un corallo o da un metallo blu ornato da motivi marini. Permette a chi lo indossa può camminare sull'acqua.

\textbf{Anello Accumula Essenza}\index{Anello Accumula Essenza}

Essenza Varie 18-24-28, 18000 mo

Un anello accumula incantesimi contiene 3 essenze con LP fino a 13, 18, 21 a secondo del potere.

L'anello informa magicamente chi lo indossa su quali incantesimi vi sono attualmente contenuti.

\textbf{Anello di Invisibilita'}\index{Anello di Invisibilita'}

Essenza Illusione 21, 20000 mo

Attivando questo semplice anello d'argento, chi lo indossa può beneficiare degli effetti dell'Essenza di Illusione rendendosi invisibile, appena attacca diviene visibile.

\textbf{Anello delle Essenze}\index{Anello delle Essenze}

Essenze Protezione e Attacco, 18,24,30,34 20.000 mo (I), 40.000 mo
(II), 70.000 mo (III), 100.000 mo (IV)

Questo anello speciale può essere di quattro tipi diversi (anello delle Essenze I, anello delle Essenz II, anello delle Essenz III, e anello delle Essenz IV), tutti destinati all'utilizzo da parte di incantatore. Oltre al bonus indicato concedono +1d6 alle prove di concentrazione.

un anello della stregoneria I concede +1d6 alla prova di Competenza Magica per lanciare una Essenza specifica

un anello della stregoneria II concede +2d6 alla prova di Competenza Magica per lanciare una Essenza specifica

un anello della stregoneria III concede +3d6 alla prova di Competenza Magica per lanciare una Essenza specifica

un anello della stregoneria IV concede +4d6 alla prova di Competenza Magica per lanciare una Essenza specifica

\textbf{Anello di Eludere}\index{Anello di Eludere}

Essenza Difesa 24 , 25.000 mo

Questo anello conferisce costantemente a chi lo indossa l'agilità di evitare i danni come se fosse dotato della abilità di Eludere. Ogni volta che chi lo indossa supera un Tiro Salvezza su Riflessi per dimezzare i danni di un attacco, non subisce alcun danno.

\textbf{Anello di Visione a Raggi X}\index{Anello di Visione a Raggi X}

Essenza Rivelazione 24, 25.000 mo

A comando, questo anello conferisce a chi lo indossa la capacità di vedere all'interno e attraverso la materia solida. La portata della vista è di 3 metri, entro i quali chi lo indossa è in grado di vedere tutto come se si trovasse esposto alla normale luce esterna, anche senza illuminazione. La vista a raggi X è in grado di oltrepassare 30 cm di pietra, 2,5 cm di metalli comuni o fino a 90 cm di legno o terra.

Sostanze più spesse o una sottile lamina di piombo impediscono la vista. Usare questo anello è fisicamente stancante, e provoca a chi lo indossa 1 danno a Potenza per ogni minuto successivo ai primi 10 minuti di utilizzo in un giorno. L'anello deve essere utilizzato in incrementi di 1 minuto.

\textbf{Anello dell'Intermittenza}\index{Anello dell'Intermittenza}

Essenza movimento 24, 27000

A comando, questo anello rende chi lo indossa intermittente. La creature ottiene +4 in difesa e se colpito può effettuare un Tiro Salvezza su Riflessi sul Tiro per colpire dell'avversario, se lo passa non si viene colpiti.

\textbf{Anello di Libertà di Movimento}\index{Anello di Libertà di Movimento}

Essenza Movimento 28, 40000 mo

Questo anello d'oro permette a chi lo indossa di muoversi senza difficoltà date da movimenti difficili, non può essere ammanettato o legato, non può essere spostato o fermato magicamente,

\textbf{Anello di Stelle Cadenti}\index{Anello di Stelle Cadenti}

Essenza Attacco, Rivelazioni 28, 50000 mo

Questo anello può funzionare in due modi: se si trova in zone con luce scarsa o all'esterno durante la notte; quando chi lo indossa si trova sottoterra o in interni durante la notte. Durante la notte, a cielo aperto o in zone con luce scarsa o buio, l'anello delle stelle cadenti può effettuare a comando le seguenti funzioni:

\textbf{Luci Danzanti} (una volta ogni ora)

\textbf{Luce} (due volte per notte)

\textbf{Globo di Fulmini} (una volta per notte, all'aperto)

\textbf{Stelle Cadenti} (tre volte a settimana, all'aperto)

La prima funzione crea 4 luci (che illuminano in raggio di 3 metri) che puoi comandare singolarmente e spostare dove vuoi, entro distanza di 18 metri. Durata 1 ora.

La seconda funzione crea una luce di raggio 6 metri, durata 10 minuti.

Globo di fulmini genera delle sfere luminose sono simili al del luci danzanti, e chi lo indossa le controlla nel movimento. Le sfere hanno una gittata di media lunghezza e una durata di 4 round, e possono essere mosse a una velocità di 36 metri (2 azioni) per round. Ogni sfera ha un diametro di circa 90 cm, e qualsiasi creatura che arrivi a distanza di mischia da una sfera ne scarica l'energia e ne subisce i relativi danni da elettricità, in base al numero delle sfere create.

Numero di Sfere Danno per Sfera

1 sfera 4d6 danni da Elettricità

2 sfere 3d6 danni da elettricità ciascuna

3 sfere 2d6 danni da elettricità ciascuna

4 sfere 1d6 danni da elettricità ciascuna

Una volta attivata la funzione globo di fulmini, le sfere possono essere lanciate in qualsiasi momento prima del sorgere del sole (più sfere possono essere lanciate nello stesso round).

La funzione Stelle Cadenti produce tre stelle cadenti, che possono essere generate dall'anello ogni settimana, simultaneamente o una alla volta. Il loro impatto infligge 12 danni e si propaga (come fossero palle di fuoco) in una sfera del raggio di mischia infliggendo 24 danni da fuoco.

Qualsiasi creatura colpita da una stella cadente subisce pieni danni dall'impatto e pieni danni dalla propagazione a meno che non superi un tiro salvezza su Riflessi con DC 13.

Le creature non colpite dalla stella ma all'interno dell'area di propagazione ignorano i danni da impatto e subiscono solo metà dei danni da fuoco se superano un tiro salvezza su Riflessi con DC 13. La gittata è di media lunghezza, al limite dei quali la stella cadente esplode a meno che non colpisca prima una creatura o un oggetto. Una stella cadente segue sempre un percorso in linea retta, e qualsiasi creatura sulla sua traiettoria deve superare un tiro salvezza per non essere colpita.

Di notte al chiuso o sottoterra, l'anello delle stelle cadenti ha le seguenti proprieta':

\textbf{Luminescenza} (due volte al giorno)

Pioggia di scintille (speciale, una volta al giorno)

Luminescenza crea una luce pari ad una torcia, 3 m raggio, 1 ora per
uso.

La pioggia di scintille è una nuvola volante di scintille violacee scoppiettanti che si sprigionano dall'anello fino a una distanza di 3 metri in un arco di dimensioni di 3 metri. Le creature all'interno dell'area subiscono 2d8 danni ognuna se non indossano armature di metallo o non trasportano armi di metallo, nel caso contrario i danni diventano 4d8.

\textbf{Anello di Telecinesi}\index{Anello di Telecinesi}

Essenza Movimento 26, 75000 mo

Questo anello permette a chi lo indossa permette di spostare fino a 100Kg di materia per 30 minuti al giorno. Se usato su una creatura viene concesso un Tiro Salvezza su Arbitrio DC 18 per resistere, se il Tiro Salvezza è riuscito quell'anello non funzionarà più per 24 ore su quel soggetto.

La capacità va usata in incrementi di 1 minuto alla volta

\textbf{Anello di Rigenerazione}\index{Anello di Rigenerazione}

Essenza Cura 28, 90000 mo

Questo anello d'oro bianco con incastonato un grosso zaffiro verde conferisce continuamente ad una creatura vivente che lo indossa la capacità di guarire 1 pf a turno (ogni 10 minuti) e l'immunità ai danni da Sanguinamento. Se chi lo indossa perde un arto, un organo o qualsiasi altra parte del corpo mentre indossa questo anello, l'anello lo rigenera con gli stessi effetti dell'Essenza Cura per Rigenerazione.

\textbf{Anello di Riflettere Essenza}\index{Anello di Riflettere Essenza}

Essenza Protezione 28, 100000 mo

A comando, una volta al giorno, questo semplice anello di platino riflette automaticamente fino a LP 24.

\textbf{Anello dei Tre Desideri}\index{Anello dei Tre Desideri}

Essenze tutte 28, 120.000 mo

Su questo anello sono incastonati tre rubini: ogni rubino contiene la capacità di esaudire un desiderio. L'anello farà di tutto per fuorviare la richiesta.

\textbf{Anello del Richiamo del Djinni}\index{Anello del Richiamo del Djinni}

Essenza Convocazione 28,

Questo anello dei geni, uno dei molti che vengono menzionati nelle favole, è un oggetto estremamente versatile. Agisce come un Portale speciale attraverso il quale uno specifico Djinni può essere evocato dal Piano Elementale dell'Aria. Quando l'anello viene strofinato (2 Azioni), il richiamo parte e il djinni compare al round seguente. Il djinni obbedisce fedelmente come servitore a chi lo indossa, ma mai per più di 1 ora al giorno. Se il djinni dell'anello dovesse essere ucciso, l'anello perderebbe ogni caratteristica magica e ogni valore.

\textbf{Anello di Comando degli Elementi}\index{Anello di Comando degli Elementi}

Essenza Convocazione, Charme 31, 200.000 mo

Tutti e quattro i tipi di anello del comando degli elementali sono molto potenti. Ognuno ha l'aspetto di un comune anello magico minore finché non viene attivato completamente (al verificarsi di certe condizioni come uccidere un elementale del tipo appropriato con un solo colpo o esporsi a materiale sacro dell'elemento appropriato), nel qual caso rivela alcuni poteri specifici oltre alle proprietà comuni sotto descritte.

Gli elementali del piano a cui l'anello è legato non possono attaccare chi lo indossa, e nemmeno avvicinarsi a più di distanza di mischia da lui. Se chi lo indossa lo desidera, può rinunciare a questa protezione e tentare invece di rendere l'elementale soggetto all'Essenza Charme, Tiro Salvezza su Volontà con DC 17 nega, ma se questo fallisce, tuttavia, la protezione viene perduta e non si possono fare ulteriori tentativi di charme.

Chi indossa l'anello è in grado di parlare con le creature del piano a cui l'anello è legato. Le creature in questione capiscono che il personaggio sta indossando quel genere di anello, e mostrano un profondo rispetto per lui se i loro allineamenti sono affini. Se gli allineamenti sono opposti, le creature avranno paura di lui, qualora si dimostri forte con loro. Se è debole, tuttavia, lo odieranno e tenteranno di ucciderlo.

Chi indossa un anello del comando degli elementali subisce le seguenti
penalità ai tiri salvezza:

\begin{tabular}{ll}
	\toprule
	\textbf{Elemento} & \textbf{Penalità ai tiri salvezza}\\
	Acqua             & -2 tiri salvezza basati sul fuoco\\
	Aria              & -2 tiri salvezza basati sul terra\\
	Fuoco             & -2 tiri salvezza basati sul freddo o acqua\\
	Terra             & -2 tiri salvezza basati sul aria o Elettricità\\
\end{tabular}

Oltre ai poteri sopra descritti, ogni specifico anello conferisce
a chi lo indossa le seguenti capacità relative al suo elemento.

\textbf{Comando degli Elementi} (Acqua)\index{Comando degli Elementi Acqua}

Camminare sull'Acqua (uso illimitato)

Creare Acqua (uso illimitato)

Respirare Sott'Acqua (uso illimitato)

Muro di Ghiaccio (una volta al giorno), come Essenza Creazione LP 16

Tempesta di Ghiaccio (due volte alla settimana), come Essenza Attacco LP 16

L'anello ha l'aspetto di un Anello di Camminare sull'Acqua finché non si verificano certe condizioni.

\textbf{Comando degli Elementi} (Aria)\index{Comando degli Elementi Aria}

Caduta Morbida (uso illimitato, solo per chi lo indossa), annulli il danno da caduta

Resistere all'Energia (Elettricità) (uso illimitato, solo per chi lo indossa), protegge da 5 di danno

Folata di Vento (due volte al giorno), Essenza Creazione LP 11

Muro di Vento (uso illimitato), Essenza Creazione 1LP 1

Camminare nell'Aria (una volta al giorno, solo per chi lo indossa, 1 ora)

Catena di Fulmini (una volta alla settimana), Essenza Attacco LP 16

L'anello ha l'aspetto di un Anello di Caduta Morbida finché non si verificano le condizioni per attivarne il pieno potenziale. Deve essere riattivato ogni volta che viene indossato da una nuova creatura.

\textbf{Comando degli elementi (Fuoco)}\index{Comando degli elementi (Fuoco)}

Resistere all'Energia (fuoco) , protegge da 12 di danno a round

Mani Brucianti (uso illimitato), Essenza Attacco LP 11

Sfera Infuocata (due volte al giorno), Essenza Attacco 1LP 1

Pirotecnica (due volte al giorno), Essenza Creazione LP 13

Muro di Fuoco (una volta al giorno), Essenza Creazione LP 11

Colpo Infuocato (due volte alla settimana), Essenza Attacco LP 16

L'anello ha l'aspetto di un Anello di Resistenza all'Energia maggiore (fuoco) finché non si verificano le condizioni stabilite.

\textbf{Comando degli Elementi (Terra)}\index{Comando degli Elementi (Terra)}

Fondersi nella Pietra (uso illimitato, solo per chi lo indossa), Essenza Trasformazione LP 11

Ammorbidire Terra e Pietra (uso illimitato), Essenza Trasformazione LP 13

Scolpire Pietra (due volte al giorno), Essenza Trasformazione LP 16

Pelle di Pietra (una volta alla settimana, solo per chi lo indossa), Essenza Difesa LP 16

Passapareti (due volte alla settimana), Essenza Movimento LP 13

Muro di Pietra (una volta al giorno), Essenza Creazione LP 16

L'anello ha l'aspetto di un Anello di Fondersi nella Pietra (permettendo chi lo indossa di Fondersi Nella Pietra finché non si verificano le condizioni stabilite.

\pagebreak

\section{Bastoni Magici Speciali}\index{Bastoni Magici}

\label{bastoni-magici-speciali}

Un bastone è un lungo pezzo di legno in grado di contenere diverse Essenze. Al momento della sua creazione ha 10 cariche a disposizione.

Lanciare un incantesimo tramite un bastone costa in genere 2 Azioni. Per essere in grado di attivare un bastone, un personaggio deve tenerlo almeno con una mano (o quello che svolge le funzioni delle mani, nel caso di creature non umanoidi).

Tirate un d100. Un risultato di 01--30 indica che qualcosa (un disegno, un'iscrizione, ecc.) fornisce qualche indizio sulla sua funzione, mentre un risultato di 31--100 indica che l'oggetto non ha rune particolari.

I bastoni utilizzano i punteggi di statistica e le Abilità relativi di chi li impugna per determinare la DC dei Tiri Salvezza contro i loro incantesimi. A differenza di altri oggetti magici, chi lo impugna può utilizzare la propria CM al posto di quello del bastone. Ciò vuol dire che i bastoni risultano molto più efficaci nelle mani di potenti incantatori.

I bastoni contengono un massimo di 10 cariche. Ogni Essenza lanciata
da un bastone consuma una o più cariche.

Quando termina le sue cariche, il bastone non può più essere usato finché non viene ricaricato. Ogni mattina l'incantatore può infondere una parte del suo potere nel bastone sempre che uno o più essenze conteniti nel bastone siano uguali alle sue.

Per infondere l'Essenza nel bastone è sufficiente superare una prova di competenza magica pari al livello di potere dell'Essenza indicata nel bastone +4. Infondere potere nel bastone in questo modo ristora al bastone una carica e l'incantatore avrà un -4 a tutti i CM del giorno (ma non risulterà avere consumato una delle magie giornaliere)

Un bastone non può guadagnare più di una carica al giorno ed un incantatore non può infondere cariche in più di un bastone al giorno.

\bigskip

Tabella Bastoni

\begin{tabularx}{0.95\textwidth}{llXl}
	\toprule
	Medio & Maggiore & Nome& Prezzo\\
	01-15 & 01-03    & Bastone dello Tuono \index{Bastone dello Tuono}     & 17600\\
	16-30 & 04-09    & Bastone del Fuoco \index{Bastone del Fuoco}         & 18950\\
	31-40 & 10-11    & Bastone del Ghiaccio \index{Bastone del Ghiaccio}   & 22800\\
	41-55 & 12-13    & Bastone dell'Alterazione della Taglia \index{Bastone dell'Alterazione della Taglia} & 26150\\
	56-75 & 14-19    & Bastone della Cura \index{Bastone della Cura}       & 29600\\
	76-90 & 20-24    & Bastone del Fortunato \index{Bastone del Fortunato} & 41400\\
	91-95 & 25-31    & Bastone d'Illuminazione \index{Bastone d'Illuminazione}             & 51500\\
	96-99 & 31-37    & Bastone della Difesa \index{Bastone della Difesa}   & 82000\\
	100   & 38-45    & Bastone della Attacco \index{Bastone della Attacco} & 82000\\
	-     & 46-52    & Bastone della Rivelazione \index{Bastone della Rivelazione}         & 82000\\
	-     & 53-60    & Bastone dello Charme \index{Bastone dello Charme}   & 82000\\
	-     & 61-68    & Bastone della Protezione \index{Bastone della Protezione}           & 82000\\
	-     & 69-76    & Bastone della Trasformazione \index{Bastone della Trasformazione}   & 82000\\
	-     & 77-85    & Bastone dell'Alterazione \index{Bastone dell'Alterazione}           & 82000\\
	-     & 86-93    & Bastone del Movimento \index{Bastone del Movimento} & 82000\\
	-     & 94-98    & Bastone Nero (Distruzione) \index{Bastone Nero (Distruzione)}       & 82000\\
	-     & 99       & Ritira 1 volta su Medio e 1 volta su Maggiore       & 130000\\
	-     & 100      & Ritira 2 volta su Medio e 1 volta su Maggiore       & 180000\\
\end{tabularx}

\bigskip

\textbf{Bastone del Tuono}\index{Bastone del Tuono}

Essenza Illusione, Attacco LP 15

Il bastone quando usato emette un forte boato e genera luce in raggio di 3 metri e può lanciare per singola carica un piccolo fulmine da 4d6 entro distanza di 18 metri (solo dopo aver tuonato), DC 25 TS Riflessi dimezza

\textbf{Bastone del Fuoco}\index{Bastone del Fuoco}

Essenza Attacco, Trasformazione LP 15

Il bastone quando usato si trasforma in una lingua di fuoco (+1d6 di danno se usato come arma) e può lanciare per singola carica un globo di fuoco da 3d6 di danno entro 18 metri, DC 25 TS Riflessi dimezza

\textbf{Bastone del Ghiaccio}\index{Bastone del Ghiaccio}

Essenza Attacco, Trasformazione LP 15

Il bastone quando usato può congelare 9 metri cubi di acqua (un cubo di lato lungo mischia) può lanciare per singola carica un globo di ghiaccio da 3d6 di danno entro 18 metri, DC 25 TS Riflessi dimezza

\textbf{Bastone dell'Alterazione della Taglia}\index{Bastone dell'Alterazione della Taglia}

Essenza Alterazione, LP 18

Il bastone permette per singola carica di aumentare di una taglia di dimensione, compreso ciò che si sta portando su di se. Durata 1 ora.

\textbf{Bastone del Fortunato}\index{Bastone del Fortunato}

Essenza Movimento, LP 18

Questo "bastone" non più grande di uno stuzzica denti, deve essere a contatto con il proprietario per poter essere attivato. Come azione immediata, anche dopo aver saputo il risultato di un tiro di dado e saputo se si supera o meno al prova (Tiro per Colpire, tiro salvezza, check\ldots ) può usare una carica per tirare 1d6 ulteriore.

\textbf{Bastone d'Illuminazione}\index{Bastone d'Illuminazione}

Essenza Creazione, LP 15

Questo bastone può illumiare in un raggio entro 6 metri. Usando una carica è possibile creare delle zone di luce di raggio 12 metri, durata 8 ore.

\textbf{Bastone dell'Essenza...}\index{Bastone dell'Essenza...}

Essenza Vari, LP18

Il bastone concede un bonus di +2 alle prove di CM per l'Essenza del Bastone. Usando una carica del bastone si aumenta il singolo check di magia (CM) di 2d6.

\pagebreak

\section{Oggetti Maledetti}\index{Oggetti Maledetti}

\label{oggetti-maledetti}

Gli oggetti maledetti sono oggetti magici dotati di un'influenza potenzialmente negativa sul personaggio. A volte tendono a confondere il male con il bene, costringendo il loro possessore a fare scelte difficili.

Gli oggetti maledetti non sono mai realizzati intenzionalmente, ma piuttosto sono il risultato di un lavoro mal riuscito, di artigiani con poca esperienza o della mancanza di componenti adeguate.

La maggior parte di questi oggetti funziona, ma non nel senso che si voleva e il loro uso produce inconvenienti dannosi.

Quando una prova di creazione di un oggetto magico fallisce di 5 o piu', tirate sulla tabella per determinare il tipo di maledizione che l'oggetto possiede.

\bigskip

\textbf{Maledizioni Comuni degli Oggetti}

\medskip
\begin{tabular}{ll}
	\toprule
	\textbf{\%} & \textbf{Maledizione}\\
	01-15       & Inganno\\
	16-40       & Effetto o Bersaglio Opposto\\
	41-50       & Funzionamento Discontinuo\\
	51-65       & Requisito\\
	66-90       & Inconveniente\\
	91-100      & Effetto completamente diverso\\
\end{tabular}

\bigskip

Gli oggetti maledetti sono identificati come qualsiasi altro oggetto magico con una sola eccezione: a meno che la prova effettuata per identificare l'oggetto non ecceda la DC di 10 (successo critico) o piu', la maledizione non viene individuata. Se la prova non eccede 10 o piu', ma riesce comunque, tutto quello che viene rivelato è l'originale scopo dell'oggetto magico.

Se si sa che l'oggetto è maledetto, la natura della maledizione può essere determinata usando la DC standard per identificare l'oggetto.

\textbf{Rimuovere Oggetti Maledetti}\index{Rimuovere Oggetti Maledetti}

Mentre alcuni oggetti maledetti possono essere semplicemente posati, altri esercitano una forte compulsione sul possessore a tenerli con sé, a qualsiasi costo. Altri riappaiono anche se abbandonati o è impossibile gettarli via.

Questi oggetti possono essere rimossi solo dopo che sul personaggio o l'oggetto viene lanciato una Essenza di Protezione Rimuovi Maledizione. La DC della prova di livello dell'incantatore per rimuovere la maledizione è pari a 10 + CM dell'incantatore che ha creato l'oggetto.

Se la prova ha successo, l'oggetto può essere rimosso nel round successivo, ma la maledizione rimane e colpisce nuovamente se l'oggetto viene usato un'altra volta.

\textbf{Effetti Comuni degli Oggetti Maledetti}

Gli effetti più comuni degli oggetti maledetti sono i seguenti. I Narratore possono inventare nuovi effetti particolari per specifici oggetti maledetti.

\textbf{Inganno}

Chi utilizza l'oggetto continua a credere che sia ciò che sembra a prima vista, ma in realtà non ha alcun potere, a parte quello di ingannare. Chi lo usa è mentalmente spinto a credere che funzioni, e non può essere convinto del contrario se non con l'uso d'Essenza di Protezione rimuovi maledizione

\textbf{Effetto o Bersaglio Opposto}

Questi oggetti maledetti tendono ad avere dei difetti di funzionamento che in alcuni casi generano effetti diametralmente opposti a quelli desiderati dal loro creatore, mentre in altri casi tendono a colpire chi li utilizza invece di qualcun altro.

Ma la cosa più interessante è che questi oggetti potrebbero anche non essere uno svantaggio per chi li possiede. La categoria degli oggetti magici dagli effetti opposti include anche le armi che infliggono penalità ai tiri per colpire e per i danni, invece che bonus.

Visto che un personaggio non dovrebbe sapere immediatamente quale sia il bonus di un oggetto magico, non dovrebbe venire a conoscenza nemmeno della natura della sua maledizione. Una volta che lo verrà a sapere, comunque, l'oggetto potrà essere abbandonato a meno che su di esso non vi sia qualche effetto magico che costringa il suo possessore a tenerlo e ad usarlo.

In questi casi, per liberarsi dall'oggetto sarà necessario l'Essenza di Protezione rimuovi maledizione

\textbf{Funzionamento Discontinuo}

Gli oggetti discontinui funzionano esattamente come dovrebbero, quando funzionano. Le tre tipologie a cui possono appartenere sono:

\textbf{Inaffidabile}: Ogni volta che l'oggetto viene attivato, c'è una probabilità del 5\% che non funzioni.

\textbf{Condizionato}: Questo oggetto funziona solo in determinate situazioni. Per determinare quali siano, scegliete una condizione di attivazione o consultato la tabella poco sotto.

\textbf{Incontrollabile}: Un oggetto incontrollabile tende ad attivarsi casualmente. Tirare un d\% ogni giorno. Con un risultato di 01--05 l'oggetto si attiva spontaneamente in un certo momento del giorno.

\bigskip

\begin{tabularx}{0.95\textwidth}{lX}
	\toprule
	\textbf{\%} & \textbf{Situazione}\\
	01-03       & Temperatura sotto lo zero\\
	04-05       & Temperatura sopra lo zero\\
	06-10       & Durante il giorno\\
	11-15       & Durante la notte\\
	16-20       & Esposto alla luce solare\\
	21-25       & In assenza di luce solare\\
	26-34       & Sott'acqua\\
	35-37       & Fuori dall'acqua\\
	38-45       & Sottoterra\\
	46-55       & In superficie\\
	56-60       & Entro 3 metri da un tipo di creatura casuale\\
	61-64       & Entro 3 metri da una razza o tipo di creatura casuale\\
	65-72       & Entro 3 metri da un incantatore\\
	73-80       & Entro 3 metri da un incantatore di un Patrono specifico\\
	81-85       & Nelle mani di un personaggio non incantatore\\
	86-90       & Nelle mani di un personaggio incantatore\\
	91-95       & Nelle mani di una creatura con particolare tratto\\
	96          & Nelle mani di una creatura di un particolare sesso\\
	97-99       & Nei giorni non sacri o durante particolari ricorrenze astrologiche\\
	100         & A più di 150 km da un determinato luogo\\
\end{tabularx}

\bigskip

\textbf{Requisito}

Alcuni oggetti hanno requisiti molto più difficili da soddisfare perché funzionino. Per far funzionare l'oggetto in questione, potrebbe essere necessario soddisfare una delle seguenti condizioni:
\begin{itemize}
	\item Il personaggio deve mangiare il doppio del normale.
	\item Il personaggio deve dormire il doppio del normale.
	\item Il personaggio deve compiere una missione specifica (solo una volta,poi l'oggetto funziona normalmente).
	\item Il personaggio deve sacrificare (distruggere) un valore pari a 100 mo di oggetti o materiali preziosi al giorno.
	\item Il personaggio deve sacrificare (distruggere) un valore pari a 2000 mo di oggetti magici ogni settimana.
	\item Il personaggio deve giurare lealtà ad un nobile in particolare, o alla sua famiglia.
	\item Il personaggio deve abbandonare tutti gli altri oggetti magici.
	\item Il personaggio deve venerare una particolare Dio
	\item Il personaggio deve cambiare il suo nome in un altro. L'oggetto funziona solo per i personaggi con un certo nome.
	\item Il personaggio deve avere un numero minimo di gradi in una particolare competenza.
	\item Il personaggio deve sacrificare parte della propria energia vitale (1 punto di Potenza) la prima volta che usa l'oggetto. Se il personaggio trova un modo di recuperare i punti di Potenza persi, l'oggetto smette immediatamente di funzionare. L'oggetto non smette di funzionare se il personaggio guadagna punti di Potenza in seguito all'avanzamento di livello o di un altro oggetto magico.
	\item L'oggetto deve essere purificato con l'acqua santa ogni giorno.
	\item L'oggetto deve essere usato per uccidere una creatura vivente al giorno.
	\item L'oggetto deve essere immerso nella lava vulcanica una volta al mese.
	\item L'oggetto deve essere usato almeno una volta al giorno, o smette di funzionare per il suo attuale possessore.
	\item Quando viene brandito, l'oggetto deve spillare sangue (solo armi). Non può essere messo da parte o cambiato con un altro oggetto finché non ha messo a segno un colpo.
\end{itemize}
I requisiti dipendono così tanto dalla convenienza dell'oggetto che non dovrebbero mai essere determinati a caso. Un oggetto intelligente con un requisito spesso impone il proprio requisito grazie alla sua personalità.

Se il requisito non viene soddisfatto, l'oggetto smette di funzionare. Se invece viene soddisfatto, di solito l'oggetto funziona per un giorno intero prima di dover di nuovo soddisfare il requisito (anche se alcuni requisiti vanno soddisfatti una volta sola, altri
una volta al mese e altri ancora in continuazione).

\textbf{Inconveniente}

Gli oggetti che hanno degli inconvenienti hanno solitamente degli effetti positivi su chi li usa, ma hanno anche degli aspetti negativi. Anche se a volte gli inconvenienti vengono alla luce solo quando gli oggetti sono utilizzati (o tenuti in mano, nel caso di oggetti come le armi), di solito rimangono presenti fino a quando il personaggio non si libera dell'oggetto in questione.

A meno che non sia indicato diversamente, gli inconvenienti rimangono attivi per tutto il tempo in cui l'oggetto rimane in possesso del personaggio. La DC dei Tiro Salvezza per evitare questi effetti è pari a 10 + il livello dell'incantatore dell'oggetto.

\bigskip

\begin{tabular}{ll}
	\toprule
	\textbf{\%} & \textbf{Inconveniente}\\
	01-04       & I capelli del PG crescono di 2,5 cm all'ora.\\
	05-09       & L'altezza del PG diminuisce di 30 cm (risultato di 01--50su un d\%) oppure  \\
	            & aumenta della stessa misura (un risultato di 51--100).   \\
	            & Accade solo una volta.\\
	10-13       & La temperatura intorno all'oggetto è di 5° C più fredda del normale.\\
	14-17       & La temperatura intorno all'oggetto è di 5° C più calda del normale.\\
	18-21       & Il colore dei capelli del PG cambia.\\
	22-25       & II colore della pelle del PG cambia.\\
	26-29       & II PG ora porta un segno distintivo (un tatuaggio, una strana
	luminescenza ecc.).\\
	30-32       & II sesso del PG cambia.\\
	33-34       & La razza o la specie del PG cambiano.\\
	35          & II PG viene colpito da una Malattia determinata casualmente,
	che non può essere curata.\\
	36-39       & L'oggetto emette costantemente suoni sgradevoli (lamenti, maledizioni, insulti...).\\
	40          & L'oggetto ha un aspetto ridicolo (colori sgargianti, forma,brilla di un alone rosa ecc.).\\
	41-45       & II PG diventa estremamente possessivo nei confronti dell'oggetto.\\
	46-49       & II PG ha una paura incontrollabile di perdere l'oggetto o che venga danneggiato.\\
	50-51       & Un tratto viene cambiato\\
	52-54       & II PG deve attaccare la creatura a lui più vicina (probabilità del 5\% ogni giorno).\\
	55-57       & II PG rimane Stordito per 1d4 round una volta che l'oggetto è servito al suo scopo       \\
	            & (o casualmente 1 volta al giorno).\\
	58-60       & La vista del PG è sfocata (penalità --2 agli attacchi, ai Tiri Salvezza e alle prove di Abilita'         \\
	            & che richiedono la vista).\\
	61-64       & II PG guadagna un livello negativo.\\
	65          & II PG guadagna due livelli negativi.\\
	66-70       & II PG deve effettuare un TS su Volontà ogni giorno o subisce 1 danno a Intelletto.\\
	71-75       & Il PG deve effettuare un TS su Volontà ogni giorno o subisce 1 danno a volontà.\\
	76-80       & II PG deve effettuare un TS su Volontà ogni giorno o subisce 1 danno a Magnetismo.\\
	81-85       & II PG deve effettuare un TS su Tempra ogni giorno o subisce 1 danno a Potenza.\\
	86-90       & II PG deve effettuare un TS su Tempra ogni giorno o subisce 1 danno a Agilità.\\
	91-95       & II PG deve effettuare un TS su Tempra ogni giorno o subisce 1 danno a Potenza ed Agilità.\\
	96          & II PG viene trasformato in una creatura specifica (probabilità del 5\% ogni giorno).\\
	97          & II PG non può più usare Essenze con difficoltà oltre 18\\
	98          & II PG non può più usare Essenze con Livello di Potere oltre 15\\
	99          & II PG non può più usare Essenze\\
	100         & Tira due volte\\
\end{tabular}

\pagebreak

\section{Draghi}\index{Draghi}
\label{draghi}

\begin{tcolorbox}[enhanced,arc=5pt,boxrule=0.3pt]{
		Oh maledetti possa Lynx chiudervi tutti i portali\\
		Oh assassini possa Sumkjir sterminarvi\\
		Oh devastatori possa Nedraf rompervi le ossa!\\
		(imprecazioni contro i Draghi)}\end{tcolorbox}\medskip

I Draghi non sono nativi di Yeru bensi' arrivati poco meno di 300 anni fa portandosi dietro morte e distruzione sia per Curyan che per Tiya.

Ta'hil, potente drago rosso, usando la magia dei viaggi interdimensionali voleva trovare nuovi tesori e terre da soggiogare purtroppo per lui si avventuro' troppo nello spazio vuoto, dove anche la luce le stelle non arriva ma si flette e torna indietro.

In questi non luoghi venne soggiogato, dominato, la sua mente rifatta da esseri oltre l'umana compresioni, pura follia e chaos.

Morto, ricostruito, distrutto, riassemblato, annichilito, ricomposto un innumerevole volte del suo essere originario non rimase nulla, solo lucida aggressiva follia.

Quando questi esseri passarono ad un nuovo gioco Ta'hil creo' un nuovo portale magico, non avendo piu' una percezione o ricordo di casa questo lo porto' su Yeru, un mondo diverso e ambiguo dove subito' si scontro contro un Patrono, Lynx.

Il suo corpo era stato rifatto, riforgiato, ricomposto della stessa non materia, la sua mente un vortice di puro potere caotico.

Il Patrono, forse una divita' per quel mondo, si dimostro' un avversario facile e si diverti' a scarificare il corpo di quella debole entita'.

Ormai giunto al colpo finale il Patrono aprì un portale sotto di e si lascio cadere dentro chiudendolo immediatamente.

Ta'hil comprese subito la magia usata e le potenzialita' del mondo, come la sua magia fosse ancora piu' efficace.
Apri centinaia di portali richiamando in quel mondo ogni abominio avesse mai immaginato e.. draghi, tanti, migliaia, tutti i draghi oscuri che potesse mai sognare.

L'invasione di Yeru era iniziata. Immediatamente Ta'hil ribadi' il suo potere e la sua leadership uccidendo in un solo giorno decine e decine di draghi di tutti i colori presenti (rosso, verde, blu, bianco, nero e viola).
Gli altri draghi su sottomisero al suo volere, rimanendo incatenati alla volonta' di Ta'hil.

Tutta Yeru venne devastata per oltre 60 anni dai draghi, pochi dei draghi perirono per mano degli avventurieri del mondo.

Con il massimo sforzo Gradh riuscì a coinvolgere i Patroni della Genesi e solo cosi' Ta'hil accetto un incontro.

La storia e' nota, Calicante vide in Ta'hil un arma diversa e sommamente potente per portare distruzione ed entropia in Yeru, Ljust vide in Deynos l'aiuto e la conoscenza di quelle arcane creature.
Si strinse una alleanza ed una finta tregua.

Calicante privo di un po' della follia Ta'hil, ma non tutta, amava il chaos e la morte che che in tanti modi riusciva a portare.

Ljust amplio' i poteri di Deynos perche' potesse trasformare i draghi oscuri catturati in draghi buoni.

Esattamente, draghi buoni.

Fino a quel momento l'unico, solo, drago dimostratosi sinceramente buono era Deynos, che per puro caso, se non per errore aveva usato un portale creato da Ta'hil prima che questo si chiudesse.

Nessun altro drago buono e' mai giunto su Yeru, i pochissimi presenti sono quelli che catturati sono stati trasformati da Deynos.

Ta'hil è diventato un jolly, un arma di pura aggressivita' buttata nel mondo, perché alzi il livello di caos, vendetta, violenza di Yeru.
I draghi, quando non comandati da Ta'hil, vanno in giro per Tiya e Curyan a distruggere, ammassare ricchezze e creare leggende.

Un drago e' una creatura praticamente incontrastata, le piu' potenti balliste possono ferirli ma il loro soffio è morte certa.

Ogni qual volta Ta'hil decide che è necessario rinforzare le sue truppe apre nuovi portali facendo giungere nuovi draghi da sottomettere e dominare.

E' una lotta impari per i poveri yeruiti, ogni volta che con estremo sacrificio umano un drago muore altri due arrivano. Dove sia la tana di Ta'hil e' un mistero, celato dal potere stesso di  Calicante e neanche Ljust è riuscita a carpire.

La speranza è che prima o poi un leggendario gruppo di eroi possa scovare la sua tana ed uccidere il maggior nemico di Yeru.

Nota: purtroppo no e' vero che i draghi hanno solo 300 anni, in realta' l'ultima vittoria del millennio ha fatto dimenticare che queste creature sono su Yeru da molti millenni e da altrettanto seminano distruzione, chaos e morte.

\subsection{I Colori dei Draghi}

Ogni Drago ha sue caratteristiche tipiche e peculiari.
Tutti i draghi su Yeru obbediscono a Ta'hil, ciecamente e senza resistenza, almeno finche' vengono trasferiti su Yeru tramite un suo portale.

Ogni volta che Ta'hil vuole convocare un drago apre un portale e da questo esce un drago, di colore casuale. Appena arrivato la magia di dominazione di Ta'hil soggioga il drago che non puo' piu' ribellarsi ai suoi ordini.

Il drago conduce la vita che piu' gli "aggrada", solitamente questa contempla la distruzione di qualche citta' e centinaia di morti, solo quando richiamato da Ta'hil interrompe le sue attivita' per volare dove richiesto. Oppure Ta'hil puo' inviare un ordine mentale senza bisogno che il drago si sposti.
Negli ultimi 300 anni solo una volta sono stati convocati tutti i draghi, diversamente solo i piu' anziani e potenti vengono richiamati perche' agiscano come luogotenenti presso il territorio.

\subsubsection{Drago Nero} \index{Drago Nero}

I Draghi Neri sono violenti ed aggressivi, vivono in paludi e acquitrini e che generalmente governano come padroni indiscussi.

I Draghi Neri sono creature minacciose che hanno grandi corna curve in avanti.
La testa si collega ad un collo relativamente corto e ad un corpo da lucertola grossa e muscoloso.

Hanno ali piccolissime che si trovano sui lati, ma riescono comunque a volare grazie alla magia.
Hanno le zampe palmate per permettere loro di nuotare con maggiore facilita' nelle zone paludose dove vivono.

I Draghi Neri tendono a fare le loro tane al centro della palude o acquitrino.
Considerano quel territorio il loro e nessuno puo' bagnarsi senza subire la loro ira.

Una tana di drago nero puo' essere un ammasso gigantesco di tronchi ma anche una caverna sotterranea sommersa d'acqua, se non il fondo di un lago.
Potendo respirare sott'acqua non si fanno preoccupazione su dove costruire la loro dimora.

La loro casa e' sempre protetta da trappole e dai loro seguaci malvagi che gli portano cibo, possibilmente vivo.

L'ambiente dove vive un drago nero ne subisce i suoi effetti, vapori acidi, distruzione, corruzione sono immediatamente percepibili.

Il Drago Nero rappresentano i tratti dell'egoismo e violenza odiando ogni forma di vita, compreso gli stessi draghi neri.

I Draghi neri hanno pieno accesso all'essenza di Distruzione.


\subsubsection{Drago Blu} \index{Drago Blu}

I Draghi Blu abitano tra le nuvole, volando (e levitando) tra le tempeste.

I Draghi Blu hanno un aspetto serpentiforme, allungato ed legante, con corna lunghe all'indietro.

La faccia di un Drago Blu e' meno segnata da increspature e rimane liscia.
Sono gli unici draghi a non avere ali pur volando meglio di ogni altro drago.

La loro magica ma naturale capacita' di volo unita al fatto di nutrirsi di elettricita' ne fa creature prettamente volanti che quasi mai scendono a terra (e mai toccano terra considerandola impura e sporca!), preferiscono rimanere tra le nubi, specialmente tra quelle piu' scure e cariche di energia per nutrirsi

La tana del Drago Blu solitamente e' tra i picchi piu' alti delle montagne possibilmente tanto alte da arrivare alle nubi. Questa non e' mai coperta e spesso assomiglia a giganteschi nidi.

I Draghi Blu possono assimilare carne ma non vegetali, non traggono nutrimenti da cio' che mangiano avendo un metabolismo puramente elettrico.

Sono draghi sociali, che amano stare con i loro simili e sono molto protettivi con la loro prole.
Solitamente non si trova mai un nido da solo, ma interi altopiani dominati da decine di draghi.

Non vanno d'accordo con i draghi viola che disprezzano per la scelta di aver rinunciato al volo per vivere sottoterra.

I Draghi Blu padroneggiano l'Essenza di Attacco solo con l'Elettricita' di cui ne sono immuni ai danni (magici o meno).


\subsubsection{Drago Verde} \index{Drago Verde}

I Draghi verdi amano le foreste e la natura incontaminata dove si reputano i padroni e re indiscussi.

I potenti draghi verdi hanno la testa tondeggiante e pronunciate orecchie all'indietro, le corna sono corte e non appuntite.
Gli artigli e le fauci sono devastanti, potenti e capace di tranciare qualsiasi cosa.
Il naso e' largo e le narici aperte come se dovesse soffiare in qualsiasi momento.

Il soffio dei draghi verde e' veleno, cosi' che possa uccidere le creature viventi ma non le piante.

La tana di un drago verde e' sempre vicino ad una sorgente d'acqua, possibilmente nella parte piu' lussureggiante ed incontaminata della foresta.

Un Drago verde non ama volare e preferisce saltare schiacciando con il suo peso e dilaniare con i suoi artigli.

Tra i tanti draghi quello verde e' forse quello che fara' parlare gli avventurieri se si dimostrano rispettosi ed impauriti dalla sua regalita'.

I Draghi Verdi padroneggiano l'Essenza di Attacco solo con il Veleno e ne sono immuni sia a quelli magici che naturali.

\subsubsection{Drago Bianco} \index{Drago Bianco}

I Draghi Bianchi sono tra i piu' selvaggi e "animali" di tutti i draghi.
Amano i posti freddi e ghiacciati, trovando rifugio nelle valli piu' fredde come i picchi ghiacciati delle montagne e le steppe gelide.

I Draghi Bianchi hanno un aspetto selvaggio quasi sempre mostrano i denti e gli artigli sono estratti per muoversi agilmente sul terreno ghiacciato.
Non hanno penalita' di movimento su questi terreni.

Sfruttano il loro naturale camuffamento per aggredire e catturare le prede, sono ottimi cacciatori, molto intelligenti nello sfruttare l'ambiente.

Poco inclini alla magia sanno pero' soffiare schegge di ghiaccio molto piu' frequentemente di altri draghi, possono usare l'Essenza di Attacco solo con il ghiaccio. E' immine gli attacchi basati sul freddo e ghiaccio.

Le loro tane sono caverne ghiacciate nelle montagne o scavate nei ghiacciai piu' massici.


\subsubsection{Drago Porpora} \index{Drago Porpora}

I Draghi Porpora vivono sotto terra e si sono perfettamente adattati alla vita sotterranea.
Capaci di vedere al buio come se fosse pieno giorno, dotati di tremorsense, hanno perso la capacita' di volare ma acquisito quella di scavare con la stessa velocita' come se corressero.

Un Drago Porpora e' molto territoriale e stabilita' un perimetro (di circa 5 km di raggio) crea, se non gia' presente un intricata serie di cunicoli e caverne per i suoi servi.

Un Drago Porpora e' molto protettivo con le sue creature, con chi gli porta da mangiare e gli offre tesori.

Dall'aspetto serpentino hanno denti fini e artigli enormi che continuamente crescono.

E' forte e coraggioso, arrogante ma non sfrontato. Non ha paura di combattere se pensa di vincere. Porta sempre la battaglia sottoterra dove puo' creare fosse per fare precipitare i nemici o scappare se necessario.

Un Drago Porpora soffia un potentissimo attacco sonico che spesso crea crolli nelle caverne, crolli che sono completamente indifferente a lui. Ha padronanza dell'essenza di Attacco ma solo come forma di attacco sonora. E' immune agli attacchi sonori.


\subsubsection{Drago Giallo}  \index{Drago Giallo}

I Draghi Gialli hanno squame di vari toni di giallo che con la crescita prendono ad assomigliare sempre di più al colore delle sabbie dove dimorano, dal giallo chiaro all'ocra mattone.

Sono molto intelligenti ma essendo per natura solitari non hanno interesse a comunicare con le altre razze.

Vivono nei deserti dove spesso tendono agguati alle loro prede nascondendosi sul fondo di ampie buche scavate nella sabbia.
Appena percepiscono un movimento sopra di loro escono e divorano qualunque creatura.
Hanno una passione per la carne dei nani che trovano saporita anche se asciutta.

Il Drago Giallo pur se intelligente e' una macchina di morte e difficilmente scende a patti, solo se si trova in serio pericolo.

Ha una naturale resistenza al danno (dimezzano) al fuoco ed alle armi non magiche.

Un Drago Giallo ha un soffio rovente, anche se non propriamente fuoco.
Padroneggia l'Essenza di Attacco con il Fuoco. E' immune agli attacchi di Fuoco.


\subsubsection{Drago Rosso}  \index{Drago Rosso}

Il Drago Rosso si crede il Re dei Draghi per via della sua potenza fisica e del soffio capace di sciogliere la pietra.

I Draghi Rossi sono i draghi piu' grandi sia per corporatura che per apertura alare.
Spesso le scaglie, di un rosso scuro quasi di sangue, hanno bordi affilati ed allungati.

I Draghi Rossi prediligono le montagne calde e se possibile direttamente direttamente dentro un vulcano.

Combattono sfruttando la loro mole le ali il morso artigli.. insomma tutto cio' che sono ed hanno a disposizione. Un Drago Rosso combatte sempre fino alla morte non si ritira ne scappa ne rinuncia ad una sfida, l'orgoglio di cui sono tronfi non gli permette di mostrarsi deboli.

Un drago rosso e' immune al fuoco naturale e magico nonche' all'essenza di attacco da fuoco.
Un drago rosso puo' usare l'essenza di attacco soffiando fiamme.


%Ta'hil rosso
%Dyenos argento 
%Curyan vita
%Tiya scuro

\pagebreak

\section{Yeru}\index{Yeru}\index{Atilantis}

\label{yeru}

Yeru è il pianeta di riferimento di TUS. Un pianeta spaccato sia fisicamente che magicamente.

Intorno a Yeru ruotano due stelle Sparka e Andhakara.\index{Sparka}\index{Andhakara}

Sparka è di un caldo colore dorato è colei che porta calore e luce, attorno a lei Yeru fa un giro completo in 336 giorni da 24 ore l'uno.

Sparka illumina sempre e solo l'emisfero nord di Yeru, chiamato Curyan \index{Curyan}.

Andhakara illumina sempre e solo l'emisfero sud di Yeru, Tiya, ed è invece una stella azzurra e fredda, priva di vita, è colei che porta tempeste energetiche e strani accadimenti naturali. Porta una fredda penombra.

Yeru compie il suo giro completo attorno a lui in 336 giorni da 24 ore l'uno.

Se le 14 (06-20) ore diurne vedono Sparka e Andhakara protagoniste in questa loro danza nel cielo; le 10 ore notturne vedono come totali protagoniste le due lune di Yeru di nome Idam e Kenatu.

Gli abitanti di Yeru le chiamano le loro lune anche se in realtà non sono propriamente solo lune ma veri e propri pianeti abitati.

Le due lune sono grandi ed imponenti sul cielo notturno, Idam di un colore grigio rossastro e Kenatu di un caldo grigio madreperlato comandano le maree e influenzano con la loro presenza la navigazione.

Yeru ha una distribuzione delle terre peculiare ed unica, frutto del capriccio degli Dei della Genesi (Ljust e Calicante), potete immaginarlo come un sistema speculare sull'equatore.

Le terre non si uniscono all'equatore, lasciando circa 200 km di mare aperto.

Le terre emerse che compongono emisfero nord ed emisfero sud sono fra loro quasi simmetriche e simbiotiche. Forma e suddivisione delle grandi isole sono fra loro molto similari. Ma dal punto di vista climatico ci sono profonde diversità.

La zona di mare aperto di confine è selvaggia ed imperscrutabile. Le più profonde e potenti tempeste scaricano di continuo la loro forza ed anche la magia non riesce a penetrare. Nell'occhio di questo perenne e gigantesco maelstrom c'è la civilizzata e potentissima Atilantis, da molti ritenuta una isola leggendaria e culla della civiltà.

Curyan è governata dalla forza della vita, questa regione vive una sorta di perenne calda stagione con gradazioni di temperatura e fenomeni atmosferici che variano a seconda della latitudine.

Si incrociano zone dal clima torrido e umido ad altre con un caldo secco e senza precipitazioni; esistono fenomeni quali tempeste di sabbia nelle zone desertiche e tempeste tropicali forti e devastanti nelle lussureggianti baie centrali.
Ci sono territori piacevolmente caldi ed altri rinfrescati da brezze fresche provenienti dai ghiacciai del nord.

Tiya invece è un emisfero semi avvizzito, la luce che arriva basta appena a permettere l'agricoltura e gli animali hanno spesso un aspetto pallido ed emaciato.

La zona piu' ricca e' quella piu' prossima all'equatore dove sono di poco si attenua la fredda luce di Andhakara fa spazio a qualche raggio di Sparka.
In questa stretta fascia l'agricoltura e' piu' fiorente e ci sono meno fenomeni metereologigi devastanti.

E' l'emisfero dove vige la legge del più forte, dove si lotta per vivere e pochi sono gli stati che hanno un sistema di protezione efficace.

Il mare che abbraccia l'equatoriale è forte e tumultuoso, pochissime barche si avventurano da un continente all'altro, questo porta che scambi tra Tiya e Curyan siano estremamente ridotti via mare.


\pagebreak

\section{I Portali}\index{Portali}

\label{i-portali}

In un mondo dove i trasferimenti marittimi non funzionano se non tra isola e isola dello stesso emisfero, l'Essenza di Movimento ed in particolare la capacità di usare dei portali per trasferire merci e persone ha preso piede in maniera significativa.

Questo proliferare di piccoli, grandi, duraturi o istantanei tunnel ha causato uno squarcio nel tessuto dimensionale di Yeru generando a sua volta un proliferare di tunnel spontanei più o meno grandi e duraturi.

E questi Portali sono la causa di tantissimi problemi sia a Tiya che a Curyan in quanto non solo legano i due emisferi ma collegano tutta Yeru ad altri mondi (o almeno così si pensa dato che pochi sono tornati per riferirlo..).

Ci sono portali conosciuti e stabili, fino ad ora, che collegano Tiya a Curyan, quasi tutto sotto il controllo, per non dire dentro il castello, di reali o potenti.

Ci sono zone dove più frequentemente si aprono portali ma la destinazione non è sempre certa.

Poi ci sono i portali dei draghi. I draghi non sono nativi di Yeru ma sono stati attirati da queste porte magiche, causando scompiglio e terrore a Tiya e Curyan.

I draghi hanno ben compreso la natura di Yeru e con la loro fine intelligenza e innata capacità di plasmare la magia hanno costruito i loro portali facendo venire centinaia di draghi. Tutti malvagi.

Si, su Yeru non ci sono draghi "buoni" se non con poche eccezioni.

Si è sempre cercato di distruggere i portali dei draghi, con sacrificio e sangue. Molti sono stati distrutti, altri sono stati generati. E' una guerra senza fine, l'unica che può unire le persone dei due emisferi.

\pagebreak

\section{I Piani}

Anche se avventure infinite vi attendono nel gioco, ci sono altri mondi oltre questo, altri continenti, altri pianeti, altre galassie. Tuttavia anche oltre l'esistenza di innumerevoli pianeti esistono altri mondi, dimensioni completamente differenti dalla realtà, conosciuti come piani di esistenza. Tranne per rari punti di collegamento che permettono di viaggiare tra loro, ogni piano è un universo a sé con le sue proprie leggi naturali. Insieme, queste altre dimensioni e piani sono conosciuti come il Grande Oltre.\\

Sebbene il numero di piani sia limitato solo dall'immaginazione, essi possono ricondursi tutti a cinque tipi generali: il Piano Materiale, i Piani di Transizione, i Piani Interni, i Piani Esterni e gli innumerevoli semipiani.\\

\subsection{Cos'è un Piano?}

I piani di esistenza in realtà sono differenti collegamenti intrecciati. Eccetto per rari punti di collegamento, ogni piano è effettivamente un universo fine a se stesso, con le proprie leggi naturali. I piani si suddividono in una serie di tipologie generali: il Piano Materiale, i Piani di Transizione, i Piani Interni, i Piani Esterni e i Semipiani.\\

\textit{Piano Materiale}: Il Piano Materiale tende ad essere quello più simile alla terra e a funzionare facendo uso delle stesse regole naturali. La sua “dimensione” dipende dalla campagna: si può conformare solo al mondo di gioco effettivo, o comprendere un intero universo di pianeti, lune, stelle e galassie. Il Piano Materiale è il piano di base per il gioco.\\
\textit{Piani di Transizione}: Questi piani hanno un importante elemento in comune: sono tutti sovrapposti agli altri piani e servono a viaggiare tra realtà in sovrapposizione. Questi piani sono fortemente interconnessi col Piano Materiale, ed è possibile accedervi usando numerosi incantesimi. Sono dotati anche essi di abitanti nativi. Qui di seguito sono descritti alcuni Piani di Transizione.\\
\begin{itemize}
\item
\textit{Piano Astrale}: Un vuoto argenteo che connette il Piano Materiale ai Piani Interni e ai Piani Esterni, il Piano Astrale è il mezzo attraverso cui le anime dei defunti giungono all'aldilà. Un viaggiatore nel Piano Astrale vede il piano come un infinito vuoto periodicamente punteggiato da minuscoli sprazzi di realtà fisica distaccatisi dagli innumerevoli piani sovrapposti. Incantatori potenti utilizzano il Piano Astrale per una breve frazione di secondo quando si teletrasportano, o possono usarlo per viaggiare tra i piani con incantesimi come Proiezione Astrale.\\
\item
\textit{Piano Etereo}: Il Piano Etereo è una dimensione nebulosa e nascosta coesistente col Piano Materiale e il Piano delle Ombre. I viaggiatori che attraversano il Piano Etereo sperimentano il mondo reale come fosse insostanziale e si possono muovere tra gli oggetti solidi senza essere visti nel mondo reale. Creature bizzarre abitano il Piano Etereo, così come fantasmi e sogni, molte delle quali possono a volte estendere la loro influenza nel mondo reale in modi misteriosi e terrificanti. Incantatori potenti utilizzano il Piano Etereo con incantesimi come Forma Eterea, Intermittenza e Transizione Eterea.\\
\item
\textit{Piano delle Ombre}: Il misterioso e mortale Piano delle Ombre è una versione grigia e priva di colori del Piano Materiale. Si sovrappone al Piano Materiale ma è più piccolo, ed è per molti versi un “riflesso” distorto e perverso del Piano Materiale, infuso di energia negativa (vedi Piani Interni) e abitato da terribili mostri come ombre o creature ancora peggiori. Incantatori potenti utilizzano il Piano delle Ombre per percorrere rapidamente immense distanze sul Piano Materiale con Camminare nelle Ombre, o attingono all'essenza mutevole del piano per creare effetti e mostri quasi reali con gli incantesimi Ombra di una Evocazione o Ombre.\\
\end{itemize}
\medskip
\textit{Piani Interni}: Questi piani sono le incarnazioni degli elementi base che costruiscono l'universo. Possono essere visti come “contenenti” il Piano Materiale, ma non sono ad esso sovrapposti come i Piani di Transizione. Sono composti da un unico tipo di energia o di elemento, superiore a tutti gli altri. Gli stessi abitanti di uno specifico Piano Interno sono composti dall'elemento del piano. Tra i Piani Interni ci sono:\\
\medskip
\begin{itemize}
\item
Piani Elementali: I quattro classici Piani Interni sono Piano dell'Acqua, Piano dell'Aria, Piano del Fuoco e Piano della Terra. Da questi piani provengono le creature note come elementali, ma sono abitati anche da altre bizzarre creature, come geni, xorn, mephit e cacciatori invisibili.\\
\item
Piani di Energia: Esistono due piani di energia, Il Piano dell'Energia Positiva (da cui provengono le scintille vitali) ed il Piano dell'Energia Negativa (da cui proviene la corruzione della non morte). L'energia di entrambi i piani è infusa nella realtà, ed il flusso di questa energia scorre in tutte le creature dalla nascita alla morte. I Devoti utilizzano il potere di questi due piani quando Incanalano Energia.\\
\end{itemize}
\medskip
\textit{Piani Esterni}: Oltre i regni mortali, oltre gli elementi della realtà, ci sono i Piani Esterni. Vasti oltre ogni immaginazione è ad essi che giungono le anime dei morti ed è qui che dimorano le divinità. Ognuno di essi ha un suo allineamento, che rappresenta un aspetto morale o etico particolare, e i loro abitanti tendono a comportarsi seguendo questo allineamento. I Piani Esterni sono anche il luogo del riposo finale degli spiriti provenienti dal Piano Materiale, sia che siano destinati a una calma introspezione che alla dannazione eterna. Gli abitanti dei Piani Esterni formano le mitologie delle civiltà, comprendendo angeli e demoni, titani e diavoli, e innumerevoli altre incarnazioni del possibile. Ogni mondo di gioco dovrebbe avere Piani Esterni diversi che si conformino ai temi e alle necessità specifiche, ma i classici Piani Esterni includono il Paradiso (Legale Buono), l'Abisso (Caotico Malvagio), l'Inferno (Legale Malvagio) e l'Elysium (Caotico Buono). Incantatori potenti possono entrare in contatto con i Piani Esterni per guida e consiglio con incantesimi come Comunione e Contattare Altri Piani, o possono evocare alleati con Evoca Mostri e Evoca Alleato Naturale.\\
\textit{Semipiani}: Questa categoria serve a raccogliere tutti gli altri spazi extradimensionali che funzionano come i piani ma che hanno accesso e dimensioni misurabili e limitate. Gli altri tipi di piani hanno in teoria dimensioni infinite, ma un semipiano potrebbe essere lungo anche soltanto poche centinaia di metri. Ci sono innumerevoli semipiani alla deriva nella realtà, e mentre molti sono connessi al Piano Astrale e al Piano Etereo, altri sono tagliati completamente fuori dai Piani di Transizione e sono raggiungibili solo attraverso portali ben nascosti o magie oscure.\\

\subsection{Piani a Più Strati}
L'infinito può essere ripartito in infiniti più piccoli e i piani in piani più piccoli correlati tra loro. Questi strati sono a tutti gli effetti dei piani di esistenza separati, e ogni strato può avere i suoi particolari tratti planari. Gli strati sono collegati tra loro attraverso numerosi portali planari, vortici naturali, sentieri e confini mutevoli.\\

L'accesso a un piano a più strati da un'altra provenienza di solito avviene su uno strato specifico, il primo strato del piano, che può essere sia quello più in alto che quello più in basso, in base al piano in questione. Molti punti d'accesso fissi (come portali e vortici naturali) conducono fino a questo strato, che diventa il portale d'accesso verso gli altri strati del piano. Anche l'incantesimo \emph{Spostamento Planare} deposita l'incantatore sul primo strato del piano.\\

\subsection{Interazione Planare}
Due piani che sono separati tra loro non si sovrappongono e non si collegano direttamente l'uno all'altro. Sono come pianeti su orbite diverse. Il solo modo di spostarsi da un piano all'altro è attraversare un terzo piano, come un Piano di Transizione.\\

\textit{Piani Adiacenti}: Quei piani che si collegano gli uni agli altri in punti specifici vengono considerati adiacenti. Laddove si toccano esiste una connessione attraverso la quale i viaggiatori possono uscire da una realtà ed entrare nell'altra.\\
\textit{Piani Coesistenti}: Se è possibile creare in ogni punto un legame tra due piani, i due piani sono coesistenti. Questi piani si sovrappongono l'uno all'altro completamente. Un piano coesistente può essere raggiunto da qualsiasi punto del piano a cui è sovrapposto. Quando ci si muove su un piano coesistente, spesso si può vedere o interagire col piano ad esso sovrapposto.

\subsection{Tratti Planari}
Ogni piano di esistenza ha le sue peculiarità; le leggi naturali del suo universo. Le caratteristiche planari si suddividono in aree generali. Tutti i piani hanno i seguenti tipi di tratti.\\

\textit{Tratti Fisici}: Determinano le leggi fisiche e naturali del piano, compreso il funzionamento della gravità e del tempo.\\
\textit{Tratti Elementali ed Energetici}: L'influenza di forze elementali ed energetiche è determinata da questi tratti.\\
\textit{Tratti di Allineamento}: Proprio come i personaggi possono essere legali neutrali o caotici buoni, così molti piani sono legati ad una particolare morale o etica.\\
\textit{Tratti Magici}: La magia funziona in modo diverso da piano a piano; i tratti magici delimitano il confine tra ciò che la magia può fare e non può fare su ogni piano.\\
\textit{Tratti Fisici}\\
Le due più importanti leggi naturali determinate dai tratti fisici riguardano la funzione della gravità e del tempo. Altri tratti fisici riguardano la grandezza e la forma di un piano ed il modo con cui si possa alterarne la natura.\\

\subsection{Gravità}

La direzione di attrazione gravitazionale può essere inusuale, e potrebbe addirittura cambiare direzioni all'interno dello stesso piano.\\

\subsection{Tempo}

Il ritmo con cui trascorre il tempo può variare nei diversi piani, sebbene rimanga costante all'interno di un qualunque piano specifico. Il tempo è sempre soggettivo per lo spettatore. La stessa soggettività si applica ai vari piani. I viaggiatori potrebbero scoprire che stanno guadagnando o perdendo tempo muovendosi tra i piani, ma dal loro punto di vista il tempo trascorre in modo naturale.\\

\textit{Tempo Normale}: Definisce il trascorrere del tempo sul Piano Materiale. Un'ora su un piano caratterizzato da tempo normale equivale ad 1 ora sul Piano Materiale. A meno che non sia diversamente specificato nella descrizione di un piano, si presume che esso sia caratterizzato da tempo normale.
\textit{Tempo Irregolare}: Alcuni piani sono caratterizzati da un tempo che rallenta ed accelera, per cui un individuo potrebbe perdere o guadagnare tempo mentre si muove tra piani come questo ed altri. Per gli abitanti di un simìle piano, il tempo trascorre in modo naturale e lo spostamento passa inosservato.\\

\textit{Tempo Fluente}: Su alcuni piani, il flusso temporale è considerevolmente più veloce o più lento. Qualcuno potrebbe viaggiare verso un altro piano, trascorrervi un anno e poi ritornare al Piano Materiale per scoprire che sono passati soltanto 6 secondi. Qualsiasi cosa del piano in cui si è tornati ha vissuto appena qualche secondo in più. Per il viaggiatore e gli oggetti, gli incantesimi e gli effetti in funzione su di lui, quell'anno di lontananza è stato completamente reale. Quando progettate il funzionamento del tempo su piani con tempo fluente, pensate per prima cosa al flusso temporale del Piano Materiale, dopodiché al flusso presente nell'altro piano.\\

\textit{Assenza di Tempo}: Sui piani che presentano questo tratto il tempo trascorre ma i suoi effetti sono limitati. Il modo in cui il tratto assenza di tempo influenza determinate attività e condizioni quali la fame, la sete, l'invecchiamento, gli effetti del veleno e la guarigione varia da piano a piano. Il pericolo di un piano con assenza di tempo è che quando un individuo abbandona tale piano per giungere in un altro dove il tempo scorre normalmente, condizioni come la fame e l'invecchiamento si verificano con effetto retroattivo. Se un piano ha assenza di tempo in relazione alla magia, qualsiasi incantesimo lanciato con durata non istantanea diviene permanente finché dissolto.\\

\subsection{Tratti Elementali ed Energetici}

Quattro elementi base e due tipologie di energia si combinano per plasmare ogni cosa; gli elementi sono acqua, aria, fuoco e terra; le tipologie di energia sono positiva e negativa. Il Piano Materiale rispecchia un bilanciamento di questi elementi ed energie: in esso è possibile trovarle tutte. Ognuno dei Piani Interni è dominato da un elemento o tipologia di energia. Altri piani potrebbero mostrare degli aspetti di questi tratti elementali. Molti piani non hanno alcuna caratteristica elementale o energetica; tali tratti vengono specificati nella descrizione di un piano solo se presenti.\\

\textit{Acqua Dominante}: I piani con questo tratto sono per lo più liquidi. I visitatori che non possono respirare sott'acqua o che non riescono a raggiungere una sacca d'aria probabilmente muoiono annegati. Le creature del Fuoco si trovano estremamente a disagio nei piani con acqua dominante. Tali creature, costituite di fuoco, subiscono 1d10 danni ogni round.\\
\textit{Aria Dominante}: Costituiti essenzialmente da spazio aperto, i piani con questo tratto ospitano giusto qualche pezzo di pietra fluttuante o di altro materiale solido. Di solito sono caratterizzati da atmosfera respirabile, sebbene un piano simile potrebbe presentare nubi di gas acido o tossico. Le creature del Sottotipo Terra si trovano a disagio sui piani con aria dominante vista la scarsa quantità o assenza di terra naturale con cui entrare in contatto. Tuttavia, esse non subiscono alcun danno effettivo.\\
\textit{Terra Dominante}: I piani con questo tratto sono per lo più solidi. I viaggiatori che vi giungono sono a rischio di soffocamento a meno che non raggiungano una caverna o un altro anfratto. Peggio ancora, gli individui che non possiedono la capacità di Scavare restano intrappolati sotto terra e devono scavare da sè una via d'uscita (1 metrio per round). 
Le creature del Sottotipo Aria si trovano a disagio sui piani con terra dominante visto che li considerano angusti e claustrofobici, ma a parte avere difficoltà nei movimenti non incappano in altri inconvenienti.\\
\textit{Fuoco Dominante}: I piani con questo tratto sono costituiti da fiamme che bruciano continuamente senza esaurire la loro fonte di alimentazione. I piani con fuoco dominante sono estremamente ostili per le creature del Piano Materiale, e coloro che non hanno Resistenza o Immunità al fuoco vengono inceneriti in poco tempo. Legno, carta, stoffa senza protezione ed altri materiali infiammabili prendono fuoco quasi istantaneamente, così come coloro che indossano vestiario non protetto ed infiammabile. In aggiunta, gli individui subiscono 3d10 danni da fuoco per ogni round in cui restano in un piano con fuoco dominante. Le creature del Sottotipo Acqua si trovano estremamente a disagio sui piani con fuoco dominante. Tali creature, costituite d'acqua, subiscono il doppio del danno ogni round.\\
\textit{Energia Negativa Dominante}: I piani cono questo tratto sono recessi vasti e vuoti che risucchiano l'essenza vitale dei viaggiatori che li attraversano. Tendono ad essere piani desertici e tormentati, spogliati del colore e riempiti da venti che trasportano i deboli lamenti di coloro che sono morti al loro interno. Esistono due tipi di tratti basati su energia negativa dominante: energia negativa dominante minore e superiore. Nei primi, le creature viventi subiscono 1d6 danni per round. A 0 Punti Ferita o meno, queste si riducono in cenere.\\
I secondi sono persino più pericolosi. Ogni round, coloro che ne sono all'interno devono effettuare un Tiro Salvezza su Tempra con CD 25 o subiscono un Livello Negativo. Una creatura che ha tanti Livelli Negativi quanti sono i suoi livelli effettivi o i suoi Dadi Vita muore, diventando un Wraith. L'incantesimo Interdizione alla Morte protegge il viaggiatore dal danno e dal risucchio d'energia di un piano con energia negativa dominante.\\
\textit{Energia Positiva Dominante}: L'abbondanza di vita contraddistingue i piani che presentano questo tratto. Come per i piani con energia negativa dominante, anche i piani con energia positiva dominante possono essere minori e superiori.
Un piano con energia positiva dominante minore è una tumultuosa esplosione di vita in tutte le sue forme. I colori sono più luminosi, i fuochi più caldi, i rumori più forti e le sensazioni più intense grazie all'energia positiva difusa nel piano. Tutti gli individui in un piano con energia positiva dominante rigenerano 2 PF a round.\\
I piani con energia positiva dominante superiore vanno persino oltre. Una creatura su uno di questi piani deve effettuare un Tiro Salvezza su Tempra con CD 15 per evitare di rimanere Accecata per 10 round dalla luminescenza dei dintorni. Il semplice fatto di trovarsi sul piano conferisce rigenerazione 5pf a round. In aggiunta, coloro che hanno il massimo dei propri Punti Ferita guadagnano 5 Punti Ferita Temporanei aggiuntivi per round. Tali Punti Ferita Temporanei svaniscono 1d20 round dopo che la creatura ha lasciato il piano. Tuttavia, una creatura deve effettuare un Tiro Salvezza su Tempra con CD 20 per ogni round che questi Punti Ferita Temporanei superano i suoi normali Punti Ferita totali. Fallire questo Tiro Salvezza fa sì che la creatura esploda in un tripudio d'energia, morendo.\\

\subsection{Tratti di Allineamento}

Alcuni piani hanno una predisposizione verso uno specifico allineamento. Gli abitanti di questi piani condividono per lo più tale allineamento, persino creature potenti come le divinità. Il tratto di allineamento di un piano ne influenza le interazioni sociali. I personaggi che hanno allineamenti diversi da quelli della maggior parte degli abitanti potrebbero avere delle difficoltà quando si confrontano con i nativi e le situazioni del piano. I tratti di allineamento hanno molteplici componenti. Prima di tutto ci sono le componenti morali (buono o malvagio) ed etiche (legale o caotico); un piano può avere una componente morale, una etica, o entrambe. In secondo luogo, lo specifico tratto di allineamento indica se ciascuna componente morale o etica si manifesta in maniera moderata o in modo più marcato. Molti piani non hanno tratti di allineamento; quest'ultimi sono specificati nella descrizione di un piano solo se presenti\\

\textit{Bene/Male}: Questi piani hanno scelto di assumere una posizione nella lotta bene contro male. Nessun piano può essere allo stesso tempo allineato verso il bene e verso il male.
Legge/Caos: Legge contro caos rappresenta il conflitto chiave per questo tipo di piani e per i loro abitanti. Nessun piano può essere allo stesso tempo allineato verso la legge o verso il caos.\\
\textit{Neutrale}: Trattasi di quei piani estranei al conflitto tra bene e male e tra legge e caos.
\textit{Moderatamente Allineato}: Le creature con allineamento opposto a quello di un piano moderatamente allineato subiscono penalità  -2 a tutte le prove basate sul Magnetismo. La penalità non si applica ad un piano moderatamente neutrale.\\
\textit{Fortemente Allineato}: Sui piani fortemente allineati si applica penalità di circostanza -2 a tutte le prove basate su Intelletto, Volontà e Magnetismo effettuate da tutte le creature con allineamento diverso da quello del piano. Le penalità derivanti dalle componenti morali ed etiche della caratteristica di allineamento sono cumulative.\\
Un piano fortemente neutrale è in opposizione a tutti gli altri principi morali ed etici: bene, male, legge e caos. Un piano simile potrebbe focalizzarsi più sull'equilibrio tra allineamenti che sull'accogliere ed accettare punti di vista alternativi. Come gli altri piani fortemente allineati, i piani fortemente neutrali applicano penalità di circostanza -2 a tutte le prove basate su Intelletto, Volontà o Magnetismo effettuate da qualsiasi creatura non neutrale. La penalità viene applicata due volte (una per legge/caos, l'altra per bene/male, così che le creature neutrali buone, neutrali malvagie, legali neutrali e caotiche neutrali subiscono penalità -2, mentre le creature legali buone, caotiche buone, caotiche malvagie e legali malvagie subiscono penalità -4.\\

\subsection{Tratti Magici}

Il tratto magico di un piano definisce come opera, la magia su quel piano rispetto al Piano Materiale. In luoghi particolari di un piano (come quelli sotto il diretto controllo delle divinità) potrebbero applicarsi un diverso tratto magico.\\

\textit{Magia Normale}: Questo tratto magico implica che tutti gli incantesimi e le capacità soprannaturali agiscano come da descrizione. A meno che non sia diversamente descritto, un piano si presume che abbia il tratto magia normale.\\
\textit{Magia Morta}: Contraddistingue i piani dove la magia non esiste affatto. Un piano con la caratteristica magia morta funziona sotto tutti gli aspetti come un Campo Antimagia. Gli incantesimi di Divinazione/Essenze di Rivelazione non possono individuare qualcuno che si trovi in un piano di magia morta, né un incantatore può utilizzare l'incantesimo Teletrasporto/Essenza Movimento per muoversi dentro e fuori di esso. L'unica eccezione alla regola "assenza di magia" è rappresentata dai portali planari permanenti, che funzionano comunque normalmente.\\
\textit{Magia Potenziata}: Sui piani con questo tratto tratto, particolari incantesimi e capacità magiche sono più facili da usare o producono effetti più potenti rispetto a come operano nel Piano Materiale. I nativi di un piano con il tratto magia potenziata sono consapevoli di quali incantesimi e capacità magiche siano potenziate, ma i viaggiatori planari potrebbero scoprirlo di loro iniziativa. Se un incantesimo è potenziato, esso funziona come se avesse fatto un critico nella prova di magia.\\
\textit{Magia Ostacolata}: Particolari incantesimi e capacità magiche sono più difficili da utilizzare sui piani con questo tratto, spesso perché la natura del piano li ostacola. Per lanciare un incantesimo ostacolato, l'incantatore deve effettuare una prova di Concentrazione (CD 20 + il livello dell'incantesimo). Se la prova fallisce, l'incantesimo non ha effetto ma viene comunque sprecato o come fosse un incantesimo preparato o uno slot incantesimo. Se la prova riesce, l'incantesimo ha effetto normalmente.\\
\textit{Magia Limitata}: I piani con questo tratto consentono il solo uso di incantesimi e capacità magiche che soddisfino particolari requisiti. La magia può essere limitata nei suoi effetti da determinate scuole o sottoscuole, da effetti con certi descrittori o da effetti di un dato livello (o da qualsiasi combinazione di questi aspetti). Gli incantesimi e le capacità magiche che non soddisfano i requisiti semplicemente non hanno effetto.\\
\textit{Magia Selvaggia}: Su un piano con il tratto magia selvaggia, incantesimi e capacità magiche funzionano in modo totalmente diverso e a volte pericoloso. C'è la possibilità che qualsiasi incantesimo o capacità magica utilizzata su un piano di magia selvaggia non abbia effetto. Quando l'incantatore lancia una magia deve effettuare due prove se anche solo una fallisce comporta che accada qualcosa di insolito; tirate un d100 e consultate la

\medskip

\textbf{Tabella: Effetti della Magia Selvaggia.}

\medskip

\begin{tabularx}{0.95\textwidth}{lX}
d100	&Effetto\\
01-19	&L'incantesimo rimbalza sull'incantatore con effetto normale. Se l'incantesimo non può influenzare l'incantatore, non produce alcun effetto.\\
20-23	&Una fossa circolare del diametro di 4,5 metri si apre sotto i piedi dell'incantatore; la sua profondità è di 3 metri per livello dell'incantatore.\\
24-27	&L'incantesimo non ha effetto, ma il bersaglio o i bersagli di quest'ultimo vengono colpiti da una pioggia di piccoli oggetti (qualsiasi cosa, dai fiori alla frutta rancida), che scompaiono non appena hanno colpito. L'attacco continua per 1 round. Durante questo periodo i bersagli sono Accecati e devono effettuare prove di Concentrazione (CD 15 + il livello dell'incantesimo) per lanciare incantesimi.\\
28-31	&L'incantesimo colpisce un bersaglio o un'area casuale. Scegliete in modo casuale un bersaglio differente fra quelli entro il raggio d'azione dell'incantesimo o centrate quest'ultimo in un luogo a caso che rientri in tale raggio d'azione. Per generare casualmente la direzione, tirate 1d8 e contate in senso orario, partendo da sud. Per generare casualmente il raggio d'azione tirate 3d6. Moltiplicate il risultato per 1 metro per gli incantesimi a corto raggio, 6 metri per quelli a medio raggio e 24 metri per quelli a lungo raggio.\\
32-35	&L'incantesimo funziona normalmente, ma qualsiasi componente materiale non viene consumata. L'incantesimo non scompare dalla mente dell'incantatore (uno slot incantesimo o un incantesimo preparato possono ancora essere utilizzati). Allo stesso modo, un oggetto non perde cariche e l'effetto non influenza il limite di utilizzo di un oggetto o di una capacità magica.\\
36-39	&L'incantesimo non ha effetto. Invece, qualcuno (amico o nemico) entro 9 metri dall'incantatore riceve l'effetto di un incantesimo Guarigione.\\
40-43	&L'incantesimo non ha effetto. Invece, degli effetti di Oscurità Profonda e Silenzio coprono un raggio di 9 metri attorno all'incantatore per 2d4 round.\\
44-47	&L'incantesimo non ha effetto. Invece, un effetto di Inversione della Gravità copre un raggio di 9 metri attorno all'incantatore per 1 round.\\
48-51	&L'incantesimo ha effetto, ma dei colori scintillanti turbinano attorno all'incantatore per 1d4 round. Considerate quest'area come un effetto di Polvere Luccicante con un Tiro Salvezza con CD 10 + il livello dell'incantesimo che ha generato tale risultato.\\
52-59	&Non accade nulla. L'incantesimo non ha effetto. Qualsiasi componente materiale viene utilizzata. L'incantesimo o lo slot incantesimo viene utilizzato, un oggetto perde cariche e l'effetto influenza il limite di utilizzo di un oggetto o di una capacità magica.\\
\end{tabularx}
\begin{tabularx}{0.95\textwidth}{lX}
60-71	&Non accade nulla. L'incantesimo non ha effetto. Qualsiasi componente materiale non viene utilizzata. L'incantesimo non scompare dalla mente dell'incantatore (uno slot incantesimo o un incantesimo preparato possono ancora essere utilizzati). Un oggetto non perde cariche e l'effetto non influenza il limite di utilizzo di un oggetto o di una capacità magica.\\
72-98	&L'incantesimo ha effetto normalmente.\\
99-100	&L'incantesimo ha effetto potenziato. La prova di magia genera automaticamente un critico\\
\end{tabularx}
\medskip

\subsection{Piano Materiale}
Il Piano Materiale è il fulcro della maggior parte delle cosmologie e definisce cosa può considerarsi normale. Si tratta del piano su cui si focalizzano gran parte delle campagne.
Il Piano Materiale presenta i seguenti tratti:\\
\textit{Gravità Normale}\\
\textit{Tempo Normale}\\
\textit{Nessun Tratto Elementale o Energetico}: Tuttavia, luoghi specifici potrebbero presentare tali tratti.\\
\textit{Moderatamente Neutrale}: Anche se in alcuni punti potrebbe presentare elevate concentrazioni di male o bene, legge o caos.\\
Magia Normale\\

\subsection{Piano delle Ombre}
Il Piano delle Ombre è una dimensione poco illuminata che allo stesso tempo coincide e coesiste con il Piano Materiale. Si sovrappone al Piano Materiale tanto quanto al Piano Etereo, per cui il viaggiatore planare può sfruttare il Piano delle Ombre per coprire grandi distanze rapidamente. Il Piano delle Ombre coincide anche con altri piani. Tramite l'incantesimo giusto, un personaggio può servirsi del Piano delle Ombre per visitare altre realtà. Il Piano delle Ombre è un mondo in bianco e nero: l'ambiente è privo di colori. Se non fosse per questo, somiglierebbe al Piano Materiale. Nonostante l'assenza di fonti luminose, alcune piante, animali e umanoidi, considerano il Piano delle Ombre come loro dimora.\\
Il Piano delle Ombre presenta i seguenti tratti:\\
\textit{Geografia imperfetta}: Parti del Piano delle Ombre fluiscono continuamente verso altri piani. Dunque, nonostante la presenza di punti di riferimento, creare una mappa precisa del piano è quasi impossibile. Inoltre, alcuni incantesimi, come Ombra di una Evocazione e Ombra di una Invocazione, modificano la struttura di base del Piano delle Ombre. Questi incantesimi all'interno del Piano delle Ombre sono particolarmente utili sia per gli esploratori che per i nativi .\\
\textit{Moderatamente Neutrale}\\
\textit{Magia Potenziata}: Gli incantesimi che lavorano con l'ombra  vengono potenziati sul Piano delle Ombre. Inoltre, specifici incantesimi diventano più potenti. Gli incantesimi Ombra di una Evocazione e Ombra di una Invocazione hanno il 30\% della potenza degli incantesimi che copiano (invece del 20\%). Ombra di una Evocazione Superiore e Ombra di una Invocazione Superiore hanno il 70\% della potenza degli incantesimi che copiano (invece del 60\%), mentre un incantesimo Ombre il 90\% (invece dell'8o\%). Nonostante la natura oscura del Piano delle Ombre, gli incantesimi che generano, utilizzano o manipolano l'oscurità non vengono influenzati dal piano.\\
\textit{Magia Ostacolata}: Gli incantesimi di luce o che utilizzino o generino luce o fuoco vengono ostacolati sul Piano delle Ombre. Gli incantesimi che generano luce sono meno efficaci in generale, dal momento che su questo piano tutte le fonti luminose hanno raggio d'azione dimezzato.\\

\subsection{Piano dell'Energia Negativa}
Per un osservatore c'è ben poco da vedere sul Piano dell'Energia Negativa. È un luogo buio e vuoto, una fossa infinita in cui il viaggiatore potrebbe precipitare finché il piano non abbia cancellato luce e vita. Il Piano dell'Energia Negativa è il più ostile fra i Piani Interni, il più indifferente ed intollerante nei confronti della vita. Soltanto le creature immuni ai suoi effetti di risucchio possono sopravvivere qui.\\
Il Piano dell'Energia Negativa presenta i seguenti tratti:\\

\textit{Energia Negativa Dominante Superiore}: Alcune zone all'interno del piano hanno solo il tratto energia negativa dominante minore, ma queste isole tendono ad essere disabitate.\\
\textit{Magia Potenziata}: Incantesimi e capacità magiche che utilizzano l'energia negativa vengono potenziati. Le Abilità che sfruttano l'energia negativa, come Incanalare Energia negativa, ottengono bonus +4 alla CD del Tiro Salvezza per resistere alla capacità.\\
\textit{Magia Ostacolata}: Incantesimi e capacità magiche che utilizzano l'energia positiva (inclusi gli incantesimi di guarigione) vengono ostacolati. I personaggi su questo piano subiscono penalità ­-10 ai Tiri Salvezza effettuati per rimuovere Livelli Negativi causati da un attacco di risucchio d'energia.\\

\subsection{Piano dell'Energia Positiva}
Il Piano dell'Energia Positiva non ha superficie ed è simile al Piano dell'Aria con il suo spazio totalmente aperto. Tuttavia, ogni angolo di questo piano è illuminato vivacemente da una potenza innata. Tale potere è pericoloso per le forme mortali, non predisposte a subirlo. Nonostante gli effetti benefici è uno dei Piani Interni più ostili. Un personaggio sprovvisto di difese, traboccherà di potenza non appena l'energia positiva viene convogliata su di lui. Ma, visto che la sua forma mortale non è in grado di contenere tale potere, verrà incenerito, come un granello di polvere catturato all'estremità di una supernova. Le visite al Piano dell'Energia Positiva sono di breve durata, ed anche in tal caso i viaggiatori devono essere adeguatamente protetti.
Il Piano dell'Energia Positiva presenta i seguenti tratti:\\

\textit{Energia Positiva Dominante Superiore}: Alcune regione del piano, invece, sono caratterizzate da energia positiva dominante minore, ma tali isole tendono ad essere disabilitate.\\
\textit{Magia Potenziata}: Incantesimi e capacità magiche che usano l'energia positiva vengono potenziati. Le Abilità che sfruttano l'energia positiva, come Incanalare Energia positiva, ottengono bonus +4 alla CD per resistere alla capacità.\\
\textit{Magia Ostacolata}: Incantesimi e capacità magiche che utilizzano l'energia negativa (inclusi gli incantesimi infliggi) vengono ostacolati.\\

\subsection{Piano dell'Acqua}
Il Piano dell'Acqua è un mare senza fondale o superficie, un ambiente liquido illuminato da una luce diffusa. E' uno dei Piani Interni più ospitali, una volta che il viaggiatore supera il problema di respirare sott'acqua. Gli infiniti oceani di questo piano spaziano tra il freddo gelido e il caldo incandescente e tra acqua salata e acqua dolce. Gli insediamenti permanenti del piano si generano attorno a pezzi di relitti sospesi in questo fluido senza fine, andando alla deriva con le maree.\\
Il Piano dell'Acqua presenta i seguenti tratti:\\

\textit{Acqua Dominante}
\textit{Magia Potenziata}: Incantesimi e capacità magiche che usano l'acqua vengono potenziati.\\
\textit{Magia Ostacolata}: Incantesimi e capacità magiche che usano o creano fuoco  e gli incantesimi che evocano elementali del fuoco o Esterni del Sottotipo Fuoco) vengono ostacolati.\\

\subsection{Piano dell'Aria}
Il Piano dell'Aria è un piano vuoto, costituito da cielo in ogni direzione. Si tratta del più confortevole e vivibile dei piani interni ed è la dimora di tutti i generi di creature dell'aria. Infatti, le creature volanti ottengono grande vantaggio su questo piano. Sebbene i viaggiatori possano sopravvivere bene qui anche senza la capacità di volare, sono comunque svantaggiati.
Il Piano dell'Aria presenta i seguenti tratti:\\

\textit{Aria Dominante}\\
\textit{Magia Potenziata}: Incantesimi e capacità magiche che usano, manipolano o generano aria vengono potenziati.\\
\textit{Magia Ostacolata}: Incantesimi e capacità magiche che usano o generano terra e gli incantesimi che evocano elementali della terra o Esterni del Sottotipo Terra) vengono ostacolate.\\

\subsection{Piano del Fuoco}
Sul Piano del Fuoco ogni cosa è illuminata. Il suolo è costituito nient'altro che da vasti e mutevoli strati di fuoco compresso. L'aria viene smossa dal calore delle continue piogge di fuoco ed il liquido più comune è il magma. Gli oceani sono composti da fiamma liquida e le montagne fanno colare lava fusa. Qui il fuoco perdura senza alimentazione o aria, ma gli elementi infiammabili introdotti sul piano vengono consumati rapidamente.\\
Il Piano del Fuoco presenta i seguenti tratti:\\

\textit{Fuoco Dominante}
\textit{Magia Potenziata}: Incantesimi e capacità magiche che utilizzano, manipolano o generano fuoco vengono potenziati.\\
\textit{Magia Ostacolata}: Incantesimi e capacità magiche che usano o generano acqua e gli incantesimi che evocano elementali dell'acqua o Esterni del Sottotipo Acqua) vengono ostacolati.\\

\subsection{Piano della Terra}
Il Piano della Terra è un luogo solido composto da terra e pietra. Un viaggiatore imprudente potrebbe ritrovarsi seppellito da questa vasta massa solida: i suoi resti polverizzati resteranno di monito per chi oserà seguirlo. Nonostante la sua natura solida e rigida, il Piano della Terra ha consistenza variabile, spaziando da terreno soffice a vene di metallo più duro e prezioso.\\
Il Piano della Terra presenta i seguenti tratti:\\

\subsection{Terra Dominante}
\textit{Magia Potenziata}: Incantesimi e capacità magiche che utilizzano, manipolano o generano terra o pietra vengono potenziati.\\
\textit{Magia Ostacolata}: Incantesimi e capacità magiche che utilizzano o generano aria e gli incantesimi che evocano elementali dell'aria o Esterni del Sottotipo Aria) vengono ostacolati.\\

\subsection{Piano Etereo}
Il Piano Etereo coesiste col Piano Materiale e spesso anche con altri piani. Lo stesso Piano Materiale è visibile dal Piano Etereo, ma appare silenzioso e indistinto; i colori si confondo fra loro e i confini sono sfocati. Sebbene sia possibile vedere il Piano Materiale dal Piano Etereo, quest'ultimo è di solito invisibile a coloro che si trovano sul Piano Materiale. Normalmente, le creature del Piano Etereo non possono attaccare quelle del Piano Materiale e viceversa. Un viaggiatore che si trovi sul Piano Etereo è invisibile, incorporeo e totalmente silenzioso per qualcuno del Piano Materiale.\\
Il Piano Etereo presenta i seguenti tratti:\\

\textit{Assenza di Gravità}\\
\textit{Moderatamente Neutrale}\\
\textit{Magia Normale}: Gli incantesimi funzionano normalmente sul Piano Etereo, anche se non attraversano il Piano Materiale. Le uniche eccezioni sono gli incantesimi e le capacità magiche e che influenzano le entità eteree.\\
Nessun attacco magico passa dal Piano Etereo al Piano Materiale, compresi gli attacchi di forza.\\

\subsection{Piano Astrale}
Il Piano Astrale è lo spazio tra piani interni ed esterni e confina con tutti i piani. Quando un personaggio attraversa un portale o proietta il suo spirito su un altro piano di esistenza, viaggia attraverso il Piano Astrale. Anche gli incantesimi che consentono movimento istantaneo attraverso un piano influenzano per poco il Piano Astrale. Quest'ultimo è una grande distesa infinita di limpido cielo argenteo, sia sopra che sotto. Occasionali pezzi di materia solida possono essere trovati qui, ma la maggior parte del Piano Astrale è uno spazio aperto e sconfinato.\\
Il Piano Astrale presenta i seguenti tratti:\\

\textit{Assenza di Tempo}: L'età, la fame, la sete, le sofferenze (come Malattie, Maledizioni e Veleni) e la guarigione naturale non hanno effetti nel Piano Astrale, sebbene riprendano il proprio funzionamento quando il viaggiatore lascia il piano.\\
\textit{Moderatamente Neutrale}\\
\textit{Magia Potenziata}: Tutti gli incantesimi e le capacità magiche usate nel Piano Astrale hanno velocita' di 1 Azione. Gli incantesimi e le capacità magiche già velocizzati non vengono influenzati, così come gli incantesimi degli oggetti magici. Gli incantesimi velocizzati in tal modo sono comunque preparati e lanciati al loro livello originario.\\

\subsection{Abaddon (Neutrale Malvagio)}
Reame di distese desolate sotto un cielo putrido, Abaddon è avvolto da una nera nebbia nauseante, e dall'opprimente crepuscolo di un'eclissi solare senza fine. Il veneficio Stige nasce in Abaddon, prima di immettersi come un serpente contorto negli altri piani. Abaddon è uno dei Piani Esterni più ostile: regno dei Daemon, immondi di male puro indifferenti al conflitto tra legge e caos, che rappresentano l'oblio e la distruzione. I Daemon governati da quattro arcidaemon con poteri simili a divinità, sono temuti come divoratori di anime.\\
Abaddon presenta i seguenti tratti:\\

\textit{Fortemente Malvagio}\\
\textit{Magia Potenziata}: Gli incantesimi e le capacità magiche malvage vengono potenziati.
\textit{Magia Ostacolata}: Gli incantesimi e le capacità magiche benevole bene vengono ostacolati.

\subsection{L'Abisso (Caotico Malvagio)}
L'Abisso, piano multi-strato, circonda la Sfera Esterna come la buccia estesa di una cipolla; è formato da giganteschi canyon e gole che si spalancano nel tessuto dei Piani Esterni ed è delimitato dalle nefaste acque del Fiume Stige. Gli infiniti strati dell'Abisso, confinanti con tutti i Piani Esterni, sono collegati l'un l'altro da sentieri in costante spostamento. Nell'Abisso non ci sono regole, leggi, ordine o speranza. L'Abisso rappresenta la corruzione della libertà, un regno da incubo di orrore assoluto dove il desiderio e la sofferenza assumono forma demoniaca, terra di proliferazione di innumerevoli razze di Demoni, tra gli esseri più antichi di tutto il creato.\\
L'Abisso presenta i seguenti tratti:\\

\textit{Fortemente Caotico e Fortemente Malvagio}\\
\textit{Magia Potenziata}: Gli incantesimi e le capacità magiche caotiche o malvage vengono potenziati.\\
\textit{Magia Ostacolata}: Gli incantesimi e le capacità magiche legali o buone vengono ostacolati.\\

\subsection{Elysium (Caotico Buono)}
Una vasta terra di natura incontaminata e selvaggia, Elysium è il piano del caos benevolo, della libertà e indipendenza, personificati nei nativi Azata. Nell'Elysium, la cooperazione disinteressata e la feroce competizione si scontrano violentemente, ma tali conflitti non mettono mai in ombra i nobili concetti di coraggio, creatività e bene non ostacolati da regole o leggi.\\
Elysium presenta i seguenti tratti:\\

Fortemente Caotico e Fortemente Buono\\
\textit{Magia Potenziata}: Gli incantesimi e le capacità magiche caotiche o buone vengono potenziati.\\
\textit{Magia Ostacolata}: Gli incantesimi e le capacità magiche legali o malvage vengono ostacolati.\\

\subsection{Inferno (Legale Malvagio)}
I nove strati dell'Inferno formano un labirinto strutturato di male premeditato dove il tormento va di pari passo con la purificazione. Piano di città di ferro, desolazioni in fiamme, ghiacciai congelati e picchi vulcanici infiniti, l'Inferno è suddiviso in nove strati concentrici, ciascuno sotto il crudele dominio di un arcidiavolo. Tortura, angoscia e sofferenza sono inevitabili nell'Inferno, ma sono impartite metodicamente, non per dispetto o capriccio, e supportano un disegno pianificato sotto i vigili sguardi dei disciplinati ranghi di Diavoli minori dell'Inferno. I nove strati dell'Inferno, dal primo all'ultimo, sono Averno, Dite, Erebo, Flegetonte, Stigia, Malebolge, Cocito, Caina e Nessus.\\
L'Inferno presenta i seguenti tratti:\\

\textit{Fortemente Legale e Fortemente Malvagio}\\
\textit{Magia Potenziata}: Gli incantesimi e le capacità magiche legali o malvage vengono potenziati.\\
\textit{Magia Ostacolata}: Gli incantesimi e le capacità magiche caotiche o buone vengono ostacolati.\\

\subsection{Limbo (Caotico Neutrale)}
Un vasto oceano di caos sfrenato e di potenziale inutilizzato circonda ciascuno dei Piani Esterni e confina con essi. Questo è il Limbo: splendido, mortale e davvero infinito. Dalle sue profondità inesplorate sono nati tutti gli altri piani e nelle sue viscere ribelli, alla fine, farà ritorno tutto il creato. Dove il mare informe del Limbo bagna le coste di altri piani, la sua massa assume un certo grado di stabilità, ed è in queste terre di confine che il viaggio è più sicuro, anche se ancora irto di pericoli derivanti dagli abitanti corrotti dal caos del Limbo. Più in profondità nel piano, i Protean nativi del Limbo si tuffano nel Caos Primordiale, creando e distruggendo la materia grezza del caos con incomprensibile trasporto.\
Il Limbo presenta i seguenti tratti:\\

\textit{Tempo Irregolare}\\
\textit{Fortemente Caotico}
\textit{Magia Selvaggia e Magia Normale}: Sulle poche isole stabili del Limbo, è più probabile che la magia sia normale. In qualsiasi altro luogo la magia è selvaggia.\\

\subsection{Nirvana (Neutrale Buono)}
Il Nirvana è un paradiso imparziale esistente tra i due estremi di Elisio e Paradiso. Le sue meravigliose montagne, colline e fitte foreste rispondo alle aspettative di paradiso pastorale da parte del viaggiatore, ma il Nirvana contiene anche misteri che conducono all'illuminazione. Il Nirvana è un santuario e un luogo di riposo per tutti coloro che sono in cerca della redenzione o dell'illuminazione. Gli Agathion nativi del Nirvana hanno volontariamente messo da parte la propria trascendenza per custodire gli enigmi del piano, mentre i celestiali combattono le forze del male presenti fra i piani.\\
il Nirvana presenta i seguenti tratti:\\

\textit{Fortemente Buono}\\
\textit{Magia Potenziata}: Incantesimi e capacità magiche buone vengono potenziati.\\
\textit{Magia Ostacolata}: Incantesimi e capacità magiche malvage vengono ostacolati.\\

\subsection{Paradiso (Legale Buono)}
La svettante montagna del Paradiso torreggia al di sopra della Sfera Esterna. Questo ordinato regno di onore e compassione è suddiviso in sette strati. I declivi del Paradiso sono pieni di città ordinate e ben strutturate e di giardini e frutteti puliti e curati. Sebbene abbiano iniziato le proprie esistenze come mortali, gli Arconti nativi del Paradiso vedono la legge ed il bene come le due metà inscindibili dello stesso sommo concetto e si schierano contro le corruzioni cosmiche del caos e del male.\\
Il Paradiso presenta i seguenti tratti:\\

\textit{Fortemente Legale e Fortemente Buono}\\
\textit{Magia Potenziata}: Gli incantesimi e le capacità magiche legali o buone vengono potenziati.\\
\textit{Magia Ostacolata}: Gli incantesimi e le capacità magiche caotiche o malvage vengono ostacolati.\\

\subsection{Purgatorio (Neutrale)}
Ogni anima transita nel Purgatorio per essere giudicata prima di essere inviata verso la sua destinazione ultima. Vasti cimiteri e terre desolate riempiono le sue cupe distese, insieme a polverosi e riecheggianti tribunali preposti al giudizio dei morti. il Purgatorio è la dimora degli Eoni, una razza che incorpora la dualistica natura dell'esistenza e i cui membri sono costantemente in guerra e in pace fra loro e con se stessi.\\
Il Purgatorio presenta i seguenti tratti:\\

\textit{Assenza di Tempo}: L'età, la fame, la sete, le sofferenze (come Malattie, Maledizioni e Veleni) e la guarigione naturale non hanno effetto nel Purgatorio, sebbene riprendano il proprio funzionamento quando il viaggiatore lascia il piano.\\
\textit{Fortemente Neutrale}\\
\textit{Magia Potenziata}: Incantesimi e capacità magiche che riguardano la morte o riposo vengono potenziati.\\

\subsection{Utopia (Legale Neutrale)}
Utopia è una roccaforte dell'ordine contrapposta al caos del Limbo e alle infinite orde demoniache dell'Abisso. Una grande città di perfezione eterna, le cui strade ed edifici sono modelli di architettura ed estetica: ogni cosa è in ordine e nulla accade per caso. Anche se Utopia non è governata da nessuna razza, Assiomiti ed Inevitabili ne fanno la loro dimora, cercando costantemente di espandere la loro città perfetta.\\
Utopia presenta i seguenti tratti:\\

\textit{Fortemente Legale}\\
\textit{Magia Potenziata}: Gli incantesimi e le capacità magiche legali vengono potenziati.\\
\textit{Magia Ostacolata}: Gli incantesimi e le capacità magiche caotiche vengono ostacolati.\\

\pagebreak
\section{Il Calendario}\index{Calendario}

\label{il-calendario}

Basato sul ciclo presenta 12 mesi da 28 giorni.

Questi i nomi dei mesi a partire da quello che si definisce inizio
anno
\bigskip

1°) Ianas

2°) Prineva

3°) Marc

4°) Epral

5°) Meea

6°) Vernam

7°) Ilai

8°) Arkast

9°) Cester

10°) Koper

11°) Narava

12°) Kartan

\bigskip
La settimana è a sua volta diviso in 7 giorni di nome

Kalint

Iratam

Munrat

Arai

Venran

Kittam

Viltar\\

Il giorno è diviso in 24 ore


\subsubsection{I cicli millenari}\index{I cicli millenari}

\begin{tcolorbox}[enhanced,arc=5pt,boxrule=0.3pt]{
		Vidi poi un angelo che scendeva dal cielo con la chiave dell'Abisso e una gran catena in mano.\\
		Afferrò il dragone, il serpente antico - cioè il diavolo, satana - e lo incatenò per mille anni; \\
		lo gettò nell'Abisso, ve lo rinchiuse e ne sigillò la porta sopra di lui, perché non seducesse più le nazioni, fino al compimento dei mille anni. Dopo questi dovrà essere sciolto per un po' di tempo. (Apocalisse 20,1-3)
	}\end{tcolorbox}\medskip


Dice la storia che ogni mille anni il Yeru muoia per rinascere nuovamente, piu' bello di prima.

Non e' proprio cosi' ma ci si avvicina molto.

E' noto a pochi eruditi di Atmos che ogni mille anni i Patroni riconosciuti e da cui molti traggono i poteri scompaiano e lascino il posto, dopo esattamente 1 anno a nuovi Patroni.

Improvvisamente le Essenze cessano di funzionare, solo gli oggetti magici che possono assorbire e conservare la magia funzionano (come ad esempio una Pozione od un Bastone che abbia delle cariche, ma non oggetti che si ricaricano automaticamente), Devoti e Seguaci non hanno piu' accesso a nessuna Essenza.

Con qualche eccezione. I Patroni della Genesi, Atmos e Lynx  ed il Vincitore sono gli unici a rimanere costanti a non mutare e solo i loro Devoti e Seguaci possono continuare ad usare le Essenze.

A partire dal sesto mese i vecchi seguaci e devoti incominciano a sentire delle voci, a sognare nuovi volti e nuovi Patroni.

Ogni nuovo Patrono, in base ai tratti che comanda, avvicina un credente e cerca di convincerlo ad accettarlo come nuovo Patrono.

Questo Seguace/Devoto dovra' avere almeno due tratti in comune con il nuovo Patrono per essergli Seguace ed almeno 3 per rimanergli Devoto.

Solo al termine dell'anno potranno essere usate le essenze che questi Patroni comandano.

E' un periodo estremamente turbolento ed agitato dove scoppiano guerre e vendette approfittando dell'assenza della magia. Per molti e' un periodi di puro odio e violenza dove vengono sfogati gli istinti piu' bassi sapendo poi che non si sara' giudicati da nessuna divinta'.

La verita' e' che ogni mille anni i Patroni delle Genesi giudicano le lore creature i Patroni, valutando chi ha fatto meglio e chi peggio. E' una sfida tra Calicante ed LJust a chi ha, tramite i Patroni, ottenuto piu' Seguaci o Devoti.

Il Patrono che piu' di tutti si e' dimostrato capace di mantenere e conquistare persone rimarra' anche nel millennio successivo, questo sara' il Vincitore ed i suoi Devoti ne canteranno per mille anni la gloria e potenza.

Inebriato dalla vittoria il Patrono della Genesi esprimera' un desiderio che l'altro dovra' cercare di rispettare il piu' possibile. Ovvio che lui/lei stessa potrebbe manifestarlo ma la soddisfazione di fare fare all'altro qualcosa che detesta e' superiore a ogni cosa.

Ed e' per questo che ogni mille anni succede qualcosa, oltre alla nascita di nuovi Patroni. Puo' essere un continente, mare.. luna, animali... qualcosa di imponente cambia per tutti gli yeruiti. E' un periodo di sconvolgimenti globali.

Solo i sommi Devoti di Atmos conoscono questa verita' come conoscono che i Patroni della Genesi dopo la vittoria giacciono insieme per sei mesi generando i nuovi Patroni.

\bigskip

Per il Narratore: valutate bene quando fare incominciare le vostre campagne, in base alla durata ed all'anno potreste incorrere in questi accadimenti.
Sfruttate a vostro favore e beneficio di avventura il mutamento dei patroni, fate giocare un po' del "riposo" dalle essenze, aiutati i giocatori con personaggi più magici a riprendersi.

\pagebreak

\section{Condizioni}\index{Condizioni}

\label{condizioni}

\textbf{Abbagliato}:\index{Abbagliato} La creatura è incapace di vedere bene a causa di un'eccessiva stimolazione degli occhi. Una creatura abbagliata subisce penalità -1 al Tiro per Colpire e alle prove di Consapevolezza basate sulla vista.

\textbf{Accecato}:\index{Accecato} Il personaggio non riesce a vedere nulla. Subisce penalità -2 alla Difesa, perde il suo bonus di Agilità alla Difesa (se presente), subisce penalità -2 alla maggior parte delle Competenze basate su Potenza e Agilità e alle prove contrapposte di Consapevolezza.

Tutte le prove o le attività basate sulla visione (come ad esempio leggere, o eventuali prove di Consapevolezza basate sulla vista) falliscono automaticamente. Tutti gli avversari vengono considerati dotati di invisibilita' nei confronti del personaggio accecato.

I personaggi accecati devono effettuare una prova di Acrobatica con DC 10 per muoversi più veloci della propria velocità dimezzata. Le creature che falliscono questa prova cadono a terra Prone. I personaggi che rimangono per lungo tempo accecati possono abituarsi ad alcune di queste penalità e iniziare a superarne alcune, a discrezione del Narratore.

\textbf{Accovacciato}:\index{Accovacciato} Un personaggio accovacciato subisce penalità -2 alla Difesa e -1 al Tiro per Colpire, perde due punti di bonus di Agilità (se posseduti).

\textbf{Affascinato}:\index{Affascinato} Una creatura affascinata è soggiogata da un effetto soprannaturale o di una Essenza. La creatura rimane in piedi o si siede tranquilla, senza effettuare alcuna azione se non prestare attenzione alla fonte del fascino, fintanto che dura l'effetto. L'effetto provoca penalità -4 alle Prove di Competenza richieste come reazione, come ad esempio le prove di Consapevolezza.

Qualsiasi potenziale minaccia, come ad esempio una creatura ostile in avvicinamento, consente alla creatura affascinata un nuovo Tiro Salvezza contro l'effetto del fascino. Qualsiasi minaccia palese, come ad esempio qualcuno che estrae un'arma, lancia un'Essenza o punta un'arma a distanza verso la creatura affascinata, interrompe automaticamente l'effetto.

Un alleato della creatura affascinata può scuoterla per liberarla dall'effetto spendendo 2 Azioni.

\textbf{Affaticato}\index{Affaticato}: Un personaggio affaticato non può correre o Caricare e subisce una penalità -1 a Potenza e Agilità. Se compie qualsiasi cosa normalmente affaticante diventa Esausto.

Ci vogliono 8 ore di riposo totale per rimuovere la condizione di affaticato o Cura a LP16 . Se un personaggio non dorme almeno 8 ore alla mattina è affaticato.

\textbf{Afferrato}\index{Afferrato}: Un personaggio afferrato non puo' muoversi, deve usare due Azioni per liberarsi (TS Tempra contrapposto).

Puo' attaccare con armi in mischia se adeguate (difficilmente potra' usare uno spadone, alabarda.. un pugnale o spada corta e' piu' probabile.)

\textbf{Amichevole}:\index{Amichevole} Una creatura amichevole non attaccherà il personaggio se non minacciata esplicitamente.

\textbf{Assordato}:\index{Assordato} Un personaggio assordato non può ascoltare. Subisce penalità -2 alle prove di Iniziativa, fallisce automaticamente tutte le prove di Consapevolezza basate sul suono e ha una probabilità del 20\% di fallire l'uso delle Essenze, presupponiamo che tutte le Essenze abbiano componenti verbali e somatiche.

I personaggi che rimangono assordati per lunghi periodi di tempo, possono abituarsi a queste penalità e superarne alcune, a discrezione del Narratore.

\textbf{Avvelenato}\index{Avvelenato}: si considera avvelenato qualsiasi soggetto sotto l'influenza di un veleno o pozione, indipendentemente che questa stia gia' producendo gli effetti o li debba ancora produrre dato il tempo dell'insorgenza.

\textbf{Charmato}:\index{Charmato} una creatura charmata tratta il giocatore con un fidato amico ed alleato. Se la creatura viene minacciata o attaccata può fare un nuovo Tiro Salvezza su Arbitrio con un +2.

L'effetto di charme non permetto il controllo del target ma questo percepisce le tue parole nel modo più favorevole. Puoi anche dare ordini ma devi riuscire in una prova di Faccia Tosta contro un Tiro Salvezza su Arbitrio.

Un target influenzato da charme non farà nulla di pericoloso per se stesso (tranne se convinto) o per altri soggetti che reputa amici.

\textbf{Confuso}: \index{Confuso}Una creatura confusa è mentalmente ottenebrata e non può agire normalmente. Una creatura confusa non riesce a distinguere un alleato da un nemico e considera tutti come nemici.

Gli alleati che vogliono utilizzare un'Essenza a vantaggio della creatura confusa devono comunque toccarla con un attacco di contatto in mischia riuscito.

Se una creatura confusa è attaccata, attacca sempre l'ultima creatura che la ha attaccata, finché quella creatura non muore o esce dalla sua visuale.

Tirate un dado sulla tabella seguente all'inizio di ogni turno della
creatura confusa ad ogni round per vedere quello che la creatura fa
in quel round.

\textbf{d100 Comportamento:}

01-25 Agisce normalmente

26-50 Non fa altro che balbettare in modo incoerente

51-75 Si infligge 1d8 + modificatore di Potenza con l'arma che tiene in mano

76-100 Attacca la creatura più vicina (a tale scopo, un Famiglio conta come parte del soggetto stesso)

Una creatura confusa che non è in grado di eseguire l'azione indicata non farà altro che balbettare in modo incoerente. Gli aggressori non hanno alcun vantaggio speciale quando attaccano una creatura confusa. Qualsiasi creatura confusa che venga attaccata, attacca automaticamente a sua volta il suo aggressore al suo turno successivo, fintanto che rimane confusa quando giunge il suo turno.

\textbf{Dominato}:\index{Dominato} si è in grado di controllare le azioni di una qualsiasi creatura Umanoide mediante un legame telepatico con la mente del soggetto.

Se si ha un linguaggio in comune, si può generalmente costringere il soggetto ad eseguire i comandi entro i limiti delle sue capacità. Se non si condivide nessun linguaggio, si possono impartire solo comandi di base come "vieni qui", "vai li"', "combatti" o "stai fermo". Si è a conoscenza di ciò che il soggetto sta provando ma non si ricevono percezioni sensoriali dirette da lui, né si può comunicare con lui telepaticamente.

Una volta impartito un ordine alla creatura dominata, questa continua a tentare di eseguirlo con l'esclusione di tutte le altre attività ad eccezione di quelle necessarie per la sopravvivenza quotidiana (come mangiare, dormire e così via). Grazie a questo limitato spettro di attività, una prova di Consapevolezza con DC 15 (invece che DC 25) può determinare se il comportamento del soggetto è stato influenzato da un effetto di ammaliamento.

Concentrandosi completamente sull'Essenza (2 Azioni), si possono ricevere percezioni sensoriali come vengono interpretate dalla mente del soggetto, anche se questo non può comunque comunicarle. Non si può in realtà vedere attraverso gli occhi del soggetto, quindi non è come se si fosse presenti, ma ci si può rendere conto di cosa sta succedendo.

Ovviamente ordini palesemente autodistruttivi non vengono eseguiti. Una volta stabilito il controllo, il raggio di azione entro il quale può essere mantenuto è illimitato purché entrambi i soggetti rimangano sullo stesso piano. Non c'è bisogno di vedere il soggetto per controllarlo. Se ogni giorno non si trascorre almeno 1 minuto a concentrarsi sull'Essenza, il soggetto riceve un nuovo Tiro Salvezza per liberarsi dal controllo.

\textbf{Esausto}:\index{Esausto} Un personaggio esausto si muove a velocità dimezzata e subisce penalità -3 a Potenza e Agilità. Dopo 1 ora di completo riposo (o Cura LP19), un personaggio esausto diventa solo Affaticato. Un personaggio Affaticato diventa esausto compiendo un'azione che normalmente lo affaticherebbe.

\textbf{Frastornato}:\index{Frastornato} La creatura è incapace di agire normalmente.
Una creatura frastornata non può eseguire azioni, ma non subisce penalità alla CA o Difesa. La condizione frastornato dura solitamente 1 round.

\textbf{Immobilizzato}:\index{Immobilizzato} Una creatura immobilizzata è strettamente limitata nei movimenti e può compiere solo alcune azioni.

Una creatura immobilizzata non può muoversi ed è Impreparata. Inoltre subisce una ulteriore penalità -4 alla sua Difesa. Una creatura immobilizzata è limitata nelle azioni che può compiere. Una creatura immobilizzata può tentare sempre di liberarsi, solitamente attraverso una prova di Artista della Fuga o un Tiro Salvezza su riflessi.

Può compiere azioni verbali e mentali, ma non può utilizzare, di norma, le Essenze. Un personaggio immobilizzato che tenta di utilizzare le Essenze o usare una Capacità Magica deve effettuare una prova di Concentrazione.

Se il soggetto è legato la prova è contro la prova di Criminalità di chi ha legato.

\textbf{Impreparato}:\index{Impreparato} Un personaggio che non ha ancora agito in combattimento è impreparato, non potendo ancora reagire alla situazione. Un personaggio impreparato perde il suo bonus di Agilità alla Difesa (se presente)

\textbf{In Lotta}:\index{Lotta} Una creatura in lotta è trattenuta da una creatura,
da una trappola o da un effetto. Le creature in lotta non possono
muoversi e subiscono penalità -2 a Agilità. Una creatura in lotta
subisce penalità -2 a Tiro per Colpire e Difesa. Inoltre, le creature
in lotta non possono compiere azioni che richiedano due mani per essere
effettuate.

\textbf{Incorporeo}:\index{Incorporeo} Le creature di questo tipo non possiedono un corpo fisico. Le creature incorporee possono essere colpite solo da armi magiche con un bonus di +2 o superiore. Dimezzano gli effetti di essenze con DC inferiore o pari a 18. Le creature incorporee subiscono danno pieno da altri soggetti ed effetti incorporei, così come tutti gli effetti di forza.

Una creatura incorporea può entrare in un oggetto corporeo o passarvi attraverso,

Gli attacchi di una creatura incorporea passano attraverso (ignorano) armature non magiche e scudi, solo la naturale Agilità e appunto armature/scudi magici offrono resistenza.

Le creature incorporee possono muoversi ed agire normalmente nell'acqua come nell'aria. Le creature incorporee non possono cadere e subire danni da caduta.

Le creature incorporee non possono effettuare attacchi per Sbilanciare o Lottare, né possono essere sbilanciate o afferrate.

Le creature incorporee non hanno peso, e non fanno scattare trappole attivate dal peso.

Una creatura incorporea si muove sempre silenziosamente e non può essere sentita con Consapevolezza a meno che non lo desideri. Non ha punteggio di Potenza, e si applica il suo bonus di Agilità agli attacchi in mischia ed a distanza

\textbf{Indifeso}:\index{Indifeso} Un personaggio addormentato, bloccato, legato, Paralizzato, Privo di Sensi, Morente o per qualche altro motivo completamente alla mercé dei suoi avversari, è considerato indifeso.

Un personaggio indifeso viene considerato come se avesse Agilità 0 e non si considerano bonus derivanti dallo scudo. Gli attacchi in mischia contro un personaggio indifeso ottengono bonus +1d6 (equivalente ad attaccare un personaggio Prono).

Gli attacchi a distanza, non ottengono alcun bonus particolare contro i bersagli indifesi.

\textbf{Colpo di Grazia}:\index{Colpo di Grazia} Come unica azione nel round, un nemico può utilizzare un'arma da mischia per infliggere un colpo di grazia ad un personaggio indifeso. Un nemico può anche usare un arco o una balestra, l'importante è che sia adiacente al bersaglio.

L'attaccante colpisce automaticamente ed infligge un colpo critico. Se il difensore sopravvive, deve superare un Tiro Salvezza su Tempra (DC 10 + danni inflitti) o muore.

Le creature immuni ai colpi critici, non subiscono danni critici, né devono superare un Tiro Salvezza su Tempra per evitare di essere uccisi da un colpo di grazia.

\textbf{Infermo}:\index{Infermo} Un personaggio infermo subisce una penalità -2 a tutti i Tiri per Colpire e per i danni delle armi, ai Tiri Salvezza, alle Prove di Competenza.

\textbf{Intralciato}:\index{Intralciato} Il personaggio rimane intrappolato. Un personaggio intralciato ha difficoltà di movimento, ma può comunque provare a muoversi, a meno che i legami che lo intralciano non siano ancorati a un oggetto immobile o impugnati da una forza contrapposta.

Una creatura intralciata può muoversi a velocità dimezzata ma non può Correre o Caricare, e subisce penalità -2 ai Tiri per colpire e penalità -1 alla Agilità.

Un personaggio intralciato che cerca di lanciare una Essenza deve superare una prova di Concentrazione (DC 15) o perde la magia.

\textbf{Invisibile}:\index{Invisibile} Le creature invisibili non sono percepibili dalla vista, ricevono bonus +1d6 al Tiro per Colpire contro avversari visibili e negano il bonus di Agilità alla Difesa dei loro avversari (se posseduto).

\textbf{Livelli Negativi}:\index{Livelli Negativi} Ci sono creature non morte con capacità soprannaturali ed Essenze con effetti magici che possono risucchiare la vita e l'energia vitale; questo terrificante attacco è noto come Risucchio di Energia e comporta dei Livelli Negativi che infliggono delle penalità alle creature.

Se i livelli negativi di una creatura sono uguali o superiori ai suoi Dadi Vita totali, la creatura muore.

\textbf{Livelli Negativi Temporanei:} Una creatura con livelli negativi temporanei ha diritto ogni giorno ad un nuovo Tiro Salvezza per rimuovere il livello negativo. La DC di questo Tiro Salvezza è la stessa dell'effetto che ha causato i livelli negativi.

\textbf{Livelli Negativi Permanenti:} Alcune capacità e l'Essenza Distruzione comportano un risucchio di livello permanente ad una creatura. Questi sono trattati come livelli negativi temporanei, ma non permettono di effettuare ogni giorno un nuovo Tiro Salvezza per rimuoverli.

\textbf{Ristorare Livelli Negativi}: Solo l'Essenza di Cura permette di recuperare i livelli persi, vuoi causati da mostri o da Distruzione. Una creatura portata a livelli negativi, ovvero sotto zero, è morta e non si può riportare in vita o recuperare i livelli mancanti.

\textbf{Morente} \index{Morente}Un personaggio morente ha -1 Punti Ferita e si considera Indifeso per le penalita' ed e' prossimo alla morte.

\textbf{Morto}:\index{Morto} L'anima del personaggio abbandona permanentemente il suo corpo. I personaggi morti non possono beneficiare delle cure normali o magiche, e non possono essere riportati in vita da una Essenza. Solo un Patrono ha sufficiente potere per riportare l'anima nel corpo e riportare in vita la creatura. L'Essenza di Distruzione può rianimare un corpo come non morto.

\textbf{Nauseato}:\index{Nauseato} Le creature nauseate soffrono di disturbi di stomaco.
Le creature nauseate non sono in grado di attaccare, utilizzare Essenze, concentrarsi sulle Essenze o fare qualsiasi altra cosa che richieda attenzione. La sola azione che un tale personaggio può compiere è una singola Azione di movimento per turno.

\textbf{Paralizzato}: \index{Paralizzato}Un personaggio paralizzato è bloccato sul posto ed è incapace di muoversi od agire. Ha punteggi effettivi di Potenza e Agilità pari a 0, è Indifeso e può compiere azioni esclusivamente mentali.

Una creatura alata in volo, nel momento in cui viene paralizzata non può più battere le ali e precipita.
Un nuotatore paralizzato non può più Nuotare e potrebbe annegare.

Una creatura può attraversare una zona occupata da una creatura paralizzata (o morta), che sia un alleato o meno e si considera come terreno difficile.

\textbf{Pietrificato}: \index{Pietrificato}Un personaggio pietrificato è stato trasformato in pietra ed è privo di sensi ed Indifeso. Se un personaggio pietrificato si incrina o si rompe, ma i pezzi rotti sono uniti al corpo quando ritorna di carne, il personaggio non viene ferito o danneggiato. Se il corpo pietrificato del personaggio è incompleto quando viene ritrasformato in carne, il corpo rimane incompleto e potrebbe avere una qualche perdita permanente di punti ferita e/o altre menomazioni.

\textbf{Paura}:\index{Paura} Essenze, Oggetti Magici e certe creature possono influenzare i personaggi con paura. In molti casi, il personaggio deve effettuare un Tiro Salvezza su Arbitrio per resistere agli effetti, e un tiro fallito indica che il personaggio è scosso, spaventato o in preda al panico.

\textbf{Prono}\index{Prono}: chi è prono ha un -1d6 ad attaccare ed un -4 alla Difesa. Alzarsi da prono costa 2 Azioni.

\textbf{Scosso}:\index{Scosso}I personaggi che sono scossi subiscono penalità -2 ai Tiri per Colpire, ai Tiri Salvezza e alle prove.

\textbf{Spaventato}:\index{Spaventato} I personaggi spaventati sono anche scossi, e inoltre fuggono dalla fonte della loro paura il più velocemente possibile, anche se possono scegliere la direzione di fuga.

A parte cio', una volta che sono fuori vista (o udito) dalla fonte della loro paura, possono agire normalmente. Se la durata della paura non è ancora arrivata al termine, qualora dovessero incontrare di nuovo la fonte della loro paura, cercherebbero nuovamente di fuggire.

I personaggi che non sono in grado di fuggire possono combattere (anche se continuano ad essere scossi).

\textbf{Stordito/Svenuto}:\index{Stordito}\index{Svenuto} si considera che sia Indifeso.

\textbf{In Preda al Panico}:\index{In Preda al Panico} I personaggi in preda al panico sono scossi e, inoltre, hanno una probabilità del 50\% di far cadere a terra qualsiasi cosa stanno tenendo in mano e di fuggire dalla fonte del loro terrore il più in fretta possibile seguendo un percorso di fuga completamente casuale.

I personaggi in preda al panico fuggono davanti a qualsiasi altro pericolo che possano trovarsi di fronte. A parte cio', una volta che sono fuori vista (o udito) dalla fonte della loro paura, possono agire normalmente.

I personaggi in preda al panico prendono anche la condizione Accovacciato se non possono fuggire.

\textbf{Terrore Crescente}:\index{Terrore Crescente} Gli effetti della paura sono cumulativi.

Un personaggio scosso che viene nuovamente scosso diventa spaventato, mentre invece un personaggio scosso che viene spaventato cade in preda al panico. Un personaggio spaventato che viene scosso o spaventato cade in preda al panico.

\textbf{Rotto}\index{Rotto}

La condizione rotto ha i seguenti effetti, a seconda dell'oggetto:

- Se l'oggetto è un'arma, tutti gli attacchi effettuati con l'oggetto subiscono penalità -2 al CA per colpire e per i danni. Tali armi ottengono un Colpo Critico soltanto con un 4 volte 6 naturale ed infliggono solo 1 volta il danno in aggiunta.

- Se l'oggetto è un'armatura o uno scudo, il bonus che concede alla Difesa è dimezzato, arrotondando per difetto. L'armatura rotta raddoppia la penalità di armatura alla Prova sulle Competenze.

- Se l'oggetto è un attrezzo necessario per una Competenza, tutte le prove di Competenza effettuate con esso subiscono penalità -2.

- Se l'oggetto è una Bacchetta o un Bastone, utilizzate il doppio delle cariche necessarie ogni volta che viene usato.

- Se l'oggetto non rientra in nessuna delle precedenti categorie, la condizione rotto non ha effetto sul suo uso. Gli oggetti con condizione rotto, a prescindere dal tipo, valgono il 75\% del loro costo normale. Se l'oggetto è magico, può essere riparato soltanto con l'Essenza di Creazione utilizzata da un incantatore di livello uguale o superiore a quello che ha creato dell'oggetto.

\textbf{Sanguinante}\index{Sanguinante} Una creatura che sta subendo danni da sanguinamento subisce la quantità di danno indicata all'inizio del suo turno. Il sanguinamento può essere interrotto superando una prova di sopravvivenza (pronto soccorso) con DC 15 o con l'uso di Essenza di Cura.

Alcuni effetti di sanguinamento causano un danno di caratteristica o persino un risucchio di caratteristica. Gli effetti di sanguinamento non si cumulano a meno che non causino tipi differenti di danno.

Quando due o più effetti di sanguinamento causano lo stesso tipo di danno, si tenga l'effetto peggiore. In questo caso, il risucchio di caratteristica è peggiore del danno di caratteristica.

Se non indicato diversamente il danno massimo da sanguinamento e' di 5 PF a round.

\pagebreak

%\begin{document}
%	\author{Andres Zanzani}
%	\title{\centering Dungeon Bell System (DBS)\\\vspace{0.5cm}Monster Manual\\\vspace{0.5cm}\includegraphics[scale=0.365] {copertina-monster}}
%	\date{\today\\v1.0.4\\}
%\maketitle
%\newpage~
%\setcounter{page}{1}
%\begin{multicols}{2}  
%{\small \tableofcontents{}}
%\end{multicols}
%\pagebreak
%\thispagestyle{empty}	
%\newpage~\newpage~
%}	
\section{Mostruario di TUS}

\textbf{Arrivano i Mostri...}

\begin{enfasiwide}{Chi lotta con i mostri deve guardarsi di non diventare, così facendo, un mostro. E se tu scruterai a lungo in un abisso, anche l'abisso scruterà dentro di te. (Friedrich Nietzsche)
		
\medskip
		
I mostri possono essere sconfitti soltanto dai loro simili. (Claymore)

\medskip

La tragedia dei mostri è di essere troppo grandi e potenti per essere accettati dal genere umano. (Ishiro Honda)
}\end{enfasiwide}\medskip

\begin{multicols}{2}

\lettrine[lines=2, lhang=0.33, loversize=0.25, findent=1.5em]{B}{envenuti} in un universo ricco di nemici cattivi violenti subdoli intelligenti meschini giganteschi.. e quant'altro tu vorrai. I mostri sono il caposaldo di qualsiasi gioco di ruolo fantasy.

Vengono qui spiegati e presentati dei mostri, non certo tutti ne tanto meno esaustivi, usateli per popolare di incubi le avventure dei vostri compagni.

\medskip

\includegraphics[width=0.8\linewidth]{immagini/sangiorgioedrago.png}
\begin{center}
\textit{San Giorgio e il drago (1460 circa) di Paolo Uccello. National Gallery di Londra}
\end{center}

\subsection{Introduzione}

Un avventura non e' solo un insieme di mostri ma di situazioni, di luoghi, di sorprese, insomma di tutto cio' che puo' affascinare, coinvolgere stupire, impegnare i giocatori. Ma anche i mostri servono. Picchiare ha un aspetto catartico, liberatorio.

Inserite nell'avventura mostri difficili e pericolosi dove serve ma ogni tanto, raramente fate sentire i giocatori potenti, fagli affrontare mostri che in pochissimi round possono risolvere. Descrivete il combattimento enfatizzando i colpi, i critici, il dolore ed il sangue dei mostri. Fate capire quanto possano essere potenti i personaggi.

Altre volte fate che i mostri incutano timore perché' sono grossi, affamati, magici e cattivi, e' necessario che i giocatori abbiano paura per i loro personaggi, che non diano mai per scontato la vittoria.

La sicurezza nel descrivere la situazione, poche battute, il fissare negli occhi i giocatori..coinvolgete i giocatori innanzitutto, una volta che i giocatori avranno la vostra attenzione anche i personaggi saranno piu' attenti. Cercate di mettere mostri coerenti all'ambiente, all'avventura, alla situazione. Non tirate a caso su tabelle, uno scontro ben organizzato da molta piu' soddisfazione che mostri a caso che "spawnano".

Non riducete tutto a un MMORG dove l'obiettivo e' solo uccidere tutto e tutti, ci possono essere sempre tante scelte se ti impegni un po'.

\begin{tcolorbox}[width=0.425\textwidth,title = Affrontare i mostri]
{
Lascia che questo vecchio ti dia un paio di consiglio giovane avventuriero!		

- Non tutti i nemici si sconfiggono con la spada, molte volte serve anche una mazza!

- A volte le armi e la forza bruta non bastano. Se non hai compagni che possono lanciare incantesimi assicurati di avere sempre la possibilità di appiccare un fuoco.

- Scappa. E' sempre una opzione valida se hai modo e vedi che la situazione non promette niente di buono.

- Organizzati! non entrare nel dungeon a testa bassa senza mai fermarti tranne quando sei morto! Riposati, esplora, controlla l'ambiente e quando sei sicuro e stai meglio prosegui! anche i tuoi nemici si organizzano e si riposano intanto.

- A volte si puo' anche parlare con i nemici, non sempre ma anche loro non vogliono morire.

- Se devi uccidere fallo con cattiveria e velocità. Non perdere tempo e ottimizza i colpi, risparmia le energie e preparati immediatamente ad un altro scontro.

}\end{tcolorbox}

\subsection{Modificare le Creature}

Nonostante la versatile collezione di mostri in questo documento, potresti comunque trovarti in imbarazzo quando si tratta di trovare la creatura perfetta per una tua avventura. Sentiti libro di modificare le creature esistenti e trasformarle in qualcosa che ti sia più utile, magari prendendo in prestito uno o due tratti da un mostro diverso.

Tieni a mente che modificare un mostro potrebbe cambiarne il grado di sfida. 

\subsection{Taglia}

Un mostro può essere di taglia Minuscola, Piccola, Media, Grande, Enorme o Mastodontica e Colossale. La tabella Categorie di Taglia mostra quanto spazio una creatura di una specifica taglia occupi in combattimento.

\medskip

\textbf{Categorie di Taglia}

\begin{tabularx}{0.45\textwidth}{llX}
\toprule
\textbf{Taglia}& \textbf{Spazio} & \textbf{Esempio}\\
Minuscola & 25 x 25 cm& Gatto, spiritello\\
Piccola & 0,5 x 0,5 m & Goblin, cane\\
Media & 1 x 1 m & Orco\\
Grande & 3 x 3 m& Ogre\\
Enorme & 5 x 5 m & Gigante, Ent\\
Mastodontico & 6 x 6 m&Kraken, Drago\\
Colossale & 12 x 12 m&Drago anziano\\
\end{tabularx}

\subsection{Tipo}

Il tipo di un mostro si riferisce alla sua natura basilare. Certi incantesimi, oggetti magici, Abilità e altri effetti del gioco interagiscono in modi speciali con le creature di un tipo specifico. Ad esempio, una \emph{freccia} \emph{ammazza draghi} infligge danni extra non solo ai draghi ma anche a tutte le altre creature del tipo drago, come i draghi tartaruga e le viverne.

Il gioco comprende i seguenti tipi di mostri:

\medskip\textbf{Aberrazioni}, creature totalmente aliene. Molte di esse possiedono innate abilità magiche che attingono alla mente aliena della creatura anziché dalle forze mistiche del mondo. Esempi classici di aberrazioni sono aboleti, osservatori, scortica mente e i batraci del caos.

\medskip\textbf{Bestie}, creature non umanoidi che sono una componente naturale di un mondo fantasy. Alcune possiedono poteri magici, ma la maggior parte è priva di Intelligenza e non ha alcuna forma di società o linguaggio. Esempi classici di bestie sono tutte le specie di animali comuni, i dinosauri e le versioni giganti degli animali. 

\medskip\textbf{Celestiali}, creature native dei Piani Superiori. Molti di loro sono servitori delle divinità, impiegati come messaggeri o agenti nel mondo dei mortali e per i piani.\\
I celestiali sono di natura buona, esempi classici di celestiali sono angeli, couatl e pegasi.

\medskip\textbf{Costrutti}, sono creati e non partoriti. Alcuni sono programmati dai loro creatori per seguire una semplice serie di istruzioni, mentre altri sono senzienti e capaci di pensare per proprio conto. I golem sono i costrutti più rappresentativi.

\medskip\textbf{Draghi}, sono grandi creature rettili di antica origine ed enorme potere. I veri draghi, compresi i buoni draghi metallici e i malvagi draghi cromatici, sono molto intelligenti e possiedono doti magiche innate. In questa categoria si collocano anche creature lontanamente imparentate con i veri draghi, ma meno potenti, meno intelligenti e meno magiche, come le viverne e gli pseudodraghi.

\medskip\textbf{Elementali}, sono creature native dei piani elementali. Alcune creature di questo tipo sono poco più che masse animate del rispettivo elemento, e includono le creature chiamate semplicemente elementali. Altre creature possiedono forme biologiche infuse di energia elementale. Le razze dei geni, compresi djinn ed efreet, formano le civiltà più importanti dei piani elementali. Altre creature elementali sono gli azer, i persecutori  invisibili e le bizzarrie d'acqua. 

\medskip\textbf{Fatati}, sono creature magiche strettamente legate alle forze della natura. Vivono in radure crepuscolari e foreste nebbiose. Esempi di fatati sono driadi, pixie e satiri.

\medskip\textbf{Giganti}, troneggiano sugli umani e i loro simili. Sono di forma umana, sebbene alcuni abbiano più teste (ettin) o deformità (fomori). Le sei varianti dei veri giganti sono gigante di collina, gigante di pietra, gigante del gelo, gigante del fuoco, gigante delle nuvole, gigante delle tempeste. Oltre questi, anche ogri e troll sono giganti. 

\medskip\textbf{Immondi}, creature perverse native dei Piani Inferi. Alcune sono al servizio di  divinità, ma molte di più operano agli ordini di arcidiavoli e principi demoni. A volte sacerdoti e maghi malvagi evocano gli immondi nel mondo materiale perché eseguano le loro volontà. Se un celestiale malvagio è una rarità, un immondo buono è praticamente inconcepibile. Gli immondi includono demoni, diavoli, segugi infernali e rakshasa. 

\medskip\textbf{Melme}, sono creature gelatinose che difficilmente hanno una forma fissa. Vivono principalmente sottoterra, stabilendosi in grotte e sotterranei, nutrendosi di rifiuti, carcasse o creature tanto sfortunate da incapparvi. I protoplasmi neri e i cubi gelatinosi sono tra le melme più riconoscibili.

\medskip\textbf{Mostruosità}, sono mostri nel senso più stretto del termine creature spaventose che non sono comuni, né davvero naturali, e quasi mai benigne. Alcune sono il risultato di esperimenti magici andati male (come l'orsogufo), mentre altri sono il prodotto di terribili maledizioni (tra cui annoveriamo il minotauro). Sfuggono a qualsiasi categorizzazione, e in qualche modo servono da categoria onnicomprensiva per quelle creature che non corrispondono a nessun altro tipo di mostro. 

\begin{center}
	\includegraphics[width=0.8\linewidth]{immagini/sanmichelesatana.png}\\
	\textit{San Michele sconfigge Satana. Raffaello ed aiuti (1518). Museo del Louvre}
\end{center}

\medskip\textbf{Non Morti}, sono creature un tempo vive condotte ad un orribile stato di non morte tramite la pratica della magia negromantica o qualche blasfema maledizione. Tra i non morti si annoverano cadaveri ambulanti, come vampiri e zombi, e spiriti incorporei, come fantasmi e spettri.

\medskip\textbf{Piante}, in questo contesto si tratta di creature vegetali, non della normale flora. La maggior parte di esse sono mobili, e alcune sono carnivore. L'esempio più classico di piante sono i tumuli ambulanti e gli ent. Anche le creature fungoidi come le spore gassose e i miconidi rientrano in questa categoria.

\medskip\textbf{Umanoidi}, sono la popolazione principale dei mondi di gioco, civilizzati e selvaggi, comprendono gli umani e un'ampia gamma di altre specie. Possiedono una lingua e una cultura, poche o nessuna abilità magica innata (sebbene molti umanoidi possano apprendere gli incantesimi), e una forma bipede. Le razze più comuni di umanoide sono quelle più adatte come personaggi del giocatore: umani, nani, elfi e halfling. Quasi altrettanto numerose, ma più brutali e selvagge, e quasi tutte malvagie, sono le razze goblinoidi (goblin, hobgoblin e bugbear), orchi, gnoll, lucertoloidi e coboldi.\\

\medskip

Queste categorie possono essere a loro volta raggruppate in tipologie di Creature:
\smallskip
\begin{itemize}
\item
Le \textbf{Creature Naturali}: sono Insetti, Rettili, Bestie, Umanoidi, Piante, Creature acquatiche, Monstrusita', Melme
\item
Le \textbf{Creature Magiche} sono: Immondi, Fatati, Spiriti, Non morti, Giganti, Celestiali, Costrutti, Aberrazioni (tutto ciò che e' alieno o innaturale) e Draghi.

Se una Creatura Naturale ha poteri magici allora si considera anche come Creatura Magica.
\end{itemize}


\medskip\textbf{Etichette}

Un mostro può presentare una o più etichette indicate tra parentesi, a seguire il suo tipo. Ad esempio un orco ha il tipo \emph{umanoide (orco)}. Le etichette tra parentesi forniscono ulteriori categorizzazioni per determinate creature. Le etichette non hanno delle proprie regole specifiche, ma alcuni elementi del gioco, come gli oggetti magici, vi possono fare riferimento. Ad esempio, una lancia particolarmente efficace contro i demoni, funzionerebbe contro qualsiasi mostro che abbia l'etichetta demone.

\subsection{Tratti}

I mostri non presentano l'elenco dettagliato dei tratti, troverete solo l'indicazione sugli assi del Chaos, Legge, Bene e Male. Ricordatevi che sono indicazioni, le eccezioni possono capitare specialmente nelle specie più intelligenti.
Determinate creature sono \textbf{disallineate}, ovvero non hanno una condotta morale o etica.

\subsection{Difesa}

Un mostro che indossa un'armatura o trasporta uno scudo ha una Difesa che tiene conto dell'armatura, lo scudo e della Destrezza. Altrimenti, la Difesa di un mostro è basata sul suo valore di Destrezza l'armatura naturale, se la possiede. Se un mostro possiede un'armatura naturale, indossa armature o trasporta uno scudo, viene indicato tra parentesi dopo il valore della sua Difesa.

Qualora il mostro fosse \textbf{colto di sorpresa} sottraete alla Difesa il valore di Destrezza e Scudo se presente.

Se il mostro viene colpito con un \textbf{effetto a tocco} (Difesa a tocco) la Difesa sarà 10 + Destrezza + bonus magici.

\subsection{Punti Ferita}

Di solito quando scende a 0 punti ferita, un mostro muore o viene distrutto.

I punti ferita di un mostro sono presentati sia come un insieme di dadi che come valore medio. Ad esempio, un mostro con 2d8 punti ferita ha di media 9 punti ferita (2 x 4,5).

La taglia di un mostro determina il dado impiegato per calcolare i suoi punti ferita, come mostrato sulla tabella Dadi Vita per Taglia.

\subsection{Dadi Vita per Taglia Mostro}

\medskip
\begin{tabular}{lll}
\toprule
Taglia & Dado Vita & PF per Dado\\
Minuscola &d4&2,5\\
Piccola &d6&3,5\\
Media&d8 &4,5\\
Grande&d10&5,5\\
Enorme&d12&6,5\\
Mastodontico&d20&10,5\\
Colossale&2d12&12\\
\end{tabular}
\medskip

Anche il valore di Costituzione di un mostro influenza il numero di punti ferita che possiede. Il suo valore di Costituzione viene moltiplicato per il numero di Dadi Vita che possiede e il risultato viene sommato ai suoi punti ferita. Ad esempio, un mostro ha Costituzione 1 e 2d8 Dadi Vita, e avrà quindi 2d8+2 punti ferita (media 11).

\subsection{Movimento}

Il Movimento di un mostro ti dice di quanto si possa muovere durante il suo round per Azione di Movimento

Tutte le creature possiedono un movimento di passeggio, detto semplicemente movimento del mostro. Le creature che non possiedono alcuna forma di spostamento terreno hanno velocità di passeggio 0 metri.

Alcune creature possiedono uno o più dei seguenti modi di movimento aggiuntivi.

\begin{center}
	\includegraphics[width=0.7\linewidth]{immagini/roc.png}\\
	\textit{Henry Justice Ford}
\end{center}


\medskip\textbf{Nuoto}

Un mostro che possiede una velocità di nuoto non deve spendere movimento extra per nuotare (non e' terreno difficile)

\medskip\textbf{Scalata}

Un mostro che possiede una velocità di scalata può usare tutto o solo parte del suo movimento per muoversi su superfici verticali. Il mostro non deve spendere movimento extra per scalare.

\medskip\textbf{Scavo}

Un mostro che possiede una velocità di scavo può usare la sua velocità per attraversare sabbia, terra, fango, ecc. Un mostro non può scavare attraverso la roccia solida a meno che non possieda un tratto speciale che glielo permetta.

\medskip\textbf{Volo}

Un mostro che possiede una velocità di volo può usare tutto o solo parte del suo movimento per volare. Alcuni mostri hanno l'abilità di \textbf{fluttuare}, che li rende difficili da abbattere. Il mostro smette di fluttuare quando muore.


\subsection{Punteggi di Caratteristica}

Ogni mostro possiede sei punteggi di caratteristica (Forza, Destrezza, Costituzione, Intelligenza, Saggezza, Carisma)

\subsection{Competenze}

La voce Competenze è riservata a quei mostri che sono capaci in una o più competenze. Ad esempio, un mostro che è molto attento e furtivo potrebbe avere bonus alle prove di Saggezza (Consapevolezza) e Destrezza. \\
Si possono applicare anche altri modificatori. Ad esempio, un mostro potrebbe avere un bonus più grande del previsto per tenere conto della sua grande perizia.

\subsection{Vulnerabilità, Resistenze e Immunità}

Alcune creature possiedono vulnerabilità, resistenze o immunità ad un certo tipo di danno. Creature particolari sono addirittura resistenti o immuni agli attacchi non magici (un attacco magico è un attacco sferrato tramite un incantesimo, un oggetto magico, o un'altra fonte di magia). \\
E' anche possibile che sia indicato uno specifico bonus magico minimo per poter danneggiare la creatura.\\
Inoltre, certe creature sono immuni a determinate condizioni. Se  un mostro è immune ad un effetto di gioco che non viene considerato danno o condizione, possiede invece un tratto speciale.


Nella tabella sottostante viene indicato quale incantamento magico dell'arma e' necessario per superare l'immunita' indicata.\\
E' anche indicato il livello minimo di attacco naturale nel caso si colpisca con calci e pugni.

\medskip

\textbf{Tabella: Equivalenza Armi Magiche}\index{Tabella Equivalenza Armi Magiche}

\medskip

\begin{tabularx}{0.45\textwidth}{lXX}
	\toprule
	\textbf{Immunità} & \textbf{Bonus Arma} & \textbf{Attacco Naturale}\\
	Incantamento +1         & +1              & Livello 3\\
	Incantamento +2         & +2              & Livello 6\\
	Ferro Freddo / Argento  & +2              & Livello 9\\
	Adamantio               & +3              & Livello 12\\
	Incantamento +3         & +3              & Livello 15\\
	Incantamento +4         & +4              & Livello 18\\
\end{tabularx}


\subsection{Sensi}

La voce Sensi elenca qualsiasi senso speciale di cui il mostro sia in possesso. I sensi speciali sono descritti di seguito. Se non e' presente la voce Sensi, la creatura ha dei sensi standard (visione...)

\subsubsection{Percezione Tellurica}

Un mostro con percezione tellurica può individuare e trovare le origini delle vibrazioni entro uno specifico raggio, purché il mostro e la fonte della vibrazione siano in contatto con lo stesso terreno o sostanza. La percezione tellurica non può essere impiegata per individuare creature volanti o incorporee. Molte creature scavatrici, come gli ankheg e i colossi di terra, possiedono questo senso speciale.

\subsubsection{Visione Crepuscolare}

Una creatura con Visione Crepuscolare può vedere nella piu' tenue delle luci, ma non nell'oscurità' completa. Molte creature che vivono sottoterra possiedono questo senso
speciale.  Vedi capitolo \hyperlink{visioneeluce}{Caratteristiche Speciali}.

\subsubsection{Visione del Vero}

Un mostro con la visione del vero può, fino ad una specifica gittata, vedere attraverso l'oscurità normale e magica, vedere creature e oggetti invisibili, automaticamente individuare le illusioni e riuscire i Tiri Salvezza contro di loro, e percepire la forma originale di un mutaforma o di una creatura trasformata dalla magia. Inoltre, la creatura può vedere nel Piano Etereo fino alla stessa gittata.


\subsubsection{Vista Cieca}

Una creatura con vista cieca può percepire l'ambiente circostante, senza fare affidamento alla vista, fino ad una specifica gittata. 

Le creature senza occhi, come i grimlock e le melme, e le creature con ecolocazione o sensi potenziati, come i pipistrelli e i draghi puri, possiedono questo senso. 

Se un mostro è cieco di natura, la cosa viene annotata tra parentesi, ad indicare che la gittata della sua vista cieca definisce anche la gittata massima della sua percezione.

\medskip
\begin{center}
	
	\includegraphics[width=0.7\linewidth]{immagini/ciclope.png}
	
	\textit{Henry Justice Ford}
	
\end{center}

\subsection{Linguaggi}

Le lingue che un mostro può parlare sono riportate in ordine alfabetico. A volte un mostro può capire una lingua ma non parlarla, e la cosa viene indicata a questa voce. Una ``-'' indica che la creatura non parla né comprende alcuna lingua.

\subsection{Telepatia}

La telepatia è un'abilità magica che permette ad un mostro di comunicare mentalmente con un'altra creatura nel raggio di azione specificato. La creatura contattata non e' necessario che parli la stessa lingua del mostro per comunicare in questo modo. Una creatura senza telepatia può ricevere e rispondere a messaggi telepatici ma non può iniziare o terminare una conversazione telepatica.

Un mostro telepatico non ha bisogno di vedere la creatura contattata e può terminare il contatto telepatico in qualsiasi momento. Il contatto è infranto non appena le due creature non si trovano più entro il raggio di azione o se il mostro telepatico contatta un'altra creatura a gittata. Un mostro telepatico può iniziare o terminare una conversazione  telepatica senza dover usare un'azione, ma mentre il mostro è inabile, non può dare inizio ad un contatto telepatico, e qualsiasi contatto in corso viene terminato.

Una creatura nell'area di un \emph{campo anti-magia} o in qualsiasi altro posto in cui la magia non funziona non può inviare o ricevere messaggi telepatici.

\subsection{Sfida}

Il \textbf{grado di sfida} (CR) di un mostro vi dice quanto sia grande la minaccia che pone. Una compagnia di quattro avventurieri equipaggiata in maniera appropriata e riposata dovrebbe essere in grado di sconfiggere un mostro dal grado di sfida pari al proprio livello medio senza subire perdite. Ad esempio, una compagnia di quattro personaggi di 3° livello dovrebbe ritenere un mostro di grado di sfida 3 una degna sfida, ma non letale.

I mostri che sono significativamente più deboli dei personaggi di 1° livello hanno un grado di sfida inferiore ad 1. I mostri con un grado di sfida 0 non presentano problemi eccetto in grandi numeri; quelli privi di reali attacchi non valgono punti esperienza.

Alcuni mostri presentano una sfida superiore a quelle che anche una compagnia di 20° livello sia in grado di gestire. Questi mostri hanno grado di sfida 21 o superiore e sono progettati proprio per mettere alla prova le capacità dei personaggi.


\subsection{Tratti Speciali}

I tratti speciali (che compaiono dopo il grado di sfida di un mostro ma prima di qualsiasi azione o reazione) sono peculiarità che avranno probabilmente un ruolo in un incontro di combattimento e che richiedono delle spiegazioni.

\subsection{Incantesimi}

Un mostro con il privilegio Incantesimi e' in grado di lanciare Incantesimi.\\
Un mostro può lanciare un incantesimo dalla sua lista senza effettuare la prova di magia e senza la possibilità di effettuare tiri critici o meno. La DC e' quella dell'incantesimo + Intelligenza o Saggezza a seconda della caratteristica migliore.


\includegraphics[width=0.7\linewidth]{immagini/lich2.png}

\medskip

\textit{Lich - Battle of Wesnoth}


\subsection{Incantesimi Innati}

Un mostro con l'abilità innata di lanciare incantesimi ha il tratto speciale Incantesimi.
Se non indicato diversamente non e' necessario effettuare la prova di magia e l'incantesimo viene lanciato alla sua Difficoltà senza conteggiare alcun critico.\\
Gli incantesimi innati di un mostro non possono essere scambiati con altri incantesimi. 

\subsection{Azioni}

Anche i mostri agiscono secondo lo schema delle 3 Azioni disponibile per round. Possono essere segnate abilità e capacità che gli permettono di eseguire un numero piu' elevato di Azioni.

Quando un mostro svolge le sue azioni, può scegliere tra le opzioni della sezione Azioni del suo blocco statistiche o impiegare una delle azioni disponibili a tutte le creature, come Scattare o Nascondersi.

\subsubsection{Attacchi da Mischia e a Distanza}

L'azione più comune che un mostro effettuerà in combattimento, sarà un attacco da mischia o a distanza. Possono essere attacchi con incantesimi o attacchi con armi, dove l'arma può essere un manufatto o un'arma naturale, come gli artigli o la coda chiodata.

\emph{\textbf{Creatura contro Bersaglio}.} Il bersaglio di un attacco da mischia o a distanza è di solito una creatura o un bersaglio, la differenza nel fatto che un ``bersaglio'' può essere una creatura o un oggetto.

\emph{\textbf{Colpisce.}} Qualsiasi danno inflitto o altro effetto che avviene come risultato di un attacco che colpisce il bersaglio viene descritto nell'annotazione ``\emph{Colpisce}''. Puoi scegliere se prendere il danno medio o tirare i dadi; per questo  motivo vengono presentati sia il danno medio che una formula di dadi. 

\textbf{\emph{Manca}.} Se un attacco ha un effetto prodotto da un colpo a vuoto, quell'informazione viene fornita dall'annotazione ``\emph{Manca}''.

\emph{\textbf{Danni.}} Se un mostro impugna armi manufatte, infligge danni appropriati all'arma. I mostri più grossi di solito impugnano armi di dimensioni superiori che infliggono danni extra quando colpiscono. Raddoppiare i dadi dell'arma se la creatura è Grande, triplicarli se Enorme e quadruplicarli se Mastodontica.

Una creatura ha -1d6 ai tiri per colpire con un'arma costruita per una taglia superiore alla sua.  Il Narratore può decidere che le armi di due o più taglie più grandi di quella dell'attaccante sono del tutto impossibili da usare.

\subsubsection{Multiattacco}

Una creatura che può effettuare più attacchi durante il suo round ha l'abilità Multiattacco. Una creatura non può usare Multiattacco quando effettua un attacco di opportunità, il quale deve essere un singolo attacco da mischia.

\begin{center}
	\includegraphics[width=0.6\linewidth]{immagini/polpo.png}
	
	\textit{Alphonse de Neuville - Hetzel edition of 20000 Lieues Sous les Mers}
\end{center}

\subsubsection{Regole dell'Afferrare per i Mostri}

Molti mostri possiedono un attacco speciale che gli permette di afferrare rapidamente la preda. Quando un mostro colpisce con un simile attacco, non deve effettuare un'ulteriore prova di caratteristica per determinare se l'afferrare riesce, a meno che l'attacco non dica altrimenti.

Una creatura afferrata dal mostro può usare un azione per tentare di sfuggirgli. Per farlo, deve riuscire una prova di Forza contro la DC di fuga nel blocco statistiche del mostro. Se non viene fornita una DC di fuga, assumere che la DC sia uguale a 10 + Forza del mostro.

\subsubsection{Munizioni}

Un mostro porta con sé munizioni sufficienti per effettuare i suoi attacchi a distanza. Puoi presumere che un mostro abbia 2d4 proiettili per un attacco con armi da lancio, e 2d10 proiettili per un'arma a proiettili come un arco o una balestra.


\includegraphics[width=0.7\linewidth]{immagini/cupido.png}

\textit{Eros con il suo arco. Musei Capitolini}

\subsubsection{Reazioni}

Se un mostro può compiere qualcosa di speciale con le sue reazioni, è riportato qui. Se una creatura non ha reazioni speciali, questa sezione è assente.

\subsubsection{Uso Limitato}

Alcune abilità speciali hanno restrizioni sul numero di volte che possono essere usate.

\textbf{\emph{X/Giorno}.} L'annotazione ``X/Giorno'' indica un'abilità speciale che può essere usata X volte prima che il sorga l'alba per recuperare gli usi consumati. Ad esempio, ``1/Giorno'' indica un'abilità speciale che può essere usata una volta prima che il mostro debba aspettare la nuova alba.

\emph{\textbf{Ricarica X-Y.}} L'annotazione ``Ricarica X-Y'' indica che il mostro può usare un'abilità speciale una volta e che l'abilità ha una probabilità casuale di ricaricarsi ogni round seguente di combattimento. All'inizio di ciascun round del mostro, tira un d6. Se il risultato è uno dei numeri dell'annotazione di ricarica, il mostro recupera l'uso dell'abilità speciale. L'abilità si ricarica anche all'alba di un nuovo giorno.

Ad esempio, "Ricarica 5-6" indica che un mostro può usare la sua abilità speciale una volta. Poi, all'inizio del round del mostro, recupera l'uso dell'abilità se tira 5 o 6 su di un d6.

\subsection{Equipaggiamento}

Il blocco statistiche si riferisce all'equipaggiamento, oltre le armi o le armature utilizzate dal mostro. Una creatura che normalmente indossa abiti, come un umanoide, si assume sia abbigliato in maniera appropriata.

Puoi equipaggiare i mostri con ulteriore equipaggiamento o ninnoli come preferisci, utilizzando il capitolo \hyperlink{equipaggiamento}{Equipaggiamento} come fonte di ispirazione, e sei tu a decidere quanto dell'equipaggiamento del mostro è recuperabile dopo che la creatura è stata uccisa o se qualsiasi parte del suo equipaggiamento sia ancora utilizzabile. Ad esempio, un'armatura ammaccata fatta per un mostro difficilmente sarà utilizzabile da qualcun altro.  Se un mostro incantatore necessita di componenti  materiali per lanciare i suoi incantesimi, dai per  scontato che abbia le componenti materiali per lanciare  gli incantesimi nel suo blocco statistiche.  

\subsection{Azioni Aggiuntive}

Certe creature possono  possono eseguire azioni speciali al di fuori del proprio  round, e alcune possono estendere il proprio potere  all'ambiente, provocando l'avvenimento di effetti magici  straordinari nelle loro vicinanze.

Una creatura con azioni aggiuntive può effettuare un certo  numero di azioni speciali -- dette azioni aggiuntive -- al  di fuori del suo round. Solo un'azione aggiuntiva può  essere usata alla volta e solo al termine del round di  un'altra creatura. Una creatura con azioni aggiuntive recupera  all'inizio del suo round le azioni aggiuntive che ha  usato. Non è obbligata ad usare le sue azioni aggiuntive, e non può usare le azioni aggiuntive mentre è inabile o altrimenti incapace di effettuare  azioni. Se sorpresa, non può usarle fin dopo il suo  primo round di combattimento.

Se una creatura assume la forma di una creatura con azioni aggiuntive, magari tramite un incantesimo, non ne  ottiene però le azioni aggiuntive, le azioni da tana, o  gli effetti regionali.

\subsubsection{La Tana di una Creatura}

Una creatura con azioni aggiuntive può presentare una sezione che ne descrive la tana e gli effetti speciali che vi può  creare mentre si trova lì, o per propria volontà o  semplicemente grazie alla sua presenza. Questa  sezione si applica solo alle creature leggendarie che  trascorrono molto tempo nelle loro tane ed è altamente  probabile che vi vengano incontrate.

\subsubsection{Azioni da Tana}

Se una creatura con azioni aggiuntive ha un'azione da tana, può  usarla per imbrigliare la magia ambientale della sua  tana. Al conteggio di iniziativa 20, perdendo i pareggi,  la creatura può usare una delle sue opzioni di azioni da  tana. Non può farlo mentre è inabile o altrimenti  incapace di effettuare azioni. Se sorpresa, non può  farne uso fino a dopo il suo primo round di combattimento.

\subsubsection{Effetti Regionali}

La semplice presenza di una creatura con azioni aggiuntive può  avere effetti strani e meravigliosi sull'ambiente, come  indicato in questa sezione. Gli effetti regionali terminano all'istante o si dissipano col tempo una volta  morta la creatura con azioni aggiuntive.


\subsubsection{Tipologie di Tesoro}

Ogni tipologia di creatura puo' preferire un tipo di tesoro (inteso come oggetti, monete, gemme...) diverso. Questi sono solo suggerimenti su come costruire il tesoro del mostro.

\medskip

\begin{itemize}

\item \textbf{Aberrazione}
Molte aberrazioni hanno scarsa considerazione per i tesori, possedendo solo quel che prendono dai resti delle loro vittime precedenti. Altre sono avversari astute che usano vari oggetti magici e tesori per potenziare le loro capacità.

\item \textbf{Animale}
Gli animali non si curano affatto dei tesori, lasciando invece monete e oggetti con i resti dei loro pasti. Per quelli con un tesoro, quest'ultimo in genere si trova nelle loro tane, sparso tra le ossa e gli altri scarti.

\item \textbf{Bestia Magica}
Curandosi poco dei valori, la maggior parte delle bestie magiche è unicamente in cerca del suo prossimo pasto. I nascondigli di queste creature sono spesso disseminati di ninnoli preziosi e oggetti magici.

\item \textbf{Costrutto}
Il solo tesoro portato dai costrutti generalmente è parte del costrutto stesso, come un'arma o un oggetto magico. I costrutti, però, vengono tipicamente usati per sorvegliare tesori o oggetti magici di maggior valore.

\item \textbf{Drago}
Noti per i loro preziosi tesori, i draghi spesso rimuginano su pile di monete, gemme, oggetti magici e altri oggetti costosi.

\item \textbf{Esterno}
Gli esterni sono tra i tipi di creature più vari e di conseguenza potrebbero avere davvero qualsiasi tipo di tesoro su di loro o nascosto nei loro rifugi. Il Narratore dovrebbe considerare la singola creatura quando determina il tipo di tesoro che più si adatta a quell'esterno.

\item \textbf{Folletto}
Sopra ogni altra cosa, i folletti danno valore agli oggetti belli e magici. Hanno scarsa considerazione per gli strumenti di scambio e commercio usati dalle razze più civilizzate, come monete e valori.

\item \textbf{Melma}
Le melme non concepiscono cose come i tesori e lasciano tutto quel che trovano nella loro ricerca del prossimo pasto. Qualsiasi tesoro possano trasportare è completamente accidentale.

\item \textbf{Non Morto}
I tesori trasportati dai non morti variano a seconda che si tratti o meno di una creatura intelligente. I non morti privi di intelletto tipicamente hanno solo i miseri valori che portavano con sé in vita, raramente davvero utilizzabili come tesori, mentre quelli intelligenti sfruttano una vasta gamma di oggetti magici per distruggere i viventi.

\item \textbf{Parassita}
Come le altre creature senza mente, i parassiti non bramano tesori, sebbene queste creature talvolta si trovino a infestare aree dove sono custoditi dei valori.

\item \textbf{Umanoide}
Le creature di questo tipo sono molto variegate, ma persino gli umanoidi più primitivi usano equipaggiamento e oggetti magici in qualche misura. In gruppi più grandi, come le comunità, gli umanoidi spesso possiedono una grande quantità di tesori che custodiscono collettivamente.

\item \textbf{Vegetale}
Come gli animali, le creature vegetali non danno peso ai tesori, e tutto ciò che si potrebbe trovare dove crescono rappresenta semplicemente i resti non digeriti di una vittima precedente.

\end{itemize}

\end{multicols}

\bigskip

\begin{narratorewide}

Le creature qui presentate vogliono essere un esempio, corposo, dei nemici che i tuoi amici potrebbero incontrare. Attenzione, non e' detto che siano tutti nemici o per forza che abbiano intenzione negative.

Creature piu' civilizzate avranno una loro condotta etica e morale individuale, anche all'interno di uno stesso gruppo di "nemici" c'e' chi potrebbe essere più "nemico" o semplicemente indifferente.

Sfrutta le peculiarità e unicità delle creature per creare incontri non scontati e sfidanti dal punto di vista tattico. Non essere scontato ma neanche assurdo nelle scelte, deve sempre esserci della coerenza nel piazzare le creature.	
	
\end{narratorewide}


\bigskip

\begin{enfasiwide}{
		
Amon Goth: Il controllo è potere. Questo è il potere.

Oskar Schindler: E' per questo che ci temono?

Amon Goth: Abbiamo il potere di uccidere. Per questo ci temono.

Oskar Schindler: Ci temono perché abbiamo il potere di uccidere arbitrariamente. Un uomo commette un reato, doveva pensarci, lo facciamo uccidere e ci sentiamo in pace. O lo uccidiamo noi stessi e ci sentiamo ancora meglio. Questo non è il potere però! Questa è giustizia, è una cosa diversa dal potere. Il potere è quando abbiamo ogni giustificazione per uccidere e non lo facciamo.

Amon Goth: E' questo il potere?

Oskar Schindler: L'avevano gli imperatori questo. Un uomo ruba qualcosa, viene portato davanti all'imperatore e si lascia cadere per terra tremante, implora per avere pietà. E' conscio che sta per andarsene. E l'imperatore lo perdona, invece. Quell'uomo, immeritevole, lo lascia libero. 

(Schindler's list - La lista di Schindler, Film, 1993)
}\end{enfasiwide}\medskip




\pagebreak
\subsection{I Mostri}

\begin{enfasiwide}{
Io sono il mostro che gli uomini che respirano bramerebbero uccidere. Io sono Dracula. (Dracula di Bram Stoker)

\medskip

Nipote del signor Wing: Senta mister, ci sono tre regole da seguire, però.

Rand: Ah, sì? E quali sarebbero?

Nipote del signor Wing: Lo tenga lontano dalla luce, lui odia la luce forte, specialmente quella del sole. Morirebbe. E lo tenga lontano dall'acqua, non lo faccia bagnare. Ma la cosa più importante, la regola che non dovrà mai dimenticare è che anche se lui piange, anche se fa scena e la supplica lei non dovrà mai, mai dargli da mangiare dopo la mezzanotte. Ha capito? (Gremlins, Film, 1984)} \end{enfasiwide}\medskip

\bigskip

\begin{multicols}{2}

\medskip\index{Mostri - Aboleth}\textbf{Aboleth}

\emph{Grande aberrazione, legale malvagio}

\textbf{FORZA} +5

\textbf{DESTREZZA} -1

\textbf{COSTITUZIONE} +2

\textbf{INTELLIGENZA} +4

\textbf{SAGGEZZA} +2

\textbf{CARISMA} +4

\textbf{Iniziativa} +4 -- \textbf{Difesa} 22

\textbf{Punti Ferita} 135 (18d10 + 36)

\textbf{Movimento} 3 m, nuoto 12 m

\textbf{Tiri Salvezza} Tempra +8, Riflessi +5, Volontà +11

\textbf{Competenze} Consapevolezza +10, Storia +12

\textbf{Sensi} scurovisione 36 m

\textbf{Linguaggi} Linguaggio delle Profondità, telepatia 36 m

\textbf{Sfida} 10 (5.900 PE)

\emph{\textbf{Anfibio.}} L'aboleth può respirare aria e acqua.

\emph{\textbf{Nube di Muco.}} Mentre è sott'acqua, l'aboleth è avvolto da muco mutante. Una creatura che entri a contatto con l'aboleth, o che lo colpisca con un attacco da mischia mentre si trova entro 1 metro da esso, deve effettuare un Tiro Salvezza di Tempra CD 14. Se lo fallisce, la creatura resta ammalata per 1d4 ore. La creatura ammalata può respirare solo sott'acqua.

\emph{\textbf{Sonda Telepatica.}} Se una creatura comunica telepaticamente con l'aboleth, e l'aboleth può vederla, l'aboleth ne apprende i più grandi desideri.

\textbf{Azioni}

\emph{\textbf{Multiattacco.}} L'aboleth effettua tre attacchi con i tentacoli

\emph{\textbf{Tentacolo.} Attacco con arma da mischia}: +9 a colpire, portata 3 m, un bersaglio.

\emph{Colpisce:} 12 (2d6 + 5) danni da botta. Se il bersaglio è una creatura, deve riuscire un Tiro Salvezza di Tempra CD 14 o divenire ammalato. La malattia non produce alcun effetto per 1 minuto e può essere rimossa da qualsiasi magia che curi le malattie. Dopo 1 minuto, la pelle della creatura ammalata diventa trasparente e viscida, la creatura non può recuperare punti ferita a meno che non sia sott'acqua, e la malattia può essere rimossa solo da \emph{guarire} o un altro incantesimo cura malattie di Difficoltà 29 o più. Quando la creatura si trova al di fuori di un corpo d'acqua, subisce 6 (1d12) danni da acido ogni 10 minuti a meno che la sua pelle non venga bagnata prima che siano passati questi 10 minuti.

\emph{\textbf{Coda.} Attacco con arma da mischia}: +9 a colpire, portata 3 m, un bersaglio.

\emph{Colpisce:} 15 (3d6 + 5) danni da botta.

\emph{\textbf{Schiavizzare (3/Giorno).}} L'aboleth prende a bersaglio una creatura che può vedere entro 9 metri da esso. Il bersaglio deve riuscire un Tiro Salvezza di Volontà CD 14 o restare affascinato magicamente dall'aboleth finché l'aboleth muore o i due si trovano su piani di esistenza differenti. Il bersaglio affascinato è sotto il controllo dell'aboleth e non può effettuare reazioni. L'aboleth e il bersaglio possono comunicare telepaticamente tra di loro a qualsiasidistanza. 

Ogniqualvolta il bersaglio affascinato subisce danni, può ripetere il Tiro Salvezza. Se lo riesce, l'effetto termina. Non più di una volta ogni 24 ore, può ripetere il Tiro Salvezza quando si trova almeno a 1,5 chilometri di distanza dall'aboleth.

\textbf{Azioni Aggiuntive}

L'aboleth può effettuare 3 Azioni aggiuntive, scelte tra le opzioni seguenti. Può usare solo un'opzione leggendaria alla volta e solo al termine del turno di un'altra creatura. L'aboleth recupera le Azioni aggiuntive spese all'inizio del proprio turno.

\textbf{Individuare.} L'aboleth effettua una prova di Saggezza (Consapevolezza).

\textbf{Risucchio Psichico (Costa 2 Azioni).} Una creatura affascinata dall'aboleth subisce 10 (3d6) danni psichici, e l'aboleth recupera un numero di punti ferita pari al danno subito dalla creatura.

\textbf{Spazzata di Coda.} L'aboleth effettua un attacco di coda.

\textbf{Ecologia}\\
Ambiente: Qualsiasi Acquatico\\
Organizzazione: Solitario, coppia, nidiata (3-6) o branco (7-19)\\
Tesoro: Doppio\\
\textbf{Descrizione}\\
Come suggerisce il loro aspetto primitivo, gli ermafroditi aboleth sono fra le più antiche forme di vita al mondo. Già antichi quando gli dei cominciarono ad interessarsi del Piano Materiale, gli aboleth hanno sempre vissuto lontani dagli altri mortali: sono alieni, freddi e sempre occupati ad intessere piani. Un tempo governavano il mondo in un vasto impero, ed oggi vedono le altre forme di vita come cibo o schiavi... a volte entrambe le cose assieme. Disprezzano gli dei, poiché ritengono di essere loro i veri signori del creato, un aboleth è lungo 7 metri e pesa circa 3,2 tonnellate. Nelle più oscure profondità del mare, gli aboleth abitano ancora nelle loro grottesche città, ciclopiche e nauseabonde. Sono serviti da innumerevoli schiavi presi da ogni nazione, sia terrestre che marina, e quelli terrestri sono doppiamente schiavi dei loro padroni e del loro muco, che permette loro di respirare sott'acqua, gli aboleth incontrati da soli sono di solito esploratori provenienti da queste città nascoste, in cerca di nuovi schiavi.\\



\subsection{Angeli}

\medskip\index{Mostri - Angelo Deva}\textbf{Angelo Deva}

\emph{Medio celestiale, legale buono}

\textbf{FORZA} +4

\textbf{DESTREZZA} +4

\textbf{COSTITUZIONE} +4

\textbf{INTELLIGENZA} +3

\textbf{SAGGEZZA} +5

\textbf{CARISMA} +5

\textbf{Iniziativa} +4 -- \textbf{Difesa} 22

\textbf{Punti Ferita} 136 (16d8 + 64)

\textbf{Movimento} 9 m, volo 27 m

\textbf{Tiri Salvezza} Tempra +16, Riflessi +13, Volontà +11

\textbf{Competenze} Percepire Emozioni +9, Consapevolezza +9

\textbf{Resistenze ai Danni} da Luce; da botta, perforante e tagliente di attacchi non magici

\textbf{Immunità alle Condizioni} affascinato, affaticamento, spaventato

\textbf{Sensi} scurovisione 36 m

\textbf{Linguaggi} tutte, telepatia 36 m

\textbf{Sfida} 10 (5.900 PE)

\emph{\textbf{Armi Angeliche.}} Gli attacchi con arma del deva sono magici. Quando il deva colpisce con qualsiasi arma, l'arma infligge 4d8 danni da Luce aggiuntivi (già compresi nell'attacco).

\emph{\textbf{Incantesimi Innati.}} La caratteristica da incantatore innato del deva è il Carisma (CD 17 per i Tiri Salvezza degli incantesimi). Il deva può lanciare in maniera innata i seguenti incantesimi, con l'uso delle sole componenti verbali: 

A volontà: \emph{individuazione del bene e del male}

1/giorno: \emph{comunione, rianimare morti}

\emph{\textbf{Resistenza alla Magia.}} Il deva ha +1d6 ai Tiri Salvezza contro incantesimi e altri effetti magici.

\textbf{Azioni}

\emph{\textbf{Multiattacco.}} Il deva effettua due attacchi da mischia.

\emph{\textbf{Mazza.} Attacco con arma da mischia}: +8 a colpire, portata 1 m, un bersaglio.

\emph{Colpisce:} 7 (1d6 + 4) danni da botta più 18 (4d8) danni da Luce.

\emph{\textbf{Tocco Guaritore (3/Giorno).}} Il deva entra a contatto con un'altra creatura. Il bersaglio recupera magicamente 20 (4d8 + 2) punti ferita ed è libero da qualsiasi cecità, malattia, maledizione, sordità o veleno.

\emph{\textbf{Mutare Forma.}} Il deva può trasformarsi magicamente in un umanoide o bestia il cui grado di sfida sia pari o inferiore al proprio, o tornare alla sua vera forma. Alla morte ritorna alla sua vera forma. Qualsiasi equipaggiamento stia indossando o trasportando viene assorbito o trasportato nella nuova forma (a scelta del deva).

Nella nuova forma, il deva mantiene le sue statistiche di gioco e la facoltà di parlare, ma la sua Difesa, metodi di movimento, Forza, Destrezza e sensi speciali vengono rimpiazzati da quelli della nuova forma, e ottiene qualsiasi statistica o capacità (Azioni aggiuntive e azioni da tana) possedute dalla sua nuova forma e non dalla sua originale.

\textbf{Ecologia}
Ambiente: Qualsiasi piano di con Tratti buono\\
Organizzazione: Solitario, coppia, o squadriglia (3-6)\\
Tesoro: Doppio (Spadone Infuocato +1, altro tesoro)\\
\textbf{Descrizione}\\
I deva movanici compongono i ranghi della fanteria delle armate celesti, sebbene trascorrano la maggior parte del loro tempo pattugliando il Piano Positivo, quello Negativo e quello Materiale. Sul Piano Positivo sorvegliano le anime buone erranti, e questa volta li mette in conflitto con gli Jyoti. Sul Piano Negativo combattono i non morti, gli Sceaduinar e altri strani esseri che cacciano nel famelico vuoto. Le loro rare volte sul Piano Materiale hanno solitamente lo scopo di portare aiuto a potenti mortali, quando un grande pericolo minaccia di far cadere nelle mani del male un intero regno.\\


\medskip\index{Mostri - Angelo Planetar}\textbf{Angelo Planetar}

\emph{Grande celestiale, legale buono}

\textbf{FORZA} +7

\textbf{DESTREZZA} +5

\textbf{COSTITUZIONE} +7

\textbf{INTELLIGENZA} +4

\textbf{SAGGEZZA} +6

\textbf{CARISMA} +7

\textbf{Iniziativa} +5 -- \textbf{Difesa} 27

\textbf{Punti Ferita} 200 (16d10 + 112)

\textbf{Movimento} 12 m, volo 36 m

\textbf{Tiri Salvezza} Tempra +19, Riflessi +11, Volontà +19

\textbf{Competenze} Consapevolezza +11

\textbf{Resistenze ai Danni} da Luce;

\textbf{Immunità alle Condizioni} affascinato, affaticamento, spaventato, armi +1

\textbf{Sensi} visione del vero 36 m

\textbf{Linguaggi} tutte, telepatia 36 m

\textbf{Sfida} 16 (15.000 PE)

\emph{\textbf{Armi Angeliche.}} Gli attacchi con arma del planetar sono magici. Quando colpisce con qualsiasi arma, l'arma infligge 5d8 danni da Luce aggiuntivi (già compresi nell'attacco).

\emph{\textbf{Consapevolezza Divina.}} Il planetar riconosce immediatamente le bugie.

\emph{\textbf{Incantesimi Innati.}} La caratteristica da incantatore innato del planetario è il Carisma (CD 20 per i Tiri Salvezza degli incantesimi). Il planetario può lanciare in maniera innata i seguenti incantesimi, senza bisogno di componenti materiali:

A volontà: \emph{individuazione del bene e del male}, \emph{invisibilità} (solo personale)

3/giorno: \emph{barriera di lame, colpo infuocato, dissolvi il bene e il} \emph{male, rianimare morti}

1/giorno: \emph{comunione, controllare tempo atmosferico, piaga degli insetti}

\emph{\textbf{Resistenza alla Magia.}} Il planetar ha +1d6 ai Tiri Salvezza contro incantesimi e altri effetti magici.

\textbf{Azioni}

\emph{\textbf{Multiattacco.}} Il planetar effettua due attacchi da mischia.

\emph{\textbf{Spadone.} Attacco con arma da mischia}: +12 a colpire, portata 1 m, un bersaglio.

\emph{Colpisce:} 21 (4d6 + 7) danni taglienti più 22 (5d8) danni da Luce.

\emph{\textbf{Tocco Guaritore (4/Giorno).}} Il planetar entra a contatto con un'altra creatura. Il bersaglio recupera magicamente 30 (6d8 + 3) punti ferita ed è libero da qualsiasi cecità, malattia, maledizione, sordità o veleno.

\textbf{Ecologia}
Ambiente: Qualsiasi piano con Tratti buono\\
Organizzazione: Solitario o coppia\\
Tesoro: Doppio (Spadone Sacro +3)\\
\textbf{Descrizione}\\
I planetar sono i generali delle armate celestiali volti alla distruzione del male. Un planetar è di norma alto 2,7 metri e pesa circa 250 kg. Sono ottimi diplomatici, ma contro gli immondi preferiscono una guerra piuttosto che negoziare una pace.\\


\medskip\index{Mostri - Angelo Solar}\textbf{Angelo Solar}

\emph{Grande celestiale, legale buono}

\textbf{FORZA} +8

\textbf{DESTREZZA} +6

\textbf{COSTITUZIONE} +8

\textbf{INTELLIGENZA} +7

\textbf{SAGGEZZA} +7

\textbf{CARISMA} +10

\textbf{Iniziativa} +7 -- \textbf{Difesa} 31

\textbf{Punti Ferita} 243 (18d10 + 144) 

\textbf{Movimento} 15 m, volo 45 m

\textbf{Tiri Salvezza} Tempra +25, Riflessi +14, Volontà +23

\textbf{Competenze} Consapevolezza +14

\textbf{Resistenze ai Danni} da Luce;

\textbf{Immunità ai Danni} da Vuoto, veleno, armi +2

\textbf{Immunità alle Condizioni} affascinato, avvelenato, affaticamento, spaventato

\textbf{Sensi} visione del vero 36 m

\textbf{Linguaggi} tutte, telepatia 36 m 

\textbf{Sfida} 21 (33.000 PE)

\emph{\textbf{Armi Angeliche.}} Gli attacchi con arma del solar sono magici. Quando colpisce con qualsiasi arma, l'arma infligge 6d8 danni da Luce aggiuntivi (già compresi nell'attacco).

\emph{\textbf{Consapevolezza Divina.}} Il solar riconosce immediatamente le bugie.

\emph{\textbf{Incantesimi Innati.}} La caratteristica da incantatore innato del solar è il Carisma (CD 25 per i Tiri Salvezza degli incantesimi). Il solar può lanciare in maniera innata i seguenti incantesimi, senza bisogno di componenti materiali:

A volontà: \emph{individuazione del bene e del male}, \emph{invisibilità} (solo personale)

3/giorno: \emph{barriera di lame, colpo infuocato, dissolvi il bene e il male, resurrezione}

1/giorno: \emph{comunione, controllare tempo atmosferico}

\emph{\textbf{Resistenza alla Magia.}} Il solar ha +1d6 ai Tiri Salvezza contro incantesimi e altri effetti magici.

\textbf{Azioni}

\emph{\textbf{Multiattacco.}} Il solar effettua due attacchi con lo spadone.

\emph{\textbf{Spadone.} Attacco con arma da mischia}: +15 a colpire, portata 1 m, un bersaglio.

\emph{Colpisce:} 22 (4d6 + 8) danni taglienti più 27 (6d8) danni da Luce.

\emph{\textbf{Arco Lungo dell'Uccisione.} Attacco con arma a distanza}: +13 a colpire, gittata 45m, un bersaglio.

\emph{Colpisce:} 15 (2d8 + 6) danni perforanti più 27 (6d8) danni da Luce. Se il bersaglio è una creatura con 100 punti ferita o meno, deve riuscire un Tiro Salvezza di Tempra CD 15 o morire.

\emph{\textbf{Spada Volante.}} Il solare libera il suo spadone perché fluttui magicamente in uno spazio non occupato entro 1 metro da lui.  Se il solare può vedere la spada, con un'azione bonus le può ordinare mentalmente di volare per un massimo di 15 metri ed effettuare un attacco contro un bersaglio o ritornare nella mano del solare. Se la spada fluttuante è bersaglio di un effetto, si considera come se fosse impugnata dal solare. Se il solare muore, la spada fluttuante cade a terra.

\emph{\textbf{Tocco Guaritore (4/Giorno).}} Il solare entra a contatto con un'altra creatura. Il bersaglio recupera magicamente 40 (8d8 + 4) punti ferita ed è libero da qualsiasi cecità, malattia, maledizione, sordità o veleno.

Il solare può effettuare 3 azioni aggiuntive, scelte tra le opzioni seguenti. Può usare solo un'Azione Aggiuntiva alla volta e solo al termine del round di un'altra creatura. Il solare recupera le azioni aggiuntive spese all'inizio del proprio round. 

\textbf{Esplosione Incandescente (Costa 2 Azioni).} Il solare emette energia magica divina. Ogni creatura di sua scelta, in un raggio di 3 metri, deve effettuare un Tiro Salvezza su Riflessi DC  30, subendo 14 (4d6) danni da fuoco più 14 (4d6) danni da Luce se fallisce il Tiro Salvezza, o la metà se lo riesce. 

\textbf{Sguardo Accecante (Costa 3 Azioni).} Il solare prende a bersaglio una creatura entro 9 metri e che possa vedere. Se il bersaglio può vedere il solare, il bersaglio deve riuscire un Tiro Salvezza su Tempra DC  18 o restare accecato finché un incantesimo come \emph{ristorare inferiore} non rimuoverà la cecità.

\textbf{Teletrasporto.} Il solare si teletrasporta magicamente fino a 36 metri di distanza, insieme a tutto l'equipaggiamento che sta indossando o trasportando, in uno spazio non occupato e che può vedere.

\textbf{Ecologia}\\
Ambiente: Qualsiasi piano con Tratti buono\\
Organizzazione: Solitario o coppia\\
Tesoro: Doppio (Armatura Completa +5, Spadone Danzante +5, Arco Lungo Composito +5 [For +9])\\
\textbf{Descrizione}\\
I solar sono i più potenti fra gli angeli, solitamente braccio destro di una divinità o campioni di cause che portano beneficio ad un intero mondo o piano. Un solar ha di solito aspetto quasi umano, anche se alcuni di essi somigliano ad altre razze umanoidi ed alcuni hanno anche forme più inusuali. Un solar è alto circa 2,7 metri, pesa circa 250 kg ed ha una voce profonda ed imperiosa, impossibile da ignorare. La maggior parte di essi ha pelle argentata o dorata.\\
Benedetti con una serie di capacità magiche più potenti, i solar sono avversari terribili in grado di uccidere da soli le più potenti creature malvagie. Fra i celestiali sono considerati i più eccellenti cercatori di tracce, ed i migliori fra loro, si dice, sono in grado di seguire tracce vecchie di giorni lasciate da un Diavolo della Fossa attraverso il Piano Astrale. Alcuni di essi prendono il manto di uccisori di mostri e danno la caccia a potenti immondi e non morti come i divoratori, le megere notturne, le Ombre Notturne ed i Diavoli della Fossa, compiendo perfino incursioni nei piani  malvagi e nel Piano dell'Energia Negativa per distruggere queste creature alla fonte, prima che possano fare del male ai mortali. Alcuni fra i solar più antichi hanno portato a termine la loro missione, ed hanno fama di uccisori di creature oggi estinte.\\
I solar accettano il ruolo di guardiani, di solito di concetti soprannaturali o di oggetti o creature di grande importanza. Su un mondo, un gruppo di solar protegge i condotti di energia del sole contro i tentativi di spegnerlo e di portare tenebre eterne operati da razze malvagie come gli Elfi. Su un altro, sette solar vegliano su sette catene mistiche che tengono gli dèi del male imprigionati in un semipiano. Su un altro ancora, un solar con una spada fiammeggiante protegge il Paradiso Terrestre, impedendo a tutte le creature di entrare.\\
Nei mondi in cui gli dèi possono prendere forma fisica, i solar vengono inviati per diventare profeti e guru (spesso in guisa di mortali), gettando così le fondamenta di culti che diverranno grandi religioni. Nei mondi oppressi dal male, i solar sono i sacerdoti clandestini che portano speranza agli oppressi o che si lasciano martirizzare così che la loro essenza possa esplodere nelle regioni circostanti e crescere nel cuore dei futuri eroi.\\
Pur non essendo divinità, il potere dei solar si avvicina a quello dei semidei, e spesso fanno da consiglieri per le divinità più giovani o deboli. In alcune fedi politeiste, i mortali venerano uno o più solar come aspetti o servitori alla pari delle vere divinità (comunque mai senza l'approvazione della divinità in questione) o considerano i solar più famosi come figli, consorti o amanti delle vere divinità (cosa che, a seconda della divinità, potrebbe corrispondere al vero).\\
A differenza degli altri angeli, la maggior parte dei solar viene creata come diretta servitrice degli dèi, amalgamando anime buone ed energia divina pura, ma sempre più spesso questi potenti angeli vengono creati attraverso la "promozione" di angeli minori come deva e planetar. Raramente accade che anime particolarmente potenti e pure ascendano direttamente allo status di solar. I più antichi fra loro sono anteriori alla creazione dei mortali, e sono fra le prime creazioni degli dèi. Questi solar sono campioni fra i loro simili, ed hanno poca o nessuna interazione con i mortali, concentrandosi invece su concetti astratti quali la gravità, l'entropia, la materia oscura ed il male primevo.\\
I solar che passano molto tempo sul Piano Materiale, specialmente quelli che prendono la forma di mortali, sono a volta fonte di linee di sangue aasimar o mezzo-celestiali nelle famiglie umane, a volte a causa di una storia d'amore, a volte semplicemente per la vicinanza dei mortali alle loro emanazioni celestiali. È raro che vi siano loro discendenti diretti, e quando ciò accade è sempre una madre mortale a portare in grembo il figlio: anche se i solar possono apparire di qualsiasi sesso, gli dèi non hanno concesso loro la possibilità di dare alla luce un figlio. È per questo che i solar tendono a cercare un'amante mortale. Gli altri solar hanno poca considerazione di un loro simile che dà un figlio ad una mortale, quindi i padri solar tendono ad evitare i contatti con la loro progenie, per evitare di attirare vergogna su di sé. I solar, però, tendono a controllare da lontano i loro figli e, in tempo di difficoltà, ad aiutarli, anche se in modi misteriosi e discreti.\\
Tutti gli angeli rispettano il potere e la saggezza dei solar, e sebbene essi tendano a lavorare da soli, a volte comandano armate guidate da planetar e fanno da generali per le grandi incursioni contro le legioni dell'Inferno o le orde dell'Abisso.\\


\medskip\index{Mostri - Ankheg}\textbf{Ankheg}

\emph{Grande mostruosità, disallineato}

\textbf{FORZA} +3

\textbf{DESTREZZA} +0

\textbf{COSTITUZIONE} +1

\textbf{INTELLIGENZA} -5

\textbf{SAGGEZZA} +1

\textbf{CARISMA} -2

\textbf{Iniziativa} +0 -- \textbf{Difesa} 15 , 12 mentre è prono

\textbf{Punti Ferita} 39 (6d10 + 6)

\textbf{Movimento} 9 m, scavo 3 m

\textbf{Sensi} scurovisione 18 m, percezione tellurica 18 m, 

\textbf{Linguaggi} -

\textbf{Sfida} 2 (450 PE)

\textbf{Azioni}

\emph{\textbf{Morso.} Attacco con arma da mischia}: +5 a colpire, portata 1 m, un bersaglio.

\emph{Colpisce:} 10 (2d6 + 3) danni taglienti più 3 (1d6) danni da acido. Se il bersaglio è una creatura di taglia Grande o inferiore, è afferrata (CD 13 per fuggire). Fino al termine dell'afferrare, l'ankheg può mordere solo la creatura afferrata e ha +1d6 ai tiri di attacco contro di essa.

\emph{\textbf{Spruzzo Acido (Ricarica 6).}} L'ankheg sputa acido in una linea lunga 9 metri e larga 1 metro, purché non stia afferrando nessuna creatura. Ogni creatura su quella linea deve effettuare un Tiro Salvezza di Riflessi CD 13, e subire 10 (3d6) danni da acido se fallisce il Tiro Salvezza, o la metà di questi danni se lo riesce.

\textbf{Ecologia}\\
Ambiente: Pianure temperate o calde\\
Organizzazione: Solitario, coppia o nido (3-6)\\
Tesoro: Accidentale\\
\textbf{Descrizione}\\
Gli ankheg sono una piaga fin troppo comune per le zone rurali. Questo mostro scavatore grande come un cavallo in genere evita le zone abitate più popolose, ma la sua predilezione per la carne del bestiame e degli esseri umani li tiene lontano dalle zone disabitate. Il loro habitat favorito è rappresentato delle campagne rurali, dato che il terriccio smosso rende loro molto facile muoversi scavando. Si narra di ankheg più grandi che vivono nei deserti remoti, che si cibano di Scorpioni e Cammelli e vengono raramente in contatto con la civiltà (un ankheg del deserto è un ankheg avanzato Enorme).\\
In combattimento, gli ankheg preferiscono attaccare con il morso. Contro più avversari, un ankheg Afferra uno dei bersagli e tenta di ritirarsi sottoterra. Una creatura trascinata sottoterra può respirare, anche se con difficoltà (anche l'ankheg deve farlo, quindi i tunnel sono abbastanza porosi), ma spesso viene mangiata viva prima che i suoi compagni possano salvarla.\\
Gli ankheg scavano con le loro gambe e le mandibole, muovendosi velocissimi attraverso il terriccio, la sabbia e la ghiaia (non la roccia). Un ankheg che scava si ferma spesso a costruire tunnel, cospargendo le pareti con una densa secrezione orale. Se un ankheg vuole costruire un tunnel mentre scava deve muoversi a metà della propria velocità di scavare. Un tipico tunnel di ankheg è alto e largo 3 metri, di forma vagamente circolare e lungo da 18 a 45 metri ([1d10+5]×10). I gruppi di ankheg condividono lo stesso territorio e creano complesse reti di tunnel sotto le campagne, a volte creando voragini nei punti in cui troppi di essi scavano allo stesso tempo.\\
Anche se gli ankheg somigliano ad immensi insetti sono più intelligenti e, con un po' di tempo e un buon addestratore, possono diventare animali domestici o da carico. Il fatto che anche "addomesticati" gli ankheg tendano a sputare acido quando spaventati o sorpresi li rende poco sicuri nelle regioni più civilizzate, ma fra le razze selvagge, come gli Hobgoblin, i Trogloditi e soprattutto gli Orchi sono popolari come guardiani o perfino animali da salotto. Un ankheg può raggiungere una lunghezza di 3 metri e pesare circa 400 kg.\\


\medskip\index{Mostri - Arpia}\textbf{Arpia}

\emph{Media mostruosità, caotico malvagio}

\textbf{FORZA} +1

\textbf{DESTREZZA} +1

\textbf{COSTITUZIONE} +1

\textbf{INTELLIGENZA} -2

\textbf{SAGGEZZA} +0

\textbf{CARISMA} +1

\textbf{Iniziativa} +1 -- \textbf{Difesa} 12

\textbf{Punti Ferita} 38 (7d8 + 7)

\textbf{Movimento} 6 m, volo 12 m

\textbf{Linguaggi} Comune

\textbf{Sfida} 1 (200 PE)

\textbf{Azioni}

\emph{\textbf{Multiattacco.}} L'armatura effettua due attacchi: uno con gli artigli e uno con il randello.

\emph{\textbf{Artigli.} Attacco con arma da mischia}: +3 a colpire, portata 1 m, un bersaglio.

\emph{Colpisce:} 6 (2d4 + 1) danni taglienti.

\emph{\textbf{Randello.} Attacco con arma da mischia}: +3 a colpire, portata 1 m, un bersaglio.

\emph{Colpisce:} 3 (1d4 + 1) danni da botta.

\emph{\textbf{Canto Ammaliatore.}} L'arpia canta una melodia magica. Ogni umanoide e gigante entro 90 metri dall'arpia e che possa udire la canzone deve riuscire un Tiro Salvezza di Volontà CD 11 o restare affascinato fino al termine della canzone. L'arpia deve effettuare un'azione bonus durante il suo prossimo turno per continuare a cantare. Può smettere di cantare in qualsiasi momento. Il canto ha termine se l'arpia è inabile.

Mentre è affascinato dall'arpia, un bersaglio è inabile e ignora le canzoni di altre arpie. Se il bersaglio affascinato si trova a più di 1 metro dall'arpia, il bersaglio deve muoversi durante il proprio turno per dirigersi verso l'arpia usando la via più diretta. Egli non eviterà attacchi di opportunità, ma prima di muoversi in un terreno pericoloso, come lava o un pozzo, e prima di subire danno da qualsiasi fonte che non sia l'arpia, il bersaglio potrà ripetere il Tiro Salvezza. Una creatura può ripetere il Tiro Salvezza al termine di ciascun proprio turno. Se il Tiro Salvezza ha successo, l'effetto ha termine per quel bersaglio.

Un bersaglio che riesce il Tiro Salvezza è immune al canto di quell'arpia per le successive 24 ore.

\textbf{Ecologia}\\
Ambiente: Paludi Temperate\\
Organizzazione: Solitario, coppia o stormo (3-12)\\
Tesoro: Standard (Armatura di Cuoio, Morning Star e altro tesoro)\\
\textbf{Descrizione}\\
Spesso viste come creature malvagie e corrotte, le arpie sanno come gli altri pensano e agiscono. Questa capacità percettiva offre loro un vantaggio nel trovare i loro pasti preferiti. Sebbene le creature selvatiche cadano facilmente vittime del canto ammaliatore, queste malvagie donne-uccello preferiscono pasti conditi con complessi pensieri senzienti. Le facili prede rendono il pasto noioso.\\
Anche se in definitiva selvagge e senza alcun rimorso per le loro azioni, diverse arpie vivono presso le società umanoidi e si divertono a sfruttare le creature che reputano potenziali pasti.\\
Le arpie tendono ad indossare ninnoli e ciondoli rubati alle loro vittime, perché amano compiacersi dei brillanti ornamenti degli uomini. Da vicino queste creature trasudano del puzzo delle loro vittime divorate e raramente lasciano che le creature non ancora ammaliate si avvicinino troppo, cosicché non sentano l'odore del sangue e della putrefazione sulle loro penne. Per questo motivo, molte arpie si cospargono di profumi e oli aromatici.\\
Le arpie sono marcatamente differenti a seconda della regione in cui vivono. Alcune assomigliano ad una mescolanza di avvoltoi e donne, mentre altri portano sulle penne i tratti regali di falchi e falconi. Rare nidiate di arpie, in luoghi isolati e tropicali del mondo, hanno anche piume colorate come i pappagalli.\\


\medskip\index{Mostri - Azer}\textbf{Azer}

\emph{Media elementale, legale neutrale}

\textbf{FORZA} +3

\textbf{DESTREZZA} +1

\textbf{COSTITUZIONE} +2

\textbf{INTELLIGENZA} +1

\textbf{SAGGEZZA} +1

\textbf{CARISMA} +0

\textbf{Iniziativa} +1 -- \textbf{Difesa} 18 (armatura naturale, scudo)

\textbf{Punti Ferita} 39 (6d8 + 12)

\textbf{Movimento} 9 m

\textbf{Tiri Salvezza} Tempra +2, Riflessi +1, Volontà +1

\textbf{Immunità ai Danni} fuoco, veleno

\textbf{Immunità alle Condizioni} avvelenato

\textbf{Linguaggi} Ignan

\textbf{Sfida} 2 (450 PE)

\emph{\textbf{Armi Riscaldate.}} Quando l'azer colpisce con un'arma da mischia in metallo, infligge 3 (1d6) danni da fuoco aggiuntivi (già inclusi nell'attacco).

\emph{\textbf{Corpo Riscaldato.}} Una creatura che entri a contatto con l'azer o lo colpisca con un attacco da mischia mentre si trova entro 1 metro da lui subisce 5 (1d10) danni da fuoco.

\emph{\textbf{Fuoco Vivente.}} Un azer non ha bisogno di cibo, bevande o di dormire.

\emph{\textbf{Illuminazione.}} L'azer irradia luce intensa in un raggio di 3 metri e luce fioca per ulteriori 3 metri.

\textbf{Azioni}

\emph{\textbf{Martello da Guerra.} Attacco con arma da mischia}: +5 a colpire, portata 1 m, un bersaglio.

\emph{Colpisce:} 7 (1d8 + 3) danni da botta, o 8 (1d10 + 3) danni da botta se usato a due mani per effettuare un attacco da mischia, più 3 (1d6) danni da fuoco.

\textbf{Ecologia}\\
Ambiente: Qualsiasi terreno (Piano del Fuoco)\\
Organizzazione: Solitario, coppia, gruppo (3-6), squadra (11-20 più 2 sergenti di 3° livello e 1 capo di 3°-6° livello) o clan (30-100 più 50\% di non combattenti più 1 sergente di 3° livello ogni 20 adulti, 5 tenenti di 5° livello e 3 capitani di 7° livello)\\
Tesoro: Standard (Corazza a Scaglie Perfetta, Martello da Guerra Perfetto, Martello Leggero, altro tesoro)\\
\textbf{Descrizione}\\
Una Razza orgogliosa e industriosa proveniente dal Piano del Fuoco, gli Azer lavorano nelle loro fortezze di bronzo e d'ottone, sempre pronti a combattere la loro lunga e ribollente guerra contro gli Efreet. Gli Azer vivono in una società in cui ogni membro sa qual è il suo posto. Nati con specifici doveri, solitamente legati alle attività del padre o della madre, gli Azer si dedicano a queste occupazioni per tutta la vita. Un sistema di caste provvede a tenere ulteriormente in riga la società Azer. I nobili, che regnano senza dover rendere conto a nessuno, indossano kilt di ottone decorato come simbolo della loro casta, mentre quelli dei mercanti e dei proprietari di negozi sono in resistente bronzo. I kilt di rame sono indossati dalla casta lavoratrice, composta da servitori, artigiani e braccianti.\\
Capaci di incanalare calore tramite le Armi e gli attrezzi in metallo, gli Azer non utilizzano quasi mai Armi non metalliche, e prediligono il corpo a corpo agli attacchi a distanza. Sono soliti fare prigionieri, riportandoli alle loro fortezze e obbligandoli a lavorare per loro per un anno e un giorno.\\
Nella leggendaria Città d'Ottone abitano più di mezzo milione di Azer. La maggior parte di questi sfortunati Azer vive una vita di Schiavitù sotto gli Efreet. Gli Azer soggiogati a questa Schiavitù continuano a eseguire i loro doveri senza porre domande, preferendo aspettare la conclusione dei loro contratti o sperando che i loro padroni muoiano o vengano sconfitti. La dedizione all'ordine arde intensa in questa Razza, al punto che alcuni degli Schiavi Azer fungono da supervisori sulla loro stessa gente. Al di fuori della Città d'Ottone, gli Azer sono liberi di vivere le loro vite, spesso in altre metropoli Planari, creando oggetti, vendendo merci e gestendo taverne.\\
A un occhio non allenato gli Azer si somigliano tra loro in modo impressionante. Sono alti 1,2 metri ma pesano 100 kg.\\


\medskip\index{Mostri - Basilisco}\textbf{Basilisco}

\emph{Media mostruosità, disallineato}

\textbf{FORZA} +3

\textbf{DESTREZZA} -1

\textbf{COSTITUZIONE} +2

\textbf{INTELLIGENZA} -4

\textbf{SAGGEZZA} -1

\textbf{CARISMA} -2

\textbf{Iniziativa} -1 -- \textbf{Difesa} 17

\textbf{Punti Ferita} 52 (8d8 + 16)

\textbf{Movimento} 6 m

\textbf{Sensi} scurovisione 18 m

\textbf{Linguaggi} -

\textbf{Sfida} 3 (700 PE)

\emph{\textbf{Sguardo Pietrificante.}} Se una creatura comincia il suo turno entro 9 metri dal basilisco e i due si possono vedere vicendevolmente, se non  inabile il basilisco può obbligare la creatura ad effettuare un Tiro Salvezza di Tempra CD 12. Se la creatura fallisce il Tiro Salvezza, inizia magicamente a trasformarsi in pietra ed è intralciata. La creatura deve ripetere il Tiro Salvezza al termine del suo prossimo turno. Se lo riesce, l'effetto termina. Se lo fallisce, la creatura è pietrificata finché non viene liberata dall'incantesimo \emph{ristorare} \emph{superiore} o altra magia.

Una creatura che non sia sorpresa, può distogliere lo sguardo per evitare il Tiro Salvezza all'inizio del suo turno. In quel caso, non potrà vedere il basilisco fino all'inizio del suo prossimo turno, quando potrà distogliere nuovamente lo sguardo. Se nel frattempo dovesse guardare il basilisco, dovrebbe immediatamente effettuare il Tiro Salvezza.

Se il basilisco si trova entro 9 metri dal suo riflesso a luce intensa e lo vede, lo scambia per un rivale e diventa il bersaglio del proprio sguardo.

\textbf{Azioni}

\emph{\textbf{Morso.} Attacco con arma da mischia}: +5 a colpire, portata 1 m, un bersaglio.

\emph{Colpisce:} 10 (2d6 + 3) danni perforanti più 7 (2d6) danni da veleno.

\textbf{Ecologia}\\
Ambiente: Qualsiasi\\
Organizzazione: Solitario, coppia o colonia (3-6)\\
Tesoro: Accidentale\\
\textbf{Descrizione}\\
Il basilisco, spesso chiamato "Re dei Serpenti" è un rettile a otto zampe di indole aggressiva che ha la capacità di trasformare le creature in pietra con il suo sguardo. La leggenda narra che, come la Cockatrice, i primi basilischi nacquero da uova deposte da serpenti e covate da galli, ma ben poco nella fisiologia del basilisco lascia spazio a questa teoria.\\
I basilischi vivono in quasi tutti gli ambienti asciutti, dalla foresta al deserto, e la loro pelle tende a rispecchiare l'ambiente che li circonda: un basilisco del deserto può essere bronzeo o marrone, mentre uno che vive nelle foreste può essere di colore verde acceso. Tendono a usare come rifugio le grotte, le tane o altre zone riparate. Questi rifugi sono spesso segnalati da statue raffiguranti persone e animali in pose naturali, che non sono altro che i resti pietrificati degli sventurati imbattutisi in un basilisco.\\
I basilischi hanno la capacità di consumare le creature pietrificate; l'acido prodotto dal loro stomaco dissolve ed estrae sostanze nutrienti dalla pietra, sebbene il processo sia lento e inefficiente, il che li rende pigri e inerti. Di conseguenza, i basilischi raramente attaccano o cacciano le prede che evitano il loro sguardo, contando sulla loro Furtività e l'elemento di sorpresa al fine di non rimanere senza cibo. Quando non sono in attesa dei piccoli mammiferi, uccelli o rettili che fanno parte della loro dieta, i basilischi passano il tempo a dormire nelle tane. Coloro che sono abbastanza coraggiosi da catturare i basilischi o da nascondere un tesoro vicino a loro, scoprono che questi esseri possono fare da custodi o da cani da guardia.\\
Un basilisco adulto è lungo quasi 4 metri, di cui la metà occupata dalla lunga coda, e pesa 135 chili. Alcune razze presentano delle piccole corna ricurve sul naso o piccole creste di pungiglioni ossuti sopra la testa simili a una corona. Sebbene siano creature in genere solitarie che si riuniscono solo per accoppiarsi e deporre le uova, in zone particolarmente pericolose possono riunirsi in piccoli gruppi per proteggersi e attaccare gli intrusi in massa.\\
Per motivi ignoti, le donnole e i furetti sono immuni allo sguardo del basilisco, e a volte si intrufolano nelle tane mentre l'adulto è a caccia per cibarsi dei suoi piccoli. Alcune leggende narrano che il sangue di un basilisco può tramutare comuni pietre in un altro materiale, ma probabilmente si tratta di testimoni che hanno mal interpretato la ristorazione magica di creature o di parti del corpo pietrificate in precedenza.\\


\medskip\index{Mostri - Behir}\textbf{Behir}

\emph{Enorme mostruosità, neutrale malvagio}

\textbf{FORZA} +6

\textbf{DESTREZZA} +3

\textbf{COSTITUZIONE} +4

\textbf{INTELLIGENZA} -2

\textbf{SAGGEZZA} +2

\textbf{CARISMA} +1

\textbf{Iniziativa} +3 -- \textbf{Difesa} 23

\textbf{Punti Ferita} 168 (16d12 + 64)

\textbf{Movimento} 15 m, scalata 12 m

\textbf{Competenze} Muoversi Silenziosamente / Nascondersi +7, Consapevolezza +6

\textbf{Immunità al Danno} fulmine

\textbf{Sensi} scurovisione 27 m

\textbf{Linguaggi} Draconico

\textbf{Sfida} 11 (7.200 PE)

\textbf{Azioni}

\emph{\textbf{Multiattacco.}} Il behir effettua due attacchi: uno con il morso e uno per stritolare.

\emph{\textbf{Morso.} Attacco con arma da mischia}: +10 a colpire, portata 3 m, un bersaglio.

\emph{Colpisce:} 22 (3d10 + 6) danni perforanti.

\emph{\textbf{Stritolare.} Attacco con arma da mischia}: +10 a colpire, portata 1 m, una creatura di taglia Grande o inferiore.

\emph{Colpisce:} 17 (2d10 + 6) danni da botta più 17 (2d10 + 6) danni taglienti. Il bersaglio è afferrato (CD 16 per fuggire) Se il behir non sta già stritolando un'altra creatura, il bersaglio è afferrato e intralciato fino al termine dell'afferrare.

\emph{\textbf{Inghiottire.}} Il behir effettua una attacco di morso contro un bersaglio di taglia Media o inferiore che sta afferrando. Se l'attacco colpisce, il bersaglio è inghiottito, e l'afferrare ha termine. Il bersaglio inghiottito è accecato e intralciato, ha copertura totale contro gli attacchi e altri effetti all'esterno del behir, e subisce 21 (6d6) danni da acido all'inizio di ciascun turno del behir. Il behir può inghiottire solo una creatura alla volta.

Se il behir subisce 30 o più danni in un singolo turno da una creatura che ha inghiottito, deve riuscire un Tiro Salvezza di Tempra CD 14 al termine di quel turno o vomitare la creatura, che ricade prona in uno spazio entro 3 metri dal behir. Se il behir muore, una creatura inghiottita non è più intralciata da esso e può uscire dal cadavere utilizzando 5 metri di movimento, uscendo prona.

\emph{\textbf{Soffio di Fulmine (Ricarica 5-6).}} Il behir esala fulmini in una linea lunga 6 metri e larga 1 metro. Ogni creatura su quella linea deve effettuare un Tiro Salvezza di Riflessi CD 16 e subire 66 (12d10) danni da fulmine se fallisce il Tiro Salvezza, o la metà di questi danni se lo riesce.

\textbf{Ecologia}\\
Ambiente: Colline e Deserti Caldi\\
Organizzazione: Solitario o coppia\\
Tesoro: Doppio\\
\textbf{Descrizione}\\
Istintivo e bramoso, il behir trascorre gran parte del tempo a strisciare per le colline sabbiose e le rocce del deserto che formano il suo territorio, dando la caccia a tutte le creature che osano entrare nel suo territorio. Le sue sei paia di zampe robuste e dotate di artigli restano piegate ai suoi fianchi per gran parte del tempo, e si stendono solo in combattimento per afferrare i nemici, per Correre al galoppo o per Scalare i pendii delle scogliere a picco, tane di queste creature.\\
In media il behir è lungo 12 metri e pesa circa 1800 Kg. Oltre alle due corna prominenti sulla testa, molti hanno aculei decorativi a intervalli regolari lungo la spina dorsale.\\
Pur essendo territoriale e bestiale nella sua furia, il behir non è né stupido né necessariamente malvagio anche se, a causa del suo egocentrismo e della tendenza a rivendicare come sua ogni cosa esistente, entra spesso in conflitto con le altre razze. In quanto tale, un behir può essere corrotto o convinto da intrepidi negoziatori disposti ad avvicinarglisi. In questi casi, la tendenza di un behir ad attaccare prima e a ragionare poi (o non ragionare affatto) significa che chiunque cerchi di trovare un accordo deve avere dei validi motivi e far subito colpo sul behir con un'offerta allettante.\\
Spesso si dice che i behir siano in qualche modo legati ai draghi blu, ma la vera natura di questo Legame rimane un mistero. Molti draghi negano qualsiasi Legame e non vedono di buon occhio i behir per la loro scarsa intelligenza: un affronto che fa infuriare i behir, di per sé già impulsivi. Proprio per questo, molti behir portano rancore verso i draghi e sono pronti ad attaccare qualunque drago entri nel loro territorio.\\


\medskip\index{Mostri - Bugbear}\textbf{Bugbear}

\emph{Media umanoide (goblinoide), caotico malvagio}

\textbf{FORZA} +2

\textbf{DESTREZZA} +2

\textbf{COSTITUZIONE} +1

\textbf{INTELLIGENZA} -1

\textbf{SAGGEZZA} +0

\textbf{CARISMA} -1

\textbf{Iniziativa} +2 -- \textbf{Difesa} 17

\textbf{Punti Ferita} 27 (5d8 + 5)

\textbf{Movimento} 9 m

\textbf{Competenze} Muoversi Silenziosamente / Nascondersi +6, Sopravvivenza +2

\textbf{Sensi} scurovisione 18 m

\textbf{Linguaggi} Comune, Goblin

\textbf{Sfida} 1 (200 PE)

\emph{\textbf{Attacco di Sorpresa.}} Se il bugbear sorprende una creatura e la colpisce con un attacco durante il primo round di combattimento, il bersaglio subisce 7 (2d6) danni aggiuntivi
dall'attacco.

\emph{\textbf{Bruto.}} Un'arma da mischia infligge un dado aggiuntivo di danno quando il bugbear colpisce con essa (già incluso nell'attacco).

\textbf{Azioni}

\emph{\textbf{Mazza Chiodata.} Attacco con arma da mischia}: +4 a colpire, portata 1 m, un bersaglio.

\emph{Colpisce:} 11 (2d8 + 2) danni perforanti.

\emph{\textbf{Giavellotto.} Attacco con arma da mischia o a Distanza}: +4 a colpire, portata 1 m o gittata 9m, un bersaglio.

\emph{Colpisce:} 9 (2d6 + 2) danni perforanti in mischia o 5 (1d6 + 2) danni perforanti a gittata.

\textbf{Ecologia}\\
Ambiente: Montagne temperate\\
Organizzazione: Solitario, coppia, gruppo (3-6) o banda da guerra (7-12 più 2 Guerrieri di 1° livello e 1 capitano di 3°-5° livello)\\
Tesoro: Equipaggiamento da PNG (Armatura di Cuoio, Scudo Leggero di Legno, Morning Star, 3 Giavellotti, altro tesoro)\\
\textbf{Descrizione}\\
Il bugbear è il più grande degli esponenti della razza Goblinoide, un bruto dai movimenti pesanti che supera di almeno una testa la maggior parte degli Umani. Sono solitari che preferiscono vivere ed uccidere da soli piuttosto che in tribù, sebbene non sia insolito trovare una piccola banda di Bugbear che collabora o vive con una tribù di Goblin od Hobgoblin fungendo da guardia d'élite o carnefici.\\
I bugbear non formano grandi insediamenti come i goblin o nazioni come gli hobgoblin; preferiscono qualcosa di più piccolo e caotico che li lasci liberi di fare quello che preferiscono (uccidere e torturare) a un livello più personale. Gli umani sono le prede preferite dei bugbear, e la maggior parte di essi annovera la carne umana come uno degli alimenti principali della propria dieta. Macabri trofei quali orecchie e dita sono decorazioni comuni tra i bugbear.\\
I bugbear, quando si rivolgono alla religione, prediligono le divinità dell'omicidio e della violenza, con i vari signori dei demoni tra i preferiti.\\
Un tipico bugbear è alto 2,1 metri e pesa 200 kg.\\


\medskip\index{Mostri - Bulette}\textbf{Bulette}

\emph{Grande mostruosità, disallineato}

\textbf{FORZA} +4

\textbf{DESTREZZA} +0

\textbf{COSTITUZIONE} +5

\textbf{INTELLIGENZA} -4

\textbf{SAGGEZZA} +0

\textbf{CARISMA} -3

\textbf{Iniziativa} +0 -- \textbf{Difesa} 20

\textbf{Punti Ferita} 94 (9d10 + 45)

\textbf{Movimento} 12 m, scavo 12 m

\textbf{Competenze} Consapevolezza +6

\textbf{Sensi} scurovisione 18 m, percezione tellurica 18 m

\textbf{Linguaggi} -

\textbf{Sfida} 5 (1.800 PE)

\emph{\textbf{Salto da Fermo.}} Un bulette può saltare in lungo fino a 9  metri e in alto fino a 5 m con o senza la rincorsa.

\textbf{Azioni}

\emph{\textbf{Morso.} Attacco con arma da mischia}: +7 a colpire, portata 1 m, un bersaglio.

\emph{Colpisce:} 30 (4d12 + 4) danni perforanti.

\emph{\textbf{Salto Letale.}} Se il bulette può saltare di almeno 4  metri come parte del suo movimento, può usare poi questa azione per  atterrare in piedi in uno spazio che contiene una o più creature.  Ciascuna di queste creature deve riuscire un Tiro Salvezza di Tempra o Riflessi CD 16 (a scelta del bersaglio) o venire gettata prona e subire  14 (3d6 + 4) danni da botta più 14 (3d6 + 4) danni taglienti. Se il  Tiro Salvezza riesce, la creatura subisce solo la metà dei danni, non è  gettata prona, e viene spinta di 1 metro fuori dello spazio del  bulette in uno spazio non occupato a scelta della creatura. Se non ci  sono spazi non occupati a gittata, la creatura cade prona nello spazio  del bulette.

\textbf{Ecologia}\\
Ambiente: Colline Temperate\\
Organizzazione: Solitario o coppia\\
Tesoro: Nessuno\\
\textbf{Descrizione}\\
Creazione di uno sconosciuto mago del passato, il bulette ora è diventato un feroce predatore di collina. Scavando rapidamente sotto il terreno, fende la superficie con la sua pinna dorsale lasciandosi dietro una scia caratteristica. Il bulette balza fuori, liberandosi da pietre e terriccio, per fare a pezzi la sua preda senza rimorsi, dando così origine al suo soprannome di "squalo terrestre".\\
I bulette sono noti per il pessimo carattere, e attaccano creature molto più grandi di loro senza alcuna paura. Bestie solitarie tranne per le occasionali coppie in fase riproduttiva, passano la maggior parte del tempo pattugliando i loro territori, che possono superare i 4 km2, cacciando e punendo gli intrusi con una furia in grado di scuotere i pendii delle colline.\\
I bulette sono macchine perfette per divorare e distruggere ossa, armature e anche oggetti magici con le loro possenti mascelle e l'acido ribollente del loro stomaco. In mancanza d'altro, un bulette potrebbe sgranocchiare oggetti comuni, ma per qualche ragione non mangia volontariamente carne di elfo, segno forse di un coinvolgimento della magia elfica nella loro creazione, o di nani, anche se può far strage dei membri di entrambe le razze. Gli Halfling, invece, sono tra i cibi preferiti di queste bestie, e non ci sono Halfling assennati che si avventurino nel territorio di un bulette a cuor leggero.\\
Il bulette è un combattente astuto, e sorprende i nemici con agilità impressionante. Una delle sue tattiche preferite è lanciarsi alla carica e balzare sulla preda attaccando con i suoi artigli affilati come rasoi. Si dice che la carne dietro la cresta dorsale della bestia sia particolarmente tenera, e che quanti vogliano o riescano ad attendere che la pinna venga sollevata nella concitazione del combattimento o dell'accoppiamento possano tentare di sferrare un colpo mortale in quel punto, anche se la maggior parte di quelli che hanno affrontato uno squalo terrestre concordano sul fatto che il miglior modo per vincere un combattimento con un bulette sia evitarlo del tutto.\\


\medskip\index{Mostri - Centauro}\textbf{Centauro}

\emph{Grande mostruosità, neutrale buono}

\textbf{FORZA} +4

\textbf{DESTREZZA} +2

\textbf{COSTITUZIONE} +2

\textbf{INTELLIGENZA} -1

\textbf{SAGGEZZA} +1

\textbf{CARISMA} +0

\textbf{Iniziativa} +2 -- \textbf{Difesa} 13

\textbf{Punti Ferita} 45 (6d10 + 12)

\textbf{Movimento} 15 m

\textbf{Competenze} Acrobatica +6, Consapevolezza +3, Sopravvivenza +3

\textbf{Linguaggi} Elfico, Silvano

\textbf{Sfida} 2 (450 PE)

\emph{\textbf{Carica.}} Se il centauro si muove di almeno 9 metri diretto verso il bersaglio e colpisce con un attacco di picca durante lo stesso turno, il bersaglio subisce 10 (3d6) danni perforanti aggiuntivi.

\textbf{Azioni}

\emph{\textbf{Multiattacco.}} Il centauro effettua due attacchi: uno con
la picca e uno con gli zoccoli o due con l'arco lungo.

\emph{\textbf{Picca.} Attacco con arma da mischia}: +6 a colpire,
portata 3 m, un bersaglio.

\emph{Colpisce:} 9 (1d10 + 4) danni perforanti.

\emph{\textbf{Zoccoli.} Attacco con arma da mischia}: +6 a colpire,
portata 1 m, un bersaglio.

\emph{Colpisce:} 11 (2d6 + 4) danni da botta.

\emph{\textbf{Arco Lungo.} Attacco con arma a Distanza}: +4 a colpire, gittata 45m, un bersaglio.

\emph{Colpisce:} 6 (1d8 + 2) danni perforanti.

\textbf{Ecologia}\\
Ambiente: Pianure e foreste temperate\\
Organizzazione: Solitario, coppia, banda (3-10), tribù (11-30 più 3 cacciatori di 3° livello e 1 capo di 6° livello)\\
Tesoro: Standard (Corazza di Piastre, Scudo Pesante di Metallo, Spada Lunga, Lancia, altro tesoro)\\
\textbf{Descrizione}\\
Leggendari cacciatori e abili guerrieri, i centauri sono in parte uomini e in parte cavalli. Generalmente collocata ai margini della civilizzazione, questa stoica popolazione varia enormemente come aspetto: di solito il colore della pelle è molto abbronzato ma simile a quello degli umani delle regioni limitrofe, mentre la parte inferiore del corpo ha le tonalità degli equini locali. Hanno capelli e occhi di colore scuro e i tratti del volto piuttosto marcati, mentre la loro stazza totale dipende dalla taglia del cavallo di cui hanno la parte inferiore del corpo. Quindi, anche se un centauro medio è alto in piedi 2,1 metri e pesa più di 1.000 kg, esistono molteplici varianti regionali, dagli esili corridori delle pianure ai massicci cacciatori di montagna.\\
I centauri vivono in media circa 60 anni. Distanti dalle altre razze e in conflitto con gli altri della loro specie, i centauri sono una razza antica che lentamente comincia ad accettare il mondo moderno. Anche se la maggioranza dei centauri vive ancora in tribù vagando per vaste pianure o ai margini di mistiche foreste, alcuni hanno abbandonato i modi isolazionisti dei loro antenati per stabilirsi in città cosmopolite. Spesso questi spiriti liberi sono considerati dei reietti e vengono disprezzati dalle loro tribù, e pertanto la decisione di abbandonarle è una scelta pesante. In alcuni casi, comunque, intere tribù guidate da capi progressisti hanno cominciato a commerciare o stringere alleanze con altre comunità di umanoidi, specie Elfi, a volte Gnomi, e più raramente Umani o Nani. Molte razze rimangono caute nei confronti dei centauri, però, per lo più a causa di leggende che li ritraggono come creature territoriali e feroci e dei periodici scontri violenti che essi hanno con i coloni testardi e i paesi in via di espansione.\\


\medskip\index{Mostri - Chimera}\textbf{Chimera}

\emph{Grande mostruosità, caotico malvagio}

\textbf{FORZA} +4

\textbf{DESTREZZA} +0

\textbf{COSTITUZIONE} +4

\textbf{INTELLIGENZA} -4

\textbf{SAGGEZZA} +2

\textbf{CARISMA} +0

\textbf{Iniziativa} +0 -- \textbf{Difesa} 17

\textbf{Punti Ferita} 114 (12d10 + 48)

\textbf{Movimento} 9 m, volo 18 m

\textbf{Competenze} Consapevolezza +8

\textbf{Sensi} scurovisione 18 m

\textbf{Linguaggi} comprende il Draconico ma non può parlare

\textbf{Sfida} 6 (2.300 PE)

\textbf{Azioni}

\emph{\textbf{Multiattacco.}} La chimera effettua tre attacchi: uno con  il morso, uno con le corna e uno con gli artigli. Quando il soffio infuocato  è disponibile, può usare il soffio al posto del morso o delle corna.

\emph{\textbf{Artigli.} Attacco con arma da mischia}: +7 a colpire,  portata 1 m, un bersaglio.

\emph{Colpisce:} 11 (2d6 + 4) danni taglienti.

\emph{\textbf{Corna.} Attacco con arma da mischia}: +7 a colpire,  portata 1 m, un bersaglio.

\emph{Colpisce:} 10 (1d12 + 4) danni da botta.

\emph{\textbf{Morso.} Attacco con arma da mischia}: +7 a colpire,  portata 1 m, un bersaglio.

\emph{Colpisce:} 11 (2d6 + 4) danni perforanti.

\emph{\textbf{Soffio Infuocato (Ricarica 5-6).}} La testa di drago esala  fuoco in un cono di 5 metri. Ogni creatura in quell'area deve  effettuare un Tiro Salvezza di Riflessi CD 15 e subire 31 (7d8) danni  da fuoco se fallisce il Tiro Salvezza, o la metà di questi danni se lo  riesce.

\textbf{Ecologia}\\
Ambiente: Colline Temperate\\
Organizzazione: Solitario, coppia, branco (3-6) o stormo (7-12)\\
Tesoro: Standard\\
\textbf{Descrizione}\\
Le chimere sono mostruose creature nate dal male primordiale. Odiose e fameliche, cacciano sia a terra che in aria. La testa di drago di una chimera può essere di qualunque tipo di drago malvagio, con il soffio corrispondente e le ali generalmente dotate delle stesse scaglie della testa. Le chimere parlano con tre voci che si sovrappongono, ma lo fanno raramente, tipicamente solo per adulare una creatura più potente. Una chimera è alta al garrese 1 metro, raggiungendo la lunghezza di 3 metri e il peso di 350 kg.\\
Le chimere preferiscono la carne, ma possono sopravvivere di vegetali se necessario (anche se quando sono costrette a farlo il loro umore peggiora ulteriormente). Il fatto che volino significa che possono scegliere con attenzione le loro prede, e generalmente cacciano in vaste aree cercando quelle facili. Sono troppo stupide e belligeranti per acquisire seguaci, anche se a volte una tribù di coboldi può far loro delle offerte. Al contrario, sono abbastanza intelligenti e caparbie da essere mediocri animali domestici, e solo una creatura molto più potente di loro può riuscire a sottometterle. Possono formare collaborazioni paritarie con umanoidi rispettosi o creature simili, e acconsentono anche ad essere usate come cavalcature. Un branco di chimere ha una gerarchia simile a quella dei leoni, con un maschio dominante che comanda il gruppo e la maggior parte delle cacce svolte dalle femmine. Una chimera solitaria può essere un giovane maschio solitario o una femmina con i cuccioli nelle vicinanze.\\


\medskip\index{Mostri - Chuul}\textbf{Chuul}

\emph{Grande aberrazione, caotico malvagio}

\textbf{FORZA} +4

\textbf{DESTREZZA} +0

\textbf{COSTITUZIONE} +3

\textbf{INTELLIGENZA} -3

\textbf{SAGGEZZA} +0

\textbf{CARISMA} -3

\textbf{Iniziativa} +0 -- \textbf{Difesa} 18

\textbf{Punti Ferita} 93 (11d10 + 33)

\textbf{Movimento} 9 m, nuoto 9 m

\textbf{Competenze} Consapevolezza +4

\textbf{Immunità ai Danni} veleno

\textbf{Immunità alle Condizioni} avvelenato

\textbf{Sensi} scurovisione 18 m

\textbf{Linguaggi} comprende la Linguaggio delle Profondità ma non può
parlare

\textbf{Sfida} 4 (1.100 PE)

\emph{\textbf{Anfibio.}} Il chuul può respirare aria e acqua.

\emph{\textbf{Senso della Magia.}} Il chuul percepisce la magia entro 36  metri da sé. Questo tratto funziona come l'incantesimo  \emph{individuazione} \emph{del magico} ma di per sé non è magico.  

\textbf{Azioni}

\emph{\textbf{Multiattacco.}} Il chuul effettua due attacchi con le  chele. Se il chuul sta afferrando una creatura, può anche usare i suoi tentacoli una volta.

\emph{\textbf{Chele.} Attacco con arma da mischia}: +6 a colpire,  portata 3 m, un bersaglio.

\emph{Colpisce:} 11 (2d6 + 4) danni da botta. Un bersaglio è  afferrato (CD 14 per fuggire) se è di taglia Grande o inferiore e il  chuul non sta già afferrando altre due creature.

\emph{\textbf{Tentacoli.}} Una creatura afferrata dal chuul deve  riuscire un Tiro Salvezza di Tempra CD 13 o restare avvelenata per  1 minuto. Fino al termine dell'avvelenamento, il bersaglio è  paralizzato. Il bersaglio può ripetere il Tiro Salvezza al termine di  ciascun suo turno, terminando l'effetto per sé in caso di successo.

\textbf{Ecologia}\\
Ambiente: Paludi Temperate\\
Organizzazione: Solitario, coppia o branco (3-6)\\
Tesoro: Standard\\
\textbf{Descrizione}\\
I chuul sono predatori corazzati simili ai crostacei, sempre in agguato sotto la superficie degli stagni e dei pantani poco profondi, che escono dal loro nascondiglio per afferrare le loro prede con le loro chele e poi paralizzarle con i tentacoli della bocca prima di mangiarle vive.\\
I chuul sono eccellenti nuotatori, ma preferiscono attaccare le creature terrestri o abituate ad acque poco profonde. Una volta afferrate le loro vittime, i chuul spesso le trascinano nell'acqua profonda. I lucertoloidi sono le prede preferite dei chuul, anche se le pallide specie di chuul che vivono nei sotterranei preferiscono morlock, duergar, incauti drow e altri sfortunati che si avvicinano troppo ai loro corsi d'acqua sotterranei, ad eccezione dei trogloditi il cui sapore i chuul trovano particolarmente disgustoso.\\
I chuul sono sorprendentemente intelligenti e molti si impegnano in inutili speculazioni sulle loro origini e motivazioni. Parlano un cinguettante e gorgogliante dialetto del Comune, ma pochi di essi sono inclini a chiacchierare con quanti non siano della loro razza, e se esiste una società chuul al di fuori del frenetico periodo degli amori, nessuno lo ha ancora scoperto. Al contrario, le menti dei chuul sembrano dedite solo alla ricerca del luogo perfetto in cui tendere un'imboscata per attaccare altre creature intelligenti e a come decorare le loro elaborate tane con trofei delle loro vittime. Anche se i chuul sembrano non interessati all'utilizzo di utensili, hanno un bisogno compulsivo di collezionare quelli delle loro vittime.\\
Un tipico chuul è alto 2,4 metri e pesa 325 kg.\\


\medskip\index{Mostri - Coboldo}\textbf{Coboldo}

\emph{Piccola umanoide (coboldo), legale malvagio}

\textbf{FORZA} -2

\textbf{DESTREZZA} +2

\textbf{COSTITUZIONE} -1

\textbf{INTELLIGENZA} -1

\textbf{SAGGEZZA} -2

\textbf{CARISMA} -1

\textbf{Iniziativa} +2 -- \textbf{Difesa} 13

\textbf{Punti Ferita} 5 (2d6 - 2)

\textbf{Movimento} 9 m

\textbf{Sensi} scurovisione 18 m

\textbf{Linguaggi} Comune, Draconico

\textbf{Sfida} 1/8 (25 PE)

\emph{\textbf{Sensibilità alla Luce}}. Mentre è alla luce del sole, il coboldo ha -1d6 ai tiri per colpire, oltre che alle prove di Saggezza (Consapevolezza) basate sulla vista.

\emph{\textbf{Tattiche di Branco.}} Il coboldo ha +1d6 ai tiri per colpire contro una creatura se almeno uno degli alleati del coboldo si trova entro 1 metro dalla creatura e quell'alleato non è inabile.

\textbf{Azioni}

\emph{\textbf{Pugnale.} Attacco con arma da mischia}: +4 a colpire,
portata 1 m, un bersaglio.

\emph{Colpisce:} 4 (1d4 + 2) danni perforanti.

\emph{\textbf{Fionda.} Attacco con arma a distanza}: +4 a colpire, gittata 9m, un bersaglio.

\emph{Colpisce:} 4 (1d4 + 2) danni da botta.

\textbf{Ecologia}\\
Ambiente: Foreste temperate o sotterranei\\
Organizzazione: solitario, gruppo (2-4), nido (5-30 più un ugual numero di non combattenti, 1 sergente di 3° livello ogni 20 adulti e 1 capo di 4°-6° livello) o tribù (31-300 più di 35\% di non combattenti, 1 sergente di 3° livello ogni 20 adulti, 2 tenenti di 4° livello, 1 capo di 6°-8° livello e 5-16 Ratti Crudeli)\\
Tesoro: Equipaggiamento da PNG (Armatura di Cuoio, Lancia, Fionda, altro tesoro)\\
\textbf{Descrizione}\\
I coboldi sono creature dell'oscurità, che si incontrano più facilmente in enormi dedali sotterranei o negli angoli bui delle foreste dove il sole non batte mai. A causa della somiglianza fisica, i coboldi si proclamano a gran voce eredi della stirpe draconica e destinati a governare la terra sotto l'ala dei loro grandi cugini divini, ma la maggior parte dei draghi li considera poco più che insetti fastidiosi. Ma, anche se proclamano discendenze divine e l'evidenza del loro destino, i coboldi sono consapevoli della loro debolezza. Codardi ed intriganti, non lottano mai apertamente se possono evitarlo, tendendo invece imboscate e trappole, rintanandosi nei loro dedali dietro una coltre di rozzi ma ingegnosi trabocchetti, o rovesciandosi sul nemico in vaste orde ululanti.\\
La tonalità dei coboldi varia anche tra i fratelli della stessa covata, spaziando tra i colori dei draghi cromatici, con una predominanza del rosso, e più di rado bianco, verde, blu e nero.\\


\medskip\index{Mostri - Cockatrice}\textbf{Cockatrice}

\emph{Piccola mostruosità, disallineato}

\textbf{FORZA} -2

\textbf{DESTREZZA} +1

\textbf{COSTITUZIONE} +1

\textbf{INTELLIGENZA} -4

\textbf{SAGGEZZA} +1

\textbf{CARISMA} -3

\textbf{Iniziativa} +1 -- \textbf{Difesa} 12

\textbf{Punti Ferita} 27 (6d6 + 6)

\textbf{Movimento} 6 m, volo 12 m

\textbf{Sensi} scurovisione 18 m

\textbf{Linguaggi} -

\textbf{Sfida} 1/2 (100 PE)

\textbf{Azioni}

\emph{\textbf{Morso.} Attacco con arma da mischia}: +3 a colpire, portata 1 m, una creatura.

\emph{Colpisce:} 3 (1d4 + 1) danni perforanti, e il bersaglio deve riuscire un Tiro Salvezza di Tempra CD 11 per non essere magicamente pietrificato. Se fallisce il Tiro Salvezza, la creatura inizia a trasformarsi in pietra ed è intralciata. Al termine del turno successivo deve ripetere il Tiro Salvezza. Se lo riesce, l'effetto ha termine. Se lo fallisce, la creatura è pietrificata per 24 ore.

\textbf{Ecologia}\\
Ambiente: Pianure temperate\\
Organizzazione: Solitario, coppia, squadriglia (3-5) o stormo (6-12)\\
Tesoro: Nessuno\\
\textbf{Descrizione}\\
Stupide, malevole e repellenti, le cockatrici sono evitate dalle altre creature per la loro capacità di trasformare la carne in pietra. Le leggende affermano che la prima cockatrice emerse da un uovo deposto da un gallo e covato da un rospo. Che questa storia sia vera o no, le cockatrici odierne si riproducono fra loro in tane terrificanti e sporche scavate a casaccio da almeno una dozzina di creature chioccianti. I maschi sono molto più numerosi delle femmine in questi stormi, e si distinguono solo per barbigli e creste. Una tipica cockatrice è alta poco più di 60 centimetri e pesa 2,5 kg.\\
Anche se la loro dieta consiste principalmente di semi e insetti pietrificati (che nello stomaco della creatura fungono sia da gastroliti che da nutrimento), le cockatrici difendono ferocemente il loro territorio da tutto ciò che ritengono una minaccia, e i vagabondaggi dei maschi raminghi in cerca di nuovi luoghi dove costruire tane a volte li portano ad involontari contatti con gli umani, con risultati devastanti.\\
La strana capacità della cockatrice di trasformare le altre creature in pietra è la sua miglior difesa, e la sua tana è invariabilmente piena di resti dei nemici pietrificati. Per ironia della sorte, tuttavia, donnole e furetti, le creature che più probabilmente finiscono nei nidi delle cockatrici per mangiarne le uova, sembrano completamente immuni a questo effetto. Per ragioni sconosciute, le cockatrici sono sia terrorizzate che furiose con i galli comuni, e c'è la stessa probabilità che fuggano o attacchino quando avviene un confronto.\\


\medskip\index{Mostri - Couatl}\textbf{Couatl}

\emph{Media celestiale, legale buono}

\textbf{FORZA} +3

\textbf{DESTREZZA} +5

\textbf{COSTITUZIONE} +3

\textbf{INTELLIGENZA} +4

\textbf{SAGGEZZA} +5

\textbf{CARISMA} +4

\textbf{Iniziativa} +5 -- \textbf{Difesa} 21

\textbf{Punti Ferita} 97 (13d8 + 39)

\textbf{Movimento} 9 m, volo 9 m

\textbf{Tiri Salvezza} Tempra +9, Riflessi +13, Volontà +14

\textbf{Resistenze al Danno} da Luce

\textbf{Immunità al Danno} psichico; da botta, perforante e tagliente di attacchi non magici

\textbf{Sensi} visione del vero 36 m

\textbf{Linguaggi} tutte, telepatia 36 m 

\textbf{Sfida} 4 (1.100 PE)

\emph{\textbf{Armi Magiche.}} Gli attacchi con armi del couatl sono magici.

\emph{\textbf{Incantesimi Innati.}} La caratteristica da incantatore innato del couatl è il Carisma. Il couatl può lanciare questi incantesimi in maniera innata, usando solo componenti verbali:

A volontà: \emph{individuazione del bene e del male, individuazione del magico, individuazione dei pensieri}

3/giorno ciascuno: \emph{benedizione, creare cibo e acqua, cura ferite,} \emph{protezione dai veleni, ristorare inferiore, santuario, scudo} 1/giorno ciascuno: \emph{ristorare superiore, scrutare, sogno}

\emph{\textbf{Mente Protetta.}} Il couatl è immune allo scrutare e qualsiasi effetto che percepisca le sue emozioni, legga i suoi pensieri o individui la sua posizione.

\textbf{Azioni}

\emph{\textbf{Morso.} Attacco con arma da mischia}: +8 a colpire, portata 1 m, una creatura.

\emph{Colpisce:} 8 (1d6 + 5) danni perforanti, e il bersaglio deve riuscire un Tiro Salvezza di Tempra CD 13 o restare avvelenato per 24 ore. Fino al termine dell'avvelenamento, il bersaglio è privo di sensi. Un'altra creatura può effettuare un'azione per risvegliare il bersaglio.

\emph{\textbf{Stritolare.} Attacco con arma da mischia}: +6 a colpire, portata 3 m, una creatura di taglia Media o inferiore.

\emph{Colpisce:} 10 (2d6 + 3) danni da botta, e il bersaglio è afferrato (CD 15 per fuggire). Fino al termine dell'afferrare, il bersaglio è intralciato, e il couatl non può stritolare un altro bersaglio.

\emph{\textbf{Mutare Forma.}} Il couatl può trasformarsi magicamente in un umanoide o bestia il cui grado di sfida sia pari o inferiore al proprio, o tornare alla sua vera forma. Alla morte ritorna alla sua vera forma. Qualsiasi equipaggiamento stia indossando o trasportando viene assorbito o trasportato nella nuova forma (a scelta del couatl).

Nella nuova forma, il couatl mantiene le sue statistiche di gioco e la facoltà di parlare, ma la sua Difesa, metodi di movimento, Forza, Destrezza e altre azioni vengono rimpiazzati da quelli della nuova forma, e ottiene qualsiasi statistica o capacità (Azioni aggiuntive e azioni da tana) possedute dalla sua nuova forma e non dalla sua originale. Se la nuova forma ha un attacco di morso, il couatl può usare il proprio morso nella nuova forma.

\textbf{Ecologia}\\
Ambiente: foreste calde\\
Organizzazione: Solitario, coppia o stormo (3-6)\\
Tesoro: Standard\\
\textbf{Descrizione}\\
I couatl sono servitori di divinità legali buone, anche se alcuni operano in maniera indipendente da qualsiasi entità superiore. Rispettati ed ammirati per la loro saggezza e bellezza, cercano di portare i mortali sulla retta via e usano i loro poteri per combattere il male, specie quelli noti per viaggiare tra i piani. Alcuni couatl sono visti come divinità benevole da società isolate e, anche se i couatl rabbrividiscono al solo pensiero di fingere di essere una divinità, consentono che si perpetuino questi malintesi poiché permettono loro di guidare queste società su sentieri di pace e cooperazione con i loro vicini. Un couatl è lungo circa 3,6 metri, con un'apertura alare di circa 5 metri e pesa 900 kg.\\
Come esterni nativi, i couatl devono mangiare. Preferiscono gli stessi alimenti dei veri serpenti, come mammiferi e uccelli, anche se è noto che divorano gli umanoidi malvagi. Poiché preferiscono passare il tempo a perseguire i loro intenti anziché cacciare, apprezzano le offerte di cibo, in particolare piccoli cinghiali e volatili. Un couatl talvolta mostra il suo apprezzamento a un avventuriero o a un gruppo che gli ha reso un servizio donandogli 1d4 delle sue brillanti piume colorate. Queste piume ottenute gratuitamente, se usate come componente materiale aggiuntivo, permettono ad un incantatore che lancia Alleato Planare di evocare quello specifico couatl senza pagare il normale costo in oro o altri valori, a condizione che il couatl approvi il servizio richiesto dall'incantatore.\\


\medskip\index{Mostri - Cumulo Strisciante}\textbf{Cumulo Strisciante}

\emph{Grande pianta, disallineato}

\textbf{FORZA} +4

\textbf{DESTREZZA} -1

\textbf{COSTITUZIONE} +3

\textbf{INTELLIGENZA} -3

\textbf{SAGGEZZA} +0

\textbf{CARISMA} -3

\textbf{Iniziativa} -1 -- \textbf{Difesa} 18

\textbf{Punti Ferita} 136 (16d10 + 48)

\textbf{Movimento} 6 m, nuoto 6 m

\textbf{Competenze} Muoversi Silenziosamente / Nascondersi +2

\textbf{Resistenze al Danno} freddo, fuoco

\textbf{Immunità al Danno} fulmine

\textbf{Immunità alle Condizioni} accecato, assordato, affaticamento

\textbf{Sensi} vista cieca 18 m (cieco oltre questo raggio)

\textbf{Linguaggi} -

\textbf{Sfida} 5 (1.800 PE)

\emph{\textbf{Assorbimento dei Fulmini.}} Ogni qual volta il cumulo strisciante subisce danni da fulmine, non subisce danni e recupera un numero di punti ferita pari al danno da fulmine inferto.

\textbf{Azioni}

\emph{\textbf{Multiattacco.}} Il cumulo strisciante effettua due attacchi di schianto. Se entrambi gli attacchi colpiscono una creatura di taglia Media o inferiore, il bersaglio è afferrato (CD 14 per fuggire) e il cumulo strisciante usa Avvolgere su di esso.

\emph{\textbf{Schianto.} Attacco con arma da mischia}: +7 a colpire, portata 1 m, un bersaglio.

\emph{Colpisce:} 13 (2d8 + 4) danni da botta.

\emph{\textbf{Avvolgere.}} Il cumulo strisciante avvolge una creatura di taglia Media o inferiore che ha afferrato. Il bersaglio avvolto è accecato, intralciato e impossibilitato a respirare, e deve riuscire un Tiro Salvezza di Tempra CD 14 all'inizio di ciascun turno del tumulo o subire 13 (2d8 + 4) danni da botta. Se il cumulo si muove, il bersaglio avvolto si muove con esso. Il cumulo può avvolgere solo una creatura alla volta.

\textbf{Ecologia}\\
Ambiente: Foreste o Paludi Temperate\\
Organizzazione: Solitario\\
Tesoro: Standard\\
\textbf{Descrizione}\\
I cumuli striscianti, chiamati anche soltanto striscianti, sembrano masse vegetali in decomposizione. Sono piante carnivore intelligenti, con un debole per la carne elfica. Il cervello e gli organi sensoriali si trovano nella parte superiore del corpo. Di solito i cumuli striscianti hanno una circonferenza di 2,4 metri e sono alti da 1,8 a 2,7 metri. Pesano circa 1.900 kg.\\
I cumuli striscianti sono strane creature, più simili a un groviglio di rampicanti parassiti che ad una singola pianta dotata di radici. Sono onnivori, capaci di trarre sostentamento da qualsiasi cosa, avvinghiandosi agli alberi per succhiarne la linfa, inserendo le radici nel terreno per assorbire nutrienti semplici o consumando la carne e le ossa dalle prede.\\
I cumuli striscianti sono incredibilmente furtivi nel loro ambiente naturale. Si confondono con il terreno circostante e possono attendere immobili per giorni l'arrivo di una potenziale preda. Possono essere praticamente ovunque ed attaccare in qualsiasi momento senza alcun preavviso e senza curarsi che ci siano o meno sopravvissuti, fintanto che hanno da mangiare.\\
Di solito i cumuli striscianti conducono un'esistenza nomade e solitaria in profonde foreste e fetide paludi ma possono essere trovati anche sottoterra, in mezzo a boschetti di funghi. Voci preoccupanti parlano di gruppi di cumuli striscianti che si radunano intorno a grandi tumuli nelle profondità di giungle e paludi, spesso durante violente tempeste di fulmini. Il motivo di questo comportamento è sconosciuto, e molti saggi si chiedono se dietro non ci sia uno scopo oscuro ed alieno.\\


\subsection{Demoni}

\medskip\index{Mostri - Balor}\textbf{Balor}

\emph{Enorme immondo (demone), caotico malvagio}

\textbf{FORZA} +8

\textbf{DESTREZZA} +2

\textbf{COSTITUZIONE} +6

\textbf{INTELLIGENZA} +5

\textbf{SAGGEZZA} +3

\textbf{CARISMA} +6

\textbf{Iniziativa} +5 -- \textbf{Difesa} 29

\textbf{Punti Ferita} 262 (21d12 + 126)

\textbf{Movimento} 12 m, volo 24 m

\textbf{Tiri Salvezza} Tempra +29, Riflessi +17, Volontà +25

\textbf{Resistenze al Danno} freddo, fulmine; 

\textbf{Immunità al Danno} fuoco, veleno , armi +1

textbf{Immunità alle Condizioni} avvelenato

\textbf{Vulnerabilità al Danno} ferro freddo

\textbf{Sensi} visione del vero 36 m

\textbf{Linguaggi} Abissale, telepatia 36 m

\textbf{Sfida} 19 (22.000 PE)

\emph{\textbf{Armi Magiche.}} Gli attacchi con arma del demone sono magici.

\emph{\textbf{Aura di Fuoco.}} All'inizio di ciascun turno del demone, ciascuna creatura entro 1 metro da lui subisce 10 (3d6) danni da fuoco, e gli oggetti infiammabili che si trovano nell'aura e che non sono indossati o trasportati prendono fuoco. Una creatura che entri a contatto con il demone o lo colpisca con un attacco da mischia mentre si trova entro 1 metro da esso subisce 10 (3d6) danni da fuoco.

\emph{\textbf{Resistenza alla Magia.}} Il demone ha +1d6 ai Tiri Salvezza contro incantesimi e altri effetti magici.

\emph{\textbf{Spasmo Mortale.}} Quando il demone muore, esplode; ciascuna creatura entro 9 metri da esso deve effettuare un Tiro Salvezza di Riflessi CD 20, subendo 70 (20d6) danni da fuoco se fallisce il Tiro Salvezza, o la metà di questi danni se lo riesce. L'esplosione appicca il fuoco agli oggetti infiammabili che non sono indossati o trasportati, e distrugge le armi del demone.

\textbf{Azioni}

\emph{\textbf{Multiattacco.}} Il demone effettua due attacchi: uno con la spada lunga e uno con la frusta.

\emph{\textbf{Frusta.} Attacco con arma da mischia}: +14 a colpire, portata 9 m, un bersaglio.

\emph{Colpisce:} 15 (2d6 + 8) danni taglienti più 10 (3d6) danni da fuoco, e il bersaglio deve riuscire un Tiro Salvezza di Tempra CD 20 o venire trascinato 7 metri verso il demone.

\emph{\textbf{Spada Lunga.} Attacco con arma da mischia}: +14 a colpire, portata 3 m, un bersaglio.

\emph{Colpisce:} 21 (3d8 + 8) danni taglienti più 13 (3d8) danni da fulmine. Se il demone ottiene un colpo critico, tira il danno tre volte, invece che due.

\emph{\textbf{Teletrasporto.}} Il demone si teletrasporta magicamente, insieme a tutto l'equipaggiamento che indossa o trasporta, in uno spazio non occupato e che può vedere entro 36 metri.

\textbf{Ecologia}\\
Ambiente: Qualsiasi (Abisso)\\
Organizzazione: Solitario o banda di guerra (1 Balor e 2-5 Glabrezu)\\
Tesoro: Standard (Spada Lunga Sacrilega+1, Frusta Infuocata+1, altro tesoro)\\
\textbf{Descrizione}\\
Quando la gente sussurra terrificanti racconti di creature demoniache, immagina per lo più un'imponente figura di fuoco e carne, un incubo cornuto armato di frusta e spada fiammeggianti, che vola nella notte in cerca delle sue prede. Il demone che queste persone temono è il balor, e questa paura è pienamente giustificata, dal momento che pochi demoni possono eguagliare il possente balor in forza o in brutalità.\\
Nell'Abisso, i balor sono per lo più al servizio dei signori dei demoni, in qualità di generali o capitani (quando non si tratti di balor estremamente potenti, noti come signori dei balor). Un balor solitamente comanda vaste legioni di demoni e, sebbene spesso consenta a questi servi bramosi e sbavanti di combattere le sue battaglie, è tutt'altro che un codardo. Se si presenta l'opportunità di unirsi ad uno scontro, sono pochi i balor che scelgono di trattenersi.\\
Un balor è alto 4,2 metri e pesa 2.250 kg. Solo le anime mortali più crudeli possono alimentare la creazione di un balor: a differenza degli altri demoni, spesso occorrono numerose anime di potenti malvagi per far nascere un nuovo balor.\\


\medskip\index{Mostri - Dretch}\textbf{Dretch}

\emph{Piccola immondo (demone), caotico malvagio}

\textbf{FORZA} +0

\textbf{DESTREZZA} +0

\textbf{COSTITUZIONE} +1

\textbf{INTELLIGENZA} -3

\textbf{SAGGEZZA} -1

\textbf{CARISMA} -4

\textbf{Iniziativa} +0 -- \textbf{Difesa} 12

\textbf{Punti Ferita} 18 (4d6 + 4)

\textbf{Movimento} 6 m

\textbf{Resistenze al Danno} freddo, fulmine, fuoco

\textbf{Immunità al Danno} veleno

\textbf{Immunità alle Condizioni} avvelenato

\textbf{Vulnerabilità al Danno} ferro freddo

\textbf{Sensi} scurovisione 18 m

\textbf{Linguaggi} Abissale, telepatia 18 m (funziona solo con le creature che comprendono l'Abissale)

\textbf{Sfida} 1/4 (50 PE)

\textbf{Azioni}

\emph{\textbf{Multiattacco.}} Il demone effettua due attacchi: uno con il morso e uno con gli artigli.

\emph{\textbf{Artigli.} Attacco con arma da mischia}: +2 a colpire, portata 1 m, un bersaglio.

\emph{Colpisce:} 5 (2d4) danni taglienti.

\emph{\textbf{Morso.} Attacco con arma da mischia}: +2 a colpire, portata 1 m, un bersaglio.

\emph{Colpisce:} 3 (1d6) danni perforanti.

\emph{\textbf{Nube Fetida (1/Giorno).}} Un disgustoso gas verde si estende in un raggio di 3 metri dal demone. Il gas si propaga intorno agli angoli, e la sua area è oscurata leggermente. Rimane per 1 minuto o finché non viene disperso da un forte vento. Qualsiasi creatura che inizi il proprio turno in quell'area deve riuscire un Tiro Salvezza di Tempra CD 11 o restare avvelenata fino all'inizio del suo prossimo turno. Mentre è avvelenato in questo modo, il bersaglio, durante il suo turno, può effettuare solo un'azione o un'azione bonus, ma non entrambe, e non può effettuare reazioni.

\textbf{Ecologia}\\
Ambiente: Qualsiasi (Abisso)\\
Organizzazione: Solitario, coppia, banda (3-5), gruppo (6-12) o folla (13+)\\
Tesoro: Nessuno\\
\textbf{Descrizione}\\
Anche il più infimo demone dell'Abisso è pericoloso e possiede la necessità impellente di spargere rovina e sgomento. Il miserabile dretch è tanto orripilante e fetido quanto crudele, anche se non possiede la forza ed il potere per riuscire a soddisfare la sua voglia di brutalizzare gli altri nel suo reame nativo. Lo scopo dell'esistenza dei dretch è quello di servire demoni più potenti come vittime sacrificabili, e solo pochi fortunati riescono a sopravvivere abbastanza a lungo da evolversi.\\
I dretch sono i bersagli preferiti dai dilettanti in evocazioni abissali. Relativamente deboli e facili da intimorire, i dretch spesso possono essere obbligati a lunghi periodi di servitù utilizzando vaghe promesse di opportunità di sfogare le loro frustrazioni e la loro rabbia contro avversari più deboli. Eppure il potenziale evocatore di dretch farebbe meglio a ricordarsi che questi demoni sono codardi ed infidi quanto gli altri demoni. Un dretch che si trova di fronte un nemico più potente sarà assai lieto di scambiare qualsiasi informazione di cui disponga in cambio della sua miserevole vita.\\
A differenza della maggior parte dei demoni, la sciatta personalità del dretch ed il suo disprezzo per il lavoro fisico prolungato raramente danno dei risultati. I dretch avanzati sono rari, ma quelli che riescono a trovare la forza in se stessi per diventare più di quello che erano al momento della loro creazione divengono i sovrani poveri dell'Abisso, crudeli ed amareggiati, che regnano su parassiti, anime spezzate, non morti privi di intelletto e altri dretch. I loro imperi sono limitati a tratti abbandonati di fogne sotto città dimenticate, instabili distese paludose evitate dalle menti più sensate ed altri sgraditi angoli dell'Abisso che persino i demoni considerano scomodi o ripugnanti. Eppure per i signori dei dretch questi regni sono i loro imperi, e li difendono con pietosa tenacia.\\
Un dretch è alto 1,2 metri e pesa 90 kg. I dretch solitamente si formano dalle anime di mortali malvagi ed indolenti: è sufficiente solo un piccolo frammento di anima per dare origine ad una nascita così orripilante. Una sola anima spesso può causare l'apparizione di una piccola armata di dretch, e la vista di un'orda di dretch appena nati che si liberano dalla protomateria pulsante dell'Abisso è al contempo nauseante e terrificante.\\


\medskip\index{Mostri - Glabrezu}\textbf{Glabrezu}

\emph{Grande immondo (demone), caotico malvagio}

\textbf{FORZA} +5

\textbf{DESTREZZA} +2

\textbf{COSTITUZIONE} +5

\textbf{INTELLIGENZA} +4

\textbf{SAGGEZZA} +3

\textbf{CARISMA} +3

\textbf{Iniziativa} +4 -- \textbf{Difesa} 22

\textbf{Punti Ferita} 157 (15d10 + 75)

\textbf{Movimento} 12 m

\textbf{Tiri Salvezza} Tempra +18, Riflessi +4, Volontà +11

\textbf{Resistenze al Danno} freddo, fulmine, fuoco; da botta, perforante e tagliente di attacchi non magici

\textbf{Immunità al Danno} veleno

\textbf{Immunità alle Condizioni} avvelenato

\textbf{Vulnerabilità al Danno} ferro freddo

\textbf{Sensi} visione del vero 36 m

\textbf{Linguaggi} Abissale, telepatia 36 m 

\textbf{Sfida} 9 (5.000 PE)

\emph{\textbf{Incantesimi Innati.}} La caratteristica da incantatore del demone è l'Intelligenza. Il demone può lanciare questi incantesimi in maniera innata, senza bisogno di componenti materiali:

A volontà: \emph{dissolvi magie, individuazione del magico, oscurità}

1/giorno ciascuno: \emph{confusione, parola del potere stordire, volare}

\emph{\textbf{Resistenza alla Magia.}} Il demone ha +1d6 ai Tiri Salvezza contro incantesimi e altri effetti magici.

\textbf{Azioni}

\emph{\textbf{Multiattacco.}} Il demone effettua quattro attacchi: due con le chele e due con i pugni. In alternativa, può effettuare due attacchi con le chele e lanciare un incantesimo.

\emph{\textbf{Chela.} Attacco con arma da mischia}: +9 a colpire, portata 3 m, un bersaglio.

\emph{Colpisce:} 16 (2d10 + 5) danni da botta. Se il bersaglio è una creatura di taglia Media o inferiore, è afferrato (CD 15 per fuggire). Il glabrezu possiede due chele, ciascuna delle quali può afferrare un bersaglio.

\emph{\textbf{Pugno.} Attacco in mischia con arma}: +9 a colpire, portata 1 m, un bersaglio.

\emph{Colpisce:} 7 (2d4 + 2) danni da botta.

\textbf{Ecologia}\\
Ambiente: Qualsiasi (Abisso)\\
Organizzazione: Solitario o drappello (1 glabrezu, 1 Succube e 2-5 Vrock)\\
Tesoro: Standard\\
\textbf{Descrizione}\\
Mentre la Succube è un demone che adesca la sua preda sfruttandone i desideri e le necessità carnali, il glabrezu è un tentatore di altro genere. Feroce e dalla forma bestiale, il glabrezu è in realtà un maestro di inganni e bugie. Con la sua abilità di nascondere la sua vera forma dietro piacenti illusioni, usa la sua magia per esaudire i desideri degli umanoidi mortali, come forma di ricompensa per coloro che soccombono ai suoi inganni e raggiri. Un desiderio esaudito da un glabrezu appaga la necessità di chi lo esprime nel modo più rovinoso possibile, sebbene queste conseguenze possano non rivelarsi immediatamente tali. Un fabbro che fatica ad affermarsi potrebbe desiderare fama ed abilità nella professione scelta, solo per scoprire che il suo miglior patrono è un crudele e sadico omicida che usa le armi per promuovere i propri distruttivi desideri. Un uomo solo che esprime il desiderio di avere una compagna, potrebbe vedere il suo desiderio avverarsi con una sua vecchia fiamma ritornata alla "vita" in forma di vampiro, ed altri esempi di questo tipo. Il glabrezu è assai creativo nel soddisfare i desideri di un mortale.\\
Un glabrezu è alto 5,4 metri e pesa poco più di 3.000 kg. Questi perfidi demoni si originano dalle anime dei traditori, dei falsi e dei sovversivi: anime di mortali che, in vita, giurarono il falso o utilizzarono il tradimento e l'inganno per rovinare le vite altrui.\\


\medskip\index{Mostri - Hezrou}\textbf{Hezrou}

\emph{Grande immondo (demone), caotico malvagio}

\textbf{FORZA} +4

\textbf{DESTREZZA} +3

\textbf{COSTITUZIONE} +5

\textbf{INTELLIGENZA} 5 (-2)

\textbf{SAGGEZZA} +1

\textbf{CARISMA} +1

\textbf{Iniziativa} +3 -- \textbf{Difesa} 20

\textbf{Punti Ferita} 136 (13d10 + 65)

\textbf{Movimento} 9 m

\textbf{Tiri Salvezza} Tempra +16, Riflessi +3, Volontà +9

\textbf{Resistenze al Danno} freddo, fulmine, fuoco; da botta, perforante e tagliente di attacchi non magici

\textbf{Immunità al Danno} veleno

\textbf{Immunità alle Condizioni} avvelenato

\textbf{Vulnerabilità al Danno} ferro freddo

\textbf{Sensi} scurovisione 36 m

\textbf{Linguaggi} Abissale, telepatia 36 m 

\textbf{Sfida} 8 (3.900 PE)

\emph{\textbf{Fetore.}} Qualsiasi creatura che inizi il suo turno entro 3 metri dal demone, deve riuscire un Tiro Salvezza di Tempra CD 14 o restare avvelenata fino all'inizio del proprio turno. Se riesce il Tiro Salvezza, la creatura è immune al fetore del demone gracidante per 24 ore.

\emph{\textbf{Resistenza alla Magia.}} Il demone ha +1d6 ai Tiri Salvezza contro incantesimi e altri effetti magici.

\textbf{Azioni}

\emph{\textbf{Multiattacco.}} Il demone effettua tre attacchi: uno con il morso e due con gli artigli.

\emph{\textbf{Artiglio.} Attacco con arma da mischia}: +7 a colpire, portata 1 m, un bersaglio.

\emph{Colpisce:} 11 (2d6 + 4) danni taglienti.

\emph{\textbf{Morso.} Attacco con arma da mischia}: +7 a colpire, portata 1 m, un bersaglio.

\emph{Colpisce:} 15 (2d10 + 4) danni perforanti.

\textbf{Ecologia}\\
Ambiente: Qualsiasi (Abisso)\\
Organizzazione: Solitario o banda (2-4)\\
Tesoro: Standard\\
\textbf{Descrizione}\\
l'hezrou vive nelle vaste paludi, acquitrini e corsi d'acqua dell'Abisso, a suo agio sia nell'acqua che sulla terraferma. La presenza di un hezrou ha un effetto dannoso su flora, causando nodosità e mutazioni, e acque circostanti, rendendole maleodoranti e dal sapore salmastro, peculiarità più facilmente individuabili nel Piano Materiale che nell'Abisso. L'esposizione prolungata a questa corruzione causa orrende trasformazioni e deformità. Spesso intere comunità isolate di mutanti deformi devono il loro aspetto contorto non tanto ai loro depravati costumi quanto alla vicinanza di un hezrou.\\
Sebbene sia abbastanza intelligente, si può onestamente dire che un hezrou sprechi il proprio intelletto. Questi esseri preferiscono i piaceri più semplici: dormire, il gusto della tortura, la beatitudine di cibarsi di carne vivente o la gioia di sentire qualcosa di bello rompersi e sbriciolarsi nella stretta dei loro pugni. Non cercano spesso di costruire imperi o porsi alla testa di culti, sebbene pochi hezrou rifiuterebbero potenziali seguaci che vengano ad offrirglisi di loro spontanea volontà.\\
Queste mostruose e bestiali creature nascono dalle anime di mortali malvagi che hanno avvelenato se stessi, i loro parenti o il loro ambiente, ad esempio, drogati, assassini ed alchimisti che non si sono preoccupati di come i loro esperimenti avvelenassero il mondo naturale.\\


\medskip\index{Mostri - Marilith}\textbf{Marilith}

\emph{Grande immondo (demone), caotico malvagio}

\textbf{FORZA} +4

\textbf{DESTREZZA} +5

\textbf{COSTITUZIONE} +5

\textbf{INTELLIGENZA} +4

\textbf{SAGGEZZA} +3

\textbf{CARISMA} +5

\textbf{Iniziativa} +5 -- \textbf{Difesa} 26

\textbf{Punti Ferita} 189 (18d10 + 90)

\textbf{Movimento} 12 m

\textbf{Tiri Salvezza} Tempra +25, Riflessi +18, Volontà +13

\textbf{Resistenze al Danno} freddo, fulmine, fuoco

\textbf{Immunità al Danno} veleno, armi +1

\textbf{Immunità alle Condizioni} avvelenato

\textbf{Vulnerabilità al Danno} ferro freddo

\textbf{Sensi} visione del vero 36 m

\textbf{Linguaggi} Abissale, telepatia 36 m 

\textbf{Sfida} 16 (15.000 PE)

\emph{\textbf{Armi Magiche.}} Gli attacchi con armi del demone sono magici.

\emph{\textbf{Reattivo.}} Il demone può effettuare una reazione durante ciascun turno di combattimento.

\emph{\textbf{Resistenza alla Magia.}} Il demone ha +1d6 ai Tiri Salvezza contro incantesimi e altri effetti magici.

\textbf{Azioni}

\emph{\textbf{Multiattacco.}} Il demone effettua sette attacchi: sei con le spade lunghe e uno con la coda.

\emph{\textbf{Coda.} Attacco con arma da mischia}: +9 a colpire, portata 3 m, una creatura.

\emph{Colpisce:} 15 (2d10 + 4) danni da botta. Se il bersaglio è di taglia Media o inferiore, è afferrato (CD 19 per fuggire). Fino al termine dell'afferrare, il bersaglio è intralciato, e il demone può colpire automaticamente il bersaglio con la coda, ma non può effettuare attacchi di coda contro altri bersagli.

\emph{\textbf{Spada Lunga.} Attacco con arma da mischia}: +9 a colpire, portata 1 m, un bersaglio.

\emph{Colpisce:} 13 (2d8 + 4) danni taglienti.

\textbf{Reazioni}

\emph{\textbf{Parata.}} Il demone somma 5 alla sua Difesa contro un attacco da mischia che lo colpirebbe. Per farlo, il demone deve poter vedere il suo attaccante e impugnare un'arma da mischia.

\textbf{Ecologia}\\
Ambiente: Qualsiasi (Abisso)\\
Organizzazione: Solitario, coppia o plotone (1 marilith, 1-3 Glabrezu e 3-14 Babau)\\
Tesoro: Doppio (6 Spade Lunghe, altro tesoro)\\
\textbf{Descrizione}\\
Sovrane di orde demoniache e regine di nazioni abissali, le temibili marilith servono i signori dei demoni come governanti, consigliere e persino amanti, eppure la loro supremazia come strateghe le rende particolarmente richieste come generali e comandanti d'armate. Le marilith più potenti non sono al servizio di nessuno e comandano invece fameliche legioni demoniache.\\
Una marilith è alta da 1,8 a 2,7 metri, lunga 6 metri dalla testa alla punta della coda, e pesa 2.000 kg. Solo le anime malvagie più arroganti ed orgogliose, solitamente quelle di crudeli sovrani, sadici generali e signori della guerra particolarmente violenti, possono causare la nascita di una marilith.\\


\medskip\index{Mostri - Nalfeshnee}\textbf{Nalfeshnee}

\emph{Grande immondo (demone), caotico malvagio}

\textbf{FORZA} +5

\textbf{DESTREZZA} +0

\textbf{COSTITUZIONE} +6

\textbf{INTELLIGENZA} +4

\textbf{SAGGEZZA} +1

\textbf{CARISMA} +2

\textbf{Iniziativa} +4 -- \textbf{Difesa} 25

\textbf{Punti Ferita} 184 (16d10 + 96)

\textbf{Movimento} 6 m, volo 9 m

\textbf{Tiri Salvezza} Tempra +22, Riflessi +9, Volontà +21

\textbf{Resistenze al Danno} freddo, fulmine, fuoco; da botta, perforante e tagliente di attacchi non magici

\textbf{Immunità al Danno} veleno

\textbf{Immunità alle Condizioni} avvelenato

\textbf{Vulnerabilità al Danno} ferro freddo

\textbf{Sensi} scurovisione 36 m

\textbf{Linguaggi} Abissale, telepatia 36 m 

\textbf{Sfida} 13 (10.000 PE)

\emph{\textbf{Resistenza alla Magia.}} Il demone ha +1d6 ai Tiri Salvezza contro incantesimi e altri effetti magici.

\textbf{Azioni}

\emph{\textbf{Multiattacco.}} Il demone usa, se possibile, Aureola di Orrore. Poi effettua tre attacchi: uno con il morso e due con gli artigli.

\emph{\textbf{Artiglio.} Attacco con arma da mischia}: +10 a colpire, portata 3 m, un bersaglio.

\emph{Colpisce:} 15 (3d6 + 5) danni taglienti.

\emph{\textbf{Morso.} Attacco con arma da mischia}: +10 a colpire, portata 1 m, un bersaglio.

\emph{Colpisce:} 32 (5d10 + 5) danni perforanti.

\emph{\textbf{Aureola di Orrore (Ricarica 5-6).}} Il demone emette una luce magica multicolore e scintillante. Ogni creatura entro 5 metri dal demone e che possa vedere la luce, deve riuscire un Tiro Salvezza su Volontà CD 15 o restare spaventata per 1 minuto. Una creatura può ripetere il Tiro Salvezza al termine di ciascun suo turno, terminando l'effetto per sé se lo riesce. Se il Tiro Salvezza della creatura riesce o l'effetto ha termine per essa, la creatura è immune all'Aureola di
Orrore del demone gemente per le successive 24 ore.

\emph{\textbf{Teletrasporto.}} Il demone si teletrasporta, insieme a tutto l'equipaggiamento che sta indossando o trasportando, in uno spazio non occupato che possa vedere fino a 36 metri di distanza.

\textbf{Ecologia}
Ambiente: Qualsiasi (Abisso)\\
Organizzazione: Solitario o banda di guerra (1 nalfeshnee, 1 Hezrou e 2-5 Vrock)\\
Tesoro: Standard\\
\textbf{Descrizione}\\
Sono pochi i demoni che comprendono le meccaniche interne che regolano l'Abisso come i nalfeshnee, e non è raro che questi demoni servano l'Abisso stesso invece che un signore dei demoni. Alcuni sovrintendono i reami organici che generano i nuovi demoni, mentre altri custodiscono luoghi di particolare importanza nei recessi nascosti del piano. Spesso il regno di un nalfeshnee nell'Abisso è superiore per forze e dimensioni al più grande dei regni mortali, in quanto questi demoni hanno una predisposizione naturale a governare ed imporre una sorta di ordine al caos dell'Abisso. Gli evocatori mortali spesso li richiamano per il loro folle ma impareggiabile intelletto, esaminando accuratamente gli accordi presi con questi demoni onde evitare eventuali conseguenze nascoste e risvolti non voluti, in quanto un nalfeshnee raramente accetta qualcosa che, in qualche modo contorto, non gli consenta di soddisfare le necessità ed i desideri dell'Abisso.\\
I nalfeshnee sono alti 6 metri e pesano 4.000 kg. Sono creati dalle anime di malvagi mortali avari o bramosi, in particolare di coloro che hanno regnato su imperi di schiavitù, furto, brigantaggio e altri vizi ancora più violenti.\\


\medskip\index{Mostri - Quasit}\textbf{Quasit}

\emph{Minuscola immondo (demone, mutaforma), caotico malvagio}

\textbf{FORZA} -3

\textbf{DESTREZZA} +3

\textbf{COSTITUZIONE} +0

\textbf{INTELLIGENZA} -2

\textbf{SAGGEZZA} +0

\textbf{CARISMA} +0

\textbf{Iniziativa} +3 -- \textbf{Difesa} 14

\textbf{Punti Ferita} 7 (3d4)

\textbf{Movimento} 12 m (3 m, volo 12 m in forma di pipistrello; 12 m, scalata 12 m in forma di centopiedi; 12 m, nuoto 12 m in forma di rospo)

\textbf{Competenze} Muoversi Silenziosamente / Nascondersi +5

\textbf{Resistenze al Danno} freddo, fulmine, fuoco; da botta, perforante e tagliente di attacchi non magici

\textbf{Immunità al Danno} veleno 

\textbf{Immunità alle Condizioni}
avvelenato

\textbf{Sensi} scurovisione 36 m

\textbf{Linguaggi} Abissale, Comune

\textbf{Sfida} 1 (200 PE)

\emph{\textbf{Mutaforma.}} Il demone può usare la sua azione per trasformarsi in una forma bestiale da pipistrello, centopiedi o rospo, o per tornare alla sua vera forma. Le sue statistiche sono le stesse in tutte le forme, sebbene gli attacchi possano variare per alcune di esse. Qualsiasi equipaggiamento stia indossando o trasportando non viene trasformato. Alla morte ritorna alla sua vera forma.

\emph{\textbf{Resistenza alla Magia.}} Il demone ha +1d6 ai Tiri Salvezza contro incantesimi e altri effetti magici.

\textbf{Azioni}

\emph{\textbf{Artigli (Morso in Forma di Bestia).} Attacco con arma da  mischia}: +4 a colpire, portata 1 m, un bersaglio. \emph{Colpisce:} 5 (1d4 + 3) danni perforanti. Se il bersaglio è una creatura, deve riuscire un Tiro Salvezza di Tempra CD 10 o subire 5 (2d4) danni da veleno e restare avvelenato per 1 minuto. La creatura può ripetere il Tiro Salvezza al termine di ciascun suo turno, ponendo termine all'effetto se lo riesce.

\emph{\textbf{Invisibilità.}} Il demone resta invisibile finché non attacca o termina la sua concentrazione. Qualsiasi cosa che il demone stia trasportando o indossando resta invisibile finché rimane in contatto con il demone.

\emph{\textbf{Spavento (1/Giorno).}} Una creatura scelta dal demone che si trovi entro 6 metri da lui, deve riuscire un Tiro Salvezza su Volontà CD 10 o restare spaventata per 1 minuto. Il bersaglio può ripetere il Tiro Salvezza al termine di ciascun suo turno, con -1d6 se il demone è in linea di visuale, ponendo termine all'effetto prematuramente se riesce il Tiro Salvezza.

\textbf{Ecologia}\\
Ambiente: Qualsiasi (Abisso)\\
Organizzazione: Solitario o stormo (2-12)\\
Tesoro: Standard\\
\textbf{Descrizione}\\
Il quasit è forse il demone meno potente, ma non è tra i meno rispettati: persino i quasit si ritengono superiori alle orde di Dretch e, fedeli alla propria natura, i Dretch mancano del coraggio o degli stimoli necessari a dimostrare loro che si sbagliano. Il ruolo primario in vita di un quasit è quello di famiglio al servizio di un incantatore, ma quei quasit che sfuggono a questa umiliante servitù acquisiscono una volontà propria e sono molto più pericolosi. Un quasit tipico è alto 45 centimetri e pesa solo 4 kg.\\
Unici tra le orde demoniache, i quasit non nascono dalle anime di malvagi mortali deceduti, ma da anime viventi: quando un incantatore cerca di richiamare a sé un quasit come famiglio, la sua anima sfiora l'Abisso ed esso reagisce, creando dalla sua materia un quasit collegato all'anima dell'incantatore e generando un potente legame tra i due.\\
I quasit appena creati vengono alla luce direttamente nel Piano Materiale, dove diventano famigli e, finché sono soggetti alla volontà del loro padrone, lo odiano e disprezzano, dal momento che possono percepire il pulsare delle sua anima e sanno che potrebbero aspirare a qualcosa di più. Un quasit serve, eppure osserva e vigila nell'attesa di errori che possano costare la vita al suo signore, o meglio, che gli consentano di rivoltarsi contro il proprio padrone. Quando il padrone di un quasit muore, questi può cercare di seguirne l'anima nel Grande Oltre, superando un Tiro Salvezza su Volontà con DC 15. Questo effetto funziona come Spostamento Planare ma influisce solo sul quasit e lo trasporta nell'Abisso, facendo diventare sua l'anima del padrone, in forma di larva, piuttosto che utilizzarla per creare nuove forme di vita demoniache. In questo modo, un quasit può usare l'anima appena catturata per contrattare con abitanti più potenti dei piani inferiori, e magari raggiungere un'abietta "promozione" che lo trasformi in una forma di vita più potente.\\
Raramente un quasit decide di ignorare la morte del proprio padrone e di rimanere nel Piano Materiale in cerca di altri modi per divertirsi: solitamente insediandosi in un'area urbana dove ci sono molti individui da tormentare\\


\medskip\index{Mostri - Vrock}\textbf{Vrock}

\emph{Grande immondo (demone), caotico malvagio}

\textbf{FORZA} +3

\textbf{DESTREZZA} +2

\textbf{COSTITUZIONE} +4

\textbf{INTELLIGENZA} -1

\textbf{SAGGEZZA} +1

\textbf{CARISMA} -1

\textbf{Iniziativa} +2 -- \textbf{Difesa} 18

\textbf{Punti Ferita} 104 (11d10 + 44)

\textbf{Movimento} 12 m, volo 18 m

\textbf{Tiri Salvezza} Tempra +13, Riflessi +10, Volontà +6

\textbf{Resistenze al Danno} freddo, fulmine, fuoco; da botta, perforante e tagliente di attacchi non magici

\textbf{Immunità al Danno} veleno 

\textbf{Immunità alle Condizioni} avvelenato

\textbf{Sensi} scurovisione 36 m

\textbf{Linguaggi} Abissale, telepatia 36 m

\textbf{Sfida} 6 (2.300 PE)

\emph{\textbf{Resistenza alla Magia.}} Il demone ha +1d6 ai Tiri Salvezza contro incantesimi e altri effetti magici.

\textbf{Azioni}

\emph{\textbf{Multiattacco.}} Il demone effettua due attacchi: uno con il becco e uno con gli speroni.

\emph{\textbf{Becco.} Attacco con arma da mischia}: +6 a colpire, portata 1 m, un bersaglio.

\emph{Colpisce:} 10 (2d6 + 3) danni perforanti.

\emph{\textbf{Speroni.} Attacco con arma da mischia}: +6 a colpire, portata 1 m, un bersaglio.

\emph{Colpisce:} 14 (2d10 + 3) danni taglienti.

\emph{\textbf{Spore (Ricarica 6).}} Una nube di spore tossiche si diffonde in un raggio di 5 metri intorno al demone. Le spore si propagano intorno agli angoli. Ogni creatura in quell'area deve riuscire un Tiro Salvezza di Tempra CD 14 o restare avvelenata. Mentre   avvelenato in questo modo, un bersaglio subisce 5 (1d10) danni da   veleno all'inizio di ciascun suo turno. Il bersaglio può ripetere il   Tiro Salvezza al termine di ciascun suo turno, ponendo termine   all'effetto se lo riesce. Anche svuotare una fiala di acqua sacra sul   bersaglio pone termine all'effetto.

\emph{\textbf{Strillo Stordente (1/Giorno).}} Il demone emette uno strillo orripilante. Ogni creatura entro 6 metri da esso e che lo possa udire, e non sia un demone, deve riuscire un Tiro Salvezza su Tempra CD 14 o restare stordita fino al termine del prossimo turno del demone.

\textbf{Ecologia}\\
Ambiente: Qualsiasi (Abisso)\\
Organizzazione: Solitario, coppia o banda (3-10)\\
Tesoro: Standard\\
\textbf{Descrizione}\\
Profani campioni dell'Abisso, i vrock incarnano tutta la rabbia, l'odio e la violenza di questo reame. Tanto voraci e grottescamente opportunisti quanto il saprofago a cui assomigliano, i vrock si deliziano nello spargimento di sangue, godendo del suono e delle sensazioni derivanti dallo strappare gli intestini ancora pulsanti da una creatura vivente.\\
Un vrock tipico è alto 2,4 metri e pesa 200 kg. Queste creature solitamente si originano dalle anime di malvagi mortali colmi di odio e di collera, in particolare coloro che erano criminali professionisti, mercenari o assassini.\\



\medskip\index{Mostri - Destriero da Incubo}\textbf{Destriero da Incubo}

\emph{Grande immondo, neutrale malvagio}

\textbf{FORZA} +4

\textbf{DESTREZZA} +2

\textbf{COSTITUZIONE} +3

\textbf{INTELLIGENZA} +0

\textbf{SAGGEZZA} +1

\textbf{CARISMA} +2

\textbf{Iniziativa} +2 -- \textbf{Difesa} 15

\textbf{Punti Ferita} 68 (8d10 + 24)

\textbf{Movimento} 18 m, volo 24 m

\textbf{Immunità al Danno} fuoco

\textbf{Linguaggi} comprende Abissale, Comune e Infernale ma non può parlare

\textbf{Sfida} 3 (700 PE)

\emph{\textbf{Conferire Resistenza al Fuoco.}} Il destriero da incubo può conferire resistenza al danno da fuoco a chiunque lo cavalchi.

\emph{\textbf{Illuminazione.}} Il destriero da incubo irradia luce intensa in un raggio di 3 metri e luce fioca per ulteriori 3 metri.

\textbf{Azioni}

\emph{\textbf{Zoccoli.} Attacco con arma da mischia}: +6 a colpire, portata 1 m, un bersaglio.

\emph{Colpisce:} 13 (2d8 + 4) danni da botta più 7 (2d6) danni da fuoco.

\emph{\textbf{Passo Etereo.}} Il destriero da incubo e fino a tre creature consenzienti entro 1 metro da esso possono entrare magicamente nel Piano Etereo dal Piano Materiale e viceversa.

\textbf{Ecologia}\\
Ambiente: Qualsiasi (Abaddon)\\
Organizzazione: Solitario\\
Tesoro: Nessuno\\
\textbf{Descrizione}\\
Gli incubi sono fiammeggianti messaggeri di morte. Permettono solo alle creature più malvagie di cavalcarli, e non sono mai soltanto cavalcature, ma correi nella distruzione provocata dai loro cavalieri.\\



\subsection{Diavoli}

\medskip\index{Mostri - Diavolo Barbuto}\textbf{Diavolo Barbuto}

\emph{Media immondo (diavolo), legale malvagio}

\textbf{FORZA} +3

\textbf{DESTREZZA} +2

\textbf{COSTITUZIONE} +2

\textbf{INTELLIGENZA} -1

\textbf{SAGGEZZA} +0

\textbf{CARISMA} +0

\textbf{Iniziativa} +2 -- \textbf{Difesa} 15

\textbf{Punti Ferita} 52 (8d8 + 16)

\textbf{Movimento} 9 m

\textbf{Tiri Salvezza} Tempra +9, Riflessi +7, Volontà +3

\textbf{Resistenze al Danno} freddo; da botta, perforante e tagliente di attacchi non magici o che non siano argentati

\textbf{Immunità al Danno} fuoco, veleno

\textbf{Immunità alle Condizioni} avvelenato

\textbf{Sensi} scurovisione 36 m

\textbf{Linguaggi} Infernale, telepatia 36 m

\textbf{Sfida} 3 (700 PE)

\emph{\textbf{Resistenza alla Magia.}} Il diavolo ha +1d6 ai Tiri Salvezza contro incantesimi e altri effetti magici.

\emph{\textbf{Risoluto.}} Il diavolo non può essere spaventato finché riesce a vedere una creatura alleata entro 9 metri da lui.

\emph{\textbf{Vista del Diavolo.}} La scurovisione del diavolo non è limitata dall'oscurità magica.

\textbf{Azioni}

\emph{\textbf{Multiattacco.}} Il diavolo effettua due attacchi: uno con la barba e uno con il falcione.

\emph{\textbf{Barba.} Attacco con arma da mischia}: +5 a colpire, portata 1 m, una creatura.

\emph{Colpisce:} 6 (1d8 + 2) danni perforanti, e il bersaglio deve riuscire un Tiro Salvezza di Tempra CD 12 o restare avvelenato per 1 minuto. Mentre è avvelenato in questo modo, il bersaglio non può recuperare punti ferita. Il bersaglio può ripetere il Tiro Salvezza al termine di ciascun suo turno, terminando l'effetto se riesce il Tiro Salvezza.

\emph{\textbf{Falcione.} Attacco con arma da mischia}: +5 a colpire, portata 3 m, un bersaglio.

\emph{Colpisce:} 8 (1d10 + 3) danni taglienti. Se il bersaglio è una creatura, ad esclusione di costrutti e non morti, deve riuscire un Tiro Salvezza su Tempra 12 o perdere 5 (1d10) punti ferita all'inizio di ciascun suo turno a causa della ferita infernale. Ogni volta che il diavolo colpisce il bersaglio ferito con questo attacco, il danno inflitto dalla ferita aumenta di 5 (1d10). Qualsiasi creatura può effettuare un'azione per bloccare la ferita con una prova riuscita di Saggezza (Pronto Soccorso) CD 12. La ferita si richiude anche nel caso in cui il bersaglio riceva della magia guaritrice.

\textbf{Ecologia}\\
Ambiente: Qualsiasi (Inferno)\\
Organizzazione: Solitario, coppia, squadra (3-10) o truppa (10-40)\\
Tesoro: Standard (Falcione, altro tesoro)\\
\textbf{Descrizione}\\
Guerrieri scelti delle legioni infernali, i diavoli barbuti, o barbazu, combattono selvaggiamente in nome dei loro signori infernali e in battaglia comandano orde brutali di dannati. Si radunano e si addestrano con i loro falcioni forgiati negli inferi, tra le volte del terzo girone dell'Inferno, Erebo, ma ritornano inevitabilmente nel primo girone, Averno, per servire al fianco del temibile signore Barbatos.\\
I barbazu amano effettuare attacchi di carica con i loro falcioni e cercano di mantenere una distanza di 3 metri tra loro ed i loro avversari, così che possono utilizzare le loro caratteristiche armi ad asta con la massima efficacia. Contro un avversario che ha una portata superiore (oppure è in grado di evitare la tattica preferita del diavolo), gettano i falcioni e si affidano ai loro artigli ed alle orribili barbe. In posizione eretta i diavoli barbuti sono alti più di 1,8 metri (sebbene la posizione accovacciata che tengono in battaglia li faccia spesso sembrare più bassi) e pesano più di 100 kg.\\


\medskip\index{Mostri - Diavolo delle Catene}\textbf{Diavolo delle Catene}

\emph{Media immondo (diavolo), legale malvagio}

\textbf{FORZA} +4

\textbf{DESTREZZA} +2

\textbf{COSTITUZIONE} +4

\textbf{INTELLIGENZA} +0

\textbf{SAGGEZZA} +1

\textbf{CARISMA} +2

\textbf{Iniziativa} +2 -- \textbf{Difesa} 20

\textbf{Punti Ferita} 85 (10d8 + 40)

\textbf{Movimento} 9 m

\textbf{Tiri Salvezza} Tempra +9, Riflessi +4, Volontà +3

\textbf{Resistenze al Danno} freddo; da botta, perforante e tagliente di attacchi non magici che non siano argentati

\textbf{Immunità al Danno} fuoco, veleno 

\textbf{Immunità alle Condizioni} avvelenato

\textbf{Sensi} scurovisione 36 m

\textbf{Linguaggi} Infernale, telepatia 36 m 

\textbf{Sfida} 8 (3.900 PE)

\emph{\textbf{Resistenza alla Magia.}} Il diavolo ha +1d6 ai Tiri Salvezza contro incantesimi e altri effetti magici.

\emph{\textbf{Vista del Diavolo.}} La scurovisione del diavolo non è limitata dall'oscurità magica.

\textbf{Azioni}

\emph{\textbf{Multiattacco.}} Il diavolo effettua due attacchi con la catena.

\emph{\textbf{Catena.} Attacco con arma da mischia}: +8 a colpire, portata 3 m, un bersaglio.

\emph{Colpisce:} 11 (2d6 + 4) danni taglienti. Il bersaglio è afferrato (CD 14 per fuggire) se il diavolo non sta già afferrando un'altra creatura. Fino al termine dell'afferrare, il bersaglio è intralciato e subisce 7 (2d6) danni perforanti all'inizio di ciascun suo turno.

\emph{\textbf{Animare Catene (Ricarica dopo un 1 ora).}} Fino a quattro catene che il diavolo possa vedere e si trovano entro 18 metri da lui producono dei bordi affilati e si animano sotto il controllo del diavolo, purché quelle catene non siano né indossate né trasportate da qualcun altro.

Ogni catena animata è un oggetto con Difesa 20, 20 punti ferita, resistenza ai danni perforanti, e immunità ai danni psichici e da tuono. Quando il diavolo usa Multiattacco durante il suo turno, può usare ciascuna catena animata per effettuare un ulteriore attacco di catena. Una catena animata può afferrare una creatura per conto proprio ma non può effettuare attacchi mentre afferra. Una catena animata ritorna al suo stato inanimato se viene ridotta a 0 punti ferita o se il diavolo è reso inabile o muore.

\textbf{Reazioni}

\emph{\textbf{Maschera Snervante.}} Quando una creatura che il diavolo può vedere inizia il proprio turno entro 9 metri dal diavolo, il diavolo può creare un'illusione per assomigliare all'amore perduto o un acerrimo rivale di quella creatura. Se la creatura può vedere il diavolo, deve riuscire un Tiro Salvezza di Volontà CD 14 o rimanere spaventata fino al termine del suo turno.

\textbf{Ecologia}\\
Ambiente: Qualsiasi\\
Organizzazione: Solitario, coppia, anello (3-6) o catena (7-20)\\
Tesoro: Standard\\
\textbf{Descrizione}
Spesso classificati dai profani tra le fila dei diavoli infernali, i sadomasochistici non sono veri diavoli. Anche se alcuni sono noti per vivere all'Inferno, essi esistono al di fuori delle gerarchie stabilite dagli dei degli inferi e dai suoi arcidiavoli e a volte si possono trovare su altri piani, in particolare sul Piano delle Ombre. Molti suggeriscono che siano nativi dell'Inferno che esisteva prima dell'avvento della stirpe diabolica, anche se altri ipotizzano che siano stati portati sul piano da qualche sadica potenza. Indipendentemente dalle loro origini vagano per i piani assecondano il loro desiderio di causare e ricevere sofferenza, ricercando il dolore attraverso violenti rapimenti e sadiche depravazioni.\\


\medskip\index{Mostri - Diavolo Cornuto}\textbf{Diavolo Cornuto}

\emph{Grande immondo (diavolo), legale malvagio}

\textbf{FORZA} +6

\textbf{DESTREZZA} +3

\textbf{COSTITUZIONE} +5

\textbf{INTELLIGENZA} +1

\textbf{SAGGEZZA} +3

\textbf{CARISMA} +3

\textbf{Iniziativa} +3 -- \textbf{Difesa} 23

\textbf{Punti Ferita} 178 (17d10 + 85)

\textbf{Movimento} 6 m, volo 18 m

\textbf{Tiri Salvezza} Tempra +18, Riflessi +17, Volontà +13

\textbf{Resistenze al Danno} freddo; da botta, perforante e tagliente di attacchi che non siano argentati

\textbf{Immunità al Danno} fuoco, veleno, armi +1

\textbf{Immunità alle Condizioni} avvelenato

\textbf{Sensi} scurovisione 36 m

\textbf{Linguaggi} Infernale, telepatia 36 m 

\textbf{Sfida} 11 (7.200 PE)

\emph{\textbf{Resistenza alla Magia.}} Il diavolo ha +1d6 ai Tiri Salvezza contro incantesimi e altri effetti magici.

\emph{\textbf{Vista del Diavolo.}} La scurovisione del diavolo non è limitata dall'oscurità magica.

\textbf{Azioni}

\emph{\textbf{Multiattacco.}} Il diavolo effettua tre attacchi da mischia: due con il forcone e uno con la coda. Può usare Scagliare Fiamma al posto di qualsiasi attacco da mischia.

\emph{\textbf{Coda.} Attacco con arma da mischia}: +10 a colpire, portata 3 m, un bersaglio.

\emph{Colpisce:} 10 (1d8 + 6) danni perforanti. Se il bersaglio è una creatura, ad esclusione di costrutti e non morti, deve riuscire un Tiro Salvezza su Tempra 17 o perdere 10 (3d6) punti ferita all'inizio di ciascun suo turno a causa della ferita infernale. Ogni volta che il diavolo ferisce il bersaglio con questo attacco, il danno inflitto dalla ferita aumenta di 10 (3d6). Qualsiasi creatura può effettuare un'azione per bloccare la ferita riuscendo una prova di Saggezza (Pronto Soccorso) CD 12. La ferita si richiude anche nel caso in cui il bersaglio riceva magia guaritrice.

\emph{\textbf{Forcone.} Attacco con arma da mischia}: +10 a colpire, portata 3 m, un bersaglio.

\emph{Colpisce:} 15 (2d8 + 6) danni perforanti.

\emph{\textbf{Pungiglione.} Attacco con arma da mischia}: +8 a colpire, portata 3 m, un bersaglio.

\emph{Colpisce:} 13 (2d8 + 4) danni perforanti più 17 (5d6) danni da veleno, e il bersaglio deve riuscire un Tiro Salvezza di Tempra CD 14, o restare avvelenato per 1 minuto. Il bersaglio può ripetere il Tiro Salvezza al termine di ciascun suo turno, terminando l'effetto se lo
riesce.

\emph{\textbf{Scagliare Fiamma.} Attacco con incantesimo a Distanza}: +7 a colpire, gittata 45 m, un bersaglio.

\emph{Colpisce:} 14 (4d6) danni da fuoco. Se il bersaglio è un oggetto infiammabile che non sia indossato o trasportato, prende fuoco.

\textbf{Ecologia}\\
Ambiente: Qualsiasi (Inferno)\\
Organizzazione: Solitario, coppia o stormo (3-10)\\
Tesoro: Standard (Catena Chiodata Sacrilega+1, altro tesoro)\\
\textbf{Descrizione}\\
Tra i più letali guerrieri degli arcidiavoli ed abili comandanti dei diavoli minori, i diavoli cornuti divulgano le regole dell'Inferno dovunque passano. Questi diavoli maggiori sono addestrati, forgiati e riforgiati per essere tra i più implacabili ed obbedienti guerrieri del multiverso. I diavoli cornuti delle truppe degli eserciti infernali sono noti come cornugon, mentre i più grandi tra loro sono chiamati malebranche.\\
Un diavolo cornuto tipico raggiunge la ragguardevole altezza di 2,7 metri, è dotato di ali con un'apertura di 4,2 metri, e pesa 350 kg.\\


\medskip\index{Mostri - Diavolo della Fossa}\textbf{Diavolo della Fossa}

\emph{Grande immondo (diavolo), legale malvagio}

\textbf{FORZA} +8

\textbf{DESTREZZA} +2

\textbf{COSTITUZIONE} +7

\textbf{INTELLIGENZA} +6

\textbf{SAGGEZZA} +4

\textbf{CARISMA} +7

\textbf{Iniziativa} +6 -- \textbf{Difesa} 29

\textbf{Punti Ferita} 300 (24d10 + 168) 

\textbf{Movimento} 9 m, volo 18 m

\textbf{Tiri Salvezza} Tempra +24, Riflessi +21, Volontà +18

\textbf{Resistenze al Danno} freddo; da botta, perforante e tagliente che non siano argentati

\textbf{Immunità al Danno} fuoco, veleno , armi +2

\textbf{Immunità alle Condizioni} avvelenato

\textbf{Sensi} visione del vero 36 m

\textbf{Linguaggi} Infernale, telepatia 36 m

\textbf{Sfida} 20 (25.000 PE)

\emph{\textbf{Arma Magica.}} Gli attacchi con arma del diavolo della fossa sono magici.

\emph{\textbf{Aura di Paura.}} Qualsiasi creatura ostile al diavolo che inizi il suo turno entro 6 metri da esso, deve effettuare un Tiro Salvezza su Volontà CD 21, a meno che il diavolo non sia inabile. Se fallisce il Tiro Salvezza, la creatura è spaventata fino all'inizio del suo prossimo turno. Se il Tiro Salvezza della creatura riesce, la creatura è immune all'Aura di Paura del diavolo per le successive 24 ore.

\emph{\textbf{Incantesimi Innati.}} La caratteristica da incantatore  diavolo della fossa è il Carisma. Il diavolo della fossa può lanciare questi incantesimi in maniera innata, senza bisogno di componenti materiali:

A volontà: \emph{individuazione del magico, palla di fuoco}

3/giorno ciascuno: \emph{blocca mostri, muro di fuoco}

\emph{\textbf{Resistenza alla Magia.}} Il diavolo ha +1d6 ai Tiri Salvezza contro incantesimi e altri effetti magici.

\textbf{Azioni}

\emph{\textbf{Multiattacco.}} Il diavolo effettua quattro attacchi: uno con il morso, uno con l'artiglio, uno con la mazza e uno con la coda.

\emph{\textbf{Artiglio.} Attacco con arma da mischia}: +14 a colpire, portata 3 m, un bersaglio.

\emph{Colpisce:} 17 (2d8 + 8) danni taglienti.

\emph{\textbf{Coda.} Attacco con arma da mischia}: +14 a colpire, portata 3 m, un bersaglio.

\emph{Colpisce:} 24 (3d10 + 8) danni da botta.

\emph{\textbf{Mazza.} Attacco con arma da mischia}: +14 a colpire, portata 3 m, un bersaglio.

\emph{Colpisce:} 15 (2d6 + 8) danni da botta più 21 (6d6) danni da fuoco.

\emph{\textbf{Morso.} Attacco con arma da mischia}: +14 a colpire, portata 1 m, un bersaglio.

\emph{Colpisce:} 22 (4d6 + 8) danni perforanti. Il bersaglio deve riuscire un Tiro Salvezza di Tempra CD 21 o restare avvelenato. Mentre è avvelenato in questo modo, il bersaglio non può recuperare punti ferita, e subisce 21 (6d6) danni da veleno all'inizio di ciascun suo turno. Il bersaglio avvelenato può ripetere il Tiro Salvezza al termine di ciascun suo turno, terminando l'effetto su di sé.

\textbf{Ecologia}\\
Ambiente: Qualsiasi (Inferno)\\
Organizzazione: Solitario, coppia o concilio (3-9)\\
Tesoro: Doppio\\
\textbf{Descrizione}
Sovrani di reami infernali, generali delle armate dell'Inferno e consiglieri degli arcidiavoli, i diavoli della fossa sono la personificazione del terribile e spaventoso apice della razza diabolica.\\
Massicci, dal fisico indomito e dotati di ingegnosi intelletti malvagi, questi diabolici tiranni possiedono grande autonomia sia al servizio degli arcidiavoli che nella loro sovranità su distese infernali di schiavi o quando sono impegnati a soggiogare i mondi mortali. Solidi muscoli si tendono sui loro giganteschi corpi, corazzati da spesse placche taglienti capaci di bloccare quasi tutti gli attacchi. Le fauci dotate di zanne grandi come pugnali ed i loro volti bestiali nascondono alcune tra le menti più insidiose dell'Inferno.\\
Nati nelle profondità di Nessus, il nono e più profondo girone dell'Inferno, i diavoli della fossa vengono creati dai ranghi dei cornugon e dei gelugon solamente dagli arcidiavoli e dai loro duchi. Sebbene molti viaggino fino ai gironi superiori e oltre l'Inferno, al comando delle legioni infernali, la maggior parte rimane nel Nessus, al servizio delle corti dei potenti dell'Inferno o in oscure congreghe dagli innominabili propositi.\\

I diavoli della fossa sono sempre alti più di 4,2 metri, con una apertura alare di oltre 6 metri ed un peso superiore ai 500 kg.\\

I diavoli della fossa sono signori del fuoco e prediligono i territori lambiti dalle fiamme. All'Inferno, questa loro predisposizione fa sì che Averno, Dite, Malebolge, Nessus, e Flegetonte siano i gironi che più facilmente ospitano i loro templi-cittadelle avvolti dalle fiamme. Fanatici ossessionati dalla superiorità diabolica e dalla più ferrea obbedienza, i diavoli della fossa, se lasciati agire indisturbati, radunano immensi eserciti, rastrellando le fosse dell'Inferno alla ricerca dei lemure più depravati per trasformarli in veri diavoli. Una volta certi di aver creato le legioni perfette, volgono la loro attenzione ai semipiani ed ai mondi mortali più vulnerabili, pregustandone la conquista.\\

Servitori degli arcidiavoli o di altri unici signori della guerra infernali, i diavoli della fossa si votano alla loro causa, obbedendo alla volontà dei nobili scelti da Asmodeus nella speranza che, un giorno, riescano ad ottenere il favore del Principe dell'Oscurità o dell'Inferno stesso. Pur obbedienti alle gerarchie della propria razza, sono anche severi nel farne rispettare le regole e, se un diavolo della fossa si trovasse a servire un padrone indegno, si riterrebbe in dovere di deporlo. Pertanto, siano essi signori o servitori, i diavoli della fossa incarnano la volontà delle implacabili leggi dell'inferno e si assicurano che solo i diavoli più potenti possano (o osino) prosperare.\\

Solo i più potenti tra gli incantatori mortali possono od osano evocare un diavolo della fossa. Le reazioni di questo tipo di diavoli all'evocazione sono rapide e premeditate, solitamente caratterizzate da una furia incontenibile all'idea che un essere così insignificante possa sprecare il loro tempo immortale. Chi non riesce a fronteggiarne la bruciante rabbia viene ucciso e la sua anima dannata all'Inferno e posta al servizio del diavolo evocato. Chi riesce a controllare questi diavoli maggiori riesce anche ad intrigarli.\\

Un diavolo della fossa può servire rispettosamente un signore mortale per secoli, ma il suo scopo rimane sempre lo stesso: corromperne sempre più l'anima, assicurarsi la sua completa dannazione e, quando questi alla fine muore, rivendicarne l'anima ed iniziare il processo per farne un servitore lemure totalmente corrotto.\\

I diavoli della fossa sono consapevoli di essere immortali e sono abbastanza intelligenti da avere una pazienza incredibilmente disciplinata. Pertanto i diavoli della fossa più antichi vedono nelle loro legioni i volti degli innumerevoli folli che un tempo hanno preteso di ritenersi loro padroni.\\


\medskip\index{Mostri - Diavolo del Ghiaccio}\textbf{Diavolo del Ghiaccio}

\emph{Grande immondo (diavolo), legale malvagio}

\textbf{FORZA} +5

\textbf{DESTREZZA} +2

\textbf{COSTITUZIONE} +4

\textbf{INTELLIGENZA} +4

\textbf{SAGGEZZA} +2

\textbf{CARISMA} +4

\textbf{Iniziativa} +4 -- \textbf{Difesa} 25

\textbf{Punti Ferita} 180 (19d10 + 76)

\textbf{Movimento} 12 m

\textbf{Tiri Salvezza} Tempra +15, Riflessi +14, Volontà +12

\textbf{Resistenze al Danno} da botta, perforante e tagliente di attacchi che non siano argentate

\textbf{Immunità al Danno} freddo, fuoco, veleno , armi +1

\textbf{Immunità alle Condizioni} avvelenato

\textbf{Sensi} vista cieca 18 m, scurovisione 36 m

\textbf{Linguaggi} Infernale, telepatia 36 m

\textbf{Sfida} 14 (11.500 PE)

\emph{\textbf{Resistenza alla Magia.}} Il diavolo ha +1d6 ai Tiri Salvezza contro incantesimi e altri effetti magici.

\emph{\textbf{Vista del Diavolo.}} La scurovisione del diavolo non è limitata dall'oscurità magica.

\textbf{Azioni}

\emph{\textbf{Multiattacco.}} Il diavolo effettua tre attacchi: uno con il morso, uno con gli artigli e uno con la coda. In alternativa effettua due attacchi: uno con la coda e uno con lancia.

\emph{\textbf{Artigli.} Attacco con arma da mischia}: +10 a colpire, portata 1 m, un bersaglio.

\emph{Colpisce:} 10 (2d4 + 5) danni taglienti più 10 (3d6) danni da freddo.

\emph{\textbf{Coda.} Attacco con arma da mischia}: +10 a colpire, portata 3 m, un bersaglio.

\emph{Colpisce:} 12 (2d6 + 5) danni da botta più 10 (3d6) danni da freddo.

\emph{\textbf{Lancia di Ghiaccio.} Attacco con arma da mischia}: +10 a colpire, portata 3 m, un bersaglio.

\emph{Colpisce:} 14 (2d8 + 5) danni perforanti più 10 (3d6) danni da freddo. Se il bersaglio è una creatura, deve riuscire un Tiro Salvezza su Tempra CD 15, o avere per 1 minuto la velocità ridotta di 3 metri; durante ciascun suo turno può effettuare solo un'azione o un'azione bonus, ma non entrambe; non può effettuare reazioni. Il bersaglio può ripetere il Tiro Salvezza al termine di ciascun suo turno, terminando l'effetto su di sé in caso di successo.

\emph{\textbf{Morso.} Attacco con arma da mischia}: +10 a colpire,
portata 1 m, un bersaglio.

\emph{Colpisce:} 12 (2d6 + 5) danni perforanti più 10 (3d6) danni da
freddo.

\emph{\textbf{Muro di Ghiaccio (Ricarica 6).}} Il diavolo forma magicamente un muro di ghiaccio opaco su di una superficie solida che possa vedere entro 18 metri da lui. Il muro è spesso 30 centimetri e largo fino a 9 metri per un massimo di 3 metri di altezza, oppure   una cupola semisferica di massimo 6 metri di diametro. Quando la   parete appare, ogni creatura nel suo spazio viene spinta fuori da esso   tramite la via più breve. La creatura sceglie su quale lato del muro   finire, a meno che la creatura non sia inabile. La creatura poi   effettua un Tiro Salvezza di Riflessi CD 17, subendo 35 (10d6) danni da freddo se lo fallisce, o la metà di questi danni se lo riesce.

Il muro rimane per 1 minuto o finché il diavolo non è reso inabile o muore. Il muro può essere danneggiato e bucato; ogni sezione di 3 metri ha Difesa 5, 30 punti ferita, vulnerabilità al danno da fuoco, e immunità al danno da acido, freddo, da Vuoto, psichico e da veleno. Se una sezione viene distrutta, lascia una patina di aria gelida nello spazio che occupava prima il muro. Ogni volta che una creatura finisce per muoversi attraverso quest'aria gelida durante un turno, consenziente o meno, deve effettuare un Tiro Salvezza di Tempra CD 17, subendo 17 (5d6)danni da freddo se lo  fallisce, o la metà di questi danni se lo riesce. L'aria gelida si dissipa quando il resto del muro svanisce.


\medskip\index{Mostri - Diavolo d'Ossa}\textbf{Diavolo d'Ossa}

\emph{Grande immondo (diavolo), legale malvagio}

\textbf{FORZA} +4

\textbf{DESTREZZA} +3

\textbf{COSTITUZIONE} +4

\textbf{INTELLIGENZA} +1

\textbf{SAGGEZZA} +2

\textbf{CARISMA} +3

\textbf{Iniziativa} +3 -- \textbf{Difesa} 24

\textbf{Punti Ferita} 142 (15d10 + 60)

\textbf{Movimento} 12 m, volo 12 m

\textbf{Tiri Salvezza} Tempra +12, Riflessi +12, Volontà +7

\textbf{Competenze} Ingannare +7, Percepire Emozioni +6

\textbf{Resistenze al Danno} freddo; da botta, perforante e tagliente di attacchi non magici o che non siano argentati

\textbf{Immunità al Danno} fuoco, veleno

\textbf{Immunità alle Condizioni} avvelenato

\textbf{Sensi} scurovisione 36 m

\textbf{Linguaggi} Infernale, telepatia 36 m 

\textbf{Sfida} 9 (5.000 PE)

\emph{\textbf{Resistenza alla Magia.}} Il diavolo ha +1d6 ai Tiri Salvezza contro incantesimi e altri effetti magici.

\emph{\textbf{Vista del Diavolo.}} La scurovisione del diavolo non è limitata dall'oscurità magica.

\textbf{Azioni}

\emph{\textbf{Multiattacco.}} Il diavolo effettua tre attacchi: due con gli artigli e uno con il pungiglione oppure uno con la sua arma inastata uncinata e uno con il pungiglione.

\emph{\textbf{Arma Inastata Uncinata.} Attacco con arma da mischia}: +8 a colpire, portata 3 m, un bersaglio.

\emph{Colpisce:} 17 (2d12 + 4) danni perforanti. Se il bersaglio è una creatura di taglia Enorme o inferiore, è afferrato (CD 14 per fuggire). Fino al termine dell'afferrare, il diavolo non può usare la sua arma inastata su di un altro bersaglio.

\emph{\textbf{Artiglio.} Attacco con arma da mischia}: +8 a colpire, portata 3 m, un bersaglio.

\emph{Colpisce:} 8 (1d8 + 4) danni taglienti.

\emph{\textbf{Pungiglione.} Attacco con arma da mischia}: +8 a colpire, portata 3 m, un bersaglio.

\emph{Colpisce:} 13 (2d8 + 4) danni perforanti più 17 (5d6) danni da veleno, e il bersaglio deve riuscire un Tiro Salvezza di Tempra CD 14, o restare avvelenato per 1 minuto. Il bersaglio può ripetere il Tiro Salvezza al termine di ciascun suo turno, terminando l'effetto se lo riesce.

\textbf{Ecologia}\\
Ambiente: Qualsiasi (Inferno)\\
Organizzazione: Solitario, squadra (2-3), concilio (4-10) o contingente (1-3 diavoli del ghiaccio, 2-6 diavoli cornuti e 1-4 diavoli d'ossa\\
Tesoro: Standard (Lancia Gelida+1, altro tesoro)\\
\textbf{Descrizione}\\
Strateghi illuminati delle armate dell'Inferno, gli insettoidi diavoli del ghiaccio sono tra le menti più ingegnose e crudeli nelle legioni dell'inferno. Noto come gelugon tra le fila dei diavoli, un diavolo del ghiaccio nasconde nel suo petto un cuore ghiacciato trafugato ad un mortale, che gli permette di prendere decisioni libero da emozioni. Nati nel girone ghiacciato di Cocito, il settimo girone infernale, la maggior parte dei diavoli del ghiaccio migra a Caina, l'ottavo girone, dove complotta per dannare il mondo da corti di gelido acciaio. Sebbene abbiano le sembianze più aliene e mostruose tra tutti i diavoli, a pochi altri viene accordato un maggiore rispetto.\\
In combattimento un gelugon manda avanti i suoi sottoposti, così da poter valutare le tattiche, i punti di forza e le debolezze dell'avversario nelle retrovie, e fornire loro supporto con le sue capacità magiche, evitando di coglierli nell'area di effetto dei suoi incantesimi: atteggiamento non dovuto ad un senso di cameratismo, bensì alla fredda e logica verità che i suoi alleati possono sopravvivere più a lungo in uno scontro se non sono esposti a fuoco amico.\\
I gelugon sono alti 3,6 metri e pesano approssimativamente 350 kg.\\


\medskip\index{Mostri - Diavolo Spinoso}\textbf{Diavolo Spinoso}

\emph{Piccola immondo (diavolo), legale malvagio}

\textbf{FORZA} +0

\textbf{DESTREZZA} +2

\textbf{COSTITUZIONE} +1

\textbf{INTELLIGENZA} +0

\textbf{SAGGEZZA} +2

\textbf{CARISMA} -1

\textbf{Iniziativa} +2 -- \textbf{Difesa} 14

\textbf{Punti Ferita} 22 (5d6 + 5)

\textbf{Movimento} 6 m, volo 12 m

\textbf{Resistenze al Danno} freddo; da botta, perforante e tagliente di attacchi non magici o che non siano argentati

\textbf{Immunità al Danno} fuoco, veleno

\textbf{Immunità alle Condizioni} avvelenato

\textbf{Sensi} scurovisione 36 m

\textbf{Linguaggi} Infernale, telepatia 36 m

\textbf{Sfida} 2 (450 PE)

\emph{\textbf{Resistenza alla Magia.}} Il diavolo ha +1d6 ai Tiri Salvezza contro incantesimi e altri effetti magici.

\emph{\textbf{Sorvolare.}} Il diavolo non provoca attacchi di opportunità quando vola via dalla portata di un nemico.

\emph{\textbf{Spine Limitate.}} Il diavolo possiede dodici spine caudali. Le spine usate ricrescono a mezzanotte.

\emph{\textbf{Vista del Diavolo.}} La scurovisione del diavolo non è limitata dall'oscurità magica.

\textbf{Azioni}

\emph{\textbf{Multiattacco.}} Il diavolo effettua due attacchi: uno con il morso e uno con il suo forcone o due con le sue spine caudali.

\emph{\textbf{Forcone.} Attacco con arma da mischia}: +2 a colpire, portata 1 m, un bersaglio.

\emph{Colpisce:} 3 (1d6) danni perforanti.

\emph{\textbf{Morso.} Attacco con arma da mischia}: +2 a colpire, portata 1 m, un bersaglio.

\emph{Colpisce:} 5 (2d4) danni taglienti.

\emph{\textbf{Spina Caudale.} Attacco con arma a Distanza}: +4 a colpire, gittata 6m, un bersaglio.

\emph{Colpisce:} 4 (1d4 + 2) danni perforanti più 3 (1d6) danni da fuoco.

\textbf{Ecologia}\\
Ambiente: Qualsiasi (Inferno)\\
Organizzazione: Solitario, coppia, gruppo (3-5) o plotone (6-11)\\
Tesoro: Standard\\
\textbf{Descrizione}\\
Sentinelle delle volte dell'Inferno, carcerieri delle anime più nere e armi viventi delle forge infernali, i diavoli uncinati, noti ai diabolisti come hamatula, impongono ai dannati i loro ceppi e custodiscono il nefasto operato dei diavoli maggiori. Un hamatula ama sentire il sangue caldo sulle proprie spine e preferisce gettarsi nella mischia quando gli viene offerta l'opportunità di combattere.\\
Gli hamatula sono collezionisti ed organizzatori, e sono gli alleati favoriti di bramosi evocatori, dal momento che spesso portano con sé tesori tentatori dalle volte dell'Inferno o conoscono il sentiero per ottenere mortali ricchezze. Se lasciati agire liberamente, nei nascondigli di questi diavoli spesso fanno mostra i trofei trafitti di vecchie vittime, appesi come perverse collezioni di insetti su muri insanguinati.\\
La maggior parte dei diavoli uncinati è alta dai 2,1 metri in su e pesa 150 kg, sebbene i loro corpi asciutti e muscolosi sembrino più grossi per via degli spuntoni in continua crescita che fuoriescono dai loro corpi, taglienti come lame.\\


\medskip\index{Mostri - Erinni}\textbf{Erinni}

\emph{Media immondo (diavolo), legale malvagio}

\textbf{FORZA} +4

\textbf{DESTREZZA} +3

\textbf{COSTITUZIONE} +4

\textbf{INTELLIGENZA} +2

\textbf{SAGGEZZA} +2

\textbf{CARISMA} +4

\textbf{Iniziativa} +3 -- \textbf{Difesa} 24 (armatura di piastre)

\textbf{Punti Ferita} 153 (18d8 + 72)

\textbf{Movimento} 9 m, volo 18 m

\textbf{Tiri Salvezza} Tempra +11, Riflessi +12, Volontà +7

\textbf{Resistenze al Danno} freddo; da botta, perforante e tagliente di attacchi non magici o che non siano argentati

\textbf{Immunità al Danno} fuoco, veleno

\textbf{Immunità alle Condizioni} avvelenato

\textbf{Sensi} visione del vero 36 m

\textbf{Linguaggi} Infernale, telepatia 36 m

\textbf{Sfida} 12 (8.400 PE)

\emph{\textbf{Armi Diaboliche.}} Gli attacchi con arma dell'erinni sono magici e infliggono 13 (3d8) danni da veleno aggiuntivi quando colpiscono (già incluso negli attacchi).

\emph{\textbf{Resistenza alla Magia.}} L'erinni ha +1d6 ai Tiri Salvezza contro incantesimi e altri effetti magici.

\textbf{Azioni}

\emph{\textbf{Multiattacco.}} L'erinni effettua tre attacchi.

\emph{\textbf{Spada Lunga.} Attacco con arma da mischia}: +8 a colpire, portata 1 m, un bersaglio.

\emph{Colpisce:} 8 (1d8 + 4) danni taglienti, o 9 (1d10 + 4) danni taglienti se usata con due mani, più 13 (3d8) danni da veleno.

\emph{\textbf{Arco Lungo.} Attacco con arma a Distanza}: +7 a colpire, gittata 45m, un bersaglio.

\emph{Colpisce:} 7 (1d8 + 4) danni perforanti più 13 (3d8) danni da veleno, e il bersaglio deve riuscire un Tiro Salvezza di Tempra CD 14 o restare avvelenato. Il veleno rimane finché non viene rimosso da un incantesimo \emph{ristorazione inferiore} o simile.

\textbf{Reazioni}

\emph{\textbf{Parata.}} L'erinni somma 4 alla sua Difesa contro un attacco da mischia che lo colpirebbe. Per farlo, l'erinni deve poter vedere il suo attaccante e impugnare un'arma da mischia.

\textbf{Ecologia}\\
Ambiente: Qualsiasi (Inferno)\\
Organizzazione: Solitario o trio\\
Tesoro: Triplo (Arco Lungo Composito Infuocato+1 [Forza +5], corda, Spada Lunga+1)\\
\textbf{Descrizione}\\
Note con molti nomi, i Caduti, le Ali Cineree e le Furie, i diavoli conosciuti come erinni insultano la loro forma angelica con la loro brama di vendetta e sanguinosa giustizia. Carnefici, non giudici, le erinni volteggiano sopra i cornicioni affilati come lame di Dite, il secondo girone cosmopolita dell'Inferno, sempre attente a cogliere ogni occasione di battaglia, che sia a difesa dell'inferno, per il capriccio dei loro diabolici signori o per l'appassionata chiamata di capricciosi evocatori mortali. Tutte le erinni intrecciano con i loro stessi capelli letali corde viventi, che utilizzano in battaglia per intralciare e sollevare in aria i loro nemici, schernendoli e condannandoli per le loro trasgressioni prima di lasciarli precipitare da grandi altezze.\\
Le erinni sono angeli bellissimi ed oscuri che accrescono deliberatamente la propria sensualità con cicatrici e lividi. Eppure, nonostante la loro bellezza, le erinni non sono seduttrici: mancano loro la sottigliezza e la pazienza richieste per questa raffinata arte emotiva, poiché preferiscono risolvere i loro problemi con atti di rapida ed atroce violenza. Spesso una erinni tratterrà il suo colpo mortale mentre tenta di uccidere un nemico, solo per prolungarne le sofferenze. La morte è in genere l'unico modo per sfuggire alle attenzioni di una erinni, e quelle più potenti sono abilissime nel tenere i loro nemici in vita ma inermi, così da prolungare il loro tormento, arrivando addirittura a mantenerli vivi con la magia. Si dice che le più potenti torturatrici erinni siano dotate di capacità che permettono alle sofferenze da loro inflitte di perdurare anche dopo la morte del soggetto.\\
La maggior parte delle erinni è alta poco meno di 1,8 metri e pesa circa 70 kg, e le loro ali nere piumate hanno una apertura superiore ai 3 metri.\\


\medskip\index{Mostri - Imp}\textbf{Imp}

\emph{Minuscola immondo (diavolo, mutaforma), legale malvagio}

\textbf{FORZA} -2

\textbf{DESTREZZA} +3

\textbf{COSTITUZIONE} +1

\textbf{INTELLIGENZA} +0

\textbf{SAGGEZZA} +1

\textbf{CARISMA} +2

\textbf{Iniziativa} +3 -- \textbf{Difesa} 14

\textbf{Punti Ferita} 10 (3d4 + 3)

\textbf{Movimento} 6 m, volo 12 m (6 m in forma di ratto; 6 m, volo 18 m in forma di corvo; 6 m, scalata 6 m in forma di ragno)

\textbf{Tiri Salvezza} Tempra +1, Riflessi +6, Volontà +4

\textbf{Competenze} Muoversi Silenziosamente / Nascondersi +5, Ingannare +4, Percepire Emozioni +3

\textbf{Resistenze al Danno} freddo; da botta, perforante e tagliente di attacchi non magici o che non siano argentati

\textbf{Immunità al Danno} fuoco, veleno

\textbf{Immunità alle Condizioni} avvelenato

\textbf{Sensi} scurovisione 36 m 

\textbf{Linguaggi} Infernale, Comune

\textbf{Sfida} 1 (200 PE)

\emph{\textbf{Mutaforma.}} Il diavolo può usare la sua azione per trasformarsi in una forma bestiale da ratto, corvo o ragno, o per tornare alla sua vera forma. Le sue statistiche sono le stesse in tutte le forme, sebbene gli attacchi possano variare per alcune di esse. Qualsiasi equipaggiamento stia indossando o trasportando non viene trasformato. Alla morte ritorna alla sua vera forma.

\emph{\textbf{Resistenza alla Magia.}} Il diavolo ha +1d6 ai Tiri Salvezza contro incantesimi e altri effetti magici.

\emph{\textbf{Vista del Diavolo.}} La scurovisione del diavolo non è limitata dall'oscurità magica.

\textbf{Azioni}

\emph{\textbf{Pungiglione (Morso in Forma di Bestia).} Attacco con arma da mischia}: +5 a colpire, portata 1 m, una creatura.

\emph{Colpisce:} 5 (1d4 + 3) danni perforanti, e il bersaglio deve effettuare un Tiro Salvezza di Tempra CD 11, subendo 10 (3d6) danni da veleno se lo fallisce, o la metà di questi danni se lo riesce. 

\emph{\textbf{Invisibilità.}} Il diavolo resta invisibile finché non attacca o termina la sua concentrazione. Qualsiasi cosa che il diavolo stia trasportando o indossando, resta invisibile finché rimane in contatto con il diavolo.

\textbf{Ecologia}\\
Ambiente: Qualsiasi (Inferno)\\
Organizzazione: Solitario, coppia o stormo (3-10)\\
Tesoro: Standard\\
\textbf{Descrizione}\\
Nati direttamente dalle fosse dell'Inferno, gli imp sono i diavoli meno potenti, anche se queste crudeli ed invadenti creature svolgono un ruolo importante nella corruzione delle anime mortali. Libere dalle gerarchie e dai doveri delle armate infernali, gli imp si dilettano ad ogni opportunità di viaggiare fino al Piano Materiale e di tentare astutamente i mortali, spingendoli a compiere atti sempre più depravati.\\

Volontariamente al servizio di incantatori nel ruolo di famigli, gli imp recitano la parte dei fedeli servitori, offrendo spesso ai loro padroni astuti consigli ed infernali intuizioni. In realtà, gli imp operano per inviare anime all'Inferno, accertandosi che l'anima del loro padrone, insieme a molte altre, sia destinata alla dannazione dopo la morte.\\

Gli imp variano molto in aspetto, in un ampio spettro di tratti bestiali e grotteschi, sebbene molti di essi abbiano la forma di un umanoide alato dalla pelle rossiccia, con lineamenti bulbosi. Il tipico imp è alto solamente 60 centimetri, ha un'apertura alare di 90 centimetri e pesa 5 kg.\\

Un imp su mille è dotato dell'abilità di comunicare telepaticamente con creature entro 15 metri ed il potere di modificare la propria forma in quella di un animale Piccolo o Minuscolo, come per effetto di un incantesimo Forma Ferina II. Questi imp consolari sono assai apprezzati dai diavoli potenti, che li inviano come servitori ai loro seguaci preferiti o per corrompere eroi mortali. Un imp consolare può essere evocato per mezzo del talento Famiglio Migliorato, ma solo da un incantatore di 8° livello o superiore. I diabolisti narrano di altre razze di imp con capacità altrettanto specializzate, ma se queste creature esistono realmente si tratta di casi estremamente rari.\\

Diversamente dagli altri diavoli, gli imp si ritrovano spesso liberi e soli nel Piano Materiale, in particolare dopo che sono stati evocati per servire come famigli ed i loro padroni sono morti (spesso, indirettamente, a causa delle macchinazioni dell'imp stesso). Senza alcun mezzo per poter fare ritorno a casa questi imp, liberi da ogni legame con padroni arcani, possono diventare pericolosi seccatori o persino porsi a capo di piccole tribù di sanguinosi umanoidi, quali Goblin o Coboldi.\\


\medskip\index{Mostri - Lemure}\textbf{Lemure}

\emph{Media immondo (diavolo), legale malvagio}

\textbf{FORZA} +0

\textbf{DESTREZZA} -3

\textbf{COSTITUZIONE} +0

\textbf{INTELLIGENZA} -5

\textbf{SAGGEZZA} +0

\textbf{CARISMA} -4

\textbf{Iniziativa} -3 -- \textbf{Difesa} 8

\textbf{Punti Ferita} 13 (3d8)

\textbf{Movimento} 5 metri

\textbf{Tiri Salvezza} Tempra +4, Riflessi +3, Volontà +0

\textbf{Resistenze al Danno} freddo

\textbf{Immunità al Danno} fuoco, veleno

\textbf{Immunità alle Condizioni} affascinato, avvelenato, spaventato

\textbf{Sensi} scurovisione 36 m

\textbf{Linguaggi} comprende l'Infernale ma non può parlare

\textbf{Sfida} 0 (10 PE)

\emph{\textbf{Rinvigorimento Diabolico.}} Un lemure che muore nei Nove Inferi ritorna in vita con tutti i suoi punti ferita in 1d10 giorni a meno che non venga ucciso da una creatura di allineamento buono su cui sia stato eseguito l'incantesimo \emph{benedire} o i suoi resti vengano
cosparsi di acqua sacra.

\emph{\textbf{Vista del Diavolo.}} La scurovisione del diavolo non è limitata dall'oscurità magica.

\textbf{Azioni}

\emph{\textbf{Pugno.} Attacco con arma da mischia}: +3 a colpire, portata 1 m, un bersaglio.

\emph{Colpisce:} 2 (1d4) danni da botta.

\textbf{Ecologia}\\
Ambiente: Qualsiasi (Inferno)\\
Organizzazione: Solitario, coppia, gruppo (3-5), sciame (6-17) o schiera (10-40 o più)\\
Tesoro: Nessuno\\
\textbf{Descrizione}\\
I più infimi tra i diavoli, i lemure hanno origine dalle fila delle anime condannate all'inferno, masse informi di carne tremolante. La scintilla di istinto o di memoria che sopravvive nella loro coscienza addormentata solitamente dà forma ai loro tratti, che imitano quelli dei suoi torturatori o delle anime torturate che li circondano. Grotteschi ed inutili, i tratti di un lemure non rivelano nulla di quello che è stato un tempo. Molti sfoggiano diversi orribili volti o sono nulla più di colonne ribollenti di carne cancerosa. Solamente i loro arti bitorzoluti, che agitano in continuazione, sembrano funzionare correttamente, e vengono usati solo per distruggere qualsiasi forma di vita non infernale che si avvicini troppo.\\
I lemure in movimento si consolidano in forme alte più di 1,2 metri e pesanti più di 100 kg, sebbene questi disgustosi diavoli, quando stanno riposando, spesso hanno l'indistinto aspetto di masse di carne disciolta dai tratti deformi.\\

Sebbene siano tra le più rivoltanti creature esistenti, i lemure rivestono un ruolo vitale nella perversa ecologia dell'Inferno. Quando, al termine della sua esistenza mortale, un'anima viene dannata, sia perché adoratrice di forze diaboliche che per mancanza di fede in altre divinità, essa si unisce alle masse delle anime sofferenti che riempiono le pianure dell'Averno, il primo girone dell'Inferno. Qui iniziano i tormenti, mentre diavoli minori le sospingono assieme ad altri spiriti, preparandole all'arduo viaggio fino ad uno dei gironi dell'inferno più profondi, solitamente uno adatto alla punizione appropriata per i crimini commessi dall'anima, oppure semplicemente verso il dominio di un diavolo che necessita di nuovi schiavi. Una volta giunte nel regno della loro dannazione, le anime affrontano innumerevoli secoli di tormento per mano dei diavoli, di altri esseri malvagi e delle letali macchinazioni dell'Inferno stesso. Mentre l'essenza mortale impazzisce lentamente, queste creature dimenticano le loro vite, divenendo prima selvagge e infine poco più che automi guidati dall'odio e dalla paura. Dopo eoni di questa esistenza, il crudele procedimento dell'Inferno distrugge totalmente l'anima oppure, nel caso degli spiriti più profani, riconsacra questi esseri dimenticati sotto la forma di lemure, la forma di vita più elementare dei diavoli, insensate orde di carne putrescente e diabolica. Questi esseri ripugnanti si radunano in grandi masse, rivoltanti ondate formate da migliaia e migliaia di queste creature.\\

I diavoli maggiori sono in grado di riconoscere i più corrotti tra loro e, per mezzo di torture misteriose o grazie ai poteri stessi dell'Inferno, le riplasmano in veri diavoli, appena rinati e pronti a servire obbedienti nelle legioni dei dannati.\\


\subsection{Dinosauri}

\medskip\index{Mostri - Plesiosauro}\textbf{Plesiosauro}

\emph{Grande bestia, disallineato}

\textbf{FORZA} +4

\textbf{DESTREZZA} +2

\textbf{COSTITUZIONE} +3

\textbf{INTELLIGENZA} -4

\textbf{SAGGEZZA} +1

\textbf{CARISMA} -3

\textbf{Iniziativa} +2 -- \textbf{Difesa} 14

\textbf{Punti Ferita} 68 (8d10 + 24)

\textbf{Movimento} 6 m, nuoto 12 m

\textbf{Tiri Salvezza} Tempra +18, Riflessi +11, Volontà +9

\textbf{Competenze} Muoversi Silenziosamente / Nascondersi +4, Consapevolezza +3

\textbf{Linguaggi} -

\textbf{Sfida} 2 (450 PE)

\emph{\textbf{Trattenere il Fiato.}} Il plesiosauro può trattenere il fiato per 1 ora.

\textbf{Azioni}

\emph{\textbf{Morso.} Attacco con arma da mischia}: +6 a colpire, portata 3 m, un bersaglio.

\emph{Colpisce:} 14 (3d6 + 4) danni perforanti.

\textbf{Ecologia}\\
Ambiente: Acquatico Caldo\\
Organizzazione: Solitario, coppia o branco (3-6)\\
Tesoro: Nessuno\\
\textbf{Descrizione}\\
Il plesiosauro è un rettile acquatico dal lungo collo. Sebbene tecnicamente non sia un dinosauro, questa creatura ed i suoi simili si trovano spesso a cacciare in laghi ed oceani nei quali è facile trovare dei dinosauri.\\


\medskip\index{Mostri - Tirannosauro}\textbf{Tirannosauro}

\emph{Enorme bestia, disallineato}

\textbf{FORZA} +7

\textbf{DESTREZZA} +0

\textbf{COSTITUZIONE} +4

\textbf{INTELLIGENZA} -4

\textbf{SAGGEZZA} +1

\textbf{CARISMA} -1

\textbf{Iniziativa} +0 -- \textbf{Difesa} 17

\textbf{Punti Ferita} 136 (13d12 + 52)

\textbf{Movimento} 15 m

\textbf{Tiri Salvezza} Tempra +15, Riflessi +12, Volontà +10

\textbf{Competenze} Consapevolezza +4

\textbf{Linguaggi} -

\textbf{Sfida} 8 (3.900 PE)

\textbf{Azioni}

\emph{\textbf{Multiattacco.}} Il tirannosauro effettua due attacchi: uno con il morso e uno con la coda. Non può effettuare entrambi gli attacchi contro lo stesso bersaglio.

\emph{\textbf{Coda.} Attacco con arma da mischia}: +10 a colpire, portata 3 m, un bersaglio.

\emph{Colpisce:} 20 (3d8 + 7) danni da botta.

\emph{\textbf{Morso.} Attacco con arma da mischia}: +10 a colpire, portata 3 m, un bersaglio.

\emph{Colpisce:} 33 (4d12 + 7) danni perforanti. Se il bersaglio è una creatura di taglia Media o inferiore, è afferrato (CD 17 per fuggire). Fino al termine dell'afferrare, il bersaglio è intralciato, e il tirannosauro non può usare il morso contro un altro bersaglio.

\textbf{Ecologia}\\
Ambiente: Foreste e Pianure Calde\\
Organizzazione: Solitario, coppia o branco (3-6)\\
Tesoro: Nessuno\\
\textbf{Descrizione}\\
Il tirannosauro è un predatore primario che misura 12 metri di lunghezza e pesa 7.000 kg.\\


\medskip\index{Mostri - Triceratopo}\textbf{Triceratopo}

\emph{Enorme bestia, disallineato}

\textbf{FORZA} +6

\textbf{DESTREZZA} -1

\textbf{COSTITUZIONE} +3

\textbf{INTELLIGENZA} -4

\textbf{SAGGEZZA} +0

\textbf{CARISMA} -3

\textbf{Iniziativa} -1 -- \textbf{Difesa} 16

\textbf{Punti Ferita} 95 (10d12 + 30)

\textbf{Movimento} 15 m

\textbf{Tiri Salvezza} Tempra +15, Riflessi +8, Volontà +5

\textbf{Linguaggi} -

\textbf{Sfida} 5 (1.800 PE)

\emph{\textbf{Carica Travolgente.}} Se il triceratopo si muove di almeno 6 metri diretto verso una creatura e la colpisce con un attacco di incornata durante lo stesso turno, il bersaglio deve riuscire un Tiro Salvezza su Tempra CD 13 o cadere prono. Se il bersaglio è prono, il triceratopo può effettuare un attacco di pestone contro di lui come azione bonus.

\textbf{Azioni}

\emph{\textbf{Incornata.} Attacco con arma da mischia}: +9 a colpire,
portata 1 m, un bersaglio.

\emph{Colpisce:} 24 (3d10 + 6) danni perforanti.

\emph{\textbf{Pestone.} Attacco con arma da mischia}: +9 a colpire,
portata 1 m, una creatura prona.

\emph{Colpisce:} 22 (3d10 + 6) danni da botta.

\textbf{Ecologia}\\
Ambiente: Pianure Calde\\
Organizzazione: Solitario, coppia o branco (5-8)\\
Tesoro: Nessuno\\
\textbf{Descrizione}\\
Il triceratopo è un erbivoro irascibile e caparbio. Un tipico triceratopo è lungo 9 metri e pesa 10.000 kg.\\


\medskip\index{Mostri - Doppelganger}\textbf{Doppelganger}

\emph{Media mostruosità (mutaforma), neutrale}

\textbf{FORZA} +0

\textbf{DESTREZZA} +4

\textbf{COSTITUZIONE} +2

\textbf{INTELLIGENZA} +0

\textbf{SAGGEZZA} +1

\textbf{CARISMA} +2

\textbf{Iniziativa} +4 -- \textbf{Difesa} 16

\textbf{Punti Ferita} 52 (8d8 + 16)

\textbf{Movimento} 9 m

\textbf{Tiri Salvezza} Tempra +4, Riflessi +5, Volontà +6

\textbf{Competenze} Ingannare +6, Percepire Emozioni +3

\textbf{Immunità alle Condizioni} affascinato

\textbf{Sensi} scurovisione 18 m

\textbf{Linguaggi} Comune

\textbf{Sfida} 3 (700 PE)

\emph{\textbf{Mutaforma.}} Il doppelganger può usare la sua azione per cambiare la propria forma in quella di un umanoide Piccolo o Medio che abbia visto, o per tornare alla sua vera forma. Le sue statistiche, a parte la taglia, sono le stesse in tutte le forme. Qualsiasi equipaggiamento stia indossando o trasportando non viene trasformato. Alla morte ritorna alla sua vera forma.

\emph{\textbf{Appostato.}} Nel primo round di combattimento, il doppelganger ha +1d6 ai tiri di attacco contro qualsiasi creatura abbia preso di sorpresa.

\emph{\textbf{Attacco di Sorpresa.}} Se il doppelganger sorprende una creatura e la colpisce con un attacco durante il primo round di combattimento, il bersaglio subisce 10 (3d6) danni aggiuntivi dall'attacco.

\textbf{Azioni}

\emph{\textbf{Multiattacco.}} Il doppelganger effettua due attacchi da
mischia.

\emph{\textbf{Schianto.} Attacco con arma da mischia}: +6 a colpire,
portata 1 m, un bersaglio.

\emph{Colpisce:} 7 (1d6 + 4) danni da botta.

\emph{\textbf{Leggere Pensieri.}} Il doppelganger legge magicamente i pensieri di superficie di una creatura entro 18 metri da lui. L'effetto può penetrare le barriere, ma 1 metro di legno o terra, 50 centimetri di pietra, 5 centimetri di metallo, o un sottile foglio di piombo lo blocca. Mentre il bersaglio è a gittata, il doppelganger può continuare a leggerne i pensieri, purché la concentrazione del doppelganger non venga infranta (come la concentrazione di un incantesimo). Mentre legge la mente di un bersaglio, il doppelganger ha +1d6 alle prove di Saggezza e Carisma contro il bersaglio.

\textbf{Ecologia}\\
Ambiente: Qualsiasi\\
Organizzazione: Solitario, coppia o banda (3-6)\\
Tesoro: Equipaggiamento da PNG\\
\textbf{Descrizione}\\
I doppelganger sono strani esseri che possono assumere la forma di coloro che incontrano. Nella sua forma naturale, la creatura somiglia più o meno ad un umanoide, ma snello e fragile, con membra magre e tratti facciali non del tutto formati. La sua carnagione è pallida, è glabro ed i suoi occhi sono bianchi e vacui.\\
I doppelganger preferiscono infiltrarsi in società dove possono ammassare ricchezza e potere, e vedono scarse prospettive nel fondare città coi loro simili. I doppelganger più giovani sperimentano le loro abilità su piccole tribù di orchi o goblin, poi si spostano in società più complesse come comunità naniche, elfiche e umane. Piuttosto che diventare bersagli occupando posizioni di comando, preferiscono mantenere il potere da dietro il trono, o usare molteplici identità per manipolare cittadini influenti o intere gilde.\\
I doppelganger fanno un uso eccellente della loro mimica naturale per tendere imboscate, trappole con esca e infiltrarsi nelle società umanoidi. Anche se di solito non sono malvagi, sono interessati solo a sé stessi e considerano tutti gli altri come giocattoli da manipolare ed ingannare. Piace loro moltissimo invadere le società umane per soddisfare i loro desideri; alcuni amano i complessi giochi politici mentre altri cercano continuamente di cambiare razza, sesso e partner amorosi. Nonostante non sia la norma, quei doppelganger che usano i loro doni per scopi crudeli e sadici sono molto famosi, e questi mutaforma sono i principali responsabili della sinistra reputazione della loro razza. Certamente, una creatura capace di cambiare forma è avvantaggiata quando cerca di evitare di essere catturata per i suoi crimini, ed alcuni doppelganger particolarmente malevoli godono nel troncare relazioni amorose inscenando tradimenti.\\

Delle voci insistenti parlano di doppelganger ancor più potenti capaci non solo di cambiare il loro aspetto, ma anche di far proprie abilità, ricordi ed anche capacità straordinarie e soprannaturali delle creature che scelgono di impersonare.\\


\subsection{Draghi Cromatici}

\medskip\index{Mostri - Drago Bianco Antico}\textbf{Drago Bianco Antico}

\emph{Mastodontica drago, caotico malvagio}

\textbf{FORZA} +8

\textbf{DESTREZZA} +0

\textbf{COSTITUZIONE} +8

\textbf{INTELLIGENZA} +0

\textbf{SAGGEZZA} +1

\textbf{CARISMA} +2

\textbf{Iniziativa} +0 -- \textbf{Difesa} 30

\textbf{Punti Ferita} 333 (18d20 + 144)

\textbf{Movimento} 12 m, nuoto 12 m, scavo 12 m, volo 24 m

\textbf{Tiri Salvezza} Tempra +19, Riflessi +14, Volontà +16

\textbf{Competenze} Muoversi Silenziosamente / Nascondersi +6, Consapevolezza +13

\textbf{Immunità al Danno} freddo, arma +1

\textbf{Sensi} scurovisione 36 m, vista cieca 18 m

\textbf{Linguaggi} Comune, Draconico

\textbf{Sfida} 20 (25.000 PE)

\emph{\textbf{Camminare sul Ghiaccio.}} Il drago può muoversi e arrampicarsi su superfici ghiacciate senza bisogno di effettuare prove di caratteristica. Inoltre, il terreno difficile composto di ghiaccio o neve non gli costa movimento aggiuntivo.

\emph{\textbf{Resistenza Leggendaria (3/Giorno).}} Se il drago fallisce un Tiro Salvezza, può scegliere invece di riuscire.

\textbf{Azioni}

\emph{\textbf{Multiattacco.}} Il drago può usare la sua Presenza Spaventosa. Poi effettuare tre attacchi: uno con il morso e due con gli artigli.

\emph{\textbf{Artiglio.} Attacco con arma da mischia}: +14 a colpire, portata 3 m, un bersaglio.

\emph{Colpisce:} 15 (2d6 + 8) danni taglienti.

\emph{\textbf{Coda.} Attacco con arma da mischia}: +14 a colpire, portata 6 m, un bersaglio.

\emph{Colpisce:} 17 (2d8 + 8) danni da botta.

\emph{\textbf{Morso.} Attacco con arma da mischia}: +14 a colpire, portata 5 metri, un bersaglio.

\emph{Colpisce:} 19 (2d10 + 8) danni perforanti più 9 (2d8) danni da freddo.

\emph{\textbf{Presenza Spaventosa.}} Ogni creatura scelta dal drago, che si trovi entro 36 metri da esso e consapevole della sua presenza, deve riuscire un Tiro Salvezza di Volontà CD 16 o restare spaventata per 1 minuto. Una creatura può ripetere il Tiro Salvezza al termine di ciascun suo turno, terminando l'effetto se lo riesce. Se il Tiro Salvezza della creatura ha successo o l'effetto ha termine per essa, la creatura è immune alla Presenza Spaventosa del drago per le successive 24 ore.

\emph{\textbf{Soffio Gelido (Ricarica 5-6).}} Il drago esala un'esplosione di ghiaccio in un cono di 27 metri. Ogni creatura in quell'area deve effettuare un Tiro Salvezza di Tempra CD 22 e subire 72 (16d8) danni da freddo se fallisce il Tiro Salvezza, o la metà di questi danni se lo riesce.

\textbf{Azioni Aggiuntive}

Il drago può effettuare 3 Azioni aggiuntive, scelte tra le opzioni seguenti. Può usare solo un'opzione leggendaria alla volta e solo al termine del turno di un'altra creatura. Il drago recupera le Azioni aggiuntive spese all'inizio del proprio turno.

\textbf{Attacco di Ala (Costa 2 Azioni).} Il drago batte le ali. Ogni creatura entro 5 metri dal drago deve riuscire un Tiro Salvezza su Riflessi CD 22 o subire 15 (2d6 + 8) danni da botta e venir gettato prono. Il drago può poi volare fino a metà del suo movimento di volo. \textbf{Attacco di Coda.} Il drago effettua un attacco di coda. \textbf{Individuare.} Il drago effettua una prova di Saggezza (Consapevolezza).

\textbf{Ecologia}\\
Ambiente: Montagne Fredde\\
Organizzazione: Solitario\\
Tesoro: Triplo\\
\textbf{Descrizione}\\
Anche se molti lo considerano il più debole e bestiale tra i draghi cromatici, il drago bianco supplisce alla sua mancanza di astuzia con la pura e semplice ferocia. I draghi bianchi vivono su remote e gelide cime montuose, e nei bassopiani artici, facendo la loro tana in scintillanti caverne piene di ghiaccio e neve. Preferiscono che i loro pasti siano completamente congelati.\\

\textbf{Incantesimi}\index{Incantesimi da Drago Bianco}\\
Il drago e' anche capace, se di eta' giovanile o piu' anziano, di lanciare incantesimi.\\
Puo' lanciare un numero di incantesimi pari al suo valore di Carisma, senza aver necessita' di componenti magici.\\
Non deve fare prove di magia, si considera che riesca sempre con un punteggio pari alla Difficoltà + il valore del Carisma.\\
Se deve fare un Tiro per Colpire ha CA = Grado di Sfida con un bonus al colpire pari alla Forza\\
Gli incantesimi a disposizione sono:\\
- Scudo di Fuoco\\
- Tempesta di ghiaccio\\
- Tempesta di Nevischio\\


\medskip\index{Mostri - Drago Bianco Adulto}\textbf{Drago Bianco Adulto}

\emph{Enorme drago, caotico malvagio}

\textbf{FORZA} +6

\textbf{DESTREZZA} +0

\textbf{COSTITUZIONE} +6

\textbf{INTELLIGENZA} -1

\textbf{SAGGEZZA} +1

\textbf{CARISMA} +1

\textbf{Iniziativa} +0 -- \textbf{Difesa} 25

\textbf{Punti Ferita} 200 (16d12 + 96)

\textbf{Movimento} 12 m, nuoto 12 m, scavo 9 m, volo 24 m

\textbf{Tiri Salvezza} Tempra +13, Riflessi +9, Volontà +10

\textbf{Competenze} Muoversi Silenziosamente / Nascondersi +5, Consapevolezza +11

\textbf{Immunità al Danno} freddo

\textbf{Sensi} scurovisione 36 m, vista cieca 18 m

\textbf{Linguaggi} Comune, Draconico

\textbf{Sfida} 13 (10.000 PE)

\emph{\textbf{Camminare sul Ghiaccio.}} Il drago può muoversi e arrampicarsi su superfici ghiacciate senza bisogno di effettuare prove di caratteristica. Inoltre, il terreno difficile composto di ghiaccio o neve non gli costa movimento aggiuntivo.

\emph{\textbf{Resistenza Leggendaria (3/Giorno).}} Se il drago fallisce un Tiro Salvezza, può scegliere invece di riuscire. 

\textbf{Azioni}

\emph{\textbf{Multiattacco.}} Il drago può usare la sua Presenza Spaventosa e poi effettuare tre attacchi: uno con il morso e due con gli artigli.

\emph{\textbf{Artiglio.} Attacco con arma da mischia}: +11 a colpire, portata 1 m, un bersaglio.

\emph{Colpisce:} 13 (2d6 + 6) danni taglienti.

\emph{\textbf{Coda.} Attacco con arma da mischia}: +11 a colpire, portata 5 metri, un bersaglio.

\emph{Colpisce:} 15 (2d8 + 6) danni da botta.

\emph{\textbf{Morso.} Attacco con arma da mischia}: +11 a colpire, portata 3 m, un bersaglio.

\emph{Colpisce:} 17 (2d10 + 6) danni perforanti più 4 (1d8) danni da freddo.

\emph{\textbf{Presenza Spaventosa.}} Ogni creatura scelta dal drago, che si trovi entro 36 metri da esso e consapevole della sua presenza, deve riuscire un Tiro Salvezza di Volontà CD 14 o restare spaventata per 1 minuto. Una creatura può ripetere il Tiro Salvezza al termine di ciascun suo turno, terminando l'effetto se lo riesce. Se il Tiro Salvezza della creatura ha successo o l'effetto ha termine per essa, la creatura è immune alla Presenza Spaventosa del drago per le successive 24 ore.

\emph{\textbf{Soffio Gelido (Ricarica 5-6).}} Il drago esala un'esplosione di ghiaccio in un cono di 18 metri. Ogni creatura in quell'area deve effettuare un Tiro Salvezza di Tempra CD 19 e subire 54 (12d8) danni da freddo se fallisce il Tiro Salvezza, o la metà di questi danni se lo riesce. 

\textbf{Azioni Aggiuntive}

Il drago può effettuare 3 Azioni aggiuntive, scelte tra le opzioni seguenti. Può usare solo un'opzione leggendaria alla volta e solo al termine del turno di un'altra creatura. Il drago recupera le Azioni aggiuntive spese all'inizio del proprio turno.

\textbf{Attacco di Ala (Costa 2 Azioni).} Il drago batte le ali. Ogni creatura entro 3 metri dal drago deve riuscire un Tiro Salvezza su Riflessi CD 19 o subire 13 (2d6 + 6) danni da botta e venir gettato prono. Il drago può poi volare fino a metà del suo movimento di volo. \textbf{Attacco di Coda.} Il drago effettua un attacco di coda 
.
\textbf{Individuare.} Il drago effettua una prova di Saggezza (Consapevolezza).

\textbf{Ecologia}\\
Ambiente: Montagne Fredde\\
Organizzazione: Solitario\\
Tesoro: Triplo\\
\textbf{Descrizione}\\
Anche se molti lo considerano il più debole e bestiale tra i draghi cromatici, il drago bianco supplisce alla sua mancanza di astuzia con la pura e semplice ferocia. I draghi bianchi vivono su remote e gelide cime montuose, e nei bassopiani artici, facendo la loro tana in scintillanti caverne piene di ghiaccio e neve. Preferiscono che i loro pasti siano completamente congelati.\\
\textbf{Incantesimi}\index{Incantesimi da Drago Bianco}\\
Il drago e' anche capace, se di eta' giovanile o piu' anziano, di lanciare incantesimi.\\
Puo' lanciare un numero di incantesimi pari al suo valore di Carisma, senza aver necessita' di componenti magici.\\
Non deve fare prove di magia, si considera che riesca sempre con un punteggio pari alla Difficoltà + il valore del Carisma.\\
Se deve fare un Tiro per Colpire ha CA = Grado di Sfida con un bonus al colpire pari alla Forza\\
Gli incantesimi a disposizione sono:\\
- Scudo di Fuoco\\
- Tempesta di ghiaccio\\
- Tempesta di Nevischio\\


\medskip\index{Mostri - Drago Bianco Giovane}\textbf{Drago Bianco Giovane}

\emph{Grande drago, caotico malvagio}

\textbf{FORZA} +4

\textbf{DESTREZZA} +0

\textbf{COSTITUZIONE} +4

\textbf{INTELLIGENZA} -2

\textbf{SAGGEZZA} +0

\textbf{CARISMA} +1

\textbf{Iniziativa} +0 -- \textbf{Difesa} 20

\textbf{Punti Ferita} 133 (14d10 + 56)

\textbf{Movimento} 12 m, nuoto 12 m, scavo 6 m, volo 24 m

\textbf{Tiri Salvezza} Tempra +8, Riflessi +7, Volontà +5

\textbf{Competenze} Muoversi Silenziosamente / Nascondersi +3, Consapevolezza +6

\textbf{Immunità al Danno} freddo

\textbf{Sensi} scurovisione 36 m, vista cieca 9 m

\textbf{Linguaggi} Comune, Draconico

\textbf{Sfida} 6 (2.300 PE)

\emph{\textbf{Camminare sul Ghiaccio.}} Il drago può muoversi e arrampicarsi su superfici ghiacciate senza bisogno di effettuare prove di caratteristica. Inoltre, il terreno difficile composto di ghiaccio o neve non gli costa movimento aggiuntivo.

\textbf{Azioni}

\emph{\textbf{Multiattacco.}} Il drago può usare la sua Presenza Spaventosa. Poi effettuare tre attacchi: uno con il morso e due con gli artigli.

\emph{\textbf{Artiglio.} Attacco con arma da mischia}: +7 a colpire, portata 1 m, un bersaglio.

\emph{Colpisce:} 11 (2d6 + 4) danni taglienti.

\emph{\textbf{Morso.} Attacco con arma da mischia}: +7 a colpire, portata 3 m, un bersaglio.

\emph{Colpisce:} 15 (2d10 + 4) danni perforanti più 4 (1d8) danni da freddo.

\emph{\textbf{Soffio Gelido (Ricarica 5-6).}} Il drago esala un'esplosione di ghiaccio in un cono di 9 metri. Ogni creatura in quell'area deve effettuare un Tiro Salvezza di Tempra CD 15 e subire 45 (10d8) danni da freddo se fallisce il Tiro Salvezza, o la metà di questi danni se lo riesce.

\textbf{Ecologia}\\
Ambiente: Montagne Fredde\\
Organizzazione: Solitario\\
Tesoro: Triplo\\
\textbf{Descrizione}\\
Anche se molti lo considerano il più debole e bestiale tra i draghi cromatici, il drago bianco supplisce alla sua mancanza di astuzia con la pura e semplice ferocia. I draghi bianchi vivono su remote e gelide cime montuose, e nei bassopiani artici, facendo la loro tana in scintillanti caverne piene di ghiaccio e neve. Preferiscono che i loro pasti siano completamente congelati.\\
\textbf{Incantesimi}\index{Incantesimi da Drago Bianco}\\
Il drago e' anche capace, se di eta' giovanile o piu' anziano, di lanciare incantesimi.\\
Puo' lanciare un numero di incantesimi pari al suo valore di Carisma, senza aver necessita' di componenti magici.\\
Non deve fare prove di magia, si considera che riesca sempre con un punteggio pari alla Difficoltà + il valore del Carisma.\\
Se deve fare un Tiro per Colpire ha CA = Grado di Sfida con un bonus al colpire pari alla Forza\\
Gli incantesimi a disposizione sono:\\
- Scudo di Fuoco\\
- Tempesta di ghiaccio\\
- Tempesta di Nevischio\\


\medskip\index{Mostri - Drago Bianco Cucciolo}\textbf{Drago Bianco Cucciolo}

\emph{Media drago, caotico malvagio}

\textbf{FORZA} +2

\textbf{DESTREZZA} +0

\textbf{COSTITUZIONE} +2

\textbf{INTELLIGENZA} -3

\textbf{SAGGEZZA} +0

\textbf{CARISMA} +0

\textbf{Iniziativa} +0 -- \textbf{Difesa} 17

\textbf{Punti Ferita} 32 (5d8 + 10)

\textbf{Movimento} 9 m, nuoto 9 m, scavo 5 metri, volo 18 m

\textbf{Tiri Salvezza} Tempra +2, Riflessi +1, Volontà +1

\textbf{Competenze} Muoversi Silenziosamente / Nascondersi +2, Consapevolezza +4

\textbf{Immunità al Danno} freddo

\textbf{Sensi} scurovisione 18 m, vista cieca 3 m

\textbf{Linguaggi} Draconico

\textbf{Sfida} 2 (450 PE)

\textbf{Azioni}

\emph{\textbf{Morso.} Attacco con arma da mischia}: +7 a colpire, portata 3 m, un bersaglio.

\emph{Colpisce:} 15 (2d10 + 4) danni perforanti più 4 (1d8) danni da freddo.

\emph{\textbf{Soffio Gelido (Ricarica 5-6).}} Il drago esala un'esplosione di ghiaccio in un cono di 5 metri. Ogni creatura in quell'area deve effettuare un Tiro Salvezza di Tempra CD 12 e subire 22 (5d8) danni da freddo se fallisce il Tiro Salvezza, o la metà di questi danni se lo riesce.

\textbf{Ecologia}\\
Ambiente: Montagne Fredde\\
Organizzazione: Solitario\\
Tesoro: Triplo\\
\textbf{Descrizione}\\
Anche se molti lo considerano il più debole e bestiale tra i draghi cromatici, il drago bianco supplisce alla sua mancanza di astuzia con la pura e semplice ferocia. I draghi bianchi vivono su remote e gelide cime montuose, e nei bassopiani artici, facendo la loro tana in scintillanti caverne piene di ghiaccio e neve. Preferiscono che i loro pasti siano completamente congelati.\\


\medskip\index{Mostri - Drago Blu Antico}\textbf{Drago Blu Antico}

\emph{Mastodontica drago, legale malvagio}

\textbf{FORZA} +9

\textbf{DESTREZZA} +0

\textbf{COSTITUZIONE} +8

\textbf{INTELLIGENZA} +4

\textbf{SAGGEZZA} +3

\textbf{CARISMA} +5

\textbf{Iniziativa} +4 -- \textbf{Difesa} 34

\textbf{Punti Ferita} 481 (26d20 + 208)

\textbf{Movimento} 12 m, scavo 12 m, volo 24 m

\textbf{Tiri Salvezza} Tempra +21, Riflessi +13, Volontà +19

\textbf{Competenze} Muoversi Silenziosamente / Nascondersi +7, Consapevolezza +17

\textbf{Immunità al Danno} fulmine, arma +1

\textbf{Sensi} scurovisione 36 m, vista cieca 18 m

\textbf{Linguaggi} Comune, Draconico

\textbf{Sfida} 23 (50.000 PE)

\emph{\textbf{Resistenza Leggendaria (3/Giorno).}} Se il drago fallisce un Tiro Salvezza, può scegliere invece di riuscire.

\textbf{Azioni}

\emph{\textbf{Multiattacco.}} Il drago può usare la sua Presenza Spaventosa. Poi effettuare tre attacchi: uno con il morso e due con gli artigli.

\emph{\textbf{Artiglio.} Attacco con arma da mischia}: +16 a colpire,
portata 3 m, un bersaglio.

\emph{Colpisce:} 16 (2d6 + 9) danni taglienti.

\emph{\textbf{Coda.} Attacco con arma da mischia}: +16 a colpire, portata 6 m, un bersaglio.

\emph{Colpisce:} 18 (2d8 + 9) danni da botta.

\emph{\textbf{Morso.} Attacco con arma da mischia}: +16 a colpire, portata 5 metri, un bersaglio.

\emph{Colpisce:} 20 (2d10 + 9) danni perforanti più 11 (2d10) danni da fulmine.

\emph{\textbf{Presenza Spaventosa.}} Ogni creatura scelta dal drago, che si trovi entro 36 metri da esso e consapevole della sua presenza, deve riuscire un Tiro Salvezza di Volontà CD 20 o restare spaventata per 1 minuto. Una creatura può ripetere il Tiro Salvezza al termine di ciascun suo turno, terminando l'effetto se lo riesce. Se il Tiro Salvezza della creatura ha successo o l'effetto ha termine per essa, la creatura è immune alla Presenza Spaventosa del drago per le successive 24 ore.

\emph{\textbf{Soffio Fulminante (Ricarica 5-6).}} Il drago esala fulmini in una linea lunga 36 metri e larga 3 metri. Ogni creatura su quella linea deve effettuare un Tiro Salvezza di Riflessi CD 23 e subire 88 (16d10) danni da fulmine se fallisce il Tiro Salvezza, o la metà di questi danni se lo riesce.

\textbf{Azioni Aggiuntive}

Il drago può effettuare 3 Azioni aggiuntive, scelte tra le opzioni seguenti. Può usare solo un'opzione leggendaria alla volta e solo al termine del turno di un'altra creatura. Il drago recupera le Azioni aggiuntive spese all'inizio del proprio turno.

\textbf{Attacco di Ala (Costa 2 Azioni).} Il drago batte le ali. Ogni creatura entro 5 metri dal drago deve riuscire un Tiro Salvezza su Riflessi CD 24 o subire 16 (2d6 + 9) danni da botta e venir gettato prono. Il drago può poi volare fino a metà del suo movimento di volo.

\textbf{Attacco di Coda.} Il drago effettua un attacco di coda.

\textbf{Individuare.} Il drago effettua una prova di Saggezza (Consapevolezza).

\textbf{Individuare.} Il drago effettua una prova di Saggezza (Consapevolezza).\\
\textbf{Ecologia}\\
Ambiente: Picchi montuosi\\
Organizzazione: Solitario\\
Tesoro: Triplo\\
\textbf{Descrizione}\\
I draghi blu sono intriganti consumati ed ossessivamente ordinati. In combattimento, i draghi blu preferiscono prendere di sorpresa i nemici, se possibile, e non esitano a ritirarsi se le cose si mettono male. Preferiscono fare la loro tana vicino a quelli che controllano, qualche volta anche entro i confini di una città.\\
\textbf{Incantesimi}\index{Incantesimi da Drago Blu}\\
Il drago e' anche capace, se di eta' giovanile o piu' anziano, di lanciare incantesimi.\\
Puo' lanciare un numero di incantesimi pari al suo valore di Carisma, senza aver necessita' di componenti magici.\\
Non deve fare prove di magia, si considera che riesca sempre con un punteggio pari alla Difficoltà + il valore del Carisma.\\
Se deve fare un Tiro per Colpire ha CA = Grado di Sfida con un bonus al colpire pari alla Forza\\
Gli incantesimi a disposizione sono:\\
- Catena di fulmini\\
- Gabbia di Forza\\
- Teletrasporto\\
- Forma Eterea\\


\medskip\index{Mostri - Drago Blu Adulto}\textbf{Drago Blu Adulto}

\emph{Enorme drago, legale malvagio}

\textbf{FORZA} +7

\textbf{DESTREZZA} +0

\textbf{COSTITUZIONE} +6

\textbf{INTELLIGENZA} +3

\textbf{SAGGEZZA} +2

\textbf{CARISMA} +4

\textbf{Iniziativa} +3 -- \textbf{Difesa} 27

\textbf{Punti Ferita} 225 (18d12 + 108)

\textbf{Movimento} 12 m, scavo 12 m, volo 24 m

\textbf{Tiri Salvezza} Tempra +15, Riflessi +10, Volontà +13

\textbf{Competenze} Muoversi Silenziosamente / Nascondersi +5, Consapevolezza +12

\textbf{Immunità al Danno} fulmine

\textbf{Sensi} scurovisione 36 m, vista cieca 18 m

\textbf{Linguaggi} Comune, Draconico

\textbf{Sfida} 16 (15.000 PE)

\emph{\textbf{Resistenza Leggendaria (3/Giorno).}} Se il drago fallisce un Tiro Salvezza, può scegliere invece di riuscire.

\textbf{Azioni}

\emph{\textbf{Multiattacco.}} Il drago può usare la sua Presenza Spaventosa. Poi effettuare tre attacchi: uno con il morso e due con gli artigli.

\emph{\textbf{Artiglio.} Attacco con arma da mischia}: +12 a colpire, portata 1 m, un bersaglio.

\emph{Colpisce:} 14 (2d6 + 7) danni taglienti.

\emph{\textbf{Coda.} Attacco con arma da mischia}: +12 a colpire, portata 5 metri, un bersaglio.

\emph{Colpisce:} 16 (2d8 + 7) danni da botta.

\emph{\textbf{Morso.} Attacco con arma da mischia}: +12 a colpire, portata 3 m, un bersaglio.

\emph{Colpisce:} 18 (2d10 + 7) danni perforanti più 5 (1d10) danni da fulmine.

\emph{\textbf{Presenza Spaventosa.}} Ogni creatura scelta dal drago, che si trovi entro 36 metri da esso e consapevole della sua presenza, deve riuscire un Tiro Salvezza di Volontà CD 17 o restare spaventata per 1 minuto. Una creatura può ripetere il Tiro Salvezza al termine di ciascun suo turno, terminando l'effetto se lo riesce. Se il Tiro Salvezza della creatura ha successo o l'effetto ha termine per essa, la creatura è immune alla Presenza Spaventosa del drago per le successive 24 ore.

\emph{\textbf{Soffio Fulminante (Ricarica 5-6).}} Il drago esala fulmini in una linea lunga 27 metri e larga 1 metro. Ogni creatura su quella linea deve effettuare un Tiro Salvezza di Riflessi CD 19 e subire 66 (12d10) danni da fulmine se fallisce il Tiro Salvezza, o la metà di questi danni se lo riesce.

\textbf{Azioni Aggiuntive}

Il drago può effettuare 3 Azioni aggiuntive, scelte tra le opzioni seguenti. Può usare solo un'opzione leggendaria alla volta e solo al termine del turno di un'altra creatura. Il drago recupera le Azioni aggiuntive spese all'inizio del proprio turno.

\textbf{Attacco di Ala (Costa 2 Azioni).} Il drago batte le ali. Ogni creatura entro 3 metri dal drago deve riuscire un Tiro Salvezza su Riflessi CD 20 o subire 14 (2d6 + 7) danni da botta e venir gettato prono. Il drago può poi volare fino a metà della del suo movimento di  volo.

\textbf{Attacco di Coda.} Il drago effettua un attacco di coda.

\textbf{Individuare.} Il drago effettua una prova di Saggezza (Consapevolezza).

\textbf{Ecologia}\\
Ambiente: Picchi montuosi\\
Organizzazione: Solitario\\
Tesoro: Triplo\\
\textbf{Descrizione}\\
I draghi blu sono intriganti consumati ed ossessivamente ordinati. In combattimento, i draghi blu preferiscono prendere di sorpresa i nemici, se possibile, e non esitano a ritirarsi se le cose si mettono male. Preferiscono fare la loro tana vicino a quelli che controllano, qualche volta anche entro i confini di una città.\\
\textbf{Incantesimi}\index{Incantesimi da Drago Blu}\\
Il drago e' anche capace, se di eta' giovanile o piu' anziano, di lanciare incantesimi.\\
Puo' lanciare un numero di incantesimi pari al suo valore di Carisma, senza aver necessita' di componenti magici.\\
Non deve fare prove di magia, si considera che riesca sempre con un punteggio pari alla Difficoltà + il valore del Carisma.\\
Se deve fare un Tiro per Colpire ha CA = Grado di Sfida con un bonus al colpire pari alla Forza\\
Gli incantesimi a disposizione sono:\\
- Catena di fulmini\\
- Gabbia di Forza\\
- Teletrasporto\\
- Forma Eterea\\


\medskip\index{Mostri - Drago Blu Giovane}\textbf{Drago Blu Giovane}

\emph{Enorme drago, legale malvagio}

\textbf{FORZA} 21(+5)

\textbf{DESTREZZA} +0

\textbf{COSTITUZIONE} +4

\textbf{INTELLIGENZA} +2

\textbf{SAGGEZZA} +1

\textbf{CARISMA} +3

\textbf{Iniziativa} +2 -- \textbf{Difesa} 23

\textbf{Punti Ferita} 152 (16d10 + 64)

\textbf{Movimento} 12 m, scavo 12 m, volo 24 m

\textbf{Tiri Salvezza} Tempra +10, Riflessi +8, Volontà +8

\textbf{Competenze} Muoversi Silenziosamente / Nascondersi +4, Consapevolezza +9

\textbf{Immunità al Danno} fulmine

\textbf{Sensi} scurovisione 36 m, vista cieca 9 m 

\textbf{Linguaggi} Comune, Draconico

\textbf{Sfida} 9 (5.000 PE)

\textbf{Azioni}

\emph{\textbf{Multiattacco.}} Il drago può effettuare tre attacchi: uno con il morso e due con gli artigli.

\emph{\textbf{Artiglio.} Attacco con arma da mischia}: +9 a colpire, portata 1 m, un bersaglio.

\emph{Colpisce:} 12 (2d6 + 5) danni taglienti.

\emph{\textbf{Morso.} Attacco con arma da mischia}: +9 a colpire, portata 3 m, un bersaglio.

\emph{Colpisce:} 16 (2d10 + 5) danni perforanti più 5 (1d10) danni da fulmine.

\emph{\textbf{Soffio Fulminante (Ricarica 5-6).}} Il drago esala fulmini in una linea lunga 18 metri e larga 1 metro. Ogni creatura su quella linea deve effettuare un Tiro Salvezza di Riflessi CD 16 e subire 55 (10d10) danni da fulmine se fallisce il Tiro Salvezza, o la metà di questi danni se lo riesce.

\textbf{Ecologia}\\
Ambiente: Picchi montuosi\\
Organizzazione: Solitario\\
Tesoro: Triplo\\
\textbf{Descrizione}\\
I draghi blu sono intriganti consumati ed ossessivamente ordinati. In combattimento, i draghi blu preferiscono prendere di sorpresa i nemici, se possibile, e non esitano a ritirarsi se le cose si mettono male. Preferiscono fare la loro tana vicino a quelli che controllano, qualche volta anche entro i confini di una città.\\
\textbf{Incantesimi}\index{Incantesimi da Drago Blu}\\
Il drago e' anche capace, se di eta' giovanile o piu' anziano, di lanciare incantesimi.\\
Puo' lanciare un numero di incantesimi pari al suo valore di Carisma, senza aver necessita' di componenti magici.\\
Non deve fare prove di magia, si considera che riesca sempre con un punteggio pari alla Difficoltà + il valore del Carisma.\\
Se deve fare un Tiro per Colpire ha CA = Grado di Sfida con un bonus al colpire pari alla Forza\\
Gli incantesimi a disposizione sono:\\
- Catena di fulmini\\
- Gabbia di Forza\\
- Teletrasporto\\
- Forma Eterea\\


\medskip\index{Mostri - Drago Blu Cucciolo}\textbf{Drago Blu Cucciolo}

\emph{Enorme drago, legale malvagio}

\textbf{FORZA} +3

\textbf{DESTREZZA} +0

\textbf{COSTITUZIONE} +2

\textbf{INTELLIGENZA} +1

\textbf{SAGGEZZA} +0

\textbf{CARISMA} +2

\textbf{Iniziativa} +1 -- \textbf{Difesa} 19

\textbf{Punti Ferita} 52 (8d8 + 16)

\textbf{Movimento} 9 m, scavo 5 metri, volo 18 m

\textbf{Tiri Salvezza} Tempra +4, Riflessi +1, Volontà +1

\textbf{Competenze} Muoversi Silenziosamente / Nascondersi +2, Consapevolezza +4

\textbf{Immunità al Danno} fulmine

\textbf{Sensi} scurovisione 18 m, vista cieca 3 m

\textbf{Linguaggi} Draconico

\textbf{Sfida} 3 (700 PE)

\textbf{Azioni}

\emph{\textbf{Morso.} Attacco con arma da mischia}: +5 a colpire, portata 1 m, un bersaglio.

\emph{Colpisce:} 8 (1d10 + 3) danni perforanti più 3 (1d6) danni da fulmine.

\emph{\textbf{Soffio Fulminante (Ricarica 5-6).}} Il drago esala fulmini in una linea lunga 9 metri e larga 1 metro. Ogni creatura su quella linea deve effettuare un Tiro Salvezza di Riflessi CD 12 e subire 22 (4d10) danni da fulmine se fallisce il Tiro Salvezza, o la metà di questi danni se lo riesce.

\textbf{Ecologia}\\
Ambiente: Picchi montuosi\\
Organizzazione: Solitario\\
Tesoro: Triplo\\
\textbf{Descrizione}\\
I draghi blu sono intriganti consumati ed ossessivamente ordinati. In combattimento, i draghi blu preferiscono prendere di sorpresa i nemici, se possibile, e non esitano a ritirarsi se le cose si mettono male. Preferiscono fare la loro tana vicino a quelli che controllano, qualche volta anche entro i confini di una città.\\

\medskip\textbf{Drago Giallo Antico}\index{Mostri - Drago Giallo Antico}\\
\emph{Mastodontica drago, neutrale malvagio}\\
\textbf{Forza}: +10\\
\textbf{Destrezza}: +1\\
\textbf{Costituzione}: +8\\
\textbf{Intelligenza}: +3\\
\textbf{Saggezza}: +2\\
\textbf{Carisma}: +4\\	
\textbf{Difesa}: 27 (armatura naturale) - \textbf{Iniziativa}: +4\\
\textbf{Punti Ferita}: 481 (26d20 + 208)\\
\textbf{Movimento}: 12 m, scavo 24 m, scalata 24, volo 12 m\\
\textbf{Tiri Salvezza}: Tempra +21, Riflessi +13, Volontà +19\\
\textbf{Competenze}: Criminalità +7, Consapevolezza +17\\
\textbf{Immunità al Danno}: fulmine\\
\textbf{Sensi}: Scurovisione 36 m, vista cieca 18 m\\
\textbf{Linguaggi} Comune, Draconico\\
\textbf{Sfida}: 23 (50.000 PE)\smallskip\\
\emph{\textbf{Resistenza Leggendaria (3/Giorno).}} Se il drago fallisce un Tiro Salvezza, può scegliere invece di riuscire. \\
\smallskip\textbf{Azioni}\\
\emph{\textbf{Multiattacco.}} Il drago può usare la sua Presenza Spaventosa. Poi effettuare tre attacchi: uno con il morso e due con gli artigli.\\
\emph{\textbf{Artiglio.} Attacco con arma da mischia}: +25 al colpire, portata 3 m, un bersaglio.\\
\emph{Colpisce:} 16 (2d6 + 9) danni taglienti.\\
\emph{\textbf{Coda.} Attacco con arma da mischia}: +25 al colpire, portata 6 m, un bersaglio.\\
\emph{Colpisce:} 18 (2d8 + 9) danni contundenti.\\
\emph{\textbf{Morso.} Attacco con arma da mischia}: +25 al colpire, portata 5 metri, un bersaglio.\\
\emph{Colpisce:} 20 (2d10 + 9) danni perforanti più 11 (2d10) danni da fulmine.\\
\emph{\textbf{Presenza Spaventosa.}} Ogni creatura scelta dal drago, che si trovi entro 36 metri da esso e consapevole della sua presenza, deve riuscire un Tiro Salvezza su Volontà DC  25 o restare spaventata per 1 minuto. Una creatura può ripetere il Tiro Salvezza al termine di ciascun suo round, terminando l'effetto se lo riesce. Se il Tiro Salvezza della creatura ha successo o l'effetto ha termine per essa, la creatura è immune alla Presenza Spaventosa del drago per le successive 24 ore.\\
\emph{\textbf{Soffio Incendiario (Ricarica 5-6).}} Il drago esala aria rovente in una linea lunga 36 metri e larga 3 metri. Ogni creatura su quella linea deve effettuare un Tiro Salvezza su Riflessi DC 30 e subire 88 (16d10) danni da fuoco se fallisce il Tiro Salvezza, o la metà di questi danni se lo riesce.\\
\textbf{Azioni Aggiuntive}\\
Il drago può effettuare 3 azioni aggiuntive, scelte tra le opzioni seguenti. Può usare solo un'Azione Aggiuntiva alla volta e solo al termine del round di un'altra creatura. Il drago recupera le Azioni Aggiuntive spese all'inizio del proprio round.\\
\textbf{Attacco di Ala (Costa 2 Azioni).} Il drago batte le ali. Ogni creatura entro 5 metri dal drago deve riuscire un Tiro Salvezza su Riflessi DC  31 o subire 16 (2d6 + 9) danni contundenti e venir gettato prono. Il drago può poi volare fino a metà della sua velocità di volo.\\
\textbf{Attacco di Coda.} Il drago effettua un attacco di coda.\\
\textbf{Individuare.} Il drago effettua una prova di Saggezza (Consapevolezza).\\
\textbf{Ecologia}\\
Ambiente: Deserti Caldi\\
Organizzazione: Solitario\\
Tesoro: Triplo\\
\textbf{Descrizione}\\
I draghi gialli sono predatori famelici e combattenti indisciplinati. Amano la caccia ed uccidere, sono consumati predatori che istintivamente aggrediscono chiunque sia nel loro territorio. Il deserto e' il loro terreno dove scavano trappole grezze per catturare le loro povere vittime.
Il soffio di un drago giallo e' un onda di calore (danno da fuoco).
\\
\textbf{Incantesimi}\index{Incantesimi da Drago Giallo}\\
Il drago e' anche capace, se di eta' giovanile o piu' anziano, di lanciare incantesimi.\\
Puo' lanciare un numero di incantesimi pari al suo valore di Carisma, senza aver necessita' di componenti magici.\\
Non deve fare prove di magia, si considera che riesca sempre con un punteggio pari alla Difficoltà + il valore del Carisma.\\
Se deve fare un Tiro per Colpire ha CA = Grado di Sfida con un bonus al colpire pari alla Forza\\
Gli incantesimi a disposizione sono:\\
- Riscaldare Metallo\\
- Palla di Fuoco\\
- Scudo di Fuoco\\


\medskip\index{Mostri - Drago Nero Antico}\textbf{Drago Nero Antico}

\emph{Mastodontica drago, caotico malvagio}

\textbf{FORZA} +8

\textbf{DESTREZZA} +2

\textbf{COSTITUZIONE} +7

\textbf{INTELLIGENZA} +3

\textbf{SAGGEZZA} +2

\textbf{CARISMA} +4

\textbf{Iniziativa} +3 -- \textbf{Difesa} 33

\textbf{Punti Ferita} 367 (21d20 + 147)

\textbf{Movimento} 12 m, scalata 12 m, volo 24 m

\textbf{Tiri Salvezza} Tempra +20, Riflessi +13, Volontà +18

\textbf{Competenze} Muoversi Silenziosamente / Nascondersi +9, Consapevolezza +16

\textbf{Immunità al Danno} acido, arma +1

\textbf{Sensi} scurovisione 36 m, vista cieca 18 m

\textbf{Linguaggi} Comune, Draconico

\textbf{Sfida} 21 (33.000 PE)

\emph{\textbf{Anfibio.}} Il drago può respirare aria e acqua.

\emph{\textbf{Resistenza Leggendaria (3/Giorno).}} Se il drago fallisce un Tiro Salvezza, può scegliere invece di riuscire.

\textbf{Azioni}

\emph{\textbf{Multiattacco.}} Il drago può usare la sua Presenza Spaventosa. Poi effettuare tre attacchi: uno con il morso e due con gli artigli.

\emph{\textbf{Artiglio.} Attacco con arma da mischia}: +15 a colpire, portata 3 m, un bersaglio.

\emph{Colpisce:} 15 (2d6 + 8) danni taglienti.

\emph{\textbf{Coda.} Attacco con arma da mischia}: +15 a colpire, portata 6 m, un bersaglio.

\emph{Colpisce:} 17 (2d8 + 8) danni da botta.

\emph{\textbf{Morso.} Attacco con arma da mischia} : +15 a colpire, portata 5 metri, un bersaglio.

\emph{Colpisce:} 19 (2d10 + 8) danni perforanti più 9 (4d6) danni da acido.

\emph{\textbf{Presenza Spaventosa.}} Ogni creatura scelta dal drago, che si trovi entro 36 metri da esso e consapevole della sua presenza, deve riuscire un Tiro Salvezza di Volontà CD 19 o restare spaventata per 1 minuto. Una creatura può ripetere il Tiro Salvezza al termine di ciascun suo turno, terminando l'effetto se lo riesce. Se il Tiro Salvezza della creatura ha successo o l'effetto ha termine per essa, la creatura è immune alla Presenza Spaventosa del drago per le successive 24 ore.

\emph{\textbf{Soffio Acido (Ricarica 5-6).}} Il drago esala acido in una linea di 27 metri larga 3 metri. Ogni creatura in quell'area deve effettuare un Tiro Salvezza di Riflessi CD 22 e subire 67 (15d8) danni da acido se fallisce il Tiro Salvezza, o la metà di questi danni se lo riesce.

\textbf{Azioni Aggiuntive}

Il drago può effettuare 3 Azioni aggiuntive, scelte tra le opzioni seguenti. Può usare solo un'opzione leggendaria alla volta e solo al termine del turno di un'altra creatura. Il drago recupera le Azioni aggiuntive spese all'inizio del proprio turno.

\textbf{Attacco di Ala (Costa 2 Azioni).} Il drago batte le ali. Ogni creatura entro 5 metri dal drago deve riuscire un Tiro Salvezza su Riflessi CD 23 o subire 15 (2d6 + 8) danni da botta e venir gettato prono. Il drago può poi volare fino a metà del suo movimento di volo.  

\textbf{Attacco di Coda.} Il drago effettua un attacco di coda.

\textbf{Individuare.} Il drago effettua una prova di Saggezza (Consapevolezza).

\textbf{Ecologia}\\
Ambiente: Paludi Calde\\
Organizzazione: Solitario\\
Tesoro: Triplo\\
\textbf{Descrizione}\\
Signori delle paludi e degli acquitrini più cupi, i draghi neri sono i padroni incontrastati del loro territorio, che dominano con crudeltà e infondendo terrore in chi abita nelle vicinanze. I draghi neri si stanziano nelle zone più remote delle paludi, specie in caverne sul fondo di pozze fetide e buie. Dentro, ammassano i loro luridi tesori e dormono tra radici e fango. I draghi neri amano il cibo un po' marcio e spesso lasciano un pasto a marcire in una pozza per giorni prima di consumarlo. I draghi neri preferiscono tesori che non si decompongono o degradano, accumulando tesori di monete, pietre preziose, gioielli e altri oggetti di pietra o metallo.\\
\textbf{Incantesimi}\index{Incantesimi da Drago Nero}\\
Il drago e' anche capace, se di eta' giovanile o piu' anziano, di lanciare incantesimi.\\
Puo' lanciare un numero di incantesimi pari al suo valore di Carisma, senza aver necessita' di componenti magici.\\
Non deve fare prove di magia, si considera che riesca sempre con un punteggio pari alla Difficoltà + il valore del Carisma.\\
Se deve fare un Tiro per Colpire ha CA = Grado di Sfida con un bonus al colpire pari alla Forza\\
Gli incantesimi a disposizione sono:\\
- Dito della morte\\
- Disintegrazione\\
- Blocca Mostri\\


\medskip\index{Mostri - Drago Nero Adulto}\textbf{Drago Nero Adulto}

\emph{Enorme drago, caotico malvagio}

\textbf{FORZA} +6

\textbf{DESTREZZA} +2

\textbf{COSTITUZIONE} +5

\textbf{INTELLIGENZA} +2

\textbf{SAGGEZZA} +1

\textbf{CARISMA} +3

\textbf{Iniziativa} +2 -- \textbf{Difesa} 28

\textbf{Punti Ferita} 195 (17d12 + 85)

\textbf{Movimento} 12 m, scalata 12 m, volo 24 m

\textbf{Tiri Salvezza} Tempra +14, Riflessi +10, Volontà +12

\textbf{Competenze} Muoversi Silenziosamente / Nascondersi +7, Consapevolezza +11

\textbf{Immunità al Danno} acido

\textbf{Sensi} scurovisione 36 m, vista cieca 18 m 

\textbf{Linguaggi} Comune, Draconico

\textbf{Sfida} 17 (18.000 PE)

\emph{\textbf{Anfibio.}} Il drago può respirare aria e acqua.

\emph{\textbf{Resistenza Leggendaria (3/Giorno).}} Se il drago fallisce un Tiro Salvezza, può scegliere invece di riuscire.

\textbf{Azioni}

\emph{\textbf{Multiattacco.}} Il drago può usare la sua Presenza Spaventosa. Poi effettuare tre attacchi: uno con il morso e due con gli artigli.

\emph{\textbf{Artiglio.} Attacco con arma da mischia}: +11 a colpire, portata 1 m, un bersaglio.

\emph{Colpisce:} 13 (2d6 + 6) danni taglienti.

\emph{\textbf{Coda.} Attacco con arma da mischia}: +11 a colpire, portata 5 metri, un bersaglio.

\emph{Colpisce:} 15 (2d8 + 6) danni da botta.

\emph{\textbf{Morso.} Attacco con arma da mischia}: +11 a colpire, portata 3 m, un bersaglio.

\emph{Colpisce:} 17 (2d10 + 6) danni perforanti più 4 (1d8) danni da acido.

\emph{\textbf{Presenza Spaventosa.}} Ogni creatura scelta dal drago, che si trovi entro 36 metri da esso e consapevole della sua presenza, deve riuscire un Tiro Salvezza di Volontà CD 16 o restare spaventata per 1 minuto. Una creatura può ripetere il Tiro Salvezza al termine di ciascun suo turno, terminando l'effetto se lo riesce. Se il Tiro Salvezza della creatura ha successo o l'effetto ha termine per essa, la creatura è immune alla Presenza Spaventosa del drago per le successive 24 ore.

\emph{\textbf{Soffio Acido (Ricarica 5-6).}} Il drago esala acido in una linea di 18 metri larga 1 metro. Ogni creatura in quell'area deve effettuare un Tiro Salvezza di Riflessi CD 18 e subire 54 (12d8) danni da acido se fallisce il Tiro Salvezza, o la metà di questi danni se lo
riesce.

\textbf{Azioni Aggiuntive}

Il drago può effettuare 3 Azioni aggiuntive, scelte tra le opzioni seguenti. Può usare solo un'opzione leggendaria alla volta e solo al termine del turno di un'altra creatura. Il drago recupera le Azioni aggiuntive spese all'inizio del proprio turno.

\textbf{Attacco di Ala (Costa 2 Azioni).} Il drago batte le ali. Ogni creatura entro 3 metri dal drago deve riuscire un Tiro Salvezza su Riflessi CD 19 o subire 13 (2d6 + 6) danni da botta e venir gettato prono. Il drago può poi volare fino a metà della del suo movimento di volo.

\textbf{Attacco di Coda.} Il drago effettua un attacco di coda.

\textbf{Individuare.} Il drago effettua una prova di Saggezza (Consapevolezza).

\textbf{Ecologia}\\
Ambiente: Paludi Calde\\
Organizzazione: Solitario\\
Tesoro: Triplo\\
\textbf{Descrizione}\\
Signori delle paludi e degli acquitrini più cupi, i draghi neri sono i padroni incontrastati del loro territorio, che dominano con crudeltà e infondendo terrore in chi abita nelle vicinanze. I draghi neri si stanziano nelle zone più remote delle paludi, specie in caverne sul fondo di pozze fetide e buie. Dentro, ammassano i loro luridi tesori e dormono tra radici e fango. I draghi neri amano il cibo un po' marcio e spesso lasciano un pasto a marcire in una pozza per giorni prima di consumarlo. I draghi neri preferiscono tesori che non si decompongono o degradano, accumulando tesori di monete, pietre preziose, gioielli e altri oggetti di pietra o metallo.\\
\textbf{Incantesimi}\index{Incantesimi da Drago Nero}\\
Il drago e' anche capace, se di eta' giovanile o piu' anziano, di lanciare incantesimi.\\
Puo' lanciare un numero di incantesimi pari al suo valore di Carisma, senza aver necessita' di componenti magici.\\
Non deve fare prove di magia, si considera che riesca sempre con un punteggio pari alla Difficoltà + il valore del Carisma.\\
Se deve fare un Tiro per Colpire ha CA = Grado di Sfida con un bonus al colpire pari alla Forza\\
Gli incantesimi a disposizione sono:\\
- Dito della morte\\
- Disintegrazione\\
- Blocca Mostri\\


\medskip\index{Mostri - Drago Nero Giovane}\textbf{Drago Nero Giovane}

\emph{Grande drago, caotico malvagio}

\textbf{FORZA} +4

\textbf{DESTREZZA} +2

\textbf{COSTITUZIONE} +3

\textbf{INTELLIGENZA} +1

\textbf{SAGGEZZA} +0

\textbf{CARISMA} +2

\textbf{Iniziativa} +2 -- \textbf{Difesa} 22

\textbf{Punti Ferita} 127 (15d10 + 45)

\textbf{Movimento} 12 m, scalata 12 m, volo 24 m

\textbf{Tiri Salvezza} Tempra +9, Riflessi +8, Volontà +7

\textbf{Competenze} Muoversi Silenziosamente / Nascondersi +5, Consapevolezza +6

\textbf{Immunità al Danno} acido

\textbf{Sensi} scurovisione 36 m, vista cieca 9 m

\textbf{Linguaggi} Comune, Draconico

\textbf{Sfida} 7 (2.900 PE)

\emph{\textbf{Anfibio.}} Il drago può respirare aria e acqua.

\textbf{Azioni}

\emph{\textbf{Multiattacco.}} Il drago può effettuare tre attacchi: uno con il morso e due con gli artigli.

\emph{\textbf{Artiglio.} Attacco con arma da mischia}: +10 a colpire, portata 1 m, un bersaglio.

\emph{Colpisce:} 11 (2d6 + 4) danni taglienti.

\emph{\textbf{Morso.} Attacco con arma da mischia}: +7 a colpire, portata 3 m, un bersaglio.

\emph{Colpisce:} 11 (2d10 + 4) danni perforanti più 4 (1d8) danni da acido.

\emph{\textbf{Soffio Acido (Ricarica 5-6).}} Il drago esala acido in una linea di 9 metri larga 1 metro. Ogni creatura in quell'area deve effettuare un Tiro Salvezza di Riflessi CD 14 e subire 49 (11d8) danni da acido se fallisce il Tiro Salvezza, o la metà di questi danni se lo riesce.

\textbf{Ecologia}\\
Ambiente: Paludi Calde\\
Organizzazione: Solitario\\
Tesoro: Triplo\\
\textbf{Descrizione}\\
Signori delle paludi e degli acquitrini più cupi, i draghi neri sono i padroni incontrastati del loro territorio, che dominano con crudeltà e infondendo terrore in chi abita nelle vicinanze. I draghi neri si stanziano nelle zone più remote delle paludi, specie in caverne sul fondo di pozze fetide e buie. Dentro, ammassano i loro luridi tesori e dormono tra radici e fango. I draghi neri amano il cibo un po' marcio e spesso lasciano un pasto a marcire in una pozza per giorni prima di consumarlo. I draghi neri preferiscono tesori che non si decompongono o degradano, accumulando tesori di monete, pietre preziose, gioielli e altri oggetti di pietra o metallo.\\
\textbf{Incantesimi}\index{Incantesimi da Drago Nero}\\
Il drago e' anche capace, se di eta' giovanile o piu' anziano, di lanciare incantesimi.\\
Puo' lanciare un numero di incantesimi pari al suo valore di Carisma, senza aver necessita' di componenti magici.\\
Non deve fare prove di magia, si considera che riesca sempre con un punteggio pari alla Difficoltà + il valore del Carisma.\\
Se deve fare un Tiro per Colpire ha CA = Grado di Sfida con un bonus al colpire pari alla Forza\\
Gli incantesimi a disposizione sono:\\
- Dito della morte\\
- Disintegrazione\\
- Blocca Mostri\\


\medskip\index{Mostri - Drago Nero Cucciolo}\textbf{Drago Nero Cucciolo}

\emph{Media drago, caotico malvagio}

\textbf{FORZA} +2

\textbf{DESTREZZA} +2

\textbf{COSTITUZIONE} +1

\textbf{INTELLIGENZA} +0

\textbf{SAGGEZZA} +0

\textbf{CARISMA} +1

\textbf{Iniziativa} +2 -- \textbf{Difesa} 18

\textbf{Punti Ferita} 33 (6d8 + 6)

\textbf{Movimento} 9 m, scalata 9 m, volo 18 m

\textbf{Tiri Salvezza} Tempra +2, Riflessi +2, Volontà +0

\textbf{Competenze} Muoversi Silenziosamente / Nascondersi +4, Consapevolezza +4

\textbf{Immunità al Danno} acido

\textbf{Sensi} scurovisione 18 m, vista cieca 3 m

\textbf{Linguaggi} Draconico

\textbf{Sfida} 2 (450 PE)

\emph{\textbf{Anfibio.}} Il drago può respirare aria e acqua.

\textbf{Azioni}

\emph{\textbf{Morso.} Attacco con arma da mischia}: +4 a colpire, portata 1 m, un bersaglio.

\emph{Colpisce:} 7 (1d10 + 2) danni perforanti più 2 (1d4) danni da acido.

\emph{\textbf{Soffio Acido (Ricarica 5-6).}} Il drago esala acido in una linea di 5 metri larga 1 metro. Ogni creatura in quell'area deve effettuare un Tiro Salvezza di Riflessi CD 11 e subire 22 (5d8) danni da acido se fallisce il Tiro Salvezza, o la metà di questi danni se lo riesce.

\textbf{Ecologia}\\
Ambiente: Paludi Calde\\
Organizzazione: Solitario\\
Tesoro: Triplo\\
\textbf{Descrizione}\\
Signori delle paludi e degli acquitrini più cupi, i draghi neri sono i padroni incontrastati del loro territorio, che dominano con crudeltà e infondendo terrore in chi abita nelle vicinanze. I draghi neri si stanziano nelle zone più remote delle paludi, specie in caverne sul fondo di pozze fetide e buie. Dentro, ammassano i loro luridi tesori e dormono tra radici e fango. I draghi neri amano il cibo un po' marcio e spesso lasciano un pasto a marcire in una pozza per giorni prima di consumarlo. I draghi neri preferiscono tesori che non si decompongono o degradano, accumulando tesori di monete, pietre preziose, gioielli e altri oggetti di pietra o metallo.\\


\medskip\index{Mostri - Drago Rosso Antico}\textbf{Drago Rosso Antico}

\emph{Mastodontica drago, caotico malvagio}

\textbf{FORZA} +10

\textbf{DESTREZZA} +0

\textbf{COSTITUZIONE} +9

\textbf{INTELLIGENZA} +4

\textbf{SAGGEZZA} +2

\textbf{CARISMA} +6

\textbf{Iniziativa} +4 -- \textbf{Difesa} 34

\textbf{Punti Ferita} 546 (28d20 + 252)

\textbf{Movimento} 12 m, scalata 12 m, volo 24 m

\textbf{Tiri Salvezza} Tempra +22, Riflessi +13, Volontà +21

\textbf{Competenze} Muoversi Silenziosamente / Nascondersi +7, Consapevolezza +16

\textbf{Immunità al Danno} fuoco, arma +1

\textbf{Sensi} scurovisione 36 m, vista cieca 18 m

\textbf{Linguaggi} Comune, Draconico

\textbf{Sfida} 24 (62.000 PE)

\emph{\textbf{Resistenza Leggendaria (3/Giorno).}} Se il drago fallisce un Tiro Salvezza, può scegliere invece di riuscire.

\textbf{Azioni}

\emph{\textbf{Multiattacco.}} Il drago può usare la sua Presenza Spaventosa e poi effettuare tre attacchi: uno con il morso e due con gli artigli.

\emph{\textbf{Artiglio.} Attacco con arma da mischia}: +17 a colpire, portata 3 m, un bersaglio.

\emph{Colpisce:} 17 (2d6 + 10) danni taglienti.

\emph{\textbf{Coda.} Attacco con arma da mischia}: +17 a colpire, portata 6 m, un bersaglio.

\emph{Colpisce:} 19 (2d8 + 10) danni da botta.

\emph{\textbf{Morso.} Attacco con arma da mischia}: +17 a colpire, portata 5 metri, un bersaglio.

\emph{Colpisce:} 21 (2d10 + 10) danni perforanti più 14 (4d6) danni da fuoco.

\emph{\textbf{Presenza Spaventosa.}} Ogni creatura scelta dal drago, che si trovi entro 36 metri da esso e consapevole della sua presenza, deve riuscire un Tiro Salvezza di Volontà CD 21 o restare spaventata per 1 minuto. Una creatura può ripetere il Tiro Salvezza al termine di ciascun suo turno, terminando l'effetto se lo riesce. Se il Tiro Salvezza della creatura ha successo o l'effetto ha termine per essa, la creatura è immune alla Presenza Spaventosa del drago per le successive 24 ore.

\emph{\textbf{Soffio Infuocato (Ricarica 5-6).}} Il drago esala fuoco in un cono di 27 metri. Ogni creatura in quell'area deve effettuare un Tiro Salvezza su Riflessi CD 24 e subire 91 (26d6) danni da fuoco se fallisce il Tiro Salvezza, o la metà di questi danni se lo riesce.

\textbf{Azioni Aggiuntive}

Il drago può effettuare 3 Azioni aggiuntive, scelte tra le opzioni seguenti. Può usare solo un'opzione leggendaria alla volta e solo al termine del turno di un'altra creatura. Il drago recupera le Azioni aggiuntive spese all'inizio del proprio turno.

\textbf{Attacco di Ala (Costa 2 Azioni).} Il drago batte le ali. Ogni creatura entro 5 metri dal drago deve riuscire un Tiro Salvezza su Riflessi CD 25 o subire 17 (2d6 + 10) danni da botta e venir gettato prono. Il drago può poi volare fino a metà del suo movimento di volo.

\textbf{Attacco di Coda.} Il drago effettua un attacco di coda.

\textbf{Individuare.} Il drago effettua una prova di Saggezza (Consapevolezza).

\textbf{Ecologia}\\
Ambiente: Montagne calde\\
Organizzazione: Solitario\\
Tesoro: Triplo\\
\textbf{Descrizione}\\
Poche creature sono più crudeli e terribili del possente drago rosso. Sovrano dei cromatici, il terribile drago rosso porta rovina e morte nelle terre minacciate dalla sua presenza.\\
\textbf{Incantesimi}\index{Incantesimi da Drago Rosso}\\
Il drago e' anche capace, se di eta' giovanile o piu' anziano, di lanciare incantesimi.\\
Puo' lanciare un numero di incantesimi pari al suo valore di Carisma, senza aver necessita' di componenti magici.\\
Non deve fare prove di magia, si considera che riesca sempre con un punteggio pari alla Difficoltà + il valore del Carisma.\\
Se deve fare un Tiro per Colpire ha CA = Grado di Sfida con un bonus al colpire pari alla Forza\\
Gli incantesimi a disposizione sono:\\
- Palla di Fuoco\\
- Nube Incendiaria\\
- Muro di fuoco\\


\medskip\index{Mostri - Drago Rosso Adulto}\textbf{Drago Rosso Adulto}

\emph{Enorme drago, caotico malvagio}

\textbf{FORZA} +8

\textbf{DESTREZZA} +0

\textbf{COSTITUZIONE} +7

\textbf{INTELLIGENZA} +3

\textbf{SAGGEZZA} +1

\textbf{CARISMA} +5

\textbf{Iniziativa} +3 -- \textbf{Difesa} 28

\textbf{Punti Ferita} 256 (19d12 + 133) 

\textbf{Movimento} 12 m, scalata 12 m, volo 24 m

\textbf{Tiri Salvezza} Tempra +16, Riflessi +10, Volontà +15

\textbf{Competenze} Muoversi Silenziosamente / Nascondersi +6, Consapevolezza +13

\textbf{Immunità al Danno} fuoco

\textbf{Sensi} scurovisione 36 m, vista cieca 18 m 

\textbf{Linguaggi} Comune, Draconico

\textbf{Sfida} 17 (18.000 PE)

\emph{\textbf{Resistenza Leggendaria (3/Giorno).}} Se il drago fallisce un Tiro Salvezza, può scegliere invece di riuscire.

\textbf{Azioni}

\emph{\textbf{Multiattacco.}} Il drago può usare la sua Presenza Spaventosa e poi effettuare tre attacchi: uno con il morso e due con gli artigli.

\emph{\textbf{Artiglio.} Attacco con arma da mischia}: +14 a colpire, portata 1 m, un bersaglio.

\emph{Colpisce:} 15 (2d6 + 8) danni taglienti.

\emph{\textbf{Coda.} Attacco con arma da mischia}: +14 a colpire, portata 5 metri, un bersaglio.

\emph{Colpisce:} 17 (2d8 + 8) danni da botta.

\emph{\textbf{Morso.} Attacco con arma da mischia}: +14 a colpire, portata 3 m, un bersaglio.

\emph{Colpisce:} 19 (2d10 + 8) danni perforanti più 7 (2d6) danni da
fuoco.

\emph{\textbf{Presenza Spaventosa.}} Ogni creatura scelta dal drago, che si trovi entro 36 metri da esso e consapevole della sua presenza, deve riuscire un Tiro Salvezza di Volontà CD 19 o restare spaventata per 1 minuto. Una creatura può ripetere il Tiro Salvezza al termine di ciascun suo turno, terminando l'effetto se lo riesce. Se il Tiro Salvezza della creatura ha successo o l'effetto ha termine per essa, la creatura è  immune alla Presenza Spaventosa del drago per le successive 24 ore.

\emph{\textbf{Soffio Infuocato (Ricarica 5-6).}} Il drago esala fuoco in un cono di 18 metri. Ogni creatura in quell'area deve effettuare un Tiro Salvezza su Riflessi CD 21 e subire 63 (18d6) danni da fuoco se fallisce il Tiro Salvezza, o la metà di questi danni se lo riesce.

\textbf{Azioni Aggiuntive}

Il drago può effettuare 3 Azioni aggiuntive, scelte tra le opzioni seguenti. Può usare solo un'opzione leggendaria alla volta e solo al termine del turno di un'altra creatura. Il drago recupera le Azioni aggiuntive spese all'inizio del proprio turno.

\textbf{Attacco di Ala (Costa 2 Azioni).} Il drago batte le ali. Ogni creatura entro 3 metri dal drago deve riuscire un Tiro Salvezza su Riflessi CD 22 o subire 15 (2d6 + 8) danni da botta e venir gettato prono. Il drago può poi volare fino a metà del suo movimento di volo. 

\textbf{Attacco di Coda.} Il drago effettua un attacco di coda. 

\textbf{Individuare.} Il drago effettua una prova di Saggezza (Consapevolezza).

\textbf{Ecologia}\\
Ambiente: Montagne calde\\
Organizzazione: Solitario\\
Tesoro: Triplo\\
\textbf{Descrizione}\\
Poche creature sono più crudeli e terribili del possente drago rosso. Sovrano dei cromatici, il terribile drago rosso porta rovina e morte nelle terre minacciate dalla sua presenza.\\
\textbf{Incantesimi}\index{Incantesimi da Drago Rosso}\\
Il drago e' anche capace, se di eta' giovanile o piu' anziano, di lanciare incantesimi.\\
Puo' lanciare un numero di incantesimi pari al suo valore di Carisma, senza aver necessita' di componenti magici.\\
Non deve fare prove di magia, si considera che riesca sempre con un punteggio pari alla Difficoltà + il valore del Carisma.\\
Se deve fare un Tiro per Colpire ha CA = Grado di Sfida con un bonus al colpire pari alla Forza\\
Gli incantesimi a disposizione sono:\\
- Palla di Fuoco\\
- Nube Incendiaria\\
- Muro di fuoco\\


\medskip\index{Mostri - Drago Rosso Giovane}\textbf{Drago Rosso Giovane}

\emph{Grande drago, caotico malvagio}

\textbf{FORZA} +6

\textbf{DESTREZZA} +0

\textbf{COSTITUZIONE} +5

\textbf{INTELLIGENZA} +2

\textbf{SAGGEZZA} +0

\textbf{CARISMA} +4

\textbf{Iniziativa} +2 -- \textbf{Difesa} 23

\textbf{Punti Ferita} 178 (17d10 + 85)

\textbf{Movimento} 12 m, scalata 12 m, volo 24 m

\textbf{Tiri Salvezza} Tempra +11, Riflessi +8, Volontà +10

\textbf{Competenze} Muoversi Silenziosamente / Nascondersi +4, Consapevolezza +8

\textbf{Immunità al Danno} fuoco

\textbf{Sensi} scurovisione 36 m, vista cieca 9 m

\textbf{Linguaggi} Comune, Draconico

\textbf{Sfida} 10 (5.900 PE)

\textbf{Azioni}

\emph{\textbf{Multiattacco.}} Il drago può effettuare tre attacchi: uno con il morso e due con gli artigli.

\emph{\textbf{Artiglio.} Attacco con arma da mischia}: +10 a colpire, portata 1 m, un bersaglio.

\emph{Colpisce:} 13 (2d6 + 6) danni taglienti.

\emph{\textbf{Morso.} Attacco con arma da mischia}: +10 a colpire, portata 3 m, un bersaglio.

\emph{Colpisce:} 17 (2d10 + 6) danni perforanti più 3 (1d6) danni da fuoco.

\emph{\textbf{Soffio Infuocato (Ricarica 5-6).}} Il drago esala fuoco in un cono di 9 metri. Ogni creatura in quell'area deve effettuare un Tiro Salvezza su Riflessi CD 17 e subire 56 (16d6) danni da fuoco se fallisce il Tiro Salvezza, o la metà di questi danni se lo riesce.

\textbf{Ecologia}\\
Ambiente: Montagne calde\\
Organizzazione: Solitario\\
Tesoro: Triplo\\
\textbf{Descrizione}\\
Poche creature sono più crudeli e terribili del possente drago rosso. Sovrano dei cromatici, il terribile drago rosso porta rovina e morte nelle terre minacciate dalla sua presenza.\\
\textbf{Incantesimi}\index{Incantesimi da Drago Rosso}\\
Il drago e' anche capace, se di eta' giovanile o piu' anziano, di lanciare incantesimi.\\
Puo' lanciare un numero di incantesimi pari al suo valore di Carisma, senza aver necessita' di componenti magici.\\
Non deve fare prove di magia, si considera che riesca sempre con un punteggio pari alla Difficoltà + il valore del Carisma.\\
Se deve fare un Tiro per Colpire ha CA = Grado di Sfida con un bonus al colpire pari alla Forza\\
Gli incantesimi a disposizione sono:\\
- Palla di Fuoco\\
- Nube Incendiaria\\
- Muro di fuoco\\


\medskip\index{Mostri - Drago Rosso Cucciolo}\textbf{Drago Rosso Cucciolo}

\emph{Media drago, caotico malvagio}

\textbf{FORZA} +4

\textbf{DESTREZZA} +0

\textbf{COSTITUZIONE} +3

\textbf{INTELLIGENZA} +1

\textbf{SAGGEZZA} +0

\textbf{CARISMA} +2

\textbf{Iniziativa} +1 -- \textbf{Difesa} 19

\textbf{Punti Ferita} 75 (10d8 + 30)

\textbf{Movimento} 9 m, scalata 9 m, volo 18 m

\textbf{Tiri Salvezza} Tempra +4, Riflessi +3, Volontà +1

\textbf{Competenze} Muoversi Silenziosamente / Nascondersi +2, Consapevolezza +4

\textbf{Immunità al Danno} fuoco

\textbf{Sensi} scurovisione 18 m, vista cieca 3 m

\textbf{Linguaggi} Draconico

\textbf{Sfida} 4 (1.100 PE)

\textbf{Azioni}

\emph{\textbf{Morso.} Attacco con arma da mischia}: +6 a colpire, portata 1 m, un bersaglio.

\emph{Colpisce:} 9 (1d10 + 4) danni perforanti più 3 (1d6) danni da fuoco.

\emph{\textbf{Soffio Infuocato (Ricarica 5-6).}} Il drago esala fuoco in un cono di 5 metri. Ogni creatura in quell'area deve effettuare un Tiro Salvezza di Riflessi CD 13 e subire 24 (7d6) danni da fuoco se fallisce il Tiro Salvezza, o la metà di questi danni se lo riesce.

\textbf{Ecologia}\\
Ambiente: Montagne calde\\
Organizzazione: Solitario\\
Tesoro: Triplo\\
\textbf{Descrizione}\\
Poche creature sono più crudeli e terribili del possente drago rosso. Sovrano dei cromatici, il terribile drago rosso porta rovina e morte nelle terre minacciate dalla sua presenza.\\


\medskip\index{Mostri - Drago Verde Antico}\textbf{Drago Verde Antico}

\emph{Mastodontica drago, legale malvagio}

\textbf{FORZA} +8

\textbf{DESTREZZA} +1

\textbf{COSTITUZIONE} +7

\textbf{INTELLIGENZA} +5

\textbf{SAGGEZZA} +3

\textbf{CARISMA} +4

\textbf{Iniziativa} +5 -- \textbf{Difesa} 32

\textbf{Punti Ferita} 385 (22d20 + 154) 

\textbf{Movimento} 12 m, nuoto 12 m, volo 24 m

\textbf{Tiri Salvezza} Tempra +20, Riflessi +12, Volontà +20

\textbf{Competenze} Muoversi Silenziosamente / Nascondersi +8, Ingannare +11, Percepire Emozioni +10, Consapevolezza + 15

\textbf{Immunità al Danno} veleno , arma +1

\textbf{Immunità alle Condizioni}
avvelenato

\textbf{Sensi} scurovisione 36 m, vista cieca 18 m

\textbf{Linguaggi} Comune, Draconico 

\textbf{Sfida} 22 (41.000 PE)

\emph{\textbf{Anfibio.}} Il drago può respirare aria e acqua.

\emph{\textbf{Resistenza Leggendaria (3/Giorno).}} Se il drago fallisce un Tiro Salvezza, può scegliere invece di riuscire.

\textbf{Azioni}

\emph{\textbf{Multiattacco.}} Il drago può usare la sua Presenza Spaventosa. Poi effettuare tre attacchi: uno con il morso e due con gli artigli.

\emph{\textbf{Artiglio.} Attacco con arma da mischia}: +15 a colpire, portata 3 m, un bersaglio.

\emph{Colpisce:} 15 (2d6 + 8) danni taglienti.

\emph{\textbf{Coda.} Attacco con arma da mischia}: +15 a colpire, portata 6 m, un bersaglio.

\emph{Colpisce:} 17 (2d8 + 8) danni da botta.

\emph{\textbf{Morso.} Attacco con arma da mischia}: +15 a colpire, portata 5 metri, un bersaglio.

\emph{Colpisce:} 19 (2d10 + 8) danni perforanti più 10 (3d6) danni da veleno.

\emph{\textbf{Presenza Spaventosa.}} Ogni creatura scelta dal drago, che si trovi entro 36 metri da esso e consapevole della sua presenza, deve riuscire un Tiro Salvezza di Volontà CD 19 o restare spaventata per 1 minuto. Una creatura può ripetere il Tiro Salvezza al termine di ciascun suo turno, terminando l'effetto se lo riesce. Se il Tiro Salvezza della creatura ha successo o l'effetto ha termine per essa, la creatura è immune alla Presenza Spaventosa del drago per le successive 24 ore.

\emph{\textbf{Soffio Velenoso (Ricarica 5-6).}} Il drago esala gas velenosi in un cono di 27 metri. Ogni creatura in quell'area deve effettuare un Tiro Salvezza di Tempra CD 22 e subire 77 (22d6) danni da veleno se fallisce il Tiro Salvezza, o la metà di questi danni se lo riesce.

\textbf{Azioni Aggiuntive}

Il drago può effettuare 3 Azioni aggiuntive, scelte tra le opzioni seguenti. Può usare solo un'opzione leggendaria alla volta e solo al termine del turno di un'altra creatura. Il drago recupera le Azioni aggiuntive spese all'inizio del proprio turno.

\textbf{Attacco di Ala (Costa 2 Azioni).} Il drago batte le ali. Ogni creatura entro 5 metri dal drago deve riuscire un Tiro Salvezza su Riflessi CD 23 o subire 15 (2d6 + 8) danni da botta e venire gettato prono. Il drago può poi volare fino a metà del suo movimento di volo.

\textbf{Attacco di Coda.} Il drago effettua un attacco di coda.

\textbf{Individuare.} Il drago effettua una prova di Saggezza (Consapevolezza).

\textbf{Ecologia}\\
Ambiente: Foreste Temperate\\
Organizzazione: Solitario\\
Tesoro: Triplo\\
\textbf{Descrizione}\\
I draghi verdi vivono nelle antiche foreste del mondo, vagando in cerca di preda sotto giganteschi tetti di foglie. Di tutti i draghi cromatici, i draghi verdi sono forse quelli con cui ci si può più facilmente accordare diplomaticamente.\\
\textbf{Incantesimi}\index{Incantesimi da Drago Verde}\\
Il drago e' anche capace, se di eta' giovanile o piu' anziano, di lanciare incantesimi.\\
Puo' lanciare un numero di incantesimi pari al suo valore di Carisma, senza aver necessita' di componenti magici.\\
Non deve fare prove di magia, si considera che riesca sempre con un punteggio pari alla Difficoltà + il valore del Carisma.\\
Se deve fare un Tiro per Colpire ha CA = Grado di Sfida con un bonus al colpire pari alla Forza.\\
Gli incantesimi a disposizione sono:\\
- Nube Mortale\\
- Terreno illusorio\\
- Rimuovi veleno\\


\medskip\index{Mostri - Drago Verde Adulto}\textbf{Drago Verde Adulto}

\emph{Enorme drago, legale malvagio}

\textbf{FORZA} +6

\textbf{DESTREZZA} +1

\textbf{COSTITUZIONE} +5

\textbf{INTELLIGENZA} +4

\textbf{SAGGEZZA} +2

\textbf{CARISMA} +3

\textbf{Iniziativa} +4 -- \textbf{Difesa} 27

\textbf{Punti Ferita} 207 (18d12 + 90)

\textbf{Movimento} 12 m, nuoto 12 m, volo 24 m

\textbf{Tiri Salvezza} Tempra +14, Riflessi +9, Volontà +14

\textbf{Competenze} Muoversi Silenziosamente / Nascondersi +6, Ingannare +8, Percepire Emozioni +7, Consapevolezza +12

\textbf{Immunità al Danno} veleno

\textbf{Immunità alle Condizioni} avvelenato

\textbf{Sensi} scurovisione 36 m, vista cieca 18 m

\textbf{Linguaggi} Comune, Draconico

\textbf{Sfida} 15 (13.000 PE)

\emph{\textbf{Anfibio.}} Il drago può respirare aria e acqua.

\emph{\textbf{Resistenza Leggendaria (3/Giorno).}} Se il drago fallisce un Tiro Salvezza, può scegliere invece di riuscire.

\textbf{Azioni}

\emph{\textbf{Multiattacco.}} Il drago può usare la sua Presenza Spaventosa. Poi effettuare tre attacchi: uno con il morso e due con gli artigli.

\emph{\textbf{Artiglio.} Attacco con arma da mischia}: +11 a colpire, portata 1 m, un bersaglio.

\emph{Colpisce:} 13 (2d6 + 6) danni taglienti.

\emph{\textbf{Coda.} Attacco con arma da mischia}: +11 a colpire, portata 5 metri, un bersaglio.

\emph{Colpisce:} 15 (2d8 + 6) danni da botta.

\emph{\textbf{Morso.} Attacco con arma da mischia}: +11 a colpire, portata 3 m, un bersaglio.

\emph{Colpisce:} 17 (2d10 + 6) danni perforanti più 7 (2d6) danni da veleno.

\emph{\textbf{Presenza Spaventosa.}} Ogni creatura scelta dal drago, che si trovi entro 36 metri da esso e consapevole della sua presenza, deve riuscire un Tiro Salvezza di Volontà CD 16 o restare spaventata per 1 minuto. Una creatura può ripetere il Tiro Salvezza al termine di ciascun suo turno, terminando l'effetto se lo riesce. Se il Tiro Salvezza della creatura ha successo o l'effetto ha termine per essa, la creatura è immune alla Presenza Spaventosa del drago per le successive 24 ore.

\emph{\textbf{Soffio Velenoso (Ricarica 5-6).}} Il drago esala gas velenosi in un cono di 18 metri. Ogni creatura in quell'area deve effettuare un Tiro Salvezza di Tempra CD 18 e subire 56 (16d6) danni da veleno se fallisce il Tiro Salvezza, o la metà di questi danni se lo riesce.

\textbf{Azioni Aggiuntive}

Il drago può effettuare 3 Azioni aggiuntive, scelte tra le opzioni seguenti. Può usare solo un'opzione leggendaria alla volta e solo al termine del turno di un'altra creatura. Il drago recupera le Azioni aggiuntive spese all'inizio del proprio turno.

\textbf{Attacco di Ala (Costa 2 Azioni).} Il drago batte le ali. Ogni creatura entro 3 metri dal drago deve riuscire un Tiro Salvezza su Riflessi CD 19 o subire 13 (2d6 + 6) danni da botta e venir  gettato prono. Il drago può poi volare fino a metà del suo movimento di volo.

\textbf{Attacco di Coda.} Il drago effettua un attacco di coda.

\textbf{Individuare.} Il drago effettua una prova di Saggezza (Consapevolezza).

\textbf{Ecologia}\\
Ambiente: Foreste Temperate\\
Organizzazione: Solitario\\
Tesoro: Triplo\\
\textbf{Descrizione}\\
I draghi verdi vivono nelle antiche foreste del mondo, vagando in cerca di preda sotto giganteschi tetti di foglie. Di tutti i draghi cromatici, i draghi verdi sono forse quelli con cui ci si può più facilmente accordare diplomaticamente.\\
\textbf{Incantesimi}\index{Incantesimi da Drago Verde}\\
Il drago e' anche capace, se di eta' giovanile o piu' anziano, di lanciare incantesimi.\\
Puo' lanciare un numero di incantesimi pari al suo valore di Carisma, senza aver necessita' di componenti magici.\\
Non deve fare prove di magia, si considera che riesca sempre con un punteggio pari alla Difficoltà + il valore del Carisma.\\
Se deve fare un Tiro per Colpire ha CA = Grado di Sfida con un bonus al colpire pari alla Forza\\
Gli incantesimi a disposizione sono:\\
- Nube Mortale\\
- Terreno illusorio\\
- Rimuovi veleno\\


\medskip\index{Mostri - Drago Verde Giovane}\textbf{Drago Verde Giovane}

\emph{Grande drago, legale malvagio}

\textbf{FORZA} +4

\textbf{DESTREZZA} +1

\textbf{COSTITUZIONE} +3

\textbf{INTELLIGENZA} +3

\textbf{SAGGEZZA} +1

\textbf{CARISMA} +2

\textbf{Iniziativa} +3 -- \textbf{Difesa} 22

\textbf{Punti Ferita} 136 (16d10 + 48)

\textbf{Movimento} 12 m, nuoto 12 m, volo 24 m

\textbf{Tiri Salvezza} Tempra +9, Riflessi +7, Volontà +9

\textbf{Competenze} Muoversi Silenziosamente / Nascondersi +4, Ingannare +5, Consapevolezza +7

\textbf{Immunità al Danno} veleno

\textbf{Immunità alle Condizioni} avvelenato

\textbf{Sensi} scurovisione 36 m, vista cieca 9 m
\textbf{Linguaggi} Comune, Draconico

\textbf{Sfida} 8 (3.900 PE)

\emph{\textbf{Anfibio.}} Il drago può respirare aria e acqua.

\textbf{Azioni}

\emph{\textbf{Multiattacco.}} Il drago può effettuare tre attacchi: uno con il morso e due con gli artigli. 

\emph{\textbf{Artiglio.} Attacco con arma da mischia}: +7 a colpire, portata 1 m, un bersaglio.

\emph{Colpisce:} 11 (2d6 + 4) danni taglienti.

\emph{\textbf{Morso.} Attacco con arma da mischia}: +7 a colpire, portata 3 m, un bersaglio.

\emph{Colpisce:} 15 (2d10 + 4) danni perforanti più 7 (2d6) danni da veleno.

\emph{\textbf{Soffio Velenoso (Ricarica 5-6).}} Il drago esala gas velenosi in un cono di 9 metri. Ogni creatura in quell'area deve effettuare un Tiro Salvezza di Tempra CD 14 e subire 42 (12d6) danni da veleno se fallisce il Tiro Salvezza, o la metà di questi danni se lo riesce.

\textbf{Ecologia}\\
Ambiente: Foreste Temperate\\
Organizzazione: Solitario\\
Tesoro: Triplo\\
\textbf{Descrizione}\\
I draghi verdi vivono nelle antiche foreste del mondo, vagando in cerca di preda sotto giganteschi tetti di foglie. Di tutti i draghi cromatici, i draghi verdi sono forse quelli con cui ci si può più facilmente accordare diplomaticamente.\\
\textbf{Incantesimi}\index{Incantesimi da Drago Verde}\\
Il drago e' anche capace, se di eta' giovanile o piu' anziano, di lanciare incantesimi.\\
Puo' lanciare un numero di incantesimi pari al suo valore di Carisma, senza aver necessita' di componenti magici.\\
Non deve fare prove di magia, si considera che riesca sempre con un punteggio pari alla Difficoltà + il valore del Carisma.\\
Se deve fare un Tiro per Colpire ha CA = Grado di Sfida con un bonus al colpire pari alla Forza\\
Gli incantesimi a disposizione sono:\\
- Nube Mortale\\
- Terreno illusorio\\
- Rimuovi veleno\\

\medskip\index{Mostri - Drago Verde Cucciolo}\textbf{Drago Verde Cucciolo}

\emph{Media drago, legale malvagio}

\textbf{FORZA} +2

\textbf{DESTREZZA} +1

\textbf{COSTITUZIONE} +1

\textbf{INTELLIGENZA} +2

\textbf{SAGGEZZA} +0

\textbf{CARISMA} +1

\textbf{Iniziativa} +2 -- \textbf{Difesa} 18

\textbf{Punti Ferita} 38 (7d8 + 7)

\textbf{Movimento} 9 m, nuoto 9 m, volo 18 m

\textbf{Tiri Salvezza} Tempra +3, Riflessi +1, Volontà +0

\textbf{Competenze} Muoversi Silenziosamente / Nascondersi +3, Consapevolezza +4 

\textbf{Immunità al Danno} veleno 

\textbf{Immunità alle Condizioni} avvelenato

\textbf{Sensi} scurovisione 18 m, vista cieca 3 m

\textbf{Linguaggi} Draconico

\textbf{Sfida} 2 (450 PE)

\emph{\textbf{Anfibio.}} Il drago può respirare aria e acqua.

\textbf{Azioni}

\emph{\textbf{Morso.} Attacco con arma da mischia}: +4 a colpire, portata 1 m, un bersaglio.

\emph{Colpisce:} 7 (1d10 + 2) danni perforanti più 3 (1d6) danni da veleno.

\emph{\textbf{Soffio Velenoso (Ricarica 5-6).}} Il drago esala gas velenosi in un cono di 5 metri. Ogni creatura in quell'area deve effettuare un Tiro Salvezza di Tempra CD 11 e subire 21 (6d6) danni da veleno se fallisce il Tiro Salvezza, o la metà di questi danni se lo riesce.

\textbf{Ecologia}\\
Ambiente: Foreste Temperate\\
Organizzazione: Solitario\\
Tesoro: Triplo\\
\textbf{Descrizione}\\
I draghi verdi vivono nelle antiche foreste del mondo, vagando in cerca di preda sotto giganteschi tetti di foglie. Di tutti i draghi cromatici, i draghi verdi sono forse quelli con cui ci si può più facilmente accordare diplomaticamente.\\


\subsection{Draghi Metallici}

\medskip\index{Mostri - Drago d'Argento Antico}\textbf{Drago d'Argento Antico}

\emph{Mastodontica drago, legale buono}

\textbf{FORZA} +10

\textbf{DESTREZZA} +0

\textbf{COSTITUZIONE} +9

\textbf{INTELLIGENZA} +4

\textbf{SAGGEZZA} +2

\textbf{CARISMA} +6

\textbf{Iniziativa} +4 -- \textbf{Difesa} 34

\textbf{Punti Ferita} 487 (25d20 + 225)

\textbf{Movimento} 12 m, volo 24 m

\textbf{Tiri Salvezza} Tempra +21, Riflessi +15, Volontà +23

\textbf{Competenze} Arcano +11, Muoversi Silenziosamente / Nascondersi +7, Consapevolezza +16, Storia +11

\textbf{Immunità al Danno} freddo, arma +1

\textbf{Sensi} scurovisione 36 m, vista cieca 18 m

\textbf{Linguaggi} Comune, Draconico

\textbf{Sfida} 23 (50.000 PE)

\emph{\textbf{Resistenza Leggendaria (3/Giorno).}} Se il drago fallisce un Tiro Salvezza, può scegliere invece di riuscire.

\textbf{Azioni}

\emph{\textbf{Multiattacco.}} Il drago può usare la sua Presenza Spaventosa. Poi effettuare tre attacchi: uno con il morso e due con gli
artigli.

\emph{\textbf{Artiglio.} Attacco con arma da mischia}: +17 a colpire, portata 3 m, un bersaglio.

\emph{Colpisce:} 17 (2d6 + 10) danni taglienti.

\emph{\textbf{Coda.} Attacco con arma da mischia}: +17 a colpire, portata 6 m, un bersaglio.

\emph{Colpisce:} 19 (2d8 + 10) danni da botta.

\emph{\textbf{Morso.} Attacco con arma da mischia}: +17 a colpire, portata 5 metri, un bersaglio.

\emph{Colpisce:} 21 (2d10 + 10) danni perforanti.

\emph{\textbf{Presenza Spaventosa.}} Ogni creatura scelta dal drago, che si trovi entro 36 metri da esso e consapevole della sua presenza, deve riuscire un Tiro Salvezza di Volontà CD 21 o restare spaventata per 1 minuto. Una creatura può ripetere il Tiro Salvezza al termine di ciascun suo turno, terminando l'effetto se lo riesce. Se il Tiro Salvezza della creatura ha successo o l'effetto ha termine per essa, la creatura è immune alla Presenza Spaventosa del drago per le successive 24 ore.

\emph{\textbf{Arma a Soffio (Ricarica 5-6).}} Il drago usa una delle seguenti armi a soffio:

\emph{Soffio Gelido.} Il drago esala un'esplosione ghiacciata in un cono di 27 metri. Ogni creatura nell'area deve effettuare un Tiro Salvezza su Tempra CD 24, subendo 67 (15d8) danni da freddo se fallisce il Tiro Salvezza, o la metà di questi danni se lo riesce.

\emph{Soffio Paralizzante.} Il drago esala un gas paralizzante in un cono di 24 metri. Ogni creatura nell'area deve riuscire un Tiro Salvezza su Tempra 24 o restare paralizzata per 1 minuto. Una creatura può ripetere il Tiro Salvezza al termine di ciascun suo turno, terminando l'effetto per sé in caso di successo.

\emph{\textbf{Mutare Forma.}} Il drago può trasformarsi magicamente in un umanoide o bestia il cui grado di sfida sia pari o inferiore al proprio, o tornare alla sua vera forma. Alla morte ritorna alla sua vera forma. 

Qualsiasi equipaggiamento stia indossando o trasportando viene assorbito o trasportato nella nuova forma (a scelta del drago).

Nella nuova forma, il drago mantiene il suo allineamento, punti ferita, Dadi Vita, la facoltà di parlare, le competenze, la Resistenza Leggendaria, le azioni da tana, e i punteggi di Intelligenza, Saggezza  e Carisma, oltre a questa azione. Le sue statistiche e capacità vengono altrimenti rimpiazzate da quelle della nuova forma, eccetto Azioni aggiuntive della nuova forma.

\textbf{Azioni Aggiuntive}

Il drago può effettuare 3 Azioni aggiuntive, scelte tra le opzioni seguenti. Può usare solo un'opzione leggendaria alla volta e solo al termine del turno di un'altra creatura. Il drago recupera le  Azioni aggiuntive spese all'inizio del proprio turno.

\textbf{Attacco di Ala (Costa 2 Azioni).} Il drago batte le ali. Ogni  creatura entro 5 metri dal drago deve riuscire un Tiro Salvezza  su Riflessi CD 25 o subire 17 (2d6 + 10) danni da botta e  venir gettato prono. Il drago può poi volare fino a metà della sua  velocità di volo.

\textbf{Attacco di Coda.} Il drago effettua un attacco di coda.

\textbf{Individuare.} Il drago effettua una prova di Saggezza (Consapevolezza).

\textbf{Ecologia}\\
Ambiente: Montagne Temperate\\
Organizzazione: Solitario\\
Tesoro: Triplo\\
\textbf{Descrizione}\\
Tra tutti i draghi, quelli d'argento sono i più coraggiosi, e si attengono ad un codice cavalleresco che impone loro di aiutare i deboli, sconfiggere il male e comportarsi in modo onorevole.\\
\textbf{Incantesimi}\index{Incantesimi da Drago Argento}\\
Il drago e' anche capace, se di eta' giovanile o piu' anziano, di lanciare incantesimi.\\
Puo' lanciare un numero di incantesimi pari al suo valore di Carisma, senza aver necessita' di componenti magici.\\
Non deve fare prove di magia, si considera che riesca sempre con un punteggio pari alla Difficoltà + il valore del Carisma.\\
Se deve fare un Tiro per Colpire ha CA = Grado di Sfida con un bonus al colpire pari alla Forza\\
Gli incantesimi a disposizione sono:\\
- Cono di freddo\\
- Tempesta di ghiaccio\\
- Creazione\\
- Capanna\\

\medskip\index{Mostri - Drago d'Argento Adulto}\textbf{Drago d'Argento Adulto}

\emph{Enorme drago, legale buono}

\textbf{FORZA} +8

\textbf{DESTREZZA} +0

\textbf{COSTITUZIONE} +7

\textbf{INTELLIGENZA} +3

\textbf{SAGGEZZA} +1

\textbf{CARISMA} +5

\textbf{Iniziativa} +3 -- \textbf{Difesa} 27

\textbf{Punti Ferita} 243 (18d12 + 126)

\textbf{Movimento} 12 m, volo 24 m

\textbf{Tiri Salvezza} Tempra +15, Riflessi +12, Volontà +17

\textbf{Competenze} Arcano +8, Muoversi Silenziosamente / Nascondersi +5, Consapevolezza +11, Storia +8

\textbf{Immunità al Danno} freddo

\textbf{Sensi} scurovisione 36 m, vista cieca 18 m 

\textbf{Linguaggi} Comune, Draconico

\textbf{Sfida} 16 (15.000 PE)

\emph{\textbf{Resistenza Leggendaria (3/Giorno).}} Se il drago fallisce un Tiro Salvezza, può scegliere invece di riuscire.

\textbf{Azioni}

\emph{\textbf{Multiattacco.}} Il drago può usare la sua Presenza Spaventosa. Poi effettuare tre attacchi: uno con il morso e due con gli artigli.

\emph{\textbf{Artiglio.} Attacco con arma da mischia}: +13 a colpire, portata 1 m, un bersaglio.

\emph{Colpisce:} 15 (2d6 + 8) danni taglienti.

\emph{\textbf{Coda.} Attacco con arma da mischia}: +13 a colpire, portata 5 metri, un bersaglio.

\emph{Colpisce:} 17 (2d8 + 8) danni da botta.

\emph{\textbf{Morso.} Attacco con arma da mischia}: +13 a colpire, portata 3 m, un bersaglio.

\emph{Colpisce:} 19 (2d10 + 8) danni perforanti.

\emph{\textbf{Presenza Spaventosa.}} Ogni creatura scelta dal drago, che si trovi entro 36 metri da esso e consapevole della sua presenza, deve riuscire un Tiro Salvezza di Volontà CD 18 o restare spaventata per 1 minuto. Una creatura può ripetere il Tiro Salvezza al termine di ciascun suo turno, terminando l'effetto se lo riesce. Se il Tiro Salvezza della creatura ha successo o l'effetto ha termine per essa, la creatura è immune alla Presenza Spaventosa del drago per le successive 24 ore.

\emph{\textbf{Arma a Soffio (Ricarica 5-6).}} Il drago usa una delle seguenti armi a soffio:

\emph{Soffio Gelido.} Il drago esala un'esplosione ghiacciata in un cono di 18 metri. Ogni creatura nell'area deve effettuare un Tiro Salvezza su Tempra CD 20, subendo 58 (13d8) danni da freddo se fallisce il Tiro Salvezza, o la metà di questi danni se lo riesce.

\emph{Soffio Paralizzante.} Il drago esala un gas paralizzante in un cono di 18 metri. Ogni creatura nell'area deve riuscire un Tiro Salvezza su Tempra 20 o restare paralizzata per 1 minuto. Una creatura può ripetere il Tiro Salvezza al termine di ciascun suo turno, terminando l'effetto per sé in caso di successo.

\emph{\textbf{Mutare Forma.}} Il drago può trasformarsi magicamente in un umanoide o bestia il cui grado di sfida sia pari o inferiore al proprio, o tornare alla sua vera forma. Alla morte ritorna alla sua vera forma. Qualsiasi equipaggiamento stia indossando o trasportando viene assorbito o trasportato nella nuova forma (a scelta del drago).

Nella nuova forma, il drago mantiene il suo allineamento, punti ferita, Dadi Vita, la facoltà di parlare, le competenze, la Resistenza Leggendaria, le azioni da tana, e i punteggi di Intelligenza, Saggezza e Carisma, oltre a questa azione. Le sue statistiche e capacità vengono altrimenti rimpiazzate da quelle della nuova forma, eccetto Azioni aggiuntive della nuova forma.

\textbf{Azioni Aggiuntive}

Il drago può effettuare 3 Azioni aggiuntive, scelte tra le opzioni seguenti. Può usare solo un'opzione leggendaria alla volta e solo al termine del turno di un'altra creatura. Il drago recupera le Azioni aggiuntive spese all'inizio del proprio turno.

\textbf{Attacco di Ala (Costa 2 Azioni).} Il drago batte le ali. Ogni creatura entro 3 metri dal drago deve riuscire un Tiro Salvezza su Riflessi CD 21 o subire 15 (2d6 + 8) danni da botta e venir gettato prono. Il drago può poi volare fino a metà del suo movimento di volo.

\textbf{Attacco di Coda.} Il drago effettua un attacco di coda.

\textbf{Individuare.} Il drago effettua una prova di Saggezza (Consapevolezza).

\textbf{Ecologia}\\
Ambiente: Montagne Temperate\\
Organizzazione: Solitario\\
Tesoro: Triplo\\
\textbf{Descrizione}\\
Tra tutti i draghi, quelli d'argento sono i più coraggiosi, e si attengono ad un codice cavalleresco che impone loro di aiutare i deboli, sconfiggere il male e comportarsi in modo onorevole.\\
\textbf{Incantesimi}\index{Incantesimi da Drago Argento}\\
Il drago e' anche capace, se di eta' giovanile o piu' anziano, di lanciare incantesimi.\\
Puo' lanciare un numero di incantesimi pari al suo valore di Carisma, senza aver necessita' di componenti magici.\\
Non deve fare prove di magia, si considera che riesca sempre con un punteggio pari alla Difficoltà + il valore del Carisma.\\
Se deve fare un Tiro per Colpire ha CA = Grado di Sfida con un bonus al colpire pari alla Forza\\
Gli incantesimi a disposizione sono:\\
- Cono di freddo\\
- Tempesta di ghiaccio\\
- Creazione\\
- Capanna\\

\medskip\index{Mostri - Drago d'Argento Giovane}\textbf{Drago d'Argento Giovane}

\emph{Grande drago, legale buono}

\textbf{FORZA} +6

\textbf{DESTREZZA} +0

\textbf{COSTITUZIONE} +5

\textbf{INTELLIGENZA} +2

\textbf{SAGGEZZA} +0

\textbf{CARISMA} +4

\textbf{Iniziativa} +2 -- \textbf{Difesa} 23

\textbf{Punti Ferita} 168 (16d10 + 80)

\textbf{Movimento} 12 m, volo 24 m

\textbf{Tiri Salvezza} Tempra +10, Riflessi +8, Volontà +12

\textbf{Competenze} Arcano +6, Muoversi Silenziosamente / Nascondersi +4, Consapevolezza +8, Storia +6

\textbf{Immunità al Danno} freddo

\textbf{Sensi} scurovisione 36 m, vista cieca 9 m

\textbf{Linguaggi} Comune, Draconico

\textbf{Sfida} 9 (5.000 PE)

\textbf{Azioni}

\emph{\textbf{Multiattacco.}} Il drago può effettuare tre attacchi: uno con il morso e due con gli artigli.

\emph{\textbf{Artiglio.} Attacco con arma da mischia}: +10 a colpire, portata 1 m, un bersaglio.

\emph{Colpisce:} 13 (2d6 + 6) danni taglienti.

\emph{\textbf{Morso.} Attacco con arma da mischia}: +10 a colpire, portata 3 m, un bersaglio.

\emph{Colpisce:} 17 (2d10 + 6) danni perforanti.

\emph{\textbf{Arma a Soffio (Ricarica 5-6).}} Il drago usa una delle seguenti armi a soffio:

\emph{Soffio Gelido.} Il drago esala un'esplosione ghiacciata in un cono di 9 metri. Ogni creatura nell'area deve effettuare un Tiro Salvezza su Tempra CD 17, subendo 54 (12d8) danni da freddo se fallisce il Tiro Salvezza, o la metà di questi danni se lo riesce.

\emph{Soffio Paralizzante.} Il drago esala un gas paralizzante in un cono di 9 metri. Ogni creatura nell'area deve riuscire un Tiro Salvezza su Tempra 17 o restare paralizzata per 1 minuto. Una creatura può ripetere il Tiro Salvezza al termine di ciascun suo turno, terminando l'effetto per sé in caso di successo.

\textbf{Ecologia}\\
Ambiente: Montagne Temperate\\
Organizzazione: Solitario\\
Tesoro: Triplo\\
\textbf{Descrizione}\\
Tra tutti i draghi, quelli d'argento sono i più coraggiosi, e si attengono ad un codice cavalleresco che impone loro di aiutare i deboli, sconfiggere il male e comportarsi in modo onorevole.\\
\textbf{Incantesimi}\index{Incantesimi da Drago Argento}\\
Il drago e' anche capace, se di eta' giovanile o piu' anziano, di lanciare incantesimi.\\
Puo' lanciare un numero di incantesimi pari al suo valore di Carisma, senza aver necessita' di componenti magici.\\
Non deve fare prove di magia, si considera che riesca sempre con un punteggio pari alla Difficoltà + il valore del Carisma.\\
Se deve fare un Tiro per Colpire ha CA = Grado di Sfida con un bonus al colpire pari alla Forza\\
Gli incantesimi a disposizione sono:\\
- Cono di freddo\\
- Tempesta di ghiaccio\\
- Creazione\\
- Capanna\\

\medskip\index{Mostri - Drago d'Argento Cucciolo}\textbf{Drago d'Argento Cucciolo}

\emph{Media drago, legale buono}

\textbf{FORZA} +4

\textbf{DESTREZZA} +0

\textbf{COSTITUZIONE} +3

\textbf{INTELLIGENZA} +1

\textbf{SAGGEZZA} +0

\textbf{CARISMA} +2

\textbf{Iniziativa} +1 -- \textbf{Difesa} 18

\textbf{Punti Ferita} 45 (6d8 + 18)

\textbf{Movimento} 9 m, volo 18 m

\textbf{Tiri Salvezza} Tempra +3, Riflessi +3, Volontà +2

\textbf{Competenze} Muoversi Silenziosamente / Nascondersi +2, Consapevolezza +4

\textbf{Immunità al Danno} freddo

\textbf{Sensi} scurovisione 18 m, vista cieca 3 m

\textbf{Linguaggi} Draconico

\textbf{Sfida} 2 (450 PE)

\textbf{Azioni}

\emph{\textbf{Morso.} Attacco con arma da mischia}: +6 a colpire, portata 1 m, un bersaglio.

\emph{Colpisce:} 9 (1d10 + 4) danni perforanti.

\emph{\textbf{Arma a Soffio (Ricarica 5-6).}} Il drago usa una delle seguenti armi a soffio:

\emph{Soffio Gelido.} Il drago esala un'esplosione ghiacciata in un cono di 5 metri. Ogni creatura nell'area deve effettuare un Tiro Salvezza su Tempra 13, subendo 18 (4d8) danni da freddo se fallisce il Tiro Salvezza, o la metà di questi danni se lo riesce.

\emph{Soffio Paralizzante.} Il drago esala un gas paralizzante in un cono di 5 metri. Ogni creatura nell'area deve riuscire un Tiro Salvezza su Tempra 13 o restare paralizzata per 1 minuto. Una creatura può ripetere il Tiro Salvezza al termine di ciascun suo turno, terminando l'effetto per sé in caso di successo.

\textbf{Ecologia}\\
Ambiente: Montagne Temperate\\
Organizzazione: Solitario\\
Tesoro: Triplo\\
\textbf{Descrizione}\\
Tra tutti i draghi, quelli d'argento sono i più coraggiosi, e si attengono ad un codice cavalleresco che impone loro di aiutare i deboli, sconfiggere il male e comportarsi in modo onorevole.\\


\medskip\index{Mostri - Drago di Bronzo Antico}\textbf{Drago di Bronzo Antico}

\emph{Mastodontica drago, caotico buono}

\textbf{FORZA} +9

\textbf{DESTREZZA} +0

\textbf{COSTITUZIONE} +8

\textbf{INTELLIGENZA} +4

\textbf{SAGGEZZA} +3

\textbf{CARISMA} +5

\textbf{Iniziativa} +4 -- \textbf{Difesa} 33

\textbf{Punti Ferita} 444 (24d20 + 192)

\textbf{Movimento} 12 m, nuoto 12 m, volo 24 m

\textbf{Tiri Salvezza} Tempra +21, Riflessi +13, Volontà +21

\textbf{Competenze} Muoversi Silenziosamente / Nascondersi +7, Percepire Emozioni +10, Consapevolezza +17

\textbf{Immunità al Danno} fulmine, arma +1

\textbf{Sensi} scurovisione 36 m, vista cieca 18 m

\textbf{Linguaggi} Comune, Draconico

\textbf{Sfida} 22 (41.000 PE)

\emph{\textbf{Anfibio.}} Il drago può respirare aria e acqua.

\emph{\textbf{Resistenza Leggendaria (3/Giorno).}} Se il drago fallisce un Tiro Salvezza, può scegliere invece di riuscire.

\textbf{Azioni}

\emph{\textbf{Multiattacco.}} Il drago può usare la sua Presenza Spaventosa. Poi effettuare tre attacchi: uno con il morso e due con gli artigli.

\emph{\textbf{Artiglio.} Attacco con arma da mischia}: +16 a colpire, portata 3 m, un bersaglio.

\emph{Colpisce:} 16 (2d6 + 9) danni taglienti.

\emph{\textbf{Coda.} Attacco con arma da mischia}: +16 a colpire, portata 6 m, un bersaglio.

\emph{Colpisce:} 18 (2d8 + 9) danni da botta.

\emph{\textbf{Morso.} Attacco con arma da mischia}: +16 a colpire, portata 5 metri, un bersaglio.

\emph{Colpisce:} 20 (2d10 + 9) danni perforanti.

\emph{\textbf{Presenza Spaventosa.}} Ogni creatura scelta dal drago, che si trovi entro 36 metri da esso e consapevole della sua presenza, deve riuscire un Tiro Salvezza di Volontà CD 20 o restare spaventata per 1 minuto. Una creatura può ripetere il Tiro Salvezza al termine di ciascun suo turno, terminando l'effetto se lo riesce. Se il Tiro Salvezza della creatura ha successo o l'effetto ha termine per essa, la creatura è immune alla Presenza Spaventosa del drago per le successive 24 ore.

\emph{\textbf{Arma a Soffio (Ricarica 5-6).}} Il drago usa una delle seguenti armi a soffio:

\emph{Soffio Fulminante.} Il drago esala fulmini in una linea lunga 36 metri e larga 3 metri. Ogni creatura sulla linea deve effettuare un Tiro Salvezza su Riflessi CD 23, subendo 88 (16d10) danni da fulmine se fallisce il Tiro Salvezza, o la metà di questi danni se lo riesce. \emph{Soffio Repulsivo.} Il drago esala dell'energia repulsiva in un cono di 9 metri. Ogni creatura in quell'area deve riuscire un Tiro Salvezza su Tempra CD 23, altrimenti viene allontana di 18 metri dal
drago.

\emph{\textbf{Mutare Forma.}} Il drago può trasformarsi magicamente in un umanoide o bestia il cui grado di sfida sia pari o inferiore al proprio, o tornare alla sua vera forma. Alla morte ritorna alla sua vera forma. Qualsiasi equipaggiamento stia indossando o trasportando viene assorbito o trasportato nella nuova forma (a scelta del drago).

Nella nuova forma, il drago mantiene il suo allineamento, punti ferita, Dadi Vita, la facoltà di parlare, le competenze, la Resistenza Leggendaria, le azioni da tana, e i punteggi di Intelligenza, Saggezza e Carisma, oltre a questa azione. Le sue statistiche e capacità vengono altrimenti rimpiazzate da quelle della nuova forma, eccetto Azioni aggiuntive della nuova forma.

\textbf{Azioni Aggiuntive}

Il drago può effettuare 3 Azioni aggiuntive, scelte tra le opzioni seguenti. Può usare solo un'opzione leggendaria alla volta e solo al termine del turno di un'altra creatura. Il drago recupera le Azioni aggiuntive spese all'inizio del proprio turno.

\textbf{Attacco di Ala (Costa 2 Azioni).} Il drago batte le ali. Ogni creatura entro 5 metri dal drago deve riuscire un Tiro Salvezza su Riflessi CD 24 o subire 16 (2d6 + 9) danni da botta e venir gettato prono. Il drago può poi volare fino a metà del suo movimento di volo.

\textbf{Attacco di Coda.} Il drago effettua un attacco di coda.

\textbf{Individuare.} Il drago effettua una prova di Saggezza (Consapevolezza).

\textbf{Ecologia}\\
Ambiente: Zone Costiere Temperate\\
Organizzazione: Solitario\\
Tesoro: Triplo\\
\textbf{Descrizione}\\
I draghi di bronzo sono noti per allearsi con viaggiatori ed avventurieri se causa e ricompensa sono giuste e adeguate\\
\textbf{Incantesimi}\index{Incantesimi da Drago Bronzo}\\
Il drago e' anche capace, se di eta' giovanile o piu' anziano, di lanciare incantesimi.\\
Puo' lanciare un numero di incantesimi pari al suo valore di Carisma, senza aver necessita' di componenti magici.\\
Non deve fare prove di magia, si considera che riesca sempre con un punteggio pari alla Difficoltà + il valore del Carisma.\\
Se deve fare un Tiro per Colpire ha CA = Grado di Sfida con un bonus al colpire pari alla Forza\\
Gli incantesimi a disposizione sono:\\
- Controllare Tempo Atmosferico\\
- Invocare il Fulmine\\


\medskip\index{Mostri - Drago di Bronzo Adulto}\textbf{Drago di Bronzo Adulto}

\emph{Enorme drago, caotico buono}

\textbf{FORZA} +7

\textbf{DESTREZZA} +0

\textbf{COSTITUZIONE} +6

\textbf{INTELLIGENZA} +3

\textbf{SAGGEZZA} +2

\textbf{CARISMA} +4

\textbf{Iniziativa} +3 -- \textbf{Difesa} 27

\textbf{Punti Ferita} 212 (17d12 + 102)

\textbf{Movimento} 12 m, nuoto 12 m, volo 24 m

\textbf{Tiri Salvezza} Tempra +15, Riflessi +10, Volontà +15

\textbf{Competenze} Muoversi Silenziosamente / Nascondersi +5, Percepire Emozioni +7, Consapevolezza +12

\textbf{Immunità al Danno} fulmine

\textbf{Sensi} scurovisione 36 m, vista cieca 18 m

\textbf{Linguaggi} Comune, Draconico

\textbf{Sfida} 15 (13.000 PE)

\emph{\textbf{Anfibio.}} Il drago può respirare aria e acqua.

\emph{\textbf{Resistenza Leggendaria (3/Giorno).}} Se il drago fallisce un Tiro Salvezza, può scegliere invece di riuscire.

\textbf{Azioni}

\emph{\textbf{Multiattacco.}} Il drago può usare la sua Presenza Spaventosa e poi effettuare tre attacchi: uno con il morso e due con gli
artigli.

\emph{\textbf{Artiglio.} Attacco con arma da mischia}: +12 a colpire, portata 1 m, un bersaglio.

\emph{Colpisce:} 14 (2d6 + 7) danni taglienti.

\emph{\textbf{Coda.} Attacco con arma da mischia}: +12 a colpire, portata 5 metri, un bersaglio.

\emph{Colpisce:} 16 (2d8 + 7) danni da botta.

\emph{\textbf{Morso.} Attacco con arma da mischia}: +12 a colpire, portata 3 m, un bersaglio.

\emph{Colpisce:} 18 (2d10 + 7) danni perforanti.

\emph{\textbf{Presenza Spaventosa.}} Ogni creatura scelta dal drago, che si trovi entro 36 metri da esso e consapevole della sua presenza, deve riuscire un Tiro Salvezza di Volontà CD 17 o restare spaventata per 1 minuto. Una creatura può ripetere il Tiro Salvezza al termine di ciascun suo turno, terminando l'effetto se lo riesce. Se il Tiro Salvezza della creatura ha successo o l'effetto ha termine per essa, la creatura è immune alla Presenza Spaventosa del drago per le successive 24 ore. 

\emph{\textbf{Arma a Soffio (Ricarica 5-6).}} Il drago usa una delle seguenti armi a soffio:

\emph{Soffio Fulminante.} Il drago esala fulmini in una linea lunga 27 metri e larga 1 metro. Ogni creatura sulla linea deve effettuare un Tiro Salvezza di Riflessi CD 19, subendo 66 (12d10) danni da fulmine se fallisce il Tiro Salvezza, o la metà di questi danni se lo riesce. \emph{Soffio Repulsivo.} Il drago esala dell'energia repulsiva in un cono di 9 metri. Ogni creatura in quell'area deve riuscire un Tiro Salvezza su Tempra CD 19, altrimenti viene allontana di 18 metri dal drago.

\emph{\textbf{Mutare Forma.}} Il drago può trasformarsi magicamente in un umanoide o bestia il cui grado di sfida sia pari o inferiore al proprio, o tornare alla sua vera forma. Alla morte ritorna alla sua vera forma. Qualsiasi equipaggiamento stia indossando o trasportando viene assorbito o trasportato nella nuova forma (a scelta del drago).

Nella nuova forma, il drago mantiene il suo allineamento, punti ferita, Dadi Vita, la facoltà di parlare, le competenze, la Resistenza Leggendaria, le azioni da tana, e i punteggi di Intelligenza, Saggezza e Carisma, oltre a questa azione. Le sue statistiche e capacità vengono altrimenti rimpiazzate da quelle della nuova forma, eccetto Azioni aggiuntive della nuova forma.

\textbf{Azioni Aggiuntive}

Il drago può effettuare 3 Azioni aggiuntive, scelte tra le opzioni seguenti. Può usare solo un'opzione leggendaria alla volta e solo al termine del turno di un'altra creatura. Il drago recupera le Azioni aggiuntive spese all'inizio del proprio turno.

\textbf{Attacco di Ala (Costa 2 Azioni).} Il drago batte le ali. Ogni creatura entro 3 metri dal drago deve riuscire un Tiro Salvezza su Riflessi CD 20 o subire 14 (2d6 + 7) danni da botta e venir gettato prono. Il drago può poi volare fino a metà del suo movimento di volo.

\textbf{Attacco di Coda.} Il drago effettua un attacco di coda. 

\textbf{Individuare.} Il drago effettua una prova di Saggezza (Consapevolezza).

\textbf{Ecologia}\\
Ambiente: Zone Costiere Temperate\\
Organizzazione: Solitario\\
Tesoro: Triplo\\
\textbf{Descrizione}\\
I draghi di bronzo sono noti per allearsi con viaggiatori ed avventurieri se causa e ricompensa sono giuste e adeguate\\
\textbf{Incantesimi}\index{Incantesimi da Drago Bronzo}\\
Il drago e' anche capace, se di eta' giovanile o piu' anziano, di lanciare incantesimi.\\
Puo' lanciare un numero di incantesimi pari al suo valore di Carisma, senza aver necessita' di componenti magici.\\
Non deve fare prove di magia, si considera che riesca sempre con un punteggio pari alla Difficoltà + il valore del Carisma.\\
Se deve fare un Tiro per Colpire ha CA = Grado di Sfida con un bonus al colpire pari alla Forza\\
Gli incantesimi a disposizione sono:\\
- Controllare Tempo Atmosferico\\
- Invocare il Fulmine\\


\medskip\index{Mostri - Drago di Bronzo Giovane}\textbf{Drago di Bronzo Giovane}

\emph{Grande drago, caotico buono}

\textbf{FORZA} +5

\textbf{DESTREZZA} +0

\textbf{COSTITUZIONE} +4

\textbf{INTELLIGENZA} +2

\textbf{SAGGEZZA} +1

\textbf{CARISMA} +3

\textbf{Iniziativa} +2 -- \textbf{Difesa} 22

\textbf{Punti Ferita} 142 (15d10 + 60)

\textbf{Movimento} 12 m, nuoto 12 m, volo 24 m

\textbf{Tiri Salvezza} Tempra +10, Riflessi +8, Volontà +10

\textbf{Competenze} Muoversi Silenziosamente / Nascondersi +3, Percepire Emozioni +4, Consapevolezza +7

\textbf{Immunità al Danno} fulmine

\textbf{Sensi} scurovisione 36 m, vista cieca 9 m

\textbf{Linguaggi} Comune, Draconico

\textbf{Sfida} 8 (3.900 PE)

\emph{\textbf{Anfibio.}} Il drago può respirare aria e acqua.

\textbf{Azioni}

\emph{\textbf{Multiattacco.}} Il drago può usare effettuare tre attacchi: uno con il morso e due con gli artigli.

\emph{\textbf{Artiglio.} Attacco con arma da mischia}: +8 a colpire, portata 1 m, un bersaglio.

\emph{Colpisce:} 12 (2d6 + 5) danni taglienti.

\emph{\textbf{Morso.} Attacco con arma da mischia}: +8 a colpire, portata 3 m, un bersaglio.

\emph{Colpisce:} 16 (2d10 + 5) danni perforanti.

\emph{\textbf{Arma a Soffio (Ricarica 5-6).}} Il drago usa una delle seguenti armi a soffio:

\emph{Soffio Fulminante.} Il drago esala fulmini in una linea lunga 18 metri e larga 1 metro. Ogni creatura sulla linea deve effettuare un Tiro Salvezza di Riflessi CD 15, subendo 55 (10d10) danni da fulmine se fallisce il Tiro Salvezza, o la metà di questi danni se lo riesce.

\emph{Soffio Repulsivo.} Il drago esala dell'energia repulsiva in un cono di 9 metri. Ogni creatura in quell'area deve riuscire un Tiro Salvezza su Tempra CD 15, altrimenti viene allontana di 12 metri dal drago.

\textbf{Ecologia}\\
Ambiente: Zone Costiere Temperate\\
Organizzazione: Solitario\\
Tesoro: Triplo\\
\textbf{Descrizione}\\
I draghi di bronzo sono noti per allearsi con viaggiatori ed avventurieri se causa e ricompensa sono giuste e adeguate\\
\textbf{Incantesimi}\index{Incantesimi da Drago Bronzo}\\
Il drago e' anche capace, se di eta' giovanile o piu' anziano, di lanciare incantesimi.\\
Puo' lanciare un numero di incantesimi pari al suo valore di Carisma, senza aver necessita' di componenti magici.\\
Non deve fare prove di magia, si considera che riesca sempre con un punteggio pari alla Difficoltà + il valore del Carisma.\\
Se deve fare un Tiro per Colpire ha CA = Grado di Sfida con un bonus al colpire pari alla Forza\\
Gli incantesimi a disposizione sono:\\
- Controllare Tempo Atmosferico\\
- Invocare il Fulmine\\


\medskip\index{Mostri - Drago di Bronzo Cucciolo}\textbf{Drago di Bronzo Cucciolo}

\emph{Media drago, caotico buono}

\textbf{FORZA} +3

\textbf{DESTREZZA} +0

\textbf{COSTITUZIONE} +2

\textbf{INTELLIGENZA} +1

\textbf{SAGGEZZA} +0

\textbf{CARISMA} +2

\textbf{Iniziativa} +1 -- \textbf{Difesa} 18

\textbf{Punti Ferita} 32 (5d8 + 10)

\textbf{Movimento} 9 m, nuoto 9 m, volo 18 m

\textbf{Tiri Salvezza} Tempra +2, Riflessi +1, Volontà +1

\textbf{Competenze} Muoversi Silenziosamente / Nascondersi +2, Consapevolezza +4

\textbf{Immunità al Danno} fulmine

\textbf{Sensi} scurovisione 18 m, vista cieca 3 m

\textbf{Linguaggi} Draconico

\textbf{Sfida} 2 (450 PE)

\emph{\textbf{Anfibio.}} Il drago può respirare aria e acqua.

\textbf{Azioni}

\emph{\textbf{Morso.} Attacco con arma da mischia}: +5 a colpire,
portata 1 m, un bersaglio.

\emph{Colpisce:} 8 (1d10 + 3) danni perforanti.

\emph{\textbf{Arma a Soffio (Ricarica 5-6).}} Il drago usa una delle seguenti armi a soffio:

\emph{Soffio Fulminante.} Il drago esala fulmini in una linea lunga 12 metri e larga 1 metro. Ogni creatura sulla linea deve effettuare un Tiro Salvezza di Riflessi CD 12, subendo 16 (3d10) danni da fulmine se fallisce il Tiro Salvezza, o la metà di questi danni se lo riesce.

\emph{Soffio Repulsivo.} Il drago esala dell'energia repulsiva in un cono di 9 metri. Ogni creatura in quell'area deve riuscire un Tiro Salvezza su Tempra CD 12, altrimenti viene allontana di 9 metri dal drago.

\textbf{Ecologia}\\
Ambiente: Zone Costiere Temperate\\
Organizzazione: Solitario\\
Tesoro: Triplo\\
\textbf{Descrizione}\\
I draghi di bronzo sono noti per allearsi con viaggiatori ed avventurieri se causa e ricompensa sono giuste e adeguate\\



\medskip\index{Mostri - Drago d'Oro Antico}\textbf{Drago d'Oro Antico}

\emph{Mastodontica drago, legale buono}

\textbf{FORZA} +10

\textbf{DESTREZZA} +2

\textbf{COSTITUZIONE} +9

\textbf{INTELLIGENZA} +4

\textbf{SAGGEZZA} +3

\textbf{CARISMA} +9

\textbf{Iniziativa} +4 -- \textbf{Difesa} 34

\textbf{Punti Ferita} 546 (28d20 + 252)

\textbf{Movimento} 12 m, nuoto 12 m, volo 24 m

\textbf{Tiri Salvezza} Tempra +23, Riflessi +14, Volontà +24

\textbf{Competenze} Muoversi Silenziosamente / Nascondersi +9, Percepire Emozioni +10, Consapevolezza +17, Ingannare +16

\textbf{Immunità al Danno} fuoco, arma +1

\textbf{Sensi} scurovisione 36 m, vista cieca 18 m

\textbf{Linguaggi} Comune, Draconico

\textbf{Sfida} 24 (62.000 PE)

\emph{\textbf{Anfibio.}} Il drago può respirare aria e acqua.

\emph{\textbf{Resistenza Leggendaria (3/Giorno).}} Se il drago fallisce un Tiro Salvezza, può scegliere invece di riuscire.

\textbf{Azioni}

\emph{\textbf{Multiattacco.}} Il drago può usare la sua Presenza Spaventosa. Poi effettuare tre attacchi: uno con il morso e due con gli artigli.

\emph{\textbf{Artiglio.} Attacco con arma da mischia}: +17 a colpire, portata 3 m, un bersaglio.

\emph{Colpisce:} 17 (2d6 + 10) danni taglienti.

\emph{\textbf{Coda.} Attacco con arma da mischia}: +17 a colpire, portata 6 m, un bersaglio.

\emph{Colpisce:} 19 (2d8 + 10) danni da botta.

\emph{\textbf{Morso.} Attacco con arma da mischia}: +17 a colpire, portata 5 metri, un bersaglio.

\emph{Colpisce:} 21 (2d10 + 10) danni perforanti.

\emph{\textbf{Presenza Spaventosa.}} Ogni creatura scelta dal drago, che si trovi entro 36 metri da esso e consapevole della sua presenza, deve riuscire un Tiro Salvezza di Volontà CD 24 o restare spaventata per 1 minuto. Una creatura può ripetere il Tiro Salvezza al termine di ciascun suo turno, terminando l'effetto se lo riesce. Se il Tiro Salvezza della creatura ha successo o l'effetto ha termine per essa, la creatura è immune alla Presenza Spaventosa del drago per le successive 24 ore.

\emph{\textbf{Arma a Soffio (Ricarica 5-6).}} Il drago usa una delle seguenti armi a soffio:

\emph{Soffio Infuocato.} Il drago esala fuoco in un cono di 27 metri. Ogni creatura nell'area deve effettuare un Tiro Salvezza di Riflessi CD 24, subendo 71(13d10) danni da fuoco se fallisce il Tiro Salvezza, o la metà di questi danni se lo riesce.

\emph{Soffio Indebolente.} Il drago esala del gas in un cono di 27 metri. Ogni creatura in quell'area deve riuscire un Tiro Salvezza su Tempra CD 24 o avere -1d6 ai tiri di attacco basati sulla Forza, prove di Forza, e Tiri Salvezza su Tempra per 1 minuto. Una creatura può ripetere il Tiro Salvezza al termine di ciascun suo turno, terminando l'effetto su di sé in caso di successo.

\emph{\textbf{Mutare Forma.}} Il drago può trasformarsi magicamente in un umanoide o bestia il cui grado di sfida sia pari o inferiore al proprio, o tornare alla sua vera forma. Alla morte ritorna alla sua vera forma. Qualsiasi equipaggiamento stia indossando o trasportando viene assorbito o trasportato nella nuova forma (a scelta del drago).

Nella nuova forma, il drago mantiene il suo allineamento, punti ferita, Dadi Vita, la facoltà di parlare, le competenze, la Resistenza Leggendaria, le azioni da tana, e i punteggi di Intelligenza, Saggezza e Carisma, oltre a questa azione. Le sue statistiche e capacità vengono altrimenti rimpiazzate da quelle della nuova forma, eccetto Azioni aggiuntive della nuova forma.

\textbf{Azioni Aggiuntive}

Il drago può effettuare 3 Azioni aggiuntive, scelte tra le opzioni seguenti. Può usare solo un'opzione leggendaria alla volta e solo al termine del turno di un'altra creatura. Il drago recupera le Azioni aggiuntive spese all'inizio del proprio turno.

\textbf{Attacco di Ala (Costa 2 Azioni).} Il drago batte le ali. Ogni creatura entro 5 metri dal drago deve riuscire un Tiro Salvezza su Riflessi CD 25 o subire 17 (2d6 + 10) danni da botta e venir gettato prono. Il drago può poi volare fino a metà del suo movimento di volo.

\textbf{Attacco di Coda.} Il drago effettua un attacco di coda.

\textbf{Individuare.} Il drago effettua una prova di Saggezza (Consapevolezza).

\textbf{Ecologia}\\
Ambiente: Pianure calde\\
Organizzazione: Solitario\\
Tesoro: Triplo\\
\textbf{Descrizione}\\
I draghi d'oro sono l'emblema della virtù. Gli altri draghi metallici li riveriscono come agenti delle potenze divine e membri esemplari della razza draconica, e spesso li cercano per consigli o aiuto.\\
\textbf{Incantesimi}\index{Incantesimi da Drago d'Oro}\\
Il drago e' anche capace, se di eta' giovanile o piu' anziano, di lanciare incantesimi.\\
Puo' lanciare un numero di incantesimi pari al suo valore di Carisma, senza aver necessita' di componenti magici.\\
Non deve fare prove di magia, si considera che riesca sempre con un punteggio pari alla Difficoltà + il valore del Carisma.\\
Se deve fare un Tiro per Colpire ha CA = Grado di Sfida con un bonus al colpire pari alla Forza\\
Gli incantesimi a disposizione sono:\\
- Blocca Mostri\\
- Muro di Fuoco\\
- Porta Dimensionale\\

\medskip\index{Mostri - Drago d'Oro Adulto}\textbf{Drago d'Oro Adulto}

\emph{Enorme drago, legale buono}

\textbf{FORZA} +8

\textbf{DESTREZZA} +2

\textbf{COSTITUZIONE} +7

\textbf{INTELLIGENZA} +3

\textbf{SAGGEZZA} +2

\textbf{CARISMA} +7

\textbf{Iniziativa} +3 -- \textbf{Difesa} 28

\textbf{Punti Ferita} 256 (19d12 + 133)

\textbf{Movimento} 12 m, nuoto 12 m, volo 24 m

\textbf{Tiri Salvezza} Tempra +17, Riflessi +11, Volontà +18

\textbf{Competenze} Muoversi Silenziosamente / Nascondersi +8, Percepire Emozioni +8, Consapevolezza +14, Ingannare +13 

\textbf{Immunità al Danno} fuoco

\textbf{Sensi} scurovisione 36 m, vista cieca 18 m

\textbf{Linguaggi} Comune, Draconico

\textbf{Sfida} 17 (18.000 PE)

\emph{\textbf{Anfibio.}} Il drago può respirare aria e acqua.

\emph{\textbf{Resistenza Leggendaria (3/Giorno).}} Se il drago fallisce un Tiro Salvezza, può scegliere invece di riuscire.

\textbf{Azioni}

\emph{\textbf{Multiattacco.}} Il drago può usare la sua Presenza Spaventosa. Poi effettuare tre attacchi: uno con il morso e due con gli artigli.

\emph{\textbf{Artiglio.} Attacco con arma da mischia}: +14 a colpire, portata 1 m, un bersaglio.

\emph{Colpisce:} 15 (2d6 + 8) danni taglienti.

\emph{\textbf{Coda.} Attacco con arma da mischia}: +14 a colpire, portata 5 metri, un bersaglio.

\emph{Colpisce:} 17 (2d8 + 8) danni da botta.

\emph{\textbf{Morso.} Attacco con arma da mischia}: +14 a colpire, portata 3 m, un bersaglio.

\emph{Colpisce:} 19 (2d10 + 8) danni perforanti.

\emph{\textbf{Presenza Spaventosa.}} Ogni creatura scelta dal drago, che si trovi entro 36 metri da esso e consapevole della sua presenza, deve riuscire un Tiro Salvezza di Volontà CD 21 o restare spaventata per 1 minuto. Una creatura può ripetere il Tiro Salvezza al termine di ciascun suo turno, terminando l'effetto se lo riesce. Se il Tiro Salvezza della creatura ha successo o l'effetto ha termine per essa, la creatura è immune alla Presenza Spaventosa del drago per le successive 24 ore.

\emph{\textbf{Arma a Soffio (Ricarica 5-6).}} Il drago usa una delle seguenti armi a soffio:

\emph{Soffio Infuocato.} Il drago esala fuoco in un cono di 18 metri. Ogni creatura nell'area deve effettuare un Tiro Salvezza di Riflessi CD 21, subendo 66 (12d10) danni da fuoco se fallisce il Tiro Salvezza, o la metà di questi danni se lo riesce.

\emph{Soffio Indebolente.} Il drago esala del gas in un cono di 18 metri. Ogni creatura in quell'area deve riuscire un Tiro Salvezza su Tempra CD 21 o avere -1d6 ai tiri di attacco basati sulla Forza, prove di Forza, e Tiri Salvezza su Tempra per 1 minuto. Una creatura può ripetere il Tiro Salvezza al termine di ciascun suo turno, terminando l'effetto su di sé in caso di successo.

\emph{\textbf{Mutare Forma.}} Il drago può trasformarsi magicamente in un umanoide o bestia il cui grado di sfida sia pari o inferiore al proprio, o tornare alla sua vera forma. Alla morte ritorna alla sua vera forma. Qualsiasi equipaggiamento stia indossando o trasportando viene assorbito o trasportato nella nuova forma (a scelta del drago).

Nella nuova forma, il drago mantiene il suo allineamento, punti ferita, Dadi Vita, la facoltà di parlare, le competenze, la Resistenza Leggendaria, le azioni da tana, e i punteggi di Intelligenza, Saggezza e Carisma, oltre a questa azione. Le sue statistiche e capacità vengono altrimenti rimpiazzate da quelle della nuova forma, eccetto Azioni aggiuntive della nuova forma. 

\textbf{Azioni Aggiuntive}

Il drago può effettuare 3 Azioni aggiuntive, scelte tra le opzioni seguenti. Può usare solo un'opzione leggendaria alla volta e solo al termine del turno di un'altra creatura. Il drago recupera le Azioni aggiuntive spese all'inizio del proprio turno.

\textbf{Attacco di Ala (Costa 2 Azioni).} Il drago batte le ali. Ogni creatura entro 3 metri dal drago deve riuscire un Tiro Salvezza su Riflessi CD 22 o subire 15 (2d6 + 8) danni da botta e venir gettato prono. Il drago può poi volare fino a metà del suo movimento di volo.

\textbf{Attacco di Coda.} Il drago effettua un attacco di coda.

\textbf{Individuare.} Il drago effettua una prova di Saggezza (Consapevolezza).

\textbf{Ecologia}\\
Ambiente: Pianure calde\\
Organizzazione: Solitario\\
Tesoro: Triplo\\
\textbf{Descrizione}\\
I draghi d'oro sono l'emblema della virtù. Gli altri draghi metallici li riveriscono come agenti delle potenze divine e membri esemplari della razza draconica, e spesso li cercano per consigli o aiuto.\\
\textbf{Incantesimi}\index{Incantesimi da Drago d'Oro}\\
Il drago e' anche capace, se di eta' giovanile o piu' anziano, di lanciare incantesimi.\\
Puo' lanciare un numero di incantesimi pari al suo valore di Carisma, senza aver necessita' di componenti magici.\\
Non deve fare prove di magia, si considera che riesca sempre con un punteggio pari alla Difficoltà + il valore del Carisma.\\
Se deve fare un Tiro per Colpire ha CA = Grado di Sfida con un bonus al colpire pari alla Forza\\
Gli incantesimi a disposizione sono:\\
- Blocca Mostri\\
- Muro di Fuoco\\
- Porta Dimensionale\\

\medskip\index{Mostri - Drago d'Oro Giovane}\textbf{Drago d'Oro Giovane}

\emph{Grande drago, legale buono}

\textbf{FORZA} +6

\textbf{DESTREZZA} +2

\textbf{COSTITUZIONE} +5

\textbf{INTELLIGENZA} +3

\textbf{SAGGEZZA} +1

\textbf{CARISMA} +5

\textbf{Iniziativa} +3 -- \textbf{Difesa} 23

\textbf{Punti Ferita} 178 (17d10 + 85)

\textbf{Movimento} 12 m, nuoto 12 m, volo 24 m

\textbf{Tiri Salvezza} Tempra +12, Riflessi +9, Volontà +13

\textbf{Competenze} Muoversi Silenziosamente / Nascondersi +6, Percepire Emozioni +5, Consapevolezza +9, Ingannare +9

\textbf{Immunità al Danno} fuoco

\textbf{Sensi} scurovisione 36 m, vista cieca 9 m

\textbf{Linguaggi} Comune, Draconico

\textbf{Sfida} 10 (5.900 PE)

\emph{\textbf{Anfibio.}} Il drago può respirare aria e acqua.

\textbf{Azioni}

\emph{\textbf{Multiattacco.}} Il drago può effettuare tre attacchi: uno con il morso e due con gli artigli.

\emph{\textbf{Artiglio.} Attacco con arma da mischia}: +10 a colpire, portata 1 m, un bersaglio.

\emph{Colpisce:} 13 (2d6 + 6) danni taglienti.

\emph{\textbf{Morso.} Attacco con arma da mischia}: +10 a colpire, portata 3 m, un bersaglio.

\emph{Colpisce:} 17 (2d10 + 6) danni perforanti.

\emph{\textbf{Arma a Soffio (Ricarica 5-6).}} Il drago usa una delle seguenti armi a soffio:

\emph{Soffio Infuocato.} Il drago esala fuoco in un cono di 9 metri. Ogni creatura nell'area deve effettuare un Tiro Salvezza di Riflessi CD 17, subendo 55 (10d10) danni da fuoco se fallisce il Tiro Salvezza, o la metà di questi danni se lo riesce.

\emph{Soffio Indebolente.} Il drago esala del gas in un cono di 9 metri. Ogni creatura in quell'area deve riuscire un Tiro Salvezza di Tempra CD 17 o avere -1d6 ai tiri di attacco basati sulla Forza, prove di Forza, e Tiri Salvezza su Tempra per 1 minuto. Una creatura può ripetere il Tiro Salvezza al termine di ciascun suo turno, terminando l'effetto su di sé in caso di successo.

\textbf{Ecologia}\\
Ambiente: Pianure calde\\
Organizzazione: Solitario\\
Tesoro: Triplo\\
\textbf{Descrizione}\\
I draghi d'oro sono l'emblema della virtù. Gli altri draghi metallici li riveriscono come agenti delle potenze divine e membri esemplari della razza draconica, e spesso li cercano per consigli o aiuto.\\
\textbf{Incantesimi}\index{Incantesimi da Drago d'Oro}\\
Il drago e' anche capace, se di eta' giovanile o piu' anziano, di lanciare incantesimi.\\
Puo' lanciare un numero di incantesimi pari al suo valore di Carisma, senza aver necessita' di componenti magici.\\
Non deve fare prove di magia, si considera che riesca sempre con un punteggio pari alla Difficoltà + il valore del Carisma.\\
Se deve fare un Tiro per Colpire ha CA = Grado di Sfida con un bonus al colpire pari alla Forza\\
Gli incantesimi a disposizione sono:\\
- Blocca Mostri\\
- Muro di Fuoco\\
- Porta Dimensionale\\

\medskip\index{Mostri - Drago d'Oro Cucciolo}\textbf{Drago d'Oro Cucciolo}

\emph{Media drago, legale buono}

\textbf{FORZA} +4

\textbf{DESTREZZA} +2

\textbf{COSTITUZIONE} +3

\textbf{INTELLIGENZA} +2

\textbf{SAGGEZZA} +0

\textbf{CARISMA} +3

\textbf{Iniziativa} +2 -- \textbf{Difesa} 19

\textbf{Punti Ferita} 60 (8d8 + 24)

\textbf{Movimento} 9 m, nuoto 9 m, volo 18 m

\textbf{Tiri Salvezza} Tempra +3, Riflessi +2, Volontà +1

\textbf{Competenze} Muoversi Silenziosamente / Nascondersi +4, Consapevolezza +4

\textbf{Immunità al Danno} fuoco

\textbf{Sensi} scurovisione 18 m, vista cieca 3 m

\textbf{Linguaggi} Draconico

\textbf{Sfida} 3 (700 PE)

\emph{\textbf{Anfibio.}} Il drago può respirare aria e acqua.

\textbf{Azioni}

\emph{\textbf{Morso.} Attacco con arma da mischia}: +6 a colpire, portata 1 m, un bersaglio.

\emph{Colpisce:} 9 (1d10 + 4) danni perforanti.

\emph{\textbf{Arma a Soffio (Ricarica 5-6).}} Il drago usa una delle seguenti armi a soffio:

\emph{Soffio Infuocato.} Il drago esala fuoco in un cono di 5 metri. Ogni creatura nell'area deve effettuare un Tiro Salvezza di Riflessi CD 13, subendo 22 (4d10) danni da fuoco se fallisce il Tiro Salvezza, o la metà di questi danni se lo riesce.

\emph{Soffio Indebolente.} Il drago esala del gas in un cono di 5 metri. Ogni creatura in quell'area deve riuscire un Tiro Salvezza su Tempra CD 13 o avere -1d6 ai tiri di attacco basati sulla Forza, prove di Forza, e Tiri Salvezza su Tempra per 1 minuto. Una creatura può ripetere il Tiro Salvezza al termine di ciascun suo turno, terminando l'effetto su di sé in caso di successo.

\textbf{Ecologia}\\
Ambiente: Pianure calde\\
Organizzazione: Solitario\\
Tesoro: Triplo\\
\textbf{Descrizione}\\
I draghi d'oro sono l'emblema della virtù. Gli altri draghi metallici li riveriscono come agenti delle potenze divine e membri esemplari della razza draconica, e spesso li cercano per consigli o aiuto.\\


\medskip\index{Mostri - Drago d'Ottone Antico}\textbf{Drago d'Ottone Antico}

\emph{Mastodontica drago, caotico buono}

\textbf{FORZA} +8

\textbf{DESTREZZA} +0

\textbf{COSTITUZIONE} +7

\textbf{INTELLIGENZA} +3

\textbf{SAGGEZZA} +2

\textbf{CARISMA} +4

\textbf{Iniziativa} +3 -- \textbf{Difesa} 30

\textbf{Punti Ferita} 297 (17d20 + 119)

\textbf{Movimento} 12 m, scavo 12 m, volo 24 m

\textbf{Tiri Salvezza} Tempra +20, Riflessi +13, Volontà +18

\textbf{Competenze} Muoversi Silenziosamente / Nascondersi +6, Consapevolezza +14, Ingannare +10, Storia +9 

\textbf{Immunità al Danno} fuoco, arma +1

\textbf{Sensi} scurovisione 36 m, vista cieca 18 m

\textbf{Linguaggi} Comune, Draconico

\textbf{Sfida} 20 (25.000 PE)

\emph{\textbf{Resistenza Leggendaria (3/Giorno).}} Se il drago fallisce un Tiro Salvezza, può scegliere invece di riuscire.

\textbf{Azioni}

\emph{\textbf{Multiattacco.}} Il drago può usare la sua Presenza Spaventosa. Poi effettuare tre attacchi: uno con il morso e due con gli artigli.

\emph{\textbf{Artiglio.} Attacco con arma da mischia}: +14 a colpire, portata 3 m, un bersaglio.

\emph{Colpisce:} 15 (2d6 + 8) danni taglienti.

\emph{\textbf{Coda.} Attacco con arma da mischia}: +14 a colpire, portata 6 m, un bersaglio.

\emph{Colpisce:} 17 (2d8 + 8) danni da botta.

\emph{\textbf{Morso.} Attacco con arma da mischia}: +14 a colpire, portata 5 metri, un bersaglio.

\emph{Colpisce:} 19 (2d10 + 8) danni perforanti.

\emph{\textbf{Presenza Spaventosa.}} Ogni creatura scelta dal drago, che si trovi entro 36 metri da esso e consapevole della sua presenza, deve riuscire un Tiro Salvezza di Volontà CD 18 o restare spaventata per 1 minuto. Una creatura può ripetere il Tiro Salvezza al termine di ciascun suo turno, terminando l'effetto se lo riesce. Se il Tiro Salvezza della creatura ha successo o l'effetto hatermine per essa, la creatura è immune alla Presenza Spaventosa del drago per le successive 24 ore.

\emph{\textbf{Arma a Soffio (Ricarica 5-6).}} Il drago usa una delle seguenti armi a soffio:

\emph{Soffio Infuocato.} Il drago esala fuoco in una linea lunga 27 metri e larga 3 metri. Ogni creatura sulla linea deve effettuare un Tiro Salvezza su Riflessi CD 21, subendo 56 (16d6) danni da fuoco se fallisce il Tiro Salvezza, o la metà di questi danni se lo riesce.

\emph{Soffio Soporifero.} Il drago esala del gas soporifero in un cono di 27 metri. Ogni creatura in quell'area deve riuscire un Tiro Salvezza su Tempra 21 o cadere svenuta per 10 minuti. Questo effetto
termina se la creatura svenuta subisce danni o qualcuno impiega un'azione per risvegliarla.

\emph{\textbf{Mutare Forma.}} Il drago può trasformarsi magicamente in un umanoide o bestia il cui grado di sfida sia pari o inferiore al proprio, o tornare alla sua vera forma. Alla morte ritorna alla sua vera forma. Qualsiasi equipaggiamento stia indossando o trasportando viene assorbito o trasportato nella nuova forma (a scelta del drago).

Nella nuova forma, il drago mantiene il suo allineamento, punti ferita, Dadi Vita, la facoltà di parlare, le competenze, la Resistenza Leggendaria, le azioni da tana, e i punteggi di Intelligenza, Saggezza e Carisma, oltre a questa azione. Le sue statistiche e capacità vengono altrimenti rimpiazzate da quelle della nuova forma, eccetto Azioni aggiuntive della nuova forma.

\textbf{Azioni Aggiuntive}

Il drago può effettuare 3 Azioni aggiuntive, scelte tra le opzioni seguenti. Può usare solo un'opzione leggendaria alla volta e solo al termine del turno di un'altra creatura. Il drago recupera le Azioni aggiuntive spese all'inizio del proprio turno.

\textbf{Attacco di Ala (Costa 2 Azioni).} Il drago batte le ali. Ogni creatura entro 5 metri dal drago deve riuscire un Tiro Salvezza su Riflessi CD 22 o subire 15 (2d6 + 8) danni da botta e venir gettato prono. Il drago può poi volare fino a metà del suo movimento di volo.

\textbf{Attacco di Coda.} Il drago effettua un attacco di coda.

\textbf{Individuare.} Il drago effettua una prova di Saggezza (Consapevolezza).

\textbf{Ecologia}\\
Ambiente: Deserti Caldi\\
Organizzazione: Solitario\\
Tesoro: Triplo\\
\textbf{Descrizione}\\
Ottimi conversatori, i draghi d'ottone preferiscono parlare invece che combattere. I draghi d'ottone fanno la tana vicino agli insediamenti umanoidi, dove possono udire le notizie e i pettegolezzi più recenti.\\
\textbf{Incantesimi}\index{Incantesimi da Drago d'Ottone}\\
Il drago e' anche capace, se di eta' giovanile o piu' anziano, di lanciare incantesimi.\\
Puo' lanciare un numero di incantesimi pari al suo valore di Carisma, senza aver necessita' di componenti magici.\\
Non deve fare prove di magia, si considera che riesca sempre con un punteggio pari alla Difficoltà + il valore del Carisma.\\
Se deve fare un Tiro per Colpire ha CA = Grado di Sfida con un bonus al colpire pari alla Forza\\
Gli incantesimi a disposizione sono:\\
- Inviare\\
- Trama Ipnotica\\
- Lingue\\


\medskip\index{Mostri - Drago d'Ottone Adulto}\textbf{Drago d'Ottone Adulto}

\emph{Enorme drago, caotico buono}

\textbf{FORZA} +6

\textbf{DESTREZZA} +0

\textbf{COSTITUZIONE} +5

\textbf{INTELLIGENZA} +2

\textbf{SAGGEZZA} +1

\textbf{CARISMA} +3

\textbf{Iniziativa} +2 -- \textbf{Difesa} 25

\textbf{Punti Ferita} 172 (15d12 + 75)

\textbf{Movimento} 12 m, scavo 9 m, volo 24 m

\textbf{Tiri Salvezza} Tempra +14, Riflessi +10, Volontà +12

\textbf{Competenze} Muoversi Silenziosamente / Nascondersi +5, Consapevolezza +11, Ingannare +8, Storia +7

\textbf{Immunità al Danno} fuoco

\textbf{Sensi} scurovisione 36 m, vista cieca 18 m 

\textbf{Linguaggi} Comune, Draconico

\textbf{Sfida} 13 (10.000 PE)

\emph{\textbf{Resistenza Leggendaria (3/Giorno).}} Se il drago fallisce un Tiro Salvezza, può scegliere invece di riuscire.

\textbf{Azioni}

\emph{\textbf{Multiattacco.}} Il drago può usare la sua Presenza Spaventosa. Poi effettuare tre attacchi: uno con il morso e due con gli artigli.

\emph{\textbf{Artiglio.} Attacco con arma da mischia}: +11 a colpire, portata 1 m, un bersaglio.

\emph{Colpisce:} 13 (2d6 + 6) danni taglienti.

\emph{\textbf{Coda.} Attacco con arma da mischia}: +11 a colpire, portata 5 metri, un bersaglio.

\emph{Colpisce:} 15 (2d8 + 6) danni da botta.

\emph{\textbf{Morso.} Attacco con arma da mischia}: +11 a colpire, portata 3 m, un bersaglio.

\emph{Colpisce:} 17 (2d10 + 6) danni perforanti.

\emph{\textbf{Presenza Spaventosa.}} Ogni creatura scelta dal drago, che si trovi entro 36 metri da esso e consapevole della sua presenza, deve riuscire un Tiro Salvezza di Volontà CD 16 o restare spaventata per 1 minuto. Una creatura può ripetere il Tiro Salvezza al termine di ciascun suo turno, terminando l'effetto se lo riesce. Se il Tiro Salvezza della creatura ha successo o l'effetto ha termine per essa, la creatura è immune alla Presenza Spaventosa del drago per le successive 24 ore.

\emph{\textbf{Arma a Soffio (Ricarica 5-6).}} Il drago usa una delle seguenti armi a soffio:

\emph{Soffio Infuocato.} Il drago esala fuoco in una linea lunga 18 metri e larga 1 metro. Ogni creatura sulla linea deve effettuare un Tiro Salvezza di Riflessi CD 18, subendo 45 (13d6) danni da fuoco se fallisce il Tiro Salvezza, o la metà di questi danni se lo riesce. 

\emph{Soffio Soporifero.} Il drago esala del gas soporifero in un cono di 18 metri. Ogni creatura in quell'area deve riuscire un Tiro Salvezza su Tempra 18 o cadere svenuta per 10 minuti. Questo effetto termina se la creatura svenuta subisce danni o qualcuno impiega un'azione per risvegliarla.

\textbf{Azioni Aggiuntive}

Il drago può effettuare 3 Azioni aggiuntive, scelte tra le opzioni seguenti. Può usare solo un'opzione leggendaria alla volta e solo al termine del turno di un'altra creatura. Il drago recupera le Azioni aggiuntive spese all'inizio del proprio turno.

\textbf{Attacco di Ala (Costa 2 Azioni).} Il drago batte le ali. Ogni creatura entro 3 metri dal drago deve riuscire un Tiro Salvezza su Riflessi CD 19 o subire 13 (2d6 + 6) danni da botta e venir gettato  prono. Il drago può poi volare fino a metà del suo movimento di volo.

\textbf{Attacco di Coda.} Il drago effettua un attacco di coda. 

\textbf{Individuare.} Il drago effettua una prova di Saggezza (Consapevolezza).

\textbf{Ecologia}\\
Ambiente: Deserti Caldi\\
Organizzazione: Solitario\\
Tesoro: Triplo\\
\textbf{Descrizione}\\
Ottimi conversatori, i draghi d'ottone preferiscono parlare invece che combattere. I draghi d'ottone fanno la tana vicino agli insediamenti umanoidi, dove possono udire le notizie e i pettegolezzi più recenti.\\
\textbf{Incantesimi}\index{Incantesimi da Drago d'Ottone}\\
Il drago e' anche capace, se di eta' giovanile o piu' anziano, di lanciare incantesimi.\\
Puo' lanciare un numero di incantesimi pari al suo valore di Carisma, senza aver necessita' di componenti magici.\\
Non deve fare prove di magia, si considera che riesca sempre con un punteggio pari alla Difficoltà + il valore del Carisma.\\
Se deve fare un Tiro per Colpire ha CA = Grado di Sfida con un bonus al colpire pari alla Forza\\
Gli incantesimi a disposizione sono:\\
- Inviare\\
- Trama Ipnotica\\
- Lingue\\

\medskip\index{Mostri - Drago d'Ottone Giovane}\textbf{Drago d'Ottone Giovane}

\emph{Grande drago, caotico buono}

\textbf{FORZA} +4

\textbf{DESTREZZA} +0

\textbf{COSTITUZIONE} +3

\textbf{INTELLIGENZA} +1

\textbf{SAGGEZZA} +0

\textbf{CARISMA} +2

\textbf{Iniziativa} +1 -- \textbf{Difesa} 20

\textbf{Punti Ferita} 110 (13d10 + 39)

\textbf{Movimento} 12 m, scavo 6 m, volo 24 m

\textbf{Tiri Salvezza} Tempra +9, Riflessi +8, Volontà +7

\textbf{Competenze} Muoversi Silenziosamente / Nascondersi +3, Consapevolezza +6, Ingannare +5

\textbf{Immunità al Danno} fuoco

\textbf{Sensi} scurovisione 36 m, vista cieca 9 m

\textbf{Linguaggi} Comune, Draconico

\textbf{Sfida} 6 (2.300 PE)

\textbf{Azioni}

\emph{\textbf{Multiattacco.}} Il drago può effettuare tre attacchi: uno con il morso e due con gli artigli.

\emph{\textbf{Artiglio.} Attacco con arma da mischia}: +7 a colpire, portata 1 m, un bersaglio.

\emph{Colpisce:} 11 (2d6 + 4) danni taglienti.

\emph{\textbf{Morso.} Attacco con arma da mischia}: +7 a colpire, portata 3 m, un bersaglio.

\emph{Colpisce:} 15 (2d10 + 4) danni perforanti.

\emph{\textbf{Arma a Soffio (Ricarica 5-6).}} Il drago usa una delle seguenti armi a soffio:

\emph{Soffio Infuocato.} Il drago esala fuoco in una linea lunga 12 metri e larga 1 metro. Ogni creatura sulla linea deve effettuare un Tiro Salvezza di Riflessi CD 14, subendo 42 (12d6) danni da fuoco se fallisce il Tiro Salvezza, o la metà di questi danni se lo riesce. \emph{Soffio Soporifero.} Il drago esala del gas soporifero in un cono di 9 metri. Ogni creatura in quell'area deve riuscire un Tiro Salvezza su Tempra 14 o cadere svenuta per 5 minuti. Questo effetto termina se la creatura svenuta subisce danni o qualcuno impiega un'azione per risvegliarla.

\textbf{Ecologia}\\
Ambiente: Deserti Caldi\\
Organizzazione: Solitario\\
Tesoro: Triplo\\
\textbf{Descrizione}\\
Ottimi conversatori, i draghi d'ottone preferiscono parlare invece che combattere. I draghi d'ottone fanno la tana vicino agli insediamenti umanoidi, dove possono udire le notizie e i pettegolezzi più recenti.\\
\textbf{Incantesimi}\index{Incantesimi da Drago d'Ottone}\\
Il drago e' anche capace, se di eta' giovanile o piu' anziano, di lanciare incantesimi.\\
Puo' lanciare un numero di incantesimi pari al suo valore di Carisma, senza aver necessita' di componenti magici.\\
Non deve fare prove di magia, si considera che riesca sempre con un punteggio pari alla Difficoltà + il valore del Carisma.\\
Se deve fare un Tiro per Colpire ha CA = Grado di Sfida con un bonus al colpire pari alla Forza\\
Gli incantesimi a disposizione sono:\\
- Inviare\\
- Trama Ipnotica\\
- Lingue\\

\medskip\index{Mostri - Drago d'Ottone Cucciolo}\textbf{Drago d'Ottone Cucciolo}

\emph{Media drago, caotico buono}

\textbf{FORZA} +2

\textbf{DESTREZZA} +0

\textbf{COSTITUZIONE} +1

\textbf{INTELLIGENZA} +0

\textbf{SAGGEZZA} +0

\textbf{CARISMA} +1

\textbf{Iniziativa} +0 -- \textbf{Difesa} 17

\textbf{Punti Ferita} 16 (3d8 + 3)

\textbf{Movimento} 9 m, scavo 5 metri, volo 18 m

\textbf{Tiri Salvezza} Tempra +2, Riflessi +0, Volontà +1

\textbf{Competenze} Muoversi Silenziosamente / Nascondersi +2, Consapevolezza +4

\textbf{Immunità al Danno} fuoco

\textbf{Sensi} scurovisione 18 m, vista cieca 3 m

\textbf{Linguaggi} Draconico

\textbf{Sfida} 1 (200 PE)

\textbf{Azioni}

\emph{\textbf{Morso.} Attacco con arma da mischia}: +4 a colpire, portata 1 m, un bersaglio.

\emph{Colpisce:} 7 (1d10 + 2) danni perforanti.

\emph{\textbf{Arma a Soffio (Ricarica 5-6).}} Il drago usa una delle seguenti armi a soffio:

\emph{Soffio Infuocato.} Il drago esala fuoco in una linea lunga 6 metri e larga 1 metro. Ogni creatura sulla linea deve effettuare un Tiro Salvezza su Riflessi CD 11, subendo 14 (4d6) danni da fuoco se fallisce il Tiro Salvezza, o la metà di questi danni se lo riesce.

\emph{Soffio Soporifero.} Il drago esala del gas soporifero in un cono di 5 metri. Ogni creatura in quell'area deve riuscire un Tiro Salvezza su Tempra 11 o cadere svenuta per 1 minuto. Questo effetto termina se la creatura svenuta subisce danni o qualcuno impiega un'azione per risvegliarla.

\textbf{Ecologia}\\
Ambiente: Deserti Caldi\\
Organizzazione: Solitario\\
Tesoro: Triplo\\
\textbf{Descrizione}\\
Ottimi conversatori, i draghi d'ottone preferiscono parlare invece che combattere. I draghi d'ottone fanno la tana vicino agli insediamenti umanoidi, dove possono udire le notizie e i pettegolezzi più recenti.\\



\medskip\index{Mostri - Drago di Rame Antico}\textbf{Drago di Rame Antico}

\emph{Mastodontica drago, caotico buono}

\textbf{FORZA} +8

\textbf{DESTREZZA} +1

\textbf{COSTITUZIONE} +7

\textbf{INTELLIGENZA} +5

\textbf{SAGGEZZA} +3

\textbf{CARISMA} +4

\textbf{Iniziativa} +5 -- \textbf{Difesa} 33

\textbf{Punti Ferita} 350 (20d20 + 140) 

\textbf{Movimento} 12 m, scalata 12 m, volo 24 m

\textbf{Tiri Salvezza} Tempra +20, Riflessi +13, Volontà +19

\textbf{Competenze} Muoversi Silenziosamente / Nascondersi +8, Ingannare +11, Consapevolezza +17

\textbf{Immunità al Danno} acido, arma +1

\textbf{Sensi} scurovisione 36 m, vista cieca 18 m

\textbf{Linguaggi} Comune, Draconico

\textbf{Sfida} 21 (33.000 PE)

\emph{\textbf{Resistenza Leggendaria (3/Giorno).}} Se il drago fallisce un Tiro Salvezza, può scegliere invece di riuscire.

\textbf{Azioni}

\emph{\textbf{Multiattacco.}} Il drago può usare la sua Presenza Spaventosa. Poi effettuare tre attacchi: uno con il morso e due con gli artigli.

\emph{\textbf{Artiglio.} Attacco con arma da mischia}: +15 a colpire, portata 3 m, un bersaglio.

\emph{Colpisce:} 15 (2d6 + 8) danni taglienti.

\emph{\textbf{Coda.} Attacco con arma da mischia}: +15 a colpire, portata 6 m, un bersaglio.

\emph{Colpisce:} 17 (2d8 + 8) danni da botta.

\emph{\textbf{Morso.} Attacco con arma da mischia}: +15 a colpire, portata 5 metri, un bersaglio.

\emph{Colpisce:} 19 (2d10 + 8) danni perforanti.

\emph{\textbf{Presenza Spaventosa.}} Ogni creatura scelta dal drago, che si trovi entro 36 metri da esso e consapevole della sua presenza, deve riuscire un Tiro Salvezza di Volontà CD 19 o restare spaventata per 1 minuto. Una creatura può ripetere il Tiro Salvezza al termine di ciascun suo turno, terminando l'effetto se lo riesce. Se il Tiro Salvezza della creatura ha successo o l'effetto ha termine per essa, la creatura è immune alla Presenza Spaventosa del drago per le successive 24 ore.

\emph{\textbf{Arma a Soffio (Ricarica 5-6).}} Il drago usa una delle seguenti armi a soffio:

\emph{Soffio Acido.} Il drago esala acido in una linea lunga 27 metri e larga 3 metri. Ogni creatura sulla linea deve effettuare un Tiro Salvezza su Riflessi CD 22, subendo 63 (14d8) danni da acido se fallisce il Tiro Salvezza, o la metà di questi danni se lo riesce.

\emph{Soffio Rallentante.} Il drago esala del gas in un cono di 27 metri. Ogni creatura in quell'area deve riuscire un Tiro Salvezza su Tempra CD 22. Se fallisce il Tiro Salvezza, la creatura non può usare la sua reazione, ha la velocità dimezzata, e non può effettuare più di un attacco durante il suo turno. Inoltre, la creatura può usare un'azione o un'azione bonus, ma non entrambe. Questi effetti permangono 1 minuto. La creatura può ripetere il Tiro Salvezza al termine di ciascun suo turno, terminando l'effetto su di sé in caso di successo.

\emph{\textbf{Mutare Forma.}} Il drago può trasformarsi magicamente in un umanoide o bestia il cui grado di sfida sia pari o inferiore al proprio, o tornare alla sua vera forma. Alla morte ritorna alla sua vera forma. Qualsiasi equipaggiamento stia indossando o trasportando viene assorbito o trasportato nella nuova forma (a scelta del drago).

Nella nuova forma, il drago mantiene il suo allineamento, punti ferita, Dadi Vita, la facoltà di parlare, le competenze, la Resistenza Leggendaria, le azioni da tana, e i punteggi di Intelligenza, Saggezza e Carisma, oltre a questa azione. Le sue statistiche e capacità

vengono altrimenti rimpiazzate da quelle della nuova forma, eccetto Azioni aggiuntive della nuova forma.

\textbf{Azioni Aggiuntive}

Il drago può effettuare 3 Azioni aggiuntive, scelte tra le opzioni seguenti. Può usare solo un'opzione leggendaria alla volta e solo al termine del turno di un'altra creatura. Il drago recupera le Azioni aggiuntive spese all'inizio del proprio turno.

\textbf{Attacco di Ala (Costa 2 Azioni).} Il drago batte le ali. Ogni creatura entro 5 metri dal drago deve riuscire un Tiro Salvezza su Riflessi CD 23 o subire 15 (2d6 + 8) danni da botta e venir gettato prono. Il drago può poi volare fino a metà del suo movimento di volo.

\textbf{Attacco di Coda.} Il drago effettua un attacco di coda.

\textbf{Individuare.} Il drago effettua una prova di Saggezza (Consapevolezza).

\textbf{Ecologia}\\
Ambiente: Colline Calde\\
Organizzazione: Solitario\\
Tesoro: Triplo\\
\textbf{Descrizione}\\
Questo drago capriccioso durante il combattimento cerca di ostacolare e frustrare i suoi nemici.\\
\textbf{Incantesimi}\index{Incantesimi da Drago di Rame}\\
Il drago e' anche capace, se di eta' giovanile o piu' anziano, di lanciare incantesimi.\\
Puo' lanciare un numero di incantesimi pari al suo valore di Carisma, senza aver necessita' di componenti magici.\\
Non deve fare prove di magia, si considera che riesca sempre con un punteggio pari alla Difficoltà + il valore del Carisma.\\
Se deve fare un Tiro per Colpire ha CA = Grado di Sfida con un bonus al colpire pari alla Forza\\
Gli incantesimi a disposizione sono:\\
- Metamorfosi\\
- Confusione\\
- Nube maleodorante\\


\medskip\index{Mostri - Drago di Rame Adulto}\textbf{Drago di Rame Adulto}

\emph{Enorme drago, caotico buono}

\textbf{FORZA} +6

\textbf{DESTREZZA} +1

\textbf{COSTITUZIONE} +5

\textbf{INTELLIGENZA} +4

\textbf{SAGGEZZA} +2

\textbf{CARISMA} +3

\textbf{Iniziativa} +4 -- \textbf{Difesa} 25

\textbf{Punti Ferita} 184 (16d12 + 80)

\textbf{Movimento} 12 m, scalata 12 m, volo 24 m

\textbf{Tiri Salvezza} Tempra +14, Riflessi +10, Volontà +13

\textbf{Competenze} Muoversi Silenziosamente / Nascondersi +6, Ingannare +8, Consapevolezza +12

\textbf{Immunità al Danno} acido

\textbf{Sensi} scurovisione 36 m, vista cieca 18 m

\textbf{Linguaggi} Comune, Draconico

\textbf{Sfida} 14 (11.500 PE)

\emph{\textbf{Resistenza Leggendaria (3/Giorno).}} Se il drago fallisce un Tiro Salvezza, può scegliere invece di riuscire.

\textbf{Azioni}

\emph{\textbf{Multiattacco.}} Il drago può usare la sua Presenza Spaventosa. Poi effettuare tre attacchi: uno con il morso e due con gli artigli.

\emph{\textbf{Artiglio.} Attacco con arma da mischia}: +11 a colpire, portata 1 m, un bersaglio.

\emph{Colpisce:} 13 (2d6 + 6) danni taglienti.

\emph{\textbf{Coda.} Attacco con arma da mischia}: +11 a colpire, portata 5 metri, un bersaglio.

\emph{Colpisce:} 15 (2d8 + 6) danni da botta.

\emph{\textbf{Morso.} Attacco con arma da mischia}: +11 a colpire, portata 3 m, un bersaglio.

\emph{Colpisce:} 17 (2d10 + 6) danni perforanti.

\emph{\textbf{Presenza Spaventosa.}} Ogni creatura scelta dal drago, che si trovi entro 36 metri da esso e consapevole della sua presenza, deve riuscire un Tiro Salvezza di Volontà CD 16 o restare spaventata per 1 minuto. Una creatura può ripetere il Tiro Salvezza al termine di ciascun suo turno, terminando l'effetto se lo riesce. Se il Tiro Salvezza della creatura ha successo o l'effetto ha termine per essa, la creatura è immune alla Presenza Spaventosa del drago per le successive 24 ore.

\emph{\textbf{Arma a Soffio (Ricarica 5-6).}} Il drago usa una delle seguenti armi a soffio:

\emph{Soffio Acido.} Il drago esala acido in una linea lunga 18 metri e larga 1 metro. Ogni creatura sulla linea deve effettuare un Tiro Salvezza su Riflessi CD 18, subendo 54 (12d8) danni da acido se fallisce il Tiro Salvezza, o la metà di questi danni se lo riesce.

\emph{Soffio Rallentante.} Il drago esala del gas in un cono di 18 metri. Ogni creatura in quell'area deve riuscire un Tiro Salvezza su Tempra CD 18. Se fallisce il Tiro Salvezza, la creatura non può usare la sua reazione, ha la velocità dimezzata, e non può effettuare più di un attacco durante il suo turno. Inoltre, la creatura può usare un'azione o un'azione bonus, ma non entrambe. Questi effetti permangono 1 minuto. La creatura può ripetere il Tiro Salvezza al termine di ciascun suo turno, terminando l'effetto su di sé in caso di successo.

\textbf{Azioni Aggiuntive}

Il drago può effettuare 3 Azioni aggiuntive, scelte tra le opzioni seguenti. Può usare solo un'opzione leggendaria alla volta e solo al termine del turno di un'altra creatura. Il drago recupera le Azioni aggiuntive spese all'inizio del proprio turno.

\textbf{Attacco di Ala (Costa 2 Azioni).} Il drago batte le ali. Ogni creatura entro 3 metri dal drago deve riuscire un Tiro Salvezza su Riflessi CD 19 o subire 13 (2d6 + 6) danni da botta e venir gettato prono. Il drago può poi volare fino a metà del suo movimento di volo.

\textbf{Attacco di Coda.} Il drago effettua un attacco di coda. 

\textbf{Individuare.} Il drago effettua una prova di Saggezza (Consapevolezza).

\textbf{Ecologia}\\
Ambiente: Colline Calde\\
Organizzazione: Solitario\\
Tesoro: Triplo\\
\textbf{Descrizione}\\
Questo drago capriccioso durante il combattimento cerca di ostacolare e frustrare i suoi nemici.\\
\textbf{Incantesimi}\index{Incantesimi da Drago di Rame}\\
Il drago e' anche capace, se di eta' giovanile o piu' anziano, di lanciare incantesimi.\\
Puo' lanciare un numero di incantesimi pari al suo valore di Carisma, senza aver necessita' di componenti magici.\\
Non deve fare prove di magia, si considera che riesca sempre con un punteggio pari alla Difficoltà + il valore del Carisma.\\
Se deve fare un Tiro per Colpire ha CA = Grado di Sfida con un bonus al colpire pari alla Forza\\
Gli incantesimi a disposizione sono:\\
- Metamorfosi\\
- Confusione\\
- Nube maleodorante\\


\medskip\index{Mostri - Drago di Rame Giovane}\textbf{Drago di Rame Giovane}

\emph{Grande drago, caotico buono}

\textbf{FORZA} +4

\textbf{DESTREZZA} +1

\textbf{COSTITUZIONE} +3

\textbf{INTELLIGENZA} +3

\textbf{SAGGEZZA} +1

\textbf{CARISMA} +2

\textbf{Iniziativa} +3 -- \textbf{Difesa} 21

\textbf{Punti Ferita} 119 (14d10 + 42)

\textbf{Movimento} 12 m, scalata 12 m, volo 24 m

\textbf{Tiri Salvezza} Tempra +9, Riflessi +8, Volontà +8

\textbf{Competenze} Muoversi Silenziosamente / Nascondersi +4, Ingannare +5, Consapevolezza +7

\textbf{Immunità al Danno} acido

\textbf{Sensi} scurovisione 36 m, vista cieca 9 m

\textbf{Linguaggi} Comune, Draconico

\textbf{Sfida} 7 (2.900 PE)

\textbf{Azioni}

\emph{\textbf{Multiattacco.}} Il drago può effettuare tre attacchi: uno con il morso e due con gli artigli.

\emph{\textbf{Artiglio.} Attacco con arma da mischia}: +7 a colpire, portata 1 m, un bersaglio.

\emph{Colpisce:} 11 (2d6 + 4) danni taglienti.

\emph{\textbf{Morso.} Attacco con arma da mischia}: +7 a colpire, portata 3 m, un bersaglio.

\emph{Colpisce:} 15 (2d10 + 4) danni perforanti.

\emph{\textbf{Arma a Soffio (Ricarica 5-6).}} Il drago usa una delle seguenti armi a soffio:

\emph{Soffio Acido.} Il drago esala acido in una linea lunga 12 metri e larga 1 metro. Ogni creatura sulla linea deve effettuare un Tiro Salvezza su Riflessi CD 14, subendo 40 (9d8) danni da acido se fallisce il Tiro Salvezza, o la metà di questi danni se lo riesce.

\emph{Soffio Rallentante.} Il drago esala del gas in un cono di 9 metri. Ogni creatura in quell'area deve riuscire un Tiro Salvezza su Tempra CD 14. Se fallisce il Tiro Salvezza, la creatura non può usare la sua reazione, ha la velocità dimezzata, e non può effettuare più di un attacco durante il suo turno. Inoltre, la creatura può usare un'azione o un'azione bonus, ma non entrambe. Questi effetti permangono 1 minuto. La creatura può ripetere il Tiro Salvezza al termine di ciascun suo turno, terminando l'effetto su di sé in caso di successo.

\textbf{Ecologia}\\
Ambiente: Colline Calde\\
Organizzazione: Solitario\\
Tesoro: Triplo\\
\textbf{Descrizione}\\
Questo drago capriccioso durante il combattimento cerca di ostacolare e frustrare i suoi nemici.\\
\textbf{Incantesimi}\index{Incantesimi da Drago di Rame}\\
Il drago e' anche capace, se di eta' giovanile o piu' anziano, di lanciare incantesimi.\\
Puo' lanciare un numero di incantesimi pari al suo valore di Carisma, senza aver necessita' di componenti magici.\\
Non deve fare prove di magia, si considera che riesca sempre con un punteggio pari alla Difficoltà + il valore del Carisma.\\
Se deve fare un Tiro per Colpire ha CA = Grado di Sfida con un bonus al colpire pari alla Forza\\
Gli incantesimi a disposizione sono:\\
- Metamorfosi\\
- Confusione\\
- Nube maleodorante\\

\textbf{Drago di Rame Cucciolo}

\emph{Media drago, caotico buono}

\textbf{FORZA} +2

\textbf{DESTREZZA} +1

\textbf{COSTITUZIONE} +1

\textbf{INTELLIGENZA} +2

\textbf{SAGGEZZA} +0

\textbf{CARISMA} +1

\textbf{Iniziativa} +2 -- \textbf{Difesa} 17

\textbf{Punti Ferita} 22 (4d8 + 4)

\textbf{Movimento} 9 m, scalata 9 m, volo 18 m

\textbf{Tiri Salvezza} Tempra +2, Riflessi +2, Volontà +0

\textbf{Competenze} Muoversi Silenziosamente / Nascondersi +3, Consapevolezza +4

\textbf{Immunità al Danno} acido

\textbf{Sensi} scurovisione 18 m, vista cieca 3 m

\textbf{Linguaggi} Draconico

\textbf{Sfida} 1 (200 PE)

\textbf{Azioni}

\emph{\textbf{Morso.} Attacco con arma da mischia}: +4 a colpire, portata 1 m, un bersaglio.

\emph{Colpisce:} 7 (1d10 + 2) danni perforanti.

\emph{\textbf{Arma a Soffio (Ricarica 5-6).}} Il drago usa una delle seguenti armi a soffio:

\emph{Soffio Acido.} Il drago esala acido in una linea lunga 6 metri e larga 1 metro. Ogni creatura sulla linea deve effettuare un Tiro Salvezza su Riflessi CD 11, subendo 18 (4d8) danni da acido se fallisce il Tiro Salvezza, o la metà di questi danni se lo riesce.

\emph{Soffio Rallentante.} Il drago esala del gas in un cono di 5 metri. Ogni creatura in quell'area deve riuscire un Tiro Salvezza su Tempra CD 11. Se fallisce il Tiro Salvezza, la creatura non può usare la sua reazione, ha la velocità dimezzata, e non può effettuare più di un attacco durante il suo turno. Inoltre, la creatura può usare un'azione o un'azione bonus, ma non entrambe. Questi effetti permangono 1 minuto. La creatura può ripetere il Tiro Salvezza al termine di ciascun suo turno, terminando l'effetto su di sé in caso di successo.

\textbf{Ecologia}\\
Ambiente: Colline Calde\\
Organizzazione: Solitario\\
Tesoro: Triplo\\
\textbf{Descrizione}\\
Questo drago capriccioso durante il combattimento cerca di ostacolare e frustrare i suoi nemici.\\


\medskip\index{Mostri - Drider}\textbf{Drider}

\emph{Grande mostruosità, caotico malvagio}

\textbf{FORZA} +3

\textbf{DESTREZZA} +3

\textbf{COSTITUZIONE} +4

\textbf{INTELLIGENZA} +1

\textbf{SAGGEZZA} +2

\textbf{CARISMA} +1

\textbf{Iniziativa} +3 -- \textbf{Difesa} 22

\textbf{Punti Ferita} 123 (13d10 + 52)

\textbf{Movimento} 9 m, scalata 9 m

\textbf{Tiri Salvezza} Tempra +7, Riflessi +5, Volontà +9

\textbf{Competenze} Muoversi Silenziosamente / Nascondersi +9, Consapevolezza +5

\textbf{Sensi} scurovisione 36 m

\textbf{Linguaggi} Elfico, Linguaggio delle Profondità

\textbf{Sfida} 6 (2.300 PE)

\emph{\textbf{Camminare sulla Tela.}} Il drider ignora le restrizioni al movimento provocate dalle ragnatele.

\emph{\textbf{Discendenza Fatata.}} Il drider ha +1d6 ai Tiri Salvezza per non restare affascinato, e la magia non può far addormentare un drider.

\emph{\textbf{Incantesimi Innati.}} La caratteristica da incantatore innato del drider è la Saggezza. Il drider può lanciare in maniera innata i seguenti incantesimi, senza bisogno di componenti materiali:

A volontà: \emph{luci danzanti}

1/Giorno: \emph{luminescenza, oscurità}

\emph{\textbf{Scalare come Ragno.}} Il drider può scalare superfici difficili, compreso lo stare a testa in giù sul soffitto, senza bisogno di effettuare una prova di abilità.

\textbf{Azioni}

\emph{\textbf{Multiattacco.}} Il drider effettua tre attacchi con la spada lunga o con l'arco lungo. Può rimpiazzare uno di questi attacchi con un attacco di morso.

\emph{\textbf{Morso.} Attacco con arma da mischia}: +6 a colpire, portata 1 m, una creatura.

\emph{Colpisce:} 2 (1d4) danni perforanti più 9 (2d8) danni da veleno.

\emph{\textbf{Spada Lunga.} Attacco con arma da mischia}: +6 a colpire, portata 1 m, un bersaglio.

\emph{Colpisce:} 7 (1d8 + 3) danni taglienti, o 8 (1d8 + 3) danni taglienti se usata con due mani.

\emph{\textbf{Arco Lungo.} Attacco con arma a Distanza}: +6 a colpire, gittata 45m, un bersaglio.

\emph{Colpisce:} 7 (1d8 + 3) danni perforanti più 4 (1d8) danni da veleno.

\textbf{Ecologia}\\
Ambiente: Qualsiasi sotterraneo\\
Organizzazione: Solitario, coppia o gruppo (3-8)\\
Tesoro: doppio (Mazza Pesante Perfetta, Arco Lungo Composito Perfetto [Forza +2] con 20 Frecce, altro tesoro)\\
\textbf{Descrizione}\\
Creato dal corpo di un drow, alterato e mutato attraverso speciali veleni ed elisir per assumere le caratteristiche di un ragno gigante, il drider è una creatura pericolosa.\\
I drider sono sessualmente dimorfici. La parte inferiore da ragno del corpo di un drider femmina è lucente ed aggraziata, spesso simile al corpo di una vedova nera, mentre il busto superiore di drow mantiene le sue curve allettanti e il bel viso (con l'eccezione delle venefiche zanne acuminate). La parte inferiore del corpo di un drider maschio è tozza come una tarantola, mentre quella superiore ha un fisico asciutto e supporta un'orrenda faccia più da ragno che da drow, completa di mandibole zannute.\\


\medskip\index{Mostri - Driade}\textbf{Driade}

\emph{Media fatato, neutrale}

\textbf{FORZA} +0

\textbf{DESTREZZA} +1

\textbf{COSTITUZIONE} +0

\textbf{INTELLIGENZA} +2

\textbf{SAGGEZZA} +2

\textbf{CARISMA} +4

\textbf{Iniziativa} +2 -- \textbf{Difesa} 12 (17 con \emph{pelle di corteccia})

\textbf{Punti Ferita} 22 (5d8)

\textbf{Vulnerabilità al Danno} ferro freddo

\textbf{Movimento} 9 m

\textbf{Tiri Salvezza} Tempra +5, Riflessi +9, Volontà +7

\textbf{Competenze} Muoversi Silenziosamente / Nascondersi +5, Consapevolezza +4

\textbf{Sensi} scurovisione 18 m

\textbf{Linguaggi} Elfico, Silvano

\textbf{Sfida} 1 (200 PE)

\emph{\textbf{Camminata Arborea.}} Uno volta durante il suo turno, la driade può usare 3 metri di movimento per entrare magicamente in un albero vivo a sua portata ed emergere da un altro albero vivo entro 18 metri dal primo albero, ricomparendo in uno spazio non occupato entro 1 metro dal secondo albero. Entrambi gli alberi devono essere di taglia Grande o superiore.

\emph{\textbf{Incantesimi Innati.}} La caratteristica da incantatore innato della driade è il Carisma (CD 14 per i Tiri Salvezza degli incantesimi). La driade può lanciare in maniera innata i seguenti incantesimi, senza aver bisogno di componenti materiali. A volontà: 

\emph{arte del druido}

3/giorno ciascuno: \emph{bacche benefiche}, \emph{intralciare} 1/giorno:
\emph{passare senza tracce, pelle coriacea, randello} \emph{incantato}

\emph{\textbf{Parlare con Animali e Piante.}} La driade può comunicare con bestie e piante come se parlassero la stessa lingua.

\emph{\textbf{Resistenza alla Magia.}} La driade ha +1d6 ai Tiri Salvezza contro incantesimi e altri effetti magici.

\textbf{Azioni}

\emph{\textbf{Randello.} Attacco con arma da mischia}: +2 a colpire (+6 a colpire con \emph{bastone}), portata 1 m, un bersaglio.

\emph{Colpisce:} 2 (1d4) danni da botta, o 8 (1d8 + 4) danni da botta con \emph{bastone}

\emph{\textbf{Fascino Fatato.}} La driade può prendere a bersaglio un umanoide o bestia entro 9 metri da lei e che possa vedere. Se il bersaglio può vedere la driade, deve riuscire un Tiro Salvezza su Volontà CD 14 o restare affascinato dalla magia. Le creature affascinate considerano la driade un'amica fidata da ascoltare e proteggere. Sebbene il bersaglio non sia sotto il controllo della driade, interpreterà le richieste o le azioni della driade nel modo più favorevole possibile.

Ogni volta che la driade o i suoi alleati arrecano danno al bersaglio, esso può ripetere il Tiro Salvezza, terminando l'effetto in caso di successo. Altrimenti, l'effetto permane 24 ore o finché la driade muore, si trova su di un piano di esistenza diverso rispetto al bersaglio, o termina l'effetto con un'azione bonus. Se il Tiro Salvezza del bersaglio riesce, il bersaglio sarà immune al Fascino Fatato della driade per le successive 24 ore.

La driade non può tenere affascinati più di un umanoide o tre bestie alla volta.

\textbf{Ecologia}\\
Ambiente: Foreste Temperate\\
Organizzazione: Solitario, coppia o boschetto (3-8)\\
Tesoro: standard (Arco Lungo Perfetto con 20 Frecce, Pugnale, altro tesoro)\\
\textbf{Descrizione}\\
Le driadi sono folletti-albero che amano i boschi appartati lontani dagli umanoidi bisognosi di legname. L'interesse principale delle driadi è la propria sopravvivenza e quella delle adorate foreste, e sono note per costringere magicamente i viaggiatori ad aiutarle in quei compiti che non possono espletare. Sono amichevoli con druidi e guardiaboschi non malvagi, dato che riconoscono la loro empatia o il loro rispetto per la natura.\\
Le driadi sono benevole guardiane degli alberi, e sebbene non siano violente di natura, possono bloccare e sventare le minacce alle loro dimore o trasformare i nemici in alleati. Alcune tengono uno o più umanoidi ammaliati nel proprio territorio per difenderlo o per sviare gli assalitori. I nemici resi inabili in genere vengono trascinati ai confini della foresta dagli alleati delle driadi e scacciati, ma quelli malvagi o ostili vengono uccisi una volta finito il combattimento.\\


\medskip\index{Mostri - Duergar}\textbf{Duergar}

\emph{Media umanoide (nano), legale malvagio}

\textbf{FORZA} +2

\textbf{DESTREZZA} +0

\textbf{COSTITUZIONE} +2

\textbf{INTELLIGENZA} +0

\textbf{SAGGEZZA} +0

\textbf{CARISMA} -1

\textbf{Iniziativa} +2 -- \textbf{Difesa} 17 (armatura di scaglie, scudo)

\textbf{Punti Ferita} 26 (4d8 + 8)

\textbf{Movimento} 8 m

\textbf{Tiri Salvezza} Tempra +4, Riflessi +0, Volontà +1

\textbf{Resistenza al Danno} veleno

\textbf{Sensi} scurovisione 36 m

\textbf{Linguaggi} Nanico, Linguaggio delle Profondità 

\textbf{Sfida} 1 (200 PE)

\emph{\textbf{Resilienza Duerga.}} Il duergar ha +1d6 ai Tiri Salvezza contro veleni, incantesimi e illusioni, oltre al resistere al restare affascinato o paralizzato.

\emph{\textbf{Sensibilità alla Luce}}. Mentre è alla luce del sole, il duergar ha -1d6 ai tiri di attacco, oltre che alle prove di Saggezza (Consapevolezza) basate sulla vista.

\textbf{Azioni}

\emph{\textbf{Ingrandire (Ricarica dopo un 1 ora).}} Per 1 minuto, il duergar aumenta magicamente di taglia, insieme a tutto ciò che sta trasportando o indossando. Mentre è ingrandito, il duergar è di taglia Grande, raddoppia i dadi di danno degli attacchi con armi basate sulla Forza (già incluso negli attacchi), e ha +1d6 alle prove di Forza e ai Tiri Salvezza di Forza. Se il duergar non ha sufficiente spazio per diventare Grande, ottiene la massima taglia concessa dallo spazio a disposizione.

\emph{\textbf{Piccone da Guerra.} Attacco con arma da mischia}: +4 a colpire, portata 1 m, un bersaglio.

\emph{Colpisce:} 6 (1d8 + 2) danni perforanti, o 11 (2d8 + 2) danni perforanti quando ingrandito.

\emph{\textbf{Giavellotto.} Attacco con arma da mischia o a Distanza}: +4 a colpire, portata 1 m o gittata 9m, un bersaglio. \emph{Colpisce:} 5 (1d6 + 2) danni perforanti o 9 (2d6 + 2) danni
perforanti quando ingrandito.

\emph{\textbf{Invisibilità (Ricarica dopo un 1 ora).}} Il duergar diventa magicamente invisibile al massimo per un'ora (come se stesse mantenendo la concentrazione per un incantesimo) o finché non attacca, lancia un incantesimo, usa Ingrandire o la sua concentrazione viene spezzata. Tutto l'equipaggiamento che il duergar indossa o trasporta diventa invisibile assieme a lui.

\textbf{Ecologia}\\
Ambiente: Qualsiasi sotterraneo\\
Organizzazione: solitario, gruppo (2-5), squadra (6-12 più 3 sergenti di 3° livello e 1 capo di 3°-8° livello), o clan (13-80 più 25\% di bambini non combattenti più 1 sergente di 3° livello ogni 5 adulti, 3-6 tenenti di 3°-6° livello, e 1-4 capitani di 9° livello)\\
Tesoro: equipaggiamento da PNG (Cotta di Maglia, Scudo Pesante di Metallo, Martello da Guerra, Balestra Leggera con 20 Quadrelli, 3d6 mo, altro tesoro)\\
\textbf{Descrizione}\\
Lontani parenti dei Nani, più cupi e deformi, i Duergar sono creature dal pessimo carattere che odiano gli intrusi nei loro reami sotterranei, ma mai più dei Nani. Vivono in comunità nelle profondità del sottosuolo. Hanno pelle grigio opaco, come fosse sporca di polvere o cenere, ma questa tonalità naturale permette di mimetizzarsi meglio nelle zone sotterranee. Sono una Razza di schiavisti, ma mentre costringono i prigionieri non Nani a lavori massacranti, uccidono senza remore i Nani catturati. In combattimento, i Duergar tirano di balestra, e poi passano al Martello da Guerra qualche round dopo. Se in inferiorità numerica, o se c'è un pericolo (e spazio) adeguato, un Duergar userà la sua capacità Ingrandire ed attacchera'.

\medskip\index{Mostri - Elementale dell'Acqua}\textbf{Elementale dell'Acqua}

\emph{Grande elementale, neutrale}

\textbf{FORZA} +4

\textbf{DESTREZZA} +2

\textbf{COSTITUZIONE} +4

\textbf{INTELLIGENZA} -3

\textbf{SAGGEZZA} +0

\textbf{CARISMA} -1

\textbf{Iniziativa} +4 -- \textbf{Difesa} 17

\textbf{Punti Ferita} 114 (12d10 + 48)

\textbf{Movimento} 9 m, nuoto 27 m

\textbf{Tiri Salvezza} Tempra +9, Riflessi +8, Volontà +2

\textbf{Resistenze al Danno} acido; da botta, perforante e tagliente di attacchi non magici

\textbf{Immunità al Danno} veleno

\textbf{Immunità alle Condizioni} afferrato, avvelenato, intralciato, paralizzato, pietrificato, privo di sensi, prono, affaticamento

\textbf{Sensi} scurovisione 18 m

\textbf{Linguaggi} Aquan

\textbf{Sfida} 5 (1.800 PE)

\emph{\textbf{Congelamento.}} Se l'elementale subisce danno da freddo, gela parzialmente; il suo movimento è ridotto di 6 metri fino al termine del suo prossimo turno.

\emph{\textbf{Forma d'Acqua.}} L'elementale può entrare nello spazio di una creatura ostile e fermarsi lì. Può muoversi attraverso uno spazio stretto fino a 3 centimetri senza doversi stringere.

\emph{\textbf{Natura Elementale.}} Un elementale non ha bisogno di aria,
cibo, bevande o sonno.

\textbf{Azioni}

\emph{\textbf{Multiattacco.}} L'elementale effettua due attacchi di
schianto.

\emph{\textbf{Schianto.} Attacco con arma da mischia}: +7 a colpire,
portata 1 m, un bersaglio.

\emph{Colpisce:} 13 (2d8 + 4) danni da botta.

\emph{\textbf{Sommergere (Ricarica 4-6).}} Ogni creatura nello spazio dell'elementale deve effettuare un Tiro Salvezza di Tempra CD 15. Se lo fallisce, il bersaglio subisce 13 (2d8 + 4) danni da botta. Se è di taglia Grande o inferiore, il bersaglio è anche afferrato (CD 14 per fuggire). Fino al termine dell'afferrare, il bersaglio è intralciato e non può respirare a meno che non sia in grado di respirare acqua. Se il Tiro Salvezza riesce, il bersaglio viene spinto fuori dallo spazio
dell'elementale.

L'elementale può afferrare una creatura Grande o fino a due Medie o più piccole alla volta. All'inizio di ciascun turno dell'elementale, ogni bersaglio afferrato subisce 13 (2d8 + 4) danni da botta. Una creatura entro 1 metro dall'elementale può trascinare fuori da esso una creatura o oggetto, impiegando un'azione per tentare di riuscire una prova di Forza CD 14.

\textbf{Ecologia}\\
Ambiente: Qualsiasi (Piano dell'Acqua)\\
Organizzazione: Solitario, coppia o gruppo (3-8)\\
Tesoro: Nessuno\\
\textbf{Descrizione}\\
Gli elementali dell'acqua sono creature pazienti e inflessibili composte di acqua vivente, dolce o salata. Preferiscono coprire d'acqua i loro avversari o trascinarveli dentro per ottenere un vantaggio.\\
Come gli altri elementali, tutti gli elementali dell'acqua hanno aspetto e forma unici. Molti sono creature dall'aspetto simile ad un'ondata con faccia vagamente umanoide e onde più piccole ai lati che fungono da braccia. Un'altra forma comune è quella di una qualche creatura acquatica, come uno squalo o un polpo, ma fatta interamente d'acqua.\\
Un elementale dell'acqua grande è alto 4,8 metri e pesa 1125kg.\\

\medskip\index{Mostri - Elementale dell'Aria}\textbf{Elementale dell'Aria}

\emph{Grande elementale, neutrale}

\textbf{FORZA} +2

\textbf{DESTREZZA} +5

\textbf{COSTITUZIONE} +2

\textbf{INTELLIGENZA} -2

\textbf{SAGGEZZA} +0

\textbf{CARISMA} -2

\textbf{Iniziativa} +5 -- \textbf{Difesa} 18

\textbf{Punti Ferita} 90 (12d10 + 24)

\textbf{Movimento} 0 m, volo 27 m (fluttua)

\textbf{Tiri Salvezza} Tempra +9, Riflessi +13, Volontà +2

\textbf{Resistenze al Danno} fulmine, tuono; da botta, perforante e tagliente di attacchi non magici

\textbf{Immunità al Danno} veleno

\textbf{Immunità alle Condizioni} afferrato, avvelenato, intralciato, paralizzato, pietrificato, privo di sensi, prono, affaticamento

\textbf{Sensi} scurovisione 18 m

\textbf{Linguaggi} Auran

\textbf{Sfida} 5 (1.800 PE)

\emph{\textbf{Forma d'Aria.}} L'elementale può entrare nello spazio di una creatura ostile e fermarsi lì. Può muoversi attraverso uno spazio stretto fino a 3 centimetri senza doversi stringere.

\emph{\textbf{Natura Elementale.}} Un elementale non ha bisogno di aria, cibo, bevande o sonno.

\textbf{Azioni}

\emph{\textbf{Multiattacco.}} L'elementale effettua due attacchi di schianto.

\emph{\textbf{Schianto.} Attacco con arma da mischia}: +8 a colpire, portata 1 m, un bersaglio.

\emph{Colpisce:} 14 (2d8 + 5) danni da botta.

\emph{\textbf{Turbine (Ricarica 4-6).}} Ogni creatura nello spazio dell'elementale deve effettuare un Tiro Salvezza di Tempra CD 13. Se lo fallisce, il bersaglio subisce 15 (3d8 + 2) danni da botta e viene scagliato a 6 metri di distanza dall'elementale in una direzione casuale e cadere prono. Se un bersaglio lanciato colpisce un oggetto, come un muro o il pavimento, subisce 3 (1d6) danni da botta per ogni 3 metri per cui è stato lanciato. Se il bersaglio viene lanciato contro un'altra creatura, quella creatura deve riuscire un Tiro Salvezza di Riflessi CD 13 o subire lo stesso danno e cadere prona.

Se il Tiro Salvezza riesce, il bersaglio subisce la metà del danno da botta e non viene scagliato via né cade prono.

\textbf{Ecologia}\\
Ambiente: Piano dell'Aria\\
Organizzazione: Solitario, coppia o gruppo (3-8)\\
Tesoro: Nessuno\\
\textbf{Descrizione}\\
Gli elementali dell'aria sono rapide creature volanti fatte d'aria. Primitivi e territoriali, non amano essere evocati o controllati dai mortali, e preferiscono trascorrere il loro tempo sul Piano dell'Aria, volando attraverso il cielo infinito.\\
Sebbene tutti gli elementali dell'aria della stessa taglia abbiano le stesse statistiche, l'aspetto esatto di ognuno varia molto da individuo a individuo: uno può apparire come un vortice animato di vento e fumo, mentre un altro come una creatura di fumo simile ad un uccello con occhi scintillanti e ali di vento.\\
Un elementale dell'aria preferisce attaccare le creature che volano, non solo per i vantaggi che ha grazie alla sua padronanza dell'aria, ma anche perché detesta toccare il terreno. Un elementale dell'aria può muoversi sott'acqua, e anche se non corre alcun rischio di annegamento, non ha gradi in Nuotare e sott'acqua perde gran parte della sua mobilità e velocità.\\
Un elementale dell'Aria Grande è alto 4,8 m e pesa 2 kg.\\


\medskip\index{Mostri - Elementale del Fuoco}\textbf{Elementale del Fuoco}

\emph{Grande elementale, neutrale}

\textbf{FORZA} +0

\textbf{DESTREZZA} +3

\textbf{COSTITUZIONE} +3

\textbf{INTELLIGENZA} -2

\textbf{SAGGEZZA} +0

\textbf{CARISMA} -2

\textbf{Iniziativa} +3 -- \textbf{Difesa} 16

\textbf{Punti Ferita} 102 (12d10 + 36)

\textbf{Movimento} 15 m

\textbf{Tiri Salvezza} Tempra +8, Riflessi +11, Volontà +4

\textbf{Resistenze al Danno} da botta, perforante e tagliente di attacchi non magici

\textbf{Immunità al Danno} fuoco, veleno

\textbf{Immunità alle Condizioni} afferrato, avvelenato, intralciato, paralizzato, pietrificato, prono, privo di sensi, affaticamento

\textbf{Sensi} scurovisione 18 m

\textbf{Linguaggi} Ignan

\textbf{Sfida} 5 (1.800 PE)

\emph{\textbf{Forma di Fuoco.}} L'elementale può spostarsi attraverso uno spazio fino a 3 centimetri di larghezza senza stringersi. Una creatura che entri a contatto o colpisca l'elementale con un attacco da mischia mentre si trova entro 1 metro da esso subisce 5 (1d10) danni da fuoco. Inoltre, l'elementale può entrare nello spazio di una creatura ostile e fermarsi lì. La prima volta che entra nello spazio di una creatura in un turno, la creatura subisce 5 (1d10) danni da fuoco e prende fuoco; finché qualcuno non impiega un'azione per spegnere le fiamme, la creatura subirà 5 (1d10) danni da fuoco all'inizio di ciascun proprio turno.

\emph{\textbf{Illuminazione.}} L'elementale emette luce intensa in un raggio di 9 metri e luce fioca per ulteriori 9 metri.

\emph{\textbf{Natura Elementale.}} Un elementale non ha bisogno di aria, cibo, bevande o sonno.

\emph{\textbf{Suscettibilità all'Acqua.}} L'elementale subisce 1 danno da freddo per ogni 1 metro che si muove in acqua o per ogni 4 litri d'acqua che gli vengono spruzzati addosso.

\textbf{Azioni}

\emph{\textbf{Multiattacco.}} L'elementale effettua due attacchi di contatto.

\emph{\textbf{Contatto.} Attacco con arma da mischia}: +6 a colpire, portata 1 m, un bersaglio.

\emph{Colpisce:} 10 (2d8 + 5) danni da fuoco. Se il bersaglio è una creatura o un oggetto infiammabile, prende fuoco. Finché una creatura non impiega un'azione per spegnere le fiamme, la creatura subirà 5 (1d10) danni da fuoco all'inizio di ciascun proprio turno.

\textbf{Ecologia}
Ambiente: Qualsiasi (Piano del Fuoco)\\
Organizzazione: Solitario, coppia o gruppo (3-8)\\
Tesoro: Nessuno\\
\textbf{Descrizione}\\
Gli elementali del fuoco sono creature veloci e crudeli fatte di fiamme viventi. Si divertono a spaventare quelli più deboli di loro, e terrorizzano qualsiasi creatura che possano incendiare. Un elementale del fuoco non può entrare nel'acqua o in qualsiasi liquido ininfiammabile. Una massa d'acqua è una barriera impenetrabile a meno che l'elementale possa scavalcarla o saltarla, oppure venga coperta con materiale infiammabile (come uno strato d'olio).\\
Gli elementali del fuoco hanno un aspetto variabile; in genere si manifestano in forma di spire serpentine fatte di fumo e fiamme, ma alcuni elementali del fuoco prendono sembianze più simili a quelle di umani, demoni o altri mostri per aumentare il terrore quando compaiono improvvisamente. Il corpo di un elementale del fuoco sembra fatto di fiamme o sbuffi di scintille, fumo o cenere semistabili.\\

Un elementale del fuoco grande è alto 4,8 metri\\

\medskip\index{Mostri - Elementale della Terra}\textbf{Elementale della Terra}

\emph{Grande elementale, neutrale}

\textbf{FORZA} +5

\textbf{DESTREZZA} -1

\textbf{COSTITUZIONE} +5

\textbf{INTELLIGENZA} -3

\textbf{SAGGEZZA} +0

\textbf{CARISMA} -3

\textbf{Iniziativa} -1 -- \textbf{Difesa} 20

\textbf{Punti Ferita} 126 (12d10 + 60)

\textbf{Movimento} 9 m, scavo 9 m

\textbf{Tiri Salvezza} Tempra +9, Riflessi +1, Volontà +6

\textbf{Vulnerabilità al Danno} tuono

\textbf{Resistenze al Danno} da botta, perforante e tagliente di attacchi non magici

\textbf{Immunità al Danno} veleno

\textbf{Immunità alle Condizioni} avvelenato, paralizzato, pietrificato, prono, privo di sensi, affaticamento,

\textbf{Sensi} percezione tellurica 18 m, scurovisione 18 m

\textbf{Linguaggi} Terran

\textbf{Sfida} 5 (1.800 PE)

\emph{\textbf{Mostro d'Assedio.}} L'elementale infligge danni doppi agli oggetti e le strutture.

\emph{\textbf{Natura Elementale.}} Un elementale non ha bisogno di aria, cibo, bevande o sonno.

\emph{\textbf{Planata Terrestre.}} L'elementale può scavare attraversa la terra e la pietra non magiche e non lavorate. Quando lo fa, l'elementale non disturba il materiale che sposta. 
\textbf{Azioni}

\emph{\textbf{Multiattacco.}} L'elementale effettua due attacchi di schianto.

\emph{\textbf{Schianto.} Attacco con arma da mischia}: +8 a colpire, portata 3 m, un bersaglio.

\emph{Colpisce:} 14 (2d8 + 5) danni da botta.

\textbf{Ecologia}
Ambiente: Qualsiasi (Piano della Terra)\\
Organizzazione: Solitario, coppia o gruppo (3-8)\\
Tesoro: Nessuno\\
\textbf{Descrizione}\\
Gli elementali della terra sono creature lente ed ostinate fatte di pietra o terra. Quando stanno completamente fermi sono indistinguibili da un mucchio di pietre o una piccola collina.\\

Quando un elementale della terra si mette pesantemente in movimento, il suo aspetto esteriore può variare, anche se le sue statistiche restano identiche a quelle dei suoi simili della stessa taglia. Gli elementali della terra sono fatti per lo più di roccia, terra o cristallo, con gemme scintillanti come occhi. Quelli più grandi hanno l'aspetto di umanoidi di pietra. Ciuffi di vegetazione spesso crescono sul suolo che costituisce parte del corpo di un elementale della terra.\\

Un elementale della terra grande è alto 4,8 metri e pesa 3000 kg.\\


\medskip\index{Mostri - Ettercap}\textbf{Ettercap}

\emph{Media mostruosità, neutrale malvagio}

\textbf{FORZA} +2

\textbf{DESTREZZA} +2

\textbf{COSTITUZIONE} +1

\textbf{INTELLIGENZA} -2

\textbf{SAGGEZZA} +1

\textbf{CARISMA} 8 (-2)

\textbf{Iniziativa} +2 -- \textbf{Difesa} 14

\textbf{Punti Ferita} 44 (8d8 + 8)

\textbf{Movimento} 9 m, scalata 9 m

\textbf{Tiri Salvezza} Tempra +6, Riflessi +4, Volontà +6

\textbf{Competenze} Muoversi Silenziosamente / Nascondersi +4, Consapevolezza +3, Sopravvivenza +3

\textbf{Sensi} scurovisione 18 m

\textbf{Linguaggi} -

\textbf{Sfida} 2 (450 PE)

\emph{\textbf{Camminare sulla Tela.}} L'ettercap ignora le restrizioni al movimento provocate dalle ragnatele.

\emph{\textbf{Scalare come Ragno.}} L'ettercap può scalare superfici difficili, compreso lo stare a testa in giù sul soffitto, senza bisogno di effettuare una prova di caratteristica.

\emph{\textbf{Senso della Tela.}} Mentre è in contatto con una ragnatela, l'ettercap sa l'esatta posizione di qualsiasi altra creatura in contatto con la stessa ragnatela.

\textbf{Azioni}

\emph{\textbf{Multiattacco.}} L'ettercap effettua due attacchi: uno con il morso e uno con gli artigli

\emph{\textbf{Artigli.} Attacco con arma da mischia}: +4 a colpire, portata 1 m, un bersaglio.

\emph{Colpisce:} 7 (2d4 + 2) danni taglienti.

\emph{\textbf{Morso.} Attacco con arma da mischia}: +4 a colpire, portata 1 m, un bersaglio.

\emph{Colpisce:} 6 (1d8 + 2) danni perforanti più 4 (1d8) danni da veleno. Il bersaglio deve riuscire un Tiro Salvezza di Tempra CD 11 o restare avvelenato per 1 minuto. La creatura può ripetere il Tiro Salvezza al termine di ciascun suo turno, terminando l'effetto se riesce il Tiro Salvezza.

\emph{\textbf{Ragnatela (Ricarica 5-6).} Attacco con arma a Distanza}: +4 a colpire, gittata 9m, una creatura di taglia Grande o minore. \emph{Colpisce:} La creatura è intralciata dalla ragnatela. Con un'azione, la creatura intralciata può effettuare una prova di Forza CD 11, liberandosi dalla tela se la riesce. L'effetto termina se la tela è distrutta. La tela ha Difesa 10, 5 punti ferita, vulnerabilità ai danni da fuoco, e immunità ai danni da botta, da veleno e psichici.

\textbf{Ecologia}\\
Ambiente: Foreste Temperate\\
Organizzazione: solitario, coppia o nido (3-6 più 2-8 ragni giganti)\\
Tesoro: Standard\\
\textbf{Descrizione}\\
Gli ettercap sono alti di solito 1,8 metri e pesano circa 100 kg. Sono solitari e raramente si uniscono ad altri della loro razza, tranne per l'accoppiamento. Quando fanno gruppo, tendono ad attrarre varie specie di ragni, formando uno strano connubio di ettercap e aracnidi.\\
Gli ettercap sono noti per la costruzione di astute trappole fatte di ragnatele e altri materiali naturali, che usano per catturare prede. Costruiscono rifugi di ragnatela, tra i rami più alti gli alberi lontano dagli altri predatori terrestri, e usano ragni mostruosi come vedette e guardiani.\\
Gli ettercap non sono coraggiosi, ma le loro trappole spesso impediscono al nemico di estrarre le armi. Un ettercap attacca con artigli e morsi velenosi. In genere evita la mischia con gli avversari che possono ancora muoversi e fugge se si liberano.\\


\medskip\index{Mostri - Ettin}\textbf{Ettin}

\emph{Grande gigante, caotico malvagio}

\textbf{FORZA} +5

\textbf{DESTREZZA} -1

\textbf{COSTITUZIONE} +3

\textbf{INTELLIGENZA} -2

\textbf{SAGGEZZA} +0

\textbf{CARISMA} -1

\textbf{Iniziativa} -1 -- \textbf{Difesa} 14

\textbf{Punti Ferita} 85 (10d10 + 30)

\textbf{Movimento} 12 m

\textbf{Tiri Salvezza} Tempra +9, Riflessi +2, Volontà +5

\textbf{Competenze} Consapevolezza +4

\textbf{Linguaggi} Gigante, Goblinoide

\textbf{Sfida} 4 (1.100 PE)

\emph{\textbf{Due Teste.}} L'ettin ha +1d6 alle prove di Saggezza (Consapevolezza) e sui Tiri Salvezza contro le condizioni accecato, affascinato, assordato, privo di sensi, spaventato e stordito.

\emph{\textbf{Veglia.}} Quando una delle due teste dell'ettin è addormentata, l'altra è sveglia.

\textbf{Azioni}

\emph{\textbf{Multiattacco.}} L'ettin effettua due attacchi: uno con l'ascia da battaglia e uno con la mazza chiodata.

\emph{\textbf{Ascia da Battaglia.} Attacco con arma da mischia}: +7 a colpire, portata 1 m, un bersaglio.

\emph{Colpisce:} 14 (2d8 + 5) danni taglienti.

\emph{\textbf{Mazza Chiodata.} Attacco con arma da mischia}: +7 a colpire, portata 1 m, un bersaglio.

\emph{Colpisce:} 14 (2d8 + 5) danni perforanti.

\textbf{Ecologia}\\
Ambiente: Colline fredde\\
Organizzazione: Solitario, coppia, gruppo (3-6), truppa (1-2 più 1-2 Orsi Bruni, banda (3-6 più 1-2 Orsi Bruni) o colonia (3-6 più 1-2 Orsi Bruni e 7-12 Orchi, o 9-16 Goblin)\\
Tesoro: Standard (Armatura di Cuoio, 2 Mazzafrusti Leggeri, 4 Giavellotti, altro tesoro)\\
\textbf{Descrizione}\\
Gli ettin, o giganti a due teste, sono cacciatori notturni malevoli e imprevedibili. Le due teste gli concedono impareggiabili poteri di percezione, facendone dei guardiani eccellenti.\\
Gli ettin sembrano Giganti delle Colline o Giganti delle Rocce, ma il volto zannuto tradisce una discendenza orchesca. Hanno pelle marrone rosata e non si lavano mai se non vi sono costretti, cosa che li rende così sporchi e sudici che la loro pelle sembra spessa e grigia.\\
Gli adulti sono alti 3,9 metri e pesano 2.600 kg. Gli ettin vivono circa 75 anni.\\
Gli ettin non hanno un loro linguaggio ma parlano un gergo misto di Gigante, Goblin e Orchesco. Le creature che parlano uno qualsiasi di questi linguaggi possono comunicare con un ettin effettuando una prova di Intelligenza con DC 15. La prova si effettua una volta per ogni frammento di informazione; se l'altra creatura parla due di questi linguaggi la DC è 10, mentre per qualcuno che li parla tutti e tre è 5.\\
Sebbene gli ettin non siano molto intelligenti, sono guerrieri astuti. Preferiscono tendere imboscate alle loro vittime anziché ingaggiarle in combattimento, ma una volta che la battaglia è cominciata, un ettin combatte furiosamente fino alla morte del nemico.\\
Gli ettin sono creature solitarie, si stabiliscono nella sicurezza di cave rocciose e cavità, spesso circondate da buche e fossi, e tengono a volte degli orsi delle caverne come animali da compagnia o guardiani.\\
Un ettin particolarmente potente può attrarre un gruppo di seguaci, specie Goblin o Orchi. Comunque, questi assembramenti sono più che altro delle eccezioni, e raramente durano a lungo, con gli individualisti ettin che vanno per la loro strada appena le opportunità di saccheggio e rapina diminuiscono o se il capo viene ucciso.\\
In genere formano delle coppie riproduttive per allevare la prole solo per brevi periodi prima di riprendere ognuno la propria strada. I giovani ettin maturano rapidamente, raggiungendo la taglia adulta in un anno, potendo così provvedere a se stessi.\\

\textbf{Ecologia}
Ambiente: qualsiasi\\
Organizzazione: solitario\\
Tesoro: equipaggiamento da PNG\\
\textbf{Descrizione}\\
Quando ad un'anima non è concesso il riposo a causa di qualche grave ingiustizia, vera o presunta, a volte essa torna come fantasma. Questi esseri sono eternamente angosciati, privi di sostanza e incapaci di rimettere le cose a posto. Sebbene i fantasmi possano avere qualsiasi Tratto, molti si aggrappano al mondo dei viventi con un forte senso di odio e rabbia, e come risultato diventano  malvagi; anche una creatura buona dopo morta può diventare un fantasma odioso e crudele.\\

Più di altri mostri, il fantasma deve avere un background ben delineato. Perché questo personaggio è diventato un fantasma? Quali leggende lo circondano? Un incontro con un fantasma non dovrebbe mai avvenire in modo accidentale: ci sono molti altri non morti incorporei, come Wraith e Spettri, per questo. Un incontro adeguato con un fantasma dovrebbe avvenire in una scena al culmine di un lungo periodo di tensione costruito con servitori minori o manifestazioni di spiriti non morti. L'esempio di fantasma sopra rappresenta una principessa umana assassinata da un amante infedele; dopo un confronto, lui la legò con delle catene e la gettò nel pozzo del castello, dove morì annegata. Le capacità del fantasma sono state selezionate in base al background, mostrando come si possa creare un potente antagonista. Applicando l'archetipo a creature con livelli e quindi Abilità proprie o con capacità razziali significative si possono creare fantasmi molto più potenti.\\

Quando viene creato un fantasma, questi ottiene le "copie" degli oggetti a cui in vita dava particolare valore (a condizione che gli originali non siano in possesso di altre creature). L'equipaggiamento funziona normalmente per il fantasma ma passa attraverso gli oggetti o le creature materiali. Un'arma +1 o con un potenziamento superiore, tuttavia, può danneggiare le creature materiali, ma tali attacchi infliggono la metà dei danni (50\%) a meno che non sia un'arma del tocco fantasma. Un fantasma può usare scudi e armature solo se hanno la capacità Tocco Fantasma.\\

Gli oggetti originali vengono lasciati indietro, proprio come le spoglie fisiche del fantasma. Se un'altra creatura impugna l'originale, la copia incorporea svanisce. Questa perdita fa inevitabilmente infuriare il fantasma, che non si ferma davanti a nulla per riportare l'oggetto nel posto in cui giaceva originariamente (e riguadagnarne l'utilizzo).\\

\medskip\index{Mostri - Fantasma}\textbf{Fantasma}

\emph{Media non morto, qualsiasi allineamento}

\textbf{FORZA} -2

\textbf{DESTREZZA} +1

\textbf{COSTITUZIONE} +0

\textbf{INTELLIGENZA} +0

\textbf{SAGGEZZA} +1

\textbf{CARISMA} +3

\textbf{Iniziativa} +1 -- \textbf{Difesa} 13

\textbf{Punti Ferita} 45 (10d8)

\textbf{Movimento} 0 m, volo 12 m (fluttua)

\textbf{Tiri Salvezza} Tempra +7, Riflessi +6, Volontà +7

\textbf{Resistenze al Danno} acido, fulmine, fuoco, tuono; da botta, perforante, tagliente di attacchi non magici

\textbf{Immunità ai Danni} freddo, da Vuoto, veleno

\textbf{Immunità alle Condizioni} affascinato, afferrato, avvelenato, intralciato, paralizzato, pietrificato, prono, affaticamento, spaventato

\textbf{Sensi} scurovisione 18 m

\textbf{Linguaggi} qualsiasi lingua conosciuta in vita

\textbf{Sfida} 4 (1.100 PE)

\emph{\textbf{Movimento Incorporeo.}} Il fantasma può attraversare altre creature e oggetti come se fossero terreno difficile. Subisce 5 (1d10) danni da forza se termina il suo turno all'interno di un oggetto. 

\emph{\textbf{Natura Non Morta.}} Il fantasma non ha bisogno di aria, cibo, bevande o di dormire.

\emph{\textbf{Vista Eterea.}} Il fantasma può vedere 18 metri nel Piano Etereo quando si trova sul Piano Materiale, e vice versa.

\textbf{Azioni}

\emph{\textbf{Tocco Avvizzente.} Attacco con arma da mischia}: +5 a colpire, portata 1 m, un bersaglio.

\emph{Colpisce:} 17 (4d6 + 3) danni da Vuoto.

\emph{\textbf{Eterealità.}} Il fantasma entra nel Piano Etereo dal Piano Materiale, o vice versa. È visibile sul Piano Materiale mentre è nel Margine Etereo, e vice versa, ma non può interagire con nulla che si trovi sull'altro piano.

\emph{\textbf{Possessione (Ricarica 6).}} Un umanoide, entro 1 metro e visibile al fantasma, deve riuscire un Tiro Salvezza di Volontà CD 13 o venire posseduto dal fantasma; il fantasma poi scompare, e il bersaglio è inabile e perde il controllo del suo corpo. Il fantasma ora controlla il corpo ma non priva il bersaglio della sua consapevolezza. Il fantasma non può essere bersaglio di attacchi, incantesimi, o altri effetti, eccetto quelli che scacciano i non morti, e mantiene il suo allineamento, Intelligenza, Saggezza, Carisma e immunità all'essere affascinato e spaventato. Per il resto usa altrimenti le statistiche del bersaglio posseduto, ma non accede al sapere e competenze del bersaglio.

La possessione dura finché il corpo scende a 0 punti ferita, il fantasma la termina con un'azione bonus, o il fantasma viene scacciato o espulso da un effetto come l'incantesimo \emph{dissolvi il bene e il male}. Quando la possessione termina, il fantasma riappare in uno spazio non occupato entro 1 metro dal corpo. Il bersaglio è immune alla Possessione di questo fantasma per 24 ore dopo aver riuscito il Tiro Salvezza o al termine della possessione.

\emph{\textbf{Viso Orripilante.}} Ogni creatura che non sia non morta, entro 18 metri dal fantasma e che lo possa vedere, deve riuscire un Tiro Salvezza di Volontà CD 13 o essere spaventata per 1 minuto. Se il Tiro Salvezza fallisce di 5 o più, il bersaglio invecchia anche di 1d4 x 10 anni. Un  bersaglio spaventato può ripetere il Tiro Salvezza al termine di ciascun proprio turno, terminando l'effetto per sé, qualora riuscisse il tiro  salvezza. Se il Tiro Salvezza del bersaglio riesce e per lui l'effetto ha fine, il bersaglio è immune al Viso Orripilante del fantasma per le successive 24  ore. L'effetto di invecchiamento può essere invertito con l'incantesimo \emph{ristorare superiore}, ma solo se eseguito entro 24 dall'effetto di  invecchiamento.

\textbf{Ecologia}
Ambiente: qualsiasi\\
Organizzazione: solitario\\
Tesoro: equipaggiamento da PNG\\
\textbf{Descrizione}\\
Quando ad un'anima non è concesso il riposo a causa di qualche grave ingiustizia, vera o presunta, a volte essa torna come fantasma. Questi esseri sono eternamente angosciati, privi di sostanza e incapaci di rimettere le cose a posto. Sebbene i fantasmi possano avere qualsiasi Tratto, molti si aggrappano al mondo dei viventi con un forte senso di odio e rabbia, e come risultato diventano  malvagi; anche una creatura buona dopo morta può diventare un fantasma odioso e crudele.\\

Più di altri mostri, il fantasma deve avere un background ben delineato. Perché questo personaggio è diventato un fantasma? Quali leggende lo circondano? Un incontro con un fantasma non dovrebbe mai avvenire in modo accidentale: ci sono molti altri non morti incorporei, come Wraith e Spettri, per questo. Un incontro adeguato con un fantasma dovrebbe avvenire in una scena al culmine di un lungo periodo di tensione costruito con servitori minori o manifestazioni di spiriti non morti. L'esempio di fantasma sopra rappresenta una principessa umana assassinata da un amante infedele; dopo un confronto, lui la legò con delle catene e la gettò nel pozzo del castello, dove morì annegata. Le capacità del fantasma sono state selezionate in base al background, mostrando come si possa creare un potente antagonista. Applicando l'archetipo a creature con livelli e quindi Abilità proprie o con capacità razziali significative si possono creare fantasmi molto più potenti.\\

Quando viene creato un fantasma, questi ottiene le "copie" degli oggetti a cui in vita dava particolare valore (a condizione che gli originali non siano in possesso di altre creature). L'equipaggiamento funziona normalmente per il fantasma ma passa attraverso gli oggetti o le creature materiali. Un'arma +1 o con un potenziamento superiore, tuttavia, può danneggiare le creature materiali, ma tali attacchi infliggono la metà dei danni (50\%) a meno che non sia un'arma del tocco fantasma. Un fantasma può usare scudi e armature solo se hanno la capacità Tocco Fantasma.\\

Gli oggetti originali vengono lasciati indietro, proprio come le spoglie fisiche del fantasma. Se un'altra creatura impugna l'originale, la copia incorporea svanisce. Questa perdita fa inevitabilmente infuriare il fantasma, che non si ferma davanti a nulla per riportare l'oggetto nel posto in cui giaceva originariamente (e riguadagnarne l'utilizzo).\\


\medskip\index{Mostri - Fauci Gorgoglianti}\textbf{Fauci Gorgoglianti}

\emph{Media aberrazione, neutrale}

\textbf{FORZA} +0

\textbf{DESTREZZA} -1

\textbf{COSTITUZIONE} +3

\textbf{INTELLIGENZA} -4

\textbf{SAGGEZZA} +0

\textbf{CARISMA} -2

\textbf{Iniziativa} -1 -- \textbf{Difesa} 10

\textbf{Punti Ferita} 67 (9d8 + 27)

\textbf{Movimento} 3 m, nuoto 3 m

\textbf{Tiri Salvezza} Tempra +8, Riflessi +4, Volontà +5

\textbf{Immunità alle Condizioni} prono

\textbf{Sensi} scurovisione 18 m

\textbf{Linguaggi} -

\textbf{Sfida} 2 (450 PE)

\emph{\textbf{Gorgoglio.}} Finché la fauce è in grado di vedere una creatura e non è inabile, pronuncia frasi incoerenti. Ogni creatura che inizi il suo turno entro 6 metri dalla fauce e può udire il suo gorgoglio deve effettuare un Tiro Salvezza di Volontà CD 10. Se lo fallisce, la creatura non può effettuare reazioni fino all'inizio del suo prossimo turno e tira un d8 per determinare cosa farà durante il proprio turno. Da 1 a 4, la creatura non fa nulla. Con 5 o 6, la creatura non svolge nessun'azione o azione bonus e usa tutto il suo movimento per muoversi in una direzione determinata casualmente. Con 7 o 8, la creatura effettua un attacco da mischia contro una creatura determinata a caso entro la sua portata o non fa nulla se non è in grado di effettuare un simile attacco.

\emph{\textbf{Terreno Aberrante.}} Il terreno in un raggio di 3 metri intorno alla fauce è considerato terreno difficile. Ogni creatura che inizi il suo turno in quell'area deve riuscire un Tiro Salvezza di Tempra CD 10 o vedere il suo movimento ridotto a 0 fino all'inizio del suo turno successivo.

\textbf{Azioni}

\emph{\textbf{Multiattacco.}} La fauce gorgogliante effettua un attacco di morso e, se può, uno Sputo Accecante.

\emph{\textbf{Morso.} Attacco con arma da mischia}: +2 a colpire, portata 1 m, una creatura.

\emph{Colpisce:} 17 (5d6) danni perforanti. Se il bersaglio è di taglia Media o inferiore, deve riuscire un Tiro Salvezza di Tempra CD 10 o venir gettato prono. Se il bersaglio viene ucciso da questo danno, viene assorbito dalla fauce.

\emph{\textbf{Sputo Accecante (Ricarica 5-6).}} La fauce sputa un globo chimico ad un punto visibile entro 5 metri da essa. Il globo esplode all'impatto in un lampo accecante di luce. Ogni creatura entro 1 metro dal lampo deve riuscire un Tiro Salvezza di Riflessi CD 13 o restare accecata fino al termine del prossimo turno della fauce.

\textbf{Ecologia}\\
Ambiente: Qualsiasi Sotterraneo\\
Organizzazione: Solitario\\
Tesoro: Standard\\
\textbf{Descrizione}\\
Disgustosa, nauseante e affamata: queste sono le uniche parole che descrivono in modo appropriato la fauce gorgogliante. Bestie ripugnanti che si nascondono nelle grotte, nelle fogne e negli incubi, le fauci non hanno altro senso sociale, ecologico o religioso diverso dalla loro capacità di far impazzire coloro che le ascoltano. Alcuni studiosi credono che le fauci gorgoglianti siano una variante più piccola del molto più pericoloso shoggoth, mentre altri teorizzano che sia una punizione di qualche potente entità o divinità inflitta a coloro che l'hanno offesa.\\




\subsection{Funghi}

\medskip\index{Mostri - Fungo Stridente}\textbf{Fungo Stridente}

\emph{Media pianta, disallineato}

\textbf{FORZA} -5

\textbf{DESTREZZA} -5

\textbf{COSTITUZIONE} +0

\textbf{INTELLIGENZA} -5

\textbf{SAGGEZZA} -4

\textbf{CARISMA} -5

\textbf{Iniziativa} -5 -- \textbf{Difesa} 6

\textbf{Punti Ferita} 13 (3d8)

\textbf{Movimento} 0 m

\textbf{Tiri Salvezza}: Fort -3, Riflessi +3, Will -4

\textbf{Immunità alle Condizioni} accecato, assordato, spaventato

\textbf{Sensi} vista cieca 9 m (cieco oltre questo raggio)

\textbf{Linguaggi} -

\textbf{Sfida} 0 (10 PE)

\emph{\textbf{Falso Aspetto.}} Mentre il fungo stridente rimane immobile, è indistinguibile da un normale fungo.

\textbf{Azioni}

\emph{\textbf{Strillo.}} Quando una luce intensa o una creatura si trova entro 9 metri dal fungo stridente, esso emette un strillo udibile fino a 90 metri di distanza. Il fungo stridente continua a strillare finché la fonte del disturbo non si è portata fuori gittata e per altri 1d4 turni successivi.

\textbf{Ecologia}\\
Ambiente: Qualsiasi sotterraneo\\
Organizzazione: Solitario, coppia o macchia (3-12)\\
Tesoro: Accidentale\\
\textbf{Descrizione}\\
Un fungo stridente e' alto circa 50 cm, dall'ampio cappello marrone. Una volta emesso l'urlo il cappello si sgonfia. \\
Si racconta di Cuochi Duergar specializzati nel cuocere questi funghi in pietanze sopraffine.
I piu' bravi riescono anche a non fare sgonfiare il cappello.\\


\medskip\index{Mostri - Fungo Violetto}\textbf{Fungo Violetto}

\emph{Media pianta, disallineato}

\textbf{FORZA} -4

\textbf{DESTREZZA} -5

\textbf{COSTITUZIONE} +0

\textbf{INTELLIGENZA} -5

\textbf{SAGGEZZA} -4

\textbf{CARISMA} -5

\textbf{Iniziativa} -5 -- \textbf{Difesa} 6

\textbf{Punti Ferita} 18 (4d8)

\textbf{Movimento} 2 m

\textbf{Tiri Salvezza}: Fort -3, Ref -3, Will -3

\textbf{Immunità alle Condizioni} accecato, assordato, spaventato 

\textbf{Sensi} vista cieca 9 m (cieco oltre questo raggio)

\textbf{Linguaggi} -

\textbf{Sfida} 1/4 (50 PE)

\emph{\textbf{Falso Aspetto.}} Mentre il fungo violetto rimane immobile, è indistinguibile da un normale fungo.

\textbf{Azioni}

\emph{\textbf{Multiattacco.}} Il fungo effettua 1d4 attacchi con Contatto Putrido.

\emph{\textbf{Contatto Putrido.} Attacco con arma da mischia}: +2 a colpire,  portata 3 m, un bersaglio.

\emph{Colpisce:} 4 (1d8) danni da Vuoto.

\textbf{Ecologia}\\
Ambiente: Qualsiasi sotterraneo\\
Organizzazione: Solitario, coppia o macchia (3-12)\\
Tesoro: Accidentale\\
\textbf{Descrizione}\\
I funghi viola sono uno dei più noti e temuti pericoli delle caverne. Un viaggiatore può spesso notare i segni lasciati dal fungo viola su coloro che vivono o cacciano nei luoghi in cui questi funghi carnivori si appostano. Queste profonde e orribili cicatrici sembrano solchi scavati nella carne: i segni di un incontro ravvicinato con un fungo viola.\\
Un fungo viola si nutre della materia organica putrefatta, ma a differenza della maggioranza dei funghi non è un consumatore passivo. I viticci di un fungo viola possono colpire con inaspettata rapidità e sono ricoperti di un veleno virulento che causa la putrefazione delle carni con nauseante velocità. Questo potente veleno, se trascurato, può far marcire rapidamente un intero braccio o una gamba, lasciandosi dietro solo ossa che presto si corroderanno anch'esse.\\
Sebbene i funghi viola possano muoversi, lo fanno solo per attaccare o cacciare la preda. Un fungo viola con un flusso regolare di putredine di cui nutrirsi si accontenta di restare in un posto. Molti abitanti del sottosuolo, in particolare Trogloditi e Vegepigmei, sfruttano questo comportamento a loro vantaggio e posizionano molteplici funghi viola in giunzioni ed entrate chiave delle loro caverne come guardiani, assicurandosi di fornire loro cadaveri a sufficienza per evitare che si addentrino nel rifugio in cerca di cibo.\\
Alcune specie di Boleto Stridente hanno un aspetto piuttosto simile a quello dei funghi viola, sebbene manchino di ramificazioni tentacolari. Non è strano trovare boleti stridenti e funghi viola nello stesso groviglio, specialmente nelle aree dove altre creature coltivano questi funghi come guardiani.\\

Un fungo viola è alto 1,2 metri e pesa 25 kg.\\


\medskip\index{Mostri - Fuoco Fatuo}\textbf{Fuoco Fatuo}

\emph{Minuscola non morto, caotico malvagio}

\textbf{FORZA} -5

\textbf{DESTREZZA} +9

\textbf{COSTITUZIONE} +0

\textbf{INTELLIGENZA} +1

\textbf{SAGGEZZA} +2

\textbf{CARISMA} +0

\textbf{Iniziativa} +9 -- \textbf{Difesa} 20

\textbf{Punti Ferita} 22 (9d4)

\textbf{Movimento} 0 m, volo 15 m (fluttua)

\textbf{Tiri Salvezza}: Tempra +3, Riflessi +12, Volontà +9

\textbf{Immunità ai Danni} fulmine, veleno

\textbf{Resistenze al Danno} acido, freddo, fuoco, da Vuoto, tuono; contendente, perforante e tagliente di attacchi non magici

\textbf{Immunità alle Condizioni} afferrato, avvelenato, intralciato, paralizzato, privo di sensi, prono, affaticamento

\textbf{Sensi} scurovisione 36 m

\textbf{Linguaggi} le lingue che conosceva in vita

\textbf{Sfida} 2 (450 PE)

\emph{\textbf{Consumare Vita.}} Con un'azione bonus, il fuoco fatuo può prendere a bersaglio una creatura che può vedere entro 1 metro da esso e che abbia 0 punti ferita e sia ancora in vita. Il bersaglio deve riuscire un Tiro Salvezza di Tempra CD 10 contro questa magia o morire. Se il bersaglio muore, il fuoco fatuo recupera 10 (3d6) punti ferita.

\emph{\textbf{Effimero.}} Il fuoco fatuo non può indossare né trasportare nulla.

\emph{\textbf{Illuminazione Variabile.}} Il fuoco fatuo promana luce intensa in un raggio da 1 a 6 metri e luce fioca per un numero di metri aggiuntivi pari al raggio scelto. Il fuoco fatuo può modificare questo raggio con un'azione bonus.

\emph{\textbf{Movimento Incorporeo.}} Il fuoco fatuo può muoversi attraverso altre creature e oggetti come se fossero terreno difficile. Subisce 5 (1d10) danni da forza se termina il suo turno all'interno di un oggetto.

\emph{\textbf{Natura Non Morta.}} Il fuoco fatuo non ha bisogno di aria, cibo o bevande.

\textbf{Azioni}

\emph{\textbf{Scossa.} Attacco con incantesimo in mischia}: +4 a colpire, portata 1 m, una creatura.

\emph{Colpisce:} 9 (2d8) danni da fulmine.

\emph{\textbf{Invisibilità.}} Il fuoco fatuo e la sua luce diventano magicamente invisibili finché non attacca o usa Consumare Vita, o finché la sua concentrazione non termina (come se si stesse concentrando su di un incantesimo).

\textbf{Ecologia}
Ambiente: Qualsiasi Palude\\
Organizzazione: Solitario, coppia o sequenza (3-4)\\
Tesoro: Accidentale\\
\textbf{Descrizione}\\
Ogni cacciatore e agricoltore che viva vicino ad un acquitrino o a una palude ha dato un nome a queste sfere di luce fioca: jack lanterna, candele dei defunti, fuochi che camminano, luci dei pini, luci fantasma, luci di giunco; ma tutti sanno che si tratta di pericolosi predatori e false guide nell'oscurità.\\

Malvagie creature che si nutrono delle forti emanazioni psichiche delle creature terrorizzate, i fuochi fatui traggono piacere nel mettere i viaggiatori creduloni in situazioni pericolose. Nelle terre selvagge, dove sono molto comuni, i fuochi fatui preferiscono tattiche semplici come posizionarsi su scogli o sabbie mobili dove possono essere scambiati facilmente per lanterne (specialmente se possono predisporre la trappola nei pressi di vere lanterne di segnalazione), così da attirare i viaggiatori verso il pericolo. In rare occasioni, i fuochi fatui in cerca di vita facile si spostano in una città e si stabiliscono vicino ai patiboli o seguono, invisibili, un'armata, così da nutrirsi della paura degli uomini morenti; perché la stragrande maggioranza scelga di rimanere nelle paludi, dove le vittime scarseggiano, rimane un mistero.\\

I fuochi fatui possono contare solo sulla loro scossa elettrica in situazioni pericolose, quindi preferiscono lasciare che altre creature o pericoli si occupino delle loro vittime mentre loro fluttuano nelle vicinanze e banchettano.\\

I fuochi fatui possono brillare di qualunque colore desiderino, ma sono più spesso gialli, bianchi, verdi o blu. Possono anche variare la loro luminosità per creare un disegno: molti fuochi fatui amano creare forme che somigliano vagamente a teschi nella loro luminescenza per aumentare il terrore nelle loro vittime. I loro veri corpi sono globi di materiale spugnoso traslucido appena visibili di circa 30 centimetri che pesano 1,5 kg e possono emettere luce su tutta la loro superficie. La luce dei fuochi fatui brilla approssimativamente come una torcia, e sebbene non sembrino utilizzare suoni per comunicare, sentono perfettamente e possono far vibrare i loro corpi così rapidamente da imitare il linguaggio.\\

Nonostante siano denigrati dalla maggioranza delle creature senzienti, i fuochi fatui sono in realtà alquanto intelligenti, sebbene ragionino in modo completamente alieno. A volte si organizzano in gruppi chiamati "sequenze"; la loro società e i loro scopi rimangono completamente sconosciuti, così come le loro origini, sebbene talvolta siano noti per stringere patti con chi offre loro una grande quantità di vittime adeguatamente terrorizzate.\\

I fuochi fatui non hanno età e sono di fatto immortali, a meno che non muoiano di morte violenta; i fuochi fatui più antichi possono essere ottimi depositari di conoscenze del passato, sebbene convincere una di queste crudeli creature a cooperare possa essere piuttosto complicato.\\


\medskip\index{Mostri - Fustigatore}\textbf{Fustigatore}

\emph{Grande mostruosità, neutrale malvagio}

\textbf{FORZA} +4

\textbf{DESTREZZA} -1

\textbf{COSTITUZIONE} +3

\textbf{INTELLIGENZA} -2

\textbf{SAGGEZZA} +3

\textbf{CARISMA} -2

\textbf{Iniziativa} -1 -- \textbf{Difesa} 23

\textbf{Punti Ferita} 93 (11d10 + 33)

\textbf{Movimento} 3 m, scalata 3 m

\textbf{Tiri Salvezza}: Tempra +13, Riflessi +5, Volontà +13

\textbf{Competenze} Muoversi Silenziosamente / Nascondersi +5, Consapevolezza +6

\textbf{Sensi} scurovisione 18 m

\textbf{Linguaggi} -

\textbf{Sfida} 5 (1.800 PE)

\emph{\textbf{Falso Aspetto.}} Quando il fustigatore rimane immobile, è indistinguibile da una normale formazione rocciosa, come una stalagmite.

\emph{\textbf{Scalare come Ragno.}} Il fustigatore può scalare superfici difficili, compreso lo stare a testa in giù sul soffitto, senza bisogno di effettuare una prova di abilità.

\emph{\textbf{Viticci Afferranti.}} Il fustigatore può avere fino a sei viticci alla volta. Ogni viticcio può essere attaccato (CA 20; 10 punti ferita; immunità ai danni psichici e da veleno). Distruggere un viticcio non infligge danni al fustigatore, che può produrre un viticcio di rimpiazzo nel suo prossimo turno. Un viticcio può essere anche rotto se una creatura effettua un'azione e riesce una prova di Forza CD 15 contro di esso.

\textbf{Azioni}

\emph{\textbf{Multiattacco.}} Il fustigatore può effettuare quattro attacchi con i suoi viticci, usare avvolgere e effettuare un attacco con il morso.

\emph{\textbf{Morso.} Attacco con arma da mischia}: +7 a colpire, portata 1 m, un bersaglio.

\emph{Colpisce:} 22 (4d8 + 4) danni perforanti.

\emph{\textbf{Viticcio.} Attacco con arma da mischia}: +7 a colpire, portata 15 m, una creatura.

\emph{Colpisce:} Il bersaglio è afferrato (CD 15 per fuggire). Fino al termine dell'afferrare, il bersaglio è intralciato e ha -1d6 alle prove di Forza e ai Tiri Salvezza su Tempra, mentre il fustigatore non può usare lo stesso viticcio contro un altro bersaglio.

\emph{\textbf{Avvolgere.}} Il fustigatore trascina le creature afferrate da lui di 7 metri verso di lui.

\textbf{Ecologia}
Ambiente: Qualsiasi Sotterraneo\\
Organizzazione: Solitario, coppia o gruppo (3-6)\\
Tesoro: Standard\\
\textbf{Descrizione}\\
Il fustigatore è un cacciatore da agguato. Capace di modificare la colorazione e la forma del suo corpo, un fustigatore nascosto sembra una stalagmite di pietra o ghiaccio (o in luoghi dal soffitto basso, una colonna di pietra o ghiaccio). Nelle aree prive di questi tratti per nascondersi un fustigatore può comprimere il suo corpo fino a sembrare un masso. Le sferze che può estroflettere non sono di carne ma di uno spesso materiale semiliquido simile a cera parzialmente fusa ma con la resistenza di una catena di ferro e la capacità di intirizzire la carne e indebolire le forze. Il fustigatore può usare queste sferze con grande maestria e farle volare fino a 15 metri per rubare gli oggetti che attraggono la sua attenzione.\\

Nonostante la sua forma aliena e mostruosa, il fustigatore è uno degli abitanti più intelligenti del sottosuolo. Non formano vaste società (anche se spesso si trovano a vivere insieme ad altre creature del sottosuolo come Divoracervelli e Neothelid, con cui a volte si alleano), ma spesso si aggregano in piccoli gruppi. Particolarmente interessato alla filosofia della vita e della morte, e agli aspetti più sottili delle religioni più sinistre e crudeli del mondo, un fustigatore può parlare o discutere per ore con quelli che inizialmente aveva semplicemente cercato di mangiare. Alcune storie parlano di oratori e filosofi particolarmente dotati che sono stati tenuti per giorni o anche anni come animali domestici o compagni di conversazione da gruppi di fustigatori; alla fine, però, se non riescono a scappare, l'appetito dei fustigatori finisce per avere la meglio sulla loro curiosa intelligenza, specialmente nei casi in cui questi animali da compagnia superano costantemente l'arguzia e la pazienza dei loro guardiani.\\
Un fustigatore è alto 2,7 metri e pesa 1.100 kg.\\


\medskip\index{Mostri - Gargoyle}\textbf{Gargoyle}

\emph{Media elementale, caotico malvagio}

\textbf{FORZA} +2

\textbf{DESTREZZA} +0

\textbf{COSTITUZIONE} +3

\textbf{INTELLIGENZA} -2

\textbf{SAGGEZZA} +0

\textbf{CARISMA} -2

\textbf{Iniziativa} +0 -- \textbf{Difesa} 16

\textbf{Punti Ferita} 52 (7d8 + 21)

\textbf{Movimento} 9 m, volo 18 m

\textbf{Tiri Salvezza}: Tempra +4, Riflessi +6, Volontà +4

\textbf{Resistenze al Danno} da botta, perforante e tagliente di attacchi non magici o che non siano di adamantio

\textbf{Immunità ai Danni} veleno

\textbf{Immunità alle Condizioni} avvelenato, pietrificato, affaticamento

\textbf{Sensi} scurovisione 18 m

\textbf{Linguaggi} Terran

\textbf{Sfida} 2 (450 PE)

\emph{\textbf{Falso Aspetto.}} Mentre la gargoyle rimane immobile, è indistinguibile da una statua inanimata.

\emph{\textbf{Natura Elementale.}} Una gargoyle non ha bisogno di aria, cibo, bevande o sonno.

\textbf{Azioni}

\emph{\textbf{Multiattacco.}} La gargoyle effettua due attacchi: uno con il morso e uno con gli artigli.

\emph{\textbf{Artigli.} Attacco con arma da mischia}: +4 a colpire, portata 1 m, un bersaglio.

\emph{Colpisce:} 5 (1d6 + 2) danni taglienti.

\emph{\textbf{Morso.} Attacco con arma da mischia}: +4 a colpire, portata 1 m, un bersaglio.

\emph{Colpisce:} 5 (1d6 + 2) danni perforanti.

\emph{Colpisce:} 5 (1d6 + 2) danni perforanti.\\
\textbf{Ecologia}
Ambiente: Qualsiasi\\
Organizzazione: Solitario, coppia o stormo (3-12)\\
Tesoro: Standard\\
\textbf{Descrizione}\\
I gargoyle spesso sembrano essere statue alate di pietra, poiché possono rimanere immobili indefinitamente per poi sorprendere i nemici. I gargoyle tendono a comportamenti ossessivo-compulsivi, tanto diversi quanto abbondante è la loro specie. Libri, ninnoli rubati, armi e trofei raccolti dai nemici caduti sono solo alcuni esempi dei tipi di oggetti che un gargoyle può collezionare per decorare la sua tana e il suo territorio.\\

I gargoyle tendono ad avere uno stile di vita solitario, anche se a volte formano temibili stormi detti "ali" per protezione e divertimento. In certe condizioni, una tribù di gargoyle può persino allearsi con altre creature, ma anche la più stabile di queste alleanze può crollare per ragioni infime; i gargoyle sono solo traditori, meschini e vendicativi.\\

I gargoyle sono noti per abitare nel cuore delle città più grandi, accovacciati tra le decorazioni di pietra delle cattedrali e degli edifici dove si nascondono in bella vista di giorno piombando giù per nutrirsi di vagabondi, mendicanti e altri sfortunati la notte.\\

Più a lungo una tribù di gargoyle dimora in un'area di edifici o rovine, più i suoi membri cominciano ad assomigliare allo stile architettonico della zona. I cambiamenti subiti dall'aspetto di un gargoyle sono lenti e sottili, ma nel corso degli anni possono diventare radicali.\\

Un'insolita variante del gargoyle non abita tra edifici e rovine ma sotto le onde del mare. Queste creature sono note come kapoacinth; hanno le stesse statistiche base dei gargoyle normali, eccetto che hanno il sottotipo acquatico e le loro ali gli garantiscono una velocità di nuotare di 12 metri (ma sono inutili per volare). I kapoacinth abitano nelle regioni costiere poco profonde dove possono strisciare fuori dalla spuma per dare la caccia ai residenti della zona. È più probabile che formino stormi, poiché i kapoacinth preferiscono la vita di gruppo a quella solitaria.\\



\subsection{Geni}

\medskip\index{Mostri - Djinni}\textbf{Djinni}

\emph{Grande elementale, caotico buono}

\textbf{FORZA} +5

\textbf{DESTREZZA} +2

\textbf{COSTITUZIONE} +6

\textbf{INTELLIGENZA} +2

\textbf{SAGGEZZA} +3

\textbf{CARISMA} +5

\textbf{Iniziativa} +2 -- \textbf{Difesa} 23

\textbf{Punti Ferita} 161 (14d10 + 84)

\textbf{Movimento} 9 m, volo 27 m

\textbf{Tiri Salvezza} Tempra +4, Riflessi +9, Volontà +7

\textbf{Immunità al Danno} fulmine, tuono

\textbf{Sensi} scurovisione 36 m

\textbf{Linguaggi} Auran

\textbf{Sfida} 11 (7.200 PE)

\emph{\textbf{Decesso Elementale.}} Se il djinni muore, il suo corpo si disintegra in una brezza calda, lasciando dietro di sé solo l'equipaggiamento che il djinni stava indossando o trasportando.

\emph{\textbf{Incantesimi Innati.}} La caratteristica da incantatore innato del djinni è il Carisma 17, +9 a colpire con attacchi da incantesimo). Può lanciare in maniera innata i seguenti incantesimi, senza bisogno di componenti materiali:

A volontà: \emph{individuazione del bene e del male, individuazione del magico, onda tonante}

3/giorno ciascuno: \emph{camminare nel vento, creare cibo e acqua} (può creare vino al posto dell'acqua), \emph{linguaggi}

1/giorno ciascuno: \emph{creazione}, \emph{evoca elementali} (solo elementale dell'aria), \emph{forma gassosa, immagine maggiore}, \emph{invisibilità,} \emph{spostamento planare}

\textbf{Azioni}

\emph{\textbf{Multiattacco.}} Il djinni effettua tre attacchi di
scimitarra.

\emph{\textbf{Scimitarra.} Attacco con arma da mischia}: +9 a colpire, portata 1 m, un bersaglio.

\emph{Colpisce:} 12 (2d6 + 5) danni taglienti più 3 (1d6) danni da fulmine o tuono (a scelta del gin).

\emph{\textbf{Creare Turbine.}} Un cilindro d'aria turbinante di 1 metro di raggio e alto 9 metri si forma magicamente in un punto visibile al djinni entro 36 metri da esso. Il turbine resta finché il djinni mantiene la concentrazione (come se si stesse concentrando su di un incantesimo). Qualsiasi creatura salvo il djinni che entri nel turbine deve riuscire un Tiro Salvezza di Tempra CD 18 o restare intralciata da esso. Il djinni può muovere il turbine di massimo 18 metri con un'azione, e le creature intralciate dal turbine si muovono con esso. Il turbine termina se il djinni lo perde di vista.

Una creatura può usare la sua azione per liberare una creatura intralciata dal turbine, compresa se stessa, riuscendo una prova di Forza CD 18. Se la prova riesce, la creatura non è più intralciata e si sposta nello spazio più vicino all'esterno del turbine.

\textbf{Ecologia}
Ambiente: Qualsiasi (Piano dell'Aria)\\
Organizzazione: Solitario, coppia, compagnia (3-6) o banda (7-10)\\
Tesoro: Standard (Scimitarra Perfetta, altro tesoro)\\
\textbf{Descrizione}\\
I Djinn (singolare djinni) sono Geni provenienti dal Piano Elementale dell'Aria. Si dice che siano fatti di nuvole e abbiano la forza delle tempeste più potenti. Un Djinni è alto circa 3 metri e pesa circa 500 kg.\\

I Djinn disdegnano il Combattimento fisico, preferendo usare i loro poteri Magici e capacità aeree contro i nemici. Un Djinni sconfitto in Combattimento generalmente prende il volo e diventa un turbine per molestare chi lo insegue. Quando non ha altra scelta che combattere in mischia, la maggioranza dei Djinn preferisce impugnare Scimitarre a Due Mani Perfette.\\

Verso gli altri Geni, i Djinn vanno d'accordo con gli Janni e i Marid. Sono frequentemente in contrasto con gli Shaitan, e sono nemici giurati degli Efreeti, disprezzando questi Geni feroci più di qualsiasi altra delle Razze di Geni. Il conflitto tra gli Efreeti e i Djinn è così leggendario che molti incantatori tentano (con vari gradi di successo) di assicurarsi il servizio di un Djinni promettendogli aiuto nella causa contro gli odiati nemici.\\


\medskip\index{Mostri - Efreeti}\textbf{Efreeti}

\emph{Grande elementale, legale malvagio}

\textbf{FORZA} +6

\textbf{DESTREZZA} +1

\textbf{COSTITUZIONE} +7

\textbf{INTELLIGENZA} +3

\textbf{SAGGEZZA} +2

\textbf{CARISMA} +3

\textbf{Iniziativa} +3 -- \textbf{Difesa} 23

\textbf{Punti Ferita} 200 (16d10 + 112)

\textbf{Movimento} 12 m, volo 18 m

\textbf{Tiri Salvezza} Tempra +7, Riflessi +10, Volontà +9

\textbf{Immunità al Danno} fuoco

\textbf{Sensi} scurovisione 36 m

\textbf{Linguaggi} Ignan

\textbf{Sfida} 11 (7.200 PE)

\emph{\textbf{Decesso Elementale.}} Se l'efreeti muore, il suo corpo si disintegra in un lampo di fuoco e uno sbuffo di fumo, lasciando dietro di sé solo l'equipaggiamento che l'efreeti stava indossando o trasportando.

\emph{\textbf{Incantesimi Innati.}} La caratteristica da incantatore innato dell'efreeti è il Carisma, +7 a colpire con attacchi da incantesimo). Può lanciare in maniera innata i seguenti incantesimi, senza bisogno di componenti materiali: 

A volontà: \emph{individuazione del magico}

3/giorno ciascuno: \emph{ingrandire/ridurre, linguaggi}

1/giorno ciascuno: \emph{evoca elementali} (solo elementale del fuoco), \emph{forma gassosa, immagine maggiore}, \emph{invisibilità, muro di fuoco, spostamento planare}

\textbf{Azioni}

\emph{\textbf{Multiattacco.}} L'efreeti effettua due attacchi di scimitarra o usa due volte Scagliare Fiamma.

\emph{\textbf{Scimitarra.} Attacco con arma da mischia}: +10 a colpire, portata 1 m, un bersaglio.

\emph{Colpisce:} 13 (2d6 + 6) danni taglienti più 7 (2d6) danni da fuoco.

\emph{\textbf{Scagliare Fiamma.} Attacco con arma a Distanza}: +7 a colpire, gittata 36 m, un bersaglio.

\emph{Colpisce:} 17 (5d6) danni da fuoco.

\textbf{Ecologia}
Ambiente: Qualsiasi (Piano del Fuoco)\\
Organizzazione: Solitario, coppia, compagnia (3-6) o banda (7-12)\\
Tesoro: Standard (Falchion Perfetto, altro tesoro)\\
\textbf{Descrizione}\\
Gli Efreet (singolare Efreeti) sono Geni provenienti dal Piano del Fuoco. Sono alti 3,6 metri e pesano circa 1.000 kg.\\
Gli Efreet hanno pochi alleati tra gli altri Geni: odiano i Djinni, e li attaccano a vista, non sopportano i Marid, e vedono i Janni come deboli e fragili. Gli Efreet spesso cooperano bene con gli Shaitan, eppure anche queste alleanze sono temporanee.\\


\subsection{Ghoul}

\medskip\index{Mostri - Ghast}\textbf{Ghast}

\emph{Media non morto, caotico malvagio}

\textbf{FORZA} +3

\textbf{DESTREZZA} +3

\textbf{COSTITUZIONE} +0

\textbf{INTELLIGENZA} +0

\textbf{SAGGEZZA} +0

\textbf{CARISMA} -1

\textbf{Iniziativa} +3 -- \textbf{Difesa} 14

\textbf{Punti Ferita} 36 (8d8)

\textbf{Movimento} 9 m

\textbf{Tiri Salvezza}: Tempra +2, Riflessi +2, Volontà +5

\textbf{Resistenze al Danno} da Vuoto

\textbf{Immunità al Danno} veleno

\textbf{Immunità alle Condizioni} affascinato, avvelenato, affaticamento

\textbf{Sensi} scurovisione 18 m

\textbf{Linguaggi} Comune

\textbf{Sfida} 2 (450 PE)

\emph{\textbf{Fetore.}} Qualsiasi creatura che inizi il suo turno entro 1 metro dal ghast deve riuscire un Tiro Salvezza di Tempra CD 10 o restare avvelenata fino all'inizio del suo prossimo turno. Se riesce il Tiro Salvezza, la creatura è immune al Fetore del ghast per le successive 24
ore.

\emph{\textbf{Ribellione allo Scacciare.}} Il ghast e tutti i ghoul entro 9 metri da esso hanno +1d6 ai Tiri Salvezza contro gli effetti che scacciano i non morti.

\textbf{Azioni}

\emph{\textbf{Artigli.} Attacco con arma da mischia}: +5 a colpire, portata 1 m, un bersaglio.

\emph{Colpisce:} 10 (2d6 + 3) danni taglienti. Se il bersaglio è una creatura, diversa da un non morto, deve riuscire un Tiro Salvezza su Tempra CD 10 o restare paralizzata per 1 minuto. Il bersaglio può ripetere il Tiro Salvezza al termine di ciascun suo turno, terminando l'effetto se riesce il Tiro Salvezza. 

\emph{\textbf{Morso.} Attacco con arma da mischia}: +3 a colpire, portata 1 m, una creatura.

\emph{Colpisce:} 12 (2d8 + 3) danni perforanti.

\textbf{Ecologia}\\
Ambiente: Qualsiasi terreno\\
Organizzazione: Solitario, gruppo (2-4) o branco (7-12)\\
Tesoro: Standard\\
\textbf{Descrizione}\\
I ghast sono Ghoul con l'archetipo semplice avanzato. La paralisi di un ghast ha effetto anche sugli Elfi. I ghast si aggirano in branchi o comandano gruppi di Ghoul comuni. Il fetore di morte e putrefazione che circonda queste creature è travolgente.\\


\medskip\index{Mostri - Ghoul}\textbf{Ghoul}

\emph{Media non morto, caotico malvagio}

\textbf{FORZA} +1

\textbf{DESTREZZA} +2

\textbf{COSTITUZIONE} +0

\textbf{INTELLIGENZA} -2

\textbf{SAGGEZZA} +0

\textbf{CARISMA} -2

\textbf{Iniziativa} +2 -- \textbf{Difesa} 13

\textbf{Punti Ferita} 22 (5d8)

\textbf{Movimento} 9 m

\textbf{Tiri Salvezza}: Tempra +1, Riflessi +2, Volontà +4

\textbf{Immunità al Danno} veleno

\textbf{Immunità alle Condizioni} affascinato, avvelenato, affaticamento

\textbf{Sensi} scurovisione 18 m

\textbf{Linguaggi} Comune

\textbf{Sfida} 1 (200 PE)

\textbf{Azioni}

\emph{\textbf{Artigli.} Attacco con arma da mischia}: +4 a colpire, portata 1 m, un bersaglio.

\emph{Colpisce:} 7 (2d4 + 2) danni taglienti. Se il bersaglio è una creatura, diversa da un elfo o un non morto, deve riuscire un Tiro Salvezza su Tempra CD 10 o restare paralizzata per 1 minuto. Il bersaglio può ripetere il Tiro Salvezza al termine di ciascun suo turno, terminando l'effetto se riesce il Tiro Salvezza.

\emph{\textbf{Morso.} Attacco con arma da mischia}: +2 a colpire, portata 1 m, una creatura.

\emph{Colpisce:} 9 (2d6 + 2) danni perforanti.

\textbf{Ecologia}
Ambiente: Qualsiasi terreno\\
Organizzazione: Solitario, gruppo (2-4) o branco (7-12)\\
Tesoro: Standard\\
\textbf{Descrizione}\\
I ghoul sono non morti che frequentano i cimiteri e mangiano i cadaveri. Le leggende sostengono che i primi ghoul fossero umani cannibali che una fame innaturale ha riportato indietro dalla morte, oppure umani che in vita si nutrivano dei resti in decomposizione dei loro simili e che morirono (e poi rinacquero) a causa di un'orrenda malattia; la vera origine di questi non morti necrofagi è incerta.\\
I ghoul si appostano ai margini della civilizzazione (dentro o nei pressi dei cimiteri o nelle fogne cittadine) dove possono reperire ampie scorte del loro cibo preferito. Sebbene preferiscano i corpi in putrefazione e spesso seppelliscano le loro vittime per migliorarne il sapore, mangiano i morti freschi se hanno abbastanza fame.\\

Anche se molti ghoul di superficie vivono in modo primitivo, delle voci parlano di città di ghoul nelle profondità del sottosuolo comandate da sacerdoti che adorano antiche divinità crudeli o strani signori dei demoni della fame. Questi ghoul "civilizzati" non sono meno orribili nelle loro abitudini alimentari, e in effetti il loro concetto di tavola ben imbandita per banchetti è forse anche più orrendo dell'idea di un pasto fresco prelevato da una bara.\\


\subsection{Giganti}

\medskip\index{Mostri - Gigante di Collina}\textbf{Gigante di Collina}

\emph{Enorme gigante, caotico malvagio}

\textbf{FORZA} +5

\textbf{DESTREZZA} -1

\textbf{COSTITUZIONE} +4

\textbf{INTELLIGENZA} -3

\textbf{SAGGEZZA} -1

\textbf{CARISMA} -2

\textbf{Iniziativa} -1 -- \textbf{Difesa} 16

\textbf{Punti Ferita} 105 (10d12 + 40)

\textbf{Movimento} 12 m

\textbf{Tiri Salvezza}: Tempra +11, Riflessi +2, Volontà +3

\textbf{Competenze} Consapevolezza +2

\textbf{Linguaggi} Gigante

\textbf{Sfida} 5 (1.800 PE)

\textbf{Azioni}

\emph{\textbf{Multiattacco.}} Il gigante effettua due attacchi con il randello pesante.

\emph{\textbf{Randello Pesante.} Attacco con arma da mischia}: +8 a colpire, portata 3 m, un bersaglio.

\emph{Colpisce:} 18 (3d8 + 5) danni da botta.

\emph{\textbf{Sasso.} Attacco con arma a Distanza}: +8 a colpire, gittata 18m, un bersaglio.

\emph{Colpisce:} 21 (3d10 + 5) danni da botta.

\textbf{Ecologia}\\
Ambiente: Colline Temperate\\
Organizzazione: Solitario, gruppo (2-5), banda (6-8), gruppo di razzia (9-12 più 1d4 Lupi Crudeli) o tribù (13-30 più 35\% non combattente più 1 capo combattente di 4°-6° livello, 11-16 Lupi Crudeli, 1-4 Ogre e 13-20 schiavi orchi)\\
Tesoro: Standard (Armatura di Pelle, Randello Pesante, altro tesoro)\\
\textbf{Descrizione}\\
I giganti delle colline hanno pelle che varia dal marrone chiaro al rossastro, capelli castani o neri, ed occhi dello stesso colore. Indossano strati di pelli rozzamente conciate con ancora il pelo. Raramente lavano o riparano i propri indumenti, e preferiscono semplicemente aggiungere nuovi strati man mano che i vecchi si logorano. Gli adulti sono alti circa 3 metri e pesano più o meno 550 kg. I giganti delle colline possono vivere fino a 200 anni, anche se raramente raggiungono quest'età.\\
I giganti delle colline preferiscono combattere dall'alto di sporgenze e rupi, da dove possono colpire gli avversari con rocce e massi, limitando così il rischio personale. Amano effettuare attacchi di oltrepassare contro creature più piccole all'inizio del combattimento, e solo dopo prendono posizione e iniziano a roteare i loro massicci randelli.\\
I giganti delle colline sono per natura nomadi e preferiscono viaggiare da un luogo all'altro per razziare e saccheggiare. Sebbene gradiscano di più i climi temperati, non disdegnano di viaggiare lontano dal loro ambiente favorito, se la razzia è abbondante e prospera. Si tratta, nel complesso, di creature molto egoiste, che raramente affrontano battaglie che non siano sicuri di vincere. I giganti delle colline sono noti per l'abitudine di spingersi l'un l'altro se devono confrontarsi con avversari temibili e non esitano a sacrificare un compagno per salvarsi la pelle. Bande erranti di giganti delle colline sono diffuse sulle colline temperate, e la loro costante aggressività li rende uno dei pericoli più temuti in questo ambiente.\\

I giganti delle colline solitari e non malvagi sono molto rari, ma li si può trovare qualche volta in altre società umanoidi, anche se non sono quasi mai accettati nelle città principali o nei centri popolati. Si trovano a proprio agio come lavoratori e soldati nelle remote città di frontiera, e spesso fungono da rudimentali diplomatici per negoziare con le bande di giganti delle colline razziatori. Sfortunatamente, i giganti delle colline che abbandonano il proprio stile di vita razziale per la civiltà vengono derisi e spesso uccisi a vista dai loro fratelli nomadi. Tuttavia, questi giganti delle colline "civilizzati" possono trovare il proprio posto nella società e molti sono riusciti a vivere un'esistenza pacifica e tranquilla.\\


\medskip\index{Mostri - Gigante del Fuoco}\textbf{Gigante del Fuoco}

\emph{Enorme gigante, legale malvagio}

\textbf{FORZA} +7

\textbf{DESTREZZA} -1

\textbf{COSTITUZIONE} +6

\textbf{INTELLIGENZA} +0

\textbf{SAGGEZZA} +2

\textbf{CARISMA} +1

\textbf{Iniziativa} +0 -- \textbf{Difesa} 27 (armatura di piastre)

\textbf{Punti Ferita} 162 (13d12 + 78)

\textbf{Movimento} 9 m

\textbf{Tiri Salvezza}: Tempra +14, Riflessi +4, Volontà +9

\textbf{Competenze} Acrobatica +11, Consapevolezza +6

\textbf{Immunità ai Danni} fuoco

\textbf{Linguaggi} Gigante

\textbf{Sfida} 9 (5.000 PE)

\textbf{Azioni}

\emph{\textbf{Multiattacco.}} Il gigante effettua due attacchi con lo spadone.

\emph{\textbf{Spadone.} Attacco con arma da mischia}: +11 a colpire, portata 3 m, un bersaglio.

\emph{Colpisce:} 28 (6d6 + 7) danni taglienti.

\emph{\textbf{Sasso.} Attacco con arma a Distanza}: +11 a colpire, gittata 18m, un bersaglio.

\emph{Colpisce:} 29 (4d10 + 7) danni da botta.

\textbf{Ecologia}
Ambiente: Montagne calde\\
Organizzazione: Solitario, gruppo (2-5), banda (6-12 più un 35\% non combattenti e 1 adepto o Devoto di 1°-2° livello), gruppo di razziatori (6-12 più 1 adepto o mago di 3°-5° livello, 2-5 Segugi Infernali e 2-3 Troll o Ettin) o tribù (20-30 più 1 adepto, mago o Devoto di 6°-7° livello; 1 re Guerriero o guardiaboschi di 8°-9° livello; e 17-38 Segugi Infernali, 12-22 Troll, 7-12 Ettin e 1-2 Draghi Rossi Giovani)\\
Tesoro: Standard (Mezza Armatura, Spadone, altro tesoro)\\
\textbf{Descrizione}\\
I giganti del fuoco sono i giganti più rigidi e marziali, sempre pronti alla guerra e a trattare brutalmente chiunque incontrino. La loro rigida struttura di comando prevede soldati, ufficiali e persino generali, e che tutti obbediscano agli ordini del loro re senza discutere.\\

I giganti del fuoco hanno capelli arancione brillante che splendono e scintillano come se fossero in fiamme. Un maschio adulto è alto tra i 3,6 e i 4,8 metri, con una cassa toracica di circa 2,7 metri, e pesa circa 3.500 kg. Le femmine sono leggermente più basse e snelle. I giganti del fuoco possono vivere fino a 350 anni.\\

I giganti del fuoco indossano abiti di tessuti robusti o di pelle di color arancione, giallo, nero o rosso. I guerrieri indossano elmi e mezze armature di acciaio brunito e impugnano grandi spadoni che mulinano per il campo di battaglia. In gruppi numerosi, i giganti del fuoco combattono con tattiche di gruppo brutali ed efficienti, e non esitano a sacrificare qualche compagno per tendere un'imboscata al nemico.\\

I giganti del fuoco preferiscono i luoghi caldi: più caldi sono meglio è. Si possono trovare nei deserti, nei vulcani, nelle fonti termali e nelle profondità della terra nei pressi di camini lavici. Vivono in castelli, insediamenti fortificati o grandi caverne, e l'architettura di questi luoghi riflette il loro stile di vita rigido e militaristico, con gli ufficiali che abitano in alloggi migliori di quelli dei loro sottoposti.\\



\medskip\index{Mostri - Gigante del Gelo}\textbf{Gigante del Gelo}

\emph{Enorme gigante, neutrale malvagio}

\textbf{FORZA} +6

\textbf{DESTREZZA} -1

\textbf{COSTITUZIONE} +5

\textbf{INTELLIGENZA} -1

\textbf{SAGGEZZA} +0

\textbf{CARISMA} +1

\textbf{Iniziativa} -1 -- \textbf{Difesa} 19 (armatura composita)

\textbf{Punti Ferita} 138 (12d12 + 60)

\textbf{Movimento} 12 m

\textbf{Tiri Salvezza} Tempra +14, Riflessi +3, Volontà +6

\textbf{Competenze} Acrobatica +9, Consapevolezza +3

\textbf{Immunità ai Danni} freddo

\textbf{Linguaggi} Gigante

\textbf{Sfida} 8 (3.900 PE)

\textbf{Azioni}

\emph{\textbf{Multiattacco.}} Il gigante effettua due attacchi con l'ascia bipenne.

\emph{\textbf{Ascia Bipenne.} Attacco con arma da mischia}: +9 a colpire, portata 3 m, un bersaglio.

\emph{Colpisce:} 25 (3d12 + 6) danni taglienti.

\emph{\textbf{Sasso.} Attacco con arma a Distanza}: +9 a colpire, gittata 18m, un bersaglio.

\emph{Colpisce:} 28 (4d10 + 6) danni da botta.

\textbf{Ecologia}\\
Ambiente: Montagne fredde\\
Organizzazione: Solitario, banda (3-5), gruppo (6-12 più 35\% non combattenti e 1 mago o Devoto di 1°-2° livello), gruppo di razziatori (6-12 più 35\% non combattenti, 1 Devoto o mago di 3°-5° livello, 1-4 Lupi Invernali e 2-3 Ogre) o tribù (21-30 più 1 adepto, mago o Devoto di 6°-7° livello; 1 jarl Barbaro o guardiaboschi 7°-9° livello; e 15-36 Lupi Invernali, 13-22 Ogre e 1-2 Draghi Bianchi Giovani)\\
Tesoro: Standard (Giaco di Maglia, Ascia Bipenne, altro tesoro)\\
\textbf{Descrizione}\\
Un gigante del gelo ha capelli azzurri o giallo sporco, e occhi in genere dello stesso colore. Si vestono con pelli e pellicce, adornandosi con qualsiasi gioiello possiedano. I giganti del gelo combattenti indossano anche giachi di maglia ed elmi di metallo decorati con corna e piume. Un maschio adulto è alto 5 metri e pesa circa 1.400 kg. Le femmine sono leggermente più basse e snelle, ma per il resto sono identiche ai maschi. I giganti del gelo possono vivere fino a 250 anni.\\

I giganti del gelo sono molto temuti, poiché la brama di distruzione e guerra ed il loro comportamento sprezzante li spingono a manifestazioni di brutalità sempre maggiori. I giganti del gelo iniziano attaccando a distanza, scagliando rocce finché finiscono le munizioni o l'avversario si avvicina, poi lo affrontano con le loro enormi asce. Una delle tattiche preferite è tendere un'imboscata nascondendosi sotto la neve al di sopra di un pendio ghiacciato o innevato, dove gli avversari avranno difficoltà a raggiungerli, e poi iniziano causando una valanga prima di scendere in battaglia. I giganti del gelo possono nascondersi molto bene negli ambienti nevosi e sono dei maestri nella furtività nel loro dominio.\\

I giganti del gelo sopravvivono cacciando e razziando da soli, dato che vivono in ambienti freddi e desolati. I gruppi di giganti del gelo sono divisi quasi equamente tra quelli che vivono in insediamenti di fortuna o castelli abbandonati e quelli che vagabondano per il gelido nord, come nomadi in cerca di bottino e provviste. I capi dei giganti del gelo si chiamano jarl e richiedono obbedienza assoluta ai loro seguaci. In ogni momento uno jarl può essere sfidato in combattimento per il comando della tribù. Queste sfide tipicamente finiscono con la morte di uno dei contendenti. Un singolo jarl può spesso contare su una dozzina o più di tribù più piccole di giganti del gelo come estensione della sua. In questi casi, i capi delle tribù minori sono noti come capitani o signori della guerra.\\

I giganti del gelo amano prendere prigionieri e li usano sia come schiavi che come materia prima. Di solito ogni gruppo di giganti del gelo tiene 1-2 schiavi umanoidi incatenati ad un addestratore di schiavi: il più meschino e crudele del gruppo dopo lo jarl. Hanno anche una certa passione per gli animali domestici mostruosi: Draghi Bianchi e Lupi Invernali sono scelte popolari, ma nella tana di un gigante del gelo si possono trovare anche Remorhaz e Yeti.\\



\medskip\index{Mostri - Gigante delle Nuvole}\textbf{Gigante delle Nuvole}

\emph{Enorme gigante, neutrale buono (50\%) o neutrale malvagio (50\%)}

\textbf{FORZA} +8

\textbf{DESTREZZA} +0

\textbf{COSTITUZIONE} +6

\textbf{INTELLIGENZA} +1

\textbf{SAGGEZZA} +3

\textbf{CARISMA} +3

\textbf{Iniziativa} +1 -- \textbf{Difesa} 19

\textbf{Punti Ferita} 200 (16d12 + 96)

\textbf{Movimento} 12 m

\textbf{Tiri Salvezza} Tempra +16, Riflessi +6, Volontà +10

\textbf{Competenze} Percepire Emozioni +7, Consapevolezza +7

\textbf{Linguaggi} Comune, Gigante

\textbf{Sfida} 9 (5.000 PE)

\emph{\textbf{Incantesimi Innati.}} La caratteristica da incantatore del gigante è il Carisma. Il gigante può lanciare questi incantesimi in maniera innata, senza bisogno di componenti materiali:

A volontà: \emph{individuazione del magico, luce, nube di nebbia}

3/giorno ciascuno: \emph{caduta morbida, passo nebbioso, telecinesi}

1/giorno ciascuno: \emph{controllare tempo atmosferico, forma gassosa}

\emph{\textbf{Olfatto Affinato.}} Il gigante ha +1d6 alle prove di Saggezza (Consapevolezza) basate sull'olfatto.

\textbf{Azioni}

\emph{\textbf{Multiattacco.}} Il gigante effettua due attacchi con la morning star.

\emph{\textbf{Morning star.} Attacco con arma da mischia}: +12 a colpire, portata 3 m, un bersaglio.

\emph{Colpisce:} 21 (3d8 + 8) danni perforanti.

\emph{\textbf{Sasso.} Attacco con arma a Distanza}: +12 a colpire, gittata 18m, un bersaglio.

\emph{Colpisce:} 30 (4d10 + 8) danni da botta.

\textbf{Ecologia}\\
Ambiente: Montagne Temperate\\
Organizzazione: Solitario, gruppo (2-5), famiglia (2-5 più 35\% non combattenti più 1 mago o Devoto di 4°-7° livello e 2-5 Grifoni) o tribù (6-20 più 1 oracolo mago o Devoto di 7°-12° livello e 2-5 Grifoni)\\
Tesoro: Standard (Giaco di Maglia, Morning Star, altro tesoro)\
\textbf{Descrizione}\\
Il colore pelle dei giganti delle nuvole varia dal bianco latte al blu polvere. I maschi adulti sono alti circa 5,4 metri e pesano approssimativamente 2.500 kg. Le femmine sono leggermente più basse e snelle. I giganti delle nuvole possono vivere fino a 400 anni, vestono con abiti preziosi e gioielli. Per molti l'aspetto indica lo status. Migliori sono i vestiti e più raffinati i gioielli, più importante è chi li indossa. Inoltre apprezzano la musica, e la maggioranza suona uno o più strumenti (l'arpa è uno dei preferiti).\\

I giganti delle nuvole possono avere Tratti insolitamente vari; circa metà sono buoni e metà malvagi. I giganti delle nuvole buoni costruiscono strade che collegano i loro insediamenti con le strade degli umani per promuovere il commercio. Non è insolito vedere un gigante delle nuvole buono camminare tra gli uomini, ad esempio, in una città umana nei pressi di un'alta catena montuosa. I giganti delle nuvole malvagi tendono a non creare insediamenti stabili e anzi preferiscono vivere in rozzi rifugi su alti picchi, da cui scendono solo per depredare i villaggi di quello di cui potrebbero aver bisogno. Queste due filosofie portano spesso allo scoppio di guerre violente e durature tra tribù vicine.\\

Sono molte le leggende che parlano di magiche città dei giganti delle nuvole situate tra le nuvole stesse, che fluttuano sui venti e circumnavigano il mondo. Mentre i giganti delle nuvole riconoscono che si tratta per lo più di fantasie, alcuni sostengono di averle viste e hanno dedicato la loro intera esistenza a ritrovarle.\\


\medskip\index{Mostri - Gigante di Pietra}\textbf{Gigante di Pietra}

\emph{Enorme gigante, neutrale}

\textbf{FORZA} +6

\textbf{DESTREZZA} +2

\textbf{COSTITUZIONE} +5

\textbf{INTELLIGENZA} +0

\textbf{SAGGEZZA} +1

\textbf{CARISMA} -1

\textbf{Iniziativa} +2 -- \textbf{Difesa} 21

\textbf{Punti Ferita} 126 (11d12 + 55)

\textbf{Movimento} 12 m

\textbf{Tiri Salvezza} Tempra +12, Riflessi +6, Volontà +7

\textbf{Competenze} Acrobatica +12, Consapevolezza +4

\textbf{Sensi} scurovisione 18 m

\textbf{Linguaggi} Gigante

\textbf{Sfida} 7 (2.900 PE)

\emph{\textbf{Mimetismo di Pietra.}} Il gigante ha +1d6 alle prove di Destrezza (Nascondersi) effettuate per nascondersi su terreni rocciosi.

\textbf{Azioni}

\emph{\textbf{Multiattacco.}} Il gigante effettua due attacchi con il randello pesante.

\emph{\textbf{Randello Pesante.} Attacco con arma da mischia}: +9 a colpire, portata 5 metri, un bersaglio.

\emph{Colpisce:} 19 (3d8 + 6) danni da botta.

\emph{\textbf{Sasso.} Attacco con arma a Distanza}: +9 a colpire, gittata 18m, un bersaglio.

\emph{Colpisce:} 28 (4d10 + 6) danni da botta. Se il bersaglio è una creatura, deve riuscire un Tiro Salvezza di Tempra CD 17 o cadere prona.

\textbf{Reazioni}

\emph{\textbf{Afferrare Sassi.}} Se un sasso o un simile oggetto viene scagliato al gigante, il gigante può, riuscendo un Tiro Salvezza su Riflessi CD 10, afferrare il proiettile e non subire danni da botta da esso.

\textbf{Ecologia}
Ambiente: Montagne temperate\\
Organizzazione: Solitario, gruppo (2-5), banda (4-8), gruppo di caccia (9-12 più 1 Anziano) o tribù (13-30 più 35\% non combattenti, 1-3 Anziani e 4-6 Orsi Crudeli)\\
Tesoro: Standard (Randello Pesante, altro tesoro)\\
\textbf{Descrizione}\\
I giganti delle rocce preferiscono spessi indumenti di cuoio, tinti con tonalità di marrone e grigio per confondersi con la pietra che li circonda. Gli adulti sono alti circa 3,6 metri, pesano circa 750 kg e possono vivere fino a 800 anni.\\
I giganti delle rocce, se possibile, combattono a distanza, ma se non possono evitare la mischia usano giganteschi randelli di pietra. Una delle tattiche favorite dai giganti delle rocce è di stare immobili, mimetizzandosi con il paesaggio, per poi avanzare scagliando rocce e sorprendere i nemici.\\

I giganti delle rocce preferiscono vivere in enormi caverne sulle cime rocciose. Raramente vivono a più di qualche giorno di viaggio da altre bande di giganti delle rocce e allevano greggi condivisi di capre e altro bestiame.\\

I giganti delle rocce più vecchi tendono ad allontanarsi dalla tribù per molto tempo, per vivere in solitudine da qualche parte o tentando di inserirsi in altre civiltà umanoidi. Dopo decadi di esilio auto imposto, chi fa ritorno è noto come Gigante delle Rocce Anziano.\\


\medskip\index{Mostri - Gigante delle Tempeste}\textbf{Gigante delle Tempeste}

\emph{Enorme gigante, caotico buono}

\textbf{FORZA} +9

\textbf{DESTREZZA} +2

\textbf{COSTITUZIONE} +5

\textbf{INTELLIGENZA} +3

\textbf{SAGGEZZA} +4

\textbf{CARISMA} +4

\textbf{Iniziativa} +3 -- \textbf{Difesa} 23 (armatura di scaglie)

\textbf{Punti Ferita} 230 (20d12 + 100)

\textbf{Movimento} 15 m, nuoto 15 m

\textbf{Tiri Salvezza} Tempra +17, Riflessi +8, Volontà +13

\textbf{Competenze} Arcano +8, Acrobatica +14, Consapevolezza +9, Storia +8

\textbf{Resistenze al Danno} freddo

\textbf{Immunità al Danno} fulmine, tuono

\textbf{Linguaggi} Comune, Gigante

\textbf{Sfida} 13 (10.000 PE)

\emph{\textbf{Anfibio.}} Il gigante può respirare aria e acqua.

\emph{\textbf{Incantesimi Innati.}} La caratteristica da incantatore del gigante è il Carisma. Il gigante può lanciare questi incantesimi in maniera innata, senza bisogno di componenti materiali:

A volontà: \emph{caduta controllata, individuazione del magico,} \emph{levitazione, luce}

3/giorno ciascuno: \emph{controllare tempo atmosferico, respirare} \emph{sott'acqua}

\textbf{Azioni}

\emph{\textbf{Multiattacco.}} Il gigante effettua due attacchi con lo spadone.

\emph{\textbf{Spadone.} Attacco con arma da mischia}: +14 a colpire, portata 3 m, un bersaglio.

\emph{Colpisce:} 30 (6d6 + 9) danni taglienti.

\emph{\textbf{Sasso.} Attacco con arma a Distanza}: +14 a colpire, gittata 18m, un bersaglio.

\emph{Colpisce:} 35 (4d12 + 9) danni da botta.

\emph{\textbf{Colpo Fulminante (Ricarica 5-6).}} Il gigante scaglia una folgore magica ad un punto visibile entro 150 metri da sé. Ogni creatura entro 3 metri da quel punto deve effettuare un Tiro Salvezza su Riflessi CD 17, subendo 54 (12d8) danni da fulmine se lo fallisce, o la metà se lo supera.

\textbf{Ecologia}\\
Ambiente: Qualsiasi caldo\\
Organizzazione: Solitario o famiglia (2-5 più 1 mago o Devoto di livello 7°-10°, 1-2 Roc, 2-6 Grifoni e 2-8 Squali)\\
Tesoro: Standard (Corazza di Piastre Perfetta, Arco Lungo Composito Perfetto [Forza +14] con 20 Frecce, Spadone Perfetto, altro tesoro)\\
\textbf{Descrizione}\\
I giganti delle tempeste tendono ad avere carnagione abbronzata, anche se rari esemplari hanno pelle viola, capelli viola o blu scuri e occhi grigio argento o porpora. Il colore viola è considerato di buon auspicio tra i giganti delle tempeste, e coloro che lo posseggono tendono a diventare capi tra la loro gente. Gli adulti sono normalmente alti 6,3 metri e pesano 6.000 kg. I giganti delle tempeste possono vivere fino a 600 anni.\\

Quando sono a riposo, preferiscono indossare tuniche corte e ampie cinte ai fianchi, sandali o piedi nudi e una fascia per capelli. Indossano pochi gioielli di semplice ma ottima fattura, i più comuni sono cavigliere (preferite dai giganti a piedi scalzi), anelli o diademi. Ma quando si equipaggiano per la battaglia, indossano corazze di piastre scintillanti e impugnano enormi spadoni e archi.\\

I giganti delle tempeste sono tendenzialmente solitari, preferendo abitare lungo remote coste o su isole tropicali. Come suggerisce il loro nome, sono inclini a violenti sbalzi di umore. I giganti delle tempeste sono facili all'ira di fronte al male e possono essere nemici brutali e pericolosi quando vengono insultati. In battaglia, preferiscono scagliare una pioggia di frecce sui loro nemici, estraendo gli spadoni solo dopo che gli avversari si sono avvicinati.\\

I giganti delle tempeste vivono in belle torri, castelli o in insediamenti cinti da mura e amano coltivare la terra. Possiedono enormi giardini ben curati e gestiscono centinaia di acri di coltivazioni per gruppo. Spesso impiegano altri umanoidi, come Elfi o Umani, come supporto per condurre le loro immense fattorie. Una enclave di giganti delle tempeste spesso si assume la responsabilità della sicurezza di un'intera isola o linea di costa.\\


\medskip\index{Mostri - Gnoll}\textbf{Gnoll}

\emph{Media umanoide (gnoll), caotico malvagio}

\textbf{FORZA} +2

\textbf{DESTREZZA} +1

\textbf{COSTITUZIONE} +0

\textbf{INTELLIGENZA} -2

\textbf{SAGGEZZA} +0

\textbf{CARISMA} -2

\textbf{Iniziativa} +1 -- \textbf{Difesa} 16 (armatura di pelle, scudo)

\textbf{Punti Ferita} 22 (5d8)

\textbf{Movimento} 9 m

\textbf{Tiri Salvezza}: Tempra +4, Riflessi +0, Volontà +0

\textbf{Sensi} scurovisione 18 m

\textbf{Linguaggi} Gnoll

\textbf{Sfida} 1/2 (100 PE)

\emph{\textbf{Rabbia.}} Quando lo gnoll riduce una creatura a 0 punti ferita con un attacco da mischia durante il proprio turno, può svolgere un'azione bonus per muoversi fino a metà del suo movimento ed effettuare un attacco di morso.

\textbf{Azioni}

\emph{\textbf{Morso.} Attacco con arma da mischia}: +4 a colpire, portata 1 m, una creatura.

\emph{Colpisce:} 4 (1d4 + 2) danni perforanti.

\emph{\textbf{Lancia.} Attacco con arma da mischia o a Distanza}: +4 a colpire, portata 1 m o gittata 6 m, un bersaglio.

\emph{Colpisce:} 5 (1d6 + 2) danni perforanti o 6 (1d8 + 2) danni perforanti se usata con due mani per effettuare un attacco da mischia.

\emph{\textbf{Arco Lungo.} Attacco con arma a Distanza}: +3 a colpire, gittata 45m, un bersaglio.

\emph{Colpisce:} 5 (1d8 + 1) danni perforanti.

\textbf{Ecologia}\\
Ambiente: Pianure calde, deserti\\
Organizzazione: Solitario, coppia, gruppo di caccia (2-5 e 1-2 Iene), banda (10-100 adulti più 50\% piccoli non combattenti, 1 sergente di 3° livello ogni 20 adulti, 1 capo di 4°-6° livello e 5-8 Iene) o tribù (20-200 più 1 sergente di 3° livello ogni 20 adulti, 1 o 2 luogotenenti di 4° o 5° livello, 1 capo di 6°-8° livello, 7-12 Iene e 4-7 ienodonti)\\
Tesoro: equipaggiamento da PNG (Armatura di Cuoio, Scudo Pesante di Legno, Lancia, altro tesoro)\\
\textbf{Descrizione}\\
Gli gnoll sono una razza di umanoidi grandi e grossi che assomigliano alle iene non solo per il semplice aspetto; mostrano un'evidente affinità con questi animali spazzini, tanto da tenerli come animali di compagnia, e riflettono molti dei comportamenti di tali animali.\\

Gli gnoll sono abili cacciatori, ma preferiscono di gran lunga ripulire o trafugare una carcassa piuttosto che cacciare una preda. Questa pigrizia li spinge a procurarsi degli schiavi di qualsiasi specie disponibile per costringerli a scavare tane, raccogliere provviste e acqua e perfino cacciare per i loro padroni gnoll.\\

Le altre creature che non siano iene o gnoll diventano pasto o schiavi, a seconda del temperamento della tribù. Anche un compagno morto o caduto diventa un pasto fresco per uno gnoll, che può onorare un membro famoso della tribù con una breve preghiera o cucinarne interamente uno morto di una devastante malattia: altrimenti, gli gnoll non vedono un loro simile morto molto diversamente da qualsiasi altra creatura. Gli gnoll più "civilizzati" non mangiano i loro prigionieri: li tengono, invece, come schiavi, per difendere o migliorare la loro tana o per scambiarli con altre tribù o bande schiaviste.\\

Gli gnoll provano gusto per il combattimento, ma solo quando sono in superiorità numerica. In altre situazioni, preferiscono evitare il combattimento tranne che come mezzo per ottenere una carcassa da un altro cacciatore, o come un'ingegnosa imboscata per abbattere un lauto pasto. Questi uomini iena non vedono alcun valore nel coraggio o nell'eroismo e preferiscono invece fuggire una volta chiaro che la vittoria non è raggiungibile, sostenendo che è meglio scappare con la coda tra le gambe piuttosto che perderla del tutto.\\

Durante il combattimento, gli gnoll usano una strana combinazione di tattiche da branco e strategie individuali. Se uno gnoll è sicuro di vincere, tenta di abbattere l'avversario più debole piuttosto che aiutare i suoi compagni. Se gli gnoll sono in difficoltà, si coalizzano contro un avversario potente e tentano di eliminarlo, sperando di costringere alla fuga i suoi alleati.\\

I capi gnoll hanno competenze da guardiaboschi ma non e' impossibile trovare anche gnoll Devoti a qualche famelico Patrono. Difficilmente padroneggiano in maniera efficage la magia.\\


\medskip\index{Mostri - Gnomo delle Profondità (Svirfneblin)}\textbf{Gnomo delle Profondità (Svirfneblin)}

\emph{Piccola umanoide (gnomo), neutrale buono}

\textbf{FORZA} +2

\textbf{DESTREZZA} +2

\textbf{COSTITUZIONE} +2

\textbf{INTELLIGENZA} +1

\textbf{SAGGEZZA} +0

\textbf{CARISMA} -1

\textbf{Iniziativa} +2 -- \textbf{Difesa} 16 (giaco di maglia)

\textbf{Punti Ferita} 16 (3d6 + 6)

\textbf{Movimento} 6 m

\textbf{Tiri Salvezza}: Tempra +6, Riflessi +6, Volontà +2

\textbf{Competenze} Muoversi Silenziosamente / Nascondersi +4, Consapevolezza +2

\textbf{Sensi} scurovisione 36 m

\textbf{Linguaggi} Gnomica, Linguaggio delle Profondità, Terran

\textbf{Sfida} 1/2 (100 PE)

\emph{\textbf{Astuzia Gnomesca.}} Lo gnomo ha +1d6 ai Tiri Salvezza contro la magia.

\emph{\textbf{Camuffamento di Pietra.}} Lo gnomo ha +1d6 alle prove di Destrezza (Nascondersi) effettuate per nascondersi su terreni rocciosi.

\emph{\textbf{Incantesimi Innati.}} La caratteristica da incantatore innato dello gnomo è l'Intelligenza. Lo gnomo può lanciare questi incantesimi in maniera innata, senza bisogno di componenti:

A volontà: \emph{anti-individuazione} (personale)

1/giorno ciascuno: \emph{camuffare sé stesso, cecità/sordità, sfocatura}

\textbf{Azioni}

\emph{\textbf{Piccone da Guerra.} Attacco con arma da mischia}: +4 a colpire, portata 1 m, un bersaglio.

\emph{Colpisce:} 6 (1d8 + 2) danni perforanti.

\emph{\textbf{Dardo Avvelenato.} Attacco con arma a Distanza}: +4 a colpire, gittata 9m, un bersaglio.

\emph{Colpisce:} 4 (1d4 + 2) danni perforanti, e il bersaglio deve riuscire un Tiro Salvezza di Tempra CD 12 o restare avvelenato per 1 minuto. Il bersaglio può ripetere il Tiro Salvezza al termine di ciascun suo turno, terminando l'effetto su di sé in caso di successo.

\textbf{Ecologia}
Ambiente: Qualsiasi sotterraneo\\
Organizzazione: Solitario, compagnia (2-4), squadra (5-20 più 1 capo 3°-6° e due sergenti di 3° livello), o banda (30-50 più 1 sergente di 3° livello ogni 20 adulti, 5 tenenti di 5° livello, 3 capitani di 7° livello, e 2-5 Elementali della Terra Medi)\\
Tesoro: Equipaggiamento da PNG (Piccone Pesante, Balestra Leggera con 10 Quadrelli, altro tesoro)\\
\textbf{Descrizione}\\
Gli svirfneblin, o "gnomi delle profondità", sono una branca della razza gnomesca. Dimorano nel sottosuolo, in città nascoste, al sicuro dagli elfi scuri e da altre razze sotterranee. La loro pelle è del colore della roccia, di solito grigia o marrone. I maschi sono calvi e le femmine hanno radi capelli grigi. Il legame di uno svirfneblin con il reame dei folletti è molto più forte di quello degli Gnomi di superficie; questo rende gli svirfneblin stranamente distaccati dalle loro emozioni oppure soggetti ad improvvise e violente manifestazioni emotive. Gli svirfneblin hanno combattuto a lungo contro i Duergar e non riescono a distinguere bene i Duergar dai Nani.\\



\subsection{Golem}

\medskip\index{Mostri - Golem di Argilla}\textbf{Golem di Argilla}

\emph{Grande costrutto, disallineato}

\textbf{FORZA} +5

\textbf{DESTREZZA} -1

\textbf{COSTITUZIONE} +4

\textbf{INTELLIGENZA} -4

\textbf{SAGGEZZA} -1

\textbf{CARISMA} -5

\textbf{Iniziativa} -1 -- \textbf{Difesa} 19

\textbf{Punti Ferita} 133 (14d10 + 56)

\textbf{Movimento} 6 m

\textbf{Tiri Salvezza}: Tempra +4, Riflessi +3, Volontà +4

\textbf{Immunità al Danno} acido, psichico, veleno; da botta, perforante e tagliente di attacchi non magici o che non siano di adamantio

\textbf{Immunità alle Condizioni} affascinato, avvelenato, paralizzato, pietrificato, affaticamento, spaventato

\textbf{Sensi} scurovisione 18 m

\textbf{Linguaggi} comprende le lingue del suo creatore ma non può parlare

\textbf{Sfida} 9 (5.000 PE)

\emph{\textbf{Berserk.}} Ogni volta che il golem inizia il suo turno con 60 punti ferita o meno, tira un d6. Se ottieni 6, il golem va in berserk. Durante ogni suo turno mentre è in berserk, il golem attacca la creatura più vicina che può vedere. Se non c'è nessuna creatura abbastanza vicina da muoversi e attaccarla, il golem attacca un oggetto, con preferenza per gli oggetti più piccoli di lui. Una volta che il golem è andato in berserk, continuerà ad esserlo finché non viene distrutto o recupera tutti i suoi punti ferita.

\emph{\textbf{Armi Magiche.}} Gli attacchi con armi del golem sono magici.

\emph{\textbf{Assorbimento dell'Acido.}} Ogni volta che il golem è vittima di danni da acido, non subisce danni ma invece recupera un pari numero di punti ferita. 

\emph{\textbf{Forma Immutabile.}} Il golem è immune a qualsiasi incantesimo o effetto che altererebbe la sua forma.

\emph{\textbf{Natura di Costrutto.}} Un golem non ha bisogno di aria, cibo, bevande o sonno.

\emph{\textbf{Resistenza alla Magia.}} Il golem ha +1d6 ai Tiri Salvezza contro incantesimi e altri effetti magici.

\textbf{Azioni}

\emph{\textbf{Multiattacco.}} Il golem effettua due attacchi di schianto.

\emph{\textbf{Schianto.} Attacco con arma da mischia}: +8 a colpire, portata 1 m, un bersaglio.

\emph{Colpisce:} 16 (2d10 + 5) danni da botta. Se il bersaglio è una creatura, deve riuscire un Tiro Salvezza di Tempra CD 15 o vedere i suoi punti ferita massimi ridotti di un ammontare pari al danno subito. Il bersaglio muore se l'attacco riduce i suoi punti ferita massimi a 0. La riduzione resta finché non viene rimossa dall'incantesimo \emph{ristorare superiore} o altra magia.

\emph{\textbf{Velocità (Ricarica 5-6).}} Fino al termine del suo prossimo turno, il golem ottiene un bonus magico di +2 alla Difesa, ha +1d6 ai Tiri Salvezza su Riflessi, e può usare gli attacchi di schianto come azione bonus.

\textbf{Ecologia}\\
Ambiente: Qualsiasi\\
Organizzazione: Solitario o gruppo (2-4)\\
Tesoro: Nessuno\\
\textbf{Descrizione}\\
I golem di argilla non indossano abiti, eccezion fatta per un indumento di cuoio trattato o metallo attorno ai fianchi. Mediamente sono alti più di 2,4 metri e pesano 300 chili.\\
\textbf{Costruzione}
Un golem d'argilla può essere scolpito a partire da un unico blocco d'argilla del peso minimo di 500 chili, trattato con polveri e oli rari per il valore di 1,500 mo.\\


\medskip\index{Mostri - Golem di Carne}\textbf{Golem di Carne}

\emph{Media costrutto, neutrale}

\textbf{FORZA} +4

\textbf{DESTREZZA} -1

\textbf{COSTITUZIONE} +4

\textbf{INTELLIGENZA} -2

\textbf{SAGGEZZA} +0

\textbf{CARISMA} -3

\textbf{Iniziativa} -1 -- \textbf{Difesa} 12

\textbf{Punti Ferita} 93 (11d8 + 44)

\textbf{Movimento} 9 m

\textbf{Tiri Salvezza}: Tempra +3, Riflessi +2, Volontà +3

\textbf{Immunità al Danno} fulmine, veleno; da botta, perforante e tagliente di attacchi non magici o che non siano di adamantio

\textbf{Immunità alle Condizioni} affascinato, avvelenato, paralizzato, pietrificato, affaticamento, spaventato

\textbf{Sensi} scurovisione 18 m

\textbf{Linguaggi} comprende le lingue del suo creatore ma non può
parlare

\textbf{Sfida} 5 (1.800 PE)

\emph{\textbf{Berserk.}} Ogni volta che il golem inizia il suo turno con 40 punti ferita o meno, tira un d6. Se ottieni 6, il golem va in berserk. Durante ogni suo turno mentre è in berserk, il golem attacca la creatura più vicina che possa vedere. Se non c'è nessuna creatura abbastanza vicina da muoversi e attaccarla, il golem attacca un oggetto, con preferenza per gli oggetti più piccoli di lui. Una volta che il golem è andato in berserk, continuerà ad esserlo finché non viene distrutto o recupera tutti i suoi punti ferita.

\emph{\textbf{Armi Magiche.}} Gli attacchi con armi del golem sono magici.

\emph{\textbf{Assorbimento dei Fulmini.}} Ogni volta che il golem sia vittima di un danno da fulmine, non subisce danni ma invece recupera un pari numero di punti ferita.

\emph{\textbf{Avversione al Fuoco.}} Se il golem subisce danni da fuoco, ha -1d6 ai tiri di attacco e le prove di competenza fino alla fine del suo prossimo turno.

\emph{\textbf{Forma Immutabile.}} Il golem è immune a qualsiasi incantesimo o effetto che altererebbe la sua forma.

\emph{\textbf{Natura di Costrutto.}} Un golem non ha bisogno di aria, cibo, bevande o sonno.

\emph{\textbf{Resistenza alla Magia.}} Il golem ha +1d6 ai Tiri Salvezza contro incantesimi e altri effetti magici.

\textbf{Azioni}

\emph{\textbf{Multiattacco.}} Il golem effettua due attacchi di
schianto.

\emph{\textbf{Schianto.} Attacco con arma da mischia}: +7 a colpire,
portata 1 m, un bersaglio.

\emph{Colpisce:} 13 (2d8 + 4) danni da botta.

\textbf{Ecologia}\\
Ambiente: Qualsiasi\\
Organizzazione: Solitario o gruppo (2-4)\\
Tesoro: Nessuno\\
\textbf{Descrizione}\\
Un golem di carne è una mostruosa collezione di parti anatomiche umanoidi trafugate e cucite insieme. La sua carne cadaverica ha tonalità verde pallido o giallognola. Un golem di carne indossa qualsiasi tipo di vestito che il suo creatore desideri, normalmente solo un logoro paio di pantaloni. Non ha Equipaggiamento né armi. Un golem di carne è alto più di 2,4 metri e pesa 250 kg.\\

Un golem di carne non parla, anche se può emettere una specie di ringhio rauco. Cammina e si muove con un'andatura a scatti, come se non avesse il pieno controllo del proprio corpo.\\

Anche se molti golem di carne sono privi di ragione, si narra di golem eccezionali che in qualche modo hanno mantenuto i ricordi della vita precedente. La testa (e quindi il cervello) di questi golem di carne deve essere la giusta combinazione di freschezza e (nella vita precedente) decisione, ma di assoluta importanza sembrano essere anche la fortuna e il caso affinchè durante la loro creazione si conservi l'intelletto. Certamente quelli che costruiscono golem di carne preferiscono avere schiavi privi di intelletto piuttosto che dotati di una propria volontà, di conseguenza i golem di carne intelligenti sono rari.\\



\medskip\index{Mostri - Golem di Ferro}\textbf{Golem di Ferro}

\emph{Grande costrutto, disallineato}

\textbf{FORZA} +7

\textbf{DESTREZZA} -1

\textbf{COSTITUZIONE} +5

\textbf{INTELLIGENZA} -4

\textbf{SAGGEZZA} +0

\textbf{CARISMA} -5

\textbf{Iniziativa} -1 -- \textbf{Difesa} 28

\textbf{Punti Ferita} 210 (20d10 + 100)

\textbf{Movimento} 9 m

\textbf{Tiri Salvezza}: Tempra +6, Riflessi +5, Volontà +6

\textbf{Immunità al Danno} fuoco, psichico, veleno; da botta, perforante e tagliente di attacchi non magici o che non siano di adamantio

\textbf{Immunità alle Condizioni} affascinato, avvelenato, paralizzato, pietrificato, affaticamento, spaventato

\textbf{Sensi} scurovisione 36 m

\textbf{Linguaggi} comprende le lingue del suo creatore ma non può parlare

\textbf{Sfida} 16 (15.000 PE)

\emph{\textbf{Armi Magiche.}} Gli attacchi con armi del golem sono magici.

\emph{\textbf{Assorbimento del Fuoco.}} Ogni volta che il golem sia vittima di un danno da fuoco, non subisce danni ma invece recupera un pari numero di punti ferita.

\emph{\textbf{Forma Immutabile.}} Il golem è immune a qualsiasi incantesimo o effetto che altererebbe la sua forma.

\emph{\textbf{Natura di Costrutto.}} Un golem non ha bisogno di aria, cibo, bevande o sonno.

\emph{\textbf{Resistenza alla Magia.}} Il golem ha +1d6 ai Tiri Salvezza contro incantesimi e altri effetti magici.

\textbf{Azioni}

\emph{\textbf{Multiattacco.}} Il golem effettua due attacchi da mischia.

\emph{\textbf{Schianto.} Attacco con arma da mischia}: +13 a colpire, portata 1 m, un bersaglio.

\emph{Colpisce:} 20 (3d8 + 7) danni da botta.

\emph{\textbf{Spada.} Attacco con arma da mischia}: +13 a colpire, portata 3 m, un bersaglio.

\emph{Colpisce:} 23 (3d10 + 7) danni taglienti.

\emph{\textbf{Soffio Velenoso (Ricarica 6).}} Il golem esala un gas velenoso in un cono di 5 metri. Ogni creatura in quell'area deve effettuare un Tiro Salvezza di Tempra CD 19, subendo 45 (10d8) danni da veleno se fallisce il Tiro Salvezza, o la metà di questi danni se lo riesce.

\textbf{Ecologia}\\
Ambiente: Qualsiasi\\
Organizzazione: Solitario o gruppo (2-4)\\
Tesoro: Nessuno\\
\textbf{Descrizione}\\
Un golem di ferro ha un corpo di forma umanoide in ferro. Il creatore può dargli qualsiasi forma desideri, ma presenta quasi sempre un'armatura di qualche tipo, sia essa cerimoniale e preziosa o semplice e d'uso. Rispetto ad un golem di pietra ha sembianze molto più definite. I golem di ferro, talvolta, portano con sé un'arma, anche se il più delle volte tendono a preferirle i loro attacchi schianto.\\

Un golem di ferro è alto 3,6m e pesa circa 2.500 chili. Un golem di ferro non può parlare né emettere voce. Inoltre, non emette nessun odore riconoscibile.\\

Anche se la pratica della costruzione di golem di ferro è gradualmente caduta in disuso, i membri venerabili di alcune grandi civiltà del passato consideravano la capacità di forgiare golem di ferro dalla forza e dalle dimensioni sconcertanti un motivo di vanto. Questi golem (di taglia maggiore o uguale a Enorme), in alcuni angoli remoti del mondo, esistono ancora, e ancora eseguono meccanicamente ordini impartiti loro da imperi ormai scomparsi.\\

\textbf{Costruzione}
Per costruire un golem di ferro occorrono 2.500 kg di ferro, fuso con tinture rare del valore minimo di 10.000 mo.\\


\medskip\index{Mostri - Golem di Pietra}\textbf{Golem di Pietra}

\emph{Grande costrutto, disallineato}

\textbf{FORZA} +6

\textbf{DESTREZZA} -1

\textbf{COSTITUZIONE} +5

\textbf{INTELLIGENZA} -4

\textbf{SAGGEZZA} +0

\textbf{CARISMA} -5

\textbf{Iniziativa} -1 -- \textbf{Difesa} 22

\textbf{Punti Ferita} 178 (17d10 + 85)

\textbf{Movimento} 9 m

\textbf{Tiri Salvezza}: Tempra +4, Riflessi +3, Volontà +4

\textbf{Immunità al Danno} psichico, veleno; da botta, perforante e tagliente di attacchi non magici o che non siano di adamantio

\textbf{Immunità alle Condizioni} affascinato, avvelenato, paralizzato, pietrificato, affaticamento, spaventato

\textbf{Sensi} scurovisione 36 m

\textbf{Linguaggi} comprende le lingue del suo creatore ma non può parlare

\textbf{Sfida} 10 (5.900 PE)

\emph{\textbf{Armi Magiche.}} Gli attacchi con armi del golem sono magici.

\emph{\textbf{Forma Immutabile.}} Il golem è immune a qualsiasi incantesimo o effetto che altererebbe la sua forma.

\emph{\textbf{Natura di Costrutto.}} Un golem non ha bisogno di aria, cibo, bevande o sonno.

\emph{\textbf{Resistenza alla Magia.}} Il golem ha +1d6 ai Tiri Salvezza contro incantesimi e altri effetti magici.

\textbf{Azioni}

\emph{\textbf{Multiattacco.}} Il golem effettua due attacchi di schianto.

\emph{\textbf{Schianto.} Attacco con arma da mischia}: +10 a colpire, portata 1 m, un bersaglio.

\emph{Colpisce:} 19 (3d8 + 6) danni da botta.

\emph{\textbf{Lentezza (Ricarica 5-6).}} Il golem prende a bersaglio una o più creature entro 3 metri da lui e che possa vedere. Ciascun bersaglio deve effettuare un Tiro Salvezza di Volontà CD 17 contro questa magia. Se fallisce il Tiro Salvezza, il bersaglio non può usare reazioni, ha la velocità dimezzata, e durante il proprio turno non può effettuare più di un attacco. Inoltre, durante il proprio turno il bersaglio può effettuare un'azione o un'azione bonus, ma non entrambe. Questi effetti durano per 1 minuto. Il bersaglio può ripetere il Tiro Salvezza al termine di ciascun suo turno, terminando l'effetto per sé, in caso di successo.

\textbf{Ecologia}\\
Ambiente: Qualsiasi\\
Organizzazione: Solitario o gruppo (2-4)\\
Tesoro: Nessuno\\
\textbf{Descrizione}\\
Un golem di pietra ha un corpo umanoide fatto di pietra, spesso stilizzato per soddisfare il suo creatore. Ad esempio, può essere scolpito in modo da indossare un'armatura, con particolari simboli scolpiti sulla corazza, o avere dei disegni intarsiati nella pietra dei suoi arti. La testa è spesso scolpita per sembrare un elmo o la testa di qualche bestia. Sebbene possa essere scolpito con uno scudo o un'arma di pietra come una spada, queste scelte estetiche non influenzano le sue capacità in combattimento.\\

Come per la maggior parte dei golem, un golem di pietra non può parlare e non emette altro suono se non quelo della pietra che sfrega sulla pietra quando si muove. Un golem di pietra è alto 2,7 metri e pesa cirda 1.000 kg.\\

\textbf{Costruzione}
Il corpo di un golem di pietra viene scolpito da un unico blocco di pietra dura, come il granito, del peso di almeno 1.500 kg. La pietra deve essere di qualità eccezionale, e costare 5.000 mo.\\


\medskip\index{Mostri - Gorgone}\textbf{Gorgone}

\emph{Grande mostruosità, disallineato}

\textbf{FORZA} +5

\textbf{DESTREZZA} +0

\textbf{COSTITUZIONE} +4

\textbf{INTELLIGENZA} -4

\textbf{SAGGEZZA} +1

\textbf{CARISMA} -2

\textbf{Iniziativa} +0 -- \textbf{Difesa} 22

\textbf{Punti Ferita} 114 (12d10 + 48)

\textbf{Movimento} 12 m

\textbf{Tiri Salvezza}: Tempra +13, Riflessi +6, Volontà +7

\textbf{Competenze} Consapevolezza +4

\textbf{Immunità alle Condizioni} Pietrificato

\textbf{Sensi} scurovisione 18 m

\textbf{Linguaggi} -

\textbf{Sfida} 5 (1.800 PE)

\emph{\textbf{Carica Travolgente.}} Se la gorgone si muove di almeno 6 metri in linea retta verso il bersaglio e lo colpisce con un attacco di incornata durante lo stesso turno, il bersaglio deve riuscire un Tiro Salvezza su Tempra CD 16 o cadere prono. Se il bersaglio è prono, la gorgone può effettuare un attacco di zoccoli contro di lui come azione bonus.

\textbf{Azioni}

\emph{\textbf{Incornata.} Attacco con arma da mischia}: +8 a colpire, portata 1 m, un bersaglio.

\emph{Colpisce:} 18 (2d12 + 5) danni perforanti.

\emph{\textbf{Zoccoli.} Attacco con arma da mischia}: +8 a colpire, portata 1 m, un bersaglio.

\emph{Colpisce:} 16 (2d10 + 5) danni da botta.

\emph{\textbf{Soffio Pietrificante (Ricarica 5-6).}} La gorgone esala un gas pietrificante in un cono di 9 metri. Ogni creatura in quell'area deve riuscire un Tiro Salvezza di Tempra CD 13. Se il Tiro Salvezza fallisce, il bersaglio inizia a trasformarsi in pietra ed è intralciato. Il bersaglio intralciato deve ripetere il Tiro Salvezza al termine del suo prossimo turno. Se lo riesce, l'effetto sul bersaglio ha termine. Se lo fallisce, il bersaglio è pietrificato finché non viene liberato dall'incantesimo \emph{ripristino superiore} o simile magia.

\textbf{Ecologia}\\
Ambiente: Pianure Temperate, Colline Rocciose e Sotterranei\\
Organizzazione: Solitario, coppia, branco (3-4) o mandria (5-12)\\
Tesoro: Nessuno\\
\textbf{Descrizione}\\
Le gorgoni sono creature magiche e irascibili: sebbene a prima vista possano sembrare dei costrutti, sotto le piastre metalliche dall'aspetto artificiale sono fatte di carne e ossa. Come tori aggressivi, sfidano qualsiasi creatura sconosciuta che incontrano, spesso travolgendo il cadavere del loro avversario o frantumando i suoi resti pietrificati finché la creatura non è più riconoscibile. Le femmine sono pericolose quanto i maschi, e i due sessi hanno l'identico aspetto. Una tipica gorgone è alta 1,8 metri e lunga 2,4 metri. Pesa circa 2.000 kg.\\

Le gorgoni ricavano il loro nutrimento consumando minerali, in particolare la pietra delle loro vittime pietrificate, e ogni statua da loro creata viene completamente divorata. Non possono digerire metallo o gemme, così il loro sterco (che assomiglia a polvere grigia dall'odore acre) spesso contiene piccoli cristalli grezzi e pepite d'oro. La loro aggressività verso tutte le altre creature fa sì che nei loro pascoli siano pochi, se non nessuno, i predatori e le prede. Ogni mandria è guidata da un toro dominante; le gorgoni solitarie sono generalmente tori adolescenti allontanati dalla mandria del toro dominante.\\

La loro carne è dura e muscolosa (una volta che viene rimossa l'armatura), e per coloro che la assaggiano è abbastanza nutriente. Molte tribù di giganti della pietra credono che mangiare la carne di gorgone aumenti la loro armatura naturale. Le corna di gorgone polverizzate valgono 250 mo come componente materiale alternativo per gli oggetti magici ed incantesimi che agiscono fulla Forza o Pietra.\\


\medskip\index{Mostri - Grick}\textbf{Grick}

\emph{Media mostruosità, neutrale}

\textbf{FORZA} +2

\textbf{DESTREZZA} +2

\textbf{COSTITUZIONE} +0

\textbf{INTELLIGENZA} -4

\textbf{SAGGEZZA} +2

\textbf{CARISMA} -3

\textbf{Iniziativa} +2 -- \textbf{Difesa} 15

\textbf{Punti Ferita} 27 (6d8)

\textbf{Movimento} 9 m, scalata 9 m

\textbf{Tiri Salvezza}: Tempra +3, Riflessi +3, Volontà +2

\textbf{Resistenza al Danno} da botta, perforante e tagliente di attacchi non magici

\textbf{Sensi} scurovisione 18 m

\textbf{Linguaggi} -

\textbf{Sfida} 2 (450 PE)

\emph{\textbf{Camuffamento di Pietra.}} Il grick ha +1d6 alle prove di Destrezza (Nascondersi) effettuate per nascondersi su terreno roccioso.

\textbf{Azioni}

\emph{\textbf{Multiattacco.}} Il grick effettua un attacco con i suoi tentacoli. Se l'attacco colpisce, il grick può effettuare un attacco di becco contro lo stesso bersaglio.

\emph{\textbf{Tentacoli.} Attacco con arma da mischia}: +4 a colpire, portata 1 m, un bersaglio.

\emph{Colpisce:} 9 (2d6 + 2) danni taglienti.

\emph{\textbf{Becco.} Attacco con arma da mischia}: +4 a colpire,
portata 1 m, un bersaglio.

\emph{Colpisce:} 5 (1d6 + 2) danni perforanti.

\textbf{Ecologia}: \\
Ambiente: Qualsiasi Sotterraneo\\
Organizzazione: Solitario o ammasso (2-5)\\
Tesoro: Accidentale\\

\textbf{Descrizione}
Il vermiforme grick è il terrore delle caverne e dei cunicoli in cui abita, attendendo in agguato nei pressi di tunnel molto trafficati o città sotterranee, per balzare fuori dal buio e catturare le sue prede. È raro che tali prede vengano consumate sul posto. Il grick preferisce portare il cibo fresco nella sua tana, uno stretto cunicolo o sull'alta sporgenza di una caverna, dove può consumarlo con piccoli morsi, in tranquillità.\\
Le origini del grick sono ignote. E anche se ha una rudimentale intelligenza, non ha alcuna società di cui parlare, e la maggior parte delle volte si incontrano singoli esemplari. Nelle occasioni in cui i viaggiatori sfortunati ne incontrano più di uno, i gruppi di grick non sembrano comunicare o lavorare tra loro: ognuno attacca invece obiettivi individuali e si ritira col suo bottino non appena riesce ad abbattere un avversario. Predatori capaci, i grick hanno anche una strana pelle resistente alle armi che li rende particolarmente pericolosi. Molti avventurieri inesperti sono periti sotto l'attacco di un grick semplicemente perché non erano in grado di danneggiare la creatura con le loro armi non magiche. Coloro che hanno familiarità con i grick (soprattutto i Nani, i Morlock e i Trogloditi) sanno che la migliore strategia per affrontarli è quella di ritirarsi e attendere rinforzi più potenti o magici.\\
I grick contano sul loro colore scuro e sulla capacità di scalare i muri per tenersi fuori vista, finché non sono pronti a scattare all'attacco. In più occasioni quando il cibo scarseggia in una determinata regione, i grick si dirigono verso la superficie e vagano per il deserto in cerca di prede, ma questi soggiorni sono quasi sempre per necessità, e alla fine i grick trovano presto entrate a nuove tane sotterranee. Preferiscono le tenebre e la comodità di un "tetto" sopra la testa, evitando il cielo aperto e facendo molto per restare coperto da alberi, nuvole basse o edifici.\\



\medskip\index{Mostri - Grifone}\textbf{Grifone}

\emph{Grande mostruosità, disallineato}

\textbf{FORZA} +4

\textbf{DESTREZZA} +2

\textbf{COSTITUZIONE} +3

\textbf{INTELLIGENZA} -4

\textbf{SAGGEZZA} +1

\textbf{CARISMA} -1

\textbf{Iniziativa} +2 -- \textbf{Difesa} 13

\textbf{Punti Ferita} 59 (7d10 + 21)

\textbf{Movimento} 9 m, volo 24 m

\textbf{Tiri Salvezza}: Tempra +7, Riflessi +6, Volontà +4

\textbf{Competenze} Consapevolezza +5

\textbf{Sensi} scurovisione 18 m

\textbf{Linguaggi} -

\textbf{Sfida} 2 (450 PE)

\emph{\textbf{Vista Affinata.}} Il grifone ha +1d6 nelle prove di Saggezza (Consapevolezza) basate sulla vista.

\textbf{Azioni}

\emph{\textbf{Multiattacco.}} Il grifone effettua due attacchi: uno con il becco e uno con gli artigli.

\emph{\textbf{Artigli.} Attacco con arma da mischia}: +6 a colpire, portata 1 m, un bersaglio.

\emph{Colpisce:} 11 (2d6 + 4) danni taglienti.

\emph{\textbf{Becco.} Attacco con arma da mischia}: +6 a colpire, portata 1 m, un bersaglio.

\emph{Colpisce:} 8 (1d8 + 4) danni perforanti.

\textbf{Ecologia}\\
Ambiente: Colline Temperate\\
Organizzazione: Solitario, coppia o branco (6-10)\\
Tesoro: Accidentale\\
\textbf{Descrizione}\\
I grifoni sono potenti predatori aerei, che piombano dai loro altissimi nidi per afferrare le loro prede con il becco e gli artigli. Aggressivi e territoriali, non sono semplici bestie, bensì combattenti astuti e compagni leali verso coloro che si guadagnano il loro rispetto, combattendo fino alla morte per proteggere i loro amici e i loro simili.\\

Del peso di oltre 250 kg e lungo 2,4 metri, dal becco aguzzo alla coda crestata, il grifone possiede un profilo imponente che è da tempo usato in araldica e in altre iconografie come simbolo di potenza, autorità e giustizia. In realtà, il grifone è meno interessato a concetti astratti e più a cacciare per nutrirsi e difendersi. Sebbene a volte possano essere addestrati o diventino amici per servire da cavalcatura, i grifoni non possiedono un'innata affinità con gli umanoidi, e spesso entrano in sanguinosi conflitti con le razze civilizzate nel tentativo di procurarsi il loro cibo preferito: carne di cavallo. La gente di città può meravigliarsi di fronte allo stile maestoso di un grifone addestrato e alla sua apertura alare di 7 metri, ma quei contadini costretti a condividere il territorio con la sua specie sanno che conviene affrettarsi in casa e mettere al sicuro le loro greggi quando sentono le urla di caccia delle bestie.\\

I grifoni si accoppiano per la vita, e spesso per anni cercano vendetta per l'uccisione del compagno o di un figlio. È stata proprio per questa innata caparbietà e fiera lealtà che li ha portati nell'uso domestico come cavalcature e guardiani di tesori. Nonostante l'insito pericolo, il commercio di grifoni catturati e di uova rubate è fervido, con le uova che valgono fino a 3.500 mo l'una e i giovani vivi fino a 7.000. I personaggi che desiderano un grifone come cavalcatura, però, dovrebbero sapere che comprare o addomesticare con la violenza le creature intelligenti come i grifoni è ritenuto schiavitù dalla maggior parte delle divinità buone, e guadagnarsi la spontanea lealtà di un grifone non è un compito facile. Raggiungere un mutuo accordo (o perfino l'amicizia) è una strada molto più elegante e sicura per assicurarsi un grifone come cavalcatura.\\

Prima che lo si possa cavalcare in combattimento, un grifone deve fare pratica nel portare il peso del suo cavaliere. Per essere ben addestrato, un grifone deve per prima cosa avere un atteggiamento amichevole verso il suo addestratore (con una prova di Addestrare Animali, Diplomazia o Intimidire). Dopodiché, 6 settimane di pratica e una prova riuscita di Addestrare Animali con DC 20 sono sufficienti perché la bestia sia a suo agio con il carico e, per la loro intelligenza, si può ritenere che i grifoni addestrati conoscano tutti i trucchi elencati nella descrizione dell'abilità Addestrare Animali, ed è anche possibile che imparino nuovi comandi, impartendo semplici richieste in Comune.\\

I grifoni possono portare fino a 150 kg come carico leggero, 300 kg come carico medio e 450 kg come carico pesante. Per cavalcare un grifone è necessaria una sella esotica.\\


\medskip\index{Mostri - Grimlock}\textbf{Grimlock}

\emph{Media umanoide (grimlock), neutrale malvagio}

\textbf{FORZA} +3

\textbf{DESTREZZA} +1

\textbf{COSTITUZIONE} +1

\textbf{INTELLIGENZA} -1

\textbf{SAGGEZZA} -1

\textbf{CARISMA} -2

\textbf{Iniziativa} +1 -- \textbf{Difesa} 12

\textbf{Punti Ferita} 11 (2d8 + 2)

\textbf{Movimento} 9 m

\textbf{Tiri Salvezza}: Tempra +3, Riflessi +1, Volontà +0

\textbf{Competenze} Acrobatica +5, Muoversi Silenziosamente / Nascondersi +3, Consapevolezza +3

\textbf{Immunità alle Condizioni} accecato

\textbf{Sensi} vista cieca 9 m o 3 m se assordato (cieco oltre questo raggio) 

\textbf{Linguaggi} Linguaggio delle Profondità

\textbf{Sfida} 1/4 (50 PE)

\emph{\textbf{Camuffamento di Pietra.}} Il grimlock ha +1d6 alle prove di Destrezza (Nascondersi) effettuate per nascondere su terreni rocciosi.

\emph{\textbf{Sensi Ciechi.}} Il grimlock non può usare la vista cieca mentre è assordato e non più fiutare.

\emph{\textbf{Olfatto e Udito Affinati.}} Il grimlock ha +1d6 alle prove di Saggezza (Consapevolezza) basate su udito o olfatto. 

\textbf{Azioni}

\emph{\textbf{Randello d'Osso Appuntito.} Attacco con arma da mischia}:
+5 a colpire, portata 1 m, un bersaglio.

\emph{Colpisce:} 5 (1d4 + 3) danni da botta più 2 (1d4) danni perforanti.

\emph{\textbf{Arco Lungo.} Attacco con arma a Distanza}: +3 a colpire, gittata 45m, un bersaglio.

\emph{Colpisce:} 5 (1d8 + 1) danni perforanti.

\textbf{Ecologia}\\
I Grimlock abitano gli insediamenti abbandonati di altre Razze e sono spesso trovati come Schiavi di altre creature più organizzate, come i Duergar ed Elfi. Si ritiene che si trattino di una propaggine ancora più degenerata dei Morlock, che viaggiano da Sekamina per cacciare i Grimlock per il cibo e considerano la loro carne una delicatezza.\\
\textbf{Descrizione}\\
I Grimlock sono creature umane cieche e selvagge che abitano nel regno delle Darklands di Nar-Voth, dove si organizzano in piccoli gruppi tribali.\\


\medskip\index{Mostri - Guardiano Protettore}\textbf{Guardiano Protettore}

\emph{Grande costrutto, disallineato}

\textbf{FORZA} +4

\textbf{DESTREZZA} -1

\textbf{COSTITUZIONE} +4

\textbf{INTELLIGENZA} -2

\textbf{SAGGEZZA} +0

\textbf{CARISMA} -4

\textbf{Iniziativa} -1 -- \textbf{Difesa} 21

\textbf{Punti Ferita} 142 (15d10 + 60)

\textbf{Movimento} 9 m

\textbf{Tiri Salvezza}: Tempra +6, Riflessi +1, Volontà +2

\textbf{Immunità al Danno} veleno

\textbf{Immunità alle Condizioni} affascinato, avvelenato, paralizzato, affaticamento, spaventato

\textbf{Sensi} scurovisione 18 m, vista cieca 3 m

\textbf{Linguaggi} comprende i comandi forniti in qualsiasi lingua ma non può parlare

\textbf{Sfida} 7 (2.900 PE)

\emph{\textbf{Accumulare Incantesimi.}} Un incantatore che indossi l'amuleto del guardiano protettore può far sì che il guardiano accumuli un incantesimo di Difficoltà 23 o più basso. Per farlo, l'incantatore deve lanciare l'incantesimo sul guardiano. L'incantesimo non ha effetto ma viene accumulato all'interno del guardiano. Quando gli viene comandato di farlo da chi indossa l'amuleto o si presenta una situazione predeterminata dall'incantatore, il guardiano lancia l'incantesimo accumulato con tutti i parametri predisposti dall'incantatore originale, senza bisogno di componenti. Quando l'incantesimo viene lanciato o qualsiasi nuovo incantesimo viene accumulato, tutti gli incantesimi precedentemente accumulati vengono persi.

\emph{\textbf{Natura di Costrutto.}} Il guardiano non ha bisogno di aria, cibo, bevande o sonno.

\emph{\textbf{Rigenerazione.}} Il guardiano protettore recupera 10 punti ferita all'inizio del proprio turno se ne possiede ancora almeno 1.

\emph{\textbf{Vincolato.}} Il guardiano protettore è vincolato magicamente ad un amuleto. Finché il guardiano e l'amuleto sono sullo stesso piano di esistenza, chi indossa l'amuleto può richiamare telepaticamente il guardiano perché lo raggiunga, e il guardiano saprà la distanza e la direzione in cui si trova l'amuleto. Se il guardiano si trova entro 18 metri da chi indossa l'amuleto, metà dei danni subiti da chi lo indossa (arrotondati per difetto) vengono trasferiti al guardiano. Se l'amuleto viene distrutto, il guardiano è inabile finché non viene creato un amuleto di rimpiazzo. L'amuleto del guardiano può essere soggetto ad un attacco diretto qualora non sia indossato o trasportato da nessuno. Ha Difesa 10, 10 punti ferita e immunità ai danni psichici e da veleno. Costruire un amuleto richiede 1 settimana e costa 10.000 mo in componenti.

\textbf{Azioni}

\emph{\textbf{Multiattacco.}} Il golem effettua due attacchi di pugno.

\emph{\textbf{Pugno.} Attacco con arma da mischia}: +7 a colpire,
portata 1 m, un bersaglio.

\emph{Colpisce:} 11 (2d6 + 4) danni da botta.

\textbf{Reazioni}

\emph{\textbf{Scudo.}} Quando una creatura attacca chi indossa l'amuleto del guardiano, il guardiano conferisce un bonus di +2 alla sua Difesa, se entro 1 metro dal suo controllore.

\medskip\index{Mostri - Hobgoblin}\textbf{Hobgoblin}

\emph{Media umanoide (goblinoide), legale malvagio}

\textbf{FORZA} +1

\textbf{DESTREZZA} +1

\textbf{COSTITUZIONE} +1

\textbf{INTELLIGENZA} +0

\textbf{SAGGEZZA} +0

\textbf{CARISMA} -1

\textbf{Iniziativa} +1 -- \textbf{Difesa} 19 (armatura di maglia, scudo)

\textbf{Punti Ferita} 11 (2d8 + 2)

\textbf{Movimento} 9 m

\textbf{Tiri Salvezza}: Tempra +5, Riflessi +2, Volontà +1

\textbf{Sensi} scurovisione 18 m

\textbf{Linguaggi} Comune, Goblin

\textbf{Sfida} 1/2 (100 PE)

\emph{\textbf{+1d6 Marziale.}} Una volta per turno, l'hobgoblin può infliggere 7 (2d6) danni aggiuntivi ad una creatura che colpisce con un attacco con arma, se quella creatura si trova entro 1 metro da un alleato dell'hobgoblin che non sia inabile.

\textbf{Azioni}

\emph{\textbf{Spada Lunga.} Attacco con arma da mischia}: +3 a colpire, portata 1 m, un bersaglio.

\emph{Colpisce:} 5 (1d8 + 1) danni taglienti o 6 (1d10 + 1) danni taglienti se usata con due mani.

\emph{\textbf{Arco Lungo.} Attacco con arma a Distanza}: +3 a colpire, gittata 45m, un bersaglio.

\emph{Colpisce:} 5 (1d8 + 1) danni perforanti.

\textbf{Ecologia}\\
Ambiente: Colline Temperate\\
Organizzazione: Gruppo (4-9), banda da guerra (10-24) o tribù (25+ più 50\% non combattenti, 1 sergente di 3° livello per 20 adulti, 1 o 2 luogotenenti di 4° o 5° livello, 1 capo di 6°-8° livello, 6-12 Leopardi e 1-4 Ogre o 1-2 Troll)\\
Tesoro: Equipaggiamento da PNG (Corazza di Cuoio Borchiato, Scudo Leggero di Metallo, Spada Lunga, Arco Lungo con 20 Frecce, altro tesoro)\\
\textbf{Descrizione}\\
Gli Hobgoblin sono militaristi e prolifici, una combinazione che li rende molto pericolosi in alcune regioni. Procreano rapidamente, rimpiazzando i membri caduti con nuovi soldati mantenendo costante il loro numero indipendentemente dalle sorti della guerra. Generalmente basta poco perché dichiarino guerra, ma nella maggior parte dei casi il motivo è per catturare nuovi Schiavi: la vita da Schiavi in un covo di Hobgoblin è brutale e breve, e nuovi Schiavi sono sempre necessari per rimpiazzare quelli che muoiono o che vengono mangiati.\\
Tra tutte le Razze Goblinoidi quella degli Hobgoblin è di gran lunga la più civilizzata.
Vedono i più grandi e solitari Bugbear come strumenti da assoldare e usare dove necessario, di solito per specifiche missioni che richiedono l'omicidio e il furto, e guardano alla specie più piccola dei Goblin con un misto di vergogna e frustrazione. Gli Hobgoblin ammirano la tenacia dei Goblin, sebbene la natura imprevedibile e la passione per il fuoco dei loro minuscoli parenti li rende sgradite aggiunte a tribù o insediamenti Hobbgoblin. Tuttavia, la maggior parte delle tribù Hobgoblin include un piccolo gruppo di Goblin, che normalmente si nascondono negli angoli peggiori dell'insediamento.\\
Molte tribù Hobgoblin uniscono l'amore per la guerra con l'intelletto acuto. La Scienza delle macchine d'assedio, l'Alchimia e le complesse imprese di ingegneria affascinano la maggior parte degli Hobgoblin, e quelli particolarmente dotati vengono trattati da eroi e ottengono sempre delle posizioni di alto rango nella tribù. Gli Schiavi dalle menti raffinate vengono apprezzati, perciò le incursioni nelle città Naniche sono cosa ordinaria.\\
È risaputo che gli Hobgoblin diffidano della Magia e la disprezzano, in particolar modo quella Arcana. I loro Sciamani sono considerati con un misto di paura e rispetto, e vengono normalmente costretti a vivere da soli ai margini del covo della tribù. Non si è mai sentito di Hobgoblin che pratichino la Magia Arcana o, come dicono gli Hobgoblin, la "Magia degli Elfi". Questa è la causa del loro odio per la Magia: gli Hobgoblin odiano gli Elfi.\\
Un Hobgoblin è alto 1 metro e pesa 80 kg.\\



\medskip\index{Mostri - Idra}\textbf{Idra}

\emph{Enorme mostruosità, disallineato}

\textbf{FORZA} +5

\textbf{DESTREZZA} +1

\textbf{COSTITUZIONE} +5

\textbf{INTELLIGENZA} -4

\textbf{SAGGEZZA} +0

\textbf{CARISMA} -2

\textbf{Iniziativa} +1 -- \textbf{Difesa} 19

\textbf{Punti Ferita} 172 (15d12 + 75)

\textbf{Movimento} 9 m, nuoto 9 m

\textbf{Tiri Salvezza}: Tempra +8, Riflessi +7, Volontà +3

\textbf{Competenze} Consapevolezza +6

\textbf{Sensi} scurovisione 18 m

\textbf{Linguaggi} -

\textbf{Sfida} 8 (3.900 PE)

\emph{\textbf{Teste Multiple.}} L'idra ha cinque teste. Finché ha più di una testa, l'idra ha +1d6 ai Tiri Salvezza contro le condizioni accecata, affascinata, assordata, spaventata, stordita o svenuta.

Ogni volta che l'idra subisce 25 o più danni in un singolo turno, una delle sue teste muore. Se tutte le teste muoiono, anche l'idra muore.

Al termine del suo turno, l'idra ricresce due teste per ciascuna delle sue teste uccise dal suo ultimo turno, a meno che non abbia subito danno da fuoco dal suo ultimo turno. L'idra recupera 10 punti ferita per ogni testa ricresciuta in questo modo.

\emph{\textbf{Teste Reattive.}} Per ogni testa posseduta oltre la prima, l'idra riceve una reazione extra che può essere usata solo per compiere attacchi di opportunità.

\emph{\textbf{Trattenere il Fiato.}} L'idra può trattenere il fiato per 1 ora.

\emph{\textbf{Veglia.}} Mentre l'idra dorme, almeno una delle sue teste resta sveglia.

\textbf{Azioni}

\emph{\textbf{Multiattacco.}} L'idra effettua tanti attacchi di morso quante sono le sue teste.

\emph{\textbf{Morso.} Attacco con arma da mischia}: +8 a colpire, portata 3 m, un bersaglio.

\emph{Colpisce:} 10 (1d10 + 5) danni perforanti.

\textbf{Ecologia}\\
Ambiente: Paludi Temperate\\
Organizzazione: Solitario\\
Tesoro: Standard\\
\textbf{Descrizione}\\
L'idra e' un drago a piu' teste, ma stupido.\\


\medskip\index{Mostri - Ippogrifo}\textbf{Ippogrifo}

\emph{Grande mostruosità, disallineato}

\textbf{FORZA} +3

\textbf{DESTREZZA} +1

\textbf{COSTITUZIONE} +1

\textbf{INTELLIGENZA} -4

\textbf{SAGGEZZA} +1

\textbf{CARISMA} -1

\textbf{Iniziativa} +1 -- \textbf{Difesa} 12

\textbf{Punti Ferita} 19 (3d10 + 3)

\textbf{Movimento} 12 m, volo 18 m

\textbf{Tiri Salvezza}: Tempra +5, Riflessi +5, Volontà +2

\textbf{Competenze} Consapevolezza +5

\textbf{Linguaggi} -

\textbf{Sfida} 1 (200 PE)

\emph{\textbf{Vista Affinata.}} L'ippogrifo ha +1d6 nelle prove di Saggezza (Consapevolezza) basate sulla vista.

\textbf{Azioni}

\emph{\textbf{Multiattacco.}} L'ippogrifo effettua due attacchi: uno con il becco e uno con gli artigli.

\emph{\textbf{Artigli.} Attacco con arma da mischia}: +5 a colpire, portata 1 m, un bersaglio.

\emph{Colpisce:} 10 (2d6 + 3) danni taglienti.

\emph{\textbf{Becco.} Attacco con arma da mischia}: +5 a colpire,
portata 1 m, un bersaglio.

\emph{Colpisce:} 8 (1d10 + 3) danni perforanti.

\textbf{Ecologia}\\
Ambiente: Colline Temperate o Pianure\\
Organizzazione: Solitario, coppia o stormo (7-12)\\
Tesoro: Nessuno\\
\textbf{Descrizione}\\
l'ippogrifo ha le ali, le zampe anteriori e la testa di un grande rapace e la coda e il corpo di un magnifico cavallo. Dato che i cavalli sono il cibo preferito dei grifoni, gli studiosi affermano che un mago con senso dell'umorismo tanto tempo fa creò come scherzo questa sfortunata fusione tra un cavallo e un falco.\\

Le piume dell'ippogrifo hanno una colorazione simile a quelle di un falco o di un'aquila; tuttavia, alcuni allevatori sono riusciti a produrre degli esemplari con piume completamente bianche o color carbone. Il torso di un ippogrifo e le estremità posteriori sono molto spesso di colore baio, nocciola o grigio, con alcuni manti che mostrano colorazioni pezzate o anche palomino. Un ippogrifo è lungo 3,3 metri e pesa fino a 680 kg.\\

I territoriali ippogrifi proteggono ferocemente il loro dominio. Gli ippogrifi devono anche sorvegliare i cieli a causa degli altri predatori, dato che sono il cibo preferito di grifoni, viverne e giovani draghi. Gli ippogrifi nidificano nelle vaste praterie erbose, aspre colline e f luenti praterie. Ippogrifi eccezionalmente resistenti stabiliscono le loro dimore all'interno di nicchie o mura di canyon, da cui setacciano i deserti rocciosi alla ricerca di coyote, cervi e a volte umanoidi. Gli ippogrifi preferiscono i mammiferi, tuttavia brucano erba dopo qualsiasi pasto di carne per aiutare la digestione. Queste loro abitudini dietetiche possono risultare pericolose sia per gli allevatori che per le loro mandrie, così spesso le comunità di allevatori mettono delle ricompense su di loro. Le vittime di queste partite di caccia vengono spesso imbalsamate, e di frequente degli ippogrifi imbalsamati decorano le taverne di frontiera e gli avamposti sperduti.\\

Di gran lunga più facili da addestrare rispetto ai grifoni e intelligenti quanto i cavalli, gli ippogrifi vengono addestrati come animali da monta da alcune compagnie scelte di soldati a cavallo, che pattugliano i cieli e piombano addosso ai nemici inconsapevoli. Sebbene siano bestie magiche, se catturati da giovani gli ippogrifi possono venire addestrati grazie a Addestrare Animali come fossero degli animali. Un ippogrifo adulto è molto più difficile da addestrare, e per farlo bisogna seguire le normali regole per l'addestramento delle bestie magiche utilizzando tale abilità. Una sella per ippogrifo deve venire fatta in modo tale da non intralciare i movimenti delle ali della creatura; queste selle sono sempre selle esotiche.\\

Gli ippogrifi sono ovipari: come regola generale, il nido di un ippogrifo contiene solo un uovo alla volta. L'uovo di ippogrifo vale 200 mo, ma un giovane ippogrifo in salute vale 500 mo. Un ippogrifo completamente addestrato come cavalcatura può veder salire il proprio valore fino a 5000 mo o anche di più. Un ippogrifo può trasportare 90 kg come carico leggero, 180 kg come carico medio e 270 kg come carico pesante.\\


\medskip\index{Mostri - Kraken}\textbf{Kraken}

\emph{Mastodontica mostruosità (titano), caotico malvagio}

\textbf{FORZA} +10

\textbf{DESTREZZA} +0

\textbf{COSTITUZIONE} +7

\textbf{INTELLIGENZA} +6

\textbf{SAGGEZZA} +4

\textbf{CARISMA} +5

\textbf{Iniziativa} +6 -- \textbf{Difesa} 30

\textbf{Punti Ferita} 472 (27d20 + 189)

\textbf{Movimento} 6 m, nuoto 18 m

\textbf{Tiri Salvezza}: Tempra +21, Riflessi +12, Volontà +11

\textbf{Immunità al Danno} fulmine, armi +1

\textbf{Immunità alle Condizioni} paralizzato, spaventato

\textbf{Sensi} visione del vero 36 m 

\textbf{Linguaggi} comprende Abissale, Celestiale, Infernale e Druidico ma non può parlare, telepatia 36 m 

\textbf{Sfida} 23 (50.000 PE)

\emph{\textbf{Anfibio.}} Il kraken può respirare aria e acqua.

\emph{\textbf{Libertà di Movimento.}} Il kraken ignora i terreni difficili, e gli effetti magici non possono ridurne la velocità o far sì che diventi intralciato. Può spendere 1 metro di movimento per liberarsi dalle restrizioni non magiche o dall'essere afferrato.

\emph{\textbf{Mostro d'Assedio.}} Il kraken infligge danni doppi agli oggetti e le strutture.

\textbf{Azioni}

\emph{\textbf{Multiattacco.}} Il kraken effettua tre attacchi di tentacolo, ciascuno dei quali può essere rimpiazzato da un uso di Fiondare.

\emph{\textbf{Morso.} Attacco con arma da mischia}: +17 a colpire, portata 1 m, un bersaglio.

\emph{Colpisce:} 23 (3d8 + 10) danni perforanti. Se il bersaglio è una creatura di taglia Grande o inferiore afferrato dal kraken, quella creatura viene inghiottita, e l'afferrare ha termine. Mentre è inghiottita, la creatura è accecata e intralciata, ha copertura totale contro gli attacchi e altri effetti provenienti dall'esterno del kraken, e subisce 42 (12d6) danni da acido all'inizio di ciascun turno del kraken.

Se il kraken subisce 50 o più danni in un singolo turno da una creatura al suo interno, il kraken deve riuscire un Tiro Salvezza di Tempra CD 25 o vomitare tutte le creature inghiottite, che cadono prone in uno spazio entro 3 metri dal kraken. Se il kraken muore, una creatura inghiottita non risulta più intralciata da esso e può fuggire dal cadavere usando 5 metri di movimento, uscendo prona.

\emph{\textbf{Tentacolo.} Attacco con arma da mischia}: +17 a colpire, portata 9 m, un bersaglio.

\emph{Colpisce:} 20 (3d6 + 10) danni da botta, e il bersaglio è afferrato (CD 18 per fuggire). Fino al termine dell'afferrare, il bersaglio è intralciato. Il kraken ha dieci tentacoli, ciascuno dei
quali può afferrare un bersaglio.

\emph{\textbf{Fiondare.}} Un oggetto impugnato o una creatura afferrata dal kraken, di taglia Grande o inferiore viene lanciato di 18 metri in una direzione casuale e gettata prona. Se il bersaglio lanciato colpisce una superficie solida, subisce 3 (1d6) danni da botta per ogni 3 metri percorsi. Se il bersaglio viene lanciato contro un'altra creatura, quella creatura deve riuscire un Tiro Salvezza di Riflessi CD 18 o subire lo stesso danno e cadere prona.

\emph{\textbf{Tempesta di Fulmini.}} Il kraken crea magicamente tre saette di energia, ciascuna delle quali può colpire un bersaglio entro 36 metri e che il kraken possa vedere. Il bersaglio deve effettuare un Tiro Salvezza di Riflessi CD 23, e subire 22 (4d10) danni da fulmine se fallisce il Tiro Salvezza, o la metà se lo riesce.

\textbf{Azioni Aggiuntive}

Il kraken può effettuare 3 Azioni aggiuntive, scelte tra le opzioni seguenti. Può usare solo un'opzione leggendaria alla volta e solo al termine del turno di un'altra creatura. Il kraken recupera le Azioni aggiuntive spese all'inizio del proprio turno.

\textbf{Attacco di Tentacolo o Fiondare.} Il kraken effettua un attacco di tentacolo o usa Fiondare.

\textbf{Nube di Inchiostro (Costa 3 Azioni).} Mentre si trova sott'acqua, il kraken espelle una nube di inchiostro con un raggio di 18 metri. La nube si propaga intorno agli angoli, e quell'area è oscurata pesantemente per tutte le creature tranne il kraken. Ciascuna creatura a parte il kraken che termini il suo turno nell'area deve riuscire un Tiro Salvezza su Tempra 23, subendo 16 (3d10) danni da veleno se fallisce il Tiro Salvezza, o la metà se lo riesce. Una forte corrente disperde lanube, che altrimenti svanisce al termine del prossimo turno  del kraken. \textbf{Tempesta di Fulmini (Costa 2 Azioni).} Il kraken usa Tempesta di Fulmini.

\textbf{Ecologia}\\
Ambiente: Qualsiasi Oceano\\
Organizzazione: Solitario\\
Tesoro: Triplo\\
\textbf{Descrizione}\\
Il leggendario kraken è una delle più grandi paure dei marinai, perché è una creatura della taglia di una balena, può colpire delle profondità senza esser visto, può comandare i venti e le condizioni meteorologiche necessarie alla nave per muoversi, e possiede il crudele intelletto della maggior parte dei più spietati e creativi criminali del mondo. Alcuni credono che i kraken siano una punizione divina, mentre altri li ritengono i veri signori delle profondità, che considerano le razze che respirano aria nient'altro che bestiame.\\

Molte leggende sono sorte in merito al fatto che comprenda il linguaggio druidico.

Un kraken è lungo quasi 30 metri e pesa 2.000 kg.\\


\medskip\index{Mostri - Lamia}\textbf{Lamia}

\emph{Grande mostruosità, caotico malvagio}

\textbf{FORZA} +3

\textbf{DESTREZZA} +1

\textbf{COSTITUZIONE} +2

\textbf{INTELLIGENZA} +2

\textbf{SAGGEZZA} +2

\textbf{CARISMA} +3

\textbf{Iniziativa} +2 -- \textbf{Difesa} 15

\textbf{Punti Ferita} 97 (13d10 + 26)

\textbf{Movimento} 9 m

\textbf{Tiri Salvezza}: Tempra +6, Riflessi +9, Volontà +11

\textbf{Competenze} Muoversi Silenziosamente / Nascondersi +3, Ingannare +7, Percepire Emozioni +4,

\textbf{Sensi} scurovisione 18 m

\textbf{Linguaggi} Abissale, Comune

\textbf{Sfida} 4 (1.100 PE)

\emph{\textbf{Incantesimi Innati.}} La caratteristica da incantatore innato della lamia è il Carisma- La lamia può lanciare in maniera innata i seguenti incantesimi, senza bisogno di componenti materiali:

A volontà: \emph{camuffare sé stesso} (qualsiasi forma umanoide)\emph{,} \emph{immagine maggiore}

3/Giorno ciascuno: \emph{charme su persone, immagine speculare,}

\emph{scrutare, suggestione}

1/Giorno: \emph{restrizione}

\textbf{Azioni}

\emph{\textbf{Multiattacco.}} La lamia effettua due attacchi: uno con
gli artigli e uno con il pugnale o il Tocco Intossicante.

\emph{\textbf{Artigli.} Attacco con arma da mischia}: +5 a colpire, portata 1 m, un bersaglio.

\emph{Colpisce:} 14 (2d10 + 3) danni taglienti.

\emph{\textbf{Pugnale.} Attacco con arma da mischia}: +5 a colpire, portata 1 m, un bersaglio.

\emph{Colpisce:} 5 (1d4 + 3) danni perforanti.

\emph{\textbf{Tocco Intossicante.} Attacco con incantesimo in mischia}: +5 a colpire, portata 1 m, una creatura.

\emph{Colpisce:} Il bersaglio è maledetto per 1 ora da questa magia. Fino al termine della maledizione, il bersaglio ha -1d6 ai Tiri Salvezza su Volontà e a tutte le prove di competenza.

\textbf{Ecologia}\\
Ambiente: Deserti Temperati\\
Organizzazione: Solitario, coppia o setta (3-12)\\
Tesoro: Doppio (Pugnale+1, altro tesoro)\\

\textbf{Descrizione}\\
Eredi piene d'odio di un'antica maledizione, le lamie hanno l'aspetto di donne snelle ed attraenti dalla cintola in su, mentre sotto hanno il corpo di un possente leone. Anche le loro fattezze umanoidi portano tratti distintivi dei felini, i loro occhi sono stretti e ferini e i loro denti somigliano alle zanne dei predatori. Una tipica lamia in piedi è alta 1,8 metri, è lunga 2,4 metri e pesa più di 325 kg.\\

Le lamie sono attratte da torrioni in rovina, città abbandonate e monumenti dimenticati che soddisfano i rozzi canoni estetici di queste letali cacciatrici; specie quelli in zone aride o sterili. Tuttavia, le lamie prediligono i templi decrepiti. Provano gioia nel vedere in rovina i templi di divinità buone e deviano dalla loro strada per mettere in difficoltà questi fiorenti luoghi sacri.\\

Le lamie vedono le femmine più anziane del loro gruppo come capi, madri e sciamane, attaccandosi a loro con fanatica reverenza. Anche se le lamie rifuggono dalla maggior parte delle religioni, vedendole come la fonte della maledizione che le ha costrette in queste forme bestiali, le lamie anziane affermano di udire i sussurri del vento che flagella il deserto e di conoscere i freddi capricci delle stelle, e fanno affidamento su queste sorgenti mistiche per guidare il loro popolo.\\

Le lamie presentate qui sono solo gli esponenti più comuni e meno potenti di questa razza maledetta; altre hanno forme serpentine, volanti e anche più perverse.\\


\medskip\index{Mostri - Lich}\textbf{Lich}

\emph{Media non morto, qualsiasi allineamento malvagio}

\textbf{FORZA} +0

\textbf{DESTREZZA} +3

\textbf{COSTITUZIONE} +3

\textbf{INTELLIGENZA} +5

\textbf{SAGGEZZA} +2

\textbf{CARISMA} +3

\textbf{Iniziativa} +5 -- \textbf{Difesa} 28

\textbf{Punti Ferita} 135 (18d8 + 54)

\textbf{Movimento} 9 m

\textbf{Tiri Salvezza}: Tempra +6, Riflessi +7, Volontà +11

\textbf{Resistenze al Danno} freddo, fulmine, da Vuoto

\textbf{Immunità al Danno} veleno; da botta, perforante e tagliente di attacchi non magici

\textbf{Immunità alle Condizioni} affascinato, avvelenato, paralizzato, affaticamento, spaventato

\textbf{Sensi} visione del vero 36 m

\textbf{Linguaggi} Comune più altre cinque lingue

\textbf{Sfida} 21 (33.000 PE)

\emph{\textbf{Incantesimi.}} Il lich ha CM 18. La sua caratteristica da incantatore è l'Intelligenza, +5 a colpire con attacchi da incantesimo). Il lich ha preparati i seguenti incantesimi:

Trucchetti (a volontà): \emph{mano magica, prestidigitazione, raggio} \emph{di gelo}

Difficoltà 16 (4 slot): \emph{dardo incantato, individuazione del magico,} \emph{onda tonante, scudo}

Difficoltà 19 (3 slot): \emph{freccia acida, immagine speculare,} \emph{individuazione dei pensieri, invisibilità}

Difficoltà 21 (3 slot): \emph{animare morti, controincantesimo, dissolvi} \emph{magie, palla di fuoco}

Difficoltà 23 (3 slot): \emph{inaridire, porta dimensionale}

Difficoltà 26 (3 slot): \emph{nube mortale, scrutare}

Difficoltà 29 (1 slot): \emph{disintegrazione, globo di invulnerabilità}

Difficoltà 31 (1 slot): \emph{dito della morte, spostamento planare}

Difficoltà 34 (1 slot): \emph{dominare mostri, parola del potere stordire}

Difficoltà 36 (1 slot): \emph{parola del potere uccidere}

\emph{\textbf{Natura Non Morta.}} Il lich non ha bisogno di aria, cibo, bevande o sonno.

\emph{\textbf{Resistenza Leggendaria (3/Giorno).}} Se il lich fallisce un Tiro Salvezza, può scegliere invece di riuscirvi.

\emph{\textbf{Resistenza allo Scacciare.}} Il lich ha +1d6 ai Tiri Salvezza contro gli effetti che scacciano i non morti.

\emph{\textbf{Rinvigorimento.}} Se possiede un filatterio, il lich distrutto ottiene un nuovo corpo in 1d10 giorni, recuperando tutti i suoi punti ferita e ritornando in attività. Il nuovo corpo compare entro 1 metro dal filatterio.

\emph{\textbf{Sacrifici di Anime.}} Un lich deve periodicamente nutrire di anime il suo filatterio per sostenere la magia che mantiene il suo corpo e la sua coscienza. Per farlo usa l'incantesimo \emph{imprigionare}. Invece di scegliere una delle normali opzioni dell'incantesimo, il lich lo impiega per intrappolare magicamente il corpo e l'anima del bersaglio all'interno del filatterio. Il filatterio deve trovarsi sullo stesso piano del lich, perché questo incantesimo funzioni. Il filatterio di un lich può contenere solo una creatura alla volta, e \emph{dissolvi magie} lanciato come incantesimo di Difficoltà 36 sul filatterio libera qualsiasi creatura imprigionata al suo interno. Una creatura imprigionata nel filatterio per 24 ore viene consumata e distrutta, dopodiché nulla salvo un intervento divino potrà riportarla in vita.

Un lich che dimentichi o non riesca a mantenere il suo corpo con le anime sacrificate inizia a cascare a pezzi, e potrebbe infine trasformarsi in un semilich.

\textbf{Azioni}

\emph{\textbf{Tocco Paralizzante.} Attacco con incantesimo in mischia}: +12 a colpire, portata 1 m, una creatura.

\emph{Colpisce:} 10 (3d6) danni da freddo. Il bersaglio deve riuscire un Tiro Salvezza di Tempra CD 18 o restare paralizzato per 1 minuto. Il bersaglio può ripetere il Tiro Salvezza al termine di ciascun suo turno, terminando l'effetto su di sé in caso di successo.

\textbf{Azioni Aggiuntive}

Il lich può effettuare 3 Azioni aggiuntive, scelte tra le opzioni seguenti. Può usare solo un'opzione leggendaria alla volta e solo al termine del turno di un'altra creatura. Il lich recupera le Azioni aggiuntive spese all'inizio del proprio turno.

\emph{\textbf{Distruggere Vita (Costa 3 Azioni).}} Ogni creatura ad eccezione dei non morti entro 6 metri dal lich deve effettuare un Tiro Salvezza su Tempra CD 18 contro questa magia, subendo 21 (6d6) danni da Vuoto se fallisce il Tiro Salvezza, o la metà di questi danni se lo riesce.

\emph{\textbf{Sguardo Spaventoso (Costa 2 Azioni).}} Il lich fissa il suo sguardo su di una creatura visibile entro 3 metri da esso. Il bersaglio deve riuscire un Tiro Salvezza di Volontà CD 18 contro questa magia o restare spaventato per 1 minuto. Il bersaglio spaventato può ripetere il Tiro Salvezza al termine di ciascun suo turno, terminando l'effetto su di sé in caso di successo. Se il Tiro Salvezza del bersaglio è riuscito o l'effetto per lui ha termine, il bersaglio è immune allo sguardo del lich per le successive 24 ore.

\emph{\textbf{Tocco Paralizzante (Costa 2 Azioni).}} Il lich usa il suo Tocco Paralizzante.

\emph{\textbf{Trucchetto.}} Il lich lancia un trucchetto.

\textbf{Ecologia}\\
Ambiente: Qualsiasi\\
Organizzazione: Solitario\\
Tesoro: Equipaggiamento da PNG (Anello di Protezione +2, Fascia della Sapienza +2 (Consapevolezza), Stivali della Levitazione, pergamena di Dominare Persone, pergamena di Teletrasporto, pozione di Invisibilità)\\

\textbf{Descrizione}
Poche creature sono più temute dei lich. Apice delle arti Necromantiche, il lich è un incantatore che ha scelto di rinunciare alla vita ed ingannare la morte diventando non morto. Anche se molti di coloro che raggiungono simili vette di potenza farebbero di tutto per raggiungere l'immortalità, l'idea di diventare un lich è aborrita da molte creature. Il processo prevede di estrarre la forza vitale dell'incantatore e imprigionarla in un filatterio preparato in modo speciale; l'incantatore cede la sua vita, ma rimane intrappolato tra la vita e la morte, e fintanto che il suo filatterio rimane intatto può continuare le sue ricerche e il suo lavoro senza temere il passare del tempo.\\



\medskip\index{Mostri - Lucertoloide}\textbf{Lucertoloide}

\emph{Media umanoide (lucertoloide), neutrale}

\textbf{FORZA} +2

\textbf{DESTREZZA} +0

\textbf{COSTITUZIONE} +1

\textbf{INTELLIGENZA} -2

\textbf{SAGGEZZA} +1

\textbf{CARISMA} -2

\textbf{Iniziativa} +0 -- \textbf{Difesa} 16 (armatura naturale, scudo)

\textbf{Punti Ferita} 22 (4d8 + 4)

\textbf{Movimento} 9 m, nuoto 9 m

\textbf{Tiri Salvezza}: Tempra +4, Riflessi +0, Volontà +0

\textbf{Competenze} Muoversi Silenziosamente / Nascondersi +4, Consapevolezza +3, Sopravvivenza +5

\textbf{Linguaggi} Draconico

\textbf{Sfida} 1/2 (100 PE)

\emph{\textbf{Trattenere il Fiato.}} Il lucertoloide può trattenere il fiato per 15 minuti.

\textbf{Azioni}

\emph{\textbf{Multiattacco.}} Il lucertoloide effettua due attacchi in mischia, ciascuno con un'arma diversa.

\emph{\textbf{Giavellotto.} Attacco con arma da mischia o a Distanza}: +4 a colpire, portata 1 m o gittata 9m, un bersaglio. \emph{Colpisce:} 5 (1d6 + 2) danni perforanti.

\emph{\textbf{Morso.} Attacco con arma da mischia}: +4 a colpire, portata 1 m, un bersaglio.

\emph{Colpisce:} 5 (1d6 + 2) danni perforanti.

\emph{\textbf{Randello Pesante.} Attacco con arma da mischia}: +4 a colpire, portata 1 m, un bersaglio.

\emph{Colpisce:} 5 (1d6 + 2) danni da botta.

\emph{\textbf{Scudo Appuntito.} Attacco con arma da mischia}: +4 a colpire, portata 1 m, un bersaglio.

\emph{Colpisce:} 5 (1d6 + 2) danni perforanti.

\textbf{Ecologia}\\
Ambiente: paludi temperate\\
Organizzazione: solitario, coppia, banda (3-12) o tribù (13-60)\\
Tesoro: Equipaggiamento da PNG (Scudo Pesante di Legno, Morning Star, 3 Giavellotti)\\

\textbf{Descrizione}\\
I lucertoloidi sono rettili predatori orgogliosi e potenti che fanno le loro case comuni in sparuti villaggi nei recessi di paludi e acquitrini. Privi di interesse verso la colonizzazione delle terre aride e soddisfatti delle loro semplici armi e dei rituali che li hanno serviti bene per millenni, i lucertoloidi sono visti da molte delle altre razze come selvaggi retrogradi, ma all'interno delle loro isolate comunità sono in realtà un popolo vitale ricco di tradizioni e con una storia orale che risale a prima che l'uomo camminasse in posizione eretta.\\

La maggior parte dei lucertoloidi è alta dagli 1,8 ai 2,1 metri e pesa dai 100 ai 125 kg, ed ha i possenti muscoli coperti da scaglie grigie, verdi o marroni. Alcune razze hanno piccole creste dorsali o collari dai colori brillanti, e tutte nuotano bene spostandosi con rapidi movimenti della loro possente coda lunga 1,2 metri. Anche se sono pienamente a loro agio in acqua, trattengono il fiato e tornano alle loro abitazioni poste su colline artificiali per riprodursi e dormire. Poiché il loro sangue da rettile li rende lenti al freddo, molti lucertoloidi cacciano e lavorano durante il giorno e si ritirano nelle loro dimore di notte per rannicchiarsi con gli altri della loro tribù a condividere il calore di grandi fuochi di torba.\\

Anche se generalmente sono neutrali, il comportamento scostante dei lucertoloidi, il loro strenuo rifiuto dei "doni" della civilizzazione, e la leggendaria ferocia in battaglia li fa mal giudicare dalla maggioranza degli umanoidi. Questi tratti derivano da buone ragioni, tuttavia, poiché il loro basso tasso di riproduzione non ha eguali tra gli umanoidi a sangue caldo, e se le tribù non difendessero i loro territori paludosi fino all'ultimo respiro si troverebbero presto sopraffatte da orde di mammiferi. Per quanto riguarda la loro propensione a mangiare i corpi dei morti sia amici che nemici, i pratici lucertoloidi sono lesti a sottolineare che la vita è dura nella palude, e nulla deve andare sprecato.\\

I lucertoloidi presentati qui vivono in ambienti paludosi. Le tribù lucertoloidi possono vivere altrettanto bene in altri ambienti, ma come velocità ottengono Scalare 5 metri al posto di Nuotare.\\



\subsection{Mannari}

\medskip\index{Mostri - Cinghiale Mannaro}\textbf{Cinghiale Mannaro}

\emph{Media umanoide (umano, mutaforma), neutrale malvagio}

\textbf{FORZA} +3

\textbf{DESTREZZA} +0

\textbf{COSTITUZIONE} +2

\textbf{INTELLIGENZA} +0

\textbf{SAGGEZZA} +0

\textbf{CARISMA} -1

\textbf{Iniziativa} +0 -- \textbf{Difesa} 12 in forma umanoide, 13 in forma di cinghiale o ibrida

\textbf{Punti Ferita} 78 (12d8 + 24)

\textbf{Movimento} 9 m (12 m in forma di cinghiale)

\textbf{Tiri Salvezza}: Tempra +7, Riflessi +1, Volontà +4

\textbf{Competenze} Consapevolezza +2

\textbf{Immunità al Danno} da botta, perforante e tagliente di attacchi non magici o che non siano argentati 

\textbf{Linguaggi} Comune (non può parlare in forma di cinghiale)

\textbf{Sfida} 4 (1.100 PE)

\emph{\textbf{Carica (Solo Forma di Cinghiale o Ibrida).}} Se il cinghiale mannaro si muove in linea retta di almeno 5 metri verso un bersaglio e poi lo colpisce con le zanne durante lo stesso turno, il bersaglio subisce 7 (2d6) danni taglienti aggiuntivi. Se il bersaglio è una creatura, deve riuscire un Tiro Salvezza di Tempra CD 13 o cadere prono.

\emph{\textbf{Implacabile (Ricarica dopo un 1 ora).}} Se il cinghiale mannaro subisce 14 danni o meno che lo ridurrebbero a 0 punti ferita, scende invece a 1 punto ferita.

\emph{\textbf{Mutaforma.}} Il cinghiale mannaro può usare la sua azione per trasformarsi in un ibrido cinghiale-umanoide o in un cinghiale, o per tornare alla sua vera forma, che è umanoide. Le sue statistiche, a parte la Difesa, sono le stesse in tutte le forme. Qualsiasi equipaggiamento stia indossando o trasportando non viene trasformato. Alla morte ritorna alla sua vera forma.

\textbf{Azioni}

\emph{\textbf{Multiattacco (Solo in Forma Umanoide o Ibrida).}} Il cinghiale mannaro effettua due attacchi, di cui solo uno può essere con le zanne.

\emph{\textbf{Maglio (Soltanto in Forma Umanoide o Ibrida).} Attacco con arma da mischia}: +5 a colpire, portata 1 m, un bersaglio. \emph{Colpisce:} 10 (2d6 + 3) danni da botta.

\emph{\textbf{Zanne (Soltanto in Forma di Cinghiale o Ibrida).} Attacco con arma da mischia}: +5 a colpire, portata 1 m, un bersaglio. \emph{Colpisce:} 10 (2d6 + 3) danni taglienti. Se il bersaglio è un umanoide, deve riuscire un Tiro Salvezza di Tempra CD 12 o venire maledetto dalla licantropia del cinghiale mannaro.

\textbf{Ecologia}\\
Ambiente: Qualsiasi Foresta o Pianura\\
Organizzazione: Solitario, coppia, famiglia (3-8) o truppa (3-8 più 1-4 Cinghiali)\\
Tesoro: Equipaggiamento da PNG (Armatura di Cuoio Borchiato, 2 Pugnali, altro tesoro)\\
\textbf{Descrizione}\\
Nella loro forma umanoide, i cinghiali mannari tendono a essere tozzi, con nasi all'insù, pelo ispido e incisivi prominenti. Hanno capelli rossi, castani o neri ma alcuni sono anche biondi, canuti o calvi. Hanno di norma peluria sul labbro superiore, e i maschi di solito non riescono a far crescere la barba. Poiché sono testardi e aggressivi, hanno piccole comunità di loro simili e non si mischiano ai non licantropi: di solito vivono in piccole fattorie dall'aspetto assolutamente normale. Tendono ad avere grandi famiglie e molti figli.\\



\medskip\index{Mostri - Lupo Mannaro}\textbf{Lupo Mannaro}

\emph{Media umanoide (umano, mutaforma), caotico malvagio}

\textbf{FORZA} +2

\textbf{DESTREZZA} +1

\textbf{COSTITUZIONE} +2

\textbf{INTELLIGENZA} +0

\textbf{SAGGEZZA} +0

\textbf{CARISMA} +0

\textbf{Iniziativa} +1 -- \textbf{Difesa} 13 in forma umanoide, 14 in forma di lupo o ibrida

\textbf{Punti Ferita} 58 (9d8 + 18)

\textbf{Movimento} 9 m (12 m in forma di lupo)

\textbf{Tiri Salvezza}: Tempra +5, Riflessi +1, Volontà +2

\textbf{Competenze} Muoversi Silenziosamente / Nascondersi +3, Consapevolezza +4

\textbf{Immunità al Danno} da botta, perforante e tagliente di attacchi non magici o che non siano argentati 

\textbf{Linguaggi} Comune (non può parlare in forma di lupo)

\textbf{Sfida} 3 (700 PE)

\emph{\textbf{Mutaforma.}} Il lupo mannaro può usare la sua azione per trasformarsi in un ibrido lupo-umanoide o in un lupo, o per tornare alla sua vera forma, che è umanoide. Le sue statistiche, a parte la Difesa, sono le stesse in tutte le forme. Qualsiasi equipaggiamento stia indossando o trasportando non viene trasformato. Alla morte ritorna alla sua vera forma.

\emph{\textbf{Udito e Olfatto Affinato.}} Il lupo mannaro ha +1d6 nelle prove di Saggezza (Consapevolezza) basate su udito o olfatto.

\textbf{Azioni}

\emph{\textbf{Multiattacco (Soltanto in Forma Umanoide o Ibrida).}} Il lupo mannaro effettua due attacchi: uno con il morso e uno con gli artigli o la lancia.

\emph{\textbf{Artigli (Soltanto in Forma Ibrida).} Attacco con arma da mischia}: +4 a colpire, portata 1 m, una creatura. \emph{Colpisce:} 7 (2d4 + 2) danni taglienti.

\emph{\textbf{Lancia (Soltanto in Forma Umanoide).} Attacco con arma da mischia o a Distanza}: +4 a colpire, portata 1 m o gittata 6m, una creatura.

\emph{Colpisce:} 5 (1d6 + 2) danni perforanti o 6 (1d8 + 2) danni perforanti se usata con due mani in un attacco di mischia.

\emph{\textbf{Morso (Soltanto in Forma di Lupo o Ibrida).} Attacco con arma da mischia}: +4 a colpire, portata 1 m, un bersaglio.

\emph{Colpisce:} 6 (1d8 + 2) danni perforanti. Se il bersaglio è un umanoide, deve riuscire un Tiro Salvezza di Tempra CD 12 o venir maledetto dalla licantropia del lupo mannaro.

\textbf{Ecologia}\\
Ambiente: Qualsiasi Terreno\\
Organizzazione: Solitario, coppia o branco (3-6)\\
Tesoro: Equipaggiamento da PNG (Cotta di Maglia, Spada Lunga, Balestra Leggera con 20 Quadrelli, altro tesoro)\\
\textbf{Descrizione}\\
Nella forma umana i lupi mannari somigliano a persone normali, anche se alcuni tendono ad avere un aspetto ferino e capelli ribelli. Sopracciglia che crescono unendosi, dito indice più lungo del medio e strane voglie sul palmo della mano sono tutti segni comunemente accettati che una persona sia in realtà un lupo mannaro. Naturalmente, questi segni rivelatori non sono sempre accurati, perché questi tratti fisici esistono anche nelle persone normali, ma nelle zone dove i lupi mannari sono un problema comune, questi tratti possono essere considerati schiaccianti a prescindere.\\


\medskip\index{Mostri - Orso Mannaro}\textbf{Orso Mannaro}

\emph{Media umanoide (umano, mutaforma), neutrale buono}

\textbf{FORZA} +4

\textbf{DESTREZZA} +0

\textbf{COSTITUZIONE} +3

\textbf{INTELLIGENZA} +0

\textbf{SAGGEZZA} +1

\textbf{CARISMA} +1

\textbf{Iniziativa} +0 -- \textbf{Difesa} 13 in forma umanoide, 14

in forma di orso o ibrida

\textbf{Punti Ferita} 135 (18d8 + 54)

\textbf{Movimento} 9 m (12 m, scalata 9 m in forma di orso o forma ibrida)

\textbf{Tiri Salvezza}:  Tempra +5, Riflessi +6, Volontà +2

\textbf{Competenze} Consapevolezza +7

\textbf{Immunità al Danno} da botta, perforante e tagliente di attacchi non magici o che non siano argentati

\textbf{Linguaggi} Comune (non può parlare in forma di orso)

\textbf{Sfida} 5 (1.800 PE)

\emph{\textbf{Mutaforma.}} L'orso mannaro può usare la sua azione per trasformarsi in un ibrido orso-umanoide o in un orso, o per tornare alla sua vera forma, che è umanoide. Le sue statistiche, a parte la Difesa, sono le stesse in tutte le forme. Qualsiasi equipaggiamento stia indossando o trasportando non viene trasformato. Alla morte ritorna alla sua vera forma.

\emph{\textbf{Olfatto Affinato.}} L'orso mannaro ha +1d6 nelle prove di Saggezza (Consapevolezza) basate sull'olfatto.

\textbf{Azioni}

\emph{\textbf{Multiattacco.}} In forma di orso, l'orso mannaro effettua due attacchi di artiglio. In forma umanoide, effettua due attacchi di ascia bipenne. In forma ibrida, può attaccare come un orso o un umanoide.  

\emph{\textbf{Artiglio (Soltanto in Forma di Orso o Ibrida).} Attacco con arma da mischia}: +7 a colpire, portata 1 m, un bersaglio. \emph{Colpisce:} 13 (2d8 + 2) danni taglienti.

\emph{\textbf{Ascia Bipenne (Soltanto in Forma Umanoide o Ibrida).} Attacco con arma da mischia}: +7 a colpire, portata 1 m, un bersaglio. \emph{Colpisce:} 10 (1d12 + 4) danni taglienti.

\emph{\textbf{Morso (Soltanto in Forma di Orso o Ibrida).} Attacco con arma da mischia}: +7 a colpire, portata 1 m, un bersaglio.

\emph{Colpisce:} 15 (2d10 + 4) danni perforanti. Se il bersaglio è un umanoide, deve riuscire un Tiro Salvezza di Tempra CD 14 o venir maledetto dalla licantropia dell'orso mannaro.


\textbf{Ecologia}\\
Ambiente: Qualsiasi Foresta\\
Organizzazione: Solitario, coppia, famiglia (3-6) o truppa (3-6 più 1-4 orsi Neri o Grigi)\\
Tesoro: Equipaggiamento da PNG (Giaco di Maglia, Ascia da Battaglia Perfetta, 2 Asce da Lancio Perfette, altro tesoro)\\
\textbf{Descrizione}\\
Nelle loro forme umanoidi, gli orsi mannari tendono a essere muscolosi e con spalle larghe, tratti aspri e occhi scuri. Hanno capelli rossi, castani o neri e sembrano abituati a una vita di duro lavoro. Anche se i più benigni fra i licantropi, sono evitati dalla maggior parte delle persone normali, che temono la loro trasformazione animalesca. Per la maggior parte vivono in zone boschive isolate o in piccole unità familiari della loro stessa specie. Evitano di affrontare gli stranieri, ma non esitano se devono scacciare umanoidi malvagi dai loro territori.\\

\medskip\index{Mostri - Ratto Mannar}\textbf{Ratto Mannaro}

\emph{Media umanoide (umano, mutaforma), legale malvagio}

\textbf{FORZA} +0

\textbf{DESTREZZA} +2

\textbf{COSTITUZIONE} +1

\textbf{INTELLIGENZA} +0

\textbf{SAGGEZZA} +0

\textbf{CARISMA} -1

\textbf{Iniziativa} +2 -- \textbf{Difesa} 13

\textbf{Punti Ferita} 33 (6d8 + 6)

\textbf{Movimento} 9 m

\textbf{Tiri Salvezza}: Tempra +2, Riflessi +5, Volontà +3

\textbf{Competenze} Muoversi Silenziosamente / Nascondersi +4, Consapevolezza +2

\textbf{Immunità al Danno} da botta, perforante e tagliente di attacchi non magici o che non siano argentati

\textbf{Sensi} scurovisione 18 m (solo in forma di ratto)

\textbf{Linguaggi} Comune (non può parlare in forma di ratto)

\textbf{Sfida} 2 (450 PE)

\emph{\textbf{Mutaforma.}} Il ratto mannaro può usare la sua azione per trasformarsi in un ibrido ratto-umanoide o in un ratto, o per tornare alla sua vera forma, che è umanoide. Le sue statistiche, a parte la Difesa, sono le stesse in tutte le forme. Qualsiasi equipaggiamento stia indossando o trasportando non viene trasformato. Alla morte ritorna alla sua vera forma.

\emph{\textbf{Olfatto Affinato.}} Il ratto mannaro ha +1d6 nelle prove di Saggezza (Consapevolezza) basate sull'olfatto.

\textbf{Azioni}

\emph{\textbf{Multiattacco (Solo in Forma Umanoide o Ibrida).}} Il ratto mannaro effettua due attacchi, di cui solo uno può essere con il morso.

\emph{\textbf{Spada Corta (Soltanto in Forma Umanoide o Ibrida).} Attacco con arma da mischia}: +4 a colpire, portata 1 m, un bersaglio. \emph{Colpisce:} 5 (1d6 + 2) danni perforanti.

\emph{\textbf{Balestra a mano (Soltanto in Forma Umanoide o Ibrida).} Attacco con arma a Distanza}: +4 a colpire, gittata 9m, un bersaglio.

\emph{Colpisce:} 5 (1d6 + 2) danni perforanti.

\emph{\textbf{Morso (Soltanto in Forma di Ratto o Ibrida).} Attacco con arma da mischia}: +4 a colpire, portata 1 m, un bersaglio.

\emph{Colpisce:} 4 (1d4 + 2) danni perforanti. Se il bersaglio è un umanoide, deve riuscire un Tiro Salvezza di Tempra CD 11 o venir maledetto dalla licantropia del ratto mannaro.

\textbf{Ecologia}\\
Ambiente: Qualsiasi Urbano\\
Organizzazione: Solitario, coppia, branco (5-10) o gilda (11-30 più 5-12 Ratti Crudeli)\\
Tesoro: Equipaggiamento da PNG (Armatura di Cuoio Borchiato Perfetta, Spada Corta, Balestra Leggera con 20 Quadrelli, altro tesoro)\\
\textbf{Descrizione}\\
I ratti mannari naturali sono bassi, asciutti e muscolosi, con occhi attenti e vispi, e hanno movimenti nervosi. I maschi spesso hanno sottili baffi striminziti.\\


\medskip\index{Mostri - Tigre Mannara}\textbf{Tigre Mannara}

\emph{Media umanoide (umano, mutaforma), neutrale}

\textbf{FORZA} +3

\textbf{DESTREZZA} +2

\textbf{COSTITUZIONE} +3

\textbf{INTELLIGENZA} +0

\textbf{SAGGEZZA} +1

\textbf{CARISMA} +0

\textbf{Iniziativa} +2 -- \textbf{Difesa} 14

\textbf{Punti Ferita} 120 (16d8 + 48)

\textbf{Movimento} 9 m (12 m in forma di tigre)

\textbf{Tiri Salvezza}: Tempra +2, Riflessi +7, Volontà +4

\textbf{Competenze} Muoversi Silenziosamente / Nascondersi +4, Consapevolezza +5

\textbf{Immunità al Danno} da botta, perforante e tagliente di attacchi non magici che non siano argentati

\textbf{Sensi} scurovisione 18 m

\textbf{Linguaggi} Comune (non può parlare in forma di tigre)

\textbf{Sfida} 4 (1.1100 PE)

\emph{\textbf{Balzo.}} Se la tigre mannara si muove di almeno 5 metri in linea retta verso una creatura e la colpisce con un attacco di artiglio durante lo stesso turno, il bersaglio deve riuscire un Tiro Salvezza su Tempra CD 14 o cadere prono. Se il bersaglio è prono, la tigre mannara può effettuare un attacco di morso contro di esso come azione bonus.

\emph{\textbf{Mutaforma.}} La tigre mannara può usare la sua azione per trasformarsi in un ibrido tigre-umanoide o in una tigre, o per tornare alla sua vera forma, che è umanoide. Le sue statistiche, a parte la Difesa, sono le stesse in tutte le forme. Qualsiasi equipaggiamento stia indossando o trasportando non viene trasformato. Alla morte ritorna alla sua vera forma.

\emph{\textbf{Olfatto e Udito Affinato.}} La tigre mannara ha +1d6 nelle prove di Saggezza (Consapevolezza) basate su olfatto e udito.

\textbf{Azioni}

\emph{\textbf{Multiattacco (Solo in Forma Umanoide o Ibrida).}} In forma umanoide, la tigre mannara effettua due attacchi di scimitarra o due attacchi di arco lungo. In forma ibrida, può attaccare come un umanoide o effettuare due attacchi di artiglio.

\emph{\textbf{Artiglio (Soltanto in Forma di Tigre o Ibrida).} Attacco con arma da mischia}: +5 a colpire, portata 1 m, un bersaglio. \emph{Colpisce:} 7 (1d8 + 3) danni taglienti.

\emph{\textbf{Morso (Soltanto in Forma di Tigre o Ibrida).} Attacco con arma da mischia}: +5 a colpire, portata 1 m, un bersaglio.

\emph{Colpisce:} 8 (1d10 + 3) danni perforanti. Se il bersaglio è un umanoide, deve riuscire un Tiro Salvezza di Tempra CD 13 o venir maledetto dalla licantropia della tigre mannara.

\emph{\textbf{Scimitarra (Soltanto in Forma Umanoide o Ibrida).} Attacco con arma da mischia}: +5 a colpire, portata 1 m, un bersaglio. \emph{Colpisce:} 6 (1d6 + 3) danni taglienti.

\emph{\textbf{Arco Lungo (Soltanto in Forma Umanoide o Ibrida).} Attacco con arma a Distanza}: +4 a colpire, gittata 45m, un bersaglio.

\emph{Colpisce:} 6 (1d8 + 2) danni perforanti.

\textbf{Ecologia}
Ambiente: Qualsiasi Pianura o Palude\\
Organizzazione: Solitario o coppia\\
Tesoro: Equipaggiamento da PNG (Armatura di Cuoio Borchiato, Spada Corta, 2 Pugnali, altro tesoro)\\
\textbf{Descrizione}\\
Le tigri mannare in forma umanoide hanno grandi occhi, nasi allungati, zigomi sporgenti e capelli castani o rossi, oppure bianchi, neri o grigio-blu. I loro movimenti sono attenti e aggraziati, e chi li guarda potrebbe scambiarli per un ottimo tagliaborse, un danzatore aggraziato o un'abile cortigiana.\\


\medskip\index{Mostri - Manticora}\textbf{Manticora}

\emph{Grande mostruosità, legale malvagio}

\textbf{FORZA} +3

\textbf{DESTREZZA} +3

\textbf{COSTITUZIONE} +3

\textbf{INTELLIGENZA} -2

\textbf{SAGGEZZA} +1

\textbf{CARISMA} -1

\textbf{Iniziativa} +3 -- \textbf{Difesa} 16

\textbf{Punti Ferita} 68 (8d10 + 24)

\textbf{Movimento} 9 m, volo 15 m

\textbf{Tiri Salvezza}: Tempra +9, Riflessi +7, Volontà +3

\textbf{Sensi} scurovisione 18 m

\textbf{Linguaggi} Comune

\textbf{Sfida} 3 (700 PE)

\emph{\textbf{Ricrescere Spine della Coda.}} La manticora possiede ventiquattro spine nella coda. Le spine usate ricrescono all'alba.

\textbf{Azioni}

\emph{\textbf{Multiattacco.}} La manticora effettua tre attacchi: uno con il morso e due con gli artigli o tre con le spine della coda.

\emph{\textbf{Artiglio.} Attacco con arma da mischia}: +5 a colpire, portata 1 m, un bersaglio.

\emph{Colpisce:} 6 (1d6 + 3) danni taglienti.

\emph{\textbf{Morso.} Attacco con arma da mischia}: +5 a colpire, portata 1 m, un bersaglio.

\emph{Colpisce:} 7 (1d8 + 3) danni perforanti.

\emph{\textbf{Spine della Coda.} Attacco con arma a Distanza}: +5 a colpire, gittata 30m, un bersaglio.

\emph{Colpisce:} 7 (1d8 + 3) danni perforanti.

\textbf{Ecologia}
Ambiente: Colline e Paludi Calde\\
Organizzazione: Solitario, coppia o branco (3-6)\\
Tesoro: Standard\\

\textbf{Descrizione}\\
Le manticore sono feroci predatori che controllano vaste aree in cerca di carne fresca. Una tipica manticora è lunga circa 3 metri e pesa circa 500 kg. Alcune hanno la testa simile a quella di un umano, in genere barbuto. Maschi e femmine sono molto simili.\\

Le manticore mangiano qualsiasi tipo di carne, anche quella delle carogne, ma preferiscono quella umana e raramente si lasciano sfuggire un'occasione di gustare questa delizia. Sono abbastanza furbe e sociali da stringere patti con umanoidi malvagi per formare alleanze o da costringerli ad offre tributi, e molte creature potenti le incaricano di sorvegliare o controllare un posto o una zona. Prediligono fare le loro tane in posti alti, come le sommità delle colline e le caverne tra le rupi.\\

Anche se le manticore sono simili a delle creazioni magiche, sono da tempo annoverate tra le specie naturali. Curiosamente, le manticore sembrano stranamente feconde e possono incrociarsi con numerose altre specie dalla forma simile, inclusi Leoni, Tigri, Lamie, Sfingi e Chimere.\\


\medskip\index{Mostri - Manto Assassino}\textbf{Manto Assassino}

\emph{Grande aberrazione, caotico neutrale}

\textbf{FORZA} +3

\textbf{DESTREZZA} +2

\textbf{COSTITUZIONE} +1

\textbf{INTELLIGENZA} +1

\textbf{SAGGEZZA} +1

\textbf{CARISMA} +2

\textbf{Iniziativa} +2 -- \textbf{Difesa} 18

\textbf{Punti Ferita} 78 (12d10 + 12)

\textbf{Movimento} 3 m, volo 12 m

\textbf{Tiri Salvezza}: Tempra +6, Riflessi +5, Volontà +7

\textbf{Competenze} Muoversi Silenziosamente / Nascondersi +5

\textbf{Sensi} scurovisione 18 m

\textbf{Linguaggi} Linguaggio delle Profondità

\textbf{Sfida} 8 (3.900 PE)

\emph{\textbf{Falso Aspetto.}} Mentre il manto assassino resta immobile senza esporre la parte inferiore del corpo, è indistinguibile da un manto di pelle nera.

\emph{\textbf{Sensibilità alla Luce}}. Mentre è alla luce del sole, il manto assassino ha -1d6 ai tiri per colpire, oltre che alle prove di Saggezza (Consapevolezza) basate sulla vista.

\emph{\textbf{Trasferimento di Danno.}} Mentre è appiccicato ad una creatura, il manto assassino subisce solo la metà dei danni che gli sono inferti (arrotondare per difetto), e la creatura vittima del manto assassino subisce l'altra metà.

\textbf{Azioni}

\emph{\textbf{Multiattacco.}} Il manto assassino effettua due attacchi:
uno con il morso e uno con la coda.

\emph{\textbf{Morso.} Attacco con arma da mischia}: +6 a colpire, portata 1 m, una creatura.

\emph{Colpisce:} 10 (2d6 + 3) danni perforanti, e se il bersaglio è di taglia Grande o inferiore, il manto assassino vi si appiccica. Se il manto assassino ha +1d6 contro il bersaglio, si appiccica alla sua testa e il bersaglio è accecato e impossibilitato a respirare finché il manto assassino vi rimane appiccicato. Mentre appiccicato il manto assassino può effettuare questo attacco solo   contro il bersaglio e ha +1d6 al tiro per colpire. Il manto   assassino può staccarsi spendendo 1 metro di movimento. Una   creatura, compreso il bersaglio, può effettuare la sua azione per   staccare il manto assassino riuscendo una prova di Forza CD 16. 


\emph{\textbf{Coda.} Attacco con arma da mischia}: +6 a colpire, portata 3 m, una creatura.

\emph{Colpisce:} 7 (1d8 + 3) danni taglienti.

\emph{\textbf{Apparizioni (Ricarica dopo un 1 ora).}} Qualora non si trovi sotto luce intensa, il manto assassino crea tre duplicati illusori di sé stesso, che si muovono assieme ad esso e ne imitano le azioni, scambiandosi di posizione per rendere impossibile capire quale sia il reale manto assassino. Se l'originale si trova in un'area di luce intensa, i duplicati svaniscono.

Ogniqualvolta una creatura prenda a bersaglio il manto assassino con un attacco o un incantesimo nocivo mentre sono ancora presenti dei duplicati, quella creatura determina casualmente se prende a bersaglio il manto assassino o uno dei duplicati. Una creatura che non possa vedere o che si affida a sensi diversi dalla vista ignora questo effetto magico.

Un duplicato possiede la Difesa e usa i Tiri Salvezza del manto assassino. Se un attacco colpisce un duplicato, o se un duplicato fallisce un Tiro Salvezza contro un effetto che infligge danni, svanisce.

\emph{\textbf{Gemito.}} Ogni creatura entro 18 metri dal manto assassino, che possa udire il suo gemito e che non sia un'aberrazione, deve riuscire un Tiro Salvezza su Volontà CD 13 o essere spaventata fino al termine del prossimo turno del manto assassino. Se il Tiro Salvezza della creatura
riesce, la creatura è immune al gemito del manto assassino per le successive 24 ore.

\textbf{Ecologia}
Ambiente: Sotterranei\\
Organizzazione: Solitario, coppia, schiera (3-6) o stormo (7-12)\\
Tesoro: Standard\\
\textbf{Descrizione}\\
Simili a mante volanti orribilmente malvagie, i manti assassini sono creature misteriose e paranoiche. Un tipico esemplare ha un'apertura alare di 2,4 metri e pesa 50 kg.\\

Le loro motivazioni sono misteriose e confuse, e diffidano perfino dei loro simili. La strana forma permette loro di essere scambiati per mantelli, arazzi o altri oggetti comuni, e alcune storie narrano di manti assassini che si alleano con altre creature, facendosi trasportare sulla loro schiena e contribuendo alla protezione dei loro alleati per ragioni imperscrutabili. Alcuni esemplari sono sacerdoti di antiche divinità, al comando di culti di manti assassini e Skum intenti a celebrare orribili riti dagli scopi sinistri.\\


\medskip\index{Mostri - Mantoscuro}\textbf{Mantoscuro}

\emph{Piccola mostruosità, disallineato}

\textbf{FORZA} +3

\textbf{DESTREZZA} +1

\textbf{COSTITUZIONE} +1

\textbf{INTELLIGENZA} -4

\textbf{SAGGEZZA} +0

\textbf{CARISMA} -3

\textbf{Iniziativa} +1 -- \textbf{Difesa} 12

\textbf{Punti Ferita} 22 (5d6 + 5)

\textbf{Movimento} 3 m, volo 9 m

\textbf{Tiri Salvezza}: Tempra +5, Riflessi +3, Volontà +0

\textbf{Competenze} Muoversi Silenziosamente / Nascondersi +3

\textbf{Sensi} vista cieca 18 m

\textbf{Linguaggi} -

\textbf{Sfida} 1/2 (100 PE)

\emph{\textbf{Ecolocazione.}} Il mantoscuro non può usare la vista cieca se assordato.

\emph{\textbf{Falso Aspetto.}} Mentre il mantoscuro rimane immobile, è indistinguibile da una formazione rocciosa come una stalattite o una stalagmite.

\textbf{Azioni}

\emph{\textbf{Spaccare.} Attacco con arma da mischia}: +5 a colpire, portata 1 m, una creatura.

\emph{Colpisce:} 6 (1d6 + 3) danni da botta e il mantoscuro si appiccica alla creatura. Se il bersaglio è di taglia Media o inferiore il mantoscuro ha +1d6 al tiro per colpire, si appiccica avvolgendo la testa del bersaglio, che è accecato e impossibilitato a respirare finché il mantoscuro resta appiccicato in questo modo.

Mentre è appiccicato al bersaglio, il mantoscuro non può attaccare nessun'altra creatura salvo il bersaglio, ma ha +1d6 ai suoi tiri per colpire. La velocità del mantoscuro diventa 0 e non può trarre beneficio da nessun bonus alla velocità, muovendosi assieme al bersaglio.

Una creatura può staccare il mantoscuro con un'azione e riuscendo una prova di Forza CD 13. Durante il suo turno, il mantoscuro può staccarsi dal bersaglio da solo usando 1 metro di movimento.

\emph{\textbf{Aura di Oscurità (1/Giorno).}} Un'oscurità magica con 5 metri di raggio si estende dal mantoscuro, muovendosi con esso, e propagandosi oltre gli angoli. L'oscurità permane finché il mantoscuro mantiene la concentrazione, massimo 10 minuti (come se si stesse concentrando su di un incantesimo). La scurovisione non può penetrare questa oscurità, né essa può essere rischiarata da alcuna luce naturale. Se qualsiasi parte dell'oscurità si sovrappone ad un'area di luce generata da un incantesimo di Difficoltà 19 o inferiore, l'incantesimo che sta creando la luce viene dissolto.

\textbf{Ecologia}
Ambiente: Qualsiasi (sotterraneo)\\
Organizzazione: Solitario, coppia o nidiata (3-12)\\
Tesoro: Nessuno\\
\textbf{Descrizione}\\
l'apertura tentacolare di un mantoscuro ha un'ampiezza di poco inferiore agli 1 m; quando è appeso alla volta di una caverna, mascherato da stalattite, la sua lunghezza varia tra i 60 ed i 90 cm. Un esemplare tipico di mantoscuro pesa 20 kg. La testa ed il corpo della creatura sono solitamente del colore del basalto o del granito scuro, ma i suoi tentacoli membranosi possono cambiare colore per adattarsi all'ambiente circostante.\\

I mantoscuro non sono scalatori particolarmente abili, ma sono in grado di appendersi alla volta di una caverna come i pipistrelli, agganciati per mezzo degli uncini posti in fondo ai loro tentacoli, così che il loro corpo penzolante risulti quasi indistinguibile da una stalattite. Da questa postazione nascosta la creatura attende che la preda passi sotto di lei e, a questo punto, si stacca lanciandosi verso di essa, sbattendo contro il bersaglio e tentando di avvolgervi attorno i suoi membranosi tentacoli. Se il mantoscuro manca la preda, risale e si lancia nuovamente contro la preda, fino a quando quest'ultima non viene sconfitta o il mantoscuro è gravemente ferito (nel qual caso svolazza sul soffitto per nascondersi, sperando che la sua "preda" lo lasci perdere). La capacità innata di questa creatura di celare la zona circostante per mezzo dell'oscurità magica le offre un ulteriore vantaggio contro gli avversari che necessitano della luce per vedere.\\

I mantoscuro preferiscono vivere e cacciare nelle caverne e nei cunicoli più vicini alla superficie, dal momento che questi offrono un più frequente passaggio di prede che questi mostri possono cacciare. Non si limitano però a queste caverne buie e talvolta possono essere incontrati in fortezze abbandonate o persino nelle fogne delle città affollate. Qualsiasi luogo dove abbondi il cibo e ci sia un soffitto a cui appendersi è un possibile covo per un mantoscuro.\\

Il ciclo vitale di un mantoscuro è rapido: i piccoli diventano adulti nell'arco di pochi mesi e la maggior parte muore di vecchiaia dopo pochi anni. Di conseguenza le generazioni di mantoscuro si susseguono rapidamente e nel corso degli anni l'evoluzione di queste creature è altrettanto rapida. Per questa ragione l'ecosistema di una caverna può avere effetti importanti sull'aspetto, le capacità e le tattiche di un mantoscuro. In caverne acquatiche possono svilupparsi mantoscuri in grado di nuotare, mentre creature che abitano luoghi soggetti a vulcanismo potrebbero sviluppare una specifica resistenza al fuoco. Altre varianti di mantoscuro potrebbero avere pelli più resistenti ed invece di cadere per stritolare la preda potrebbero semplicemente gettarsi cercando di trafiggerla analogamente a vere e proprie stalattiti. Si mormora che le caverne più oscure e profonde nascondano mantoscuri di dimensioni incredibili, in grado di soffocare contemporaneamente diversi bersagli di dimensioni umane nel loro abbraccio avvolgente.\\


\medskip\index{Mostri - Medusa}\textbf{Medusa}

\emph{Media mostruosità, legale malvagio}

\textbf{FORZA} +0

\textbf{DESTREZZA} +2

\textbf{COSTITUZIONE} +3

\textbf{INTELLIGENZA} +1

\textbf{SAGGEZZA} +1

\textbf{CARISMA} +2

\textbf{Iniziativa} +2 -- \textbf{Difesa} 18

\textbf{Punti Ferita} 127 (17d8 + 51)

\textbf{Movimento} 9 m

\textbf{Tiri Salvezza}: Tempra +6, Riflessi +8, Volontà +7

\textbf{Competenze} Muoversi Silenziosamente / Nascondersi +5, Ingannare +5, Percepire Emozioni +4, Consapevolezza +4

\textbf{Sensi} scurovisione 18 m

\textbf{Linguaggi} Comune

\textbf{Sfida} 6 (2.300 PE)

\emph{\textbf{Sguardo Pietrificante.}} Se una creatura comincia il suo turno entro 9 metri da una medusa di cui possa vedere gli occhi, la medusa, qualora la non sia inabile e possa vedere a sua volta la creatura, può obbligarla ad effettuare un Tiro Salvezza di Tempra CD 14. Se la creatura fallisce il Tiro Salvezza di 5 o più, viene pietrificata all'istante, altrimenti inizia magicamente a trasformarsi in pietra ed è intralciata. La creatura intralciata deve ripetere il Tiro Salvezza al termine del suo prossimo turno. Se lo riesce, l'effetto termina. Se lo fallisce, la creatura è pietrificata finché non viene liberata dall'incantesimo \emph{ristorare superiore} o altra magia.

Una creatura che non sia sorpresa può distogliere lo sguardo per evitare il Tiro Salvezza all'inizio del proprio turno. In quel caso, non potrà vedere la medusa fino all'inizio del suo prossimo turno, quando potrà distogliere nuovamente lo sguardo. Se nel frattempo dovesse guardare la medusa, dovrebbe immediatamente effettuare il Tiro Salvezza.

Se la medusa vede il suo riflesso su di una superficie riflettente entro 9 metri da lei in un'area di luce intensa, a causa della propria maledizione subirà gli effetti del suo stesso sguardo.

\textbf{Azioni}

\emph{\textbf{Multiattacco.}} La medusa effettua tre attacchi -- uno con i capelli serpentini e due con la spada corta -- o due attacchi a distanza con l'arco lungo.

\emph{\textbf{Capelli Serpentini.} Attacco con arma da mischia}: +5 a colpire, portata 1 m, un bersaglio.

\emph{Colpisce:} 4 (1d4 + 2) danni perforanti più 14 (4d6) danni da veleno.

\emph{\textbf{Spada Corta.} Attacco con arma da mischia}: +5 a colpire, portata 1 m, un bersaglio.

\emph{Colpisce:} 5 (1d6 + 2) danni perforanti.

\emph{\textbf{Arco Lungo.} Attacco con arma a Distanza}: +5 a colpire, gittata 45m, un bersaglio.

\emph{Colpisce:} 6 (1d8 + 2) danni perforanti più 7 (2d6) danni da veleno.

\textbf{Ecologia}\\
Ambiente: Paludi temperate e sotterranei\\
Organizzazione: Solitario\\
Tesoro: Doppio (Pugnale, Arco Lungo Perfetto con 20 Frecce, altro tesoro)\\
\textbf{Descrizione}\\
Le meduse sono creature simili agli umani con serpenti al posto dei capelli. Dalla distanza di 9 metri o più, una medusa può passare facilmente per una bella donna se indossa qualcosa che copre la sua chioma serpentina; quando indossa un abbigliamento che ne cela la testa e il volto può essere scambiata per un'umana anche a distanza ravvicinata. Le meduse usano bugie e travestimenti per celare il loro volto fino a che gli avversari non sono abbastanza vicini da usare il loro sguardo pietrificante, anche se gli piace giocare con la loro preda e possono usare delle frecce fiammeggianti per intrappolare i nemici a distanza. Alcune si divertono a creare intricate decorazioni con le loro vittime, usando la pietrificazione per dare un certo tocco ai loro nascondigli paludosi, ma molte meduse hanno cura di nascondere le prove dei loro scontri precedenti così che i loro nuovi nemici non si accorgano della loro pericolosa presenza.\\

Avvezze a nascondersi, le meduse cittadine generalmente sono ladre, mentre quelle delle zone selvagge spesso finiscono per essere guardiaboschi. Le meduse delle leggende più note, tuttavia, sono quelle che prendono livelli da incatatore. Carismatiche ed intelligenti, le meduse urbane sono spesso coinvolte in gilde di ladri ed altri aspetti del mondo criminale. Le meduse possono formare alleanze con creature cieche o non morti intelligenti, entrambi immuni al loro sguardo pietrificante. Le meduse incantatrici fungono spesso da oracoli o profetesse, vivendo generalmente in remote zone di leggendaria potenza o dalla storia infausta. Queste meduse oracoli traggono grande diletto dal loro ruolo, e se ci si presenta con i giusti doni e adulazioni, i segreti che offrono possono essere veramente utili. Naturalmente, i nascondigli di queste potenti creature sono decorati con le statue di coloro che le hanno offese, come monito ad usare le dovute cautele durante gli incontri.\\

Tutte le meduse sono femmine. Raramente, una medusa decide di prendere un maschio umanoide come compagno, generalmente grazie all'aiuto di una Elisir d'Amore o qualche magia simile, ed hanno sempre cura di non pietrificare il loro prigioniero, a meno che non si siano annoiate della sua compagnia.\\


\subsection{Mefiti}

\medskip\index{Mostri - Mefito di Ghiaccio}\textbf{Mefito di Ghiaccio}

\emph{Piccola elementale, neutrale malvagio}

\textbf{FORZA} -2

\textbf{DESTREZZA} +1

\textbf{COSTITUZIONE} +0

\textbf{INTELLIGENZA} -1

\textbf{SAGGEZZA} +0

\textbf{CARISMA} +1

\textbf{Iniziativa} +1 -- \textbf{Difesa} 12

\textbf{Punti Ferita} 21 (6d6)

\textbf{Movimento} 9 m, volo 9 m

\textbf{Tiri Salvezza}: Tempra +2, Riflessi +5, Volontà +3

\textbf{Competenze} Muoversi Silenziosamente / Nascondersi +3, Consapevolezza +2

\textbf{Vulnerabilità ai Danni} da botta, fuoco

\textbf{Immunità ai Danni} freddo, veleno

\textbf{Immunità alle Condizioni} avvelenato

\textbf{Sensi} scurovisione 18 m

\textbf{Linguaggi} Aquan, Auran

\textbf{Sfida} 1/2 (100 PE)

\emph{\textbf{Falso Aspetto.}} Mentre il mefito rimane immobile, è indistinguibile da un ordinario frammento di ghiaccio.

\emph{\textbf{Incantesimi Innati (1/Giorno).}} Il mefito può lanciare in maniera innata \emph{nube di nebbia}, senza bisogno di componenti materiali. La sua caratteristica da incantatore innato è il Carisma.

\emph{\textbf{Natura Elementale.}} Un mefito non ha bisogno di cibo, bevande o sonno.

\emph{\textbf{Scoppio Mortale.}} Quando il mefito muore, esplode in uno scoppio di frammenti di ghiaccio. Ogni creatura entro 1 metro da esso deve effettuare un Tiro Salvezza di Riflessi CD 10 o subire 4 (1d8) danni taglienti in caso di fallimento, o la metà di questi danni in caso
di successo.

\textbf{Azioni}

\emph{\textbf{Artigli.} Attacco con arma da mischia}: +3 a colpire,
portata 1 m, una creatura.

\emph{Colpisce:} 3 (1d4 + 1) danni taglienti più 2 (1d4) danni da
freddo.

\emph{\textbf{Soffio Gelido (Ricarica 6).}} Il mefito esala un cono di 5 metri di aria fredda. Ogni creatura nell'area deve effettuare un Tiro Salvezza di Riflessi CD 10, subendo 5 (2d4) danni da freddo in caso di fallimento, o la metà di questi danni in caso di successo.

\textbf{Ecologia}\\
Ambiente: Qualsiasi (piano elementale dell'aria)\\
Organizzazione: Solitario, coppia, gruppo (3-6) o stormo (7-12)\\
Tesoro: Standard\\
\textbf{Descrizione}\\
I mephit sono i servitori di potenti creature elementali. I siti e le locazioni chiave dei piani elementali sono pieni di mephit che si affannano per svolgere un importante dovere o incarico.\\

I mephit del ghiaccio comunemente si trovano sul Piano dell'Aria. Questi mephit sono distanti e crudeli.\\


\medskip\index{Mostri - Mefito di Magma}\textbf{Mefito di Magma}

\emph{Piccola elementale, neutrale malvagio}

\textbf{FORZA} -1

\textbf{DESTREZZA} +1

\textbf{COSTITUZIONE} +1

\textbf{INTELLIGENZA} -2

\textbf{SAGGEZZA} +0

\textbf{CARISMA} +0

\textbf{Iniziativa} +1 -- \textbf{Difesa} 12

\textbf{Punti Ferita} 22 (5d6 + 5)

\textbf{Movimento} 9 m, volo 9 m

\textbf{Tiri Salvezza}: Tempra +2, Riflessi +5, Volontà +3

\textbf{Competenze} Muoversi Silenziosamente / Nascondersi +3

\textbf{Vulnerabilità ai Danni} freddo

\textbf{Immunità ai Danni} fuoco, veleno

\textbf{Immunità alle Condizioni} avvelenato

\textbf{Sensi} scurovisione 18 m

\textbf{Linguaggi} Ignan, Terran

\textbf{Sfida} 1/2 (100 PE)

\emph{\textbf{Falso Aspetto.}} Mentre il mefito rimane immobile, è indistinguibile da un'ordinaria pozza di magma.

\emph{\textbf{Incantesimi Innati (1/Giorno).}} Il mefito può lanciare in maniera innata \emph{riscaldare metallo} (CD del Tiro Salvezza dell'incantesimo 10), senza bisogno di componenti materiali. La sua caratteristica da incantatore innato è il Carisma.

\emph{\textbf{Natura Elementale.}} Un mefito non ha bisogno di cibo, bevande o sonno.

\emph{\textbf{Scoppio Mortale.}} Quando il mefito muore, esplode in uno scoppio di lava. Ogni creatura entro 1 metro da esso deve effettuare un Tiro Salvezza di Riflessi CD 11 o subire 7 (2d6) danni da fuoco in caso di fallimento, o la metà di questi danni in caso di successo.

\textbf{Azioni}

\emph{\textbf{Artigli.} Attacco con arma da mischia}: +3 a colpire, portata 1 m, una creatura.

\emph{Colpisce:} 3 (1d4 + 1) danni taglienti più 2 (1d4) danni da fuoco.

\emph{\textbf{Soffio Infuocato (Ricarica 6).}} Il mefito esala un cono di 5 metri di fuoco. Ogni creatura nell'area deve effettuare un tiro salvezza su Riflessi CD 11, subendo 7 (2d6) danni da fuoco in caso di fallimento, o la metà di questi danni in caso di successo.

\textbf{Ecologia}\\\
Ambiente: Qualsiasi (piano elementale del fuoco)\\
Organizzazione: Solitario, coppia, gruppo (3-6) o stormo (7-12)\\
Tesoro: Standard\\
\textbf{Descrizione}\\
I mephit sono i servitori di potenti creature elementali. I siti e le locazioni chiave dei piani elementali sono pieni di mephit che si affannano per svolgere un importante dovere o incarico.\\
I mephit del magma comunemente si trovano sul Piano del Fuoco. Questi mephit sono stupidi bruti.\\


\medskip\index{Mostri - Mefito di Polvere}\textbf{Mefito di Polvere}

\emph{Piccola elementale, neutrale malvagio}

\textbf{FORZA} -3

\textbf{DESTREZZA} +2

\textbf{COSTITUZIONE} +0

\textbf{INTELLIGENZA} -1

\textbf{SAGGEZZA} +0

\textbf{CARISMA} +0

\textbf{Iniziativa} +2 -- \textbf{Difesa} 13

\textbf{Punti Ferita} 17 (5d6)

\textbf{Movimento} 9 m, volo 9 m

\textbf{Tiri Salvezza}: Tempra +2, Riflessi +5, Volontà +3

\textbf{Competenze} Muoversi Silenziosamente / Nascondersi +4, Consapevolezza +2

\textbf{Vulnerabilità ai Danni} fuoco

\textbf{Immunità ai Danni} veleno

\textbf{Immunità alle Condizioni} avvelenato

\textbf{Sensi} scurovisione 18 m

\textbf{Linguaggi} Auran, Terran

\textbf{Sfida} 1/2 (100 PE)

\emph{\textbf{Incantesimi Innati (1/Giorno).}} Il mefito può eseguire in
maniera innata \emph{sonno} (CD del Tiro Salvezza dell'incantesimo 10),
senza bisogno di componenti materiali. La sua abilità da incantatore
innato è il Carisma.

\emph{\textbf{Natura Elementale.}} Un mefito non ha bisogno di cibo,
bevande o sonno.

\emph{\textbf{Scoppio Mortale.}} Quando il mefito muore, esplode in uno scoppio di polvere. Ogni creatura entro 1 metro da esso deve riuscire un Tiro Salvezza di Tempra CD 10 o restare accecata per 1 minuto. Una creatura accecata può ripetere il Tiro Salvezza durante ciascun suo turno, terminando l'effetto su di sé in caso di successo.

\textbf{Azioni}

\emph{\textbf{Artigli.} Attacco con arma da mischia}: +4 a colpire,
portata 1 m, una creatura.

\emph{Colpisce:} 4 (1d4 + 2) danni taglienti.

\emph{\textbf{Soffio Accecante (Ricarica 6).}} Il mefito esala un cono di 5 metri di polvere accecante. Ogni creatura nell'area deve riuscire un Tiro Salvezza di Riflessi CD 10 o restare accecata per 1 minuto. Una creatura accecata può ripetere il Tiro Salvezza durante ciascun suo turno, terminando l'effetto su di sé in caso di successo.

\textbf{Ecologia}\\
Ambiente: Qualsiasi (piano elementale dell'aria)\\
Organizzazione: Solitario, coppia, gruppo (3-6) o stormo (7-12)\\
Tesoro: Standard\\
\textbf{Descrizione}\\
I mephit sono i servitori di potenti creature elementali. I siti e le locazioni chiave dei piani elementali sono pieni di mephit che si affannano per svolgere un importante dovere o incarico.\\
I mephit della polvere comunemente si trovano sul Piano dell'Aria. Questi mephit sono irritanti ed insistenti.\\

\medskip\index{Mostri - Mefito di Vapore}\textbf{Mefito di Vapore}

\emph{Piccola elementale, neutrale malvagio}

\textbf{FORZA} -3

\textbf{DESTREZZA} +0

\textbf{COSTITUZIONE} +0

\textbf{INTELLIGENZA} +0

\textbf{SAGGEZZA} +0

\textbf{CARISMA} +1

\textbf{Iniziativa} +0 -- \textbf{Difesa} 11

\textbf{Punti Ferita} 21 (6d6)

\textbf{Movimento} 9 m, volo 9 m

\textbf{Tiri Salvezza}: Tempra +2, Riflessi +5, Volontà +3

\textbf{Immunità ai Danni} fuoco, veleno

\textbf{Immunità alle Condizioni} avvelenato

\textbf{Sensi} scurovisione 18 m

\textbf{Linguaggi} Aquan, Ignan

\textbf{Sfida} 1/4 (50 PE)

\emph{\textbf{Incantesimi Innati (1/Giorno).}} Il mefito può eseguire in maniera innata \emph{sfocatura}, senza bisogno di componenti materiali. La sua abilità da incantatore innato è il Carisma.

\emph{\textbf{Natura Elementale.}} Un mefito non ha bisogno di cibo,
bevande o sonno.

\emph{\textbf{Scoppio Mortale.}} Quando il mefito muore, esplode in nube
di vapore. Ogni creatura entro 1 metro da esso deve riuscire un tiro
salvezza su Riflessi CD 10 o subire 4 (1d8) danni da fuoco.

\textbf{Azioni}

\emph{\textbf{Artigli.} Attacco con arma da mischia}: +2 a colpire,
portata 1 m, una creatura.

\emph{Colpisce:} 2 (1d4) danni taglienti più 2 (1d4) danni da fuoco.

\emph{\textbf{Soffio Vaporoso (Ricarica 6).}} Il mefito esala un cono di 5 metri di vapore caldo. Ogni creatura nell'area deve effettuare un Tiro Salvezza di Riflessi CD 10, subendo 4 (1d8) danni da fuoco in caso di fallimento, o la metà di questi danni in caso di successo.

\textbf{Ecologia}\\
Ambiente: Qualsiasi (piano elementale del fuoco)\\
Organizzazione: Solitario, coppia, gruppo (3-6) o stormo (7-12)\\
Tesoro: Standard\\
\textbf{Descrizione}\\
I mephit sono i servitori di potenti creature elementali. I siti e le locazioni chiave dei piani elementali sono pieni di mephit che si affannano per svolgere un importante dovere o incarico.\\
I mephit del vapore comunemente si trovano sul Piano del Fuoco. Questi mephit sono insolenti e sprezzanti.\\



\subsection{Megere}

\medskip\index{Mostri - Megera Marina}\textbf{Megera Marina}

\emph{Media fatato, caotico malvagio}

\textbf{FORZA} +3

\textbf{DESTREZZA} +1

\textbf{COSTITUZIONE} +3

\textbf{INTELLIGENZA} +1

\textbf{SAGGEZZA} +1

\textbf{CARISMA} +1

\textbf{Iniziativa} +1 -- \textbf{Difesa} 15

\textbf{Punti Ferita} 52 (7d8 + 21)

\textbf{Vulnerabilità al Danno} ferro freddo

\textbf{Movimento} 9 m, nuoto 12 m

\textbf{Tiri Salvezza}: Tempra +5, Riflessi +7, Volontà +5

\textbf{Sensi} scurovisione 18 m

\textbf{Linguaggi} Aquan, Comune, Gigante

\textbf{Sfida} 2 (450 PE)

\emph{\textbf{Anfibio.}} La megera può respirare aria e acqua.

\emph{\textbf{Aspetto Orripilante.}} Qualsiasi umanoide che inizi il suo turno entro 9 metri dalla megera e ne può vedere la vera forma deve effettuare un Tiro Salvezza di Volontà CD 11. Se fallisce il Tiro Salvezza, la creatura resta spaventata per 1 minuto. Una creatura può ripetere il Tiro Salvezza al termine di ciascun suo turno, con -1d6 se la megera è in linea di visuale, e terminando l'effetto se riesce il Tiro Salvezza. Se il Tiro Salvezza della creatura riesce o l'effetto ha termine su di essa, la creatura è immune all'Aspetto Orripilante per le successive 24 ore.

A meno che il bersaglio non sia sorpreso o la rivelazione della vera forma della megera non sia improvvisa, il bersaglio può distogliere lo sguardo e evitare di effettuare il Tiro Salvezza iniziale. Fino all'inizio del suo prossimo turno, una creatura che distolga lo sguardo
ha -1d6 ai tiri di attacco contro la megera.

\textbf{Azioni}

\emph{\textbf{Artigli.} Attacco in mischia con arma}: +5 a colpire,
portata 1 m, un bersaglio.

\emph{Colpisce:} 10 (2d6 + 3) danni taglienti.

\emph{\textbf{Aspetto Illusorio.}} La megera ricopre se stessa e tutto
quello che sta indossando o trasportando in un'illusione magica che le
dona l'aspetto di una creatura ripugnante all'incirca della stessa
taglia e forma umanoide. L'illusione termina se la megera effettua
un'azione bonus per terminarla o se muore.

I cambiamenti apportati da questo effetto non sono in grado di superare le ispezioni fisiche. Ad esempio, la megera potrebbe apparire come una creatura priva di artigli, ma una persona in contatto con le sue mani li avvertirebbe. Altrimenti, una creatura deve effettuare un'azione per ispezionare visivamente l'illusione e riuscire una prova di Intelligenza CD 16 per comprendere che la megera si è camuffata.

\emph{\textbf{Occhiata Mortale.}} La megera prende a bersaglio una creatura spaventata visibile entro 9 metri da lei. Se il bersaglio può vedere la megera, deve riuscire un Tiro Salvezza di Volontà CD 11 contro questa magia o scendere a 0 punti ferita.

\textbf{Ecologia}\\
Ambiente: qualsiasi acquatico\\
Organizzazione: solitario o congrega (3 megere di qualsiasi specie)\\
Tesoro: standard\\
\textbf{Descrizione}\\
Queste perfide e mostruose megere possiedono dei tratti terrificanti che pochi osano fissare, traggono piacere dalla discordia e dalla morte dei marinai, e tormentano la gente di mare con ineluttabili sciagure. Le megere marine hanno sempre un aspetto terribile e, malgrado la loro natura famelica, in genere sono creature emaciate che sembrano sul punto di morir di fame. Sono alte 1,8 metri e pesano 75 kg.\\

Le megere marine preferiscono vivere vicino alla riva dove i pescherecci e i mercantili sono più comuni, e comunque lontano dalle aree urbane di modo che le loro azioni non attraggano troppo l'attenzione di possibili nemici, anche se non è insolito che una megera marina coraggiosa o avida si stabilisca in una città portuale o alla foce di un fiume profondo.\\

Le megere marine formano congreghe simili a quelle delle altre megere, ma la loro natura acquatica generalmente le spinge ad astenersi dal formare congreghe miste. Nel caso in cui una Megera Verde abiti lungo la costa (spesso in una palude salmastra o in una palude costiera), una congrega è formata da due megere marine che rispettano la Megera Verde come madre e capo. Molto comunemente, una congrega di megere marine consiste in un gruppo di megere marine particolarmente amiche e vicine.\\


\medskip\index{Mostri - Megera Notturna}\textbf{Megera Notturna}

\emph{Media immondo, neutrale malvagio}

\textbf{FORZA} +4

\textbf{DESTREZZA} +2

\textbf{COSTITUZIONE} +3

\textbf{INTELLIGENZA} +3

\textbf{SAGGEZZA} +2

\textbf{CARISMA} +3

\textbf{Iniziativa} +3 -- \textbf{Difesa} 20

\textbf{Punti Ferita} 112 (15d8 + 45)

\textbf{Movimento} 9 m

\textbf{Tiri Salvezza}: Tempra +14, Riflessi +8, Volontà +11

\textbf{Competenze} Muoversi Silenziosamente / Nascondersi +6, Ingannare +7, Percepire Emozioni +6, Consapevolezza +6,

\textbf{Resistenze al Danno} freddo, fuoco; da botta, perforante e tagliente di attacchi non magici o non siano argentati

\textbf{Sensi} scurovisione 36 m

\textbf{Linguaggi} Abissale, Comune, Infernale, Druidico

\textbf{Sfida} 5 (1.800 PE)

\emph{\textbf{Incantesimi Innati.}} La caratteristica da incantatore innato della megera è il Carisma (CD 14 per i Tiri Salvezza degli incantesimi, +6 a colpire con attacchi da incantesimo). La megera può lanciare in maniera innata i seguenti incantesimi, senza aver bisogno di
componenti materiali.

A volontà: \emph{dardo incantato, individuazione del magico} 2/giorno ciascuno: \emph{raggio di indebolimento, sonno, spostamento} \emph{planare} (personale)

\emph{\textbf{Resistenza alla Magia.}} La megera ha +1d6 ai tiri salvezza contro incantesimi e altri effetti magici.

\textbf{Azioni}

\emph{\textbf{Artigli (Solo in Forma di Megera).} Attacco con arma da 	mischia}: +7 a colpire, portata 1 m, un bersaglio. \emph{Colpisce:} 13 (2d8 + 4) danni taglienti.

\emph{\textbf{Forma Eeterea.}} La megera entra magicamente nel Piano Etereo dal Piano Materiale, e viceversa. Per farlo deve essere in possesso di un \emph{cuore di pietra}.

\emph{\textbf{Infestare Incubi (1/Giorno).}} Mentre si trova sul Piano Etereo, la megera entra magicamente in contatto con un umanoide addormentato che si trova sul Piano Materiale. L'incantesimo \emph{protezione dal bene e dal male} lanciato sul bersaglio previene questo contatto, così come \emph{cerchio magico}. Finché il contatto persiste, il bersaglio soffre di orribili visioni. Se queste visioni durano per almeno 1 ora, il bersaglio non ottiene benefici dal suo riposo, e i suoi punti ferita massimi sono ridotti di 5 (1d10). Se questo effetto riduce i punti ferita massimi del bersaglio a 0, il bersaglio muore, e se il bersaglio era malvagio, la sua anima resta intrappolata nella \emph{borsa} \emph{delle anime} della megera. La riduzione dei punti ferita massimi del bersaglio rimane finché non viene rimossa dall'incantesimo \emph{ristorare} \emph{superiore} o simile magia.

\emph{\textbf{Mutare Forma.}} La megera può trasformarsi magicamente in una femmina umanoide di taglia Piccola o Media, o tornare alla sua vera forma. Le sue statistiche sono le stesse in qualsiasi forma. Tutto l'equipaggiamento che stava trasportando o indossando non viene trasformato. Alla morte, ritorna alla sua vera forma.



\medskip\index{Mostri - Megera Verde}\textbf{Megera Verde}

\emph{Media fatato, neutrale malvagio}

\textbf{FORZA} +4

\textbf{DESTREZZA} +1

\textbf{COSTITUZIONE} +3

\textbf{INTELLIGENZA} +1

\textbf{SAGGEZZA} +2

\textbf{CARISMA} +2

\textbf{Iniziativa} +1 -- \textbf{Difesa} 19

\textbf{Punti Ferita} 82 (11d8 + 33)

\textbf{Vulnerabilità al Danno} ferro freddo

\textbf{Movimento} 9 m

\textbf{Tiri Salvezza}: Temp. +6, Rifl. +7, Vol. +7

\textbf{Competenze} Arcano +3, Muoversi Silenziosamente / Nascondersi +3, Ingannare +4, Consapevolezza +4

\textbf{Sensi} scurovisione 18 m

\textbf{Linguaggi} Comune, Draconico, Silvano

\textbf{Sfida} 3 (700 PE)

\emph{\textbf{Anfibio.}} La megera può respirare aria e acqua.

\emph{\textbf{Imitazione.}} La megera può imitare suoni animali e voci umanoidi. Una creatura che senta questi rumori può determinare che si tratti di un'imitazione riuscendo una prova di Saggezza CD 14.

\emph{\textbf{Incantesimi Innati.}} La caratteristica da incantatore innato della megera è il Carisma (CD 12 per i Tiri Salvezza degli incantesimi). La megera può lanciare in maniera innata i seguenti incantesimi, senza aver bisogno di componenti materiali.

A volontà: \emph{illusione minore, luci danzanti, beffa maligna}

\textbf{Azioni}

\emph{\textbf{Artigli.} Attacco con arma da mischia}: +6 a colpire,
portata 1 m, un bersaglio.

\emph{Colpisce:} 13 (2d8 + 4) danni taglienti.

\emph{\textbf{Aspetto Illusorio.}} La megera ricopre sé stessa e tutto quello che sta indossando o trasportando in un'illusione magica che le dona l'aspetto di un'altra creatura all'incirca della stessa taglia e forma umanoide. L'illusione termina se la megera effettua un'azione bonus per terminarla o se muore.

I cambiamenti apportati da questo effetto non sono in grado di superare le ispezioni fisiche. Ad esempio, la megera potrebbe apparire come una creatura dalla pelle liscia, ma il contatto rivelerebbe la sua pelle ruvida. Altrimenti, una creatura deve effettuare un'azione per ispezionare visivamente l'illusione e riuscire una prova di Intelligenza CD 20 per comprendere che si tratta di una megera camuffata.

\emph{\textbf{Passaggio Invisibile.}} La megera può rendersi invisibile finché non attacca o lancia un incantesimo, o finché non termina la concentrazione (come se si stesse concentrando su di un incantesimo). Mentre è invisibile, non lascia traccia fisica del suo passaggio, quindi le sue tracce possono essere seguite solo dalla magia. Tutto l'equipaggiamento che sta trasportando o indossando diventa invisibile assieme a lei.

\textbf{Ecologia}
Ambiente: Paludi temperate\\
Organizzazione: Solitario o congrega (3 megere di qualsiasi tipo)\\
Tesoro: Standard\\
\textbf{Descrizione}\\
Terrificanti vecchie rugose che frequentano ripugnanti paludi e foreste intricate, le megere verdi nutrono un odio intenso per tutto ciò che è bello e puro. Facendo uso delle loro svariate capacità illusorie, queste vegliarde si dilettano nell'uccidere gli innocenti, nello sconvolgere gli animi nobili e nell'avvilire i cuori puri. Amano utilizzare Camuffare Se Stesso per assumere le forme di giovani e attraenti ragazze così da sedurre e strappare giovani uomini ai loro affetti e parenti, e corrompere nobili e onesti cittadini con ogni sorta di depravazione e scandalo. Alcune megere verdi preferiscono rivelare la loro reale natura ai loro amati in un momento attentamente architettato per spingere l'uomo alla pazzia per l'orrore e la vergogna. Altre prolungano il loro amoreggiamento e fanno di tutto per rovinare completamente la vita degli uomini da loro sedotti prima di mostrare loro la verità. Infine, i più fortunati di questi sventurati finiscono per essere divorati dalla megera verde loro amante: per gli sfortunati, il destino finale può essere molto peggiore, dato che la crudele fantasia della megera verde è immensa. Una tipica megera verde è alta tra 1,5 e 1,8 metri e pesa poco meno di 80 kg.\\


\subsection{Melme}

\medskip\index{Mostri - Ameba Paglierina}\textbf{Ameba Paglierina}

\emph{Grande melma, disallineato}

\textbf{FORZA} +2

\textbf{DESTREZZA} -2

\textbf{COSTITUZIONE} +2

\textbf{INTELLIGENZA} -4

\textbf{SAGGEZZA} -2

\textbf{CARISMA} -5

\textbf{Iniziativa} +2 -- \textbf{Difesa} 9

\textbf{Punti Ferita} 45 (6d10 + 12)

\textbf{Movimento} 3 m, scalata 3 m

\textbf{Tiri Salvezza}: Tempra +8, Riflessi -3, Volontà -3

\textbf{Resistenze al Danno} acido

\textbf{Immunità al Danno} fulmine, tagliente

\textbf{Immunità alle Condizioni} accecato, affascinato, assordato, prono, affaticamento, spaventato

\textbf{Sensi} vista cieca 18 m (cieca oltre questo raggio)

\textbf{Linguaggi} -

\textbf{Sfida} 2 (450 PE)

\emph{\textbf{Amorfo.}} L'ameba può muoversi attraverso uno spazio fino a 3 centimetri di larghezza senza doversi stringere.

\emph{\textbf{Natura di Melma.}} L'ameba non necessita di dormire.

\emph{\textbf{Scalare come Ragno.}} L'ameba può scalare superfici
difficili, compreso lo stare a testa in giù sul soffitto, senza bisogno
di effettuare una prova di abilità.

\textbf{Azioni}

\emph{\textbf{Pseudopodo.} Attacco con arma da mischia}: +4 a colpire,
portata 1 m, un bersaglio.

\emph{Colpisce:} 9 (2d6 + 2) danni da botta più 3 (1d6) danni da
acido.

\textbf{Reazioni}

\emph{\textbf{Divisione.}} Quando un'ameba Media o più grande subisce danni da fulmine o taglienti, si divide in due nuove amebe che hanno almeno 10 punti ferita. Ogni nuova ameba ha un numero di punti ferita pari alla metà dell'ameba originale, arrotondati per difetto. Le nuove amebe sono di una taglia più piccola di quella originale.

\textbf{Ecologia}
Ambiente: Sotterranei o Paludi Temperati\\
Organizzazione: Solitario\\
Tesoro: Nessuno\\
\textbf{Descrizione}\\
Le Ameba Paglierina sono masse animate di protoplasma di colore simile ad un repellente amalgama di giallo, arancio e marrone. Quando a riposo, il loro corpo piatto e pulsante è alto circa 15 centimetri e si estende tutto intorno; in movimento, si raccolgono in una forma vagamente sferica e sembrano quasi spostarsi rotolando. I loro corpi malleabili permettono loro di attraversare fessure e buchi molto più piccoli dello spazio che occupano. Le creature che vivono sottoterra spesso sigillano tutte le aperture per difendersi dalle Ameba Paglierina.\\

l'acido altamente specializzato dell'Ameba Paglierina dissolve solo la carne. Questa scoperta ha portato molti maestri avvelenatori ed alchimisti a cercarne esemplari per studiarli. Da questi esperimenti sono nate diverse armi specifiche ideate per distruggere i corpi. Si racconta dell'esistenza di un veleno ad azione lenta che distrugge ad una ad una le cellule delle creature viventi, il cui segreto è ben conservato dal suo creatore.\\

Alcune note in un tomo dimenticato parlano di un rituale funebre utilizzato in luoghi lontani. Invece di bruciare il corpo, esso veniva sigillato in un sarcofago di pietra con una Ameba Paglierina, che ne dissolveva il corpo. In seguito, i becchini inserivano la gelatina in un'urna completa di targa in bronzo con il nome del deceduto. Questa pratica protegge gli oggetti interrati con il morto (che viene ridotto in poco tempo ad uno scheletro lucido) e l'essenza della creatura, che si riteneva vivere ancora all'interno della gelatina.\\

L'Ameba Paglierina sono alte circa 15 centimetri, hanno un diametro che puo' arrivare a 3 metri e pesano circa 1.300 chili. In combattimento, si raccolgono su loro stesse e producono lunghi pseudopodi umidi per colpire ed afferrare qualunque cosa si muova.\\

Anche se la tipica Ameba Paglierina ha le statistiche qui presentate, nelle profondità della terra questi predatori possono raggiungere dimensioni mostruose. Si parla anche di Ameba Paglierina che hanno sviluppato altri modi di catturare la preda. Ad esempio, gelatine che avvelenano con il tocco e che espellono nubi di gas tossico che fa bruciare occhi e bocca, lasciando indifesi ma coscienti mentre questa bestia protoplasmatica scivola sui corpi e se ne ciba.


\medskip\index{Mostri - Cubo Gelatinoso}\textbf{Cubo Gelatinoso}

\emph{Grande melma, disallineato}

\textbf{FORZA} +2

\textbf{DESTREZZA} -4

\textbf{COSTITUZIONE} +5

\textbf{INTELLIGENZA} -5

\textbf{SAGGEZZA} -2

\textbf{CARISMA} -5

\textbf{Iniziativa} -4 -- \textbf{Difesa} 7

\textbf{Punti Ferita} 84 (8d10 + 40)

\textbf{Movimento} 5 metri

\textbf{Tiri Salvezza}: Tempra +9, Riflessi -4, Volontà -4

\textbf{Immunità alle Condizioni} accecato, affascinato, assordato, prono, affaticamento, spaventato

\textbf{Sensi} vista cieca 18 m (cieca oltre questo raggio)

\textbf{Linguaggi} -

\textbf{Sfida} 2 (450 PE)

\emph{\textbf{Cubo di Melma.}} Il cubo occupa il suo intero spazio. Le altre creature possono entrare nello spazio, ma rimangono vittima del Sommergere del cubo e hanno -1d6 al Tiro Salvezza.

Le creature all'interno del cubo sono visibili ma godono di copertura totale.

Una creatura entro 1 metro dal cubo può effettuare un'azione per tirare una creatura od oggetto fuori dal cubo. Farlo richiede la riuscita di una prova di Forza CD 12, e la creatura che effettua il tentativo subisce 10 (3d6) danni da acido.

Il cubo può contenere solo una creatura Grande o un massimo di quattro creature Medie o più piccole alla volta.

\emph{\textbf{Natura di Melma.}} Il cubo non necessita di dormire.

\emph{\textbf{Trasparente.}} Anche quando è in piena vista, è necessario riuscire una prova di Saggezza (Consapevolezza) CD 15 per notare un cubo che non si è mosso o non ha attaccato. Una creatura che cerchi di entrare nello spazio del cubo mentre è inconsapevole della sua presenza resta sorpresa dal cubo.

\textbf{Azioni}

\emph{\textbf{Pseudopodo.} Attacco con arma da mischia}: +4 a colpire, portata 1 m, un bersaglio.

\emph{Colpisce:} 10 (3d6) danni da acido.

\emph{\textbf{Sommergere.}} Il cubo si muove fino al massimo del suo movimento. Nel farlo, può entrare nello spazio di una creatura di taglia Grande o più piccola. Ogni volta che il cubo entra nello spazio di una creatura, la creatura deve effettuare un Tiro Salvezza di Riflessi CD 12.

Se il Tiro Salvezza riesce, la creatura può scegliere di essere spinta indietro o di lato di 1 metro. Una creatura che decida di non farsi spingere subisce le conseguenze di un Tiro Salvezza fallito.

Se il Tiro Salvezza fallisce, il cubo entra nello spazio della creatura, che subisce 10 (3d6) danni da acido ed è sommersa. La creatura sommersa non può respirare, è intralciata e subisce 21 (6d6) danni da acido all'inizio del turno del cubo. Quando il cubo si muove, la creatura sommersa si muove con esso.

Una creatura sommersa può tentare di fuggire effettuando un'azione per compiere una prova di Forza CD 12. Se la riesce, la creatura sfugge e ed entra nello spazio di sua scelta entro 1 metro dal cubo.

\textbf{Ecologia}
Ambiente: Qualsiasi sotterraneo\\
Organizzazione: Solitario\\
Tesoro: Accidentale\\
\textbf{Descrizione}\\
Tra i predatori più insoliti e peculiari dei dungeon, i cubi gelatinosi trascorrono la loro esistenza vagabondando senza meta per i cunicoli sotterranei e le oscure caverne, inglobando materiali organici come piante, rifiuti, carogne e anche creature viventi. La materia che il cubo non può digerire, come metalli e pietra, riempie di detriti il volume della creatura, e a volte questa può espellerne una parte dal suo corpo. Spesso il tesoro e gli averi delle vittime passate restano dentro il cubo gelatinoso: immagine spettrale dei loro resti materiali.\\

Alcuni saggi credono che queste creature si siano evolute delle Melme Grigie. Alcuni esseri usano i cubi gelatinosi come guardiani di dungeon e fortificazioni sotterranee, intrappolando queste immense creature in casse di metallo massiccio e trasportandole con poteri o magie fino al loro posto di guardia finale. Sono dei meccanismi di smaltimento rifiuti particolarmente efficaci; una tribù può intrappolare un cubo gelatinoso in una fossa o un'altra area che non possa scalare usandolo come letamaio o anche trappola mortale, a seconda dell'ingegnosità delle creature che l'hanno catturato.\\

I cubi gelatinosi in genere hanno uno spigolo di 3 metri e pesano più di 7.500 kg, sebbene alcuni esploratori sotterranei affermino che nel sottosuolo esistano esemplari più grandi. In zone in cui il cibo abbonda, i cubi gelatinosi possono vivere per centinaia, se non migliaia, di anni. Tuttavia, se viene a mancare la materia organica per più di 6 mesi, un cubo gelatinoso comincia a deperire, e le sue pareti iniziano a colare, disfacendosi rapidamente in muco liquido finché l'intero corpo non collassa e scompare completamente.\\


\medskip\index{Mostri - Melma Grigia}\textbf{Melma Grigia}

\emph{Media melma, disallineato}

\textbf{FORZA} +1

\textbf{DESTREZZA} -2

\textbf{COSTITUZIONE} +3

\textbf{INTELLIGENZA} -5

\textbf{SAGGEZZA} -2

\textbf{CARISMA} -4

\textbf{Iniziativa} -2 -- \textbf{Difesa} 9

\textbf{Punti Ferita} 22 (3d8 + 9)

\textbf{Movimento} 3 m, scalata 3 m

\textbf{Tiri Salvezza}: Tempra +9, Riflessi -4, Volontà -4

\textbf{Resistenze al Danno} acido, freddo, fuoco

\textbf{Immunità alle Condizioni} accecato, affascinato, assordato, prono, affaticamento, spaventato

\textbf{Sensi} vista cieca 18 m (cieca oltre questo raggio)

\textbf{Linguaggi} -

\textbf{Sfida} 1/2 (100 PE)

\emph{\textbf{Amorfo.}} La melma può muoversi attraverso uno spazio fino a  centimetri di larghezza senza doversi stringere.

\emph{\textbf{Corrodere Metallo.}} Qualsiasi arma non magica fatta di metallo che colpisca la melma si corrode. Dopo aver inflitto il danno, l'arma subisce una penalità permanente e cumulativa di -1 ai tiri di danno. Se la penalità arriva a -5, l'arma è distrutta. Le munizioni non magiche fatte di metallo che colpiscano la melma, si distruggono dopo aver inflitto il danno.

La melma può divorare metallo non magico dello spessore di 5 centimetri in un 1 round.

\emph{\textbf{Falso Aspetto.}} Quando la melma rimane immobile, è indistinguibile da una pozza d'olio o una pietra bagnata.

\emph{\textbf{Natura di Melma.}} La melma non necessita di dormire.

\textbf{Azioni}

\emph{\textbf{Pseudopodo.} Attacco con arma da mischia}: +3 a colpire, portata 1 m, un bersaglio.

\emph{Colpisce:} 4 (1d6 + 1) danni da botta più 7 (2d6) danni da acido, e se il bersaglio sta indossando un'armatura di metallo, questa viene parzialmente dissolta e subisce una penalità permanente e cumulativa di -1 alla Difesa che offre. L'armatura è distrutta se la penalità riduce la sua Difesa a 10.

\textbf{Ecologia}\\
Ambiente: Paludi fredde e sotterranei\\
Organizzazione: Solitario\\
Tesoro: Nessuno\\
\textbf{Descrizione}\\
Strisciando attraverso le fredde paludi e gli acquitrini nebbiosi o, a volte in sotterranei e caverne, le melme grigie consumano ogni sostanza organica che incontrano. Sebbene priva di intelligenza, la melma grigia è una delle creature che dà non pochi problemi per la sua trasparenza. Anche se non può arrampicarsi facilmente sui muri o nuotare, la sua abitudine di nascondersi nel fango spesso lungo le rive paludose o di rimanere immobile in pozze dall'aspetto innocuo sul pavimento grigio di un sotterraneo, la rendono molto difficile da notare e da evitare.\\

Alcuni saggi credono che le melme grigie siano il risultato di un esperimento alchemico fallito, mentre altri teorizzano che le prime melme grigie siano nate spontaneamente da un pozzo di detriti magici. Naturalmente, queste teorie che non le considerano organismi viventi, bensì il risultato di una sfortunata mistura di fluidi caustici e residui magici, sono derisi da chi vive nelle zone infestate da queste creature, che non hanno una storia di inquinamento magico.\\


\medskip\index{Mostri - Protoplasma Nero}\textbf{Protoplasma Nero}

\emph{Grande melma, disallineato}

\textbf{FORZA} +3

\textbf{DESTREZZA} -3

\textbf{COSTITUZIONE} +3

\textbf{INTELLIGENZA} -5

\textbf{SAGGEZZA} -2

\textbf{CARISMA} -5

\textbf{Iniziativa} -3 -- \textbf{Difesa} 9

\textbf{Punti Ferita} 85 (10d10 + 30)

\textbf{Movimento} 6 m, scalata 6 m

\textbf{Tiri Salvezza}: Tempra +9, Riflessi -2, Volontà -2

\textbf{Immunità al Danno} acido, freddo, fulmine, tagliente

\textbf{Immunità alle Condizioni} accecato, affascinato, assordato, prono, affaticamento, spaventato

\textbf{Sensi} vista cieca 18 m (cieco oltre questo raggio)

\textbf{Linguaggi} -

\textbf{Sfida} 4 (1.100 PE)

\emph{\textbf{Amorfo.}} Il protoplasma nero può muoversi attraverso uno spazio fino a 3 centimetri di larghezza senza doversi stringere.

\emph{\textbf{Forma Corrosiva.}} Una creatura che entri a contatto col protoplasma nero o lo colpisca con un attacco da mischia mentre si trova entro 1 metro da esso subisce 4 (1d8) danni da acido. Qualsiasi arma non magica fatta di metallo o legno che colpisca il protoplasma nero si corrode. Dopo aver inflitto il danno, l'arma subisce una penalità permanente e cumulativa di -1 ai tiri di danno. Se la penalità arriva a -5, l'arma è distrutta. Le munizioni non magiche fatte di metallo o legno che colpiscano il protoplasma nero, si distruggono dopo aver inflitto il danno.

Il protoplasma nero può divorare legno o metallo non magico dello spessore di 5 centimetri in un 1 round.

\emph{\textbf{Natura di Melma.}} Il protoplasma nero non necessita di dormire.

\emph{\textbf{Scalare come Ragno.}} Il protoplasma nero può scalare superfici difficili, compreso lo stare a testa in giù sul soffitto, senza bisogno di effettuare una prova di abilità.

\textbf{Azioni}

\emph{\textbf{Pseudopodo.} Attacco con arma da mischia}: +5 a colpire, portata 1 m, un bersaglio.

\emph{Colpisce:} 6 (1d6 + 3) danni da botta più 18 (4d8) danni da acido. Inoltre, un'armatura non magica indossata dal bersaglio viene parzialmente dissolta e subisce una penalità permanente e cumulativa di -1 alla Difesa che offre. L'armatura è distrutta se la penalità riduce la sua Difesa a 10.

\textbf{Reazioni}

\emph{\textbf{Divisione.}} Quando un protoplasma nero di taglia Media o più grande subisce danni da fulmine o taglienti, si divide in due nuovi protoplasma neri di almeno 10 punti ferita ciascuno. Ogni nuovo protoplasma nero ha un numero di punti ferita pari alla metà del protoplasma nero originale, arrotondati per difetto. I nuovi protoplasmi neri sono di una taglia più piccola di quella originale.

\textbf{Ecologia}\\
Ambiente: Qualsiasi sotterraneo\\
Organizzazione: Solitario\\
Tesoro: Nessuno\\
\textbf{Descrizione}\\
I protoplasmi neri sono gli spazzini del mondo sotterraneo, costantemente alla ricerca di cibo. Possono percepire corpi organici o metallici nel raggio di 18 metri e attaccano in modo istintivo tali oggetti o esseri finché non li dissolvono, o finché la melma non viene uccisa. Un protoplasma nero si riproduce staccando un pezzo del proprio corpo e formando un nuovo protoplasma più piccolo che raggiunge l'età adulta nel giro di un mese. Alcune tra le creature più intelligenti nel mondo sotterraneo usano i protoplasmi neri per smaltire in modo naturale la spazzatura, creando cave di pietra atte ad ospitare il protoplasma, per poi gettarvi i rifiuti organici o i nemici.\\
Gli esemplari più grandi di protoplasmi neri sono stati avvistati nelle regioni più profonde del mondo: individui Mastodontici che possiedono fino a 30 DV. Si dice che esistano anche protoplasmi colorati: alcuni bianchi che vivono nelle zone artiche, marroni nelle paludi e di colore rossiccio che popolano il deserto.\\


\medskip\index{Mostri - Mimic}\textbf{Mimic}

\emph{Media mostruosità (mutaforma), neutrale}

\textbf{FORZA} +3

\textbf{DESTREZZA} +1

\textbf{COSTITUZIONE} +2

\textbf{INTELLIGENZA} -3

\textbf{SAGGEZZA} +1

\textbf{CARISMA} -1

\textbf{Iniziativa} +1 -- \textbf{Difesa} 13

\textbf{Punti Ferita} 58 (9d8 + 18)

\textbf{Movimento} 5 metri

\textbf{Tiri Salvezza}: Tempra +5, Riflessi +5, Volontà +6

\textbf{Competenze} Muoversi Silenziosamente / Nascondersi +5

\textbf{Immunità al Danno} acido

\textbf{Immunità alle Condizioni} prono

\textbf{Sensi} scurovisione 18 m

\textbf{Linguaggi} -

\textbf{Sfida} 2 (450 PE)

\emph{\textbf{Aderente (Solo Forma di Oggetto).}} Il mimic aderisce a qualsiasi cosa con cui entri in contatto. Una creatura di taglia Enorme o inferiore a cui il mimic aderisce è considerata afferrata da esso (CD 13 per fuggire). Le prove di caratteristica effettuare per fuggire da
questo afferrare hanno -1d6.

\emph{\textbf{Afferratore.}} Il mimic ha +1d6 ai tiri per colpire contro una creatura da esso afferrata.

\emph{\textbf{Falso Aspetto (Solo Forma di Oggetto).}} Mentre il mimic rimane immobile, è indistinguibile da un comune oggetto.

\emph{\textbf{Mutaforma.}} Il mimic può usare la sua azione per trasformarsi in un oggetto, o per tornare alla sua vera forma amorfa. Le sue statistiche sono le stesse in qualsiasi forma. Qualsiasi equipaggiamento stia indossando o trasportando non si trasforma. Alla morte ritorna al suo vero aspetto.

\textbf{Azioni}

\emph{\textbf{Morso.} Attacco con arma da mischia}: +5 a colpire, portata 1 m, un bersaglio.

\emph{Colpisce:} 7 (1d8 + 3) danni perforanti più 4 (1d8) danni da acido.

\emph{\textbf{Pseudopodo.} Attacco con arma da mischia}: +5 a colpire, portata 1 m, un bersaglio.

\emph{Colpisce:} 7 (1d8 + 3) danni da botta. Se il mimic è in forma di oggetto, il bersaglio è vittima del tratto Aderente.

\textbf{Ecologia}
Ambiente: Qualsiasi\\
Organizzazione: Solitario\\
Tesoro: Accidentale\\
\textbf{Descrizione}\\
Si ritiene che i mimic siano il risultato del tentativo di un alchimista di dar vita ad un oggetto inanimato attraverso l'applicazione di un reagente mistico, la cui formula è andata perduta. Nel corso degli anni, queste creature strane ma intelligenti hanno appreso la capacità di trasformarsi in simulacri degli oggetti manufatti, in particolare nei luoghi frequentati poco da un ristretto numero di creature, dove aumentano le loro probabilità di successo con un attacco alle loro vittime.\\
Anche se i mimic non sono intrinsecamente malvagi, alcuni saggi suggeriscono che attacchino gli uomini e le altre creature intelligenti più per passatempo che per sfamarsi. Il desiderio di ingannare gli altri è parte del loro essere, e i loro attacchi a sorpresa rappresentano il culmine di questo desiderio.\\
Un tipico mimic ha un volume di 2 metri cubi (1 m per 1 m per 2 m) e pesa circa 450 kg. Leggende e storie parlano di mimic di taglie maggiori, con la capacità di assumere la forma di case, navi o interi complessi sotterranei che guarniscono con dei tesori (sia veri che falsi) per attirare al loro interno il loro ignaro cibo.\\


\medskip\index{Mostri - Minotauro}\textbf{Minotauro}

\emph{Grande mostruosità, caotico malvagio}

\textbf{FORZA} +4

\textbf{DESTREZZA} +0

\textbf{COSTITUZIONE} +3

\textbf{INTELLIGENZA} -2

\textbf{SAGGEZZA} +3

\textbf{CARISMA} -1

\textbf{Iniziativa} +0 -- \textbf{Difesa} 16

\textbf{Punti Ferita} 76 (9d10 + 27)

\textbf{Movimento} 12 m

\textbf{Tiri Salvezza}: Tempra +6, Riflessi +5, Volontà +5

\textbf{Competenze} Consapevolezza +7

\textbf{Sensi} scurovisione 18 m

\textbf{Linguaggi} Abissale

\textbf{Sfida} 3 (700 PE)

\emph{\textbf{Carica.}} Se il minotauro si muove di almeno 3 metri diretto verso un bersaglio e lo colpisce con un attacco di incornata durante lo stesso turno, il bersaglio subisce 9 (2d8) danni perforanti aggiuntivi. Se il bersaglio è una creatura, deve riuscire un Tiro Salvezza su Tempra CD 14 o venire spinto via fino a 3 metri di distanza e cadere prono.

\emph{\textbf{Incauto.}} All'inizio del suo turno, il minotauro può ottenere +1d6 su tutti i tiri per colpire con armi da mischia effettuati durante quel turno, ma i tiri per colpire contro di esso hanno +1d6 fino all'inizio del suo prossimo turno.

\emph{\textbf{Ricordare Labirinto.}} Il minotauro può ricordare perfettamente qualsiasi tragitto abbia percorso.

\textbf{Azioni}

\emph{\textbf{Ascia Bipenne.} Attacco con arma da mischia}: +6 a colpire, portata 1 m, un bersaglio.

\emph{Colpisce:} 17 (2d12 + 4) danni taglienti.

\emph{\textbf{Incornata.} Attacco con arma da mischia}: +6 a colpire, portata 1 m, un bersaglio.

\emph{Colpisce:} 13 (2d8 + 4) danni perforanti.

\textbf{Ecologia}\\
Ambiente: Rovine Temperate e Sotterranei\\
Organizzazione: Solitario, coppia o gruppo (3-4)\\
Tesoro: Standard (Ascia Bipenne, altro tesoro)\\
\textbf{Descrizione}\\
Nessuno porta rancore come un minotauro. Disprezzati dalle razze civilizzate e nati secoli fa da una maledizione divina, i minotauri hanno cacciato, ucciso e divorato gli umanoidi inferiori per punire offese vere o presunte più a lungo di quanto riescono a ricordare. La maggior parte delle culture ha leggende su come i minotauri furono creati da divinità vendicative o offese che punirono gli umani deformando le loro sembianze, sottraendo loro bellezza e intelligenza, e dotandoli di teste di toro. Eppure la maggioranza dei minotauri moderni disprezza queste leggende e non crede di essere lo scherzo di qualche divinità, ma modelli di perfezione divina creati dal crudele e potente signore dei demoni Baphomet.\\
I nascondigli tradizionali dei minotauri sono i labirinti, sia i dedali costruiti per confondere e sconcertare, sia quelli naturali creati da un intrico di caverne o altri passaggi sotterranei. Grazie alla loro astuzia naturale, i minotauri usano i loro nascondigli labirintici per scoraggiare gli incauti nemici che cercano di scovarli o che semplicemente incappano nei loro nascondigli e si perdono, dando lentamente la caccia agli intrusi che cercano inutilmente di trovare una via d'uscita. Solo quando la disperazione ha nettamente preso il sopravvento il minotauro colpisce le sue perdute vittime. Quando hanno a che fare con un gruppo, spesso i minotauri lasciano scappare uno creatura, affinché diffonda il suo terribile racconto e attiri altri, che sperano di uccidere queste bestie, nei loro labirinti. Naturalmente, per i minotauri, questi aspiranti eroi rappresentano delle pietanze deliziose.\\
I minotauri si possono trovare anche al servizio di un mostro o una creatura malvagia più potente, e lo servono fintanto che possono cacciare e mangiare a loro piacimento. Generalmente questo significa fare la guardia a qualche potente oggetto o preziosa locazione, ma può anche significare lavorare come mercenario, dando la caccia ai nemici del padrone.\\
I minotauri sono combattenti relativamente diretti, usando le loro corna per incornare orribilmente le creature viventi più vicine quando cominciano a combattere.\\


\subsection{Mummie}

\medskip\index{Mostri - Mummia}\textbf{Mummia}

\emph{Media non morto, legale malvagio}

\textbf{FORZA} +3

\textbf{DESTREZZA} -1

\textbf{COSTITUZIONE} +2

\textbf{INTELLIGENZA} -2

\textbf{SAGGEZZA} +0

\textbf{CARISMA} +1

\textbf{Iniziativa} -1 -- \textbf{Difesa} 13

\textbf{Punti Ferita} 58 (9d8 + 18)

\textbf{Movimento} 6 m

\textbf{Tiri Salvezza}: Tempra +4, Riflessi +2, Volontà +8

\textbf{Vulnerabilità al Danno} fuoco

\textbf{Resistenze al Danno} da botta, perforante e tagliente di attacchi non magici

\textbf{Immunità al Danno} da Vuoto, veleno

\textbf{Immunità alle Condizioni} affascinato, avvelenato, paralizzato, affaticamento, spaventato

\textbf{Sensi} scurovisione 18 m 

\textbf{Linguaggi} le lingue che conosceva in vita

\textbf{Sfida} 3 (700 PE)

\emph{\textbf{Natura Non Morta.}} Un mummia non ha bisogno di aria, cibo, bevande o sonno.

\textbf{Azioni}

\emph{\textbf{Multiattacco.}} La mummia può usare la sua Occhiata Temibile ed effettuare un attacco con il pugno putrefacente.

\emph{\textbf{Pugno Putrefacente.} Attacco con arma da mischia}: +5 a colpire, portata 1 m, un bersaglio.

\emph{Colpisce:} 10 (2d6 + 3) danni da botta più 10 (3d6) danni da Vuoto. Se il bersaglio è una creatura deve riuscire un Tiro Salvezza su Tempra 12 o venire maledetto dalla putrefazione della mummia. Il bersaglio maledetto non può recuperare punti ferita, e i suoi punti ferita massimi diminuiscono di 10 (3d6) ogni 24 ore di durata della maledizione. Se la maledizione riduce i punti ferita massimi del bersaglio a 0, il bersaglio muore, e il suo corpo si tramuta in polvere. La maledizione dura finché non viene rimossa dall'incantesimo \emph{rimuovi maledizione} o altra magia.

\emph{\textbf{Occhiata Temibile.}} La mummia prende a bersaglio una creatura che possa vedere e si trovi entro 18 metri da lei. Se il bersaglio può vedere la mummia, deve riuscire un Tiro Salvezza su Volontà CD 11 contro questa magia o restare spaventato fino al termine del prossimo turno della mummia. Se il bersaglio fallisce il Tiro Salvezza di 5 o più, è anche paralizzato per la stessa durata. Un bersaglio che riesca il Tiro Salvezza è immune all'Occhiata Terribile di tutte le mummie (ma non delle mummie sovrane) per le successive 24 ore.

\medskip\index{Mostri - Mummia Sovrana}\textbf{Mummia Sovrana}

\emph{Media non morto, legale malvagio}

\textbf{FORZA} +4

\textbf{DESTREZZA} +0

\textbf{COSTITUZIONE} +3

\textbf{INTELLIGENZA} +0

\textbf{SAGGEZZA} +4

\textbf{CARISMA} +3

\textbf{Iniziativa} +0 -- \textbf{Difesa} 25

\textbf{Punti Ferita} 97 (13d8 + 39)

\textbf{Movimento} 6 m

\textbf{Tiri Salvezza}: Tempra +12, Riflessi +6, Volontà +16

\textbf{Competenze} Religione +5, Storia +5

\textbf{Vulnerabilità al Danno} fuoco

\textbf{Immunità al Danno} da Vuoto, veleno; armi +1

\textbf{Immunità alle Condizioni} affascinato, avvelenato, paralizzato, affaticamento, spaventato

\textbf{Sensi} scurovisione 18 m

\textbf{Linguaggi} le lingue che conosceva in vita

\textbf{Sfida} 15 (13.000 PE)

\emph{\textbf{Cuore della Mummia Sovrana.}} Come parte del rituale che crea una mummia sovrana, il cuore e le viscere della creatura vengono rimossi dal cadavere e piazzati all'interno di contenitori sigillati. Questi contenitori sono di solito fatti in pietra o ceramica, incisi o dipinti con geroglifici religiosi.

Finché il suo cuore avvizzito rimane intatto, la mummia sovrana non può essere permanentemente distrutta. Quando scende a 0 punti ferita, la mummia sovrana si riduce in polvere e si riforma a piena forza 24 ore più tardi, riemergendo dalla polvere in prossimità della giara sigillata che contiene il suo cuore. Per impedire che una mummia sovrana si riformi e distruggerla una volta per tutte, bisogna ridurne il cuore in cenere. Per questo motivo, la mummia sovrana di solito tiene il cuore e le viscere nascoste all'interno di una tomba nascosta.

Il cuore della mummia sovrana ha Difesa 5, 25 punti ferita e immunità a tutti i danni eccetto il fuoco.

\emph{\textbf{Incantesimi.}} La mummia ha CM 10. La sua caratteristica da incantatore è la Saggezza, +9 a colpire con attacchi da incantesimo. La mummia ha preparati i seguenti incantesimi: Trucchetti (a volontà): \emph{fiamma sacra, taumaturgia}

Difficoltà 16 (4 slot): \emph{comando, dardo tracciante, scudo della fede}

Difficoltà 19 (3 slot): \emph{arma spirituale, blocca persone, silenzio}

Difficoltà 21 (3 slot): \emph{animare morti, dissolvi magie}

Difficoltà 23 (3 slot): \emph{divinazione, guardiano della fede}

Difficoltà 26 (2 slot): \emph{contagio, piaga degli insetti}

Difficoltà 29 (1 slot): \emph{ferire}

\emph{\textbf{Natura Non Morta.}} Un mummia non ha bisogno di aria, cibo, bevande o sonno.

\emph{\textbf{Resistenza alla Magia.}} La mummia sovrana ha +1d6 ai Tiri Salvezza contro incantesimi o altri effetti magici.

\emph{\textbf{Rinvigorimento.}} Una mummia sovrana forma un nuovo corpo entro 24 ore se il suo cuore resta intatto, recuperando tutti i punti ferita e potendo agire nuovamente. Il nuovo corpo compare entro 1 metro dal cuore della mummia sovrana.

\textbf{Azioni}

\emph{\textbf{Multiattacco.}} La mummia può usare la sua Occhiata Temibile ed effettuare un attacco con il pugno putrefacente.

\emph{\textbf{Pugno Putrefacente.} Attacco con arma da mischia}: +9 a colpire, portata 1 m, un bersaglio.

\emph{Colpisce:} 14 (3d6 + 4) danni da botta più 21 (6d6) danni da Vuoto. Se il bersaglio è una creatura deve riuscire un Tiro Salvezza su Tempra 16 o venire maledetto dalla putrefazione della mummia. Il bersaglio maledetto non può recuperare punti ferita, e i suoi punti ferita massimi diminuiscono di 10 (3d6) ogni 24 ore di durata della maledizione. Se la maledizione riduce i punti ferita massimi del bersaglio a 0, il bersaglio muore, e il suo corpo si tramuta in polvere. Lamaledizione dura finché non viene rimossa dall'incantesimo  \emph{rimuovere maledizione} o altra magia.

\emph{\textbf{Occhiata Temibile.}} La mummia prende a bersaglio una creatura che possa vedere e si trovi entro 18 metri da lei. Se il bersaglio può vedere la mummia, deve riuscire un Tiro Salvezza su Volontà CD 16 contro questa magia o restare spaventato fino al termine del prossimo turno della mummia. Se il bersaglio fallisce il Tiro Salvezza di 5 o più, è anche paralizzato per la stessa durata. Un bersaglio che riesca il Tiro Salvezza è immune all'Occhiata Terribile di tutte le mummie (ma non delle mummie sovrane) per le successive 24 ore.

\textbf{Azioni Aggiuntive}

La mummia sovrana può effettuare 3 Azioni aggiuntive, scelte tra le opzioni seguenti. Può usare solo un'opzione leggendaria alla volta e solo al termine del turno di un'altra creatura. La mummia sovrana recupera le Azioni aggiuntive spese all'inizio del proprio turno.

\emph{\textbf{Attaccare.}} La mummia sovrana effettua un attacco con il pugno putrefacente o usa la sua Occhiata Temibile.

\emph{\textbf{Incanalare Energia Negativa (Costa 2 Azioni).}} La mummia sovrana può scatenare magicamente l'energia negativa. Le creature entro 18 metri dalla mummia sovrana, comprese quelle dietro barriere o angoli, non possono recuperare punti ferita fino al termine del prossimo turno della mummia sovrana.

\emph{\textbf{Parola Blasfema (Costa 2 Azioni).}} La mummia sovrana pronuncia una parola blasfema. Ciascuna creatura, esclusi i non morti, entro 3 metri dalla mummia sovrana e che possa udire questa frase magica deve riuscire un Tiro Salvezza di Tempra CD 16 o restare stordita fino al termine del prossimo turno della mummia sovrana.

\emph{\textbf{Polvere Accecante.}} Polvere e sabbia accecanti turbinano magicamente intorno alla mummia sovrana. Ogni creatura entro 1 metro dalla mummia sovrana deve riuscire un Tiro Salvezza di Tempra CD 16 o restare accecata fino al termine del prossimo turno della creatura.

\emph{\textbf{Turbine di Sabbia (Costa 2 Azioni).}} La mummia sovrana può trasformarsi magicamente in un turbine di sabbia, muovendosi di massimo 18 metri, e tornando poi alla sua forma normale. Mentre è in forma di turbine, la mummia sovrana è immune a tutti i danni, e non può essere afferrata, pietrificata, gettata prona, intralciata o stordita. L'equipaggiamento indossato o trasportato dalla mummia sovrana rimane in suo possesso.

\subsection{Naga}

\medskip\index{Mostri - Naga Guardiano}\textbf{Naga Guardiano}

\emph{Grande mostruosità, legale buono}

\textbf{FORZA} +4

\textbf{DESTREZZA} +4

\textbf{COSTITUZIONE} +3

\textbf{INTELLIGENZA} +3

\textbf{SAGGEZZA} +4

\textbf{CARISMA} +4

\textbf{Iniziativa} +4 -- \textbf{Difesa} 23

\textbf{Punti Ferita} 127 (15d10 + 45)

\textbf{Movimento} 12 m

\textbf{Tiri Salvezza}: Tempra +9, Rif +12, Volontà +12

\textbf{Immunità ai Danni} veleno

\textbf{Immunità alle Condizioni} affascinato, avvelenato 

\textbf{Sensi} scurovisione 18 m 

\textbf{Linguaggi} Celestiale, Comune 

\textbf{Sfida} 10 (5.900 PE)

\emph{\textbf{Incantesimi.}} Il naga ha CM 11. La sua caratteristica da incantatore è la Saggezza (+8 a colpire con attacchi con incantesimo), e ha bisogno solo delle componenti verbali per lanciare i suoi incantesimi. Il naga prepara i seguenti incantesimi:

Trucchetti (a volontà): \emph{fiamma sacra, riparare, taumaturgia}

Difficoltà 16 (4 slot): \emph{comando, cura ferite, scudo della fede}

Difficoltà 19 (3 slot): \emph{bloccare persone, calmare emozioni}

Difficoltà 21 (3 slot): \emph{chiaroveggenza, scagliare maledizione}

Difficoltà 23 (3 slot): \emph{esilio, libertà di movimento}

Difficoltà 26 (2 slot): \emph{colpo infuocato, costrizione}

Difficoltà 29 (1 slot): \emph{visione del vero}

\emph{\textbf{Rinvigorimento.}} Se muore, il naga ritorna in vita in 1d6 giorni e recupera tutti i suoi punti ferita. Solo l'incantesimo \emph{desiderio} può impedire a questo tratto di funzionare.

\textbf{Azioni}

\emph{\textbf{Morso.} Attacco con arma da mischia}: +8 a colpire, portata 3 m, una creatura.

\emph{Colpisce:} 8 (1d8 + 4) danni perforanti, e il bersaglio deve effettuare un Tiro Salvezza di Tempra CD 15, subendo 45 (10d8) danni da veleno se fallisce il Tiro Salvezza, o la metà di questi danni se lo riesce.

\emph{\textbf{Sputare Veleno.} Attacco con arma a Distanza}: +8 a colpire, gittata 5m, una creatura.

\emph{Colpisce:} Il bersaglio deve effettuare un Tiro Salvezza su Tempra CD 15, subendo 45 (10d8) danni da veleno se fallisce il Tiro Salvezza, o la metà di questi danni se lo riesce.

\textbf{Ecologia}\\
Ambiente: Pianure Temperate\\
Organizzazione: Solitario, coppia o nido (3-6)\\
Tesoro: Standard\\
\textbf{Descrizione}\\
Sebbene abbiano un aspetto feroce, con scaglie brillanti, cappucci simili a quelli dei cobra e potenti corpi serpentini, i naga guardiani fungono da coscienziosi protettori di luoghi di eccezionale potere e sacralità. Spesso le loro scaglie sfoggiano disegni elaborati simili a quelli degli esotici serpenti della giungla. Un tipico naga guardiano raggiunge la lunghezza di 4,2 metri e un peso approssimativo di 175 kg.\\
Mentre alcuni naga guardiani aderiscono a pratiche esotiche di divinità antiche o dimenticate, altri sono semplicemente attratti da siti dalla spiccata bellezza naturale, quali templi su imponenti cascate, pinnacoli naturali e cime di montagne, custodendoli con il massimo della reverenza e del senso del dovere. Spesso questi naga si uniscono a fedi ancora attive, servendo come protettori di santuari o antichi tesori. Una coppia di naga può stabilirsi nei pressi di un sito che ritengono meritevole di protezione, covandovi una nidiata e crescendovi la prole. Quando i giovani raggiungono l'età adulta, possono scegliere di partire per cercare la propria casa o rimanere a proteggere la zona sorvegliata dai loro genitori. A volte, un naga guardiano che custodisce delle rovine od un tempio è solo l'ultimo di una successione di sentinelle che si sono avvicendate nel corso dei secoli. Queste sentinelle spesso prendono lo stesso nome dei loro predecessori sembrando un unico individuo eccezionalmente longevo.\\


\medskip\index{Mostri - Naga Spirituale}\textbf{Naga Spirituale}

\emph{Grande mostruosità, caotico malvagio}

\textbf{FORZA} +4

\textbf{DESTREZZA} +3

\textbf{COSTITUZIONE} +2

\textbf{INTELLIGENZA} +3

\textbf{SAGGEZZA} +2

\textbf{CARISMA} +3

\textbf{Iniziativa} +3 -- \textbf{Difesa} 19

\textbf{Punti Ferita} 75 (10d10 + 20)

\textbf{Movimento} 12 m

\textbf{Tiri Salvezza}: Tempra +8, Rif +10, Volontà +10

\textbf{Immunità al Danno} veleno

\textbf{Immunità alle Condizioni} affascinato, avvelenato

\textbf{Sensi} scurovisione 18 m

\textbf{Linguaggi} Abissale, Comune

\textbf{Sfida} 8 (3.900 PE)

\emph{\textbf{Incantesimi.}} Il naga ha CM 10. La sua abilità da incantatore è l'Intelligenza (+6 a colpire con attacchi con incantesimo), e ha bisogno solo delle componenti verbali per eseguire i suoi incantesimi. Il naga prepara i seguenti incantesimi:

Trucchetti (a volontà): \emph{illusione minore, mano magica, raggio di}
\emph{gelo}

Difficoltà 16 (4 slot): \emph{charme su persone, individuazione del
	magico,} \emph{sonno}

Difficoltà 19 (3 slot): \emph{blocca persone, individuazione dei pensieri}

Difficoltà 21 (3 slot): \emph{fulmine, respirare sott'acqua}

Difficoltà 23 (3 slot): \emph{inaridire, porta dimensionale}

Difficoltà 26 (2 slot): \emph{dominare persone}

\emph{\textbf{Rinvigorimento.}} Se muore, il naga ritorna in vita in 1d6 giorni e recupera tutti i suoi punti ferita. Solo l'incantesimo \emph{desiderio} può impedire a questo tratto di funzionare.

\textbf{Azioni}

\emph{\textbf{Morso.} Attacco con arma da mischia}: +7 a colpire, portata 3 m, una creatura.

\emph{Colpisce:} 7 (1d8 + 4) danni perforanti, e il bersaglio deve effettuare un Tiro Salvezza di Tempra CD 13, subendo 31 (7d8) danni da veleno se fallisce il Tiro Salvezza, o la metà di questi danni se lo riesce.

\subsection{Oggetti Animati}

\medskip\index{Mostri - Armatura Animata}\textbf{Armatura Animata}

\emph{Media costrutto, disallineato}

\textbf{FORZA} +2

\textbf{DESTREZZA} +0

\textbf{COSTITUZIONE} +1

\textbf{INTELLIGENZA} -5

\textbf{SAGGEZZA} -4

\textbf{CARISMA} -5

\textbf{Iniziativa} +0 -- \textbf{Difesa} 19

\textbf{Punti Ferita} 33 (6d8 + 6)

\textbf{Movimento} 7 m

\textbf{Tiri Salvezza}: Tempra +2, Rif +0, Volontà -4

\textbf{Immunità al Danno} psichico, veleno

\textbf{Immunità alle Condizioni} accecato, affascinato, assordato, avvelenato, paralizzato, pietrificato, affaticamento, spaventato

\textbf{Sensi} vista cieca 18 m (cieco oltre questo raggio)

\textbf{Linguaggi} -

\textbf{Sfida} 1 (200 PE)

\emph{\textbf{Falso Aspetto.}} Mentre l'armatura rimane immobile, è indistinguibile da una normale armatura.

\emph{\textbf{Suscettibilità all'Anti Magia.}} L'armatura è inabile se si trova nell'area di un \emph{campo anti-magia}. Se è bersaglio di \emph{dissolvi} \emph{magie}, l'armatura deve riuscire un Tiro Salvezza su Tempra contro la CD del Tiro Salvezza dell'incantesimo o restare svenuta per 1 minuto.

\textbf{Azioni}

\emph{\textbf{Multiattacco.}} L'armatura effettua due attacchi da mischia.

\emph{\textbf{Schianto.} Attacco con arma da mischia}: +4 a colpire, portata 1 m, un bersaglio.

\emph{Colpisce:} 5 (1d6 + 2) danni da botta.

\medskip\index{Mostri - Spada Volante}\textbf{Spada Volante}

\emph{Piccola costrutto, disallineato}

\textbf{FORZA} +1

\textbf{DESTREZZA} +2

\textbf{COSTITUZIONE} +0

\textbf{INTELLIGENZA} -5

\textbf{SAGGEZZA} -3

\textbf{CARISMA} -5

\textbf{Iniziativa} +2 -- \textbf{Difesa} 18

\textbf{Punti Ferita} 17 (5d6)

\textbf{Movimento} 0 m, volo 15 m (fluttua)

\textbf{Tiri Salvezza}  Tempra +1, Riflessi +3, Volontà -4

\textbf{Immunità al Danno} psichico, veleno

\textbf{Immunità alle Condizioni} accecato, affascinato, assordato, avvelenato, paralizzato, pietrificato, spaventato

\textbf{Sensi} vista cieca 18 m (cieco oltre questo raggio)

\textbf{Linguaggi} -

\textbf{Sfida} 1/4 (50 PE)

\emph{\textbf{Falso Aspetto.}} Mentre l'arma rimane immobile e non sta volando, è indistinguibile da una normale spada.

\emph{\textbf{Suscettibilità all'Anti Magia.}} La spada è inabile se si trova nell'area di un \emph{campo anti-magia}. Se è bersaglio di \emph{dissolvi} \emph{magie}, la spada deve riuscire un Tiro Salvezza su Tempra contro la CD del Tiro Salvezza dell'incantesimo o restare svenuta per 1 minuto.

\textbf{Azioni}

\emph{\textbf{Spada Lunga.} Attacco con arma da mischia}: +3 a colpire, portata 1 m, un bersaglio.

\emph{Colpisce:} 5 (1d8 + 1) danni taglienti.


\medskip\index{Mostri - Tappeto del Soffocamento}\textbf{Tappeto del Soffocamento}

\emph{Grande costrutto, disallineato}

\textbf{FORZA} +3

\textbf{DESTREZZA} +2

\textbf{COSTITUZIONE} +0

\textbf{INTELLIGENZA} -5

\textbf{SAGGEZZA} -4

\textbf{CARISMA} -5

\textbf{Iniziativa} +2 -- \textbf{Difesa} 13

\textbf{Punti Ferita} 33 (6d10)

\textbf{Movimento} 3 m

\textbf{Tiri Salvezza}: Tempra +4, Riflessi +2, Volontà -4

\textbf{Immunità al Danno} psichico, veleno

\textbf{Immunità alle Condizioni} accecato, affascinato, assordato, avvelenato, paralizzato, pietrificato, spaventato

\textbf{Sensi} vista cieca 18 m (cieco oltre questo raggio)

\textbf{Linguaggi} -

\textbf{Sfida} 2 (450 PE)

\emph{\textbf{Falso Aspetto.}} Mentre il tappeto resta immobile, è indistinguibile da un normale tappeto.

\emph{\textbf{Suscettibilità all'Anti Magia.}} Il tappeto è inabile mentre si trova nell'area di un \emph{campo anti-magia}. Se è il bersaglio di \emph{dissolvi} \emph{magie}, il tappeto deve riuscire un Tiro Salvezza di Tempra contro la CD del Tiro Salvezza dell'incantatore o cadere privo di sensi per 1 minuto.

\emph{\textbf{Trasferimento di Danno.}} Mentre afferra una creatura, il tappeto subisce solo la metà dei danni che gli sono inferti, e la creatura afferrata dal tappeto subisce l'altra metà.

\textbf{Azioni}

\emph{\textbf{Soffocare.} Attacco con arma da mischia}: +5 a colpire, portata 1 m, una creatura di taglia Media o inferiore.

\emph{Colpisce:} La creatura è afferrata (CD 13 per fuggire). Fino al termine dell'afferrare, il bersaglio è intralciato, accecato e rischia di soffocare, ma il tappeto non può soffocare un altro bersaglio. Inoltre, all'inizio di ciascun turno del bersaglio, il bersaglio subisce 10 (2d6 + 3) danni da botta.

\medskip\index{Mostri - Ogre}\textbf{Ogre}

\emph{Grande gigante, caotico malvagio}

\textbf{FORZA} +4

\textbf{DESTREZZA} -1

\textbf{COSTITUZIONE} +3

\textbf{INTELLIGENZA} -3

\textbf{SAGGEZZA} -2

\textbf{CARISMA} -2

\textbf{Iniziativa} -1 -- \textbf{Difesa} 12 (armatura di pelle)

\textbf{Punti Ferita} 59 (7d10 + 21)

\textbf{Movimento} 12 m

\textbf{Tiri Salvezza}: Tempra +6, Riflessi +0, Volontà +1

\textbf{Sensi} scurovisione 18 m

\textbf{Linguaggi} Comune, Gigante

\textbf{Sfida} 2 (450 PE)

\textbf{Azioni}

\emph{\textbf{Randello Pesante.} Attacco con arma da mischia}: +6 a colpire, portata 1 m, un bersaglio.

\emph{Colpisce:} 13 (2d8 + 4) danni da botta. 

\emph{\textbf{Giavellotto.} Attacco con arma da mischia o a Distanza}: +6 a colpire, portata 1 m o gittata 9m, un bersaglio. 

\emph{Colpisce:} 11 (2d6 + 4) danni perforanti.

\textbf{Ecologia}\\
Ambiente: Colline fredde o temperate\\
Organizzazione: Solitario, coppia, gruppo (3-4) o famiglia (5-16)\\
Tesoro: Standard (Armatura di Pelle, Randello Pesante, 4 Giavellotti, altro)\\
\textbf{Descrizione}\\
Nelle storie riguardanti gli ogre ci sono elementi orrendi: brutalità e ferocia, cannibalismo e tortura. Poi stupri, smembramenti, necrofilia, incesto, mutilazioni e altri esempi di crudeltà. Coloro che non hanno mai incontrato gli ogre ritengono queste storie un avvertimento. Chi è sopravvissuto ad un simile incontro sa che le storie sono niente in confronto alla realtà.\\

Gli ogre godono della sofferenza altrui. Se non hanno a disposizione le razze più piccole da schiacciare fra le loro grasse mani o da violare in amplessi violenti, si divertono fra loro. Per gli ogre non esiste tabù. Si potrebbe pensare che, lasciata a sé stessa, una tribù di ogre si farebbe a pezzi da sola e che soltanto i più forti sopravvivrebbero: se c'è una cosa che gli ogre rispettano, però, è la famiglia.\\

Le tribù ogre sono conosciute come famiglie, e molte delle loro deformità sono causate dalla pratica comune dell'incesto. Il capo della tribù è spesso il padre, ma in alcuni casi un'ogre femmina è in grado di reclamare il titolo di madre. Le tribù ogre litigano fra loro, cosa che li tiene impegnati ed impedisce loro di tormentare i loro vicini. Di quando in quando, però, emerge un patriarca particolarmente violento o temuto, capace di unire più famiglie sotto il suo comando.\\

Le regioni abitate degli ogre sono luoghi tristi e degradati, dato che questi giganti vivono nello squallore e non sentono il bisogno di essere in armonia con quanto li circonda. Il confine fra le terre civilizzate e quelle degli ogre è un luogo di disperazione abitato da reietti, dove vivono gli Ogremanni, progenie deformi che nascono dalle razzie che gli ogre effettuano nelle terre degli umani.\\

I giochi degli ogre sono violenti e crudeli: le vittime utilizzate come giocattolo sono fortunate a morire il primo giorno. Il crudele senso dell'umorismo degli ogre è il solo caso in cui mostrano di possedere creatività: i metodi e gli strumenti di tortura ogre sembrano usciti dagli incubi.\\

La grande forza e la mancanza di immaginazione li rendono particolarmente adatti ai lavori pesanti, nelle miniere, come fabbri o nel disboscamento. I giganti più potenti (soprattutto quelli delle Colline e delle Rocce) spesso soggiogano le famiglie ogre perché diventino loro servitori.\\
Un ogre adulto è alto sui 3 metri e pesa circa 325 kg.\\


\medskip\index{Mostri - Ombra}\textbf{Ombra}

\emph{Media non morto, caotico malvagio}

\textbf{FORZA} -2

\textbf{DESTREZZA} +2

\textbf{COSTITUZIONE} +1

\textbf{INTELLIGENZA} -2

\textbf{SAGGEZZA} +0

\textbf{CARISMA} -1

\textbf{Iniziativa} +2 -- \textbf{Difesa} 13

\textbf{Punti Ferita} 16 (3d8 + 3)

\textbf{Movimento} 12 m

\textbf{Tiri Salvezza}: Tempra +3, Riflessi +3, Volontà +4

\textbf{Competenze} Muoversi Silenziosamente / Nascondersi +4 (+6 a luce fioca o oscurità)

\textbf{Vulnerabilità al Danno} da Luce

\textbf{Resistenze al Danno} acido, freddo, fulmine, fuoco, tuono; da botta, perforante e tagliente di attacchi non magici

\textbf{Immunità al Danno} da Vuoto, veleno

\textbf{Immunità alle Condizioni} afferrato, avvelenato, intralciato, paralizzato, pietrificato, prono, affaticamento, spaventato

\textbf{Sensi} scurovisione 18 m

\textbf{Linguaggi} -

\textbf{Sfida} 1/2 (100 PE)

\emph{\textbf{Amorfo.}} L'ombra può muoversi attraverso uno spazio stretto fino a 3 centimetri senza stringersi.

\emph{\textbf{Debolezza alla Luce del Sole.}} Mentre si trova alla luce del sole, l'ombra ha -1d6 ai tiri per colpire, le prove di competenza e i Tiri Salvezza.

\emph{\textbf{Furtività d'Ombra.}} Quando si trova a luce fioca o all'oscurità, l'ombra può effettuare l'azione Nascondersi come azione bonus. 

\emph{\textbf{Natura Non Morta.}} Un'ombra non necessita aria, cibo, bevande o sonno.

\textbf{Azioni}

\emph{\textbf{Risucchio di Forza.} Attacco con arma da mischia}: +4 a colpire, portata 1 m, una creatura.

\emph{Colpisce:} 9 (2d6 + 2) danni da Vuoto, e il punteggio di Forza del bersaglio viene ridotto di 1d4. Il bersaglio muore se ciò riduce la sua Forza a 0. Altrimenti, la riduzione resta finché il bersaglio non riposa 8 ore.

Se un umanoide non malvagio muore a causa di questo attacco, entro 1d4 ore dal suo cadavere si animerà una nuova ombra.

\textbf{Ecologia}
Ambiente: Qualsiasi\\
Organizzazione: Solitario, coppia, gruppo (3–6) o sciame (7–12)\\
Tesoro: Standard\\

\textbf{Descrizione}\\
La malvagia ombra si muove lungo il confine tra il buio delle tenebre e la dura verità della luce. L’ombra preferisce infestare le rovine che la civiltà si lascia alle spalle, dove dà la caccia alle creature viventi tanto sciocche da incappare nel suo territorio. L’ombra è un orribile non morto, e come tale non ha scopi o motivazioni apparenti oltre a risucchiare forza vitale e vitalità dagli esseri viventi.


\medskip\index{Mostri - Omuncolo}\textbf{Omuncolo}

\emph{Minuscola costrutto, neutrale}

\textbf{FORZA} -3

\textbf{DESTREZZA} +2

\textbf{COSTITUZIONE} +0

\textbf{INTELLIGENZA} +0

\textbf{SAGGEZZA} +0

\textbf{CARISMA} -2

\textbf{Iniziativa} +2 -- \textbf{Difesa} 14

\textbf{Punti Ferita} 5 (2d4)

\textbf{Movimento} 6 m, volo 12 m

\textbf{Tiri Salvezza}:  Tempra +0, Riflessi +4, Volontà +1

\textbf{Immunità al Danno} veleno

\textbf{Immunità alle Condizioni} affascinato, avvelenato

\textbf{Sensi} scurovisione 18 m, vista cieca 3 m

\textbf{Linguaggi} comprende le lingue del suo creatore ma non può parlare

\textbf{Sfida} 0 (10 PE)

\emph{\textbf{Legame Telepatico.}} Mentre l'omuncolo si trova sullo stesso piano di esistenza del suo padrone, può comunicare magicamente al suo padrone quello che percepisce, e i due possono comunicare telepaticamente.

\textbf{Azioni}

\emph{\textbf{Morso.} Attacco con arma da mischia}: +4 a colpire, portata 1 m, una creatura.

\emph{Colpisce:} 1 danno perforante, e il bersaglio deve riuscire un Tiro Salvezza di Tempra CD 10 o restare avvelenato per 1 minuto. Se il Tiro Salvezza viene fallito di 5 o più, il bersaglio resta invece avvelenato per 5 (1d10) minuti e mentre è avvelenato in questo modo è anche privo di sensi.

\medskip\index{Mostri - Oni}\textbf{Oni}

\emph{Grande gigante, legale malvagio}

\textbf{FORZA} +4

\textbf{DESTREZZA} +0

\textbf{COSTITUZIONE} +3

\textbf{INTELLIGENZA} +2

\textbf{SAGGEZZA} +1

\textbf{CARISMA} +2

\textbf{Iniziativa} +2 -- \textbf{Difesa} 20 (cotta di maglia)

\textbf{Punti Ferita} 110 (13d10 + 39)

\textbf{Movimento} 9 m, volo 9 m

\textbf{Tiri Salvezza}: Tempra +7, Riflessi +4, Volontà +6

\textbf{Competenze} Arcano +5, Ingannare +8, Consapevolezza +4 

\textbf{Sensi} scurovisione 18 m 

\textbf{Linguaggi} Comune, Gigante

\textbf{Sfida} 7 (2.900 PE)

\emph{\textbf{Armi Magiche.}} Gli attacchi con armi dell'oni sono magici.

\emph{\textbf{Incantesimi Innati.}} La caratteristica da incantatore dell'oni è il Carisma. L'oni può lanciare questi incantesimi in maniera innata, senza bisogno di componenti materiali:

A volontà: \emph{invisibilità, oscurità}

1/giorno: \emph{charme su persone, cono di freddo, forma gassosa,}
\emph{sonno}

\emph{\textbf{Rigenerazione.}} Se ha almeno 1 punto ferita, l'oni recupera 10 punti ferita all'inizio del suo turno.

\textbf{Azioni}

\emph{\textbf{Multiattacco.}} L'oni effettua due attacchi, con gli artigli o con il falcione.

\emph{\textbf{Artiglio (Solo Forma di Oni).} Attacco con arma da mischia}: +7 a colpire, portata 1 m, un bersaglio. \emph{Colpisce:} 8 (1d8 + 4) danni taglienti.

\emph{\textbf{Falcione.} Attacco con arma da mischia}: +7 a colpire, portata 3 m, un bersaglio.

\emph{Colpisce:} 15 (2d10 + 4) danni taglienti, o 9 (1d10 + 4) danni taglienti in forma Piccola o Media.

\emph{\textbf{Mutare Forma.}} L'oni può trasformarsi magicamente in un umanoide Piccolo o Medio, in un gigante Grande, o tornare alla sua vera forma. A parte la taglia, le sue statistiche sono le stesse in ciascuna forma. L'unico equipaggiamento che viene trasformato è il falcione, che rimpicciolisce in modo da essere impugnato anche in forma umanoide. Se l'oni muore, ritorna alla sua vera forma, e il falcione ritorna alla sua taglia originale.

\medskip\index{Mostri - Orco}\textbf{Orco}

\emph{Media umanoide (orco), caotico malvagio}

\textbf{FORZA} +3

\textbf{DESTREZZA} +1

\textbf{COSTITUZIONE} +3

\textbf{INTELLIGENZA} -2

\textbf{SAGGEZZA} +0

\textbf{CARISMA} +0

\textbf{Iniziativa} +1 -- \textbf{Difesa} 14 (armatura di pelle)

\textbf{Punti Ferita} 15 (2d8 + 6)

\textbf{Movimento} 9 m

\textbf{Tiri Salvezza}: Tempra +3, Riflessi +1, Volontà +1

\textbf{Competenze} Intimidire +2

\textbf{Sensi} scurovisione 18 m

\textbf{Linguaggi} Comune, Goblinoide

\textbf{Sfida} 1/2 (100 PE)

\emph{\textbf{Aggressivo.}} Come azione bonus, l'orco può muoversi fino a metà del suo movimento verso una creatura ostile che possa vedere.

\textbf{Azioni}

\emph{\textbf{Ascia Bipenne.} Attacco con arma da mischia}: +5 a colpire, portata 1 m, un bersaglio.

\emph{Colpisce:} 9 (1d12 + 3) danni taglienti.

\emph{\textbf{Giavellotto.} Attacco con arma da mischia o a Distanza}: +5 a colpire, portata 1 m o gittata 9m, un bersaglio. \emph{Colpisce:} 6 (1d6 + 3) danni perforanti.

\textbf{Ecologia}\\
Ambiente: Colline e montagne temperate o sotterranei\\
Organizzazione: solitario, gruppo (2-4), squadra (11-20 più 2 sergenti di 3° livello e 1 capo di 3°-6° livello) o banda \\
Tesoro: Equipaggiamento da PNG (Armatura di Cuoio Borchiato, Falchion, 4 Giavellotti, altro tesoro)\\
\textbf{Descrizione}\\
La differenza principale fra gli orchi e gli umanoidi civilizzati, oltre alla loro forza bruta ed all'intelligenza inferiore, è il loro carattere. Come cultura, gli orchi sono violenti ed aggressivi, ed il forte domina il debole attraverso paura e brutalità. Prendono ciò che vogliono con la forza e non si fanno scrupoli a prendere interi villaggi come schiavi se ne hanno la possibilità. Non si curano delle comodità, ed i loro villaggi e campi tendono ad essere luoghi sporchi e precari, pieni di risse fra ubriachi, arene per i combattimenti ed altri divertimenti sadici. Privi della pazienza necessaria a coltivare e capaci di allevare solo gli animali più robusti ed autosufficienti, gli orchi ritengono più semplice prendere agli altri il frutto del loro lavoro. Sono arroganti e lesti ad infuriarsi quando sfidati, ma si preoccupano dell'onore solo finché farlo porta loro beneficio.\\

Un orco maschio adulto è alto 1,8 metri e pesa circa 105 kg. Gli orchi e gli umani possono accoppiarsi, anche se di solito ciò avviene durante le razzie, e non come unione consensuale. Molte tribù orchesche allevano i mezzorchi di proposito, dato che sono ottimi strateghi e capitribù.\\


\medskip\index{Mostri - Orsogufo}\textbf{Orsogufo}

\emph{Grande mostruosità, disallineato}

\textbf{FORZA} +5

\textbf{DESTREZZA} +1

\textbf{COSTITUZIONE} +3

\textbf{INTELLIGENZA} -4

\textbf{SAGGEZZA} +1

\textbf{CARISMA} -2

\textbf{Iniziativa} +1 -- \textbf{Difesa} 15

\textbf{Punti Ferita} 59 (7d10 + 21)

\textbf{Movimento} 12 m

\textbf{Tiri Salvezza}: Tempra +10, Riflessi +5, Volontà +2

\textbf{Competenze} Consapevolezza +3

\textbf{Sensi} scurovisione 18 m

\textbf{Linguaggi} -

\textbf{Sfida} 3 (700 PE)

\emph{\textbf{Olfatto e Vista Affinati.}} L'orsogufo ha +1d6 nelle prove di Saggezza (Consapevolezza) basate su olfatto o vista. 

\textbf{Azioni}

\emph{\textbf{Multiattacco.}} L'orsogufo effettua due attacchi: uno con il becco e uno con gli artigli.

\emph{\textbf{Artigli.} Attacco con arma da mischia}: +7 a colpire, portata 1 m, un bersaglio.

\emph{Colpisce:} 14 (2d8 + 5) danni taglienti.

\emph{\textbf{Becco.} Attacco con arma da mischia}: +7 a colpire, portata 1 m, una creatura.

\emph{Colpisce:} 10 (1d10 + 5) danni perforanti.

\textbf{Ecologia}\\
\textbf{Ambiente: Foreste Temperate}
Organizzazione: Solitario, coppia o branco (3-8)\\
Tesoro: Accidentale\\
\textbf{Descrizione}\\
Le origini dell'orsogufo sono oggetto di dibattito fra gli studiosi delle creature mostruose. La maggior parte di essi concorda che fu un Mago, in passato, a crearne il primo esemplare unendo un orso con un gufo gigante; forse come esperimento su qualche folle concetto della natura della vita, ma più probabilmente a causa della sua totale pazzia. Quale che fosse lo scopo originale di una creazione tanto folle come l'orsogufo, la creatura ha iniziato a riprodursi, ed è divenuta uno dei predatori più conosciuto delle zone boschive.\\
Gli orsigufo sono selvaggi predatori, noti per il loro pessimo temperamento, la loro aggressività e la loro ferocia. Tendono ad attaccare tutto ciò che si muove loro davanti, anche se questo non mostra intenzioni bellicose. Molti studiosi che hanno incontrato queste creature nelle terre selvagge hanno notato che hanno sempre occhi iniettati di sangue che ruotano tutto attorno poco prima di un attacco. Questo è generalmente visto come segno di follia, che suggerisce che tutti gli orsigufo nascano con un bisogno patologico di combattere ed uccidere, ma i ricercatori più realisti ritengono sia dovuto alla struttura dei loro occhi acuti.\\
Gli orsigufo abitano le zone più interne e nascoste dei boschi, e preparano le loro tane all'interno di foreste intricate o di buie e profonde caverne. Possono cacciare sia di giorno che di notte, a seconda delle abitudini delle prede che popolano i territori circostanti alla loro tana.\\
Gli orsigufo adulti vivono in coppia e cacciano le prede in branco, lasciando i cuccioli nelle tane. In una tana si possono trovare di solito 1d6 cuccioli, che possono valere fino a 3.000 mo nei mercati cittadini.\\

Anche se è pressochè impossibile addomesticarli a causa della loro natura selvaggia, gli orsigufo possono essere sfruttati come guardiani di un territorio specifico, sempre che vengano lasciati liberi di spostarsi al suo interno per cacciare. Gli addestratori professionisti chiedono fino a 2.000 mo per addestrare un orsogufo perché diventi un guardiano che obbedisca a comandi semplici (DC 23 per un cucciolo di orsogufo, DC 30 per un orsogufo adulto).\\


\medskip\index{Mostri - Otyugh}\textbf{Otyugh}

\emph{Grande aberrazione, neutrale}

\textbf{FORZA} +3

\textbf{DESTREZZA} +0

\textbf{COSTITUZIONE} +4

\textbf{INTELLIGENZA} -2

\textbf{SAGGEZZA} +1

\textbf{CARISMA} -2

\textbf{Iniziativa} +0 -- \textbf{Difesa} 17

\textbf{Punti Ferita} 114 (12d10 + 48)

\textbf{Movimento} 9 m

\textbf{Tiri Salvezza}: Tempra +3, Riflessi +2, Volontà +6

\textbf{Sensi} scurovisione 36 m

\textbf{Linguaggi} Otyugh

\textbf{Sfida} 5 (1.800 PE)

\emph{\textbf{Telepatia Limitata.}} L'otyugh può trasmettere magicamente dei semplici messaggi e immagini a qualsiasi creatura entro 36 metri da esso e che possa comprendere una lingua. Questa forma di telepatia non permette alla creatura ricevente di rispondere telepaticamente.

\textbf{Azioni}

\emph{\textbf{Multiattacco.}} L'otyugh effettua tre attacchi: uno con il morso e due con i tentacoli.

\emph{\textbf{Morso.} Attacco con arma da mischia}: +6 a colpire, portata 1 m, un bersaglio.

\emph{Colpisce:} 12 (2d8 + 3) danni perforanti. Se il bersaglio è una creatura, deve riuscire un Tiro Salvezza di Tempra CD 15 contro malattia o restare avvelenato finché la malattia non viene curata. Ogni 24 ore successive, il bersaglio deve ripetere il Tiro Salvezza, riducendo il suo massimo di punti ferita di 5 (1d10) se lo fallisce. Se il Tiro Salvezza riesce, la malattia è passata. Il bersaglio muore se la malattia riduce i suoi punti ferita massimi a 0.

Questa riduzione dei punti ferita massimi del personaggio, perdura finché la malattia non viene curata.
\
\emph{\textbf{Tentacolo.} Attacco con arma da mischia}: +6 a colpire, portata 3 m, un bersaglio.

\emph{Colpisce:} 7 (1d8 + 3) danni da botta più 4 (1d8) danni perforanti. Se il bersaglio è di taglia Media o inferiore, è afferrato (CD 13 per fuggire) e intralciato fino al termine dell'afferrare. L'otyugh ha due tentacoli, ciascun dei quali può afferrare un bersaglio diverso.

\emph{\textbf{Schianto di Tentacolo.}} L'otyugh schianta le creature afferrate dai suoi tentacoli, l'una contro l'altra o sul pavimento. Ogni creatura deve riuscire un Tiro Salvezza di Tempra CD 14 o subire 10 (2d6 + 3) danni da botta e restare stordita fino al termine del prossimo turno dell'otyugh. Se il Tiro Salvezza riesce, il bersaglio subisce la metà dei danni da botta e non è stordito.

\textbf{Ecologia}\\
Ambiente: Qualsiasi Sotterraneo\\
Organizzazione: Solitario, coppia o gruppo (3-4)\\
Tesoro: Standard\\
\textbf{Descrizione}\\
Gli otyugh sono creature particolarmente luride ed orride che vivono in luoghi che le persone sane di mente tendono ad evitare. Le loro tane si trovano nelle fogne, nei pozzi neri, nelle discariche e nelle paludi più mefitiche: più un luogo è sporco, più attira gli otyugh. Amano il ruolo dello spazzino, e vagano per le caverne sotterranee in cerca di nuovi bocconcini in mezzo ai rifiuti. Una volta trovati, si ingozzano e riportano alla loro tana quello che non riescono a consumare in una volta sola. Gli otyugh passano parecchio tempo nelle loro luride tane, che riempiono di carogne e letame, che rilasciano effluvi mefitici.\\
Le creature intelligenti che vivono nelle zone sotterranee vicino agli otyugh a volte formano alleanze di convenienza con essi. Forniscono loro rifiuti e carne cruda agli otyugh, rendendoli un vero e proprio mezzo di smaltimento. In cambio, gli otyugh lasciano in pace i loro benefattori, non li attaccano e possono anche fare da guardiani.\\
La cosa che la maggior parte delle razze trova terrificante degli otyugh non è la loro dieta o l'odore delle loro tane, ma il fatto che creature con i loro gusti non siano solo spazzini senza cervello. Gli otyugh si mostrano infatti sorprendentemente intelligenti, ed amano formare alleanze con coloro che li riforniscono di cibi più raffinati di letame e sporcizia. La maggior parte degli otyugh si rende conto che le altre creature li trovano rivoltanti, ma sono pochi quelli a cui importa davvero.\\


\medskip\index{Mostri - Pegaso}\textbf{Pegaso}

\emph{Grande celestiale, caotico buono}

\textbf{FORZA} +4

\textbf{DESTREZZA} +2

\textbf{COSTITUZIONE} +3

\textbf{INTELLIGENZA} +0

\textbf{SAGGEZZA} +2

\textbf{CARISMA} +1

\textbf{Iniziativa} +2 -- \textbf{Difesa} 13

\textbf{Punti Ferita} 59 (7d10 + 21)

\textbf{Movimento} 18 m, volo 27 m

\textbf{Tiri Salvezza} Tempra +7, Riflessi +6, Volontà +4

\textbf{Competenze} Consapevolezza +6

\textbf{Linguaggi} comprende Celestiale, Comune, Elfico e Silvano ma non può parlare

\textbf{Sfida} 2 (450 PE)

\textbf{Azioni}

\emph{\textbf{Zoccoli.} Attacco con arma da mischia}: +6 a colpire, portata 1 m, un bersaglio.

\emph{Colpisce:} 11 (2d6 + 4) danni da botta.

\textbf{Ecologia}
Ambiente: Pianure Temperate e Calde\\
Organizzazione: Solitario, coppia o branco (6-10)\\
Tesoro: Nessuno\\
\textbf{Descrizione}\\
Il pegaso è un magnifico cavallo alato che a volte serve la causa del bene. Seppur molto apprezzati come cavalcature volanti, i pegasi sono creature timide che difficilmente stringono amicizie. Un tipico pegaso è alto 1,8 metri al garrese, pesa 750 kg ed ha un'apertura alare di 6 metri. La maggior parte dei pegasi è bianca, ma a volte alcuni esemplari hanno colori diversi.\\

Il pegaso, nonostante le apparenze, è intelligente quanto un umano. Chi cerca di addestrarne uno a fare da cavalcatura, scoprirà che il pegaso è ricalcitrante e perfino violento. Un pegaso non può parlare, ma capisce il Comune e preferisce la compagnia di creature buone. Il metodo corretto per convincere un pegaso a fare da cavalcatura è farselo amico con Diplomazia, favori e buone azioni. Un pegaso ha di norma atteggiamento indifferente verso le creature buone, maldisposto verso quelle neutrali ed ostile verso quelle malvagie. Prima che possa servire come cavalcatura, un pegaso deve essere reso amichevole tramite una prova di Diplomazia o in altro modo. Cavalcare un pegaso richiede una sella esotica o Cavalcare a pelo, dato che una sella normale interferisce con le sue ali. Un pegaso può combattere portando un cavaliere, ma il cavaliere non può attaccare a sua volta se non supera una prova di Cavalcare. I pegasi addestrati non temono il combattimento ed il cavaliere non deve effettuare una prova di Cavalcare per controllarlo.\\

I pegasi depongono uova che sul mercato valgono 2.000 mo l'una, mentre i piccoli arrivano alle 3.000 mo a testa. Essendo creature intelligenti e buone, vendere uova e piccoli è essenzialmente schiavismo: nelle società buone chi lo fa è disprezzato o punito dalla legge.\\

I pegasi maturano come i cavalli. Gli addestratori professionisti chiedono 1.000 per addestrare un pegaso, che servirà un cavaliere buono o neutrale fedelmente per tutta la vita.\\

Un carico leggero per un pegaso è fino a 150 kg; un carico medio è 150,5-300 kg; un carico pesante è 300,5-450 kg.\\

In alcuni pegasi il sangue di un antenato che era un eroico stallone è ancora forte. Questi campioni hanno la durata della vita di un umano, l'archetipo avanzato, manovrabilità perfetta, resistenza al fuoco 10, un bonus razziale di +4 ai Tiri Salvezza contro i Veleni e immunità alla Pietrificazione. Alcuni riescono a dire poche parole in Celestiale o Comune. Si rendono conto della loro superiorità agli altri cavalli ed ai pegasi, e non devono essere addestrati a volare con un cavaliere, ma permettono solo ai più grandi eroi di cavalcarli.\\


\medskip\index{Mostri - Persecutore Invisibile}\textbf{Persecutore Invisibile}

\emph{Media elementale, neutrale}

\textbf{FORZA} +3

\textbf{DESTREZZA} +4

\textbf{COSTITUZIONE} +2

\textbf{INTELLIGENZA} +0

\textbf{SAGGEZZA} +2

\textbf{CARISMA} +0

\textbf{Iniziativa} +4 -- \textbf{Difesa} 17

\textbf{Punti Ferita} 104 (16d8 + 32)

\textbf{Movimento} 15 m, volo 15 m (fluttua)

\textbf{Tiri Salvezza}: Tempra +13, Riflessi +11, Volontà +4

\textbf{Competenze} Muoversi Silenziosamente / Nascondersi +10, Consapevolezza +8

\textbf{Resistenze al Danno} da botta, perforante e tagliente di attacchi non magici

\textbf{Immunità ai Danni} veleno

\textbf{Immunità alle Condizioni} afferrato, avvelenato, intralciato, paralizzato, pietrificato, privo di sensi, prono, affaticamento

\textbf{Sensi} scurovisione 18 m

\textbf{Linguaggi} Auran, comprende il Comune ma non lo parla

\textbf{Sfida} 6 (2.300 PE)

\emph{\textbf{Cacciatore Infallibile.}} Il convocatore assegna una preda
al persecutore. Il persecutore sa la direzione e la distanza a cui si
trova la preda finché entrambi si trovano sullo stesso piano di
esistenza. Il persecutore conosce anche la posizione del suo
convocatore.

\emph{\textbf{Invisibilità.}} Il persecutore è invisibile.

\emph{\textbf{Natura Elementale.}} Un persecutore invisibile non ha bisogno di aria, cibo, bevande o sonno.

\textbf{Azioni}

\emph{\textbf{Multiattacco.}} La persecutore effettua due attacchi di schianto.

\emph{\textbf{Schianto.} Attacco con arma da mischia}: +6 a colpire, portata 1 m, un bersaglio.

\emph{Colpisce:} 10 (2d6 + 3) danni da botta.

\medskip\index{Mostri - Pseudodrago}\textbf{Pseudodrago}

\emph{Minuscola drago, neutrale buono}

\textbf{FORZA} -2

\textbf{DESTREZZA} +2

\textbf{COSTITUZIONE} +1

\textbf{INTELLIGENZA} +0

\textbf{SAGGEZZA} +1

\textbf{CARISMA} +0

\textbf{Iniziativa} +2 -- \textbf{Difesa} 14

\textbf{Punti Ferita} 7 (2d4 + 2)

\textbf{Movimento} 5 metri, volo 18 m

\textbf{Tiri Salvezza}: Tempra +4, Riflessi +5, Volontà +4

\textbf{Competenze} Muoversi Silenziosamente / Nascondersi +4, Consapevolezza +3

\textbf{Sensi} scurovisione 18 m, vista cieca 3 m

\textbf{Linguaggi} comprende il Comune e il Draconico ma non parla

\textbf{Sfida} 1/4 (50 PE)

\emph{\textbf{Resistenza alla Magia.}} Lo pseudodrago ha +1d6 ai Tiri Salvezza contro incantesimi e altri effetti magici.

\emph{\textbf{Sensi Affinati.}} Lo pseudodrago ha +1d6 alle prove di Saggezza (Consapevolezza) basate su vista, udito e olfatto.

\emph{\textbf{Telepatia Limitata.}} Lo pseudodrago può comunicare semplici idee, emozioni e immagini telepaticamente con qualsiasi creatura entro 30 metri da esso che può comprendere una lingua.

\textbf{Azioni}

\emph{\textbf{Morso.} Attacco con arma da mischia}: +4 a colpire, portata 1 m, un bersaglio.

\emph{Colpisce:} 4 (1d4 + 2) danni perforanti.

\emph{\textbf{Pungiglione.} Attacco con arma da mischia}: +4 a colpire, portata 1 m, una creatura.

\emph{Colpisce:} 4 (1d4 + 2) danni perforanti, e il bersaglio deve riuscire un Tiro Salvezza di Tempra CD 11 o restare avvelenato per 1 ora. Se il risultato del Tiro Salvezza è 6 o meno, il bersaglio cade privo di sensi per la stessa durata, o finché non subisce danni o un'altra creatura usa un'azione per risvegliarlo.

\textbf{Ecologia}\\
Ambiente: Foreste temperate\\
Organizzazione: Solitario, coppia o nido (3-5)\\
Tesoro: Standard\\
\textbf{Descrizione}\\
Gli pseudodraghi sono piccoli parenti dei veri draghi, giocosi e timidi. Parlano cinguettando, sibilando, ringhiando e facendo le fusa, ma possono comunicare telepaticamente con qualsiasi creatura intelligente. Se avvicinati pacificamente con offerte di cibo, sono disposti a condividere informazioni su quanto si trova nel loro territorio, ma minacce e violenza li fanno fuggire.\\

Gli pseudodraghi sono carnivori e mangiano insetti, roditori, uccellini e serpenti, anche se mangiano uova ed amano burro, formaggio e pesce. A volte cacciano a terra come le lucertole o volando come gli uccelli predatori. Intelligenti come la maggior parte degli umanoidi, non amano essere trattati come animali domestici, e preferiscono essere considerati amici. Diffidano delle creature malvagie, possono unirsi a incantatori e Devoti come Famigli e alcuni hanno stretto amicizia con Druidi e guardiaboschi o collaborano con i draghi buoni come sentinelle. Gli pseudodraghi diventano Famigli solo se apprezzano la personalità dell'incantatore (e se questi ha l'Abilita' Famiglio e Carisma almeno 1), ma possono anche legarsi a persone delle quali apprezzano la compagnia. Uno pseudodrago potrebbe seguire in questo modo un personaggio per giorni, settimane, anni o perfino per tutta la vita, posto che siano ben nutriti e trattati con affetto.\\

Raggiunta l'età adulta, il corpo di uno pseudodrago è lungo 30 centimetri con una coda di 60 centimetri, e pesa circa 3,5 kg. Le uova di uno pseudodrago sono grandi come quelle di gallina, ma di consistenza simile al cuoio e macchiate di marrone, e le femmine le depongono in gruppi di 2-5 ogni primavera. Un nido di pseudodraghi (che costituiscono un gruppo familiare, e non sono nati dallo stesso gruppo di uova) di solito consiste di una coppia di adulti e diversi cuccioli quasi adulti.\\


\medskip\index{Mostri - Rakshasa}\textbf{Rakshasa}

\emph{Media immondo, legale malvagio}

\textbf{FORZA} +2

\textbf{DESTREZZA} +3

\textbf{COSTITUZIONE} +4

\textbf{INTELLIGENZA} +1

\textbf{SAGGEZZA} +3

\textbf{CARISMA} +5

\textbf{Iniziativa} +3 -- \textbf{Difesa} 23

\textbf{Punti Ferita} 110 (13d8 + 52)

\textbf{Movimento} 12 m

\textbf{Tiri Salvezza}: Tempra +9, Riflessi +12, Volontà +8 

\textbf{Competenze} Ingannare +10, Percepire Emozioni +8

\textbf{Vulnerabilità al Danno} perforante di armi magiche impugnate da
creatura buone

\textbf{Immunità al Danno} da botta, armi +1

\textbf{Sensi} scurovisione 18 m

\textbf{Linguaggi} Comune, Infernale

\textbf{Sfida} 13 (10.000 PE)

\emph{\textbf{Immunità alla Magia Limitata.}} Il rakshasa è immune agli affetti o all'individuazione tramite incantesimi di Difficoltà 29 o più basso a meno che non desideri esserne soggetto. Ha +1d6 ai Tiri Salvezza contro tutti gli altri incantesimi ed effetti magici.

\emph{\textbf{Incantesimi Innati.}} La caratteristica da incantatore del rakshasa il Carisma (+10 a colpire   con attacchi con incantesimi). Il rakshasa può lanciare in maniera innata i seguenti incantesimi, senza aver bisogno di componenti materiali:

A volontà: \emph{camuffare sé stesso, illusione minore, individuazione} \emph{dei pensieri, mano magica}

3/Giorno ciascuno: \emph{charme su persone, immagine maggiore,} \emph{individuazione del magico, invisibilità, suggestione} 1/Giorno: \emph{dominare persone, spostamento planare, visione del} \emph{vero, volare}

\textbf{Azioni}

\emph{\textbf{Multiattacco.}} Il rakshasa può effettuare due attacchi di artiglio.

\emph{\textbf{Artiglio.} Attacco con arma da mischia}: +7 a colpire, portata 1 m, un bersaglio.

\emph{Colpisce:} 9 (2d6 + 2) danni taglienti, e se il bersaglio è una creatura rimane maledetto. La maledizione magica ha effetto ogni qualvolta il bersaglio riposa, riempiendo i pensieri del bersaglio di immagini e sogni orribili. Il bersaglio maledetto non riceve beneficio dall'aver terminato un riposo. La maledizione perdura finché non viene rimossa dall'incantesimo \emph{rimuovi maledizione} o simile magia.

\textbf{Ecologia}
Ambiente: Qualsiasi\\
Organizzazione: Solitario, coppia o culto (3-12)\\
Tesoro: Doppio (Pugnale+1, altro tesoro)\\
\textbf{Descrizione}\\
Il rakshasa è uno spirito maligno che si traveste da creatura umanoide così da poter seguire la sua preda in incognito. Personificazione dei tabù della maggioranza delle società e capace di assumere l'aspetto di quelli che cerca di corrompere, un rakshasa compie moltissime azioni orribili. Se fossero umani, la loro blasfemia, il cannibalismo e gli atti ancora peggiori che compiono li marchierebbero come criminali meritevoli del più crudele degli inferni.\\
Quando non ha un altro aspetto, il rakshasa appare come un umanoide con la testa di un animale. Spesso ha il capo di un grosso felino (come tigri o pantere) o serpente (quali cobra o vipere) e, seppur sia più raro, può avere testa di gorilla, sciacallo, avvoltoio, elefante, mantide, lucertola, rinoceronte, cinghiale e molte altre ancora. In molti casi, il tipo di testa posseduta da un rakshasa dice qualcosa della sua personalità: un rakshasa dalla testa di tigre è furtivo e famelico, mentre uno con la testa di cinghiale può essere ghiotto e crudele. Queste differenze raramente incidono sulle statistiche base del rakshasa, anche se esistono varianti più potenti della standard con molteplici teste, poteri magici più potenti, e strane e letali capacità speciali aggiuntive.\\
I rakshasa disprezzano le religioni; riconoscono il potere degli dei, ma si vedono come i soli esseri degni di venerazione da parte delle razze mortali. I Devoti rakshasa sono quindi piuttosto rari. Sebbene i rakshasa siano esterni, sono anche creature del Piano Materiale, e alcuni credono che i primi rakshasa scelsero questo esilio al posto di qualche altro ruolo offertogli da un dio da tempo dimenticato. Anche se in genere sono solitari, non è raro trovare grandi famiglie di rakshasa che lavorano insieme per provocare la caduta di una civiltà mortale dall'interno, attraverso il succedersi di molte generazioni.\\
Un rakshasa è alto 1,8 metri e pesa 90 kg.\\


\medskip\index{Mostri - Remorhaz}\textbf{Remorhaz}

\emph{Enorme mostruosità, disallineato}

\textbf{FORZA} +7

\textbf{DESTREZZA} +1

\textbf{COSTITUZIONE} +5

\textbf{INTELLIGENZA} -3

\textbf{SAGGEZZA} +0

\textbf{CARISMA} -3

\textbf{Iniziativa} +1 -- \textbf{Difesa} 23

\textbf{Punti Ferita} 195 (17d12 + 85)

\textbf{Movimento} 9 m, scavo 6 m

\textbf{Tiri Salvezza}: Tempra +11, Riflessi +7, Volontà +4

\textbf{Immunità ai Danni} freddo, fuoco

\textbf{Sensi} scurovisione 18 m, senso tellurico 18 m

\textbf{Linguaggi} -

\textbf{Sfida} 11 (7.200 PE)

\emph{\textbf{Corpo Riscaldato.}} Una creatura che entri a contatto con il remorhaz o lo colpisca con un attacco da mischia mentre si trova entro 1 metro da esso, subisce 10 (3d6) danni da fuoco.

\textbf{Azioni}

\emph{\textbf{Morso.} Attacco in mischia con arma}: +11 a colpire, portata 3 m, un bersaglio.

\emph{Colpisce:} 40 (6d10 + 7) danni perforanti più 10 (3d6) danni da fuoco. Se il bersaglio è una creatura, è afferrato (CD 17 per fuggire). Fino al termine dell'afferrare, il bersaglio è intralciato, e il remorhaz non può attaccare con il morso un altro bersaglio.

\emph{\textbf{Inghiottire.}} Il remorhaz effettua una attacco di morso contro un bersaglio di taglia Media o inferiore che sta afferrando. Se l'attacco colpisce, la creatura subisce il danno da morso ed è inghiottita, e l'afferrare ha termine. Il bersaglio inghiottito è accecato e intralciato, ha copertura totale contro gli attacchi e altri effetti all'esterno del remorhaz, e subisce 21 (6d6) danni da acido all'inizio di ciascun turno del remorhaz.

Se il remorhaz subisce 30 o più danni in un singolo turno da una creatura al suo interno, il remorhaz deve riuscire un Tiro Salvezza su Tempra CD 15 al termine di quel turno o vomitare tutte le creature inghiottite, che cadono prone in uno spazio entro 3 metri dal remorhaz. Se il remorhaz muore, una creatura inghiottita non   più intralciata da esso e può uscire dal cadavere utilizzando 5 metri di movimento, uscendo prona.

\textbf{Ecologia}\\
Ambiente: Deserti Freddi e Ghiacciai\\
Organizzazione: Solitario\\
Tesoro: Nessuno\\
\textbf{Descrizione}\\
In un mondo di ghiaccio e neve, i remorhaz sono particolarmente temuti per il terribile fuoco che brucia dentro i loro corpi. Questo fuoco interiore fa sì che le piastre lungo il suo dorso divengano roventi quando la creatura è particolarmente arrabbiata, eccitata o nel panico. Le creature che si sono adattate alle regioni artiche spesso sono particolarmente vulnerabili al fuoco, il che rende la principale difesa del remorhaz incredibilmente potente e gli assicura il ruolo di pericoloso predatore delle zone ghiacciate. I remorhaz vivono in estesi labirinti scavati nel cuore dei ghiacciai. Queste bestie usano il loro calore per scavare tunnel nel ghiaccio, tunnel le cui lisce pareti vitree si ricongelano rapidamente lungo la loro scia creando numerosi dedali incredibilmente stabili.\\

Anche se il remorhaz ha molto in comune con i più piccoli parassiti di superficie, questa bestia è sorprendentemente intelligente. Sebbene incapace di palare, il tipico remorhaz capisce bene il Gigante, e spesso le tribù di giganti ne approfittano per stipulare alleanze con questi bestioni. I Giganti del Gelo sono particolarmente ossessionati da essi; questi giganti affrontano le crudeli e letali bruciature che un remorhaz può infliggere per diventare "amici del verme" ottenendo una potente arma da usare contro i loro nemici, un assassino capace di scavare attraverso il pavimento delle fortificazioni glaciali per colpire direttamente con la maggiore debolezza di un gigante del gelo: il fuoco. Altri giganti usano queste bestie come forge viventi, poiché il loro dorso è abbastanza caldo da sciogliere il metallo.\\
Un remorhaz è lungo 7 metri e pesa 5.000 kg.\\



\medskip\index{Mostri - Rugginofago}\textbf{Rugginofago}

\emph{Media Mostruosità, disallineato}

\textbf{FORZA} +1

\textbf{DESTREZZA} +1

\textbf{COSTITUZIONE} +1

\textbf{INTELLIGENZA} -4

\textbf{SAGGEZZA} +1

\textbf{CARISMA} -2

\textbf{Iniziativa} +1 -- \textbf{Difesa} 15

\textbf{Punti Ferita} 27 (5d8 + 5)

\textbf{Movimento} 12 m

\textbf{Tiri Salvezza}: Tempra +2, Riflessi +4, Volontà +5

\textbf{Sensi} scurovisione 18 m

\textbf{Linguaggi} -

\textbf{Sfida} 1/2 (100 PE)

\emph{\textbf{Fiuto del Ferro.}} Il rugginofago può individuare, con l'olfatto, l'esatta posizione di metalli ferrosi entro 9 metri.

\emph{\textbf{Arrugginire Metallo.}} Qualsiasi arma non magica fatta di metallo che colpisca il rugginofago si corrode. Dopo aver inflitto il danno, l'arma subisce una penalità permanente e cumulativa di - 1 ai tiri di danno. Se la penalità scende fino a -5, l'arma è distrutta. Le munizioni non magiche fatte di metallo e che colpiscono il rugginofago, sono considerate distrutte dopo aver inflitto il danno.

\textbf{Azioni}

\emph{\textbf{Morso.} Attacco con arma da mischia}: +3 a colpire, portata 1 m, un bersaglio.

\emph{Colpisce:} 5 (1d8 + 1) danni perforanti.

\emph{\textbf{Antenne.}} Il rugginofago corrode gli oggetti di metallo ferroso non magici che può vedere e si trovano entro 1 metro. Se l'oggetto non è indossato o trasportato, il contatto col rugginofago ne distrugge un cubo di 30 centimetri di spigolo. Se l'oggetto è indossato o trasportato da una creatura, la creatura può effettuare un Tiro Salvezza su Riflessi CD 11 per evitare il contatto con il rugginofago.

Se l'oggetto con cui entra in contatto è un'armatura o scudo di metallo indossati o trasportati, questi subiscono una penalità permanente e cumulativa di -1 alla Difesa che forniscono. Le armature ridotte a Difesa 0 o gli scudi che scendono ad un bonus di +0 sono distrutti. Se l'oggetto con cuientra in contatto è un'arma di metallo impugnata da qualcuno, la  arrugginisce come descritto nel tratto Arrugginire Metallo.

\textbf{Ecologia}
Ambiente: Qualsiasi Sotterraneo\\
Organizzazione: Solitario, coppia o nido (3-10)\\
Tesoro: Accidentale (nessun tesoro di metallo)\\
\textbf{Descrizione}\\
Di tutte le bestie terrificanti che un esploratore può incontrare nel sottosuolo, solo il rugginofago ha come obbiettivo quello cui l'avventuriero medio da più valore: il suo tesoro.\\
Lungo in genere 1 metro e dal peso di almeno 100 kg, il rugginofago somiglia ad un crostaceo e sarebbe già abbastanza spaventoso anche senza l'alieno processo nutritivo da cui prende il nome. I rugginofagi mangiano gli oggetti di metallo, preferendo quelli di ferro e di leghe ferrose come l'acciaio ma divorano anche mithral, Adamantio e metalli incantati con uguale facilità. Qualsiasi metallo toccato dalle delicate antenne del rugginofago o dalla sua pelle corazzata si corrode e si riduce in polvere in pochi secondi, facendone una delle bestie più temute dagli avventurieri sotterranei e dai Nani minatori che devono difendere le loro forge e competere con loro per l'oro.\\
Anche se i rugginofagi non hanno una tendenza innata per la violenza, la loro fame insaziabile li spinge a caricare qualunque cosa gli si avvicini con addosso abbastanza metallo, e qualsiasi resistenza viene accolta con inaspettata ferocia. Non è insolito che i rugginofagi in zone povere di metallo seguano le vittime in fuga per giorni usando la loro capacità di fiutare metalli, purché queste abbiano ancora oggetti di metallo intatti.\\
Fortunatamente, spesso è possibile sfuggire alle attenzioni di un rugginofago lanciandogli un oggetto di metallo denso, come uno scudo, e correndo nella direzione opposta. Quanti frequentano le aree infestate dai rugginofagi imparano velocemente a tenere a portata di mano armi di legno o pietra.\\



\medskip\index{Mostri - Sahuagin}\textbf{Sahuagin}

\emph{Media umanoide (sahuagin), legale malvagio}

\textbf{FORZA} +1

\textbf{DESTREZZA} +0

\textbf{COSTITUZIONE} +1

\textbf{INTELLIGENZA} +1

\textbf{SAGGEZZA} +1

\textbf{CARISMA} -1

\textbf{Iniziativa} +1 -- \textbf{Difesa} 13

\textbf{Punti Ferita} 22 (4d8 + 4)

\textbf{Movimento} 9 m, nuoto 12 m

\textbf{Tiri Salvezza}: Tempra +4, Riflessi +4, Volontà +4

\textbf{Competenze} Consapevolezza +5

\textbf{Sensi} scurovisione 36 m

\textbf{Linguaggi} Sahuagin

\textbf{Sfida} 1/2 (100 PE)

\emph{\textbf{Anfibio Limitato.}} Il sahuagin può respirare aria e acqua, ma deve restare sommerso almeno una volta ogni 4 ore per evitare di soffocare.

\emph{\textbf{Frenesia Sanguinaria.}} Il sahuagin ha +1d6 ai tiri per colpire in mischia contro qualsiasi creatura che non sia al massimo dei suoi punti ferita.

\emph{\textbf{Telepatia con gli Squali}}. Il sahuagin può comandare magicamente qualsiasi squalo entro 36 metri da sé, usando una forma limitata di telepatia. 

\textbf{Azioni}

\emph{\textbf{Multiattacco.}} Il sahuagin può effettuare due attacchi da mischia:  uno con il morso e uno con gli artigli o la lancia.

\emph{\textbf{Artigli.} Attacco con arma da mischia}: +3 a colpire, portata 1 m, un bersaglio.

\emph{Colpisce:} 3 (1d4 + 1) danni taglienti.

\emph{\textbf{Lancia.} Attacco con arma da mischia o a Distanza}: +3 a colpire, portata 1 m o gittata 6m, un bersaglio.

\emph{Colpisce:} 4 (1d6 + 1) danni perforanti, o 5 (1d8 + 1) danni perforanti se usata con due mani per effettuare un attacco da mischia.

\emph{\textbf{Morso.} Attacco con arma da mischia}: +3 a colpire, portata 1 m, un bersaglio.

\emph{Colpisce:} 3 (1d4 + 1) danni perforanti.

\textbf{Ecologia}\\
Ambiente: Oceani Temperati o Caldi\\
Organizzazione: Solitario, coppia, squadra (5-8), pattuglia (11-20 più 1 tenente di 3° livello e 1-2 Squali), banda (20-80 più 100\% non combattenti, 1 tenente di 3° livello e 1 capitano di 4° livello ogni 20 adulti, e 1-2 Squali) o tribù (70-160 più 100\% non combattenti, 1 tenente di 3° livello ogni 20 adulti, 1 capitano di 4° livello ogni 40 adulti, 9 guardie di 4° livello, 1-4 novizie di 3°-6° livello, 1 sacerdotessa di 7° livello, 1 barone di 6°-8° livello, e 5-8 Squali)
Tesoro: Equipaggiamento da PNG (Tridente, Balestra Pesante con 10 Quadrelli, altro tesoro)\\
\textbf{Descrizione}\\
Famelici e crudeli, i sahuagin sono, sfortunatamente, tra le razze oceaniche più prosperose. Grandi città sono state costruite da questa razza nelle buie profondità delle fosse oceaniche, e alcune fortezze sorgono nei pressi delle coste da dove lanciano assalti continui contro i nemici che respirano aria che vivono vicino alla riva. Orgogliosi e bellicosi, i sahuagin si alleano raramente con altri, e vedono le altre razze acquatiche, come aboleth, marinidi e simili come concorrenti. Le sole creature che sembrano rispettare oltre ai loro simili sono gli squali; in questi implacabili predatori, infatti, i sahuagin rivedono molto di loro stessi. Un sahuagin è alto 2,1 metri e pesa circa 125 kg.\\

I sahuagin sono soggetti a mutazioni genetiche, e quando nasce un mutante assurge quasi sempre ai ranghi nobiliari o di comando nella società. La mutazione sahuagin più comune consiste in un paio di braccia extra (che concedono due attacchi addizionali con gli artigli o la possibilità di maneggiare più armi). Alcuni parlano dei rari malenti, sahuagin che non sembrano uomini squalo ma elfi acquatici, malgrado condividano la sete di sangue e la natura crudele dei loro simili. I malenti spesso servono come spie o assassini i governanti sahuagin, ma si narra di intere tribù composte di malenti in remote zone del mare.\\


\medskip\index{Mostri - Salamandra}\textbf{Salamandra}

\emph{Grande elementale, neutrale malvagio}

\textbf{FORZA} +4

\textbf{DESTREZZA} +2

\textbf{COSTITUZIONE} +2

\textbf{INTELLIGENZA} +0

\textbf{SAGGEZZA} +0

\textbf{CARISMA} +1

\textbf{Iniziativa} +2 -- \textbf{Difesa} 18

\textbf{Punti Ferita} 90 (12d10 + 24)

\textbf{Movimento} 9 m

\textbf{Tiri Salvezza}: Tempra +10, Riflessi +7, Volontà +6

\textbf{Vulnerabilità al Danno} freddo

\textbf{Resistenze al Danno} da botta, perforante e tagliente di attacchi non magici

\textbf{Immunità ai Danni} fuoco

\textbf{Sensi} scurovisione 18 m

\textbf{Linguaggi} Ignan

\textbf{Sfida} 5 (1.800 PE)

\emph{\textbf{Armi Riscaldate.}} Qualsiasi arma da mischia metallica che la salamandra impugni infligge 3 (1d6) danni da fuoco aggiuntivi per colpo (già incluso nell'attacco).

\emph{\textbf{Corpo Riscaldato.}} Una creatura che entri a contatto con la salamandra o la colpisce con un attacco da mischia mentre si trova entro 1 metro da essa subisce 7 (2d6) danni da fuoco.

\textbf{Azioni}

\emph{\textbf{Multiattacco.}} La salamandra effettua due attacchi: uno con la lancia e uno con la coda.

\emph{\textbf{Coda.} Attacco con arma da mischia}: +7 a colpire, portata 3 m, un bersaglio.

\emph{Colpisce:} 11 (2d6 + 4) danni da botta più 7 (2d6) danni da fuoco, e il bersaglio è afferrato (CD 14 per fuggire). Fino al termine dell'afferrare, il bersaglio è intralciato, la salamandra può colpire automaticamente il bersaglio con la coda, e la salamandra non può effettuare attacchi di coda contro altri bersagli.

\emph{\textbf{Lancia.} Attacco con arma da mischia o a Distanza}: +7 a colpire, portata 1 m, gittata 6m, un bersaglio.

\emph{Colpisce:} 11 (2d6 + 4) danni perforanti, o 13 (2d8 +4) danni perforanti se usata con due mani per effettuare un attacco da mischia, più 3 (1d6) danni da fuoco.

\textbf{Ecologia}
Ambiente: Qualsiasi (Piano del Fuoco)\\
Organizzazione: Solitario, coppia o gruppo (3-5)\\
Tesoro: Standard (Lancia, altro tesoro ininfiammabile)\\
\textbf{Descrizione}\\
Le Salamandre sono native del Piano del Fuoco, dove le loro legioni di fieri combattenti sono molto temute dagli altri abitanti del Piano. Poiché molte delle più forti Razze Elementali del Fuoco Schiavizzano le Salamandre per la loro Abilità nella metallurgia e capacità combattiva, le Salamandre odiano gli Efreet e gli altri con fervore.\\

Anche se i loro nascondigli superano i 250 gradi C di temperatura, le Salamandre possono tollerare temperature più basse. Generalmente lo fanno se costrette, e sono anche più burbere e irascibili del normale in questi ambienti. Sebbene provenga dal Piano del Fuoco, la Razza delle Salamandre si identifica di più con l'Abisso, e ha un grande rispetto per i Demoni (in particolare quelli associati col fuoco, come i Balor e certi Signori dei Demoni legati alle fiamme). Per questo non è insolito incontrare un grosso gruppo di Salamandre nell'Abisso.\\

Le Salamandre sono spesso evocate nel Piano Materiale per servire come guardiani o, più comunemente, come fabbricanti di Armature, Armi e altri oggetti metallurgici, dato che la loro Abilità in questo campo è leggendaria. Le Salamandre infestano anche quelle aree del Piano Materiale dove il confine tra questo mondo e il Piano del Fuoco si è fatto labile, come vicino e dentro i Vulcani.\\

Abitando zone così estreme, le Salamandre posseggono solo tesori che resistono alle alte temperature, come Spade, Armature, gioielli, Verghe e altri oggetti che hanno un alto punto di fusione. La società delle Salamandre è crudele e basata sul potere e sulla capacità di soggiogare chi è inferiore a loro. Gli esseri inferiori alle Salamandre che causano problemi affrontano una morte lenta e dolorosa.\\



\medskip\index{Mostri - Satiro}\textbf{Satiro}

\emph{Media fatato, caotico neutrale}

\textbf{FORZA} +1

\textbf{DESTREZZA} +3

\textbf{COSTITUZIONE} +0

\textbf{INTELLIGENZA} +1

\textbf{SAGGEZZA} +0

\textbf{CARISMA} +2

\textbf{Iniziativa} +3 -- \textbf{Difesa} 15 (armatura di cuoio)

\textbf{Punti Ferita} 31 (7d8)

\textbf{Vulnerabilità al Danno} ferro freddo

\textbf{Movimento} 12 m

\textbf{Tiri Salvezza}: Tempra +4, Riflessi +8, Volontà +8

\textbf{Competenze} Muoversi Silenziosamente / Nascondersi +5, Intrattenere +6, Consapevolezza +2

\textbf{Linguaggi} Comune, Elfico, Silvano

\textbf{Sfida} 1/2 (100 PE)

\emph{\textbf{Resistenza alla Magia.}} Il satiro ha +1d6 ai Tiri Salvezza contro incantesimi e altri effetti magici.

\textbf{Azioni}

\emph{\textbf{Incornata.} Attacco con arma da mischia}: +3 a colpire, portata 1 m, un bersaglio.

\emph{Colpisce:} 6 (2d4 + 1) danni da botta.

\emph{\textbf{Spada Corta.} Attacco con arma da mischia}: +5 a colpire, portata 1 m, un bersaglio.

\emph{Colpisce:} 6 (1d6 + 3) danni perforanti.

\emph{\textbf{Arco Corto.} Attacco con arma a Distanza}: +5 a colpire, gittata 24m, un bersaglio.

\emph{Colpisce:} 6 (1d6 + 3) danni perforanti.

\textbf{Ecologia}\\
Ambiente: Foreste Temperate\\
Organizzazione: Solitario, coppia, banda (3-6) o festino (7-11)\\
Tesoro: Standard (Pugnale, Arco Corto più 20 Frecce, flauto di pan perfetto, altro tesoro)\\
\textbf{Descrizione}\\
I satiri, conosciuti in molte regioni come fauni, sono creature debosciate ed edoniste delle parti più profonde e primordiali delle foreste. Adorano il vino, la musica e i piaceri della carne, sono rinomati come libertini e bellimbusti che corteggiano le fanciulle sprovvedute e i pastorelli e si lasciano dietro una scia di spiegazioni imbarazzanti e gravidanze indesiderate.\\

Anche se i loro corpi sono quasi sempre quelli di uomini attraenti e ben proporzionati, le capacità seduttive dei satiri risiedono nel loro talento musicale. Con l'aiuto del suo flauto, un satiro è capace di tessere una vasta gamma di incantesimi melodici ideati per affascinare gli altri e farli accondiscendere ai suoi capricciosi desideri.\\

Oltre ad amoreggiare costantemente, i satiri spesso fungono da guardiani delle loro foreste, e quanti riescono a trasformare la lussuria del fauno in ira probabilmente si troveranno di fronte i più pericolosi tra gli animali che circondano il fauno. Inoltre, anche se i satiri tendono a mettere il loro divertimento al di sopra dei diritti altrui, non covano alcun risentimento contro quelli che seducono.\\

I bambini nati da questi incontri sono sempre satiri di sangue puro e vengono generalmente portati via dai loro sfrenati padri subito dopo la nascita.\\


\medskip\index{Mostri - Scheletro}\textbf{Scheletro}

\emph{Media non morto, legale malvagio}

\textbf{FORZA} +0

\textbf{DESTREZZA} +2

\textbf{COSTITUZIONE} +2

\textbf{INTELLIGENZA} -2

\textbf{SAGGEZZA} -1

\textbf{CARISMA} -3

\textbf{Iniziativa} +2 -- \textbf{Difesa} 14 (pezzi di armatura)

\textbf{Punti Ferita} 13 (2d8 + 4)

\textbf{Movimento} 9 m

\textbf{Tiri Salvezza}: Tempra +0, Riflessi +2, Volontà +2

\textbf{Vulnerabilità al Danno} da botta

\textbf{Resistenze al Danno} perforante e tagliente di attacchi non magici

\textbf{Immunità al Danno} veleno

\textbf{Immunità alle Condizioni} avvelenato, affaticamento

\textbf{Sensi} scurovisione 18 m

\textbf{Linguaggi} comprende tutte le lingue che parlava in vita ma non può parlare

\textbf{Sfida} 1/4 (50 PE)

\emph{\textbf{Natura Non Morta.}} Lo scheletro non necessita aria, cibo, bevande o sonno.

\textbf{Azioni}

\emph{\textbf{Spada Corta.} Attacco con arma da mischia}: +4 a colpire, portata 1 m, un bersaglio.

\emph{Colpisce:} 5 (1d6 + 2) danni perforanti.

\emph{\textbf{Arco Corto.} Attacco con arma a Distanza}: +4 a colpire, gittata 24m, un bersaglio.

\emph{Colpisce:} 5 (1d6 + 2) danni perforanti.

\textbf{Ecologia}\\
Ambiente: Qualsiasi\\
Organizzazione: Qualsiasi\\
Tesoro: Nessuno (Giaco di Maglia Rotto, Scimitarra Rotta)\\
\textbf{Descrizione}\\
Gli scheletri sono ossa di morti animate, portate alla non vita da magie sacrileghe. Per la maggior parte, gli scheletri sono automi privi di volontà, ma possiedono un'astuzia malvagia concessa loro dalla forza che li anima: un'astuzia che permette loro di portare armi ed indossare armature.\\


\medskip\index{Mostri - Scheletro di Cavallo da Guerra}\textbf{Scheletro di Cavallo da Guerra}

\emph{Grande non morto, legale malvagio}

\textbf{FORZA} +4

\textbf{DESTREZZA} +1

\textbf{COSTITUZIONE} +2

\textbf{INTELLIGENZA} -4

\textbf{SAGGEZZA} -1

\textbf{CARISMA} -3

\textbf{Iniziativa} +1 -- \textbf{Difesa} 14 (pezzi di bardatura)

\textbf{Punti Ferita} 22 (3d10 + 6)

\textbf{Movimento} 18 m

\textbf{Tiri Salvezza}: Tempra +4, Riflessi +3, Volontà +1

\textbf{Vulnerabilità al Danno} da botta

\textbf{Resistenze al Danno} perforante e tagliente di attacchi non magici

\textbf{Immunità al Danno} veleno

\textbf{Immunità alle Condizioni} avvelenato, affaticamento

\textbf{Sensi} scurovisione 18 m 

\textbf{Linguaggi} -

\textbf{Sfida} 1/2 (100 PE)

\emph{\textbf{Natura Non Morta.}} Lo scheletro non necessita aria, cibo, bevande o sonno.

\textbf{Azioni}

\emph{\textbf{Zoccoli.} Attacco con arma da mischia}: +6 a colpire, portata 1 m, un bersaglio.

\emph{Colpisce:} 11 (2d6 + 4) danni da botta.

\medskip\index{Mostri - Scheletro di Minotauro}\textbf{Scheletro di Minotauro}

\emph{Grande non morto, legale malvagio}

\textbf{FORZA} +4

\textbf{DESTREZZA} +0

\textbf{COSTITUZIONE} +2

\textbf{INTELLIGENZA} -2

\textbf{SAGGEZZA} -1

\textbf{CARISMA} -3

\textbf{Iniziativa} +0 -- \textbf{Difesa} 13

\textbf{Punti Ferita} 67 (9d10 + 18)

\textbf{Movimento} 12 m

\textbf{Tiri Salvezza}: Tempra +6, Riflessi +3, Volontà +2

\textbf{Vulnerabilità al Danno} da botta

\textbf{Immunità al Danno} veleno

\textbf{Resistenze al Danno} perforante e tagliente di attacchi non magici

\textbf{Immunità alle Condizioni} avvelenato, affaticamento

\textbf{Sensi} scurovisione 18 m

\textbf{Linguaggi} comprende l'Abissale ma non può parlare

\textbf{Sfida} 2 (450 PE)

\emph{\textbf{Carica.}} Se lo scheletro di minotauro si muove di almeno 3 metri in linea retta verso il bersaglio e poi lo colpisce con un attacco di incornata durante lo stesso turno, il bersaglio subisce 9 (2d8) danni perforanti aggiuntivi. Se il bersaglio è una creatura, deve riuscire un Tiro Salvezza di Tempra CD 14 o venire spinto di 3 metri indietro e cadere prono.

\emph{\textbf{Natura Non Morta.}} Lo scheletro non necessita aria, cibo, bevande o sonno.

\textbf{Azioni}

\emph{\textbf{Ascia Bipenne.} Attacco con arma da mischia}: +6 a colpire, portata 1 m, un bersaglio.

\emph{Colpisce:} 17 (2d12 + 4) danni taglienti.

\emph{\textbf{Incornata.} Attacco con arma da mischia}: +6 a colpire, portata 1 m, un bersaglio.

\emph{Colpisce:} 13 (2d8 + 4) danni perforanti.

\medskip\index{Mostri - Segugio Infernale}\textbf{Segugio Infernale}

\emph{Media immondo, legale malvagio}

\textbf{FORZA} +3

\textbf{DESTREZZA} +1

\textbf{COSTITUZIONE} +2

\textbf{INTELLIGENZA} -2

\textbf{SAGGEZZA} +1

\textbf{CARISMA} -2

\textbf{Iniziativa} +1 -- \textbf{Difesa} 17

\textbf{Punti Ferita} 45 (7d8 + 14)

\textbf{Movimento} 15 m

\textbf{Tiri Salvezza}: Tempra +6, Riflessi +5, Volontà +1

\textbf{Competenze} Consapevolezza +5

\textbf{Immunità al Danno} fuoco

\textbf{Sensi} scurovisione 18 m

\textbf{Linguaggi} comprende l'Infernale ma non può parlare

\textbf{Sfida} 3 (700 PE)

\emph{\textbf{Udito e Olfatto Affinato.}} Il segugio ha +1d6 nelle prove di Saggezza (Consapevolezza) basate su udito od olfatto.

\emph{\textbf{Tattiche di Branco.}} Il segugio ha +1d6 ai tiri per colpire contro una creatura se almeno uno degli alleati del segugio si trova entro 1 metro dalla creatura e quell'alleato non è inabile.

\textbf{Azioni}

\emph{\textbf{Morso.} Attacco con arma da mischia}: +5 a colpire,
portata 1 m, un bersaglio.

\emph{Colpisce:} 7 (1d6 + 3) danni perforanti più 7 (2d6) danni da
fuoco.

\emph{\textbf{Soffio Infuocato (Ricarica 5-6).}} Il segugio esala fuoco in un cono di 5 metri. Ogni creatura in quell'area deve effettuare un Tiro Salvezza di Riflessi CD 12, e subire 21 (6d6) danni da fuoco se fallisce il Tiro Salvezza, o la metà di questi danni se lo riesce.



\subsection{Sfingi}

\medskip\index{Mostri - Androsfinge}\textbf{Androsfinge}

\emph{Grande mostruosità, legale neutrale}

\textbf{FORZA} +6

\textbf{DESTREZZA} +0

\textbf{COSTITUZIONE} +5

\textbf{INTELLIGENZA} +3

\textbf{SAGGEZZA} +4

\textbf{CARISMA} +6

\textbf{Iniziativa} +3 -- \textbf{Difesa} 26

\textbf{Punti Ferita} 199 (19d10 + 95)

\textbf{Movimento} 12 m, volo 18 m

\textbf{Tiri Salvezza}: Tempra +12, Riflessi +8, Volontà +7

\textbf{Competenze} Arcano +9, Consapevolezza +10, Religione +15

\textbf{Immunità al Danno} psichico; da botta, perforante e tagliente di attacchi non magici

\textbf{Immunità alle Condizioni} affascinato, spaventato 

\textbf{Sensi} visione del vero 36 m 

\textbf{Linguaggi} Comune, Sfinge

\textbf{Sfida} 17 (18.000 PE)

\emph{\textbf{Armi Magiche.}} Gli attacchi con armi della sfinge sono magici.

\emph{\textbf{Imperscrutabile.}} La sfinge è immune a qualsiasi effetto in grado di percepirne le emozioni o leggerne i pensieri, oltre che a qualsiasi incantesimo di divinazione che rifiuti. Le prove di Saggezza (Percepire Emozioni) per discernere le intenzioni o la sincerità della sfinge hanno -1d6.

\emph{\textbf{Incantesimi.}} La sfinge ha CM 12.
La sua caratteristica da incantatore è la Saggezza (CD del Tiro Salvezza degli incantesimi 18, +10 a colpire con attacchi con incantesimo). Non ha bisogno di componenti materiali per lanciare i suoi incantesimi. La sfinge tiene preparati i seguenti incantesimi:

Trucchetti (a volontà): \emph{fiamma sacra, salvare i morenti,} \emph{taumaturgia}

Difficoltà 16 (4 slot): \emph{comando, individuazione del magico,} \emph{individuare male e bene}

Difficoltà 19 (3 slot): \emph{ristorare inferiore, zona di verità}

Difficoltà 21 (3 slot): \emph{dissolvi magie, linguaggi}

Difficoltà 23 (3 slot): \emph{esilio, libertà di movimento}

Difficoltà 26 (2 slot): \emph{colpo infuocato, ristorare superiore}

Difficoltà 29 (1 slot): \emph{banchetto degli eroi}

\textbf{Azioni}

\emph{\textbf{Multiattacco.}} La sfinge può effettuare due attacchi di artiglio.

\emph{\textbf{Artiglio.} Attacco con arma da mischia}: +12 a colpire, portata 1 m, un bersaglio.

\emph{Colpisce:} 17 (2d6 + 10) danni taglienti.

\emph{\textbf{Ruggito (3/Giorno).}} La sfinge emette un ruggito magico. Ogni volta che ruggisce prima di una nuova alba, il ruggito più forte e l'effetto è diverso, come dettagliato di seguito. Ogni creatura entro 150 metri dalla sfinge e capace di udirne il ruggito deve effettuare un Tiro Salvezza.

\textbf{Primo Ruggito.} Ogni creatura che fallisce un Tiro Salvezza su Volontà CD 18 resta spaventata per 1 minuto. Una creatura spaventata può ripetere il Tiro Salvezza al termine di ciascun suo turno, terminandone l'effetto per sé, se lo riesce.

\textbf{Secondo Ruggito.} Ogni creatura che fallisce un Tiro Salvezza su Volontà CD 18 resta assordata e spaventata per 1 minuto. Una creatura spaventata è paralizzata e può ripetere il Tiro Salvezza al termine di ciascun suo turno, terminandone l'effetto per sé, se lo riesce.

\textbf{Terzo Ruggito.} Ogni creatura effettua un Tiro Salvezza su Tempra CD 18. Chi fallisce il Tiro Salvezza subisce 44 (8d10) danni da tuono ed è gettato prono. Se il Tiro Salvezza riesce, la creatura subisce la metà di questi danni e non viene gettata prona. 

\textbf{Azioni Aggiuntive}

La sfinge può effettuare 3 Azioni aggiuntive, scelte tra le opzioni seguenti. Può usare solo un'opzione leggendaria alla volta e solo al termine del turno di un'altra creatura. La sfinge recupera le Azioni aggiuntive spese all'inizio del proprio turno.

\textbf{Attacco di Artiglio.} La sfinge effettua un attacco di artiglio. 

\textbf{Eseguire un Incantesimo (Costa 3 Azioni).} La sfinge lancia un incantesimo dalla lista degli incantesimi preparati, utilizzando uno slot incantesimo come di norma. 

\textbf{Teletrasporto (Costa 2 Azioni).} La sfinge si teletrasporta magicamente, insieme a tutto l'equipaggiamento che sta indossando o trasportando, in uno spazio non occupato che possa vedere, fino a 36 metri di distanza.


\textbf{Ecologia}\\
Ambiente: Colline o Deserti Caldi\\
Organizzazione: Solitario\\
Tesoro: Standard\\
\textbf{Descrizione}\\
Le androsfingi, le più potenti tra le sfingi comuni, ritengono di rappresentare tutto ciò che c'è di degno e nobile nella loro specie e si atteggiano come se il peso del mondo intero poggiasse sul loro buon esempio. Guardano le Criosfingi con sufficienza paternalistica, le Ieracosfingi con malcelato disgusto e le Ginosfingi come le uniche altre sfingi degne del loro tempo.\\

Le androsfingi ostentano una facciata scorbutica e astiosa nei confronti degli stranieri. Non si sforzano in alcun modo di celare il loro fastidio quando sono irritate. Tendono inoltre a essere gelose del loro territorio, anche se meno delle altre sfingi. Quasi inevitabilmente lanciano avvertimenti e proclami roboanti prima di attaccare, e rispettano quasi sempre un appello a trattare. Le androsfingi barattano informazioni e conversazioni, e non tesori, in cambio di un passaggio sicuro.\\
Le androsfingi sono alte 3,6 metri e pesano 500 kg.\\


\medskip\index{Mostri - Ginosfinge}\textbf{Ginosfinge}

\emph{Grande mostruosità, legale neutrale}

\textbf{FORZA} +4

\textbf{DESTREZZA} +2

\textbf{COSTITUZIONE} +3

\textbf{INTELLIGENZA} +4

\textbf{SAGGEZZA} +4

\textbf{CARISMA} +4

\textbf{Iniziativa} +4 -- \textbf{Difesa} 23

\textbf{Punti Ferita} 136 (16d10 + 48)

\textbf{Movimento} 12 m, volo 18 m

\textbf{Tiri Salvezza}: Tempra +11, Riflessi +9, Volontà +10

\textbf{Competenze} Arcano +14, Consapevolezza +9, Religione +9, Storia +14

\textbf{Resistenze al Danno} da botta, perforante e tagliente di attacchi non magici

\textbf{Immunità al Danno} psichico

\textbf{Immunità alle Condizioni} affascinato, spaventato 

\textbf{Sensi} visione del vero 36 m

\textbf{Linguaggi} Comune, Sfinge

\textbf{Sfida} 11 (7.200 PE)

\emph{\textbf{Armi Magiche.}} Gli attacchi con armi della sfinge sono magici.

\emph{\textbf{Imperscrutabile.}} La sfinge è immune a qualsiasi effetto in grado di percepirne le emozioni o leggerne i pensieri, oltre che a qualsiasi incantesimo di divinazione che rifiuti. Le prove di Saggezza (Percepire Inganni) per discernere le intenzioni o la sincerità della sfinge hanno -1d6.

\emph{\textbf{Incantesimi.}} La sfinge ha CM 9. La sua abilità da incantatore è l'Intelligenza (CD del Tiro Salvezza degli incantesimi 17, +9 a colpire con attacchi da incantesimo). Non ha bisogno di componenti materiali per eseguire i suoi incantesimi. La sfinge tiene preparati i seguenti incantesimi: Trucchetti (a volontà): \emph{illusione minore, mano magica,} \emph{prestidigitazione}

Difficoltà 16 (4 slot): \emph{identificare, individuazione del magico, scudo}

Difficoltà 19 (3 slot): \emph{localizza oggetto, oscurità, suggestione} 

Difficoltà 21 (3 slot): \emph{dissolvi magie, linguaggi, rimuovi maledizione}

Difficoltà 23 (3 slot): \emph{esilio, invisibilità superiore}

Difficoltà 26 (2 slot): \emph{conoscenza delle leggende}

\textbf{Azioni}

\emph{\textbf{Multiattacco.}} La sfinge può effettuare due attacchi di artiglio.

\emph{\textbf{Artiglio.} Attacco con arma da mischia}: +9 a colpire, portata 1 m, un bersaglio.

\emph{Colpisce:} 13 (2d8 + 4) danni taglienti.

\textbf{Azioni Aggiuntive}

La sfinge può effettuare 3 Azioni aggiuntive, scelte tra le opzioni seguenti. Può usare solo un'opzione leggendaria alla volta e solo al termine del turno di un'altra creatura. La sfinge recupera le Azioni aggiuntive spese all'inizio del proprio turno.

\textbf{Attacco di Artiglio.} La sfinge effettua un attacco di artiglio. 

\textbf{Eseguire un Incantesimo (Costa 3 Azioni).} La sfinge esegue un incantesimo dalla lista degli incantesimi preparati, utilizzando uno slot incantesimo come di norma.

\textbf{Teletrasporto (Costa 2 Azioni).} La sfinge si teletrasporta magicamente, insieme a tutto l'equipaggiamento che sta indossando o trasportando, in uno spazio non occupato che possa vedere, fino a 36 metri di distanza.

\textbf{Ecologia}
Ambiente: Deserti e colline caldi\\
Organizzazione: Solitario, coppia o culto (3-6)\\
Tesoro: Doppio\\
\textbf{Descrizione}\\
Anche se esistono diversi tipi di sfinge, quella alla quale gli studiosi si riferiscono come Ginosfinge (un nome che molte sfingi trovano offensivo) è una creatura saggia e maestosa ma al contempo terrificante se arrabbiata. Meno moraliste delle loro controparti maschili (le Androsfingi, creature totalmente differenti da quella presentata qui), le sfingi sono prudenti e metodiche quando prendono delle decisioni, e sono orgogliose della loro fredda logica e della loro imparzialità. Hanno poca pazienza con le varianti inferiori di sfingi, considerandole poco più che animali. Le sfingi amano gli enigmi e gli indovinelli complicati, e fanno tesoro di fatti insoliti e dilemmi arcani molto più che di oro o gemme.\\

Pur non essendo grandi studiose in senso tradizionale, il grande apprezzamento delle sfingi per gli enigmi le porta a compiere ricerche in una grande varietà di materie, rendendole spesso una preziosa fonte di informazioni, specialmente quando fanno uso delle loro capacità magiche. Di solito sono felici di avere contatti con altre razze, ed offrono regolarmente beni materiali in cambio di informazioni o di indovinelli nuovi ed interessanti. Sono eccellenti guardiane di templi, tombe ed altri luoghi importanti, fintanto che vengono intrattenute in maniera adeguata. Le sfingi danno grande importanza alla gentilezza, ma possono essere capricciose: possono decidere altruisticamente di dividere i loro ultimi enigmi con dei viaggiatori ma non ci pensano due volte a divorarli se non vi prestano abbastanza attenzione o non forniscono alcun indizio utile alla loro risoluzione.\\

Una tipica sfinge è lunga 3 metri e pesa circa 400 kg. Anche se le loro ali possono tenerle in aria per lunghi periodi di tempo, sono delle volatrici scarse, e preferiscono atterrare prima di iniziare a combattere, attaccando con i loro poderosi artigli. Nonostante siano estremamente territoriali, le sfingi tendono ad avvisare gli intrusi varie volte prima di attaccare.\\


\medskip\index{Mostri - Spiritello}\textbf{Spiritello}

\emph{Minuscola fatato, neutrale buono}

\textbf{FORZA} -4

\textbf{DESTREZZA} +4

\textbf{COSTITUZIONE} +0

\textbf{INTELLIGENZA} +2

\textbf{SAGGEZZA} +1

\textbf{CARISMA} +0

\textbf{Iniziativa} +4 -- \textbf{Difesa} 16 (armatura di cuoio)

\textbf{Punti Ferita} 2 (1d4)

\textbf{Vulnerabilità al Danno} ferro freddo

\textbf{Movimento} 3 m, volo 12 m

\textbf{Tiri Salvezza}: Tempra +0, Riflessi +5, Volontà +2

\textbf{Competenze} Muoversi Silenziosamente / Nascondersi +8 (la prova è fatta con -1d6 se lo spiritello sta volando), Consapevolezza +3

\textbf{Linguaggi} Comune, Elfico, Silvano

\textbf{Sfida} 1/4 (50 PE)

\textbf{Azioni}

\emph{\textbf{Spada Lunga.} Attacco con arma da mischia}: +2 a colpire,

portata 1 m, un bersaglio.

\emph{Colpisce:} 1 danno tagliente.

\emph{\textbf{Arco Corto.} Attacco con arma a Distanza}: +6 a colpire, gittata 12 m, un bersaglio.

\emph{Colpisce:} 1 danno perforante. Se il bersaglio è una creatura, deve riuscire un Tiro Salvezza di Tempra CD 10 o restare avvelenata per 1 minuto. Se il risultato di questo Tiro Salvezza è 5 o meno, il bersaglio cade privo di sensi per la stessa durata, o finché subisce danni o un'altra creatura usa un'azione per risvegliarlo. 

\emph{\textbf{Invisibilità.}} Lo spiritello resta invisibile finché non attacca o termina la sua concentrazione. Qualsiasi cosa che lo spiritello stia trasportando o indossando resta invisibile finché rimane in contatto con lo spiritello.

\emph{\textbf{Vista del Cuore.}} Lo spiritello entra in contatto con una creatura e ne apprende l'attuale stato emotivo. Se il bersaglio fallisce un Tiro Salvezza di Tempra CD 10, lo spiritello apprende anche l'allineamento della creatura. Celestiali, immondi e non morti falliscono automaticamente questo Tiro Salvezza.

\textbf{Descrizione}\\
Gli spiritelli si riuniscono in gruppi nelle profondità di regioni boschive, uniti nella causa per proteggere la natura. Intere tribù di spiritelli si sono dichiarate protettrici di una determinata persona, di un luogo o di una creatura di particolare rilievo nelle loro terre, anche nel caso in cui l'essere non desideri o non necessiti di alcuna protezione.\\

Il corpo di uno spiritello è luminoso per natura, sebbene la creatura possa variare il colore e l'intensità della luce emessa dal suo corpo a piacimento. Subito dopo la sua morte, il corpo di uno spiritello si dissolve in una nebbia luccicante. Gli spiritelli sono i più piccoli tra i folletti, alti poco più di 22 centimetri e di un peso che raramente supera 1 kg.\\

Sotto molti aspetti gli spiritelli sono più primitivi della maggior parte dei folletti. Apprezzano la compagnia dei propri simili, ma tendono a diffidare degli altri folletti e presumono che qualsiasi umanoide o creatura che non hanno espressamente scelto di proteggere voglia far loro del male. Persino gli animali vengono da loro solitamente considerati pericolosi. La ragione di questa diffidenza è per buona parte dovuta alla taglia minuscola di queste creature, che le rende facili prede per i predatori. Pertanto la reazione iniziale di uno spiritello di fronte a un pericolo è darsi alla fuga: in genere utilizza le sue capacità magiche per rallentare o distrarre gli inseguitori, e in seguito si affida alla sua velocità di volare e alla sua taglia per riuscire a fuggire.\\

Sebbene gli spiritelli di per sé abbiano una natura incolta e selvaggia, hanno una sana curiosità per tutte le cose dotate di magia innata. Sono particolarmente attratti dai luoghi di grande potere magico latente, quali le rovine di antichi templi. Questa curiosità li rende anche insolitamente adatti al ruolo di famigli. Un incantatore caotico neutrale di 5° livello può ottenere uno spiritello come famiglio se ha l'Abilita' Famiglio.\\


\medskip\index{Mostri - Strige (Uccello Stigeo)}\textbf{Strige (Uccello Stigeo)}

\emph{Minuscola bestia, disallineato}

\textbf{FORZA} -3

\textbf{DESTREZZA} +3

\textbf{COSTITUZIONE} +0

\textbf{INTELLIGENZA} -4

\textbf{SAGGEZZA} -1

\textbf{CARISMA} -2

\textbf{Iniziativa} +3 -- \textbf{Difesa} 15

\textbf{Punti Ferita} 2 (1d4)

\textbf{Movimento} 3 m, volo 12 m

\textbf{Tiri Salvezza}: Tempra +2, Riflessi +6, Volontà +1

\textbf{Sensi} scurovisione 18 m

\textbf{Linguaggi} -

\textbf{Sfida} 1/8 (25 PE)

\textbf{Azioni}

\emph{\textbf{Risucchio di Sangue.} Attacco con arma da mischia}: +5 a colpire, portata 1 m, una creatura.

\emph{Colpisce:} 5 (1d4 + 3) danni perforanti e lo strige si attacca al bersaglio. Mentre è attaccato, lo strige non attacca. Invece, all'inizio di ciascun turno dello strige, il bersaglio perde 5 (1d4 + 3) punti ferita a causa della perdita di sangue.

Lo strige può staccarsi spendendo 1 metro di movimento. Lo fa automaticamente dopo aver risucchiato 10 punti ferita dal bersaglio o alla morte del bersaglio. Una creatura, compreso il bersaglio, può usare la sua azione per staccare lo strige.

\textbf{Ecologia}
Ambiente: Paludi temperate e calde\\
Organizzazione: Solitario, colonia (2-4), stormo (5-8), nugolo (9-14) o sciame (15-40)\\
Tesoro: Nessuno\\
\textbf{Descrizione}\\
Gli strige sono pericolosi succhiasangue che infestano le paludi e predano animali selvatici, bestiame ed ignari viaggiatori. Pur essendo deboli individualmente, sciami di queste creature sono capaci di prosciugare un uomo in pochi minuti, lasciando dietro a loro solo un cadavere essiccato.\\

Più simili ai mammiferi che agli insetti, gli strige si alzano in volo con le loro quattro ali di carne, cercando prede a sangue caldo. Spesso si nascondono vicino a pozze di acqua bevibile aspettando che i viaggiatori abbassino la guardia per poi attaccarli e bere a sazietà, conficcando le loro proboscidi nelle vene scoperte. Dopo essersi nutriti, volano via a nascondersi tra la fanghiglia e tra i canneti per deporre le loro uova e riposare finché la fame non li spinge a cacciare di nuovo.\\

Di solito gli strige sono lunghi circa 30 centimetri, con un'apertura alare di circa il doppio, e pesano meno di 0,5 kg. Sono color rosso ruggine o marrone rossastro, ed hanno il ventre color giallo sporco, ma quelli che non si sono nutriti adeguatamente sono di colore rosa pallido.\\


\medskip\index{Mostri - Succube}\textbf{Succube}

\emph{Media immondo (mutaforma), neutrale malvagio}

\textbf{FORZA} -1

\textbf{DESTREZZA} +3

\textbf{COSTITUZIONE} +1

\textbf{INTELLIGENZA} +2

\textbf{SAGGEZZA} +1

\textbf{CARISMA} +5

\textbf{Iniziativa} +3 -- \textbf{Difesa} 17

\textbf{Punti Ferita} 66 (12d8 + 12)

\textbf{Movimento} 9 m, volo 18 m

\textbf{Tiri Salvezza}: Tempra +7, Riflessi +9, Volontà +10

\textbf{Competenze} Furtività 5, Percepire Emozioni +5, Consapevolezza +5, Ingannare +9

\textbf{Resistenze al Danno} freddo, fulmine, fuoco, veleno; da botta, perforante e tagliente di attacchi non magici

\textbf{Sensi} scurovisione 18 m

\textbf{Linguaggi} Abissale, Comune, Infernale, telepatia 18 m

\textbf{Sfida} 4 (1.100 PE)

\emph{\textbf{Legame Telepatico.}} L'immondo ignora le restrizioni di raggio di azione della sua telepatia quando comunica con una creatura che ha affascinato. I due non sono neppure costretti a trovarsi sullo stesso piano di esistenza.

\emph{\textbf{Mutaforma.}} L'immondo può usare la sua azione per trasformarsi in un umanoide di taglia Piccola o Media, o per tornare alla sua vera forma. Senza le ali, l'immondo perde la velocità di volo. A parte la taglia e la velocità, le sue statistiche sono le stesse in tutte le forme. Qualsiasi equipaggiamento stia indossando o trasportando non viene trasformato. Alla morte ritorna alla sua vera forma.

\textbf{Azioni}

\emph{\textbf{Artiglio (Solo Forma Immonda).} Attacco con arma da
	mischia}:

+5 a colpire, portata 1 m, un bersaglio.

\emph{Colpisce:} 6 (1d6 + 3) danni taglienti.

\emph{\textbf{Affascinare.}} Un umanoide visibile all'immondo entro 9 metri da esso deve riuscire un Tiro Salvezza di Volontà CD 15 o restare magicamente affascinato per 1 giorno. Il bersaglio affascinato obbedisce ai comandi verbali o telepatici dell'immondo. Se il bersaglio subisce danni o riceve un comando suicida, può ripetere il Tiro Salvezza, terminando l'effetto se lo riesce. Se ilbersaglio riesce il tiro  salvezza contro l'effetto, o se l'effetto termina, il bersaglio è immune all'Affascinare dell'immondo per le successive 24 ore.

L'immondo può tenere affascinato solo un bersaglio alla volta. Se ne affascina un altro, l'effetto sul bersaglio precedente termina.

\emph{\textbf{Bacio Risucchiante.}} L'immondo bacia una creatura affascinata o una creatura consenziente. Il bersaglio deve effettuare un Tiro Salvezza di Tempra CD 15 contro questa magia, subendo 32 (5d10 + 5) danni psichici se lo fallisce, o la metà di questi danni se lo riesce. I punti ferita massimi del bersaglio vengono ridotti di un ammontare pari ai danni subiti. Questa riduzione perdura finché non sorge l'alba. Il bersaglio muore se questo effetto riduce i suoi punti ferita massimi a 0.

\emph{\textbf{Forma Eterea.}} L'immondo entra magicamente nel Piano Etereo dal Piano Materiale, e viceversa.

\textbf{Ecologia}\\
Ambiente: Qualsiasi (Abisso)\\
Organizzazione: Solitario, coppia o harem (3-12)\\
Tesoro: doppio\\
\textbf{Descrizione}\\
Tra le orde demoniache una succube spesso può raggiungere altissimi livelli di potere, utilizzando le sue manipolazioni ed il suo fascino sensuale, e molte guerre demoniache imperversano a causa delle subdole macchinazioni di queste creature. Una succube si origina dalle anime di malvagi mortali particolarmente libidinosi ed avidi.\\


\medskip\index{Mostri - Tarrasque}\textbf{Tarrasque}

\emph{Mastodontica mostruosità (titano), disallineato}

\textbf{FORZA} +10

\textbf{DESTREZZA} +0

\textbf{COSTITUZIONE} +10

\textbf{INTELLIGENZA} -4

\textbf{SAGGEZZA} +0

\textbf{CARISMA} +0

\textbf{Iniziativa} +0 -- \textbf{Difesa} 35

\textbf{Punti Ferita} 676 (33d20 + 330)

\textbf{Movimento} 12 m

\textbf{Tiri Salvezza}: Tempra +31, Riflessi +22, Volontà +12

\textbf{Immunità al Danno} fuoco, veleno; armi +2

\textbf{Immunità alle Condizioni} affascinato, avvelenato, paralizzato, spaventato

\textbf{Sensi} vista cieca 36 m

\textbf{Linguaggi} -

\textbf{Sfida} 30 (155.000 PE)

\emph{\textbf{Carapace Riflettente.}} Ogni volta che il tarrasque è il bersaglio di un incantesimo \emph{dardo incantato}, un incantesimo a linea, o un incantesimo che richiede un tiro di attacco a gittata, tira un d6. Da 1 a 5, il tarrasque lo ignora. Con 6, il tarrasque lo ignora, e l'effetto viene riflesso contro l'incantatore come se fosse originato dal tarrasque, trasformando l'incantatore nel bersaglio. 

\emph{\textbf{Mostro d'Assedio.}} Il tarrasque infligge danni doppi agli oggetti e le strutture.

\emph{\textbf{Resistenza Leggendaria (3/Giorno).}} Se il tarrasque fallisce un Tiro Salvezza, può scegliere invece di riuscire.

\emph{\textbf{Resistenza alla Magia.}} Il tarrasque ha +1d6 ai Tiri Salvezza contro incantesimi o altri effetti magici.

\textbf{Azioni}

\emph{\textbf{Multiattacco.}} Il tarrasque può usare la sua Presenza Spaventosa. Poi effettua cinque attacchi: uno con il morso, due con gli artigli, uno con le corna, e uno con la coda. Al posto del morso può usare Inghiottire.

\emph{\textbf{Artiglio.} Attacco con arma da mischia}: +19 a colpire, portata 5 metri, un bersaglio.

\emph{Colpisce:} 28 (4d8 + 10) danni taglienti.

\emph{\textbf{Coda.} Attacco con arma da mischia}: +19 a colpire,
portata 6 m, un bersaglio.

\emph{Colpisce:} 24 (4d6 + 10) danni da botta. Se il bersaglio è una creatura, deve riuscire un Tiro Salvezza di Tempra CD 20 o cadere prona.

\emph{\textbf{Corna.} Attacco con arma da mischia}: +19 a colpire, portata 3 m, un bersaglio.

\emph{Colpisce:} 32 (4d10 + 10) danni perforanti.

\emph{\textbf{Morso.} Attacco con arma da mischia}: +19 a colpire, portata 3 m, un bersaglio.

\emph{Colpisce:} 36 (4d12 + 10) danni perforanti. Se il bersaglio è una creatura, è afferrata (CD 20 per fuggire). Fino al termine dell'afferrare, il bersaglio è intralciato, e il tarrasque non può usare il morso contro un altro bersaglio.

\emph{\textbf{Inghiottire.}} Il tarrasque effettua una attacco di morso contro un bersaglio di taglia Grande o inferiore che sta afferrando. Se l'attacco colpisce, il bersaglio è inghiottito, e l'afferrare ha termine. Il bersaglio inghiottito è accecato e intralciato, ha copertura totale contro gli attacchi e altri effetti all'esterno del tarrasque, e subisce 56 (16d6) danni da acido all'inizio di ciascun turno del tarrasque.

Se il tarrasque subisce 60 o più danni in un singolo turno da una creatura al suo interno, il tarrasque deve riuscire un Tiro Salvezza su Tempra CD 30 al termine di quel turno o vomitare tutte le creature inghiottite, che cadono prone in uno spazio entro 3 metri dal tarrasque. Se il tarrasque muore, una creatura inghiottita non  più intralciata da esso e può uscire dal cadavere utilizzando 9 metri  di movimento, uscendo prona. 

\emph{\textbf{Presenza Spaventosa.}} Ogni creatura scelta dal tarrasque, che si trovi entro 36 metri da esso e consapevole della sua presenza, deve riuscire un Tiro Salvezza di Volontà CD 17 o restare spaventata per 1 minuto. Una creatura può ripetere il Tiro Salvezza al termine di ciascun suo turno, con -1d6 se il tarrasque è in linea di visuale, terminando l'effetto per sé, se lo riesce. Se il Tiro Salvezza della creatura ha successo o l'effetto ha termine per essa, la creatura è immune alla Presenza Spaventosa del tarrasque per le successive 24 ore.

\textbf{Azioni Aggiuntive}

Il tarrasque può effettuare 3 Azioni aggiuntive, scelte tra le opzioni seguenti. Può usare solo un'opzione leggendaria alla volta e solo al termine del turno di un'altra creatura. Il tarrasque recupera le azioni aggiuntive spese all'inizio del proprio turno.

\textbf{Attacco.} Il tarrasque effettua un attacco di artiglio o di coda. \textbf{Masticare (Costa 2 Azioni).} Il tarrasque effettua un attacco di morso o usa Inghiottire.

\textbf{Muoversi.} Il tarrasque si muove fino a metà del suo movimento.


\textbf{Ecologia}\\
Ambiente: Qualsiasi\\
Organizzazione: Solitario\\
Tesoro: Nessuno\\
\textbf{Descrizione}\\
Il leggendario tarrasque è fra i mostri più distruttivi del mondo. Fortunatamente, passa la maggior parte del suo tempo in una specie di profondo letargo in una sconosciuta caverna in un remoto angolo del mondo. Quando si risveglia, però, muoiono interi regni.\\
Pur non particolarmente intelligente, il tarrasque è abbastanza intelligente da capire alcune parole in Parlata delle Profondità (pur non potendo parlare). Allo stesso modo, la furia non è incontrollata: si concentra sulla creatura che l'ha danneggiato maggiormente ed è difficile distrarlo con l'inganno.\\


\medskip\index{Mostri - Testuggine Dragona}\textbf{Testuggine Dragona}

\emph{Mastodontica drago, neutrale}

\textbf{FORZA} +7

\textbf{DESTREZZA} +0

\textbf{COSTITUZIONE} +5

\textbf{INTELLIGENZA} +0

\textbf{SAGGEZZA} +1

\textbf{CARISMA} +1

\textbf{Iniziativa} +0 -- \textbf{Difesa} 29

\textbf{Punti Ferita} 341 (22d20 + 110) 

\textbf{Movimento} 6 m, nuoto 12 m

\textbf{Tiri Salvezza} Tempra +12, Riflessi +8, Volontà +9

\textbf{Sensi} scurovisione 18 m

\textbf{Linguaggi} Aquan, Draconico

\textbf{Sfida} 17 (18.000 PE)

\emph{\textbf{Anfibio.}} La testuggine dragona può respirare aria e acqua.

\textbf{Azioni}

\emph{\textbf{Multiattacco.}} Il drago può effettuare tre attacchi: uno con il morso e due con gli artigli. Può effettuare un attacco di coda al posto di due attacchi di artiglio.

\emph{\textbf{Artiglio.} Attacco con arma da mischia}: +13 a colpire, portata 3 m, un bersaglio.

\emph{Colpisce:} 16 (2d8 + 7) danni taglienti.

\emph{\textbf{Coda.} Attacco con arma da mischia}: +13 a colpire, portata 5 metri, un bersaglio.

\emph{Colpisce:} 26 (3d12 + 7) danni da botta. Se il bersaglio è una creatura, deve riuscire un Tiro Salvezza di Tempra CD 20 o venire spinta di 3 metri lontano dalla testuggine dragona e cadere prona.

\emph{\textbf{Morso.} Attacco con arma da mischia}: +13 a colpire, portata 5 metri, un bersaglio.

\emph{Colpisce:} 26 (3d12 + 7) danni perforanti.

\emph{\textbf{Soffio di Vapore (Ricarica 5-6).}} La testuggine dragona esala un vapore caldo in un cono di 18 metri. Ogni creatura in quell'area deve effettuare un Tiro Salvezza di Tempra CD 18 e subire 52 (15d6) danni da fuoco se fallisce il Tiro Salvezza, o la metà di questi danni se lo riesce. Trovarsi sott'acqua non dà resistenza contro questo tipo di danno.

\textbf{Ecologia}
Ambiente: Acquatico temperato\\
Organizzazione: Solitario\\
Tesoro: Doppio\\
\textbf{Descrizione}\\
Le testuggini dragone abitano nelle acque dolci e salate, dove si attestano tra i più grandi pericoli per i marinai e coloro che viaggiano per nave attraverso le rotte marine del mondo. Gli esperti marinai sanno quello che vogliono le testuggini dragone della zona e frequentemente fanno offerte in oro e magia per garantirsi un passaggio sicuro o evitano completamente l'area. Da parte sua, una testuggine dragona apprezza ed anche si aspetta tali pedaggi e regalie, e una testuggine dragona che si aspetta regali ma viene ignorata è davvero un nemico pericoloso.\\

Il colore del guscio di una testuggine dragona varia da individuo a individuo. Alcuni hanno gusci opachi marrone e rosso ruggine, mentre altri hanno carapaci di un intenso color verde-blu con riflessi argentei sulle punte rocciose. La colorazione di testa, coda e zampe è lievemente più pallida del guscio e comprende striature dorate lungo la cresta e le spine.\\
Le testuggini dragone reclamano enormi territori in mare aperto, che comprendono regioni che spesso superano i 75 km quadrati. Qui, queste bestie pericolose capovolgono le navi che non rispettano i loro territori, aggiungendo relitti sommersi ed i loro preziosi carichi ai loro nascondigli. Le testuggini dragone generalmente fanno le loro tane in profonde caverne accessibili solo attraverso l'acqua, e spesso non solo le decorano con le ricchezze trafugate dalle navi che hanno affondato, ma anche coi relitti di queste sfortunate imbarcazioni. La loro natura territoriale e la loro predilezione per questo tipo di tane le mettono in conflitto diretto con le altre razze sottomarine come Marinidi e Sahuagin.\\

I grandi pesci, come tonni, storioni ed anche squali sono compresi tra i cibi preferiti dalle testuggini dragone, ma essendo onnivore, qualche volta si alimentano anche di grandi campi subacquei di alghe marine. Certamente non disdegnano di integrare la loro dieta coi passeggeri delle navi che affondano, anche se tale pratica non è dovuta né a malvagità né a crudeltà. Le testuggini dragone hanno gusci del diametro di 5 metri, con gli arti che si estendono pochi metri più in là, e misurano 7 metri dalla punta del naso all'estremità della loro possente coda.\\

\medskip\index{Mostri - Troll}\textbf{Troll}

\emph{Grande gigante, caotico malvagio}

\textbf{FORZA} +4

\textbf{DESTREZZA} +1

\textbf{COSTITUZIONE} +5

\textbf{INTELLIGENZA} -2

\textbf{SAGGEZZA} -1

\textbf{CARISMA} -2

\textbf{Iniziativa} +1 -- \textbf{Difesa} 18

\textbf{Punti Ferita} 84 (8d10 + 40)

\textbf{Movimento} 9 m

\textbf{Tiri Salvezza}: Tempra +11, Riflessi +4, Volontà +3

\textbf{Competenze} Consapevolezza +2

\textbf{Sensi} scurovisione 18 m

\textbf{Linguaggi} Gigante

\textbf{Sfida} 5 (1.800 PE)

\emph{\textbf{Olfatto Affinato.}} Il troll ha +1d6 alle prove di Saggezza (Consapevolezza) basate sull'olfatto.

\emph{\textbf{Rigenerazione.}} Il troll recupera 10 punti ferita all'inizio del suo turno. Se il troll subisce danno da acido o da fuoco, questo tratto non funziona all'inizio del prossimo turno del troll. Il troll muore solo se inizia il suo turno a -5 punti ferita e non può rigenerarsi.

\textbf{Azioni}

\emph{\textbf{Multiattacco.}} Il troll può effettuare tre attacchi: uno con il morso e due con gli artigli.

\emph{\textbf{Artiglio.} Attacco con arma da mischia}: +7 a colpire, portata 1 m, un bersaglio.

\emph{Colpisce:} 11 (2d6 + 4) danni taglienti.

\emph{\textbf{Morso.} Attacco con arma da mischia}: +7 a colpire, portata 1 m, un bersaglio.

\emph{Colpisce:} 7 (1d6 + 4) danni perforanti.

\textbf{Ecologia}\\
Ambiente: Montagne Fredde\\
Organizzazione: Solitario o banda (2-4)\\
Tesoro: Standard\\
\textbf{Descrizione}\\
I troll possiedono artigli affilati ed incredibili capacità rigenerative che permettono loro di guarire quasi tutte le ferite. Sono gobbi, brutti ma fortissimi: combinata con i loro artigli, la loro forza gli permette di lacerare la carne a mani nude. I troll sono alti circa 4 metri, ma la loro postura li fa apparire più bassi. Un troll adulto pesa circa 500 kg.\\

l'appetito di un troll e le sue capacità rigenerative lo rendono un combattente indomito, che carica a testa bassa la creatura vivente più vicina ed attacca con tutta la sua furia. Solo il fuoco fa esitare un troll, ma perfino quello che per lui è un pericolo mortale non ferma la sua avanzata. Chi affronta i troll sa di dover localizzare e bruciare qualsiasi sua parte dopo un combattimento, perché perfino dal brandello più piccolo del suo corpo, con il tempo può rinascere un troll completo. Fortunatamente, solo le parti più grandi di un troll, come gli arti, ricrescono in questo modo.\\

Nonostante la loro ferocia, i troll sono straordinariamente teneri e gentili verso i loro piccoli. I troll femmina lavorano in gruppo, passando molto tempo ad insegnare ai cuccioli come cacciare e difendersi prima di mandarli a cercare un proprio territorio. Un troll maschio vive un'esistenza solitaria, incontrando brevemente le femmine solo per accoppiarsi. Tutti i troll trascorrono il loro tempo a cercare cibo, dato che devono consumarne enormi quantità ogni giorno o muoiono di fame. Per questo, la maggior parte dei troll si crea un proprio territorio di caccia che viene spesso difeso combattendo con i rivali. Simili scontri sono di solito non letali, ma i troll conoscono bene le proprie debolezze, sfruttandole per uccidere l'avversario nei periodi di magra.\\


\medskip\index{Mostri - Uomo Acquatico}\textbf{Uomo Acquatico}

\emph{Media umanoide (uomo acquatico), neutrale}

\textbf{FORZA} +0

\textbf{DESTREZZA} +1

\textbf{COSTITUZIONE} +1

\textbf{INTELLIGENZA} +0

\textbf{SAGGEZZA} +0

\textbf{CARISMA} +1

\textbf{Iniziativa} +1 -- \textbf{Difesa} 12

\textbf{Punti Ferita} 11 (2d8 + 2)

\textbf{Movimento} 3 m, nuoto 12 m

\textbf{Tiri Salvezza}:  Tempra +3, Riflessi +1, Volontà -1; +2 contro Ammaliamento

\textbf{Competenze} Consapevolezza +2

\textbf{Linguaggi} Aquan, Comune

\textbf{Sfida} 1/8 (25 PE)

\emph{\textbf{Anfibio.}} L'uomo acquatico può respirare aria e acqua.

\textbf{Azioni}

\emph{\textbf{Lancia.} Attacco con arma da mischia o a Distanza}: +2 a colpire, portata 1 m o gittata 6m, un bersaglio.

\emph{Colpisce:} 3 (1d6) danni perforanti, o 4 (1d8) danni perforanti se
usata con due mani per effettuare un attacco da mischia.

\textbf{Ecologia}\\
Ambiente: Oceani temperati\\
Organizzazione: Solitario, pattuglia (2-6), banda (6-10 più un tenete di 3° livello, compagnia (11-60 più 3 tenenti di 3° livello, 2 comandanti di 5° livello, 1 commodoro di 7° livello e 3-12 Calamari\\
Tesoro: Equipaggiamento da PNG (Tridente, Balestra Leggera con 10 Quadrelli, altro tesoro)\\
\textbf{Descrizione}\\
Fisicamente, gli Uomini Pesce somigliano ai loro antenati, con fronti espressive, pelle pallida, capelli scuri e occhi porpora. Hanno tre sottili branchie sul collo, ma possono passare per Umani per brevi periodi, se lo desiderano.\\


\medskip\index{Mostri - Uomo Albero (Treant)}\textbf{Uomo Albero (Treant)}

\emph{Enorme pianta, caotico buono}

\textbf{FORZA} +6

\textbf{DESTREZZA} -1

\textbf{COSTITUZIONE} +5

\textbf{INTELLIGENZA} +1

\textbf{SAGGEZZA} +3

\textbf{CARISMA} +1

\textbf{Iniziativa} +1 -- \textbf{Difesa} 21

\textbf{Punti Ferita} 138 (12d12 + 60)

\textbf{Movimento} 9 m

\textbf{Tiri Salvezza}: Tempra +13, Riflessi +3, Volontà +9

\textbf{Resistenze al Danno} da botta, perforante

\textbf{Vulnerabilità al Danno} fuoco

\textbf{Linguaggi} Comune, Druidico, Elfico, Silvano

\textbf{Sfida} 9 (5.000 PE)

\emph{\textbf{Falso Aspetto.}} Mentre l'uomo albero rimane immobile, è indistinguibile da un normale albero.

\emph{\textbf{Mostro d'Assedio.}} L'uomo albero infligge danni doppi agli oggetti e le strutture.

\textbf{Azioni}

\emph{\textbf{Multiattacco.}} L'uomo albero effettua due attacchi di schianto.

\emph{\textbf{Schianto.} Attacco con arma da mischia}: +10 a colpire, portata 1 m, un bersaglio.

\emph{Colpisce:} 16 (3d6 + 6) danni da botta.

\emph{\textbf{Sasso.} Attacco con arma a Distanza}: +10 a colpire, gittata 18m, un bersaglio.

\emph{Colpisce:} 28 (4d10 + 6) danni da botta.

\emph{\textbf{Animare Alberi (1/Giorno).}} L'uomo albero anima magicamente uno o due alberi visibili entro 18 metri da lui. Questi albeti hanno le stesse statistiche dell'ent, eccetto che hanno punteggio di Intelligenza e Carisma -3, non possono parlare, e hanno solo l'opzione di attacco Schianto. Un albero animato agisce come alleato dell'uomo albero. L'albero resta per 1 giorno o finché muore; finché l'uomo albero muore o si trova più di 36 metri lontano dall'albero, o finché l'uomo albero non effettua un'azione bonus per ritrasformarlo in un albero inanimato. Poi l'albero prenderà radici, se possibile.

\textbf{Ecologia}\\
Ambiente: Qualsiasi foresta\\
Organizzazione: Solitario o macchia (2-7)\\
Tesoro: Standard\\
\textbf{Descrizione}\\
I treant sono guardiani delle foreste ed ambasciatori degli alberi. Antichi quanto le foreste stesse, si vedono come genitori e pastori piuttosto che giardinieri: sono lenti e metodici, ma terrificanti quando costretti a combattere per difendere il loro gregge. Anche se raramente cercano la compagnia delle razze dalla vita breve ed hanno un'innata sfiducia verso i cambiamenti, mostrano tolleranza verso chi desidera imparare dai loro lunghi, lenti monologhi, specialmente coloro nei cui occhi leggono il desiderio di proteggere le regioni selvagge. Contro coloro che minacciano le loro foreste, specialmente i boscaioli che raccolgono legna o coloro che vorrebbero disboscare una foresta per costruire una strada o un forte, la rabbia dei treant si scatena rapida e devastante. Sono in grado di demolire ciò che gli altri costruiscono: un tratto che li aiuta durante i loro eccessi di furia.\\

I treant sono principalmente creature solitarie, ed un singolo individuo è spesso responsabile di un'intera foresta, ma a volte si raccolgono in gruppi detti boschetti per scambiarsi le ultime notizie e riprodursi.\\

In tempi di grave pericolo, tutti i boschetti di una regione si uniscono per una riunione della durata di mesi detta concilio, ma simili eventi sono molto rari, e fra i concili passano anche millenni.\\

Un tipico treant è alto 9 metri, con un tronco del diametro di 60 centimetri, e pesa circa 2.250 kg. I treant somigliano agli alberi più comuni dei territori dove vivono.\\



\medskip\index{Mostri - Uomo Magma (Magmin)}\textbf{Uomo Magma (Magmin)}

\emph{Piccola elementale, caotico neutrale}

\textbf{FORZA} -2

\textbf{DESTREZZA} +2

\textbf{COSTITUZIONE} +1

\textbf{INTELLIGENZA} -1

\textbf{SAGGEZZA} +0

\textbf{CARISMA} +0

\textbf{Iniziativa} +2 -- \textbf{Difesa} 15

\textbf{Punti Ferita} 9 (2d6 + 2)

\textbf{Movimento} 9 m

\textbf{Tiri Salvezza}: Tempra +6, Riflessi +4, Volontà +3

\textbf{Resistenze al Danno} da botta, perforante e tagliente di attacchi non magici

\textbf{Immunità ai Danni} fuoco

\textbf{Sensi} scurovisione 18 m

\textbf{Linguaggi} Ignan

\textbf{Sfida} 1/2 (100 PE)

\emph{\textbf{Illuminazione Incendiaria.}} Come azione bonus, l'uomo magma può accendere o spegnere le sue fiamme. Mentre la fiamma è accesa, l'uomo magma irradia luce intensa in un raggio di 3 metri e luce fioca per ulteriori 3 metri.

\emph{\textbf{Scoppio Mortale.}} Quando l'uomo magma muore, esplode in uno scoppio di fuoco e magma. Ogni creatura entro 3 metri da esso deve effettuare un Tiro Salvezza di Riflessi CD 11, subendo 7 (2d6) danni da fuoco se fallisce il Tiro Salvezza, o la metà di questi danni se lo riesce. Gli oggetti infiammabili che non siano indossati o trasportati e che si trovino nell'area, prendono fuoco.

\textbf{Azioni}

\emph{\textbf{Tocco.} Attacco con arma da mischia}: +4 a colpire, portata 1 m, un bersaglio.

\emph{Colpisce:} 7 (2d6) danni da fuoco. Se il bersaglio è una creatura o un oggetto infiammabile, questi prende fuoco. Fino a che una creatura effettua un'azione per estinguere la fiamma, la creatura subisce 3 (1d6) danni da fuoco al termine di ciascun suo turno.

\textbf{Ecologia}\\
Ambiente: Qualsiasi terreno (Piano del Fuoco)\\
Organizzazione: Solitario o banda (2-8)\\
Tesoro: Standard\\
\textbf{Descrizione}\\
Benché i magmin popolino il Piano del Fuoco, a volte scivolano nelle crepe elementali nel Piano Materiale. Queste crepe di solito si formano in luoghi di forte calore, come vulcani o fiumi sotterranei di magma, o in luoghi di forte e imprevedibile magia. Quest'ultimo scenario termina di solito in eventi più complessi, dato che i magmin tendono ad appiccare involontariamente fuoco agli oggetti infiammabili vicini.\\

Anche se non sono coraggiosi, questi piccoli esterni sono tuttavia temibili nemici delle creature senza resistenza al loro intenso calore. Il loro tocco incenerisce gli abiti, e le creature che colpiscono i loro corpi con l'acciaio corrono il rischio di ridurre in scorie le loro armi. La miglior difesa dei magmin in patria sul Piano del Fuoco è il loro numero. Gli insediamenti, costellati di laghi di magma e spruzzanti geyser di roccia fusa, brulicano di incredibili quantità di queste creature.\\

I magmin sono paranoici e diffidenti. Sempre spaventati dagli abitanti più grandi del Piano del Fuoco, i magmin sommergono gli intrusi con migliaia di domande, chiedendo dove vanno, da dove vengono, e che cosa fanno vicino ai loro preziosi laghi di magma. Se le risposte dei viaggiatori non sono soddisfacenti, i magmin tentano di sbarazzarsi il più rapidamente possibile delle creature. Chi si rifiuta di andarsene rischia di essere gettato in un lago di roccia liquida.\\

I magmin sono molto orgogliosi di come curano i loro laghi di magma. Ogni lago ha un diverso scopo: per farsi il bagno, per cucinare i pasti o per rilassarsi. I magmin inseriscono minerali e sali in questi laghi per adeguarli al loro scopo. I laghi per cucinare (a volte chiamati dagli stra-nieri "laghi assassini") sono più caldi degli altri, e quelli per lo svago sono di solito più scuri di quelli da bagno.\\

Alla maturità, i magmin sono alti 1,2 metri, la loro densa composizione li fa pesera 150 kg.\\


\medskip\index{Mostri - Unicorno}\textbf{Unicorno}

\emph{Grande celestiale, legale buono}

\textbf{FORZA} +4

\textbf{DESTREZZA} +2

\textbf{COSTITUZIONE} +2

\textbf{INTELLIGENZA} +0

\textbf{SAGGEZZA} +3

\textbf{CARISMA} +3

\textbf{Iniziativa} +2 -- \textbf{Difesa} 15

\textbf{Punti Ferita} 67 (9d10 + 18)

\textbf{Movimento} 15 m

\textbf{Tiri Salvezza}: Tempra +7, Riflessi +7, Volontà +6; +2 resistenza contro il Vuoto, Energia Negativa

\textbf{Immunità al Danno} veleno

\textbf{Immunità alle Condizioni} affascinato, avvelenato, paralizzato

\textbf{Sensi} scurovisione 18 m

\textbf{Linguaggi} Celestiale, Elfico, Silvano, telepatia 18 m

\textbf{Sfida} 5 (1.800 PE)

\emph{\textbf{Armi Magiche.}} Gli attacchi con armi dell'unicorno sono magici.

\emph{\textbf{Carica.}} Se l'unicorno si muove di almeno 6 metri in linea retta verso il bersaglio e lo colpisce con un attacco di corno durante lo stesso turno, il bersaglio subisce 9 (2d8) danni perforanti aggiuntivi. Se il bersaglio è una creatura, deve riuscire un Tiro Salvezza su Tempra CD 15 o cadere prono.

\emph{\textbf{Incantesimi Innati.}} La caratteristica da incantatore innato dell'unicorno è il Carisma (CD 14 per i Tiri Salvezza degli incantesimi). L'unicorno può lanciare in maniera innata i seguenti incantesimi, senza bisogno di componenti:

A volontà: \emph{arte del druido, individuazione del bene e male,} \emph{passare senza tracce}

1/giorno ciascuno: \emph{calmare emozioni, dissolvi il bene e il male,} \emph{intralciare}

\emph{\textbf{Resistenza alla Magia.}} L'unicorno ha +1d6 ai Tiri Salvezza contro incantesimi e altri effetti magici. 

\textbf{Azioni}

\emph{\textbf{Multiattacco.}} L'unicorno effettua due attacchi: uno con gli zoccoli e uno con il corno.

\emph{\textbf{Corno.} Attacco con arma da mischia}: +7 a colpire, portata 1 m, un bersaglio.

\emph{Colpisce:} 8 (1d8 + 4) danni perforanti.

\emph{\textbf{Zoccoli.} Attacco con arma da mischia}: +7 a colpire, portata 1 m, un bersaglio.

\emph{Colpisce:} 11 (2d6 + 4) danni da botta.

\emph{\textbf{Telestraporto (1/Giorno).}} L'unicorno può teletrasportare  magicamente sé stesso e fino a tre altre creature consenzienti  visibili entro 1 metro da esso, insieme a tutto  l'equipaggiamento che stanno indossando o trasportando, in un  luogo familiare all'unicorno, che si trova ad un massimo di 1,5 chilometri di distanza.

\emph{\textbf{Tocco Guaritore (3/Giorno).}} L'unicorno entra a contatto tramite il corno con un'altra creatura. Il bersaglio recupera magicamente 11 (2d8 + 2) punti ferita. Inoltre, il contatto rimuove tutte le malattie e neutralizza tutti i veleni che affliggono il bersaglio.

\textbf{Azioni Aggiuntive}

L'unicorno può effettuare 3 Azioni aggiuntive, scelte tra le opzioni seguenti. Può usare solo un'opzione leggendaria alla volta e solo al termine del turno di un'altra creatura. L'unicorno recupera le azioni aggiuntive spese all'inizio del proprio turno.

\textbf{Autoguarigione (Costa 3 Azioni).} L'unicorno recupera magicamente 11 (2d8 + 2) punti ferita.

\textbf{Scudo Scintillante (Costa 2 Azioni).} L'unicorno crea un campo magico scintillante che circonda lui o un'altra creatura visibile a lui entro 18 metri. Il bersaglio ottiene un bonus di +2 alla Difesa fino al termine del prossimo turno dell'unicorno.

\textbf{Zoccoli.} L'unicorno effettua un attacco con gli zoccoli.

\textbf{Ecologia}\\
Ambiente: Foreste Temperate\\
Organizzazione: Solitario, coppia o benedizione (3-6)\\
Tesoro: Nessuno\\
\textbf{Descrizione}\\
Gli unicorni sono fiere, intelligenti creature silvane che preferiscono rimanere isolate, apparendo solo per difendere le loro dimore dal male. Evitano tutte le creature tranne i folletti buoni, le donne umanoidi buone e gli animali nativi della loro foresta, ma potrebbero unirsi ad altre creature buone contro nemici comuni. Un tipico unicorno è lungo 2,4 metri, alto 1 metro al garrese e pesa 600 kg.\\

Le coppie di unicorni rimangono insieme per tutta la vita e dimorano in radure particolari o all'interno delle foreste che difendono. Permettono alle creature buone e neutrali di attraversarle, cacciare o di abitarvi, ma le creature malvagie o quelle che vorrebbero turbarne l'ecosistema, ad esempio cacciando per divertimento o abbattendone gli alberi per venderne il legname, vengono in fretta allontanati o uccisi. In alcune rare occasioni, gli unicorni il cui partner è stato ucciso prendono giovani donne di rara virtù come surrogati, permettendo loro di cavalcarli divenendo loro guardiani per tutta la vita. Se la donna si lega a qualcun altro, come un figlio o un amante, il legame con l'unicorno si scioglie amorevolmente, generando la leggenda che gli unicorni diventano amici solo delle vergini.\\

Il corno di un unicorno è la fonte dei suoi poteri, e per utilizzare le proprie capacità magiche su altre creature questi deve toccarle con esso. Le creature malvagie danno grande valore ai corni di unicorno come reagenti per pozioni di guarigione e per riti oscuri: un corno di unicorno in polvere vale 1.600 mo quando utilizzato per creare un oggetto magico di guarigione.\\


\subsection{Vampiri}

\medskip\index{Mostri - Vampiro}\textbf{Vampiro}

\emph{Media non morto (mutaforma), legale malvagio}

\textbf{FORZA} +4

\textbf{DESTREZZA} +4

\textbf{COSTITUZIONE} +4

\textbf{INTELLIGENZA} +3

\textbf{SAGGEZZA} +2

\textbf{CARISMA} +4

\textbf{Iniziativa} +4 -- \textbf{Difesa} 23

\textbf{Punti Ferita} 144 (17d8 + 68)

\textbf{Movimento} 9 m

\textbf{Tiri Salvezza} : Tempra +13, Riflessi +11, Volontà +12

\textbf{Competenze} Muoversi Silenziosamente / Nascondersi +9, Consapevolezza +17

\textbf{Immunità al Danno} da Vuoto; da botta, perforante e tagliente di attacchi non magici

\textbf{Sensi} scurovisione 36 m

\textbf{Linguaggi} le lingue che conosceva in vita

\textbf{Sfida} 13 (10.000 PE)

\emph{\textbf{Mutaforma.}} Se il vampiro non è sotto la luce del sole o immerso in acqua corrente, può usare la sua azione per trasformarsi in un Minuscolo pipistrello, una nube di foschia Media, o per tornare alla sua vera forma.

Mentre è in forma di pipistrello, il vampiro non può parlare, la sua velocità di passeggio è 1 metro e ha velocità di volo 9 metri. Le sue statistiche, a parte la taglia e la velocità, sono immutate. Qualsiasi equipaggiamento stia indossando si trasforma con esso, ma quello che stava trasportando viene fatto cadere a terra. Alla morte ritorna alla sua vera forma.

Mentre è in forma di foschia, il vampiro non può effettuare azioni, parlare o manipolare oggetti. È privo di peso, ha velocità di volo 6 metri, può fluttuare, e può entrare nello spazio di una creatura ostile e fermarsi lì. Inoltre, se in uno spazio vi passa dell'aria, la foschia può fare altrettanto senza stringersi, ma non può attraversare l'acqua. Ha +1d6 ai Tiri Salvezza su Tempra e Riflessi ed è immune a tutti i danni non magici, eccetto i danni subiti dalla luce del
sole.

\emph{\textbf{Debolezze del Vampiro.}} Il vampiro ha i seguenti difetti:

\emph{Danneggiato dall'Acqua Corrente.} Il vampiro subisce 20 danni da acido se termina il suo turno all'interno dell'acqua corrente.

\emph{Ipersensibilità alla Luce.} Il vampiro subisce 20 danni da Luce quando inizia il suo turno alla luce del sole. Mentre è alla luce del sole, ha -1d6 ai tiri di attacco e le prove di competenza. 

\emph{Paletto nel Cuore.} Se un'arma perforante fatta di legno viene conficcata nel cuore del vampiro mentre il vampiro è inabile nel suo luogo di riposo, il vampiro resta paralizzato finché il paletto non viene rimosso.

\emph{Proibizione.} Il vampiro non può entrare in un'abitazione senza invito da parte dei suoi occupanti.

\emph{\textbf{Fuga nella Foschia.}} Quando scende a 0 punti ferita al di fuori del suo luogo di riposo, il vampiro si trasforma in una nube di foschia (come per il tratto Mutaforma) invece di cadere privo di sensi, purché non sia esposto alla luce del sole o all'acqua corrente. Se non può trasformarsi, viene distrutto.

Mentre si trova a 0 punti ferita in questa forma, non può tornare alla sua forma di vampiro, e deve raggiungere il suo luogo di riposo entro 2 ore o venire distrutto. Una volta raggiunto il suo luogo di riposo, ritorna alla sua forma di vampiro. Resterà quindi paralizzato finché non avrà recuperato almeno 1 punto ferita. Dopo aver trascorso almeno 1 ora nel suo luogo di riposo a 0 punti ferita, il vampiro recupererà 1 punto ferita.

\emph{\textbf{Natura Non Morta.}} Il vampiro non ha bisogno di aria.

\emph{\textbf{Resistenza Leggendaria (3/Giorno).}} Se il vampiro fallisce un Tiro Salvezza, può scegliere invece di riuscire.

\emph{\textbf{Rigenerazione.}} Il vampiro recupera 20 punti ferita all'inizio del suo turno se possiede almeno 1 punto ferita e non è esposto alla luce del sole o l'acqua corrente. Se il vampiro subisce danno da Luce o danno dall'acqua sacra, questo tratto non funziona all'inizio del prossimo turno del vampiro.

\emph{\textbf{Scalare come Ragno.}} Il vampiro può scalare superfici difficili, compreso lo stare a testa in giù sul soffitto, senza bisogno di effettuare una prova di abilità.

\textbf{Azioni}

\emph{\textbf{Multiattacco.}} Il vampiro può effettuare due attacchi, ma solo uno di essi può essere un attacco con morso.

\emph{\textbf{Colpo Disarmato (Solo in Forma di Vampiro).} Attacco con arma da mischia}: +9 a colpire, portata 1 m, una creatura.

\emph{Colpisce:} 8 (1d8 + 4) danni da botta. Invece di infliggere danno, il vampiro può afferrare il bersaglio (CD per fuggire 18).

\emph{\textbf{Morso (Solo in Forma di Pipistrello o Vampiro).} Attacco con arma da mischia}: +9 a colpire, portata 1 m, una creatura consenziente o una creatura afferrata dal vampiro, inabile o intralciata.

\emph{Colpisce:} 7 (1d6 + 4) danni perforanti più 10 (3d6) danni da Vuoto. I punti ferita massimi del bersaglio sono ridotti di un ammontare pari al danno da Vuoto subito, e il vampiro recupera un numero di punti ferita pari a quell'ammontare. Questa riduzione permane fino alla nuova alba. Il bersaglio muore se questo effetto riduce i suoi punti ferita massimi a 0. Un umanoide ucciso in questo modo e poi sepolto nel terreno si rianima la notte seguente come progenie vampirica sotto il controllo del vampiro.

\emph{\textbf{Affascinare.}} Il vampiro prende a bersaglio un umanoide entro 9 metri che può vedere. Se il bersaglio può vedere il vampiro, deve effettuare un Tiro Salvezza di Volontà CD 17 contro questa magia o esserne affascinato. Il bersaglio affascinato considera il vampiro un amico fidato da ascoltare e proteggere. Sebbene il bersaglio non sia sotto il controllo del vampiro, prende le richieste e le azioni del vampiro nel modo più favorevole possibile, ed è un bersaglio consenziente dell'attacco con morso del vampiro.

Ogni volta che il vampiro o i compagni del vampiro fanno qualcosa di nocivo al bersaglio, questi può ripetere il Tiro Salvezza, terminando l'effetto su di sé in caso di successo. Altrimenti, l'effetto persiste 24 ore o finché il vampiro non viene distrutto, si trova su di un piano di esistenza diverso dal bersaglio, o effettua un'azione bonus per terminare l'effetto.

\emph{\textbf{Figli della Notte (1/Giorno).}} Il vampiro richiama magicamente 2d4 sciami di pipistrelli o ratti, purché il sole non sia sorto. Mentre è all'esterno, il vampiro può richiamare invece 3d6 lupi. Le creature richiamate arrivano in 1d4 round, agendo da alleati del vampiro e obbedendo ai suoi comandi. Le bestie restano per 1 ora, finché il vampiro non muore, o finché non le congeda con un'azione bonus.

\textbf{Azioni Aggiuntive}

Il vampiro può effettuare 3 Azioni aggiuntive, scelte tra le opzioni seguenti. Può usare solo un'opzione leggendaria alla volta e solo al termine del turno di un'altra creatura. Il vampiro recupera all'inizio del proprio turno le Azioni aggiuntive che ha speso.

\textbf{Colpo Disarmato.} Il vampiro effettua un colpo disarmato. \textbf{Morso (Costa 2 Azioni).} Il vampiro effettua un attacco con
morso.

\textbf{Muoversi.} Il vampiro si muove del suo movimento senza provocare attacchi di opportunità.

\textbf{Ecologia}
Ambiente: Qualsiasi\\
Organizzazione: Solitario o famiglia (vampiro più 2-8 Progenie)\\
Tesoro: Equipaggiamento da PNG (Anello della Protezione +2, Fascia della Seduzione +4, Mantello della Resistenza +3)\\
\textbf{Descrizione}\\
I vampiri sono creature umanoidi non morte che si nutrono del sangue dei viventi. Hanno un aspetto molto simile a quando erano in vita, diventando spesso più attraenti, sebbene alcuni appaiano invece duri e ferini.\\



\medskip\index{Mostri - Progenie Vampirica}\textbf{Progenie Vampirica}

\emph{Media non morto, neutrale malvagio}

\textbf{Iniziativa} +0 -- \textbf{Difesa} 18

\textbf{Punti Ferita} 82 (11d8 + 33)

\textbf{Movimento} 9 m

\textbf{Tiri Salvezza} Tempra +3, Riflessi +2, Volontà +5

\textbf{FORZA} +3

\textbf{DESTREZZA} +3

\textbf{COSTITUZIONE} +3

\textbf{INTELLIGENZA} +0

\textbf{SAGGEZZA} +0

\textbf{CARISMA} +1

\textbf{Competenze} Muoversi Silenziosamente / Nascondersi +6, Consapevolezza +3

\textbf{Resistenze ai Danni} da Vuoto; da botta, perforante e tagliente di attacchi non magici

\textbf{Sensi} scurovisione 18 m

\textbf{Linguaggi} le lingue che conosceva in vita 

\textbf{Sfida} 5 (1.800 PE)

\emph{\textbf{Debolezze della Progenie Vampirica.}} La Progenie Vampirica ha i seguenti difetti:

\emph{Danneggiato dall'Acqua Corrente.} La Progenie Vampirica subisce 20 danni da acido se termina il suo turno all'interno dell'acqua corrente.

\emph{Ipersensibilità alla Luce.} La Progenie Vampirica subisce 20 danni da Luce quando inizia il suo turno alla luce del sole. Mentre è alla luce del sole, ha -1d6 ai tiri di attacco e le prove di competenza.

\emph{Paletto nel Cuore.} La Progenie Vampirica è distrutto se un'arma perforante di legno gli viene conficcata nel cuore mentre è inabile all'interno del suo luogo di riposo.

\emph{Proibizione.} La Progenie Vampirica non può entrare in un'abitazione senza invito da parte dei suoi occupanti.

\emph{\textbf{Natura Non Morta.}} La Progenie Vampirica non ha bisogno di aria.

\emph{\textbf{Rigenerazione.}} La Progenie Vampirica recupera 10 punti ferita all'inizio del suo turno se possiede almeno 1 punto ferita e non è esposto alla luce del sole o l'acqua corrente. Se la Progenie Vampirica subisce danno da Luce o danno dall'acqua sacra, questo tratto non funziona all'inizio del prossimo turno del vampiro.

\emph{\textbf{Scalare come Ragno.}} La Progenie Vampirica può scalare superfici difficili, compreso lo stare a testa in giù sul soffitto, senza bisogno di effettuare una prova di abilità.

\textbf{Azioni}

\emph{\textbf{Multiattacco.}} La progenie vampirica può effettuare due attacchi, ma solo uno di essi può essere un attacco con morso.

\emph{\textbf{Artigli.} Attacco con arma da mischia}: +6 a colpire, portata 1 m, una creatura.

\emph{Colpisce:} 8 (2d4 + 3) danni taglienti. Invece di infliggere danno, il vampiro può afferrare il bersaglio (CD per fuggire 13). 

\emph{\textbf{Morso.} Attacco con arma da mischia}: +6 a colpire, portata 1 m, una creatura afferrata dal vampiro, inabile o intralciata.

\emph{Colpisce:} 6 (1d6 + 3) danni perforanti più 7 (2d6) danni da Vuoto. I punti ferita massimi del bersaglio sono ridotti di un ammontare pari al danno da Vuoto subito, e il vampiro recupera un numero di punti ferita pari a quell'ammontare. Questa riduzione permane fino alla nuova alba. Il bersaglio muore se questo effetto riduce i suoi punti ferita massimi a 0.

\textbf{Ecologia}\\
Ambiente: Qualsiasi\\
Organizzazione: Solitario, coppia, gruppo (3-6) o turba (7-12)\\
Tesoro: Standard\\
\textbf{Descrizione}\\
Un Vampiro può decidere di creare da una vittima una progenie vampirica anziché farne un vampiro completo solo quando usa la sua capacità creare progenie su una creatura umanoide. Questa decisione deve essere presa come azione gratuita appena un vampiro uccide una creatura appropriata usando risucchio di sangue o risucchio di energia. \\


\medskip\index{Mostri - Verme Purpureo}\textbf{Verme Purpureo}

\emph{Mastodontica mostruosità, disallineato}

\textbf{FORZA} +9

\textbf{DESTREZZA} -2

\textbf{COSTITUZIONE} +6

\textbf{INTELLIGENZA} -5

\textbf{SAGGEZZA} -1

\textbf{CARISMA} -3

\textbf{Iniziativa} -2 -- \textbf{Difesa} 26

\textbf{Punti Ferita} 247 (15d20 + 90)

\textbf{Movimento} 15 m, scavo 9 m

\textbf{Tiri Salvezza}: Tempra +17, Riflessi +8, Volontà +4

\textbf{Sensi} vista cieca 9 m, senso tellurico 18 m

\textbf{Linguaggi} -

\textbf{Sfida} 15 (13.000 PE)

\emph{\textbf{Scavatore di Tunnel.}} Il verme può scavare attraverso la roccia solida a metà della velocità di scavare e lascia un tunnel di 3 metri di diametro dietro di sè.

\textbf{Azioni}

\emph{\textbf{Multiattacco.}} Il verme effettua due attacchi: uno con il morso e uno con il pungiglione.

\emph{\textbf{Morso.} Attacco con arma da mischia}: +9 a colpire,
portata 3 m, un bersaglio.

\emph{Colpisce:} 22 (3d8 + 9) danni perforanti. Se il bersaglio è una creatura di taglia Grande, deve riuscire un Tiro Salvezza di Riflessi CD 19 o venire inghiottita dal verme. Mentre è inghiottita, la creatura è accecata e intralciata, ha copertura totale contro gli attacchi e altri effetti provenienti dall'esterno del verme, e subisce 21 (6d6) danni da acido all'inizio di ciascun turno del verme.

Se il verme subisce 30 o più danni in un singolo turno da una creatura al suo interno, il verme deve riuscire un Tiro Salvezza di Tempra CD 21 al termine del suo turno o vomitare tutte le creature inghiottite, che cadono prone in uno spazio entro 3 metri dal verme. Se il verme muore, una creatura inghiottita non risulta più intralciata da esso e può fuggire dal cadavere usando 6 metri di movimento, uscendo prona.

\emph{\textbf{Pungiglione.} Attacco con arma da mischia}: +9 a colpire, portata 3 m, una creatura.

\emph{Colpisce:} 19 (3d6 + 9) danni perforanti, e il bersaglio deve effettuare un Tiro Salvezza di Tempra CD 19, subendo 42 (12d6) danni da veleno se fallisce il Tiro Salvezza, o la metà di questi danni se lo riesce.

\textbf{Ecologia}\\
Ambiente: Qualsiasi sotterraneo\\
Organizzazione: Solitario\\
Tesoro: Accidentale\\
\textbf{Descrizione}\\
I vermi purpurei sono giganteschi necrofagi che abitano nelle regioni più profonde del mondo, mangiando qualsiasi materiale organico incontrino. Sono noti per inghiottire le loro prede intere. Non è insolito sentire di un gruppo di avventurieri scomparso all'interno delle fameliche fauci di un verme purpureo, gridando di terrore mentre i suoi membri sparivano uno alla volta.\\

Mentre vanno in cerca di creature viventi per divorarle, i vermi purpurei ingoiano anche un'enorme quantità di terra e minerali scavando nel sottosuolo. Le interiora di un verme purpureo possono contenere un considerevole numero di gemme e altri oggetti in grado di resistere all'acido corrosivo all'interno del suo esofago. In zone ricche di minerali preziosi, come quelle vicine alle miniere naniche, i tunnel naturali creati dagli scavi dei vermi purpurei sono spesso pieni di un notevole numero di pepite d'oro grezzo.\\

Un verme purpureo generalmente reclama una grande caverna sotterranea come sua tana, e anche se vi torna per riposare e digerire il cibo, passa la maggior parte del suo tempo in cerca di preda, scavando attraverso l'oscurità senza fine o scivolando lungo tunnel preesistenti alla costante ricerca di cibo per saziare la sua immensa fame. Sebbene quasi privi di intelletto, i vermi purpurei raramente sono stupidi. Sono diffusi come guardiani fra chi riesce a controllarli magicamente o hanno nel loro covo una stanza abbastanza grande da ospitarli.\\


\medskip\index{Mostri - Viverna}\textbf{Viverna}

\emph{Grande drago, disallineato}

\textbf{FORZA} +4

\textbf{DESTREZZA} +0

\textbf{COSTITUZIONE} +3

\textbf{INTELLIGENZA} -3

\textbf{SAGGEZZA} +1

\textbf{CARISMA} -2

\textbf{Iniziativa} +0 -- \textbf{Difesa} 16

\textbf{Punti Ferita} 110 (13d10 + 39)

\textbf{Movimento} 6 m, volo 24 m

\textbf{Tiri Salvezza}: Tempra +9, Riflessi +6, Volontà +8

\textbf{Competenze} Consapevolezza +4

\textbf{Sensi} scurovisione 18 m

\textbf{Linguaggi} -

\textbf{Sfida} 6 (2.300 PE)

\textbf{Azioni}

\emph{\textbf{Multiattacco.}} La viverna può effettuare due attacchi: uno con il morso e uno con il pungiglione. Mentre vola, può usare i suoi artigli al posto di uno degli altri attacchi.

\emph{\textbf{Artigli.} Attacco con arma da mischia}: +7 a colpire, portata 1 m, un bersaglio.

\emph{Colpisce:} 13 (2d8 + 4) danni taglienti.

\emph{\textbf{Morso.} Attacco con arma da mischia}: +7 a colpire, portata 3 m, una creatura.

\emph{Colpisce:} 11 (2d6 + 4) danni perforanti.

\emph{\textbf{Pungiglione.} Attacco con arma da mischia}: +7 a colpire, portata 3 m, una creatura.

\emph{Colpisce:} 11 (2d6 + 4) danni perforanti. Il bersaglio deve effettuare un Tiro Salvezza di Tempra CD 15, e subire 24 (7d6) danni da veleno se lo fallisce, o la metà di questi danni se lo riesce.

\textbf{Ecologia}\\
Ambiente: Colline temperate o calde\\
Organizzazione: Solitario, coppia o stormo (3-6)\\
Tesoro: standard\\
\textbf{Descrizione}\\
Le viverne sono rettili brutali e violenti imparentati con i draghi. Sono sempre aggressive ed impazienti e preferiscono raggiungere i loro scopi utilizzando la forza. Per questa ragione, i draghi guardano alle viverne con superiorità, considerando questi loro lontani parenti come selvaggi primitivi privi di stile ed intelligenza.\\

Nella maggior parte dei casi, questa generalizzazione è azzeccata. Anche se non certo di intelletto animale e capace di parola, la maggior parte delle viverne non si cura della diplomazia, preferendo combattere prima e discutere poi, solo se si trovano davanti ad un avversario che non possono sconfiggere o da cui non possono fuggire.\\

Le viverne sono creature territoriali. Pur cacciando occasionalmente prede più grandi in gruppi più estesi, sono creature solitarie il cui territorio di caccia si estende dai 160 ai 320 km quadrati. È noto che le viverne combattono spesso fra loro fino alla morte per le contese su un territorio ricco di prede.\\


Seppur costantemente affamate ed inclini ad attaccare, una viverna può essere resa amichevole attraverso un'attenta combinazione di lusinghe, intimidazione, cibo e tesoro, per farne un potente alleato. Spesso servono Giganti e Umanoidi Mostruosi come guardiani come guardiani, ed alcune tribù di Boggard e Lucertoloidi le usano come cavalcature, anche se tali accordi spesso risultano parecchio costosi in termini di cibo ed oro, poiché sono poche le viverne che accettano di servire a lungo creature simili come cavalcature.\\

Una viverna è lunga circa 4,8 metri e la coda rappresenta da sola circa metà della lunghezza. Una viverna pesa in media 1.000 kg.\\


\medskip\index{Mostri - Wight}\textbf{Wight}

\emph{Media non morto, neutrale malvagio}

\textbf{FORZA} +2

\textbf{DESTREZZA} +2

\textbf{COSTITUZIONE} +3

\textbf{INTELLIGENZA} +0

\textbf{SAGGEZZA} +1

\textbf{CARISMA} +2

\textbf{Iniziativa} +2 -- \textbf{Difesa} 16 (armatura borchiata)

\textbf{Punti Ferita} 45 (6d8 + 18)

\textbf{Movimento} 9 m

\textbf{Tiri Salvezza}: Tempra +3, Riflessi +2, Volontà +5

\textbf{Competenze} Muoversi Silenziosamente / Nascondersi +4, Consapevolezza +3

\textbf{Resistenze al Danno} da Vuoto; da botta, perforante e tagliente di attacchi non magici o che non siano argentati

\textbf{Immunità al Danno} veleno

\textbf{Immunità alle Condizioni} avvelenato, affaticamento

\textbf{Sensi} scurovisione 18 m

\textbf{Linguaggi} le lingue che conosceva in vita

\textbf{Sfida} 3 (700 PE)

\emph{\textbf{Natura Non Morta.}} Il wight non ha bisogno di aria, cibo, bevande o sonno.

\emph{\textbf{Sensibilità alla Luce}}. Mentre è alla luce del sole, il wight ha -1d6 ai tiri di attacco, oltre che alle prove di Saggezza (Consapevolezza) basate sulla vista.

\textbf{Azioni}

\emph{\textbf{Multiattacco.}} Il wight può effettuare due attacchi con la spada lungha o due attacchi con l'arco lungo. Può usare Risucchiare Vita al posto di uno dei suoi attacchi con la spada lungha.

\emph{\textbf{Risucchiare Vita.} Attacco con arma da mischia}: +4 a colpire, portata 1 m, una creatura.

\emph{Colpisce:} 5 (1d6 + 2) danni da Vuoto. Il bersaglio deve riuscire un Tiro Salvezza di Tempra CD 13 o vedere i suoi punti ferita massimi ridotti di un ammontare pari al danno subito. Questa riduzione perdura fino al sorgere della nuova alba. Il bersaglio muore se l'effetto riduce i suoi punti ferita massimi a 0.

Un umanoide ucciso da questo attacco si rianima 24 ore più tardi come zombi sotto il controllo del wight, a meno che l'umanoide non venga prima riportato in vita o il corpo sia distrutto. Il wight non può controllare più di dodici zombi alla volta.

\emph{\textbf{Spada Lunga.} Attacco con arma da mischia}: +4 a colpire, portata 1 m, un bersaglio.

\emph{Colpisce:} 6 (1d8 + 2) danni taglienti o 7 (1d10 + 2) danni taglienti se usata con due mani.

\emph{\textbf{Arco Lungo.} Attacco con arma a Distanza}: +4 a colpire, gittata 45m, un bersaglio.

\emph{Colpisce:} 6 (1d8 + 2) danni perforanti.

\textbf{Ecologia}\\
Ambiente: qualsiasi\\
Organizzazione: Solitario, coppia, gruppo (3-6) o branco (7-12)\\
Tesoro: Standard\\
\textbf{Descrizione}\\
I wight sono umanoidi risorti come non morti a causa della necromanzia, di una morte violenta o di una personalità estremamente malevola. In alcuni casi, un wight sorge quando uno spirito non morto si lega permanentemente ad un cadavere, spesso quello di un guerriero. Sono appena riconoscibili da chi li conosceva in vita: le loro carni sono corrotte dalla malvagità e dalla non morte, gli occhi ardono d'odio ed i denti divengono quelli di una bestia. In un certo senso, un wight è l'anello di congiunzione tra ghoul e spettri: un cadavere deforme che risucchia energia vitale col tocco.\\

Essendo non morti, i wight non hanno bisogno di respirare, così a volte si possono trovare sott'acqua, sebbene non siano nuotatori particolarmente abili a meno che non siano originati da creature nuotatrici quali elfi acquatici e marinidi. Sott'acqua i wight preferiscono le caverne dal soffitto basso dove le loro scarse capacità di nuoto non sono una limitazione.\\


\medskip\index{Mostri - Wraith}\textbf{Wraith}

\emph{Media non morto, neutrale malvagio}

\textbf{FORZA} -2

\textbf{DESTREZZA} +3

\textbf{COSTITUZIONE} +3

\textbf{INTELLIGENZA} +1

\textbf{SAGGEZZA} +2

\textbf{CARISMA} +2

\textbf{Iniziativa} +3 -- \textbf{Difesa} 16

\textbf{Punti Ferita} 67 (9d8 + 27)

\textbf{Movimento} 0 m, volo 18 m (fluttua)

\textbf{Tiri Salvezza}: Tempra +6, Riflessi +4, Volontà +6

\textbf{Resistenze al Danno} acido, freddo, fulmine, fuoco, tuono; da botta, perforante e tagliente di attacchi non magici o che non siano argentati

\textbf{Immunità al Danno} da Vuoto, veleno

\textbf{Immunità alle Condizioni} affascinato, afferrato, avvelenato, intralciato, paralizzato, pietrificato, prono, affaticamento

\textbf{Sensi} scurovisione 18 m 

\textbf{Linguaggi} le lingue
che conosceva in vita 

\textbf{Sfida} 5 (1.800 PE)

\emph{\textbf{Movimento Incorporeo.}} Il wraith può attraversare creature e oggetti come fossero terreno difficile. Subisce 5 (1d10) danni da forza se termina il proprio turno all'interno di un oggetto.

\emph{\textbf{Natura Non Morta.}} Il wraith non ha bisogno di aria, cibo, bevande o sonno.

\emph{\textbf{Sensibilità alla Luce}}. Mentre è alla luce del sole, il wraith ha -1d6 ai tiri di attacco, oltre che alle prove di Saggezza (Consapevolezza) basate sulla vista.

\textbf{Azioni}

\emph{\textbf{Risucchiare Vita.} Attacco con arma da mischia}: +6 a colpire, portata 1 m, una creatura.

\emph{Colpisce:} 21 (4d8 + 3) danni da Vuoto. Il bersaglio deve riuscire un Tiro Salvezza di Tempra CD 14 o vedere i suoi punti ferita massimi ridotti di un ammontare pari al danno subito. Questa riduzione perdura fino al sorgere della nuova alba. Il bersaglio muore se l'effetto riduce i suoi punti ferita massimi a 0.

\emph{\textbf{Creare Spettro.}} Il wraith prende a bersaglio un umanoide entro 3 metri da esso e che sia morto da non più di 1 minuto e per cause violente. Lo spirito del bersaglio si anima come spettro nello spazio del suo cadavere e nello spazio più vicino non occupato. Lo spettro è sotto ilcontrollo del wraith. Il wraith non può tenere più di sette  spettri alla volta sotto il suo controllo.

\textbf{Ecologia}\\
Ambiente: Qualsiasi\\
Organizzazione: Solitario, coppia, gruppo (3-6) o branco (7-12)\\
Tesoro: Nessuno\\
\textbf{Descrizione}\\
I wraith sono creature nate dal male e dall'oscurità. Detestano la luce e le creature viventi, avendo perduto la maggior parte del legame con la loro vita precedente.\\


\medskip\index{Mostri - Xorn}\textbf{Xorn}

\emph{Media elementale, neutrale}

\textbf{FORZA} +3

\textbf{DESTREZZA} +0

\textbf{COSTITUZIONE} +6

\textbf{INTELLIGENZA} +0

\textbf{SAGGEZZA} +0

\textbf{CARISMA} +0

\textbf{Iniziativa} +0 -- \textbf{Difesa} 22

\textbf{Punti Ferita} 73 (7d8 + 42)

\textbf{Movimento} 6 m, scavo 6 m

\textbf{Tiri Salvezza}: Tempra +8, Riflessi +2, Volontà +5

\textbf{Competenze} Muoversi Silenziosamente / Nascondersi +3, Consapevolezza +6

\textbf{Resistenze al Danno} perforante e tagliente di attacchi non magici che non siano di adamantio

\textbf{Sensi} scurovisione 18 m, senso tellurico 18 m

\textbf{Linguaggi} Terran

\textbf{Sfida} 5 (1.800 PE)

\emph{\textbf{Mimetismo di Pietra.}} Lo xorn ha +1d6 alle prove di Destrezza (Nascondersi) effettuate per nascondersi su terreno roccioso.

\emph{\textbf{Scorrere sulla Terra.}} Lo xorn può scavare attraversa la terra e la pietra non magiche e non lavorate. Quando lo fa, lo xorn non disturba il materiale che sposta.

\emph{\textbf{Senso del Tesoro.}} Lo xorn può individuare precisamente, con l'olfatto, la posizione di metalli e pietre preziose, come monete e gemme, entro 18 metri da esso.

\textbf{Azioni}

\emph{\textbf{Multiattacco.}} Lo xorn effettua tre attacchi di artiglio e un attacco di morso.

\emph{\textbf{Artiglio.} Attacco con arma da mischia}: +6 a colpire, portata 1 m, un bersaglio.

\emph{Colpisce:} 6 (1d6 + 3) danni taglienti.

\emph{\textbf{Morso.} Attacco con arma da mischia}: +6 a colpire, portata 1 m, un bersaglio.

\emph{Colpisce:} 13 (3d6 + 3) danni perforanti.

\textbf{Ecologia}\\
Ambiente: Qualsiasi (Piano della Terra)\\
Organizzazione: Solitario, coppia o gruppo (3-6)\\
Tesoro: Standard (solo metalli preziosi, gemme e gioielli e gemme magiche)\\
\textbf{Descrizione}
Strane creature larghe quanto alte, gli xorn hanno poco interesse verso i nativi del Piano Materiale, non fosse per le gemme ed i metalli preziosi che potrebbero avere con sé. Nascosti sotto la superficie del terreno per un tempo che ad un umano potrebbe sembrare lunghissimo, uno xorn può attendere mesi, perfino anni, per la preda ideale, per poi assalire chi porta con sé il suo cibo preferito, come una gemma particolare o un determinato tipo di argento. Gli avventurieri che si addentrano nelle regioni abitate dagli xorn portano spesso con sé piccole pepite di minerali o gemme e cristalli di scarso valore da utilizzare come tributo. Anche se il suo valore è solitamente direttamente proporzionale al suo sapore e all'appetibilità che esso può avere, la maggior parte degli xorn è piuttosto ingorda, e preferisce la quantità alla qualità.\\

Il tesoro che uno xorn porta con sé o nasconde nella sua tana consiste in uno spuntino che ha conservato per il giorno successivo. Offrire un gioiello o un metallo preziosi particolarmente deliziosi (e costosi) ad uno xorn può cementare un'alleanza temporanea. Dato che gli xorn possono attraversare la roccia con facilità sono ottime guide nelle regioni sotterranee.\\

Gli xorn non sono molto religiosi, ma quelli fra loro che trovano la fede sono solitamente Druidi (anche se è raro, se non improbabile, che gli xorn abbiano Compagni Animali, dato che non possono seguirli nella roccia, e scelgono invece il dominio della Terra). Bardi e Devoti xorn non sono sconosciuti: i Bardi scelgono di solito Intrattenere (canto), e gli Devoti hanno invariabilmente la Stirpe Elementale (terra).\\


\medskip\index{Mostri - Zombi}\textbf{Zombi}

\emph{Media non morto, neutrale malvagio}

\textbf{FORZA} +1

\textbf{DESTREZZA} -2

\textbf{COSTITUZIONE} +3

\textbf{INTELLIGENZA} -4

\textbf{SAGGEZZA} -2

\textbf{CARISMA} -3

\textbf{Iniziativa} -2 -- \textbf{Difesa} 9

\textbf{Punti Ferita} 22 (3d8 + 9)

\textbf{Movimento} 6 m

\textbf{Tiri Salvezza}  Tempra +0, Riflessi +0, Volontà +3

\textbf{Immunità al Danno} veleno

\textbf{Immunità alle Condizioni} avvelenato

\textbf{Sensi} scurovisione 18 m

\textbf{Linguaggi} comprende tutte le lingue che parlava in vita ma non può parlare

\textbf{Sfida} 1/4 (50 PE)

\emph{\textbf{Natura Non Morta.}} Lo zombi non ha bisogno di aria, cibo, bevande o sonno.

\emph{\textbf{Tempra dei Non Morti.}} Se il danno riduce lo zombi a 0 punti ferita, lo zombi deve effettuare un Tiro Salvezza di Tempra CD 5 + il danno subito, a meno che il danno non sia da Luce o un colpo critico. Se riesce, lo zombi scende invece a 1 punto ferita.

\textbf{Azioni}

\emph{\textbf{Schianto.} Attacco con arma da mischia}: +3 a colpire, portata 1 m, un bersaglio.

\emph{Colpisce:} 4 (1d6 + 1) danni da botta.

\textbf{Ecologia}\\
Ambiente: Qualsiasi\\
Organizzazione: Qualsiasi\\
Tesoro: Nessuno\\
\textbf{Descrizione}\\
Gli zombi sono i cadaveri animati di creature morte, costretti a muoversi da magie necromantiche come Animare Morti. Anche se gli zombi incontrati di norma sono lenti e robusti, altri possiedono tratti differenti, che permettono loro di diffondere una malattia o di muoversi più rapidi.\\

Gli zombi sono automi senza mente e non possono fare altro che seguire gli ordini. Se lasciati a loro stessi, attendono immobili o si spostano alla ricerca di creature viventi da massacrare e divorare. Gli zombi attaccano fino alla distruzione, senza curarsi della loro sicurezza.\\

Sebbene siano in grado di seguire gli ordini, gli zombi vengono spesso lasciati liberi con l'ordine di uccidere tutte le creature viventi. Spesso vengono incontrati in branchi che infestano le terre frequentate dai viventi, in cerca di preda. La maggior parte degli zombi viene creata attraverso Animare Morti. Simili zombi sono sempre standard, a meno che il creatore lanci anche Velocità o Rimuovi Paralisi per creare Zombi Rapidi o Contagio per creare Zombi Infetti.\\


\medskip\index{Mostri - Zombi Ogre}\textbf{Zombi Ogre}

\emph{Grande non morto, neutrale malvagio}

\textbf{FORZA} +4

\textbf{DESTREZZA} -2

\textbf{COSTITUZIONE} +4

\textbf{INTELLIGENZA} -4

\textbf{SAGGEZZA} -2

\textbf{CARISMA} -3

\textbf{Iniziativa} -2 -- \textbf{Difesa} 9

\textbf{Punti Ferita} 85 (9d10 + 36)

\textbf{Movimento} 9 m

\textbf{Tiri Salvezza}: Tempra +6, Riflessi +0, Volontà +3

\textbf{Immunità al Danno} veleno

\textbf{Immunità alle Condizioni} avvelenato

\textbf{Sensi} scurovisione 18 m

\textbf{Linguaggi} comprende Comune e Gigante ma non può parlare

\textbf{Sfida} 2 (450 PE)

\emph{\textbf{Natura Non Morta.}} Lo zombi non ha bisogno di aria, cibo, bevande o sonno.

\emph{\textbf{Tempra dei Non Morti.}} Se il danno riduce lo zombi a 0 punti ferita, lo zombi deve effettuare un Tiro Salvezza di Tempra CD 5 + il danno subito, a meno che il danno non sia da Luce o un colpo critico. Se riesce, lo zombi scende invece a 1 punto ferita.

\textbf{Azioni}

\emph{\textbf{Mazza Chiodata.} Attacco con arma da mischia}: +6 a colpire, portata 1 m, un bersaglio.

\emph{Colpisce:} 13 (2d8 + 4) danni da botta.


\subsection{Appendice A: Creature Varie}

Questa appendice contiene le statistiche di vari animali, parassiti e
altre creature. Le statistiche sono organizzate in ordine alfabetico.

\medskip\textbf{Albero Risvegliato}\index{Mostri - Albero Risvegliato}

L'albero risvegliato è un normale albero fornito dalla magia di capacità
senziente e mobilità.

\emph{Enorme pianta, disallineato}

\textbf{FORZA} +4

\textbf{DESTREZZA} -2

\textbf{COSTITUZIONE} +2

\textbf{INTELLIGENZA} +0

\textbf{SAGGEZZA} +0

\textbf{CARISMA} -2

\textbf{Iniziativa} +0 -- \textbf{Difesa} 14

\textbf{Punti Ferita} 59 (7d12 + 14)

\textbf{Movimento} 6 m

\textbf{Tiri Salvezza}: Tempra +6, Riflessi -1, Volontà +1

\textbf{Vulnerabilità al Danno} fuoco

\textbf{Resistenze al Danno} da botta, perforante

\textbf{Lingue} una lingua conosciuta dal suo creatore

\textbf{Sfida} 2 (450 PE)

\emph{\textbf{Falso Aspetto.}} Mentre l'albero rimane immobile, è indistinguibile da un normale albero.

\textbf{Azioni}

\emph{\textbf{Schianto.} Attacco con Arma da Mischia}: +6 a colpire, portata 3 m, un bersaglio.

\emph{Colpisce:} 14 (3d6 + 4) danni da botta.

\medskip\textbf{Alce}\index{Mostri - Alce}

\emph{Grande bestia, disallineato}

\textbf{FORZA} +3

\textbf{DESTREZZA} +0

\textbf{COSTITUZIONE} +1

\textbf{INTELLIGENZA} -4

\textbf{SAGGEZZA} +0

\textbf{CARISMA} -2

\textbf{Iniziativa} +0 -- \textbf{Difesa} 11

\textbf{Punti Ferita} 13 (2d10 + 2)

\textbf{Movimento} 15 m

\textbf{Tiri Salvezza}:  Tempra +4, Riflessi +1, Volontà +0

\textbf{Lingue} -

\textbf{Sfida} 1/4 (50 PE)

\emph{\textbf{Carica.}} Se l'alce si muove di almeno 6 metri diretto verso il bersaglio e lo colpisce con un attacco di rostro durante lo stesso turno, il bersaglio subisce 7 (2d6) danni da botta aggiuntivi. Se il bersaglio è una creatura, deve riuscire un Tiro Salvezza di Tempra
CD 13 o cadere prono.

\textbf{Azioni}

\emph{\textbf{Rostro.} Attacco con Arma da Mischia}: +5 a colpire, portata 1 m, un bersaglio.

\emph{Colpisce:} 6 (1d6 + 3) danni da botta.

\emph{\textbf{Zoccoli.} Attacco con Arma da Mischia}: +5 a colpire, portata 1 m, una creatura prona.

\emph{Colpisce:} 8 (2d4 + 3) danni da botta.

\medskip\textbf{Alce Gigante}\index{Mostri - Alce Gigante}

\emph{Enorme bestia, disallineato}

\textbf{FORZA} +4

\textbf{DESTREZZA} +3

\textbf{COSTITUZIONE} +2

\textbf{INTELLIGENZA} -2

\textbf{SAGGEZZA} +2

\textbf{CARISMA} +0

\textbf{Iniziativa} +3 -- \textbf{Difesa} 15

\textbf{Punti Ferita} 42 (5d12 + 10)

\textbf{Movimento} 18 m

\textbf{Tiri Salvezza}:  Tempra +8, Riflessi +7, Volontà +2

\textbf{Competenze} Consapevolezza +4

\textbf{Lingue} Alce Gigante, comprende il Comune, l'Elfico e il

Silvano ma non può parlarli

\textbf{Sfida} 2 (450 PE)

\emph{\textbf{Carica.}} Se l'alce si muove di almeno 6 metri diretto verso il bersaglio e lo colpisce con un attacco di rostro durante lo stesso turno, il bersaglio subisce 7 (2d6) danni da botta aggiuntivi. Se il bersaglio è una creatura, deve riuscire un Tiro Salvezza di Tempra CD 14 o cadere prono.

\textbf{Azioni}

\emph{\textbf{Rostro.} Attacco con Arma da Mischia}: +6 a colpire, portata 3 m, un bersaglio.

\emph{Colpisce:} 11 (2d6 + 4) danni perforanti.

\emph{\textbf{Zoccoli.} Attacco con Arma da Mischia}: +6 a colpire, portata 1 m, una creatura prona.

\emph{Colpisce:} 22 (4d4 + 4) danni da botta.

\medskip\textbf{Aquila}\index{Mostri - Aquila}

\emph{Piccola bestia, disallineato}

\textbf{FORZA} -2

\textbf{DESTREZZA} +2

\textbf{COSTITUZIONE} +0

\textbf{INTELLIGENZA} -4

\textbf{SAGGEZZA} +2

\textbf{CARISMA} -2

\textbf{Iniziativa} +2 -- \textbf{Difesa} 13

\textbf{Punti Ferita} 3 (1d6)

\textbf{Movimento} 3 m, volo 18 m

\textbf{Tiri Salvezza}: Tempra +3, Riflessi +4, Volontà +2

\textbf{Competenze} Consapevolezza +4

\textbf{Lingue} -

\textbf{Sfida} 0 (10 PE)

\emph{\textbf{Vista Affinata.}} L'aquila ha +1d6 nelle prove di Saggezza (Consapevolezza) basate sulla vista.

\textbf{Azioni}

\emph{\textbf{Speroni.} Attacco con Arma da Mischia}: +4 a colpire, portata 1 m, un bersaglio.

\emph{Colpisce:} 4 (1d4 + 2) danni taglienti.

\medskip\textbf{Aquila Gigante}\index{Mostri - Aquila Gigante}

L'aquila gigante è una nobile creatura che parla la propria lingua e comprende quella di altre razze.

\emph{Grande bestia, neutrale buono}

\textbf{FORZA} +3

\textbf{DESTREZZA} +3

\textbf{COSTITUZIONE} +1

\textbf{INTELLIGENZA} -1

\textbf{SAGGEZZA} +2

\textbf{CARISMA} +0

\textbf{Iniziativa} +3 -- \textbf{Difesa} 14

\textbf{Punti Ferita} 26 (4d10 + 4)

\textbf{Movimento} 3 m, volo 24 m

\textbf{Tiri Salvezza}: Tempra +5, Riflessi +7, Volontà +3

\textbf{Competenze} Consapevolezza +4

\textbf{Lingue} Aquila Gigante, comprende il Comune e l'Auran ma non può parlarli

\textbf{Sfida} 1 (200 PE)

\emph{\textbf{Vista Affinata.}} L'aquila ha +1d6 nelle prove di Saggezza (Consapevolezza) basate sulla vista.

\textbf{Azioni}

\emph{\textbf{Multiattacco.}} L'aquila effettua due attacchi: uno con il becco e uno con gli speroni.

\emph{\textbf{Becco.} Attacco con Arma da Mischia}: +5 a colpire, portata 1 m, un bersaglio.

\emph{Colpisce:} 6 (1d6 + 3) danni perforanti.

\emph{\textbf{Speroni.} Attacco con Arma da Mischia}: +5 a colpire, portata 1 m, un bersaglio.

\emph{Colpisce:} 10 (2d6 + 3) danni taglienti.

\medskip\textbf{Avvoltoio}\index{Mostri - Avvoltoio}

\emph{Media bestia, disallineato}

\textbf{FORZA} -2

\textbf{DESTREZZA} +0

\textbf{COSTITUZIONE} +1

\textbf{INTELLIGENZA} -4

\textbf{SAGGEZZA} +1

\textbf{CARISMA} -3

\textbf{Iniziativa} +0 -- \textbf{Difesa} 11

\textbf{Punti Ferita} 5 (1d8 + 1)

\textbf{Movimento} 3 m, volo 15 m

\textbf{Tiri Salvezza}: Tempra +6, Riflessi +3, Volontà +1; +4 contro malattie

\textbf{Competenze} Consapevolezza +3

\textbf{Lingue} -

\textbf{Sfida} 0 (10 PE)

\emph{\textbf{Olfatto e Vista Affinati.}} L'avvoltoio ha +1d6 nelle prove di Saggezza (Consapevolezza) basate su olfatto o vista.

\emph{\textbf{Tattiche di Branco.}} L'avvoltoio ha +1d6 al tiro di attacco contro una creatura se almeno uno degli alleati dell'avvoltoio si trova entro 1 metro dalla creatura e quell'alleato non è inabile.

\textbf{Azioni}

\emph{\textbf{Becco.} Attacco con Arma da Mischia}: +2 a colpire, portata 1 m, un bersaglio.

\emph{Colpisce:} 2 (1d4) danni perforanti.

\medskip\textbf{Avvoltoio Gigante}\index{Mostri - Avvoltoio Gigante}

L'avvoltoio gigante possiede un'intelligenza superiore e un'attitudine maligna.

\emph{Grande bestia, neutrale malvagio}

\textbf{FORZA} +2

\textbf{DESTREZZA} +0

\textbf{COSTITUZIONE} +2

\textbf{INTELLIGENZA} -2

\textbf{SAGGEZZA} +1

\textbf{CARISMA} -2

\textbf{Iniziativa} +0 -- \textbf{Difesa} 11

\textbf{Punti Ferita} 22 (3d10 + 6)

\textbf{Movimento} 3 m, volo 18 m

\textbf{Tiri Salvezza}: Tempra +10, Riflessi +6, Volontà +3; +4 contro malattie

\textbf{Competenze} Consapevolezza +3

\textbf{Lingue} comprende il Comune ma non può parlare

\textbf{Sfida} 1 (200 PE)

\emph{\textbf{Olfatto e Vista Affinati.}} L'avvoltoio ha +1d6 nelle prove di Saggezza (Consapevolezza) basate su olfatto o vista.

\emph{\textbf{Tattiche di Branco.}} L'avvoltoio ha +1d6 al tiro di attacco contro una creatura se almeno uno degli alleati dell'avvoltoio si trova entro 1 metro dalla creatura e quell'alleato non è inabile.

\textbf{Azioni}

\emph{\textbf{Multiattacco.}} L'avvoltoio effettua due attacchi: uno con il becco e uno con gli speroni.

\emph{\textbf{Becco.} Attacco con Arma da Mischia}: +4 a colpire, portata 1 m, un bersaglio.

\emph{Colpisce:} 7 (2d4 + 2) danni perforanti.

\emph{\textbf{Speroni.} Attacco con Arma da Mischia}: +4 a colpire, portata 1 m, un bersaglio.

\emph{Colpisce:} 9 (2d6 + 2) danni taglienti.

\medskip\textbf{Babbuino}\index{Mostri - Babbuino}

\emph{Piccola bestia, disallineato}

\textbf{FORZA} -1

\textbf{DESTREZZA} +2

\textbf{COSTITUZIONE} +0

\textbf{INTELLIGENZA} -3

\textbf{SAGGEZZA} +1

\textbf{CARISMA} -2

\textbf{Iniziativa} +2 -- \textbf{Difesa} 13

\textbf{Punti Ferita} 3 (1d6)

\textbf{Movimento} 9 m, scalata 9 m

\textbf{Tiri Salvezza}: Tempra +3, Riflessi +4, Volontà +1

\textbf{Lingue} -

\textbf{Sfida} 0 (10 PE)

\emph{\textbf{Tattiche di Branco.}} Il babbuino ha +1d6 al tiro di attacco contro una creatura se almeno uno degli alleati del babbuino si trova entro 1 metro dalla creatura e quell'alleato non è inabile.

\textbf{Azioni}

\emph{\textbf{Morso.} Attacco con Arma da Mischia}: +1 a colpire, portata 1 m, un bersaglio.

\emph{Colpisce:} 1 (1d4 - 1) danni perforanti.


\medskip\textbf{Balena Assassina (Orca)}\index{Mostri - Orca}

\emph{Enorme bestia, disallineato}

\textbf{FORZA} +4

\textbf{DESTREZZA} +0

\textbf{COSTITUZIONE} +1

\textbf{INTELLIGENZA} -4

\textbf{SAGGEZZA} +1

\textbf{CARISMA} -2

\textbf{Iniziativa} +0 -- \textbf{Difesa} 14

\textbf{Punti Ferita} 90 (12d12 + 12)

\textbf{Movimento} 0 m, nuoto 18 m

\textbf{Tiri Salvezza}: Tempra +9, Riflessi +8, Volontà +5

\textbf{Competenze} Consapevolezza +3

\textbf{Sensi} vista cieca 36 m

\textbf{Lingue} -

\textbf{Sfida} 3 (700 PE)

\emph{\textbf{Ecolocazione.}} La balena non può usare la vista cieca se assordata.

\emph{\textbf{Trattenere il Fiato.}} La balena può trattenere il fiato per 30 minuti

\emph{\textbf{Udito Affinato.}} La balena ha +1d6 alle prove di Saggezza (Consapevolezza) basate sull'udito.

\textbf{Azioni}

\emph{\textbf{Morso.} Attacco con Arma da Mischia}: +6 a colpire, portata 1 m, un bersaglio.

\emph{Colpisce:} 21 (5d6 + 4) danni perforanti.

\medskip\textbf{Becco d'Ascia}\index{Mostri - Becco d'Ascia}

Il becco d'ascia è un grosso e slanciato volatile privo di ali ma con potenti gambe, un becco a cuneo, e un pessimo carattere.

\emph{Grande bestia, disallineato}

\textbf{FORZA} +2

\textbf{DESTREZZA} +1

\textbf{COSTITUZIONE} +1

\textbf{INTELLIGENZA} -4

\textbf{SAGGEZZA} +0

\textbf{CARISMA} -3

\textbf{Iniziativa} +1 -- \textbf{Difesa} 12

\textbf{Punti Ferita} 19 (3d10 + 3)

\textbf{Movimento} 15 m

\textbf{Tiri Salvezza}: Tempra +3, Riflessi +1, Volontà +1

\textbf{Lingue} -

\textbf{Sfida} 1/4 (50 PE)

\textbf{Azioni}

\emph{\textbf{Becco.} Attacco con Arma da Mischia}: +4 a colpire, portata 1 m, un bersaglio.

\emph{Colpisce:} 6 (1d8 + 2) danni taglienti.

\medskip\textbf{Cammello}\index{Mostri - Cammello}

\emph{Grande bestia, disallineato}

\textbf{FORZA} +3

\textbf{DESTREZZA} -1

\textbf{COSTITUZIONE} +2

\textbf{INTELLIGENZA} -4

\textbf{SAGGEZZA} -1

\textbf{CARISMA} -3

\textbf{Iniziativa} -1 -- \textbf{Difesa} 10

\textbf{Punti Ferita} 15 (2d10 + 4)

\textbf{Movimento} 15 m

\textbf{Tiri Salvezza}: Tempra +5, Riflessi +6, Volontà +0 

\textbf{Lingue} -

\textbf{Sfida} 1/8 (25 PE)

\textbf{Azioni}

\emph{\textbf{Morso.} Attacco con Arma da Mischia}: +5 a colpire, portata 1 m, un bersaglio.

\emph{Colpisce:} 2 (1d4) danni da botta.

\medskip\textbf{Cane della Morte}\index{Mostri - Cane della Morte}

Il cane della morte è un orribile segugio a due teste che si aggira per pianure, deserti e sotterranei.

\emph{Media mostruosità, neutrale malvagio}

\textbf{FORZA} +2

\textbf{DESTREZZA} +2

\textbf{COSTITUZIONE} +2

\textbf{INTELLIGENZA} -4

\textbf{SAGGEZZA} +1

\textbf{CARISMA} -2

\textbf{Iniziativa} +2 -- \textbf{Difesa} 13

\textbf{Punti Ferita} 39 (6d8 + 12)

\textbf{Movimento} 12 m

\textbf{Tiri Salvezza}: Tempra +4, Riflessi +5, Volontà +2

\textbf{Competenze} Muoversi Silenziosamente / Nascondersi +4, Consapevolezza +5

\textbf{Sensi} visione al buio 36 m

\textbf{Lingue} -

\textbf{Sfida} 1 (200 PE)

\emph{\textbf{Bicefalo.}} Il cane ha +1d6 nelle prove di Saggezza (Consapevolezza) e nei Tiri Salvezza contro le condizioni accecato, affascinato, assordato, spaventato, stordito o svenuto.

\textbf{Azioni}

\emph{\textbf{Multiattacco.}} Il cane effettua due attacchi di morso. 

\emph{\textbf{Morso.} Attacco con Arma da Mischia}: +4 a colpire, portata 1 m, un bersaglio.

\emph{Colpisce:} 5 (1d6 + 2) danni perforanti. Se il bersaglio è una creatura, deve riuscire un Tiro Salvezza di Tempra CD 12 contro la malattia o restare avvelenato finché la malattia non viene curata. Dopo ogni 24 ore, la creatura deve ripetere il Tiro Salvezza, riducendo i suoi punti ferita massimi di 5 (1d10) in caso di fallimento. Questa riduzione perdura finché la malattia non viene curata. La creatura muore se la malattia riduce i suoi punti ferita massimi a 0.

\medskip\textbf{Cane Intermittente}\index{Mostri - Cane Intermittente}

Il cane intermittente deriva il nome dalla sua abilità di entrare e uscire dalla realtà, un talento che usa per attaccare ed evitare di essere attaccato.

\emph{Media fatato, legale buono}

\textbf{FORZA} +1

\textbf{DESTREZZA} +3

\textbf{COSTITUZIONE} +1

\textbf{INTELLIGENZA} +0

\textbf{SAGGEZZA} +1

\textbf{CARISMA} +0

\textbf{Iniziativa} +3 -- \textbf{Difesa} 14

\textbf{Punti Ferita} 22 (4d8 + 4)

\textbf{Vulnerabilità al Danno} ferro freddo

\textbf{Movimento} 12 m

\textbf{Tiri Salvezza}:  Tempra +5, Riflessi +5, Volontà +4

\textbf{Competenze} Muoversi Silenziosamente / Nascondersi +5, Consapevolezza +3

\textbf{Lingue} Cane Intermittente, comprende il Silvano ma non può parlarlo

\textbf{Sfida} 1/4 (50 PE)

\emph{\textbf{Udito e Olfatto Affinato.}} Il cane ha +1d6 nelle prove di Saggezza (Consapevolezza) basate su udito o olfatto.

\textbf{Azioni}

\emph{\textbf{Morso.} Attacco con Arma da Mischia}: +3 a colpire, portata 1 m, un bersaglio.

\emph{Colpisce:} 4 (1d6 + 1) danni perforanti.

\emph{\textbf{Teletrasporto (Ricarica 4-6).}} Il cane si teletrasporta magicamente, insieme a qualsiasi cosa stia indossando o trasportando, fino a 12 metri in uno spazio non occupato che possa vedere. Prima o dopo il teletrasporto, il cane può effettuare un attacco di morso.

\medskip\textbf{Caprone}\index{Mostri - Caprone}

\emph{Media bestia, disallineato}

\textbf{FORZA} +1

\textbf{DESTREZZA} +0

\textbf{COSTITUZIONE} +0

\textbf{INTELLIGENZA} -4

\textbf{SAGGEZZA} +0

\textbf{CARISMA} -3

\textbf{Iniziativa} +0 -- \textbf{Difesa} 11

\textbf{Punti Ferita} 4 (1d8)

\textbf{Movimento} 12 m

\textbf{Tiri Salvezza}: Tempra +1, Riflessi +1, Volontà +0 

\textbf{Lingue} -

\textbf{Sfida} 0 (10 PE)

\emph{\textbf{Carica.}} Se il caprone si muove di almeno 6 metri diretto verso il bersaglio e colpisce con un attacco di rostro durante lo stesso turno, il bersaglio subisce 2 (1d4) danni da botta aggiuntivi. Se il bersaglio è una creatura, deve riuscire un Tiro Salvezza di Tempra CD 10
o cadere prona.

\emph{\textbf{Piedi Saldi.}} Il caprone ha +1d6 ai Tiri Salvezza su Tempra e Riflessi effettuati contro effetti che lo farebbero cadere prono.

\textbf{Azioni}

\emph{\textbf{Rostro.} Attacco con Arma da Mischia}: +3 a colpire, portata 1 m, un bersaglio.

\emph{Colpisce:} 3 (1d4 + 1) danni da botta.

\medskip\textbf{Caprone Gigante}\index{Mostri - Caprone Gigante}

\emph{Grande bestia, disallineato}

\textbf{FORZA} +3

\textbf{DESTREZZA} +0

\textbf{COSTITUZIONE} +1

\textbf{INTELLIGENZA} -4

\textbf{SAGGEZZA} +1

\textbf{CARISMA} -2

\textbf{Iniziativa} +0 -- \textbf{Difesa} 12

\textbf{Punti Ferita} 19 (3d10 + 3)

\textbf{Movimento} 12 m

\textbf{Tiri Salvezza}: Tempra +4, Riflessi +1, Volontà +1 

\textbf{Lingue} -

\textbf{Sfida} 1/2 (100 PE)

\emph{\textbf{Carica.}} Se il caprone si muove di almeno 6 metri diretto verso il bersaglio e colpisce con un attacco di rostro durante lo stesso turno, il bersaglio subisce 5 (2d4) danni da botta aggiuntivi. Se il bersaglio è una creatura, deve riuscire un Tiro Salvezza di Tempra CD 13 o cadere prona.

\emph{\textbf{Piedi Saldi.}} Il caprone ha +1d6 ai Tiri Salvezza su Tempra e Riflessi effettuati contro effetti che lo farebbero cadere prono.

\textbf{Azioni}

\emph{\textbf{Rostro.} Attacco con Arma da Mischia}: +5 a colpire, portata 1 m, un bersaglio.

\emph{Colpisce:} 8 (2d4 + 3) danni da botta.

\medskip\textbf{Cavallo da Corsa}\index{Mostri - Cavallo da Corsa}

\emph{Grande bestia, disallineato}

\textbf{FORZA} +3

\textbf{DESTREZZA} +0

\textbf{COSTITUZIONE} +1

\textbf{INTELLIGENZA} -4

\textbf{SAGGEZZA} +0

\textbf{CARISMA} -2

\textbf{Iniziativa} +0 -- \textbf{Difesa} 11

\textbf{Punti Ferita} 13 (2d10 + 2)

\textbf{Movimento} 18 m

\textbf{Tiri Salvezza}: Tempra +3, Riflessi +1, Volontà +1 

\textbf{Lingue} -

\textbf{Sfida} 1/4 (50 PE)

\textbf{Azioni}

\emph{\textbf{Zoccoli.} Attacco con Arma da Mischia}: +5 a colpire, portata 1 m, un bersaglio.

\emph{Colpisce:} 8 (2d4 + 3) danni da botta.

\medskip\textbf{Cavallo da Guerra}\index{Mostri - Cavallo da Guerra}

\emph{Grande bestia, disallineato}

\textbf{FORZA} +4

\textbf{DESTREZZA} +1

\textbf{COSTITUZIONE} +1

\textbf{INTELLIGENZA} -4

\textbf{SAGGEZZA} +1

\textbf{CARISMA} -2

\textbf{Iniziativa} +1 -- \textbf{Difesa} 12 (più possibile bardatura)

\textbf{Punti Ferita} 19 (3d10 + 3)

\textbf{Movimento} 18 m

\textbf{Tiri Salvezza}:  Tempra +4, Riflessi +2, Volontà +1 

\textbf{Lingue} -

\textbf{Sfida} 1/2 (100 PE)

\emph{\textbf{Carica Travolgente.}} Se il cavallo si muove di almeno 6 metri diretto verso il bersaglio e lo colpisce con un attacco di zoccoli durante lo stesso turno, il bersaglio deve riuscire un Tiro Salvezza su Tempra CD 14 o cadere prono. Se il bersaglio è prono, il cavallo può effettuare un altro attacco di zoccoli contro di lui come azione bonus.

\textbf{Azioni}

\emph{\textbf{Zoccoli.} Attacco con Arma da Mischia}: +6 a colpire, portata 1 m, un bersaglio.

\emph{Colpisce:} 11 (2d6 + 4) danni da botta.

\medskip\textbf{Cavallo da Tiro}\index{Mostri - Cavallo da Tiro}

\emph{Grande bestia, disallineato}

\textbf{FORZA} +4

\textbf{DESTREZZA} +0

\textbf{COSTITUZIONE} +1

\textbf{INTELLIGENZA} -4

\textbf{SAGGEZZA} +0

\textbf{CARISMA} -2

\textbf{Iniziativa} +0 -- \textbf{Difesa} 11

\textbf{Punti Ferita} 19 (3d10 + 3)

\textbf{Movimento} 12 m

\textbf{Tiri Salvezza}:  Tempra +5, Riflessi +1, Volontà +2 

\textbf{Lingue} -

\textbf{Sfida} 1/4 (50 PE)

\textbf{Azioni}

\emph{\textbf{Zoccoli.} Attacco con Arma da Mischia}: +6 a colpire, portata 1 m, un bersaglio.

\emph{Colpisce:} 9 (2d4 + 4) danni da botta.

\medskip\textbf{Cavallo Marino Gigante}\index{Mostri - Cavallo Marino Gigante}

Il cavallo marino gigante viene spesso impiegato come cavalcatura dagli umanoidi acquatici.

\emph{Grande bestia, disallineato}

\textbf{FORZA} +1

\textbf{DESTREZZA} +2

\textbf{COSTITUZIONE} +0

\textbf{INTELLIGENZA} -4

\textbf{SAGGEZZA} +1

\textbf{CARISMA} -3

\textbf{Iniziativa} +2 -- \textbf{Difesa} 14

\textbf{Punti Ferita} 16 (3d10)

\textbf{Movimento} 0 m, nuoto 12 m

\textbf{Tiri Salvezza}: Tempra +2, Riflessi +3, Volontà +1

\textbf{Lingue} -

\textbf{Sfida} 1/2 (100 PE)

\emph{\textbf{Carica.}} Se il cavallo marino si muove di almeno 6 metri diretto verso il bersaglio e colpisce con un attacco di rostro durante lo stesso turno, il bersaglio subisce 7 (2d6) danni da botta aggiuntivi. Se il bersaglio è una creatura, deve riuscire un Tiro Salvezza su Tempra CD 11 o cadere prona.

\emph{\textbf{Respirare Acqua.}} Il cavallo marino può respirare solo sottacqua.

\textbf{Azioni}

\emph{\textbf{Rostro.} Attacco con Arma da Mischia}: +3 a colpire, portata 1 m, un bersaglio.

\emph{Colpisce:} 4 (1d6 + 1) danni da botta.

\medskip\textbf{Centopiedi Gigante}\index{Centopiedi Gigante}

\emph{Piccola bestia, disallineato}

\textbf{FORZA} -3

\textbf{DESTREZZA} +2

\textbf{COSTITUZIONE} +1

\textbf{INTELLIGENZA} -5

\textbf{SAGGEZZA} -2

\textbf{CARISMA} -4

\textbf{Iniziativa} +2 -- \textbf{Difesa} 14

\textbf{Punti Ferita} 4 (1d6 + 1)

\textbf{Movimento} 9 m, scalata 9 m

\textbf{Tiri Salvezza}: Tempra -2, Riflessi +3, Volontà -2 

\textbf{Sensi} vista cieca 9 m

\textbf{Lingue} -

\textbf{Sfida} 1/4 (50 PE)

\textbf{Azioni}

\emph{\textbf{Morso.} Attacco con Arma da Mischia}: +4 a colpire, portata 1 m, una creatura.

\emph{Colpisce:} 4 (1d4 + 2) danni perforanti e il bersaglio deve riuscire un Tiro Salvezza di Tempra CD 11 o subire 10 (3d6) danni da veleno. Se il danno da veleno riduce il bersaglio a 0 punti ferita, il bersaglio è stabile ma resta avvelenato per 1 ora, anche dopo aver recuperato i punti ferita, e mentre è avvelenato in questo modo resta paralizzato.

\medskip\textbf{Cervo}\index{Mostri - Cervo}

\emph{Media bestia, disallineato}

\textbf{FORZA} +0

\textbf{DESTREZZA} +3

\textbf{COSTITUZIONE} +0

\textbf{INTELLIGENZA} -4

\textbf{SAGGEZZA} +2

\textbf{CARISMA} -3

\textbf{Iniziativa} +3 -- \textbf{Difesa} 14

\textbf{Punti Ferita} 4 (1d8)

\textbf{Movimento} 15 m

\textbf{Tiri Salvezza}: Tempra +2, Riflessi +3, Volontà +2 

\textbf{Lingue} -

\textbf{Sfida} 0 (10 PE)

\textbf{Azioni}

\emph{\textbf{Morso.} Attacco con Arma da Mischia}: +2 a colpire, portata 1 m, un bersaglio.

\emph{Colpisce:} 2 (1d4) danni perforanti.

\medskip\textbf{Cinghiale}\index{Mostri - Cinghiale}

\emph{Media bestia, disallineato}

\textbf{FORZA} +1

\textbf{DESTREZZA} +0

\textbf{COSTITUZIONE} +1

\textbf{INTELLIGENZA} -4

\textbf{SAGGEZZA} -1

\textbf{CARISMA} -3

\textbf{Iniziativa} +0 -- \textbf{Difesa} 12

\textbf{Punti Ferita} 11 (2d8 + 2)

\textbf{Movimento} 12 m

\textbf{Tiri Salvezza}: Tempra +2, Riflessi +1, Volontà -1 

\textbf{Lingue} -

\textbf{Sfida} 1/4 (50 PE)

\emph{\textbf{Carica.}} Se il cinghiale si muove di almeno 6 metri diretto verso il bersaglio e colpisce con un attacco di zanna durante lo stesso turno, il bersaglio subisce 3 (1d6) danni taglienti aggiuntivi. Se il bersaglio è una creatura, deve riuscire un Tiro Salvezza di Tempra
CD 11 o cadere prono.

\emph{\textbf{Implacabile (Ricarica dopo un 1 ora).}} Se il cinghiale subisce 7 danni o meno che lo ridurrebbero a 0 punti ferita, scende invece a 1 punto ferita.

\textbf{Azioni}

\emph{\textbf{Zanna.} Attacco con Arma da Mischia}: +3 a colpire, portata 1 m, un bersaglio.

\emph{Colpisce:} 4 (1d6 + 1) danni taglienti.

\medskip\textbf{Cinghiale Gigante}\index{Mostri - Cinghiale Gigante}

\emph{Grande bestia, disallineato}

\textbf{FORZA} +3

\textbf{DESTREZZA} +0

\textbf{COSTITUZIONE} +3

\textbf{INTELLIGENZA} -4

\textbf{SAGGEZZA} -2

\textbf{CARISMA} -3

\textbf{Iniziativa} +0 -- \textbf{Difesa} 13

\textbf{Punti Ferita} 42 (5d10 + 15)

\textbf{Movimento} 12 m

\textbf{Tiri Salvezza}: Tempra +4, Riflessi +2, Volontà +0

\textbf{Lingue} -

\textbf{Sfida} 2 (450 PE)

\emph{\textbf{Carica.}} Se il cinghiale si muove di almeno 6 metri diretto verso il bersaglio e colpisce con un attacco di zanna durante lo stesso turno, il bersaglio subisce 7 (2d6) danni taglienti aggiuntivi. Se il bersaglio è una creatura, deve riuscire un Tiro Salvezza di Tempra CD 13 o cadere prono.

\emph{\textbf{Implacabile (Ricarica dopo un 1 ora).}} Se il cinghiale subisce 10 danni o meno che lo ridurrebbero a 0 punti ferita, scende invece a 1 punto ferita.

\textbf{Azioni}

\emph{\textbf{Zanna.} Attacco con Arma da Mischia}: +5 a colpire, portata 1 m, un bersaglio.

\emph{Colpisce:} 10 (2d6 + 3) danni taglienti.

\medskip\textbf{Coccodrillo}\index{Mostri - Coccodrillo}

\emph{Grande bestia, disallineato}

\textbf{FORZA} +2

\textbf{DESTREZZA} +0

\textbf{COSTITUZIONE} +1

\textbf{INTELLIGENZA} -4

\textbf{SAGGEZZA} +0

\textbf{CARISMA} -3

\textbf{Iniziativa} +0 -- \textbf{Difesa} 13

\textbf{Punti Ferita} 19 (3d10 + 3)

\textbf{Movimento} 6 m, nuoto 9 m

\textbf{Tiri Salvezza}: Tempra +6, Riflessi +4, Volontà +2 

\textbf{Competenze} Muoversi Silenziosamente / Nascondersi +2

\textbf{Lingue} -

\textbf{Sfida} 1/2 (100 PE)

\emph{\textbf{Trattenere il Fiato.}} Il coccodrillo può trattenere il fiato per 15 minuti.

\textbf{Azioni}

\emph{\textbf{Morso.} Attacco con Arma da Mischia}: +4 a colpire, portata 1 m, una creatura.

\emph{Colpisce:} 7 (1d10 + 2) danni perforanti, e il bersaglio è afferrato (CD 12 per fuggire). Fino al termine dell'afferrare, il bersaglio è intralciato, e il coccodrillo non può usare il morso contro un altro bersaglio.

\medskip\textbf{Coccodrillo Gigante}\index{Mostri - Coccodrillo Gigante}

\emph{Enorme bestia, disallineato}

\textbf{FORZA} +5

\textbf{DESTREZZA} -1

\textbf{COSTITUZIONE} +3

\textbf{INTELLIGENZA} -4

\textbf{SAGGEZZA} +0

\textbf{CARISMA} -2

\textbf{Iniziativa} -1 -- \textbf{Difesa} 15

\textbf{Punti Ferita} 85 (9d12 + 27)

\textbf{Movimento} 9 m, nuoto 15 m

\textbf{Tiri Salvezza}: Tempra +15, Riflessi +8, Volontà +8

\textbf{Competenze} Muoversi Silenziosamente / Nascondersi +5

\textbf{Lingue} -

\textbf{Sfida} 5 (1.800 PE)

\emph{\textbf{Trattenere il Fiato.}} Il coccodrillo può trattenere il fiato per 30 minuti.

\textbf{Azioni}

\emph{\textbf{Multiattacco.}} Il coccodrillo effettua due attacchi: uno con il morso e uno con la coda.

\emph{\textbf{Coda.} Attacco con Arma da Mischia}: +8 a colpire, portata 3 m, un bersaglio non afferrato dal coccodrillo.

\emph{Colpisce:} 14 (2d8 + 5) danni da botta. Se il bersaglio è una creatura, deve riuscire un Tiro Salvezza di Tempra CD 16 o cadere prono.

\emph{\textbf{Morso.} Attacco con Arma da Mischia}: +8 a colpire, portata 1 m, un bersaglio.

\emph{Colpisce:} 21 (3d10 + 5) danni perforanti, e il bersaglio è afferrato (CD 16 per fuggire). Fino al termine dell'afferrare, il bersaglio è intralciato, e il coccodrillo non può usare il morso contro un altro bersaglio.

\medskip\textbf{Corvo}\index{Mostri - Corvo}

\emph{Minuscola bestia, disallineato}

\textbf{FORZA} -4

\textbf{DESTREZZA} +2

\textbf{COSTITUZIONE} -1

\textbf{INTELLIGENZA} -4

\textbf{SAGGEZZA} +1

\textbf{CARISMA} -2

\textbf{Iniziativa} +2 -- \textbf{Difesa} 13

\textbf{Punti Ferita} 1 (1d4 - 1)

\textbf{Movimento} 3 m, volo 15 m

\textbf{Tiri Salvezza}: Tempra +1, Riflessi +4, Volontà +2

\textbf{Competenze} Consapevolezza +3

\textbf{Lingue} -

\textbf{Sfida} 0 (10 PE)

\emph{\textbf{Imitazione.}} Il corvo può imitare dei semplici suoni che ha udito, come il sussurro di una persona, il pianto di un bambino o il verso di un animale. Una creatura che ode il suono può identificarlo come imitazione riuscendo una prova di Saggezza (Sopravvivenza) CD 10.

\textbf{Azioni}

\emph{\textbf{Becco.} Attacco con Arma da Mischia}: +4 a colpire, portata 1 m, un bersaglio.

\emph{Colpisce:} 1 danno perforante.

\medskip\textbf{Donnola}\index{Mostri - Donnola}

\emph{Minuscola bestia, disallineato}

\textbf{FORZA} -4

\textbf{DESTREZZA} +3

\textbf{COSTITUZIONE} -1

\textbf{INTELLIGENZA} -4

\textbf{SAGGEZZA} +1

\textbf{CARISMA} -4

\textbf{Iniziativa} +3 -- \textbf{Difesa} 14

\textbf{Punti Ferita} 1 (1d4 - 1)

\textbf{Movimento} 9 m

\textbf{Tiri Salvezza}: Tempra +2, Riflessi +4, Volontà +1

\textbf{Competenze} Muoversi Silenziosamente / Nascondersi +5, Consapevolezza +3

\textbf{Lingue} -

\textbf{Sfida} 0 (10 PE)

\emph{\textbf{Udito e Olfatto Affinati.}} La donnola ha +1d6 nelle prove di Saggezza (Consapevolezza) basate su udito o olfatto.

\textbf{Azioni}

\emph{\textbf{Morso.} Attacco con Arma da Mischia}: +5 a colpire, portata 1 m, un bersaglio.

\emph{Colpisce:} 1 danno perforante.

\medskip\textbf{Donnola Gigante}\index{Mostri - Donnola Gigante}

\emph{Media bestia, disallineato}

\textbf{FORZA} +0

\textbf{DESTREZZA} +3

\textbf{COSTITUZIONE} +0

\textbf{INTELLIGENZA} -3

\textbf{SAGGEZZA} +1

\textbf{CARISMA} -3

\textbf{Iniziativa} +3 -- \textbf{Difesa} 14

\textbf{Punti Ferita} 9 (2d8)

\textbf{Movimento} 12 m

\textbf{Tiri Salvezza}:  Tempra +6, Riflessi +7, Volontà +2 

\textbf{Competenze} Muoversi Silenziosamente / Nascondersi +5, Consapevolezza +3

\textbf{Sensi} visione al buio 18 m

\textbf{Lingue} -

\textbf{Sfida} 1/8 (25 PE)

\emph{\textbf{Udito e Olfatto Affinati.}} La donnola ha +1d6 nelle prove di Saggezza (Consapevolezza) basate su udito o olfatto.

\textbf{Azioni}

\emph{\textbf{Morso.} Attacco con Arma da Mischia}: +5 a colpire, portata 1 m, un bersaglio.

\emph{Colpisce:} 5 (1d4 + 3) danni perforanti.

\medskip\textbf{Elefante}\index{Mostri - Elefante}

\emph{Enorme bestia, disallineato}

\textbf{FORZA} +6

\textbf{DESTREZZA} -1

\textbf{COSTITUZIONE} +3

\textbf{INTELLIGENZA} -4

\textbf{SAGGEZZA} +0

\textbf{CARISMA} -2

\textbf{Iniziativa} -1 -- \textbf{Difesa} 14

\textbf{Punti Ferita} 76 (8d12 + 24)

\textbf{Movimento} 12 m

\textbf{Tiri Salvezza}: Tempra +13, Riflessi +7, Volontà +6 

\textbf{Lingue} -

\textbf{Sfida} 4 (1.000 PE)

\emph{\textbf{Carica Travolgente.}} Se l'elefante si muove di almeno 6 metri diretto verso una creatura e la colpisce con un attacco di incornata durante lo stesso turno, il bersaglio deve riuscire un Tiro Salvezza su Tempra CD 12 o cadere prono. Se il bersaglio è prono, l'elefante può effettuare un attacco di pestone contro di lui come azione bonus.

\textbf{Azioni}

\emph{\textbf{Incornata.} Attacco con Arma da Mischia}: +8 a colpire, portata 1 m, un bersaglio.

\emph{Colpisce:} 19 (3d8 + 6) danni perforanti. 

\emph{\textbf{Pestone.} Attacco con Arma da Mischia}: +8 a colpire, portata 1 m, un bersaglio prono.

\emph{Colpisce:} 22 (3d10 + 6) danni da botta.

\medskip\textbf{Falco}\index{Mostri - Falco}

\emph{Minuscola bestia, disallineato}

\textbf{FORZA} -3

\textbf{DESTREZZA} +3

\textbf{COSTITUZIONE} -1

\textbf{INTELLIGENZA} -4

\textbf{SAGGEZZA} +2

\textbf{CARISMA} -2

\textbf{Iniziativa} +3 -- \textbf{Difesa} 14

\textbf{Punti Ferita} 1 (1d4 - 1)

\textbf{Movimento} 3 m, volo 18 m

\textbf{Tiri Salvezza}: Tempra +2, Riflessi +5, Volontà +2 

\textbf{Competenze} Consapevolezza +4

\textbf{Lingue} -

\textbf{Sfida} 0 (10 PE)

\emph{\textbf{Vista Affinata.}} Il falco ha +1d6 alle prove di Saggezza (Consapevolezza) basate sulla vista.

\textbf{Azioni}

\emph{\textbf{Speroni.} Attacco con Arma da Mischia}: +5 a colpire, portata 1 m, un bersaglio.

\emph{Colpisce:} 1 danno tagliente.

\medskip\textbf{Falco di Sangue}\index{Mostri - Falco di Sangue}

Dovendo il suo nome alle sue piume cremisi e alla sua natura aggressiva, il falco di sangue attacca senza timore usando il suo becco appuntito.

\emph{Piccola bestia, disallineato}

\textbf{FORZA} -2

\textbf{DESTREZZA} +2

\textbf{COSTITUZIONE} +0

\textbf{INTELLIGENZA} -4

\textbf{SAGGEZZA} +2

\textbf{CARISMA} -3

\textbf{Iniziativa} +2 -- \textbf{Difesa} 13

\textbf{Punti Ferita} 7 (2d6)

\textbf{Movimento} 3 m, volo 18 m

\textbf{Tiri Salvezza}: Tempra +3, Riflessi +6, Volontà +3 

\textbf{Competenze} Consapevolezza +4

\textbf{Lingue} -

\textbf{Sfida} 1/8 (25 PE)

\emph{\textbf{Tattiche di Branco.}} Il falco ha +1d6 ai tiri di attacco contro una creatura se almeno uno degli alleati del falco si trova entro 1 metro dalla creatura e quell'alleato non è inabile.

\emph{\textbf{Vista Affinata.}} Il falco ha +1d6 alle prove di Saggezza (Consapevolezza) basate sulla vista.

\textbf{Azioni}

\emph{\textbf{Becco.} Attacco con Arma da Mischia}: +4 a colpire, portata 1 m, un bersaglio.

\emph{Colpisce:} 4 (1d4 + 2) danni perforanti.

\medskip\textbf{Pirana}\index{Mostri - Pirana}

Il pirana è un pesce carnivoro dai denti affilati.

\emph{Minuscola bestia, disallineato}

\textbf{FORZA} -4

\textbf{DESTREZZA} +3

\textbf{COSTITUZIONE} -1

\textbf{INTELLIGENZA} -5

\textbf{SAGGEZZA} -2

\textbf{CARISMA} -4

\textbf{Iniziativa} +3 -- \textbf{Difesa} 14

\textbf{Punti Ferita} 1 (1d4 - 1)

\textbf{Movimento} 0 m, nuoto 12 m

\textbf{Tiri Salvezza}: Tempra -4, Riflessi +3, Volontà -2 

\textbf{Sensi} visione al buio 18 m

\textbf{Lingue} -

\textbf{Sfida} 0 (10 PE)

\emph{\textbf{Frenesia Sanguinaria.}} Il pirana ha +1d6 ai tiri di attacco in mischia contro qualsiasi creatura che non sia al massimo dei punti ferita.

\emph{\textbf{Respirare Acqua.}} Il pirana può respirare solo sottacqua. 

\textbf{Azioni}

\emph{\textbf{Morso.} Attacco con Arma da Mischia}: +5 a colpire, portata 1 m, un bersaglio.

\emph{Colpisce:} 1 danno perforante.

\medskip\textbf{Gatto}\index{Mostri - Gatto}

\emph{Minuscola bestia, disallineato}

\textbf{FORZA} -4

\textbf{DESTREZZA} +2

\textbf{COSTITUZIONE} +0

\textbf{INTELLIGENZA} -4

\textbf{SAGGEZZA} +1

\textbf{CARISMA} -2

\textbf{Iniziativa} +2 -- \textbf{Difesa} 13

\textbf{Punti Ferita} 2 (1d4)

\textbf{Movimento} 12 m, scalata 9 m

\textbf{Tiri Salvezza}:  Tempra +1, Riflessi +4, Volontà +1

\textbf{Competenze} Muoversi Silenziosamente / Nascondersi +4, Consapevolezza +3

\textbf{Lingue} -

\textbf{Sfida} 0 (10 PE)

\emph{\textbf{Olfatto Affinato.}} Il gatto ha +1d6 alle prove di Saggezza (Consapevolezza) basate sull'olfatto.

\textbf{Azioni}

\emph{\textbf{Artigli.} Attacco con Arma da Mischia}: +0 a colpire, portata 1 m, un bersaglio.

\emph{Colpisce:} 1 danno tagliente.

\medskip\textbf{Granchio Gigante}\index{Mostri - Granchio Gigante}

\emph{Media bestia, disallineato}

\textbf{FORZA} +1

\textbf{DESTREZZA} +2

\textbf{COSTITUZIONE} +0

\textbf{INTELLIGENZA} -5

\textbf{SAGGEZZA} -1

\textbf{CARISMA} -4

\textbf{Iniziativa} +2 -- \textbf{Difesa} 16

\textbf{Punti Ferita} 13 (3d8)

\textbf{Movimento} 9 m, nuoto 9 m

\textbf{Tiri Salvezza}: Tempra +5, Riflessi +2, Volontà +1

\textbf{Competenze} Muoversi Silenziosamente / Nascondersi +4

\textbf{Sensi} vista cieca 9 m

\textbf{Lingue} -

\textbf{Sfida} 1/8 (25 PE)

\emph{\textbf{Anfibio.}} Il granchio può respirare aria e acqua.

\textbf{Azioni}

\emph{\textbf{Artiglio (Chela).} Attacco con Arma da Mischia}: +3 a colpire, portata 1 m, un bersaglio.

\emph{Colpisce:} 4 (1d6 + 1) danni da botta e il bersaglio è afferrato (CD 11 per fuggire). Il granchio ha due chele, ciascuna delle quali può afferrare un solo bersaglio.

\medskip\textbf{Gufo}\index{Mostri - Gufo}

\emph{Minuscola bestia, disallineato}

\textbf{FORZA} -4

\textbf{DESTREZZA} +1

\textbf{COSTITUZIONE} -1

\textbf{INTELLIGENZA} -4

\textbf{SAGGEZZA} +1

\textbf{CARISMA} -2

\textbf{Iniziativa} +1 -- \textbf{Difesa} 12

\textbf{Punti Ferita} 1 (1d4 - 1)

\textbf{Movimento} 1 m, volo 18 m

\textbf{Tiri Salvezza}: Tempra +2, Riflessi +5, Volontà +2 

\textbf{Competenze} Muoversi Silenziosamente / Nascondersi +3, Consapevolezza +3

\textbf{Sensi} visione al buio 36 m

\textbf{Lingue} -

\textbf{Sfida} 0 (10 PE)

\emph{\textbf{Sorvolare.}} Il gufo non provoca attacchi di opportunità quando vola via dalla portata di un nemico.

\emph{\textbf{Udito e Vista Affinati.}} Il gufo ha +1d6 nelle prove di Saggezza (Consapevolezza) basate su udito o vista.

\textbf{Azioni}

\emph{\textbf{Speroni.} Attacco con Arma da Mischia}: +3 a colpire, portata 1 m, un bersaglio.

\emph{Colpisce:} 1 danno tagliente.

\medskip\textbf{Gufo Gigante}\index{Mostri - Gufo Gigante}

I gufi giganti sono creature intelligenti che proteggono i regni silvani.

\emph{Grande bestia, neutrale}

\textbf{FORZA} +1

\textbf{DESTREZZA} +2

\textbf{COSTITUZIONE} +1

\textbf{INTELLIGENZA} -1

\textbf{SAGGEZZA} +1

\textbf{CARISMA} +0

\textbf{Iniziativa} +2 -- \textbf{Difesa} 13

\textbf{Punti Ferita} 19 (3d10 + 3)

\textbf{Movimento} 1 m, volo 18 m

\textbf{Tiri Salvezza}: Tempra +1, Riflessi +4, Volontà +1 

\textbf{Competenze} Muoversi Silenziosamente / Nascondersi +4, Consapevolezza +5

\textbf{Sensi} visione al buio 36 m

\textbf{Lingue} Gufo Gigante, comprende Comune, Elfico e Silvano ma non può parlarli

\textbf{Sfida} 1/4 (50 PE)

\emph{\textbf{Sorvolare.}} Il gufo non provoca attacchi di opportunità quando vola via dalla portata di un nemico.

\emph{\textbf{Udito e Vista Affinati.}} Il gufo ha +1d6 nelle prove di Saggezza (Consapevolezza) basate su udito o vista.

\textbf{Azioni}

\emph{\textbf{Speroni.} Attacco con Arma da Mischia}: +3 a colpire, portata 1 m, un bersaglio.

\emph{Colpisce:} 8 (2d6 + 1) danni perforanti.

\medskip\textbf{Iena}\index{Mostri - Iena}

\emph{Media bestia, disallineato}

\textbf{FORZA} +0

\textbf{DESTREZZA} +1

\textbf{COSTITUZIONE} +1

\textbf{INTELLIGENZA} -4

\textbf{SAGGEZZA} +1

\textbf{CARISMA} -3

\textbf{Iniziativa} +1 -- \textbf{Difesa} 12

\textbf{Punti Ferita} 5 (1d8 + 1)

\textbf{Movimento} 15 m

\textbf{Tiri Salvezza}: Tempra +5, Riflessi +5, Volontà +1

\textbf{Competenze} Consapevolezza +3

\textbf{Lingue} -

\textbf{Sfida} 0 (10 PE)

\emph{\textbf{Tattiche di Branco.}} La iena ha +1d6 ai tiri di attacco contro una creatura se almeno uno degli alleati della iena si trova entro 1 metro dalla creatura e quell'alleato non è inabile.

\textbf{Azioni}

\emph{\textbf{Morso.} Attacco con Arma da Mischia}: +2 a colpire, portata 1 m, un bersaglio.

\emph{Colpisce:} 3 (1d6) danni perforanti.

\medskip\textbf{Iena Gigante}\index{Mostri - Iena Gigante}

\emph{Grande bestia, disallineato}

\textbf{FORZA} +3

\textbf{DESTREZZA} +2

\textbf{COSTITUZIONE} +2

\textbf{INTELLIGENZA} -4

\textbf{SAGGEZZA} +1

\textbf{CARISMA} -2

\textbf{Iniziativa} +2 -- \textbf{Difesa} 13

\textbf{Punti Ferita} 45 (6d10 + 12)

\textbf{Movimento} 15 m

\textbf{Tiri Salvezza}: Tempra +6, Riflessi +6, Volontà +2 

\textbf{Competenze} Consapevolezza +3

\textbf{Lingue} -

\textbf{Sfida} 1 (200 PE)

\emph{\textbf{Rabbia.}} Quando la iena riduce una creatura a 0 punti ferita con un attacco di mischia durante il proprio turno, la iena può svolgere un'azione bonus per muoversi fino a metà del suo movimento effettuare un attacco di morso.

\textbf{Azioni}

\emph{\textbf{Morso.} Attacco con Arma da Mischia}: +5 a colpire, portata 1 m, un bersaglio.

\emph{Colpisce:} 10 (2d6 + 3) danni perforanti.

\medskip\textbf{Leone}\index{Mostri - Leone}

\emph{Grande bestia, disallineato}

\textbf{FORZA} +3

\textbf{DESTREZZA} +2

\textbf{COSTITUZIONE} +1

\textbf{INTELLIGENZA} -4

\textbf{SAGGEZZA} +1

\textbf{CARISMA} -1

\textbf{Iniziativa} +2 -- \textbf{Difesa} 13

\textbf{Punti Ferita} 26 (4d10 + 4)

\textbf{Movimento} 15 m

\textbf{Tiri Salvezza}: Tempra +6, Riflessi +7, Volontà +2 

\textbf{Competenze} Muoversi Silenziosamente / Nascondersi +6, Consapevolezza +3

\textbf{Lingue} -

\textbf{Sfida} 1 (200 PE)

\emph{\textbf{Balzo.}} Se il leone si muove di almeno 6 metri diretto verso una creatura e la colpisce con un attacco di artiglio durante lo stesso turno, il bersaglio deve riuscire un Tiro Salvezza di Tempra CD 13 o cadere prono. Se il bersaglio è prono, il leone può effettuare un
attacco di morso come azione bonus.

\emph{\textbf{Olfatto Affinato.}} Il leone ha +1d6 alle prove di Saggezza (Consapevolezza) basate sull'olfatto.

\emph{\textbf{Salto con Rincorsa.}} Con 3 metri di rincorsa, il leone può saltare in lungo fino a 7 metri.

\emph{\textbf{Tattiche di Branco.}} Il leone ha +1d6 ai tiri di attacco contro una creatura se almeno uno degli alleati del leone si trova entro 1 metro dalla creatura e quell'alleato non è inabile.

\textbf{Azioni}

\emph{\textbf{Artiglio.} Attacco con Arma da Mischia}: +5 a colpire, portata 1 m, un bersaglio.

\emph{Colpisce:} 6 (1d6 + 3) danni taglienti. 

\emph{\textbf{Morso.} Attacco con Arma da Mischia}: +5 a colpire, portata 1 m, un bersaglio.

\emph{Colpisce:} 7 (1d8 + 3) danni perforanti.

\medskip\textbf{Lucertola}\index{Mostri - Lucertola}

\emph{Minuscola bestia, disallineato}

\textbf{FORZA} -4

\textbf{DESTREZZA} +0

\textbf{COSTITUZIONE} +0

\textbf{INTELLIGENZA} -5

\textbf{SAGGEZZA} -1

\textbf{CARISMA} -4

\textbf{Iniziativa} +0 -- \textbf{Difesa} 11

\textbf{Punti Ferita} 2 (1d4)

\textbf{Movimento} 6 m, scalata 6 m

\textbf{Tiri Salvezza}:  Tempra +1, Riflessi +4, Volontà +1 

\textbf{Sensi} visione al buio 9 m

\textbf{Lingue} -

\textbf{Sfida} 0 (10 PE)

\emph{\textbf{Scalare come Ragno.}} La lucertola può scalare superfici difficili, compreso lo stare a testa in giù sul soffitto, senza bisogno di effettuare una prova di abilità.

\textbf{Azioni}

\emph{\textbf{Morso.} Attacco con Arma da Mischia}: +0 a colpire, portata 1 m, un bersaglio.

\emph{Colpisce:} 1 danno perforante.

\medskip\textbf{Lucertola Gigante}\index{Mostri - Lucertola Gigante}

Le lucertole giganti sono temibili predatori e spesso vengono impiegate come cavalcature o animali da tiro da umanoidi rettiloidi e residenti del sottosuolo.

\emph{Grande bestia, disallineato}

\textbf{FORZA} +2

\textbf{DESTREZZA} +1

\textbf{COSTITUZIONE} +1

\textbf{INTELLIGENZA} -4

\textbf{SAGGEZZA} +0

\textbf{CARISMA} -3

\textbf{Iniziativa} +1 -- \textbf{Difesa} 13

\textbf{Punti Ferita} 19 (3d10 + 3)

\textbf{Movimento} 9 m, scalata 9 m

\textbf{Tiri Salvezza}: Tempra +11, Riflessi +8, Volontà +4 

\textbf{Sensi} visione al buio 9 m

\textbf{Lingue} -

\textbf{Sfida} 1/4 (50 PE)

\textbf{Azioni}

\emph{\textbf{Morso.} Attacco con Arma da Mischia}: +4 a colpire, portata 1 m, un bersaglio.

\emph{Colpisce:} 6 (1d8 + 2) danni perforanti.

\textbf{VARIANTE}

Alcune lucertole giganti possiedono uno o entrambi i seguenti tratti.

\emph{\textbf{Scalare come Ragno.}} La lucertola può scalare superfici difficili, compreso lo stare a testa in giù sul soffitto, senza bisogno di effettuare una prova di abilità. 

\emph{\textbf{Trattenere il Fiato.}} La lucertola può trattenere il fiato per 15 minuti. (Una lucertola con questo tratto possiede anche velocità di nuoto 9 metri).

\medskip\textbf{Lupo}\index{Mostri - Lupo}

\emph{Media bestia, disallineato}

\textbf{FORZA} +1

\textbf{DESTREZZA} +2

\textbf{COSTITUZIONE} +1

\textbf{INTELLIGENZA} -4

\textbf{SAGGEZZA} +1

\textbf{CARISMA} -2

\textbf{Iniziativa} +2 -- \textbf{Difesa} 14

\textbf{Punti Ferita} 11 (2d8 + 2)

\textbf{Movimento} 12 m

\textbf{Tiri Salvezza}: Tempra +5, Riflessi +5, Volontà +1 

\textbf{Competenze} Muoversi Silenziosamente / Nascondersi +4, Consapevolezza +3

\textbf{Lingue} -

\textbf{Sfida} 1/4 (50 PE)

\emph{\textbf{Udito e Olfatto Affinato.}} Il lupo ha +1d6 nelle prove di Saggezza (Consapevolezza) basate su udito o olfatto.

\emph{\textbf{Tattiche di Branco.}} Il lupo ha +1d6 ai tiri di attacco contro una creatura se almeno uno degli alleati del lupo si trova entro 1 metro dalla creatura e quell'alleato non è inabile.

\textbf{Azioni}

\emph{\textbf{Morso.} Attacco con Arma da Mischia}: +4 a colpire, portata 1 m, un bersaglio.

\emph{Colpisce:} 7 (2d4 + 2) danni perforanti. Se il bersaglio è una creatura, deve riuscire un Tiro Salvezza di Tempra CD 11 o cadere prona.

\medskip\textbf{Dinolupo (Metalupo)}\index{Mostri - Dinolupo (Metalupo}

\emph{Grande bestia, disallineato}

\textbf{FORZA} +3

\textbf{DESTREZZA} +2

\textbf{COSTITUZIONE} +2

\textbf{INTELLIGENZA} -2

\textbf{SAGGEZZA} +1

\textbf{CARISMA} -2

\textbf{Iniziativa} +2 -- \textbf{Difesa} 15

\textbf{Punti Ferita} 37 (5d10 + 10)

\textbf{Movimento} 15 m

\textbf{Tiri Salvezza}: Tempra +7, Riflessi +6, Volontà +2 

\textbf{Competenze} Muoversi Silenziosamente / Nascondersi +4, Consapevolezza +3

\textbf{Lingue} -

\textbf{Sfida} 1 (200 PE)

\emph{\textbf{Udito e Olfatto Affinato.}} Il lupo ha +1d6 nelle prove di Saggezza (Consapevolezza) basate su udito o olfatto.

\emph{\textbf{Tattiche di Branco.}} Il lupo ha +1d6 ai tiri di attacco contro una creatura se almeno uno degli alleati del lupo si trova entro 1 metro dalla creatura e quell'alleato non è inabile.

\textbf{Azioni}

\emph{\textbf{Morso.} Attacco con Arma da Mischia}: +5 a colpire, portata 1 m, un bersaglio.

\emph{Colpisce:} 10 (2d6 + 3) danni perforanti. Se il bersaglio è una creatura, deve riuscire un Tiro Salvezza di Tempra CD 13 o cadere prona.

\medskip\textbf{Lupo Invernale}\index{Mostri - Lupo Invernale}

I lupi invernali abitano nelle regioni artiche e sono creature malvagie e intelligenti dal manto bianco come la neve e gli occhi color del ghiaccio.

\emph{Grande mostruosità, neutrale malvagio}

\textbf{FORZA} +4

\textbf{DESTREZZA} +1

\textbf{COSTITUZIONE} +2

\textbf{INTELLIGENZA} -2

\textbf{SAGGEZZA} +1

\textbf{CARISMA} -1

\textbf{Iniziativa} +1 -- \textbf{Difesa} 15

\textbf{Punti Ferita} 75 (10d10 + 20)

\textbf{Movimento} 15 m

\textbf{Tiri Salvezza}: Tempra +9, Riflessi +6, Volontà +3 

\textbf{Competenze} Muoversi Silenziosamente / Nascondersi +3, Consapevolezza +5

\textbf{Immunità al Danno} freddo

\textbf{Lingue} Comune, Gigante, Lupo Invernale

\textbf{Sfida} 3 (700 PE)

\emph{\textbf{Camuffamento di Neve.}} Il lupo ha +1d6 alle prove di Destrezza (Nascondersi) effettuate per nascondersi su terreno innevato.

\emph{\textbf{Udito e Olfatto Affinato.}} Il lupo ha +1d6 nelle prove di Saggezza (Consapevolezza) basate su udito o olfatto.

\emph{\textbf{Tattiche di Branco.}} Il lupo ha +1d6 ai tiri di attacco contro una creatura se almeno uno degli alleati del lupo si trova entro 1 metro dalla creatura e quell'alleato non è inabile.

\textbf{Azioni}

\emph{\textbf{Morso.} Attacco con Arma da Mischia}: +6 a colpire, portata 1 m, un bersaglio.

\emph{Colpisce:} 11 (2d6 + 4) danni perforanti. Se il bersaglio è una creatura, deve riuscire un Tiro Salvezza di Tempra CD 14 o cadere prona.

\emph{\textbf{Soffio Gelido (Ricarica 5-6).}} Il lupo esala un'esplosione di vento gelido in un cono di 5 metri. Ogni creatura in quell'area deve effettuare un Tiro Salvezza di Riflessi CD 12, e subire 18 (4d8) danni da freddo se fallisce il Tiro Salvezza, o la metà di questi danni se lo riesce.

\medskip\textbf{Mammut}\index{Mostri - Mammut}

Il mammut è una creatura simile all'elefante dalla folta pelliccia e lunghe zanne.

\emph{Enorme bestia, disallineato}

\textbf{FORZA} +7

\textbf{DESTREZZA} -1

\textbf{COSTITUZIONE} +5

\textbf{INTELLIGENZA} -4

\textbf{SAGGEZZA} +0

\textbf{CARISMA} -2

\textbf{Iniziativa} -1 -- \textbf{Difesa} 16

\textbf{Punti Ferita} 126 (11d12 + 55)

\textbf{Movimento} 12 m

\textbf{Tiri Salvezza}: Tempra +14, Riflessi +10, Volontà +7 

\textbf{Lingue} -

\textbf{Sfida} 6 (2.300 PE)

\emph{\textbf{Carica Travolgente.}} Se il mammut si muove di almeno 6 metri diretto verso una creatura e la colpisce con un attacco di incornata durante lo stesso turno, il bersaglio deve riuscire un Tiro Salvezza su Tempra CD 18 o cadere prono. Se il bersaglio è prono, il mammut può effettuare un attacco di pestone contro di lui come azione bonus.

\textbf{Azioni}

\emph{\textbf{Incornata.} Attacco con Arma da Mischia}: +10 a colpire, portata 3 m, un bersaglio.

\emph{Colpisce:} 25 (4d8 + 7) danni perforanti. 

\emph{\textbf{Pestone.} Attacco con Arma da Mischia}: +10 a colpire, portata 1 m, una creatura prona.

\emph{Colpisce:} 29 (4d10 + 7) danni da botta.

\medskip\textbf{Mastino}\index{Mostri - Mastino}

\textbf{I} mastini sono impressionanti segugi apprezzati dagli umanoidi per la loro realtà e sensi affinati.

\emph{Media bestia, disallineato}

\textbf{FORZA} +1

\textbf{DESTREZZA} +2

\textbf{COSTITUZIONE} +1

\textbf{INTELLIGENZA} -4

\textbf{SAGGEZZA} +1

\textbf{CARISMA} -2

\textbf{Iniziativa} +2 -- \textbf{Difesa} 13

\textbf{Punti Ferita} 5 (1d8 + 1)

\textbf{Movimento} 12 m

\textbf{Tiri Salvezza}: Tempra +3, Riflessi +3, Volontà +1 

\textbf{Competenze} Consapevolezza +3, Sopravvivenza (Seguire Tracce) +3

\textbf{Lingue} -

\textbf{Sfida} 1/8 (25 PE)

\emph{\textbf{Udito e Olfatto Affinato.}} Il mastino ha +1d6 nelle prove di Saggezza (Consapevolezza) basate su udito o olfatto.

\textbf{Azioni}

\emph{\textbf{Morso.} Attacco con Arma da Mischia}: +3 a colpire, portata 1 m, un bersaglio.

\emph{Colpisce:} 4 (1d6 + 1) danni perforanti. Se il bersaglio è una creatura, deve riuscire un Tiro Salvezza di Tempra CD 11 o cadere prono.

\medskip\textbf{Mulo}\index{Mostri - Mulo}

\emph{Media bestia, disallineato}

\textbf{FORZA} +2

\textbf{DESTREZZA} +0

\textbf{COSTITUZIONE} +1

\textbf{INTELLIGENZA} -4

\textbf{SAGGEZZA} +0

\textbf{CARISMA} -3

\textbf{Iniziativa} +0 -- \textbf{Difesa} 11

\textbf{Punti Ferita} 11 (2d8 + 2)

\textbf{Movimento} 12 m

\textbf{Tiri Salvezza}: Tempra +3, Riflessi +1, Volontà +1 

\textbf{Lingue} -

\textbf{Sfida} 1/8 (25 PE)

\emph{\textbf{Bestia da Soma.}} Il mulo è considerato un animale Grande al fine di determinare la sua capacità di carico.

\emph{\textbf{Piedi Saldi.}} Il mulo ha +1d6 ai Tiri Salvezza su Tempra e Riflessi effettuati contro effetti che lo farebbero cadere prono.

\textbf{Azioni}

\emph{\textbf{Zoccoli.} Attacco con Arma da Mischia}: +2 a colpire, portata 1 m, un bersaglio.

\emph{Colpisce:} 4 (1d4 + 2) danni da botta.

\medskip\textbf{Orso Bruno}\index{Mostri - Orso Bruno}

\emph{Grande bestia, disallineato}

\textbf{FORZA} +4

\textbf{DESTREZZA} +0

\textbf{COSTITUZIONE} +3

\textbf{INTELLIGENZA} -4

\textbf{SAGGEZZA} +1

\textbf{CARISMA} -2

\textbf{Iniziativa} +0 -- \textbf{Difesa} 12

\textbf{Punti Ferita} 34 (4d10 + 12)

\textbf{Movimento} 12 m, scalata 9 m

\textbf{Tiri Salvezza}: Tempra +6, Riflessi +2, Volontà +3 

\textbf{Competenze} Consapevolezza +3

\textbf{Lingue} -

\textbf{Sfida} 1 (200 PE)

\emph{\textbf{Olfatto Affinato.}} L'orso ha +1d6 alle prove di Saggezza (Consapevolezza) basate sull'olfatto.

\textbf{Azioni}

\emph{\textbf{Multiattacco.}} L'orso effettua due attacchi: uno con il morso e uno con gli artigli.

\emph{\textbf{Artigli.} Attacco con Arma da Mischia}: +5 a colpire, portata 1 m, un bersaglio.

\emph{Colpisce:} 11 (2d6 + 4) danni taglienti.

\emph{\textbf{Morso.} Attacco con Arma da Mischia}: +5 a colpire, portata 1 m, un bersaglio.

\emph{Colpisce:} 8 (1d8 + 4) danni perforanti.

\medskip\textbf{Orso Nero}\index{Mostri - Orso Nero}

\emph{Media bestia, disallineato}

\textbf{FORZA} +2

\textbf{DESTREZZA} +0

\textbf{COSTITUZIONE} +2

\textbf{INTELLIGENZA} -4

\textbf{SAGGEZZA} +1

\textbf{CARISMA} -2

\textbf{Iniziativa} +0 -- \textbf{Difesa} 12

\textbf{Punti Ferita} 19 (3d8 + 6)

\textbf{Movimento} 12 m, scalata 9 m

\textbf{Tiri Salvezza}: Tempra +4, Riflessi +1, Volontà +1 

\textbf{Competenze} Consapevolezza +3

\textbf{Lingue} -

\textbf{Sfida} 1/2 (100 PE)

\emph{\textbf{Olfatto Affinato.}} L'orso ha +1d6 alle prove di Saggezza (Consapevolezza) basate sull'olfatto.

\textbf{Azioni}

\emph{\textbf{Multiattacco.}} L'orso nero effettua due attacchi: uno con il morso e uno con gli artigli.

\emph{\textbf{Artigli.} Attacco con Arma da Mischia}: +3 a colpire, portata 1 m, un bersaglio.

\emph{Colpisce:} 7 (2d4 + 2) danni taglienti.

\emph{\textbf{Morso.} Attacco con Arma da Mischia}: +3 a colpire, portata 1 m, un bersaglio.

\emph{Colpisce:} 5 (1d6 + 2) danni perforanti.

\medskip\textbf{Orso Polare}\index{Mostri - Orso Polare}

\emph{Grande bestia, disallineato}

\textbf{FORZA} +5

\textbf{DESTREZZA} +0

\textbf{COSTITUZIONE} +3

\textbf{INTELLIGENZA} -4

\textbf{SAGGEZZA} +1

\textbf{CARISMA} -2

\textbf{Iniziativa} +0 -- \textbf{Difesa} 13

\textbf{Punti Ferita} 42 (5d10 + 15)

\textbf{Movimento} 12 m, nuoto 9 m

\textbf{Tiri Salvezza}: Tempra +10, Riflessi +7, Volontà +4 

\textbf{Competenze} Consapevolezza +3

\textbf{Lingue} -

\textbf{Sfida} 2 (450 PE)

\emph{\textbf{Olfatto Affinato.}} L'orso ha +1d6 alle prove di Saggezza (Consapevolezza) basate sull'olfatto.

\textbf{Azioni}

\emph{\textbf{Multiattacco.}} L'orso effettua due attacchi: uno con il morso e uno con gli artigli.

\emph{\textbf{Artigli.} Attacco con Arma da Mischia}: +7 a colpire, portata 1 m, un bersaglio.

\emph{Colpisce:} 12 (2d6 + 5) danni taglienti.

\emph{\textbf{Morso.} Attacco con Arma da Mischia}: +7 a colpire, portata 1 m, un bersaglio.

\emph{Colpisce:} 9 (1d8 + 5) danni perforanti.

\textbf{VARIANTE: ORSO DELLE CAVERNE}\index{Mostri - Orso delle Caverne}

Alcuni orsi si sono adattati alla vita sottoterra. Costoro hanno le stesse statistiche degli orsi polari, ma con visione al buio 18 m.

\medskip\textbf{Pantera}\index{Mostri - Pantera}

\emph{Media bestia, disallineato}

\textbf{FORZA} +2

\textbf{DESTREZZA} +2

\textbf{COSTITUZIONE} +0

\textbf{INTELLIGENZA} -4

\textbf{SAGGEZZA} +2

\textbf{CARISMA} -2

\textbf{Iniziativa} +2 -- \textbf{Difesa} 13

\textbf{Punti Ferita} 13 (3d8)

\textbf{Movimento} 15 m, scalata 12 m

\textbf{Tiri Salvezza}: Tempra +3, Riflessi +5, Volontà +3 

\textbf{Competenze} Muoversi Silenziosamente / Nascondersi +6, Consapevolezza +4

\textbf{Lingue} -

\textbf{Sfida} 1/4 (50 PE)

\emph{\textbf{Balzo.}} Se la pantera si muove di almeno 6 metri diretta verso una creatura e la colpisce con un attacco di artiglio durante lo stesso turno, il bersaglio deve riuscire un Tiro Salvezza di Tempra CD 12 o cadere prono. Se il bersaglio è prono, la pantera può effettuare un  attacco di morso contro di esso come azione bonus.

\emph{\textbf{Olfatto Affinato.}} La pantera ha +1d6 alle prove di Saggezza (Consapevolezza) basate sull'olfatto.

\textbf{Azioni}

\emph{\textbf{Artiglio.} Attacco con Arma da Mischia}: +4 a colpire, portata 1 m, un bersaglio.

\emph{Colpisce:} 4 (1d4 + 2) danni taglienti.

\emph{\textbf{Morso.} Attacco con Arma da Mischia}: +4 a colpire, portata 1 m, un bersaglio.

\emph{Colpisce:} 5 (1d6 + 2) danni perforanti.


\medskip\textbf{Pony}\index{Mostri - Pony}

\emph{Media bestia, disallineato}

\textbf{FORZA} +2

\textbf{DESTREZZA} +0

\textbf{COSTITUZIONE} +1

\textbf{INTELLIGENZA} -4

\textbf{SAGGEZZA} +0

\textbf{CARISMA} -2

\textbf{Iniziativa} +0 -- \textbf{Difesa} 11

\textbf{Punti Ferita} 11 (2d8 + 2)

\textbf{Movimento} 12 m

\textbf{Tiri Salvezza}: Tempra +5, Riflessi +4, Volontà +0

\textbf{Lingue} -

\textbf{Sfida} 1/8 (25 PE)

\textbf{Azioni}

\emph{\textbf{Zoccoli.} Attacco con Arma da Mischia}: +4 a colpire, portata 1 m, un bersaglio.

\emph{Colpisce:} 7 (2d4 + 2) danni da botta.

\medskip\textbf{Ragno}\index{Mostri - Ragno}

\emph{Minuscola bestia, disallineato}

\textbf{FORZA} 2 (-5)

\textbf{DESTREZZA} +2

\textbf{COSTITUZIONE} -1

\textbf{INTELLIGENZA} -5

\textbf{SAGGEZZA} +0

\textbf{CARISMA} -4

\textbf{Iniziativa} +2 -- \textbf{Difesa} 13

\textbf{Punti Ferita} 1 (1d4 - 1)

\textbf{Movimento} 6 m, scalata 6 m

\textbf{Tiri Salvezza}: Tempra -4, Riflessi +2, Volontà -4 

\textbf{Competenze} Muoversi Silenziosamente / Nascondersi +4

\textbf{Sensi} visione al buio 9 m

\textbf{Lingue} -

\textbf{Sfida} 0 (10 PE)

\emph{\textbf{Camminare sulla Tela.}} Il ragno ignora le restrizioni al movimento provocate dalle ragnatele.

\emph{\textbf{Scalare come Ragno.}} Il ragno può scalare superfici difficili, compreso lo stare a testa in giù sul soffitto, senza bisogno  di effettuare una prova di abilità.

\emph{\textbf{Senso della Tela.}} Mentre è in contatto con una ragnatela, il ragno sa l'esatta posizione di qualsiasi altra creatura in contatto con la stessa ragnatela.

\textbf{Azioni}

\emph{\textbf{Morso.} Attacco con Arma da Mischia}: +4 a colpire, portata 1 m, una creatura.

\emph{Colpisce:} 1 danno perforante e il bersaglio deve riuscire un Tiro Salvezza su Tempra 9 o subire 2 (1d4) danni da veleno.

\medskip\textbf{Ragno Fase}\index{Mostri - Ragno Fase}

Il ragno fase possiede l'abilità magica di entrare ed uscire dal Piano Etereo. Sembra apparire dal nulla e scompare rapidamente dopo aver attaccato.

\emph{Grande mostruosità, disallineato}

\textbf{FORZA} +2

\textbf{DESTREZZA} +2

\textbf{COSTITUZIONE} +1

\textbf{INTELLIGENZA} -2

\textbf{SAGGEZZA} +0

\textbf{CARISMA} -2

\textbf{Iniziativa} +2 -- \textbf{Difesa} 15

\textbf{Punti Ferita} 32 (5d10 + 5)

\textbf{Movimento} 9 m, scalata 9 m

\textbf{Tiri Salvezza}: Tempra +8, Riflessi +8, Volontà +3 

\textbf{Competenze} Muoversi Silenziosamente / Nascondersi +6

\textbf{Sensi} visione al buio 18 m

\textbf{Lingue} -

\textbf{Sfida} 3 (700 PE)

\emph{\textbf{Camminare sulla Tela.}} Il ragno ignora le restrizioni al movimento provocate dalle ragnatele.

\emph{\textbf{Gita Eterea.}} Come azione bonus, il ragno può magicamente spostarsi dal Piano Materiale al Piano Etereo, o viceversa.

\emph{\textbf{Scalare come Ragno.}} Il ragno può scalare superfici difficili, compreso lo stare a testa in giù sul soffitto, senza bisogno di effettuare una prova di abilità.

\textbf{Azioni}

\emph{\textbf{Morso.} Attacco con Arma da Mischia}: +4 a colpire, portata 1 m, una creatura.

\emph{Colpisce:} 7 (1d10 + 2) danni perforanti e il bersaglio deve effettuare un Tiro Salvezza di Tempra CD 11, e subire 18 (4d8) danni da veleno se fallisce il Tiro Salvezza, o la metà di questo danno se lo riesce. Se il danno da veleno riduce il bersaglio a 0 punti ferita, il bersaglio è stabile ma avvelenato per 1 ora, anche dopo aver recuperato i punti ferita, e mentre è avvelenato in questo modo resta paralizzato.

\medskip\textbf{Ragno Gigante}\index{Mostri - Ragno Gigante}

\emph{Grande bestia, disallineato}

\textbf{FORZA} +2

\textbf{DESTREZZA} +3

\textbf{COSTITUZIONE} +1

\textbf{INTELLIGENZA} -4

\textbf{SAGGEZZA} +0

\textbf{CARISMA} -3

\textbf{Iniziativa} +2 -- \textbf{Difesa} 15

\textbf{Punti Ferita} 26 (4d10 + 4)

\textbf{Movimento} 9 m, scalata 9 m

\textbf{Tiri Salvezza}:  Tempra +4, Riflessi +4, Volontà +1 

\textbf{Competenze} Muoversi Silenziosamente / Nascondersi +7

\textbf{Sensi} vista cieca 3 m, visione al buio 18 m

\textbf{Lingue} -

\textbf{Sfida} 1 (200 PE)

\emph{\textbf{Camminare sulla Tela.}} Il ragno ignora le restrizioni al movimento provocate dalle ragnatele.

\emph{\textbf{Scalare come Ragno.}} Il ragno può scalare superfici difficili, compreso lo stare a testa in giù sul soffitto, senza bisogno di effettuare una prova di abilità.

\emph{\textbf{Senso della Tela.}} Mentre è in contatto con una ragnatela, il ragno sa l'esatta posizione di qualsiasi altra creatura in contatto con la stessa ragnatela. 

\textbf{Azioni}

\emph{\textbf{Morso.} Attacco con Arma da Mischia}: +5 a colpire, portata 1 m, una creatura.

\emph{Colpisce:} 7 (1d8 + 3) danni perforanti e il bersaglio deve effettuare un Tiro Salvezza di Tempra CD 11, e subire 9

(2d8) danni da veleno se fallisce il Tiro Salvezza, o la metà di questi danni se lo riesce. Se il danno da veleno riduce il bersaglio a 0 punti ferita, il bersaglio è stabile ma avvelenato per 1 ora, anche dopo aver recuperato i punti ferita, e mentre è avvelenato in questo modo resta paralizzato.

\emph{\textbf{Ragnatela (Ricarica 5-6).} Attacco con Arma a Gittata}: +5 a colpire, gittata 9m, una creatura.

\emph{Colpisce:} Il bersaglio è intralciato dalla ragnatela. Con un'azione, il bersaglio intralciato può effettuare una prova di Forza CD 12 e, in caso di successo, spezzare la tela. La ragnatela può essere anche attaccata e distrutta (CA 10; pf 5; vulnerabilità al danno da fuoco; immunità ai danni da botta, psichici e da veleno). 

\medskip\textbf{Ragno Lupo Gigante}\index{Mostri - Ragno Lupo Gigante}

Un ragno lupo gigante caccia le prede su terreno aperto o si nasconde in tane o fessure del terreno per tendere imboscate.

\emph{Media bestia, disallineato}

\textbf{FORZA} +1

\textbf{DESTREZZA} +3

\textbf{COSTITUZIONE} +1

\textbf{INTELLIGENZA} -4

\textbf{SAGGEZZA} +1

\textbf{CARISMA} -3

\textbf{Iniziativa} +3 -- \textbf{Difesa} 14

\textbf{Punti Ferita} 11 (2d8 + 2)

\textbf{Movimento} 12 m, scalata 12 m

\textbf{Tiri Salvezza}:  Tempra +2, Riflessi +4, Volontà +1 

\textbf{Competenze} Muoversi Silenziosamente / Nascondersi +7, Consapevolezza +3

\textbf{Sensi} vista cieca 3 m, visione al buio 18 m

\textbf{Lingue} -

\textbf{Sfida} 1/4 (50 PE)

\emph{\textbf{Camminare sulla Tela.}} Il ragno ignora le restrizioni al movimento provocate dalle ragnatele.

\emph{\textbf{Scalare come Ragno.}} Il ragno può scalare superfici difficili, compreso lo stare a testa in giù sul soffitto, senza bisogno di effettuare una prova di abilità.

\emph{\textbf{Senso della Tela.}} Mentre è in contatto con una ragnatela, il ragno sa l'esatta posizione di qualsiasi altra creatura in contatto con la stessa ragnatela.

\textbf{Azioni}

\emph{\textbf{Morso.} Attacco con Arma da Mischia}: +3 a colpire, portata 1 m, una creatura.

\emph{Colpisce:} 4 (1d6 + 1) danni perforanti e il bersaglio deve effettuare un Tiro Salvezza di Tempra CD 11, e subire 7 (2d6) danni da veleno se fallisce il Tiro Salvezza, o la metà di questi danni se lo riesce. Se il danno da veleno riduce il bersaglio a 0 punti ferita, il bersaglio è stabile ma avvelenato per 1 ora, anche dopo aver recuperato i punti ferita, e mentre è avvelenato in questo modo resta paralizzato.

\medskip\textbf{Rana}\index{Mostri - Rana}

\emph{Minuscola bestia, disallineato}

\textbf{FORZA} -5

\textbf{DESTREZZA} +1

\textbf{COSTITUZIONE} -1

\textbf{INTELLIGENZA} -5

\textbf{SAGGEZZA} -1

\textbf{CARISMA} -4

\textbf{Iniziativa} +1 -- \textbf{Difesa} 12

\textbf{Punti Ferita} 1 (1d4 - 1)

\textbf{Movimento} 6 m, nuoto 6 m

\textbf{Tiri Salvezza}:  Tempra -4, Riflessi +1, Volontà -2

\textbf{Competenze} Muoversi Silenziosamente / Nascondersi +3, Consapevolezza +1

\textbf{Sensi} visione al buio 9 m

\textbf{Lingue} -

\textbf{Sfida} 0 (0 PE)

\emph{\textbf{Anfibio.}} La rana può respirare aria e acqua.

\emph{\textbf{Salto da Fermo.}} Una rana può saltare in lungo fino a 3 metri e in alto fino a 1 metro, con o senza la rincorsa.

Una \textbf{rana} è sprovvista di attacchi. Si nutre di piccoli insetti e di solito vive in prossimità di acquitrini, dentro gli alberi o sottoterra.

\medskip\textbf{Rana Gigante}\index{Mostri - Rana Gigante}

\emph{Media bestia, disallineato}

\textbf{FORZA} +1

\textbf{DESTREZZA} +1

\textbf{COSTITUZIONE} +0

\textbf{INTELLIGENZA} -4

\textbf{SAGGEZZA} +0

\textbf{CARISMA} -4

\textbf{Iniziativa} +1 -- \textbf{Difesa} 12

\textbf{Punti Ferita} 18 (4d8)

\textbf{Movimento} 9 m, nuoto 9 m

\textbf{Tiri Salvezza}: Tempra +2, Riflessi +2, Volontà +0 

\textbf{Competenze} Muoversi Silenziosamente / Nascondersi +3, Consapevolezza +2

\textbf{Sensi} visione al buio 9 m

\textbf{Lingue} -

\textbf{Sfida} 1/4 (50 PE)

\emph{\textbf{Anfibio.}} La rana può respirare aria e acqua.

\emph{\textbf{Salto da Fermo.}} Una rana può saltare in lungo fino a 6 metri e in alto fino a 3 metri, con o senza la rincorsa.

\textbf{Azioni}

\emph{\textbf{Morso.} Attacco con Arma da Mischia}: +3 a colpire, portata 1 m, un bersaglio.

\emph{Colpisce:} 4 (1d6 + 1) danni perforanti e il bersaglio è afferrato (CD 11 per fuggire). Fino al termine dell'afferrare, il bersaglio è intralciato, e la rana non può usare il morso contro un altro bersaglio.

\emph{\textbf{Inghiottire.}} La rana effettua una attacco di morso contro un bersaglio di taglia Piccola o inferiore che sta afferrando. Se l'attacco colpisce, il bersaglio è inghiottito, e l'afferrare ha termine. Il bersaglio inghiottito è accecato e intralciato, ha copertura totale contro gli attacchi e altri effetti all'esterno della rana, e subisce 5 (2d4) danni da acido all'inizio di ciascun turno della rana. La rana può inghiottire solo un bersaglio alla volta. Se la rana muore, una creatura inghiottita non è più intralciata da essa e può uscire dal cadavere utilizzando 1 metro di movimento, uscendo prona.

\medskip\textbf{Ratto}\index{Mostri - Ratto}

\emph{Minuscola bestia, disallineato}

\textbf{FORZA} -4

\textbf{DESTREZZA} +0

\textbf{COSTITUZIONE} -1

\textbf{INTELLIGENZA} -4

\textbf{SAGGEZZA} +0

\textbf{CARISMA} -3

\textbf{Iniziativa} +0 -- \textbf{Difesa} 11

\textbf{Punti Ferita} 1 (1d4 - 1)

\textbf{Movimento} 6 m

\textbf{Tiri Salvezza}: Tempra -4, Riflessi +0, Volontà +0 

\textbf{Sensi} visione al buio 9 m

\textbf{Lingue} -

\textbf{Sfida} 0 (10 PE)

\emph{\textbf{Olfatto Affinato.}} Il ratto ha +1d6 alle prove di Saggezza (Consapevolezza) basate sull'olfatto.

\textbf{Azioni}

\emph{\textbf{Morso.} Attacco con Arma da Mischia}: +0 a colpire, portata 1 m, un bersaglio.

\emph{Colpisce:} 1 danno perforante.

\medskip\textbf{Ratto Gigante}\index{Mostri - Ratto Gigante}

\emph{Piccola bestia, disallineato}

\textbf{FORZA} -2

\textbf{DESTREZZA} +2

\textbf{COSTITUZIONE} +0

\textbf{INTELLIGENZA} -4

\textbf{SAGGEZZA} +0

\textbf{CARISMA} -3

\textbf{Iniziativa} +2 -- \textbf{Difesa} 13

\textbf{Punti Ferita} 7 (2d6)

\textbf{Movimento} 9 m

\textbf{Tiri Salvezza}: Tempra +3, Riflessi +5, Volontà +1 

\textbf{Sensi} visione al buio 18 m

\textbf{Lingue} -

\textbf{Sfida} 1/8 (25 PE)

\emph{\textbf{Olfatto Affinato.}} Il ratto ha +1d6 alle prove di Saggezza (Consapevolezza) basate sull'olfatto.

\emph{\textbf{Tattiche di Branco.}} Il ratto ha +1d6 al tiro di attacco contro una creatura se almeno uno degli alleati del ratto si trova entro 1 metro dalla creatura e quell'alleato non è inabile.

\textbf{Azioni}

\emph{\textbf{Morso.} Attacco con Arma da Mischia}: +4 a colpire, portata 1 m, un bersaglio.

\emph{Colpisce:} 4 (1d4 + 2) danni perforanti.

\textbf{VARIANTE: RATTO GIGANTE AMMALATO}\index{Mostri - Ratto Gigante ammalato}

Alcuni ratti giganti recano una terribile malattia che diffondono tramite il morso. Un ratto gigante ammalato ha grado di sfida 1/8 (25 PE) e la seguente azione invece del suo normale attacco di morso.

\emph{\textbf{Morso.} Attacco con Arma da Mischia}: +4 a colpire, portata 1 m, un bersaglio.

\emph{Colpisce:} 4 (1d4 + 2) danni perforanti. Se il bersaglio è una creatura, deve riuscire un Tiro Salvezza di Tempra CD 10 o contrarre una malattia. Fino a che la malattia non viene curata, il bersaglio non può recuperare punti ferita eccetto tramite metodi magici, e i punti ferita massimi del bersaglio diminuiscono di 3 (1d6) ogni 24 ore. Se i punti ferita massimi del bersaglio scendono a 0 come risultato della malattia, il bersaglio muore.

\medskip\textbf{Rinoceronte}\index{Mostri - Rinoceronte}

\emph{Grande bestia, disallineato}

\textbf{FORZA} +5

\textbf{DESTREZZA} -1

\textbf{COSTITUZIONE} +2

\textbf{INTELLIGENZA} -4

\textbf{SAGGEZZA} +1

\textbf{CARISMA} -2

\textbf{Iniziativa} -1 -- \textbf{Difesa} 12

\textbf{Punti Ferita} 45 (6d10 + 12)

\textbf{Movimento} 12 m

\textbf{Tiri Salvezza}: Tempra +10, Riflessi +4, Volontà +2

\textbf{Lingue} -

\textbf{Sfida} 2 (450 PE)

\emph{\textbf{Carica.}} Se il rinoceronte si muove di almeno 6 metri diretto verso un bersaglio e lo colpisce con un attacco di incornata durante lo stesso turno, il bersaglio subisce 9 (2d8) danni da botta aggiuntivi. Se il bersaglio è una creatura, deve riuscire un Tiro Salvezza su Tempra CD 15 o cadere prono.

\textbf{Azioni}

\emph{\textbf{Incornata.} Attacco con Arma da Mischia}: +7 a colpire, portata 1 m, un bersaglio.

\emph{Colpisce:} 14 (2d8 + 5) danni da botta.

\medskip\textbf{Rospo Gigante}\index{Mostri - Rospo Gigante}

\emph{Grande bestia, disallineato}

\textbf{FORZA} +2

\textbf{DESTREZZA} +1

\textbf{COSTITUZIONE} +1

\textbf{INTELLIGENZA} -4

\textbf{SAGGEZZA} +0

\textbf{CARISMA} -4

\textbf{Iniziativa} +1 -- \textbf{Difesa} 12

\textbf{Punti Ferita} 39 (6d10 + 6)

\textbf{Movimento} 6 m, nuoto 12 m

\textbf{Tiri Salvezza}: Tempra +6, Riflessi +6, Volontà +0

\textbf{Sensi} visione al buio 9 m

\textbf{Lingue} -

\textbf{Sfida} 1 (200 PE)

\emph{\textbf{Anfibio.}} Il rospo può respirare aria e acqua.

\emph{\textbf{Salto da Fermo.}} Un rospo può saltare in lungo fino a 6 metri e in alto fino a 3 metri, con o senza la rincorsa.

\textbf{Azioni}

\emph{\textbf{Morso.} Attacco con Arma da Mischia}: +4 a colpire, portata 1 m, un bersaglio.

\emph{Colpisce:} 7 (1d10 + 2) danni perforanti più 5 (1d10) danni da veleno, e il bersaglio è afferrato (CD 13 per fuggire). Fino al termine dell'afferrare, il bersaglio è intralciato, e il rospo non può usare il morso contro un altro bersaglio.

\emph{\textbf{Inghiottire.}} Il rospo effettua una attacco di morso contro un bersaglio di taglia Media o inferiore che sta afferrando. Se l'attacco colpisce, il bersaglio è inghiottito, e l'afferrare ha termine. Il bersaglio inghiottito è accecato e intralciato, ha copertura totale contro gli attacchi e altri effetti all'esterno della rana, e subisce 10 (3d6) danni da acido all'inizio di ciascun turno del rospo. Il rospo può inghiottire solo un bersaglio alla volta.

Se il rospo muore, una creatura inghiottita non è più intralciata da esso e può uscire dal cadavere utilizzando 1 metro di movimento, uscendo prono.

\medskip\textbf{Scarabeo di Fuoco Gigante}\index{Mostri - Scarabeo di Fuoco Gigante}

Uno scarabeo di fuoco gigante è una creatura notturna che possiede una coppia di ghiandole luminose capaci di emettere luce per 1d6 giorni dopo la morte dello scarabeo.

\emph{Piccola bestia, disallineato}

\textbf{FORZA} -1

\textbf{DESTREZZA} +0

\textbf{COSTITUZIONE} +1

\textbf{INTELLIGENZA} -5

\textbf{SAGGEZZA} -2

\textbf{CARISMA} -4

\textbf{Iniziativa} +0 -- \textbf{Difesa} 14

\textbf{Punti Ferita} 4 (1d6 + 1)

\textbf{Movimento} 9 m

\textbf{Tiri Salvezza}: Tempra +2, Riflessi +0, Volontà +0

\textbf{Sensi} vista cieca 9 m

\textbf{Lingue} -

\textbf{Sfida} 0 (10 PE)

\emph{\textbf{Illuminazione.}} Lo scarabeo irradia luce intensa in un raggio di 3 metri e luce fioca per ulteriori 3 metri.

\textbf{Azioni}

\emph{\textbf{Morso.} Attacco con Arma da Mischia}: +1 a colpire, portata 1 m, un bersaglio.

\emph{Colpisce:} 2 (1d6 - 1) danni taglienti.

\medskip\textbf{Sciacallo}\index{Mostri - Sciacallo}

\emph{Piccola bestia, disallineato}

\textbf{FORZA} -1

\textbf{DESTREZZA} +2

\textbf{COSTITUZIONE} +0

\textbf{INTELLIGENZA} -4

\textbf{SAGGEZZA} +1

\textbf{CARISMA} -2

\textbf{Iniziativa} +2 -- \textbf{Difesa} 13

\textbf{Punti Ferita} 3 (1d6)

\textbf{Movimento} 12 m

\textbf{Tiri Salvezza}: Tempra -1, Riflessi +3, Volontà +1

\textbf{Competenze} Consapevolezza +3

\textbf{Lingue} -

\textbf{Sfida} 0 (10 PE)

\emph{\textbf{Tattiche di Branco.}} Lo sciacallo ha +1d6 ai tiri di attacco contro una creatura se almeno uno degli alleati dello sciacallo si trova entro 1 metro dalla creatura e quell'alleato non è inabile.

\emph{\textbf{Udito e Olfatto Affinato.}} Lo sciacallo ha +1d6 nelle prove di Saggezza (Consapevolezza) basate su udito o olfatto.

\textbf{Azioni}

\emph{\textbf{Morso.} Attacco con Arma da Mischia}: +1 a colpire, portata 1 m, un bersaglio.

\emph{Colpisce:} 1 (1d4 - 1) danni perforanti.

\medskip\textbf{Sciami}\index{Mostri - Sciami}

Gli sciami presentati qui di seguito non sono dei normali o benigni raduni di piccole creature. Si formano invece come risultato di un'influenza esterna, spesso maligna. Anche i druidi non sono in grado di affascinare questi sciami, e la loro aggressività è quasi innaturale. 

\textbf{Sciame di Centopiedi}\index{Mostri - Sciame di Centopiedi}

\emph{Medio sciame di Minuscole bestie, disallineato}

\textbf{FORZA} -4

\textbf{DESTREZZA} +1

\textbf{COSTITUZIONE} +0

\textbf{INTELLIGENZA} -5

\textbf{SAGGEZZA} -2

\textbf{CARISMA} -5

\textbf{Iniziativa} +1 -- \textbf{Difesa} 13

\textbf{Punti Ferita} 22 (5d8)

\textbf{Movimento} 6 m, scalata 6 m

\textbf{Tiri Salvezza}: Tempra -1, Riflessi +3, Volontà +1

\textbf{Resistenze al Danno} da botta, perforante, tagliente

\textbf{Immunità alle Condizioni} affascinato, afferrato, intralciato,

paralizzato, pietrificato, prono, spaventato, stordito

\textbf{Sensi} vista cieca 3 m 

\textbf{Lingue} -

\textbf{Sfida} 1/2 (100 PE)

\emph{\textbf{Sciame.}} Lo sciame può occupare lo spazio di un'altra creatura e viceversa, e lo sciame può muoversi attraverso qualsiasi apertura grande abbastanza per un Minuscolo insetto. Lo sciame non può recuperare punti ferita né ottenere punti ferita temporanei.

\textbf{Azioni}

\emph{\textbf{Morsi.} Attacco con Arma da Mischia}: +3 a colpire, portata 0 m, un bersaglio nello spazio dello sciame. 

\emph{Colpisce:} 10 (4d4) danni perforanti, o 5 (2d4) danni perforanti se lo sciame è ha metà o meno dei suoi punti ferita. Una creatura ridotta a 0 punti ferita da uno sciame di centopiedi e stabile resta avvelenata per 1 ora, anche dopo aver recuperato i punti ferita, e rimane paralizzata dal veleno durante questo periodo.

\medskip\textbf{Sciame di Corvi}\index{Mostri - Sciame di Corvi}

\emph{Medio sciame di Minuscole bestie, disallineato}

\textbf{FORZA} -2

\textbf{DESTREZZA} +2

\textbf{COSTITUZIONE} -1

\textbf{INTELLIGENZA} -4

\textbf{SAGGEZZA} +1

\textbf{CARISMA} -2

\textbf{Iniziativa} +2 -- \textbf{Difesa} 13

\textbf{Punti Ferita} 24 (7d8 -- 7)

\textbf{Movimento} 3 m, volo 15 m

\textbf{Tiri Salvezza}: Tempra -1, Riflessi +3, Volontà +2

\textbf{Competenze} Consapevolezza +5

\textbf{Resistenze al Danno} da botta, perforante, tagliente \textbf{Immunità alle Condizioni} affascinato, afferrato, intralciato, paralizzato, pietrificato, prono, spaventato, stordito

\textbf{Lingue} -

\textbf{Sfida} 1/4 (50 PE)

\emph{\textbf{Sciame.}} Lo sciame può occupare lo spazio di un'altra creatura e viceversa, e lo sciame può muoversi attraverso qualsiasi apertura grande abbastanza per un Minuscolo corvo. Lo sciame non può recuperare punti ferita né ottenere punti ferita temporanei.

\textbf{Azioni}

\emph{\textbf{Becchi.} Attacco con Arma da Mischia}: +4 a colpire, portata 1 m, un bersaglio nello spazio dello sciame.

\emph{Colpisce:} 7 (2d6) danni perforanti, o 3 (1d6) danni perforanti se lo sciame è ha metà o meno dei suoi punti ferita.

\medskip\textbf{Sciame di Pirana}\index{Mostri - Sciame di Pirana}

\emph{Medio sciame di Minuscole bestie, disallineato}

\textbf{FORZA} +1

\textbf{DESTREZZA} +3

\textbf{COSTITUZIONE} -1

\textbf{INTELLIGENZA} -5

\textbf{SAGGEZZA} -2

\textbf{CARISMA} -4

\textbf{Iniziativa} +3 -- \textbf{Difesa} 14

\textbf{Punti Ferita} 28 (8d8 -- 8)

\textbf{Movimento} 0 m, nuoto 12 m

\textbf{Tiri Salvezza}: Tempra -3, Riflessi +4, Volontà -1

\textbf{Resistenze al Danno} da botta, perforante, tagliente

\textbf{Immunità alle Condizioni} affascinato, afferrato, intralciato, paralizzato, pietrificato, prono, spaventato, stordito

\textbf{Sensi} visione al buio 18 m

\textbf{Lingue} -

\textbf{Sfida} 1 (200 PE)

\emph{\textbf{Frenesia Sanguinaria.}} Lo sciame ha +1d6 ai tiri di attacco in mischia contro qualsiasi creatura che non sia al massimo dei punti ferita.

\emph{\textbf{Respirare Acqua.}} Lo sciame può respirare solo sottacqua.

\emph{\textbf{Sciame.}} Lo sciame può occupare lo spazio di un'altra creatura e viceversa, e lo sciame può muoversi attraverso qualsiasi apertura grande abbastanza per un Minuscolo pirana. Lo sciame non può recuperare punti ferita né ottenere punti ferita temporanei.

\textbf{Azioni}

\emph{\textbf{Morsi.} Attacco con Arma da Mischia}: +5 a colpire, portata 0 m, una creatura nello spazio dello sciame.

\emph{Colpisce:} 14 (4d6) danni perforanti, o 7 (2d6) danni perforanti se lo sciame è ha metà o meno dei suoi punti ferita.

\medskip\textbf{Sciame di Insetti}\index{Mostri - Sciame di Insetti}

\emph{Medio sciame di Minuscole bestie, disallineato}

\textbf{FORZA} -4

\textbf{DESTREZZA} +1

\textbf{COSTITUZIONE} +0

\textbf{INTELLIGENZA} -5

\textbf{SAGGEZZA} -2

\textbf{CARISMA} -5

\textbf{Iniziativa} +1 -- \textbf{Difesa} 13

\textbf{Punti Ferita} 22 (5d8)

\textbf{Movimento} 6 m, scalata 6 m

\textbf{Tiri Salvezza}: Tempra -3, Riflessi +2, Volontà -1

\textbf{Resistenze al Danno} da botta, perforante, tagliente

\textbf{Immunità alle Condizioni} affascinato, afferrato, intralciato, paralizzato, pietrificato, prono, spaventato, stordito

\textbf{Sensi} vista cieca 3 m

\textbf{Lingue} -

\textbf{Sfida} 1/2 (100 PE)

\emph{\textbf{Sciame.}} Lo sciame può occupare lo spazio di un'altra creatura e viceversa, e lo sciame può muoversi attraverso qualsiasi apertura grande abbastanza per un Minuscolo insetto. Lo sciame non può recuperare punti ferita né ottenere punti ferita temporanei.

\textbf{Azioni}

\emph{\textbf{Morsi.} Attacco con Arma da Mischia}: +3 a colpire, portata 0 m, un bersaglio nello spazio dello sciame.

\emph{Colpisce:} 10 (4d4) danni perforanti, o 5 (2d4) danni perforanti se lo sciame è ha metà o meno dei suoi punti ferita.

\medskip\textbf{Sciame di Pipistrelli}\index{Mostri - Sciame di Pipistrelli}

\emph{Medio sciame di Minuscole bestie, disallineato}

\textbf{FORZA} -3

\textbf{DESTREZZA} +2

\textbf{COSTITUZIONE} +0

\textbf{INTELLIGENZA} -4

\textbf{SAGGEZZA} +1

\textbf{CARISMA} -3

\textbf{Iniziativa} +2 -- \textbf{Difesa} 13

\textbf{Punti Ferita} 22 (5d8)

\textbf{Movimento} 0 m, volo 9 m

\textbf{Tiri Salvezza}: Tempra -2, Riflessi +4, Volontà +2

\textbf{Resistenze al Danno} da botta, perforante, tagliente

\textbf{Immunità alle Condizioni} affascinato, afferrato, intralciato, paralizzato, pietrificato, prono, spaventato, stordito

\textbf{Sensi} vista cieca 18 m

\textbf{Lingue} -

\textbf{Sfida} 1/4 (50 PE)

\emph{\textbf{Ecolocazione.}} Lo sciame non può usare la vista cieca se assordato.

\emph{\textbf{Sciame.}} Lo sciame può occupare lo spazio di un'altra creatura e viceversa, e lo sciame può muoversi attraverso qualsiasi apertura grande abbastanza per un Minuscolo pipistrello. Lo sciame non può recuperare punti ferita né ottenere punti ferita temporanei.

\emph{\textbf{Udito Affinato.}} Lo sciame ha +1d6 alle prove di Saggezza (Consapevolezza) basate sull'udito.

\textbf{Azioni}

\emph{\textbf{Morsi.} Attacco con Arma da Mischia}: +4 a colpire, portata 0 m, una creatura nello spazio dello sciame.

\emph{Colpisce:} 5 (2d4) danni perforanti, o 2 (1d4) danni perforanti se lo sciame è ha metà o meno dei suoi punti ferita.

\medskip\textbf{Sciame di Ragni}\index{Mostri - Sciame di Ragni}

\emph{Medio sciame di Minuscole bestie, disallineato}

\textbf{FORZA} -4

\textbf{DESTREZZA} +1

\textbf{COSTITUZIONE} +0

\textbf{INTELLIGENZA} -5

\textbf{SAGGEZZA} -2

\textbf{CARISMA} -5

\textbf{Iniziativa} +1 -- \textbf{Difesa} 13

\textbf{Punti Ferita} 22 (5d8)

\textbf{Movimento} 6 m, scalata 6 m

\textbf{Tiri Salvezza}: Tempra -3, Riflessi +2, Volontà -1

\textbf{Resistenze al Danno} da botta, perforante, tagliente

\textbf{Immunità alle Condizioni} affascinato, afferrato, intralciato, paralizzato, pietrificato, prono, spaventato, stordito

\textbf{Sensi} vista cieca 3 m

\textbf{Lingue} -

\textbf{Sfida} 1/2 (100 PE)

\emph{\textbf{Camminare sulla Tela.}} Lo sciame ignora le restrizioni al movimento provocate dalle ragnatele.

\emph{\textbf{Scalare come Ragno.}} Lo sciame può scalare superfici difficili, compreso lo stare a testa in giù sul soffitto, senza bisogno di effettuare una prova di abilità.

\emph{\textbf{Senso della Tela.}} Mentre è in contatto con una ragnatela, lo sciame sa l'esatta posizione di qualsiasi altra creatura in contatto con la stessa ragnatela.

\emph{\textbf{Sciame.}} Lo sciame può occupare lo spazio di un'altra creatura e viceversa, e lo sciame può muoversi attraverso qualsiasi apertura grande abbastanza per un Minuscolo insetto. Lo sciame non può recuperare punti ferita né ottenere punti ferita temporanei.

\textbf{Azioni}

\emph{\textbf{Morsi.} Attacco con Arma da Mischia}: +3 a colpire, portata 0 m, un bersaglio nello spazio dello sciame.

\emph{Colpisce:} 10 (4d4) danni perforanti, o 5 (2d4) danni perforanti se lo sciame è ha metà o meno dei suoi punti ferita.

\medskip\textbf{Sciame di Ratti}\index{Mostri - Sciame di Ratti}

\emph{Medio sciame di Minuscole bestie, disallineato}

\textbf{FORZA} -1

\textbf{DESTREZZA} +0

\textbf{COSTITUZIONE} -1

\textbf{INTELLIGENZA} -4

\textbf{SAGGEZZA} +0

\textbf{CARISMA} -4

\textbf{Iniziativa} +0 -- \textbf{Difesa} 11

\textbf{Punti Ferita} 24 (7d8 - 7)

\textbf{Movimento} 9 m

\textbf{Tiri Salvezza}: Tempra +0, Riflessi +1, Volontà +1

\textbf{Resistenze al Danno} da botta, perforante, tagliente

\textbf{Immunità alle Condizioni} affascinato, afferrato, intralciato, paralizzato, pietrificato, prono, spaventato, stordito

\textbf{Sensi} visione al buio 9 m
\textbf{Lingue} -

\textbf{Sfida} 1/4 (50 PE)

\emph{\textbf{Olfatto Affinato.}} Lo sciame ha +1d6 alle prove di Saggezza (Consapevolezza) basate sull'olfatto.

\emph{\textbf{Sciame.}} Lo sciame può occupare lo spazio di un'altra creatura e viceversa, e lo sciame può muoversi attraverso qualsiasi apertura grande abbastanza per un Minuscolo ratto. Lo sciame non può recuperare punti ferita né ottenere punti ferita temporanei.

\textbf{Azioni}

\emph{\textbf{Morsi.} Attacco con Arma da Mischia}: +2 a colpire, portata 0 m, un bersaglio nello spazio dello sciame.

\emph{Colpisce:} 7 (2d6) danni perforanti, o 3 (1d6) danni perforanti se lo sciame è ha metà o meno dei suoi punti ferita.

\medskip\textbf{Sciame di Scarabei}\index{Sciame di Scarabei}

\emph{Medio sciame di Minuscole bestie, disallineato}

\textbf{FORZA} -4

\textbf{DESTREZZA} +1

\textbf{COSTITUZIONE} +0

\textbf{INTELLIGENZA} -5

\textbf{SAGGEZZA} -2

\textbf{CARISMA} -5

\textbf{Iniziativa} +1 -- \textbf{Difesa} 13

\textbf{Punti Ferita} 22 (5d8)

\textbf{Movimento} 6 m, scalata 6 m, scavo 6 m

\textbf{Tiri Salvezza}: Tempra -3, Riflessi +2, Volontà -1

\textbf{Resistenze al Danno} da botta, perforante, tagliente

\textbf{Immunità alle Condizioni} affascinato, afferrato, intralciato, paralizzato, pietrificato, prono, spaventato, stordito

\textbf{Sensi} vista cieca 3 m

\textbf{Lingue} -

\textbf{Sfida} 1/2 (100 PE)

\emph{\textbf{Sciame.}} Lo sciame può occupare lo spazio di un'altra creatura e viceversa, e lo sciame può muoversi attraverso qualsiasi apertura grande abbastanza per un Minuscolo insetto. Lo sciame non può recuperare punti ferita né ottenere punti ferita temporanei.

\textbf{Azioni}

\emph{\textbf{Morsi.} Attacco con Arma da Mischia}: +3 a colpire, portata 0 m, un bersaglio nello spazio dello sciame.

\emph{Colpisce:} 10 (4d4) danni perforanti, o 5 (2d4) danni perforanti se lo sciame è ha metà o meno dei suoi punti ferita.

\medskip\textbf{Sciame di Serpenti Velenosi}\index{Sciame di Serpenti Velenosi}

\emph{Medio sciame di Minuscole bestie, disallineato}

\textbf{FORZA} -1

\textbf{DESTREZZA} +4

\textbf{COSTITUZIONE} +0

\textbf{INTELLIGENZA} -5

\textbf{SAGGEZZA} +0

\textbf{CARISMA} -4

\textbf{Iniziativa} +4 -- \textbf{Difesa} 15

\textbf{Punti Ferita} 36 (8d8)

\textbf{Movimento} 9 m, nuoto 9 m

\textbf{Tiri Salvezza}: Tempra +0, Riflessi +5, Volontà +1

\textbf{Resistenze al Danno} da botta, perforante, tagliente

\textbf{Immunità alle Condizioni} affascinato, afferrato, intralciato, paralizzato, pietrificato, prono, spaventato, stordito

\textbf{Sensi} vista cieca 3 m

\textbf{Lingue} -

\textbf{Sfida} 2 (450 PE)

\emph{\textbf{Sciame.}} Lo sciame può occupare lo spazio di un'altra creatura e viceversa, e lo sciame può muoversi attraverso qualsiasi apertura grande abbastanza per un Minuscolo serpente. Lo sciame non può recuperare punti ferita né ottenere punti ferita temporanei.

\textbf{Azioni}

\emph{\textbf{Morsi.} Attacco con Arma da Mischia}: +6 a colpire, portata 0 m, una creatura nello spazio dello sciame.

\emph{Colpisce:} 7 (2d6) danni perforanti, o 3 (1d6) danni perforanti se lo sciame è ha metà o meno dei suoi punti ferita, e il bersaglio deve effettuare un Tiro Salvezza di Tempra CD 10, e subire 14 (4d6) danni da veleno se fallisce il Tiro Salvezza, o la metà di questi danni se lo riesce.

\medskip\textbf{Sciame di Vespe}\index{Sciame di Serpenti Velenosi}

\emph{Medio sciame di Minuscole bestie, disallineato}

\textbf{FORZA} -4

\textbf{DESTREZZA} +1

\textbf{COSTITUZIONE} +0

\textbf{INTELLIGENZA} -5

\textbf{SAGGEZZA} -2

\textbf{CARISMA} -5

\textbf{Iniziativa} +1 -- \textbf{Difesa} 13

\textbf{Punti Ferita} 22 (5d8)

\textbf{Movimento} 1 m, volo 9 m

\textbf{Tiri Salvezza}: Tempra -3, Riflessi +2, Volontà -1

\textbf{Resistenze al Danno} da botta, perforante, tagliente

\textbf{Immunità alle Condizioni} affascinato, afferrato, intralciato, paralizzato, pietrificato, prono, spaventato, stordito

\textbf{Sensi} vista cieca 3 m

\textbf{Lingue} -

\textbf{Sfida} 1/2 (100 PE)

\emph{\textbf{Sciame.}} Lo sciame può occupare lo spazio di un'altra creatura e viceversa, e lo sciame può muoversi attraverso qualsiasi apertura grande abbastanza per un Minuscolo insetto. Lo sciame non può recuperare punti ferita né ottenere punti ferita temporanei.

\textbf{Azioni}

\emph{\textbf{Morsi.} Attacco con Arma da Mischia}: +3 a colpire, portata 0 m, un bersaglio nello spazio dello sciame.

\emph{Colpisce:} 10 (4d4) danni perforanti, o 5 (2d4) danni perforanti se lo sciame è ha metà o meno dei suoi punti ferita.

\medskip\textbf{Scimmione}\index{Mostri - Scimmione}

\emph{Media bestia, disallineato}

\textbf{FORZA} +3

\textbf{DESTREZZA} +2

\textbf{COSTITUZIONE} +2

\textbf{INTELLIGENZA} -2

\textbf{SAGGEZZA} +1

\textbf{CARISMA} -2

\textbf{Iniziativa} +2 -- \textbf{Difesa} 13

\textbf{Punti Ferita} 19 (3d8 + 6)

\textbf{Movimento} 9 m, scalata 9 m

\textbf{Tiri Salvezza}: Tempra +3, Riflessi +3, Volontà +2

\textbf{Competenze} Acrobatica +5, Consapevolezza +3

\textbf{Lingue} -

\textbf{Sfida} 1/2 (100 PE)

\textbf{Azioni}

\emph{\textbf{Multiattacco.}} Lo scimmione effettua due attacchi di pugno.

\emph{\textbf{Pugno.} Attacco con Arma da Mischia}: +5 a colpire, portata 1 m, un bersaglio.

\emph{Colpisce:} 6 (1d6 + 3) danni da botta.

\emph{\textbf{Sasso.} Attacco con Arma a Gittata}: +5 a colpire, gittata 8m, un bersaglio.

\emph{Colpisce:} 6 (1d6 + 3) danni da botta.

\medskip\textbf{Scimmione Gigante}\index{Mostri - Scimmione Gigante}

\emph{Enorme bestia, disallineato}

\textbf{FORZA} +6

\textbf{DESTREZZA} +2

\textbf{COSTITUZIONE} +4

\textbf{INTELLIGENZA} -2

\textbf{SAGGEZZA} +1

\textbf{CARISMA} -2

\textbf{Iniziativa} +2 -- \textbf{Difesa} 16

\textbf{Punti Ferita} 157 (15d12 + 60)

\textbf{Movimento} 12 m, scalata 12 m

\textbf{Tiri Salvezza}: Tempra +7, Riflessi +6, Volontà +4

\textbf{Competenze} Acrobatica +9, Consapevolezza +4

\textbf{Lingue} -

\textbf{Sfida} 7 (2.900 PE)

\textbf{Azioni}

\emph{\textbf{Multiattacco.}} Lo scimmione effettua due attacchi di pugno.

\emph{\textbf{Pugno.} Attacco con Arma da Mischia}: +9 a colpire, portata 3 m, un bersaglio.

\emph{Colpisce:} 22 (3d10 + 6) danni da botta.

\emph{\textbf{Sasso.} Attacco con Arma a Gittata}: +9 a colpire, gittata 15m, un bersaglio.

\emph{Colpisce:} 30 (7d6 + 6) danni da botta.

\medskip\textbf{Scorpione}\index{Mostri - Scorpione}

\emph{Minuscola bestia, disallineato}

\textbf{FORZA} -4

\textbf{DESTREZZA} +0

\textbf{COSTITUZIONE} -1

\textbf{INTELLIGENZA} -5

\textbf{SAGGEZZA} -1

\textbf{CARISMA} -4

\textbf{Iniziativa} +0 -- \textbf{Difesa} 12

\textbf{Punti Ferita} 1 (1d4 - 1)

\textbf{Movimento} 3 m

\textbf{Tiri Salvezza}: Tempra -3, Riflessi +2, Volontà -1

\textbf{Sensi} vista cieca 3 m

\textbf{Lingue} -

\textbf{Sfida} 0 (10 PE)

\textbf{Azioni}

\emph{\textbf{Pungiglione.} Attacco con Arma da Mischia}: +2 a colpire, portata 1 m, una creatura.

\emph{Colpisce:} 1 danno perforante e il bersaglio deve effettuare un Tiro Salvezza di Tempra CD 9, e subire 4 (1d8) danni da veleno se fallisce il Tiro Salvezza, o la metà di questi danni se lo riesce.

\medskip\textbf{Scorpione Gigante}\index{Mostri - Scorpione Gigante}

\emph{Grande bestia, disallineato}

\textbf{FORZA} +2

\textbf{DESTREZZA} +1

\textbf{COSTITUZIONE} +2

\textbf{INTELLIGENZA} -5

\textbf{SAGGEZZA} -1

\textbf{CARISMA} -4

\textbf{Iniziativa} +1 -- \textbf{Difesa} 17

\textbf{Punti Ferita} 52 (7d10 + 14)

\textbf{Movimento} 12 m

\textbf{Tiri Salvezza}: Tempra +7, Riflessi +1, Volontà +1

\textbf{Sensi} vista cieca 18 m

\textbf{Lingue} -

\textbf{Sfida} 3 (700 PE)

\textbf{Azioni}

\emph{\textbf{Multiattacco.}} Lo scorpione effettua tre attacchi: due con gli artigli e uno con il pungiglione.

\emph{\textbf{Artiglio.} Attacco con Arma da Mischia}: +4 a colpire, portata 1 m, un bersaglio.

\emph{Colpisce:} 6 (1d8 + 2) danni da botta e il bersaglio è afferrato (CD 12 per fuggire). Lo scorpione ha due artigli, ciascuno dei quali può afferrare solo un bersaglio.

\emph{\textbf{Pungiglione.} Attacco con Arma da Mischia}: +4 a colpire, portata 1 m, una creatura.

\emph{Colpisce:} 7 (1d10 + 2) danni perforanti e il bersaglio deve effettuare un Tiro Salvezza di Tempra CD 12, e subire 22 (4d10) danni da veleno se fallisce il Tiro Salvezza, o la metà di questi danni se lo riesce.

\medskip\textbf{Serpente Costrittore}\index{Mostri - Serpente Costrittore}

\emph{Grande bestia, disallineato}

\textbf{FORZA} +2

\textbf{DESTREZZA} +2

\textbf{COSTITUZIONE} +1

\textbf{INTELLIGENZA} -5

\textbf{SAGGEZZA} +0

\textbf{CARISMA} -4

\textbf{Iniziativa} +2 -- \textbf{Difesa} 13

\textbf{Punti Ferita} 13 (2d10 + 2)

\textbf{Movimento} 9 m, nuoto 9 m

\textbf{Tiri Salvezza}: Tempra +3, Riflessi +2, Volontà +0

\textbf{Sensi} vista cieca 3 m

\textbf{Lingue} -

\textbf{Sfida} 1/4 (50 PE)

\textbf{Azioni}

\emph{\textbf{Morso.} Attacco con Arma da Mischia}: +4 a colpire, portata 1 m, una creatura.

\emph{Colpisce:} 5 (1d6 + 2) danni perforanti.

\emph{\textbf{Stritolare.} Attacco con Arma da Mischia}: +4 a colpire, portata 1 m, una creatura.

\emph{Colpisce:} 6 (1d8 + 2) danni da botta, e il bersaglio è afferrato (CD 14 per fuggire). Fino al termine dell'afferrare, la creatura è intralciata, e il serpente non può stritolare un altro bersaglio.

\medskip\textbf{Serpente Costrittore Gigante}\index{Mostri - Serpente Costrittore Gigante}

\emph{Enorme bestia, disallineato}

\textbf{FORZA} +4

\textbf{DESTREZZA} +2

\textbf{COSTITUZIONE} +1

\textbf{INTELLIGENZA} -5

\textbf{SAGGEZZA} +0

\textbf{CARISMA} -4

\textbf{Iniziativa} +2 -- \textbf{Difesa} 13

\textbf{Punti Ferita} 60 (8d12 + 8)

\textbf{Movimento} 9 m, nuoto 9 m

\textbf{Tiri Salvezza}: Tempra +3, Riflessi +2, Volontà +0

\textbf{Competenze} Consapevolezza +2

\textbf{Sensi} vista cieca 3 m

\textbf{Lingue} -

\textbf{Sfida} 2 (450 PE)

\textbf{Azioni}

\emph{\textbf{Morso.} Attacco con Arma da Mischia}: +6 a colpire, portata 3 m, una creatura.

\emph{Colpisce:} 11 (2d6 + 4) danni perforanti.

\emph{\textbf{Stritolare.} Attacco con Arma da Mischia}: +6 a colpire, portata 1 m, una creatura.

\emph{Colpisce:} 13 (2d8 + 4) danni da botta, e il bersaglio è afferrato (CD 16 per fuggire). Fino al termine dell'afferrare, la creatura è intralciata, e il serpente non può stritolare un altro bersaglio.

\medskip\textbf{Serpente Velenoso}\index{Mostri - Serpente Velenoso}

\emph{Minuscola bestia, disallineato}

\textbf{FORZA} -4

\textbf{DESTREZZA} +3

\textbf{COSTITUZIONE} +0

\textbf{INTELLIGENZA} -5

\textbf{SAGGEZZA} +0

\textbf{CARISMA} -4

\textbf{Iniziativa} +3 -- \textbf{Difesa} 14

\textbf{Punti Ferita} 2 (1d4)

\textbf{Movimento} 9 m, nuoto 9 m

\textbf{Tiri Salvezza}: Tempra +1, Riflessi +4, Volontà +1

\textbf{Sensi} vista cieca 3 m

\textbf{Lingue} -

\textbf{Sfida} 1/8 (25 PE)

\textbf{Azioni}

\emph{\textbf{Morso.} Attacco con Arma da Mischia}: +5 a colpire, portata 1 m, un bersaglio.

\emph{Colpisce:} 1 danno perforante e il bersaglio deve effettuare un Tiro Salvezza di Tempra CD 10, e subire 5 (2d4) danni da veleno se fallisce il Tiro Salvezza, o la metà di questi danni se lo riesce.

\medskip\textbf{Serpente Velenoso Gigante}\index{Mostri - Serpente Velenoso Gigante}

\emph{Media bestia, disallineato}

\textbf{FORZA} +0

\textbf{DESTREZZA} +4

\textbf{COSTITUZIONE} +1

\textbf{INTELLIGENZA} -4

\textbf{SAGGEZZA} +0

\textbf{CARISMA} -4

\textbf{Iniziativa} +4 -- \textbf{Difesa} 15

\textbf{Punti Ferita} 11 (2d8 + 2)

\textbf{Movimento} 9 m, nuoto 9 m

\textbf{Tiri Salvezza}: Tempra +1, Riflessi +5, Volontà +2

\textbf{Competenze} Consapevolezza +2

\textbf{Sensi} vista cieca 3 m

\textbf{Lingue} -

\textbf{Sfida} 1/4 (50 PE)

\textbf{Azioni}

\emph{\textbf{Morso.} Attacco con Arma da Mischia}: +6 a colpire, portata 3 m, un bersaglio.

\emph{Colpisce:} 6 (1d4 + 4) danni perforanti e il bersaglio deve effettuare un Tiro Salvezza di Tempra CD 11, e subire 10 (3d6) danni da veleno se fallisce il Tiro Salvezza, o la metà di questi danni se lo riesce.

\medskip\textbf{Serpente Volante}\index{Mostri - Serpente Volante}

Un serpente volante è una serpe alata, dai colori intensi, rinvenuta in giungle remote.

\emph{Minuscola bestia, disallineato}

\textbf{FORZA} -3

\textbf{DESTREZZA} +4

\textbf{COSTITUZIONE} +0

\textbf{INTELLIGENZA} -4

\textbf{SAGGEZZA} +1

\textbf{CARISMA} -3

\textbf{Iniziativa} +4 -- \textbf{Difesa} 15

\textbf{Punti Ferita} 5 (2d4)

\textbf{Movimento} 9 m, nuoto 9 m, volo 18 m

\textbf{Tiri Salvezza}: Tempra -2, Riflessi +5, Volontà +1

\textbf{Sensi} vista cieca 3 m

\textbf{Lingue} -

\textbf{Sfida} 1/8 (25 PE)

\emph{\textbf{Sorvolare.}} Il serpente non provoca attacchi di opportunità quando vola via dalla portata di un nemico.

\textbf{Azioni}

\emph{\textbf{Morso.} Attacco con Arma da Mischia}: +6 a colpire, portata 1 m, un bersaglio.

\emph{Colpisce:} 1 danno perforante più 7 (3d4) danni da veleno.

\medskip\textbf{Squalo Cacciatore}\index{Mostri - Squalo Cacciatore}

Uno squalo cacciatore è lungo da 4 a 6 metri e di solito caccia in solitario nelle acque più profonde.

\emph{Grande bestia, disallineato}

\textbf{FORZA} +4

\textbf{DESTREZZA} +1

\textbf{COSTITUZIONE} +2

\textbf{INTELLIGENZA} -5

\textbf{SAGGEZZA} +0

\textbf{CARISMA} -3

\textbf{Iniziativa} +1 -- \textbf{Difesa} 13

\textbf{Punti Ferita} 45 (6d10 + 12)

\textbf{Movimento} 0 m, nuoto 12 m

\textbf{Tiri Salvezza}: Tempra +4, Riflessi +2, Volontà +0

\textbf{Competenze} Consapevolezza +2

\textbf{Sensi} vista cieca 9 m

\textbf{Lingue} -

\textbf{Sfida} 2 (450 PE)

\emph{\textbf{Frenesia Sanguinaria.}} Lo squalo ha +1d6 ai tiri di attacco in mischia contro qualsiasi creatura che non sia al massimo dei punti ferita.

\emph{\textbf{Respirare Acqua.}} Lo squalo può respirare solo sott'acqua.

\textbf{Azioni}

\emph{\textbf{Morso.} Attacco con Arma da Mischia}: +6 a colpire, portata 1 m, un bersaglio.

\emph{Colpisce:} 13 (2d8 + 4) danni perforanti.

\medskip\textbf{Squalo Corallino}\index{Mostri - Squalo Corallino}

Gli squali corallini sono lunghi da 2 a 3 metri e vivono nelle acque meno profonde e lungo le barriere coralline.

\emph{Media bestia, disallineato}

\textbf{FORZA} +2

\textbf{DESTREZZA} +1

\textbf{COSTITUZIONE} +1

\textbf{INTELLIGENZA} -5

\textbf{SAGGEZZA} +0

\textbf{CARISMA} -3

\textbf{Iniziativa} +1 -- \textbf{Difesa} 13

\textbf{Punti Ferita} 22 (4d8 + 4)

\textbf{Movimento} 0 m, nuoto 12 m

\textbf{Tiri Salvezza}: Tempra +2, Riflessi +2, Volontà +1

\textbf{Competenze} Consapevolezza +2

\textbf{Sensi} vista cieca 9 m

\textbf{Lingue} -

\textbf{Sfida} 1/2 (100 PE)

\emph{\textbf{Respirare Acqua.}} Lo squalo può respirare solo sottacqua.

\emph{\textbf{Tattiche di Branco.}} Lo squalo ha +1d6 al tiro di attacco contro una creatura se almeno uno degli alleati dello squalo si trova entro 1 metro dalla creatura e quell'alleato non è inabile.

\textbf{Azioni}

\emph{\textbf{Morso.} Attacco con Arma da Mischia}: +4 a colpire, portata 1 m, un bersaglio.

\emph{Colpisce:} 6 (1d8 + 2) danni perforanti.

\medskip\textbf{Squalo Gigante}\index{Mostri - Squalo Gigante}

Lo squalo gigante è lungo 9 metri e lo si incontra

normalmente solo negli oceani più profondi.

\emph{Enorme bestia, disallineato}

\textbf{FORZA} +6

\textbf{DESTREZZA} +0

\textbf{COSTITUZIONE} +5

\textbf{INTELLIGENZA} -5

\textbf{SAGGEZZA} +0

\textbf{CARISMA} -3

\textbf{Iniziativa} +0 -- \textbf{Difesa} 16

\textbf{Punti Ferita} 126 (11d12 + 55)

\textbf{Movimento} 0 m, nuoto 15 m

\textbf{Tiri Salvezza}: Tempra +7, Riflessi +2, Volontà +1

\textbf{Competenze} Consapevolezza +3

\textbf{Sensi} vista cieca 18 m

\textbf{Lingue} -

\textbf{Sfida} 5 (1.800 PE)

\emph{\textbf{Frenesia Sanguinaria.}} Lo squalo ha +1d6 ai tiri di attacco

in mischia contro qualsiasi creatura che non sia al massimo dei punti ferita.

\emph{\textbf{Respirare Acqua.}} Lo squalo può respirare solo sottacqua. 

\textbf{Azioni}

\emph{\textbf{Morso.} Attacco con Arma da Mischia}: +9 a colpire, portata 1 m, un bersaglio.

\emph{Colpisce:} 22 (3d10 + 6) danni perforanti.

\medskip\textbf{Strige}\index{Mostri - Strige}

Questo orrendo mostro sembra un incrocio tra un grosso pipistrello e una zanzara sovradimensionata. Le sue zampe terminano in lunghe pinze, e la sua lunga proboscide, simile ad un ago, fende l'aria mentre cerca di nutrirsi del sangue delle creature viventi. 

\emph{Minuscola bestia, disallineato}

\textbf{FORZA} -3

\textbf{DESTREZZA} +3

\textbf{COSTITUZIONE} +0

\textbf{INTELLIGENZA} -4

\textbf{SAGGEZZA} -1

\textbf{CARISMA} -2

\textbf{Iniziativa} +3 -- \textbf{Difesa} 15

\textbf{Punti Ferita} 2 (1d4)

\textbf{Movimento} 3 m, volo 12 m

\textbf{Tiri Salvezza}: Tempra -3, Riflessi +4, Volontà -1

\textbf{Sensi} visione al buio 18 m

\textbf{Lingue} -

\textbf{Sfida} 1/8 (25 PE)

\textbf{Azioni}

\emph{\textbf{Risucchio di Sangue.} Attacco con Arma da Mischia}: +5 a colpire, portata 1 m, una creatura.

\emph{Colpisce:} 5 (1d4 + 3) danni perforanti e lo strige si attacca al bersaglio. Mentre è attaccato, lo strige non attacca. Invece, all'inizio di ciascun turno dello strige, il bersaglio perde 5 (1d4 + 3) punti ferita a causa della perdita di sangue.

Lo strige può staccarsi spendendo 1 metro di movimento. Lo fa automaticamente dopo aver risucchiato 10 punti ferita dal bersaglio o alla morte del bersaglio. Una creatura, compreso il bersaglio, può usare la sua azione per staccare lo strige.

\medskip\textbf{Tasso}\index{Mostri - Tasso}

\emph{Minuscola bestia, disallineato}

\textbf{FORZA} -3

\textbf{DESTREZZA} +0

\textbf{COSTITUZIONE} +1

\textbf{INTELLIGENZA} -4

\textbf{SAGGEZZA} +1

\textbf{CARISMA} -3

\textbf{Iniziativa} +0 -- \textbf{Difesa} 11

\textbf{Punti Ferita} 3 (1d4 + 1)

\textbf{Movimento} 6 m, scavo 1 m

\textbf{Tiri Salvezza}: Tempra -3, Riflessi +1, Volontà +1

\textbf{Sensi} visione al buio 9 m

\textbf{Lingue} -

\textbf{Sfida} 0 (10 PE)

\emph{\textbf{Olfatto Affinato.}} Il tasso ha +1d6 alle prove di Saggezza (Consapevolezza) basate sull'olfatto.

\textbf{Azioni}

\emph{\textbf{Morso.} Attacco con Arma da Mischia}: +2 a colpire, portata 1 m, un bersaglio.

\emph{Colpisce:} 1 danno perforante.

\medskip\textbf{Tasso Gigante}\index{Mostri - Tasso Gigante}

\emph{Media bestia, disallineato}

\textbf{FORZA} +1

\textbf{DESTREZZA} +0

\textbf{COSTITUZIONE} +2

\textbf{INTELLIGENZA} -4

\textbf{SAGGEZZA} +1

\textbf{CARISMA} -3

\textbf{Iniziativa} +0 -- \textbf{Difesa} 11

\textbf{Punti Ferita} 13 (2d8 + 4)

\textbf{Movimento} 9 m, scavo 3 m

\textbf{Tiri Salvezza}: Tempra +2, Riflessi +1, Volontà +2

\textbf{Sensi} visione al buio 9 m

\textbf{Lingue} -

\textbf{Sfida} 1/4 (50 PE)

\emph{\textbf{Olfatto Affinato.}} Il tasso ha +1d6 alle prove di Saggezza (Consapevolezza) basate sull'olfatto.

\textbf{Azioni}

\emph{\textbf{Multiattacco.}} Il tasso effettua due attacchi: uno con il morso e uno con gli artigli.

\emph{\textbf{Artigli.} Attacco con Arma da Mischia}: +3 a colpire,  portata 1 m, un bersaglio.

\emph{Colpisce:} 6 (2d4 + 1) danni taglienti.

\emph{\textbf{Morso.} Attacco con Arma da Mischia}: +3 a colpire, portata 1 m, un bersaglio.

\emph{Colpisce:} 4 (1d6 + 1) danni perforanti.

\medskip\textbf{Tigre}\index{Mostri - Tigre}

\emph{Grande bestia, disallineato}

\textbf{FORZA} +3

\textbf{DESTREZZA} +2

\textbf{COSTITUZIONE} +2

\textbf{INTELLIGENZA} -4

\textbf{SAGGEZZA} +1

\textbf{CARISMA} -1

\textbf{Iniziativa} +2 -- \textbf{Difesa} 13

\textbf{Punti Ferita} 37 (5d10 + 10)

\textbf{Movimento} 12 m

\textbf{Tiri Salvezza}: Tempra +4, Riflessi +4, Volontà +2

\textbf{Competenze} Muoversi Silenziosamente / Nascondersi +6, Consapevolezza +3

\textbf{Sensi} visione al buio 18 m

\textbf{Lingue} -

\textbf{Sfida} 1 (200 PE)

\emph{\textbf{Balzo.}} Se la tigre si muove di almeno 6 metri diretta verso una creatura e la colpisce con un attacco di artiglio durante lo stesso turno, il bersaglio deve riuscire un Tiro Salvezza di Tempra CD 13 o cadere prono. Se il bersaglio è prono, la tigre può effettuare un attacco di morso contro di esso come azione bonus.

\emph{\textbf{Olfatto Affinato.}} La tigre ha +1d6 alle prove di Saggezza (Consapevolezza) basate sull'olfatto.

\textbf{Azioni}

\emph{\textbf{Artiglio.} Attacco con Arma da Mischia}: +5 a colpire, portata 1 m, un bersaglio.

\emph{Colpisce:} 7 (1d8 + 3) danni taglienti.

\emph{\textbf{Morso.} Attacco con Arma da Mischia}: +5 a colpire, portata 1 m, un bersaglio.

\emph{Colpisce:} 8 (1d10 + 3) danni perforanti.

\medskip\textbf{Tigre dai Denti a Sciabola}\index{Mostri - Tigre dai Denti a Sciabola}

\emph{Grande bestia, disallineato}

\textbf{FORZA} +4

\textbf{DESTREZZA} +2

\textbf{COSTITUZIONE} +2

\textbf{INTELLIGENZA} -4

\textbf{SAGGEZZA} +1

\textbf{CARISMA} -1

\textbf{Iniziativa} +2 -- \textbf{Difesa} 13

\textbf{Punti Ferita} 52 (7d10 + 14)

\textbf{Movimento} 12 m

\textbf{Tiri Salvezza}: Tempra +5, Riflessi +3, Volontà +2

\textbf{Competenze} Muoversi Silenziosamente / Nascondersi +6, Consapevolezza +3

\textbf{Lingue} -

\textbf{Sfida} 2 (450 PE)

\emph{\textbf{Balzo.}} Se la tigre si muove di almeno 6 metri diretta verso una creatura e la colpisce con un attacco di artiglio durante lo stesso turno, il bersaglio deve riuscire un Tiro Salvezza di Tempra CD 14 o cadere prono. Se il bersaglio è prono, la tigre può effettuare un attacco di morso contro di esso come azione bonus.

\emph{\textbf{Olfatto Affinato.}} La tigre ha +1d6 alle prove di Saggezza (Consapevolezza) basate sull'olfatto.

\textbf{Azioni}

\emph{\textbf{Artiglio.} Attacco con Arma da Mischia}: +6 a colpire, portata 1 m, un bersaglio.

\emph{Colpisce:} 12 (2d6 + 5) danni taglienti.

\emph{\textbf{Morso.} Attacco con Arma da Mischia}: +6 a colpire, portata 1 m, un bersaglio.

\emph{Colpisce:} 10 (1d10 + 5) danni perforanti.

\medskip\textbf{Topi}\\\index{Mostri - Topi}
\emph{Minuscola fatata}\\
\textbf{Forza}: -1\\
\textbf{Destrezza}: +4\\
\textbf{Costituzione}: +0\\
\textbf{Intelligenza}: +6\\
\textbf{Saggezza}: +2\\
\textbf{Carisma}: +6\\
\textbf{Difesa}: 17 -- \textbf{Iniziativa}: +15\\
\textbf{Punti Ferita}: 4 (1d10 - 1)\\
\textbf{Movimento}: 6 m\\
\textbf{Tiri Salvezza}: Tempra +20, Riflessi +30, Volontà +20 \\
\textbf{Sensi}: Senso tellurico 30 , Scurovisione 9 m, Visione del Vero 30 m\\
\textbf{Lingue}: tutte\\
\textbf{Sfida}: 0 (10 PE)\smallskip\\
\textbf{Immunità}: al danno delle armi con bonus magico inferiore a +6\\
\textbf{Immunità}: a qualsiasi magia la Topi non voglia essere influenzata\\
\emph{\textbf{E' la Topi}} La Topi ha +3d6 (oppure +18) ogni volta che deve tirare dei dadi o contare un valore.
Qualsiasi attacco effettuato dalla Topi e' considerato magico +5 e non e' resistibile.\\
\smallskip\textbf{Azioni}\\
\emph{\textbf{Musetto}} ogni creatura a scelta di Topi, entro 30 metri, subisce un Musetto. La creatura viene allontanata di 2d6 metri e subisce 3d6 danni\\
\emph{\textbf{Morso topetto} Attacco con Arma da Mischia}: +26 al colpire, portata 1 m, un bersaglio.\\
\emph{Colpisce:} 6 danno perforante.\\
\emph{\textbf{Graffiotto} fino a 8 Attacchi con Arma da Mischia}: colpisce automaticamente, portata 1 m, fino a 4 bersagli.\\
\emph{Colpisce:} 1 danno perforante, ignora ogni Resistenza, immunità o protezione.\\


\medskip\textbf{Vespa Gigante}\index{Mostri - Vespa Gigante}

\emph{Media bestia, disallineato}

\textbf{FORZA} +0

\textbf{DESTREZZA} +2

\textbf{COSTITUZIONE} +0

\textbf{INTELLIGENZA} -5

\textbf{SAGGEZZA} +0

\textbf{CARISMA} -4

\textbf{Iniziativa} +2 -- \textbf{Difesa} 13

\textbf{Punti Ferita} 13 (3d8)

\textbf{Movimento} 3 m, volo 15 m

\textbf{Tiri Salvezza}: Tempra +1, Riflessi +3, Volontà +0 

\textbf{Lingue} -

\textbf{Sfida} 1/2 (100 PE)

\textbf{Azioni}

\emph{\textbf{Pungiglione.} Attacco con Arma da Mischia}: +4 a colpire, portata 1 m, una creatura.

\emph{Colpisce:} 5 (1d6 + 2) danni perforanti e il bersaglio deve effettuare un Tiro Salvezza di Tempra CD 11, e subire 10 (3d6) danni da veleno se fallisce il Tiro Salvezza, o la metà di questi danni se lo riesce. Se il danno da veleno riduce il bersaglio a 0 punti ferita, il bersaglio è stabile ma avvelenato per 1 ora, anche dopo aver recuperato i punti ferita, e mentre è avvelenato in questo modo resta paralizzato.

\medskip\textbf{Worg}\index{Mostri - Worg}

I worg sono mostruosi predatori dall'aspetto simile ad un lupo che amano cacciare e divorare le creature più deboli di loro.

\emph{Grande mostruosità, neutrale malvagio}

\textbf{FORZA} +3

\textbf{DESTREZZA} +1

\textbf{COSTITUZIONE} +1

\textbf{INTELLIGENZA} -2

\textbf{SAGGEZZA} +0

\textbf{CARISMA} -1

\textbf{Iniziativa} +1 -- \textbf{Difesa} 14

\textbf{Punti Ferita} 26 (4d10 + 4)

\textbf{Movimento} 15 m

\textbf{Tiri Salvezza}: Tempra +3, Riflessi +2, Volontà +2 

\textbf{Competenze} Consapevolezza +4

\textbf{Sensi} visione al buio 18 m

\textbf{Lingue} Goblin, Worg

\textbf{Sfida} 1/2 (100 PE)

\emph{\textbf{Udito e Olfatto Affinato.}} Il worg ha +1d6 nelle prove di Saggezza (Consapevolezza) basate su udito o olfatto.

\textbf{Azioni}

\emph{\textbf{Morso.} Attacco con Arma da Mischia}: +5 a colpire, portata 1 m, un bersaglio.

\emph{Colpisce:} 10 (2d6 + 3) danni perforanti. Se il bersaglio è una creatura, deve riuscire un Tiro Salvezza di Tempra CD 13 o cadere prona.

\subsection{Appendice B: Personaggi Non Giocanti}\index{Mostri - Personaggi Non Giocanti}

Questa appendice contiene le statistiche di vari personaggi non giocanti (PNG) umanoidi che gli avventurieri possono incontrare nel corso di una campagna, da infimi popolani a potenti arcimaghi. Queste statistiche possono essere utilizzate per rappresentare PNG umani e non.

Personalizzare i PNG

Esistono molti semplici modi di personalizzare i PNG di questa appendice per l'uso nella tua campagna casalinga.

\emph{\textbf{Cambiare Incantesimi.}} Un modo per personalizzare un PNG incantatore è quello di rimpiazzare uno o più dei suoi incantesimi. Puoi sostituire qualsiasi incantesimo della lista di
incantesimi del PNG con un diverso incantesimo della stessa Difficoltà. Cambiare incantesimi in questo modo non modifica il grado di sfida del PNG.

\textbf{\emph{Cambiare Armi e Armatura}.} Puoi migliorare o peggiorare l'armatura del PNG o aggiungere o cambiare armi. Le modifiche alla Difesa e ai danni possono modificare il grado di sfida del PNG.

\emph{\textbf{Oggetti Magici}}. Più potente è un PNG, maggiori le probabilità che possieda uno o più
oggetti magici. Un mago, ad esempio, potrebbe avere una bacchetta o un bastone magico, oltre ad una o più pozioni e pergamene. Fornire un PNG di un potente oggetto magico capace di infliggere danni potrebbe modificarne il grado di sfida.

Alcuni oggetti magici di esempio sono descritti più avanti in questo documento.

\textbf{Combattenti}

I combattenti sono individui che si guadagnano da vivere mettendo la loro spada al servizio di un individuo o un ideale.

\medskip\textbf{Guardia}

Le guardie comprendono membri della ronda cittadina, sentinelle di una cittadella o città fortificata e le guardie del corpo di nobili e mercanti.

\emph{Media umanoide (qualsiasi razza), qualsiasi allineamento} 

\textbf{FORZA} +1

\textbf{DESTREZZA} +1

\textbf{COSTITUZIONE} +1

\textbf{INTELLIGENZA} +0

\textbf{SAGGEZZA} +0

\textbf{CARISMA} +0

\textbf{Iniziativa} +1 -- \textbf{Difesa} 17 (giaco di maglia, scudo)

\textbf{Punti Ferita} 11 (2d8 + 2)

\textbf{Movimento} 9 m

\textbf{Tiri Salvezza}: Tempra +3, Riflessi +1, Volontà +1 

\textbf{Competenze} Consapevolezza +2


\textbf{Lingue} una qualsiasi lingua (di solito il Comune)

\textbf{Sfida} 1/8 (25 PE)

\textbf{Azioni}

\emph{\textbf{Lancia.} Attacco con Arma da Mischia o a Gittata}: +3 a colpire, portata 1 m o gittata 6m, un bersaglio.

\emph{Colpisce:} 4 (1d6 + 1) danni perforanti o 5 (1d8 + 1) danni perforanti se impiegata con due mani per effettuare un attacco da mischia.

\medskip\textbf{Veterano}

Guerrieri sopravvissuti a lungo, guadagnandosi una grande fama di esperti e abili combattenti.

\emph{Media umanoide (qualsiasi razza), qualsiasi allineamento}

\textbf{FORZA} +3

\textbf{DESTREZZA} +1

\textbf{COSTITUZIONE} +2

\textbf{INTELLIGENZA} +0

\textbf{SAGGEZZA} +0

\textbf{CARISMA} +0

\textbf{Iniziativa} +1 -- \textbf{Difesa} 19 (armatura di strisce)

\textbf{Punti Ferita} 58 (9d8 + 18)

\textbf{Movimento} 9 m

\textbf{Tiri Salvezza}: Tempra +4, Riflessi +2, Volontà +3 

\textbf{Competenze} Acrobatica +5, Consapevolezza +2

\textbf{Lingue} una lingua qualsiasi (di solito il Comune)

\textbf{Sfida} 3 (700 PE)

\textbf{Azioni}

\emph{\textbf{Multiattacco.}} Il veterano effettua due attacchi con la spada lunga. Se ha estratto una spada corta, può effettuare anche un attacco con la spada corta.

\emph{\textbf{Spada Lunga.} Attacco con Arma da Mischia}: +5 a colpire, portata 1 m, un bersaglio.

\emph{Colpisce:} 7 (1d8 + 3) danni taglienti, o 8 (1d10 + 3) danni taglienti se usata con due mani.

\emph{\textbf{Spada Corta.} Attacco con Arma da Mischia}: +5 a colpire, portata 1 m, un bersaglio.

\emph{Colpisce:} 6 (1d6 + 3) danni perforanti.

\emph{\textbf{Balestra Pesante.} Attacco con Arma a Gittata}: +3 a colpire, gittata 30m, un bersaglio. \emph{Colpisce:} 6 (1d10 + 1) danni perforanti.

\medskip\textbf{Cavaliere}

I cavalieri sono combattenti che giurano fedeltà a sovrani, ordini religiosi, e nobili cause. L'allineamento del cavaliere determina fino a che punto è disposto ad onorare il suo giuramento.

\emph{Media umanoide (qualsiasi razza), qualsiasi allineamento}

\textbf{FORZA} +3

\textbf{DESTREZZA} +0

\textbf{COSTITUZIONE} +2

\textbf{INTELLIGENZA} +0

\textbf{SAGGEZZA} +0

\textbf{CARISMA} +2

\textbf{Iniziativa} +0 -- \textbf{Difesa} 20 (armatura di piastre)

\textbf{Punti Ferita} 52 (8d8 + 16)

\textbf{Movimento} 9 m

\textbf{Tiri Salvezza}: Tempra +4, Riflessi +1, Volontà +3

\textbf{Lingue} una qualsiasi lingua (di solito il Comune)

\textbf{Sfida} 3 (700 PE)

\emph{\textbf{Coraggioso.}} Il cavaliere ha +1d6 ai Tiri Salvezza contro l'essere spaventato.

\textbf{Azioni}

\emph{\textbf{Multiattacco.}} Il cavaliere effettua due attacchi da mischia.

\emph{\textbf{Spada Grossa.} Attacco con Arma da Mischia}: +5 a colpire, portata 1 m, un bersaglio.

\emph{Colpisce:} 10 (2d6 + 3) danni taglienti.

\emph{\textbf{Balestra Pesante.} Attacco con Arma a Gittata}: +2 a colpire, gittata 30m, un bersaglio.

\emph{Colpisce:} 5 (1d10) perforanti.

\emph{\textbf{Autorità (Ricarica dopo un 1 ora)}}. Per 1 minuto, il cavaliere può pronunciare un comando speciale o avvertimento ogni qualvolta una creatura non ostile entro 9 metri da lui, e che possa vedere, effettua un tiro di attacco o Tiro Salvezza. La creatura può sommare un d4 al suo tiro purchè possa udire e comprendere il cavaliere. Una creatura può beneficiare di un solo dado Autorità alla volta. Questo effetto termina se il cavaliere è inabile.

\textbf{Reazioni}

\emph{\textbf{Parata.}} Il cavaliere può aggiungere 2 alla sua Difesa contro un attacco da mischia che lo colpirebbe. Per farlo, il cavaliere deve vedere l'attaccante e star impugnando un'arma da mischia.

\medskip\textbf{Gladiatore}

Addestrati per intrattenere le folle, sono tra i combattenti più pericolosi in circolazione.

\emph{Media umanoide (qualsiasi razza), qualsiasi allineamento}
\textbf{FORZA} +4

\textbf{DESTREZZA} +2

\textbf{COSTITUZIONE} +3

\textbf{INTELLIGENZA} +0

\textbf{SAGGEZZA} +1

\textbf{CARISMA} +2

\textbf{Iniziativa} +2 -- \textbf{Difesa} 19 (armatura di cuoio borchiato, scudo)

\textbf{Punti Ferita} 112 (15d8 + 45)

\textbf{Movimento} 9 m

\textbf{Tiri Salvezza}: Tempra +5, Riflessi +5, Volontà +3 

\textbf{Competenze} Acrobatica +10, Intimidazione +5

\textbf{Lingue} una lingua qualsiasi (di solito il Comune)

\textbf{Sfida} 5 (1.800 PE)

\emph{\textbf{Bruto.}} Un'arma da mischia infligge un dado aggiuntivo di danno

quando un gladiatore colpisce con essa (già incluso nell'attacco).

\emph{\textbf{Coraggioso.}} Il gladiatore ha +1d6 ai Tiri Salvezza contro l'essere spaventato.

\textbf{Azioni}

\emph{\textbf{Multiattacco.}} Il gladiatore effettua tre attacchi da mischia o due attacchi a gittata.

\emph{\textbf{Lancia.} Attacco con Arma da Mischia o a Gittata}: +7 a colpire, portata 1 m o gittata 6m, un bersaglio.

\emph{Colpisce:} 11 (2d6 + 4) danni perforanti, o 13 (2d8 + 4) danni taglienti se usata con due mani.

\emph{\textbf{Botta di Scudo.} Attacco con Arma da Mischia}: +7 a colpire, portata 1 m, un bersaglio.

\emph{Colpisce:} 9 (2d4 + 4) danni da botta. Se il bersaglio è una creatura di taglia Media o inferiore, deve riuscire un Tiro Salvezza su Tempra CD 15 o cadere prono.

\textbf{Reazioni}

\emph{\textbf{Parata.}} Il gladiatore somma 3 alla sua Difesa contro un attacco da mischia che lo colpirebbe. Per farlo, il gladiatore deve vedere l'attaccante e impugnare un'arma da mischia.

\medskip\textbf{Cittadini}

In questa categoria rientrano quegli individui che si occupano di mandare avanti il mondo, svolgendo le mansioni necessarie affinché i campi vengano coltivati, le città amministrate, il cibo coltivato e
nuovi territori esplorati.

\medskip\textbf{Nobile}

I nobili comandano sulla popolazione, in virtù di un diritto di nascita o per le ricchezze accumulate. Tra costoro si annoverano anche i cortigiani che affollano le corti dei ricchi e dei potenti.

\emph{Media umanoide (qualsiasi razza), qualsiasi allineamento}

\textbf{FORZA} +0

\textbf{DESTREZZA} +1

\textbf{COSTITUZIONE} +0

\textbf{INTELLIGENZA} +1

\textbf{SAGGEZZA} +2

\textbf{CARISMA} +3

\textbf{Iniziativa} +1 -- \textbf{Difesa} 16 (pettorale)

\textbf{Punti Ferita} 9 (2d8)

\textbf{Movimento} 9 m

\textbf{Tiri Salvezza}: Tempra +1, Riflessi +1, Volontà +2 

\textbf{Competenze} Percepire Emozioni +4, Ingannare +5

\textbf{Lingue} due lingue qualsiasi

\textbf{Sfida} 1/8 (25 PE)

\textbf{Azioni}

\emph{\textbf{Stocco.} Attacco con Arma da Mischia}: +3 a colpire, portata 1 m, un bersaglio.

\emph{Colpisce:} 5 (1d8 + 1) danni perforanti.

\textbf{Reazioni}

\emph{\textbf{Parata.}} Il nobile somma 2 alla sua Difesa contro un attacco da mischia che lo colpirebbe. Per farlo, il nobile deve vedere

l'attaccante e impugnare un'arma da mischia.

\medskip\textbf{Popolano}

I popolani comprendono contadini, servi, schiavi, servitori, pellegrini, mercanti, artigiani ed eremiti.

\emph{Media umanoide (qualsiasi razza), qualsiasi allineamento}

\textbf{FORZA} +0

\textbf{DESTREZZA} +0

\textbf{COSTITUZIONE} +0

\textbf{INTELLIGENZA} +0

\textbf{SAGGEZZA} +0

\textbf{CARISMA} +0

\textbf{Iniziativa} +0 -- \textbf{Difesa} 11

\textbf{Punti Ferita} 4 (1d8)

\textbf{Movimento} 9 m

\textbf{Tiri Salvezza}: Tempra +0, Riflessi +0, Volontà +0 

\textbf{Lingue} una qualsiasi lingua (di solito il Comune)

\textbf{Sfida} 0 (10 PE)

\textbf{Azioni}

\emph{\textbf{Randello.} Attacco con Arma da Mischia}: +2 a colpire, portata 1 m, un bersaglio.

\emph{Colpisce:} 2 (1d4) danni da botta.

\medskip\textbf{Criminali}

I criminali sono individui che vivono al margine della legalità, procurandosi il pane svolgendo attività spesso considerate illecite e immorali.

\medskip\textbf{Picchiatore}

I picchiatori sono criminali spietati abili nell'intimidire e perpetrare atti di violenza. Lavorano per soldi e si fanno pochi scrupoli.

\emph{Media umanoide (qualsiasi razza), qualsiasi allineamento non buono}

\textbf{FORZA} +2

\textbf{DESTREZZA} +0

\textbf{COSTITUZIONE} +2

\textbf{INTELLIGENZA} +0

\textbf{SAGGEZZA} +0

\textbf{CARISMA} +0

\textbf{Iniziativa} +0 -- \textbf{Difesa} 12 (armatura di cuoio)

\textbf{Punti Ferita} 32 (5d8 + 10)

\textbf{Movimento} 9 m

\textbf{Tiri Salvezza}: Tempra +3, Riflessi +1, Volontà +0 

\textbf{Competenze} Intimidazione +2

\textbf{Lingue} una lingua qualsiasi (di solito il Comune)

\textbf{Sfida} 1/2 (100 PE)

\emph{\textbf{Tattiche di Branco.}} Il picchiatore ha +1d6 ai tiri di attacco contro una creatura se almeno uno degli alleati del picchiatore si trova entro 1 metro dalla creatura e quell'alleato non
è inabile.

\textbf{Azioni}

\emph{\textbf{Multiattacco.}} Il picchiatore effettua due attacchi da mischia.

\emph{\textbf{Mazza.} Attacco con Arma da Mischia}: +4 a colpire, portata 1 m, una creatura.

\emph{Colpisce:} 5 (1d6 + 2) danni da botta.

\emph{\textbf{Balestra Pesante.} Attacco con Arma a Gittata}: +2 a colpire, gittata 30m, un bersaglio. \emph{Colpisce:} 5 (1d10) danni perforanti.

\medskip\textbf{Bandito/Pirata}

Che siano uomini di strada o di mare (pirati) costoro guadagnano da vivere depredando il prossimo.

\emph{Media umanoide (qualsiasi razza), qualsiasi allineamento non legale}

\textbf{FORZA} +0

\textbf{DESTREZZA} +1

\textbf{COSTITUZIONE} +1

\textbf{INTELLIGENZA} +0

\textbf{SAGGEZZA} +0

\textbf{CARISMA} +0

\textbf{Iniziativa} +1 -- \textbf{Difesa} 13 (armatura di cuoio)

\textbf{Punti Ferita} 11 (2d8 + 2)

\textbf{Movimento} 9 m

\textbf{Tiri Salvezza}: Tempra +1, Riflessi +2, Volontà +1 

\textbf{Lingue} una qualsiasi lingua (di solito il Comune)

\textbf{Sfida} 1/8 (25 PE)

\textbf{Azioni}

\emph{\textbf{Scimitarra.} Attacco con Arma da Mischia}: +3 a colpire, portata 1 m, un bersaglio.

\emph{Colpisce:} 4 (1d6 + 1) danni taglienti.

\emph{\textbf{Balestra Leggera.} Attacco con Arma a Gittata}: +3 a colpire, gittata 24m, un bersaglio. \emph{Colpisce:} 5 (1d8 + 1) danni taglienti.

\medskip\textbf{Spia}

Una spia è un individuo addestramento nel reperire segreti per conto di qualcuno, o a volte per rivenderli al miglior offerente.

\emph{Media umanoide (qualsiasi razza), qualsiasi allineamento}

\textbf{FORZA} +0

\textbf{DESTREZZA} +2

\textbf{COSTITUZIONE} +0

\textbf{INTELLIGENZA} +1

\textbf{SAGGEZZA} +2

\textbf{CARISMA} +3

\textbf{Iniziativa} +2 -- \textbf{Difesa} 13

\textbf{Punti Ferita} 27 (6d8)

\textbf{Movimento} 9 m

\textbf{Tiri Salvezza}: Tempra +2, Riflessi +3, Volontà +3 

\textbf{Competenze} Muoversi Silenziosamente / Nascondersi +4, Percepire Emozioni +4, Investigazione +5, Consapevolezza +6, Ingannare +5, Mani di fata +4

\textbf{Lingue} due lingue qualsiasi

\textbf{Sfida} 1 (200 PE)

\emph{\textbf{Attacco Furtivo (1/Turno).}} La spia infligge 7 (2d6) danni aggiuntivi quando colpisce un bersaglio con un attacco con arma e ha +1d6 al tiro di attacco, o quando il bersaglio è entro 1 metro da un alleato dell'assassino che non è inabile e l'assassino non ha -1d6 al tiro di attacco.

\emph{\textbf{Azione Astuta.}} Durante ciascun suo turno, la spia può usare un'azione bonus per effettuare l'azione Ritirarsi, Nascondersi o Scattare.

\textbf{Azioni}

\emph{\textbf{Multiattacco.}} La spia effettua due attacchi da mischia.

\emph{\textbf{Spada Corta.} Attacco con Arma da Mischia}: +4 a colpire, portata 1 m, un bersaglio.

\emph{Colpisce:} 5 (1d6 + 2) danni perforanti.

\emph{\textbf{Balestrino.} Attacco con Arma a Gittata}: +4 a colpire, gittata 9m, un bersaglio. \emph{Colpisce:} 5 (1d6 + 2) danni perforanti.


\medskip\textbf{Capitano dei Banditi/Pirata}

Che viva in terra o in mare, è un individuo munito di una grande personalità che riesce a tenere in riga la marmaglia che risponde ai suoi ordini.

\emph{Media umanoide (qualsiasi razza), qualsiasi allineamento non legale}

\textbf{FORZA} +2

\textbf{DESTREZZA} +3

\textbf{COSTITUZIONE} +2

\textbf{INTELLIGENZA} +2

\textbf{SAGGEZZA} +0

\textbf{CARISMA} +2

\textbf{Iniziativa} +2 -- \textbf{Difesa} 16 (armatura di cuoio borchiato)

\textbf{Punti Ferita} 65 (10d8 + 8)

\textbf{Movimento} 9 m

\textbf{Tiri Salvezza}: Tempra +5, Riflessi +5, Volontà +3 

\textbf{Competenze} Acrobatica +4, Raggiro +4 

\textbf{Lingue} due lingue qualsiasi

\textbf{Sfida} 2 (450 PE)

\textbf{Azioni}

\emph{\textbf{Multiattacco.}} Il capitano effettua tre attacchi da mischia: due con la scimitarra e uno con il pugnale. Oppure il capitano effettua due attacchi a gittata con i pugnali.

\emph{\textbf{Scimitarra.} Attacco con Arma da Mischia}: +5 a colpire, portata 1 m, un bersaglio.

\emph{Colpisce:} 6 (1d6 + 3) danni taglienti.

\emph{\textbf{Pugnale.} Attacco con Arma da Mischia o a Gittata}: +5 a colpire, portata 1 m o gittata 6m, un bersaglio. \emph{Colpisce:} 5 (1d4 + 3) danni perforanti.

\textbf{Reazioni}

\emph{\textbf{Parata.}} Il capitano somma 2 alla sua Difesa contro un attacco da mischia che lo colpirebbe. Per farlo, il capitano deve vedere l'attaccante e impugnare un'arma da mischia.

\medskip\textbf{Assassino}

Solitari o membri di una gilda, gli assassini sono pagati per eliminare, spesso in modo silenzioso e discreto, rivali e nemici dei loro datori di lavoro.

\emph{Media umanoide (qualsiasi razza), qualsiasi allineamento non buono}

\textbf{FORZA} +0

\textbf{DESTREZZA} +3

\textbf{COSTITUZIONE} +2

\textbf{INTELLIGENZA} +1

\textbf{SAGGEZZA} +0

\textbf{CARISMA} +0

\textbf{Iniziativa} +3 -- \textbf{Difesa} 19 (armatura di cuoio borchiato)

\textbf{Punti Ferita} 78 (12d8 + 24)

\textbf{Movimento} 9 m

\textbf{Tiri Salvezza}: Tempra +4, Riflessi +6, Volontà +3 

\textbf{Competenze} Acrobazia +6, Muoversi Silenziosamente / Nascondersi +9, Consapevolezza +3, Raggiro +3


\textbf{Lingue} Gergo dei Ladri più due altre lingue

\textbf{Sfida} 8 (3.900 PE)

\emph{\textbf{Assassinare.}} Durante il suo primo turno, l'assassino ha +1d6 ai tiri di attacco contro le creature che non hanno ancora svolto nessun turno. Qualsiasi colpo che l'assassino mandi a segno contro una creatura sorpresa, è un colpo critico.

\emph{\textbf{Attacco Furtivo (1/Turno).}} L'assassino infligge 14 (4d6) danni aggiuntivi quando colpisce un bersaglio con un attacco con arma e ha +1d6 al tiro di attacco, o quando il bersaglio è entro 1 metro da un alleato dell'assassino che non è inabile e l'assassino non ha -1d6 al tiro di attacco.

\emph{\textbf{Evasione.}} Se l'assassino è vittima di un effetto che permette di effettuare un Tiro Salvezza di Riflessi per dimezzare i danni, l'assassino non prende danni se riesce il Tiro Salvezza, e solo la metà se lo fallisce.

\textbf{Azioni}

\emph{\textbf{Multiattacco.}} L'assassino effettua due attacchi con le spade corte.

\emph{\textbf{Spada Corta.} Attacco con Arma da Mischia}: +6 a colpire, portata 1 m, un bersaglio.

\emph{Colpisce:} 6 (1d6 + 3) danni perforanti, e il bersaglio deve effettuare un Tiro Salvezza di Tempra CD 15, subendo 24 (7d6) danni da veleno se fallisce il Tiro Salvezza, o la metà di questi danni se lo riesce.

\emph{\textbf{Balestra Leggera.} Attacco con Arma a Gittata}: +6 a colpire, gittata 24m, un bersaglio.

\emph{Colpisce:} 7 (1d8 + 3) danni perforanti, e il bersaglio deve effettuare un Tiro Salvezza di Tempra CD 15, subendo 24 (7d6) danni da veleno se fallisce il Tiro Salvezza, o la metà di questi danni se lo riesce.

\medskip\textbf{Mago}

Il mago trascorrono la vita nello studio e la pratica della magia.

\textbf{VARIANTE: FAMIGLI}

Qualsiasi incantatore che possa eseguire l'incantesimo \emph{trovare} \emph{famiglio} è probabile che abbia un famiglio. Il famiglio può essere una delle creature descritte nell'incantesimo (vedi le \emph{Regole Base}) o qualche altro mostro Minuscolo, come un artiglio strisciante, un diavoletto, uno pseudodrago o un demonietto.

\medskip\textbf{Mago Avventuriero}

Un Mago novizio, che ha superato con successo le sue prime avventure e ha iniziato a stabilire una reputazione come nobile o famigerato avventuriero.

\emph{Media umanoide (qualsiasi razza), qualsiasi malvagio}

\textbf{FORZA} -1

\textbf{DESTREZZA} +2

\textbf{COSTITUZIONE} +0

\textbf{INTELLIGENZA} +3

\textbf{SAGGEZZA} +1

\textbf{CARISMA} +0

\textbf{Iniziativa} +3 -- \textbf{Difesa} 13

\textbf{Punti Ferita} 22 (5d8)

\textbf{Movimento} 9 m

\textbf{Tiri Salvezza}: Tempra +0, Riflessi +3, Volontà +2 

\textbf{Competenze} Arcano +5, Storia +5

\textbf{Lingue} quattro lingue qualsiasi

\textbf{Sfida} 1 (200 PE)

\emph{\textbf{Incantesimi.}} Il mago ha CM 4. La sua abilità da incantatore è l'Intelligenza (+5 al colpire con attacchi con incantesimo). Il Mago ha preparato i seguenti incantesimi: Trucchetti (a volontà): 

\emph{luce, mano magica, stretta folgorante}

Difficoltà 16 (4 slot): \emph{charme su persone, dardo incantato}

Difficoltà 19 (3 slot): \emph{bloccare persona, passo velato}

\textbf{Azioni}

\emph{\textbf{Bastone.} Attacco con Arma da Mischia}: +1 a colpire, portata 1 m, un bersaglio.

\emph{Colpisce:} 3 (1d8 - 1) danni da botta.

\medskip\textbf{Grande Mago}

Un Mago che ha stabilito una discreta fama nel territorio e che attira intorno a sé studenti da ogni dove.

\emph{Media umanoide (qualsiasi razza), qualsiasi allineamento}

\textbf{FORZA} -1

\textbf{DESTREZZA} +2

\textbf{COSTITUZIONE} +0

\textbf{INTELLIGENZA} +3

\textbf{SAGGEZZA} +1

\textbf{CARISMA} +0

\textbf{Iniziativa} +3 -- \textbf{Difesa} 15 (18 con \emph{armatura del Mago})

\textbf{Punti Ferita} 40 (9d8)

\textbf{Movimento} 9 m

\textbf{Tiri Salvezza}: Tempra +1, Riflessi +4, Volontà +3 

\textbf{Competenze} Arcano +6, Storia +6

\textbf{Lingue} quattro lingue qualsiasi

\textbf{Sfida} 6 (2.300 PE)

\emph{\textbf{Incantesimi.}} Il mago ha CM 9. La sua abilità da incantatore è l'Intelligenza (+6 al colpire con attacchi con incantesimo). Il Mago ha preparato i seguenti incantesimi:

Trucchetti (a volontà): \emph{dardo infuocato, luce, mano magica,}
\emph{prestidigitazione}

Difficoltà 16 (4 slot): \emph{armatura del Mago, dardo incantato,}
\emph{individuare magia, scudo}

Difficoltà 19 (3 slot): \emph{passo velato, suggestione}

Difficoltà 21 (3 slot): \emph{controincantesimo, palla di fuoco, volare}

Difficoltà 23 (3 slot): \emph{invisibilità superiore, tempesta di ghiaccio}

Difficoltà 26 (1 slot): \emph{cono di freddo}

\textbf{Azioni}

\emph{\textbf{Pugnale.} Attacco con Arma da Mischia o a Gittata}: +5 a colpire, portata 1 m o gittata 6m, un bersaglio. \emph{Colpisce:} 4 (1d4 + 2) danni perforanti.

\medskip\textbf{ArciMago}

Un mago molto potente (e anche molto anziano) che studia i segreti del multiverso.

\emph{Media umanoide (qualsiasi razza), qualsiasi allineamento}

\textbf{FORZA} +0

\textbf{DESTREZZA} +2

\textbf{COSTITUZIONE} +1

\textbf{INTELLIGENZA} +5

\textbf{SAGGEZZA} +2

\textbf{CARISMA} +3

\textbf{Iniziativa} +5 -- \textbf{Difesa} 18 (21 con \emph{armatura del Mago})

\textbf{Punti Ferita} 99 (18d8 + 18)

\textbf{Movimento} 9 m

\textbf{Tiri Salvezza}: Tempra +8, Riflessi +10, Volontà +12 

\textbf{Competenze} Arcano +13, Storia +13

\textbf{Resistenze al Danno} danno degli incantesimi; da botta, perforante e tagliente non magico (da \emph{pelle di pietra})

\textbf{Lingue} sei lingue qualsiasi

\textbf{Sfida} 12 (8.400 PE)

\emph{\textbf{Incantesimi.}} Il mago ha CM 18. La sua abilità da incantatore è l'Intelligenza (+9 al colpire con attacchi con incantesimo).

L'arciMago può eseguire \emph{camuffare sé stesso} e \emph{invisibilità} a volontà e ha preparato i seguenti incantesimi: Trucchetti (a volontà): \emph{dardo infuocato, luce, mano magica,}
\emph{prestidigitazione, stretta folgorante}

Difficoltà 16 (4 slot): \emph{armatura magica*, dardo incantato,}
\emph{identificare, individuare magia}

Difficoltà 19 (3 slot): \emph{immagine speculare, individuazione dei}
\emph{pensieri, passo velato}

Difficoltà 21 (3 slot): \emph{controincantesimo, fulmine}

Difficoltà 23 (3 slot): \emph{esilio, pelle di pietra*, scudo di fuoco}

Difficoltà 26 (3 slot): \emph{cono di freddo, muro di forza, scrutare}

Difficoltà 29 (1 slot): \emph{globo di invulnerabilità}

Difficoltà 31 (1 slot): \emph{teletrasporto}

Difficoltà 34 (1 slot): \emph{vuoto mentale*}

Difficoltà 36 (1 slot): \emph{fermare il tempo}

L'arciMago esegue questi incantesimi su di sé prima del combattimento.

\textbf{Azioni}

\emph{\textbf{Pugnale.} Attacco con Arma da Mischia o a Gittata}: +6 a colpire, portata 1 m o gittata 6m, un bersaglio. \emph{Colpisce:} 4 (1d4 + 2) danni perforanti.


\medskip\textbf{Sacerdoti}

I sacerdoti sono devoti di una divinità o una fede che si prendono cura di impartire gli insegnamenti divini al loro gregge.

\medskip\textbf{Cultista}

I cultisti giurano fedeltà ai poteri oscuri, e nelle loro credenze e pratiche mostrano spesso segni di follia.

\emph{Media umanoide (qualsiasi razza), qualsiasi allineamento non buono}

\textbf{FORZA} +0

\textbf{DESTREZZA} +1

\textbf{COSTITUZIONE} +0

\textbf{INTELLIGENZA} +0

\textbf{SAGGEZZA} +0

\textbf{CARISMA} +0

\textbf{Iniziativa} +0- \textbf{Difesa} 13 (armatura di cuoio)

\textbf{Punti Ferita} 9 (2d8)

\textbf{Movimento} 9 m

\textbf{Tiri Salvezza}: Tempra +1, Riflessi +1, Volontà +2 

\textbf{Competenze} Raggiro +2, Religione +2

\textbf{Lingue} una qualsiasi lingua (di solito il Comune)

\textbf{Sfida} 1/8 (25 PE)

\emph{\textbf{Oscura Devozione.}} Il cultista ha +1d6 sui Tiri Salvezza contro l'essere affascinato o spaventato.

\textbf{Azioni}

\emph{\textbf{Scimitarra.} Attacco con Arma da Mischia}: +3 a colpire, portata 1 m, una creatura.

\emph{Colpisce:} 4 (1d6 + 1) danni taglienti.

\medskip\textbf{Accolito}

Gli accoliti sono membri di grado minore del clero, e di solito rispondono ad un sacerdote di rango superiore. Svolgono diverse funzioni in un tempio e gli viene conferita dalla loro divinità l'abilità di eseguire incantesimi minori.

\emph{Media umanoide (qualsiasi razza), qualsiasi allineamento}

\textbf{FORZA} +0

\textbf{DESTREZZA} +0

\textbf{COSTITUZIONE} +0

\textbf{INTELLIGENZA} +0

\textbf{SAGGEZZA} +2

\textbf{CARISMA} +0

\textbf{Iniziativa} +0 -- \textbf{Difesa} 11

\textbf{Punti Ferita} 9 (2d8)

\textbf{Movimento} 9 m

\textbf{Tiri Salvezza}: Tempra +0, Riflessi +0, Volontà +3 

\textbf{Competenze} Pronto Soccorso +4, Religione +2

\textbf{Lingue} una qualsiasi lingua (di solito il Comune)

\textbf{Sfida} 1/4 (50 PE)

\emph{\textbf{Incantesimi.}} L'accolito ha CM 1. La sua abilità da incantatore è la Saggezza (+4 al colpire con attacchi con incantesimo). L'accolito ha preparato i seguenti incantesimi: Trucchetti (a volontà): \emph{fiamma sacra, luce, taumaturgia} Difficoltà 16 (3 slot): \emph{benedizione}, \emph{cura ferite, santuario}

\medskip\textbf{Azioni}

\emph{\textbf{Randello.} Attacco con Arma da Mischia}: +2 a colpire, portata 1 m, un bersaglio.

\emph{Colpisce:} 2 (1d4) danni da botta.

\textbf{Fanatico del Culto}

Sono i capi di un culto, che usano il proprio carisma e i propri dogmi per influenzare i deboli di volontà.

\emph{Media umanoide (qualsiasi razza), qualsiasi allineamento non buono}

\textbf{FORZA} +0

\textbf{DESTREZZA} +2

\textbf{COSTITUZIONE} +1

\textbf{INTELLIGENZA} +0

\textbf{SAGGEZZA} +1

\textbf{CARISMA} +2

\textbf{Iniziativa} +2 -- \textbf{Difesa} 14 (armatura di cuoio)

\textbf{Punti Ferita} 33 (6d8 + 6)

\textbf{Movimento} 9 m

\textbf{Tiri Salvezza}: Tempra +2, Riflessi +2, Volontà +3 

\textbf{Competenze} Ingannare +4, Raggiro +4, Religione +2 

\textbf{Lingue} una qualsiasi lingua (di solito il Comune)

\textbf{Sfida} 2 (450 PE)

\emph{\textbf{Incantesimi.}} Il sacerdote ha CM 4. La sua abilità da incantatore è la Saggezza (+3 al colpire con attacchi con incantesimo). Il sacerdote ha preparato i seguenti incantesimi: Trucchetti (a volontà): \emph{fiamma sacra, luce, taumaturgia}

Difficoltà 16 (4 slot): \emph{comando, infliggi ferite, scudo della fede}

Difficoltà 19 (3 slot): \emph{arma spirituale, blocca persona}

\emph{\textbf{Oscura Devozione.}} Il cultista ha +1d6 sui Tiri Salvezza contro l'essere affascinato o spaventato.

\textbf{Azioni}

\emph{\textbf{Multiattacco.}} Il fanatico effettua due attacchi da mischia.

\emph{\textbf{Pugnale.} Attacco con Arma da Mischia o a Gittata}: +4 a colpire, portata 1 m o gittata 6m, una creatura. \emph{Colpisce:} 4 (1d4 + 2) danni perforanti.

\medskip\textbf{Gran Sacerdote}

Sono individui al comando di un tempio o altro luogo sacro e che hanno a loro disposizione diversi accoliti.

\emph{Media umanoide (qualsiasi razza), qualsiasi allineamento}

\textbf{FORZA} +0

\textbf{DESTREZZA} +0

\textbf{COSTITUZIONE} +1

\textbf{INTELLIGENZA} +1

\textbf{SAGGEZZA} +3

\textbf{CARISMA} +1

\textbf{Iniziativa} +1 -- \textbf{Difesa} 14 (giaco di maglia)

\textbf{Punti Ferita} 27 (5d8 + 5)

\textbf{Movimento} 7 m

\textbf{Tiri Salvezza}: Tempra +1, Riflessi +1, Volontà +4 

\textbf{Competenze} Pronto Soccorso +7, Ingannare +3, Religione +4

\textbf{Lingue} due lingue qualsiasi

\textbf{Sfida} 2 (450 PE)

\emph{\textbf{Eminenza Divina.}} Come azione bonus, il sacerdote può spendere uno slot incantesimo per far sì che il suo attacco con arma da mischia infligge 10 (3d6) danni da Luce aggiuntivi. Il beneficio dura fino al termine del turno.

\emph{\textbf{Incantesimi.}} Il sacerdote ha CM 5. La sua abilità da incantatore è la Saggezza (+5 al colpire con attacchi con incantesimo). Il sacerdote ha preparato i seguenti incantesimi: Trucchetti (a volontà): \emph{fiamma sacra, luce, taumaturgia}

Difficoltà 16 (4 slot): \emph{cura ferite, dardo tracciante, santuario}

Difficoltà 19 (3 slot): \emph{arma spirituale, ristorare inferiore}

Difficoltà 21 (2 slot): \emph{dissolvi magie}, \emph{guardiani spirituali}

\textbf{Azioni}

\emph{\textbf{Mazza.} Attacco con Arma da Mischia}: +2 a colpire, portata 1 m, un bersaglio.

\emph{Colpisce:} 3 (1d6) danni da botta.


\medskip\textbf{Selvaggi}

Questi individui vivono ai margini della civiltà, a volte entrandovi raramente in contatto. A disagio tra le mura e nelle terre civilizzate, si trovano nel loro ambiente quando possono muoversi tra le terre selvagge.

\medskip\textbf{Berserker}

Provenienti da terre selvagge, gli imprevedibili berserker si radunano in compagnie di guerra e sono sempre alla ricerca di conflitti in cui combattere.

\emph{Media umanoide (qualsiasi razza), qualsiasi allineamento caotico}

\textbf{FORZA} +3

\textbf{DESTREZZA} +1

\textbf{COSTITUZIONE} +3

\textbf{INTELLIGENZA} -1

\textbf{SAGGEZZA} +0

\textbf{CARISMA} -1

\textbf{Iniziativa} +1 -- \textbf{Difesa} 14 (armatura di pelle)

\textbf{Punti Ferita} 67 (9d8 + 27)

\textbf{Movimento} 9 m

\textbf{Tiri Salvezza}: Tempra +4, Riflessi +3, Volontà +2 

\textbf{Lingue} una qualsiasi lingua (di solito il Comune)

\textbf{Sfida} 2 (450 PE)

\emph{\textbf{Incauto.}} All'inizio del suo turno, il berserker può ottenere +1d6 su tutti i tiri di attacco con armi da mischia effettuati durante quel turno, ma i tiri di attacco contro di esso hanno
+1d6 fino all'inizio del suo prossimo turno.

\textbf{Azioni}

\emph{\textbf{Ascia Grossa.} Attacco con Arma da Mischia}: +5 a colpire, portata 1 m, un bersaglio.

\emph{Colpisce:} 9 (1d12 + 3) danni taglienti.

\textbf{Combattente Tribale}

Sono i difensori delle tribù che vivono ai margini della civiltà.

\emph{Media umanoide (qualsiasi razza), qualsiasi allineamento}

\textbf{FORZA} +1

\textbf{DESTREZZA} +0

\textbf{COSTITUZIONE} +1

\textbf{INTELLIGENZA} -1

\textbf{SAGGEZZA} +0

\textbf{CARISMA} -1

\textbf{Iniziativa} +0 -- \textbf{Difesa} 13 (armatura di pelle)

\textbf{Punti Ferita} 11 (2d8 + 2)

\textbf{Movimento} 9 m

\textbf{Tiri Salvezza}: Tempra +2, Riflessi +1, Volontà +1 

\textbf{Lingue} una qualsiasi lingua

\textbf{Sfida} 1/8 (25 PE)

\emph{\textbf{Tattiche di Branco.}} Il combattente tribale ha +1d6 ai tiri di attacco contro una creatura se almeno uno degli alleati del picchiatore si trova entro 1 metro dalla creatura e quell'alleato non è inabile.

\textbf{Azioni}

\emph{\textbf{Lancia.} Attacco con Arma da Mischia o a Gittata}: +3 a colpire, portata 1 m o gittata 6m, un bersaglio.

\emph{Colpisce:} 4 (1d6 + 1) danni perforanti, o 5 (1d8 + 1) danni perforanti se usata con due mani per effettuare un attacco da mischia.

\medskip\textbf{Druido}

I druidi proteggono il mondo naturale dai mostri e dall'avanzare della civiltà. Alcuni sono sciamani tribali che curano i malati, pregano agli spiriti animali e forniscono consigli spirituali.

\emph{Media umanoide (qualsiasi razza), qualsiasi allineamento}

\textbf{FORZA} +0

\textbf{DESTREZZA} +1

\textbf{COSTITUZIONE} +1

\textbf{INTELLIGENZA} +1

\textbf{SAGGEZZA} +2

\textbf{CARISMA} +0

\textbf{Iniziativa} +1 -- \textbf{Difesa} 12 (17 con \emph{pelle di corteccia}*)

\textbf{Punti Ferita} 27 (5d8 + 5)

\textbf{Movimento} 9 m

\textbf{Tiri Salvezza}: Tempra +1, Riflessi +2, Volontà +3 \\

\textbf{Competenze} Pronto Soccorso +4, Natura +3, Consapevolezza +4 

\textbf{Lingue} Druidico più due altre lingue

\textbf{Sfida} 2 (450 PE)

\emph{\textbf{Incantesimi.}} Il sacerdote ha CM 4. La sua abilità da incantatore è la Saggezza (+4 al colpire con attacchi con incantesimo). Il sacerdote ha preparato i seguenti incantesimi: Trucchetti (a volontà): \emph{arte druidica, bastone, produrre fiamma}

Difficoltà 16 (4 slot): \emph{intralciare, onda tonante, parlare con gli}
\emph{animali, passo veloce}

Difficoltà 19 (3 slot): \emph{animale messaggero, pelle di corteccia}

\textbf{Azioni}

\emph{\textbf{Bastone da Combattimento.} Attacco con Arma da Mischia}: +2 a colpire (+4 a colpire con \emph{bastone*}), portata 1 m o gittata 6m, un bersaglio.

\emph{Colpisce:} 3 (1d6) danni da botta, o 6 (1d8 + 2) danni da botta con \emph{bastone} o se impugnato con due mani.

\medskip\textbf{Esploratore}

Abili cacciatori e battitori di piste.

\emph{Media umanoide (qualsiasi razza), qualsiasi allineamento}

\textbf{FORZA} +0

\textbf{DESTREZZA} +2

\textbf{COSTITUZIONE} +1

\textbf{INTELLIGENZA} +0

\textbf{SAGGEZZA} +1

\textbf{CARISMA} +0

\textbf{Iniziativa} +2 -- \textbf{Difesa} 14 (armatura di cuoio)

\textbf{Punti Ferita} 16 (3d8 + 3)

\textbf{Movimento} 9 m

\textbf{Tiri Salvezza}: Tempra +1, Riflessi +2, Volontà +3

\textbf{Competenze} Muoversi Silenziosamente / Nascondersi +6, Natura +4, Consapevolezza +5, Sopravvivenza +5

\textbf{Lingue} una qualsiasi lingua (di solito Comune)

\textbf{Sfida} 1/2 (100 PE)

\emph{\textbf{Olfatto e Vista Affinati.}} L'esploratore ha +1d6 nelle prove di Saggezza (Consapevolezza) basate su olfatto o vista.

\textbf{Azioni}

\emph{\textbf{Multiattacco.}} L'esploratore effettua due attacchi da mischia o due attacchi a gittata.

\emph{\textbf{Spada Corta.} Attacco con Arma da Mischia}: +4 a colpire, portata 1 m, un bersaglio.

\emph{Colpisce:} 5 (1d6 + 2) danni perforanti.

\emph{\textbf{Arco Lungo.} Attacco con Arma da Mischia}: +4 a colpire, gittata 45m, un bersaglio.

\emph{Colpisce:} 6 (1d8 + 2) danni perforanti.


\end{multicols}

%{\scriptsize 
%	\printindex}
%\end{document}

\pagebreak


\section*{Conversione Mostri}\index{Conversione Mostri}

\bigskip

Per aggiungere altri mostri a TUS vi invito a convertire da Pathfinder o dalla 5ed del famoso gioco di ruolo i vari mostri.\\

TUS è di base un sistema D20 fortemente modificato nelle dinamiche ma non nelle fondamenta dei valori numerici.

Prendiamo ad esempio l'Orco comune da https://www.d20pfsrd.com/bestiary/monster-listings/humanoids/orcs/orc/ , tralasciamo le parti descrittive e concentriamoci sui numeri e valori.

\bigskip

\textbf{Orc (CR 1/3)} \textgreater{} questo valore rimane il medesimo in TUS

\textbf{XP 135} \textgreater{} questo valore non ha più senso

\textbf{Orc warrior 1} \textgreater{} non ci interessa

\textbf{CE Medium humanoid} \textgreater{} ci indica che la creatura è di taglia media, umanoide e malvagio, ai fini dei tratti la creatura non è di livello tale da aver attirato l'attenzione di un Patrono.

\textbf{Init +0} \textgreater{} è l'iniziativa

\textbf{Senses} darkvision 60 ft.; Perception --1 \textgreater{} rimane uguale, si tengono gli stessi valori ed abilità. In questo caso 60 piedi indica che la distanza è di 20 metri

\textbf{Weakness} light sensitivity \textgreater{} si cerca dove possibile l'equivalente in TUS, in questo caso fotofobia leggera oppure si applicano direttamente le penalità indicate.

\textbf{AC} 13, touch 10, flat-footed 13 (+3 armor) \textgreater{} la AC deve essere aumentata di 2 in ogni componente

\textbf{Competenza Armi}: è pari al BAB indicato

\textbf{Competenza Magia}: di base è metà del CR. Utile solo se la creatura ha poteri magici.

\textbf{hp} 6 (1d10+1) \textgreater{} rimane uguale

\textbf{Fort} +3, Ref +0, Will --1 \textgreater{} si traducono in Tempra, Agilità e Arbitrio. Il punteggio rimane uguale

\textbf{Speed} 30 ft. \textgreater{} è il movimento, in questo
caso è 9 metri

\textbf{Melee} falchion +5 (2d4+4/18--20) \textgreater{} è il mio Tiro per Colpire e danno. Rimane uguale

\textbf{Ranged} javelin +1 (1d6+3) \textgreater{} è il Tiro per Colpire. Rimane uguale

\textbf{Str} 17, Dex 11, Con 12, Int 7, Wis 8, Cha 6 \textgreater{} togli 10 dalla somma di Forza e Costituzione e poi fai la media arrotondata per eccesso per determinare la Potenza, in questo caso ((17+12)/2-10)/2=3. Devi prendere solo la parte bonus.

\textbf{Base Atk} +1; CMB +4; CMD 14 \textgreater{} il primo valore determina la CA. Suggerisco di usare direttamente i bonus al colpire indicati nel melee.

\textbf{Feats} Weapon Focus (falchion) \textgreater{} Arma Focalizzata. Il bonus dell'abilità è già calcolata nei valori di Melee

\textbf{Skills} Intimidate +2 \textgreater{} rimane uguale. In questo caso è Faccia Tosta

\textbf{Ferocity} (Ex): An orc remains conscious and can continue fighting even if its hit point total is below 0. It is still staggered and loses 1 hit point each round. A creature with ferocity still dies when its hit point total reaches a negative amount equal to its Constitution score. \textgreater{} si prende la abilità così come e'.

\pagebreak

\section{Scheda e Manuale}\index{Scheda}

\label{scheda-e-manuale}

Il link diretto per l'ultima versione compilata di TUS e'\\ \url{https://github.com/buzzqw/TUS/blob/master/The\%20Untitled\%20Bell\%20System.pdf}\\

\medskip

Questo il link per la scheda
\url{https://github.com/buzzqw/TUS/blob/master/TUS-schedav6.pdf}\\

\medskip
Questa versione non e' propriamente la piu' indicata da usare in quanto contiene tutte le modifiche attive e non definite apportate al sistema.\\

\medskip
Le versioni "stabili" le trovi su Versioni (\url{https://github.com/buzzqw/TUS/releases/latest})\\

\includepdf[pages={1,2},scale=0.95]{TUS-schedav6.pdf}

\pagebreak

\section{Autore}\index{Autore}

\bigskip

\textsc{Autore ed Ideatore}: Andres Zanzani  - azanzani@gmail.com

\bigskip
\textsc{Coautore}: Roberta Giorgini - madgiorgini@yahoo.it

\bigskip

Playtesting: Fabrizio Bonetti, Emanuele Pezzi, Nicola Ricottone, Marco Valmori, Edoardo Zanzani, Isotta Zanzani, Federica Angeli

\bigskip

Un ringraziamento speciale a tutta la mia famiglia che mi ha sopportato e supportato in questi anni disperati!

\bigskip

Powered by \Large\LaTeX\ \normalfont\& \Large\textbf{GitHub}

\bigskip

Andres Zanzani

\pagebreak

\section{Licenza}\index{Licenza} 

{{\scriptsize Permission to copy, modify and distribute the files collectively known as the System Reference Document 5.1 (“SRD5”) is granted solely through the use of the Open Gaming License, Version 1.0a. This material is being released using the Open Gaming License Version 1.0a and you should read and understand the terms of that License before using this material.

The text of the Open Gaming License itself is not Open Game Content. Instructions on using the License are provided within the License itself. The following items are designated Product Identity, as defined in Section 1(e) of the Open Game License Version 1.0a, and are subject to the Conditions set forth in Section 7 of the OGL, and are not Open Content: Dungeons \& Dragons, D\&D, Player’s Handbook, Dungeon Master, Monster Manual, d20 System, Wizards of the Coast, d20 (when used as a trademark), Forgotten Realms, Faerûn, proper names (including those used in the names of Spells or items), places, Underdark, Red Wizard of Thay, the City of Union, Heroic Domains of Ysgard, EverChanging Chaos of Limbo, Windswept Depths of Pandemonium, Infinite Layers of the Abyss, Tarterian Depths of Carceri, Gray Waste of Hades, Bleak Eternity of Gehenna, Nine Hells of Baator, Infernal Battlefield of Acheron, Clockwork Nirvana of Mechanus, Peaceable Kingdoms of Arcadia, Seven Mounting Heavens of Celestia, Twin Paradises of Bytopia, Blessed Fields of Elysium, Wilderness of the Beastlands, Olympian Glades of Arborea, Concordant Domain of the Outlands, Sigil, Lady of Pain, Book of Exalted Deeds, Book of Vile Darkness, Beholder, gauth, Carrion Crawler, tanar’ri, baatezu, Displacer Beast, Githyanki, Githzerai, Mind Flayer, illithid, Umber Hulk, Yuan-ti.

All of the rest of the SRD5 is Open Game Content as described in Section 1(d) of the License. The terms of the Open Gaming License Version 1.0a are as follows:

OPEN GAME License Version 1.0a The following text is the property of Wizards of the Coast, LLC. and is Copyright 2000 Wizards of the Coast, Inc ("Wizards"). All Rights Reserved.

1. Definitions: (a)"Contributors" means the copyright and/or trademark owners who have contributed Open Game Content; (b)"Derivative Material" means copyrighted material including derivative works and translations (including into other computer languages), potation, modification, correction, addition, extension, upgrade, improvement, compilation, abridgment or other form in which an existing work may be recast, transformed or adapted; (c) "Distribute" means to reproduce, License, rent, lease, sell, broadcast, publicly display, transmit or otherwise distribute; (d)"Open Game Content" means the game mechanic and includes the methods, procedures, processes and routines to the extent such content does not embody the Product Identity and is an enhancement over the prior art and any additional content clearly identified as Open Game Content by the Contributor, and means any work covered by this License, including translations and derivative works under copyright law, but specifically excludes Product Identity. (e) "Product Identity" means product and product line names, logos and identifying marks including trade dress; artifacts; creatures characters; stories, storylines, plots, thematic elements, dialogue, incidents, language, artwork, symbols, designs, depictions, likenesses, formats, poses, concepts, themes and graphic, photographic and other visual or audio representations; names and descriptions of characters, Spells, enchantments, personalities, teams, personas, likenesses and Special abilities; places, locations, environments, creatures, Equipment, magical or supernatural Abilities or Effects, logos, symbols, or graphic designs; and any other trademark or registered trademark clearly identified as Product identity by the owner of the Product Identity, and which specifically excludes the OPEN Game Content; (f) "Trademark" means the logos, names, mark, sign, motto, designs that are used by a Contributor to Identify itself or its products or the associated products contributed to the Open Game License by the Contributor (g) "Use", "Used" or "Using" means to use, Distribute, copy, edit, format, modify, translate and otherwise create Derivative Material of Open Game Content. (h) "You" or "Your" means the licensee in terms of this agreement.

2. The License: This License applies to any Open Game Content that contains a notice indicating that the Open Game Content may only be Used under and in terms of this License. You must affix such a notice to any Open Game Content that you Use. No terms may be added to or subtracted from this License except as described by the License itself. No other terms or Conditions may be applied to any Open Game Content distributed using this License.

3.Offer and Acceptance: By Using the Open Game Content You indicate Your acceptance of the terms of this License.

4. Grant and Consideration: In consideration for agreeing to use this License, the Contributors grant You a perpetual, worldwide, royalty-free, nonexclusive License with the exact terms of this License to Use, the Open Game Content.

5.Representation of Authority to Contribute: If You are contributing original material as Open Game Content, You represent that Your Contributions are Your original Creation and/or You have sufficient rights to grant the rights conveyed by this License.

6.Notice of License Copyright: You must update the COPYRIGHT NOTICE portion of this License to include the exact text of the COPYRIGHT NOTICE of any Open Game Content You are copying, modifying or distributing, and You must add the title, the copyright date, and the copyright holder's name to the COPYRIGHT NOTICE of any original Open Game Content you Distribute.

7. Use of Product Identity: You agree not to Use any Product Identity, including as an indication as to compatibility, except as expressly licensed in another, independent Agreement with the owner of each element of that Product Identity. You agree not to indicate compatibility or co-adaptability with any Trademark or Registered Trademark in conjunction with a work containing Open Game Content except as expressly licensed in another, independent Agreement with the owner of such Trademark or Registered Trademark. The use of any Product Identity in Open Game Content does not constitute a Challenge to the ownership of that Product Identity. The owner of any Product Identity used in Open Game Content shall retain all rights, title and interest in and to that Product Identity.

8. Identification: If you distribute Open Game Content You must clearly indicate which portions of the work that you are distributing are Open Game Content.

9. Updating the License: Wizards or its designated Agents may publish updated versions of this License. You may use any authorized version of this License to copy, modify and distribute any Open Game Content originally distributed under any version of this License.

10. Copy of this License: You MUST include a copy of this License with every copy of the Open Game Content You Distribute.

11. Use of Contributor Credits: You may not market or advertise the Open Game Content using the name of any Contributor unless You have written permission from the Contributor to do so.

12. Inability to Comply: If it is impossible for You to comply with any of the terms of this License with respect to some or all of the Open Game Content due to statute, judicial order, or governmental regulation then You may not Use any Open Game Material so affected.

13. Termination: This License will terminate automatically if You fail to comply with all terms herein and fail to cure such breach within 30 days of becoming aware of the breach. All sublicenses shall survive the termination of this License.

14. Reformation: If any provision of this License is held to be unenforceable, such provision shall be reformed only to the extent necessary to make it enforceable.

15. COPYRIGHT NOTICE Open Game License v 1.0a Copyright 2000, Wizards of the Coast, LLC. System Reference Document 5.1 Copyright 2016, Wizards of the Coast, LLC.; Authors Mike Mearls, Jeremy Crawford, Chris Perkins, Rodney Thompson, Peter Lee, James Wyatt, Robert J. Schwalb, Bruce R. Cordell, Chris Sims, and Steve Townshend, based on original material by E. Gary Gygax and Dave Arneson.}

\pagebreak

{\footnotesize
	\section{Changelog}\index{Changelog}

	1.0.1 aggiunto changelog, aggiunti ambiti agli dei con vantaggi e svantaggi, modificato da +2 a +1 bonus da riselezionare essenza

	1.0.2 layout, layout, layout, prime correzioni ad ambiente, prime correzioni a masterizzare

	1.0.3 perfezionati e sistemati dei, aggiornate razze,

	1.0.4 rivisti costi base magia, aggiornate descrizione magia

	1.0.5 aggiornati dei

	1.0.6 vari errori di scrittura, aggiunti kender (al posto di halfling), aggiunto sgambetto, modificato combattimenti a due mani, chiarimenti, dettagli su incanalare energia ed imposizioni delle mani, Fare infuriare, riviste divinatà, armature

	1.0.7 aggiornati dettagli divinità, modificati penalità al CM per Armature

	07/06/2018 STAMPA

	1.0.8 correzioni divinità, aggiornati Drow, aggiunto CRP, gestione attacchi multipli,

	1.0.9 rifiniture, precisazione su armature, iniziativa carte

	1.0.10 modificati bonus e gestione armature, layout, riordinati e sistemati termini di base, aggiunti scudi alla tipologia di arma, modificati costi essenza distruzione su elementi, chiarimenti su magia, layout, piccole correzioni e chiarimenti

	15/10/2018 STAMPA

	2.0.10 nell'iniziativa con carte si pescano carte in base al valore di Intelletto se si lanciano Essenze, chiarimenti, più magie nello stesso round, rimosso danno da sanguinamento in distruzione e corretti riferimenti a potenze ed Agilità, precisazione su movimento

	2.0.11 sistemato elenco armi semplici, aggiunti nuovi svantaggi, aggiornato incanalare energia, +1 caratteristiche ogni 4 livelli, reso più chiaro combattimento a due mani

	21/11/2018 STAMPA

	2.0.12 modificata e semplificata magia, ridotti costi essenza, rimosso residui di usare oggetti magici, modificato check concentrazione, piccole correzioni, aggiunto link a scheda online, modificato valori base delle caratteristiche, 0 è normale, 1 buono, 2 ottimo , 3 eccezionale, precisazioni sul linguaggio, avviato Monster Manual, aggiunta levitazione e volo, aggiunta abilità ferocia, aggiunta breve descrizioni di yeru e portali, modificata iniziativa

	2.0.13 aggiunti tratti, ripristinate e modificati divinità di Codex, modificato recupero pf dopo nottata, modificato il sistema di modificatore in base +1d6 o -1d6 a seconda di un bonus o malus con lo scopo di semplificare e ridurre tanti modifiche ad un vantaggio o svantaggio di dado, migliorata la terminologia dimagus e incatatore e devoto, aggiornate tabelle abilità armi specifiche, aggiunte tabelle anelli magici e bastoni magici

	24/12/2018 STAMPA

	2.0.15 rilettura e correzioni varie, aggiunta versione semplificata della magia (scelta 2 spell), aggiunta parte come creare il personaggio, le abilità si prendono ogni livello dispari, precisazione sulle razze ed patroni collegati, piccoli aggiornamenti su tridente e sassi, aggiunte frasi di inizio capitolo (wip), rimosso vincolo dei due tratti per essere un incantatore, corretto livello di potere nei bonus dati dall'affinità di tratto 1=10, 2=13, 3=15, 4=17, corretta abilità trattenere il respiro

	2.0.16 correzioni varie, sostituito competenza armi e competenza difesa dove usata genericamente con Tiro per colpire e Difesa, aggiornata scheda, aggiornata scheda su manuale, aggiunto svantaggio seguire la legge, aggiornato svantaggio seguire il chaos, precisato e modificato collegamento tratti, Patrono e magia adesso è vincolante scegliere un Patrono per avere la magia

	2.0.17 piccole correzioni, ricalibrato livelli di potere, modificato esplosione del 6 nella magia, rimossi i kender, semplificata l'iniziativa, aggiunti dettagli in cappello cosmologia, aggiunta abilità lo scudo è mio amico, rivisti i tratti, rivisti assegnazione tratti a dei, rivisto costo contingenze, sostituito la dizione Tiro salvezza su volontà in Tiro salvezza su Arbitrio per evitare di avere una volontà come caratteristica e come nome del tiro salvezza, aggiornata la scheda con tiro salvezza su arbitrio, piccola correzione su anello dell'ariete, modificato Colpi Poderosi limitandone l'esplosione, chiarimenti e correzioni sulla magia, modificate alcune capacità dati dai gruppi d'arma (in particolare spade, spade e scudo) , aggiornate descrizioni e poteri abilità, aggiornata e dettagliata tabella creazione e distruzione introdotto concetto di cubo base

	2.0.18 semplificato combattimento a due mani, aggiunte Abilità collegate al Energia Psichica, sostituito termine congiurazione con convocazione, adesso la convocazione convoca in base ai CR non agli HD, anche charme influenza sui CR

	01/02/2019 STAMPA

	2.0.19 modificata come l'essenza di difesa può essere usata come controincantesimo, dettaglio durata su essenza creazione, modficata iniziativa adesso è Agilità o Intelletto + CD, modificata Essenza Attacco: adesso è basata su Intelletto, modificata iniziativa: Dal più veloce al più lento c'è la risoluzione delle dichiarazioni e delle azioni, Fiancheggiare: corretto bonus a +2 al compire o difesa. default al colpire, Combattimento a due armi. Senza competenza hai un -4 al colpire su ogni arma, con la competenza il -2 rimane solo sulla secondaria, la Difesa adesso è di base 11, modificato energia chi in Energia Interiore, un incantatore può formulare nel giorno un numero di Essenze pari a CM+3. Per ogni critico (esplosione di magia) che ottiene il numero di Essenze lanciabili nel giorno diminuisce di 1 a causa del grande stress fisico sostenuto, modificato il livello di potere, si parte da 11 ed aumento di 3 per ogni livello di potere (\textless11, 13, 16, 19, 22, 25, 28\ldots ), Essenza Illusione adesso è basata su magnetismo, corretto Consapevolezza su volonta'

	2.0.20 aggiunta Resistenza alla Magia, piccole correzioni, aggiornata tabella energia psichica

	13/02/2019 STAMPA
	2.1.0 cambiata gestione del movimento adesso basata su raggi di effetto, aggiornata la scheda, layout, correzioni, rifacimento indice, aggiornata scheda, avanzamento indice, correzione doppi spazi, correzioni, modificato metodo di ricarica dei bastoni magici, modificato livello di punti ferita per morte 10+3*pot, aggiunto chiarimento su applicazione veleno ad armi, aggiornati vantaggi, corretto sistema magia opzionale, abbassato base difesa a 10, eliminati alcuni residui di distanza in metri, corrette residui citazione quadretti 2.1.1. aggiornata magia con gestione differenziata dei target influenzabili, piccole correzioni su essenza attacco, creata tabella collegamento valore tratto bonus ricevuti, layout su patroni , aggiornata scheda, aggiornata parte di conversione mostri, correzioni, dettagli su check contrapposti, aggiunta magia basata su carte, specificazione su movimento, ulteriori chiarimenti su movimento, aggiornate armatura per nuova gestione movimento, correzioni su morte, chiarimenti su Seguace e Devoto, dettagli su neve e nebbia connessi a movimento

	07/03/2019 STAMPA

	2.1.2 modificato essenza cura per livello potere \textless11, gestione sorpresa reciproca, precisazioni sulla magia, in caso di critico magico si può lanciare un incantesimo in più al giorno, correzioni, modificato numero massimo di essenze al giorno pari a cm/2+3, modificati parametri di conversione mostri, abbassato il bonus di difesa delle armature

	2.1.3 correzioni, aggiunta abilità furia, passare da distanza corta a mischia costa 1 azione

	2.2.0 tolta competenza difesa, correzioni, precisazione su bonus dovuti al valore del tratto, senza competenza il check e’ 1d6, rimossa scurovisione adesso rimane solo la visione crepuscolare, se una essenza si risolve con un tiro per compire in mischia si usa il valore di CM al posto di CA, aggiunta categoria armi versatili dove si puo’ usare agilita’ al posto della forza per tc e danno, le capacita’ del famiglio si basano non sul livello del padrone ma sulla sua competenza magica, aggiunto dettaglio a prova di concentrazione in caso di distrazione, sostituito riferimento da bonus di circostanza a bonus, aggiornata procedura calcolo distanze e movimenti, modificate azioni per round rese moltopiu’ snelle ed immediate nella scelta.

	25/07/2019 STAMPA

	2.3.0 aggiungo gestione afferrare, aggiunto gestione fare cadere, dettagli su prono, corretti numerosi riferimenti ad azione standard e di round completo, rimosso distanza generica e ripristinati movimenti a metri e quadretti,correzioni al layout, modificata gestione carica adesso è possibile ad alti livelli fare più attacchi, dettaglio su bonus e malus quando usare dadi e quando valori, modifiche su prendere la mira adesso hai un bonus all'iniziativa del 4' round, modifiche a maestria del combattimento, modificata finta l'effetto è fino alla fine del round, fare cadere specificato che è un prova contrapposta di tiri salvezza potenza/agilità, dettagli su lista d'armi, aggiornate e riviste abilità per lista armi, rimossi ultimi riferimenti a CD, aggiornate quasi tutte le abilità, chiarimenti su assegnazione cm a essenze, rimosso Assenza come suddivisione, aggiornato cubo base a cubo di 1m di lato, riviste essenze fino a distruzione .essenza movimento dettagli su tiro salvezza, correzione su esempio movimento, precisazione su illusione ed allarmi e contingenze, essenza movimento precisazioni importanti su caduta oggetti, modifiche importanti su essenza protezione rimossa la possibiltà di rimuovere stati, rivista essenza trasformazione su creature, aggiustamenti a vantaggi, aggiustamenti a svantaggi, correzioni in cosmologia, correzioni su poteri per valore tratto comune, chiarimenti su armature, aggiornata copertura, rifatta parte relativa a invisibilità adesso è più organico, introdotto concetto di reazione, impostata versione a 2.3.0, cecchino reso omogeneo le penalità al colpire, modificate qualche abilità perché usino reazione o immediata, aggiornamenti layout, modificata penalità per portare armatura senza competenza, adeguato penalità movimento armatura a nuova gestione, aggiornate tabelle movimento, modificati valori capacità di trasporto, aggiornata procedura creazione oggetti magici, corretto riparare oggetti magici, aggiornati triboli, modificato armi in argento alchemico, scritto meglio armi in adamantio, aggiunta spiegazione su importanza declamazione essenza, rimosso riferimenti residui a classe armatura, rimossi riferimenti ad azione gratuita (adesso è immediata o di reazione), aggiunto abilità Segugio, aggiunta abilità Esperto, correzioni varie, inserite regole per attacco di opportunità, adeguate Abilità alla gestione delle Azioni, aggiornate essenze, specificato costo in azioni per abilità legate ai tratti, aggiornati poteri dati dai tratti per trattare con le azioni, corretti valori difesa delle armature, aggiunto dettagli se prova fallisce di 10 o piu', gestione bonus recitazione, aggiornato sommario, corretti vari riferimenti ad azione generica, correzione nei termini comuni su Difesa,chiarimenti su attacchi multipli usando più azioni singole di attacco, chiarimenti su riuscita o fallimento tiro salvezza, indicazioni su fallimento o riuscita critica di un tiro salvezza, indicazioni di successo e fallimento critico nella descrizione delle essenze, aggiornata struttura indice analitico, chiarimenti su attacchi multipli usando più azioni singole di attacco, chiarimenti su riuscita o fallimento tiro salvezza, indicazioni su fallimento o riuscita critica di un tiro salvezza, indicazioni di successo e fallimento critico nella descrizione delle essenze, aggiornata struttura indice analitico

	19/09/2019 STAMPA

	2.3.1 layout, piccole correzioni, aggiornamento indice, correzioni in termini comuni, pagine al centro,aggiunta informazione su gestione personalizzata dei tratti,chiarimenti su prendere 10, chiarimenti su fornire aiuto, sistemazione layout, chiarimenti su scelta linguaggi, corretta lingua scritta da Terran, precisazione su più attacchi eseguiti con singoli attacchi, specificato quando azioni di reazioni/immediate possono intervenire, modificato bonus per differenza di taglia da +4 a +2, chiarimenti su prono, in caso di svenuto dopo 5 tiri si prende il risultato con più successi,aggiunta morte per danni temporanei, aggiunta abilità Incantatore Combattente, chiarimenti su linguaggio per Fare Infuriare, chiarimenti su Tentare Essenza con impedimenti, chiarimenti su prova magia, chiarimenti su Eludere, chiarimenti su Forgiato nella furia, chiarimenti su kensai, chiarimenti su proseguire, correzioni, ridotto il bonus alla prova di senso trappola, chiarimenti su segugio, reso più italiano il manuale (tolti alcuni termini inglesi), reso più chiaro il recupero della essenza a seguito di critico, aggiornata decrizione tipi di essenze, chiarimenti su penalità armature alla prove di agilità, corretti pesi armatura, dettagli su caduta, chiarimenti su caduta in vulcano, corrette ed aggiornate pozioni, chiarimento su dipendenza, aggiornata e corretta descrizione veleno da contatto, modificate probabilità di auto avvelenarsi applicando il veleno, correzioni varie

	14/01/2020 STAMPA

	2.3.2 chiarimenti su fumble (tirare 3 volte 1 al TC) , modificato limite per attacco multipli (-4 invece che -5), colpo potente modificat da dove si può togliere il valore al colpire, allineato bonus/malus in maestria del combattimento a -4, dettaglio su critici su creature invisibile, chiarimenti su termini comuni, cambiare svantaggi di razza, dettagli su scelta tratti, la dc dei ts richiesti è segreta, con intelletto 2+ una lingua in piu', chiarimenti su quando è possibile prendere il 18, chiarimenti su quando eseguire reazioni ed azioni immediate, chiarimenti su attacchi con essenze, chiarimenti su considerare un 1d6 con un +-4, chiarimenti su disarmo, chiarimenti su spingere l'avversario, piccoli aggiornamenti su lista armi, aggiunta abilità Guerriero dell'Essenza, effetti minimi di lanciare un essenza con impedimento, rinominata abilità Incantatore Combattente in Incantatore Prudente, migliorata descrizione abilità opportunista, scritto meglio passo tattico, specificato che percettivo si può prendere più volte ma il bonus aumenta solo di +1, persona veramente malvagia costa 1 azione, chiarito meglio quando si può usare rappresaglia, rinominato salto e schivo in toccata e fuga, dettagliato meglio schivata prodigiosa, piccole correzioni nelle abilità, in famiglio corretto Livello del padrone con CM del patrone nella tabella delle capacità del famiglio, modificato scrutare su famiglio, corretto modificatore di intelletto del famiglio

	3.0.0 passaggio a Latex, varie correzioni

	3.0.1 aggiunta possibilità di cambiare una Abilità scelta, cambiato font, specificato che la specializzazione magica si usa anche nelle prove di concentrazione, aggiunti punti esperienza per oro guadagnato, aggiunti suggerimenti per il Narratore, aggiornati vantaggi

	3.0.2b tornato alla vecchia impaginazione senza template dnd, voglio cercare in mio stile, gestito ingombro non dato dal peso degli oggetti ma da un valore di ingombro relativo, chiarimenti su descrizione competenze, dettagli su competenze, modificato stile della sezione, piccole dimenticanze in ingombro lasciati in kg, prova di pronto soccorso per ridurre sanguinamento, continuo lavoro su Ingombro, piccole correzioni, modificato template per sezioni, chiarimenti per svenuto, fixato hyperlink capitoli mettendo titlesec prima di hyperref, messe " al posto di '', aggiornati elfi resi in pò diversi dai canoni, aggiornamenti e chiarimenti su invisibilita'

	3.0.2c riformattato i quotebox, aggiunto capitolo su draghi, aggiunte informazioni su cicli millenari, aggiornato (v2)script di git perche' chieda esplicitamente il commento alla commit, rimosso andare veloci;chiarimenti su quadretto e mezzi metri, precisazioni su afferrare, modifiche su attacchi multipli, corretto valore di ingombro e penalita'

	3.0.3 inserito drago giallo, rinominato drago viola in drago porpora, usato tcolorbox per maggiore compatibilita' con pandoc, sistemato errore in conversione pandoc verso odt/doc, chiarimenti su AoE

	15/06/2020 STAMPA

	19/07/2020 layout, corretto attacchi multipli

	3.0.4 il minimo di punteggio preso nelle competenze al passaggio di livello passa a +1 da 0,il punteggio massimo di una competenza base e' pari al lv+1, aggiunta informazione su come apprendere una nuova competenza, layout, spostato muoversi silenziosamente in criminalita', aggiunta a Sopravvivenza la specifica creature naturali ed ad Arcano creature Magiche, specificato che il bonus di +4 all'iniziativa si ha quando la propria arma ha una portata maggiore dell'arma dell'avversario, specificato che il bonus di iniziativa non si applica per le armi da lancio, aggiunto tirare 3 volte 6, in carica si puo' percorrere il doppio del proprio movimento, dettagli su danno da controcarica, dettagli su azione di disingaggio, aggiunta possibilita' di disingaggio da creature piu' grandi, aggiornata versione alla 3.0.4, aggiornamento su Stato Indifeso, aggiornata versione scheda, riassegnazione punti CA, dettagli su tiro per colpire per essenza, nota sulle razze, il tiro per colpire dell'essenza e' sempre a tocco, chiarimenti su manifestazioni offensive di essenze non di attacco, correzioni, chiarimenti su check concentrazione, corretta prova di dungeon in sopravvivenza, chiarimenti su ingombro e trasporto, modificati massimali per ingombro trasportabile, aggiunta regola per poter spingere in piu' persona, modificata manovrabilita' di creature che volano ma non sanno farlo (adesso e' scarso), corretta tabella bonus manovrabilita' volare, chiarimenti su invisibilita', reso opzionale il recupero da 0 pf,chiarimenti su invisibilita', rimossa energia radiosa e' ridondante con energia positiva, aggiunta come forma di attacco il vuoto e riordinato le altre forme mettendo elettricita' neutrale, aggiornate elementi favoriti per le divinità, abbassato il bonus di essenza favorita a +2, aggiornata scheda, modificati costi di Durata delle Essenze, layout, chiarimento su danno da Luce e Vuoto

	01/10/2020 STAMPA
	
    3.0.5 aggiunto riferimento al razzismo, aggiornate razze, layout, riportato malus per essenze limitate a -4, aggiornata versione a 3.0.5, aggiunta opzione per gli dei antichi,  rifacimento indice, aggiunte/sistemate voci di indice, ritornato malus a -2, aggiornato famiglio corretta difesa,aggiornato famiglio, inserite distanze per usare i poteri, aggiunta pozioni generiche, aggiornamento divinità antiche, aggiunti simboli sacri divinità antiche, rifatta tabella aspetti generici, specificata durata pozioni generiche, dettagli su dei antichi,l'essenza favorita di Zarkor e' alterazione e non trasformazione, precisazioni su costo ed effetti magici, esempi di utilizzo di Essenza, aggiornamenti su magia, modificata lista elementi e creature, mantenere la concentrazione di una essenza costa 1 azione, ristrutturata sezione dei potenziamenti magici e messi in un unica tabella, Cura agisce su Creature ed Elementi (oggetti), rimosse Vita e Morte dai concetti, aggiornato indice, rifatta tabella elementi creature,chiarimenti su difficolta' data da area effetto e Essenze con cui base, dettagli sui potenziamenti delle essenze, dettagli su effetti essenza in linea  ed a cono, chiarimenti su riuscire e fallire nella magia, layout e precisazioni, modificata prova di concentrazione, chiarimenti su Essenza Alterazione, una Essenza non di attacco usata per attaccare causa danno con 2 livelli inferiori di LP, essenza charme inserito modificatori emotivi, chiarimenti su Essenza Creazione, esempio di contingenza su cura, dettagli su costo pozioni naturali, dettagli su modifica empatia con essenza charme, aggiornata tabella charme per cambio amichevole/ostile, aggiornata tabella effetti distruzione su creature, aggiornata essenza movimento, modificato uso di tabular con le misure L e usa di tabularx X per una esportazione su word delle tabelle almeno possibile, rimossi spazi di formattazione ridondanti, ogni singola pozione naturale puo' essere assunta una volta al giorno le successive non hanno effetto, rifatta tabella armi dovrebbe essere meglio formattata, aggiornato tabella equipaggiamento armi con armi da carceriere, rivista lista armi da carceriere, perfezionata tabella equipaggiamento armi eliminando i doppioni ed i raggruppamenti, modificato il danno per il pugno/calcio, chiarimenti su utilizzo lista arma pugno/calcio, aggiornata tabella veleni naturali, aggiunto veleno Mistura Rossa, allineate tabelle pozioni e veleni reso piu' chiaro il tipo di pozione e veleno, separate le droghe e creata tabella apposita, rimosso occultamento e sostituito con copertura, gestione alternativa del danno critico, chiarimenti su lista armi archi, rimosso dal danno critico il valore di potenza si applica, chiarimenti su invibilita', essenze cumulate ovvero piu' Essenze nello stesso round, mancava la descrizione dell'abilita' ferocia, valori non chiari su rompere porte, rilettura competenze piccole correzioni, prima bozza di layout, spiegazione sulle lingue elementali, dettagli su altre lingue, spiegazione su lingue speciali, chiarimenti su pronto soccorso, chiarimenti su lista armi archi e attacco aggiuntivo, ulteriore lettura su azioni in combattimento, chiarimenti su punti liste d'arma, modificata e ridotta la progressione dei punti necessari per prendere i vantaggi dalle liste d'arma, rinominata l'abilita' Cecchino in Occhio di Falco, rinominata l'abilita' Cecchino in Occhio di Falco, chiarito colpo paralizzante, aggiunto terzo livello combattere alla cieca, gestito terzo livello con arma secondaria, chiarito conteggi bonus e malus multiattacco con arma secondaria, cambiato nome da katana a uchigatana, dettagli su come gestire una pergamena (lettura, comprensione..), chiarimenti su raggio psichico, aggiunto potenziamento su raggio psichico e colpo psichico, potenziamento energia chi possibilita' di recuperare energia chi ogni ora, modificato il costo dell'azione per guerriero dell'essenza, immunita' ai veleni da un +4 contro veleni magigi e non immunita', modificata area di effetto di Incanalare energia portata a 3 metri, dettagli su montagna umana, dettagli su passo tattico, varie correzioni di layout ad Abilita', chiarimenti su Tiro Rapido, modifiche a kensai adesso il bonus e' sulla iniziativa o TC, rinominata l'Abilita' Kensei in Iaijutsu, chiarimenti su Magie Efficaci, aggiunta abilita' Passo Veloce, rifatta Abilita' Schivare trappole, altri dettagli minori su abilita', chiarimenti su famiglio, modificato trasmettere essenze a contatto (famiglio), chiarimenti su vedere attraverso il famiglio, piccoli chiarimenti su assorbimento danni armatura,  rimosso concetto di arma e armatura perfetta, sostituito riferimento a carisma con magnetismo in abiti, alcuni riferimenti in indice erano minuscoli, inseriti pesi in area di effetto,  La trasformazione di Elementi o Energia e' permanente ed ha difficolta' come Durata 8, rivista completamente parte della magia sul calcolo della difficolta' data dalla Durata/Massa e Volume adesso e' piu' chiaro come computare, rese Creazione, Distruzione, Trasformazione permanenti come effetti, aggiornata tabella delle difficolta' base aggiungendo massa/volume, aggiunto include con esempi magie, specificato che la Creazione non puo' creare qualcosa di magico, chiarimenti su trasformazione, aggiunta casistica di trasformazione elemento in elemento magico, aggiunto resistenza al danno e chiarimenti su riduzione del danno
    
    3.1.0 aggiunta dell'opzione magica basata su incantesimi di DnD 5e, tolta durata concentrazione da Invibilita', dai Muri, dalle Nubi, Oscurità, passare senza traccia, paura, pelle di corteccia, resistenze, volare e molti tanti altri incantesimi, aggiunta vulnerabilita' al danno, aggiunto chiarimento punti ferita temporanei, aggiunto chiarimenti per afferrato, cambiata licenza lasciato solo OGL 5.1, aggiunto capitolo sui mostri, inserito allineamento, chiarimenti per utilizzo tratti ed allineamento, inseriti i piani, diverse correzioni minori e proseguimento lavori su mostri, aggiunta abilita' scacciare i non morti, aggiunta scurovisione, dettagli su statistiche basse, aggiunte scuole di magia agli incantesimi ed ai patroni e divinita', rimossa magia 5ed
}

\pagebreak

{\scriptsize 
\printindex}

\end{document}
