\subsubsection{Esempi di formulazioni Essenze}\index{Esempi Essenze}

Questi sono alcuni esempi di Essenze formulate.

Vengono presentate cosi' che sia piu' facile compredere il sistema magico e fornire una base di partenza per le vostre creazioni


Questo il template di base


\flushleft \textbf{Nome Magia}: \\ \index{Esempio Magia}
\textbf{Verbo}: \\
\textbf{Nome}: \\
\textbf{Tempo di Lancio} (): \\
\textbf{Distanza} (): \\
\textbf{Area di Effetto}/\textbf{Massa/Volume} (): \\
\textbf{Durata} (): \\
\textbf{Difficolta' base}: \\
\textbf{Descrizione}: \\


\flushleft \textbf{Nome Magia}: Acqua benedetta\\  \index{Acqua benedetta}
\textbf{Verbo}: Trasformazione\\
\textbf{Nome}: Elementi\\
\textbf{Tempo di Lancio} (0): 2 Azioni\\
\textbf{Distanza} (0): tocco\\
\textbf{Massa/Volume} (2+6): tre boccette di acqua, per un totale di circa 500ml. Il +6 e' dato dal fatto che si vuole trasformare l'acqua in un elemento magico\\
\textbf{Durata}: permanente (8)\\
\textbf{Difficolta' base}: 16\\
\textbf{Descrizione}: Con questa formulazione trasformi fino a mezzo litro d'acqua in acqua benedetta o maledetta (a seconda del Patrono). Riuscire nella prova di Competenza Magica con un punteggio oltre 16 non cambia il risultato finale.\\


\flushleft \textbf{Nome Magia}: Trasforma Pietra in Carne \\ \index{Esempio Magia}
\textbf{Verbo}: Trasformare\\
\textbf{Nome}: Creature\\
\textbf{Tempo di Lancio} (0): 2 Azioni\\
\textbf{Area di Effetto} (1): una creatura\\
\textbf{Distanza} (3): entro 50m\\
\textbf{Durata} (8): permamente\\
\textbf{Difficolta' base}: 12 \\
\textbf{Descrizione}: In base al risultato ottenuto verificare se si e' raggiunto un Livello di Potere sufficiente a trasformare l'obiettivo in Pietra.
Ricordarsi che in caso di creatura magica si passa al LP successivo.\\


\flushleft \textbf{Nome Magia}: \\ \index{Esempio Magia}
\textbf{Verbo}: \\
\textbf{Nome}: \\
\textbf{Tempo di Lancio} (): \\
\textbf{Distanza} (): \\
\textbf{Area di Effetto} (): \\
\textbf{Massa/Volume} (): \\
\textbf{Durata} (): \\
\textbf{Difficolta' base}: \\
\textbf{Descrizione}: \\
