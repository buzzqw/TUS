\documentclass[a4paper,twoside,openany]{book}
%\documentclass[a4paper,draft,twoside,openany]{book}
\usepackage{quoting}
\usepackage{tcolorbox}
\usepackage{tikz}
\usetikzlibrary{shadows}
\usepackage{multicol}
\usepackage{tocloft}
\usepackage{lmodern}
\usepackage{caption}
\usepackage[utf8]{inputenc}
%\usepackage[utf8x]{inputenc}
\usepackage[T1]{fontenc}  %\usepackage[B1,T1]{fontenc}
\usepackage{setspace}
\usepackage[a4paper]{geometry}
\geometry{verbose,tmargin=2cm,bmargin=2cm,lmargin=2cm,rmargin=2cm}  %std
\setcounter{secnumdepth}{-1}
\usepackage{booktabs}
\usepackage{url}
\usepackage[english]{babel}
\usepackage{setspace}
\usepackage{graphicx}

\usepackage{amssymb}
\usepackage{makeidx}
\usepackage{multirow}
\usepackage{titlesec}
\usepackage[unicode=true, bookmarks=true,
pdftitle={OBSS - Old Bell School System},pdfauthor={Andres Zanzani},
breaklinks=false,pdfborder={0 0 1},backref=section,colorlinks=false]
{hyperref}
\hypersetup{colorlinks=true,linkcolor=blue,pdfcreator={LaTeX}}
\usepackage{bookmark}
\usepackage{yfonts}
\usepackage{lettrine}
\usepackage{calligra}
\renewcommand{\LettrineFontHook}{\calligra}
\usepackage{accanthis}
\usepackage{auncial}
\usepackage{fontspec}
\usepackage{ragged2e}

\setmainfont[Path=../altro/fonts/,BoldItalicFont=DejaVuSerif-BoldItalic.ttf,ItalicFont=DejaVuSerif-Italic.ttf,BoldFont=ReadexPro-bold.ttf,Ligatures=TeX,Scale=0.94]{ReadexPro-Regular.ttf}

\usepackage{wrapfig}
\usepackage{fancyhdr}
\usepackage{tcolorbox}
\tcbuselibrary{skins}
\tcbset{colback=brown!10, fonttitle=\scshape}
\usepackage{imakeidx}
\usepackage{cancel}

\def\CountIndexOccurrences#1{%
\expandafter\newcount\csname #1\endcsname%
\expandafter\newcount\csname #1\endcsname%
\def\indexentry##1##2{\expandafter\advance\csname #1\endcsname 1}%
\IfFileExists{#1.idx}{\input{#1.idx}}{}%
}
\CountIndexOccurrences{OBSS}
\CountIndexOccurrences{Incantesimi}
\CountIndexOccurrences{Mostruario}
\CountIndexOccurrences{OggettiMagici}
\def\TotalBox#1{%
\fbox{Ci sono \expandafter\the\csname #1\endcsname\ voci in questo indice}\par}

\makeindex[columns=3, title=Indice Analitico, intoc=true]
\makeindex[columns=3, name=Incantesimi, title=Lista degli Incantesimi, intoc=true]
\makeindex[columns=3, name=Mostruario, title=Lista dei Mostri, intoc=true]
\makeindex[columns=3, name=OggettiMagici, title=Lista degli Oggetti Magici, intoc=true]

\usetikzlibrary{shapes.misc,calc}
\definecolor{lightgray}{gray}{0.95}

\usetikzlibrary{shapes.misc,calc}
\definecolor{lightgray}{gray}{0.95}

\usepackage{fancyhdr}
\pagestyle{fancy}
\fancyhf{}
\fancyhead[LE,RO]{\leftmark}
\fancyhead[RE,LO]{}

\fancyfoot[C]{\thepage}
\renewcommand{\sectionmark}[1]{\markboth{#1}{}}

\usepackage{xltabular}
\usepackage{tabularx}
\usepackage{pdfpages}
\usepackage{hyperref}
\usepackage{tikz}
\usepackage[absolute,overlay]{textpos}
\usepackage{etoolbox}

\raggedbottom

\usepackage{array}
\newcolumntype{L}[1]{>{\raggedright\let\newline\\\arraybackslash\hspace{0pt}}m{#1}}
\newcolumntype{k}[1]{>{\centering\let\newline\\\arraybackslash\hspace{0pt}}m{#1}}
\newcolumntype{R}[1]{>{\raggedleft\let\newline\\\arraybackslash\hspace{0pt}}m{#1}}
\newcolumntype{D}[1]{>{\centering}m{#1}}
\newcolumntype{M}[1]{>{\centering\arraybackslash}m{#1}}

\titleformat{\section}{\filcenter\huge\bfseries\accanthis}{\thesection}{1em}\textsc{}
\titleformat{\subsection}{\Large\bfseries\accanthis}{\thesubsection}{1em}\textsc{}
\titleformat{\subsubsection}{\normalsize\bfseries\accanthis}{\thesubsubsection}{1em}\textsc{}

\def\changemargin#1#2{\list{}{\rightmargin#2\leftmargin#1}\item[]}
\let\endchangemargin=\endlist

\setcounter{tocdepth}{3}

\newtcolorbox{Storytellere}{
enhanced, % enable advanced settings
%left = 3mm,
%width=0.45\textwidth,
left = 9mm, % pushes text away from the left edge by 10mm
sharp corners, % disables rounded corners
rounded corners = southeast, % "round" the bottom right corner
arc is angular, % make the "round" corner an angle
arc = 3mm, % controls corner cut
boxrule=0.6pt, % sets box line thickness
underlay={%
\path[fill=black] ([yshift=3mm]interior.south east)--++(-0.4,-0.1)--++(0.1,-0.2); % triangle
\path[draw=black,shorten <=-0.05mm,shorten >=-0.05mm] ([yshift=3mm]interior.south east)--++(-0.4,-0.1)--++(0.1,-0.2); % triangle edge
\path[fill=gray!50!black,draw=none] (interior.south west) rectangle node[brown!10]{\Huge\bfseries ?!} ([xshift=8mm]interior.north west);

},
drop fuzzy shadow % adds drop shadow
}

\newtcolorbox{enfasi}{
enhanced,
arc=5pt,
boxrule=0.3pt
}

\begin{document}

\def \versione {0.50} \fontsize{10}{12} \selectfont
\thispagestyle{empty} \tikz[remember picture, overlay] \node[opacity = 1] at (current page.center){\includegraphics[width = 21cm, height = \pdfpageheight]{copertina.png}}; \begin{textblock*}{20cm} (5.5cm, 7cm) \Huge {Old Bell School System} \\ \end{textblock*} \begin{textblock*}{22cm} (9cm, 8cm) \Large {\textbf{(OBSS)}} \\ \end{textblock*} \begin{textblock*}{20cm} (4cm, 27.5cm){\color{red} \calligra \Huge{Fantasy Adventure Game} \LARGE \textbf{v \versione}} \end{textblock*} \newpage ~\thispagestyle{empty} \newpage ~\thispagestyle{empty} %riga 131 %rimuovere questa riga per togliere la copertina

\bigskip
Dedicated to the only woman ever loved, the one who accompanies me every day in my dreams

Never give up on your dreams, persevere until they are real.

\vspace{\fill}
\begin{center} \textbf{\versione} - \today \end{center}
\thispagestyle{empty}


\newpage ~\thispagestyle{empty} %%\newpage~\thispagestyle{empty}


\pagebreak

\setcounter{page}{1}

\begin{multicols}{2}
{\small \tableofcontents{}}
\end{multicols}

\pagebreak

\section{Introduction}

\begin{changemargin}{0.3cm}{0.3cm} \begin{tcolorbox}[enhanced, arc = 5pt, boxrule = 0.3pt]{You can find out more about a person in one hour of play than in one year of conversation. (Plato)} \end{tcolorbox} \end{changemargin} \medskip

\smallskip

\begin{multicols}{2}
\lettrine[lines = 2, lhang = 0.33, loversize = 0.25, findent = 1.5em]{O}{BSS} is a narrative and collaborative RPG in which players create characters who will experience fantastic and mind-blowing adventures. The Storyteller will take care of unraveling the story and involving the characters. As in a storytelling game, each character will actively contribute to the story with their choices, decisions and actions.

If you are a gamer you will soon realize that it is not an easy world, things are not given away or simple. There will be more times that they will want to kill you and rob you of those who will want to offer you a drink.

Don't get too attached to your character, only the strong survive and the mighty rule. Show who you are and what you want. Do you want to be the shining hero in armor? he does, but a rat is more likely to survive.

In each case of your character you will be the one to decide everything, from the appearance, to the name, to his abilities and what he possesses. Will he want to be a heartthrob pirate or a shy knight, a steppe barbarian or a sorcerer? Gold, honor, treasure, looting, your character's adventures will be filled with choices and battles, revelry and ... whatever you want!

If you are the Storyteller instead you rule the world, history, adventure. Your role is to illustrate the scenario in which players move and make decisions. Will you lead them into the depths of the earth in search of the forgotten Tome of Atmos or to challenge the great Dragons for the crown of Omniscience?

Your task is not easy, use imagination, common sense and the main rule - have fun. When you are in difficulty, do not look for the precise rule, use your greatest ally: imagination, combine a pinch of sensibility and try to amaze the players. The aim is always and only one, to have fun together and grow, as players, as characters, as friends.

Beyond this manual you will also need some dice, the classics used in role-playing games.
Usually called d4, d6, d8, d10, d12, d20 they indicate a 4-sided die, the 6-sided die (of these you must have 3 or 4 at least!), The 8-sided data, the 10-sided one (usually are sold in pairs, in order to obtain the d100), the solitary 12-sided die and the everlasting 20-sided die. Whenever you are asked to roll a die it will be written with the notation XdZ, that is, I roll a die with Z faces X times. Ex. 4d6 tells you to roll the 6-sided die 4 times.

Even some miniatures may be necessary, otherwise even the gifts of snacks or chocolate eggs may be sufficient.

Inside this manual you will find everything you need, such as rules, to play, you (you) will need imagination, friendship, dice, a few sheets of paper and fun (sorry, chips and drinks are not included in the manual!)

Draw and use a map whenever the description and the situation need accurate details and precise positioning, otherwise close your eyes and use your imagination, let yourself be accompanied by the voice of the Storyteller and reconstruct the structures and locations in your mind.

Create and play the character that suits you best, that you feel most yours and make you have fun, do not look for the combinations of Skills and Abilities that give you more power otherwise sooner or later the character will get bored. The more you play, the more experience your character will gain and you will play him better too. The character will acquire wonderful items, shiny armor, flying weapons, gold, precious and jewels and who knows what else.

The Storyteller will take care to tell you how much experience your character has gained, based on how you played, how you collaborated in the group, how much you helped the group of players to have fun. It will keep you engaged in dangerous, often deadly, encounters, it will put a strain on your character and as a group you will be able, perhaps not always, to solve the intricate situations that the Storyteller has prepared. Remember that the Storyteller always has the last word in any discussion.

This manual is complete, that is, inside you will find everything (apart from the dice!) To start playing!

You will also find many rules, yet many situations will have to be handled using the first rule: have fun. Common sense, experience and trust in the Storyteller will solve any situation.

Whether you decide to be the Storyteller or you decide to play a character, you need to carefully read the following chapters.
It is important that you have a good knowledge of the basic rules and, above all, that you know where to look for anything when you need it!

\textit{For experts ...}

The OBSS project, Old Bell School System, was created to evolve DBS in a different way and make the magical system less frustrating. Moreover, the evolution will bring more and more simplifications and reductions of rules.

At the base is the dissatisfaction with playing the 5ed of the famous role-playing game. The 5ed flattens the characters too much and the rules system, however efficient it is, does not allow that diversification, even if often exaggerated, that you could have in Pathfinder.

I needed a middle ground, a game based on the d20 and OGL but that took a little bit of the best of what had already been created and added what I liked of the countless RPGs studied and played. Do not try to recreate the 5ed or Pathfinder classes, you will not succeed, nor does OBSS want to lend itself to this job! In OBSS the classes do not exist and the characters acquire depth and ability depending on what they learn to do. The skills are dictated by the chosen profession and do not reach exaggerated scores. The combat does not reach the epic complexity of Pathfinder or the flatness of the 5ed, but rather tries to be quick and tactical, effective and spectacular. The Golden Rules, the management of the critic, give that extra something that allows players to have fun every time a die is rolled.

The magic takes up the standard canons of the 5th but deeply revisited. A lot of spells have lost focus and the concept of spell enhancement using a higher slot is gone. The Golden Rules also apply to spells and this allows you to diversify the outcome, adding more tension, to each spell.

The approach to alignment is completely changed, becoming now a fundamental aspect for the construction of the character; no longer two skimpy letters (LB, CB ...) but a choice based on character, morals and ethics.
The deities, pardon Patron, have a dirtier and more direct role. Read them carefully, they are not the usual gods.
The monsters are the OGL ones of the 5ed, modified to be tougher as there is no longer "bounded accuracy", there are better results on the attack roll and on the saving throws, as rightly was in the 3.5ed.

The license is the trusted one of 5.1ed OGL which grants anyone to create and produce wonderful adventures and expansions for OBSS.

One last note. The system aims to be more lethal than 5ed, more wounds, more suffering, without the concept of short or long rest or hit point recovery based on hit dice. Enough characters heroes always and in any case.

Don't worry, good role and group plays and good dice will always guarantee excellent results, even in spells !. Participation and identification always.

Last but not least, OBSS refers to the OSR movement, it would like to be played according to those principles. Read the chapter \hyperlink{OSR}{Masterizzazione} (pag. \pageref{masterizzare}).

\medskip

\begin{center}
Buona lettura e Buon Divertimento!
\end{center}

\begin{flushright}
Andres Zanzani
\end{flushright}

\end{multicols}

\smallskip

\begin{changemargin}{0.3cm}{0.3cm}\begin{tcolorbox}
"Although the masculine form of appellation is typically used when listing the level titles of the various types of characters, these names can easily be changed to the feminine if desired. This is fantasy, what's in a name? In all but a few cases sex makes no difference to ability!" Gary Gygax, Advanced Dungeons \& Dragons Players Handbook
\end{tcolorbox}\end{changemargin}

\begin{changemargin}{0.3cm}{0.3cm}\begin{tcolorbox}
"May you make all your Saving Throws!" Frank Mentzer, Spring 1985. Master Player's Book
\end{tcolorbox}\end{changemargin}

\medskip

\begin{center}
\includegraphics[keepaspectratio,width=0.80\textwidth]{immagini/Dragon_by_Henry_Justice_Ford_grey.png}

\textit{The End of the Dragon - Henry Justice Ford}
\end{center}

%\begin{wrapfigure}{l}{0.2\textwidth}
%\centering
%\includegraphics[width=4cm]{Dragon_by_Henry_Justice_Ford.png}
%\end{wrapfigure}


%\begin{figure}[h]\centering\includegraphics[width=18cm]{Dragon_by_Henry_Justice_Ford.png}\end{figure}

\pagebreak{}

\subsection{Common term}\label{Termini Comuni}

\begin{multicols}{2}

\lettrine[lines=2, lhang=0.33, loversize=0.25, findent=1.5em]{I}{l}ist a few terms \index{Common terms} that you will find repeated several times in the book.

\textbf{Roundings}: \index{Roundings} always rounded down unless otherwise specified. Ex. 7/2 = 3, 9/4 = 2.

\textbf{Abilities}: \index{Abilities} are the particular skills the character has learned to use. They are often similar to magical abilities, they allow particular actions and also to subvert the rules at times. They are rare and are taken at level crossings.

\textbf{Action}: \index{Action} is what you do over a period of time. Everything the character does is measured in Actions. Fighting, casting spells, breaking in, drinking potions, moving around ... 3 Actions can be done in each round.

\textbf{Bonus}: \index{Bonus} any modification due to external factors, environmental, magical, circumstance or that the Storyteller decides is a bonus or malus to be applied to the die roll or difficulty in the check.

\textbf{Check / Test}: \index{Check} \index{Test} a check (or test) is the roll of 3d6 plus a score indicated by the Characteristic and Competence involved, modifiers given by Skills and circumstances.

\textbf{Class}: There are no classes in OBSS. Each character is built around what they can do. So you won't find the word Class in the manual.

\textbf{Critical Hit} \index{Critical Hit}: when in the attack roll he gets more than 6 times.

\textbf{Magic Check} \index{Magic Check}: The Magic Check may be due for particular situations, such as when the character is injured or distracted, but it can also be requested by the player.

The Trial of Magic allows the player to go further in casting the spell and try to tap into and harness more magic.

Depending on the results it could get advantages or disadvantages.

\textbf{Casting Spells under attack, threat, distraction ..}: \index{Test of Concentration} \index{Casting Spells under attack, threat, distraction ..} \index{Distracted} when a spellcaster wants to use a Spell but is disturbed, attacked, injured or otherwise distracted while casting a spell then he must make a Magic Check.

\textbf{Difficulty Class (DC)}: \index{Difficulty Class} \index{DC} indicates how difficult it is to succeed in a check. It can be used to check skills (swimming ..) as well as knowledge (poisons ..). In spells it is the difficulty of resisting the same spell. It indicates what value to reach to pass and succeed in the check.

\textbf{Competence} \index{Competence} (skill) \index{Skill}: competence tells us what we know and its value indicates the degree of knowledge of a single ability. May it be learning a language, climbing, noticing little things.

\textbf{Proficiency with Weapons (melee or ranged)} \index{Proficiency with Weapons} is the ability to know how to hit the opponent with melee (swords, clubs ...) or shooting / distance ( throwing daggers, bows, crossbows ..)

\textbf{Magical Proficiency (CM)}: \index{Magical Proficiency} \index{CM} is your ability to use spells, the higher this value the more effective the spells will be.

\begin{center}
\includegraphics[keepaspectratio, width = 0.40 \textwidth]{immagini/spiritomagia2.png}
\end{center}

\textbf{Hit Dice} \index{Hit Dice}: Hit dice are the levels of a creature. Basically they are used to indicate how many hit points and level he has. If not listed, a creature has 6 hit points per hit die.

\textbf{Defense}: \index{Defense} Defense means the total value obtained from 10 + Shield + Armor + Dexterity + various and any bonuses. Represents the difficulty in being hit, the higher it is the harder it is to be hit.

\textbf{+ 1d6 or -1d6}: it is a bonus or malus to a check. Add or subtract a 6 die roll to the check, or a + 4 / -4

\textbf{Distance}: \index{Distance} The distance, when it comes to combat, is measured in 1 meter squares.

\textbf{Devotee} \index{Devotee}: a character who is linked to a Patron and has at least 3 Traits in common.

\textbf{Follower}: A character who has bonded with a Patron with 2 Traits in common

\textbf{Explosion of 6}: \index{Explosion of 6} when, you make the attack roll, saving throw, Magic Check (read the specifics in the dedicated chapter) or anytime it is indicated that the Explosion of 6 means that for each die rolled that rolled a 6, the die must be marked and re-rolled. The result of the new roll is also added up and if you roll a 6 you continue to roll as long as you continue to roll a 6.

\textbf{Initiative}: \index{Initiative} is a Dexterity or Intelligence check. Establishes the order of actions in combat. Whoever has the highest score acts first.

\textbf{Level}: \index{Level} the Level indicates the skill and power achieved by the character. It can indicate when the enemy is strong.

\textbf{Spell Level}: Indicates the scale (1 to 9) of the spell's magical power.

\textbf{Spellcaster, Mage:} \index{Spellcaster} means any user of magic in any capacity.

\textbf{Melee}: \index{Melee} melee refers to contact combat, hand-to-hand, sword to sword, meaning when your character fights not with a weapon that has range (bow, crossbows, slingshots. ..) against an opponent.
Any creature the character can reach with his non-missile weapon is considered in melee. A large enemy (or one with a long weapon) may be in melee with the character but not vice versa.

\textbf{Movement}: \index{Movement} Movement represents the ability to move. A move Action represents the movement of the character. The higher the movement value, the more meters a creature can move.


\begin{center}
\includegraphics[width = 0.40 \textwidth]{immagini/merlin.png}

\textit{Merlin dictating his prophecies to his scribe. Robert de Boron's Merlin en prose (written ca 1200)}
\end{center}

\textbf{Storyteller:} \index{Storyteller} is the person who leads the adventure, sets the rules and controls the elements of the story. The duty of every Storyteller is to entertain, be fair and use common sense. The Storyteller has the final say in any matter.

\textbf{Patron}: \index{Patron} or deity. The Patron is a superior being who can grant powers and guarantee advantages.

\textbf{Penalties / Malus} \index{Penalties}: like the bonus, penalties or malus are values, numbers, which indicate unfavorable circumstances, penalizing spells or anything else that makes the test more difficult. Unfortunately, unlike bonuses, penalties, unless otherwise specified, are always added together.


\medskip
%\begin{wrapfigure}{c}{0.5\textwidth}
\begin{center}
\includegraphics[width = 0.35 \textwidth]{immagini/Sakramentarz_tyniecki_02.png}

\textit{Sakramentarz Tyniecki: Majuskuła "V".}
\end{center}

\medskip

\textbf{PC, Character}: \index{Character} is the creature that is led, managed, rolled by the player.

\textbf{NPC}: \index{NPC} non-player character. They are particular characters, important or not, that the Storyteller keeps to lead the adventure.

\textbf{Experience Points / XP}: \index{Experience Points} \index{PX} whenever you solve difficulties, monsters, riddles, find treasure, play the character well and have fun you gain experience . These points accumulated over time establish the level and therefore the abilities of the character.

\textbf{Trait Scores}: \index{Trait Scores} \index{Stats} also abbreviated to trait or stats. Each character has 6 Characteristics: Strength (STR), Dexterity (DES), Intelligence (INT), Wisdom (WIS) and Charisma (CH). The higher the value, the greater the value or ability of the character in that specific area.

\textbf{Fate Points}: \index{Fate Points} \index{Beginner's Luck} or Beginner's Luck are available points that the player can convert into d6 to add to saving throws or attack rolls or rolls Skills. They are called Luck of Beginners because their number decreases as the character increases.

\textbf{Hit points (Hit points)}: \index{Hit points} \index{Hit points} indicate life energy, Fortitude, luck in resisting the creature's wounds. As long as the creature has 1 hit point it will fight at its best, without problems (sure .. it could also decide to run away rather than die ..).

At each level up, a certain number of hit points is earned, established by the rules. Each wound subtracts from this sum of energies and when 0 (zero) hit points are reached, he faints, unable to act. If you are further wounded or in any case your hit points drop to 10 + double the Constitution value then you are dead.

\textbf{Damage Reduction (DR)}: \index{Damage Reduction} \index{DR} some creatures have innate resistance to damage and wounds. This resistance is denoted as DR.

\begin{center}
\includegraphics[width = 0.4 \textwidth]{immagini/cavaliere2.png}
\end{center}

\medskip

\textbf{Damage Resistance (DR)}, \textbf{Resistance}: \index{Damage Resistance} \index{DR}: A creature may have resistance to one type of damage. In this case, it is assumed that it automatically halves the damage taken before applying any saving throws.

\textbf{Round}: \index{Round} Combat or actions are divided into rounds. A round represents a time unit of approximately 10 seconds. During the round, each creature has a chance to act on its initiative and perform up to 3 Actions.

\textbf{Critical Success / Critical Magic Failure} \index{Critical Magic Success} \index{Critical Magic Failure}: in case the player passes the Magic Check with critics (two 1s or two 6s). Magical critical success leads to spectacular changes in the spell, otherwise bad things could happen to the caster.

\textbf{Critical Success / Critical Saving Failure} \index{Magical Critical Saving Throw Success} \index{Critical Saving Throw Failure}: depending on the spell on the critical saving throw success by 10 or more) further halves the effects while in case of critical failure (failing by 10 or more) you suffer even more damage.

\textbf{Attack Roll (TC)}: \index{Attack Roll} \index{TC} is an Attack (Weapon Proficiency) vs Defense (Armor + Shield + Abilities + Magic ...) check. The attack roll can be from melee (ie for creatures close to your weapon, at melee range) or from distance (for bows, crossbows, but also thrown daggers ..). Read the combat chapter carefully.

\textbf{Saving Throw (Save)}: \index{Saving Throw} \index{Save} When a creature is subjected to a particular effect, a saving throw is often granted to mitigate or negate the effects. The saving throw is an action that takes no time or action.

Saving throws are about reflexes and dodging (Dexterity), resisting poison / disease or body changes (Constitution), or resisting mental attacks and effects that affect agency (Wisdom).

\medskip

\begin{center}
	\includegraphics[width = 0.45 \textwidth]{immagini/Jan_Steen2.png}

	\textit{Jan Havicksz. Steen}
\end{center}

\medskip

\textbf{Stroke}: \index{Stroke} indicates a character component. Each character chooses 4 Traits to compose and build his personality.

\textbf{Turn}: \index{Turn} are 10 minutes, or 60 rounds

\textbf{One is bad for you}: \index{One is bad for you} if you roll a 1 with the data you subtract 1 from the total result. This does not mean that a rolled 6 becomes a 5, the explosion of the 6 remains .. only that you subtract 1 from the final result. In other words, 1 is equal to 0.

\begin{changemargin}{0.3cm}{0.3cm} \begin{enfasi}{
The D\&D game has neither losers nor winners, it has only gamers who relish exercising their imagination. The players and the DM share in creating adventures in fantastic lands where heroes abound and magic really works. In a sense, the D\&D game has no rules, only rule suggestions. No rule is inviolate, particularly if a new or altered rule will encourage creativity and imagination. The important thing is to enjoy the adventure. (Tom Moldvay, 03/12/1980) } \end{enfasi} \end{changemargin}

\medskip

In the Manual you will find different types of boxes, each one has a precise meaning

\medskip


\begin{changemargin}{0.3cm}{0.3cm} \begin{enfasi}{Example of box with citation or motivational motto} \end{enfasi} \end{changemargin}

\begin{changemargin}{0.3cm}{0.3cm} \begin{tcolorbox}[title = Information for the player] Box containing information and explanations for the Player. \end{tcolorbox} \end{changemargin}

\begin{changemargin}{0.3cm}{0.3cm} \begin{Storytellere} Box containing information and suggestions for the Storyteller \end{Storytellere} \end{changemargin}

\end{multicols}

\pagebreak

\section{Races} \index{Races}

\begin{changemargin}{0.3cm}{0.3cm} \begin{enfasi}{The real voyage of discovery does not consist in finding new territories, but in possessing other eyes, seeing the universe through the eyes of another, of hundreds of others: of observing the hundreds of universes that each of them observes, that each of theirs is. (Marcel Proust)

\medskip

It is not the most intelligent of species that survives; it is not even the strongest; the species that survives is the one that is able to adapt and adapt better to the changes in the environment in which it finds itself. (Leon C. Megginson)} \end{enfasi} \end{changemargin} \medskip

\begin{multicols}{2}

\lettrine[lines = 2, lhang = 0.33, loversize = 0.25, findent = 1.5em]{Y}{eru} is a multifaceted world rich in cultural, natural and human diversity.
Creatures make the planet vital and rich, each one nourishes, contributes, enriches the knowledge of all the others.


\subsection{Humans} \index{Humans} \label{umani}

Humans with their desire for discovery, power, glory and violence and reproductive capacity are the dominating race.

The physical characteristics of humans are extremely varied, the color of the skin, clothing, cultural and food traditions, lifestyles can be the most disparate and original and everything just makes the character more \textit{human}.

Leave the racism out of Yeru, there are already enough wars, you don't need to create new ones just because someone uses axes instead of swords.

Humans were the race created by Ljust and Calicante together so that with their chaotic, shifting and vital thrust they could do and undo by continually starting over and continually improving.

\begin{center}
\includegraphics[height = 0.3 \textwidth]{immagini/uomovitruviano2.png}

\textit{Vitruvian Man - Leonardo da Vinci}
\end{center}


\textbf{Racial modifiers:} +1 to any characteristic

\textbf{Physical characteristics}: height 150-185cm, 50-130kg, life expectancy 65 years (50 + 2d10 years)

\textbf{Dimensions:} Medium

\textbf{Speed}: 9m

\textbf{Languages}: Common

\textbf{Advantage}: +1 Ability at 1st level

\subsection{Elves} \index{Elves} \label{elfi}

Elves are the race created directly by Ljust to lead the world with the elegance, intelligence and foresight of an immortal race.

After millennia of peace and life in the whole world, after natural and architectural beauties had spread in harmony in the world, the creation of the new races and their expansionist drive led the elves to revise their mission.

Suddenly from bearers of beauty, culture, passion and stimulus for the arts they became more secluded, in them the fear that all beauty could be lost arose.

So they decided to isolate themselves more and more to keep every form of art and beauty, from the written word to the arts, safe from those who could despise or make them ugly.

Their desire to preserve creation takes concrete form in pure struggle against everything and everyone, against any creature who wanted to live and create something new in a nutshell. Cattalm had poisoned their blood in such a profound way that they didn't even notice.

A period of extremely nefarious and violent centuries followed, dominated by the absolute will of the elves to destroy everything using any means.
All nations and civilizations paid a very high price both in terms of lives and in terms of returning to barbaric times.

It took nearly 500 years and the annihilation of entire nations and civilizations until all other creatures decided to join forces against the elves in a last-ditch attempt at salvation.

In what is called Shadow Week, hundreds of thousands of creatures and nearly all the elves perished.

A radical purge ensued, every elf found was killed and so on for nearly another century.

A little less than 100 years ago a new Elven Queen, Licenea, embarked on a journey to the court of every nation, risking lynching numerous times. He managed to get a peace treaty that could safeguard the very few remaining elves.

The surviving elves, even if "old", have not lost their hatred, their blood remains dyed by Cattalm and they mull over a treaty that they say prevents them from fulfilling their true destiny.

Fortunately, the new elves, but not all, do not have this visceral hatred and would like to live a normal life even in contact with all other creatures even if they are well aware of how they are seen and treated by everyone else.

They are the young elves who not only want to preserve beauty but to live an almost infinite life where it is itself wonder.

Elves are generally taller and leaner than humans. The eyes are always with light shades.

Elves appreciate the written word and magic. They are a rational race, driven by a sharp mind and excellent senses, an interest in the extraordinary and knowledge.

\begin{center}
\includegraphics[height = 0.5 \linewidth]{immagini/elf.png}
\end{center}

\textbf{Racial modifiers:} +1 Intelligence, +1 Dexterity, -1 Charisma

\textbf{Physical characteristics}: height 165-195cm, 50-110kg, life expectancy 1000+ years

\textbf{Dimensions:} Medium

\textbf{Speed}: 9m

\textbf{Languages}: Elven

\textbf{Advantage}: Twilight vision of 60 feet


\subsection{Dwarfs} \index{Dwarfs} \label{nani}


The dwarves are a stoic and severe race accustomed to the purest communism, without a true concept of ownership but of pure community of goods according to the idea that every dwarf works for the community and not for himself.

Dwarves are a short, set breed, reaching a maximum height of around 140cm with a sturdy, compact build that gives them a massive appearance. Both males and females proudly wear long hair, and men often decorate their beards with various kinds of intricate hair clips and braids, and bald dwarfs are common, but not beardless. Dwarf women have no beard or excess hair. Sex is free and socialist.

The dwarves are led by honor and tradition and communism. They are often seen as gruff, but have a strong feeling of friendship and justice and respect for those who work hard and are committed to the community and group.

The dwarves are the race created by Erondil with the help of Atmos.

They judge the Elves with severity because they have not been able to carry out and indeed betrayed the dictates of Creation and therefore they feel the task, the burden and the honor of forging creation and in creation the beauty and majesty of Erondil.

\begin{center}
\includegraphics[height = 0.5 \linewidth]{immagini/DnD_Dwarf.png}
\end{center}

\textbf{Racial modifiers:} +1 Constitution, +1 Wisdom, -1 Dexterity

\textbf{Physical characteristics}: height 100-140cm, 45-90kg, life expectancy 450 years (400 + 1d100 years)

\textbf{Dimensions:} Medium

\textbf{Speed}: 6m

\textbf{Languages}: Dwarven

\textbf{Special:} Profession: A skill linked to the initial profession gets +1

\textbf{Advantage}: Darkvision of 60 feet

\textbf{Disadvantage}: Bad character


\subsection{Gnomes} \index{Gnomes} \label{gnomi}


Gnomes are small-sized beings but full of energy and life. Gnomes are a race that appeared just over 1000 years ago, it is not known exactly where from.
In a short time, thanks to their innate curiosity, tenacity and inventiveness, they managed to create populous and rich cities, almost always within virgin forests.

Gnomes are deeply linked to nature, their relationship is almost symbiotic, a Gnome will never give up seeing a tree and building with what nature provides.

The Gnomes have a deep respect for nature, the environment and animals, their cities, even if perfectly functional and indeed modern, are built and obtained in the forest and never destroying it and indeed enriching it.

Many Gnomes are inventors and builders capable of extraordinary tests of imagination and ingenuity. Many of their inventions are of help and support to the whole community and their social life is rich and supportive.

The most curious Gnomes often leave their communities, which are not closed to those who accept the lifestyle of the gnomes, and embark on a life of adventure to discover wonders and new works of genius that they can pass on to the community.

A gnome forced to stay away from a natural environment suffers the situation by becoming sad and apathetic, his need for light and nature is something physical and innate.

Gnomes get along well with anyone who respects nature and does not abuse it.

\textbf{Racial modifiers:} +1 Intelligence, +1 Charisma, -1 Strength

\textbf{Physical characteristics}: height 70-110cm, 30-50kg, life expectancy 650 years (600 + 1d100 years)

\textbf{Size:} Small

\textbf{Speed}: 6m

\textbf{Languages}: Gnomic, Sylvan

\textbf{Special:} Profession: A skill linked to the initial profession gets +1

\textbf{Advantage}: Druidic Artifice 1 per day.

\textbf{Disadvantage}: Depression if more than 7 days away from a natural environment.


\subsection{Half-elf} \index{Half-elf} \label{mezzelfo}


There is nothing more impure to an elf than a half-elf. No half-elf is born by the will of an Elf. Every half-elf is a child of violence. At least that's what the elves keep saying.

There are also rare half-elves born of romantic relationships. While usually short-lived, even by human standards, these secret encounters usually lead to the birth of half-elves, a race that is descended from two cultures but is the heir of neither. Half-elves can breed with each other, but even these "pure blood" half-elves are seen as bastards by elves.

Many elves see in a half-elf the betrayal of original beauty, of the purity of creation.
Very few see a gesture of love and gift towards an increasingly ugly world.

Half-elves are taller than humans but shorter than elves. They inherit the slender build and attractive features of their elven lineage, but the color of their skin is normally dictated by their human side. Their eyes tend to be similar to humans in shape but feature an exotic range of colors from amber to purple to emerald green and dark blue.

Half-elves understand loneliness and know that character is often more a product of life experience than of one's race.

\medskip

\begin{center}
\includegraphics[height = 0.6 \linewidth]{immagini/halfelf.png}
\end{center}


\textbf{Racial Modifiers:} +1 to a Trait of your choice

\textbf{Physical characteristics}: height 160-185cm, 50-100kg, life expectancy 210 years (180 + 5d10 years)

\textbf{Dimensions:} Medium

\textbf{Speed}: 9m

\textbf{Languages}: Common or Elven

\textbf{Advantage}: Twilight vision of 30 feet

\index{Half-orc}

\subsection{Half-orc} \label{mezzorco}

In the eyes of civilized cultures, half-orcs are monstrosities, the result of perversion and violence and are rarely the result of amorous unions, as such they are usually forced to grow up fast and hard, continually struggling to protect themselves or make a name for themselves. Some half-orcs spend their entire lives proving pure-blooded orcs that they are just as ferocious as they are.

Half-orcs stand an average of 1.9 meters tall, with powerful physique and greenish or gray skin. Their canines often grow quite long to protrude from their mouths and these "fangs", combined with a broad forehead and slightly pointed ears, give them that well-known "bestial" appearance. Despite these obvious orc traits, half-orcs are as diverse as their human parents.

If within the orc tribes they must continually earn the respect of the "thoroughbreds", in human society it is no better. Mocked, mocked, excluded and abandoned, the half-orcs often find refuge in crime.

Orcs were created directly by Cattalm with the help of Calicanthus. Much of their creator's chaotic and destructive tendency remains in the nature of the half-orcs.

Half-orcs are often victims of prejudice.

\begin{center}
	\includegraphics[height = 0.6 \linewidth]{immagini/halforc.png}
\end{center}

\textbf{Racial modifiers:} +2 Strength -1 Charisma

\textbf{Physical characteristics}: height 160-210 cm, 60 - 140 kg, life expectancy 70 years (50 + 5d10 years)

\textbf{Dimensions:} Medium

\textbf{Speed}: 9m

\textbf{Languages}: Common or Orcish

\textbf{Advantage}: Twilight vision of 30 feet

\textbf{Disadvantage}: Follow Chaos

\subsection{Nibali} \index{Nibali} \label{nibali}


The Nibals are a race magically created to be a slave to the great spellcasters of the north.

The legend says that the terrible frost wizards, starting from a couple of humans (after thousands had died atrociously in previous experiments) managed to create by manipulating with magic, a more robust, stronger, more intelligent and at the same time more docile and disciplined with merit that every child generated would be absolutely physically identical to the father or mother.

These things were happening now more than 2000 years ago and the kingdom of eternal evil collapsed under its own inability to evolve and understand the new challenges (and probably also thanks to the intervention of some Patron).

The Nibals continued to thrive and by taking advantage of what the kingdom of ice had left them, they created one of the most modern, democratic and civilized civilizations in the world.

For many the extreme efficiency and dedication of the Nibals is hateful, a yoke that leaves no room for personal freedoms, for the Nibals it is only a natural way to progress.

All Nibals are equal to each other for the same sex but the fact that they cannot have children with other races does not make them a closed or racist people, on the contrary, absorbing the best of each culture makes them better and also excellent diplomats. What really distinguishes a Nibali from another is the hairstyle, the tattoos, the clothing ... Respect for others and the Law are inextricably linked to their nature and yet there is nothing more free than a Nibali.

For a Nibali rules and laws must promote peace and freedom, they must be just and those who keep them understanding and wise.

\textbf{Racial modifiers:} +1 Constitution, +1 Intelligence, - 1 Wisdom

\textbf{Physical characteristics}: height 183cm males, 170cm females, 50 - 120 kg, life expectancy 130 years

\textbf{Dimensions:} Medium

\textbf{Speed}: 9m

\textbf{Languages}: Common

\textbf{Disadvantage}: Follow the Law

\subsection{Diverse} \index{Diverse} \label{diversi} \hypertarget{diverso}{}

Blessed or cursed the Diverse are not like us. They are not the friends you expect. A Diverse is the result of a corrupt union. If Patrons cannot act directly in the world, or at least that's what Gradh tries to avoid, they often use their powers to create a bloodline loyal to them instead.

A Diverse is faithful to his Patron and cannot do otherwise. Luckily they are sterile with humans, otherwise they would have already ruled the world.

A Diverse is more robust and smarter. Unfortunately, their hectic life is marked by a short duration. Usually a Diverse does not exceed 50 years of life.

A Diverse is branded, somewhere on his body there is the symbol, a desire, of his Patron. Almost all the Diverse have 3 or more concentric golden circles on the left wrist which can indicate the Patron (or Patrons in very rare cases) of which they are "sons".

\begin{center}
\includegraphics[width = 0.8 \linewidth]{immagini/diverso.png}
\end{center}

\textbf{Racial modifiers:} +1 Constitution, +1 Intelligence

\textbf{Physical characteristics}: height 155-185 cm, 50-110 kg, life expectancy 45 years (40 + 1d10 years)

\textbf{Dimensions:} Medium

\textbf{Speed}: 9m

\textbf{Languages}: Common

\textbf{Special}: Must identify a Patron and have at least 3 common Traits.

\medskip

\index{Races} \index{Race}
\begin{changemargin}{0.3cm}{0.3cm} \begin{tcolorbox}[title = Note on Disadvantages]
The player, in agreement with the Storyteller, can choose a disadvantage other than the one indicated as long as it is consistent with the character's story.
\end{tcolorbox} \end{changemargin}

\begin{changemargin}{0.3cm}{0.3cm} \begin{tcolorbox}[title = Note on Race]
No description of a race can ever harness and subdue a character. Each player is free to create the character of the favorite race (granted by the Storyteller) and describe him, frame him, hear him, make him alive as he likes.
Do not limit yourself to the descriptions proposed here, they are just ideas, do not feel limited in your choices because your race says this or that.
Give birth to the most beautiful and complete characters possible.
Each character is alive and is a person and as such will always be different from each other, each fantastic in a different way to the detriment of any race and prejudice.
\end{tcolorbox} \end{changemargin}

\begin{changemargin}{0.3cm}{0.3cm} \begin{tcolorbox}[title = Note on Sex] \index{Sex}
In case you were so dull I repeat that there is no difference in skills or statistics based on sex. Each player and player is invited to play the character of the sex (or not) they prefer.

If the topic is not fun for you, clarify it with the Storyteller, he will be able to orchestrate the adventure in a suitable way.
\end{tcolorbox} \end{changemargin}

\end{multicols}

\pagebreak

\section{Special Features}


\begin{changemargin}{0.3cm}{0.3cm} \begin{enfasi}{It is not enough to have eyes to see (anonymous)} \end{enfasi} \end{changemargin} \medskip

\begin{multicols}{2}


\lettrine[lines = 2, lhang = 0.33, loversize = 0.25, findent = 1.5em]{E}{very} creature is special and unique yet there are beings even more unique and special due to their characteristics. These are the peculiarities of some of these.

\subsection{Low-light vision} \index{Low-light vision} \label{visionecrepuscolare}

What for many is darkness for those who have \hypertarget{visioneeluce}{visione crepuscolare} is seeing well as long as there is a minimal source of light.

Twilight vision is color vision.
A spellcaster with low-light vision can read a scroll as long as he has even the faintest of candles as a source of light.

Characters with low-light vision can see outside on moonlit nights as if they were in daylight.
In the absolute lack of light, twilight vision does not help, it remains impenetrable pitch dark.

\subsection{Darkvision} \index{Darkvision} \label{scurovisione}

Darkvision is the extraordinary ability to see without light sources, up to a maximum distance indicated for each creature.

Darkvision is black and white only (does not allow the character to distinguish colors). It does not allow characters to see anything they cannot otherwise see: Invisible objects are still Invisible, and Illusions are still visible for what they appear to be.

Similarly, darkvision makes a creature subject to gaze attacks normally. The presence of light does not affect darkvision.
Making a Survival check to look for traps or visual Awareness only takes a 1d6 penalty.

\subsection{Scent} \index{Scent} \label{fiuto}

This special quality allows a creature to use its sense of smell to locate hidden or approaching enemies and to follow the trail. Creatures with a sense of smell can smell familiar smells as humans do with what they see.

The creature can detect creatures within 20 feet of it by smell. If the opponent is downwind, the range increases to 60 feet; if it is upwind, the radius decreases at a distance of 3 meters.
Stronger odors, such as smoke, garbage or decaying bodies, can be detected at twice the radius indicated above.

When a creature detects a scent, the exact location of its source is not revealed, only its presence within range. The creature can use an Action to detect the direction from which the scent is coming. When in melee distance from the source, it pinpoints its location.

A creature with scent can track tracks using smell, making a Follow Trail check to find and follow a trail. The typical DC of a fresh track is 10 (regardless of the surface the track is on). DC increases or decreases depending on the intensity of the track, the number of creatures leaving it, and the time elapsed since it was left. For every hour spent, the DC increases by 2.

\begin{center}
\includegraphics[width = 0.9 \linewidth]{immagini/mostro.png}

\textit{John D. Batten}
\end{center}

Otherwise, this ability follows the rules of the Survival skill. Creatures that track with their nose ignore the effects of surfaces the trail is on and poor visibility.

A creature with the Scent ability identifies familiar smells just as a human might identify a familiar place. Water, and especially running water, negates the ability to track creatures.

Some strong odors can easily mask others. The presence of such a smell makes it impossible to locate or identify exactly a creature by means of the Smell; the basic DC of the Survival skill for following traces in the presence of opaque odors goes from 10 to 20.


\begin{center}
\includegraphics[width = 0.9 \linewidth]{immagini/argus2.png}

\textit{Argus Panoptes Guarding the Heifer (Io), Red Figure pitcher, c. 460 BC Museum of Fine Arts, Boston}
\end{center}


\subsection{Blind Sight} \index{Blind Sight} \label{vistacieca}

Using senses other than sight, such as the perception of vibrations, sensitive smell, acute hearing, or sonar, a sighted creature moves and fights as well as a sighted creature.

Invisibility, darkness are useless, even if the creature with blind sight must have a line of effect to notice a certain creature or object.

However, a creature with cover still has its Defense advantage.

The range of the ability is listed in the creature's description. The creature typically doesn't have to make Awareness checks to notice creatures within range of its blind sight.

Unless otherwise noted, blind sight is always active, and the creature doesn't have to take any action to activate it. Some forms of blind sight must be activated as an immediate action. In this case, it is indicated in the description of the creature.

If a creature must activate blind sight, it gains the benefit only during its own round.

An ethereal creature is not visible to blind sight.

\subsection{Telluric sense} \index{Telluric sense} \label{sensotellurico}
A creature with Telluric Sense is sensitive to vibrations from the ground, and can automatically detect anything in contact with the ground within the radius specified by the Telluric Sense.

Aquatic Creatures with Telluric Sense (echolocation) can sense the position of creatures in contact with water.

The radius of the ability is specified in the creature's description text

\end{multicols}

\medskip

\begin{center}
\includegraphics[height = 0.65 \linewidth]{immagini/grabroid.png}

\medskip

\textit{Grabroid. Also known as Grippers. Tremors (Film)}
\end{center}

\pagebreak

\section{Features} \index{Features}


\begin{changemargin}{0.3cm}{0.3cm} \begin{enfasi}{Living is not breathing: it is acting, it is making use of the organs, the senses, the faculties, all those parts of ourselves for which we have the feeling of existing. (Jean-Jacques Rousseau)} \end{enfasi} \end{changemargin} \medskip

\begin{multicols}{2}

\lettrine[lines = 2, lhang = 0.33, loversize = 0.25, findent = 1.5em]{E}{very} character has 6 Characteristics (also called Statistics) which represent his basic attributes and constitute his potential talent and ability innate.

While it is not common for a character to make a check using only one of his Abilities, Ability scores affect virtually every aspect of a character's abilities and skills.


\subsection{Description of Features} \label{decrizionedellecaratteristiche}

Ability scoring is not all in a character, much less in a monster.

The more "instinctive" and aggressive monsters will certainly have negative Intelligence and Charisma scores, but they are not "stupid" for this, they simply act according to their natural patterns.

A monster with Strength -4 is not close to death, it simply has very little strength (imagine giving a Strength value to a mouse or a squirrel if not a small spider ..)

\subsubsection{Strength} \index{Strength} \label{forza}

\begin{changemargin}{0.3cm}{0.3cm} \begin{enfasi}{
Ah, it is excellent to possess the strength of a giant, but to use it from giant, it's tyranny! (William Shakespeare, Isabella: from “Measure for measure ", act II, scene II)
} \end{enfasi} \end{changemargin}

Strength measures the physical power, athleticism, and limits of brute force you can express. Strength applies to melee damage and hand-drawn weapons.

A Strength check can be used for any attempt to lift, push, pull, or smash something, to push your body into a space, or any other application of brute force.

A character with a Strength score of -5 is dead.


\subsubsection{Dexterity} \index{Dexterity} \label{destrezza}

\begin{changemargin}{0.3cm}{0.3cm} \begin{enfasi}{
Tired of barking. Strength matters nothing in life. Knowing how to dodge is what matters. (Daniel Pennac)
} \end{enfasi} \end{changemargin}

Dexterity measures agility, reflexes, balance and coordination; determines the Defense and Attack Rolls with Ranged Weapons.

A Dexterity check can be used for any attempt to move smoothly, dodge a blow, or to avoid losing balance or pickpocketing.

A character with a Dexterity score of -5 is unable to move and is completely immobile (but not unconscious).

\subsubsection{Constitution} \index{Constitution} \label{costituzione}

\begin{changemargin}{0.3cm}{0.3cm} \begin{enfasi}{
A little health every now and then is the best remedy for the sick. (Friedrich Nietzsche)
} \end{enfasi} \end{changemargin}

The Constitution measures health, vigor and life force. as well as resistance to stress.

A Constitution check can be used for your attempts to push yourself beyond the normal limits of your body and for endurance and durability tests.

A character with a Constitution -5 no longer has control of his body and is dead.

\subsubsection{Intelligence} \index{Intelligence} \label{intelligenza}

\begin{changemargin}{0.3cm}{0.3cm} \begin{enfasi}{
Strength without intelligence ruins under its own weight. (Horace)
} \end{enfasi} \end{changemargin}

Intelligence measures mental acuity, the accuracy of memories and the ability to reason.
An Intelligence check comes into play when you need to rely on logic, education, memory or deductive skills.

Your Intelligence (Arcana) tests measure your ability to remember information about spells, magical items, esoteric symbols, magical lore, the planes of existence, and the inhabitants of those planes. Rummaging through ancient scrolls for a piece of knowledge may require an Intelligence check.

A character with an Intelligence score of -5 is in a coma.

\subsubsection{Wisdom} \index{Wisdom} \label{saggezza}

\begin{changemargin}{0.3cm}{0.3cm} \begin{enfasi}{
Strength does not come from physical ability. It comes from a indomitable will. (Mahatma Gandhi)} \end{enfasi} \end{changemargin}

Wisdom reflects your attunement to the world around you and represents insight, intuition, willpower and common sense.

A Wisdom check reflects an effort to interpret body language, understand someone's feelings, notice details of the environment, or heal an injured person.

A character with a Wisdom score of -5 is incapable of rational thinking and is unconscious.

\subsubsection{Charisma} \index{Charisma} \label{carisma}

\begin{changemargin}{0.3cm}{0.3cm} \begin{enfasi}{
Kogami, do you know what charisma is?

- The way I see it, it's an innate attitude, like that of a hero or of a leader.

- [...] The elements that identify the charisma are three: the nature innate of heroes and prophets, the ability to instill well-being to others with mere presence and a culture that allows you one brilliant conversation on any topic. (Psycho-Pass)
} \end{enfasi} \end{changemargin}

Charisma measures your ability to interact effectively with others. It encompasses factors such as confidence and eloquence, and can represent a charming or bossy personality.

A Charisma check may be required when you try to influence or entertain other people, when you try to make an impression or tell a lie, or when you have to navigate a complicated social situation.

Typical Charisma usage situations include attempts to trick a guard, defraud a merchant, earn gambling money, pass yourself off as someone else in a disguise, dispel someone's suspicions with false reassurances, or keep an unabashed face while a blatant lie is told.

A character with a Charisma score of -5 is unconscious.

\subsubsection{Reading Ability Scores} \index{Reading Ability Scores} \label{leggereipunteggidellecaratteristiche}

Each Ability score typically ranges from 0 to 3, a good Ability score is 1, 2 is excellent, 0 is "normal", 3 is rated "outstanding".

A score of -1 is judged to be weak, a -2 subnormal, a -3 severely problematic, a -4 almost leads to a non-use of the characteristic, a -5 should just stay in bed (if it is not already in a coffin).

\subsubsection{Optional - Age of character} \index{Optional - Age of character} \hypertarget{etadelpersonaggio}{} \label{etadelpersonaggio}

The age of the character affects the physical and mental characteristics.

\begin{tabular}{llllll}
Period & FOR & DES & COS & INT & SAG \\
	\hline
Young 	& & & +1 & -1 & \\
	\hline
Adult 	& +1 & & & & +1 \\
	\hline
Mature 	& -1 & & -1 & +1 & +1 \\
	\hline
Senior 	& -2 & -1 & -1 & & +1 \\
	\hline
Venerable 	& -1 & -1 & -1 & +1 & +1 \\
\end{tabular}

\medskip

The modifiers shown stack.

\subsection{Characteristics scores} \hypertarget{assegnazione.punteggi.caratteristica}{} \label{assegnazionepunteggicaratteristica}

Ability scores play an important but not critical role. The player must understand that a "low" score does not mean having a bad character, but rather he will have more fun playing it by leveraging on skills, abilities and peculiar abilities, using ingenuity and wit. Multiple systems for pulling features are presented.

I personally suggest the \textbf{Basic Mode} approach. In OBSS the characters are not heroes, they are not the chosen ones who stand up as defenders of the planet. The characters are normal people often involved in spite of themselves in situations at the limit if not beyond survival.

\begin{center}
\includegraphics[width = 0.55 \linewidth]{immagini/dice4.png}
\end{center}

The undoubted advantage of pulling the values in order of the characteristics is that it allows you to mess up the schemes and avoid builds done at the table.

It is likely that they will not come out the results you were hoping for or even have come in features that you did not care about. That's okay. Change your mind, be inspired by the values obtained! Have fun with the new character, build something new and different, let yourself be amazed.

\textbf{Mode I would like} allows you to assign a certain weight (number of dice) to the statistic so that you are more likely to get the desired score.

Finally, the \textbf{Optional Mode} of assigning points should only be granted for adventures where the characters are the heroes, creatures that stand out and stand out among the masses and not quite ordinary people who have the misfortune to face deadly situations.

In all proposed systems, the rolls for Characteristics are made in order, so the first roll is for Strength, then for Dexterity, Constitution, Intelligence, Wisdom and finally Charisma.

Finally, remember that OBSS is an RPG where the death of the character happens, even more often than in other RPGs. Create valid and concrete characters and let the adventure forge the details.

\textbf{\textit{Racial modifiers cannot raise or lower scores beyond + 4 / -4}}.

\subsubsection{Basic Mode} \index{Features - Basic Mode} \label{modalitabase}

The player rolls 3d6 for each characteristic and in order, he can only reroll a 1 rolled for triplet (3d6) once. He then throws a seventh triplet which he can substitute for another triplet. For each characteristic rolled check the sum of the dice rolled with \textbf{Table: Character Roll}


\subsubsection{Mode I would like ..} \index{Features - Mode I would like it} \label{modalitavorreiche}

The value of the Character's Characteristics is entrusted to a set of dice that the player assigns to the statistics. The player can assign a total of 18d6 divided between the 6 characteristics. He could assign 2d6 to Strength, 4d6 to Dexterity, 3d6 to Constitution, 4d6 to Intelligence, 2d6 to Wisdom, 3d6 to Charisma.
It is not possible to assign more than 4d6 or less than 2d6. Roll the assigned dice and check the totals obtained by comparing them to \textbf{Table: Ability Roll}

\subsubsection{Table: Ability Shot} \index{Ability Shot Table}

The sum of the dice rolled for the Characteristics should be compared against this table to determine the actual values of the Characteristics. \\

\begin{tabularx}{0.45\textwidth}{lX | lX}
\textbf{Val. pulled} & \textbf{Character} & \textbf{Val. pulled} & \textbf{Character} \\
\toprule
3 (or less) & - 3 & 13-14-15 & + 1 \\
4-5 & -2 & 16-17 & + 2 \\
6-7-8 & -1 & 18 (or more) & + 3 \\
9-10-11-12 & + 0 && \\
\end{tabularx}

\subsubsection{Optional mode (for cowards!)} \index{Features - Optional mode for cowards} \label{modalitapericodardi}

Each player distributes 4 points among the 6 Characteristics, each Characteristics must have a minimum of -1 and a maximum of 2 before the racial modifiers. \\

\textbf{Remember to apply racial modifiers!}

\begin{changemargin}{0.3cm}{0.3cm} \begin{tcolorbox}[title = Let's roll Tups Features]

		\textbf{First triplet}: \cancel{1}, 1,4,3 total 8. Strength is -1

		\textbf{Second}: 5,6,6 total 17. Dexterity is +2

		\textbf{Third}: \cancel{1}, 2,1,4 total 7. Constitution is -1

		\textbf{Fourth}: 6,6,6 total 18. Intelligence is +3

		\textbf{Fifth}: 3,4,2 total 9. Wisdom is +0

		\textbf{Sixth}: 3,4,4 total 11. Charisma is +0

		\textbf{Seventh}: 3,5,2 total 10. Replacing Strength (-1 to +0)

		Being \hyperlink{diverso}{Diverso}, Tups gets +1 in Constitution and +1 in Intelligence

\end{tcolorbox} \end{changemargin}


\subsection{Enhance Features} \label{aumentarelecaratteristiche}

\textbf{Every four levels} (4, 8, 12 and up to 16) you can increase a Characteristic by one point, up to a maximum of 4 + the racial bonus, or malus, of the characteristic. Magic items or spells are required to increase beyond this value. The Characteristic increase has a retroactive effect only for Constitution increases, affecting the maximum hit points. The Characteristic increase immediately applies the modifier to saving throws and attack and initiative rolls, the increase in intelligence is reflected in the next level in the number of skills acquired.

\begin{center}
\includegraphics[width = 0.45 \linewidth]{immagini/guerrieroispirato.png}

\textit{Brian Boru, High King of Ireland}
\end{center}

\end{multicols}

\begin{changemargin}{0.3cm}{0.3cm} \begin{Storytellere}
However, players will complain about the rolled dice, it is normal, especially the more inexperienced players. Try to make him understand that he must not limit himself to looking at the Characteristics but to see the general whole of the character. Suggest Abilities that can make up for the disadvantage.
\end{Storytellere} \end{changemargin}

\begin{changemargin}{0.3cm}{0.3cm} \begin{tcolorbox}[title = Low characteristics!] \index{Low characteristics}
Having low stats is not the death of the character! Instead, try to play so that there is no need to roll dice or check! Strive to be witty, intuitive, proactive, crafty… in short, everything that can make you solve the situation without necessarily having to roll the dice. In OBSS, the Storyteller rewards players who describe and exalt themselves in what the character does with bonuses on checks!
\end{tcolorbox} \end{changemargin}


\pagebreak

\begin{multicols}{2}

\section{Hit Points} \index{Hit Points} \index{Hit Points}

\begin{changemargin}{0.3cm}{0.3cm} \begin{enfasi}{Those who do not value life do not deserve it. (Leonardo da Vinci)} \end{enfasi} \end{changemargin}


\lettrine[lines = 2, lhang = 0.33, loversize = 0.25, findent = 1.5em]{T}{he Hit Points} represent the life energy of the character but also the ability, luck, ability of the character to resist and fight . As long as the character / opponent has at least 1 hit point (HP) he will fight and fight to the best of his ability.

- Each character starts with 4 hit points at first level + Constitution score.

- At each level, beyond the first, he gains 1d4 hit points + the Constitution score.

Each point taken in Weapon Proficiency increases the Hit Points taken by 3. Further Abilities can raise this score.

Mark on the character sheet the maximum hit points you have and indicate the current value from time to time that you lose or recover. Always mark on the sheet what the current hit point amount is, after each hit or damage. The amount of hit points when the character is \textit{perfectly healthy} is also referred to as maximum hit points \\

\begin{changemargin}{0.3cm}{0.3cm} \begin{tcolorbox}[title = I'm going to die!] \index{I'm going to die!}
ESCAPE! Retreat, hide, get out of combat and help your comrades. There is no glory in being dead. Always carefully consider when it's time to get out of the fight, better a retreat than a TPK (Total Party Kill or death of the whole group).
\end{tcolorbox} \end{changemargin}

Hit Points are recovered in several ways: \\

- for each night of rest (at least 8 hours) you recover in hit points the Constitution value + 2 * WP + CM (with a minimum of 1) \index{HP recovery while sleeping}

- through healing spells (spells, potions ... or other magical items)

- competence \hyperlink{prontosoccorso}{First Aid} (page \pageref{prontosoccorso}), through more or less long treatments

Hit Points can also be \textbf{temporary} \index{Temporary Hit Points} or temporarily added to or removed from your current ones.

A spell that grants +10 temporary hit points will raise your current hit points by 10, so if you take 8 damage, you'll have 2 temporary hit points left. If you take 13 damage in addition to losing all temporary hit points you will also suffer 3 normal hit points and further damage will be calculated on his actual hit points.

- When you get temporary hit points you have to choose whether the effect replaces the previous one. The gain does not stack.

- At the end of the effect that grants temporary hit points, they disappear, leaving the creature at its previous hit points.

- Unless otherwise stated, Temporary Hit Points disappear after one hour from when they are added.

- Temporary Hit Points are removed first when you are wounded.

- Temporary hit points cannot be higher than half the maximum hit points.

A weapon or effect that causes non-lethal damage means that it causes \hyperlink{recuperopuntiferitanonletali}{temporary loss of hit points} \label{feritetemporanee}.

\section{Fate Points} \index{Fate Points} \index{Beginner's Luck}

\begin{changemargin}{0.3cm}{0.3cm} \begin{enfasi}{If fate is against us, the worse for him. (motto of the 1st Regiment Carabinieri Paratroopers "Tuscania")} \end{enfasi} \end{changemargin}

\lettrine[lines = 2, lhang = 0.33, loversize = 0.25, findent = 1.5em]{I}{n} a not easy world Fate Points helps those who have no experience.
Each character has a number of Fate Points equal to (20 - Level) / 5, with a minimum of 1. Fate Points are counted per game session.

At each session they are reset and recalculated, so Fate Points are not accumulated between one game session and the next.

Calculation example:

A level 6 character has: 20-6 = 14/5 = 3 (round to the nearest integer) Fate points to use in the session.

A Fate Point is used as an Immediate or Reaction Action and each Fate Point used grants a + 1d6 bonus to the current check. The Fate Point can be used to have an extra dice in the saving throw or attack rolls, potentially exploding the dice, or a proficiency check. It cannot be used to increase fixed values (such as Defense) or the damage of a weapon.

\begin{changemargin}{0.3cm}{0.3cm} \begin{tcolorbox}[title = Fate Points - Use them] \index{Fate Points - use them}
A common mistake for beginners playing OBSS is not to use Fate Points, either because you forget them or want to keep them. My suggestion is to use them whenever it is needed more than the confrontation itself to the situation as a whole. You learn with experience when to use them well and when it is good to use them, in any case the first few times .. just use them. Better to have a + 1d6 bonus than to have it and die.
\end{tcolorbox} \end{changemargin}

How many Fate Points to use is declared before the roll, once you have declared the amount of Fate Points you want to use it is not possible to use more or less.

\end{multicols}

\pagebreak

\section{The Traits} \index{Traits} \hypertarget{traits}{}\label{traits}

\begin{changemargin}{0.3cm}{0.3cm} \begin{enfasi}{Therefore, whoever knows how to do good and does not do it, commits sin. (James the Just 4.17, Letter of James. NdA referring to the selected Traits)
\smallskip

It is a natural right to satiate the soul with revenge. (Attila)
\smallskip

Est Sularus Oth Mithas. ("My honor is my life", Pledge of the Knights of Solamnia.)} \end{enfasi} \end{changemargin} \medskip

\begin{multicols}{2}

\index{Traits}
\lettrine[lines = 2, lhang = 0.33, loversize = 0.25, findent = 1.5em]{I}{n} OBSS there is no clear distinction between good and evil, law and chaos, between what is right and what's wrong.

In OBSS there are Traits, aspects and character nuances that contribute to the background of the character, help the player to perform better and can provide him with those guidelines to interpret the character he wanted to create more correctly.

A Trait is a detail that helps better to frame the character, outlines the main characters, granting him different shades.

\textbf{Each player chooses 5 Traits for his character at character creation.} These will be the \textit{moral, ethical and behavioral compasses} that will aid the character in acting and making choices.

\smallskip

\begin{changemargin}{0.3cm}{0.3cm} \begin{tcolorbox}[title = Choosing the Traits] %box giocatore
Traits are not the character, they do not block him or fix him eternally in time. A character is always in constant evolution and so is his Traits, morals, behavior and desires. Don't be stiff but use Traits to give you suggestions to get inspired by.
\end{tcolorbox} \end{changemargin}

\smallskip

\textbf{Of the Traits chosen at the first level, identify one, this will start, again at the first level, with a value of 1, the others will start at a value of 0.}

Over time and adventures can be earned or replaced (in concert between Storyteller and player based on how played) by other Traits. The higher a Trait value, the more present and permeating the character's choices.

During the adventures, certain Traits can also be emphasized, i.e. the Storyteller following particular scenes and roles can increase a Character Trait by a point, or a fraction of a point.

For example, following a particular choice and climax of adventure, the Storyteller could grant everyone or someone Courageous Trait or give +1 to Courageous to those who already have this Trait. For Traits not taken, the base value in points is -1. So the first point is used to take the Trait (to zero) and the following ones to emphasize them.

While it is "relatively" easy to acquire new Traits, it is extremely difficult to change existing ones. Talk to the Storyteller, he will be able to prepare situations and adventures that will help you understand how to evolve the character and possibly change the chosen Traits.

\textbf{Any particularly important action where the character has followed a Trait leads the character to approach the Patron (or Patrons) responsible for that Trait}.

In the sheet you will find \textbf{check} to be placed next to the Traits, these are marked as a result of suitable actions to increase the value of the Trait; once the 10 checks are reached, the stretch will increase by 1 point and a new ten will start again.

During the adventure, the Storyteller will tell you when to score, or cancel, partial points. \textbf{It is generally assumed that a character acquires at least one Trait point per level.}

As the value of the Trait increases, the character will be able to acquire powers, regardless of whether he is a believer (Follower or Devotee) of that Patron or not.

- At \textbf{5} points you can begin to feel the presence of a Patron linked to a Tract

- At \textbf{10} points you can feel the closeness of a Patron linked to a Tract

- A \textbf{15} points is linked to Patron by a Tract

- At \textbf{20} points you are a Champion of the Patron linked to a Trait.


It is not necessary to believe in a Patron to feel his closeness, it is simply his own nature (his Traits) that is akin to the Patron, whether one likes it or not.

Since the purpose of a Patron is to make their Traits dominate over others, having people of high level and power who are so close to him will come in handy in the 1000 year judgment.

To identify the most similar Patron, check your most high Trait on \hyperlink{tabellacollegamentoPatronotratto}{Patron List - Trait} (page \pageref{tabellacollegamentoPatronotratto}) and identify the Patron with whom you share it. In case of similarity choose the Patron you prefer. Check in \hyperlink{cosmologia}{Cosmologia} (page \pageref{Patroni}) the powers granted by the Patron. These checks should be done every time you increase a Trait value.

The Storyteller is free to insert new Traits as he likes or requested by the players, it is suggested to attribute these new Traits to the Patrons as well.

\end{multicols}

\textbf{Trait Table} \index{Trait Table}

\begin{multicols}{6}
{\small
\begin{flushleft}
Habitual \\
Accumulator \\
Affable \\
Affectionate \\
Reliable \\
Aggressive \\
Allegro \\
Haughty \\
Selfless \\
Ambitious \\
Friendly match \\
Anarchist \\
Anxious \\
Nonconformist \\
Disliked \\
Apathetic \\
Angry \\
Arrogant \\
Careful \\
Bold \\
Austere \\
Reckless \\
Greedy \\
Warlike \\
Bully \\
Brutal \\
Liar \\
Good \\
Grumpy \\
Joker \\
Calculator \\
Calm \\
Candide \\
Chaotic \\
Charitable \\
Casinista \\
Bad \\
Cynic \\
Clemente \\
Stubborn \\
Coward \\
Combatant \\
Compassionate \\
Competitive \\
Comprehensive \\
Conformist \\
Confusionary \\
Controlled \\
Brave \\
Correct \\
Corrupted \\
Cortese \\
Creative \\
Gullible \\
Gloomy \\
Curious \\
Weak \\
Decided \\
Determined \\
Devout \\
Distrustful \\
Disciplined \\
Dishonest \\
Dishonorable \\
Messy \\
Detached \\
Destructive \\
Docile \\
Double agent \\
Polite \\
Selfish \\
Emotional \\
Empathic \\
Enthusiastic \\
Expansive \\
Extrovert \\
Exuberant \\
False \\
Imaginative \\
Fatalist \\
Cold \\
Jealous \\
Generous \\
Joyful \\
Right \\
Idealist \\
Immature \\
Immoral \\
Clumsy \\
Impartial \\
Impassive \\
Impatient \\
Impetuous \\
Relentless \\
Imprudent \\
Uncertain \\
Unstoppable \\
Inconstant \\
Indifferent \\
Unruly \\
Indomitable \\
Indulgent \\
Industrious \\
Infant \\
Misleading \\
Naive \\
Insensitive \\
Insolent \\
Honest \\
Intolerant \\
Enterprising \\
Introvert \\
Envious \\
Hypocrite \\
Wrathful \\
Ironic \\
Unreasonable \\
Irritable \\
Instinctive \\
Complaining \\
Loyal \\
Legal \\
Lethargic \\
Liberal \\
Licentious \\
Litigious \\
Logorrhoeic \\
Lunatic \\
Lustful \\
Rude \\
Melancholic \\
Mischievous \\
Wicked \\
Martyr \\
Masochist \\
Maternal \\
Mattacchione \\
Miserly \\
Meticulous \\
Merciful \\
Measured \\
Mild \\
Moderate \\
Modesto \\
Worldly \\
Morigerato \\
Narcissist \\
Negligent \\
Nervous \\
Neutral \\
Non-violent \\
Shadowy \\
Honorable \\
Ordered \\
Observer \\
Hostile \\
Obstinate \\
Optimistic \\
Pacific \\
Paranoid \\
Passionate \\
Bungler \\
Scary \\
Patient \\
Perfectionist \\
Touchy \\
Crybaby \\
Planner \\
Pompous \\
Pragmatic \\
Thoughtful \\
Overbearing \\
Presumptuous \\
Provident \\
Protective \\
Provocator \\
Prudent \\
Rabid \\
Rationale \\
Reactor \\
Rebel \\
Thoughtful \\
Rigid \\
Relaxed \\
Reserved \\
Respectful \\
Resolute \\
Saccente \\
Sadistic \\
Sadomasochist \\
Sarcastic \\
Skeptical \\
Whimsical \\
Straightforward \\
Shy \\
Thoughtless \\
Grumpy \\
Grumpy \\
Simple \\
Sensitive \\
Serious \\
Unrestrained \\
Safe \\
Silent \\
Sincere \\
Unfair \\
Snob \\
Understated \\
Sociable \\
Dreamer \\
Solitaire \\
Suspicious \\
Carefree \\
Ruthless \\
Spontaneous \\
Naive \\
Stoic \\
Extravagant \\
Superb \\
Superficial \\
Susceptible \\
Tenacious \\
Shy \\
Traitor \\
Traditionalist \\
Quiet \\
Sad \\
Swindler \\
Humble \\
Vain \\
Valiant \\
Vengeful \\
Violent \\
Fickle \\

\end{flushleft}}
\end{multicols}

%valutare le motivazioni, una tabella delle motivazioni

If a player does not roll the character traits he will not allow the character to gain experience points.

\medskip

\begin{center}
\includegraphics[height = 0.35 \linewidth]{immagini/troll.png}
\end{center}

\begin{changemargin}{0.3cm}{0.3cm} \begin{enfasi}{If a traveler does not bring back something to share, he is not a Hero but an impostor, a selfish devoid of wisdom. (The Hero's Journey, Christopher Vogler)} \end{enfasi} \end{changemargin}

\pagebreak

- \section{Skills} \index{Skills}

\begin{changemargin}{0.3cm}{0.3cm} \begin{enfasi}{
Whoever says that something is impossible should not disturb those who are doing it.

\medskip

You haven't truly understood something until you are able to explain it to your grandmother. (Albert Einstein)}
\end{enfasi} \end{changemargin} \medskip

\begin{multicols}{2}

\lettrine[lines = 2, lhang = 0.33, loversize = 0.25, findent = 1.5em]{T}{he} Skills represent what you know and what you can do. The scores represent how the competence is well known and therefore the higher the value, the more experienced one is.

Unless otherwise specified, three basic rules \index{Basic Rules} called \textbf{Golden Rules}: \index{Golden Rules} apply to all proficiency checks (Basic, Active)

\begin{itemize}
\item
The \textbf{6 explode}, that is, if in the 3d6 check a die is six, add the result and re-roll, and if it is 6 you add the result again and re-roll again and again ..
\item
The \textbf{1 are bad}, if you roll 1 with the die you subtract 1 from the sum of the dice rolled (and therefore the die that made one counts zero)
\item
\textbf{Relying on fate}. Every 4 points between Skills (Basic or Active) and Characteristic that you give up adding in the check, you roll one die more to 6 (Attack Roll, Saving Throw, Proficiency checks). This four cannot be removed from the score given by Abilities or magical items.

\end{itemize}

\subsection{Basic Skills} \index{Basic Skills} \label{competenzebase}


\begin{changemargin}{0.3cm}{0.3cm} \begin{enfasi}{
%Anche se indubbiamente il desiderio di conoscere è naturale per tutti gli uomini, la voglia di imparare non è cosa da tutti; la maggior parte, anzi, assaggiato quanto lo studio sia fatica e provata la stanchezza sulla propria pelle, butta alla leggera la noce ancor prima di aver rotto il guscio per gustarne il gheriglio. (Richard de Bury)\medskip

Studying is for losers! (Lobo)} \end{enfasi} \end{changemargin}

Each character has an initial profession, a path of life and work that led him to learn certain skills or what he did (and wanting to continue to do) before embarking on dangerous adventures.

Some Professions and their relative skills are listed, the character acquires these skills with the score indicated in the table.

\end{multicols}

\textbf{Table: List of Professions and related Skills} \index{Table List of Professions and related Skills} \index{Professions}

\medskip

\begin{tabularx}{0.95\textwidth}{lllll}
\textbf{\textbf{Profession}} & \textbf{1 point} & \textbf{2 points} & \textbf{2 points} & \textbf{3 points} \\
\toprule
\textbf{Acolyte} & Occult & History or Geography & Arcane & Religion \\
\textbf{Alchemist} & Evaluate & Nature & Herbalism & Arcana \\
\textbf{Breeder} & Survival & Tracking & Nature & Handling animals \\
\textbf{Magician's Appr.} & History, Geography & Occult & Myths & Arcana \\
\textbf{Pettifogger} & Evaluate & Deceive & Perceive Emotions & Local Traditions \\
\textbf{Librarian} & Nature & Local Traditions & Religion, Arcana & History, Geography \\
\textbf{Lumberjack} & Using Rope & Nature & Orientation & Survival \\
\textbf{Hunter} & Move silently & Track & Survival & Nature \\
\textbf{Caravaner} & History or Geography & Evaluation & Riding & Orientation \\
\textbf{Theatrical} & Perceiving emotions & Languages & Entertaining & Acrobatics \\
\textbf{Herbalist} & Myths & Geography & Nature & Herbalism \\
\textbf{Card Player} & Disguise & Evaluate & Entertain & Deceive \\
\textbf{Guard} & Perceiving Emotions & Knowledge Law & Riding & Intimidating \\
\textbf{Guide} & Myths & Dungeons & Nature & Geography \\
\textbf{Pickpocket} & Disarming Gadgets & Escape Artist & Moving Silent & Fairy Hands \\
\textbf{Delinquent} & Survival & Ride & Evaluate & Hide \\
\textbf{Innkeeper} & First Aid & Assess & Perceive Emotions & Diplomacy \\
\textbf{Merchant} & Local Languages & Traditions & Evaluate & Deceive \\
\textbf{Miner} & Using Ropes & Evaluating & Orientation & Dungeon \\
\textbf{Fisherman} & Orientation & Swimming & Using ropes & Nature \\
\textbf{Soldier} & Swimming & Handling animals & Jumping & Riding \\
\textbf{Carter} & Local Traditions & Orientation & Handling Animals & Riding \\
\textbf{Medicine Man} & Myths & Nature & Herbalism & First Aid \\
\textbf{Forest Guard} & Myths & Herbalism & Riding & Nature \\
\textbf{Farmer} & Survival & Herbalism & Handling Animals & Nature \\
\end{tabularx}

\bigskip

\begin{multicols}{2}

The initial profession and the skills acquired must be marked on the sheet, obviously in agreement with the Storyteller it is possible to select different skills and even choose different professions!

With each new profession that you are going to create associated 4 skills taken from this list, one skill will start with a score of 1, two skill will start with a score of 2 and the more specific and professional one will start with a score of 3.

Player \textbf{increases the score of a Characteristic that links to Profession or background by 1} up to the maximum value of 4 + racial changes. It could be Intelligence for a Magician Apprentice, but if he is a hobby bodybuilder it could also be Strength.

Obviously a profession is not expressed in just 4 skills but these are the ones that will come into use most during the adventures, the Storyteller will be helped by your profession to understand how your character can solve situations and how he will interact with the other characters.

Below is the \textbf{Skills List Table} from which to choose for any new professions or customizations of the same.

In agreement with the Storyteller it is also possible to change the order of the scores making the character more capable in some skills rather than others.

\begin{changemargin}{0.3cm}{0.3cm} \begin{tcolorbox}[title = Profession ???]
Do not underestimate the choice of profession !. Not everything can be resolved with axes or magic. Knowing how to untangle knots, follow traces, recognize herbs or diseases make the character an expert, they create a profession. You don't have to define the character only based on the Skills he has but based on what and how well he can do it. A low-level but survival-savvy character will always be more useful than an experienced fighter when traversing a desert. \end{tcolorbox} \end{changemargin}

\subsubsection{Fifth competence} \label{quintacompetenza}

A player at the creation of the character can also ask to lower a competence from 2 points to 1 point and the one from three points to 2 points, to have a fifth competence to 1 point, as long as it is consistent with the character's story.

\end{multicols}

\medskip

\textbf{Table: List of Competences and relative Use Characteristics} \index{List of Competences and relative Use Characteristics}

\medskip

\begin{tabular}{lllll}
	\textbf{Strength} & \textbf{Dexterity} & \textbf{Intelligence} & \textbf{Wisdom} & \textbf{Charisma} \\
	\toprule
Climbing & 	Acrobatics & 				Arcana & 					Riding & 		Diplomacy \\
Intimidate & 		Escape Artist 		& Craft * 			& \textit{Awareness} & Entertain \\
Swimming & 		Fairy Hands 			& Knowledge * 			& Animal Handling & Tricking \\
Jumping & 		Moving Silently & Disabling Devices 		& Nature 			& Local Traditions \\
& 				Hiding 				& Herbalism 			& Orientation 	& \\
& 				Using rope 				& Faking	  		& Perceiving Emotions & \\
&				 						& Assess 				& First Aid & \\
&& 																& Follow Tracks 	& \\
&&										 						& Survival 	& \\
&&& 																			& \\
\end{tabular}

The \textbf{Knowledge} must be explained on which topic it concerns: Dungeon, Law, Languages, Planes, Occult, Architecture and Engineering, Nobility and Heraldry, Myths and Legends, Religion, History, Geography ...

\begin{multicols}{2}

At each \textbf{level after the first} distribute a number of points equal to half the Intelligence score +1 (Int / 2) +1 with a minimum of 1 point, among the skills already known or perfected in the adventure or learned from scratch.

\textbf{No Basic or Active skills can have a score higher than character level + 2.}


\subsubsection{Awareness} \label{consapevolezza}

A skill that all characters have is \textbf{Awareness}, which is the ability to perceive the environment around them. This skill has a fixed score of 1/3 of the character's level (rounded up) and can be increased to a maximum of 2/3 of the level through Skills such as Perceptive.

More than using Awareness to search for information, players should ask questions, investigate, poke around, speculate and confront, and not just ask for an Awareness roll to find something.

\subsubsection{Learning new skills, professions} \label{apprenderenuovecompetenze}

A character can learn a new skill by studying / practicing at least 4 hours a day for at least 4 months with a teacher who has a proficiency score equal to or higher than what the character is aiming for. After this time, the player can award one point to the basic skill he applied for.

To learn a new profession he must spend at least 6 months for 6 hours a day with those who practice that profession. After 6 months, the character acquires the 4 skills of the profession. If he already has some of those skills, he increases his score by 1 for each skill he already has.

\subsubsection{Skills and their areas of use} \label{competenzeambitidiutilizzo}

The Skills and their areas of use are briefly described. They are general indications on what to use the skills. The number of Actions necessary to carry out the typical check is also indicated, obviously more complex uses require more time and Actions.

The actions required for the check may vary depending on the character's ability and the complexity of the trial.

In any case, always remember to carefully evaluate how the player declares to carry out the actions to understand their duration and effects. \\

\textbf{Acrobatics (DES)}: This skill is for maintaining balance on narrow or precarious surfaces, for diving, rolling, somersaulting, somersaulting, overcoming obstacles as well as falling and not getting hurt. 1 Action.

\textbf{Arcana (INT)}: With this proficiency one is proficient in magic and spells, with magical objects and is able to identify the spells being cast. 1 Action.

\textbf{Climb (FOR)}: With this skill you can climb vertical surfaces, from city walls to rock faces. It is related to the action of movement.

\textbf{Craft (INT)}: It is necessary to specify the type of Craft in which you are competent. One is competent, but not at the level of Profession, in a form of craftsmanship.

\textbf{Escape Artist (DES)}: With this skill you can free yourself from ties and handcuffs. 1 Action every 10 of DC. With 6 points the time is 1 Action every 15 DC, with 12 it is 1 Action every 20 DC.

\textbf{Ride (SAG)}: With this skill you can ride professionally and give commands to your mount. 1 Action.

\textbf{Awareness (SAG)}: to seek, notice, notice. It is something active. 2 Actions. Using 1 Action imposes a -1d6 penalty on the check.

\textbf{Knowledge of Architecture and Engineering (INT)}: You are an expert builder and you know how to evaluate the structure of buildings. You also know how to recognize architectural styles and create interior and furnishing projects. 1 Action.

\textbf{Knowledge of Dungeons (INT)}: With this skill you have knowledge of aberrations, caves, underground exploration, oozes. 1 Action.

\textbf{Knowledge of Geography (INT)}: With this skill you have knowledge on climate, population, land, territories, nations and borders. 1 Action.

\textbf{Knowledge of the Law (INT)}: With this skill the Law of a region is known. You are an expert in knowing the rules and the quibbles. Cases are known to be cited, and other hustlers and judges are known. 2 Actions.

\textbf{Knowledge of Languages (INT)}: Each point in this skill allows you to learn a new written and spoken language. A good Language score helps you understand unfamiliar languages and make yourself understood. It is also used to understand complex texts. Variable cost.

\textbf{Knowledge of Myths and Legends (INT)}: There is a real passion for traditional and more remote myths and legends. Learn about locations, history and legendary creatures. 1 Action.

\textbf{Knowledge of Nobility and Heraldry (INT)}: Know noble lines, houses, rumors, heraldic coats of arms, personalities and major possessions and treasures. It also applies to famous and important people. 1 Action.

\textbf{Knowledge of Planes (INT)}: With this skill one is an expert in Planes and their inhabitants. 1 Action.

\textbf{Occult Knowledge (INT)}: With this skill one is expert in the occult, fiend creatures. 1 Action.

\textbf{Religion Knowledge (INT)}: With this skill you have knowledge about Patrons, mythology, Celestials, Undead, sacred symbols, ecclesiastical tradition, feasts and liturgical celebrations. 1 Action.

\textbf{Knowledge of History (INT)}: With this skill you have knowledge of History such as wars, migrations, colonies, foundations of cities, important events .. 1 Action.

\textbf{Diplomacy (CAR)}: With this skill you can resolve disputes, and gather valuable information and rumors from people. Competence is also used to negotiate effectively with the right etiquette and conduct suited to the controversial situation. Variable cost.

\textbf{Disable Devices (INT)}: With this skill you can disarm Traps and open locks, sabotage simple mechanical devices, such as catapults, wagon wheels or doors. 1 Action every 10 of DC. With 6 points the time is 1 Action every 15 DC, with 12 points it's 1 Action every 20 DC.

\textbf{Herbalism (INT)}: With this skill you have knowledge of how to recognize and prepare natural potions and poisons. Scoring applies to checks to brew potions. Recognize Natural Potions 1 Action for every 10 of DC. With 6 points the time is 1 Action every 15 DC, with 12 points it's 1 Action every 20 DC.

\textbf{Falsify (INT)}: With this skill one knows how to falsify works of art, maps, signatures ... 1 Minute

\textbf{Handling animals (SAG)}: With this skill it is possible to train and tame animals. 1 minute every 5 of DC. With 6 points the time is 1 minute every 10 of DC, with 12 it is 1 minute every 15 DC.

\textbf{Intimidate (FOR)}: Intimidate relies on the physical approach to persuade the person concerned. 2 Actions. With a score of 12 it costs 1 action.

\textbf{Deceive (CH)}: The Deceive skill can be used to Bluff (thus telling lies) or Persuade (adapting the truth) in order to convince the person concerned of one's words. Variable cost.

\textbf{Entertain (CAR)}: With this skill one is expert in an artistic expression, from singing to acting, from dancing to playing musical instruments. The form of entertainment must be specified. Variable cost.

\textbf{Hands of Fairy (DES)}: With this skill you can pickpocket, draw a hidden weapon, or perform other actions unnoticed. 1 Action.

\textbf{Move silently (DES)}: With this skill you are able to move without causing noise. 1 Action.

\textbf{Hide (DES)}: With this skill one is able to go unnoticed while standing still. 1 Action.

\textbf{Nature (SAG)}: With this skill you have knowledge of Animals, Fairies, seasons and cycles, weather, plants. 1 Action.

\textbf{Swim (FOR)}: With this skill one is able to swim, even in stormy waters. Without competence one knows how to stay afloat in placid water. Linked to the action of movement.

\textbf{Orientation (SAG)}: With this skill one has a sense of direction and orientation making it impossible to get lost regardless of the environment in which one finds oneself. 2 Actions.

\textbf{Perceiving Emotions (SAG)}: With this skill you can understand if someone is lying or you can guess their true intentions. 1 Action.

\textbf{First Aid (SAG)}: With this skill you can heal injuries and illnesses. Variable cost.

\textbf{Jumping (FOR)}: With this skill one is an expert and skilled jumper. 1 Action.

\textbf{Follow traces (SAG)}: With this skill you know how to follow the traces left in the environment. 1 Action every 10 of DC. With 6 points the time is 1 Action every 15 DC, with 12 points it's 1 Action every 20 DC.

\textbf{Survival (SAG)}: With this skill you can survive and navigate the wilderness. Proficiency is also used to actively search for traps and pits. 1 minute to look for traps on 3x3 meters, scoring 6 costs 3 rounds, scoring 12 costs 1 round, scoring 18 costs 1 Action.

\textbf{Local traditions (CAR)}: With this skill you have knowledge of the inhabitants (best known), customs, legends, laws, personalities, traditions. It is necessary to specify a geographic region where the knowledge is applicable. 1 Action.

\textbf{Disguise (CAR)}: With this skill one is able to put on make-up and disguise to disguise and look differently. 10 Minutes With 6 points it costs 5 minutes, with 12 points 2 minutes, with 18 points 1 minute.

\textbf{Using Rope (DES)}: With this skill you are an expert in tying and knotting to secure and lock objects or people. 2 Actions.

\textbf{Evaluate (INT)}: With this skill you know how to estimate the monetary value of an object. 1 Action every 5 of DC. With 6 points the time is 1 Action every 10 DC, with a score 12 is 1 Action every 20 DC.

\medskip

\subsubsection{Optional - Don't use core skills} \index{Optional - Don't use core skills} \label{nonusarecompetenze}

Let the players choose the profession and do not score any base Proficiency scores or values.
Think with an open mind and understand, both you Storyteller and you Player, for each situation who has the profession and Skills that are best suited to the check.
The check if relevant to the profession is resolved with a 3d6 + Wisdom + 1 / 2LV, if it is not relevant the Storyteller will reduce the bonus given by the level, using the most appropriate characteristic. Better still based on the description of how the check is done will decide the outcome.

\subsection{Active Skills} \index{Active Skills} \label{competenzeattive}

Each character takes 1 point to distribute in the Active Skills at level.

\medskip

The \textbf{Active skills} are: Magical Proficiency, Weapon Proficiency, Saving Throws (Reflex, Fortitude, Will).

\textbf{Magical Proficiency (CM)}: \index{CM} \index{Magical Proficiency} indicates the ability and proficiency in casting a spell.

\textbf{Weapon Proficiency (WP)}: \index{WP} \index{Weapon Proficiency} is the ability and skill to fight with a melee or ranged weapon.

The \textbf{Saving Throws} are usually increased by Abilities choices but it is always possible to attribute the point taken to level.

\begin{changemargin}{0.3cm}{0.3cm} \begin{enfasi}{There is only one way to train: the right one. (Carl Lewis)

\medskip

Wang Chi: Are you ready?

Jack Burton: I was born ready! (Big Trouble in Little China, 1986 Film)

} \end{enfasi} \end{changemargin}


\subsubsection{Saving Throws} \index{Saving Throws} \label{tirisavellza}

\textbf{Saving Throws} (abbreviated to TS) are used when the character is subjected to exertion, whether of physical or mental fortitude, or exceptional agility. The saving throw score is usually modified by the chosen Abilities. More physical Abilities will tend to improve the character's Fortitude aspect, more athletic or attention Abilities will increase Reflexes, purely mental Abilities will strengthen the character's Will.

The \textbf{Fortitude save} indicates how well you are able to endure physical suffering or attacks on your vitality and health. The value of \textbf{Constitution} is added to the Fortitude saving throws.

The \textbf{Will saving throw} indicates resistance against mind-influence and other magical effects, which is intended to modify your free will in your choices and actions. Added to the Will saving throws a value of \textbf{Wisdom}.

The \textbf{Reflex saving throw} indicates how agile and ready you are to avoid obstacles or spells. Add to the Reflex saving throws the value of \textbf{Dexterity}.

\begin{changemargin}{0.3cm}{0.3cm} \begin{tcolorbox}[title = Non-standard saving throws]
Saving throws with different modifiers may be required, such as a Fortitude saving throw with a Strength modifier or a Will saving throw with a Charisma modifier. The Storyteller will tell you when a different modifier is applied.
\end{tcolorbox} \end{changemargin}

When a saving throw is called, it means making a check on the required Active Proficiency, whether it be Will, Fortitude, or Reflexes.
The check will be carried out by rolling 3d6 + the score in the Saving Throw on Will, Reflexes or Fortitude + the value of the Characteristic linked to the type Active Skill (Wisdom, Dexterity or Constitution) + Abilities + magic bonuses (objects affecting the saving throw) and various modifiers present.

\subsubsection{Weapons Proficiency} \label{competenzaarmi}

The \textbf{Weapon Proficiency} (abbreviated to \textbf{CA}) indicates the competence in using a weapon. Proficiency is directly reflected in the check for hitting the opponent with weapons.

The \textbf{Attack Roll for melee weapons} \index{Melee weapons} resolves with a Weapon Proficiency check (\textbf{WP}) + \textbf{Strength} + any abilities and bonuses magic as opposed to the opponent's Defense (Dexterity + armor / shields / bonus).

The \textbf{Ranged Attack Shooting} \index{Ranged weapons} (bows, crossbows, throwing daggers, javelins, stones ...) is resolved with a Weapon Proficiency check (\textbf{CA}) + \textbf{Dexterity} + any magic abilities and bonuses as opposed to the opponent's Defense (Dexterity + armor / shields / bonus).

When assigning a point to \textbf{CA} it must always be specified which weapon group is taken, if not declared then it is like having taken it in the Simple Weapons group.
Check the list \hyperlink{lista.armi}{Weapon by homogeneous type} (page \pageref{lista.armi}). \index{Homogeneous type}

The character can decide to assign this point to a type he already knows, thus improving his ability on using these weapons or learn another type of weapon.

The player must consider that the better his ability with a type of weapon the easier he can take advantage of it, but he will know fewer weapons.

If the player has not scored any points in \textbf{WP} he may use, without penalty on hitting, only weapons stacked as Simple Weapons.

A character who uses a weapon in the Lists of Weapons he knows or in Simple Weapons will always apply his Weapon Proficiency (WP) value to the attack roll, only when using an unknown weapon will he have penalties.

The \textbf{Simple Weapons} are: Dagger, Light Mace, Club, Spiked Club, Short Foot Spear, Staff, Crossbow (Light), Javelin \index{Simple Weapons}

Using a '\textbf{Weapon without proper proficiency} in the group you belong to imposes a -1d6 on the attack roll. \index{Weapon without proficiency}

In order to use \textbf{Light Armor} you must have at least one point in Weapons Proficiency. \index{Light Armor}

In order to use \textbf{Medium Armor} and \textbf{Light Shields} or \textbf{Medium} it is necessary to have at least 2 points in Weapon Proficiency. \index{Medium Armor}

With at least 3 points in Weapon Proficiency and 1 in Strength they can be used without penalty \textbf{Heavy Armor} and \textbf{Heavy Shields}. \index{Heavy Armor} \index{Heavy Shields}

Using a '\textbf{Armor without proper proficiency} prevents you from using the Dexterity value in Defense and the bonus granted by armor to Defense is reduced by 1. \index{Armor without proficiency}

Using a \textbf{Shield without the proper proficiency} worsens the attack roll by 1 and the shield grants a maximum Defense bonus of 1. \index{Shield without proficiency}

\subsubsection{Magical Proficiency} \label{competenzamagica}

\textbf{Magical Proficiency} (abbreviated to \textbf{CM}) allows the character to know more spells, more powerful, more effective and easier.

A character with high \textbf{Magical Proficiency} can manipulate more spells and with better results.

The value of Magical Proficiency establishes, together with the Adept of Magic Skill and the Ability score, the maximum level of spells that can be cast.

\begin{changemargin}{0.3cm}{0.3cm} \begin{tcolorbox}[title = Tups reaches 4th level!]

Tups has reached the 4th level! Here is how he distributed the points of Active skill.

\textbf{1 level}: +1 weapon proficiency, abilities: devotee's armor (+2 will, +1 reflex), weapon focus (+1 reflex, +1 fortitude)

\textbf{2nd level}: +1 Magical Proficiency

\textbf{3rd level}: +1 Magical Proficiency, Ability: Faithful (+2 Will, +1 Fortitude)

\textbf{4th level}: +1 Reflex saving throw

\textbf{\textit{Total}}: +1 WP, +2 CM, +3 Reflex save, +2 Fortitude save, +4 Will save
\end{tcolorbox} \end{changemargin}

Each point awarded in the Basic or Active Skills (WP, CM, TS) allows you to take advantage of +1 in the relative check (attack roll, magic skill or chosen saving throw).

\end{multicols}

%\begin{center}
%\includegraphics[width=0.25\linewidth]{immagini/giavellottiragazzo4.png}
%\end{center}


\pagebreak

\section{Let's Build the Character} \index{Character}

\begin{changemargin}{0.3cm}{0.3cm} \begin{enfasi}{
"Never forget who you are, because certainly the world will not forget it. Transform who you are into your strength so it can never be yours weakness. Make armor of it, and it can never be used against it you. "(Tyrion Lannister)
} \end{enfasi} \end{changemargin} \medskip

OBSS is a tough, tiring, dangerous, deadly system. Your characters are not heroes, they are not chosen. They are unfortunate that they find themselves in enterprises where perhaps they will survive and it will be at the expense of some comrade. It is not you who chooses the adventure but it is she who drags you impetuously inside. Be strong, brave, witty but not stupid. Survive and claim the Law of the Prize. \textit{Spes ultima dea}!

\begin{multicols}{2}

\lettrine[lines = 2, lhang = 0.33, loversize = 0.25, findent = 1.5em]{A}{s} first prepare the card in front of you and a sheet to take notes and notes.

To create a character try answering these questions, they can help you imagine and shape him.

- Imagine what it looks like

- What is the main trait of the character

- What are his tics, ways of doing, habits

- What are its primary objectives

- A curious thing, a funny thing, an embarrassing thing and a typical expression of the character

- What he is good at, what he is committed to, what he is denied

- The three main flaws and three main strengths of the character

\begin{center}
\includegraphics[width = 0.6 \linewidth]{immagini/Leonidas_I_of_Sparta.png}

\textit{Leonidas of Sparta}
\end{center}

\medskip

He grew up in a family, in a clan, a wanderer, on the street .. what brought him and what choices did he make to get up to now?

What is your typical fighting style and strategy? Magic, Sword, from the rear .. incite the comrades .. escape ...

And last but not least: what is its purpose? what made him leave the house, his certainties .. from a "normal" life and embarked on that of an adventurer?

To get started, read the chapter on Races and identify your character's.

Always remember that this is a cruel world, full of risks, traps and monsters, but also opportunities that can make you powerful or very rich.

\textbf{Not everyone is born \textit{Powerful} but whoever evolves is destined to become one}.

Collect some d6 and roll!

Consult the chapter of \hyperlink{assegnazione.punteggi.caratteristica}{Characteristic} to understand how lucky you were (page \pageref{assegnazionepunteggicaratteristica}).

If you have Intelligence equal to or greater than 2, choose another \hyperlink{linguaggi}{languages} (page \pageref{linguaggi}) spoken / written in addition to the Common, and if you have 3 you can choose 2 more languages.

And if the Characteristic values didn't turn out as you expected then let yourself be guided by the chaos and create something different but equally fun and magnificent.

Switch to Active Skill, here you have 1 point to distribute between Weapon Proficiency, Magical Proficiency and Saving Throws.

Weapon Proficiency helps you hit better. Magical Proficiency is the only thing that allows you to use magic. Also remember that the points in Weapon Proficiency must be declared to which \hyperlink{lista.armi}{weapon list} (pg. \pageref{lista.armi}) they were applied to.

If you have no points in Weapons Proficiency you can only use \hyperlink{armi.semplici}{simple weapons} (p. \pageref{listaarmisemplice}) without incurring a penalty on the attack roll.

You can assign this point to a saving throw as well, but remember that taking an Ability is more effective as it increases saving throws. Your saving throw scores determine your ability to survive and resist trauma and spells

Basic Skills are awarded based on the Profession chosen. Choose it with attention and care. The Profession determines what you can do, and remember that based on the background and profession you choose, you increase a stat by 1, up to a maximum of 4 + racial modifier.

Hit points are equal to 1d4 + Constitution + 3 if you have 1 point in Weapons Proficiency (WP).

At this point choose the \hyperlink{tratti}{Traits} (pag. \pageref{traits}). Do it carefully, you are building your character and the Traits outline the character with strong strokes. Remember that they will be fundamental for the choice of \hyperlink{Patroni}{Patron} (pag. \pageref{Patroni}).

In the sheet on in the Traits write down where there is Patron, mark with an X the Traits that connect you to the Patron you have chosen (if you have chosen it).

Finally, remember that a "lonely" and "cruel" character sounds good in a story where he is the only protagonist, but here you play in a group, do not take Traits in obvious opposition to the others or in any case do not play as \textit{asshole}, otherwise the character will naturally be removed from the other players.

Choose the role \hyperlink{svantaggi}{disadvantage} (page \pageref{svantaggidiruolo}) and if you want disadvantages and \hyperlink{linguaggi}{vantage} (page \pageref{vantaggi}). Remember to play it, otherwise it's not fun and you won't get Experience Points.

If you have any points in Magical Proficiency remember that you must also have taken the Adept of Magic Skill otherwise you do not have access to any List of Spells and Spells (apart from those of the Universal List).

At this point you have to choose which Spells you know.
On your Tome of Magic you can mark a number of spells equal to 2 + your ability modifier for spells, of these spells you can know, or cast, 2 + half the value of the ability modifier for spells.

Go to \hyperlink{abilita}{Ability} (pag. \pageref{abilita}), at the first level you choose two, pay attention to the prerequisites and also to any Ability that your race grants you. Each odd level will take another Ability. Ability choice heavily modifies your saving throws!

Choose the \hyperlink{equipaggiamento}{equipment} (page \pageref{equipaggiamento}), \hyperlink{equipaggiamento.armature.scudi}{ armor} (page \pageref{equipaggiamentoarmature}), \hyperlink{equipaggiamento.armi}{weapons} (page \pageref{equipaggiamentoarmi}), backpack, two torches, a few rations of food .. a soft toy. what seems indispensable to you for the adventure.

Then update the part of the sheet relating to Defense by noting what bonus the armor and shield you are wearing gives you.

Get into the part, allow yourself to play this amazing character. If you ever get tired of playing it and want to try something different, talk to the Storyteller, he will be able to advise you and suggest the best way.
Moreover you have the advantage that in OBSS the classes do not exist, the character grows evolves and learns based on what you do and experience. You can prepare the "build" at the table but you will never be sure that your character evolves as you thought. Let it live and grow!

Finally, remember the Law of the Prize \index{Law of the Prize}. Yeru is ferocious, often evil, even more will want to kill you, yet for those who survive there is the Law of the Prize, a law that not even the Patrons can violate. The Law is quite simple in its basic concept "Whoever survives will go the treasures and the glory".

\subsection{Advance in Level} \index{Level} \index{Advance in Level} \label{avanzamentodilivello}

\begin{changemargin}{0.3cm}{0.3cm} \begin{enfasi}{
But there are things that cannot be understood with reflection, you have to live them. (The Neverending Story, Michael Ende)
} \end{enfasi} \end{changemargin}


Whenever the Storyteller confirms that you have passed the level, several operations must be carried out to update the character.

\textit{First take the sheet, pencil and eraser and dice (at least the d4).}

- Upgrade the Level by increasing it by 1

- Upgrade Experience Points

- Give 1 point between Weapons Proficiency, Magical Proficiency, Saving Throws (Fortitude, Reflexes, Will).

- Increase Hit Points by 1d4 + Constitution and add 3 if you gave 1 point in Weapons Proficiency

- If you have assigned a point in Weapons Proficiency, decide whether to take a new Weapon List or deepen your knowledge of a list already learned.

- At each odd level (3,5,7,9 ..) take a new Ability or upgrade an existing Ability, pay attention to the prerequisites.

- Update the saving throws total based on the points distributed or Ability taken.

- Update the part of the attack rolls according to the new value of the Weapons and Ability Proficiency

- Distributed (Int / 2) +1, with a minimum of 1 point, among the Basic Skills known or learned during the adventures.

- If you have reached level 4, 8, 12 or 16, increase a characteristic score of your choice by 1 (and possibly update your saving throw and attack roll), up to a maximum score of 4 + racial modifiers.

- Update Fate Score (20-level) / 5

- Increase a Trait score as the Storyteller tells you. Check if you have achieved a high enough score to acquire Trait-related powers.

- Check based on the new Magical Proficiency score, the Adept of Magic Skill and the Ability score, the maximum level of the spell that can be cast.

- If you have increased Magical Proficiency you learn 2 new spells or by sacrificing one you can learn two tricks (Level 0 Spells) present in your Tome of Magic. You can also replace a learned spell with another not learned but present in the Tome.

- Updated the second part of the sheet based on the new Magical Proficiency score

\medskip

\begin{center}
\includegraphics[width = 0.9 \linewidth]{immagini/Alexander_and_Bucephalus_-_Battle_of_Issus_mosaic.png}
\textit{Alexander the Great}
\end{center}

\medskip

As you may have noticed, the scores of the Skills are reduced, you take a few points to distribute at a time.
As a player you have the opportunity to prefer a specialized approach or "bet" on a few specific Skills or to dilute the points on several skills to know a little bit of everything and not have malus in the checks (the check is done only with 1d6 + Characteristic if you have no points in the Proficiency).

A tip is also to use Ability, and in particular Expert, which grants you a +2 bonus on Skill checks.

\subsection{Tips for enjoying and surviving the adventures of OBSS} \index{Guidelines for players} \label{suggerimentigiocatori}


\begin{changemargin}{0.3cm}{0.3cm} \begin{enfasi}{
- It takes a plan.

- Since when do heroes need plans? (Final Fantasy XIII)

\medskip

I'm crazy for successful plans! (Colonel John "Hannibal" Smith, A-Team)} \end{enfasi} \end{changemargin} \medskip

These are advice derived from the principles of the OSR that help the characters stay alive.

\medskip

\begin{itemize}

\item
Any fight is potentially lethal. Decide wisely and approach it carefully. Learn to escape, don't be afraid to survive.

\item
There is not everything in the sheet. A character sheet is the perimeter of the character but does not define what he can or cannot do. Get creative and talk to the Storyteller.

\item
Don't fix it all with a throw of giving. Ask the right questions, talk to peers, and carefully describe what you intend to do. Remember that the Storyteller rewards accurate descriptions and alternative actions. Describing how something is done often avoids rolling the dice and therefore failing.

\item
Low characteristics are only low characteristics and not the character. Take advantage of the skills, the skills, make sure you have to roll as few dice as possible to solve the problem.

\item
Improvise, adapt and reach the purpose! (Tom Highway - Gunny, Film). Or how some of my players preferred "Improvise, \textbf{Deceive}, reach the goal"

\item
Live your character fully. Amplify its history brings its past into the present. Help your mates get to know you and the Storyteller build better stories around your stories.

\item
Wise is he who is wise proves it, not he who says it.

One thing that no one can ever take away from you is being heroic, intelligent, resolute, stubborn, stubborn but not stupid. Live the adventure to the full but never be afraid to survive.

\item
Describe in a realistic way what you do, you will help the Storyteller and the companions around you. It is definitely better than saying "I do an Awareness check". Excited in describing the most important actions, the Storyteller will take them into account.

\item
Push your brains out and be creative, alternative, curious but not suicidal or reckless.

\item
And until you can say "\textit{I'm bad, pissed off and tired. I'm someone who eats barbed wire, pisses napalm and can put a ball in a flea's ass 200 meters away}." (Tom Highway - Gunny, Film) so stay in your seat and don't be a braggart, there is always someone bigger and more angry than you.

\item

Always remember that the greater the danger the greater the experience gained. The deeper the dungeon, the greater the treasures and experience gained!

\item
The aim is to have fun, entertain and savor the challenge. Don't create a character who is against other characters or always annoys and hassle. You mediate your desire with the needs of the group, because always and \textbf{only as a group will you survive} and never only as an individual.

\end{itemize}

\end{multicols}

\begin{changemargin}{0.3cm}{0.3cm} \begin{enfasi}{
"The candle lit on both sides lasts half (Anonymous)"
} \end{enfasi} \end{changemargin} \medskip


\begin{center}
	\includegraphics[width = 0.5 \linewidth]{immagini/threasure2.png}
\end{center}

\pagebreak

\section{Skills Rules} \index{Skills Rules} \index{Skills}

\begin{changemargin}{0.3cm}{0.3cm} \begin{enfasi}{
The law must be short, because it is easier for people bad to practice it remember. (Seneca)} \end{enfasi} \end{changemargin}

\begin{multicols}{2}

\lettrine[lines = 2, lhang = 0.33, loversize = 0.25, findent = 1.5em]{T}{he} \textbf{Skills checks are performed by rolling 3d6, at the result of the dice is added the score of the competence (basic or active) and of the connected characteristic and any magic and circumstance bonuses or Skills, the result obtained must be communicated to the Storyteller, who will compare it with the difficulty (DC) of the check }.

When you have to establish a difficulty you start thinking that the check must be reported by a "normal" person. Don't think "if I had to do it then the check would be impossible", "if Arsenio Lupine does it, the check is very easy". Start from the assumption that the difficulty must contain all the circumstantial elements.

Think if it rains, there is little light, the character is running, he is injured, he does things in a hurry and also the complexity of the thing he has to do, jumping a 10-foot ditch is not like a 10-foot ditch in the dark, without shoes, in the rain and chased and with pockets full of coins ...

Deciphering ancient writing may be a piece of cake for an experienced linguist, but for a "normal person" who has no idea what the proof may be facing is simply impossible. This "impossible" is your DC, the difficulty of the check.

And don't be scared if the characters fail the checks, it will make the adventure more interesting and allow the Storyteller to introduce facts, clues and new adventures.

%\medskip
%\begin{center}
%\includegraphics[width=0.8\linewidth]{immagini/master2.png}
%
%\textit{The Master of the Gamblers}
%\end{center}

\medskip

\textbf{When you have to check for a basic skill in which you are unprepared, ie you have no points, you only need to roll 1d6 + score of the related ability}. \index{Test without skills}.


When -1d6 is written it means that one less die is rolled (or two if it is -2d6), likewise if it is written + 1d6 one roll an additional 6 and add up.

\begin{center}
	\includegraphics[width = 0.6 \linewidth]{immagini/Foster_Bible_Pictures.png}

	\textit{Bible Pictures and What They Teach Us. Feel Awareness, Move Silently}
\end{center}


The table below is used to relate the difficulty to the minimum skill necessary to succeed in the check with an average shot (a score of 10 by casting 3d6). Use these indications to get an idea of the difficulty scales.

The Storyteller will not tell you give me a check on difficulty 10, but he will say that the check does not present elements of particular difficulty.

%\begin{center}
%\includegraphics[width=0.9\linewidth]{immagini/difficulty.png}
%
%\textit{A City on a Rock, long attributed to Goya, is now thought to have been painted by 19th-century artist Eugenio Lucas Velázquez. Ottima prova di falsificazione}
%\end{center}
%\medskip

\medskip

\textbf{Table: Difficulty class} \index{Difficulty class table}

\medskip
\begin{tabularx}{0.45\textwidth}{lll}
\textbf{Diff.} & \textbf{Description} & \textbf{Level} \\
\textbf{DC} & \textbf{difficulty} & \textbf{Competence} \\
\toprule
5 & Extremely Easy & Mediocre \\
10 & Easy & Normal \\
15 & Normal & Good \\
20 & Difficult & Great \\
25 & Very difficult & Excellent \\
30 & Heroic & Amazing \\
35 & Nearly Impossible & Epic \\
40 & Impossible & Beyond Human \\
\end{tabularx}

\begin{changemargin}{0.3cm}{0.3cm} \begin{Storytellere}
Avoid asking for proof when players declare HOW they do the test, how and where they look for, what dialogue they strike up to intimidate the target. Carefully evaluate how the player describes what they do because this is already the check. It is not only to speed up the game, it serves to stimulate players to think fully and to immerse themselves in the character and the environment. In addition to making the game more dynamic, all players will participate in the situation and collaborate by declaring what and how they act. Always use common sense and save dice rolls!
\end{Storytellere} \end{changemargin}

\bigskip

If you have to check a Characteristic you must roll 3d6 and add the score of the Characteristic and other modifiers. Communicate this result to the Storyteller who will compare it with the difficulty (DC).

\subsection{Pass or fail the check} \index{Pass or fail the check by so much} \label{superareofallirelaprova} \index{Critical success in the checks}

Whenever the check is successfully passed (by 10 or more compared to the necessary difficulty) the Storyteller can decide to give more information, grant bonuses to subsequent actions (+ 1 / + 2) .. anything that can enhance how easily the check is been exceeded.

Conversely, if the check fails by 10 or more than the necessary value, the Storyteller could describe how miserably the check failed and how the bad result affects the action or subsequent ones.

If the sum of the dice is 0 (zero) because you rolled 3 times 1 the check is not automatically failed, it will simply count your Proficiency and Characteristic score to determine if you have passed the check or not. These situations do not apply to the attack or saving throw check, you will find the specific rules in the combat and spells chapters.

\begin{changemargin}{0.3cm}{0.3cm} \begin{Storytellere}
This award clearly clashes with the suggestion to "reward" the HOW to do and not the WHAT. The Storyteller, even on the basis of the group he has to manage, will have to juggle well and seek a balance. \end{Storytellere} \end{changemargin}

\subsection{Awareness} \index{Awareness} \label{consapevolezza2}

Awareness is one of those skills that comes into play very often.

Make sure that the questions and the reasoning of the players reveal the clues, an Awareness check can be made whenever there is to look for something not obvious, something that must be sought otherwise it is not perceptible or intuitive, something that players want to find but don't ask the right question.

\begin{changemargin}{0.3cm}{0.3cm} \begin{tcolorbox}[title = It's not just the card!]{
Do not necessarily look for the solution in the card. Use your ability to imagine, to solve, to intuit to go out and solve situations. The card represents only a small part of what your character can do. } \end{tcolorbox} \end{changemargin}

\subsection{Opposing checks} \index{Opposing checks} \label{proveopposte} \index{Opposing checks}

There are situations in which the character must make a check in opposition to an opponent, for example moving silently behind a guard, stealing from the merchant's pockets, intimidating the orchetto to get directions ..

In these cases the character and the Storyteller perform a check, whoever gets the highest value wins, in case of a tie the one who has the highest value in the Competence, then in the Characteristic and finally the possible "opponent" wins.

To \textbf{speed up} the game you can decide to set the difficulty (for the \textit{player}) to 10 + Saving Throw / Proficiency / opponent's modifiers ... avoiding rolling the dice for \textit{enemies} and providing a static DC to beat. \index{Static DC in opposing checks}

\bigskip

\textbf{Some opposing proof examples}

- Fooling Someone: Fooling Vs Perceiving Emotions

- Disguising to Look Like Someone Else: Entertaining Vs Awareness

- Create a Fake Map: Fake Vs Evaluate

- Hiding: Hiding Vs Awareness

- Moving Silently: Moving Silently vs Awareness (as long as not seen)

Intimidate: Intimidate vs Will saving throw (with Charisma modifier)

- Stealing: Mani di Fata Vs Awareness, or Mani di Fata if possessed (evaluate bonuses and malus well)

- Untie from Ropes: Use Ropes Vs Fugue Artist

- Arm wrestling: opposed check Fortitude saving throw (with strength modifier)

\begin{changemargin}{0.3cm}{0.3cm} \begin{Storytellere}
Don't let trials rule your game. \textbf{Have the players play}, have them recite, let them participate and according to what they say decide whether the check has passed or not.

If they tell you "I convince the guard to let us through" do a check of Intimidate (or Diplomacy), if instead they engage in a convincing dialogue you can consider that the check was done with a positive result (or negative if they were unable to argue!) Reward the HOW more than the WHAT.
\end{Storytellere} \end{changemargin}

Whenever the opposite check concerns a \textbf{trait}, have a saving throw be made as an opposed value by evaluating the type and trait to use.

\subsection{Advantages and Disadvantages} \index{Advantages} \index{Bonus} \index{Malus} \index{Disadvantages} \label{vantaggi}

\begin{changemargin}{0.3cm}{0.3cm} \begin{enfasi}{Audentes fortuna iuvat, Virgilio) } \end{enfasi} \end{changemargin}


Depending on the circumstances, the Storyteller may grant you a bonus or a disadvantage on the check.

The value in \textbf{dynamic checks} \index{dynamic checks} is to be used when the check is made by rolling 3d6, in this case you can add bonuses (+2) or even roll more dice (+ 2d6) or if at a disadvantage, dice fewer, up to not rolling (with a 3d6 penalty).

If the accumulated penalties bring the dice of the check below zero, only the value of the Competence and Characteristic is counted.

We mean \textbf{fixed value checks} \index{fixed value checks} when it is not necessary to roll dice (eg Defense), in this case the score increases / decreases by the indicated number.

Try to always stay between these values of advantage and disadvantage, otherwise you can say that the check is directly successful or failed.

The player can still request to carry out the check even if the result is certain.

\medskip

\textbf{Table: Bonuses and Malus, Advantages and Disadvantages}: \index{Table Bonuses and Malus, Advantages and Disadvantages}

\medskip

\begin{tabular}{lll}
\multirow{2} *{\textbf{Advantage / Disadvantage}} & \multicolumn{2}{c}{\textbf{Evidence}} \\
\cmidrule (lr){2-3} & \textbf{Dynamics} & \textbf{Fixed} \\
\toprule
Light Bonus & +1 & +1 \\
Normal bonus & +2 & +2 \\
Strong bonus & + 1d6 & +4 \\
Very strong bonus & + 2d6 & +8 \\
Slight disadvantage & -1 & -1 \\
Normal disadvantage & -2 & -2 \\
Strong disadvantage & -1d6 & -4 \\
Very strong disadvantage & -2d6 & -8 \\
\end{tabular}

\begin{changemargin}{0.3cm}{0.3cm} \begin{Storytellere}
The bonuses and penalties in the roll of 3d6 are more \textit{important} than in the check made with the d20. Try to always remain within the range of + -2 and only in particular situations of effective and strong advantage or disadvantage apply greater bonuses or penalties.
\end{Storytellere} \end{changemargin}

\subsubsection{Time factor} \index{Time factor} \label{fattoretempo}

\textbf{If a character is not in difficulty or pressure} \index{No time problems} \index{Taking 10} in making the check can take 10 (+ Abilities + skills ..), that is do not roll the dice and consider that he has rolled 10 on the dice. The action takes 10 rounds.

\textbf{If the character has no compelling time limits}, ie he can spend at least 10 minutes working on it (60 rounds) he can consider taking 15. Or as if he had made the check and rolled 15 with the 3d6.

\textbf{If time becomes a factor not to be considered}, i.e. the character has at least 1 hour to think and work and has no penalty considering rolling 18 (but there is no dice explosion even if the total is 18).

If you want to take these values ask the Storyteller, he will be the one who will allow you or not to use these scores, based on the situation, urgency, danger of what surrounds you. Cracking a door in a dungeon asking for 10 requires extreme coolness and recklessness.

\subsubsection{Helping Another} \label{aiutarealtro}

\index{Helping Another} You can help a friend in a check by giving him support and suggestions. The helper must make a check at half difficulty (example if the engaged character has to make a check at difficulty 25, the helper does it at difficulty 13), if he succeeds he gives a +1 to the partner's check. If he rolls a critical hit (+10) then the bonus is +2.

Multiple characters can help the same friend; bonuses of this type can be accumulated up to a bonus equal to a quarter of the difficulty to beat (eg +6 in the case of difficulty 25).

The Storyteller will evaluate the possibility that more than one character provides help considering spaces, ways and times (it is not easy to help someone thread a thread through the eye of a needle).

\subsection{Tests made by the Storyteller} \label{provefattedalStorytellere}

Avoid doing the checks yourself instead of the Players. Be descriptive but don't go and tell the Player that they might need proof of something. Should it be necessary to perform checks secretly from the player, do not roll any dice but add to 10 the value of the Characteristic and the competence score or the value of the saving throw in question of the subject and compare the result with the difficulty of the check.

\subsection{To Roll or Not to Roll Dice} \label{tirarenontiraredadi}

Do not roll for checks that have no chance of failing, for checks that do not have \textbf{problems} if they fail or can be retried without problems. Roll the dice whenever the check could have a result \textbf{spectacular} or \textbf{failure}, make the player enjoy success or fear critical failure.

\subsection{Golden Rules} \label{goldenrules}

\begin{itemize}

\item
\textbf{Blast of 6}: \index{Blast of 6} in basic and active skills checks. If you roll a dice with a 6, you add it up and roll it up and continue like this if you still roll a 6.

\item
\textbf{Rolling a 1}: \index{Rolling a 1 in checks} is also bad for basic and active skills checks, ie it does not stack. It's not an automatic failure it's just a very low roll (zero + proficiency value)

\item
\textbf{Attempt your luck} \index{Attempt your luck}: I can subtract 4 points between the skill score and the Characteristic to add a 1d6 to the roll (attack roll, saving throw, proficiency check)
\end{itemize}

Use the \textbf{Golden Rules} to your advantage! Dare, try, take risks when the situation does not allow other solutions!

\subsubsection{Optional - Partial Success} \index{Partial Success} \index{Trial with Risk} \index{Optional - Partial Success} \hypertarget{successoparziale}{} \label{successoparziale}

The Storyteller may also decide to rate a failed trial as a partial success.

If the check fails by 1 it can be considered successful even if with a slight problem, if it fails by 2 it carries with it a serious problem if it fails by 4 it succeeds with a critical problem, if it fails by more than 4 the check is not in any case successful. Applied to skills such as Knowledge, you can decide to provide information that is not complete or partly true and false, or even if it is about opening a lock, the pick in the lock could be broken!

It may also be the player who requests a \textbf{"Proof with Risk}" in situations of particular tension and urgency in which the final result is more important than the risk taken. This request must be made before rolling the dice.

\subsection{Examples of Skills check} \label{esempiprovecompetenza} \hypertarget{esempiprovecompetenze}{}

\textbf{Fantastic check}\index{Fantastic check}. The player is invited to find uses, solutions, approaches that go beyond the most obvious checks. Be creative and describe to the Storyteller the wonderful action you want to do! He will determine from your description of the action to check and how difficult it can be.\\


For \textbf{recognize a magical object} \index{Recognize a magical object} and its abilities, a check of \textbf{Arcana} on difficulty 25 is required. This can give general indications on the powers and areas of use, but only through \hyperlink{incantesimoidentificare}{Identificare} can you learn the details, magic bonuses and charges. \textbf{1 minute}. Arcana scoring 6 costs 3 rounds, 12 costs 1 round, 18 costs 1 Action.

\medskip

\textbf{Recognize a spell} \index{Recognize a spell} while it is being cast is a check of \textbf{Arcana} at DC equal to the spell's 11+ level. It costs one \textbf{Reaction}. If done in conjunction with casting a counter spell, it costs no Reaction.

\medskip

\textbf{Acrobatics} \index{Acrobatics} \textit{Penalties due to armor}

A successful Acrobatics check with DC 15 allows the character to halve the damage when falling from less than 30 feet (\textbf{Reaction}).

\medskip

\textbf{Climb / Climb} \index{Climb} \index{Climb} \textit{Penalty due to armor.} \\

\begin{tabularx}{0.45\textwidth}{Xl}
	\textbf{Example of Surface} & \textbf{DC} \\
	\toprule
	Raw wall with handholds, protruding bricks & 10 \\
	A tree, a rope without knots & 15 \\
	A smooth wall with grips & 20 \\
	A wall with very few holds & 25 \\
	A smooth natural wall without grips & 30 \\
	You can lean against 2 opposite walls & -8 \\
	You can lean against 2 corner walls & -4 \\
	You can use a string & -8 \\
	Slippery surface & + 5 \\
	Rope with knots & -4 \\
	Descending with a rope & -4 \\
\end{tabularx}

\smallskip

Using a rope \index{Climbing a short one} \index{Climbing a rope} or climbing is equivalent to moving in doubly difficult terrain. Increasing the DC by 10, to declare before rolling the check, moves at half speed (difficult terrain only). If you fail the check, you take the Action without moving. If you fail by 10 or more you lose your grip and fall, you can make a Reflex saving throw at the same difficulty to grab something, if you fail the save you also fall to the bottom.

\medskip

For \textbf{recognize a monster}, a particular creature is made a Knowledge check (\textbf{1 Action}) on:

\textit{Arcana}: Giants, Constructs, Spirits, Monstrosities, Aberrations, Dragons

\textit{Plans}: Elementals

\textit{Occult}: Fiend, Spirits, Undead

\textit{Religion}: Spirits, Undead, Celestials

\textit{Dungeon}: Aberrations, Monstrosities, Oozes, and subterranean creatures

\textit{Nature}: Beasts, Plants, Fairies

The DC of the check is equal to the creature's Challenge rank + 10 + any rarity factor (2/8, uncommon, rare, very rare, legendary), a non-critical check failure will grant less information.

\medskip

For \textbf{identify a potion or natural poison} \index{Identify Poison} \index{Herbalism} \index{Identify Potion} a check of \textbf{Herbalism} at DC 12 + rarity factor is required of the plant. It costs 1 Action for every 10 of DC. With 6 in Herbalism the time is 1 Action every 15 DC, with 12 points it is 1 Action every 20 DC. If you fail the check by 10 or more you have come into contact / ingested part of the potion and in case of poisons you suffer the effects.

\medskip

\textbf{Intimidate} \index{Intimidate}. Character uses \textbf{2 Actions} and counters the check against the Will saving throw with bonus from the opponent's Charisma.
If the saving throw fails, the opponent until the start of the next round has -2 on the attack roll and -2 on the defense against the intimidating one. The opponent must have Intelligence -3 or more and for each size of difference he takes a +/- 2 to the check.

If the survivor of the Intimidate check succeeds critically, by 10 or more, it is that they have tried to intimidate that they must make a Will saving throw with a DC 10 Charisma modifier + the opponent's Challenge Rating (or Level) or suffer. the same penalties as if he had been intimidated.
If the opposing roll is critically successful (the opponent fails the saving throw by more than 10) the duration of the effect lasts until the end of the fight.

\medskip

\textbf{Taming an animal} is a check of \textbf{Handling Animals} at DC 12 + CR of the animal. 1 minute every 5 of DC. With 6 points the time is 1 minute every 10 of DC, with 12 it is 1 minute every 15 DC. The creature must have Intelligence -3 or higher.

\medskip

\textbf{Move Silently} \index{Move Silently} \textit{Penalty due to armor.}

The character who moves silently is as if he were on difficult ground. The Move Silently check is compared to the opponent's Awareness that he may have heard but must not have seen. Moving at full speed imposes a -2d6 penalty on the Move Silently check.

\medskip

\textbf{Hide} \index{Hide} \textit{Penalty due to armor.}

Everyone can hide behind a low wall, for example, but only those who have the competence can be sure that it will hardly be noticed. The check of Hiding is opposed by the check of Awareness. You can be seen on a DC 13 Awareness check if you don't use the Hide skill. Using \textbf{1 Action} you can try to hide from opponents' sight. It is not possible to hide if the environment does not allow it, for when your check may be high you cannot hide if there is not something that can hide you.

You can \textbf{combine the Move Silently and Hide Actions}. The character uses 1 Action to Hide and one or more Actions to Move Silently, always on difficult terrain. Performed both checks takes the worst for the control of "observers".

\medskip

\textbf{Swimming} \index{Swimming} \textit{Penalty due to armor}

In calm water DC 10, in choppy water it has DC 15, in very choppy water DC 20, stormy DC 25. Proof is needed to float or swim. Swimming in water is considered difficult terrain.

\medskip

Any check on \textbf{Profession} is made with 3d6 + Wisdom + half the level.

\medskip

\textbf{First Aid} \hypertarget{prontosoccorso}{} \label{prontosoccorso} \index{First Aid}. A successful check (DC 15) recovers 1d4 hit points \textbf{after a fight} or grants a +2 to a Fortitude save to resist a poison. To be done within 1 turn of the end of the fight. Cost \textbf{2 minutes}. With a score of 6 it costs 1 minute. Scoring 12 costs 3 rounds, scoring 18 costs 1 round.

A successful check (base DC 12) reduces damage from \hyperlink{sanguinamento}{\textbf{Bleeding}} by 1. For each Bleed value above 1 the difficulty increases by 2. Cost \textbf{2 Actions}. A 1 minute treatment guarantees 1 success, without trial. Each critical hit reduces the bleeding by one more point.

A successful check (base DC 13) for \textbf{caring for 8 hours} of a patient causes them to recover double their hit points ((2 * WP + Constitution + CM) * 2 with a minimum of 4) and grants a new Fortitude save to defeat any natural disease or poison already in progress.

\medskip

\textbf{Jumping} \index{Jumping Table} \textit{Penalty due to Armor.} \textbf{1 Action} \\
%\begin{tabular}{lc|lc}
%\textbf{Salto in Lungo} & \textbf{DC}&\textbf{Salto in Alto} & \textbf{DC}\\
%\multicolumn{2}{c}{Lunghezza} &\multicolumn{2}{c}{Altezza} \\
%\toprule
%1.5 m                              & 5 & 0.02 m                           & 4\\
%3 m                                & 10 &0.5 m                            & 8\\
%5 m                                & 15 & 1 m                              & 12\\
%7 m                                & 20 & 1.5 m                            & 16\\
%+1,5 m                             & +5 &+0.5 m                           & +4\\
%\end{tabular}

The \textbf{long jump distance} is equal to 0.50m + 0.50m for check result over 8. Eg if in the jump check I make 11, the jump will be long 0.50 +[(11-8) * 0.50] = 2m, is rounded down, with 16 in the check 0.50 +[(16-8) * 0.50] = 4m. The \textbf{distance jumped up} follows the same rules but the base jumped is 0.25m + 0.25m for score over 8, rounded to half a meter. In a \textbf{long jump} the highest point of the jump is equal to 1/4 of the jump length. If you perform a 4 meter long jump in mid jump you are 1 meter high. \\

Descending from less than 1m does not use Actions. If you do not have at least 3 meters of run-up you will skip the half. In the long run one jumps to the maximum of one's movement and in the upper half. \\

Damage (p. \pageref{cadute}): 1d6X fall height (in meters). Acrobatics DC 15 1/2 damage (for falls within 9m).

\medskip

\textbf{Survival} \index{Survival}

\smallskip

\textbf{Pursue a creature}:

\begin{tabular}{ll}
Basic difficulty & DC 10 \\
\toprule
If the ground is very soft & DC +5 \\
If the ground is soft & DC +10 \\
If the ground is stable & DC +15 \\
If the ground is hard & DC +20 \\
Every 3 creatures chased & DC -1 \\
According to size & DC + -4 \\
Every 24 Hours Past / Low Visibility & DC +2 \\
Every hour of rain & DC +4 \\
%Visibilità scarsa&DC +2\\
Try to hide the tracks & DC +5 \\
\end{tabular} \\

The check is modified by covers, more covering more difficult to seek creature.

Survival can be used in place of \textbf{Deactivate Gadgets} with a -2d6 to deactivate traps or locks 1 Action per DC.

For every four points scored on the Survival check beyond 13 the character is able to \textbf{get food} for himself and another person as long as he is in a life-sustaining environment.

\medskip


The Check of \textbf{Evaluate} \index{Evaluate} is a basis for the item's rarity, DC 12 + 2 common, 4 uncommon, 6 rare, 12 very rare, 16 legendary. \textbf{3 Actions}. With a score of 6 it costs 2 Actions, with 12 it costs 1 Action.

\begin{changemargin}{0.3cm}{0.3cm}\begin{Storytellere}
Checkd are a very important part of the game and your approach to them, like the fighting, determines the kind of adventure.

Always be very broad-minded, listen to the player and perceive his enthusiasm, try to understand the final intentions, the purposes.
An involved and participatory player spreads his enthusiasm to other players too!

Listen carefully to the proposals they make to you even if they seem \textit{not very sensible} or \textit{simply crazy}, on the contrary nothing prevents you from warning of the potential dangerousness of the choices but in the same way try to understand how in the regulation you can make the trial.

And if you don't find a rule, go with \textbf{common sense}, with similarity to other actions, get involved in the description of the facts, be theatrical when necessary! The good spirit of the group will certainly benefit!

And if the players tell you to take an Awareness check or "I make a jump check" follow them in their intentions but at the same time do everything to involve them further.

\textbf{There is not always a rule for everything but common sense and fun must never be missing}!

\end{Storytellere}\end{changemargin}

\subsubsection{Languages} \index{Languages} \hypertarget{linguaggi}{} \label{linguaggi}

In Yeru each culture is the guardian of its own spoken and written language. Any character who has at least Intelligence -2 speaks the language of his own culture, with 0 he writes it. For each point greater than or equal to 2 he speaks and writes another language that will be chosen when the character is created. For each point in Language Knowledge he speaks and writes another language. Some languages * cannot be spoken except by creatures belonging to that species or cultural groups. A member of a race may very well have as their first language not that of their race if the background justifies it (eg a Dwarf raised in a Goblin tribe).

Use the narrative richness that different languages ​​can give. An adventure, in an environment where no one understands each other languages, can lead to very interesting plots.
f

\end{multicols}

\smallskip

\textbf{Table of Languages} \index{Table of Languages}

\medskip

{\small \begin{tabular}{lll|lll}
\textbf{Cultural area} & \textbf{Spoken} & \textbf{Written} & \textbf{Cultural area} & \textbf{Spoken} & \textbf{Written} \\
\toprule
Human & Common & Common & Dwarven & Dwarven & Dwarven \\ Elvish & Elven & Elven & Gnomic & Gnomic & Gnomic \\ Gnoll & Gnoll & Goblinoid & Giants & Giant & Giant \\
Orc & Orc & Orc & sentient sea creatures & Acquan & Elvish \\
Sentient birds & Auran & Elvish & Wood dwellers * & Silvano & Silvano \\
Druidic * & Druidic & - & Goblinoid & Goblinoid & Goblinoid \\
Celestial & Celestial & Celestial & Hellish & Hellish & Hellish \\
Abyssal & Abyssal & Abyssal & Dragons & Draconic & Draconic \\
Fire Elementals * & Ignam & - & Earth Elementals * & Terran & - \\
Water Elementals * & Acquan & - & Air Elementals * & Auran & - \\
Sign Language * & Sign & Sign & Bury & Depth & Depth \\
\end{tabular}}
\medskip

\textbf{Telepathy} \index{Telepathy} is a means of speaking to any creature that has a language and Intelligence greater than -3. There is no language constraint, telepathy acts as a universal translator.


\vfill

\begin{center}
	\includegraphics[width=0.8\linewidth]{immagini/Pieter_Bruegel_the_Elder-The_Tower_of_Babel.png}

\textit{The Tower of Babel, Pieter Bruegel the Elder. \\
... This is why it was called Babel, because there the Lord confused the language of the whole earth and from there the Lord scattered them over all the earth. (Genesis 11, 1-9.)}
\end{center}

\pagebreak

\section{Combat} \index{Combat}

\begin{changemargin}{0.3cm}{0.3cm} \begin{enfasi}{
Si vis pacem, para bellum ("If you want peace, prepare for war", anonymous)

\medskip

It doesn't matter how you fall, but if and how you get up (anonymous)

\medskip

I am not a hero. No and I never will be. I'm just a bad guy coming paid to beat up guys worse than me. (Deadpool)

\medskip

An eye for an eye ... and the world goes blind (Mahatma Gandhi, NdA his Traits abhorred violence!)
} \end{enfasi} \end{changemargin} \medskip


\begin{multicols}{2}

\lettrine[lines = 2, lhang = 0.33, loversize = 0.25, findent = 1.5em]{T}{he} combat is one of the main phases of an adventure and is when the brave or fearful show off their mastery with weapons or spells.

\bigskip

The combat is divided into 2 phases: \index{Combat}
\begin{itemize}
\item verification of the initiative
\item resolution of actions (movement, attack, various actions ..)
\end{itemize}

\begin{center}
\includegraphics[width = 0.8 \linewidth]{immagini/Achildbookofwarriors.png}

\textit{A child's book of warriors (1907), William Canton}
\end{center}

\subsection{The Initiative} \index{Initiative} \label{iniziativa}

Initiative is a (3d6) check of any inherent Dexterity or Intelligence and Skills you may have.

The player chooses the Trait he prefers. If Dexterity is chosen, reflexes will determine the character's reaction, while Intelligence will guide the ability to grasp the opponent's tactics and anticipate them.

Whoever has the highest initiative between players and enemies starts first and then the others act in descending order, declaring the Actions and executing them. In the event of an Initiative with the same score, whoever has the highest Characteristic score acts first, otherwise the fight will be simultaneous. The initiative is valid for the entire fight and is withdrawn when the opponent changes.

\begin{changemargin}{0.3cm}{0.3cm} \begin{Storytellere} %box Storytellere
Try to make the fight flow naturally. Do not interrupt the flow of actions, but by describing the effects involve the players (and enemies) in the following actions. I recommend reading the article \href{https://theangrygm.com/manage-combat-like-a-dolphin/}{How to Manage Combat Like a \textit{xxx} Dolphin} to understand the method in detail.
\end{Storytellere} \end{changemargin}


\textbf{The Golden Rules do not apply to the Initiative check.}

\subsubsection{Resolution of Actions} \index{Resolution of Actions} \label{risoluzionedelleazioni}

\begin{changemargin}{0.3cm}{0.3cm} \begin{enfasi}{
... the past is the prologue and the future is in your hands and in the my. (Antonio, The Tempest, Shakespeare)}
\end{enfasi} \end{changemargin} \medskip

From fastest to slowest there is the resolution of Actions.

The Storyteller will ask the fastest, the one with the highest initiative, to declare his Actions and act, he will then go on to ask and make other players and enemies act.

In this way the choice of the action takes place when it is the player's round who can also act on the basis of the actions and resolutions already occurred.

\subsubsection{Optional - Initiative Variant} \index{Optional - Initiative Variant} \hypertarget{varianteiniziativa}{} \label{varianteiniziativa}

This variant of the initiative aims to stimulate the diversification of actions based on the situations that arise from time to time. The Initiative check is always calculated as 3d6 + Dexterity or Intelligence, but further modifiers are then applied which are counted round by round based on the Actions you take. The player will declare round by round if the initiative is changed and by what value.

- If a Small weapon is used, the Dexterity value is added.

- If you use a Medium sized weapon, you keep the value of the Initiative rolled.

- If you use a large weapon, you subtract the Dexterity value.

- If the spell you are going to cast has only Verbal (V) components, the Intelligence value is added.

- If the spell you are going to cast has Verbal and Somatic (VS) components, you keep the value of the Initiative rolled.

- If the spell you are going to cast has Verbal, Somatic, Material (VSM) components, the Intelligence value is subtracted.

- If you use a move action before attacking (or casting a spell) the initiative decreases by 2.

- If you use an Immediate Action or Reaction before attacking (or casting a Spell) the initiative decreases by 1.

- If you change your weapon, the initiative decreases by 4 but no Actions are used.

- The enemy uses standard initiative (3d6 + indicated modifier).

\bigskip

This system generates an extremely tactical approach to resolving the initiative. The player is encouraged to change attack methods, weapons or spells based on the opponent in front of him, if this is fast, slow ... if you have to hit first or just hit after, if the action of the partner was resolutive or not.

Initially this system slows down the game but as soon as the players become aware of their options, the initiative and collaboration between the characters become fundamental and supporting elements, not slowing down the flow of actions.

\begin{center}
\includegraphics[width = 0.9 \linewidth]{immagini/Arthur-Pyle_Two_Knights.png}
\textit{Howard Pyle, from the 1903 edition of The Story of King Arthur and His Knights}
\end{center}


\subsubsection{Time (Rounds, Minutes and Turns)} \index{Round} \label{iltempo}

\begin{changemargin}{0.3cm}{0.3cm} \begin{enfasi}{
"Hesitation is the death of advantage" (Magic, VE Schwab)} \end{enfasi} \end{changemargin} \medskip

A \textbf{round} lasts about 10 seconds, which is enough time to act, run, talk… fight. A minute is therefore 6 rounds, and a round lasts 10 minutes (or 60 rounds).

The rounds are used in combat scenes or where the tension must remain constantly high and each Action corresponds to an evolution of the situation.

\begin{center}
\includegraphics[width = 0.7 \linewidth]{immagini/hjford-fight.png}

\textit{\\ Henry Justice Ford, Fairy book - Fairytale illustration}
\end{center}

\subsubsection{Object and ability activation time} \index{Object and ability activation time} \label{temporiattivazioneoggetti}

Unless otherwise specified, an item or Ability that has a certain number of uses per day \textit{"eg once a day"} "recharges" at dawn after use.

\subsection{Actions in Round} \index{Actions in Round} \index{Action} \label{azioninelround}

A character can perform 3 Actions per round, 1 Immediate Action, 1 Reaction Action.

These Actions can be performed in any order you prefer.

The table below shows the main actions that a character can do, they are guidelines to follow. In the chapter dedicated to combat other Actions are listed and their relative costs in Actions.

An Action cannot be interrupted \index{Interrupt Actions} \index{Actions, Interrupt} by another Action, but it can be followed by a Reaction Action or an Immediate Action, if in its own round.

If a character wants to make more attacks while moving around the battlefield he can, for example, use an Action to perform an attack, use a Move Action to move up to all his available movement, and use a last attack action to perform. a last single attack, this second (and single) attack still counts as a multiple attack with the necessary penalties.

It is possible \textbf{delay} one or more Actions \index{Delay Actions} to wait for the scenes to unfold. The character who delays one of his actions acts first among the subjects acting in that initiative value, in subsequent rounds he will continue to act with the new initiative value. Deciding to act on a specific situation must be declared / requested before there is the declaration or execution of that situation, otherwise one acts immediately afterwards.

A player who claims to wait for a certain situation in order to act is equivalent to performing one or more \textbf{Prepared Actions} \index{Prepared Actions}. In this case the character (or enemy) acts after the triggering Action with his Prepared Actions but remains at his original initiative value at the end of the round.

If the character has already taken all the Actions then he can act outside his initiative only through a Reaction, if available. The Reaction Action always comes after the triggering Action.

\bigskip

\textbf{Table: Actions per Round} \index{Table of Actions per Round}


\begin{tabularx}{0.45\textwidth}{Xc}
\textbf{What to do} & \textbf{Actions} \\
\toprule
Performing an attack 				& 1 \\
Perform two attacks 				& 2 \\
Perform more than two attacks 			& 3 \\
Casting a Spell * 			& 2 \\
Perform a Move Action * 		& 1 \\
Click 						& 1 \\
Getting up from prone 				& 2 \\
Helping Someone 				& 2 \\
Exchanging a dialogue with someone * 		& 3 \\
\small{Scambiare poche battute con qualcuno*} 	& 0 \\
Look for something in the backpack 			& 2 \\
Using a hand held object 			& 1 \\
Drinking a potion on the belt 		& 1 \\
Draw / Holster Weapon 					& 1 \\
Take up the shield 				& 1 \\
Using a magic item 				& 2 \\
Test a skill * 		& 1 \\
Hide 					& 1 \\
Focus on a Spell 			& 1 \\
Mount or dismount & 1\\
Action \textbf{I} immediate - Action \textbf{R} action 		& I - R \\
Drinking a hand held potion 			& I \\
Throwing a Hand Held Object 		& R \\
Fall Prone 				& R \\
Recognizing a Spell 			& R \\
\end{tabularx}

\medskip

Attack means both the use of melee weapons and the use of throwing or shooting weapons such as bows, crossbows or throwing daggers. In the case of shooting or throwing weapons, each shot or throw counts as an attack.

It is possible to use an Attack Action and Cast a Spell only if it is sufficient to use a single Action, Reaction Action or Immediate Action, to cast the spell. If the character is under attack then he will be considered Distracted for that round.

\medskip

\textbf{Move Action *}: A Move Action is an Action dedicated to moving. You can move up to all your movement (9 meters for humans, 6 meters for dwarves ..).

\textbf{Cast a Spell *}: Usually 2 Actions are required. The number of Actions required is indicated in the description of the spell. In the chapter of Magic \hyperlink{piumagieround}{regole} are specified (pag. \pageref{piumagieround}).

\textbf{Exchanging a dialogue with someone *}: A dialogue can be a few seconds if not minutes. The Storyteller will evaluate how long this lasts.

\textbf{Exchanging a few lines with someone *}: As long as there are very few lines or a glance does not consume Actions, if this becomes more complex then it uses Actions. The goal is not to interrupt the flow of Actions with a dense dialogue but still allow interaction between the players.

\textbf{Test a skill *}: if they yield a fraction of the round they cost 1 Action, otherwise 2 or more. Check the reported costs in \hyperlink{esempiprovecompetenze}{Example of Skills check}.

An Action "\textbf{Reaction (R)}" \index{Action Reaction} can be performed freely even outside its own round. This Action is usually due to a particular Skill or situation. Unless otherwise indicated, a Reaction Action occurs immediately after the cause that triggers it.

An Action "\textbf{Immediate (I)}" \index{Immediate Action} can be performed freely in your Round, before or after your Action. An Immediate Action is usually granted by particular Skills.

It is possible if not specifically described in the Skill to perform only one Immediate Action and one Reaction Action per round.

\medskip

This \textbf{list is not complete}, take it as a guideline for determining the weight of player decisions and actions.

The \textbf{order} in which the Actions are performed is not important except for logical and physical correlation. The Move Action can be sandwiched between other Actions (move, attack / spell, other action, move).

A character could attack, move and attack again, this second attack would have the penalties described in multiple attacks.
\smallskip


\begin{center}
\includegraphics[width = 0.8 \linewidth]{immagini/Perseus_Fighting_Phineus_and_his_Companions.png}

\textit{Luca Giordano: Perseus turning Phineas and his Followers to Stone}
\end{center}

\end{multicols}

%\bigskip
%\begin{changemargin}{0.3cm}{0.3cm}\begin{Storytellere}Una creatura che ha una distanza di mischia (portata) superiore all'avversario si considera che abbia un bonus di \textbf{+2 al Tiro per Colpire} finché l'avversario non lo raggiunge in mischia.
%
%Questo bonus non si applica con le armi da lancio (archi, balestre, pugnali, asce... per quelle armi che usano la \textit{gittata}). \end{Storytellere}\end{changemargin}

\pagebreak

\subsection{Movement} \index{Movement} \label{movimento}


\begin{changemargin}{0.3cm}{0.3cm} \begin{enfasi}{""A slower piece of furniture cannot be reached by a faster one; since what follows must come to the point that it occupied what it is
followed and where this is no longer (when the second arrives); thereby the first always retains an advantage over the second "(Zeno's Paradox)} \end{enfasi} \end{changemargin}

\begin{multicols}{2}

The movement of a character is given by his size and race and by what he carries, by weights, encumbrances but also spells and magical objects.

The Movement written in the character's race is an indication of how many meters per Action (of Movement) the character can make.

A creature or character may also decide to move faster than usual or run (Dash Action).

The Sprint Action is a particular Move Action, it consists of running for that Action.
If you perform an Action of \textbf{Sprint} \index{Sprint} you double the meters covered (2x9 meters for a human), for a dwarf (6m move) it means to do 12 meters in one Action.
It is also possible to take multiple Sprint Actions (up to 3 in a round, i.e. run 6 times your movement).

The character who makes a Dash Action \index{Dash Action} runs and has a 1d6 penalty on the attack roll and Defense decreases by 4 for the entire round in which he has used the Dash Action and is considered Distracted for casting spells.

It is not possible to move even 1 meter if no Move Actions are spent.

These clarifications make sense and should be used when it comes to fighting and the displacement on the territory, map, is fundamental. During normal movements, while riding or walking free without danger, normal clockwise movement management is used.

In the case of diagonal movement \index{Diagonal movement} \index{Moving sideways} a distance of 1.5 meters per square is counted, in case of rounding on the last square it is done by default, that is to say you go back to last crossed.

By \textbf{Touch Distance} \index{Touch Distance} \index{Touch} we mean a distance that allows you to touch the opponent, so no more than one meter for medium-sized creatures without long weapons or with scope. Touch distance is melee distance when not using long guns.

By \textbf{Melee distance} \index{Melee distance} \index{Melee} we mean a distance that allows hand-to-hand combat (1 meter around the character, or 2 meters in the case of a long weapon) . In monsters this distance is indicated by range, for thrown weapons it is called range.

If not indicated in the opponent / monster the melee / touch distance increases by 1 meter for each size above average. \index{Size and melee distance}

\begin{changemargin}{0.3cm}{0.3cm} \begin{tcolorbox}[title = Combat Range Examples]
Eg for a creature with a spear, the melee range is 2 meters because the weapon is long. For a gnome armed with a hammer, the melee distance is 1 meter.
For a Hill Giant the melee distance (range) is 3 meters. Attacks with bows, crossbows or thrown weapons are referred to as Range.
\end{tcolorbox} \end{changemargin}

When we speak of "\textbf{small square}" \index{Small square} to indicate a distance or an influence we mean a square of the map measuring 1 meter x 1 meter.

\textbf{If moving into "difficult" terrain, a human covers 4 meters per move action (each square crossed counts for two).}

At melee range, a medium-sized creature can have a maximum of 8 medium-sized creatures.

\subsubsection{Optional - Large and Small Creatures in Combat} \index{Optional - Large and Small Creatures in Combat} \label{creaturegrandipiccole}

The table below shows how many creatures a medium-sized creature can surround depending on its size.

\medskip

\textbf{Table: Size, Scale of the Creatures and number per square} \index{Table Size, Scale of the Creatures and number per square}

\medskip

\begin{tabular}{lll}
\multirow{2}{*} \textbf{Size} & \multirow{2}{*} \textbf{Creatures} & \multirow{2}{*} \textbf{n. creature} \\
{Creature} &{in melee} &{in the picture} \\
\toprule
Very small & 100 & 16 \\
Minute & 64 & 8 \\
Lowercase & 32 & 4 \\
Small & 16 & 2 \\
Average & 8 & 1 \\
\end{tabular}

\smallskip
These are typical creature values for the indicated size. There are frequent exceptions.

\end{multicols}

\begin{center}
\includegraphics[width = 0.4 \linewidth]{immagini/camminata.png}
\end{center}

\pagebreak

\subsection{Life and Death} \index{Dying} \label{morire}

\begin{changemargin}{0.3cm}{0.3cm} \begin{enfasi}{Whoever does not know death does not know life. (Grand Hotel, 1932 film)

\medskip

The worthy GM never purposely kills players' PCs. He presents opportunities for the rash and unthinking players to do that all on their own (Gary Gygax)

} \end{enfasi} \end{changemargin} \medskip

\begin{multicols}{2}

Weapon damage is calculated as the sum of the weapon die, Strength (or Dexterity if indicated by Skill) whether positive or negative, Weapon List bonuses, Skill bonuses, weapon dice bonuses, and circumstantial bonuses. \index{How to calculate weapon damage} \index{Weapon damage}

When a character reaches 0 (zero) hit points he is considered unconscious, that is, unable to do anything. A magical cure (spell, potion ...) will bring him conscious and healed to the hit points. A First Aid check, 3 Actions, at DC 12 will bring him to 1 hit point. After one hour, if something hasn't happened to change the situation, the character can make a Fortitude save at DC 15, if successful, it returns to 1 hit point, if it fails it goes to -1 and becomes dying.

A dying character has negative hit points (-1 or less) and is unconscious and \hyperlink{morente}{prossimo alla morte}. He will continue to lose one hit point per round if the value does not reach double the Constitution + 10 and the character will die if not healed.

A spell (spell or potion) of Healing, of any level will bring him to 1 Hit Point, subsequent heals will function normally.

A check of \hyperlink{prontosoccorso}{Pronto Soccorso}, 3 Actions, at difficulty 11 plus the negative hit point value will bring the character to 0 hit points, that is, passed out. Each time the character falls below 0 hit points the difficulty of the First Aid check increases by 2 and increases the Fatigue level.

\begin{changemargin}{0.3cm}{0.3cm} \begin{tcolorbox}[title = Tups is dying]
Eg Tups is badly injured and currently has -6 hit points, Jade decides to try to heal him (after moving him to a safer place). Jade tries a First Aid check to at least stabilize her partner, her difficulty on the check is 11 + 6 or she must overcome with First Aid DC 17 to bring him back to 0 Hit Points (passed out)

A subsequent First Aid check at DC 15 can bring him to 1 Hit Points and a magical cure will heal him for the stated amount.
\end{tcolorbox} \end{changemargin}

A dying character who takes further damage, enemies that rage on the body or spells directed at him or in the area, continues to subtract hit points to see if he gets to die.

Mental-type Conditions \index{Mental Conditions} such as Fascinated, Charmed, Confused end when the character becomes dying.

\begin{center}
\includegraphics[width = 0.8 \linewidth]{immagini/Nuremberg_chronicles.png}

\textit{The Dance of Death (1493) by Michael Wolgemut, Nuremberg Chronicle of Hartmann Schedel}
\end{center}

If an attack or spell brings the character directly to -10-COS * 2, the character dies \index{Immediate Death} \index{Massive Damage} with no chance of being healed.

When a character returns to positive Hit Points after going negative, he loses half of his remaining Magic Points with a minimum of 10 and becomes further \hyperlink{affaticato}{Affaticato}.

When a character reaches negative hit points equal to 10 + double his Constitution score it is \hyperlink{morto}{\textbf{morto}} (-10- (COS * 2)).

\begin{changemargin}{0.3cm}{0.3cm} \begin{tcolorbox}[title = The death of the character]
Try to understand why he died, what are the causes, the mistakes he made. What are the choices that led him up to there. Each character who dies is a personal wound but also experience and awareness. Treasure both you and the whole group. If something did not work, try to understand it together, without accusing or blaming each other but with the conscious spirit that you can improve, everyone.
\end{tcolorbox} \end{changemargin}

Ex. If he has Constitution 2 he will die at -10-4 = -14 Hit Points, if he has Constitution 0 he will die at -10 Hit Points, if he has Constitution -2 he will die at -10 + 4 = -6 Hit Points. In case of Constitution values equal to or lower than -3 the character dies at -5 hit points.

If a character's nonlethal damage reaches negative hit points equal to 20 + 4 * Constitution the character is dead. \hypertarget{puntiferitatemporaneimorte}{}

\begin{changemargin}{0.3cm}{0.3cm} \begin{Storytellere}
Describe the character's fall with pathos and transport, make people understand the suffering experienced. Emphasized the fall to the ground, the gushing blood, the gasps. Be theatrical.
Clear that if you are dealing with easily impressionable players then it is better to reduce the "gore".
\end{Storytellere} \end{changemargin}

A dead character cannot benefit from normal or magical healing, and cannot be brought back to life by a spell. Only a Patron has enough power to bring the soul back into the body and bring the creature back to life. The animate the dead spell can revive a body, but as an undead.

\subsubsection{Optional - 0 hit point recovery} \index{recovery} \index{Unconscious} \index{Optional - 0 hit point recovery} \label{recuperozeropf}

\textbf{In case you want a less lethal system (less OSR) you can apply this optional rule.}

Each round following a 0 hit point or less, then unconscious or dying, the character must make a Fortitude save at difficulty 15, regain consciousness if successful and go to 1 hit point.

If he fails the check he can make another one at DC +1 from the previous one the next round. When the difficulty reaches 18 (or 3 failed checks in a row) the character dies.

As soon as the check succeeds (within 3 failures) the character returns to 1 hit point.

\subsubsection{Recovery feature points} \index{Recovery feature points} \label{recuperopunticcaratteristica}

Any lost ability points are recovered at the rate of 1 point per day, if not indicated as a permanent loss.

\subsubsection{Natural hit point recovery} \index{Natural hit point recovery} \label{recuperopuntiferitanaturale}

Resting for 8 hours recovers your Constitution + 2xWeapon Proficiency + Magical Profiency score per day in hit points, with a minimum of 1.

\subsubsection{Non-lethal hit point recovery} \index{Non-lethal hit point recovery} \index{Non-lethal hit points} \label{recuperopuntiferitanonletali} \hypertarget{recuperopuntiferitanonletali}{}

Every hour you recover your Constitution value with a minimum of 1 Hit Point.

\subsubsection{Maximum Hit Points} \index{Maximum Hit Point Recovery} \index{Maximum Hit Points} \label{puntiferitamassimi}

Whenever the character suffers damage that lowers the maximum hit points, he must subtract the indicated amount from his current hit points and also from the maximum hit points he has.

Every 8 hours of rest, the Constitution score in Maximum Hit Points is recovered, with a minimum of 1.

\end{multicols}

\vfill

\begin{center}

%\includegraphics[width=0.7\linewidth]{immagini/caravaggioSalomeLondon.png}

%\textit{Salomè con la testa del Battista è un dipinto di Caravaggio realizzato in olio su tela (91x106 cm) tra il 1607 e il 1610.\\ È conservato nella National Gallery di Londra.}
\includegraphics[width = 0.45 \linewidth]{immagini/giantdeath.png}

\textit{Henry Justice Ford}

\end{center}

\pagebreak

\subsection{Attack and Defense Roll} \index{Attack Roll} \index{Defense} \label{tiropercolpireedifesa}

\begin{changemargin}{0.3cm}{0.3cm} \begin{enfasi}{Always apply the right force, never too much, never too little. (Kano Jigoro)} \end{enfasi} \end{changemargin} \medskip

\begin{multicols}{2}

The attack roll is a check versus the opponent's Defense.

If the attacker uses:

\begin{itemize}
\item \textbf{Melee or Contact Weapons}: Attacker must make a \textbf{Attack Roll (TC)} = 3d6 + Weapon Proficiency + Strength and any skill and magic bonuses of the weapon and factors circumstantial (environment, curses ..)

\item
\textbf{Ranged or Versatile Weapons}: The attacker must make an attack roll (TC) = 3d6 + Weapon Proficiency + Dexterity + and any skill and magic bonuses of the weapon and circumstantial factors ( environment, curses ..). Applies to bows, crossbows, drawn daggers, scimitars ...

\item
\textbf{Spell}: The attacker must make an attack roll (TC) = 3d6 + Weapon Proficiency + Strength (for melee spells) or + Dexterity (for ranged spells) any abilities and circumstantial modifiers.
\end{itemize}

The defender has a \textbf{Defense} equal to: 10 + Dexterity + Shield + Armor + any magic bonuses and Ability and circumstantial bonuses, for monsters the Defense value is already marked.
The player can decide to forfeit the bonus given by the Weapon Proficiency in order to have a better Defense score. These points will not be available in the next attack (see chapter on combat).

\begin{center}
\includegraphics[width = 0.9 \linewidth]{immagini/Coypel_Charles-Antoine_-_Fury_of_Achilles_-_1737.png}
\textit{Charles Antoine Coypel - Fury of Roland - 1737}
\end{center}

\subsection{Defense and Attack} \index{Defense} \index{Attack} \label{difesaeattacco}

\begin{changemargin}{0.3cm}{0.3cm} \begin{enfasi}{Defense is always legitimate (anonymous victim)} \end{enfasi} \end{changemargin} \medskip

Each Hit Roll (3d6 + Proficiency with Weapons + Strength or Dexterity + any bonus / malus) compares the Defense or a value equal to 10 + Dexterity + Shield + Armor + any bonus / malus.

If the attack roll is equal to or greater than the Defense value the opponent has been hit and the damage of the wound will be established, given by the weapon + Strength score and other factors such as magic and Skill bonuses.

If the TC (Attack Roll) is lower than the Defense then the opponent will have parried, dodged, avoided .. The choice is left to the player (or Storyteller), avoided the attack will not suffer injuries.

There are situations that can benefit the Defense such as covers, hiding places, as if it were, doors, companions much larger than your own. Consult the paragraphs related to \hyperlink{coperture}{Nascondigli e Coperture} to understand the advantage they can give.

There are occasions when it is not important to penetrate the defense and hurt the opponent but simply touch the opponent.

Other times the opponent is surprised and cannot fully defend himself.

If it is \textbf{just touch the opponent} the Defense will be 10 + Dexterity + Magic bonuses, with no Shield and Armor bonuses. This is the value of \textbf{Touch Defense}.

If \textbf{opponent is surprised} or does not expect the attack the Defense will be 10 + magic bonuses + Armor, with no Shield and Dexterity bonuses. This is the value of \textbf{Surprise Defense}.

\textbf{The Golden Rules also apply to the attack roll. The d6s explode if you roll a 6 on the dice, do 1 bad door and rely on luck (i.e. remove 4 points between Weapon Proficiency and Strength or Dexterity to add 1d6 to the attack roll, not from the bonus given by Skill).}

If the modifiers and circumstances cause the damage dealt to be negative you will still do 1 damage.
This rule applies to weapon damage modifiers that cannot bring the total damage to less than 1, if there are magical protections or damage reductions this can become zero and therefore you will not hurt the opponent (but if it becomes negative don't cure it!).

First, remember that for every 6 rolled (in the 3d6 of the attack roll) you must roll another one and keep rolling as long as you keep rolling a 6 on the die.

If you hit, \textbf{every two 6s rolled} (counting those of the attack roll and the subsequent ones resulting from having rolled 6), the weapon does the extra damage or a critical roll. Re-roll the weapon's damage die, with no magic or Strength or Ability for every two 6s rolled on the attack roll.

You can \textbf{remove 4} or multiples to your attack to roll an extra d6. The choice is to be made in the most desperate situations where only luck can solve the duel. The value takes it away from the Weapon Proficiency and Strength or Dexterity values not from scores given by Skills or magic bonuses.

If you roll a 1 on the attack roll, this lowers the total value by 1 (so 1 doesn't count) but doesn't affect whether you are critical or not.

\textbf{Rolling a critical roll is no guarantee that you have hit, you must always pass the Defense}.

The basic rules of skill also apply to the attack roll. Defense is a fixed value and as such uses modifiers for fixed value checks.

\subsection{Throw 3 times 1} \index{Throw 3 times 1} \label{tiraretrevolteuno}

If in the first 3 attack rolls you roll three times 1 you will miss the opponent (regardless of the final result of the attack roll) and the Storyteller could decide bad things about your attack (you drop your weapon, hit a friend, you will break 'weapon, you get hurt, you fall, a \hyperlink{diavolodellafossa}{Diavolo della Fossa} appears to mock you ...)

\begin{changemargin}{0.3cm}{0.3cm} \begin{Storytellere}
OBSS wants to be fun to play, it wants players to have fun and see the results of the dice (and of course their choices). The Golden Rules and the Explosion of Damage really want to remove the patina from the dice and entertain. A player will appreciate, even more if experienced, how the rolls of the dice are not just a number but also open up the possibility of making a difference. Ask the player to describe the critical hit and have it act in its mighty glory!
\end{Storytellere} \end{changemargin}

\subsection{Throw 3 times 6} \index{Throw 3 times 6} \label{tiraretrevoltesei}

If in the first 3 attack rolls you roll 6 three times, you will catch the opponent regardless of the final result of the attack roll. In addition to being certain that he has made a Critical Roll (see below), the Storyteller may decide to apply some additional descriptive (or actual) effect.

\subsection{Critical Shot} \index{Critical Shot} \index{Critical Damage} \label{tirocritico}

Whenever you hit, you roll \textbf{additional weapon damage} (with no magic or Skill or Strength bonuses, just the weapon die) for every two times you roll a 6 on the attack roll, this damage is also called \textbf{critical damage}. If you made two Critical Rolls it means that you must roll 2 more weapon dice.


\begin{center}
	\includegraphics[width = 0.9 \linewidth]{immagini/critico.png}

	\textit{Henry Justice Ford}
\end{center}


\begin{changemargin}{0.3cm}{0.3cm} \begin{tcolorbox}[title = Critical Shot Example]
Eg shot 6 4 5, additional shot 6, additional shot 6, additional shot 4: as damage you roll twice the damage of the weapon, once because I hit one because you rolled three times 6 (if you had rolled an additional 6 would have been Weapon + Strength + bonus / ability + 3{*} Weapon).
\end{tcolorbox} \end{changemargin}



\subsection{Optional - Variant Critical Shot} \index{Variant Critical Shot - Optional} \index{Optional - Variant Critical Shot} \hypertarget{tirocriticovariante}{} \label{tirocriticovariante}

The player may prefer chance less and handle critics based on the character's "skill" in using the weapon.
An alternative method is to grant a critical roll for each multiple of 6 in which the attack roll is higher than the defense, regardless of the number of 6 rolled.

The choice to use this variant of the critical roll must be made at the time of character creation and in agreement with the Storyteller. It can be changed only by giving up a point of Weapons Proficiency (you have to relearn how to fight in a different way).

\begin{changemargin}{0.3cm}{0.3cm} \begin{tcolorbox}[title = Variant Critical Shot]
Ex. Hit Roll 21, the opponent's Defense is 13. I hit him with a margin of 8, ie I add 1 more weapon damage
If the attack roll had been 26 it would have added 2 Critical damage or two weapon damage.
It is the Storyteller who says how many critics have been obtained.
\end{tcolorbox} \end{changemargin}

\subsection{Blast of Damage} \index{Blast of Damage} \label{esplosionedeldanno}

Whenever you get the maximum value from the roll of the weapon die (in the classic d8 for the long sword, for example, you make 8 and it is therefore the maximum value of the die), you re-roll the die and add the value again (of the die only).

In the case of weapons with multiple dice (eg 2d4, the maximum value must be obtained as the sum of the two dice, or 8). There is no damage explosion for weapons with maximum damage less than or equal to 6.

\smallskip

\begin{center}
\includegraphics[width = 0.7 \linewidth]{immagini/esplosionedanno.png}

\textit{Henry Justice Ford}
\end{center}

Some weapons have a different blast of damage. In the weapons table where EDX is marked (e.g. ED9), the X value stands for the minimum value sufficient to roll the damage again, so in the case of ED9 you can criticize with 9 or more with the weapon die .

This is characteristic of a few extremely lethal weapons.

The blast of damage does not explode as well, even if you make the most of the die with the added die it does not explode again.

The dice rolls added thanks to the critical (obtained by casting at least two 6s) do not have the advantage of the explosion of damage. If the die of the extra weapon rolled thanks to the Critical Roll makes the maximum, do not reroll and add the damage. When rolling damage, declare which die is for the weapon and which is for Critical Rolls.

\subsection{Multiple Attacks} \index{Multiple Attacks} \label{attacchimultiplimischia}

With one Action, the character can make a single attack.

With two Actions the character can make up to two attack rolls on the same opponent. \textbf{If he wants to make 3 or more attacks he must use 3 Actions}.

Any single arrow, dart, dagger, or weapon with range fired counts as one attack.

The first attack action has no penalty while the second attack action has -5 to hit. Subsequent attack rolls will accumulate -5 to hit, so a third attack will have -10 and a fourth attack -15 ...

If the cumulative hit penalty becomes greater than the attack roll, it is no longer possible to make further attacks.

If I have Weapon Proficiency 5, Strength 1, +2 to perform as a bonus from the Weapon List and +1 to hit given by a Skill, +2 for flanking and +1 for magic weapon the first attack roll will be 3d6 + 12, the second will be 3d6 + 7, the third 3d6 + 2. It is not possible to make a fourth attack as the hit bonus would become negative.

Any dynamic hitting bonuses, eg. + 1d6, they apply to the attack roll and not to the calculation of the bonus for calculating the number of multiple attacks. In the case of the example the attack roll becomes 4d6 + 12 / + 7 + 2.

The player can declare to make attacks on different targets. Each attack can be interspersed with a Move Action, as long as it has enough Actions.


\subsubsection{Optional - Multiple Attack Variant} \index{Optional - Multiple Attack Variant} \label{varianteattacchimultipli}

The player who sets his hit bonus makes a single attack roll.
If he hits, he rolls the damage and for each multiple of 6 of the hit bonus he adds a critical roll. This attack consumes 2 Actions and is the only attack that can be made in the round.


This variation serves to speed up the game by making a single attack roll. This variant is not compatible with Optional - Critical Shot Variant.

\begin{center}
	\includegraphics[width = 0.9 \linewidth]{immagini/archer.png}

	\textit{Scythian archers in ancient attic vase painting}
\end{center}


\subsection{Missile Weapons - Bows - Crossbows (Bow / Crossbows / Daggers ..)} \index{Multiple Attacks Missile Weapons} \label{armidatiro}

The damage bonus given by Strength is automatically applied to slings, daggers, javelins .. that is, with all weapons that are thrown "by hand", the bows apply this bonus only if they are of the composite type, crossbows do not apply it never.

Dexterity modifies only the attack roll.

\textbf{Projectiles thrown by Bows, Slings, Magic Crossbows are not considered magical. \\
In case of magic bullets they add their magic bonus to the attack roll and damage}

In each thrown weapon, the range is marked, i.e. within what distance the projectile can be fired without penalty. Each thrown weapon can hit within three times the indicated range.

If the target is within the indicated range there is no penalty on hitting, if the target is between the first and second increase the penalty on hitting is -1d6. If the target is between the second and third increase, the penalty on hitting is -2d6.

A dagger thrown within 6 meters has no penalty, but shot between 6 and 12 meters has a -1d6 to hit, at a distance between 12 and 18 meters a -2d6 to hit, beyond it cannot be thrown.

\begin{center}
	\includegraphics[width = 0.75 \linewidth]{immagini/fenice.png}

	\textit{Henry Justice Ford ... watch out for the fall of the Phoenix feather ..}
\end{center}

\subsection{Spread Weapons Attacks} \index{Spread Weapons} \index{Holy Water} \index{Incediato Oil} \label{attacchiarmidaspargimento} \hypertarget{spargimento}{}

Spreading weapons are those that "scatter" their contents where they fall, for example burning oil / holy water ... A scatter weapon has a range of 6 meters \index{Throwing Spreading Weapons} \index{Weapon Range to shedding}.

In case the attack is missing (by at least 5) roll a d8 and consult this diagram to understand where the ball fell:

\medskip

\begin{tabularx}{0.30\textwidth}{ccc}
1 & 2 & 3 \\
4 & \textbf{X} & 5 \\
6 & 7 & 8 \\
& \textbf{0} & \\
\end{tabularx}

\smallskip

\textbf{X} is considered the target of the object shot. \textbf{0} the point of origin of the launch.

If the roll misses by 5 or more roll a 2d4 to determine in the direction indicated by the previous d8 how many meters has fallen away from the target, that is, count the meters from the target.

For example, with the d8 roll I make 5 and then when I roll 2d6 I make 4, it means that the bottle has fallen to the right of the target at 4 meters.

It is also possible that you have pulled the bottle over your feet (eg I do 7 and then 6 .. I could have thrown it at a partner or behind me!).


\subsection{Unprepared - Caught by Surprise} \index{Unprepared} \index{Surprise} \label{coltidisorpresa}

If characters are taken by surprise, that is, they do not expect to be attacked, this first round should be considered as a surprise round. When surprised, you cannot use your Dexterity or Shield in Defense.

For that round and for the attacks of that round your Defense, and any Reflex saving throws, will not count any bonuses given by Shield or Dexterity, you will not be able to react, you will not use any Actions or Reactions unless explicitly allowed; from the next round you will be able to declare the initiative and act normally. The same considerations apply to the opponents.

To see if a character is surprised, make a Reflex saving throw, comparing it to the opponents' ability to hide (or move silently), if the saving throw is less than the character is surprised.

If the character is at attention and expects an ambush have him make an Awareness check at DC 15, if the check passes then the character is not surprised.

When characters and enemies are both taken by surprise to see who is actually surprised, make a Reflex saving throw, whoever rolls more than 15 is not surprised.

\subsection{Attack or defense modifiers for particular situations} \index{Special situations} \label{modificatoriattaccodifesaparticolari}

The best suggestion you can give in managing the most chaotic combat situations is to think of them as a movie, evaluate the cinematic nature of the situation.

It is not a question of miniatures, spaces, pictures .. it is a question of fun and visualization of the scene. Unorthodox solutions for unorthodox situations.

Grant a bonus or malus (+ -1 / 2 unless otherwise stated) whenever the player has an advantage or disadvantage and in the same way to the opponent.

\bigskip

\textbf{Examples in an Attack situation (bonus or penalty on attack rolls). In case more bonuses or penalties are applied, the one with the highest score must be chosen.}

\begin{itemize}
\item Situations with +2 bonus: flanking opponent, elevated position, back attack, long weapon ...

\item Situations with 1d6 bonus: you are invisible, charge ...

\item Situations with a disadvantage of -2, you are dazzled, you are hindered ...

\item Situations with 1d6 disadvantage: prone, restricted in movement, scared, use a thrown weapon against an opponent in melee, attack with long weapon in melee range, use an unknown weapon, attack an invisible creature but identified ...
\end{itemize}

\textbf{Examples in defense situations:}

\begin{itemize}
\item Situations with + 2 / + 4 bonus (Defense bonus): you have cover (see below) ...

\item Situations with a disadvantage of -2 (defense penalty): you are in the way ...

\item Situations with -4 disadvantage: you are blinded, trapped, you are kneeling / sitting / prone, you are confined in a space, you are stunned, you cast a spell while under attack ...
\end{itemize}

\textbf{When -1d6 is written it means that one less die is rolled (or two if it is -2d6), likewise if it is written + 1d6 one roll to 6 more and add up}.

\textbf{When the penalty is on Defense treat each -1d6 as a -4 on Defense}.

\textbf{In principle in combat a light bonus is +1, medium +2, high + 1d6 (or +4), a very high bonus is + 2d6 (or +8), vice versa for malus} .

\bigskip
Always remember the purpose is to have fun, at the expense (for the Storyteller) of some monster, do not be rigid but dynamic and adapt to situations.


\begin{center}
	\includegraphics[width = 0.55 \linewidth]{immagini/vantaggio.png}

	\textit{Henry Justice Ford}
\end{center}

\subsection{Special actions in combat} \label{azioniparticolaricombattimento}

\subsubsection{Strike with bare hands} \index{Punch} \index{Kicks} \index{Fighting} \label{attaccomaninude}

two weapons that no one will ever miss are one's punches and kicks.

If you didn't get the "Empty Fist" weapon list, a punch or kick will do 1d2 + Strength of non-lethal damage. Only with the "Empty Fist" Weapons List can you become a martial artist.

\subsubsection{Getting up from prone} \index{Getting up from prone} \label{alzarsidaprono}

it costs two Actions. The character can make an Acrobatics check if you do 13 or more it costs 1 Action to get up. If you roll three 1's in the check check you cannot take any more actions that round and you remain prone.

When the Acrobatics score reaches 6, getting up prone costs 1 Action if you succeed in the DC 13 test. With Acrobatics 8 it always costs an Action without testing.

When prone you can crawl \index{crawl} \index{all fours} or crawl. The terrain is considered difficult and you are still considered prone until you stand up.

\begin{center}
	\includegraphics[width = 0.9 \linewidth]{immagini/carica.png}

	\textit{A Connecticut Yankee in King Arthur's Court / Samuel Clemens. New York: Charles L. Webster \& Co., 1889}
\end{center}

\subsubsection{Charge} \index{Charge} \label{carica}

the opponent must be within 2 Movement Actions (18 or 12 meters usually) and no less than 3 meters. You have to run until you are within melee range.

You get a + 1d6 on the attack roll, -4 on the Defense until the end of the round, the attack after the first takes a -10 to hit and a possible subsequent -15/20 ... The charge and attack / attacks costs 2 Actions +1 Action if move more than 1 Action movement. No other penalties are considered for having run beyond those indicated.
The Charge action brings you, in melee, with the opponent. The attack if made with a long weapon still has the +2 bonus to the attack roll and hits from a distance, and then ends up in contact with the opponent.

\begin{center}
	\includegraphics[width = 0.9 \linewidth]{immagini/pilum.png}

	\textit{Roman soldiers armed with Pilum, ready for a counter-charge.}
\end{center}

\subsubsection{Counter-charge} \index{Counter-charge} \label{controcarica}

an attack roll made with a weapon with the counter-charge talent when used against a charging opponent / mount inflicts a critical roll and hits first, unless the opponent has a long weapon or greater range than the preparer of the counter-charge , in this case the attack is governed by initiative rolls.

\subsubsection{Preparing a long weapon / counterload against a charge} \index{Preparing a long weapon against a charge} \label{prepararearmalungacontrocarica}

It is a reaction.

\subsubsection{Counter-Charge Weapon} \index{Counter-Charge} \label{caricaarmadacontrocarica}

if the attack roll is successful when you use a weapon with the countercharge talent to charge an opponent, it inflicts a critical roll.

\subsubsection{Helping another} \index{Helping} \label{aiutare}

you can help a teammate attack or defend themselves in melee fights by distracting or interfering with the opponent. You can make a melee attack (1 Action) against an opponent who has already engaged in battle with an ally of his own.

An attack roll is made against the opponent's Defense with a 1d6 bonus. If the attack hits, no damage is done, the partner gains a +2 bonus on attack roll with the next attack (by the end of the next round) towards that opponent or a +2 defense bonus against the opponent. that opponent's next attack (your choice) within the next round.

Multiple characters can help the same ally; bonuses of this type are cumulative (maximum 4 on medium size), as long as the opponent is surrounded.

\subsubsection{Coup de Grace} \index{Coup de Grace} \label{colpodigrazia}

costs 3 Actions, you can use a melee weapon to inflict a final blow on a defenseless (unconscious or trapped) target. You can also use a bow or a crossbow, as long as you are adjacent to the target.

The attacker automatically hits and deals three critical hits.

\begin{center}
\includegraphics[width = 0.9 \linewidth]{immagini/colpodigrazia.png}
\textit{Beheading of St. John the Baptist. St John's Co-Cathedral in Valletta (Malta). (Caravaggio). \\ Coup de Grace}
\end{center}

\subsubsection{Aimed Shots} \index{Aimed Shots} \label{tirimirati} \index{Aim at specific parts}

OBSS does not provide the ability to make targeted shots with any weapon or spell, unless specified.
When you hit the target you hit it generically, without the possibility of specifying whether to the head, leg or other, the same concept applies in the case of hitting objects, eg. if you aim for a hinge of a door you hit the whole door. This does not prevent the Storyteller from evaluating adequate consequences.

\subsubsection{Non-lethal damage} \index{Non-lethal damage} \label{dannononletale}

Non-lethal damage is a form of damage caused by particular weapons or when the intention is to make the enemy faint and not kill him.

Non-lethal damage is treated like normal damage but must be marked separately on the sheet.

\subsubsection{Non-lethal damage with unsuitable weapon} \index{Non-lethal damage with unsuitable weapon} \label{dannononletalearmanonidonea}

if you want to do nonlethal damage with a weapon that is not predisposed to nonlethal damage, you have -1d6 on the attack roll.

\subsubsection{Without Competence} \index{Without Competence} \label{senzacompetenza}

using a weapon without adequate proficiency, that is, not having the weapon's Weapon List, imposes a -1d6 on the attack roll. You can't use a weapon's Versatile ability if you don't know how to use it. A Simple Weapon is usable even without specific skills.

\subsubsection{Light Weapons} \index{Light Weapons} \label{armileggere}

these weapons are light and indicated for \hyperlink{combattimentoaduemani}{combattimento a due mani}.

\subsubsection{Versatile Weapons} \index{Versatile Weapons} \label{armiversatili}

on weapons with the Versatile feat you can freely use Dexterity instead of Strength on attack rolls. Strength is always used for damage.

\subsubsection{Throwing weapons} \index{Throwing weapons} \label{lanciarearmi}

a sword or in any case a weapon not made to be thrown, without range, can still be thrown at the opponent.

The attack roll takes a -1d6 and the weapon does a lower damage category (the longsword does 1d6, a shortsword 1d4 ..). The launch range is 3 meters.

\subsubsection{Powerful Blows} \index{Powerful Blows} \label{colpipotenti}

the player can freely add +1 to the damage by removing 1 from the attack roll (weapon proficiency requirement +1). You cannot remove more than Weapon Proficiency / 4 on the attack roll.

\subsubsection{Flanking} \index{Flanking} \label{fiancheggiare}

if two characters are around the same target but are not side by side, they get +2 on the attack roll or on the Defense roll (their choice which bonus to take).

At most, there can be 4 characters around a medium-sized creature who get the flanking bonus. The type of bonus is chosen round by round, if not declared it counts as +2 on the attack roll.

If, by pulling a hypothetical line that connects the two characters, it crosses the opponent's square in full, then there is a flanking situation.

\bigskip

Flanking Example \index{Flanking Examples}

\medskip

\begin{tabularx}{0.45\textwidth}{lll}
\toprule
A & G & D \\
B & \textbf{X} & E \\
C & H & F \\
\end{tabularx}

\bigskip

In this scheme the flanking is taken by the couples: AF, BE, CD, GH

\bigskip

If the creature can face multiple creatures at the same time, they won't get the flanking bonus.

\subsubsection{Using a weapon with two hands} \index{Using a weapon with two hands} \label{usarearmaconduemani}

a weapon that is not light when used with two hands allows you to apply one and a half times the damage due to the Force. This bonus does not apply if the weapon is too large or unknown.

\subsubsection{Double Weapon} \index{Double Weapon} \label{armadippia}

a double weapon is a weapon that is dangerous from both ends. It can be used as a single weapon, or, incurring the penalties of combat with two weapons, just like two weapons. Unless specified, a double weapon used for two-weapon combat is equivalent to using two medium weapons.

\subsubsection{Combat Mastery} \index{Combat Mastery} \label{maestriacombattimento}

the player can freely add +4 to the Defense for each -1d6 to the attack roll. The bonus is only applicable for melee attacks. Conversely, he can take a -4 Defense to increase his attack roll by + 1d6 and thus improve his attack.

It is not possible to assign more than + -2d6 in this way.

\subsubsection{Sighting (sniper)} \index{Sighting (sniper)} \label{cecchino}

for each round in which you take aim, 2 Actions, you gain +1 to the attack roll, up to a maximum of +3 in the third round, when you have to use the third Action to shoot the arrow (or dart or dagger ...) .

\subsubsection{Using a thrown weapon targeting an opponent engaged in combat} \index{Using a thrown weapon targeting an opponent engaged in combat} \label{usarearmalancioinmischia}

it's not easy to aim correctly and not hit your partner, you have a -1d6 attack roll. The bonus is canceled if there is a difference of 2 or more sizes between opponent and partner.

\subsubsection{Using a thrown weapon under threat} \index{Using a thrown weapon under threat} \label{usarearmalanciosottominaccia}

using a throwing weapon such as a bow, crossbow, or dagger (which you want to throw) while fighting in melee forces the Dexterity bonus to be negated on Defense and the attack roll has a -1d6.


\begin{center}
	\includegraphics[width = 0.9 \linewidth]{immagini/twoweapon.png}
\end{center}

\subsubsection{Combat with two weapons} \index{Combat with two weapons} \hypertarget{combattimentoaduemani}{} \label{combattimentoduemani}

Combat with two weapons is only possible if the secondary weapon is light or a dual weapon is used.

Attacks made with the secondary weapon are considered multiple attacks.
If I attack a first time, regardless of whether it is with the primary or secondary weapon, this will have the attack roll at full bonus, the other attacks will have -5 to hit and so on.

The damage bonus given by strength on the secondary weapon is halved.

It is possible to use the secondary weapon to improve the Defense by one point but you cannot make attacks.

\subsubsection{Get on the defensive} \index{Get on the defensive} \label{mettersisulladifensiva}

You use an Action to better prepare yourself for the next attacks of your opponents. Until the start of the next round you have +2 Defense.

The player can freely add +1 to the Defense by subtracting 2 from the attack roll. You cannot remove more than Weapon Proficiency / 4 on the attack roll.

\subsubsection{Total defense} \index{Total defense} \label{difesatotale}

it costs 2 Actions, you cannot perform any attacks or spell casts, you can only do one Action and you gain +4 in Defense.

\subsubsection{Weapon too large} \index{Weapon too large} \label{armatroppogrande}

attacking with a '\textbf{Weapon too large} \index{Weapon too large} relative to one's size is problematic.

Normally, a creature can use a weapon up to its size or by using two hands a weapon of only one size larger.

If the weapon is not "usable", such as a halberd (large weapon) for a small creature the attack roll penalty is -1d6.

In the weapons table the size is marked as P (small), M (medium), G (large), E (huge). A "larger" version of a weapon increases the weapon's damage by one category (1d4-> 1d6, 1d6-> 1d8, 1d8-> 1d10, 1d10-> 2d6, 2d6-> 2d8, 2d8-> 2d10, 2d10 -> 3d6 ...)


\begin{center}
	\includegraphics[width = 0.8 \linewidth]{immagini/angelospadone.png}
\end{center}


\subsubsection{Disengage} \index{Disengage} \label{disingaggiare}

it costs 2 Actions and you move up to 3 meters. An opponent can hit you if they have better initiative than yours or chase you (and they are as fast as you), you still have +2 Defense.

It can be used to get out of melee without causing attacks of opportunity and then move or make an attack using the remaining Action even if at -1d6 to hit.

If it is necessary to move more than 10 feet to disengage the opponent, due to the opponent's reach, a move action must also be used.

\subsubsection{Long Weapon} \index{Long Weapon} \label{armalunga}

the long weapon gives the right to hit farther, ie 2 meters. Grants a +2 bonus on attack rolls. This bonus remains valid until the opponent enters their melee range.

In case the opponent also has a long weapon do not consider the bonus (they are both in their own melee area).

\subsubsection{Long weapon at short range} \index{Long weapon at short range} \label{armalungabrevedistanza}

It is possible to use a long weapon in melee with an opponent with a non-long weapon or with a range of less than 2 meters with a -1d6 attack roll, with the exception of the staff.


\begin{changemargin}{0.3cm}{0.3cm} \begin{tcolorbox}[title = Long weapon combat]
Eg Tups armed with a long sword faces a brigand armed with a long spear. Tups has initiative 15, brigand 12.

Tups exploiting his agility comes under the brigand hitting him powerfully. The brigand finding himself in melee with Tups fails to exploit his long weapon which indeed penalizes him.

Use an Action to move two meters away and then attack with a +2 bonus because the opponent is far away.

As a third action he moves away another 30 feet and yells curses at Tups.

Tups is at this point 11 meters from the opponent, he decides to charge thus opening his defense but getting a bonus on hitting.

Charge the brigand hitting him and coming on him. The very wounded brigand tries to hit him trusting that his difficulty in using a long weapon so closely is balanced by the penalties given by Tups' run.

Tups is hit and the brigand throws his spear to the ground and draws a short dagger and gets on the defensive.

\end{tcolorbox} \end{changemargin}

\subsubsection{Magic in combat} \index{Magic in combat} \label{magiaincombattimento}

The caster who casts a spell while in combat (has an opponent in melee or is being targeted from a distance) takes a -4 to Defense and is considered Distracted.

\subsection{Optional Combat Actions} \label{azioniopzionaliincombattimento}

These combat Actions are at the discretion of the Storyteller who can grant them or not.

\medskip

\subsubsection{Disarm *} \index{Disarm} \label{disarmare}

make an opposing check Weapons Proficiency + Dexterity / Strength (who disarms) vs Weapons Proficiency (or degree of challenge) + Dexterity / Strength (who is disarmed).

A two-handed weapon grants a + 1d6 bonus, a light weapon a -2 penalty for disarming. If you fail by 10 or more you have disarmed yourself and not the opponent. It costs 2 Actions.

%\begin{center}
%\includegraphics[width=0.9\linewidth]{immagini/alfieri37.png}
%\end{center}

\subsubsection{Fake *} \index{Fake} \label{finta}

make an opposing check of Weapons Proficiency + Deceive (who feints) vs Weapons Proficiency + Awareness (who suffers the feint). If the check is successful, the opponent loses the Dexterity bonus on Defense until the end of the next round.

If you fail by 10 or more you lose the Dexterity bonus until the next round. It costs 1 Action.

\subsubsection{Push an opponent *} \index{Push an opponent} \label{spingereavversario} \hypertarget{spingereavversario}{}

it is an opposed Strength check (Fortitude save with a Strength bonus and no opposed Constitution). Whoever has a larger size gains a bonus of + 1d6 per size of difference.

If you win, push the opponent up to 0.5 meters in the direction you want for success in the check (up to the maximum of your movement), otherwise the opponent pushes you in the direction he wants up to 0.5 meters for success. Eg if you win the check by 7 you move the opponent up to 3 meters. It costs 2 Actions.

\subsubsection{Grasping an opponent *} \index{Grasping an opponent} \label{afferrareunavversario}

it is an opposed Strength check (Fortitude save but with a Strength bonus and not opposed Constitution). Whoever has a larger size gains a bonus of + 1d6 per size of difference.

It costs 2 Actions to do and maintain and free yourself from the grip. It is considered that the grabber is also grabbed and has at least one hand occupied in grabbing.

The two contenders lose the Dexterity bonus on Defense and Reflex saving throws.

Moving a grasped creature requires \hyperlink{spingereavversario}{Spingere un avversario}.

Each contender can attack the other grabbed with a small weapon usable with one hand or with punches and kicks.

\subsubsection{Dropping an opponent *} \index{Dropping an opponent} \label{farecadereavversario}

it is an opposed check of Strength or Dexterity, each contender chooses the one they prefer.

Each makes a Fortitude saving throw (with Strength modifier) or Reflex (with Dexterity modifier) and the results are compared.

For each additional leg / paw of difference you get a +2 bonus on the check.

It costs 2 Actions. Anyone who fails the check becomes prone.

\begin{changemargin}{0.3cm}{0.3cm} \begin{Storytellere} %box Storytellere
Trials such as \textbf{Disarm}, \textbf{Feint}, \textbf{Push}, \textbf{Grab}, \textbf{Drop} can be solved by setting the difficulty of the check taker ( and not the player) to a fixed value of 10 + relative modifiers (WP + Strength / Dexterity, WP + Awareness, Fortitude (strength) save).
\end{Storytellere} \end{changemargin}


\subsubsection{Change your size *} \index{Change your size} \label{modificatedimensioni}

if the character changes size \index{Change size} his Defense changes accordingly

\bigskip

\begin{tabular}{ll}
	\textbf{New Size} & \textbf{Defense} \\
	\toprule
	Very small & +8 \\
	Minute		 	& +4 \\
	Lowercase & +2 \\
	Small & +1 \\
	Average & +0 \\
	Large & -1 \\
	Huge & -2 \\
	Gargantuan & -4 \\
	Colossal & -8 \\
\end{tabular}


\subsection{Optional - The One Rule} \index{Optional - The One Rule} \hypertarget{lunicaregola}{} \label{lunicaregola}

This option wants to simplify the management of the fight by making everything more homogeneous and easy to remember. It is the various contenders who choose their own advantages and disadvantages.

- When there is any kind of advantage a bonus of + 1d6 (or +4 if static) is applied

- When there is any disadvantage, a penalty of -1d6 is applied (or -4 if static, such as Defense)

- When the advantage is significant, an advantage is assessed for one contender and a disadvantage for the other.

Eg if I fight an invisible creature it has + 1d6 on the attack roll (or +4 on the Defense) and I have a -1d6 (or I could choose a -4 on the Defense) on the attack roll.

Eg if I fight while I have cover I get a +4 to Defense, the opponent has no hit bonus.

Eg if I fight flanking an opponent I get a + 1d6 to attack roll.

Carefully consider how to distribute Advantages and Disadvantages. Always start with the advantage. A monster that is not particularly tactical and intelligent will always take the attack roll as a bonus.

In case of opposing actions between opponents, resolve with an opposed saving throw. Evaluate what the type and the most related Feature can be to provide the bonus.

This system works correctly if you apply modifiers to characters as well as opponents. Only by being fair you will not have unbalanced situations. This option encourages players to always seek a functional and advantageous tactical approach.

\subsection{Mounts} \index{Combat on horseback} \index{Horse} \label{cavalcature}

\begin{changemargin}{0.3cm}{0.3cm} \begin{enfasi}{
- And you can find yourself another wife!

- Ah, yes. but the trouble is, he took away my rifle and the horse! Too bad, it was so beautiful, I was fond of it. I gave her a few lashes, but she didn't notice.

- Who, your wife?

- No, my mare. Finding another wife is quick, but one more like that I don't find her anymore. (Red Shadows, 1939 film)} \end{enfasi} \end{changemargin} \medskip

A mount also has its 3 Actions and are usually used to move or to react and obey your commands.

A mount acts in your round and you decide when it performs its Actions in relation to yours. Don't take initiative, use yours.

To move a mount to where you want it to move, you must use your own Action, just as for making it act.

Attacks towards a character on horseback (or mount in general), unless otherwise stated, are aimed at the rider and not at the horse.


\begin{center}
	\includegraphics[width = 0.4 \textwidth]{immagini/napoleone.png}

	\textit{Jacques-Louis David, Bonaparte crossing the Great St Bernard Pass, 1801, Malmaison Castle}
\end{center}


\subsubsection{Situations and rules} \label{cavallosituazioniregole}

\begin{itemize}
\item
Whenever the mount is hit, the rider must make a Ride check at DC 15 or be unseated from the mount.

If the mount is "war" (combat-trained) the check has difficulty 12.

\item
Fighting from an elevated position grants a +2 to the attack roll if the opponent is on foot (or not at your height ...).

\item
Getting on or off the mount costs 1 Action if you have the Ride skill, otherwise 2 Actions.

\item
If a spell or situation moves the mount (abruptly) against your will, you must make a DC 13 Reflex saving throw, or a Ride check at the same difficulty, or be unsaddled.
\end{itemize}


\subsubsection{Being thrown off} \label{esseredisarcionato}

If you get thrown off an Acrobatics check at DC 15, you need a Reaction, it will avoid falling prone. If the check fails by 5 or more, you take 1d6 damage from the fall.


\subsubsection{Controlling a Mount} \label{controllocavalcatura}

While in the saddle, you have two choices:

\begin{itemize}
\item you can give orders to your mount
\item allow it to act alone.
\end{itemize}

Particularly intelligent mounts tend to favor autonomy of action rather than being commanded.

You can only control a mount if it has been trained to accept a rider. Trained horses, mules and similar creatures are assumed to have received such training.

The initiative of a controlled mount changes to match that of its rider. It moves according to your directions and has only three action options: Move, Attack, Disengage.
Any bonuses and penalties reported by these 3 Actions are only valid for the mount.

Making a mount carry out an order costs 1 Action to the rider.

If the mount is smart, having a rider does not restrict the actions the mount can perform and it moves and acts as it wishes. It may flee combat, attack and devour a badly wounded enemy, or act in some other way against your will.

\end{multicols}

%\vfill

%\begin{center}
%	\includegraphics[width=0.7\linewidth]{immagini/fauchard.png}
%\end{center}

\pagebreak

\section{Hiding places and coverings} \index{Hiding places} \index{Covering} \hypertarget{coperture}{}

\begin{changemargin}{0.3cm}{0.3cm} \begin{enfasi} Where there is a lot of light, the shadow is darker. (Johann Wolfgang von Goethe) \end{enfasi} \end{changemargin} \medskip

\begin{multicols}{2}

\lettrine[lines = 2, lhang = 0.33, loversize = 0.25, findent = 1.5em]{N}{ot} always the opponent reveals himself in front of us, often this can be hidden if not invisible.

It could be hidden behind a wall or barrels, if not behind a muscular and gigantic familiar.
What if it's behind us and we haven't even seen it?

\subsection{The Cover} \index{Cover} \label{copertura}

If the target is known to exist but is hidden in some way then it is said to have "cover".

\begin{itemize}
\item
If the target has more than half (but not all) of the "visible" surface then the cover is defined \textbf{light}, ie it has +2 to Defense. This can be the case of a creature behind another creature of the same size or 1 size larger.

This could be the case of an archer standing behind a 1 meter wall.


\begin{center}
	\includegraphics[width = 0.9 \linewidth]{immagini/hide.png}
	\textit{British Soldiers Hiding From Boer Fire At The Battle Of Majuba Hill.}
\end{center}

\item
If the target has less than half (but at least one third) of the "visible" surface then the coverage is defined as \textbf{medium}, ie it has +4 to Defense. This could be the case with a creature behind another creature 2 sizes larger.

This could be the case of an enemy armed with a crossbow who leans out just enough to keep the crossbow leaning against the wall and shoot (shoulders, arms and head visible).

\item
If the target knows where it is but hides completely looking out just to control the characters or shoot an arrow every now and then, behind a wall, window, door, table, a creature bigger than him (at least 3 sizes) .. then cover is defined as \textbf{complete}, ie it has +8 to Defense.
This can also be the case of a creature completely concealed by darkness, in which the presence is assumed for sounds, traces, spells cast or projectiles fired.

Clearly an opponent who is unknown where he is cannot be hit normally ...

\end{itemize}

Half the cover bonus also applies to saving throws against spells that have an area effect (eg fireballs exploding around ...).

\subsection{Invisibility} \index{Invisibility} \hypertarget{invisibilita}{} \label{invisibilita}

If an opponent is invisible or you don't know where he is, follow the rules of invisibility.

\begin{center}
\includegraphics[width = 0.8 \linewidth]{immagini/brickwall.png}

\textit{is there anyone in front of this wall?}
\end{center}

Even if you are invisible, it does not mean that you cannot be perceived differently through other senses, such as smell, hearing or touch. Invisibility renders a creature undetectable by sight but does not itself make a creature imperceptible or immune to critical rolls or blasts of damage.

A blinded creature, battling an invisible creature or battling in complete darkness, without darkvision, can make a Consciousness check, 1 Action, at difficulty 20, or 2 Actions at difficulty 15, if the opponent is within range. its melee range to try to \textbf{spot it}. \index{Detect targets}

Depending on the distance of the invisible creature or what it does, there are different modifiers to the Consciousness check to detect it.

\medskip

\textbf{Table Awareness Modifiers for Detecting Invisible Creatures} \index{Table Awareness Modifiers for Detecting Invisible Creatures}

\medskip

\begin{tabularx}{0.45\textwidth}{ll}
\textbf{The Invisible Creature is ...} & \textbf{Mod.} \\
	\toprule
Moving at full speed & -5 \\
Running or charging & -10 \\
Using Move Silently & check + 15 \\
Stop & +15 \\
Every 1 meter beyond the first & +2 \\
Coverage Light / Medium / Full & +4/8/12 \\
\end{tabularx}

These modifiers are cumulative with each other.

If the invisible creature attacked in melee and did not move it is considered \textbf{automatically detected}.

If the check is successful, the observer has the feeling that "there is something" but cannot see it or target it accurately with an attack.

Whoever attacks a creature invisible to her but spotted has -1d6 on the attack roll, the creature attacking the one who does not see it has + 1d6 on the attack roll.

\bigskip

If an invisible character picks up a visible object, the object remains visible. An invisible creature can pick up a small visible object and hide it on itself (putting it in a pocket or under the cloak, closing it in its fist) and effectively make it invisible.

Someone might sprinkle flour on an invisible object to at least keep track of its location (until the flour falls completely or is blown away).

Invisible creatures leave footprints. Their tracks can be followed without problems. Footprints on sand, mud, or other soft surfaces can give enemies clues to the invisible creature's location by making it spotted.

An invisible creature in the water moves the liquid, revealing its location. The invisible creature still remains difficult to hit and enjoys the benefits of light cover (+2 Defense).

An invisible lit torch still gives off light (as does an invisible object subject to a magic of light).

Invisible creatures cannot use gaze attacks. Invisibility does not affect the target being of a divination spell.

\end{multicols}

%\vspace{4cm}

\vfill

\begin{center}
\includegraphics[keepaspectratio, width = 0.75 \textwidth]{immagini/impronteneve.png}

\textit{Can help find an invisible wolf ...}
\end{center}

\pagebreak

\section{List of Weapons by Homogeneous Type} \index{List of Weapons} \index{Homogeneous Type} \hypertarget{lista.armi}{} \label{lista.armi}

\begin{changemargin}{0.3cm}{0.3cm} \begin{enfasi}{Strength resides not in a Sword, but in the arms of a valiant. (The Legend of Zelda: Twilight Princess)} \end{enfasi} \end{changemargin} \medskip

\begin{multicols}{2}

\lettrine[lines = 2, lhang = 0.33, loversize = 0.25, findent = 1.5em]{W}{henever} a point is assigned to Weapons Proficiency it is possible to decide whether to continue to improve on an already known List of Weapons or learn a new one, if not declared the use it is assigned to the List of Simple Weapons.

On the sheet, note which Weapon List you assign the Weapons Proficiency point to. Bonuses marked are cumulative unless otherwise indicated.

At least 4 hours of training for 4 months are required to reassign a weapon proficiency point to another list.

Using a weapon without the proper proficiency imposes a -1d6 on the attack roll. The points awarded in a Weapon List do not add up to the attack roll, it is necessary to check the score in the Weapon List with any bonuses that the same list lists.

All Lists of Arms grant, unless otherwise stated, these advantages:

\begin{itemize}

\item 6 points: You have a 1d6 bonus on the attack roll with these weapons

\item 18 points: When making an attack roll you also count the 5 for the Critical count (but do not re-roll the die).

\end{itemize}

The bonuses listed in the Weapon Lists only apply when fighting with the weapons listed.

\subsection{Light Weapons} \index{Light Weapons} Short Sword, Light Mace, Rapier, Scimitar, One-handed Ax \label{listaarmileggere}

\begin{itemize}

\item 4 points: You can use Dexterity instead of Strength in the attack roll

\item 8 points: you increase the damage die of the weapon by one degree (d4 - d6 - d8 - d10 - 2d6 - 2d8 - 2d10 - 3d6 ..). If the damage die becomes 8 or more the weapon gains the EDX on the maximum value of the die

\item 12 points: increase the damage die of the weapon by one degree (d4 - d6 - d8 - d10 - 2d6 - 2d8 - 2d10 - 3d6 ..). EDX is reduced by 1

\end{itemize}

\subsection{Axes} \index{Axes} One Handed Ax, Battle Ax, Hammer Ax, Great Double Ax \label{listaasce}

\begin{itemize}

\item 4 points: The fury of your attacks is such that you gain +2 to damage on hit

\item 8 points: The wounds you cause are so deep that it causes Bleeding. On the first attack of the round if successful, deal 1 point of extra damage from Bleed. The damage applies until the next round.

\item 12 points: The wounds you cause are so deep that it causes Bleeding. Each successful attack increases your bleed by one up to a maximum of 5 Bleed. Damage applies as long as the wound is not medicated (not cumulative with the 8-point advantage).

\end{itemize}

\subsection{Skull Breaker} \index{Skull Breaker} Club, Heavy Club, Spiked Club, Scourge, Warhammer, Big Club, Heavy Scourge \label{listaarmirompicranio}

\begin{center}
\includegraphics[width = 0.9 \linewidth]{immagini/arma-mazza3.png}
\end{center}


\begin{itemize}
\item 4 points: You have become so skilled that you can control the strength of your blows, you can do non-lethal damage with no penalty on the hit (otherwise -1d6 on the attack roll).

You can choose to reduce the attack roll by 4 to increase the damage by 8 (cannot be combined with Powerful Blows).

\item 8 points: Your blows stun the enemy. If you criticize and hit with the attack roll the opponent must make a Fortitude saving throw (DC equal to your attack roll) if he fails he will suffer -2 Defense until the end of the next round.

\item 12 points: increase the damage die of the weapon by one degree (d4 - d6 - d8 - d10 - 2d6 -2d8 - 2d10)

\end{itemize}

\subsection{Archi} \index{Archi} Slingshot, Longbow, Short Bow, Composite Longbow, Composite Short Bow \label{listaarmiarchi}

\begin{center}
\includegraphics[width = 0.9 \linewidth]{immagini/arma-arco.png}
\end{center}


\begin{itemize}

\item 4 points: Add Strength value to damage, even if the bow is not composite. On a short bow you can add up to +1 damage, on a long bow up to +2 damage.

\item 8 points: Your skill in using the bow in combat is such that you suffer no penalty from shooting arrows at enemies in melee or with light or less cover.

\item 12 points: Shoot an extra arrow with a -5 penalty on the attack roll, the penalty does not stack with the multi attack. (TC, TC-5, TC-5, TC -10 ...)

\end{itemize}

\subsection{Crossbows} \index{Crossbows} Light Crossbow, Heavy Crossbow, One Handed Crossbow \label{listaarmibalestr}

\begin{center}
\includegraphics[width = 0.9 \linewidth]{immagini/arma-balestra.png}
\end{center}


\begin{itemize}

\item 4 points: You gain the Quick Shot skill.

\item 8 points: Your mastery in the use of crossbows in combat is such that you suffer no penalty from shooting arrows at enemies in melee or with equal or less than light cover.

\item 12 points: Your aim and coolness are legendary. You can decide to aim (2 Actions) on an enemy, if within the next round you hit it your bolt will do critical damage. It only applies to the first hit of the round.

\end{itemize}

\subsection{Dual Weapons} \index{Dual Weapons} Great Double Ax, Double Flail, Two-Bladed Sword, Urgrosh \label{listaarmidoppie}

\begin{itemize}
\item 4 points: Your proficiency in the use of these weapons makes you extremely versatile by giving you the possibility at the beginning of your round to choose whether to be defensive or offensive by increasing your attack roll or defense by 2. It does not cost Actions.

\item 8 points: your technique is unpredictable for the opponent you can choose to have +4 damage with all your attacks or +4 defense. Cannot be combined with the 4-point bonus.

\item 12 points: your mastery is such that the opponent sees 3 weapons. You can make two off-hand attacks both with a -6 to attack roll. Cannot be combined with Two-Weapon Combat Skills or the 4 or 8 point bonus.

\end{itemize}

\begin{center}
\includegraphics[width = 0.8 \linewidth]{immagini/arma-mazzafrusto-mazza8.png}
\end{center}

\subsection{Whirling Balls} Flail, Heavy Flail, Double Flail, Spiked Chain, Whip \label{listaarmipallerotanti}

\begin{itemize}
\item 4 points: After an Attack Action (1 or two Actions) you can make an additional attack (consuming 1 Action) at -5 on the attack roll on another opponent in melee (cannot be the same target as the others strokes). The attack roll is made even if the bonus is negative.

\item 8 points: The impact of your blows is such that it stuns the enemies. If you criticize and hit with the attack roll the opponent must make a Fortitude saving throw (DC equal to your attack roll) if he fails he will suffer -2 Defense until the end of the next round.

\item 12 points: the accuracy and skill in swinging your weapon is such as to confuse the enemy's defense, ignore the protection given by the shield.

\end{itemize}

\subsection{Graceful weapons} \index{Graceful weapons} Rapeseed, Scimitar, Falcione \\ \label{listaarmiaggraziate}

\begin{center}
\includegraphics[width = 0.8 \linewidth]{immagini/sciabole.png}
\end{center}

\begin{itemize}
\item 4 points: You can choose to use Dexterity to determine the hit bonus. The damage bonus is always determined by Strength

\item 8 points: your style looks a lot like a dance. You can use the Charisma value on the attack roll in place of Strength or Dexterity.

\item 12 points: For each -1 to damage you take your initiative increases by 2, up to a maximum of +6. It is a Reaction Action to be declared each round before the actions are resolved.

\end{itemize}

\subsection{Weapons of death} \index{Weapons of death} Light Pike, Heavy Pike, Scythe, Sickle \label{listaarmidelamorte}

\begin{itemize}
\item 4 points: You can perform a Stroke of Grace for 1 Action

\item 8 points: you increase the damage die by one degree (d4 - d6 - d8 - d10 - 2d6 - 2d8 - 2d10 - 3d6 ...)

\item 12 points: increase the damage die by one degree (d6 - d8 - d10 - 2d6 - 2d8 - 2d10 - 3d6 ...)

\end{itemize}

\subsection{Stunning Weapons} \index{Stunning Weapons} Empty Fist, Baton, Spiked Glove \label{listaarmistordimento}

\begin{itemize}
\item 4 points: An unaware opponent if hit with these weapons (during the surprise round) must make a DC 15 Fortitude save or be stunned for 1d6 rounds.

\item 8 points: You double your Strength damage bonus.

\item 12 points: Your stun weapon does 1d6 more nonlethal damage.

\end{itemize}

\subsection{Lance} \index{Lance} Halberd, Urgrosh, Infantry spear, Naginata, Pole glaive, Spear, Brandistocco, Trident. \label{listarmilance}

\begin{center}
\includegraphics[width = 0.8 \linewidth]{immagini/arma-asta.png}

\textit{1 Spit of the lansquenets; 2 Pike; 3 Lancia; 4 hunting spit; 6 Buttafuoco; 7 Falcione at auction; 8 Partisan; 9 Halberd; 10 Halberd; 11 Roncone; 12 Mazzapicchio} \end{center}


\begin{itemize}
\item 4 points: you can also use it against opponents at a distance of 1 meter without penalty.

\item 8 points: Used against a charge or in charge, as long as it has the Counter Charge ability, do two critical damage in addition to the damage.

\item 12 points: If you are not in melee with an opponent you can use the Piercing Blow technique (this action requires all 3 Actions). Charged, you can sacrifice 1 WP and gain 4 to damage (max 10 WP / 40 damage) then perform a weapon only attack. This strike brings you into melee with the opponent and leaves you open for that round giving you a -4 Defense.

\end{itemize}

\subsection{Lethal weapons} Katana, Machete \index{Lethal weapons} \label{listarmiletali}

\begin{itemize}

\item 4 points: Against surprised opponents, add your Weapon Proficiency to the damage

\item 8 points: Your weapon does more damage. You gain a damage category (d4 - d6 - d8 - d10 - 2d6 - 2d8 - 2d10 ..). If this causes the weapon to have the d8 as a damage die, it also gains EDX equal to 8.

\item 12 points: EDX earnings. It is applied only by doing the maximum damage with the die, if the weapon already has an EDX (for example because with the previous bonus it reached 1d8 damage) this decreases by 1.
\end{itemize}

\subsection{Aste} \index{Aste} Javelin, Estoc, Short foot spear, Trident, Halberd \label{listaarmiaste}

\begin{itemize}

\item 4 points: If you make at least one critical roll with the attack roll you can leave the weapon in the opponent's body, penalizing him with a -1 Dexterity. The weapon when removed does critical damage.

\item 8 points: You can use the long weapon in melee within one meter without penalty.

\item 12 points: Double the range if any

\end{itemize}

\subsection{Swords} \index{Swords} Short Sword, Long Sword, Two-Handed Sword, Bastard Sword, Two-Bladed Sword, Katana \label{listarmikatana}, Broad Sword

\begin{itemize}

\item 4 points: Your mastery of the sword technique grants you +1 to damage and an attack roll.

\item 8 points: Your mastery of the sword technique grants you +2 damage and an attack roll.

\item 12 points: you have reached the pinnacle of sword mastery your strikes are precise and difficult to predict you get +1 to damage, attack roll and defense (when you have your sword in hand), the EDX of the sword if present is lowered by 1.

\end{itemize}

\begin{center}
	\includegraphics[width = 0.9 \linewidth]{immagini/arma-tipi-di-spade.png}

	\textit{A Saber, B Scimitar, C One-handed sword, D Broad sword, D Stocco, E Long sword, F One-and-a-half or bastard sword, G Two-handed broadsword}
\end{center}


\subsection{Shields} \index{Shields} Light, Medium, Heavy Shields \label{listaarmiscudi}

You are a master in the use of shields, even as a weapon.

\begin{center}
\includegraphics[width = 0.9 \linewidth]{immagini/scudotorre.png}
\textit{Henry Justice Ford. Heavy Shield}
\end{center}

You can use the shield as a weapon, a small shield does 1d4 damage (B / T), a medium shield does 1d6 damage (B / T), a heavy shield does 1d8 damage (B / T).
You have no penalty on hitting with the shield, for you the shield is not a makeshift weapon. This Weapon List does not have the 6-point bonus and the 18-point bonus common to other Weapons Lists.

\begin{itemize}
\item 1 point: you are competent in all types of shields. You do not have the Strength limit 1 constraint on Heavy Shields.

\item 2 points: Defense bonus increases by 1 and every 4 times you take the proficiency. Using the shield as a weapon does not make you lose the defense bonus given by the shield.

\item 3 points: The Magical Proficiency penalty given by the shield decreases by one die and every 4 times you take the proficiency.

\item 4 points: The weapon proficiency penalty decreases by 1 and 1 every 4 times you take the proficiency.

Your technique effectively mixes defense and attack. Your shield provides an additional +1 bonus to Defense. You can throw your shield with a range of 6 meters.

\item 5 points: increases the damage category of the shield by 1 (1d4 - 1d6 - 1d8 - 1d10 - 2d6 - 2d8 - 2d10) and every 4 additional points in the list (9,13,17 ..).

\item 8 points: Each ally adjacent (within 1 meter) to you has +1 Defense. If you wish, you can take damage from a direct attack on an ally within 1 meter. Using this Skill is a Reaction. You can throw your shield with a range of 30 feet.

\item 12 points: You can throw your shield as if it were a weapon with a range of 12 meters. If you roll a critical roll, the shield once thrown returns to your hand at the end of the round.

\item 18 points: the shield thrown has a range of 18 meters and returns immediately and always, if not unable, in your hands. This allows you to make multiple attacks even from launch.

These bonuses cannot be applied if more than one shield is used.

\end{itemize}

\subsection{Swords and Shields} \index{Swords and Shields} Short Sword, Long Sword, Small Shield, Medium Shield \label{listaarmispadescudi}

\begin{itemize}

\item 4 points: Your mastery of the sword and shield technique grants you +1 to Defense and to Attack Rolls.

\item 8 points: Your mastery of the sword and shield technique grants you +1 to defense and attack rolls.

\item 12 points: You have reached the pinnacle of mastery with the sword and shield, your ability grants you +2 to Defense, damage and attack rolls.
\end{itemize}

\subsection{Missile weapons} Short foot spear, One-handed ax, Javelin, Trident \index{Missile weapons} \label{listarmitiro}

Gain the ability of \textbf{Shattering Shot}: You can throw one of your weapons with such force that it does two critical damage, but accuracy suffers -5 on attack rolls. It costs 2 Actions.

\begin{itemize}
\item 4 points: you have become extremely accurate in throwing your weapon you have +2 to hit and +1 to damage

\item 8 points: your skill allows you to have no downtime after throwing a weapon you can instantly draw another without consuming any actions.

\item 12 points: You double the range of the weapon

\end{itemize}

\subsection{Empty Fist} Punches and Kicks \index{Empty Fist} \hypertarget{pugnovuoto}{} \label{listarmipugnonudo}

This Weapon List does not have the common bonuses of 6 and 18 Weapon Lists points.

\textbf{Empty Fist}: Each time you take this Ability the damage increases following this progression: 1d6 (taken the list twice), 1d8 (3), 2d6 (5), 2d8 (7), 2d10 (9), 3d6 (11), 3d8 (13), 3d10 (15), 4d6 (17).

The player can also decide to do non-lethal damage without incurring any penalty, to the damage he can apply the value of Strength or Dexterity at will.

\begin{itemize}
\item 1 point: Your punches deal lethal damage (1d4). You can use the Strength or Dexterity value on the attack roll and damage.

\item 4 points: Wisdom of the empty hand - you can use the Wisdom value on Hitting and Damage instead of Strength or Dexterity

\item 8 points: Your Defense increases by 1 point.

\end{itemize}

See \hyperlink{equivalenzemagiche}{Vulnerabilità, Resistenza e Immunità} for how magical your strike is.

\subsection{Simple Weapons} Dagger, Light Mace, Club, Spiked Mace, Short Foot Spear, Staff, Crossbow (Light), Javelin. \index{Simple Weapons} \hypertarget{armi.semplici}{} \label{listaarmisemplice}

This subdivision can also be chosen by those who have not assigned points to Weapons Proficiency.

This Weapon List does not grant specific bonuses.

\subsection{Additional Weapons Weapon Lists} \index{Additional Weapons Weapons Lists} \label{listaarmiinpiuliste}

When a character uses a weapon present in more than one known Weapon Lists, he can apply only one combat technique (a Weapon List) per opponent, he does not accumulate the advantages of any other lists.

Using 2 Actions he can concentrate and switch to using the bonuses resulting from applying a different Weapon List.

\end{multicols}

\vfill


\begin{center}
	\includegraphics[width = 0.6 \linewidth]{immagini/brancastle.png}

	\textit{Bran Castle, Transylvania}
\end{center}

\pagebreak

\section{Ability} \index{Ability} \hypertarget{abilita}{} \label{abilita}

\begin{changemargin}{0.3cm}{0.3cm} \begin{enfasi}{Martyrdom is the only way for a man to become famous if he hasn't skill (George Bernard Shaw, The Devil's Disciple)} \end{enfasi} \end{changemargin} \medskip

\begin{multicols}{2}

\lettrine[lines = 2, lhang = 0.33, loversize = 0.25, findent = 1.5em]{T}{he} Ability are peculiar abilities, the result of training or particular skills. Skills always have a practical effect.

Ability make up a large part of what the character can do, they must be chosen with care and attention. It is by choosing the Ability that you establish the style and ability of the character, if you want him more warrior or magician or healer .. or any combination and peculiarity.

\textbf{At first level you take two Ability}. Every 2 subsequent levels, then at 3, 5, 7, 9 ..., you take another Skill which can be the same one already taken or a new Ability learned during the adventures.

It is possible that Prerequisites are indicated under the name of the Ability, in this case they must be respected to take the Skill in question.
Any subsequent prerequisites are indicated on a case-by-case basis.

Don't take Abilities based on power, strength, combination with others but because they are in line with the character's story.
Choosing a jumble of Ability just because strong doesn't make a character powerful but unbalanced, don't be a power-player at any cost.

\medskip

\textbf{Ability must be taken according to the evolutionary path of the character, based on what has been lived and learned during the adventures.}

\medskip

It is possible to change a chosen Ability, while respecting the requirements, retraining for at least 4 months for 4 hours a day.

The abilities provided by the Abilities unless otherwise described are cumulative or if it is the same bonus the greater one applies. If not explicit, a skill cannot be taken multiple times.

\subsection{Saving Throws and Ability} \label{tirisalvezzaedabilita}

Each Ability, except those that directly modify saving throws, grants bonuses to saving throws that stack between skills, even when the same Ability is taken multiple times.

When choosing a Ability, also note which saving throws it increases!

\subsection{Add new Ability} \label{aggiungereabilita}

This list can never be exhaustive given the imagination of the players! But try to understand if what the player wants is a Ability or Competence, having a Ability or knowing how to do something particular.
Evaluate well the prerequisites and the advantages it grants, always try to be balanced, rather grant the advantages to scale, or by taking the Ability several times.

Also remember to mark the bonuses related to saving throws. Usually a concrete and practical Ability grants a bonus of +3 divided between 2 saving throws, a more general Ability grants a 2 points to be divided between a single saving throw or two.

\subsection*{Adept of Magic} \index{Adept of Magic} \hypertarget{scuoladimagia}{} \label{adeptodellamagia}

\textbf{Requirement}: Magical Proficiency 1

\textbf{Saving Throws}: +1 to two saving throws of your choice.

It is only through this Skill that a Magic List can be accessed.

By taking this Ability several times and always selecting the same School, it is possible to access the higher levels of spell.

A spellcaster can take the Adept of Magic Skill multiple times and apply it to a new or already known Magic List.

\subsection*{Wings of the Phoenix} \index{Wings of the Phoenix} \label{alidellafenice}

\textbf{Requirement}: Empty Fist List 4, Dexterity 3

\textbf{Saving Throws}: +2 Reflex, +1 Fortitude

Your fighting style emphasizes long-range hits such as flying kicks and punches.

The \textbf{first time} you take this Ability, Empty Fist List 2, Silver Crane 1 your melee range becomes 2 meters.

The \textbf{second time} you take this Ability, Empty Fist List 6, Silver Crane 3, Iron Fist 1, your melee range becomes 10 feet.

The \textbf{third time} you take this Ability, Empty Fist List 9, Silver Crane 4, Iron Fist 2, your melee range becomes 4 meters.

As long as the opponent does not reach his melee range the character will have a +2 bonus on hitting, as if using a long weapon.

\subsection*{Animalia} \index{Animalia} \label{amimalia}

\textbf{Requirement}: Follower or Devotee of Ephrem or Shayalia, Magical Proficiency 2.

\textbf{Saving Throws}: +2 Will, +1 Fortitude

The ability to transform into a known animal is acquired. Cost 2 Actions.

Your healing spells also work on normal and magical animals or plants.

The \textbf{first time} you take this Ability, you can transform into small nonmagical animals and plants for 10 minutes per Magical Proficiency score. You can only transform once a day. Any equipment is left on the ground during the transformation.

The \textbf{second time}, Magical Proficiency 6, that you take this Ability gains the ability to transform into tiny or medium nonmagical animals and plants, and can transform into a total of 3 times per day for 20 minutes per score of Magical Proficiency in total. Any equipment is left on the ground during the transformation.

The \textbf{third time} you take this Ability, Magical Proficiency 10, you gain the ability to transform into large nonmagical animals and plants, and can transform yourself 5 times per day. The minimum daily transformation time is 16 hours.

Any non-magical equipment is left on the ground during the transformation.
The magical one is absorbed in the transformation and continues to take effect if possible. Armor and shields or magical items do not apply any bonus to Defense but if they have magical abilities that are always active they remain valid. Items that require activation cannot be used.

The \textbf{fourth time}, Magical Proficiency 16, is acquired by transforming oneself into animals or plants of enormous size and also magical (always within the limit of the size and the maximum Challenge Degree is equal to half the Magical Proficiency score). The minimum transformation time is 24 hours a day, and it can transform as many times as you like per day.

All equipment is absorbed into the new form. The magic one continues to take effect if possible. Armor and Shields apply the magic bonus to the creature's Defense, and any magical abilities can be activated.

\textbf{Basic rules for transformation}

Character Traits are replaced by the animal or plant's stats, but the character retains their own Traits, Personalities, and Intelligence, Wisdom, and Charisma scores. He also retains all his Abilitys (but the new form doesn't necessarily allow him to use them) and Saving Throws, as well as gaining those of the new creature.

If the creature has a proficiency that the character also has and the creature's bonus is higher than the character's then use the creature's bonus instead of his own. If the creature has any additional or lair actions, the character can't use them.

The first time he transforms, the character retains his hit points and regains 1d12 hit points.
If the character in transformed form has 14 hit points and takes 14 (or more) damage, he returns to his normal form with the new amount of hit points (0 or negative).

The character cannot cast spells, and his ability to speak or take any action that requires his hands is limited to the capabilities of his transformed form. The transformation does not interrupt the character's concentration on a spell he has already cast, and it does not prevent him from taking actions that are part of a spell already cast, such as Invoke Lightning.

If the character has darkvision, he is lost unless he also has it in his new form. The creature's maximum Challenge Rating is equal to half the caster's Magical Proficiency. It costs 2 Actions to change form and before switching from one form to another you must return to normal form (does not regain hit points every time).

\begin{center}
\includegraphics[width = 0.9 \linewidth]{immagini/animalia3.png}
\textit{Henry Justice Ford}
\end{center}

\subsection*{Pet / Familiar} \index{Familiar} \label{famiglio-abilita}

\textbf{Saving Throws}: +1 Will, +1 Fortitude

You earn a natural animal. This pet has at most a number of hit dice equal to your Wisdom. You can teach your pet basic actions and make him do simple tasks.

\textbf{Requirement}: Magical Proficiency 1, if \textbf{get twice} this Ability gains a Familiar (see specific chapter).

\subsection*{Armor of the Devout} \index{Armor of the Devotee} \label{armaturadeldoveto}

\textbf{Requirement}: Traits in common 2 (sum of the Traits in common with the Patron)

\textbf{Saving Throws}: +2 Will, +1 Reflex

Constant training with your armor allows you to wear light armor without having to do a Magic Check.

The \textbf{second time} you take this Ability, Common Traits 6, make the Magic Check with no additional dice given by medium armor.

The \textbf{third time} you take this Ability, Common Traits 8, make the Magic Check with only 1 additional die given by heavy armor.

The \textbf{fourth time} you take the Ability, Traits in common with Patron 12, perform the Magic Check with no additional dice given by heavy armor.

\subsection*{Armor of the Enchanted Mountain} \index{Armor of the Enchanted Mountain} \label{armaturamontagnaincantata}

\textbf{Requirement}: Empty Fist Weapon List, Weapon Proficiency 1, Magical Proficiency 1, Wisdom 1

\textbf{Saving Throws}: +2 Fortitude, +1 Reflex

The constant training in spirit and body allows you to harden your skin and make it more difficult to injure. To take advantage of these bonuses, you don't need to wear armor or shields or items that improve Defense.

The \textbf{first time} you take this Ability your Defense is 10 + Dexterity + 1/3 the points in Empty Fist.

The \textbf{second time} you take this Ability you can add your Wisdom value to the Defense, up to 1 point (even if you have higher Wisdom).

The \textbf{third time} you take this Ability you can add the full Wisdom value to the Defense.

The \textbf{fourth time} you take this Ability, Hollow Fist 12, your Defense is 10 + Dexterity + Wisdom + 1/2 the points in Hollow Fist.

If you are surprised you lose only the bonus due to Dexterity, your Defense has no penalty on touch.

\subsection*{Archer on horseback} \index{Archer on horseback} \label{arcereacavallo}

\textbf{Saving Throws}: +1 Reflex, +1 Fortitude

The penalty for shooting horse arrows decreases by 2 each time you take this Ability.

Standard penalties are -4 and -6 depending on whether you are trotting (move x2) or gallop (move x3)

\begin{center}
\includegraphics[width = 0.9 \linewidth]{immagini/horsearcher.png}
\textit{Assyrian Archer}
\end{center}

\subsection*{Weapon Focused} \index{Weapon Focused} \label{armafocalizzata}

\textbf{Saving Throws}: +1 Reflex, +1 Fortitude

Choose a weapon. You gain +1 to initiative and attack roll when using this weapon of which you have proficiency.

\subsection*{Gun Artist} \index{Gun Artist} \label{artistadellafuca}

\textbf{Requirement}: Weapon Proficiency 2.

\textbf{Saving Throws}: +1 Will, +1 Fortitude

Choose a Weapon List, on these weapons you get +1 to hit.

The Ability can be taken multiple times and the Weapon List score must be 4 times higher the times this Ability is taken.

If you take \textbf{4 times} this Ability on the same Weapon List the hit bonuses are reduced to +1, rather than +4, but you make two attack rolls for the first two attacks of the round and choose which roll hold.

\subsection*{Whirlwind Attack} \index{Whirlwind Attack} \label{attaccoturbinante}

\textbf{Requirement}: Weapon Proficiency 12

\textbf{Saving Throws}: +2 Reflex, +1 Fortitude

Using 3 Actions you can perform a single attack (with a penalty of 5 on the attack roll) against all melee opponents around you.

\subsection*{Magic Battery} \index{Magic Battery} \label{batteriamagica}

\textbf{Requirement}: Magical Proficiency 3

\textbf{Saving Throws}: +2 Will, +1 Fortitude

You have a particular connection with the magic that remains Yeru.

Each time you take this Ability your Magic Points increase by your Magic Proficiency score.

Each time you take this Ability the value of the Magical Proficiency must be four times the number of times you took this Ability.

\begin{center}
\includegraphics[width = 0.9 \linewidth]{immagini/Historia_Mundi_Naturalis.png}
\textit{Woodcut illustration from an edition of Pliny the Elder's Naturalis Historia (1582)}
\end{center}

\subsection*{Extended Battery} \index{Extended Battery} \label{batteriaestesa}

\textbf{Requirement}: Magical Proficiency 1

\textbf{Saving Throws}: + Fortitude, +1 Will

You can handle the mental stress of casting spells better.

The \textbf{first time} you take this Ability the effects of \hyperlink{quandosihannopochipuntimagia}{When you have low magic points} activate at 60 \% of the Magic Point usage.

The \textbf{second time} you take this Ability the effects of "When You Have Low Magic Points" activate at 70 \% of the Magic Point usage.

The \textbf{third time} you take this Ability the effects of "When You Have Low Magic Points" activate at 80 \% of the Magic Point usage.

The \textbf{fourth time} you take this Ability the effects of "When you have low magic points" no longer apply.

\subsection*{Powerful blows} \index{Powerful blows} \label{colpipoderosi}

\textbf{Requirement}: Weapon Proficiency 1

\textbf{Saving Throws}: +2 Fortitude

Your style emphasizes powerful strokes.

You gain +1 to damage with a Weapon List.

\subsection*{Sneak Strike} \index{Sneak Strike} \index{Back Attack} \label{attaccoallespalle} \label{colpofurtivo}

\textbf{Requirement}: Weapon Proficiency 3

\textbf{Saving Throws}: +2 Reflexes, +1 Will

When the opponent is melee attacked from behind, the first successful melee attack in the fight deals two additional critical damage.

The \textbf{second time} you take this Ability, WP 6, you deal 3 critical damage

The \textbf{third time} you take this Ability, WP 10, you deal 4 critical damage

The \textbf{fourth} you take this Ability, WP 12, deals 5 critical damage

\begin{center}
\includegraphics[width = 0.8 \linewidth]{immagini/teseo.png}

\textit{Henry Justice Ford - Back shot!}
\end{center}


\subsection*{Weak Strike} \index{Weak Strike} \label{colpoindebolente}

\textbf{Requirement}: Stealth Strike 3, Weapon Proficiency 12

\textbf{Saving Throws}: +2 Reflexes, +1 Will

Soft Strike is an advanced form of stealth strike. Each weakening strike lowers Strength or Dexterity (player choice) by the number of times sneak hit.

The opponent is granted a DC Reflex saving throw of 10 + WP / 2 + DES. The additional damage from sneak hit or loss of ability points is caused.

\subsection*{Killing Blow} \index{Killing Blow} \label{colpomortale}

\textbf{Requirement}: Weapon Proficiency 5

\textbf{Saving Throws}: +2 Reflexes, +1 Will

You make the attack roll but ignore any critical rolls rolled. If you hit, with a penalty of 5, you roll an additional critical damage and add up the Strength value for the damage once again. Subsequent attack rolls start at -10 to hit.

\subsection*{Paralyzing Strike} \index{Paralyzing Strike} \label{colpoparalizzante}

\textbf{Requirement}: Weak Strike, Stealth Strike 4, Weapon Proficiency 18

\textbf{Saving Throws}: +2 Reflex, +1 Fortitude

The target after it has been studied for 6 rounds (2 Actions per round) with your next hit in melee, within 6 rounds of studying, must make a Fortitude save with DC equal to 10 + WP / 2 + DES or be paralyzed for 3d6 rounds.

\subsection*{Blind Fighting} \index{Blind Fighting} \label{combattereallacieca}

It is the ability to attack opponents that are not clearly perceptible.

\textbf{Requirement}: Awareness 2

\textbf{Saving Throws}: +2 Reflexes, +1 Will

An opponent with light cover gets no Defense bonus, with medium cover he has a +2 to Defense, with full cover he has a +4 to Defense.

An invisible attacker gains no advantage from hitting the character in melee. The invisible attacker's bonuses still apply to ranged attacks only.

There is no need to perform Acrobatics checks to move at full speed while blinded.

The \textbf{second time} you take the (Awareness to 3) ability, you reduce the Defense bonus from covered or invisible creatures with cover by two more.

\textit{"Zatoichi Level"}, the \textbf{third time} you take the Ability (Awareness at 5), an invisible creature has no advantage against you nor does you have penalties against it.

\subsection*{Combat with two weapons} \index{Combat with two weapons} \index{Two weapons} \label{duearmi}

\textbf{Requirement}: Dexterity 2, Strength 1, Weapon Proficiency 2

\textbf{Saving Throws}: +2 Reflex, +1 Fortitude

The \textbf{first time} that you take this Ability the constant and continuous training allows you to reduce the multi-attack penalty given by the attack with the secondary weapon. When you attack with your secondary weapon, you gain a penalty of -4 instead of -5 on hit.

\textbf{Requirement} Dexterity 3, Weapon Proficiency 12

The \textbf{second time} you can use a medium weapon as a secondary weapon.

\textbf{Requirement} Dexterity 3, Weapon Proficiency 18

The \textbf{third time} the first attack made with the secondary weapon does not accumulate the penalty of multiple attacks.

\subsection*{Concentrate} \index{Concentrate} \label{concentratp}

\textbf{Requirement}: Magical Proficiency 2

\textbf{Saving Throws}: +1 Fortitude, +1 Will

Choose a Spell List, the saving throw DC of your spells on that list increases by 1.

The Ability can be taken multiple times on the same Spell List or other lists as long as the time you take this Ability is less than CM / 4.


\begin{center}
	\includegraphics[width = 0.8 \linewidth]{immagini/oggettimagiciuomo.png}

	\textit{Henry Purcell - King Arthur}
\end{center}

\subsection*{Creating Magic Items} \index{Creating Magic Items} \label{creaoggettimagici}

\textbf{Requirement}: Magical Proficiency 6

\textbf{Saving Throws}: +1 Fortitude, +1 Will

Through this Ability the caster is able to infuse a spell up to level 3 into a magical item.

\subsection*{Creating Greater Magic Items} \index{Creating Greater Magic Items} \label{creaoggettimagicisuperiori}

\textbf{Requirement}: Craft Magical Items, Magical Proficiency 12

\textbf{Saving Throws}: +1 Fortitude, +1 Will

Through this Ability the caster is able to infuse a spell up to level 5 into a magical item.

\subsection*{Creating Wondrous Magical Items} \index{Creating Wondrous Magic Items} \label{creaoggettimagicimeravigliosi}

\textbf{Requirement}: Create Greater Magical Items, Magical Proficiency 16

\textbf{Saving Throws}: +1 Fortitude, +1 Will

Through this Ability the caster is able to infuse a spell up to level 8 into a magical item.


\subsection*{Create Mythic Magical Objects} \index{Create Mythic Magical Objects} \label{creaoggettimagicimitici}

\textbf{Requirement}: Craft Wondrous Magical Items, Magical Proficiency 18

\textbf{Saving Throws}: +1 Fortitude, +1 Will

Through this Ability the caster is able to infuse a spell up to level 9 into a magical item.

\subsection*{Dance of the Blade} \index{Dance of the Blade} \label{danzadellalama}

\textbf{Requirement}: Weapon List: Weapons Graceful 2, Dexterity or Charisma 1, Perform 1

\textbf{Saving Throws}: +2 Reflex, +1 Fortitude

You can replace just the damage given by Strength in melee attacks with half the value of your Charisma or Dexterity.

The \textbf{second time}, Graceful Weapons 4, Entertain 3, that you take the Ability you can fully add Charisma to the damage of the weapon, always ignoring the damage given by the Force.

The \textbf{third time}, Graceful Weapons 7, Entertain 5, that you take the Ability you can totally add Dexterity to the weapon damage, again ignoring the damage given by the Force.

The second and third benefits are not cumulative.

\subsection*{Deciphering magical writings} \index{Deciphering magical writings} \label{decifrarescrittimagici}

\textbf{Requirement}: Magical Proficiency 1

\textbf{Saving Throws}: +1 Fortitude, +1 Will

Knowing how to read magical writings. Has a + 1d6 bonus on understanding the contents of a scroll and casting the spell it contains. The bonus also applies to the check to copy a spell on your tome of spell.

\subsection*{Defending Mount} \index{Defending Mount} \label{difenderecavalcatura}

\textbf{Saving Throws}: +1 Fortitude, +1 Reflex

Whenever the mount is hit, you can make a Ride check to negate the hit. Your Ride check must be greater than your opponent's attack roll

The Ability can only be used once per round, for a single attack, it costs the Reaction.

\subsection*{Distill Potions} \index{Distill Potions} \label{distillarepozioni}

\textbf{Requirement}: Magical Proficiency 1

\textbf{Saving Throws}: +1 Fortitude, +1 Will

Expertise in brewing potions.

You gain a + 1d6 bonus on Herbalism Knowledge and in distilling and creating natural potions and poisons.

\subsection*{Double portion} \index{Double portion} \label{doppiaporzione}

\textbf{Requirement}: Two-Weapon Combat, Weapon Proficiency 4

\textbf{Saving Throws}: +2 Fortitude, +1 Reflex

The constant training with two weapons allows you to apply the bonus to the damage due to Strength in a full way even to the secondary weapon.

\subsection*{Psychic Energy} \index{Psychic Energy} \label{energiapsichica}

\textbf{Requirement}: Strength 1, Wisdom 2, Weapon Proficiency 1, Magical Proficiency 1

\textbf{Saving Throws}: +2 Will, +1 Fortitude

After years of training, meditation and internships at Nanda Parbat you are able to harvest your Chi Energy.

Every day after at least 6 hours of rest and 2 hours of meditation / training, fill your body with Chi energy equal to (Weapon Proficiency + Magical Proficiency) / 2 + Wisdom / 2

The \textbf{second time} you take this Ability: \textbf{Requirement}: Strength 1, Wisdom 2, Weapon Proficiency 4, Magical Proficiency 4

You recover 2 Chi points for each hour of rest.


\begin{center}
\includegraphics[width = 0.75 \linewidth]{immagini/distillare.png}

\textit{The Alchemist Discovering Phosphorus. Joseph Wright of Derby (1771-1795)}
\end{center}

\subsection*{Psychic Strike} \index{Psychic Strike} \label{colpopsichico}

\textbf{Requirement}: Psychic Energy, Dexterity 2

\textbf{Saving Throws}: +2 Will, +1 Fortitude

Concentrate your Chi in your hands.
You can concentrate a number of Chi points equal to Wisdom.
With your first hit, Touch Defense, in the round you drain the energy that deals 1d6 points of force damage per chi point used.

The \textbf{second time} you take this Ability: \textbf{Requirement}: Psychic Strike, Wisdom 3, Weapon Proficiency 7

You can use up to double your Wisdom score to boost Psychic Strike.

\subsection*{Psychic Ray} \index{Psychic Ray} \label{raggiopsichico}

\textbf{Requirement}: Psychic Strike, Wisdom 3, Weapon Proficiency 5

\textbf{Saving Throws}: +2 Reflexes, +1 Will

You can make a ranged attack within 30 feet using Psychic Energy.
The hit, an attack roll against touch defense, deals 1d6 points of force damage per psychic point spent focused on the damage.

It is possible to focus on one or more Psychic points to increase the distance by 30 feet each time.
You cannot use a total number of Chi points (for distance and damage) greater than Wisdom.

The \textbf{second time} you take this Ability: \textbf{Requirement}: Psychic Strike, Wisdom 3, Weapon Proficiency 9

You can use up to double your Wisdom score to upgrade Psychic Range.

\subsection*{Expert} \index{Expert} \label{esperto}

\textbf{Requirement}: Characteristic linked to at least -1

\textbf{Saving Throws}: +1 to two saving throws of your choice.

You are an expert in a topic. Whenever you take this Ability you gain +2 on checks on a Skill of your choice.

You cannot take this Skill more than twice on the same Skill.

It is not usable on Awareness (see Perceptive).

\subsection*{Infuriate} \index{Infuriate} \label{fareinfuriare}

Your dialectical skills are amazing.

\textbf{Requirement}: Weapon Proficiency 2 and Charisma / Strength 2

\textbf{Saving Throws}: +2 Will, +1 Fortitude

It takes 2 Actions to slander and rail against an opponent. The target must make a Will saving throw against your Entertain or Intimidate proficiency check or lose the Dexterity bonus (saving throws, attack rolls, and defense) until the end of the next round.

The opponent may not understand your language but must have Intelligence equal to -3 or more, the opponent takes +/- 1d6 on the saving throw for every 2 sizes of difference (bigger is more difficult).

\subsection*{Faithful} \index{Faithful} \label{fedele}

\textbf{Requirement}: Magical Proficiency 1, Sum value of Traits in common 2

\textbf{Saving Throws}: +2 Will, +1 Fortitude

Your connection with the Patron is strong and energetic. Each time you take this Ability you can your Magic Points increase by the sum of the Traits common with the Patron.

Each time you take this Ability the value of the sum of the Traits in common with your Patron must be four times the number of times you took this Ability. This Ability does not stack with the Magic Battery Ability.

\subsection*{Ferocity} \index{Ferocity} \label{ferocia}

\textbf{Requirement}: Weapon Proficiency 1

\textbf{Saving Throws}: +2 Fortitude, +1 Will

Your anger is such that it temporarily defeats death.

When you drop below 0 hit points you don't faint and you start to lose 1 hit point per round.

A creature with Ferocity faints when it has a negative hit point score equal to twice its Constitution points and dies when its hit points drop to a negative score equal to its triple Constitution score + 5 (COS * 3 + 5)

The \textbf{second time} you take this Ability, Weapon Proficiency 4 requirement, you can have your Strength increase by 2 in combat and gain 6 temporary hit points for 10 minutes. At the end of the fight, your fatigue level increases by 1 for 10 minutes.

The \textbf{third time} you take this Ability, Weapon Proficiency 6 requirement, you can have your Strength increase by 3 in combat and gain 12 temporary hit points for 10 minutes. At the end of the fight, your fatigue level increases by 2 for 20 minutes.

The \textbf{fourth time} you take this Ability, Weapon Proficiency requirement 8, you can have your Strength increase by 4 in combat and gain 18 temporary hit points for 10 minutes. At the end of the fight, your fatigue level increases by 2 for 30 minutes.

The player can choose which degree of Ferocity to use in the fight (2, 3, 4).

\subsection*{Daughter of Shayalia} \index{Daughter of Shayalia} \label{figliadishayalia}

Your connection with nature is strong and concrete

\textbf{Requirement}: Devotee or Follower of Shayalia

\textbf{Saving Throws}: +1 Fortitude, +2 Will

The \textbf{first time} you take this Ability, you get +2 on Nature checks and +2 on saving throws against natural poisons.

The \textbf{second time} you take this Ability, common traits requirement 6, you get a +4 to Nature check and a +4 to saving throws against effects, even magical, from animals or plants.

The \textbf{third time} you take this Ability, Common traits requirement 12, you are always under the effect of the Sanctuary spell to any nonmagical animal.

\subsection*{Fake Death} \index{Fake Death} \label{fintamorte}

You are able to simulate death by slowing the heart.

\textbf{Saving Throws}: +2 Fortitude, +1 Will

As a Reaction Action you are able to fall to the ground (fall to the ground!) Dead. Only a DC 20 First Aid check can reveal that you are alive.

The effect lasts a maximum of 2 minutes. The fake death is not repeatable in intervals of less than 10 minutes from each other.

\subsection*{Dancing Scourge} \index{Dancing Scourge} \label{flagellodanzante}

\textbf{Requirement}: Weapon Proficiency 1

\textbf{Saving Throws}: +1 Fortitude, +1 Will

When you use your Scourge you have a +1 bonus on attack rolls and +1 on defense

\subsection*{Forged in fury} \index{Forged in fury} \label{forgiatonellafuria}

\textbf{Requirement}: Weapon Proficiency 5

\textbf{Saving Throws}: +1 Fortitude, +1 Reflex

When you make a critical roll, i.e. you rolled at least 2 times 6, you are considered to have rolled an extra 6 for the total count of the number of critics

\subsection*{Match Dice} \index{Match Dice} \label{daditruccati}

\textbf{Requirement}: Magical Proficiency 6

\textbf{Saving Throws}: +1 Fortitude, +1 Reflex

You can increase a die in the Magic Check by 1, within the value of 6.


\subsection*{Elemental Form} \index{Elemental Form} \label{formaelementale}

\textbf{Requirement}: Follower or Devotee of Erondil, Gaya, Ephrem. Must have Magic Lists on an Element. Magical Proficiency 6

\textbf{Saving Throws}: +1 Fortitude, +1 Will

The \textbf{first time} you take this Ability you can transform once per day, for 1 minute, into a Minor Elemental (Challenge Rank 2) from the Spell List you know.

The \textbf{second time} you take this Ability, Magical Proficiency 11, you can transform twice a day, for a total of 20 minutes, into an Elemental (Challenge Rank 5) from the Spell List you know.

The \textbf{third time} you take this Ability, Magical Proficiency 16, you can transform four times a day, for a total of 1 hour, into a Major Elemental (Challenge Rank 9) from the Spell List you know.

It costs 2 Actions to transform and to change elemental, you must first return to your original form.

\subsection*{Arrow called, arrow delivered} \index{Arrow called, arrow delivered} \label{frecciachiamata}

\textbf{Requirement}: Weapon Proficiency 2

\textbf{Saving Throws}: +2 Reflexes

You can shoot 1 arrow, once per day, as a reaction, with no penalty to hit from multi-attack.

\subsection*{Fury} \index{Fury} \label{furia}

\textbf{Requirement}: Weapon Proficiency 1

\textbf{Saving Throws}: +2 Fortitude, +1 Will

Your fighting style is blind killing spree. Add + 1d6 to damage to each successful melee attack and your opponents gain + 1d6 to hit towards you. You can decide to activate this Ability round by round.

\subsection*{Juggler} \index{Juggler} \label{giocoliere}

\textbf{Requirement}: Dexterity 2

\textbf{Saving Throws}: +2 Reflexes

You have a natural talent for handling objects.

Any Acrobatics check that involves handling objects or balancing has a +2 bonus.

You can throw a second dagger as an immediate action to the dagger throw attack action with a -3 on the attack roll. Any third dagger thrown has the normal penalty of -5 (and -10 .. and so on).

\begin{center}
\includegraphics[width = 0.9 \linewidth]{immagini/Early_Egyptian_juggling_art.png}

\textit{This ancient wall painting appears to depict jugglers.}
\end{center}

\subsection*{Warrior of Magic} \index{Warrior of Magic} \label{guerrierodellamagia}

\textbf{Saving Throws}: +1 Will, +1 Reflex

You don't just follow the path of magic or the path of the sword, your style embraces both in a slash of pure magic. Each time you cast a spell through the weapon you must make a Magic Check without fail. The spell must have a casting time of 2 Actions or less, and the origin is considered to be the creature hit.

The \textbf{first time} you take this Ability, Weapon Proficiency 2, Magic Proficiency 2: You are able to unload a melee range spell with your weapon. You make a normal attack roll (3d6 + WP + Strength + ...) and if you hit, in addition to the damage of the attack, you also unload the spell. It costs 3 Actions.

The \textbf{second time} you take this Weapon Proficiency 6 Ability, Magical Proficiency 3: By consuming 3 Actions you are able to combine up to two melee attacks with the discharge of one spell. It costs 3 Actions.

You cannot download spells above level 3 with this Ability.

\subsection*{Silver Crane} \index{Silver Crane} \label{grudargento}

\textbf{Requirement}: Empty Fist List 3

\textbf{Saving Throws}: +2 Reflexes, +1 Will

Your unarmed fighting style is based on agility and counterattack.

The \textbf{first time} you take this Ability your Defense increases by 1.

The \textbf{second time} you take this Ability, \textbf{Requirement} Empty Fist List 4, your Initiative increases by 2 (only with unarmed attacks).

The \textbf{third time} you take this Ability, \textbf{Requirement} List Empty Fist 9 and Dexterity 2, you have a bonus on Reflex and Fortitude saving throws of 2 (cumulative).

The \textbf{fourth time} you take this Ability, \textbf{Requirement} Empty Fist List 11, your Defense and Initiative increase by 2 (cumulative).

The \textbf{fifth time} you take this Ability, \textbf{Requirement} List Empty Fist 13 and Dexterity 3, you have a bonus of 2 (cumulative) on Reflex and Will saving throws.

\subsection*{I said Down!} \index{I said Down!} \label{hodettocadi}

\textbf{Requirement}: Weapon Proficiency 4

\textbf{Saving Throws}: +2 Fortitude, +1 Will

If you hit 3 times within 2 rounds an opponent must make a Fortitude save (DC equal to the attack roll of the last second hit) or fall prone.

\subsection*{The Patron is with me} \index{The Patron is with me} \label{ilPatronoeconme}

\textbf{Requirement}: Sum common traits with the Patron 2

\textbf{Saving Throws}: +1 Will, +1 Reflex

The \textbf{first time} you take this Ability 1 time per day, you can re-roll a die obtained in the Magic Check for spellcasting. The Ability can also be declared after the roll of the dice.

The \textbf{third time} you take this Ability, add common Traits with Patron 6, 2 times per day you can re-roll 2 dice obtained in the Magic Check for spellcasting. The Ability can also be declared after the roll of the dice.

The \textbf{fifth time} you take this Ability, add Common Traits with Patron 12, 3 times per day you can reroll the 1,2,3 obtained in the Magic Check for spellcasting. The Ability can also be declared after the roll of the dice.

Any new values obtained with the new roll must be kept or this Ability is used again.

\subsection*{Improvising} \index{Improvising} \label{improvvisare}

\textbf{Requirement}: Weapon Proficiency 1

\textbf{Saving Throws}: +1 Fortitude, +1 Reflex

Any item that can be called an improvised weapon for you is not improvised.
You suffer no penalty on hitting when using an improvised weapon. Gain +1 to damage when using an improvised weapon.

\subsection*{Combat Spellcaster} \index{Combat Spellcaster} \label{incantatoredacombattimento}

\textbf{Saving Throws}: +1 Fortitude, +1 Will

When you are under attack and need to make a Magic Check you can roll one less die.

The Ability can be taken multiple times but the score must be less than or equal to Magical Proficiency / 6

\subsection*{Wary Spellcaster} \index{Wary Spellcaster} \label{incantatoreprudente}

\textbf{Saving Throws}: +2 Reflex, +1 Fortitude

The first time you take this Ability the defense penalty while casting a spell under threat decreases by 2 (from -4 to -2).

The \textbf{second time} you take this Ability, Weapon Proficiency minimum 3, the Defense penalty decreases by 1 (and goes to -1)

The \textbf{third time} you take this Ability, minimum weapon proficiency 6, the defense penalty becomes 0.

This Ability does not affect whether you are still Distracted in casting.

\subsection*{Immunity to poisons} \index{Immunity to poisons} \label{immunitaaiveleni} \index{Mithridatism}

\textbf{Saving Throws}: +2 Fortitude, +1 Will

The body becomes accustomed to the poisons, the character gains a +2 saving throw on the poisons.

The \textbf{second time} you take the Ability you become immune to natural poisons. You can't get drunk normally anymore.

The \textbf{third time} you have a + 1d6 saving throw on magical poisons and suffer the effects of toxic fumes (but you can always choke).

\subsection*{Lay on of hands} \index{Lay on of hands} \label{imposizionedellemani}

\textbf{Requirement}: Magical Proficiency 3, Common Traits 3

\textbf{Saving Throws}: +2 Will, +1 Fortitude

If your Traits are in common with a positive Patron you can channel positive energy (healing effect), if they are in common with a neutral or evil Patron you can channel negative energy (harmful effect). Usable a number of times per day equal to the Wisdom value. Healing / damaging effect equal to 1d6 + Wisdom

The \textbf{second time}, \textbf{Requirement} Common Trait score 6, that you take this Ability the effect becomes 2d6 and I increase the usage by 1 times the usage.

The \textbf{third time}, \textbf{Requirement} Common Trait score 12, that you take this Ability the effect becomes 3d6 and I increase the usage by 2 times the usage.

The \textbf{fourth time}, \textbf{Requirement} Common trait score 18: that you take this Ability the effect becomes 4d6 and I increase the use by 3 times.

The energy comes from the hands (it doesn't matter if there are gloves) and applies only to touch (or Touch Defense). Use 2 Actions.

\begin{center}
\includegraphics[width = 0.65 \linewidth]{immagini/Portrait_of_V_Greatrakesv2.png}

\textit{Portrait of V. Greatrakes laying on his hands, window, in right-hand corner showing several successful cures, possibly. By W. Faithorne}
\end{center}


\subsection*{Channel energy} \index{Channel energy} \label{incanalareenergia}

\textbf{Requirement}: Magical Proficiency 1, Common Traits 3

\textbf{Saving Throws}: +2 Will, +1 Reflex

You are able to channel the positive or negative energy of your Patron.

If your Traits are in common with a positive Patron you can channel positive energy (healing), if they are in common with a neutral or evil Patron you can channel negative energy. Usable a number of times with the same Wisdom score. Healing / damaging effect equal to 1d6 + Wisdom. Affect 1 creature.

The \textbf{second time}, Trait score in Common 6, that you take this Ability the effect becomes 2d6. Affect up to 2 creatures.

The \textbf{third time}, Trait score in Common 12, that you take this Ability the effect becomes 3d6. Affect up to 3 creatures.

The \textbf{fourth time}. Common Traits score 18, that you take this Ability the effect becomes 4d6. Affect up to 4 creatures. Radius of 6 meters around you.

The energy comes from the hands (it does not matter if there are gloves) and affects one or more creatures within three meters of you, possible Reflex save DC 10 + sum traits in common with the Patron to avoid the effect. You choose the creatures you influence. Use 2 Actions.

\subsection*{Channel energy at a distance} \index{Channel energy at a distance} \label{incanalareenergiaadistanza}

\textbf{Requirement}: Channel energy

You can cast energy, as per Channel Energy, up to 30 feet, affects a range of 10 feet. Use 2 actions.

The \textbf{second time} you take this Ability the energy goes up to the distance of 60 feet.

The \textbf{third time} you take this Ability the energy goes up to a distance of 36 meters.

\subsection*{Channel concentrated energy} \index{Channel concentrated energy} \label{incanalareenergiaconcentrata}

\textbf{Requirement}: Channel energy

\textbf{Saving Throws}: +2 Will, +1 Reflex

You can throw energy up to 60 feet away. Single goal. Use 2 actions.

Each time you take this Ability you add an objective within 6 meters of the previous objective on which to divide the dice available to channel energy at will.

The ability cannot be combined with "Channel energy at a distance".


\begin{center}
\includegraphics[width = 0.8 \linewidth]{immagini/kameame.png}
\textit{Kamehameha!}
\end{center}

\subsection*{Infuse Courage} \index{Infuse Courage} \label{infonderecoraggio}

\textbf{Requirement}: Charisma 2, Entertain 1

\textbf{Saving Throws}: +2 Will, +1 Fortitude

Through your performance, singing, ballet, oratory .. you are able to instill courage in your companions who are able to hear or see you, within a radius of 6 meters.

The first time you take this Ability, your teammates have a +1 bonus to attack rolls and damage in combat.

The \textbf{second time} you take this Ability, Perform 4 requirement, you may choose to instill up to 2 of these bonuses. +2 TC, +2 Defense, +2 Damage, +2 Will save. Your teammates must be within 12 meters of range.

The \textbf{third time} you take this Ability, Perform 12 requirement, you may choose to instill up to 2 of these bonuses. + 1d6 attack roll, +4 defense, +4 damage, + 1d6 save. Your teammates must be within 24 meters of range.

Activating, maintaining, or changing the Ability's effect requires 1 Action. You can keep the Ability a number of rounds, even non-consecutive, equal to Charisma x 5 per day. Creatures to remain affected must continue to see / hear the performance.

\subsection*{Infuse Fear} \index{Infuse Fear} \label{infonderepaura}

\textbf{Requirement}: Charisma 2

\textbf{Saving Throws}: +2 Will, +1 Fortitude

Through your performance, singing, ballet, oratory .. you are able to instill fear in opponents who can hear you, within a radius of 6 meters.

The first time you take this Ability your enemies have a -1 penalty on attack rolls and damage in combat.

The \textbf{second time} you take this Ability, Entertain requirement 4, the strength of your art attacks enemies and you can select two effects from: -2 Attack Roll, -2 Combat Damage, -2 Defense, -2 on Will saving throws. Your enemies must be within 12 meters of range.

The \textbf{third time} you take this Ability, Entertain requirement 12, the strength of your art attacks enemies and you can select two effects from: -1d6 Attack Roll, -4 Defense, -4 Damage, -1d6 TS. Your enemies must be within 24 meters of range.

The opponent is allowed a DC Will save of 10 + CH + Perform score. A creature that succeeds at the saving throw is immune to new manifestations of this power of yours for 24 hours that day.

Activating, maintaining, or changing the Ability's effect requires 1 Action. You can keep the Ability a number of rounds, even non-consecutive, equal to Charisma x 5 per day. Creatures to remain affected must continue to see / hear the performance.

\subsection*{Improved Initiative} \index{Improved Initiative} \label{iniziativamigliorata}

\textbf{Saving Throws}: +2 Reflexes

Increase initiative by +1. The Ability can be taken up to 2 times and the bonus stacks.

\subsection*{Iaijutsu} \index{Iaijutsu} \label{iaijutsu}

\textbf{Saving Throws}: +2 Reflexes, +1 Will

For every -5 to the attack roll you gain +5 to Initiative and vice versa.
The bonus must be used by the end of the next round. The declaration must be made every round that is intended to be used when checking the initiatives.

\subsection*{My death your death} \index{My death your death} \label{lamiamortelatuamorte}

\textbf{Saving Throws}: +2 Fortitude, +1 Will

For each individual combat opponent, you can have the first hit of the fight cause an additional damage equal to double the Weapon Proficiency. The opponent gains a bonus on attack rolls and damage equal to the value of your weapon proficiency. It must be declared before the attack roll.

\subsection*{My Head is Harder} \index{My Head is Harder} \label{lamiatestaepiudura}

\textbf{Requirement}: Weapon Proficiency 1

\textbf{Saving Throws}: +1 Fortitude, +1 Will

Your Skull Break Weapon does +2 damage

\subsection*{Versatile litany} \index{Versatile litany} \label{itaniaversatile}

\textbf{Requirement}: Performing Competence 6

\textbf{Saving Throws}: +1 Will, +1 Reflex

Through your performance you can decide to instill courage or fear in creatures within 30 feet of you. Each round you can decide to apply up to 2 modifiers between: bonus of + 1d6 to the attack roll or +4 to the defense or -1d6 to the attack roll or -4 to the defense. Changing bonuses costs 1 Action.

The opponent is allowed a DC Will save of 10 + CH + Perform score. A creature that succeeds at the saving throw is immune to new manifestations of this power of yours for 24 hours that day.

Activating and maintaining the Ability requires 2 Actions. You can keep the Ability a number of rounds, even non-consecutive, equal to Charisma x 5 per day. Creatures to remain affected must continue to see / hear the performance.

\subsection*{The shield is my friend} \index{The shield is my friend} \label{loscudoemioamico}

\textbf{Requirement}: Weapon Proficiency 1

\textbf{Saving Throws}: +1 Fortitude, +1 Reflex

You can use a Buckler without having to make a Magic Check when casting a spell.

The \textbf{second time} you take this Ability, Weapon Proficiency 3, the weapon proficiency penalty decreases by 1.

The \textbf{third time} you take this Ability, Weapon Proficiency 5, the weapon proficiency penalty decreases by 3, you can use light shields without having to make a Magic Check.

\subsection*{Mighty Magic} \index{Mighty Magic} \label{magiepotenti}

\textbf{Requirement}: Magical Proficiency 5

\textbf{Saving Throws}: +2 Will

Your spells are extraordinarily effective.

Choose a List of Magic, get + 1d6 to the Magic Check when casting spells from this school. The ability can be taken multiple times but the total must be less than CM / 4 and the bonus adds up or applies to other Magic Lists.

%\begin{center}
%	\includegraphics[width=0.65\linewidth]{immagini/elcolosso.png}
%
%	\textit{The Colossus (also known as The Giant), is known in Spanish as El Coloso. Non è di Goya ma di un allievo.}
%\end{center}

\subsection*{Human mountain} \index{Human mountain} \label{montagnaumana}

\textbf{Saving Throws}: +3 Fortitude

Maybe you were once frail and weak, now you are a mountain of muscle.

When you take this Ability, you increase the Hit Points taken per level by 1.

The \textbf{second time} you take this Ability, you increase Hit Points taken per level by 1.

The \textbf{third time} you take this Ability you increase the die to roll Hit Points (from d4 to d6).

Bonuses are cumulative and retroactive to previous levels, except for the health die increase.

The \textbf{fourth time} you take this Ability you increase by one size (P> M> G> E).

\subsection*{Clinical Eye} \index{Clinical Eye} \label{occhioclinico}

\textbf{Requirement}: Weapon Proficiency 3

\textbf{Saving Throws}: +2 Reflexes

You are able to do critical damage to creatures that are normally immune to crit (roll 6 multiple times, damage explosion).

\subsection*{Hawkeye} \index{Hawkeye} \label{occhiodifalco}

\textbf{Requirement}: Weapon Proficiency 3

\textbf{Saving Throws}: +2 Reflexes, +1 Will

The penalty for shots between the first and second increment has no penalty

The \textbf{second time} you take this Ability, the penalty for rolls up to the third range increase is 1d6.

The \textbf{third time} you take this Ability you are able to extend your roll even further and take it to a fifth increase with a -2d6 hit penalty. You have no penalty within the first 3 increments while you have -1d6 to hit between the third and fourth increments. \\

\textbf{Example. Tups uses a Short Bow, range 15 meters.}

Normally if he has to shoot an arrow at an orc within 15 meters he has no penalty, if the oret is between 15 and 30 meters he has 1d6 to attack roll, if he is between 30 and 45 meters he has -2d6 to hit. Beyond can not shoot.

- \textbf{Tups takes Hawkeye Ability}.

If he has to shoot an arrow at an orc within 15m he has no penalty, if he is between 15 and 30m he has no penalty, if he is between 30 and 45m he has -2d6 to hit. Beyond can not shoot.

- \textbf{Tups takes the Hawkeye Ability a second time}.

If he has to shoot an arrow at an orc within 15m he has no penalty, if he is between 15 and 30m he has no penalty, if he is between 30 and 45m he has -1d6 to hit, if he is between 45 and 60m has a -2d6 on hitting. Beyond can not shoot.

- \textbf{Tups takes the Hawkeye Ability a third time}.

If he has to shoot an arrow at an orc within 15 m he has no penalty, if the orchetto is between 15 and 30 meters he has no penalty, if he is between 30 and 45 meters he has no penalty to hit, if he wants to he can shoot the arrow between 45 meters and 60 meters with -2d6 to hit.

\subsection*{Opportunist} \index{Opportunist} \label{opportunista} \hypertarget{opportunista}{}

\textbf{Saving Throws}: +2 Reflexes, +1 Will

\textbf{Requirement}: Weapon Proficiency 2

You may attempt to hit an opponent who \textbf{decoy} or \textbf{crosses} a melee area that you threaten. The ability can be used once per round as a Reaction. This attack is also referred to as an attack of opportunity in the manual, and there are several creatures that don't trigger the reaction.

\subsection*{Fast Step} \index{Fast Step} \label{passoveloce}

\textbf{Saving Throws}: +2 Reflex, +1 Fortitude

Your pace is naturally quick.
If you have 6m move you step to 7m move, if you have 9m move you step to 10m move.

Each additional two times you take the Ability, your movement increases by 1 meter per Move Action, to a maximum of +3 meters per round.


\subsection*{Mocassin step}\index{Mocassin step}\label{mocassinstep}

\textbf {Saving Throws}: +1 Reflex, +1 Fortitude

\textbf {Requirement}: Move Silently to 1 or more

Your step is naturally silent.
The \textbf{first} time you take this skill the penalty for moving at full speed with the Move Silently competence becomes -1d6.

The \textbf{second time} you take this Skill, Dexterity 3 requirement, Move Silently 8, you have no penalty for moving at full speed.


\subsection*{Safe step} \index{Safe step} \label{passosicuro}

\textbf{Saving Throws}: +2 Fortitude, +1 Reflex

It is the ability not to be slowed down in a hostile environment. It is necessary to declare on which environment the Ability is taken. In these environments the terrain is not difficult for you.

\bigskip

\begin{tabular}{l|l}
\textbf{Environment} & \textbf{Environment} \\
\toprule
Jungle & Aquatic \\
Mountains & Hills and Forest \\
Plain & Desert and barren \\
Swamp & Glaciers and Tundra \\
Urban & Underground \\
\end{tabular}

\bigskip

Each time you take this Ability again, you choose a different environment and add to the previous one.

\subsection*{Leathery skin} \index{Leathery skin} \label{pellecoriacea}

\textbf{Requirement}: Constitution 2

\textbf{Saving Throws}: +2 Fortitude

Your skin is extremely resistant. Take 1 less damage when hit by slashing, piercing, or bludgeoning weapons.

The \textbf{second time} you take this Ability, Weapon Proficiency 4 requirement, take 1 less damage when struck by slashing, piercing, or bludgeoning weapons. Reduce Bleeding by 1 when acquired.

The \textbf{third time} you take this Ability, Weapon Proficiency 8 and Constitution 3 requirement, take 2 less damage when hit by slashing, piercing, or bludgeoning weapons. Take 1 less damage when hit by magic. Reduce Bleeding by 1 when acquired.

The \textbf{fourth time} you take this Ability, Weapon Proficiency 16 requirement, you ignore 1 critical roll when hit by slashing, piercing, or bludgeoning weapons and take 1 hand damage when hit by magic. Reduce Bleeding by 1 when acquired.

Bonuses are cumulative.

\subsection*{Perceptive} \index{Perceptive} \label{percettivo}

\textbf{Saving Throws}: +1 Reflex, +1 Will

Your Awareness and attention to detail is above average.
You take a +1 bonus on Awareness check. The Ability can be taken up to 3 times.

\subsection*{Truly Evil Person} \index{Truly Evil Person} \label{personaveramentemalvagia}

\textbf{Requirement}: Weapon Proficiency 1

\textbf{Saving Throws}: +1 Reflex, +1 Will

Twice a day, add your Weapon Proficiency value to the damage to an opponent you want to hit in melee. The Ability must be declared before the attack roll. It costs one Action.

\subsection*{The bigger they are, the more they make noise when they fall} \index{The bigger they are, the more they make noise when they fall} \index{Giant Killers} \label{piusonogrossipiufannorumore}

\textbf{Requirement}: Weapon Proficiency 1

\textbf{Saving Throws}: +2 Fortitude, +1 Will

When you attack a creature at least 2 sizes bigger than you, you do +1 additional damage for every 2 Weapon Proficiency points. If it's only one size larger, add 1 more damage for every 3 Weapon Proficiency points.

\subsection*{Continue} \index{Continue} \label{proseguire}

\textbf{Requirement}: Weapon Proficiency 1

\textbf{Saving Throws}: +1 Fortitude, +1 Will

If you kill the opponent with your last Attack Action, in melee, you can take a bonus attack action with the same modifiers as the last Attack Action you made and attack the next opponent in melee range, if you kill this creature with a hit, you can't make any other attacks on other creatures.

The \textbf{second time} you take this Ability \textbf{Requirement}: Carry on, Weapon Proficiency 6

If you kill the opponent with your last Attack Action, you can take a bonus attack action with the same modifiers as the last Attack Action you made with the weapon and attack the next creature in melee range. If you kill it you can continue with another bonus attack (and move within 1 meter) with the next creature and so on.

Each bonus attack beyond the first has a -2 to hit and a -1 to cumulative damage.

\subsection*{Iron Fist} \index{Iron Fist} \label{pugnodiferro}

\textbf{Requirement}: Empty Fist List 3

\textbf{Saving Throws}: +2 Fortitude, +1 Will

Your unarmed combat technique is extremely precise and powerful.

The \textbf{first time} you take this Ability the damage caused by your punches (and kicks) and the attack roll increases by 1. Your hits are treated as silver weapons.

The \textbf{second time} you take this Ability, \textbf{Requirement} Empty Fist 6. Damage +2, TC +1. Your hits are treated as a +1 weapon.

The \textbf{third time} you take this Ability, \textbf{Requirement} Hollow Fist 9. Damage +1, TC +2. Your hits are treated as an adamantium weapon.

The \textbf{fourth time} you take this Ability, \textbf{Requirement} Empty Fist 12. Damage +2, TC +1. Your hits are treated as a +2 weapon.

The \textbf{fifth time} you take this Ability, \textbf{Requirement} Empty Fist 15. Damage +1, TC +2. Your hits are treated as a +3 weapon.

The \textbf{sixth time} you take this Ability, \textbf{Requirement} Hollow Fist 18. Damage +2, TC +1. Your hits are treated as a +4 weapon.

The bonuses acquired are cumulative apart to consider the magic level of the hit.

\subsection*{Powerful Punch} \index{Powerful Punch} \label{pugnopotente}

\textbf{Requirement}: Empty Fist List 3

\textbf{Saving Throws}: +1 Fortitude, +2 Will

Consume 2 Actions. Make a single attack roll with a -5 penalty. If you hit, in addition to the damage, the opponent, which must be a maximum of two bounties higher than yours, must make a Fortitude save with DC equal to Weapon Proficiency + Empty Fist List x2 or be pushed 10 feet in a direction of your choice. If she fails the saving throw, she also takes one critical damage.

\subsection*{This is my dagger} \index{This is my dagger} \label{questoeilmiopugnale}

\textbf{Saving Throws}: +2 Fortitude, +1 Reflex

\textbf{Requirement}: Weapon Proficiency 1

When you do critical damage with your dagger, you add your Weapon Proficiency to the damage. The Ability is usable once per opponent and automatically applies to the first critical damage taken.

\subsection*{This is my weapon!} \index{This is my weapon!} \label{questaelamiaarma}

\textbf{Requirement}: Weapon Proficiency 1

\textbf{Saving Throws}: +2 Fortitude, +1 Will

Each time you hit the same opponent, starting from the second round, you do an additional damage (Max +1 per combat round, even if you hit him several times in the round) up to a maximum +5. The first time you don't hit the opponent in the round the bonus returns to +0. The bonus can only be held on one opponent at a time.

The \textbf{second time} you take this Ability you can miss the opponent with one hit and not lose the benefits.

\subsection*{Magic roots} \index{Magic roots} \label{radicimagiche}

\textbf{Requirement}: Magical Proficiency 1

\textbf{Saving Throws}: +2 Will, +1 Fortitude

As long as you are affected by your spell, using an Action your weapon gains +1 to hit and damage and is considered a magical weapon. For each of your spells that affect you in the round, beyond the first (not from magic items) the bonus increases by + 1 / + 1 to a maximum of + 3 / + 3.

\begin{center}
\includegraphics[width = 0.55 \linewidth]{immagini/streghegoya.png}

\textit{Witches' Sabbath (Goya, 1798)}
\end{center}

\subsection*{Retaliation} \index{Retaliation} \label{rappresaglia}

\textbf{Saving Throws}: +2 Will, +1 Fortitude

Seeing your friends hurt fills you with anger.
When a teammate (or yourself) drops below half their hit points, you gain +1 to attack and saving throws. The maximum duration of the effect is 1 minute (6 rounds) per day and must be consecutive. The player chooses whether or not to activate the Ability.
You can take this Ability up to 3 times, each time the bonus on attack and saving throws increase by 1.

\subsection*{Resistance of stone} \index{Resistance of stone} \label{resistenzadellapietra}

Over time you have trained your Constitution to withstand shocks, transformations, poisons and anything else that wanted to change your body. The first time you take this Ability, you gain a +2 bonus on the Fortitude saving throw. The bonus is cumulative, +2 the first time, +1 the second, +1 the third.

The fourth time you take this Ability, you can decide to automatically succeed in a Fortitude saving throw once per day as a Reaction. It must be declared and does not result in a saving throw.

\subsection*{Detect the Magic} \index{Detect the Magic} \label{rilevareilmagico} \index{Eyes of Magic}

\textbf{Requirement}: Magical Proficiency 1

\textbf{Saving Throws}: +1 Will, +1 Fortitude

If you can see it, you also know if it is magical. It costs one Action to activate magical sight and lasts for one round.

The \textbf{second time}, Magical Proficiency 1 requirement, that you take the ability to activate magical sight costs the Reaction.

\subsection*{Lightning reflexes} \index{Lightning reflexes} \label{riflessifulminei}

Over time you have trained your reflexes to dodge and predict any obstacle. The first time you take this Ability, you get a +2 bonus on Reflex saving throws. The bonus is cumulative, +2 the first time, +1 the second, +1 the third.

The fourth time you take this Ability, you can decide to automatically succeed in a Reflex saving throw once per day as a Reaction. It must be declared and does not result in a saving throw.

\subsection*{Wise} \index{Wise} \label{sapiente}

\textbf{Requirement}: Magical Proficiency 4

\textbf{Saving Throws}: +2 Will

Your interest and connection with magic is unmatched. You can know one more spell (while respecting the constraints of the maximum selectable level).

The Ability can be taken again as long as the Magical Proficiency value is at least 4 times the number of times this Ability is taken. So with a minimum value of Magical Proficiency 4, 8, 12 ..


\begin{center}
	\includegraphics[width = 0.55 \linewidth]{immagini/turning-undead-six.png}
\end{center}

\subsection*{Turning out the undead} \index{Turning out the undead} \label{scacciarenonmorti}

\textbf{Saving Throws}: +2 Will, +1 Fortitude

By focusing on the power of your Patron you channel the positive energy and drive away or destroy the undead.

Make a check by rolling 1d6 (Golden Rules apply) for each time you take this Ability and then add the Traits in common with your Patron. Compare the result with this table to understand the effects obtained.

\end{multicols}

%\addvspace{5cm}

\textbf{Table: Turning Undead} \index{Turning Off Undead Table}

\medskip

\begin{tabularx}{0.9\textwidth}{Xllllllllll}
\textbf{Undead} & \multicolumn{6}{c} \textbf{\textbf{Evidence Value}} \\
\textbf{or Degree of Challenge} & 2-4 & 5-8 & 9-11 & 12-15 & 16-18 & 19-22 & 23-25 & 26-29 & 30-32 & 33-36 \\
\toprule
\textbf{Skeleton} & - & T & T & T & D & D & D & D * & D * & D * \\
\textbf{Zombie} & - & - & T & T & T & D & D & D & D * & D * \\
\textbf{Ghoul} & - & - & - & T & T & T & D & D & D & D * \\
\textbf{Ghast}& - & -   & -   & -   & T   & T   & T   & D   & D   & D   \\
\textbf{Wraith}& - & -   &-    & -   & -   & T   & T   & T   & D   & D   \\
\textbf{Mummia}& - & -   &-    &-    & -   & -   & T   & T   & T   & D   \\
\textbf{Spettro}&-  & -   &-    &-    &-    & -   & -   & T   & T   & T   \\
\textbf{Vampiro}&-  & -   & -   &-    &-    &-    & -   & -   & T   & T   \\
\textbf{Fantasma}&-  & -   &-    & -   &-    &-    &-    & -   & -   & T   \\
\textbf{Lich} & - & - & - & - & - & - & - & - & - & T \\
\end{tabularx}

\bigskip

\textit{Legend}:

T: 1d4 creatures run away for 1 minute as far as possible. If attacked they are no longer frightened and respond to the attack.

D: 1d4 creatures are destroyed

D *: 2d4 creatures are destroyed

\begin{multicols}{2}

The creatures make a DC Will save equal to the Turn undead check to resist the effect.

Ability is usable a number of times per day equal to Wisdom but an undead can only be affected once per day by your effect.

\subsection*{Dodging Traps} \index{Dodging Traps} \label{schivaretrappole}

\textbf{Saving Throws}: +2 Reflex, +1 Fortitude

The \textbf{first time} you take the Ability, Dexterity 2 requirement, you gain a + 1d6 bonus on your saving throw to avoid the effect of traps.

The \textbf{second time} you take the Ability, weapon proficiency requirement 5, even if the trap doesn't grant a saving throw your natural propensity to avoid damage grants you a Reflex saving throw to halve the damage.

It is also possible to use this Ability, use a Reaction, to avoid Sneak Attack (Reflex save higher than opponent's Hit Roll).

%La \textbf{terza volta} che prendi l'Abilità requisiti Competenza Armi 9, il Tiro Salvezza se riuscito ti permette di evitare qualsiasi effetto della trappola, se fisicamente possibile.

\subsection*{Uncanny Dodge} \index{Uncanny Dodge} \label{schivataprodigioso}

\textbf{Saving Throws}: +2 Reflexes

As a Reaction to an attack Action you can add +2 to your Defense. You can apply the bonus after your opponent's attack roll even after you know he has hit you.

You can use the Ability up to 3 times per day.

The \textbf{second time} you take the Ability, Weapon Proficiency 4 and Dexterity 3 requirement, an opponent does not take the flanking hit bonus against you.

The \textbf{third time} you take the Ability, Weapon Proficiency 8 and Dexterity 3 requirement, an opponent does not take the hit bonus for attacking you from behind.

\subsection*{Second Skin} \index{Second Skin} \label{secondapelle} \hypertarget{secondapelle}{}

\textbf{Requirement}: Weapon Proficiency 1

\textbf{Saving Throws}: +2 Fortitude

The constant use of armor allows you to wear them without major penalties.

Armor's penalty on proficiency checks decreases by 1.

The \textbf{second time} you take this Ability, Weapon Proficiency 6, the penalty on proficiency checks decreases by an additional 1. The penalty on movement penalties decreases by 1 meter.
You can sleep in medium armor without being fatigued.

The \textbf{third time} you take this Ability, Weapon Proficiency 11, the penalty on proficiency checks decreases by an additional 1. The penalty on movement penalties decreases by an additional 1 meter.
You can sleep in heavy armor without being fatigued.

\subsection*{Hound} \index{Hound} \label{segugio}

\textbf{Requirement}: Intelligence 1, Wisdom 1, Weapon Proficiency 1

\textbf{Saving Throws}: +1 Reflex, +1 Will

You have a natural talent for following people

With two Actions and you focus on a target that you can see and as long as you see it you stay focused. All your Actions involving that target have a +1 bonus. Staying focused costs 1 Action per round.

The \textbf{second} time you take this Ability, Weapon Proficiency 6, the bonus increases to +2.

The \textbf{third} time you take this Ability, Weapon Proficiency 12, the bonus increases to +3.

The bonus can be used on attack rolls, saving throws caused by the opponent, proficiency checks ... but not on damage.

\subsection*{Without Trace} \index{Without Trace} \label{senzatraccia}

\textbf{Requirement}: Safe step

\textbf{Saving Throws}: +2 Will, +1 Reflex

The ability not to leave footprints in the chosen environment. Each time you take this Ability, you can choose a different environment (see Sure Step Ability) whose Ability you have taken. The difficulty of the Follow Traces to Pursue check increased by 10.

\subsection*{Specialist} \index{Specialist} \label{specialista}

\textbf{Saving Throws}: +2 Fortitude

Choose a spell you know, the Magic Points spent to cast this spell decrease by 2.

The Ability can be taken multiple times on different spells and even on the same one as long as the Magic Points to cast the spell are not less than or equal to 50 \% of the original cost.

\subsection*{Stay down!} \index{Stay down!} \label{staigiu}

\textbf{Saving Throws}: +2 Fortitude, +1 Will

When you make two critical rolls on an opponent the strength of your blow is such that it knocks them prone. The opponent must make a Fortitude saving throw (DC equal to the last attack roll) or fall prone. The ability works on creatures that are equal to or smaller than the character's size.

The \textbf{second time} you take the ability, you can affect creatures of a larger size as well.

The \textbf{third time} you take the ability, you can affect creatures two sizes higher as well.

\subsection*{Storm of Fury} \index{Storm of Fury} \label{tempestadifuria}

\textbf{Requirement}: Empty Fist List 2, Dexterity 1, Strength 1

\textbf{Saving Throws}: +2 Reflexes, +1 Will

When using this Ability you can declare that you are using Storm of Fury as your only action (3 Actions).

Make a single attack roll at -5 and if you hit you deal critical damage equal to Weapon Proficiency / 4 times.

\begin{center}
\includegraphics[width = 0.5 \linewidth]{immagini/kenilguerriero.png}
\end{center}

\subsection*{Precise Shot} \index{Precise Shot} \label{tiropreciso}

\textbf{Requirement}: Dexterity 3, Weapon Proficiency 1

\textbf{Saving Throws}: +2 Reflex

You gain +1 to hit and +1 to damage and attack rolls, with ranged weapons or bows, with targets within 30 feet.

\subsection*{Quick Shot} \index{Quick Shot} \label{tirorapido}

\textbf{Requirement}: Dexterity 3, Precise Shot, Weapon Proficiency 2

\textbf{Saving Throws}: +2 Reflexes

When using a bow, crossbow or throwing a weapon the penalties for multiple attack are lower.

Any bullet thrown beyond the first takes a -4 cumulative attack roll (and not a -5). The first shot has a normal attack roll, the second has a -4, the third a -8 ...

\subsection*{Touch and flight} \index{Touch and flight} \label{toccataefuga}

\textbf{Saving Throws}: +2 Reflex

Your attacks have a base penalty of -5 and you can perform an Action of 1 more move. It is not possible to perform more than one bonus move action in this manner. It costs an immediate action.

\subsection*{Merciful touch} \index{Merciful touch} \label{toccopietoso}

\textbf{Requirement}: Good Patron, Laying on of hands, Traits in common with the Patron 3

\textbf{Saving Throws}: +2 Will, +1 Fortitude

Your touch not only soothes wounds but also aches and pains. Whenever you use the Lay on Hands Ability you can add this Ability as an Immediate Action as well.

Using the Lay on of Hands you can, by giving up a number of indicated healing d6s, remove the following afflictions.

\textbf{2d6} Common Trait score 3: Fatigued (1 rank)

\textbf{3d6} Common Trait score 6: Stunned - Confused - Nauseated - Frightened

\textbf{4d6} Common Trait score 9: Sick - Poisoned - Paralyzed - With reduced Max Hit Points (reduce by 2d6) - Exhausted (reduce by one rank)

The \textbf{second time} you take the Ability, Common Traits 11, you can also, by completely forgoing the heal dice, regenerate limbs - remove the condition of Blinded or Deaf.

\subsection*{One with magic} \index{One with magic} \label{tuttunoconlamagia}

\textbf{Requirement}: Adept of Magic

\textbf{Saving Throws}: +1 on two saving throws of your choice

Each time you take the One with Magic Ability you must determine which List of Magic it links to.
Your Characteristic linked to that Spell List (eg Constitution for the Earth List) has a +1 value to determine the maximum spell level you can cast.

\subsection*{One arm, one weapon} \index{One arm, one weapon} \label{unbracciounarma}

\textbf{Requirement}: Weapon Proficiency, 2

\textbf{Saving Throw}: +1 Fortitude, +1 Will

Choose a Weapon List. The Strength damage applied by the weapons on that list increases by 2.

The Ability can be taken multiple times and the Weapon List score must be 4 times higher the times this Ability is taken if on the same Weapon List.

If you take \textbf{4 times} this Ability on the same Weapon List the damage bonuses are reduced to +4 but you roll the damage twice and choose the best result.

\subsection*{A shot for a dead person} \index{A gesture for a dead person} \label{ungestounmorto}

\textbf{Requirement}: Magical Proficiency 1, Adept of Magic 1

\textbf{Saving Throws}: +1 Will, +1 Reflex

This Ability can be taken a number of times equal to the Adept of Magic score.
Whenever it is taken, it grants +1 to attack rolls for spells that require an attack roll.

\subsection*{One body, one mind, one spirit} \index{One body, one mind, one spirit} \label{USCMS} \index{USCMS}

\textbf{Saving Throws}: +1 of your choice

Assign a point to Weapon Proficiency or Magical Proficiency. This Ability can be taken a maximum of 2 times.

In the manual you will also find this Ability under the name of \textbf{USCMS}.

\subsection*{Vampire} \index{Vampire} \label{abilitavampiro}

\textbf{Requirement}: Smell of blood (Benefits)

\textbf{Saving Throws}: +2 Fortitude, +1 Will

Your bloodlust becomes cure. The bloodlust bonus can be increased up to +5.

If the bonus increases from +3 to +4 or +5 you can, by gulping down the opponent's blood, you can heal for 1d6 using a 2 Actions

\subsection*{Iron Will} \index{Iron Will} \label{volontaferrea}

Over time you have trained your will to resist any weakness and fear. The first time you take this Ability, you get a +2 bonus on Will saving throws. The bonus is cumulative, +2 the first time, +1 the second, +1 the third.

The fourth time you take this Ability, you can decide to automatically succeed in a Will saving throw once per day as a Reaction.

\subsection*{My skin} \index{My skin} \label{My skin}

\textbf{Saving Throws}: +3 Fortitude

You have an almost symbiotic relationship with your armor.

The \textbf{first time} you take this Ability, the Defense that the armor you carry grants you increases by 1.

The \textbf{second time} you take this Ability,  Weapon Proficiency 4, you choose a type of armor (light, medium or heavy), all armor that falls within the chosen type grants +1 to Defense.

\textbf{Each subsequent time} you take this Ability, Weapon Proficiency +4 compared to the previous time, choose a new or existing type of armor, your Defense with that kind of armor increases by 1.

\end{multicols}

%\begin{center}

%\medskip

%\includegraphics[width=0.3\linewidth]{immagini/Philip_Burne-Jones_-_The_Vampire4.png}
%\textit{Philip Burne Jones - The Vampire}
%\end{center}

\pagebreak

\begin{multicols}{2}


\subsection{Grouping Enable by Style}

To facilitate the transition from those coming from other RPGs with classes, the Abilitys for the more canonical classes are broken down here.

These are clearly just indications, in OBSS the character can be built as he likes best and how the story he lives is instructing him.

These are suggestions to facilitate the construction of a character for those who are new to the Old Bell School System, of course it is possible to "fish" from all the groupings presented!

\end{multicols}

\bigskip

\begin{multicols}{3}


\begin{flushleft}
	\subsubsection*{Warrior}

Weapon Focused

Whirlwind attack

Powerful blows

Kill shot

Fighting blindly

Dance of the Blades

Dancing Scourge

I said CADI!

Iaijutsu

My head is harder

Truly evil person

Continue

This is my weapon!

Second skin

Stay down!

One arm, one weapon

\subsubsection*{Barbarian}

Ferocity

Forged in fury

Fury

My death your death

One arm, one weapon

\subsubsection*{Thief}

Sneak Shot

Weak Shot

Paralyzing Blow

Enrage

Arrow called, arrow delivered

Juggler

Improvise

Opportunist

This is my dagger

Dodging traps

Prodigious dodge

Touch and run

\subsubsection*{Paladin}

Laying on of hands

Channel energy from a distance

Channel energy

Channel concentrated energy

Retaliation

Pitiful touch

\subsubsection*{Bardo}

Instill Courage

Instill fear

\subsubsection*{Ranger}

Archer on horseback

Combat with two weapons

Defending Mount

Double portion

Clinical Eye

Hawk eye

Sure step

The bigger they are, the more noise they make when they fall

Hound

Without a trace

Accurate shot

Quick shot

\subsubsection*{Druid}

Animalia

Elemental form

Brew potions

The Patron is with me

\subsubsection*{Cleric}

Armor of the Devout

Loyal

The Patron is with me

Drive out the undead

Specialist

One with the magic

\subsubsection*{Wizard / Sorcerer}

Adept of Magic

Pet / Familiar

Magic Battery

Extended Battery

Create Magic Items

Create Higher Magical Objects

Create Wonderful Magical Items

Create Mythical Magical Items

Deciphering magical writings

The Patron is with me

Effective magic

Detect the Magic

Knowledgeable

Specialist

One with the magic

A gesture a dead man

\subsubsection*{Monaco}

Wings of the Phoenix

Armor of the Enchanted Mountain

Psychic Energy

Psychic Strike

Psychic Ray

Silver crane

Human mountain

Fast pace

Iron fist

Storm of Fury

\subsubsection*{Gish (Warrior / Mage)}

Armor of the Devout

Warrior of Magic

Combat Spellcaster

Prudent Charmer

The shield is my friend

Magic roots

One with the magic

\subsubsection*{General}

Expert

Fake Death

Immunity to poisons

Improved initiative

Perceptive

Resistance of the stone

Lightning reflexes

Vampire

Iron Will
\end{flushleft}

\end{multicols}

\pagebreak

\begin{multicols}{2}

\subsection*{Example characters for Style}

These are \textbf{examples} of how to build characters according to classical archetypes. \index{Class archetypes} \index{Example characters}. Do not try to faithfully reconstruct the classes of other systems, it is not the purpose of OBSS. OBSS was born without classes and as versatile as it is, it does not allow you to perfectly recreate other classes. If you just can't do without a class, change your game.

All archetypes shown here do NOT have any Ability due to race and one (1) skill at the first level is also missing, it is left to the player's choice.

\subsubsection*{Basic Warrior} \index{Basic Warrior} \index{Warrior}

At each level, he always takes Weapon Proficiency as an active Proficiency

Profession: Soldier or at your choice

Advantages / Disadvantages: none

Weapons List: Sword and Shield or your choice

Skills suggested: depending on your style

\textbf{Variants}:

\textit{Myrmidon}: Weapon List: Spear, Ability: see Warrior (see above)

\textit{Archer}: Weapon List: Bows, Skills: Hawkeye, Clinical Eye, Precise Shot, Quick Shot

\subsubsection*{Ranger} \index{Ranger}

List of Weapons: Bows or Sword and Shield, or other choice

Suggested skills: see above

\medskip

\begin{tabular}{lllll}
LV 	& PF 		& WP 		& CM 		& Skills \\
	& (+ COS) &&& \\
\hline
1 	& 1d4 + 3 	& 1 		& 0 		& Sure-footed \\
2 	& 1d4 + 3 	& 2 		& 0 		& - \\
3 	& 1d4 + 3 	& 3 		& 0 		& Untracked \\
4 	& 1d4 + 3 	& 4 		& 0 		& - \\
5 	& 1d4 + 3 	& 5 		& 0 		& Adept of Magic \\
6 	& 1d4 	& 5 		& 1 		& - \\
7 	& 1d4 + 3 	& 6 		& 1 		& Expert \\
8 	& 1d4 	& 6 		& 2 		& - \\
9 	& 1d4 + 3 	& 7 		& 2 		& Adept of Magic \\
10 	& 1d4 	& 7 		& 3 		& - \\
11 	& 1d4 + 3 	& 8 		& 3 		& at your choice \\
12 	& 1d4 	& 8 		& 4 		& \\
13 	& 1d4 + 3 	& 9 		& 4 		& of your choice \\
14 	& 1d4 	& 9 		& 5 		& - \\
15 	& 1d4 + 3 	& 10 		& 5 		& of your choice \\
16 	& 1d4 	& 10 		& 6 		& - \\
17 	& 1d4 + 3 	& 11 		& 6 		& your choice \\
18 	& 1d4 	& 11 		& 7 		& - \\
19 	& 1d4 + 3 	& 12 		& 7 		& at your choice \\
20 	& 1d4 + 3 	& 13 		& 7 		& - \\
\end{tabular}

\subsubsection*{Paladin} \index{Paladin}

Weapon List: Sword and Shield or at your choice

Suggested Skills: See below

Advantages and Disadvantages: select what you prefer (disease immunity, bonus on saving throws ... to be paid with disadvantages such as poverty, code of ethics ...)

\medskip

\begin{tabular}{lllll}
LV 	& PF 		& WP 		& CM 		& Skills \\
& (+ COS) &&& \\
\hline
1 	& 1d4 + 3 	& 1 		& 0 		& Retaliation \\
2 	& 1d4 + 3 	& 2 		& 0 		& - \\
3 	& 1d4 + 3 	& 3 		& 0 		& of your choice \\
4 	& 1d4 + 3 	& 4 		& 0 		& - \\
5 	& 1d4 + 3 	& 5 		& 0 		& Adept of Magic \\
6 	& 1d4 	& 5 		& 1 		& - \\
7 	& 1d4 + 3 	& 6 		& 1 		& Laying on of hands \\
8 	& 1d4 	& 6 		& 2 		& - \\
9 	& 1d4 + 3 	& 7 		& 2 		& Channel energy \\
10 	& 1d4 	& 7 		& 3 		& - \\
11 	& 1d4 + 3 	& 8 		& 3 		& Adept of Magic \\
12 	& 1d4 	& 8 		& 4 		& \\
13 	& 1d4 + 3 	& 9 		& 4 		& Merciful Touch \\
14 	& 1d4 	& 9 		& 5 		& - \\
15 	& 1d4 + 3 	& 10 		& 5 		& of your choice \\
16 	& 1d4 	& 10 		& 6 		& - \\
17 	& 1d4 + 3 	& 11 		& 6 		& your choice \\
18 	& 1d4 	& 11 		& 7 		& - \\
19 	& 1d4 + 3 	& 12 		& 7 		& your choice \\
20 	& 1d4 + 3 	& 13 		& 7 		& - \\

\end{tabular}

\subsubsection*{Wizard} \index{Wizard} \index{Sorcerer}

At each level, he always takes magical proficiency as an active proficiency.

He can take Weapon Proficiency 5 times and still cast 9-level spells.

Weapon List: Simple weapons

Suggested Skills: See below

\medskip

\begin{tabular}{lllll}
LV 	& PF 		& WP 		& CM 		& Skills \\
& (+ COS) &&& \\
	\hline
	1 	& 1d4 	& 0 & 1 	& Adept of Magic \\
	2 	& 1d4 	& 0 & 2 	& - \\
	3 	& 1d4 	& 0 & 3 	& Detect the Magic \\
	4 	& 1d4 	& 0 & 4 	& - \\
	5 	& 1d4 + 3 	& 1 & 4 	& Effective Magic \\
	6 	& 1d4 	& 1 & 5 	& - \\
	7 	& 1d4 	& 1 & 6 	& Create Magic Items \\
	8 	& 1d4 	& 1 & 7 	& - \\
	9 	& 1d4 + 3 	& 2 & 7 	& Adept of Magic \\
	10 	& 1d4 	& 2 & 8 	& - \\
	11 	& 1d4 	& 2 & 9 	& Scholar \\
	12 	& 1d4 	& 2 & 10 	& - \\
	13 	& 1d4 + 3 	& 3 & 10 	& Adept of Magic \\
	14 	& 1d4 	& 3 & 11 	& - \\
	15 	& 1d4 	& 3 & 12 	& at your choice \\
	16 	& 1d4 	& 3 & 13 	& - \\
	17 	& 1d4 + 3 	& 4 & 13 	& at your choice \\
	18 	& 1d4 	& 4 & 14 	& - \\
	19 	& 1d4 	& 4 & 15 	& at your choice \\
	20 	& 1d4 + 3 	& 5 & 15 	& - \\
\end{tabular}

\subsubsection*{Cleric} \index{Cleric}

At each level, he always takes magical proficiency as an active proficiency.

He can take Weapon Proficiency 5 times and still cast 9-level spells.

List of weapons: as suggested by the Patron

Suggested Skills: See below

\medskip


\begin{tabular}{lllll}
	LV 	& PF 		& CA 		& CM 		& Skills \\
	& (+ COS) &&& \\
	\hline
	1 	& 1d4 	& 0 & 1 	& Adept of Magic \\
	2 	& 1d4 	& 0 & 2 	& - \\
	3 	& 1d4 	& 0 & 3 	& Devotee's Armor \\
	4 	& 1d4 	& 0 & 4 	& - \\
	5 	& 1d4 + 3 	& 1 & 4 	& The Patron is with me \\
	6 	& 1d4 	& 1 & 5 	& - \\
	7 	& 1d4 	& 1 & 6 	& Turn the undead out \\
	8 	& 1d4 	& 1 & 7 	& - \\
	9 	& 1d4 + 3 	& 2 & 7 	& Adept of Magic \\
	10 	& 1d4 	& 2 & 8 	& - \\
	11 	& 1d4 	& 2 & 9 	& Laying on of hands \\
	12 	& 1d4 	& 2 & 10 	& - \\
	13 	& 1d4 + 3 	& 3 & 10 	& Adept of Magic \\
	14 	& 1d4 	& 3 & 11 	& - \\
	15 	& 1d4 	& 3 & 12 	& Faithful \\
	16 	& 1d4 	& 3 & 13 	& - \\
	17 	& 1d4 + 3 	& 4 & 13 	& at your choice \\
	18 	& 1d4 	& 4 & 14 	& - \\
	19 	& 1d4 	& 4 & 15 	& at your choice \\
	20 	& 1d4 + 3 	& 5 & 15 	& - \\
\end{tabular}

\subsubsection*{Druid} \index{Druid}

At each level, he always takes magical proficiency as an active proficiency.

He can take Weapon Proficiency 5 times and still cast 9-level spells.

Suggested Skills: See below

Weapon List: Simple weapons

\medskip


\begin{tabular}{lllll}
	LV 	& PF 		& CA 		& CM 		& Skills \\
	&(+COS)&&&\\
	\hline
	1 	&1d4 	&0&1 	&Get the Magic delta\\
	2 	&1d4 	&0&2 	&-\\
	3 	&1d4 	&0&3 	& Animals\\
	4 	&1d4 	&0&4 	&-\\
	5 	&1d4+3 	&1&4 	& Animals\\
	6 	&1d4 	&1&5 	&-\\
	7 	&1d4 	&1&6 	&a scelta\\
	8 	&1d4 	&1&7 	&-\\
	9 	&1d4+3 	&2&7 	&Get the Magic delta\\
	10 	&1d4 	&2&8 	&-\\
	11 	&1d4 	&2&9 	&Animals\\
	12 	&1d4 	&2&10 	&-\\
	13 	&1d4+3 	&3&10 	& Get the Magic delta\\
	14 	&1d4 	&3&11 	&-\\
	15 	& 1d4 	& 3 & 12 	& Elemental form \\
	16 	& 1d4 	& 3 & 13 	& - \\
	17 	& 1d4 + 3 	& 4 & 13 	& Animalia \\
	18 	& 1d4 	& 4 & 14 	& - \\
	19 	& 1d4 	& 4 & 15 	& at your choice \\
	20 	& 1d4 + 3 	& 5 & 15 	& - \\
\end{tabular}

\subsubsection*{Warrior-Cleric} \index{Warrior-Cleric} \index{Warpriest}

Profession: Soldier, Acolyte, Medicine Man or by choice

Advantages / Disadvantages: none

Weapons List: as suggested by the Patron

Suggested Skills: See Warrior, Cleric, and Paladin

\medskip

\begin{tabular}{lllll}
	LV 	& PF 		& CA 		& CM 		& Skills \\
	& (+ COS) &&& \\
	\hline
	1 	& 1d4 + 3 	& 1 		& 0 		& Devotee's Armor \\
	2 	& 1d4 	& 1 		& 1 		& - \\
	3 	& 1d4 + 3 	& 2 		& 1 		& Adept of Magic \\
	4 	& 1d4 	& 2 		& 2 		& - \\
	5 	& 1d4 + 3 	& 3 		& 3 		& USCMS (+1 CM) \\
	6 	& 1d4 	& 3 		& 4 		& - \\
	7 	& 1d4 + 3 	& 4 		& 5 		& USCMS (+1 CM) \\
	8 	& 1d4 	& 4 		& 6 		& - \\
	9 	& 1d4 + 3 	& 5 		& 6 		& Channel energy \\
	10 	& 1d4 	& 5 		& 7 		& - \\
	11 	& 1d4 + 3 	& 6 		& 7 		& Adept of Magic \\
	12 	& 1d4 	& 6 		& 8 		& \\
	13 	& 1d4 + 3 	& 7 		& 8 		& Merciful Touch \\
	14 	& 1d4 	& 7 		& 9 		& \\
	15 	& 1d4 + 3 	& 8 		& 9 		& of your choice \\
	16 	& 1d4 	& 8 		& 10 		& \\
	17 	& 1d4 + 3 	& 9 		& 10 		& of your choice \\
	18 	& 1d4 	& 9 		& 11 		& \\
	19 	& 1d4 + 3 	& 10 		& 11 		& of your choice \\
	20 	& 1d4 + 3 	& 11 		& 11 		& \\

\end{tabular}

\subsubsection*{Warrior-Mage} \index{Warrior-Mage} \index{Gish}

Profession: Magician's Apprentice or Acolyte or by choice

Advantages / Disadvantages: none

Weapon List: Sword or of your choice

Suggested Skills: See below

\medskip

\begin{tabular}{lllll}
	LV 	& PF 		& CA 		& CM 		& Skills \\
	& (+ COS) &&& \\
	\hline
	1 	& 1d4 + 3 	& 1 		& 0 		& Devotee's Armor \\
	2 	& 1d4 	& 1 		& 1 		& - \\
	3 	& 1d4 + 3 	& 2 		& 1 		& Adept of Magic \\
	4 	& 1d4 	& 2 		& 2 		& - \\
	5 	& 1d4 + 3 	& 3 		& 3 		& USCMS (+1 CM) \\
	6 	& 1d4 	& 3 		& 4 		& - \\
	7 	& 1d4 + 3 	& 4 		& 5 		& USCMS (+1 CM) \\
	8 	& 1d4 	& 4 		& 6 		& - \\
	9 	& 1d4 + 3 	& 5 		& 6 		& Weapon Focus \\
	10 	& 1d4 	& 5 		& 7 		& - \\
	11 	& 1d4 + 3 	& 6 		& 7 		& Adept of Magic \\
	12 	& 1d4 	& 6 		& 8 		& \\
	13 	& 1d4 + 3 	& 7 		& 8 		& Specialist \\
	14 	& 1d4 	& 7 		& 9 		& \\
	15 	& 1d4 + 3 	& 8 		& 9 		& of your choice \\
	16 	& 1d4 	& 8 		& 10 		& \\
	17 	& 1d4 + 3 	& 9 		& 10 		& of your choice \\
	18 	& 1d4 	& 9 		& 11 		& \\
	19 	& 1d4 + 3 	& 10 		& 11 		& of your choice \\
	20 	& 1d4 + 3 	& 11 		& 11 		& \\

\end{tabular}


\end{multicols}

\pagebreak

\section{Familiar} \index{Familiar} \label{famiglio}
\medskip

\begin{changemargin}{0cm}{0.5cm} \begin{enfasi}{
Mr. Wing's nephew: Listen mister, there are three rules to follow, However.

Rand: Oh yeah? And what would they be?

Mr. Wing's nephew: Keep him away from the light, he hates the light strong, especially that of the sun. Would die. And keep it away from the water, do not let it get wet. But most importantly, the rule what he must never forget is that even if he cries, even if he does scene and pleading she will never, ever have to feed him after the midnight. Understood? (Gremlins, Film, 1984)
} \end{enfasi} \end{changemargin} \medskip

\begin{multicols}{2}

\lettrine[lines = 2, lhang = 0.33, loversize = 0.25, findent = 1.5em]{T}{he} familiars are animals chosen by the character, through the Familiar Skill, to help him in adventures and for company . A familiar has a special bond with its master.

A familiar is a normal animal that retains the normal animal's appearance, Hit Dice, Weapon Proficiency, saving throw bonus, Ability, but is treated as a magical creature for the purposes of determining any effects that depend on its type.

Only a normal, unmodified animal can become a familiar.

A familiar grants special abilities to its master, these special abilities only apply when the master and familiar are within 100 yards of each other.

A special 4-hour ritual is required in the animal's native environment to make it a familiar.

If a familiar is dismissed, lost, or dies, it can be replaced a week later with a special ritual that costs 2 points of the character's temporary Constitution. It takes 8 hours to complete the ritual.

\medskip


\textbf{Table: Familiar Types} \index{Familiar Types Table}

\medskip

\begin{tabularx}{0.45\textwidth}{lX}
\textbf{Familiar} & \textbf{Skill acquired by master} \\
\toprule
Owl & +2 to the checks of Arcana \\
Crow & +2 Intimidate checks \\
Dobi 		& +2 to Will save \\
Weasel & +1 to Intelligence checks \\
Hawk & +2 on Sight Awareness \\
Cat & +2 to Move Silently checks \\
Owl & +2 on Hearing Awareness \\
Otter 	& +2 Swim checks \\
Lizard & +2 Survival checks \\
Bat & +2 Acrobatics checks \\
Rat & +1 Fortitude save \\
Hedgehog & +1 to Will save \\
Toad & +2 to save on poison \\
Monkey & +2 to Fairy Hands rehearsal, Escape Artist \\
\textit{Topi} & become the companion of the Topi !!! \\
Fox & +1 Reflex saving throw \\
\end{tabularx}

\bigskip

Use the base stats of a creature of the familiar's species by making the following changes.

\medskip

\textbf{Hit Dice}: For effects related to the number of Hit Dice, use the master character's Magical Proficiency score or the familiar's normal HD total, whichever is higher.

\medskip

\textbf{Attacks}: Use the master's Weapon Proficiency if higher. Use the familiar's Dexterity or Strength modifier, whichever is higher, to calculate the familiar's attack bonus with Natural Attacks. The damage is equal to that of a normal creature of the familiar's species. The familiar rolls its own initiative.

\medskip

\textbf{Defense}: The familiar has a Defense equal to that of the standard animal plus bonus due to the master's Magical Proficiency. See the Familiar Skills table.

\medskip

\textbf{Saving Throw}: For each saving throw, use the familiar's (Fortitude +2, Reflex +2, Will +0) or master's saving throw bonuses, whichever is best. The familiar applies its ability values as a bonus on saving throws and takes none of the bonuses its master may have.

\medskip

\begin{center}
	\includegraphics[width = 0.65 \linewidth]{immagini/donnadrago2.png}

	\textit{Henry Justice Ford}
\end{center}

\medskip

\textbf{Description of the Pet's Abilities}

All familiars have special abilities (or attribute them to their masters) depending on the master's Magical Proficiency score. The special abilities listed in the table are cumulative.


\end{multicols}


\textbf{Table: Familiar Skills and Bonuses} \index{Familiar Skills Table}

\medskip

\begin{tabularx}{0.95\textwidth}{cccX}
\textbf{Master CM} & \textbf{Defense Bonus} & \textbf{Intelligence Bonus} & \textbf{Special} \\
\toprule
1-2 & +1 & 0 & Alert, Share Spells, \\
& &				  	& Empathic Bond \\
3-4 & +1 & +1 & Transmit Touch Spells \\
5-6 & +2 & +1 & Talking to Animals of His Kind \\
7-8 & +3 & +1 & Talk to the Master \\
9-10 & +3 & +2 & - \\
11-12 & +4 & +2 & See through Familiar \\
13-14 & +5 & +2 & - \\
15-16 & +5 & +3 & - \\
17-18 & +6 & +3 & - \\
19-20 & +7 & +3 & - \\
\end{tabularx}

\begin{multicols}{2}

\bigskip

\textbf{Master's Magical Proficiency}: The number shown here is the familiar's master's Magical Proficiency value, broken down into bands.

\textbf{Defense Bonus}: The number shown here is in addition to the familiar's Defense.

\textbf{Intelligence Bonus}: This value is added to the familiar's Intelligence score.

\textbf{Special}: The special abilities acquired by the familiar (and / or master).

\textbf{Alert}: When the familiar is within arm's reach of the master, he gains +1 on Awareness checks

\textbf{Share Spells}: At his discretion, the master may cast any Spells that affect "himself" on his familiar (such as a touch spell), in place of himself.

The master can cast spells on his familiar even though they normally have no effect on creatures of the familiar type (magical creatures).

\textbf{Empathic Bond}: Master has an empathic bond with his familiar up to a distance of 1 km. The master cannot see through the familiar's eyes, but can telepathically communicate with it. Due to the limited nature of the bond, only generic emotions can be communicated (fear, tranquility, joy ...).

\textbf{Cast Touch Spells}: The familiar can cast touch spells for him. If the master and familiar are within 30 feet when the master casts a touch spell, he can designate his familiar as "the deliverer of the spell" (if touched).

The familiar can cast the spell just like the master. The familiar's attack must be in the same round, but later as an action of casting the Spell.

\textbf{Talk to the Master}: The familiar and the master can communicate verbally, as if using a common language. Other creatures or animals are unable to understand their conversation except by using magical aids. Capacity works within 50m and must feel.

\textbf{Talking with Animals of His Kind}: The familiar is able to communicate with animals of its general species: bats with bats, mice with rodents, cats with felines, hawks and owls and crows with birds, snakes and lizards with reptiles, toads with amphibians, monkeys with other primates, weasels with stoats and mustelids ... Communication is limited by the Intelligence of the creatures with which the familiar communicates.

\textbf{See Through Familiar}: Master can see through familiar. Activating this Skill costs 1 Immediate Action. The familiar must be within 30 meters.

\medskip

As special, intelligent and unique as a familiar is, it remains an animal and as such it cannot use magic items or scrolls, it can use a potion if it has the ability to drink it. An intelligent familiar could perform simple and straightforward tasks.

\end{multicols}

\medskip

\begin{center}
	\includegraphics[width = 0.35 \linewidth]{immagini/familiar.png}

	\textit{Henry Justice Ford, a very good familiar ...}
\end{center}



\pagebreak

\section{Other Special Abilities}


\begin{multicols}{2}

\lettrine[lines = 2, lhang = 0.33, loversize = 0.25, findent = 1.5em]{T}{hese} Abilities are not selectable by the player, but can be innate in creatures.

\subsection{Ethereal} \index{Ethereal} \label{etero}

A creature that has become Ethereal is located in the Ethereal Plane which is superimposed on the Material Plane.

An ethereal creature is Invisible, without substance, and capable of moving in any direction, even up and down, but only at half speed. An ethereal creature can move through solid objects, including other living creatures. An ethereal creature can see and hear what happens on the Material Plane, but everything appears gray and ephemeral. The sight and hearing of an ethereal creature on the Material Plane are limited to a distance of 30 feet.

Spells if not properly formulated and modified do not act on ethereal creatures. An ethereal creature has Damage Resistance towards Light or Void, and ignores all other forms of Energy.

An ethereal creature cannot attack a material creature, and Spells cast while in ethereal condition can only affect ethereal elements. Some creatures or material objects have special attacks or effects that also work on the Ethereal Plane. An ethereal creature treats all other ethereal creatures as if they were all material.

\subsection{Damage Resistance} \index{Damage Resistance} \label{resistenzaaldanno}

Certain creatures or wards grant the ability to resist a type of damage.

Being resistant to damage automatically means halving the damage taken before applying any other protection or saving throws.

Resistance to damage can also take on values. When it says Damage Resistance: Lightning Bolt, the subject automatically halves its electricity damage, if it says Damage Resistance: Lightning Bolt 10, it means it reduces the electricity damage by 10 points before applying the saving throw or other bonuses.

A creature with Fire Resistance halves (reduces) all damage it takes from flames, magical or otherwise.

There may be abilities or spells that ignore this Resistance. More equal resistances do not add up, due to the fact that two objects give me fire resistance I do not reduce the damage to a quarter, only one is applied.
If an ability ignores the damage resistance it will pass the resistance even if I have two sources of resistance.

\subsection{Damage Reduction - DR} \index{Damage Reduction} \label{resistenzaaldannodr}

Certain creatures or abilities grant the supernatural ability to resist damage from certain types of weapons or up to a certain amount (per attack).

Usually it assumes the value of XX / ZZ or how much damage (XX) is ignored if you are not attacked with (ZZ). Ignoring the damage also means that attack-related effects don't work, such as poisons on the weapon.

A single DR is applicable in case there are more than one at the same time, the choice must be made at the beginning of the fight and remains the same until the fight is over.

\begin{center}
\includegraphics[width = 0.8 \linewidth]{immagini/morteachille.png}

\textit{Paris shot Achilles with an arrow - Pieter Paul Rubens - Date 1630-1632}
\end{center}


Certain weapons, particularly magical, can ignore DR \index{Ignore DR}

\medskip

\textbf{Table: Magic Weapons Equivalence} \index{Magic Weapons Equivalence Table}


\begin{tabular}{lll}
\textbf{DR to overcome} & \textbf{Enchant} & \textbf{Attack} \\
& \textbf{weapon} & \textbf{Natural} \\
\toprule
Enchanting +1 & +1 & Level 3 \\
Enchantment +2 & +2 & Level 6 \\
Cold Iron & +2 & Level 9 \\
Silver & +2 & Level 9 \\
Adamantium & +3 & Level 12 \\
\end{tabular}

\medskip

\textbf{Projectiles (arrows, darts, stones) fired by magical weapons are NOT considered magical.} \index{Magic arrows}

\subsection{Magic Resistance} \index{Magic Resistance} \label{resistenzaallamagia}

It is indicated as Resistance to Magic (or with the abbreviation RM) followed by a number. The creature isn't affected by spells of that level or less. A spell is considered to be one level higher for each Magic Critical roll obtained in the Magic Check.

Although not affected by direct effects it is still affected by indirect effects, for example it can fall into the pit created by a disintegration spell.

Magic Resistance cannot be lowered even by the creature possessing it.

\subsection{Immunity to damage} \index{Immunity to damage} \label{immunitaaldanno}

It is extremely rare but there are creatures or magical effects that make you immune to a form of damage, be it physical (weapon damage ..) or magical (various forms of energy).

A creature immune to a form of damage takes no damage from that attack. A creature that has the ability to have its own irresistible damage, meaning it can't be reduced by resistance, will only partially penetrate the creature's immunity, making it only resistant to that damage.

A creature that says "Void, Poison Damage Immunity; weapons +2" means that they are immune to Void, Poison damage, and that you need a weapon with a +3 or greater magic bonus to wound them, or a character you attack with weapons. natural and is level 12 or higher or has taken the Empty Fist Weapon List at least 6 times (see p. \pageref{equivalenzaarmimagiche}).

\subsection{Vulnerability to Damage} \index{Vulnerability to Damage} \label{vulnerabilitadanno}

Certain creatures or spells make some effects more effective by causing more damage to the vulnerable subject.

Being vulnerable to a specific type of damage automatically means doubling the damage taken before applying any other protection or saving throws.

A creature with a Fire Vulnerability doubles all damage taken, then makes the saving throw indicated by the spell or effect if possible.

\subsection{Fear} \index{Fear} \label{paura}

Spells, magical items, and certain creatures can affect characters with fear. A creature with fear cannot suppress its aura if it is innate unless otherwise described. The difficulty with which to make the Will saving throw is always marked. A creature immune to fear cannot be frightened whether the source is natural or magical.

\textbf{Scared} \label{spaventato} \index{Scared}

A frightened creature has -1d6 on attack rolls, saving throws, and proficiency checks as long as the source of its fear is visible. A frightened creature cannot voluntarily approach the source of its fear.

\subsection{Paralyzed} \index{Paralyzed} \label{paralizzato}

There are several methods of Paralyzing a creature, both magical and natural. While natural ones often have systems to free themselves later, magical systems can expect to free themselves from paralysis or not, perhaps only after a certain amount of time.

A paralyzed character is stuck in place and unable to move or act

He has effective Strength and Dexterity scores of -5, is \hyperlink{morente}{Indifeso}, and can only perform mental actions. A winged creature in flight, the moment it is paralyzed, can no longer flap its wings and falls. A paralyzed swimmer can no longer swim and could drown.

\end{multicols}

\medskip

\begin{center}
	\includegraphics[width = 0.4 \linewidth]{immagini/the-scream.png}

	\textit{The Scream (original title: Skrik) \\ Edvard Munch - Date 1893–1910}
\end{center}

\pagebreak

\section{The Magic} \index{Magic} \label{lamagia}

\begin{changemargin}{0.3cm}{0.3cm} \begin{enfasi}{
The magic is not in the pendulum, but in the one who uses it. (NCIS - Unity anti-crime)


You will not let the one who practices magics. (Book of the Exodus) (Always depending on one's traits ...)


A sorcerer is never late, Frodo Baggins. Nor in advance. He arrives precisely when it intends to do so. (Gandalf, The Lord of the Rings - The Fellowship of the Ring. JRR Tolkien)} \end{enfasi} \end{changemargin} \medskip

\begin{multicols}{2}

\lettrine[lines = 2, lhang = 0.33, loversize = 0.25, findent = 1.5em]{M}{agic} permeates the game worlds and its most common form is that of a spell. This chapter provides the rules for casting spells.

\medskip

\subsection{What is a spell?} \index{What is a spell} \index{Spell definition}

A spell is a manifestation of power. Each spell is the result of power and knowledge, the caster is a superior medium who channels the power of the Patrons. In casting a spell, a character composes gestures, words and objects that simply connect it to the source. A spellcaster could be an empty but tough shell like the most learned of the wise, only those who have the skills can have the power.

Spells can manifest protective weapons or barriers, can inflict damage or heal life energies. Countless thousands of spells have been created throughout the history of the multiverse, many of which have been forgotten. Some may still be hidden among the pages of dusty spellbooks within ancient ruins or segregated in the minds of dead deities. Or they may someday be reinvented by a character who has amassed enough power and ability to do so.

\subsection{How to do magic! (In summary)} \index{How to do magic! (In summary)}

In short, your character must have taken a Magic List via the Adept of Magic Skill and invested a few points in Magical Proficiency.

The Adept of Magic Skill allows you to access a List of Magic and therefore be able to learn the spells that belong to it, without that you would only learn the spells of the Universal list. Then, with the passing of the Levels you will invest other Skill points always in Adept of Magic and this will allow you to be able to cast higher level spells in already known schools or to give access to spells of new schools.

Magical Proficiency allows you to have more Magic Points, therefore more spells, to make your spells more difficult to resist and together with the Adept of Magic Skill to access higher level spells. Your Tome of Spells is where you keep your spells, it's a precious item, be very careful.

However don't worry, everything you need to know is written in this chapter!

\medskip

\begin{center}

\includegraphics[width = 0.7 \linewidth]{immagini/Hex32.png}

\textit{The Witchcraft Art of Jacques de Gheyn II}

\end{center}

\subsection{The characteristics of spells} \index{The characteristics of spells} \label{caratteristicheincantesimi}

The description of each spell begins with a block of information that includes the spell's level, lists of magic, casting time, range, components, and spell duration. The rest of the description informs us of the spell's effect.

When a character casts any spell, the following basic rules are used regardless of the spell's effect.

\subsubsection{Time to Cast} \index{Time to Cast Spells} \label{magietempodilancio} \index{Spells, Actions to cast}

Most spells can be cast with two Actions. Some spells require an immediate action, a reaction action, or much longer to cast.

\textbf{Immediate Action}

A spell cast with an immediate action is particularly swift. You can use an immediate action during your round to cast the spell that is immediate, as long as you have not already taken an immediate action during your round. During the same round, you cannot cast another spell, unless it is a level 0 spell (called Tricks).

\textbf{Reactions}

Some spells can be cast as reactions. These spells take a split second to create and can be cast in response to an event. If a spell can be cast as a reaction, the spell description tells you exactly when you can do it. You must have a Reaction Action available and not have used it already.

\textbf{Longest Casting Time}

Certain spells take longer to cast: minutes or even hours. When you cast a spell with a casting time longer than two Actions, you must spend one action each subsequent round to maintain the spell, which is to maintain concentration (see "Concentration" below). If you want to try to cast the spell again, you'll have to start over. Initiative can be cast on the last round of casting when the spell manifests.

\subsubsection{The Lists of Magic} \hypertarget{lescuoledimagia}{} \index{The Lists of Magic} \label{magielistadimagia} \index{Spells, Lists of Magic}

Magical traditions throughout Yeru have formalized spells over the millennia into lists homogeneous by type and effect. There are therefore lists concerning Fire, or other elements, illusions, healing energies ...

The lists presented here are only those codified and taught in wizarding universities. Ancient legends tell of further lists created, edited and disseminated in restricted circles or sects. One such secret list is that of the Devoted gnomes of Shayalia, a purely natural list that mixes the traditional list of Animals and Plants with some spells from the element lists.
Other darker lists are demonic or Aboleth ones, some others are related to belonging to a group of Devotees. Some of the more nefarious lists go so far as to corrupt the soul of the characters by imposing traits as well. These lists will normally be closed to the character, but it is not certain that as the Magical Proficiency increases, he will not be the one to create new lists of spells.

Lists of Magic help describe spells; they do not have their own rules, although some rules may refer to these lists.

\begin{itemize}
\item
\textit{Abjuration} relates to spells of a protective nature, although it also contains some for aggressive use. These spells create magical barriers, negate harmful effects, harm violators, or banish creatures to other planes of existence.

\item
\textit{Water} are the spells that act on the element of water and cold and to a small extent the cures

\item
\textit{Air} relates to spells that manipulate and use air but also electricity.

\item
\textit{Enchantment} relates to spells that act on the mind of others, influencing or controlling their behavior. These spells can cause enemies to see you as a friend, force creatures to perform certain actions, or even control another creature as if it were a puppet.

\item
\textit{Animals and Plants} these are spells that act on animals and plants, natural or magical.

\item
\textit{Healing} deals with spells that allow you to recover physical and mental energies and cancel weaknesses and poisons.

\item
\textit{Divination} concerns spells that reveal information in the form of long-forgotten secrets, visions of the future, the location of hidden objects, the truth behind illusions or images of distant people and places.

\item
\textit{Summon} deals with spells that transport objects and creatures from one place to another. Some spells summon creatures or objects to the caster's side, while others allow the caster to teleport from one location to another. Some summons create objects or effects out of thin air.

\textit{Fire}
The most dangerous spells are in here, with everything it takes to burn and incinerate.

\textit{Illusion} deals with spells that deceive the senses and minds of others. They make people see things that don't exist, they don't point out things that exist, they hear bogus noises or remember things that never happened. Some illusions create ghostly images that anyone can see, but more insidious illusions implant an image directly into a creature's mind.

\item
\textit{Invocation} concerns spells that manipulate magical energy to produce a desired effect.

\item
\textit{Necromancy} concerns spells that manipulate the energies of life and death. These spells can grant an additional supply of life force, drain life energy from another creature, create undead, or even bring the dead back to life (if granted).

Creating undead through the use of necromancy spells such as animating dead is not a good deed, and only evil spellcasters frequently use this spell.

In OBSS only a Patron has enough power to bring the dead back to life.

\item
\textit{Earth} Spells that act and move the earth

\begin{center}
	\includegraphics[width = 0.65 \linewidth]{immagini/Leonids-1833.png}

	\textit{The most famous depiction of the famous 1833 Leonids \hyperlink{sciamedimeteore}{Meteor Storm} (Meteor Swarm!)}
\end{center}

\item
\textit{Transmutation} relates to spells that change the properties of a creature, object, or environment.

\item
\textit{Universal} Some spells are cornerstones of magic themselves and as such accessible to all spellcasters. To access the spells contained in this List of Magic it is not necessary to take Adept of Magic, to access the spells of higher level it is sufficient to have taken Adept of Magic in another School.

\end{itemize}

\begin{changemargin}{0.3cm}{0.3cm} \begin{Storytellere}{In OBSS, players potentially have access to all spells present in the known lists with therefore the possibility of having one extremely varied and powerful set of spells. To do in so that the choices are different between different players and between adventures insist on the characterizing aspect of a list over a other. The Lists of Magic taken give a different thickness and role between them. Also make sure that new spells are not easy to be found like a treasure.
} \end{Storytellere}
\end{changemargin}


\subsubsection{Range} \index{Range} \label{magiegittata} \index{Spells, Range}

The target of a spell must be within the spell's range. For a spell such as magic missile, the target is a creature. For a spell like fireball, the target is the point in space from which the fireball explodes. Most spells have a range expressed in meters. Some spells can only target a creature (including you) with whom you are in physical contact. Other spells, such as the shield spell, only affect you - these spells have a personal range. A spell that has an "ally" area of effect can also be cast on itself.

Spells that create cones or lines of effect that originate from you also have a personal range, indicating that you are the point of origin of the spell's effect (see "Areas of Effect" later in this chapter).

\subsubsection{Casting Spells in Armor} \index{Casting Spells in Armor} \label{magielanciareincantesimiinarmatura} \index{Spells, in Armor}

Given the mental concentration and precise gestures required, the armor distracts and unbalances the flows. The Magic Check in casting the spell is modified as noted in the section of \hyperlink{armatureemagie}{armature}.

\subsubsection{Optional - Armor Spells} \index{Optional - Armor Spells}

Armor blocks magical flows and does not allow for proper channeling.
This option allows all spells cast by the caster to become with touch range, that is, downloadable only through the caster's hand. No Magic Checks are required.

\subsubsection{Duration} \index{Spell Duration} \label{magiedurata} \index{Spells, Duration}

The duration of a spell is the length of time it persists. The duration can be expressed in rounds, minutes, hours or even years. Some spells specify that their effects last until the spell is dispelled or destroyed. A spell can be interrupted by your caster as an immediate action.

\begin{itemize}

\item
\textit{Instantaneous}

Many spells are instantaneous. The spell injures, heals, creates, or alters a creature or object so that it cannot be dispelled, since its magic exists only for an instant.

\item

\textit{Concentration}

Some spells require you to maintain concentration to keep their magic active. If you lose focus, the spell will end. If a spell is to be maintained through concentration, this is indicated under Duration, the spell specifies how long you can maintain concentration there. You can end the concentration at any time by using a reaction.

Normal activities, such as moving and attacking, do not interfere with concentration. Maintaining concentration costs 1 Action per round.
\end{itemize}

\subsubsection{Components} \index{Components} \label{magiecomponenti} \index{Spells, Components}

The components of a spell are the physical requirements you must meet to cast it. The description of each spell indicates whether it requires verbal (V), somatic (S), or material (M) components. If you are unable to provide one or more of the components of the spell, you will not be able to cast it.

Most spells require you to chant mystical words. Words, rhythm, cadence and resonance set the threads of magic in motion.

\textbf{Somatic (S)} \index{Spell Components}

The gestures of casting a spell can include forced gesticulation or an intricate series of gestures. If a spell requires a somatic component, the caster must be free to use at least one hand to perform these gestures.

\medskip

\textbf{Material (M)}

Casting certain spells requires special items, specified in parentheses under the item components. A character can use a bag of components or an enchantment focus in place of the components specified by a spell. If the component has an indicated cost, the character must obtain that specific component before the spell can be cast.

If a spell indicates that the material component is consumed by the spell, the caster must provide this component for each casting of the spell.
A caster must have a free hand to access these components, but it can be the same hand used to perform the somatic components.

\subsubsection{Being incapacitated or killed} \index{Being incapacitated or killed by Magic} \label{magieessereucciso} \index{Spells, Incapacitated}

If you drop below zero Hit Points, you lose half of your remaining Magic Points, with a minimum of 10 points lost. Any spells you are focusing on are broken.

\subsubsection{Targets} \index{Targets} \label{magiebersagli} \index{Spells, Targets}

A normal spell requires you to choose one or more targets that are affected by its spell. The spell description tells you whether the spell targets creatures, objects, or a point of origin to generate an area of effect (described below). Unless the spell has a perceptible effect, a creature may never know that it has been the target of a spell. An effect like crackling lightning is obvious, but a more subtle effect, like trying to read a creature's thoughts, usually goes unnoticed unless the spell says otherwise.

Casting a spell is an action that does not go unnoticed. A Hide check on difficulty 13 or cast the spell as if you were Distracted allows you to conceal the cast, if it doesn't happen right in front of the target.

\textbf{Clear Path To Target} \index{Spells, see target}

To target a creature or object, you must see it and have clear trajectory towards it, and therefore it cannot be behind full cover. If you place an area of effect in a spot that you cannot see and an obstruction, such as a wall, is between you and that spot, the point of origin is created on your closest side of the obstruction (a Fire behind a closed door explodes on contact with the door on your side and does not manifest beyond the door.) \index{Magic see target}

\textbf{Target yourself} \index{Target yourself} \index{Spells, target yourself}

If a spell targets a creature of your choice or an ally, you can choose yourself as well, unless the creature is hostile or it is specified that it cannot be you. If you are in the area of effect of a spell cast by you, you too will be affected.

%\begin{center}
%	\includegraphics[width=0.6\linewidth]{immagini/tarothanged.png}
%
%	\textit{Tarocchi - L'Impiccato}
%\end{center}

\subsubsection{Areas of Effect} \index{Area of Spell Effect} \label{magieareedieffetto} \index{Spells, Area of Effect}

Spells like burning hands and cone of cold cover an area, allowing it to hit multiple creatures at a time.

A spell description specifies its area of effect, which usually falls into one of five shapes: cylinder, cone, cube, line, or sphere. Each area of effect has a point of origin, a place from which the spell's energy erupts. The rules for each shape specify how to place its point of origin. Usually the point of origin is a point in space, but some spells have an area whose origin is a creature or object.

\begin{center}
	\includegraphics[width = 0.6 \linewidth]{immagini/3dforme.png}

	\textit{Cone, Sphere, Cylinder, Cube}
\end{center}

\begin{itemize}
\item \textit{\textbf{Cylinder}}: The point of origin of a cylinder is the center of a circle of specific radius, as indicated in the spell description. The circle must be on the floor or level with the spell's effect. The energy in a cylinder expands in straight lines from the point of origin to the perimeter of the circle, forming the base of the cylinder. The effect of the spell then starts from the bottom up or from the top to the bottom, up to a distance equal to the height of the cylinder. The point of origin of the cylinder is included in its area of effect.

\item

\textit{\textbf{Cone}}: A cone extends in a direction of your choice from its point of origin. The diameter of a cone at a given point along its length is equal to the distance of that point from the point of origin. The area of effect of a cone specifies its maximum length. The point of origin of the cone is not included in its area of effect, unless you decide otherwise.

\item
\textit{\textbf{Cube}}: select the origin point of a corner of the cube. The dimensions of the cube are expressed as the length of each of its edges. The cube's point of origin is not included in its area of effect, unless you decide otherwise.

\item
\textit{\textbf{Line}}: A line extends from its point of origin in a straight path along its entire length and covers an area defined by its width. The point of origin of the line is not included in its area of effect, unless you decide otherwise.

\item
\textit{\textbf{Sphere}}: you select the origin point of a sphere, and the sphere will extend from that point until it meets an insurmountable obstacle or its size expressed in the radius. The size of the sphere is referred to as the radius in meters that extends from that point.
The point of origin of the sphere is included in its area of effect.

A fireball that is generated in a 9x9m room will take a good part of it, as in a 6x6m room it will fill it all. In a 3x3m room, if it has a way to get out of a door or a window, it will continue its explosion up to a radius of 6 meters. A fireball in a 3x3m corridor will saturate it 6m back and forth from the point of origin.

\end{itemize}

\subsubsection{Spell Rarity} \index{Spell Rarity} \label{magieraritaincantesimi} \index{Spells, Spell Rarity}

Some Spells indicate Rarity or how likely it is to find this spell or how much it can be known. The rarity depends not only on the level of the spell itself, obviously the most powerful spells are also the rarest, but also on how widespread and known they are normally in the list. The Storyteller will use this scale to evaluate what can be found most easily: Common (60 \%) - Uncommon (24 \%) - Rare (10 \%) - Very Rare (5 \% ) - Legendary (1 \%), (1 / 13,14 / 15,16,17,18).

\subsubsection{Combining Magic Effects} \index{Combining Magic Effects} \label{magiecombinareeffettimagici} \index{Spells, Combining Effects}

The effects of different spells stack together until their duration overlaps. However, the effects of the same spell or that give the same bonus cast multiple times on the same target do not combine. Instead, the most powerful spell among those cast (for example the one that cost the most Magic Points) will be applied as long as the durations overlap.

In the case of instantaneous effects, the effects act individually if they act in the same initiative segment. Ex. If I am struck by a lightning bolt at initiative segment 4 and then by another bolt at initiative segment 8 I will make two separate saving throws with relative damage management.

\subsection{Basic Rules} \index{Basic Rules for Magic} \label{magieregoledibase} \index{Spells, Basic Rules}

\begin{itemize}

\item
The caster on casting his first spell chooses whether to use Intelligence, the Characteristic linked to the first Magic List taken as a modifier to the Magical Proficiency check, or if he is a Devotee, he can choose the Characteristic indicated by the Patron. \index{Spells , Modifier to the CM test}

Once the choice has been made, it is no longer possible to change it. This modifier is called \textit{ability modifier for spells}.
\item
When the character assigns the first point of Magical Proficiency, he learns two spells + half the ability modifier (rounded down)
\item
Whenever a spellcaster gains a point in magical proficiency, he can learn two new spells that he has available in his tome that are within the maximum castable level.
\item
Whenever the caster gains a point in magical proficiency, he can forgo learning a level 1 or higher spell to learn two (level 0) tricks he knows.
\item
Each time the caster acquires a point in Magical Proficiency, it is possible to forget a learned spell and replace it with another available in the Tome, as long as it is within the maximum castable level.
\item
The number of spells you can cast per day depends on the caster's ability. See \textbf{Magic Points and Magical Proficiency Table}.
\item
If the spell is cast with a Magic Check requested by the player and at least one Magic Critical Success is achieved (twice 6 on the roll of the dice), the cost of the spell is not subtracted from the Magic Points. Any extra Magic Points from the cost of the spell are paid regularly.
\item
A follower gains + 1d6 on Magic Checks in the Patron's favorite schools. He can use the Patron's favorite energy in your spells.
\item
A Devotee adds + 1d6 to the Magic Check in the Patron's preferred schools and can ignore a die rolled in the Magic Check. He must use the Patron's favorite energy in his spells.
\end{itemize}

\subsubsection{Magic Check} \index{Magic Check} \index{Magic Critical Success} \index{Magic Critical Failure} \label{magieprovadimagia} \index{Spells, Magic Check}

Casting a spell is not always enough, many times it is necessary for this to work and well, indeed it acts beyond its normal expectations. The caster can decide to call more energy in the casting of the spell, or perform a \textit{\textbf{Magic Check}} and roll \textbf{3d6 + 1d6 for every 4 points of Magical Proficiency} + any bonuses .

If in the set of dice rolled there are at least two 1s or a 1 and two bad things have happened, this case is called \textbf{magical critical failure}, the spell does not manifest and the Magic Points are spent.
To check how many magical critical failures have been made, first check how many pairs of 1s are present, then check if there is another 1 left to associate with a 1 or two 2s.

Checked for the absence of critical failure if in the roll of the dice there are at least two 6s you will have obtained a \textbf{magic critical success}, as for the Golden Rules you will continue to roll a die for every 6 done or that you go to do. Count the 6 you make, every two is a magical critical hit! Any 1 rolled as a result of the critical success do not count towards the critical failure. \textbf{For each critical roll, the saving throw DC increases by 2}.


\begin{changemargin}{0.3cm}{0.3cm} \begin{Storytellere}
Grant a + 1d6 or ignore a 2 in the Magic Check, if required, when the player expertly and transported the casting of the spell. If he says "\textit{I throw a fireball}" he will gain no advantage but if with transport he declaims "\textit{For all hell fires may Nedraf drag you to hell with his sacred flames. Burn unworthy. Fireball !} "then a bonus of + 1d6 or ignoring a 2 is more than a must.
\end{Storytellere} \end{changemargin}

If there is no "particular" result (neither failure nor critical success) the spell, unless otherwise requested, manifests itself normally.

By paying an additional magic point cost the caster \textbf{can ignore 1 or 2 rolled}. \index{Spells, Ignore 1 or 2 in Magic Check} Paying an additional time the spell cost can ignore a die rolled, paying double it can ignore two, paying quadruple it can ignore 3 ... This payment is done before making the Magic Check. In the event of a Magic Critical, the Magic Points of the additional cost of the spell are paid the same.

A spellcaster can also voluntarily fail the Magic Check.


\subsubsection{Access to Magic Lists} \hypertarget{scuoleelivelli}{} \index{Spell Level by Skill} \label{magieaccessoallelistedimagia} \index{Spells, Accessing Magic Lists}

The number of times the Ability \hyperlink{scuoladimagia}{Adept of Magic} on the same Spell List is taken establishes, along with the Magical Proficiency and the Ability score linked to the List, the maximum spell level that can be cast from that Spell List, as indicated in the table.

Eg. With Magical Proficiency 11 if I have taken 0 times the Adept of Magic Ability I can only cast level 0 spells (Tricks), if I have taken the Skill once (on the Fire list for example) I can reach the fourth level, if I took it twice up to the sixth level, if I took it three times I can still go up to the sixth level.

\medskip

\textbf{Table: Adept of Magic - Magical Proficiency - Spell Level} \index{Table of Adept of Magic - Magical Proficiency - Spell Level}

\medskip

{\small
	\begin{tabular}{l|llll|ll}

CM & \multicolumn{4}{c}{Adept of Magic} & Valor & Level Inc. \\
		& 0 				& 1 				& 2 			& 3 			& Char. & Max \\
		\hline
1 		& \checkmark 	& \checkmark 	& \checkmark & \checkmark & 0 & 0 \\
1 		& 				& \checkmark 	& \checkmark & \checkmark & 0 & 1 \\
2-4 	& 				& \checkmark 	& \checkmark & \checkmark & 1 & 2 \\
5-6 	& 				& \checkmark 	& \checkmark & \checkmark & 2 & 3 \\
7-8 	& 				 & \checkmark	& \checkmark & \checkmark & 2 &4\\
9-10 	& 				 & 				& \checkmark & \checkmark & 2 &5\\
11-12 	& 				 & 				& \checkmark & \checkmark & 3 &6\\
13-14 	& 				 & 				& 			 & \checkmark & 3 &7\\
15+		& 				 &			    & 	         & \checkmark & 4 &8\\
15+ 	& 				 & 				& 			 & \checkmark & 5 &9\\
\end{tabular}}

\medskip

\textit{Read the table like this}: go to the row of your Magical Proficiency and cross the column with the times you have taken Adept of Magic on that List of Magic, you have access to the spell levels indicated since the checks start upward.

The maximum spell level you find must be verified with the minimum required ability score from the spell list. In the example above, I must have Strength 3 or higher to cast 6-level spells, otherwise it can be cast up to the level that the Ability score grants me. Here the list of \hyperlink{elencoscuole}{Liste di Magia} with their spells and their \hyperlink{lescuoledimagia}{descrizione}


\subsubsection{Optional - Supreme Magic} \index{It's over 9000!} \index{Optional - Supreme Magic} \label{} \label{opzionalemagiasuprema}

One of these rules can be applied to handle a spell of the highest level made by supreme or highly skilled spellcasters.

- \textbf{Accessible Magic} \index{Accessible Magic}: if you want a more available spell with the possibility of taking more Lists to use ignored in \textit{Table: Adept of Magic - Magical Proficiency - Spell Level} the Adept column of rank 2 magic and transform the one at rank 3 into the new one at rank 2.

- \textbf{Supreme Magic} \index{Supreme Magic} \hypertarget{magiasuprema}{}: if you want high-level spellcasters to master magic in the most complete way, establish that every 6 points of Magical Proficiency the player can ignore a die rolled in the Trial of Magic.

- \textbf{Specialist} \index{Specialist} \hypertarget{magiaspecialista}{}: The caster can add a 1d6 and can ignore a die rolled in the Magic Check for every two times he has taken the Adept of Magic skill. The advantage is counted and applied per Spell List.

- \textbf{Magic Art} \index{Magic Art}: for each spell you want to cast beyond the number given by CM / 2 the caster must make a Magic Check and compare the result with 12 + 2 x Level Spell you want to cast. If the check is successful the spell is cast, if the check fails the spell is not cast and the difficulty (12) increases by 1 for the next spells. If the check succeeds by 10 or more the difficulty is lowered by 1. The rule of critical success or critical failure in the Magic Check does not apply, the Golden Rules remain valid.

%\medskip
%
%\begin{center}
%	\includegraphics[width=0.65\linewidth]{immagini/Arthur-Pyle_The_Enchanter_Merlin.png}
%
%	\textit{Merlin. Howard Pyle, The Story of King Arthur and His Knights (1903)}
%\end{center}


\subsubsection{Critical Failure in Magic Check} \index{Critical Failure in Magic Check} \label{magiefallimentocriticonellaprovadimagia} \index{Spells, Magic Check Failure}

If the Magic Check had a magical critical failure roll 3d6 and consult the following table. For each additional magical critical failure to the first subtract 1d6, until you roll a single 1d6 to check its effects.

\end{multicols}

\textbf{Table: Magic Critical Failure Effects} \index{Magic Check Critical Failure Effects Table}

\medskip
{\small
\begin{tabularx}{0.95\textwidth}{lX}
\hline
1 & Increase Fatigue by 2 degrees \\
2 & For 1 day you are no longer able to channel magical energies. You cannot cast spells except by making a critical spell success in the Magic Check \\
3 & You manifest a minor bodily change \\
4 & You are hit by a thundering column of Light and Void. Within a 10-foot radius around you, everyone must make a DC 15 Reflex save to halve or take 1d6 points of damage per spell level \\
5 & For 3 rounds you are under the influence of the Confusion spell \\
6 & You are paralyzed for 3 rounds \\
7 & Increase Fatigue by 1 degree \\
8 & Become Invisible and unable to speak for 6 rounds \\
9 & Only you are enveloped in an impenetrable curtain of magical darkness for 6 rounds \\
10 & You can't speak well, you are a stutterer. Each casting of spells forces you to pass a Magic Check. Duration 3 rounds \\
11 & The next spell you cast has effects minimized if possible \\
12 & Your heartbeat is like the beating of a drum, you can hear it within 50 meters \\
13 & All your body hair falls out, luckily it can grow back \\
14 & Emit a noisy and pestilent flatulence. A 1m x 50cm luminous sign above your head points and mocks you \\
15 & Any object you hold in your hand falls to the ground \\
16 & Earn 2d6 Magic Points \\
17 & An anvil falls, 3d6 damage Reflex save DC 15 for half, on a random creature, not including you, within six meters \\
18 & All creatures, except you, within 20 feet of you take 1d10 non-resistable damage \\
\end{tabularx}}

\begin{multicols}{2}

\bigskip

\subsubsection{Magic Points} \index{Magic Points} \label{magiepuntimagia} \index{Spells, Magic Points}

Depending on the Magical Proficiency score, the caster has a certain amount of Magic Points at his disposal.

\medskip

\begin{tabularx}{0.45\textwidth}{XX | XX}
\textbf{Comp. Magic} & \textbf{Magic Points} & \textbf{Comp. Magic} & \textbf{Magic Points} \\
\hline
1 & 	5 	& 11 & 61 \\
2&	8	&12&72\\
3&	11	&13&84\\
4&	14	&14&97\\
5&	17	&15&111\\
6&	21	&16&116\\
7&	26	&17&132\\
8&	34	&18&149\\
9&	42	&19&167\\
10&	51	&20&186\\
20+&prec.+ 19&&\\
\end{tabularx}

\medskip

\textbf{Spells have a Magic Point cost equal to the spell level +1.} \index{Spells, Magic Point Cost}

Whenever you cast a spell, you subtract the +1 level of the spell from the total Magic Points available for the day.
In case of Tricks these do not consume Magic Points but it is necessary to have at least 1 remaining Magic Point.

If you roll a Magic Critical Success in the Magic Check you do not subtract Magic Points.
You have a magic point bonus equal to your ability modifier for spells.

Magic Points are all recovered with 8 hours of rest. \index{Spells, Magic Bridge Recovery}

\begin{changemargin}{0.3cm}{0.3cm} \begin{Storytellere}
If you want a more difficult approach, for each degree of Fatigue you recover 20 \% fewer Magic Points per night of rest.
\end{Storytellere} \end{changemargin}


\subsubsection{When you have few Magic Points} \index{When you have few Magic Points} \hypertarget{quandosihannopochipuntimagia}{} \label{magiequandosihannopochipuntimagia} \index{Spells, Few Magic Points}

When the caster drops below 50 \% of the available Magic Points, any further casting of spells must be made with a Magic Check.

\subsubsection{Automatic Critical Success Magic} \index{Automatic Critical Success Magic} \index{Nova} \label{magienova} \index{Spells, Critical Success Automatic}

The caster may decide to additionally spend double the spell's normal Magic Points to automatically have a magical critical success.
The choice can be made several times and each time the cost of the spell doubles compared to the previous one. It is necessary to pass a Magic Check.

The casting time of a spell empowered in this manner increases by 1 Action.

Eg Fireball, I want him to do 2 Magic Critics, I pay 4 Magic Points to cast it, plus 8 for the first Magic Critical Success plus 16 for the Second Magic Critical Success, and possibly 32 for a third Magic Critical Success. In this case, all magic points used are always paid regardless of the critical success of the Magic Check.

You cannot spend more than half of your Magic Points to empower a spell.

\subsubsection{Cast the same spell several times} \index{Cast the same spell several times} \label{magielanciarepiuvolteincantesimo} \index{Spells, Cast the same spell several times}

Each time a spell already used in the day is cast, the cost in Magic Points increases by the same cost. Eg I throw the Fireball for the first time (4 Magic Points), the second time it will cost 8 Magic Points, the third 12, then 16 ...

This effect does not occur with the casting of Tricks.

\subsubsection{The Tome of Magic} \index{Tome of Magic} \index{The Tome of Magic} \label{magietomodellamagia} \index{Spells, Tome of Magic}

If the Patrons guarantee access to the source of magic it is only the application of ancient rites and formulas that allows this raw energy to be manifested in a form and expression that we call spell.

Each user of magic has one or more \textbf{Tome} of spells, don't just think of a big ancient Tome bound in leather, different cultures have developed over time the ability to inscribe the runes of spells in cards, Staff, plates stone, tattoos ... make your choice when creating the character.
This choice will not prevent you from copying spells from \textbf{Tomes} made differently (tobacco leaves, liquids of knowledge ...) it will always be easy for you (Arcana DC 12) to understand if you are facing a Tome of some kind .

The character with Magical Proficiency 1 will start with 2 + ability modifier for spells of spells of his choice from the List of Magic learned and within the maximum castable level, or from the Universal List, transcribed in his Tome and any other spells he wants to learn will have to find it and write it down in your book.

Each spell occupies a number of pages in the Tome equal to its level, with a minimum of one, copying a spell page takes 1 hour of work and 10 gp of precious inks.

A tome (book) of spells costs 10 gp per page.

A spellcaster can copy to his tome spells that belong to a list of magic known to him (or Universal), the maximum copiable level is one level higher than his maximum castable (see \hyperlink{scuoleelivelli}{Liste di Magia}).
If the spell is of a higher level or an unknown Spell List, the caster must make a Magic Check and gain a critical magic hit. If the character is a Devotee and the spell belongs to the List of Magic known to him and the Patron's favorite, then the Magic Check is performed only if the spell is three or more levels higher than the maximum castable.

If he does not get at least one magical critical success he cannot attempt to copy that spell until the next point of Magical Proficiency acquired. If he rolls a magical critical failure, bad things will happen to the tome and 1d4 spells will be canceled from the tome itself.

\medskip
\begin{center}
	\includegraphics[width = 0.7 \linewidth]{immagini/spellbook.png}
\end{center}

\medskip

The source of new spells can be another tome, staff, scroll… in short, anything that the previous spellcaster used to keep spells. A magical object (magic staff, ring, rod .. wand ..) is not suitable as a source from which to copy the spell it contains, it must be copied from the equivalent tome or scroll of another spellcaster.

During the adventures your spellcaster will be able to copy many and numerous spells on his Tome but he will not be able to learn them immediately. When the character acquires a new Magical Proficiency point he will be able to forget a learned spell to replace it with a spell present in his Tome that is from a known Spell List and learn the new spells.

\begin{changemargin}{0.3cm}{0.3cm} \begin{tcolorbox}[title = Choose Spells]
You read that right, spells are not learned by themselves, they are not chosen from a ready-made list. Each spell is a precious art that must be found and learned.

Characters will have to embark on perilous adventures, pay mercenaries, search for ancient tomes, and unravel the darkest and most forgotten secrets in order to learn new spells.

Each spell is like a magical item, a real treasure to seek and obtain!
\end{tcolorbox} \end{changemargin}

Through a difficult and expensive magical ritual, he will be able to replace 1 spell with another one present in the Tome for a single day. This rite, lasting 1 hour per level and costing 50 gp always per spell level, will allow for the next 24 hours to replace a learned spell with one present in a Tome and learnable given the level but not bound to the List of known Magic.

\begin{changemargin}{0.3cm}{0.3cm} \begin{Storytellere}
Spells become full-fledged magical items and rewards. Harness the characters' thirst for knowledge and power to build interesting adventures that revolve around ancient tomes and legendary lost spells.
\end{Storytellere} \end{changemargin}

\subsubsection{Studying Spells} \index{Studying Spells} \label{magiestudiareincantesimi} \index{Spells, Studying}

The character who wants to cast spells must review the ancient formulas on his Tome every day. This is pretty quick, taking only 2 minutes for Magical Proficiency.

If the caster has not revised the spells upon awakening or before casting them, he must make a Magic Check for each spell until he has revised.

\subsubsection{attack roll with spells} \index{attack roll with spells} \label{magietiropercolpireconlemagie} \index{spells, attack roll}

When the spell tells you to make an attack roll if it is a "\textbf{ranged attack with spell}" you must make an attack roll against the opponent's Defense. This attack roll is made with 3d6 + \textbf{Weapon Proficiency} + Dexterity + Abilities and various modifiers.

If the Roll \textbf{to attack with spell is in melee} you will make an attack roll against the opponent's Defense with the Strength value. This attack roll is made with 3d6 + \textbf{Weapon Proficiency} + Strength and Ability and various modifier.

It is also possible that an attack roll with a touch spell is required, in which case the subject's Defense becomes 10 + Dexterity + magic items, ignoring armor and shield.

\medskip

When the spell is area-based it is not necessary to make an attack roll except for difficult and specified areas, that is, you aim in a well-defined area and want to avoid hitting someone with an area spell.

\subsubsection{Optional - Spell Attack Roll} \index{Optional - Spell Attack Roll}

If you want to make it easier for spellcasters to hit, I suggest two possible approaches:

- The attack roll is modified not by Strength or Dexterity but by the ability modifier for spells

- The attack roll is based not on Weapon Proficiency + Strength / Dexterity but on Magical Proficiency + ability modifier for spells


\subsubsection{Explosion of 6 in Magic} \index{Explosion of 6 in Magic} \label{magieesplosionedelsei}

Even in the Trial of Magic the 6s explode, the 6s rolled in the Trial of Magic are withdrawn, and withdrawn again in the case.

Keep track of how many critics (two 6s rolled) you make, it could be used to get "special" effects in the spell! Remember that if you hit a crit, the spell's Magic Points are not scaled and the DC of the saving throw increases by 2 per magic critical hit. Any "extra" Magic Points paid are always paid, regardless of the critical gained.

\subsubsection{Saving Throw - Resist the spell} \index{Saving Throw - Resist the spell} \index{Saving Throw Spells} \label{magietirosalvezza}

The saving throw as required by the spell has difficulty (DC) equal to 10 + spell level + ability modifier per spell + 1 x Skills taken in that Magic List +2 per critical success on the Magic Check.

When you cast a spell, such as Lightning Bolt, force a Reflex saving throw to try to avoid it.

If you had a magic critical hit on the Magic Check, you would have done 9d6 damage and your saving throw DC would have increased by 2.

In the spell's description it says that saving throw is required.

\begin{changemargin}{0.3cm}{0.3cm} \begin{tcolorbox}[title = Tups launches a Lightning Bolt!]
Tups who has Intelligence 4 and has 1 Adept of Magic on the Air list cast the Lightning bolt spell. The difficulty (DC) of the Reflex saving throw will be 10 + 3 (spell level) + 4 (ability modifier for spell, Intelligence) + 1 x 1 (Adept of Magic taken 1 time in the Air list) or 10 + 3 + 4 + 2x1 = 19 to halve the damage. If he had done a Magic Check and it had a magical critical success the DC would have become 21.
\end{tcolorbox} \end{changemargin}

If you have to resist a spell, the Storyteller won't tell you to make a saving throw on difficulty 18, he is the one who compares your roll to the difficulty, he can tell you that the check is complex, difficult or easy ...

If the saving throw succeeds or fails by more than 10 (\textbf{critical saving success} \index{critical spell success} or \textbf{critical saving failure} \index{critical spell failure}) the Storyteller may decide to apply disadvantages or advantages to the final result. \index{More than 10}.
It is also possible that the description of the spell shows what happens in the event of a critical success or failure of the saving throw.

For monsters or otherwise for a casting of spells given by innate magical abilities, if not specified, the DC of the saving throw is equal to 10 + 2 x spell level + Intelligence.



\subsubsection{Distracted - Problems casting the spell} \index{Distracted - Problems casting the spell} \index{Distracted} \label{magiedistratto}

If the caster is severely \textbf{Distracted}, hindered, disturbed, bleeding, is under attack while trying to cast a spell he must make a Magic Check.

If the check rolls a magical critical failure the spell fails but suffers no consequences, if the check rolls a magical critical failure this is not counted.
If the check is successful without a critical magic failure, the spell will manifest.

A failed spell due to distraction subtracts half the Magic Points of the spell's cost.

The player forced to take the Magic Check can be the one to request it and suffer any consequences, failure or critical magic success and pay the Magic Points in full. A spellcaster who casts spells while in combat has a -4 Defense.

\subsubsection{Concentration} \index{Struck while concentrated} \index{Concentration} \label{magieconcentrazione}

You lose focus on a spell if you cast another spell that requires concentration. You can't focus on two spells at a time.

Breaking concentration costs a Reaction.

If you get hit while focusing on a spell you must make a Magic Check and get at least 1 Magic Critical Success or lose focus. Again, you can pay the additional Magic Point cost to ignore 1 or 2.

While focused, you can only cast Tricks.

\subsubsection{Optional - Multiple Concentrations} \index{Optional - Multiple Concentrations}

Every 6 CM points you can keep your concentration on one additional spell. If concentration is interrupted, all spells held in concentration are lost.

\subsubsection{Storing the magic} \index{Storing the magic} \label{magieconservare}

The caster can cast the spell (usually 2 Actions) and hold it in his fist, without manifesting it. To do so, he must cast the spell after which he can withhold it for up to 1 round per spell ability modifier point +2 rounds for times he has taken Adept of Magic in the spell's membership list.

To hold the spell, the caster must remain Focused (cost 1 Action per round) and pay a quarter of the spell's Magic Point cost per round.
To cast the preserved spell, simply roll initiative and use 1 Action. No more spells other than Tricks can be cast as long as a spell is retained and in the round it is cast.

\subsubsection{Influenced by Multiple Spells} \index{Influenced by Multiple Spells} \label{magieinfluenzatodapiumagie}

When a character is affected by \textbf{two or more magic effects} that give the same type of bonus, malus or damage in the same initiative segment (protection against fire, defense bonus or save ..., multiple balls of acid ), only the one with the highest saving throw or bonus is counted

A character who takes 2 fireballs in the same Initiative segment will make the saving throw only for the one with the highest saving throw, regardless of which is the one with the most damage. If he catches a fireball at two different times of the same round, he will make two separate saving throws, taking the relative damage.

\subsubsection{Attempt multiple Spells in the same round} \index{Attempt multiple Spells in the same round} \index{Multiple Spells in the same round} \hypertarget{piumagieround}{} \label{piumagieround}

Normally it is not possible to cast multiple spells per round even if the sum of the Actions allows it. Some dark rites and esoteric practices allow with great risk to try to cast even more spells, as long as always in the 3 Actions per round. You must have at least 3 Magical Proficiency.

Once the spellcaster has cast the first spell normally, he must make a Magic Check. If he succeeds in a magic critical success (at least 2 times 6) then he succeeds in casting the second spell, if the Magic Check does not get a magic critical then it is considered as a magic critical failure, with the effects of chance.

\subsubsection{Alter Spells} \index{Alter Spells} \label{magiealteraremagie}

The caster can modify spells in several ways. These possibilities add versatility to the caster and it is advisable for the player to always have them present in the most critical situations.

\begin{itemize}
\item
\textbf{Effective Magic} \index{Effective Magic}: The caster reducing his maximum hit points can increase the success of the Magic Check. Every 4 hit points and decrease the maximum hit points by the same amount you can ignore a die roll in the Magic Check, these are declared and lost in advance of the Magic Check. Effective Spells can also be used by a companion of the caster by sacrificing even half the Magic Points spent by the caster and using the same number of Actions.
\item
\textbf{Ethereal Magic} \index{Ethereal Magic}: Increasing the Magic Points spent on the spell by 3 will give your spells full effect on ethereal or incorporeal creatures
\item
\textbf{Merciful Magic}: Increasing Spells spent by 3 spells deal temporary damage.
Spells that deal damage of a particular type (such as fire) deal temporary damage of the same type.
\item
\textbf{Increase time} \index{Increase time} of casting from 2 Actions to 1 round decreases by 1 in Magic Points spent on casting spell.
\item
\textbf{Collaborative Spells} \index{Collaborative Spells}: Another spellcaster, sacrificing half the magic points of the partner who casts the spell, using the same number of Actions, can grant + 1d6 to the partner's Magic Check . Collaborative Magic can be combined with Effective Magic. Magical Proficiency Requirement 3
\item
\textbf{Circle of Power} \index{Circle of Power}: multiple spellcasters who are all Devotees or Followers of the same Patron can work together to make one of them more successful in casting a spell.
Each spellcaster sacrificing half the Magic Points of the spell cast by his partner can grant + 1d6 to his partner's Magic Check, up to a maximum of + 7d6. The casting time of a spell in the Circle of Power becomes at least 1 turn.
One or more companions can alternately use Effective Spells. Magical Proficiency Requirement 5
\item
\textbf{Minor changes} \index{Minor changes to spells} to the manifestation of the spell can be agreed with the Storyteller for an additional Magic Point cost or with a successful Magic Check.
\end{itemize}

The possibilities granted by Alter Spells are cumulative with each other.

\subsubsection{Attempting Spells with Impediments} \index{Attempting Spells with Impediments} \index{Impediments} \label{magieconimpedimenti}

The casting of a spell is bound to particular and unique gestures and words. When the character is in a situation where he cannot gesture or speak then he can attempt to cast the spell anyway even if it becomes much more difficult.

The Magic Points required for casting spells if he cannot gesticulate are tripled and if he cannot speak they are further tripled, it is also necessary in any case to make a Magic Check.

If the spell also has material components these must still be provided (placed within 30 cm of the caster) or it is not possible to cast the spell.

\subsubsection{Objective Definitions of Spells} \index{Objectives of Spells} \label{magiedefinizioniobiettivi}

In the spells listed below you will often find references to the types of subjects and objectives that can be influenced as well as to different types of energy and elements.

- The \textbf{Creatures} \textbf{Natural} are Insects, Reptiles, Beasts, Humanoids, Plants, Aquatic Creatures, Monstrosities, Oils.

- The \textbf{Creatures} \textbf{Magical} are: Fiend, Fairies, Spirits, Undead, Giants, Celestials, Elementals, Constructs, Aberrations (anything alien or unnatural) and Dragons.
If a Natural Creature has magical powers then it also considers itself as a Magical Creature. A more complete description of these "categories" can be found in the Chapter of Monstrosities.

- \textbf{Energy} includes: Force, Fire, Light, Sound, Electricity, Positive Energy, Negative Energy, Cold, Void.

\subsubsection{Energy or Light Damage}

The damage caused by \textbf{Light} \index{Light} is half fire and half positive energy, meaning a resistance to fire or positive energy only applies to half the damage caused by the attack.

The damage caused by \textbf{Void} \index{Void} is half from cold and half from negative energy, any protections apply to the respective halves of the damage.

The \textbf{negative energy} alone harms \index{Negative Energy} the living and heals the undead, the \textbf{positive energy} \index{Positive Energy} alone harms the undead but does not cure the living (at the Storyteller's discretion, exposure for one round may be equivalent to a lesser restoration spell), see also descriptions of the Plans. A target takes full damage from Light or Void if it has no inherent resistances.

A special case is \textbf{positive healing energy} \index{positive healing energy} which heals the living and harms the undead. This energy is that of Laying on of Hands, Channeling energy and Healing spells.

\end{multicols}

\vfill

\begin{center}
\includegraphics[width = 0.2 \linewidth]{immagini/Voynich_Manuscript.png}

\medskip

\textit{A page from the mysterious Voynich manuscript, which is undeciphered to this day.}
\end{center}

\pagebreak

\begin{multicols}{2}

\medskip

\begin{changemargin}{0.3cm}{0.3cm} \begin{tcolorbox}[title = More special effects!]
The spells listed are those of 5ed plus some of my proposals and other reinterpretations. If you have any suggestions for the Storyteller to deal with unexpected critics, talk to him! The spirit of collaboration must always be constructive.
\end{tcolorbox} \end{changemargin}

\section{The Spells}

~

\begin{changemargin}{0.3cm}{0.3cm} \begin{Storytellere} \index{Optional - Alternative to Magic Critical Success damage}
An alternative to the effects of magical critical success may be that with spells that cause direct damage or heal, instead of the additional die, you directly add half the value of the die rounded up. So 1d10 becomes 6, 1d8 becomes 5, 1d6 becomes 4, 1d4 becomes 3.
\end{Storytellere} \end{changemargin}


\medskip \textbf{Help}\index[Incantesimi]{Help} \\
\textbf{School}: Healing, Necromancy \\
\textbf{Level}: 2, Uncommon \\
\textbf{Casting Time}: 2 Actions \\
\textbf{Range}: 9 meters \\
\textbf{Components}: V, S, M (a thin strip of white fabric) \\
\textbf{Duration}: 8 hours \\
Your spell increases the toughness and resolve of your allies. Choose up to three creatures at range. Over the duration, each target's maximum hit points and current hit points increase by 5. \\
\textbf{For each Magic Critical Success rolled} in the Magic Check the target's hit points are increased by an additional 5 points

\medskip \textbf{Alarm}\index[Incantesimi]{Alarm} \\
\textbf{School}: Abjuration \\
\textbf{Level}: 1, Common \\
\textbf{Casting Time}: 1 minute \\
\textbf{Range}: 9 meters \\
\textbf{Components}: V, S, M (a bell and a piece of precious silver wire) \\
\textbf{Duration}: 8 hours \\
Set up an alarm against unwanted intrusions. Choose a door, window, or area with range that is no larger than a 6-meter edge cube. Until the spell ends, you will be alerted by an alarm whenever a creature of Tiny or larger size comes into contact or enters the protected area. When you cast the spell, you can designate creatures that will not set off the alarm. Choose whether the alarm is audible or just mental. A mental alarm, if you are within 1.5 kilometers of the protected area, warns you with a noise in your mind. Noise can wake you up if you are sleeping. An audible alarm produces the sound of a bell for 10 seconds, audible within 60 feet. \\
\textbf{For each Magic Critical Success rolled} in the Magic Check the duration is doubled.

\medskip \textbf{Mortal Hallucination}\index[Incantesimi]{Mortal Hallucination} \\
\textbf{School}: Illusion \\
\textbf{Level}: 4, Uncommon \\
\textbf{Casting Time}: 2 Actions \\
\textbf{Range}: 36 meters \\
\textbf{Components}: V, S \\
\textbf{Duration}: Instantaneous \\
You tap into the nightmares of a creature in range that you can see, and create an illusory manifestation of its innermost fears, visible only to that creature. The target must make a Will saving throw. \\
On a failed save, the target is frightened for 1 minute and takes 4d10 damage. \\
\textbf{For each magical critical success rolled} on the Magic Check the damage is increased by 1d10

\medskip \textbf{Alter Self}\index[Incantesimi]{Alter Self} \\
\textbf{School}: Transmutation \\
\textbf{Level}: 2, Common \\
\textbf{Casting Time}: 2 Actions \\
\textbf{Duration}: Self \\
\textbf{Components}: V, S \\
\textbf{Duration}: 1 minute per Magical Proficiency \\
Take a different shape. When casting this spell, choose one of the following options, the effect of which lasts for the duration of the spell. For the duration of the spell, you may end one option to gain the benefits of another. \\
Aquatic Adaptation. Adapt your body to an aquatic environment by developing webbed gills and fingers. You can breathe underwater and achieve swimming speeds equal to your walking speed. \\
\textit{Natural Weapons}. Develop claws, fangs, spikes, horns or a different natural weapon of your choice. Your unarmed strikes deal 1d6 points of slashing, piercing, or slashing damage, as appropriate to the natural weapon you are proficient with. Finally, the natural weapon is magical and you receive a +1 bonus on attack and damage rolls made when you use it. \\
\textit{Change of Appearance}. Transform your appearance. Decide on your outward appearance, including your height, weight, facial features, the sound of your voice, hair length, complexion, and any features you desire. You can appear as a member of another race, although none of your stats change. Also, you cannot appear as a creature of a different size than yours, and your base form remains the same; if you are bipedal, you cannot use this spell to become quadrupedal, for example. \\
At any point during the spell's duration, you can use two Actions to change your appearance again this way. \\
\textbf{For each Magic Critical Success you roll} in the Magic Check you can alter another subject or double the duration.

\medskip \textbf{Friendship with Animals}\index[Incantesimi]{Friendship with Animals} \\
\textbf{School}: Animals and Plants \\
\textbf{Level}: 1, Uncommon \\
\textbf{Casting Time}: 2 Actions \\
\textbf{Range}: 9 meters \\
\textbf{Components}: V, S, M (some food) \\
\textbf{Duration}: 24 hours \\
This spell allows you to convince a natural beast that you don't want to harm it. Choose a range beast that you can see. This must see and hear you. If the beast's Intelligence is 4 or more, the spell fails. Otherwise, the beast must succeed on a Will saving throw or be fascinated by you for the duration of the spell. If you or one of your companions damage the target, the spell ends. \\
\textbf{For each Magic Critical Success you roll} in the Magic Check you can act on one additional beast.

\medskip \textbf{Anathema}\index[Incantesimi]{Anathema} \\
\textbf{School}: Enchantment \\
\textbf{Level}: 1, Common \\
\textbf{Casting Time}: 1 minute \\
\textbf{Range}: 9 meters \\
\textbf{Components}: V, S, M (a drop of blood) \\
\textbf{Duration}: 1 minute \\
Up to three creatures of your choice that you can see, and that are in range, must make a Will saving throw. Any target who fails this saving throw and makes an attack or saving throw before the spell ends must roll a d4 and subtract the number so rolled from the attack or saving throw. \\
\textbf{For each Magic Critical Success you roll} in the Magic Check, you can target one additional creature.

\medskip \textbf{Messenger Animal}\index[Incantesimi]{Messenger Animal} \\
\textbf{School}: Animals and Plants \\
\textbf{Level}: 2, Common \\
\textbf{Casting Time}: 2 Actions \\
\textbf{Range}: 9 meters \\
\textbf{Components}: V, S, M (some food) \\
\textbf{Duration}: 24 hours \\
With this spell, you use an animal to deliver a message. Pick a Tiny beast that is within range and that you can see, such as a squirrel, jay or bat. You specify a place, which you must have visited in the past, and a recipient that matches a generic description, such as "a man or woman wearing the uniform of the city guard" or "a red-haired dwarf wearing a fedora". Also say a message of up to twenty-five words. The target beast travels for the duration of the spell to the specified location, covering approximately 75 kilometers in 24 hours for a flying messenger, or 40 kilometers for other animals. When the messenger arrives at its destination, it delivers the message to the creature you describe, replicating the sound of your voice. The messenger only speaks to a creature that matches your description. If the messenger fails to reach the destination before the spell ends, the message is lost, and the beast returns to where you cast the spell. \\
\textbf{For each Magic Critical Success obtained} in the Magic Check the spell duration is increased by 8 hours

\medskip \textbf{Animate Dead}\index[Incantesimi]{Animate Dead} \\
\textbf{School}: Necromancy \\
\textbf{Level}: 3, Common \\
\textbf{Casting Time}: 1 minute \\
\textbf{Range}: 3 meters \\
\textbf{Components}: V, S, M (a drop of blood, a piece of meat and a pinch of bone dust) \\
\textbf{Duration}: Instantaneous \\
This spell creates an undead minion. Choose a pile of bones or a corpse of a Medium or Small humanoid at range. Your spell imbues the target with a nefarious semblance of life, reviving them as an undead creature. The target becomes a skeleton if you choose bones or a zombie if you choose a corpse. During each of your rounds, you can use an Action to mentally command any creature you create with this spell that is within 60 feet of you (if you control multiple creatures, you can command all or some of them at the same time, sending the same command at all). Decide what action the creature will take and where it will move during its next round, or send it a general command, such as to stand guard over a particular room or corridor. If you don't send any commands, the creature just fends off hostile creatures. Once an order is received, the creature will continue to carry it out until it is fulfilled. The creature is under your control for 24 hours, after which it will stop executing your commands. To maintain control of the creature for another 24 hours, you must re-cast this spell on it before the current 24-hour period expires. This use of the spell reaffirms your control over up to four creatures you animated with this spell, rather than animating a new one. \\
\textbf{For each Magic Critical Success you roll} in the Magic Check you animate or reassert control over two undead creatures. Each of these creatures must come from a different corpse or pile of bones.

\medskip \textbf{Animate Objects}\index[Incantesimi]{Animate Objects} \\
\textbf{School}: Transmutation \\
\textbf{Level}: 5, Common \\
\textbf{Casting Time}: 1 minute \\
\textbf{Range}: 36 meters \\
\textbf{Components}: V, S \\
\textbf{Duration}: Concentration, maximum 1 minute \\
Objects come to life at your command. Choose up to ten nonmagical items that are within range and that are not worn or carried. Medium targets count as two objects, Large targets count as four objects, Huge targets count as eight objects. You cannot animate objects larger than Huge. Each target animates and becomes a creature under your control until the spell ends or until it is reduced to 0 hit points. \\
With an Action, you can mentally command any creature you've spawned with this spell that's within 150 meters of you (if you control multiple creatures, you can only command some or all of them at the same time, giving the same command to each). You decide what action the creature will take and where it will move during its next round, or you can issue a generic command, such as guarding a particular room or corridor. If you don't give commands, the creature will just defend itself from hostile creatures. Once an order is given, the creature will continue to follow it until it has completed its task. \\
\textbf{For each Magic Critical Success rolled} in the Magic Check the maximum duration is doubled.
\bigskip

\end{multicols}

\textbf{Animated Object Statistics}
\bigskip

\begin{tabular}{llllll}
Size & Hit Points & Defense & WP, Damage & Strength & Dexterity \\
\toprule
Lowercase & 20 & 18 & 8,{1d4 + 4} & -3 & 4 \\
Small & 25 & 16 & 6,{1d8 + 2} & -2 & 2 \\
Average & 40 & 13 & 5,{2d6 + 1} & 0 & 1 \\
Large & 50 & 10 & 6,{2d10 + 2} & 2 & 0 \\
Huge & 80 & 10 & 8,{2d12 + 4} & 4 & -2 \\
\end{tabular}

\bigskip

\begin{multicols}{2}

An animated object is a construct with Defense, Hit Points, Attacks, Strength, and Dexterity based on its size. His Intelligence and Wisdom score is -3, while his Charisma is -4. It has movement 9 meters; if the object has no legs or other appendages that it can use to move, it has 0 movement, has 30 feet flying movement, and can float. If the object is anchored to a surface or a larger object, such as a chain attached to a wall, its speed is 0. It has blind sight with a radius of 30 feet and is blind beyond this distance. \\
When the animated object drops to 0 hit points, it reverts to its normal object form, and any excess damage is dealt to its original form. \\
If you command an object to attack, it can make a single melee attack against a creature within 1 meter of it. Make an attack and damage determined by size (see table). The Storyteller might determine that depending on its shape, an object might instead deal slashing or piercing damage. \\
\textbf{For each Magic Critical Success you roll} in the Magic Check you can animate two additional items.

\medskip \textbf{Anti-Detection}\index[Incantesimi]{Anti-Detection} \\
\textbf{School}: Abjuration \\
\textbf{Level}: 3, Uncommon \\
\textbf{Casting Time}: 2 Actions \\
\textbf{Range}: Contact \\
\textbf{Components}: V, S, M (a pinch of diamond dust worth 25 gp scattered on the target, which the spell consumes) \\
\textbf{Duration}: 8 hours \\
For the duration, hide the target you have been in contact with from divination magic. The target can be a consenting creature or a place or object that occupies a space equivalent to a cube no greater than 10 feet of edge. The target cannot become the target of any divination spell or be perceived by magical scrying senses.

\medskip \textbf{Dislike / Sympathy}\index[Incantesimi]{Dislike / Sympathy} \\
\textbf{School}: Enchantment \\
\textbf{Level}: 8, Rare \\
\textbf{Casting Time}: 1 hour \\
\textbf{Range}: 18 meters \\
\textbf{Components}: V, S, M (or a piece of alum dipped in vinegar for the antipathy effect or a drop of honey for the sympathy effect) \\
\textbf{Duration}: 10 days \\
This spell attracts or repels creatures of your choice. Take a target at range, be it a Huge or smaller object or a creature or area no larger than a 60-meter-edged cube. Then specify some kind of intelligent creature, such as red dragons, goblins, or vampires. Invest the target in an aura that attracts or repels specified creatures for the duration. Choose dislike or sympathy as the effect of the aura. \\
Dislike. The enchantment causes creatures of the type indicated by you to feel a strong urge to leave the area and avoid the target. When such a creature can see the target or comes within 60 feet of it, the creature must succeed on a Will saving throw or become frightened. The creature is scared as long as it can see the target or stays within 60 feet of it. While frightened of the target, the creature must use its movement to move to the nearest safe spot from which it can no longer see the target. If the creature moves more than 60 feet away from the target and cannot see it, the creature is no longer scared, but it becomes scared again if it sees the target again or moves within 60 feet of it. \\
Sympathy. The enchantment causes specified creatures to feel a strong urge to approach the target if they are within 60 feet of it or can see it. When such a creature can see the target or comes within 60 feet of it, the creature must succeed at a Will saving throw or use its move during each round to enter the area, or move within range of the target. Once the creature has done so, it will no longer be able to voluntarily move away from the target. If the target damages or otherwise harms the affected creature, it can make a Will saving throw to end the effect, as described below. \\
End the Effect. If a subject creature ends its round while it is farther than 60 feet from the target or cannot see it, the creature makes a Will saving throw. On a successful save, the creature is no longer subject to the target and recognizes the feeling of repugnance or attraction as magical. In addition, a creature subject to the spell is entitled to another Will saving throw every 24 hours of the spell's duration. A creature that saves against this effect is immune to it for 1 minute, after which it can suffer it again.

\medskip \textbf{Energy Weapon}\index[Incantesimi]{Energy Weapon} \index{Lightsaber} \\
\textbf{School}: Air, Water, Earth, Fire \\
\textbf{Level}: 2, Very Rare \\
\textbf{Casting Time}: 1 Action \\
\textbf{Range}: Contact \\
\textbf{Components}: V, S, M (Fairy hair) \\
\textbf{Duration}: 6 rounds, Concentration \\
When you cast the spell on contact with a weapon, it acquires powers depending on the List of Spells used and is considered magical, as if it had a +1 bonus.
If Energy Weapon is launched using the Air List the weapon becomes struck by electricity, in case of Water the weapon becomes extremely cold, in case of Earth the weapon gushes acid, in case of Fire it becomes flaming. Whichever List is used, the effect is that the weapon deals 1d6 of additional damage of the indicated type per hit.
A weapon can only have one Energy Weapon effect active at the same time.
If you have all the indicated Lists the additional damage becomes 2d6 and each round using 1 action it is possible to change the damage type. \\
\textbf{For every two magical critical successes rolled} on the Magic Check the damage is increased by + 1d6.

\medskip \textbf{Magic Weapon}\index[Incantesimi]{Magic Weapon} \\
\textbf{School}: Transmutation \\
\textbf{Level}: 2, Common \\
\textbf{Casting Time}: 1 Immediate Action \\
\textbf{Range}: Contact \\
\textbf{Components}: V, S \\
\textbf{Duration}: 10 minutes \\
You cast the spell on contact with a non-magical weapon. Until the spell ends, the weapon becomes a magical weapon with a +1 bonus on attack and damage rolls. \\
\textbf{For each Magic Critical Success rolled} in the Magic Check the bonus increases to +1.

\medskip \textbf{Spiritual Weapon}\index[Incantesimi]{Spiritual Weapon} \\
\textbf{School}: Invocation \\
\textbf{Level}: 2, Common \\
\textbf{Casting Time}: 2 Actions \\
\textbf{Range}: 18 meters \\
\textbf{Components}: V, S \\
\textbf{Duration}: 3 minutes, Concentration \\
At one point in range, you create a floating ghostly weapon, which remains for the duration or until you cast this spell again. When you cast the spell, you can make a melee spell attack against a creature within 1 meter of the weapon with a bonus to hit equal to Magic Proficiency / 4. If you hit, the target takes force damage equal to 1d4 + your ability modifier for spellcasting spells + magical proficiency / 4. During your round, with an Action, you can move the weapon 20 feet and make the attack against a creature within 1 meter of the weapon. The weapon can take any shape you want, perhaps akin to the Patron. It is considered to have a magic bonus equal to Magical Proficiency / 4. \\
The bonuses granted by Magical Proficiency / 4 can be replaced by the sum of the Traits in common with the Patron / 4. \\
\textbf{For each Magic Critical Success rolled} on the Magic Check the damage is increased by 2.

\medskip \textbf{Magic Armor}\index[Incantesimi]{Magic Weapon} \\
\textbf{School}: Abjuration \\
\textbf{Level}: 1, Uncommon \\
\textbf{Casting Time}: 2 Actions \\
\textbf{Range}: Contact \\
\textbf{Components}: V, S, M (a piece of worked leather) \\
\textbf{Duration}: 8 hours \\
You cast the spell on contact with a consenting creature who is not wearing armor. A protective magical force surrounds the target until the spell ends. The target's Defense becomes 13 + Dexterity +1/6 Magical Proficiency. The spell ends if the target is wearing armor or interrupts the spell with an action. \\
\textbf{For each Magic Critical Success rolled} in the Magic Check the Defense increases by 1.

\medskip \textbf{Druidic Artifice} \index{Trick - Druidic Artifice} \\
\textbf{School}: Universal \\
\textbf{Level}: 0, Uncommon \\
\textbf{Casting Time}: 2 Actions \\
\textbf{Range}: 9 meters \\
\textbf{Components}: V, S \\
\textbf{Duration}: Instantaneous \\
Whispering to the spirits of nature, you create one of the following effects at range:

- You create a tiny, harmless sensory effect that predicts what climate your current location will be like for the next 24 hours. The effect could manifest as a golden sphere for clear skies, a cloud for rain, snowflakes for snow, and so on. The effect persists for 1 round. \\

- Immediately make a flower, seed or similar plant bloom. \\

- Create an instant and harmless sensory effect, such as falling leaves, a puff of wind, the sound of a small animal, or the faint smell of a skunk. The effect must fit into a 1 meter cube. \\

- Instantly light or extinguish a candle, torch or small bonfire. \\

This spell can only be cast by Followers or Devotees of Ephrem, Erondil, Gaya, Shayalia.

\medskip \textbf{Arcanist's Magical Aura}\index[Incantesimi]{Arcanist's Magical Aura} \\
\textbf{School}: Illusion \\
\textbf{Level}: 2, Uncommon \\
\textbf{Casting Time}: 2 Actions \\
\textbf{Range}: Contact \\
\textbf{Components}: V, S, M (a small silk square) \\
\textbf{Duration}: 24 hours \\
Place an illusion on a creature or object you are in contact with, so that divination spells reveal false information about it. The target can be a willing creature or an object that isn't carried or worn by another creature. When you cast this spell, you choose one or both of the following effects. The effect lasts for the duration. If you cast this spell on the same creature or object every day for 30 days, placing the same effect each time, the illusion will persist until dispelled. \\
\textit{False Aura}. You change the way the target results in spells and magical effects, such as detect magic, which detect magical auras. You can make a normal item appear magical, non-magical a magical item, or change the item's magical aura so that it appears to belong to a Magic List of your choice. When you use this effect on an object, you can make the false magic appear apparent to any creature manipulating it. \\
\textit{Mask}. You change how the target results in spells and magical effects that detect creature type or traits, such as symbol spell activation. Choose a creature type or trait, and other spells and magical effects will treat the target as a creature of that type or trait, and no more than the original one.

\medskip \textbf{Sacred Aura}\index[Incantesimi]{Sacred Aura} \\
\textbf{School}: Abjuration \\
\textbf{Level}: 8, Common \\
\textbf{Casting Time}: 2 Actions \\
\textbf{Duration}: Self \\
\textbf{Components}: V, S, M (a tiny reliquary worth at least 1000 gp containing a holy relic, such as a piece of cloth from a Devotee's robe or a piece of parchment from a religious text)\\
\textbf{Duration}: Concentration, 1 minute \\
You radiate divine light from you which collects in a dim luminosity with a radius of 30 feet around you. When you cast the spell, your chosen creatures in this ray emanate dim light with a radius of 1 meter and have{+ 2d6} on all saving throws, while other creatures have{-2d6} on attack rolls against you. them until the spell ends. In addition, when a demon or undead hits target creature with a melee attack, the aura glows with a bright light and must succeed on a Fortitude save or be blinded until the spell ends.

\medskip \textbf{Beneficial Berries}\index[Incantesimi]{Beneficial Berries} \\
\textbf{School}: Animals and Plants \\
\textbf{Level}: 2, Common \\
\textbf{Casting Time}: 2 Actions \\
\textbf{Range}: Contact \\
\textbf{Components}: V, S, M (a sprig of mistletoe) \\
\textbf{Duration}: Instantaneous \\
Spell up to 2d4 Berries in your hand which are infused with magic for the duration. A creature can use 1 immediate action to eat a berry. Eating a berry restores 1 hit point, and the berry also provides enough nutrition to feed a creature for a day. Only the first berry is effective on the day. \\
Berries lose their effectiveness if not consumed within 72 hours of casting the spell. \\
\textbf{For each Magic Critical Success obtained} in the Magic Check the Berries last one more day or enchant one more Berries (up to a maximum total of 8).

\medskip \textbf{Solar Flare}\index[Incantesimi]{Solar Flare} \index{Yamato Motive Wave Cannon} \\
\textbf{School}: Invocation \\
\textbf{Level}: 6, Uncommon \\
\textbf{Casting Time}: 2 Actions \\
\textbf{Range}: Personnel (line of 18 meters) \\
\textbf{Components}: V, S, M (a magnifying glass) \\
\textbf{Duration}: Concentration, maximum 1 minute \\
A bright beam of light explodes from your hand in a line 1 meter wide and 18 meters long. Each creature on the line must make a Fortitude saving throw. On a failed save, the creature takes 6d8 points of light damage and is blinded until your next round. On a successful save, she takes half damage and is not blinded. The undead and the oozes have -1d6 on this saving throw. You can create a new line of luminosity with an action during any of your rounds until the spell ends. \\
For the duration, a particle of bright light shines in your hand. It produces light in a radius of 9 meters and penumbra for a further 9 meters. This light is considered sunlight. \\
\textbf{In case of two magical critical successes rolled} the spell ends after the first ray but the line is 20 feet wide, 108 meters long, the damage from Light becomes 12d8.

\medskip \textbf{Feast of Heroes}\index[Incantesimi]{Feast of Heroes} \\
\textbf{School}: Summon \\
\textbf{Level}: 6, Uncommon \\
\textbf{Casting Time}: 10 minutes \\
\textbf{Range}: 9 meters \\
\textbf{Components}: V, S, M (a gem encrusted bowl worth at least 500 gp, which the spell consumes) \\
\textbf{Duration}: Instantaneous \\
Create a magnificent feast, including delicious food and drink. The feast is consumed in 1 hour and disappears at the end of this period, but the beneficial effects will not be felt until the hour ends. Up to twelve other creatures can
attend the banquet. A creature participating in the feast gains several benefits. The creature is healed of all disease and poison, becomes immune to poison and being scared, and has + 2d6 on all Will saving throws. His maximum hit points increase by 2d10, and he heals the same amount of hit points as he currently does. These benefits last for 24 hours. \\
\textbf{On case of two magical critical successes rolled} in the Magic Check the bowl is not consumed.

\medskip \textbf{Barrier of Blades}\index[Incantesimi]{Barrier of Blades} \\
\textbf{School}: Invocation \\
\textbf{Level}: 6, Common \\
\textbf{Casting Time}: 2 Actions \\
\textbf{Range}: 18 meters \\
\textbf{Components}: V, S \\
\textbf{Duration}: 10 minutes \\
You create a vertical wall of rotating blades made of magical energy, sharp as razors. The wall appears at range and remains for the duration. You can create a straight wall up to 30 meters long, 6 meters high and 1 meter thick, or a circular wall with a maximum diameter of 18 meters, 6 meters high and 1 meter thick. The wall provides three-quarters of cover for the creatures behind it, and its space is difficult terrain. \\
When a creature first enters the wall area in a round or begins its round there, the creature must make a Reflex saving throw. If the creature fails its saving throw, it takes 6d10 slashing damage, or half if it succeeds. \\
A spellcaster who is within one meter of the Barrier of Blades is considered Distracted.

\medskip \textbf{Cruel prank} \index{Trick - Cruel prank} \\
\textbf{School}: Enchantment \\
\textbf{Level}: 0, Common \\
\textbf{Casting Time}: 1 Action \\
\textbf{Range}: 18 meters \\
\textbf{Components}: V \\
\textbf{Duration}: Instantaneous \\
Unleash a series of insults wrapped in a devious spell against a creature in range that you can see. If the target can hear you (though they don't need to understand you), they must succeed on a Will saving throw or take 1d4 damage and have -1d6 on the next attack roll they make before their next round ends. \\
The damage of the spell increases by 1d4 when you reach CM 5, CM 11 and CM 17, but it costs 2 Actions to cast it boosted and 2 Magic Points, it is also necessary to have taken Adept of Magic in this Spell List a number of times equal to the upgrades that you want to apply. \\
\textbf{Every 2 magical critical successes rolled} on the Magic Check affect another creature.

\medskip \textbf{Bless Water}\index[Incantesimi]{Bless Water} \\
\textbf{School}: Universal \\
\textbf{Level}: 2, Common \\
\textbf{Launch Time}: 10 Minutes \\
\textbf{Range}: Touch \\
\textbf{Components}: V, S, M (25 gold coins offered to the church) \\
\textbf{Duration}: Instantaneous \\
Bless up to a liter of liquid, enough to create 5 bottles of holy water. \\
You must be a Follower or Devotee to be able to cast this spell. \\
\textbf{For each Magic Critical Success achieved} in the Magic Check bless an extra liter of liquid.

\medskip \textbf{Blessing}\index[Incantesimi]{Blessing} \\
\textbf{School}: Universal \\
\textbf{Level}: 1, Common \\
\textbf{Casting Time}: 2 Actions \\
\textbf{Range}: 9 meters \\
\textbf{Components}: V, S, M (a splash of holy water) \\
\textbf{Duration}: 1 minute \\
Bless up to three creatures in range, chosen by you. Targets gain +1 on saving throws and attack rolls. \\
More blessings, even from different Patrons don't add up. You must be a Follower or Devotee to be able to cast this spell. \\
\textbf{For each magical critical success rolled} in the Magic Check, you can add a creature as a target.

\medskip \textbf{Greater Blessing}\index[Incantesimi]{Greater Blessing} \\
\textbf{School}: Invocation \\
\textbf{Level}: 2, Uncommon \\
\textbf{Casting Time}: 1 Minute \\
\textbf{Range}: 18 meters \\
\textbf{Components}: V, S, M (a splash of holy water, 10 gold) \\
\textbf{Duration}: 1 hour \\
Bless a creature of your choice. The creature within the duration can add 1d6 to a roll before knowing if the check (TC / TS / Check) was successful or not. This bonus can be used 2 times per hour. You must be a Follower or Devotee to be able to cast this spell. \\
\textbf{For each Magic Critical Success you roll} in the Magic Check, you can add a creature as a target or add an hour to the duration.

\medskip \textbf{Supreme Blessing}\index[Incantesimi]{Supreme Blessing} \\
\textbf{School}: Invocation \\
\textbf{Level}: 3, Rare \\
\textbf{Casting Time}: 1 Reaction \\
\textbf{Range}: 27 meters \\
\textbf{Components}: V, S, M (a splash of holy water, 25 gold) \\
\textbf{Duration}: Instantaneous \\
Bless a creature of your choice. The creature can re-roll two dice from a single check before knowing whether or not the check was successful. The creature chooses whether to take the new rolls or keep the old ones. You must be a Follower or Devotee to be able to cast this spell. \\
\textbf{For each magical critical success rolled} on the spell check, the creature gets a +1 bonus on the check.

\medskip \textbf{Block Monsters}\index[Incantesimi]{Block Monsters} \\
\textbf{School}: Enchantment \\
\textbf{Level}: 5, Common \\
\textbf{Casting Time}: 2 Actions \\
\textbf{Range}: 27 meters \\
\textbf{Components}: V, S, M (a small straight piece of iron) \\
\textbf{Duration}: 1 minute \\
Choose a creature in range and that you can see. The target must succeed on a Will saving throw, or be paralyzed for the duration. This spell has no effect on undead or constructs. At the end of each of its rounds, the target can make another Will saving throw. If he succeeds, the spell ends for that target. \\
\textbf{For each magical critical success rolled} in the Magic Check, you can add a creature as a target as long as they are within 30 feet of each other.

\medskip \textbf{Lock Person}\index[Incantesimi]{Lock Person} \\
\textbf{School}: Enchantment \\
\textbf{Level}: 2, Common \\
\textbf{Casting Time}: 2 Actions \\
\textbf{Range}: 18 meters \\
\textbf{Components}: V, S, M (a small straight piece of iron) \\
\textbf{Duration}: 1 minute \\
Choose a humanoid in range that you can see. The spell has no effect on creatures with CR 4 or higher. The target must succeed on a Will saving throw or be paralyzed for the duration. \\
\textbf{For each magical critical success rolled} in the Magic Check, you can add a creature as a target as long as they are within 30 feet of each other.

\medskip \textbf{Advanced Person Block}\index[Incantesimi]{Advanced Person Block} \\
\textbf{School}: Enchantment \\
\textbf{Level}: 4, Uncommon \\
\textbf{Casting Time}: 2 Actions \\
\textbf{Range}: 18 meters, radius 6 meters \\
\textbf{Components}: V, S, M (a small straight piece of silver) \\
\textbf{Duration}: 1 minute \\
Blocks up to 2d4 GS (or levels) of creatures within 60 feet of you within a 20-foot radius. You start by blocking creatures with the lowest CR and subtracting the CR rolled from 2d4s, proceeding until you have no more points to block creatures. Targets must succeed on a Will saving throw or be paralyzed for the duration. \\
\textbf{For each Magic Critical Success you roll} in the Magic Check you can add 2 points to the 2d4 rolled.

\medskip \textbf{Magic Mouth}\index[Incantesimi]{Magic Mouth} \\
\textbf{School}: Illusion \\
\textbf{Level}: 2, Common \\
\textbf{Casting Time}: 1 minute \\
\textbf{Range}: 9 meters \\
\textbf{Components}: V, S, M (a small piece of honeycomb and jade dust worth at least 10 gp, which the spell consumes) \\
\textbf{Duration}: Until dissolved \\
Implants a message in a range object, a message that is spoken when the activation condition is satisfied. Choose an item that you can see that isn't worn or carried by another creature. Then say the message, which must be 25 words or less, but can be distributed over a period of up to 10 minutes. Finally, determine the circumstance that will activate the spell, in order for it to convey your message. \\
When the circumstance occurs, a magic mouth appears on the object and recites the message with your voice and at the same volume with which you pronounced it. If your chosen object has a mouth or something that looks like a mouth (for example, the mouth of a statue), the magic mouth appears so that the words appear to come from the mouth of the object. When you cast this spell, you can cause the spell to end after transmitting its message, or to persist and repeat the message each time the condition triggers. \\
The circumstance of activation can be as general or as detailed as you wish, but it must be based on visible or audible conditions occurring within 30 feet of the object. For example, you could instruct the mouth to speak when any creature approaches within 30 feet of the object or when a silver bell rings within 30 feet of it.

\medskip \textbf{Vital bubble}\index[Incantesimi]{Vital bubble} \\
\textbf{School}: Air, Abjuration \\
\textbf{Level}: 4, Uncommon \\
\textbf{Casting Time}: 1 minute \\
\textbf{Range}: 9 meters \\
\textbf{Components}: V, S, M (silver dust and diamond per 100 gp consumed) \\
\textbf{Duration}: 1 hour per Magical Proficiency \\
You can create up to 6 bubbles surrounding your designated creatures.
The total duration is 1 hour per point in Magical Proficiency divided at will among the creatures in the bubbles.
This bubble allows subjects to breathe freely, even underwater or in a vacuum, and renders them immune to noxious gases and vapors, including Inhalation Diseases and Poisons, and spells such as Stinking Cloud and Death Cloud. The bubble protects subjects from extreme temperatures (but not that they cause damage every round) and extreme pressures.

Vital bubble does not provide protection from negative or positive energy (e.g. on the Negative and Positive Energy planes), the ability to see in poor visibility conditions (such as in smoke or fog), nor the ability to move or act normally in conditions that prevent movement (such as underwater).

\medskip \textbf{Falling Feather}\index[Incantesimi]{Falling Feather} \\
\textbf{School}: Air \\
\textbf{Level}: 1, Common \\
\textbf{Casting Time}: 1 Reaction, which you perform when you or a creature within 60 feet of you falls \\
\textbf{Range}: 18 meters \\
\textbf{Components}: V, M (a small feather or a piece of feather) \\
\textbf{Duration}: 1 minute \\
Choose up to five creatures at range. A falling creature's rate of descent decreases to 60 feet per round until the spell ends. If the creature lands before the spell ends, it takes no falling damage and can land on its feet; for that creature the spell ends. \\
\textbf{For each magical critical success rolled} in the Magic Check you can move 1 meter sideways or affect another creature.

\medskip \textbf{Calm Emotions}\index[Incantesimi]{Calm Emotions} \\
\textbf{School}: Enchantment \\
\textbf{Level}: 2, Common \\
\textbf{Casting Time}: 2 Actions \\
\textbf{Range}: 18 meters \\
\textbf{Components}: V, S \\
\textbf{Duration}: Concentration, maximum 1 minute \\
You try to suppress strong emotions in a group of people. Each humanoid in a 20-foot-radius sphere centered on a point at range of your choice must make a Will saving throw; if it wishes, a creature can choose to fail this saving throw. If a creature fails its saving throw, choose one of these two effects. \\
\textit{Appease}. You can suppress any effects that make the target fascinated or frightened. When this spell ends, the suppressed effects resume, provided their duration is not expired in the meantime. \\
\textit{Indifference}. You can make a target indifferent to a creature of your choice, towards which it is hostile. This indifference ends if the target is attacked or damaged by a spell or if he sees one of his friends being damaged. When the spell ends, the creature becomes hostile again, unless the Storyteller determines otherwise.

\medskip \textbf{Walk on Water}\index[Incantesimi]{Walk on Water} \\
\textbf{School}: Water \\
\textbf{Level}: 3, Common \\
\textbf{Casting Time}: 2 Actions \\
\textbf{Range}: 9 meters \\
\textbf{Components}: V, S, M (a piece of cork) \\
\textbf{Duration}: 1 hour \\
This spell grants the ability to move across liquid surfaces (such as water, acid, mud, snow, quicksand, or lava) as if they were harmless solid ground (creatures passing through molten lava can still take damage from heat or melt in acid. ). Up to ten willing creatures within range that you can see receive this ability for the duration. If your target is immersed in a liquid, the spell returns the target to the surface of the liquid at a speed of 30 feet per round. \\
\textbf{For each magical critical success rolled} in the Magic Check the spell lasts 1 hour longer or you affect another creature.

\medskip \textbf{Walking in the Wind}\index[Incantesimi]{Walking in the Wind} \\
\textbf{School}: Air \\
\textbf{Level}: 6, Uncommon \\
\textbf{Casting Time}: 1 minute \\
\textbf{Range}: 9 meters \\
\textbf{Components}: V, S, M (fire and holy water) \\
\textbf{Duration}: 8 hours \\
For the duration, you and up to ten other consenting creatures at range you can see take on gaseous form, becoming clouds. While in cloud form, a creature has 90m flight speed and has resistance to damage from non-magical weapons. Returning to normal form takes 1 minute, during which time the creature is incapacitated and cannot move. Until the spell ends, a creature can revert to cloud form, which requires a one-minute transformation. If a creature is in cloud form and is flying when the effect ends, the creature drops 60 feet per round per minute until it lands, safely. If it fails to land after 1 minute, the creature will fall the remaining distance.

\medskip \textbf{Charm on People}\index[Incantesimi]{Charm on People} \\
\textbf{School}: Enchantment \\
\textbf{Level}: 1, Common \\
\textbf{Casting Time}: 2 Actions \\
\textbf{Range}: 9 meters \\
\textbf{Components}: V, S \\ fb
You try to fascinate a ranged humanoid that you can see. He must make a Will saving throw, and will have + 1d6 if he's fighting you or your allies. If he fails his saving throw, he is fascinated by you until the spell ends or until you or your allies do something harmful to him. The fascinated creature considers you a friendly acquaintance. When the spell ends, the creature is aware that it has been fascinated by you. \\
\textbf{For each Magic Critical Success you roll} in the Magic Check, you may add a creature as a target. When you cast the spell, target creatures must be within 30 feet of each other.

\medskip \textbf{Anti-Magic Field}\index[Incantesimi]{Anti-Magic Field} \\
\textbf{School}: Abjuration \\
\textbf{Level}: 8, Rare \\
\textbf{Casting Time}: 2 Actions \\
\textbf{Range}: Personnel (sphere of 3 meters radius) \\
\textbf{Components}: V, S, M (a pinch of powdered iron or iron lime) \\
\textbf{Duration}: Concentration, maximum 1 hour \\
Come surrounded by an invisible sphere of anti-magic with a 10-foot radius. This area is separated from the magical energy that permeates the multiverse. Within the sphere, spells cannot be cast, creatures summoned disappear, and magical items also become normal. Until the spell ends, the sphere moves with you, centered on you. Spells and other magical effects, except those created by an artifact or Patron, are suppressed within the sphere and cannot enter it. A slot spent to cast a suppressed spell is used up. While an effect is suppressed, it doesn't work, but the elapsed time that is suppressed is counted towards its duration. \\

\textit{Effects with Target}. Spells and other magical effects, such as magic missile and charm on people, that target a creature or object within the sphere have no effect on that target. \\
\textit{Areas of Magic}. The area of another spell or magical effect, such as a fireball, cannot extend into the sphere. If the sphere overlaps an area of magic, the part of that area covered by the sphere is suppressed. For example, the flames generated by a wall of fire are suppressed within the sphere, creating a hole in the wall if the overlap is large enough. Spells. Any spell or other magical effect active on a creature or object inside the sphere is suppressed as long as the creature or object is inside the sphere. \\
\textit{Magic Items}. The properties and powers of magical items are suppressed from the sphere. For example, a +1 longsword inside the sphere functions as a nonmagical longsword. The properties and powers of magical weapons are suppressed if they are used against a target within the sphere or wielded by an attacker within the sphere. If a magic weapon or magical ammunition leaves the sphere entirely (for example, if you shoot a magic arrow or shoot a magic spear at a target outside the sphere), the item's magic is no longer suppressed as soon as it exits the sphere. . \\
\textit{Travel Magic}. Teleportation and planar travel do not work inside the sphere, whether the sphere is the destination or starting point of this magical journey. Within the sphere, a portal to another place, world, or plane of existence, as well as an extradimensional space such as that created by the rope trick spell, remains closed. \\
\textit{Creatures and Objects}. Within the sphere, a creature or object summoned or created by magic temporarily vanishes from existence. The creature or object reappears instantly once the space it occupies is no longer within the sphere. \\
\textit{Dispel Magic}. Spells and magical effects such as dispel magic have no effect on the sphere. Likewise, spheres created by other antimagic field spells do not cancel each other out.

\medskip \textbf{Disguise Yourself}\index[Incantesimi]{Disguise Yourself} \\
\textbf{School}: Illusion \\
\textbf{Level}: 1, Common \\
\textbf{Casting Time}: 2 Actions \\
\textbf{Duration}: Self \\
\textbf{Components}: V, S \\
\textbf{Duration}: 1 hour \\
You change your appearance, along with that of your clothing, armor, weapons, and other items you are wearing, until the spell ends or until you take an action to break the spell. You can appear 12 inches shorter or taller, thin, fat, or somewhere in between. You cannot change your physical conformation, so you must adopt a form that has the same distribution of limbs. For everything else, the illusion is limited only by your imagination. \\
The changes made by this spell are unable to withstand physical inspection. For example, if you use this spell to add a hat to your outfit, objects go through the hat, and anyone who touches it will feel nothing and end up touching your head and hair. If you use this spell to appear thinner than you are, a person's hand trying to touch you would bounce off of you, while the sight of it would appear to stop in midair. To distinguish your camouflage, a creature can use 2 actions to inspect your appearance and must make a +4 Consciousness check against the spell's saving throw DC.

\medskip \textbf{Hut}\index[Incantesimi]{Hut} \\
\textbf{School}: Invocation \\
\textbf{Level}: 3, Uncommon \\
\textbf{Casting Time}: 1 minute \\
\textbf{Range}: Personal (hemisphere of 3 meters radius) \\
\textbf{Components}: V, S, M (a small crystal marble) \\
\textbf{Duration}: 8 hours \\
A half sphere of motionless force with a radius of 3 meters forms around and above you, remaining stationary for the duration. The spell ends if you leave the area. Eight Medium or smaller creatures can enter the dome with you. The spell fails if the area includes one creature larger or more than nine creatures. Creatures and objects inside the dome, when you cast this spell, can freely pass through it. All other creatures and objects must make a Fortitude save or are unable to pass through for that round. Spells and other magical effects cannot extend beyond the dome or pass through if it is level 1 or lower. The atmosphere inside the space is comfortable and dry, whatever the climate outside. \\
Until the spell ends, you can command the interior to become dimly lit or dark. The dome is opaque from the outside, whatever color you choose, but it is transparent from the inside. \\
\textbf{For each Magic Critical Success obtained} in the Magic Check the spell lasts 2 hours longer.

\medskip \textbf{Enhanced Charatcteristic}\index[Incantesimi]{Enhanced Characteristic} \\
\textbf{School}: Transmutation \\
\textbf{Level}: 2, Common \\
\textbf{Casting Time}: 2 Actions \\
\textbf{Range}: Contact \\
\textbf{Components}: V, S, M (fur or feather of a beast) \\
\textbf{Duration}: maximum 10 minutes \\
Grant a magical boost to a creature you are in contact with. Choose one of the following effects; the target gains that effect until the spell ends. \\
\textit{Cunning of the Fox}. Target has + 1d6 on Intelligence and Strength checks \\
\textit{Strength of the Bull}. The target has + 1d6 on Strength checks, and its encumbrance capacity doubles. \\
\textit{Grace of Light Energy}. The target has + 1d6 on Dexterity checks. In addition, if he is not incapacitated, he does not suffer damage from falls of 6 meters or less. \\
\textit{Bear Resistance}. The target has + 1d6 on Constitution checks. He also gains 2d6 temporary hit points, which are lost at the end of the spell. \\
\textit{Wisdom of the Owl}. The target has + 1d6 on Wisdom checks. \\
\textit{Splendor of the Eagle}. The target has + 1d6 on Charisma checks. \\
\textbf{For each Magic Critical Success you roll} in the Magic Check, you can target one more creature

\medskip \textbf{Flesh to Stone - Stone to Flesh}\index[Incantesimi]{Flesh to Stone} \\
\textbf{School}: Earth \\
\textbf{Level}: 6, Uncommon - Rare \\
\textbf{Casting Time}: 2 Actions \\
\textbf{Range}: 18 meters \\
\textbf{Components}: V, S, M (a pinch of lime, water and earth) \\
\textbf{Duration}: Permanent \\
You try to turn a ranged creature you can see to stone. If the target's body is made of flesh, the creature must make a Fortitude saving throw. If she fails her saving throw, she is entangled and her flesh begins to harden. If the saving throw is successful, the creature is not affected by the spell. A creature hampered by this spell must make another Fortitude saving throw at the end of each of its rounds. On a successful saving throw three times, the spell ends. If she fails her saving throw three times, she is turned to stone and becomes a victim of the petrified condition for the duration. Successes and failures don't have to be continuous; keep track of both until the target gets three of a kind. \\
If the creature is physically damaged while petrified, it suffers from deformities similar to damage done to the stone if it reverts to its original state. If you keep your focus on this spell for its full possible duration, the creature is turned to stone until the effect is removed. \\
The spell \textit{Stone to Flesh} returns a creature of flesh as long as it hasn't been transformed for more than a year.

\medskip \textbf{Chain of Lightning}\index[Incantesimi]{Chain of Lightning} \\
\textbf{School}: Aria \\
\textbf{Level}: 6, Rare \\
\textbf{Casting Time}: 2 Actions \\
\textbf{Range}: 45 meters \\
\textbf{Components}: V, S, M (some fur; a piece of amber, glass or a crystal rod; and three silver pins) \\
\textbf{Duration}: Instantaneous \\
You create a bolt of light that hits a target in range that you can see, chosen by you. From this a further bolt is generated which hits the nearest target within 6 meters. The process continues until 7 targets have been hit or there are no more new ranged opponents left. A target can be at least a medium-sized creature or object and can only be the target of one bolt. A target must make a Reflex saving throw. The target takes 8d6 points of lightning damage on a failed save, or half that damage if a successful one. \\
\textbf{For each Magic Critical Success rolled} in the Magic Check, the bolt reaches out to another target. \\
\textbf{Saving Throw Success / Critical Failure}: On critical failure the damage doubles, on critical success the damage is further halved

\medskip \textbf{Blindness / Deafness}\index[Incantesimi]{Blindness / Deafness} \\
\textbf{School}: Necromancy \\
\textbf{Level}: 2, Common \\
\textbf{Casting Time}: 2 Actions \\
\textbf{Range}: 9 meters \\
\textbf{Components}: V \\
\textbf{Duration}: 1 minute, Concentration \\
You can blind or deafen an enemy. Choose a creature in range and that you can see. The target must make a Fortitude saving throw. If unsuccessful, the target is either blinded or deafened (your choice) for the duration. \\
\textbf{For every two Magic Critical Success rolled} in the Magic Check you can add another target in range. If you roll 3 Magic Critical Success, the target is affected by the spell throughout the day.

\medskip \textbf{Blindness / Advanced Deafness}\index[Incantesimi]{Blindness / Deafness} \\
\textbf{School}: Necromancy \\
\textbf{Level}: 3, Uncommon \\
\textbf{Casting Time}: 2 Actions \\
\textbf{Range}: 9 meters \\
\textbf{Components}: V, S, M (some ear wax or a piece of black cloth) \\
\textbf{Duration}: 1 minute \\
You can blind or deafen an enemy. Choose a creature in range and that you can see. The target must make a Fortitude saving throw. If unsuccessful, the target is either blinded or deafened (your choice) for the duration. \\
\textbf{For each Magic Critical Success you roll} in the Magic Check, you can target one additional creature.

\medskip \textbf{Conceal}\index[Incantesimi]{Conceal} \\
\textbf{School}: Transmutation \\
\textbf{Level}: 7, Rare \\
\textbf{Casting Time}: 2 Actions \\
\textbf{Range}: Contact \\
\textbf{Components}: V, S, M (a dust composed of diamond, emerald, ruby, and sapphire dust worth at least 50,000 gp, which the spell consumes) \\
\textbf{Duration}: Until dissolved \\
Through this spell, a consenting creature or object can be hidden, impossible to detect for the duration. By casting this spell and making contact with a target, it becomes invisible and cannot be targeted by divination spells, nor perceived by scrying sensors created by divination spells. \\
If the target is a creature, it falls into a suspended animation state. For him, time ceases to flow, and he does not age. \\
You can set up a condition for the spell to end prematurely. The condition can be anything you want, but it must happen or be visible within 1.5 kilometers of the target. Examples include "at the next judgment of the Patrons" or "when the tarrasque awakens". This spell also ends if the target takes damage.

\medskip \textbf{Magic Circle}\index[Incantesimi]{Magic Circle} \\
\textbf{School}: Abjuration \\
\textbf{Level}: 3, Common \\
\textbf{Casting Time}: 1 minute \\
\textbf{Range}: 3 meters \\
\textbf{Components}: V, S, M (Holy Water or silver and powdered iron worth at least 100 gp, which the spell consumes) \\
\textbf{Duration}: 1 hour \\
You create a 10-foot-high, 10-foot-high cylinder of magical energy that is centered on a point on the ground at range and that you can see. Luminous runes appear wherever the cylinder intersects the floor or other surface. \\
Choose one or more of the following creature types: celestial, elemental, fairy, demon, or undead. The circle affects a creature of the chosen type in the following ways: \\

- The creature cannot consciously enter the cylinder by any non-magical means. If the creature tries to use teleportation or plane travel to do so, it must first make a Will save. \\

- The creature has -1d6 on attack rolls against targets inside the cylinder. \\

- Targets inside the cylinder cannot be fascinated, frightened, or possessed by the creature. When you cast this spell, you can decide that the spell works in the opposite direction, preventing a creature of the specified type from leaving the cylinder and protecting targets outside. \\

\textbf{For each Magic Critical Success you roll} in the Magic Check you can increase the duration by 1 hour.

\medskip \textbf{Circle of Death}\index[Incantesimi]{Circle of Death} \\
\textbf{School}: Invocation \\
\textbf{Level}: 6, Very Rare \\
\textbf{Casting Time}: 2 Actions \\
\textbf{Range}: 45 meters \\
\textbf{Components}: V, S, M (a powdered black pearl worth at least 500 gp) \\
\textbf{Duration}: Instantaneous \\
A sphere of negative energy with a radius of 60 feet erupts at one point at range. Each creature in that area must make a Fortitude saving throw. A target takes 8d6 Void damage on a failed save, or half that damage on a successful save. \\
\textbf{For each magical critical success rolled} on the Magic Check the damage is increased by 1d6. \\
\textbf{Saving Throw Success / Critical Failure}: On critical failure the damage doubles, on critical success the damage is further halved

\medskip \textbf{Teleportation Circle}\index[Incantesimi]{Teleportation Circle} \\
\textbf{School}: Summon \\
\textbf{Level}: 5, Uncommon \\
\textbf{Casting Time}: 1 minute \\
\textbf{Range}: 3 meters \\
\textbf{Components}: V, M (rare gem-infused chalks and inks worth at least 50 gp, which the spell consumes) \\
\textbf{Duration}: 1 round \\
As you cast the spell, you draw a 10-foot-diameter circle on the floor, inscribed with seals that connect your whereabouts to a permanent teleportation circle of your choice, whose sequence of seals you know and which is on the same plane of existence in which you are. A glowing portal opens within the circle you have drawn and remains open until the end of your next round. Any creature that enters the portal reappears instantly within 1 meter of the target circle or in non-space
closest occupied, if he cannot appear within 1 meter of it. \\
Many large temples, guilds, and other important places have permanent teleportation circles engraved somewhere in their vicinity. Each of these circles has a unique seal sequence: a series of magical runes arranged in a precise pattern. \\ When you gain the ability to cast this spell, you learn the seal sequences of
two destinations on the Material Plane, determined by the Storyteller. In the course of your adventures you can learn new sequences of seals. You can memorize a sequence of seals after studying it for at least 1 minute. \\
You can create a permanent teleportation circle by casting this spell in the same place every day for a year. You don't have to use the teleportation circle when casting the spell this way.

\medskip \textbf{Clairvoyance}\index[Incantesimi]{Clairvoyance} \\
\textbf{School}: Divination \\
\textbf{Level}: 3, Common \\
\textbf{Casting Time}: 10 minutes \\
\textbf{Range}: 0.9 miles \\
\textbf{Components}: V, S, M (a focus worth at least 100 gp, be it a bejeweled horn to hear or a glass eye to look at) \\
\textbf{Duration}: Concentration, maximum 10 minutes \\
You create an invisible sensor in a place that is familiar to you and that is within range (a place you have already visited or seen previously) or in an obvious place that is not familiar to you (such as behind a door or a corner, or in the middle of a grove of trees). The sensor remains in place for the duration, and cannot be attached or interacted with. When you cast this spell, you choose to see or hear. You can use the sense chosen through the sensor, as if you were in its space. With two actions, you can switch between hearing and feeling and vice versa. A creature that can see the sensor (a creature with invisibility or true seeing) perceives it as an intangible, luminous orb the size of your fist. \\
\textbf{For each Magic Critical Success rolled} in the Magic Check the duration increases by 10 minutes or the range increases by 500m.

\medskip \textbf{Clone}\index[Incantesimi]{Clone} \\
\textbf{School}: Necromancy \\
\textbf{Level}: 8, Uncommon \\
\textbf{Range}: Contact
\textbf{Components}: V, S, M (a diamond worth at least 1000 gp and at least 16 cubic centimeters of flesh from the creature to be cloned, which the spell consumes, and a vessel worth at least 2000 gp in value that has a sealable lid and is large enough to hold a Medium creature, such as a large urn, coffin, mud-filled pit in the ground, or a crystal container filled with salt water) \\
\textbf{Duration}: Instantaneous \\
This spell produces the inert duplicate of a living creature as a safeguard from death. This clone forms inside a sealed container and reaches its maximum size and maturity after 120 days; you can also decide that the clone is a younger version of the same creature. It remains inert and survives indefinitely as long as the vessel remains undisturbed. \\
At any time after the clone matures, if the original creature dies, its soul moves into the clone, as long as the soul is free and willing to return. The clone is physically identical to the original and has the same personality, memories and characteristics, but nothing of the equipment of the original. The physical remains of the original creature, if they still exist, become inert and cannot be brought back to life, since the creature's soul is elsewhere. \\
\textbf{This spell is not selectable if Patrons are active}

\medskip \textbf{Accurate Strike} \index{Trick - Accurate Strike} \\
\textbf{School}: Divination \\
\textbf{Level}: 0, Common \\
\textbf{Casting Time}: 2 Actions \\
\textbf{Range}: 9 meters \\
\textbf{Components}: S \\
\textbf{Duration}: 1 round \\
You reach out and point your finger at a target in range. Your spell gives you a brief understanding of the target's defenses. During your next round, as long as this spell has not ended, you get + 1d6 on your first attack roll against that target. \\
\textbf{For each Magic Critical Success rolled} the bonus lasts for one more round.

\medskip \textbf{Flame Strike}\index[Incantesimi]{Flame Strike} \\
\textbf{School}: Fire \\
\textbf{Level}: 5, Common \\
\textbf{Casting Time}: 2 Actions \\
\textbf{Range}: 18 meters \\
\textbf{Components}: V, S, M (pinch of sulfur) \\
\textbf{Duration}: Instantaneous \\
A vertical column of divine fire descends from the sky and strikes at the place you specify. Any creature in a 10-foot-high, 10-foot-high cylinder centered on a point at range must make a Reflex saving throw. A creature takes 8d6 points of light damage if it fails its saving throw, or half that damage if it succeeds. \\
\textbf{For each magical critical success rolled} on the Magic Check the Light damage is increased by 1d6. \\
\textbf{Saving Throw Success / Critical Failure}: On critical failure the damage doubles, on critical success the damage is further halved

\medskip \textbf{Command}\index[Incantesimi]{Command} \\
\textbf{School}: Enchantment \\
\textbf{Level}: 1, Common \\
\textbf{Casting Time}: 2 Actions \\
\textbf{Range}: 18 meters \\
\textbf{Components}: V \\
\textbf{Duration}: 1 round \\
You speak a one-word command to a creature in range that you can see. The target must succeed at a Will saving throw or carry out the command within its next round. The spell has no effect if the target is undead, does not understand your language, or if your command would damage them. Here are some typical commands and their effects. You can give commands other than those described here, and in that case the Storyteller will determine the target's behavior. If the target cannot carry out your command, the spell ends. \\

- \textit{Come closer}. The target moves towards you the shortest and most direct route, ending its round if it gets within 1 meter of you. \\

- \textit{Still}. The target does not move and then ends its round. A flying creature stays in place as long as it can. If it has to move to stay in the air, it flies the minimum distance necessary to do so. \\

- \textit{Throw}. The target throws whatever he is holding and then ends his round. \\

- \textit{Run away}. The target spends its round moving away from you with the fastest means at its disposal. \\

- \textit{Strip}. The target falls prone and then ends his round. \\

\textbf{For each Magic Critical Success you roll} in the Magic Check, you can act on an additional creature. By the time you cast the spell, target creatures must be within 30 feet of each other and carry out the same command.

\medskip \textbf{CTRLC + CTRLV (Copy Paste)}\index[Incantesimi]{CTRLC + CTRLV (Copy Paste)} \\
\textbf{School}: Universal \\
\textbf{Level}: 1, Very Rare \\
\textbf{Casting Time}: 2 Actions \\
\textbf{Duration}: Self \\
\textbf{Components}: V, S, M (three small ceramic cubes bearing the letter C, the letter V and the glyph CTRL) \\
\textbf{Duration}: 1 minute per Magical Proficiency \\
This spell allows you to copy text from one source to another. In the case of a non-magical source this can be a book, a parchment, some runes on a plate or a stick. The destination that is placed on the source will copy the symbols in shape and size up to its capacity, for a maximum of 1 (destination) page per minute.

If the writing is a spell, therefore on a Tome or Scroll, the rules and limitations provided for the copy of Spells on the Tome must still be respected. This spell allows you to avoid the Magic Check in case of a Spell of a level higher than the maximum allowed. Once a spell is copied, this spell ends.

\medskip \textbf{Understanding Languages}\index[Incantesimi]{Understanding Languages} \\
\textbf{School}: Divination \\
\textbf{Level}: 1, Common \\
\textbf{Casting Time}: 2 Actions \\
\textbf{Duration}: Self \\
\textbf{Components}: V, S, M (a pinch of salt and soot) \\
\textbf{Duration}: 1 hour \\
For the duration, understand the literal meaning of any spoken language you hear. \\
\textbf{For each Magic Critical Success rolled} in the Magic Check the duration is doubled.

\medskip \textbf{Comprehension of Writings}\index[Incantesimi]{Comprehension of Writings} \\
\textbf{School}: Divination \\
\textbf{Level}: 2, Uncommon \\
\textbf{Casting Time}: 2 Actions \\
\textbf{Duration}: Self \\
\textbf{Components}: V, S, M (a pinch of silver and dry ink) \\
\textbf{Duration}: 1 hour \\
For the duration you understand any non-magical written language you see. You have to be in contact with the surface on which the words are written. It takes 1 minute to read a page of text. This spell does not decode secret messages into a text or glyph, such as an arcane seal, that is not part of a written language. \\
\textbf{For each Magic Critical Success rolled} in the Magic Check the duration is doubled.

\medskip \textbf{Compulsion}\index[Incantesimi]{Compulsion} \\
\textbf{School}: Enchantment \\
\textbf{Level}: 4, Uncommon \\
\textbf{Casting Time}: 2 Actions \\
\textbf{Range}: 9 meters \\
\textbf{Components}: V, S \\
\textbf{Duration}: Concentration, maximum 1 minute \\
Creatures of your choice within range that you can see and can hear you must make a Will saving throw. A target automatically succeeds at its saving throw if it cannot be charmed. Until the spell ends, you can use an Action during each of your rounds to indicate a horizontal direction from you. Each target subject to the spell must use as much of its movement as possible, during its next round, to move in that direction. The target cannot take any action before moving. After moving in this way, the target can make another Will saving throw to attempt to end the effect. \\
A target cannot be forced to move into a manifestly lethal hazard, such as flames or pits.

\medskip \textbf{Communion}\index[Incantesimi]{Communion} \\
\textbf{School}: Divination \\
\textbf{Level}: 5, Rare \\
\textbf{Casting Time}: 1 minute \\
\textbf{Duration}: Self \\
\textbf{Components}: V, S, M (incense and a vial of Blessed or blasphemous Water) \\
\textbf{Duration}: 1 minute \\
You communicate with your Patron and ask him up to three questions that can be answered with a yes or a no. You must ask the questions before the spell ends. You will receive the correct answer to each question. Divine creatures are not necessarily omniscient, so you may receive "not clear" as an answer to a question regarding information not relevant to the Patron's knowledge. In the event that a one-word response could be misleading or contrary to the interests of the Patron, the Storyteller could instead give a short sentence as an answer. \\
If you cast the spell two or more times before the new dawn has risen there is a cumulative 25 \% chance that for every cast after the first you will get no response. The Storyteller makes this roll in secret. \\
\textbf{This spell is not selectable if the Patrons are not active}

\medskip \textbf{Communion with Nature}\index[Incantesimi]{Communion with Nature} \\
\textbf{School}: Divination \\
\textbf{Level}: 5, Very Rare \\
\textbf{Casting Time}: 1 minute \\
\textbf{Duration}: Self \\
\textbf{Components}: V, S \\
\textbf{Duration}: Instantaneous \\
For a moment you become one with nature and get information about the surrounding area. Outdoors, the spell gives you information about the territory within 5 kilometers of you. In caves and other natural underground environments, the range is limited to 100 meters. The spell does not work in places where nature has been supplanted by buildings, such as in dungeons and villages. \\
Instantly learn information on up to three topics of your choice on one of the following subjects, related to the area:

- land and water bodies \\
- plants, minerals, animals and prevailing populations \\
- mighty celestial, elemental, fairy, demon or undead \\
- influences from other planes of existence \\
- buildings \\
\textbf{For each Magic Critical Success obtained} in the Magic Check, you learn an additional topic.

\medskip \textbf{Confusion}\index[Incantesimi]{Confusion} \\
\textbf{School}: Enchantment \\
\textbf{Level}: 4, Common \\
\textbf{Casting Time}: 2 Actions \\
\textbf{Range}: 27 meters \\
\textbf{Components}: V, S, M (three walnut shells) \\
\textbf{Duration}: 1 minute \\
This spell assaults and bends the minds of creatures, generating illusions and provoking uncontrolled actions. When you cast this spell, each creature in a 10-foot-radius sphere centered on a point of your choice within range must succeed at a Will saving throw or be affected. A target subject to the spell cannot react and must roll a d10 at the start of each of its rounds to determine its behavior for that round.

\medskip

\begin{tabularx}{0.45\textwidth}{lX}
\hline
d10 & Behavior \\
1 & The creature uses all of its movement to move in a random direction. To determine the direction, roll a d8 giving each face a cardinal point. There
creature will take no action this round. \\
2-6 & The creature cannot move or attack this round. \\
7-8 & The creature uses its 2 Actions (and none other) to make a melee attack against a randomly determined creature within its range. If there is no creature within range, the creature will do nothing this round. \\
9-10 & The creature can act and move normally. \\
\end{tabularx}

\medskip

At the end of each of its rounds, a target subject to the spell can make a Will saving throw. If he gets over it, the effect ends for him. \\
\textbf{For each Magic Critical Success rolled} in the Magic Check the radius of the sphere increases by 1 meter.

\medskip \textbf{Cone of Cold}\index[Incantesimi]{Cone of Cold} \\
\textbf{School}: Water \\
\textbf{Level}: 5, Common \\
\textbf{Casting Time}: 2 Actions \\
\textbf{Range}: Staff (18 meter cone) \\
\textbf{Components}: V, S, M (a small crystal or glass cone) \\
\textbf{Duration}: Instantaneous \\
A blast of cold air erupts from your hands. Each creature in a 60-foot cone must make a Fortitude saving throw. A creature takes 8d8 cold damage if it fails its saving throw, or half that damage if it succeeds. A creature killed by this spell becomes an ice statue until it thaws. \\
\textbf{For each magical critical success rolled} in the Magic Check the damage is increased by 1d8 \\
\textbf{Saving Throw Success / Critical Failure}: On critical failure the damage doubles, on critical success the damage is further halved

\medskip \textbf{Knowledge of Legends}\index[Incantesimi]{Knowledge of Legends} \\
\textbf{School}: Divination \\
\textbf{Level}: 5, Common \\
\textbf{Casting Time}: 10 minutes \\
\textbf{Duration}: Self \\
\textbf{Components}: V, S, M (incense worth at least 250 gp, which the spell consumes, and four ivory strips worth at least 50 gp) \\
\textbf{Duration}: Instantaneous \\
Name or describe a person, place or object. The spell brings to mind a brief summary of the most important knowledge on the topic you mentioned. If the thing you mention has no legendary significance, you get no information. The more information you have on the subject, the more precise and detailed the information you will receive. The information you will receive will be accurate, but perhaps concealed in metaphorical language.

\medskip \textbf{Contagion}\index[Incantesimi]{Contagion} \\
\textbf{School}: Necromancy \\
\textbf{Level}: 5, Uncommon \\
\textbf{Casting Time}: 2 Actions \\
\textbf{Range}: Contact \\
\textbf{Components}: V, S \\
\textbf{Duration}: 7 days \\
Through contact you can inflict disease. Make a melee attack against a creature in range. If you hit, you infect the creature with a disease of your choice from those described below. At the end of each round the target must make a Fortitude saving throw. After failing three of these saving throws, the disease's effects persist for the duration, and the creature no longer makes any saving throws. After making three of these saving throws, the creature recovers from the disease, and the spell ends. \\
Since this spell induces a natural disease on its target, any effect that removes disease or improves the effects of disease applies to it. \\
- \textit{Putrid Meat}. The creature's skin rots. The creature has -1d6 on Charisma checks and any damage is doubled. \\
- \textit{Blinding Weakness}. Pain grips the creature's mind as its eyes turn milky white. The creature has -1d6 on Wisdom checks and Will saving throws, and is blinded. \\
- \textit{Filthy Fever}. A devastating fever overwhelms the creature's body. The creature has -1d6 on Strength checks and Fortitude saving throws, and on attack rolls that use Strength. \\
- \textit{Fitte}. The creature is overcome with tremors. The creature has -1d6 on Dexterity checks, Reflex saving throws, and attack rolls that use Dexterity. \\
- \textit{Mindfire}. The creature's mind is in the grip of fever. The creature has -1d6 on Intelligence checks and Will saving throws, and behaves as if it were under the effect of the confusion spell in combat. \\
- \textit{Oozy Death}. The creature begins to bleed incessantly. The creature has -1d6 on Constitution checks and Fortitude saving throws. Additionally, whenever the creature takes damage, it is stunned until the end of its next round. \\


\medskip \textbf{Contingency}\index[Incantesimi]{Contingency} \\
\textbf{School}: Invocation \\
\textbf{Level}: 6, Uncommon \\
\textbf{Casting Time}: 10 minutes \\
\textbf{Duration}: Self \\
\textbf{Components}: V, S, M (a statuette of yourself carved from ivory and decorated with gems worth at least 1,500 gp) \\
\textbf{Duration}: 10 days \\
Choose a Level 4 or lower spell that you can cast, that has a casting time of 2 Actions, and that can target you. You cast that spell (called the contingent spell) as part of the contingency cast, spending the spell slots of both, but without the contingent spell taking effect. Instead, it will take effect when a certain circumstance occurs. Describe this circumstance as you cast the two spells. For example, contingency thrown together with breathing underwater might stipulate that breathing underwater comes into effect when immersed in water or similar liquid. \\
The contingent spell takes effect immediately after the circumstance first occurs, whether you want it to or not, and then contingency ends. The contingent spell affects only you, although it can normally target others as well. You can only use one contingency spell at a time. If you cast this spell again, another contingency spell's effect on you will end. Furthermore, contingency for you ends if the material component is no longer on your person. \\
\textbf{For each Magic Critical Success obtained} in the Magic Check the contingency lasts 10 more days.

\medskip \textbf{Counterspell}\index[Incantesimi]{Counterspell} \\
\textbf{School}: Abjuration \\
\textbf{Level}: 3, Common \\
\textbf{Casting Time}: 1 Reaction, you perform when you see a creature / object within 60 feet manifesting a spell \\
\textbf{Range}: 18 meters \\
\textbf{Components}: S \\
\textbf{Duration}: Instantaneous \\
You use a Reaction Action to make an Arcana check at DC 13. If the check is successful you understand if you can negate the spell's effect with a Counter-Spell. The canceled spell must be Level 3 or lower, regardless of whether it is manifested by a spellcaster or object. Each magical critical or enhancement success obtained from the original spell raises the spell's level by 1. \\
\textbf{For every two Magic Critical Success obtained} in the Magic Check you can cancel one spell of a higher level.

\medskip \textbf{Control Water}\index[Incantesimi]{Control Water} \\
\textbf{School}: Water \\
\textbf{Level}: 4, Common \\
\textbf{Casting Time}: 2 Actions \\
\textbf{Range}: 90 meters \\
\textbf{Components}: V, S, M (a drop of water and a pinch of powder) \\
\textbf{Duration}: Concentration, maximum 10 minutes \\
Until the spell ends, you control any free water within the area you have chosen up to a 30-meter edge cube. When you cast this spell, you can choose any of the following effects. As an action, during your round, you can repeat the same effect or choose a different one. \\

- \textit{Flooding}. Let the level of all the water in the area increase by up to 6 meters. If the area includes a coastline, the water floods the mainland. If you choose an area within a large body of water, you instead create a 6-meter-high wave that travels from one side of the area to the other before crashing. Any Huge or smaller vehicle in the wave path is carried to the other side. Any vehicle of Huge size or smaller hit by water has a 25 \% roll over rate. \\
The water level remains elevated until the spell ends or until you choose a different effect. If this effect produced a wave, the wave repeats at the beginning of your next round, as long as the flooding effect persists. \\
- \textit{Divide the Waters}. Allow the water in the area to move to the side to create an opening. The gap extends across the spell's area, and the divided water forms a wall on either side of the gap. The gap remains until the spell ends or until you choose a different effect. The water will then slowly return to fill the gap over the course of the next round, until it has risen to its normal level. \\
- \textit{Redirect the Stream}. Have the running water in the area move in a direction of your choosing, even if the water has to overcome obstacles, climb walls, or head in other unlikely directions. The water in the area moves according to your directions, but once it gets beyond the spell's area, it resumes its flow based on the conditions of the terrain. The water continues to move in the direction you choose until the spell ends or until you choose a different effect. \\
- \textit{Turbine}. This effect requires a body of water that covers a square of 15 meters on each side and has a depth of 7 meters. Make a whirlwind form in the center of the area. The whirlwind produces a vortex 1 meter wide at the base, up to 15 meters wide at the top and 7 meters high. Any creature or object in the water that is within 7 meters of the vortex is pulled 10 feet towards it. A creature can swim out of the vortex by making a Dexterity (Athletics) check against the spell's saving throw DC.

When a creature enters the vortex for the first time during a round or begins its round there, it must make a Fortitude saving throw. If it fails, the creature takes 2d8 points of hit damage and is captured by the vortex until the spell ends. On a successful save, the creature takes half this damage, and is not captured by the vortex. A creature captured by the vortex can use 3 Actions to try to swim away from the vortex as described above, but has -1d6 on Dexterity (Athletics) checks to do so. The first time during each round that an object enters the vortex, the object takes 2d8 points of hit damage; this damage is taken every round the object remains in the vortex.

\medskip \textbf{Check Weather}\index[Incantesimi]{Check Weather} \\
\textbf{School}: Water, Air \\
\textbf{Level}: 8, Very Rare \\
\textbf{Casting Time}: 10 minutes \\
\textbf{Range}: Personnel (radius of 1.5 kilometers) \\
\textbf{Components}: V, S, M (incense burned and some earth and wood mixed in the water) \\
\textbf{Duration}: Concentration, maximum 8 hours \\
For the duration, take control of the climate within 7.5 kilometers of you. To cast this spell you must be outside. Moving to a place where you don't have an open view to the sky ends the spell prematurely. When you cast this spell, the current weather conditions change, as determined by the Storyteller based on the season and latitude. You can change the rainfall, temperature and wind. It takes 1d4 x 10 minutes for the new condition to take effect. Once the condition takes effect, you can change it again. When the spell ends, the weather will gradually return to normal. \\
When changing weather conditions, find the current condition on the following table and change it by one stage, up or down. When you change the wind, you can change its direction as well.

\medskip

\textit{Precipitation}

- 1 Clear

- 2 A few clouds

- 3 Overcast or haze on the ground

- 4 Rain, hail or snow

- 5 Torrential rain, heavy hail, blizzard \\

\textit{Temperature}

- 1 Unbearable heat

- 2 Hot

- 3 Warm

- 4 Fresh

- 5 Cold

- 6 Polar cold \\

\textit{Wind}

- 1 Calm

- 2 Moderate wind

- 3 Moderate wind

- 4 Fortunale

- 5 Storm \\

\textbf{For each Magic Critical Success obtained} in the Magic Check the duration is increased by 8 hours.

\medskip \textbf{Constraint}\index[Incantesimi]{Constraint} \\
\textbf{School}: Enchantment \\
\textbf{Level}: 5, Rare \\
\textbf{Casting Time}: 1 minute \\
\textbf{Range}: 18 meters \\
\textbf{Components}: V \\
\textbf{Duration}: 30 days \\
You impose a magical command on a ranged creature you can see, forcing it to perform a certain task or forbidding it from taking an action or course of activity you decide. If the creature can understand you, it must succeed on a Will saving throw or be fascinated by you for the duration. While the creature is fascinated by you, it takes 3d10 damage each time it acts directly contrary to your instructions, but no more than once per day. A creature that cannot understand you ignores the effects of this spell. You can give any command of your choice, except an activity that would result in certain death. Should you issue a suicidal command, the spell would end. \\
You can end the spell using an action. Also remove curse, higher refresh or desire put an end to it. \\
\textbf{If you get at least two Critics} in the Trial of Magic the duration is 1 year. If you get 3 Critics, the spell lasts until one of the aforementioned spells ends.

\medskip \textbf{Creating Food and Water}\index[Incantesimi]{Creating Food and Water} \\
\textbf{School}: Summon \\
\textbf{Level}: 3, Common \\
\textbf{Casting Time}: 2 Actions \\
\textbf{Range}: 9 meters \\
\textbf{Components}: V, S \\
\textbf{Duration}: Instantaneous \\
You create food and water in containers at range, enough to hold up to five humanoids or 2 mounts for 24 hours. Food is bland but nutritious, and rots after 24 hours if not consumed, as is water. \\
\textbf{For each Magic Critical Success you roll} in the Magic Check you create food for 3 other people or 1 mount.

\medskip \textbf{Create Beer}\index[Incantesimi]{Create Beer} \\
\textbf{School}: Summon \\
\textbf{Level}: 0, Rare \\
\textbf{Casting Time}: 2 Actions or more \\
\textbf{Range}: 9 meters \\
\textbf{Components}: V, S, M (brewer's yeast, malt, water) \\
\textbf{Duration}: 1 hour \\
Create 1 liter of beer. The quality and type of beer depends on the yeast, malt and water used.
The greater the casting time of the spell, the higher the alcohol content, with a casting time of two actions the alcohol content is 4.3, if 1 action is non-alcoholic, each action spent increases the alcohol content by 0.3 vol up to a maximum of 12.5 vol.
After an hour the beer vanishes, when consumed after an hour the alcoholic effects of the beer on the people who drank it also end. \\
\textbf{For each Magic Critical Success you roll} in the Magic Check you increase the duration by one liter or one hour.

\medskip \textbf{Create or Destroy Water}\index[Incantesimi]{Create or Destroy Water} \\
\textbf{School}: Water \\
\textbf{Level}: 1, Common \\
\textbf{Casting Time}: 2 Actions \\
\textbf{Range}: 9 meters \\
\textbf{Components}: V, S, M (a drop of water to create water or a few grains of salt to destroy it) \\
\textbf{Duration}: Instantaneous \\
Create or destroy water. \\
\textit{Creating Water}. You create up to 40 liters of clear water from your hands that spray up to 9 meters. Alternatively, the water falls as rain into a 30-foot-edged cube within range, extinguishing the flames exposed in the area. \\
The spell cannot be used on magical flames. \\
\textit{Destroy Water}. Destroy up to 40 liters of water in an open container at range. Alternatively, you can destroy the fog in a 30-foot-edged cube within range. \\
\textbf{For each Magic Critical Success obtained} in the Magic Check you create or destroy an additional 40 liters of water, or the size of the cube increases by 1 meter of edge in case of fog. \\
Water is drinkable and quenches thirst if drunk within one round of creation.

\medskip \textbf{Create Undead}\index[Incantesimi]{Create Undead} \\
\textbf{School}: Necromancy \\
\textbf{Level}: 6, Uncommon \\
\textbf{Casting Time}: 2 Actions \\
\textbf{Range}: 3 meters \\
\textbf{Components}: V, S, M (an earthen jar filled with graveyard earth, an earthen jar filled with brackish water, and a black onyx worth 50 gp for each corpse) \\
\textbf{Duration}: Instantaneous \\
You can only cast this spell at night. Choose up to three Medium or Small humanoid corpses at range. Each corpse becomes a ghoul under your control (the Storyteller owns the game stats of these creatures). During your round, with two Actions, you can mentally command any creature you animate with this spell, if the creature is within 100 feet of you (if you control multiple creatures, you can command all or just one at the same time by imparting the same command). You decide what action the creature will take and where it will move during its next round, or you can issue a generic command, such as guarding a specific room or corridor. If you don't give commands, the creatures will just defend themselves from hostile creatures. Once a command is received, the creature will continue to execute it until the task is complete. The creature is under your control for 24 hours, after which it will stop responding to your commands. To maintain control of the creature for another 24 hours, you must cast this spell on the creature before the current 24-hour period ends. This use of the spell reasserts your control over up to three creatures you animated with this spell, rather than animating new ones. \\
\textbf{If you get a Crit} in the Magic Check you can revive or reassert control over four ghouls. With two Critics you can animate or reassert control over five
ghoul or two ghasts or wights. With three Critics you can animate or reassert control over six ghouls, three ghast or wights, or two mummies.

\medskip \textbf{Creation}\index[Incantesimi]{Creation} \\
\textbf{School}: Illusion \\
\textbf{Level}: 5, Rare \\
\textbf{Casting Time}: 1 minute \\
\textbf{Range}: 9 meters \\
\textbf{Components}: V, S, M (a tiny piece of material of the same type of object you intend to create) \\
\textbf{Duration}: Special \\
You grab chunks of shadow matter from the Shadow plane to create, at range, non-living objects of plant matter: soft goods, rope, wood, or something similar. You can also use this spell to craft mineral items such as stone, crystal, or metal. The created object cannot be larger than a 1 meter edge cube, and the object must be of a shape and material that you have seen in the past. \\
The duration depends on the material of the object. If the item is made up of multiple materials, use the shortest duration.
\medskip
Material Table - Lasted
\medskip

\begin{tabularx}{0.45\textwidth}{lX}
\hline
Plant matter & 1 day \\
Stone or crystal & 12 hours \\
Precious Metals & 1 hour \\
Gems & 10 minutes \\
Adamantium or mithral & 1 minute \\
\end{tabularx}
\medskip

Using any material created by this spell as a material component of another spell will cause the new spell to fail. \\
\textbf{For each Magic Critical Success rolled} in the Magic Check the cube increases by 1 meter edge.

\medskip \textbf{Growth of Spikes}\index[Incantesimi]{Growth of Spikes} \\
\textbf{School}: Animals and Plants \\
\textbf{Level}: 2, Common \\
\textbf{Casting Time}: 2 Actions \\
\textbf{Range}: 45 meters \\
\textbf{Components}: V, S, M (seven sharp thorns or seven twigs, each pointed at one end) \\
\textbf{Duration}: 10 minutes \\
The ground within a radius of 6 meters centered on a point at range twists and generates very sharp spikes and thorns. For the duration, the area becomes difficult terrain. When a creature enters or moves within the area, it takes 2d4 points of damage for every 1 meter traveled.
The transformation of the terrain is so well camouflaged that it seems natural. Any creature that did not see the area when the spell was cast must make a Consciousness check against the spell's saving throw DC to recognize the danger posed by the terrain before entering it.

\medskip \textbf{Plant Growth}\index[Incantesimi]{Plant Growth} \\
\textbf{School}: Animals and Plants \\
\textbf{Level}: 3, Uncommon \\
\textbf{Casting Time}: 2 Actions or 8 hours \\
\textbf{Range}: 45 meters \\
\textbf{Components}: V, S \\
\textbf{Duration}: Instantaneous \\
This spell channels vitality into plants within a specific area. There are two possible uses for this spell, which confer immediate or long-term benefits. If you cast this spell using 1 action, you choose a point at range. All normal plants within a 100-foot radius centered on that spot become dense and thick. A creature that crosses the area quadruples the cost of its movement. \\
You may exclude one or more areas of any size within the spell's area from its effects. \\
If you cast this spell over the course of 8 hours, you nourish the earth. All plants within a 750 meter radius centered on a range point become super productive for 1 year. Vegetables produce double the normal amount of food at harvest time. \\
\textbf{If you get two Magic Critical Successes} you have the effect of the 8 hours of casting even if the spell was cast with 2 Actions.

\medskip \textbf{Invisible Cook}\index[Incantesimi]{Invisible Cook} \\
\textbf{School}: Summon \\
\textbf{Level}: 1, Common \\
\textbf{Casting Time}: 2 Actions \\
\textbf{Range}: 18 meters \\
\textbf{Components}: V, S, M (a wooden ladle and a few drops of olive oil, the food you want to cook) \\
\textbf{Duration}: 2 hours \\
This spell creates an almost invisible force only bounded by a slight aura (color of your choice) capable and competent in cooking. Together with the cook there is also a set of pots and pans as well as dishes and a small camp stove. \\
Based on the ingredients available or herbs and vegetables within 100 meters (the cook does not go hunting) the cook will cook the best of ingredients by preparing excellent food for up to 4 people. The spell does not create food or water, this must be available when the spell is cast. \\
Once the ingredients are available within two hours, the invisible cook will prepare the food. It is also possible to hasten the execution but to the detriment of quality. \\
None of the pots, dishes or fire can be used except by the invisible cook. \\
\textbf{If he rolls two Critical Magic Successes} the Cook is summoned with food needed to feed 2 people

\medskip \textbf{Cure Light Wounds}\index[Incantesimi]{Cure Light Wounds} \\
\textbf{School}: Water, Healing \\
\textbf{Level}: 1, Common \\
\textbf{Casting Time}: 2 Actions \\
\textbf{Range}: Contact \\
\textbf{Components}: V, S \\
\textbf{Duration}: Instantaneous \\
Your hand fills with positive healing energy, a creature you touch regains a number of hit points equal to 1d8 + ability modifier for spells. This spell when used on an undead, hit roll with a touch spell, damages it by the same amount. \\
This spell, unless explicitly stated otherwise, cannot be used on animals or plants. \\
\textbf{For each magical critical success rolled} on the Magic Check you heal an additional 1d6 HP. \\
If the caster and the healed creature are both Followers of the same Patron, the spell heals 1d8 more. \\
If the spellcaster and the healed creature are both devoted to the same Patron, each die value of 1,2,3 will count as 4.

\medskip \textbf{Cure Serious Wounds}\index[Incantesimi]{Cure Serious Wounds} \\
\textbf{School}: Healing \\
\textbf{Level}: 3, Uncommon \\
\textbf{Casting Time}: 2 Actions \\
\textbf{Range}: Contact \\
\textbf{Components}: V, S \\
\textbf{Duration}: Instantaneous \\
Your hand fills with healing positive energy, a creature you touch regains a number of hit points equal to 3d8 + 2 * ability modifier for spells. This spell when used on an undead, hit roll with a touch spell, damages it by the same amount. \\
This spell, unless explicitly stated otherwise, cannot be used on animals or plants. \\
\textbf{For each magical critical success you roll} on the Magic Check you heal an additional 1d6 HP. \\
If the caster and the healed creature are both Followers of the same Patron, the spell heals 1d8 more. \\
If the spellcaster and the healed creature are both devoted to the same Patron, each die value of 1,2,3 will count as 4.

\medskip \textbf{Cure Critical Wounds}\index[Incantesimi]{Cure Critical Wounds} \\
\textbf{School}: Healing \\
\textbf{Level}: 5, Uncommon \\
\textbf{Casting Time}: 2 Actions \\
\textbf{Range}: Contact \\
\textbf{Components}: V, S \\
\textbf{Duration}: Instantaneous \\
Your hand fills with positive healing energy, a creature you touch regains a number of hit points equal to 5d8 + 3 * ability modifier for spells. This spell when used on an undead, hit roll with touch spell, damages it by the same amount. \\
This spell, unless explicitly stated otherwise, cannot be used on animals or plants. \\
\textbf{For each magical critical success rolled} on the Magic Check you heal 1d6 HP more. \\
If the caster and the healed creature are both Followers of the same Patron, the spell heals 1d8 more. \\
If the spellcaster and the healed creature are both devoted to the same Patron, each die value of 1,2,3 will count as 4.

\medskip \textbf{Cure Mass Wounds}\index[Incantesimi]{Cure Mass Wounds} \\
\textbf{School}: Healing \\
\textbf{Rarity}: Uncommon \\
Like Heal Wounds but heal up to 4 creatures, within 30 feet. \\
You use three times more Magic Points than the selected Heal Wounds. \\
\textbf{For each magical critical success rolled} on the check you heal one more creature. \\
If the spellcaster and the healed creature are both Followers of the same Patron, the spell heals 1d8 more. \\
This spell, unless explicitly stated otherwise, cannot be used on animals or plants. \\
If the spellcaster and the healed creature are both devoted to the same Patron, each die value of 1,2,3 will count as 4.

\medskip \textbf{Fire Bolt}\index[Incantesimi]{Fire Bolt} \\
\textbf{School}: Fire \\
\textbf{Level}: 1, Common \\
\textbf{Casting Time}: 2 Actions \\
\textbf{Range}: 36 meters \\
\textbf{Components}: V, S \\
\textbf{Duration}: Instantaneous \\
You throw a fiery spark at a creature or object in range. Make a ranged spell attack against the target. If you hit, the target takes 1d10 points of fire damage. A flammable object affected by this spell catches fire if not worn or carried. \\
The spell's damage increases by 1d8 when you reach CM 5, CM 11, and CM 17. \\
\textbf{For every two magical critical successes rolled} in the Magic Check you cast one more spark.

\medskip \textbf{Tracer Bolt}\index[Incantesimi]{Tracer Bolt} \\
\textbf{School}: Invocation \\
\textbf{Level}: 1, Uncommon \\
\textbf{Casting Time}: 2 Actions \\
\textbf{Range}: 36 meters \\
\textbf{Components}: V, S \\
\textbf{Duration}: 1 round \\
A flash of light travels to a creature in range of your choice. Make a ranged spell attack against the target. If you hit, the target takes 4d6 points of Light damage, and the next hit roll against them before your
next round has + 1d6 TC, thanks to the mystical dim light that will continue to shine around the target until then. \\
\textbf{For each magical critical success rolled} on the Magic Check the damage is increased by 1d6.

\medskip \textbf{Irresistible Dance}\index[Incantesimi]{Irresistible Dance} \\
\textbf{School}: Enchantment \\
\textbf{Level}: 8, Legendary \\
\textbf{Casting Time}: 2 Actions \\
\textbf{Range}: 9 meters \\
\textbf{Components}: V \\
\textbf{Duration}: 1 minute \\
Choose a creature in range and that you can see. The target begins a comical ballet on the spot: shaking his legs, stamping his feet and hopping for the duration. Creatures that cannot be charmed are immune to this spell. \\
A dancing creature must use 2 move actions to dance without leaving its space and has -1d6 on Reflex saving throws and attack rolls. While the target is subject to this spell, other creatures have + 1d6 attack rolls against it. By spending 2 Actions, the dancing creature can make a new Will save to regain control of itself. If he succeeds, the spell ends. While dancing he considers himself Distracted \\
\textbf{If you roll 2 Magic Critical Success} the duration is increased by 1 hour

\medskip \textbf{Magic Bolt}\index[Incantesimi]{Magic Bolt} \\
\textbf{School}: Universal \\
\textbf{Level}: 1, Common \\
\textbf{Casting Time}: 2 Actions \\
\textbf{Range}: 36 meters \\
\textbf{Components}: V, S \\
\textbf{Duration}: 1 Turn, Concentration \\
You create a luminous dart of magical force. Throwing one or more darts already summoned costs 1 Action. The dart hits a creature in range that you can see, chosen by you. A dart deals 1d4 + 1 force damage to its target and you can direct them to hit one or more creatures. \\
You create an additional bolt when you reach CM 3, CM 5, CM 7 and CM 9. The damage increases by 2 each time you have Adept of Magic on the Universal List up to a maximum of 5 increases. \\
\textbf{For each magical critical success rolled} in the Magic Check the spell creates an additional bolt.

\medskip \textbf{Hidden Blast} \index{Trick - Hidden Blast} \\
\textbf{School}: Invocation \\
\textbf{Level}: 1, Common \\
\textbf{Casting Time}: 1 Actions \\
\textbf{Range}: 36 meters \\
\textbf{Components}: V, S \\
\textbf{Duration}: Instantaneous \\
A crackling beam of energy moves towards a creature in range. Make a ranged spell attack against the target. If you hit, the target takes 1d8 points of force damage. \\
The damage of the spell increases by 1d8 when you reach CM 5, CM 11 and CM 17 but it costs 2 Actions to cast it boosted and 2 Magic Points, it is also necessary to have taken Adept of Magic in this Spell List a number of times equal to the enhancements that you want to apply. \\
\textbf{Every 2 magical critical successes rolled} in the Magic Check you create another beam of energy.

\medskip \textbf{Wish}\index[Incantesimi]{Wish} \\
\textbf{School}: Summon \\
\textbf{Level}: 9, Uncommon \\
\textbf{Casting Time}: 2 Actions \\
\textbf{Duration}: Self \\
\textbf{Components}: V \\
\textbf{Duration}: Instantaneous \\
Desire is the most powerful spell a mortal creature can cast. By simply speaking out loud, you can change the very foundations of reality according to your needs. \\
The basic use of this spell is to reproduce the effect of any other spell with a level of 8 or less. You don't have to meet any of the spell's requirements, including expensive material components. The spell simply takes effect. \\
Alternatively, you can create any of the following effects of your choice: \\
- You create an item with a maximum value of 25,000 gp, which is not a magical item. The object cannot be larger than 90 meters in any dimension, and appears in an unoccupied space on the ground. \\
- Allow up to twenty creatures you can see to recover all hit points, and end all effects on them described by the greater restore spell. \\
- Grant up to ten creatures you can see resistance to a damage type of your choice. \\
- Grant up to ten creatures you can see immunity to a single spell or other magical effect for 8 hours. For example, you could make you and all of your companions immune to the lich's life-draining attack. \\
- Cancel any recent event by forcing you to reroll any rolls made in the last round (including your last round). Reality reshapes itself to accommodate the new result. You can make the new roll have + 2d6 or -2d6, you can choose whether to use the original roll or the new roll. You may also be able to achieve more than the goals in the examples above. \\

\medskip
Define your wishes as much as possible to the Storyteller. The Storyteller has a lot of room for maneuver in deciding what happens in these cases; the greater the desire, the greater the likelihood of something going wrong. The spell may simply fail, the desired effect only partially manifesting itself, or you may suffer unforeseen consequences, depending on how you uttered the wish. The stress of casting this spell to create any effect other than reproducing another spell weakens you. \\
After you handle the stress, each time you cast a spell, until you finish a night's rest, you will take 1d10 Void damage per level / 2 of the spell. This damage cannot be reduced or decreased in any way. Also, your Constitution drops to -3, if it's not already -3 or less, for 2d4 days. \\
For each day you spend resting and doing nothing but light activity, your remaining recovery time decreases by 2 days.
Roll 1d100, if you do from 1 to 33 \% you will never be able to cast wish again due to stress, 34 \% - 66 \% 5 years old, 67 \% -99 \% no particular effect happens, 100 \% recover immediately from the stress of the launch. \\
\textbf{On 2 Magic Critical Successes rolled} You suffer no side effects from casting Wish.

\medskip \textbf{Phantom Steed}\index[Incantesimi]{Phantom Steed} \\
\textbf{School}: Illusion \\
\textbf{Level}: 3, Common \\
\textbf{Casting Time}: 1 minute \\
\textbf{Range}: 9 meters \\
\textbf{Components}: V, S \\
\textbf{Duration}: 1 hour \\
A near-real creature similar to a large horse appears on the ground in an unoccupied space of your choice and at range. You decide the creature's appearance, and it appears equipped with a saddle, bit, and bridle. Any equipment created by the spell vanishes in a cloud of smoke if brought more than 10 feet away from the steed. For the duration, you or a creature of your choice can ride the steed. The creature uses the racehorse's stats, except it has a speed of 30 meters and can travel 15 kilometers in an hour, or 20 kilometers at a fast pace. When the spell ends, the steed gradually begins to fade, giving the rider 1 minute to dismount. The spell ends if you use an action to interrupt it or if the steed takes damage. \\
\textbf{For each Magic Critical Success rolled} in the Magic Check the duration is doubled.

\medskip \textbf{Floating Disc}\index[Incantesimi]{Floating Disc} \\
\textbf{School}: Summon \\
\textbf{Level}: 1, Common \\
\textbf{Casting Time}: 2 Actions \\
\textbf{Range}: 9 meters \\
\textbf{Components}: V, S, M (a drop of mercury) \\
\textbf{Duration}: 1 hour \\
This spell creates a perfectly circular, horizontal plane of force 1 meter in diameter and 2.5 centimeters thick that floats 1 meter above the ground, in an unoccupied space of your choice at range and that you can see. The disc remains active for the duration, and can support 250 pounds. If a heavier weight is placed on it, the spell ends and everything on it falls to the ground. As long as you are within 6 meters of it, the puck is motionless. If you move more than 6 meters away from it, the puck follows you so that it always stays 6 meters away from you. It can move through uneven terrain, up and down stairs, slopes and the like, but cannot overcome changes in altitude of 3 meters or more. For example, the disc cannot cross a 3 meter deep moat, nor could it leave the moat if it were created at the bottom of it. The disc can be grabbed by the caster and moved manually. If you move more than 100 feet away from the puck (usually because it cannot get around an obstacle while following you) the spell ends. \\
\textbf{For each Magic Critical Success rolled} in the Magic Check the duration is doubled.

\medskip \textbf{Disintegration}\index[Incantesimi]{Disintegration} \\
\textbf{School}: Transmutation \\
\textbf{Level}: 6, Uncommon \\
\textbf{Casting Time}: 2 Actions \\
\textbf{Range}: 18 meters \\
\textbf{Components}: V, S, M (a magnet and a pinch of dust) \\
\textbf{Duration}: Instantaneous \\
A thin green beam emanates from your finger pointing at a target in range that you can see. The target can be a creature, object, or creation of magical force, such as a wall created by a wall of force. A creature targeted by this spell must make a Reflex saving throw. The target takes 10d6 + 40 force damage on a failed saving throw. If this damage reduces the target to 0 hit points, the target is disintegrated. A disintegrated creature and everything it wears and carries, except magical items, is reduced to a pile of fine gray dust. The creature can only be brought back to life through the intervention of a Patron \\
This spell automatically disintegrates non-magical items or a creation of magical force of large or smaller size. If the target is a nonmagical object or a creation of force of Huge or larger size, this spell disintegrates a portion of it equal to a 10-foot-edged cube. Magic items ignore this spell. \\
\textbf{For each Magic Critical Success rolled} on the Magic Check, damage is increased by 2d6.

\medskip \textbf{Dissolve Good and Evil}\index[Incantesimi]{Dissolve Good and Evil} \\
\textbf{School}: Abjuration \\
\textbf{Level}: 5, Rare \\
\textbf{Casting Time}: 2 Actions \\
\textbf{Duration}: Self \\
\textbf{Components}: V, S, M (Holy Water or silver and powdered iron) \\
\textbf{Duration}: Concentration, 1 minute \\
A luminous energy surrounds you and protects you from fairies, undead, and creatures originating from places beyond the Material Plane. For the duration, celestials, elementals, faeries, demons, and undead have -1d6 attack rolls against you. You can end the spell early by using one of the following special functions. \\
\textit{Break Charm}. With an action, you can make contact with a creature fascinated, frightened, or possessed by a celestial, elemental, fairy, demon, or undead. The creature you are in contact with is no longer fascinated, frightened, or possessed by these creatures. \\
\textit{Leave}. As an action, make a melee attack against a celestial, elemental, fairy, demon, or undead in your range. If you hit it, you can try to send the creature back to its home plane. The creature must succeed on a Will saving throw or be sent back to its home plane (if not already there). If not on their home plane, the undead are sent back to the Shadow World and the fairies to the First World.

\medskip \textbf{Dispel magic}\index[Incantesimi]{dispel magic} \hypertarget{dissolvimagie}{} \\
\textbf{School}: Abjuration \\
\textbf{Level}: 3, Common \\
\textbf{Casting Time}: 2 Actions \\
\textbf{Range}: 36 meters \\
\textbf{Components}: V, S \\
\textbf{Duration}: Instantaneous \\
Choose a creature, object, or magical effect at range. Any level 3 or lower spell on the target ends. If cast on an object that manifests a spell it is deactivated for 10 minutes. \\
\textbf{For each Magic Critical Success obtained} in the Magic Check the dispellable level increases by 1. In case of 3 critical successes you can permanently disperse an effect on a non-artifact object.

\medskip \textbf{Dispel magic advanced}\index[Incantesimi]{dispel magic advanced} \hypertarget{dissolvimagie}{} \\
\textbf{School}: Abjuration \\
\textbf{Level}: 5, Rare \\
\textbf{Casting Time}: 3 Actions \\
\textbf{Range}: 36 meters \\
\textbf{Components}: V, S, M (diamond dust's worth 200 gp) \\
\textbf{Duration}: Instantaneous \\
Choose a creature, object, or magical effect at range. Any level 5 or lower spell on the target ends. If cast on an object that manifests a spell it is deactivated for 10 minutes. \\
\textbf{For each Magic Critical Success obtained} in the Magic Check the dispellable level increases by 2.

\medskip \textbf{Finger}\index[Incantesimi]{Finger} \\
\textbf{School}: Enchantment \\
\textbf{Level}: 0, Uncommon \\
\textbf{Casting Time}: 1 Immediate Action \\
\textbf{Range}: 18 meters \\
\textbf{Components}: S \\
\textbf{Duration}: 3 rounds \\
Finger (or raspberry or umbrella gesture) to the opponent who must be able to see (or hear) \\
This must make a Will saving throw, if successful, nothing happens.
If he fails the save by 5 or more he is humiliated, for the next 3 rounds he has a 1d6 penalty on attack rolls, saves, and Proficiency checks. \\
If he fails the save by 3 or 4, he is mortified, for the next 3 rounds he has a 1d6 penalty on attack and proficiency rolls. \\
If he fails the save by 2 or 1, he is penalized, for the next 3 rounds he has a 2 penalty on attack rolls. \\
\textbf{For each magical critical success you roll} in the Magic Check, you can affect another creature that can see the finger.

\medskip \textbf{Finger of Death}\index[Incantesimi]{Finger of Death} \\
\textbf{School}: Necromancy \\
\textbf{Level}: 6, Common \\
\textbf{Casting Time}: 2 Actions \\
\textbf{Range}: 18 meters \\
\textbf{Components}: V, S \\
\textbf{Duration}: Instantaneous \\
You send a burst of negative energy to a creature in range that you can see, causing it deep pain. The target must make a Fortitude saving throw. The target takes 7d8 + 30 Void damage on a failed save, or half that damage if it succeeds. \\
A humanoid killed by this spell revives as a zombie under your permanent command at the start of your next round, and will follow your verbal orders to the best of its ability. \\
\textbf{Saving Throw Success / Critical Failure}: On critical failure the damage doubles, on critical success the damage is further halved

\medskip \textbf{Divination}\index[Incantesimi]{Divination} \\
\textbf{School}: Divination \\
\textbf{Level}: 6, Rare \\
\textbf{Casting Time}: 2 Actions \\
\textbf{Duration}: Self \\
\textbf{Components}: V, S, M (incense and a sacrificial offering appropriate to your religion, whose total value is 25 gp, which will be consumed by the spell) \\
\textbf{Duration}: Instantaneous \\
Your magic and a votive offering put you in communication with a Patron or the servant of a Patron. You can ask them a single question about a specific goal, event or activity that needs to occur within 7 days. The Storyteller gives a truthful answer. The reply could be a short sentence, a cryptic rhyme, or an omen. \\
The spell ignores every possible circumstance that could alter the outcome, such as the casting of additional spells or the loss or arrival of an ally. \\
If you cast the spell two or more times before the long day is over, there is a cumulative 25 \% chance that for every cast after the first you will get an erroneous reading. The Storyteller makes this roll in secret.

\medskip \textbf{Dominate Beasts}\index[Incantesimi]{Dominate Beasts} \\
\textbf{School}: Charm, Animals and Plants \\
\textbf{Level}: 4, Very Rare - Common \\
\textbf{Casting Time}: 2 Actions \\
\textbf{Range}: 18 meters \\
\textbf{Components}: V, S \\
\textbf{Duration}: Concentration, maximum 1 minute \\
You try to charm a ranged beast that you can see. She must succeed on a Will save or be fascinated for the duration, receiving + 1d6 to the roll if you or your allies are fighting her. \\
While the beast is fascinated, as long as you two are on the same plane of existence, you maintain a telepathic connection with it. You can use this telepathic link to send commands to the creature while conscious (requires 1 action), which it will obey at its best. You can specify a simple and generic course of action, such as "Attack that creature", "Run over there", or "Get that item". If the creature completes the order and receives no further guidance from you, it will defend itself and preserve itself to the best of its ability. \\
You can take 2 of your actions to take complete and precise control of the target. Until the end of your next round, the target will only take the actions you decide, and will do nothing that you do not allow them to do. During this time, you can also have the target use a Reaction Action, but this requires the use of your reaction. \\
Each time the target takes damage, make a new Will saving throw against the spell. On a successful save, the spell ends. \\
\textbf{For each Magic Critical Success obtained} in the Magic Check the duration is doubled to a maximum of 8 hours.

\medskip \textbf{Dominate Monster}\index[Incantesimi]{Dominate Monster} \\
\textbf{School}: Enchantment \\
\textbf{Level}: 8, Common \\
\textbf{Casting Time}: 2 Actions \\
\textbf{Range}: 18 meters \\
\textbf{Components}: V, S \\
\textbf{Duration}: Concentration, maximum 1 hour \\
You try to charm a creature in range that you can see. She must succeed on a Will save or be fascinated for the duration, receiving + 1d6 to the roll if you or your allies are fighting her. \\
While the creature is fascinated, as long as you two are on the same plane of existence, you maintain a telepathic connection with it. You can use this telepathic link to send commands to the creature while conscious (requires 1 action), which it will obey at its best. You can specify a simple and generic course of action, such as "Attack that creature", "Run over there", or "Get that item". If the creature completes the order and receives no further guidance from you, it will defend itself and preserve itself to the best of its ability. \\
You can take two of your Actions to take total and precise control of the target. Until the end of your next round, the creature will only take the actions you decide, and it won't do anything you don't allow it to do. During this time, you can also have the creature use a Reaction Action, but that requires the use of your reaction. Each time the target takes damage, make a new Will saving throw against the spell. On a successful saving throw, the spell ends. \\
\textbf{For each Magic Critical Success obtained} in the Magic Check the duration is doubled to a maximum of 8 hours.

\medskip \textbf{Dominate People}\index[Incantesimi]{Dominate People} \\
\textbf{School}: Enchantment \\
\textbf{Level}: 5, Common \\
\textbf{Casting Time}: 2 Actions \\
\textbf{Range}: 18 meters \\
\textbf{Components}: V, S \\
\textbf{Duration}: Concentration, maximum 1 minute \\
You try to charm a range humanoid that you can see. It must succeed on a Will saving throw or be fascinated for the duration, receiving + 1d6 to the roll if you or your allies are fighting it. \\
While the target is fascinated, as long as you two are on the same plane of existence, maintain a telepathic connection with it. You can use this telepathic link to send commands to the target while conscious (requires 1 action), which it will obey at its best. You can specify a simple and generic course of action, such as "Attack that creature", "Run over there", or "Get that item". If the target completes the order and receives no further guidance from you, he will defend himself and preserve himself to the best of his ability. \\
You can take 2 Actions to take full and precise control of the target. Until the end of your next round, the target will only take the actions you decide, and will do nothing that you do not allow them to do. During this time, you can also have the target use a Reaction Action, but that requires the use of your reaction. Each time the target takes damage, make a new Will saving throw against the spell. On a successful save, the spell ends. \\
\textbf{For each Magic Critical Success obtained} in the Magic Check the duration is doubled to a maximum of 8 hours.

\medskip \textbf{Heroism}\index[Incantesimi]{Heroism} \\
\textbf{School}: Enchantment \\
\textbf{Level}: 1, Uncommon \\
\textbf{Casting Time}: 2 Actions \\
\textbf{Range}: Contact \\
\textbf{Components}: V, S \\
\textbf{Duration}: 1 minute \\
A consenting creature you are in contact with is infused with courage. Until the spell ends, the creature is immune to being frightened, and at the start of each of its rounds, it gains temporary hit points equal to your Intelligence value or spell modifier. When the spell ends, the target loses all remaining temporary hit points from this spell. \\
\textbf{For each Magic Critical Success you roll} in the Magic Check, you can affect another creature.

\medskip \textbf{Exile}\index[Incantesimi]{Exile} \\
\textbf{School}: Abjuration \\
\textbf{Level}: 4, Common \\
\textbf{Casting Time}: 2 Actions \\
\textbf{Range}: 18 meters \\
\textbf{Components}: V, S, M (an object despised by the target) \\
\textbf{Duration}: 1 minute \\
You try to send a creature in range and that you can see in another plane of existence. The target must succeed on a Will saving throw or be exiled. If the target is native to the plane of existence you are in, exile the target to a harmless demiplan. While there, the target is incapacitated. The target remains there until the spell ends, when it reappears in the space it left off or the nearest unoccupied space, if its original space is now occupied. If the target is native to a different plane of existence than you are on, the target vanishes with a soft pop, returning to its home plane. If the spell ends before 1 minute has elapsed, the target reappears in the space it had left or the nearest unoccupied space, if its original space is occupied. \\
\textbf{For each magical critical success you roll} in the Magic Check you can affect another creature.

\medskip \textbf{Solar Blast}\index[Incantesimi]{Solar Blast} \\
\textbf{School}: Invocation \\
\textbf{Level}: 8, Rare \\
\textbf{Casting Time}: 2 Actions \\
\textbf{Range}: 45 meters \\
\textbf{Components}: V, S, M (fire and a piece of sun stone) \\
\textbf{Duration}: Instantaneous \\
Intense sunlight illuminates in a radius of 18 meters centered on a range point, chosen by you. All creatures within the light must make a Fortitude saving throw. On a failed save, a creature takes 12d6 points of light damage and is blinded for 1 minute. If she succeeds, she takes half the damage and is not blinded by the spell. Undead and ooze have -2d6 on this saving throw. A creature blinded by this spell makes another Fortitude saving throw at the end of each of its rounds. If she succeeds at her saving throw, she is no longer blinded. \\
In its area, this spell dispels any darkness generated by a spell. \\
\textbf{For each magical critical success rolled} on the Magic Check the damage is increased by 2d6.

\medskip \textbf{Enrapture}\index[Incantesimi]{Enrapture} \\
\textbf{School}: Enchantment \\
\textbf{Level}: 2, Common \\
\textbf{Casting Time}: 2 Actions \\
\textbf{Duration}: Self \\
\textbf{Components}: V, S \\
\textbf{Duration}: 1 minute \\
You interweave a series of misleading words, causing creatures of your choice within range that you can see and hear you to make a Will saving throw. Any creature that can't be fascinated automatically succeeds at its saving throw, and if you or your companions are battling a creature, it has + 1d6 on the saving throw. If the saving throw fails, the target has -1d6 on Awareness checks made to sense any creature other than you until the spell ends or until the target can no longer hear you.
The spell ends if you are incapacitated or can no longer speak.

\medskip \textbf{Summon Animals}\index[Incantesimi]{Summon Animals} \\
\textbf{School}: Animals and Plants \\
\textbf{Level}: 3, Uncommon \\
\textbf{Casting Time}: 2 Actions \\
\textbf{Range}: 18 meters \\
\textbf{Components}: V, S \\
\textbf{Duration}: 1 hour \\
You summon fairy spirits that take on the appearance of beasts and appear in unoccupied spaces at range that you can see. Choose one of the following options to determine what appears: \\

- A beast of challenge grade 2 or lower \\
- Two beasts of challenge grade 1 or lower \\
- Four beasts of challenge grade 1/2 or lower \\
- Eight beasts of challenge grade 1/4 or lower \\
\medskip

Each beast is also considered a fairy, and disappears when it drops to 0 hit points or when the spell ends. \\
Summoned creatures are friendly towards you and your companions. Roll initiative for creatures summoned as a group, which acts during its own round. They obey any verbal command that is given to them (without needing you to perform any action). If you do not give commands to the beasts, they will defend themselves against hostile creatures, but take no further action. \\
\textbf{For each Magic Critical Success rolled} in the Magic Check, two more beasts will appear.

\medskip \textbf{Summon Mount}\index[Incantesimi]{Summon Mount} \\
\textbf{School}: Animals and Plants \\
\textbf{Level}: 2, Common \\
\textbf{Casting Time}: 10 minutes \\
\textbf{Range}: 9 meters \\
\textbf{Components}: V, S \\
\textbf{Duration}: 1 hour \\
You summon a spirit that takes the form of an unusually intelligent, strong, and loyal mount, establishing a lasting bond with it. Appearing in a ranged, unoccupied space, the steed takes the form of your choice, such as that of a warhorse, pony, camel, moose, or hound (the Storyteller may give you the ability to summon as steeds as well. other types of animals). The steed has the stats of the chosen form, although it is a celestial, fairy, or demon type (your choice) instead of its normal type. Also, if your steed has Intelligence -3 or less, its Intelligence becomes -2, and you gain the ability to understand a language of your choice from those you are able to speak. Your steed serves as a mount, both in and out of combat, and you have an instinctive bond with it, allowing you to fight as a whole. \\
When the steed drops to 0 hit points, it disappears, leaving no physical form behind. you can dismiss the steed at any time with an action, making it disappear. Either way, casting this spell again summons the same steed, restored to its maximum hit points. \\
You cannot have more than one steed tied by this spell at a time. With an action, you can free the steed from this bond at any time, causing it to disappear. \\
\textbf{For each Magic Critical Success obtained} in the Magic Check the spell lasts an extra hour.

\medskip \textbf{Summon Elemental}\index[Incantesimi]{Summon Elemental} \\
\textbf{School}: Air, Water, Earth, Fire \\
\textbf{Level}: 5, Rare \\
\textbf{Casting Time}: 1 minute \\
\textbf{Range}: 27 meters \\
\textbf{Components}: V, S, M (incense burned for air, malleable clay for earth, sulfur and phosphorus for fire, or water and sand for water) \\
\textbf{Duration}: 1 hour \\
You summon an elemental servant. Choose an area with a range made up of water, air, fire or earth and that fills a cube with a 10-foot edge. An elemental of challenge rating 5 or lower appropriate to your chosen area appears in an unoccupied space within 10 feet of it. The elemental disappears when it drops to 0 hit points or the spell ends. \\
The elemental is friendly towards you and your companions for the duration of the spell. Roll initiative for the elemental, which acts during its own round. He obeys any verbal command that is given to him (if the command is complex it consumes actions). If you do not give commands to the elemental, it will defend itself against hostile creatures, but it will take no further action. \\
Each List of Spells can only summon its own specific Elemental \\
\textbf{For two Magical Critical Success gained} in the Magic Check the challenge rating of the summoned elemental increases by 1

\medskip \textbf{Summon Minor Elementals}\index[Incantesimi]{Summon Minor Elementals} \\
\textbf{School}: Air, Water, Earth, Fire \\
\textbf{Level}: 4, Uncommon \\
\textbf{Casting Time}: 1 minute \\
\textbf{Range}: 27 meters \\
\textbf{Components}: V, S \\
\textbf{Duration}: 1 hour \\
Summons elementals that will spawn in unoccupied spaces at range and that you can see. Choose one of the following options to decide what appears: \\

- An elemental of challenge grade 2 or less \\
- Two elementals of challenge grade 1 or less \\
- Four elementals of challenge grade 1/2 or less \\
- Eight elementals of challenge grade 1/4 or less \\

\medskip
A summoned elemental disappears when it drops to 0 hit points or the spell ends. A summoned elemental is friendly towards you and your companions. Roll initiative for the elementals summoned as a group, which acts during its own round. They obey any verbal command that is given to them (if the command is complex it consumes actions). If you do not give commands to the elementals, they will defend themselves from hostile creatures, but take no further action. \\
Each List of Spells can only summon its own specific Elemental \\
\textbf{For each Magic Critical Success rolled} in the Magic Check, two more Elementals will appear.

\medskip \textbf{Instant Summons}\index[Incantesimi]{Instant Summons} \\
\textbf{School}: Summon \\
\textbf{Level}: 6, Rare \\
\textbf{Casting Time}: 1 minute \\
\textbf{Range}: Contact \\
\textbf{Components}: V, S, M (a sapphire worth 1000 gp) \\
\textbf{Duration}: Until dissolved \\
You come into contact with an object weighing 5 kilos or less and whose largest size does not exceed 180 centimeters. The spell leaves a mark on the surface of the object and invisibly engraves its name on the sapphire used as a material component. Each time you cast this spell, you must use a different sapphire. \\
At any later time, you can use 2 Actions to say the item's name and shatter the sapphire. The object appears instantly in your hand regardless of the physical or planar distance between you, and the spell ends. \\
If another creature is holding or carrying the item, shattering the sapphire will not carry the item away from you, but instead you will learn who the creature is in possession of it and roughly where it is at the moment. \\
Dispel magic, or a similar effect successfully applied to sapphire, ends the spell's effect.

\medskip \textbf{Fabricate}\index[Incantesimi]{Fabricate} \\
\textbf{School}: Transmutation \\
\textbf{Level}: 4, Common \\
\textbf{Casting Time}: 10 minutes \\
\textbf{Range}: 36 meters \\
\textbf{Components}: V, S \\
\textbf{Duration}: Instantaneous \\
Convert raw materials into finished products of the same material. For example, you can make a small wooden bridge from a pile of trees, a rope from a pile of hemp, and clothes from linen or wool. Choose the raw materials that you can see at range. You can craft a Large or smaller object (contained in a 3-meter-edge cube, or eight connected 1-meter-edge cubes) given a sufficient amount of raw materials. If you are working with metal, stone, or other mineral substances, the crafted object cannot be larger than Medium (contained in a single 1 meter edge cube). The quality of the items created by this spell is commensurate with the quality of the raw materials. \\
You cannot create or transmute magical creatures or objects through this spell. You also can't use it to craft items that normally require a high level of craftsmanship, such as jewelry, weapons, glass, or armor, unless you are proficient with the type of craftsman tools used to craft these items. In case of criticism in the Magic Check you can process more volumes or produce with higher quality.

\medskip \textbf{Beacon of Hope}\index[Incantesimi]{Beacon of Hope} \\
\textbf{School}: Abjuration \\
\textbf{Level}: 3, Rare \\
\textbf{Casting Time}: 2 Actions \\
\textbf{Range}: 9 meters \\
\textbf{Components}: V, S \\
\textbf{Duration}: 1 minute, Concentration \\
This spell confers hope and vitality. Choose up to 6 creatures at range. For the duration, each target has +2 on Will saving throws and each heal die gains +1 hit point healed. \\
\textbf{If you get 2 Magic Critical Success and also have the Heal List} each round the chosen creatures regain 1 hit point.

\medskip \textbf{Fatal}\index[Incantesimi]{Fatal} \\
\textbf{School}: Illusion \\
\textbf{Level}: 9, Rare \\
\textbf{Casting Time}: 2 Actions \\
\textbf{Range}: 36 meters \\
\textbf{Components}: V, S \\
\textbf{Duration}: Concentration, maximum 1 minute \\
By tapping into the innermost fears of a group of creatures, you create illusory creatures in their minds, visible only to them. Each creature in a 30-foot-radius sphere centered on a point of your choice in range must make a Will saving throw. If the saving throw fails, the creature becomes frightened for the duration. The illusion sinks into the creature's innermost fears, manifesting its worst nightmares as a relentless threat. At the end of each round of the frightened creature, it must succeed at a Will saving throw or take 4d10 points of damage. If the saving throw is successful, the spell ends for that creature.

\medskip \textbf{Divine Favor}\index[Incantesimi]{Divine Favor} \\
\textbf{School}: Invocation \\
\textbf{Level}: 1, Uncommon \\
\textbf{Casting Time}: 1 Immediate Action \\
\textbf{Duration}: Self \\
\textbf{Components}: V, S \\
\textbf{Duration}: 1 minute \\
Your prayers empower you and your weapon. Until the spell ends, when it hits, your weapon deals an additional 1d4 points of Light damage. \\
\textbf{For each Magic Critical Success rolled} in the Magic Check your weapon deals +1 additional Light damage.

\medskip \textbf{Wound}\index[Incantesimi]{Wound} \\
\textbf{School}: Necromancy \\
\textbf{Level}: 6, Uncommon \\
\textbf{Casting Time}: 2 Actions \\
\textbf{Range}: 18 meters \\
\textbf{Components}: V, S \\
\textbf{Duration}: Instantaneous \\
Unleash a virulent disease on a creature in range that you can see. The target must make a Fortitude saving throw. The target takes 14d6 Void damage on a failed save, or half that damage if it succeeds. the damage cannot reduce the target's hit points below 1. If the target fails its saving throw, its maximum hit points are reduced for 1 hour by an amount equal to the Void damage taken. Any effect that removes a disease allows the character's maximum hit points to return to normal before that time elapses.

\medskip \textbf{Stop Time}\index[Incantesimi]{Stop Time} \\
\textbf{School}: Transmutation \\
\textbf{Level}: 9, Rare \\
\textbf{Casting Time}: 2 Actions \\
\textbf{Duration}: Self \\
\textbf{Components}: V \\
\textbf{Duration}: Instantaneous \\
You briefly stop the flow of time for everyone but you. Time doesn't run for other creatures, while you take 1d4 + 1 rounds in a row, during which time you can take actions and move as usual. This spell ends if any of the actions you use during this period, or any effect you create during this period, affects a creature other than yourself or an object worn or carried by someone other than you. Also, the spell ends if you move to a place more than 300 meters away from where you cast it. \\
\textbf{For each magical critical success rolled} in the Magic Check the duration is increased by 1 round. In the case of two critical spell successes, you can exclude another creature from the time stop.

\medskip \textbf{Perennial Flame}\index[Incantesimi]{Perennial Flame} \\
\textbf{School}: Universal \\
\textbf{Level}: 2, Legendary \\
\textbf{Casting Time}: 2 Actions \\
\textbf{Range}: Contact \\
\textbf{Components}: V, S, M (ruby dust worth 75 gp, which the spell consumes) \\
\textbf{Duration}: Until dissolved \\
A brightness similar to that produced by a torch emanates from an object you are in contact with. The effect looks like a normal flame, but it does not produce heat or need oxygen. A perpetual flame can be concealed or hidden but it cannot be muffled or extinguished.

\medskip \textbf{Sacred Flame} \index{Trick - Sacred Flame} \\
\textbf{School}: Universal \\
\textbf{Level}: 0, Common \\
\textbf{Casting Time}: 1 Action \\
\textbf{Range}: 18 meters \\
\textbf{Components}: V, S \\
\textbf{Duration}: Instantaneous \\
A torch-like brightness descends upon a creature in range that you can see. The target must succeed on a Reflex saving throw or take 1d8 points of Light damage. The target does not receive the cover benefit for this saving throw. \\
The damage of the spell increases by 1d8 when the sum of the Traits in common with the Patron reaches 5, 11 and 17, but it costs 2 Actions to cast it empowered and 2 Magic Points. \\
\textbf{For every two Magic Critical Success rolled} in the Magic Check, one more flame descends that must hit a different target within range.

\medskip \textbf{Acid Splash} \index{Trick - Acid Splash} \\
\textbf{School}: Summon \\
\textbf{Level}: 0, Common \\
\textbf{Casting Time}: 1 Action \\
\textbf{Range}: 18 meters \\
\textbf{Components}: V, S \\
\textbf{Duration}: Instantaneous \\
Throw a bubble of acid. Choose a creature in range or two creatures in range that are within 1 meter of each other. The target must succeed on a Reflex saving throw or take 1d6 points of acid damage. \\
The damage of the spell increases by 1d8 when you reach CM 5, CM 11 and CM 17, but it costs 2 Actions to cast it boosted and 2 Magic Points, it is also necessary to have taken Adept of Magic in this Spell List a number of times equal to the upgrades that you want to apply. \\
\textbf{For every two magical critical successes rolled} in the Magic Check you throw an extra bubble of acid into range.

\medskip \textbf{Gust of Wind}\index[Incantesimi]{Gust of Wind} \\
\textbf{School}: Aria \\
\textbf{Level}: 2, Common \\
\textbf{Casting Time}: 2 Actions \\
\textbf{Range}: Personnel (line of 18 meters) \\
\textbf{Components}: V, S, M (a legume seed) \\
\textbf{Duration}: Concentration, maximum 1 minute \\
A line of strong wind 60 feet long and 10 feet wide explodes starting at you in a direction of your choice for the duration of the spell. Each creature that begins its round inside the line must succeed at a Fortitude save or be pushed 4 meters away from you, following the direction of the line. \\
Any creature on the line must spend double the movement to get close to you. \\
The gust disperses gases or vapors, extinguishes candles, torches, and similar unprotected flames in the area. Protected flames, such as those from lanterns, flare up, and have a 50 \% chance of being extinguished. As 1 Action during each of your rounds, before the spell ends, you can change the direction the line projects from you. \\
A thrown weapon that goes through a gust of wind has 50 \% miss.

\medskip \textbf{Melting into Stone}\index[Incantesimi]{Melting into Stone} \\
\textbf{School}: Earth \\
\textbf{Level}: 3, Common \\
\textbf{Casting Time}: 2 Actions \\
\textbf{Range}: Contact \\
\textbf{Components}: V, S \\
\textbf{Duration}: 8 hours \\
You enter an object or surface of stone large enough to hold your entire body, merging with the stone along with all the equipment you carry for the duration. Using your movement, you enter the stone at a point you are in contact with. There is nothing left of your presence that remains visible or otherwise can be detected by non-magical senses. While you are fused with the stone, you cannot see what is going on outside, and any Consciousness checks you make to hear the sounds made outside of it are made with -1d6. You remain aware of the passage of time and can cast spells on you while merged with the stone. You can use your movement to leave the stone and reappear where you entered it, thus ending the spell. Otherwise you cannot move. \\
Minor damage to the stone does not harm you, but its partial destruction or shape-shifting (so you never enter it again) expels you from it and deals 6d6 points of blow damage to you. Completely destroying the stone (or transmuting it into another substance) causes you to be expelled and deals 50 hit damage to you. If you are ejected, you fall prone in an unoccupied space, closest to where you entered the stone. \\
\textbf{For each Magic Critical Success rolled} in the Magic Check the maximum duration is increased by 1 hour.

\medskip \textbf{Ethereal Form}\index[Incantesimi]{Ethereal Form} \\
\textbf{School}: Transmutation \\
\textbf{Level}: 7, Rare \\
\textbf{Casting Time}: 2 Actions \\
\textbf{Duration}: Self \\
\textbf{Components}: V, S \\
\textbf{Duration}: Maximum 8 hours \\
You enter the border regions of the Ethereal Plane, the area that overlaps your current plane. You stay on the Ethereal Edge for the duration or until you use an action to break the spell. If you move up or down, the cost of the move is doubled, if you move horizontally the move is doubled per move action. You can see and hear the plane you come from, but everything there appears gray to you, and you cannot see more than 60 feet away. \\
While on the Ethereal Plane, it can only interact with other creatures on that plane. Creatures that are not on the Ethereal Plane cannot perceive you or interact with you, unless a special ability or magic gives them the ability to do so. \\
You ignore all objects and effects that are not on the ethereal plane, thus being able to pass through the objects you perceive on the plane from which you come. When the spell ends, you immediately return to the floor you came from at the point you currently occupy. If when this happens you occupy the same space as a solid object or creature, you are immediately moved to the closest unoccupied space you can occupy and take 6 force damage for each meter you are moved (or a fraction of it). This spell has no effect if you cast it while you are already in the Ethereal Plane or on a plane that does not border on it, such as one of the Outer Planes. \\
\textbf{For each Magic Critical Success you roll} in the Magic Check you may bring another creature with you.

\medskip \textbf{Gaseous Form}\index[Incantesimi]{Gaseous Form} \\
\textbf{School}: Transmutation \\
\textbf{Level}: 3, Uncommon \\
\textbf{Casting Time}: 2 Actions \\
\textbf{Range}: Contact \\
\textbf{Components}: V, S, M (a piece of gauze and a wisp of smoke) \\
\textbf{Duration}: Concentration, maximum 1 hour \\
You transform a consenting creature along with everything it is wearing and carrying, into a vaporous cloud for the duration. The spell ends if the creature drops to 0 hit points. Incorporeal creatures ignore this effect. While in this form, the target's only method of movement is a 10-foot flight speed. The target can enter and occupy another creature's space. The target has resistance to nonmagical damage, and has + 1d6 on Fortitude and Reflex saving throws. The target can pass through small holes, bottlenecks, and even simple holes, although it regards liquids as solid surfaces. The target cannot fall and remains floating in the air even if stunned or otherwise incapacitated. \\
While in the form of a vaporous cloud, the target cannot speak or manipulate objects, and any object they are wearing or carrying cannot be thrown, used or otherwise employed. The target cannot attack or cast spells. \\
\textbf{For every two magical critical successes rolled} in the Magic Check, you can affect another creature.


\medskip \textbf{Animal Shapes}\index[Incantesimi]{Animal Shapes} \\
\textbf{School}: Animals and Plants \\
\textbf{Level}: 8, Rare \\
\textbf{Casting Time}: 2 Actions \\
\textbf{Range}: 9 meters \\
\textbf{Components}: V, S \\
\textbf{Duration}: 24 hours \\
Magically transform other creatures into beasts. Choose any number of willing creatures within range that you can see. Transform each target into the form of a Large or smaller beast with a challenge rating of 4 or less. On subsequent turns, you can use 2 Actions to transform subject creatures into new forms. \\
The transformation persists for each target for the duration of the spell, or until that target drops to 0 hit points or dies. You can choose a different shape for each target. The target's game stats are replaced by the chosen beast's stats, with the exception of the Traits and Intelligence, Wisdom, and Charisma scores which remain those of the
target. The target takes on the hit points of its new form, and when it returns to its normal form, it reverts to the number of hit points it had before transforming. If it transforms back because it dropped to 0 hit points, the excess damage is applied to the original form. As long as the excess damage doesn't reduce the creature's normal form to 0 hit points, it isn't unconscious. The creature is limited in the actions it can perform by the nature of its new form, and cannot speak or cast spells. \\
The target's equipment merges into the new form. The target cannot activate, wield, or otherwise benefit from its equipment.

\medskip \textbf{Shatter}\index[Incantesimi]{Shatter} \\
\textbf{School}: Invocation \\
\textbf{Level}: 2, Common \\
\textbf{Casting Time}: 2 Actions \\
\textbf{Range}: 18 meters \\
\textbf{Components}: V, S, M (a metal fragment) \\
\textbf{Duration}: Instantaneous \\
A loud, very intense rumble erupts from a range point of your choice. Each creature in a 10-foot-radius sphere centered on that spot must make a Fortitude saving throw. A creature takes 3d8 points of sonic damage if it fails its saving throw, or half that damage if it succeeds. A creature made of inorganic material, such as stone, crystal, or metal, has -1d6 on its saving throw. A nonmagical object that is neither worn nor carried also takes damage if it is in the spell's area. \\
\textbf{For each magical critical success rolled} on the Magic Check the damage is increased by 1d8. \\
\textbf{Saving Throw Success / Critical Failure}: On critical failure the damage doubles, on critical success the damage is further halved

\medskip \textbf{Acid Arrow}\index[Incantesimi]{Acid Arrow} \\
\textbf{School}: Water \\
\textbf{Level}: 2, Common \\
\textbf{Casting Time}: 2 Actions \\
\textbf{Range}: 27 meters \\
\textbf{Components}: V, S, M (a powdered rhubarb leaf and a python stomach) \\
\textbf{Duration}: Instantaneous \\
A bright green arrow shoots at a target in range and explodes with a splash of acid. Make a ranged spell attack against the target. If you hit, the target immediately takes 4d4 points of acid damage and 2d4 points of acid damage at the end of its next round. If you miss, the arrow sprays the target with acid dealing half the initial damage and dealing no damage at the end of the target's next round. \\
\textbf{For each magical critical success rolled} on the Magic Check the damage is increased by 2d4.

\medskip \textbf{Lightning}\index[Incantesimi]{Lightning} \\
\textbf{School}: Air \\
\textbf{Level}: 3, Common \\
\textbf{Casting Time}: 2 Actions \\
\textbf{Range}: Personnel (30m line) \\
\textbf{Components}: V, S, M (a piece of fur and a rod of amber, crystal or glass) \\
\textbf{Duration}: Instantaneous \\
A lightning bolt explodes forming a line 30 meters long and 1 meter wide that starts from where you are in a direction chosen by you. Each creature on the line must succeed at a Reflex saving throw. The creature takes 8d6 points of lightning damage on a failed save, or half that damage if it succeeds. \\
Lightning ignites flammable objects in the area that are not worn or carried. \\
Lightning when thrown against hard worked stone bounces at an angle of 180 degrees - the angle of entry. Lightning thrown into the water creates a 10-foot-radius sphere of electricity where it enters. \\
\textbf{For each magical critical success rolled} on the Magic Check the damage is increased by 1d6. \\
\textbf{Saving Throw Success / Critical Failure}: On critical failure the damage doubles, on critical success the damage is further halved

\medskip \textbf{Mislead}\index[Incantesimi]{Mislead} \\
\textbf{School}: Illusion \\
\textbf{Level}: 5, Uncommon \\
\textbf{Casting Time}: 2 Actions \\
\textbf{Duration}: Self \\
\textbf{Components}: S \\
\textbf{Duration}: 1 hour \\
You become invisible at the same time that your illusory double appears in your current location. The duplicate remains for the duration of the spell, but invisibility ends if you attack or cast a spell. You can use 2 Actions to make the illusory double move up to double your speed and make him gesture, speak and behave in any way you want. \\
You can see through his eyes and hear through his ears as if you were in the space he is in. During each of your rounds, with an Action, you can switch from using her senses to using yours, or vice versa. While you are using his senses, you are blinded and deafened about your surroundings.

\medskip \textbf{Cage of Strength}\index[Incantesimi]{Cage of Strength} \\
\textbf{School}: Invocation \\
\textbf{Level}: 6, Rare \\
\textbf{Casting Time}: 2 Actions \\
\textbf{Range}: 30 meters \\
\textbf{Components}: V, S, M (ruby dust worth 1,500 gp) \\
\textbf{Duration}: 1 hour \\
A cubic prison, immobile and invisible, composed of magical force appears around an area at range of your choice. The prison can be a solid cage or box, of your choice. A prison in the form of a cage can have 6 meters on each side and consist of 1.5cm bars separated by 1.5cm from each other and provides complete coverage for the creatures inside. A box-shaped prison can have 10 feet on each side, creating a solid barrier that prevents any matter from passing through it and blocking any spells cast from inside or outside the area. When you cast this spell, any creature that is completely inside the cage is trapped. Creatures only partially in the cage area, or creatures too large to enter, are pushed out of the center of the cage until they are completely out of it. \\
A creature inside the cage cannot leave it by nonmagical means. If the creature tries to use teleportation or interplanar travel to leave the cage, it must first make a Will saving throw. If it passes, the creature can use that spell to escape the cage. If it fails, the creature can't get out of the cage and wastes the use of the spell or effect. The cage also extends to the Ethereal Plane, thus blocking the ethereal journey. \\
This spell cannot be dispelled by \hyperlink{dissolvimagie}{Dissolvi Magie}.

\medskip \textbf{Magic Jar}\index[Incantesimi]{Magic Jar} \\
\textbf{School}: Necromancy \\
\textbf{Level}: 6, Very Rare \\
\textbf{Casting Time}: 1 minute \\
\textbf{Duration}: Self \\
\textbf{Components}: V, S, M (a gem, crystal, reliquary, or some other ornamental container worth at least 500 gp) \\
\textbf{Duration}: Until dissolved \\
Your body enters a catatonic state as your soul leaves it and enters the container you use as a material component. As your soul occupies the container, you are aware of your surroundings as if you are in the space of the container. You cannot move or use reactions. The only action you can take is to project your soul up to 30 meters away from the container, returning to your living body (and ending the spell) or trying to possess a humanoid body. \\
You can attempt to possess any humanoid within 100 feet of you that you can see (creatures protected by protection from good and evil or magic circle spells cannot be possessed). The target must make a Will saving throw, and if it fails, your soul enters the target's body, while the target's soul remains trapped in the container. If it passes, the target resists your attempts to possess it, and you cannot attempt to possess it again until 24 hours have elapsed. \\
Once you own a creature's body, you can control it. Your game stats are replaced by the creature's stats, with the exception of your Traits and your Intelligence, Wisdom, and Charisma scores. Maintain the benefits provided by the Skills. If the target has any Skills, you can't use any. \\
Meanwhile, the possessed creature's soul can sense the container's surroundings using its senses, but it cannot move or perform any action. \\
While you have a body, you can use 2 Actions to return from the host body to the container, if you are within 100 feet of it, returning the host creature's soul to its body. If the host body dies while you are inside, the creature dies, and you must make a Will saving throw against your DC spell saving throws. If you pass it, you return to the container, if it is within 30 meters of you. Otherwise, you will die. \\
If the container is destroyed or the spell ends, your soul immediately returns to your body. If your body is more than 100 feet away or if it died while trying to return, your soul will also die. If another creature's soul is in the container when it is destroyed, the creature's soul returns to its body, if the body is alive and within 100 feet, otherwise, the creature dies. When the spell ends, the container is destroyed.

\medskip \textbf{Glyph of Interdiction}\index[Incantesimi]{Glyph of Interdiction} \\
\textbf{School}: Abjuration \\
\textbf{Level}: 3, Common \\
\textbf{Casting Time}: 2 Actions \\
\textbf{Range}: Contact \\
\textbf{Components}: V. S, M (incense and diamond powder worth at least 200 gp, which the spell consumes) \\
\textbf{Duration}: Until dissolved or activated \\
When you cast this spell, you inscribe a glyph that damages other creatures on a surface (such as a table or section of floor or wall) or within an object that can be closed (such as a book, scroll, or chest) to conceal the glyph. If you choose a surface, the glyph can cover a surface area no greater than 3 meters in diameter. If you choose an object, that object must stay in place; if the object is moved more than 10 feet from where the spell was cast, the glyph is broken, and the spell ends without being activated. \\
The glyph is nearly invisible and can be found on an Intelligence check against the saving throw DC of your spells. You decide what activates the glyph when the spell is cast. \\
For glyphs inscribed on a surface, typical activation involves touching or standing over the glyph, removing another object covering the glyph, approaching some distance from the glyph, or manipulating the object on which the glyph is inscribed. glyph. For glyphs inscribed on an object, typical activation involves opening the object, approaching some distance from the object, or seeing or reading the glyph. Once the glyph is activated, the spell ends. \\
You can better define the activation so that the spell is activated only under certain circumstances or according to certain physical characteristics (such as height or weight), species of creature (for example, the interdiction could act against aberrations or elves dark), or specific Traits. You can also set up conditions to prevent the glyph from being triggered, such as the pronunciation of a password. \\
When writing the glyph, choose explosive runes or spell glyph.

\medskip

- \textit{Spell Glyph}. You can insert a prepared spell of level 2 or lower into the glyph by casting it as part of creating the glyph. The spell must target a single creature or area. The spell that is placed has no immediate effect if cast this way. When the glyph is activated, the spell entered is cast. If the spell has a target, it targets the creature that activated the glyph. If the spell affects an area, the area is centered on that creature. If the spell summons hostile creatures or creates noxious objects or traps, they appear as close to the intruder as possible and attack him. If the spell requires concentration, it is maintained until the end of its normal duration. \\

- \textit{Explosive Runes}. When activated, the glyph erupts magical energy into a 20-foot-radius sphere centered on the glyph. The sphere propagates around the corners. Each creature in the area must make a Reflex saving throw. A creature takes 5d8 points of acid, lightning, fire, cold, or sound damage if it fails its saving throw (your choice when creating the glyph), or half that damage if it makes the saving throw. \\
\textbf{For each magical critical success rolled} on the Magic Check the damage of the explosive rune glyph increases by 1d8.

\medskip \textbf{Orb of Invulnerability}\index[Incantesimi]{Orb of Invulnerability} \\
\textbf{School}: Abjuration \\
\textbf{Level}: 6, Common \\
\textbf{Casting Time}: 2 Actions \\
\textbf{Range}: Personnel (3 meters radius) \\
\textbf{Components}: V. S, M (a glass or crystal ball that shatters when the spell ends) \\
\textbf{Duration}: Concentration, maximum 1 minute \\
A still, faintly shimmering barrier rises in a 10-foot radius around you and remains there for the duration. \\
Any Level 4 spell (excluding higher results from magic crit) or lower cast from outside the barrier cannot affect creatures or objects within it. These spells are suppressed if they target creatures and objects within the barrier or involve the area over which the barrier is. \\
\textbf{For every two Magic Critical Success obtained} in the Magic Check you can block one higher level of spell.

\medskip \textbf{Kyrin's Acorn Rainfall}\index[Incantesimi]{Kyrin's Acorn Rainfall} \\
\textbf{School}: Animals and Plants \\
\textbf{Level}: 2, Uncommon \\
\textbf{Casting Time}: 1 Action \\
\textbf{Range}: 50 meters \\
\textbf{Components}: V, S, M (9 acorns that are consumed, a piece of rubber) \\
\textbf{Duration}: 1 minute for Magical Proficiency, Concentration \\
You charm 9 acorns of magical energy and they begin to swirl 12 inches above your shoulder.
Each round, by spending 1 Action, you can throw up to 5 acorns at one or more targets. Make a single attack roll with ranged spells, with a bonus equal to the times you have taken Animal and Plant Lists, per target regardless of the number of acorns you throw at it. Each acorn does 1d4 hit damage if it hits. \\
\textbf{For each Magic Critical Success you roll} in the Magic Check you can enchant two more acorns.

\medskip \textbf{Kyrin's Flaming Acorn Rainfall}\index[Incantesimi]{Kyrin's Flaming Acorn Rainfall} \\
\textbf{School}: Animals and Plants, Fire \\
\textbf{Level}: 3, Rare \\
\textbf{Casting Time}: 2 Actions \\
\textbf{Range}: 30 meters \\
\textbf{Components}: V, S, M (9 acorns that are consumed, a piece of rubber) \\
\textbf{Duration}: 1 minute for Magical Proficiency, Concentration \\
You charm 9 acorns of magical fire and they begin to swirl 12 inches above your shoulder.
Each round, by spending 1 Action, you can throw up to 5 acorns at one or more targets. Make a single attack roll with ranged spells, with a bonus equal to the times you took Animal and Plant Lists or Fire, per target regardless of the number of acorns you throw at it. Each acorn if hit does 1d4 hit damage and 2 fire damage. \\
\textbf{For each Magic Critical Success you roll} in the Magic Check you can enchant two more acorns.

\medskip \textbf{Kyrin's Chestnut Tree}\index[Incantesimi]{Kyrin's Chestnut Tree} \\
\textbf{School}: Animals and Plants \\
\textbf{Level}: 5, Very Rare \\
\textbf{Casting Time}: 1 Action \\
\textbf{Range}: 60 meters \\
\textbf{Components}: V, S, M (9 browns that are consumed, a piece of rubber) \\
\textbf{Duration}: 1 minute for Magical Proficiency, Concentration \\
Charm 9 browns of magical energy and these begin to swirl 60cm over your shoulder.
Each round, by spending 1 Action, you can throw up to 5 browns at one or more targets. Make a single attack roll with ranged spells, with a bonus equal to the times you have taken Animal and Plant Lists, per target regardless of the number of acorns you throw at it. Each acorn if hit does 2d8 hit damage \\
\textbf{For each Magic Critical Success you roll} in the Magic Check you can enchant two more browns.

\medskip \textbf{Healing}\index[Incantesimi]{Healing} \\
\textbf{School}: Healing \\
\textbf{Level}: 6, Rare \\
\textbf{Casting Time}: 2 Actions \\
\textbf{Range}: 18 meters \\
\textbf{Components}: V, S \\
\textbf{Duration}: Instantaneous \\
Choose a creature in range and that you can see. a wave of positive healing energy overwhelms the creature, causing it to recover 70 hit points. The spell also ends any blindness, deafness, and disease (even magical) that afflicts the target. This spell deals 50 hit points of damage to an undead on a touch attack roll. \\
\textbf{For each Magic Critical Success rolled} in the Magic Check the amount healed increases by 20.

If the caster and creature healed are both \textbf{Followers} of the same Patron the spell heals 90 hit points.

If the caster and the healed creature are both \textbf{Devotees} of the same Patron, the spell restores full hit points.

\medskip \textbf{Mass Healing}\index[Incantesimi]{Mass Healing} \\
\textbf{School}: Healing \\
\textbf{Level}: 9, Legendary \\
\textbf{Casting Time}: 2 Actions \\
\textbf{Range}: 18 meters \\
\textbf{Components}: V, S \\
\textbf{Duration}: Instantaneous \\
A stream of healing energy flows from you to the wounded creatures around you. Restores up to 700 hit points, split as you like between any creature at range and that you can see (with a maximum of 70 HP per creature). Creatures healed by this spell are also cured of all diseases and any effects that make them blind or deafened. This spell can deal up to 120 HP of damage to an undead. TS on Strength to cancel the effect.

If the caster and creature healed are both \textbf{Followers} of the same Patron the heal granted increases by 20 \%

If the caster and creature healed are both \textbf{Devotees} of the same Patron the heal granted increases by 50 \%

\medskip \textbf{Help}\index[Incantesimi]{Help} \\
\textbf{School}: Divination \\
\textbf{Level}: 0, Common \\
\textbf{Casting Time}: 2 Actions \\
\textbf{Range}: Contact \\
\textbf{Components}: V, S \\
\textbf{Duration}: Concentration, maximum 1 minute \\
You cast the spell on contact with a consenting creature. Once, before the spell ends, the target can roll a d4 and add the rolled result to an ability check of their choice. He may roll the die either before or after making the ability check. The spell then ends.

\medskip \textbf{Anti-Life Shell}\index[Incantesimi]{Anti-Life Shell} \\
\textbf{School}: Animals and Plants \\
\textbf{Level}: 5, Uncommon \\
\textbf{Casting Time}: 2 Actions \\
\textbf{Range}: Personnel (3 meters radius) \\
\textbf{Components}: V, S \\
\textbf{Duration}: maximum 1 hour \\
A light barrier extends up to a 10-foot radius around you, moving with you and staying centered on you, keeping creatures other than undead or constructs at a distance. The barrier remains for the duration. \\
The barrier prevents a subject creature from crossing it in any way. A subject creature can cast spells or make ranged or ranged weapon attacks across the barrier. If you move so that a subject creature is forced to cross the barrier, the spell ends.

\medskip \textbf{Identify}\index[Incantesimi]{Identify} \hypertarget{incantesimoidentificare}{} \\
\textbf{School}: Universal \\
\textbf{Level}: 1, Common \\
\textbf{Casting Time}: 1 minute \\
\textbf{Range}: Contact \\
\textbf{Components}: V, S, M (a gem worth at least 10 gp and an owl feather that the spell consumes) \\
\textbf{Duration}: Instantaneous \\
Choose an object that you must keep in contact with throughout the casting of the spell. If it is a magical object or other object imbued with magic, make an Arcana check at DC 25 with a +10 bonus, if you succeed you learn its properties and how to use them and how many charges it has, if any. \\
Learn if spells are affecting the object and what they are. If the item was created by a spell, you learn which spell created it. If, on the other hand, you remain in contact with a creature during the execution, you learn if any spells are acting on it and what they are. \\
\textbf{Only If you roll a Magic Critical Success} you learn if the item is \hyperlink{oggettimaledettiid}{maledetto}.

\medskip \textbf{Minor Illusion}\index[Incantesimi]{Minor Illusion} \\
\textbf{School}: Universal \\
\textbf{Level}: 0, Common \\
\textbf{Casting Time}: 2 Actions \\
\textbf{Range}: 9 meters \\
\textbf{Components}: S, M (a piece of fleece) \\
\textbf{Duration}: 1 minute \\
You create an image of an object or sound at range for the duration of the spell. The illusion ends if you interrupt it with an action or cast this spell again. \\
If you create a sound, its volume can vary from that of a whisper to a scream. It can be your voice, someone else's voice, the roar of a lion, the beating of drums, or any other sound you choose. The sound continues incessantly throughout the duration, or you may produce different sounds at different times before the spell ends. \\
If you create an image of an object (such as a chair, a muddy footprint, or a small chest) it cannot be larger than a 1 meter edge cube. The image cannot produce sounds, lights, smells or any other sensory effects. Physical interaction with the object reveals it as an illusion, because things can pass through it. \\
A creature that uses 3 Actions to examine the sound or image can determine that it is an illusion with a successful Intelligence (Investigate) check against your spell's saving throw DC. If a creature recognizes the illusion for what it is, for it the illusion fades.

\medskip \textbf{Programmed Illusion}\index[Incantesimi]{Programmed Illusion} \\
\textbf{School}: Illusion \\
\textbf{Level}: 6, Uncommon \\
\textbf{Casting Time}: 2 Actions \\
\textbf{Range}: 36 meters \\
\textbf{Components}: V, S, M (a piece of fleece and jade dust worth at least 25 gp) \\
\textbf{Duration}: Until dissolved \\
You create, at range, the illusion of an object, creature or some other visible phenomenon that is activated when a specific condition is met. Until then the illusion is imperceptible. It can't be bigger than a 30-foot-edged cube, and you decide when you cast the spell, how the illusion behaves and what sounds it produces. The scheduled performance can last up to 5 minutes. When the conditions you specify are met, the illusion manifests and behaves in the way you described. Once the illusion has finished its performance, it disappears and lies dormant for 10 minutes. After this period, the illusion can be activated again. \\
The trigger condition can be as general or as detailed as you like, although it must be based on visible or audible conditions occurring within 30 feet of the area. For example, you could create an illusion of yourself that appears and warns anyone who tries to open a door with a trap, or you could set up the illusion to activate only when a creature utters the right word or phrase. \\
Physical interaction with the image reveals it as an illusion, as things pass through it. A creature who uses 3 Actions to examine the image can determine that it is an illusion with a successful Intelligence (Investigate) check against the spell's saving throw DC. If a creature recognizes the illusion for what it is, it can see through the image, and any sound the image makes sounds to it an artifact.

\medskip \textbf{Major Image}\index[Incantesimi]{Major Image} \\
\textbf{School}: Illusion \\
\textbf{Level}: 3, Common \\
\textbf{Casting Time}: 2 Actions \\
\textbf{Range}: 36 meters \\
\textbf{Components}: V, S, M (a piece of fleece) \\
\textbf{Duration}: Concentration, maximum 1 minute per Magical Proficiency \\
You create an image of an object, creature, or some other visible phenomenon that is no larger than a 6-meter edge cube. The image appears at a point in range that you can see and remains there for the duration of the spell. The image looks completely real, and includes sounds, smells, and the temperature appropriate to the thing depicted. You can't generate enough heat or cold to cause damage, no sound loud enough to deal sonic damage or deafening a creature, or a smell that can make a creature sick (like the stench of a troglodyte). As long as you stay in range of the illusion, you can use an action to make the image move to any other point in range. \\
When the image changes position, you can alter its appearance so that its movements appear natural. For example, if you create an image of a creature and move it, you can alter the image so that it appears to walk. Similarly, you can use the illusion to produce different sounds at different times, until it carries on a conversation. \\
Physical interaction with the image reveals it as an illusion, as things pass through it. A creature that uses 3 Actions to examine the image can determine that it is an illusion with a successful Intelligence (Investigate) check against your spell's saving throw DC. If a creature recognizes the illusion for what it is, the creature can see through it, and for that creature all other sensory qualities vanish. \\
\textbf{If you get a critical} the spell lasts until dispelled, without requiring your concentration.

\medskip \textbf{Projected Image}\index[Incantesimi]{Projected Image} \\
\textbf{School}: Illusion \\
\textbf{Level}: 7, Uncommon \\
\textbf{Casting Time}: 2 Actions \\
\textbf{Range}: 750 kilometers \\
\textbf{Components}: V, S, M (a small reproduction of yours made of materials worth at least 5 gp) \\
\textbf{Duration}: 1 day \\
You create an illusory copy of yourself that lasts for the duration. The copy can appear anywhere within range that you have already seen, ignoring any obstacles in the way. The illusion reproduces your appearance and your noises but it is intangible. If the illusion takes damage, it disappears, and the spell ends. \\
You can use 2 Actions to make this illusion move up to double your speed and make it gesture, speak and behave in any way you want. Perfectly imitate your behaviors. \\
You can see through its eyes and hear through its ears as if you were in the space it is in. During each of your rounds, with an Action, you can switch from using her senses to using yours, or vice versa. While you are using his senses, you are blinded and deafened about your surroundings. \\
Physical interaction with the image reveals it as an illusion, as things pass through it. A creature who uses 3 Actions to examine the image can determine that it is an illusion with a successful Awareness check against the spell's saving throw DC. If a creature recognizes the illusion for what it is, it can see through the image, and any sound the image makes sounds to it an artifact.

\medskip \textbf{Silent Image}\index[Incantesimi]{Silent Image} \\
\textbf{School}: Illusion \\
\textbf{Level}: 1, Common \\
\textbf{Casting Time}: 2 Actions \\
\textbf{Range}: 36 meters \\
\textbf{Components}: V, S, M (a piece of fleece) \\
\textbf{Duration}: Concentration, maximum 3 minutes per Magical Proficiency \\
You create an image of an object, creature, or some other visible phenomenon that is no larger than a 4-meter edge cube. The image appears in a point in range that you can see and remains for the duration of the spell. The image is purely visual; it is not accompanied by sounds, smells or other sensory effects. You can use an action to make the image move to any other point in range. When the image changes position, you can alter its appearance so that its movements appear natural. For example, if you create an image of a creature and move it, you can alter the image so that it appears to be walking. \\
Physical interaction with the image reveals it as an illusion, as things pass through it. A creature that uses 3 Actions to examine the image can determine that it is an illusion with an Awareness check against your spell's saving throw DC. If a creature recognizes the illusion for what it is, the creature can see through it.

\medskip \textbf{Mirror Image}\index[Incantesimi]{Mirror Image} \\
\textbf{School}: Illusion \\
\textbf{Level}: 2, Common \\
\textbf{Casting Time}: 2 Actions \\
\textbf{Duration}: Self \\
\textbf{Components}: V, S \\
\textbf{Duration}: 1 minute \\
2d4 illusory duplicates of yourself appear in your space. Until the spell ends, the duplicates move with you and mimic your actions, swapping places in a way that makes it impossible to determine what the real picture is. You can use 1 Action to dismiss illusory duplicates. \\
Whenever a creature takes you it actually hits an illusory image.
If a creature makes multiple attacks in turn, it can scatter one image for each successful attack. If you are hit by an area spell all images will vanish. \\
A creature who cannot see, or relies on senses other than sight (such as blind sight), or who can distinguish illusions as false (such as true vision), ignores the effects of this spell. \\
\textbf{For each Magic Critical Success obtained} in the Magic Check you create one more duplicate image up to a maximum total of 8 images.

\medskip \textbf{Imprison}\index[Incantesimi]{Imprison} \\
\textbf{School}: Abjuration \\
\textbf{Level}: 9, Rare \\
\textbf{Casting Time}: 2 Actions \\
\textbf{Range}: 9 meters \\
\textbf{Components}: V, S, M (a fleece depiction or figurine engraved with the features of the target, and a special component that varies depending on which version of the spell you choose, worth at least 500 gp for Target's Wound Die) \\
\textbf{Duration}: Until dissolve \\
You create magical constraints to block a creature in range that you can see. The target must succeed on a Will saving throw or be bound by the spell; if he succeeds, he is immune to the spell if you cast it again. While subject to this spell, the creature does not need to breathe, eat or drink and does not age. Divination spells cannot locate or sense the target. \\
When you cast this spell, you choose one of the following forms of captivity. \\

- \textit{Chaining}. Heavy chains, well welded to the ground, keep the target anchored. The target is hampered until the spell ends, and cannot move or be moved in any way until then. The special component for this version of the spell is a precious metal chain. \\
- \textit{Minimum Insulation}. The target shrinks to 2.5cm in height and is imprisoned in a gem or similar object. Light can pass through the gem normally (allowing the target to see outside and other creatures to see inside), but nothing else can pass through it, not even by teleportation or plane travel. The gem cannot be cut or broken as long as the spell remains in effect. The special component for this version of the spell is a large transparent gem, such as corundum, diamond or ruby. \\
- \textit{Confined Prison}. The spell transports the target into a tiny demiplane that is forbidden to teleportation and planar travel. The demiplane can be a labyrinth, a cage, a tower, or any other enclosed structure of your choice. The special component for this version of the spell is a miniature representation of the prison made of jade. \\
- \textit{Burial}. The target is buried deep within the earth in a sphere of magical force large enough to hold the target. Nothing can pass through the sphere, nor can any creature teleport or use planar travel to enter or exit it. The special component for this version of the spell is a small mithral sphere. \\
- \textit{Sopore}. The target falls asleep and cannot be awakened. The special component for this version of the spell consists of rare sleep herbs. \\

\medskip
\textit{\textbf{End the spell}}. While casting the spell, in any of its versions, you can specify a condition that can end the spell and free the target. The condition can be as specific or as elaborate as you wish, but the Storyteller must agree that the condition is reasonable and can be fulfilled. Conditions may be based on a creature's name, identity, or Patron, but still based on perceivable actions or qualities and not on intangible things like level, abilities, or hit points. \\
A dispel magic spell can end the spell only if cast by a character with Magical Proficiency at least 18, targeting the prison or material component used to create it. \\
You can use a particular special component to create only one prison at a time. If you cast the spell again using the same component, the target of the first casting of the spell is immediately released from its binding.

\medskip \textbf{Wither}\index[Incantesimi]{Wither} \\
\textbf{School}: Necromancy \\
\textbf{Level}: 4, Uncommon \\
\textbf{Casting Time}: 2 Actions \\
\textbf{Range}: 9 meters \\
\textbf{Components}: V, S \\
\textbf{Duration}: Instantaneous \\
Necromantic Energy engulfs a creature of your choice within range that you can see, robbing it of sap and vitality. The target must make a Fortitude saving throw. On a failed save, the target takes 8d8 points of Void damage, or half that damage on a successful save. The spell has no effect on undead or constructs. \\
If the target is a nonmagical vegetable that is not also a creature, such as a tree or bush, it makes no saving throws, withers and dies instantly. \\
\textbf{For each magical critical success rolled} on the Magic Check the damage is increased by 1d8. \\
\textbf{Saving Throw Success / Critical Failure}: On critical failure the damage doubles, on critical success the damage is further halved

\medskip \textbf{Detection of Good and Evil}\index[Incantesimi]{Detection of Good and Evil} \\
\textbf{School}: Divination \\
\textbf{Level}: 1, Common \\
\textbf{Casting Time}: 2 Actions \\
\textbf{Duration}: Self \\
\textbf{Components}: V, S \\
\textbf{Duration}: 1 round per Magical Proficiency \\
For the duration, you learn whether an aberration, celestial, elemental, fairy, demon, or undead is within 30 feet of you, and its location. Likewise, you learn if within 30 feet of you there is a place or object that has been magically consecrated or desecrated. \\
the spell can penetrate most barriers, but is blocked by 12 inches of stone, 1 inch of common metal, a thin sheet of lead, or 1 meter of wood or earth. \\
\textbf{For each magical critical success rolled} in the Magic Check duration doubles. \\
\textbf{Note}: This spell has no effect on creatures following traits. At the Storyteller's discretion it can be used to identify the Patron of a Follower or Devotee.

\medskip \textbf{Detect Magic}\index[Incantesimi]{Detect Magic} \\
\textbf{School}: Universal \\
\textbf{Level}: 1, Common \\
\textbf{Casting Time}: 2 Actions \\
\textbf{Duration}: Self \\
\textbf{Components}: V, S \\
\textbf{Duration}: 1d4 +1 rounds per magical proficiency \\
For the duration, feel the presence of the magic within 30 feet of you. You can use 1 Action to see a faint aura extending around any visible creature or object in the area that carries magic. With two Actions, you also learn the Magic List, if it has one. \\
The spell can penetrate most barriers, but is blocked by 12 inches of stone, 1 inch of common metal, a thin sheet of lead, or 1 meter of wood or earth. \\
\textbf{For each Magic Critical Success rolled} in the Magic Check duration doubles.

\medskip \textbf{Detection of Diseases and Poisons}\index[Incantesimi]{Detection of Diseases and Poisons} \\
\textbf{School}: Divination \\
\textbf{Level}: 1, Uncommon \\
\textbf{Casting Time}: 2 Actions \\
\textbf{Duration}: Self \\
\textbf{Components}: V, S, M (a yew leaf) \\
\textbf{Duration}: 1 round per Magical Proficiency \\
For the duration, sense the presence and location of poisons, poisonous creatures and diseases within 30 feet of you. Also you can identify the type of poison, poisonous creature or disease. The spell can penetrate most barriers, but is blocked by 12 inches of stone, 1 inch of common metal, a thin sheet of lead, or 1 meter of wood or earth. \\
\textbf{For each Magic Critical Success rolled} in the Magic Check duration doubles.

\medskip \textbf{Detection of Thoughts}\index[Incantesimi]{Detection of Thoughts} \\
\textbf{School}: Divination \\
\textbf{Level}: 2, Rare \\
\textbf{Casting Time}: 2 Actions \\
\textbf{Duration}: Self \\
\textbf{Components}: V, S, M (a piece of copper) \\
\textbf{Duration}: 1 minute \\
For the duration, you can read the thoughts of certain creatures. When you cast this spell and with two other Actions in each subsequent round until the spell ends, you can focus your mind on any creature you can see that is within 30 feet of you. If the creature you chose has an Intelligence score -3 or less or doesn't speak any language, the creature ignores the effect. \\
Initially, you only learn the creature's surface thoughts - the most recurring ones. With an action, you can either shift your attention to another creature's thoughts or attempt to probe deeper into the same creature's mind. If you probe deeper, the target must make a Will saving throw. If he fails, you get a sense of his reasoning (if any), of his emotional state, and of everything that prevails in his thoughts (such as worry, love, or hate). On a successful saving throw, the spell ends. Either way, the target knows you're probing its mind, and unless you shift your attention to another creature's mind, in its round the creature can use the 2 Actions to make an Intelligence check contested by the your proof of Intelligence; if he wins it, the spell ends. \\
The verbal questions posed to the target creature, of course, shape the course of his thoughts, so that this spell is particularly effective in interrogations. \\
You can also use this spell to detect the presence of thinking creatures that you cannot see. When you cast this spell or with 2 Actions in its duration, you can search for thoughts within 30 feet of you. The spell can penetrate barriers, but is blocked by two inches of stone, two inches of metal other than lead, or a thin sheet of lead. You can't spot a creature with Intelligence -3 or less, or a creature that doesn't speak any language. Once you've identified a creature's presence in this way, you can read its thoughts for the duration of the spell as long as it stays in range, as described above, even if you can't see it.
While you have activated this spell to cast other spells you will be Distracted.

\medskip \textbf{Inflict Wounds}\index[Incantesimi]{Inflict Wounds} \\
\textbf{School}: Necromancy \\
\textbf{Level}: 1, Common \\
\textbf{Casting Time}: 2 Actions \\
\textbf{Range}: Contact \\
\textbf{Components}: V, S \\
\textbf{Duration}: Instantaneous \\
Make a spell melee attack against a creature in range. If you hit, the target takes 3d10 Void damage. \\
\textbf{For each magical critical success rolled} on the Magic Check the damage is increased by 1d8.

\medskip \textbf{Enlarge / Reduce}\index[Incantesimi]{Enlarge / Reduce} \\
\textbf{School}: Transmutation \\
\textbf{Level}: 2, Common \\
\textbf{Casting Time}: 2 Actions \\
\textbf{Range}: 9 meters \\
\textbf{Components}: V, S, M (a pinch of powdered iron) \\
\textbf{Duration}: 1 minute \\
Make a creature or object in range that you can see zoom in or out for the duration of the spell. Choose a creature or object that is neither worn nor carried. If the target is unwilling, he can make a Fortitude saving throw; if he succeeds, the spell has no effect. If the target is a creature, everything it is wearing and carrying changes size with it. Any item dropped by a creature subject to this spell immediately returns to its normal size. \\

- \textit{Enlarge}. The target's size doubles in all dimensions, and its weight is multiplied by eight. This growth increases its size by one category: from Medium to Large, for example. If there is not enough space for the target to double its size, the creature or object assumes the largest possible size allowed by the available space. Until the spell ends, the target has + 1d6 on Strength checks and Fortitude saving throws. The target's weapons grow to reach the new size. While these weapons are enlarged, the target's attacks with them will do an additional category of damage. \\
- \textit{Reduce}. The target's size is halved in all sizes, and its weight is reduced to one eighth. This growth decreases its size by one category: from Medium to Small, for example. Until the spell ends, the target has -1d6 on Strength checks and Fortitude saving throws. The target's weapons shrink to reach the new size. While these weapons are shrunk, the target's attacks with them will do a lower damage category (but without reducing the weapon damage to less than 1). \\
\textbf{For every two Critics rolled} in the Magic Check the creature increases by one more size.

\medskip \textbf{Giant Bug}\index[Incantesimi]{Giant Bug} \\
\textbf{School}: Animals and Plants \\
\textbf{Level}: 4, Uncommon \\
\textbf{Casting Time}: 2 Actions \\
\textbf{Range}: 9 meters \\
\textbf{Components}: V, S \\
\textbf{Duration}: 10 minutes \\
For the duration of the spell, transform up to ten centipedes, three spiders, five wasps, or a ranged scorpion into giant versions of their natural form. A centipede becomes a giant centipede, a spider becomes a giant spider, a wasp becomes a giant wasp, and a scorpion becomes a giant scorpion. Each creature obeys your voice commands and, in combat, acts in each round during your round. The Storyteller has the stats of these creatures, and The Storyteller will always resolve their actions and movements. A creature remains in its giant form for the duration, until it drops to 0 hit points, or until you use an action to stop the effect on it. \\
The Storyteller can allow you to choose different targets. For example, if you transform a bee, its giant version might have the same stats as the giant wasp.

\medskip \textbf{Death ward}\index[Incantesimi]{Death ward} \\
\textbf{School}: Abjuration \\
\textbf{Level}: 4, Uncommon \\
\textbf{Casting Time}: 2 Actions \\
\textbf{Range}: Contact \\
\textbf{Components}: V, S \\
\textbf{Duration}: 8 hours \\
You cast the spell on contact with a creature. Grant the target protection from death. The first time the target drops to 0 hit points as a result of the damage taken, the target drops to 1 hit point instead and the spell ends. If the spell is still active when the target is the victim of an effect that would kill them instantly without dealing damage, that effect is instead negated on the target and the spell ends. \\
\textbf{For every two magical critical successes rolled} in the Magic Check the spell protects one more time. \\

\medskip \textbf{Intermittence}\index[Incantesimi]{Intermittence} \\
\textbf{School}: Transmutation \\
\textbf{Level}: 3, Uncommon \\
\textbf{Casting Time}: 2 Actions \\
\textbf{Duration}: Self \\
\textbf{Components}: V, S \\
\textbf{Duration}: 1 round per Magical Proficiency \\
Roll a 1d6 at the end of each of your rounds for the duration of this spell. If you roll an odd number you vanish from your current plane of existence and reappear on the Ethereal Plane (the spell fails and the casting is wasted if you were already on that plane). At the start of your next round, and when the spell ends, if you were on the Ethereal Plane, you return to an unoccupied space of your choice that you can see within 10 feet of the space you vanished from. If no unoccupied space is available within this range, you appear in the nearest unoccupied space (determined randomly if more than one space is available). You can interrupt the spell with an action. \\
While on the Ethereal Plane, you can see and hear the plane you come from, which you perceive in shades of gray, but you still cannot perceive anything more than 60 feet away. You can only interact with creatures that are on the Ethereal Plane. Creatures that are not there can neither perceive you nor interact with you unless they have the ability to do so.

\medskip \textbf{Cow of Hell}\index[Incantesimi]{Cow of Hell} \\
\textbf{School}: Evocation \\
\textbf{Level}: 1, Uncommon \\
\textbf{Casting Time}: 1 Reaction, which you can perform in response to damage done to you by a creature within 60 feet of you that you can see \\
\textbf{Range}: 18 meters \\
\textbf{Components}: V, S \\
\textbf{Duration}: Instantaneous \\
You point your finger, and the creature that harmed you is momentarily engulfed in diabolical flames. The creature must make a Reflex saving throw. It takes 2d10 points of fire damage on a failed save, or half that damage if it succeeds. \\
\textbf{For each magical critical success rolled} on the Magic Check the damage is increased by 1d6.

\medskip \textbf{Entangle}\index[Incantesimi]{Entangle} \\
\textbf{School}: Animals and Plants \\
\textbf{Level}: 1, Common \\
\textbf{Casting Time}: 2 Actions \\
\textbf{Range}: 27 meters \\
\textbf{Components}: V, S \\
\textbf{Duration}: 1 minute \\
Creepers and crushing branches sprout from the ground in a square of 6 meters on each side starting from a point at range. For the duration, these plants transform the soil in the area into difficult terrain. \\
A creature in the area when you cast this spell must succeed on a Fortitude saving throw or be hampered by these vegetables until the spell ends. A creature hampered by plants can use two Actions to make a new saving throw. If he gets over it, he breaks free. When the spell ends, the plants summoned vanish.

\medskip \textbf{Reverse Gravity}\index[Incantesimi]{Reverse Gravity} \\
\textbf{School}: Transmutation \\
\textbf{Level}: 7, Rare \\
\textbf{Casting Time}: 2 Actions \\
\textbf{Range}: 30 meters \\
\textbf{Components}: V, S, M (a magnet and a wire) \\
\textbf{Duration}: Concentration, maximum 1 minute
This spell reverses gravity in a 15 meter radius cylinder, 30 meters high, centered at a point at range. When you cast this spell, any creatures and objects that aren't anchored to the ground in some way fall straight up and reach the top of the area. A creature can attempt a Reflex saving throw to grab a stationary object in range to avoid falling that way if you pass it. \\
If a solid object (the ceiling) is encountered along this fall, the falling objects and creatures impact it as they would during a normal fall. If an object or creature reaches the top of the area without hitting anything, it stays there, swaying slightly, for the duration. \\
At the end of the duration, objects and creatures hit fall back down.

\medskip \textbf{Send}\index[Incantesimi]{Send} \\
\textbf{School}: Invocation \\
\textbf{Level}: 3, Common \\
\textbf{Casting Time}: 2 Actions \\
\textbf{Range}: Unlimited \\
\textbf{Components}: V, S, M (a small piece of copper wire) \\
\textbf{Duration}: 1 round \\
You send a short message of 25 words or less to a creature you are familiar with. The creature hears the message in its mind, recognizes you as the sender, and can respond to you in a similar way. The spell allows creatures with an Intelligence score of at least -2 to understand the meaning of your message even if it doesn't understand your language. \\
You can send the message across any distance and even to other planes of existence, but if the target is on a plane other than yours, there is a 5 \% chance that the message will not arrive. \\
\textbf{For each Magic Critical Success you roll} in the Magic Check you increase the message by 25 words or the duration by one round.

\medskip \textbf{Invisibility}\index[Incantesimi]{Invisibility} \\
\textbf{School}: Illusion \\
\textbf{Level}: 2, Common \\
\textbf{Casting Time}: 2 Actions \\
\textbf{Range}: Contact \\
\textbf{Components}: V, S, M (an eyelash wrapped in gum arabic) \\
\textbf{Duration}: 1 minute per Magical Proficiency \\
You cast the spell on contact with a creature. The target becomes invisible until the spell ends. Whatever the target is wearing or carrying becomes invisible as long as it remains on the target. The spell ends for the target attacking or casting a spell. \\
\textbf{For each Magic Critical Success rolled} in the Magic Check, you may choose an additional target creature or double the duration.

\medskip \textbf{Greater Invisibility}\index[Incantesimi]{Greater Invisibility} \\
\textbf{School}: Illusion \\
\textbf{Level}: 4, Uncommon \\
\textbf{Casting Time}: 2 Actions \\
\textbf{Range}: Contact \\
\textbf{Components}: V, S \\
\textbf{Duration}: 1 minute \\
You cast the spell on contact with a creature. The target becomes invisible until the spell ends. Anything worn or carried by the target becomes invisible as long as it remains on the target. \\
Performing spells or attack actions does not cause it to become visible.

\medskip \textbf{Invoke Lightning}\index[Incantesimi]{Invoke Lightning} \\
\textbf{School}: Air \\
\textbf{Level}: 3, Common \\
\textbf{Casting Time}: 1 round \\
\textbf{Range}: 36 meters \\
\textbf{Components}: V, S \\
\textbf{Duration}: Concentration, maximum 10 minutes \\
A storm cloud appears in the form of a 10-foot-high cylinder with a radius of 60 feet, centered on a point you can see, 100 feet above you. The spell automatically fails if you cannot see the spot in the air where the storm cloud will appear (for example, if you are in a room that cannot accommodate the cloud). When you cast the spell, pick a spot that you can see within range. Lightning will strike from the cloud on that point. Any creature within 1 meter of that point must make a Reflex saving throw. A creature takes 3d10 points of lightning damage if it fails its saving throw, or half that damage if it succeeds. During each of your rounds until the spell ends, you can use two Actions to summon another bolt in this way, targeting the same or a different spot. \\
If you are outside in stormy conditions when you cast this spell, the spell gives you control of the existing storm instead of creating a new one. Under these conditions, the spell's damage increases by 1d10. \\
\textbf{For each magical critical success rolled} on the Magic Check the damage is increased by 1d8.

\medskip \textbf{Labyrinth}\index[Incantesimi]{Labyrinth} \\
\textbf{School}: Summon \\
\textbf{Level}: 8, Rare \\
\textbf{Casting Time}: 2 Actions \\
\textbf{Range}: 18 meters \\
\textbf{Components}: V, S \\
\textbf{Duration}: maximum 10 minutes \\
Banish a creature in range and that you can see in a labyrinthine demiplane. The target remains there for the duration of the spell or until it escapes the maze. The target can take 3 Actions to try to escape. When he does so, he makes a DC 25 Intelligence check. If he succeeds, he flees, and the spell ends (a minotaur or gorister demon succeeds automatically). \\
When the spell ends, the target reappears in the space it had left or, if that space is occupied, in the nearest unoccupied space. \\
\textbf{For each Magic Critical Success obtained} in the Magic Check the duration is increased by 10 minutes. \\

\medskip \textbf{Burning Blade}\index[Incantesimi]{Burning Blade} \\
\textbf{School}: Fire \\
\textbf{Level}: 2, Common \\
\textbf{Casting Time}: 1 Immediate Action \\
\textbf{Range}: Staff \\
\textbf{Components}: V, S, M (a sumac leaf) \\
\textbf{Duration}: Concentration, maximum 10 minutes \\
You create a fiery blade in your hand. The blade is similar in size and shape to a scimitar, and remains for the duration. If you let go of the blade, it disappears, but you can create another with an Action. You can use 2 Actions to make a melee attack with the flaming blade. If you hit, the target takes 3d6 points of fire damage. The fiery blade emits intense light within a 10-foot radius and dim light for an additional 10-foot. \\
\textbf{For every two Critics rolled} in the Magic Check the damage increases by 1d6.

\medskip \textbf{Flamethrower}\index[Incantesimi]{Flamethrower} \\
\textbf{School}: Fire \\
\textbf{Level}: 2, Rare \\
\textbf{Casting Time}: 2 Actions \\
\textbf{Duration}: Self \\
\textbf{Components}: V, S, M (a 30 cm iron pipe, some beans) \\
\textbf{Duration}: 1 minute, Concentration \\
A flame appears at the end of the metal tube you hold in your hand. The flame stays there for the duration of the spell during which you need to stay focused and does not damage either you or your equipment. The flame produces bright light within 1 meter and dim light within 1 meter. The spell ends if you interrupt it with an action or if you cast it again. \\
With a ranged attack roll and spending 1 Action you can stretch the flame up to 30 feet to hit a target. If you hit, the target takes 2d6 points of fire damage, if you hold onto the target you get +2 to hit the next round. \\
\textbf{For each Critical roll} on the Magic Check the damage is increased by 1d6.

\medskip \textbf{Telepathic Bond}\index[Incantesimi]{Telepathic Bond} \\
\textbf{School}: Divination \\
\textbf{Level}: 5, Rare \\
\textbf{Casting Time}: 2 Actions \\
\textbf{Range}: 9 meters \\
\textbf{Components}: V, S, M (pieces of eggshell from two different species of creatures) \\
\textbf{Duration}: 1 hour \\
Establish a telepathic link between up to eight consenting creatures at range of your choice, psychically linking each creature to the others for the duration of the spell. Creatures with an Intelligence score of -3 or less ignore this spell. Until the spell ends, targets can communicate telepathically via this bond, whether or not they share a common language. Communication is possible at any distance, but it cannot extend over different planes of existence. \\
\textbf{For each Magic Critical Success rolled} in the Magic Check the duration is increased by 1 hour.

\medskip \hypertarget{lentezza}{\textbf{Lentezza}}\index[Incantesimi]{Slowness} \\
\textbf{School}: Transmutation \\
\textbf{Level}: 3, Uncommon \\
\textbf{Casting Time}: 2 Actions \\
\textbf{Range}: 36 meters \\
\textbf{Components}: V, S, M (a drop of molasses) \\
\textbf{Duration}: 1 minute, Concentration \\
You change the passage of time around up to six creatures of your choice in a 6-meter-edge cube at range. Each target must succeed on a Will saving throw or suffer the spell's effects for its duration. \\
The speed of a target subject to the spell is halved, it takes a -2 penalty on defense and Reflex saving throws, and it can't use reactions. During his round, he can use more than one Action or one Immediate Action, but not both. Whatever the creature's abilities or magical items, it can't make more than one melee or ranged attack during its round. \\
If the creature attempts to cast a spell with a casting time of 2 actions, roll a 1d6. On a 4 or more, the spell won't take effect until the creature's next round, and the creature will have to use 3 Actions in that round to complete the spell. If he cannot do so, the spell is wasted. \\
A creature affected by this spell makes another Will saving throw at the end of its round. If it succeeds at this saving throw, the effect ends.

\medskip \textbf{Levitation}\index[Incantesimi]{Levitation} \\
\textbf{School}: Air \\
\textbf{Level}: 2, Common \\
\textbf{Casting Time}: 2 Actions \\
\textbf{Range}: 18 meters \\
\textbf{Components}: V, S, M (either a small leather lace or a cup-shaped piece of gold cord folded with a long stem at the end) \\
\textbf{Duration}: 10 minutes \\
A creature or object at range that you can see, chosen by you, rises vertically up to 20 feet and remains suspended for the duration of the spell. The spell can levitate a target weighing up to 250 pounds. An unwilling creature that succeeds at a Fortitude saving throw ignores the effect. \\
The target can only move by pushing or pulling towards a fixed object or surface within range (such as a wall or ceiling). During your round, you can change the target's altitude by up to 20 feet in either direction. If you are the target, you can move up or down as part of your movement. Otherwise, you can use 1 Action to move the target, which must remain within the spell's range. When the spell ends, if the target is still in the air, it floats smoothly to the ground. \\
While under the influence of this spell you are considered Distracted in casting spells. \\
\textbf{For each Magic Critical Success you roll} in the Magic Check you can move 1 meter sideways or affect another creature.

\medskip \textbf{Magic Reading}\index[Incantesimi]{Magic Reading} \\
\textbf{School}: Universal \\
\textbf{Level}: 1, Common \\
\textbf{Casting Time}: 1 Action \\
\textbf{Range}: Touch \\
\textbf{Components}: V, S, M (a fragment of an enchanted item) \\
\textbf{Duration}: 1 minute, while used \\
The caster grants the ability to read a scroll or magical writing to a target. For the duration of 1 minute or as long as used once, whichever comes first, the creature will automatically be able to grasp a magical scroll or cast the contents of the scroll according to the criteria and rules for casting scroll spells.
\textbf{For each Magic Critical Success you roll} in the Magic Check you may read or understand an extra scroll.

\medskip \textbf{Freedom of Movement}\index[Incantesimi]{Freedom of Movement} \\
\textbf{School}: Abjuration \\
\textbf{Level}: 4, Common \\
\textbf{Casting Time}: 2 Actions \\
\textbf{Range}: Contact \\
\textbf{Components}: V, S, M (a strip of leather, wrapped around an arm or similar appendage) \\
\textbf{Duration}: 1 hour \\
You cast the spell on contact with a consenting creature. For its duration, the target's movement ignores hindering terrain, while spells or other magical effects cannot reduce its speed or cause the target to be paralyzed or hindered. \\
The target can use two Actions to automatically break free from any non-magical restrictions, such as handcuffs or a creature he's grabbed by. Finally, being underwater does not incur a penalty to the target's movement or attacks. \\
\textbf{For two magical critical success rolled} in the Magic Check you can affect another creature.

\medskip \textbf{Languages}\index[Incantesimi]{Languages} \\
\textbf{School}: Divination \\
\textbf{Level}: 3, Common \\
\textbf{Casting Time}: 2 Actions \\
\textbf{Range}: Contact \\
\textbf{Components}: V, M (a small clay model of a ziggurat) \\
\textbf{Duration}: 1 hour \\
This spell grants the creature you were in contact with at the time of casting the spell the ability to understand any spoken language it hears. Also, when the target speaks, any creature that knows at least one language and can hear the target understands what it says. \\
\textbf{For each magical critical success rolled} on the Magic Check the duration doubles or affects another creature.

\medskip \textbf{Locate Animals and Plants}\index[Incantesimi]{Locate Animals and Plants} \\
\textbf{School}: Animals and Plants \\
\textbf{Level}: 2, Uncommon \\
\textbf{Casting Time}: 2 Actions \\
\textbf{Duration}: Self \\
\textbf{Components}: V, S, M (a piece of fur from a hound) \\
\textbf{Duration}: Instantaneous \\
Describe or name a specific type of beast or plant. By focusing on the voice of nature in your surroundings, you learn the direction and distance to the closest creature or plant of that species, if any are within 7.5 kilometers. \\
\textbf{For each Magic Critical Success you roll} in the Magic Check you increase the controlled area by 1km

\medskip \textbf{Locate Creature}\index[Incantesimi]{Locate Creature} \\
\textbf{School}: Divination \\
\textbf{Level}: 4, Common \\
\textbf{Casting Time}: 2 Actions \\
\textbf{Duration}: Self \\
\textbf{Components}: V, S, M (a piece of hound fur) \\
\textbf{Duration}: Concentration, maximum 1 hour \\
Describe or name a creature that is familiar to you. You sense the direction of the creature's location, as long as that creature is within 300 meters of you. If the creature moves, you also know the direction of its movement. \\
The spell can locate a specific creature known to you, or the closest creature of a species (such as human or unicorn), as long as you've seen such a creature up close (within 30 feet) at least once. If the creature you describe or name has a different form, for example it is under the effects of the morph spell, this spell will not be able to locate the creature. \\
This spell cannot locate a creature if a stream of running water at least 10 feet wide blocks a direct path between you and the creature. \\
\textbf{For each Magic Critical Success rolled} in the Magic Check, increase the distance by another 300m.

\medskip \textbf{Locate Object}\index[Incantesimi]{Locate Object} \\
\textbf{School}: Divination \\
\textbf{Level}: 2, Common \\
\textbf{Casting Time}: 2 Actions \\
\textbf{Duration}: Self \\
\textbf{Components}: V, S, M (a forked twig) \\
\textbf{Duration}: Concentration, maximum 10 minutes \\
Describe or name an object that is familiar to you. You sense the direction of the object's location as long as that object is within 300 meters of you. If the object is moving, you also know the direction of its movement. \\
The spell can locate a specific object known to you as long as you have seen it up close (within 30 feet) at least once. Alternatively, the spell can locate the closest object of a particular type, such as certain types of clothing, jewelry, furniture, tools, or weapons. \\
This spell cannot locate an object if any thickness of lead, even a thin sheet of paper, blocks a direct path between you and the object. \\
\textbf{For each Magic Critical Success you roll} in the Magic Check you double the duration.

\medskip \textbf{Loquacity}\index[Incantesimi]{Loquacity} \\
\textbf{School}: Transmutation \\
\textbf{Level}: 8, Rare \\
\textbf{Casting Time}: 2 Actions \\
\textbf{Range}: Staff \\
\textbf{Components}: V \\
\textbf{Duration}: 1 hour \\
Until the spell ends, when you make a Charisma-based check you can replace the rolled number with 15. Also, no matter what you say, the spell or analysis that determines whether you are telling the truth will always indicate that you are honest. \\
\textbf{For each Magic Critical Success you roll} in the Magic Check you double the duration.

\medskip \textbf{Light}\index[Incantesimi]{Light} \\
\textbf{School}: Universal \\
\textbf{Level}: 1, Common \\
\textbf{Casting Time}: 2 Actions \\
\textbf{Range}: Contact \\
\textbf{Components}: V, M (a firefly or some phosphorescent moss) \\
\textbf{Duration}: 1 Turn per Magical Proficiency \\
You cast the spell on contact with an object that is no larger than 10 feet in any direction. Until the spell ends, the object radiates intense light within a 10-foot radius and twilight for an additional 10-feet. The light can be any color you want. Completely covering the object with something opaque blocks the light. If target object is held or worn by a hostile creature, that creature must make a Reflex saving throw to avoid the spell. A creature affected by the spell must make a Fortitude save or be blinded until the end of the next round. You cannot have more than one Light spell active at a time, a subsequent cast extinguishes the previous Light. \\
\textbf{For each Critical roll} in the Magic Check the duration is doubled.

\medskip \textbf{Daylight}\index[Incantesimi]{Daylight} \\
\textbf{School}: Invocation \\
\textbf{Level}: 3, Common \\
\textbf{Casting Time}: 2 Actions \\
\textbf{Range}: 18 meters \\
\textbf{Components}: V, S \\
\textbf{Duration}: 1 hour \\
A sphere of light with a radius of 30 feet expands from a point of your choice within range. The sphere radiates intense light and dim light for an additional 9 meters. If you pick a spot on an object that you are holding or that is not being worn or carried, the light radiates from the object and moves with it. Completely covering an object with something opaque, such as a vase or helmet, blocks the light. If any part of the area of this spell overlaps with the area of darkness created by a spell level 3 or lower, the spell that created the darkness is dispelled. The light created is considered to be sunlight.

\medskip \textbf{Dancing Lights} \index{Trick - Dancing Lights} \\
\textbf{School}: Invocation \\
\textbf{Level}: 1, Uncommon \\
\textbf{Casting Time}: 2 Actions \\
\textbf{Range}: 36 meters \\
\textbf{Components}: V, S, M (a piece of phosphorus or bewitched wood, or an earthworm) \\
\textbf{Duration}: 1 Turn \\
You create, at range, up to four lights the size of a torch, making them appear as torches, lanterns or luminous orbs that float in the air for the duration of the spell. You can also combine the four lights into a single medium-sized vaguely humanoid light form. Whichever shape you choose, each light emits a dim light within a 10-foot radius. As 1 move action during your round, you can move the lights up to 60 feet to a new point in range. \\
A light must be within 20 feet of another light created with this spell, and the lights vanish if they exceed the spell's range. \\
\textbf{For each Critical roll} in the Magic Check the duration is increased by 1 hour.

\medskip \textbf{Luminescence}\index[Incantesimi]{Luminescence} \\
\textbf{School}: Invocation \\
\textbf{Level}: 1, Uncommon \\
\textbf{Casting Time}: 2 Actions \\
\textbf{Range}: 18 meters \\
\textbf{Components}: V \\
\textbf{Duration}: 1 minute \\
All objects in a 6m edge cube at range are surrounded by a blue, green, or purple light (your choice). Any creature in the area when the spell is cast is also surrounded by light if it fails a Reflex saving throw. For the duration of the spell, objects and subject creatures emit a dim light with a 10-foot radius. Any attack roll against a subject creature or object has + 1d6 if the attacker can see it, and the creature or object cannot benefit from invisibility.

\medskip \textbf{Burning Hands}\index[Incantesimi]{Burning Hands} \\
\textbf{School}: Fire \\
\textbf{Level}: 1, Common \\
\textbf{Casting Time}: 2 Actions \\
\textbf{Range}: Staff (4 meter cone) \\
\textbf{Components}: V, S \\
\textbf{Duration}: Instantaneous \\
As you hold your hands with your thumbs touching and your fingers outstretched, a subtle burst of flame emanates from each of your fingertips. Each creature in a 5-meter cone must make a Reflex saving throw. A creature takes 1d4 damage per spell-like proficiency, up to a maximum of 5d4, fire damage if it fails its saving throw, or half if it succeeds. Fire ignites flammable objects in the area that are not worn or carried. \\
\textbf{For each magical critical success rolled} on the Magic Check the damage is increased by 1d4.

\medskip \textbf{Arcane Hand}\index[Incantesimi]{Arcane Hand} \\
\textbf{School}: Invocation \\
\textbf{Level}: 5, Uncommon \\
\textbf{Casting Time}: 2 Actions \\
\textbf{Range}: 36 meters \\
\textbf{Components}: V, S, M (an eggshell and a snakeskin glove) \\
\textbf{Duration}: Concentration, 1 minute \\
You create a Great hand, composed of transparent and luminous energy, in an unoccupied space at range and that you can see. The hand remains for the duration of the spell, and moves at your command, mimicking the movements of your hand. \\
The hand is an object that has Defense 25 and hit points equal to your maximum hit points. It has Strength 4 and Dexterity 0. The hand does not fill its space.
When you cast the spell and as 2 Actions during your subsequent rounds, you can move your hand up to 60 feet and then generate one of the following effects. \\

- \textit{Grabbing Hand}. The hand attempts to grab a Huge or smaller creature within 1 meter of it. To resolve the grapple action, you use Hand Strength. If the target is Medium or smaller, you have + 1d6 on the check. While the hand is gripping the target, you can use an Action to have the hand grasp the target. When you do, the target takes hit damage equal to 2d6 + your Intelligence or Wisdom value \\
- \textit{Hand of Strength}. The hand tries to push a 1 meter creature in a direction of your choice. Make a Strength check of the contested hand from the target's Strength check. If the target is Medium or smaller, you have + 1d6 on the check. If you win the contest, the hand pushes the target 1 meter plus 1 meter multiplied by the Intelligence or Wisdom value (minimum 1 meter). The hand moves with the target to stay within 1 meter of him. \\
- \textit{Mixed Hand}. The hand comes between you and a creature of your choice until you give it a different command. The hand moves so that it stays between you and the target, giving you half cover against the target. The target cannot move through the hand space if its Strength score is equal to or less than the Hand's Strength score. If its Strength score is higher than the Hand's Strength score, the target can move through the hand space, but treat that space as difficult terrain. \\
- \textit{Clenched Fist}. The hand hits a creature or object within 1 meter of it. Make a spell melee attack using your hand. If you hit, the target takes 4d8 points of force damage. \\

\medskip
\textbf{For each Magic Critical Success rolled} in the Magic Check the damage of the clenched fist option increases by 1d8 and the damage of the grabbing hand option increases by 1d6.

\medskip \textbf{Magic Hand} \index{Trick - Magic Hand} \\
\textbf{School}: Summon \\
\textbf{Level}: 0, Common \\
\textbf{Casting Time}: 2 Actions \\
\textbf{Range}: 9 meters \\
\textbf{Components}: V, S \\
\textbf{Duration}: 1d4 rounds +1 per point of magical proficiency \\
A floating ghostly hand appears at a point in range, chosen by you. The hand remains for the duration of the spell or until it is interrupted by an action. The hand vanishes if it is more than 30 feet away from you or if you cast the spell again. \\
The Actions required to move and use the magic hand are the same as you would use to use your hand. You can use your hand to manipulate an object, open an unlocked door or container, insert or retrieve an object from an open container, or pour out the contents of a vial. You can move your hand 30 feet each time you use it. The hand cannot attack, activate magic items, or carry more than 2 pounds. \\
\textbf{For each Magic Critical Success obtained} in the Magic Check the weight lifted increases by 1 kg or doubles the duration.

\medskip \textbf{Magic Mark} \index{Trick - Magic Mark} \\
\textbf{School}: Universal \\
\textbf{Level}: 0, Common \\
\textbf{Casting Time}: 2 Actions \\
\textbf{Range}: touch \\
\textbf{Components}: V, S \\
\textbf{Duration}: Permanent \\
This spell allows you to inscribe a personal rune or mark on an object. The writing cannot be more than 15 cm long. The writing can be visible or invisible depending on how you decide at the time of casting the spell.
A detect magic spell or read magic spell shows if invisible.
If the writing is placed on a creature it disappears within a month. \\
\textbf{For each Magic Critical Success obtained} in the Magic Check write one more logo.

\medskip \textbf{Message} \index{Trick - Message} \\
\textbf{School}: Transmutation \\
\textbf{Level}: 0, Common \\
\textbf{Casting Time}: 2 Actions \\
\textbf{Range}: 36 meters \\
\textbf{Components}: V, S, M (a small piece of copper wire) \\
\textbf{Duration}: 1 round \\
You point your finger at a creature in range and whisper a short message. The target (and only the target) hears the message and can reply in a whisper that only you can hear. \\
You can also cast this spell through solid objects if you are familiar with the target and know that the target is behind the barrier. Magical silence, 30 centimeters of stone, 2.5 centimeters of ordinary metal, a thin sheet of lead or 1 meter of wood block the spell. The spell does not have to follow a straight line, and can freely go around corners or through openings. \\
\textbf{For each magical critical success rolled} in the Magic Check the spell lasts 1 more round.

\medskip \textbf{Metamorphosis}\index[Incantesimi]{Metamorphosis} \\
\textbf{School}: Animals and Plants \\
\textbf{Level}: 4, Common \\
\textbf{Casting Time}: 2 Actions \\
\textbf{Range}: 18 meters \\
\textbf{Components}: V, S, M (a caterpillar cocoon) \\
\textbf{Duration}: 1 hour \\
This spell transforms a ranged creature, which you can see, into a new form. An unwilling creature must make a Will saving throw to avoid the effect. The shapeshifters automatically make a successful saving throw. The spell has no effect on a target with 0 hit points. \\
The transformation persists for the duration of the spell or until the target drops to 0 hit points or dies. The new form can be any beast whose challenge rating is half the spell's Magical Proficiency score (or sum of the Traits if Shayalia's Pious) of the caster. The target's game stats, including mental ability scores, are replaced by the stats of the chosen beast. However, he retains his traits and personalities. \\
The target retains the same hit points and regains 1d12 hit points in its new form. When it returns to its normal form, the creature retains the hit points it currently has. If it reaches 0, or less, hit points in the new form then it returns to normal and any effects affect the current form as well. \\
The creature is limited in the actions it can perform by the nature of its new form, and cannot speak, cast spells, or perform any other action that requires hands or speaking. The target's equipment merges into the new form. The creature cannot activate, use, wield, or otherwise benefit from its equipment.

\medskip \textbf{Pure Metamorphosis}\index[Incantesimi]{Pure Metamorphosis} \\
\textbf{School}: Animals and Plants \\
\textbf{Level}: 9, Rare \\
\textbf{Casting Time}: 2 Actions \\
\textbf{Range}: 9 meters \\
\textbf{Components}: V, S, M (a drop of mercury, a mound of gum arabic, and a puff of smoke) \\
\textbf{Duration}: 1 hour \\
Choose a creature or nonmagical object in range that you can see. The spell has no effect on a target with 0 hit points. You transform the creature into a different creature, the creature into an object, or the object into a creature (the object must not be worn or carried by another creature). The transformation persists for the duration of the spell or until the target drops to 0 hit points or dies. If you focus on this spell for the entire duration, the transformation becomes permanent. \\
The shapeshifters ignore this spell. An unwilling creature can make a Will saving throw and, if successful, ignore this spell's effect. \\

- \textit{Creature in Creature}. If you transform a creature into another species of creature, the new form can be any species you want, whose challenge rating is equal to or less than your Magical Proficiency score (or sum Traits in common if Devotee of Shayalia) . The target's game stats, including mental ability scores, are replaced by the new form's stats. However, he retains his traits and personality. \\
The target retains the same hit points and regains 1d12 hit points in its new form. When it returns to its normal form, the creature retains the hit points it currently has. If it reaches 0 or less hit points in the new form then it returns to normal and any effects also affect the current form. The creature is limited in the actions it can perform by the nature of its new form, and cannot converse, cast spells, or perform any other action that requires hands or speaking, unless the new form is capable of performing these actions. The target's equipment merges into the new form. The creature cannot activate, use, wield, or otherwise benefit from its equipment. \\

- \textit{Object into Creature.} You can transform an object into any type of creature, as long as the creature's size is no greater than the object's size and the creature's challenge rating is 9 or less. The creature is friendly towards you and your companions. It acts during your turns. You decide what actions it will perform and how it moves. The Storyteller has the creature's stats and will resolve all of its actions and movements.
If the spell becomes permanent, you lose control of the creature. Depending on how you have treated her, she may remain friendly towards you. \\

- \textit{Creature in Object}. If you transform a creature into an object, it transforms with whatever it is wearing or carrying. The creature's stats become those of the object, and after the spell ends and the creature returns to its normal form, it no longer has any memories of time spent as an object.

\medskip \textbf{Arcane Mirage}\index[Incantesimi]{Arcane Mirage} \\
\textbf{School}: Illusion \\
\textbf{Level}: 7, Rare \\
\textbf{Casting Time}: 10 minutes \\
\textbf{Range}: View \\
\textbf{Components}: V, S \\
\textbf{Duration}: 10 days \\
Make a patch of land at range, in a square area of up to 1.5 kilometers, appear, resonate, and smell like some other type of land. However, the general conformation of the terrain remains the same. Open fields or a road can be turned into a swamp, hills, crevasse, or some other type of difficult or impassable terrain. A pond can be transformed into a grassy clearing, a precipice on a gentle slope, a rock strewn ravine into a wide and smooth road. \\
Likewise, you can change the appearance of structures, or add them where there aren't any. The spell does not disguise, conceal, or add creatures. \\
The illusion encompasses auditory, visual, tactile and olfactory elements, so that clear terrain can be transformed into difficult terrain (or vice versa) or otherwise prevent movement in the area. Any piece of illusory terrain (such as a stone or a staff) that is removed from the spell's area immediately vanishes. True-seeing creatures can see beyond the illusion and discern the true shape of the terrain; however, the other elements of the illusion remain, so, although the creature is aware of the presence of the illusion, it can still physically interact with it. \\
\textbf{With three magical critical successes rolled} on the Magic Check the duration is permanent.

\medskip \textbf{Modify Memory}\index[Incantesimi]{Modify Memory} \\
\textbf{School}: Enchantment \\
\textbf{Level}: 5, Very Rare \\
\textbf{Casting Time}: 2 Actions \\
\textbf{Range}: 9 meters \\
\textbf{Components}: V, S \\
\textbf{Duration}: Concentration, maximum 1 minute \\
You try to reshape the memories of another creature. A creature you can see must make a Will saving throw. If you're fighting it, the creature has + 1d6 on the saving throw. If the saving throw fails, the target becomes fascinated with you for the duration of the spell. The fascinated target is incapacitated and unaware of their surroundings, although they are still able to hear you. If it takes damage or becomes the target of another spell, that spell ends, and none of the target's memories are changed. \\
While the target is fascinated by this spell, you can act on the target's memories of an event that has been in the past 24 hours and has lasted no more than 10 minutes. You can permanently delete all memories of the event, allow the target to remember the event with perfect clarity and detail, change the memory of the event details, or create the memory of another event. You need to be able to speak to the target to describe how their memories will be affected, and they need to be able to understand your language in order for the modified memories to settle into their memory. If the spell ends before you finish describing the modified memories, the creature's memory is not affected. Otherwise, the modified memories are established at the end of the spell. \\
A modified memory does not necessarily affect the creature's behavior, particularly if its memories contradict the creature's natural inclinations, traits, or faith. An illogically modified memory, such as implanting the memory of how much the creature loves immersing itself in acid, is removed, as if it were a bad dream. The Storyteller may find a modified memory too foolish to have any effect on a creature. A remove curse or greater restore spell cast on the target restores the target's true memories. \\
\textbf{For each Magic Critical Success you roll} in the Magic Check, you can alter a target's memories of an event that occurred up to 7 days ago, 30 days ago, 1 year ago, or anywhere in the creature's past.

\medskip \textbf{Spider Movements}\index[Incantesimi]{Spider Movements} \\
\textbf{School}: Transmutation \\
\textbf{Level}: 2, Uncommon \\
\textbf{Casting Time}: 2 Actions \\
\textbf{Range}: Contact \\
\textbf{Components}: V, S, M (a drop of bitumen and a spider) \\
\textbf{Duration}: 10 minutes \\
You cast the spell on contact with a consenting creature. Until the spell ends, the creature gains the ability to move up, down, and along vertical surfaces or by standing upside down on the ceiling, keeping its hands free. The target also gains climb speed equal to its movement speed. The creature subject to the spell is considered Distracted when casting other spells. \\

\medskip \textbf{Move Terrain}\index[Incantesimi]{Move Terrain} \\
\textbf{School}: Earth \\
\textbf{Level}: 6, Uncommon \\
\textbf{Casting Time}: 2 Actions \\
\textbf{Range}: 36 meters \\
\textbf{Components}: V, S, M (an iron shovel and a small bag containing a mixture of soil types - clay, manure and sand) \\
\textbf{Duration}: Concentration, maximum 2 hours \\
Choose an area on the terrain at range, no larger than 12 meters on each side. For the duration, you can reshape topsoil, sand, or clay in the area in any way you like. You can raise or lower the altitude of the area, create or fill a moat, erect or lower a wall, or form a pillar. The extent of these changes cannot exceed half the largest size of the area. Thus, if you operate on a square of 12 meters on each side, you can create a pillar 6 meters high, raise or lower the altitude of the ground by 6 meters, dig a ditch 6 meters deep, and so on. It takes 10 minutes to complete these changes. At the end of every 10 minutes spent concentrating on the spell, you can choose a new area of terrain to operate on. \\
Since the transformation of the terrain occurs slowly, creatures in the area usually cannot be trapped or injured by the movement of the terrain. The spell cannot manipulate natural stone or stone buildings. The rocks and structures move to adapt to the new terrain. If the way you shape the terrain would make a structure unstable, it could collapse. Likewise, this spell does not directly affect plant growth. The loose earth carries any vegetable present with it.

\medskip \textbf{Wall of Force}\index[Incantesimi]{Wall of Force} \\
\textbf{School}: Invocation \\
\textbf{Level}: 5, Common \\
\textbf{Casting Time}: 2 Actions \\
\textbf{Range}: 36 meters \\
\textbf{Components}: V, S, M (a pinch of powder produced by shattering a transparent gem) \\
\textbf{Duration}: 10 minutes \\
An invisible wall of force forms at a point of range chosen by you. The wall appears in any orientation you want, such as a horizontal or vertical barrier or an angled one. It can float in the air or rest on a solid surface. You can give it the shape of a hemispherical dome or sphere with a maximum radius of 3 meters, or give it the appearance of a flat surface made up of up to ten panels measuring 3 meters by 3 meters. Each panel must be contiguous to another panel. In any form, the wall is 75 centimeters thick and remains for the duration of the spell. If the wall cuts a space on a creature, when it appears, the creature is pushed to one side of the wall (at your discretion). Nothing can physically cross the wall. It is immune to all damage and cannot be dispelled by dispel magic. However, the wall is instantly destroyed by the disintegrate spell. The wall also extends over the Ethereal Plane, preventing ethereal travelers from crossing it.

\medskip \textbf{Wall of Fire}\index[Incantesimi]{Wall of Fire} \\
\textbf{School}: Fire \\
\textbf{Level}: 4, Common \\
\textbf{Casting Time}: 2 Actions \\
\textbf{Range}: 36 meters \\
\textbf{Components}: V, S, M (a small piece of phosphorus) \\
\textbf{Duration}: 1 minute \\
You create a wall of fire on a solid surface at range. You can create a wall up to 18 meters long, up to 6 meters high and 30 centimeters thick, or a circular wall that is 6 meters in diameter, 6 meters high and 30 centimeters thick. The wall is opaque and remains for the duration of the spell. \\
When the wall appears, each creature in its area must make a Reflex saving throw. A creature takes 5d8 points of fire damage on a failed save, or half if it succeeds. One side of the wall, selected by you when you cast this spell, deals 5d8 points of fire damage to each creature that ends its round within 10 feet of that side or within the wall. A creature takes the same damage when it enters the wall for the first time during a round. The other side of the wall does no damage. \\
\textbf{For each magical critical success rolled} on the Magic Check the damage is increased by 1d8.

\medskip \textbf{Wall of Ice}\index[Incantesimi]{Wall of Ice} \\
\textbf{School}: Water \\
\textbf{Level}: 6, Common \\
\textbf{Casting Time}: 2 Actions \\
\textbf{Range}: 36 meters \\
\textbf{Components}: V, S, M (a small piece of quartz) \\
\textbf{Duration}: 10 minutes \\
Create a wall of ice on a solid surface at range. You can create a hemispherical dome or a sphere with a maximum radius of 3 meters, or you can create a flat surface consisting of up to ten square panels of 3 meters on each side. Each panel must be contiguous to at least one other panel. In each form, the wall is 12 inches thick and remains for the duration of the spell. \\
If, when it appears, the wall crosses a creature's space, the creature is pushed to one side of the wall (your choice) and must make a Reflex saving throw. On a failed save, the creature takes 10d6 cold damage, or half that damage if it succeeds. \\
The wall is an object that can be damaged and broken through. Each 10-foot section has Defense 12 and 30 hit points, and is vulnerable to fire damage. Reducing a 10-foot section to 0 hit points destroys it and leaves a breeze of freezing wind in the space that was occupied by the wall. A creature that moves through this breeze of chill wind for the first time in a round must make a Fortitude save. If it fails, the creature takes 5d6 cold damage, or half that damage if it succeeds. \\
\textbf{For each Magic Critical Success rolled} in the Magic Check the damage is increased by 1d8.

\medskip \textbf{Wall of Stone}\index[Incantesimi]{Wall of Stone} \\
\textbf{School}: Invocation \\
\textbf{Level}: 5, Common \\
\textbf{Casting Time}: 2 Actions \\
\textbf{Range}: 36 meters \\
\textbf{Components}: V, S, M (a small block of granite) \\
\textbf{Duration}: 10 minutes \\
A non-magical solid stone wall forms at a point in range, chosen by you. The wall is 15 centimeters thick and is made up of 10 panels of 3 by 3 meters. Each panel must be contiguous to at least one other panel. Alternatively, you can create 3 x 6 meter panels that are only 7.5 centimeters thick. \\
If the wall crosses a creature's space when it appears, the creature is pushed to one side of the wall (your choice). If the creature was surrounded on all sides by the wall (or by the wall and another solid surface), the creature can make a Reflex saving throw. If she passes it, she can use her Reaction Action to move at her speed so that she is no longer trapped in the wall. \\
The wall can be any shape you want, although it cannot occupy the same space as a creature or object. The wall may also not be vertical or rest on a plane. It must, however, merge with and be supported by existing stone. Then, you can use this spell to bridge a chasm or create a ramp. \\
If you create such a non-vertical wall, longer than 6 meters, you must halve the size of each panel to create supports. You can roughly shape the stone to create battlements, battlements, and so on. The wall is an object made of stone that can be damaged and broken through. Each tile has Defense 15, Hardness 15, and Hit Points 15 per inch of thickness. Reducing a panel to 0 hit points destroys it and may cause connected panels to collapse, at the discretion of the Storyteller. If you keep your concentration on this spell for its entire duration, the wall becomes permanent and cannot be dispelled. Otherwise, the wall disappears when the spell ends.

\medskip \textbf{Prismatic Wall}\index[Incantesimi]{Prismatic Wall} \\
\textbf{School}: Abjuration \\
\textbf{Level}: 9, Rare \\
\textbf{Casting Time}: 2 Actions \\
\textbf{Range}: 18 meters \\
\textbf{Components}: V, S \\
\textbf{Duration}: 10 minutes \\
A plane of bright, multicolored lights forms an opaque vertical wall, up to 27 meters wide, 9 meters high and 2.5 centimeters thick, centered on a range point that you can see. Alternatively, you can model the wall into a sphere, up to 9 meters in diameter, centered on a range point of your choice. The wall remains fixed in place for the duration of the spell. If you position the wall so that it crosses the space occupied by a creature, the spell fails and the spell slot is wasted. The wall radiates intense light up to a range of 30 meters and dim light for an additional 30 meters. You and the creatures indicated by you when casting the spell can cross and stay close to the wall without danger. If another creature that can see the wall moves within 20 feet of it or begins its round there, it must succeed at a Fortitude save or be blinded for 1 minute. The wall consists of seven layers, each of a different color. When a creature tries to dive or cross the wall, it does so one layer at a time, through all the layers of the wall. As it submerges or traverses each layer, the creature must succeed on a Reflex saving throw or suffer the properties of each layer, one at a time, as described below. \\
The wall can be destroyed, one layer at a time, in order from red to violet, in a specific way for each layer. Once a layer is destroyed, it will be destroyed for the duration of the spell. An erasing rod destroys a prismatic wall, but an anti-magic field has no effect on it. \\

- \textit{1. Red}. The target takes 10d6 points of fire damage on a failed save, or half that damage if it succeeds. As long as this layer exists, non-magical ranged attacks cannot pass through the wall. The layer can be destroyed by dealing 25 Cold damage to it. \\
- \textit{2. Orange}. The target takes 10d6 points of acid damage on a failed save, or half that damage on a successful save. As long as this layer exists, magical ranged attacks cannot pass through the wall. The layer can be destroyed by a strong wind. 3. Yellow. The target takes 10d6 points of lightning damage on a failed save, or half that damage if a successful one. This layer can be destroyed by inflicting 60 Force damage on it. \\
- \textit{4. Green}. The target takes 10d6 points of poison damage on a failed save, or half that damage on a successful save. A wall pass spell, or another spell of the same level or higher that can open a portal on a solid surface, destroys this layer. \\
- \textit{5. Blue}. The target takes 10d6 cold damage on a failed save, or half that damage if it succeeds. The layer can be destroyed by inflicting at least 25 Fire damage to it. \\
- \textit{6. Indigo}. If the saving throw fails, the target is in the way. He then must make a Fortitude saving throw at the start of each of his rounds. On a successful save three times, the spell ends. If he fails his saving throw three times, he is permanently turned to stone and becomes a victim of the petrified condition. Successes and failures do not have to be consecutive; keep track of both until the target has three of the same type. As long as this layer exists, spells cannot be cast through the wall. The layer is destroyed by the intense light emanating from the daylight spell or similar spell of a higher level. \\
- \textit{7. Violet}. If the saving throw fails, the target is blinded. It must then make a Will saving throw at the start of your next round. On a successful saving throw, the blindness ends. If the saving throw fails, the creature is transported to another plane of existence of the Storyteller's choice and is no longer blinded (usually, a creature not on its home plane is exiled to it, while other creatures are usually cast into the Astral or Ethereal planes). This layer is destroyed by the dispel magic spell or a similar spell of the same level or higher that can end spells and magical effects.

\medskip \textbf{Wall of Thorns}\index[Incantesimi]{Wall of Thorns} \\
\textbf{School}: Animals and Plants \\
\textbf{Level}: 6, Uncommon \\
\textbf{Casting Time}: 2 Actions \\
\textbf{Range}: 36 meters \\
\textbf{Components}: V, S, M (a handful of thorns) \\
\textbf{Duration}: maximum 10 minutes \\
You create a wall of sturdy, malleable, entangled bushes filled with sharp thorns. The wall appears at range on a solid surface and remains for the duration of the spell. The wall you can create can be up to 18 meters long, up to 3 meters high, and up to 1 meter thick or a circle that is 6 meters in diameter and up to 6 meters high and 1 meter thick. The wall blocks line of sight. \\
When the wall appears, each creature in its area must make a Reflex saving throw. On a failed save, a creature takes 7d8 points of piercing damage, or half that damage if it succeeds. A creature can move through the wall, albeit slowly and painfully. For every 1 meter the creature moves through the wall, it must spend 6 meters of movement. In addition, the first time a creature enters the wall during a round or ends its round inside it, the creature must make a Reflex saving throw. It takes 7d8 slashing damage on a failed save, or half that damage if it succeeds. \\
\textbf{For each magical critical success rolled} on the Magic Check the damage is increased by 1d8.

\medskip \textbf{Wall of Wind}\index[Incantesimi]{Wall of Wind} \\
\textbf{School}: Aria \\
\textbf{Level}: 3, Uncommon \\
\textbf{Casting Time}: 2 Actions \\
\textbf{Range}: 36 meters \\
\textbf{Components}: V, S, M (a tiny fan and a feather of exotic origins) \\
\textbf{Duration}: 1 minute \\
A strong wind wall rises from the ground at a range point of your choice. You can create a wall up to 15 meters long, 4 meters high and 30 centimeters thick. You can shape the wall in any way you want as long as it makes a continuous path on the ground. The wall remains for the duration of the spell. When the wall appears, each creature within its area must make a Fortitude saving throw. A creature takes 3d8 damage from a hit on a failed save, or half that damage if it succeeds. Strong wind keeps haze, smoke and other gases away. Small or smaller flying creatures cannot pass through the wall. Lightweight materials dragged into the wall fly upward. Arrows, bolts, and other normal ammunition are deflected and automatically miss their target (boulders thrown by giants and siege engines, and similar ammunition, ignore their effects instead). Creatures in gaseous form cannot pass through it. \\
\textbf{For each Magic Critical Success rolled} in the Magic Check the duration is increased by 1 minute.

\medskip \textbf{Incendiary Cloud}\index[Incantesimi]{Incendiary Cloud} \\
\textbf{School}: Fire \\
\textbf{Level}: 8, Rare \\
\textbf{Casting Time}: 2 Actions \\
\textbf{Range}: 45 meters \\
\textbf{Components}: V, S \\
\textbf{Duration}: 1 minute \\
A swirling cloud of smoke crossed by incandescent lapilli forms in a sphere with a radius of 6 meters centered on a point at range. The cloud spreads around the corners and is in dim light. It remains for the duration of the spell or until a wind of moderate or greater speed (at least 15 kilometers per hour) disperses it. \\
When the cloud appears, each creature within it must make a Reflex saving throw. A creature takes 10d8 points of fire damage on a failed save, and half that damage on a successful save. A creature must make the saving throw even when it first enters the area or ends its round there. \\
At the start of each of your rounds, the cloud moves 10 feet away from you in a direction of your choice.

\medskip \textbf{Stinking Cloud}\index[Incantesimi]{Stinking Cloud} \\
\textbf{School}: Water, Air \\
\textbf{Level}: 3, Uncommon \\
\textbf{Casting Time}: 2 Actions \\
\textbf{Range}: 27 meters \\
\textbf{Components}: V, S, M (a rotten egg or stinking cabbage leaves) \\
\textbf{Duration}: 10 minutes \\
You create, at a point at range, a sphere of 6 meters of radius composed of a yellow and nauseating gas. The cloud spreads around corners and its area is in dim light. The cloud remains in the air for the duration. Any creature completely inside the cloud at the start of its round must make a Fortitude save against the poison. If the saving throw fails, the creature spends 2 Actions of that round vomiting and staggering. Creatures that do not need to breathe or that are immune to poison automatically succeed at the saving throw. A moderate wind (at least 15km / h) disperses the cloud after 4 rounds. A strong wind (at least 30km / h) blows it away after 1 round.

\medskip \textbf{Cloud of Death}\index[Incantesimi]{Cloud of Death} \\
\textbf{School}: Water, Air \\
\textbf{Level}: 5, Rare \\
\textbf{Casting Time}: 2 Actions \\
\textbf{Range}: 36 meters \\
\textbf{Components}: V, S \\
\textbf{Duration}: 10 minutes \\
You create a 20-foot-radius sphere formed from a yellow-green poisonous mist centered at a range point of your choice. The fog spreads around the corners. It remains for the duration of the spell or until a strong wind clears the mist, ending the spell. Its area is in dim light. When a creature enters the spell's area for the first time in a round or begins its round there, that creature must make a Fortitude saving throw. The creature takes 5d8 points of poison damage on a failed save, or half that damage if it succeeds. Creatures are affected even if they hold their breath or do not need to breathe. The fog travels 10 feet away from you at the start of each of your rounds, moving along the surface of the ground. The vapors, being heavier than air, tend to descend downwards, even going so far as to creep into the openings. \\
\textbf{For each magical critical success rolled} on the Magic Check the damage is increased by 1d8.

\medskip \textbf{Cloud of Mist}\index[Incantesimi]{Cloud of Mist} \\
\textbf{School}: Water, Air \\
\textbf{Level}: 1, Common \\
\textbf{Casting Time}: 2 Actions \\
\textbf{Range}: 36 meters \\
\textbf{Components}: V, S \\
\textbf{Duration}: 1 hour \\
You create a sphere of haze with a radius of 6 meters centered on a point at range. The sphere propagates around the corners, and its area is in dim light. It remains for the duration of the spell or until a wind of moderate or greater speed (at least 15 kilometers per hour) disperses it. \\
\textbf{For each Magic Critical Success rolled} in the Magic Check, the radius of the haze increases by 20 feet.

\medskip \textbf{Arcane Eye}\index[Incantesimi]{Arcane Eye} \\
\textbf{School}: Divination, Common \\
\textbf{Level}: 4 \\
\textbf{Casting Time}: 2 Actions \\
\textbf{Range}: 9 meters \\
\textbf{Components}: V, S, M (a piece of bat's coat) \\
\textbf{Duration}: Concentration, maximum 1 hour \\
You create an invisible magical eye at range, which floats in the air for the duration of the spell. \\
Mentally receive visual information from the eye, which has normal vision and darkvision up to 30 feet. The eye can look in all directions. With a move action, you can move the eye 30 feet in any direction. There is no limit to how far the eye can move, but it cannot enter another plane of existence. A solid barrier blocks the movement of the eye, but the eye can pass through an opening as small as 2.5 centimeters in diameter.

\medskip \textbf{Thundering Wave}\index[Incantesimi]{Thundering Wave} \\
\textbf{School}: Air \\
\textbf{Level}: 1, Common \\
\textbf{Casting Time}: 2 Actions \\
\textbf{Range}: Staff (4-meter edge cube) \\
\textbf{Components}: V, S \\
\textbf{Duration}: Instantaneous \\
A wave of thundering force is projected from you. Each creature in a 10-foot-edged cube that originates from you must make a Fortitude saving throw. On a failed saving throw, a creature takes 2d8 points of sonic damage and is 10 feet away from you. On a successful saving throw, the creature takes half the damage and isn't driven away. Additionally, non-anchored objects that are totally within the area are pushed 10 feet away from you by the spell's effect, and the spell produces a thundering rumble that is audible up to 90 meters. \\
\textbf{For each magical critical success rolled} on the Magic Check the damage increases by 1d8.

\medskip \textbf{Darkness}\index[Incantesimi]{Darkness} \\
\textbf{School}: Invocation \\
\textbf{Level}: 1, Common \\
\textbf{Casting Time}: 2 Actions \\
\textbf{Range}: 18 meters \\
\textbf{Components}: V, M (bat hair and a pinch of bitumen or a piece of coal) \\
\textbf{Duration}: 10 minutes \\
The magical darkness spreads from a point at range, chosen by you, to fill a sphere of 4 meters in radius for the duration of the spell. Darkness spreads around the corners. A creature with darkvision cannot see in this darkness, and nonmagical light cannot illuminate it. \\
If the spot you have chosen is on an object you are carrying or one that is not being worn or carried, darkness emanates from the object and moves with it. Completely covering the source of darkness with an opaque object, such as a vase or helmet, blocks the darkness. \\
If any part of this spell's area overlaps with the area of light created by a spell with difficulty 13 or lower, the spell that created the light is dispelled.

\medskip

\begin{changemargin}{0.3cm}{0.3cm} \begin{enfasi}{
I scatter around to avoid area spells (called by a player to avoid a Fireball)
} \end{enfasi} \end{changemargin}

\medskip \textbf{Fireball}\index[Incantesimi]{Fireball} \\
\textbf{School}: Fire \\
\textbf{Level}: 3, Common \\
\textbf{Casting Time}: 2 Actions \\
\textbf{Range}: 45 meters \\
\textbf{Components}: V, S, M (a tiny ball of bat guano and sulfur) \\
\textbf{Duration}: Instantaneous \\
A beam of yellow light starts from your finger pointing at a range point of your choice and then explodes with a thunderous roar and turns into a tongue of flames. \\
Each creature in a 20-foot-radius sphere centered there must make a Reflex saving throw. A creature takes 8d6 points of fire damage on a failed save, or half that damage if it succeeds. \\
The fire spreads and occupies the entire available volume within 6 meters from the explosion point. Fire ignites flammable objects in the area that are not worn or carried. \\
\textbf{For each magical critical success rolled} in the Magic Check the base damage is increased by 1d6. \\
\textbf{Saving Throw Success / Critical Failure}: On critical failure the damage doubles, on critical success the damage is further halved

\medskip \textbf{Delayed Fireball}\index[Incantesimi]{Delayed Fireball} \\
\textbf{School}: Fire \\
\textbf{Level}: 7, Rare \\
\textbf{Casting Time}: 2 Actions \\
\textbf{Range}: 45 meters \\
\textbf{Components}: V, S, M (a large ball of bat guano and sulfur) \\
\textbf{Duration}: Concentration, 1 minute \\
A beam of yellow light emanates from your pointing finger, condenses for the duration of the spell in the form of a ball of light at a range point of your choice. When the spell ends, either because your concentration is broken or because you decide to end it, the ball explodes with a soft roar and turns into a jet of flame that spreads around corners. Each creature in a 20-foot-radius sphere centered there must make a Reflex saving throw. A creature takes fire damage equal to the total accumulated damage if it fails its saving throw, or half that damage if it succeeds. The spell's base damage is 12d6. If at the end of your round the ball hasn't detonated yet, the damage increases by 1d6. \\
If the glowing ball is touched before the spell ends, the creature touching it must make a Reflex saving throw. On a failed saving throw, the spell ends immediately, causing the ball to erupt. If the saving throw is successful, the creature can throw the ball up to 40 feet away. When it hits a creature or solid object, the spell ends and the ball explodes. \\
The fire damages objects in the area and ignites flammable objects that are not worn or carried. \\
\textbf{For each magical critical success rolled} on the Magic Check the damage is increased by 1d6. \\
\textbf{Saving Throw Success / Critical Failure}: On critical failure the damage doubles, on critical success the damage is further halved.

\medskip \textbf{Talking with Animals}\index[Incantesimi]{Talking with Animals} \\
\textbf{School}: Animals and Plants \\
\textbf{Level}: 1, Common \\
\textbf{Casting Time}: 2 Actions \\
\textbf{Duration}: Self \\
\textbf{Components}: V, S \\
\textbf{Duration}: 10 minutes \\
For the duration of the spell, you gain the ability to understand and communicate verbally with beasts. The knowledge and awareness of many beasts are limited by their intellect but, at a minimum, the beasts can provide you with information about nearby places and monsters, including those they may perceive or have sensed in days past. At the Storyteller's discretion, you may be able to persuade a beast to do you a small favor. \\
\textbf{For each magical critical success rolled} in the Magic Check the duration is doubled.

\medskip \textbf{Talking with the Dead}\index[Incantesimi]{Talking with the Dead} \\
\textbf{School}: Necromancy \\
\textbf{Level}: 3, Rare \\
\textbf{Casting Time}: 2 Actions \\
\textbf{Range}: 3 meters \\
\textbf{Components}: V, S, M (incense lit) \\
\textbf{Duration}: 10 minutes \\
Give a semblance of life and Intelligence to a corpse in range, chosen by you, by allowing it to answer the questions you ask. The corpse must still have a mouth and cannot be undead. The spell fails if the corpse has already been the target of this spell in the past 10 days. Until the spell ends, you can ask the corpse up to five questions. The corpse only knows what it already knew in life, including spoken languages. The answers are usually short, cryptic, or repetitive, and the corpse is under no obligation to give you truthful answers if you are hostile to him or recognize you as his enemy. This spell does not return the creature's soul to the body, only the spirit that moves it. As a result, the corpse cannot learn new information, understand nothing of what has happened since he died, and cannot make assessments of future events.

\medskip \textbf{Talking with Plants}\index[Incantesimi]{Talking with Plants} \\
\textbf{School}: Animals and plants \\
\textbf{Level}: 3, Rare \\
\textbf{Casting Time}: 2 Actions \\
\textbf{Range}: Personnel (9 meters radius) \\
\textbf{Components}: V, S \\
\textbf{Duration}: 10 minutes \\
Infuse plants within 30 feet of you with sentience and limited mobility, giving them the ability to communicate with you and execute simple commands. You can interrogate the plants about events that occurred on the last day in the spell area, obtaining information about passing creatures, the weather, and more. You can also turn difficult soil produced by plant growth (such as bushes and thick undergrowth) into ordinary soil for the duration of the spell. \\
Or you can transform normal terrain with plants into difficult terrain, which remains for the duration of the spell by causing vines and branches to slow down pursuers, for example. \\
At the Storyteller's discretion, the plants may also perform other tasks on your behalf. The spell does not allow plants to uproot and move, but they can move branches, stems and stems freely. If a plant creature is in the area, you can communicate with it as if you spoke the same language, but you gain no magical ability to affect it. This spell can cause the plants created by the entangle spell to release an entangled creature.

\medskip \textbf{Divine Word}\index[Incantesimi]{Divine Word} \\
\textbf{School}: Invocation \\
\textbf{Level}: 7, Very Rare \\
\textbf{Casting Time}: 1 Immediate Action \\
\textbf{Range}: 9 meters \\
\textbf{Components}: V \\
\textbf{Duration}: Instantaneous \\
Speak a divine word, infused with the power of your Patron. Choose any number of creatures within range that you can see. Any creature that can hear you must make a Will saving throw. If the saving throw fails, the creature takes an effect based on its current hit points: \\

- 100 Hit Points or less: Stunned for 1 minute \\
- 40 hit points or less: deafened and blinded for 10 minutes \\
- 30 Hit Points or less: Blinded, deafened and stunned for 1 hour \\
- 20 Hit Points or Less: Killed Instantly \\

Whatever his current hit points, a celestial, elemental, fairy, or demon who fails his saving throw is forced to return to his plane of origin (if he is not there already) and cannot return to your current plane before they are after 24 hours, unless you use the wish spell.

\medskip \textbf{Healing Word}\index[Incantesimi]{Healing Word} \\
\textbf{School}: Healing \\
\textbf{Level}: 1, Uncommon \\
\textbf{Casting Time}: 1 Immediate Action \\
\textbf{Range}: 18 meters \\
\textbf{Components}: V \\
\textbf{Duration}: Instantaneous \\
A creature at range you can see, chosen by you, regains hit points equal to 1d4 + your ability modifier for spellcasting spells. This spell deals the same amount of damage to an undead. \\
\textbf{For each Magic Critical Success rolled} on the Magic Check the healing is increased by 1d4. \\
If the spellcaster and the healed creature are both Followers of the same Patron, the spell heals 1d4 more. \\
If the caster and the healed creature are both devoted to the same Patron, the spell heals 2d4 more.

\medskip \textbf{Mass Healing Word}\index[Incantesimi]{Mass Healing Word} \\
\textbf{School}: Healing \\
\textbf{Level}: 3, Rare \\
\textbf{Casting Time}: 1 Immediate Action \\
\textbf{Range}: 18 meters \\
\textbf{Components}: V \\
\textbf{Duration}: Instantaneous \\
As you speak words of healing, up to six creatures at range you can see, chosen by you, regain hit points equal to 1d4 + your ability modifier for spells. This spell deals the same amount of damage to the undead. \\
\textbf{For each Magic Critical Success rolled} on the Magic Check the healing is increased by 1d4. \\
If the spellcaster and the healed creature are both Followers of the same Patron, the spell heals 1d4 more. \\
If the caster and the healed creature are both devoted to the same Patron, the spell heals 2d4 more.

\medskip \textbf{Power Word Stun}\index[Incantesimi]{Power Word Stun} \\
\textbf{School}: Enchantment \\
\textbf{Level}: 8, Uncommon \\
\textbf{Casting Time}: 1 Immediate Action \\
\textbf{Range}: 18 meters \\
\textbf{Components}: V \\
\textbf{Duration}: 1 minutes \\
You speak a word of power that can overwhelm the mind of a creature in range and that you can see, leaving it confused. If the target has 150 hit points or less, they are stunned. Otherwise, the spell has no effect.

\medskip \textbf{Word of Power to Kill}\index[Incantesimi]{Word of Power to Kill} \\
\textbf{School}: Enchantment \\
\textbf{Level}: 9, Rare \\
\textbf{Casting Time}: 1 Immediate Action \\
\textbf{Range}: 18 meters \\
\textbf{Components}: V \\
\textbf{Duration}: Instantaneous \\
You speak a word of power that forces a creature in range you can see to die instantly. If the creature you choose has 100 hit points or less, it dies. Otherwise, the spell has no effect.

\medskip \textbf{Word of Retreat}\index[Incantesimi]{Word of Retreat} \\
\textbf{School}: Summon \\
\textbf{Level}: 6, Rare \\
\textbf{Casting Time}: 2 Actions \\
\textbf{Range}: 1 meter \\
\textbf{Components}: V \\
\textbf{Duration}: Instantaneous \\
You and up to five consenting creatures within 1 meter of you instantly teleport you to a safe location mentioned earlier, called the sanctuary. You and all creatures teleported with you reappear in the unoccupied space closest to the spot indicated when you set up this shrine (see below). If you cast this spell without first preparing a shrine, the spell has no effect. \\
You must indicate a shrine, whether it is dedicated or strongly connected to your Patron. If you attempt to cast the spell because you are taking yourself to an area that is not dedicated by your Patron, the spell has no effect.

\medskip \textbf{Passapareti}\index[Incantesimi]{Passapareti} \\
\textbf{School}: Earth \\
\textbf{Level}: 5, Uncommon \\
\textbf{Casting Time}: 2 Actions \\
\textbf{Range}: 9 meters \\
\textbf{Components}: V, S, M (a pinch of sesame seeds) \\
\textbf{Duration}: 1 hour \\
For the duration of the spell, a passage appears at a point in range that you can see, on a surface of wood, wall, or stone (such as a wall, ceiling, or floor) of your choice. Choose the size of the opening: maximum 1 meter wide, 2.4 meters high and 6 meters deep. The passage does not create instability in the surrounding structure. \\
When the opening disappears, any creatures or objects still in the passage created by the spell are ejected safely into the unoccupied space closest to the surface you cast the spell on.

\medskip \textbf{Pass Without a Trace}\index[Incantesimi]{Pass Without a Trace} \\
\textbf{School}: Earth, Animals and Plants \\
\textbf{Level}: 2, Common \\
\textbf{Casting Time}: 2 Actions \\
\textbf{Duration}: Self \\
\textbf{Components}: V, S, M (ashes of a burnt mistletoe leaf and a spruce twig) \\
\textbf{Duration}: Concentration, 1 hour
For the duration of the spell, your tracks cannot be tracked except by magical means. The creature receiving this bonus leaves no traces or other signs of its passage. \\
\textbf{For each magical critical success rolled} in the Magic Check, you may include another creature in the spell's benefits.

\medskip \textbf{Veiled Step}\index[Incantesimi]{Veiled Step} \\
\textbf{School}: Summon \\
\textbf{Level}: 2, Uncommon \\
\textbf{Casting Time}: 1 Immediate Action \\
\textbf{Duration}: Self \\
\textbf{Components}: V \\
\textbf{Duration}: Instantaneous \\
Swiftly enveloped in a silver haze, you teleport up to 30 feet into an unoccupied space that you can see. \\
\textbf{If you roll two magical critical successes rolled} in the Magic Check, you can trade with a willing creature.

\medskip \textbf{Quick Step}\index[Incantesimi]{Quick Step} \\
\textbf{School}: Transmutation \\
\textbf{Level}: 1, Very Rare \\
\textbf{Casting Time}: 2 Actions \\
\textbf{Range}: Contact \\
\textbf{Components}: V, S, M (a hare's paw) \\
\textbf{Duration}: 1 hour \\
A creature's movement increases by 10 feet until the spell ends. \\
\textbf{For each Magic Critical Success you roll} in the Magic Check, you can target one more creature.

\medskip \textbf{Fear}\index[Incantesimi]{Fear} \\
\textbf{School}: Illusion \\
\textbf{Level}: 3, Uncommon \\
\textbf{Casting Time}: 2 Actions \\
\textbf{Range}: Staff (9 meter cone) \\
\textbf{Components}: V, S, M (a white feather or the heart of a hen) \\
\textbf{Duration}: 1 minute \\
You project an illusory image of a creature's worst fears. Each creature in a 30-foot cone must succeed on a Will saving throw or drop whatever it is holding and be scared for the duration of the spell. \\
While frightened by this spell, a creature must, during each of its rounds, make the Sprint action and move away from you via the safest route, unless it has room to move. If the creature ends its round in a place where it has no line of sight to you, it can make a Will save, if it succeeds, the spell for that creature ends.

\medskip \textbf{Bark Hide}\index[Incantesimi]{Bark Hide} \\
\textbf{School}: Animals and Plants \\
\textbf{Level}: 2, Common \\
\textbf{Casting Time}: 2 Actions \\
\textbf{Range}: Contact \\
\textbf{Components}: V, S, M (a handful of oak bark) \\
\textbf{Duration}: 1 hour \\
The skin of the target you are in contact with at the time of casting the spell becomes rough and bark-like in appearance until the spell ends, and the target's Defense cannot be less than 16, regardless of armor. she is wearing.

\medskip \textbf{Stoneskin}\index[Incantesimi]{Stoneskin} \\
\textbf{School}: Earth \\
\textbf{Level}: 4, Uncommon \\
\textbf{Casting Time}: 2 Actions \\
\textbf{Range}: Contact \\
\textbf{Components}: V, S, M (diamond dust worth 100 gp, which the spell consumes) \\
\textbf{Duration}: 1 hour \\
You cast the spell on contact with a willing creature, whose skin turns into a substance as hard as stone. Roll 1d4 + half the CM value, the resulting sum being the times a melee or ranged weapon attack is canceled (regardless of whether it hits or not). \\
\textbf{For each Magic Critical Success you roll} in the Magic Check you increase negated attacks by 1.

\medskip \textbf{Plague of Insects}\index[Incantesimi]{Plague of Insects} \\
\textbf{School}: Animals and Plants \\
\textbf{Level}: 5, Rare \\
\textbf{Casting Time}: 2 Actions \\
\textbf{Range}: 90 meters \\
\textbf{Components}: V, S, M (a few grains of sugar, a few grains of wheat, a little lard) \\
\textbf{Duration}: 10 minutes \\
A swarm of hungry locusts fills a 20-foot-radius sphere centered at a range point of your choice. The sphere propagates around the corners. The sphere remains for the duration of the spell, and its area is dimmed. The sphere area is difficult terrain. \\
When the area appears, each creature within it must make a Fortitude saving throw. A creature takes 4d10 points of damage on a failed save, or half that damage if it succeeds. A creature must also make this saving throw when it first enters the spell's area during a round or if it ends its round within it. \\
\textbf{For each magical critical success rolled} on the Magic Check the damage is increased by 1d8.

\medskip \textbf{Stone in Mud - Mud in Stone}\index[Incantesimi]{Stone in Mud}\index[Incantesimi]{Mud in Rock} \\
\textbf{School}: Earth \\
\textbf{Level}: 5, Uncommon - Very Rare \\
\textbf{Casting Time}: 2 Actions \\
\textbf{Range}: 45 meters \\
\textbf{Components}: V, S, M (water and clay) \\
\textbf{Duration}: Instantaneous \\
This spell transforms any type of natural rock into an equal volume of mud. The magic stone is not affected by the spell. The spell affects up to 2 cubes of 3x3x3 meters. The depth of the created mud cannot exceed 3 meters. Creatures unable to fly, levitate, or move away from the mud in any way sink to the waist or chest; the terrain becomes doubly difficult and they are hampered. Creatures large enough to walk to the bottom of the mud pit can wade the area as difficult terrain.

If Stone in Mud is thrown onto the ceiling of a cave or tunnel, the mud pours onto the floor and expands to form a pool 1.5 meters deep. The falling mud and the landslide that follows deals 8d6 slam damage to anyone directly below the area if they don't halve the damage with a Reflex saving throw.

Castles and large stone buildings are generally immune to the spell's effects, as turning stone into mud doesn't go deep enough to undermine the foundation. However, other smaller buildings often stand on foundations shallow enough to be damaged or even destroyed by the spell's effects.

The mud remains until a dispel magic or mud to stone spell is successfully used, which restores its substance, but not necessarily its shape. Natural evaporation transforms the mud into normal soil over several days depending on exposure to the sun, wind and natural drying.
If a creature is in the mud at the time of the Mud to Stone spell, it can make a Reflex saving throw to break free or a DC 22 Strength check or 30 damage is required to break the stone. \\
\textbf{For each Magic Critical Success you roll} in the Magic Check you affect an extra 3x3x3 meter cube.

\medskip \textbf{Expert Pyro}\index[Incantesimi]{Expert Pyro} \\
\textbf{School}: Fire \\
\textbf{Level}: 2, Uncommon \\
\textbf{Casting Time}: 2 Actions \\
\textbf{Range}: 18 meters \\
\textbf{Components}: V, S, M (a match that is consumed) \\
\textbf{Duration}: Instantaneous \\
The caster chooses an area with a fire, at least 1 meter from the edge, within range that is directly visible to him. Extinguishing the flames can create fireworks or smoke.

- \textit{Fireworks}. The target fire explodes in a luminous show of flame and color. Any creature within 10 feet of the target must succeed at a Fortitude save or become blinded until the end of the next round.

- \textit{Smoke}. Thick black smoke rises from the target fire and spreads over a 20-foot radius, moving around corners. The smoke area is heavily darkened and provides medium coverage. The smoke persists for 1 minute or until a strong wind blows it away.

\medskip \textbf{Shimmer Dust}\index[Incantesimi]{Shimmer Dust} \\
\textbf{School}: Fire, Air \\
\textbf{Level}: 2, Uncommon \\
\textbf{Casting Time}: 2 Actions \\
\textbf{Range}: 36 meters \\
\textbf{Components}: V, S, M (silver dust) \\
\textbf{Duration}: 1 round per Magical Proficiency \\
In a sphere of 3 meters in diameter, whoever is is covered with shimmering and luminous dust. The cloud outlines the creatures present, even those that are invisible, and anyone who remains in the area must make a Reflex saving throw at the beginning of the round or be blinded for the round. The dust naturally disappears after the duration or if blown away by even a light wind.

\medskip \textbf{Dimensional Gate}\index[Incantesimi]{Dimensional Gate} \\
\textbf{School}: Summon \\
\textbf{Level}: 4, Common \\
\textbf{Casting Time}: 2 Actions \\
\textbf{Range}: 150 meters \\
\textbf{Components}: V \\
\textbf{Duration}: Instantaneous \\
You teleport from your current location to any other location in range. You arrive exactly where you want. It can be a place that you can see, one that you can visualize, or one that you can describe by indicating distance and direction, such as "30 meters down" or "90 meters up northwest at a 45 degree angle." \\
You can carry items whose weight does not exceed your encumbrance capacity. You can also take a consenting creature of your size or smaller with equipment with you up to the limit of its carrying capacity. The creature must be within 1 meter of you when you cast this spell. \\
If you arrive at a place already occupied by an object or creature, you and the creature traveling with you each take 4d6 points of force damage, and the spell fails to teleport you. \\
\textbf{For every two magical critical successes rolled} in the Magic Check, you can bring one more creature.

\medskip \textbf{Prayer of Healing}\index[Incantesimi]{Prayer of Healing} \\
\textbf{School}: Healing \\
\textbf{Level}: 2, Common \\
\textbf{Casting Time}: 10 minutes \\
\textbf{Range}: 9 meters \\
\textbf{Components}: V \\
\textbf{Duration}: Instantaneous \\
Up to six creatures at range you can see, chosen by you, each recover hit points equal to 2d6 + your ability modifier for spells. This spell deals the same amount of damage to the undead. \\
\textbf{For each Magic Critical Success rolled} in the Magic Check the healing is increased by 1d8.

\medskip \textbf{Omen}\index[Incantesimi]{Omen} \\
\textbf{School}: Divination \\
\textbf{Level}: 2, Common \\
\textbf{Casting Time}: 1 minute \\
\textbf{Duration}: Self \\
\textbf{Components}: V, S, M (specially marked Staff, bones or similar items worth at least 25 gp) \\
\textbf{Duration}: Instantaneous \\
By throwing in gem-inlaid Staff, rolling dragon bones, stacking elaborate cards, or employing some other divination tool, you receive an omen from an otherworldly entity as to the outcome of a specific course of action you intend to take over the next 30 minutes. The Storyteller chooses from the following omens: \\

- Prosperity, due to the positive results \\
- Disasters, for negative results \\
- Prosperity and calamity, for both positive and negative results \\
- Nothing, for the results that are neither particularly positive nor negative \\

The spell does not take into account every possible circumstance that could change the outcome, such as the casting of further spells or the loss or arrival of an ally. If you cast the spell two or more times before the new sun has risen, there is a cumulative 25 \% chance that for every cast after the first you will get an erroneous reading. The Storyteller makes this roll in secret.

\medskip \textbf{Prestidigitation} \index{Trick - Prestidigitation} \\
\textbf{School}: Universal \\
\textbf{Level}: 0, Common \\
\textbf{Casting Time}: 2 Actions \\
\textbf{Range}: 3 meters \\
\textbf{Components}: V, S \\
\textbf{Duration}: Maximum 1 hour \\
This spell is a minor magic trick that novice spellcasters employ to practice. You create one of the following magical effects at range: \\

- Create a harmless and instantaneous sensory effect such as a shower of sparks, a gust of wind, a faint musical note or a strange smell. \\
- Instantly light or extinguish a candle, torch or small campfire. \\
- Instantly clean up or soil an object no larger than 0.03 cubic meters. \\
- Cool, heat or tasteless for 1 hour up to 0.03 cubic meters of non-living material. \\
- Make a color, a small mark or a symbol appear on an object or surface for 1 hour. \\
- You create a non-magical trinket or illusory image that fits into your hand and remains until the end of your next round. \\

If you cast this spell multiple times, you can have up to three non-instant effects active at a time, and you can interrupt one of these effects with an action. \\
\textbf{For each Magic Critical Success you roll} in the Magic Check you can activate one more magical effect.


\medskip \textbf{Forecast}\index[Incantesimi]{Forecast} \\
\textbf{School}: Divination \\
\textbf{Level}: 9, Uncommon \\
\textbf{Casting Time}: 1 minute \\
\textbf{Range}: Contact \\
\textbf{Components}: V, S, M (a hummingbird feather) \\
\textbf{Duration}: 8 hours \\
You cast the spell on contact with a consenting creature to give it a limited ability to see into the immediate future. For the duration, the target cannot be surprised and has + 1d6 on attack rolls, ability checks, and saving throws. Also, again for the duration, other creatures have -1d6 attack rolls against the target. The spell ends immediately if you cast it again before its duration ends.

\medskip \textbf{Produce Flame} \index{Trick - Produce Flame} \\
\textbf{School}: Fire \\
\textbf{Level}: 0, Common \\
\textbf{Casting Time}: 1 Action \\
\textbf{Duration}: Self \\
\textbf{Components}: V, S \\
\textbf{Duration}: 10 minutes \\
A flame appears in your hand. The flame stays there for the duration of the spell and does not damage you or your equipment. The flame produces dim light within 1 meter. The spell ends if you interrupt it with an action or if you cast it again. \\
You can also use the flame to attack, although doing so ends the spell. When you cast this spell, or with an action in a later round, you can hurl the flame at a creature within 30 feet of you. Make a ranged spell attack. If you hit, the target takes 1d8 points of fire damage. \\
The damage of the spell increases by 1d8 when you reach CM 5, CM 11 and CM 17, but it costs 2 Actions to cast it boosted and 2 Magic Points, it is also necessary to have taken Adept of Magic in this Spell List a number of times equal to the upgrades that you want to apply. \\
\textbf{For each Magic Critical Success you roll} in the Magic Check, you can attack one more creature without ending the spell.


\medskip \textbf{Prohibition}\index[Incantesimi]{Prohibition} \\
\textbf{School}: Abjuration \\
\textbf{Level}: 6, Uncommon \\
\textbf{Casting Time}: 10 minutes \\
\textbf{Range}: Contact \\
\textbf{Components}: V, S, M (a splash of Holy Water, rare incense, and a ruby powder worth 1000 gp) \\
\textbf{Duration}: 1 day \\
Create a magic travel ban that protects up to 4000 square meters of floor, up to a height of 9 meters above the ground. For the duration of the spell, creatures cannot teleport to the area or use passages, such as the one created by the portal spell, to enter the area. The spell protects the area from planar travel, and thus prevents creatures from entering the area via the Astral Plane, the Ethereal Plane, the Fairy Lands or the Shadow World, or the planar displacement spell. \\
In addition, the spell damages the creature types you choose when casting. Choose one or more of the following: celestial, elemental, fairy, demon, and undead. When a selected creature enters the spell's area for the first time in a round or begins its round here, the creature takes 5d10 Light or Void damage (your choice, when you cast the spell). \\
When you cast this spell, you can establish a password. A creature that utters the password as it enters the spell's area takes no damage from it. \\
The area of the spell cannot overlap the area of another prohibition spell. If you perform prohibition every day for 30 days in the same place, the spell will last until dispelled, and the material components will be consumed during the last cast.

\medskip \textbf{Astral Projection}\index[Incantesimi]{Astral Projection} \\
\textbf{School}: Necromancy \\
\textbf{Level}: 9, Very Rare \\
\textbf{Casting Time}: 2 Actions \\
\textbf{Range}: 3 meters \\
\textbf{Components}: V, S, M (For each creature subject to this spell, you must provide a hyacinth worth at least 1000 gp and an elegantly carved silver ingot worth at least 100 gp, all of which are consumed by the spell) \\
\textbf{Duration}: Special \\
You and up to eight other consenting creatures within range project your astral bodies into the Astral Plane (the spell fails and the casting is wasted if you were already in that plane). The material body you leave behind is unconscious and in a state of suspended animation; it does not need food or water and does not age. \\
Your astral body resembles your mortal form in all respects, replicating your game stats and items. The main difference is the addition of a silver cord that extends from the shoulder blades 30 centimeters behind you, then becoming invisible. The cord is your connection to your material body. As long as this connection remains intact, you can go home. If the cord is cut (an event that only happens when a specific effect indicates it) your soul and body are separated, killing you instantly. \\
Your astral form can travel freely around the Astral Plane and pass through the portals that lead from there to other planes. If you enter a new plane or return to the plane you were on when the spell was cast, your body and items are carried along the silver cord, allowing you to re-enter your body upon entering the new plane. Your astral form is a separate incarnation. Any damage or other effect that applies to it has no effect on your physical body, nor does it appear there when you return. \\
The spell ends for you and your companions when you use an action to break it. When the spell ends, the creature it affects returns to its physical body, and awakens. The spell may also have an early end for you or one of your companions. A successfully dispel magic spell used on the astral or physical body ends the spell for that creature. If the creature's original body or astral form drops to 0 hit points, the spell ends for that creature. If the spell ends and the silver cord is intact, the cord pulls the creature's astral form back to its body, ending its state of suspended animation. \\
If you are returned to your body prematurely, your companions must remain in their astral form and find their way back to their bodies on their own, usually dropping to 0 hit points.

\medskip \textbf{Protection from Good and Evil}\index[Incantesimi]{Protection from Good and Evil} \\
\textbf{School}: Abjuration \\
\textbf{Level}: 1, Common \\
\textbf{Casting Time}: 2 Actions \\
\textbf{Range}: Contact \\
\textbf{Components}: V, S, M (Holy Water or silver and powdered iron, which the spell consumes) \\
\textbf{Duration}: 10 minutes \\
Until the spell ends, a consenting creature in contact with you at the moment of execution is protected by certain types of creatures: aberrations, celestials, elementals, faeries, demons, and undead. \\
The protection confers several benefits. Creatures of those types have -1d6 on attack rolls against the target. The target cannot be fascinated, frightened or possessed by them. If the target is already fascinated, frightened, or possessed by such a creature, the target has + 1d6 on any new saving throw against that effect. \\
\textbf{This spell is not usable when using traits. The Storyteller can grant the same effects to the Followers and Patrons of other Patrons}

\medskip \textbf{Protection from Energy}\index[Incantesimi]{Protection from Energy} \\
\textbf{School}: Abjuration \\
\textbf{Level}: 3, Common \\
\textbf{Casting Time}: 2 Actions \\
\textbf{Range}: Contact \\
\textbf{Components}: V, S \\
\textbf{Duration}: 10 minutes \\
You cast the spell on contact with a consenting creature. For the duration of the spell, the target has resistance to a damage type of your choice: acid, cold, fire, lightning, or sound. You can sacrifice the spell's entire duration, ending it, to completely undo the damage taken from an energy source. \\
\textbf{For each Magic Critical Success you roll} in the Magic Check you can influence another person or double the duration.

\medskip \textbf{Minor Energy Protection}\index[Incantesimi]{Minor Energy Protection} \\
\textbf{School}: Abjuration \\
\textbf{Level}: 1, Rare \\
\textbf{Casting Time}: 1 Reaction \\
\textbf{Range}: Contact \\
\textbf{Components}: V, S \\
\textbf{Duration}: 1 minute \\
You cast the spell on contact with a consenting creature. For the duration of the spell, the target has damage reduction from the chosen energy equal to 5. You can sacrifice the entire duration of the spell, ending it, to reduce the damage taken from an energy source by 20 (as if you had resistance to Damage 20 from that power source.) \\
\textbf{For each Magic Critical Success you roll} in the Magic Check you can influence another person or double the duration.

\medskip \textbf{Protection from Poisons}\index[Incantesimi]{Protection from Poisons} \\
\textbf{School}: Abjuration \\
\textbf{Level}: 2, Uncommon \\
\textbf{Casting Time}: 2 Actions \\
\textbf{Range}: Contact \\
\textbf{Components}: V, S \\
\textbf{Duration}: 1 hour \\
For the duration of the spell, the target has + 1d6 on saving throws against the poisoned being, and has resistance to poison damage. \\
\textbf{On case of two magical critical successes rolled} in the Magic Check you can cancel a poison that is circulating on the target.

\medskip \textbf{Marching Punishment}\index[Incantesimi]{Marching Punishment} \\
\textbf{School}: Invocation \\
\textbf{Level}: 2, Common \\
\textbf{Casting Time}: 1 Immediate Action \\
\textbf{Duration}: Self \\
\textbf{Components}: V \\
\textbf{Duration}: 1 minute \\
The next time you hit a creature with a weapon melee attack within the spell's duration, the weapon glows with a magical glow as you strike. The attack deals an additional 1d6 points of Light damage to the target, which becomes visible if it is invisible and emits dim light within a 1 meter radius. Also, the target cannot become invisible until the spell ends. \\
\textbf{For each magical critical success rolled} in the Magic Check the additional damage increases by 1d6.

\medskip \textbf{Purify Food and Drink}\index[Incantesimi]{Purify Food and Drink} \\
\textbf{School}: Animals and Plants \\
\textbf{Level}: 1, Common \\
\textbf{Casting Time}: 2 Actions \\
\textbf{Range}: 3 meters \\
\textbf{Components}: V, S \\
\textbf{Duration}: Instantaneous \\
All non-magical food and drinks within a 1 meter radius sphere, centered at a range point of your choice, are purified and freed from poisons and disease. Decaying food is cleaned up and made edible.

\medskip \textbf{Ray of Frost} \index{Trick - Ray of Frost} \\
\textbf{School}: Water \\
\textbf{Level}: 0, Common \\
\textbf{Casting Time}: 1 Action \\
\textbf{Range}: 18 meters \\
\textbf{Components}: V, S \\
\textbf{Duration}: Instantaneous \\
A frozen beam of blue light strikes a creature in range. Make a ranged spell attack against the target. If you hit, he takes 1d8 points of cold damage, and his speed is reduced by 10 feet until the start of your next round. \\
The damage of the spell increases by 1d8 when you reach CM 5, CM 11 and CM 17, but it costs 2 Actions to cast it boosted and 2 Magic Points, it is also necessary to have taken Adept of Magic in this Spell List a number of times equal to the upgrades that you want to apply. \\
\textbf{For every two Magic Critical Success rolled} in the Magic Check you create an additional ice bundle.

\medskip \textbf{Radius of Fatigue}\index[Incantesimi]{Radius of Fatigue} \\
\textbf{School}: Necromancy \\
\textbf{Level}: 2, Common \\
\textbf{Casting Time}: 2 Actions \\
\textbf{Range}: 18 meters \\
\textbf{Components}: V, S \\
\textbf{Duration}: 1 minute \\
A black beam of debilitating energy fires from your finger aimed at a creature in range. Make a ranged spell attack against the target. If you hit, the target will deal half damage with weapon attacks using Strength until the spell ends.

\medskip \textbf{Searing Ray}\index[Incantesimi]{Searing Ray} \\
\textbf{School}: Fire \\
\textbf{Level}: 2, Common \\
\textbf{Casting Time}: 2 Actions \\
\textbf{Range}: 36 meters \\
\textbf{Components}: V, S \\
\textbf{Duration}: Instantaneous \\
You create three beams of fire and project them towards three targets in range. You can throw them at the same target or at different targets. \\
Make one ranged spell attack for each ray. If you hit, the target takes 2d6 points of fire damage. \\
\textbf{For each Magic Critical Success rolled} in the Magic Check you create an additional ray.

\medskip \textbf{Web}\index[Incantesimi]{Web} \\
\textbf{School}: Animals and Plants \\
\textbf{Level}: 2, Common \\
\textbf{Casting Time}: 2 Actions \\
\textbf{Range}: 18 meters \\
\textbf{Components}: V, S, M (a piece of spider's web) \\
\textbf{Duration}: 1 hour \\
You summon a thick mass of dense, sticky canvas at a point in range, chosen by you. For the duration, the web fills a 6-meter cube of edge from that point. The web is difficult terrain and makes that area darkened slightly. \\
If the web is not anchored between two solid masses (such as walls or trees) or stretched along a floor, wall or ceiling, the summoned web collapses on itself, and the spell ends at the start of your next round. The canvases stretched on a flat surface have a depth of 1 meter. \\
Any creature that begins its round in the web or enters it during its round must make a Reflex saving throw. If it fails, the creature is entangled as long as it remains in the web or until it breaks free. \\
A creature entangled by webs can use 2 Actions to make a new saving throw. If it exceeds it, it is no longer in the way. \\
The web is flammable and if exposed to flames it catches fire immediately and for 2 rounds causing any creature within its area 2d4 of fire damage.

\medskip \textbf{Enchanted Club} \index{Trick - Enchanted Club} \\
\textbf{School}: Animals and Plants \\
\textbf{Level}: 0, Common \\
\textbf{Casting Time}: 1 Immediate Action \\
\textbf{Range}: Contact \\
\textbf{Components}: V, S, M (mistletoe, a four-leaf clover leaf, and a club or fighting stick) \\
\textbf{Duration}: 1 minute \\
The wood of a club or fighting stick you are holding is infused with the power of nature. For the duration of the spell, using that weapon you can use your spellcasting characteristic in place of Strength for attack rolls and melee damage, and the weapon's damage die becomes a d8. The weapon also becomes magical, if it isn't already. The spell ends if you cast it again or leave the weapon. \\
\textbf{For each Magic Critical Success rolled} in the Magic Check the duration is doubled or you get +1 to damage.

\medskip \textbf{Wonderful Palace}\index[Incantesimi]{Wonderful Palace} \\
\textbf{School}: Summon \\
\textbf{Level}: 7, Legendary \\
\textbf{Casting Time}: 1 minute \\
\textbf{Range}: 90 meters \\
\textbf{Components}: V, S, M (a miniature portal carved from ivory, a small piece of polished marble, and a tiny silver spoon, each of these items must be at least 5 gp worth) \\
\textbf{Duration}: 24 hours \\
Within range, you summon an extradimensional dwelling that remains for the duration of the spell. Choose where its front door is located. The entrance door emits a slight brightness and is 1 meter wide by 3 meters high. You and any creatures you indicated when you cast the spell can enter the extradimensional dwelling as long as the door is open. You can open or close the door if you are within 9 meters of it. While closed, the door is invisible. \\
Beyond the door there is a magnificent entrance, beyond which numerous rooms unfold. The atmosphere is clean, fresh and welcoming. You can create as many floors as you like, but the space cannot exceed 50 cubes each with 3 meters of edge. The place is furnished and decorated as you like. Contains enough food to satisfy a 9-course banquet for 100 people. A staff of 100 almost transparent servants is at the service of anyone who enters it. It is up to you to decide the visual appearance of these minions and their attire. They absolutely obey your orders. Each minion can perform any task a normal human minion can do, but they can't attack or take any action that could directly do damage to another creature. Servants can then collect items, clean, repair, fold clothes, light fires, serve food, pour wines, and so on. Servants can go anywhere in the mansion, but they cannot leave it. Furniture and other items created by this spell turn smoke when taken out of the mansion. When the spell ends, any creature within extradimensional space is ejected into the open space closest to the exit. \\
\textbf{Note}: the spell cast for one year every day always in the same place becomes permanent. \\
\textbf{For each Magic Critical Success rolled} in the Magic Check the duration doubles or takes one month off the count to make it permanent.

\medskip \textbf{Mental Regression}\index[Incantesimi]{Mental Regression} \\
\textbf{School}: Enchantment \\
\textbf{Level}: 8, Rare \\
\textbf{Casting Time}: 2 Actions \\
\textbf{Range}: 45 meters \\
\textbf{Components}: V, S, M (a handful of clay, crystal, glass or mineral spheres) \\
\textbf{Duration}: Instantaneous \\
You assault the mind of a ranged creature that you can see, trying to fragment its intellect and personality. The target takes 4d6 points of damage and must make a Will saving throw. If it fails the saving throw, the creature's Intelligence and Charisma scores drop to -4. The creature cannot cast spells, activate magical items, understand languages, or communicate in any understandable way. The creature can, however, identify its friends, follow them, and even protect them. After 30 days, the creature can re-roll the saving throw against the spell. If he succeeds, the spell ends if it fails the effect is permanent. \\
the spell can be ended within 30 days of higher restoring, healing, or desire.

\medskip \textbf{Reincarnation}\index[Incantesimi]{Reincarnation} \\
\textbf{School}: Animals and Plants \\
\textbf{Level}: 5, Rare \\
\textbf{Casting Time}: 1 hour \\
\textbf{Range}: Contact \\
\textbf{Components}: V, S, M (rare oils and ointments worth at least 1000 gp, which the spell consumes) \\
\textbf{Duration}: Instantaneous \\
You come into contact with a dead humanoid or a fragment of a dead humanoid. Provided the creature has not been dead for more than 10 days, the spell forms a new adult body and then calls its soul to enter the body. If the target's soul is not free or willing to do so, the spell fails. \\
The magic shapes a new body, which will likely cause the creature to change its race. The Storyteller rolls a d10 and consults the following table to determine what form the creature takes when it is brought back to life, or The Storyteller chooses the form. \\

\medskip
\begin{tabular}{ll}
\textbf{d100} & \textbf{Race} \\
\toprule
0 & Wolf / Eagle / Fox / Lynx (roll 1d4) \\
1 & Nano \\
2 & Elf \\
3 & Half-elf \\
4 & Mezzorco \\
5 & Boar / Badger / Dog / Rat (roll 1d4) \\
6 & Nibali \\
7 & Several \\
8 & Bear / Owl / Raccoon / Cat (roll 1d4) \\
9 & Human \\
10 & Same previous breed \\
\end{tabular}

The reincarnated creature remembers his life and past experiences. It retains the abilities it had in its original form if it is able to apply them. \\
\textbf{This spell is not available except to the Devotees and Followers of Shayalia or Efrem} \\
\textit{Note}: a Devotee or Follower of Shayalia or Ephrem will always reincarnate the creature in an animal, but being able to choose the type. \\
It is not possible to reincarnate as a gnome if you were not a gnome first.

\medskip \textbf{Resistance} \index{Trick - Resistance} \\
\textbf{School}: Abjuration \\
\textbf{Level}: 0, Common \\
\textbf{Casting Time}: 2 Actions \\
\textbf{Range}: Contact \\
\textbf{Components}: V, S, M (a miniature cloak) \\
\textbf{Duration}: Concentration, 1 minute \\
You cast the spell on contact with a consenting creature. Once before the spell ends, the target can roll a d4 and add the result to a saving throw of their choice. He can roll the dice before or after making the saving throw. Then the spell ends. \\
\textbf{For each Magic Critical Success obtained} in the Magic Check you can take advantage of the bonus on another trial.

\medskip \textbf{Breathe Underwater}\index[Incantesimi]{Breathe Underwater} \\
\textbf{School}: Water, Air \\
\textbf{Level}: 3, Common \\
\textbf{Casting Time}: 2 Actions \\
\textbf{Range}: 9 meters \\
\textbf{Components}: V, S, M (a straw or a straw) \\
\textbf{Duration}: 24 hours \\
This spell allows up to ten consenting creatures at range and that you can see to breathe underwater until the spell ends. Subject creatures also retain their normal breathing pattern. \\
\textbf{For each Magic Critical Success you roll} in the Magic Check you may choose one additional creature.

\medskip \textbf{Raise the Dead}\index[Incantesimi]{Raise the Dead} \\
\textbf{School}: Necromancy \\
\textbf{Level}: 5, Legendary \\
\textbf{Casting Time}: 1 hour \\
\textbf{Range}: Contact \\
\textbf{Components}: V, S, M (a diamond worth at least 500 gp, which the spell consumes) \\
\textbf{Duration}: Instantaneous \\
Bring a dead creature back to life, as long as it hasn't been dead for more than 10 days. If the creature's soul is both consenting and free to reunite with the body, the creature returns to life with 1 hit point. \\
This spell also neutralizes any poison and cures non-magical diseases that plagued the creature at the time of death. This spell, however, does not remove magical diseases, curses, or similar effects; if these are not removed before the spell is cast, they will resume manifesting when the creature comes to life. The spell cannot revive an undead creature. \\
This spell closes all mortal wounds, but does not restore missing body parts. If the creature lacks body parts or organs essential for survival (the head, for example), the spell automatically fails. \\
Returning from the dead is an ordeal. The target takes a -4 penalty on all attack rolls, saving throws, and ability checks. Each time the target ends a night's rest the penalty is reduced by 1 until it disappears. \\
\textbf{This spell should not be available. Only a Patron can bring back to life.}

\medskip \textbf{Regeneration}\index[Incantesimi]{Regeneration} \\
\textbf{School}: Transmutation \\
\textbf{Level}: 7, Legendary \\
\textbf{Casting Time}: 1 minute \\
\textbf{Range}: Contact \\
\textbf{Components}: V, S, M (a rosary and Holy Water) \\
\textbf{Duration}: 1 hour \\
You cast the spell on contact with a creature to stimulate its natural healing ability. The target recovers 4d8 + 15 hit points. For the duration of the spell, the target regains 1 hit point at the start of each of its rounds (6 hit points per minute). Severed limbs of the target's body (fingers, legs, tails, and so on), if any, are restored in 2 minutes. If you have the severed part and keep it resting on the stump, the spell causes the limb to heal with the stump in 3 rounds. \\
\textbf{For each magical Critical Success you roll} in the Magic Check you double the hit points recovered per round.

\medskip \textbf{Remove Disease}\index[Incantesimi]{Remove Disease} \\
\textbf{School}: Healing \\
\textbf{Level}: 4, Common \\
\textbf{Casting Time}: 1 turn \\
\textbf{Range}: Contact \\
\textbf{Components}: V, S \\
\textbf{Duration}: Instantaneous \\
You can put an end to even a magical disease. In the case of magical diseases, your magical proficiency must be higher than that of the person who caused the disease.

You may eventually make a Magic Check and for each critical success add 6 to your Magical Proficiency to see if it can remove the curse. \\
\textbf{For each Magic Critical Success you roll} in the Magic Check you can heal one more person.

\medskip \textbf{Remove Curse}\index[Incantesimi]{Remove Curse} \\
\textbf{School}: Abjuration \\
\textbf{Level}: 3, Common \\
\textbf{Casting Time}: 2 Actions \\
\textbf{Range}: Contact \\
\textbf{Components}: V, S \\
\textbf{Duration}: Instantaneous \\
If the object or person was cursed with the Cast Curse spell, or otherwise the Storyteller decides that the object has a specific curse for that character then the Magical Proficiency required to remove the curse must be greater than that of the one who 'he threw.

You can eventually make a Magic Check and for each critical success add 6 to your Magical Proficiency to see if it is able to remove the curse.

If the item is a cursed magical item, the curse remains, but the spell allows the item to be removed and thrown. \\
\textbf{With three critical hits} at the Storyteller's discretion it is possible to permanently remove the curse from the item.

\medskip \textbf{Remove Poison}\index[Incantesimi]{Remove Poison} \\
\textbf{School}: Water, Healing \\
\textbf{Level}: 3, Common \\
\textbf{Casting Time}: 1 round \\
\textbf{Range}: Contact \\
\textbf{Components}: V, S \\
\textbf{Duration}: Instantaneous \\
The target object of the spell is no longer poisoned.

\medskip \textbf{Rebirth}\index[Incantesimi]{Rebirth} \\
\textbf{School}: Healing, Necromancy \\
\textbf{Level}: 3, Very Rare \\
\textbf{Casting Time}: 2 Actions \\
\textbf{Range}: Contact \\
\textbf{Components}: V, S, M (diamond worth 300 gp, which the spell consumes) \\
\textbf{Duration}: Instantaneous \\
A creature that is dead in the last minute and with which you are in contact returns to life with 1 hit point. This spell cannot revive people who have died of old age, nor can it restore missing body parts. \\
The creature brought back to life must make a Fortitude save at DC 15 or because of the trauma suffered it does not come back to life, if it comes back to life it is Exhausted 2. \\
\textbf{Note}: At the Storyteller's discretion this may be the only spell allowed to revive a creature, otherwise the rule that only a Patron can revive applies.

\medskip \textbf{Repair} \index{Trick - Repair} \\
\textbf{School}: Earth \\
\textbf{Level}: 0, Common \\
\textbf{Casting Time}: 1 minute \\
\textbf{Range}: Contact \\
\textbf{Components}: V, S, M (two magnets) \\
\textbf{Duration}: Instantaneous \\
This spell repairs a single break or crack in an object you are in contact with, such as a broken chain, two halves of a broken key, a torn cloak, or a leaking skin. As long as the break or crack is no larger than 30 centimeters in any size, you are able to repair them, leaving no trace of the damage. This spell can physically repair a magical object or construct, but it cannot restore the magical functions of these objects.

\medskip \textbf{Inviolate Rest}\index[Incantesimi]{Inviolated Rest} \\
\textbf{School}: Necromancy \\
\textbf{Level}: 2, Uncommon \\
\textbf{Casting Time}: 2 Actions \\
\textbf{Range}: Contact \\
\textbf{Components}: V, S, M (a pinch of salt and a piece of copper placed on each eye of the corpse, which must remain there for the duration) \\
\textbf{Duration}: 10 days \\
You come into contact with a dead body or other remains. For the duration, the target is protected from rot and cannot become undead. \\
\textbf{For each Magic Critical Success obtained} in the Magic Check you double the duration to a maximum of one year.

\medskip \textbf{Uncontainable Laughter}\index[Incantesimi]{Uncontainable Laughter} \\
\textbf{School}: Enchantment \\
\textbf{Level}: 1, Uncommon \\
\textbf{Casting Time}: 2 Actions \\
\textbf{Range}: 9 meters \\
\textbf{Components}: V, S, M (small cakes and a feather that is waved in the air) \\
\textbf{Duration}: 1 minute
A ranged creature of your choice that you can see perceives everything as tremendously hilarious and funny, bursting into thunderous laughter as long as it is subject to this spell. The target must succeed on a Will saving throw or fall prone, remaining incapacitated and unable to stand up for the duration. Creatures with an Intelligence score of -2 or less ignore the effect. \\
At the end of each of its rounds, and each time it takes damage, the target can make another Will saving throw. The target has + 1d6 on the saving throw if it took damage in the round. If he succeeds, the spell ends.

\medskip \textbf{Heat Metal}\index[Incantesimi]{Heat Metal} \\
\textbf{School}: Fire \\
\textbf{Level}: 2, Uncommon \\
\textbf{Casting Time}: 2 Actions \\
\textbf{Range}: 18 meters \\
\textbf{Components}: V, S, M (a piece of iron and a flame) \\
\textbf{Duration}: 1 minute \\
Choose a metal artifact, such as a metal weapon or medium or heavy metal armor, that is at range and that you can see. Make the object glow red from the heat. Any creature in physical contact with the object takes 1d8 points of fire damage when you cast this spell. Until the spell ends, you can use 2 Actions to inflict this damage again in your subsequent rounds. \\
If a creature is holding or wearing the item and takes damage from it, the creature must succeed at a Fortitude save or discard the item if able. If he does not throw the item, he has -1d6 on attack rolls and ability checks until the start of his next round. \\
\textbf{For each magical critical success rolled} on the Magic Check the damage is increased by 1d8.

\medskip \textbf{Lesser Restoration}\index[Incantesimi]{Lesser Restoration} \\
\textbf{School}: Healing \\
\textbf{Level}: 2, Common \\
\textbf{Casting Time}: 2 Actions \\
\textbf{Range}: Contact \\
\textbf{Components}: V, S \\
\textbf{Duration}: Instantaneous \\
You can end a nonmagical disease or condition that afflicts a creature you are in contact with. The condition can be blinded, deafened or paralyzed. Exhausted (1) can lead to Exhausted or Exhausted (3) to Exhausted (2). It recovers 2d6 maximum hit points lost, but does not increase the current hit points. You can recover 1 lost Characteristic point not permanently.

At the Storyteller's discretion if the condition was caused by a spell, the caster's Magical Proficiency value must exceed the originator's Magical Proficiency.
You can eventually make a Magic Check and for each critical success add 6 to your Magical Proficiency to see if you are able to remove the effect.

\medskip \textbf{Greater Refreshment}\index[Incantesimi]{Greater Refreshment} \\
\textbf{School}: Healing \\
\textbf{Level}: 5, Uncommon \\
\textbf{Casting Time}: 2 Actions \\
\textbf{Range}: Contact \\
\textbf{Components}: V, S, M (diamond dust worth at least 100 gp, which the spell consumes) \\
\textbf{Duration}: Instantaneous \\
Imbue a creature on contact with healing positive energy to nullify a debilitating effect.
At the Storyteller's discretion if the condition was caused by a spell, the caster's Magical Proficiency value must exceed the originator's Magical Proficiency.
You may eventually make a Magic Check and for each critical success add 6 to your Magical Proficiency to see if you are able to remove the effect. \\

- An effect that fascinated the target. \\
- Make the target recover 2 points to a stat. Recover 1 point if the loss was permanent. \\
- Maximum hit points return to normal, but do not increase current hit points. \\
- You are able to relieve Fatigue and Exhausted conditions by two degrees

\medskip \textbf{Awakening}\index[Incantesimi]{Awakening} \\
\textbf{School}: Animals and Plants \\
\textbf{Level}: 5, Rare \\
\textbf{Casting Time}: 8 hours \\
\textbf{Range}: Contact \\
\textbf{Components}: V, S, M (an agate worth at least 1000 gp, which the spell consumes) \\
\textbf{Duration}: Instantaneous \\
After spending your casting time drawing magical paths with a precious gem, you come into contact with a Huge or smaller beast or vegetable. The target must have no Intelligence score or have Intelligence -3 or less. The target gains 0 Intelligence. The target also gains the ability to speak a language you know. If the target is a vegetable, it gains the ability to move its limbs, roots, vines, vines, and so on, and gains senses similar to those of a human. The Storyteller will choose the appropriate stats for the type of awakened plant, such as the stats for the awakened bush or awakened tree. \\
The awakened beast or vegetable is fascinated by you for 30 days or until you or your companions do it harm. When the charmed condition ends, the awakened creature chooses whether to remain friendly to you, based on how you treated it while charmed. \\
\textbf{For each Magical Critical Success obtained} in the Magic Check you double the duration of your fascination to a maximum of 1 year.

\medskip \textbf{Quick Retreat}\index[Incantesimi]{Quick Retreat} \\
\textbf{School}: Transmutation \\
\textbf{Level}: 1, Uncommon \\
\textbf{Casting Time}: 1 Immediate Action \\
\textbf{Duration}: Self \\
\textbf{Components}: V, S \\
\textbf{Duration}: Concentration, 1 minute \\
This spell allows you to move at an incredible pace. When you cast this spell you gain a bonus move action. \\
\textbf{For each Magic Critical Success rolled} in the Magic Check the duration is increased by 1 round.

\medskip \textbf{Jumping}\index[Incantesimi]{Jumping} \\
\textbf{School}: Aria \\
\textbf{Level}: 1, Common \\
\textbf{Casting Time}: 2 Actions \\
\textbf{Range}: Contact \\
\textbf{Components}: V, S, M (the hind leg of a grasshopper) \\
\textbf{Duration}: 1 minute \\
The jump distance of the creature you are in contact with at the time of casting is tripled until the spell ends.

\medskip \textbf{Sanctify}\index[Incantesimi]{Sanctify} \\
\textbf{School}: Invocation \\
\textbf{Level}: 5, Rare \\
\textbf{Casting Time}: 24 hours \\
\textbf{Range}: Contact \\
\textbf{Components}: V, S, M (herbs, oils, and incense worth at least 1000 gp, which the spell consumes) \\
\textbf{Duration}: Until dissolved \\
Infuse the surrounding area with a point where you are in touch with the power of your Patron. The area can have a maximum range of 60 feet, and the spell fails if it includes an area already under the effect of a sanctify spell. The area subject to the spell generates the following effects.
\textit{First}, celestials, elementals, faeries, demons, and undead cannot enter the area, nor can such a creature fascinate, frighten, or possess others within it. Any creature fascinated, frightened, or possessed by such a creature is no longer fascinated, frightened, or possessed by the time it enters this area. You may exclude one or more types of these creatures from this effect. \\
\textit{Second thing}, you can bind an additional effect to the area. Choose the effect from the list below, or choose one presented to you by the Storyteller. Some of these effects apply to creatures in the area; you can decide whether the effects apply to all creatures, Devotees or Followers of a specific Patron, or creatures of a specific type, such as orcs or trolls. When a creature subject to the spell enters this area for the first time during a round or begins its round here, it must make a Will saving throw. If it passes, the creature ignores the additional effect until it leaves the area. \\

- \textit{Courage}. Subject creatures cannot be frightened while they remain in this area. Extradimensional Interference. Subject creatures cannot move or travel using teleportation or other extradimensional or interplanar means. \\
\textit{Languages}. Subject creatures can communicate with any other creature in the area, even if they don't share a common language. \\
- \textit{Daylight}. Bright light fills the area. Magical darkness created by lower-level spells than that used to cast this spell cannot extinguish the light. \\
- \textit{Darkness}. Darkness fills the area. Normal light, and even magical light created by spells of a lower level than that used to cast this spell, cannot illuminate the area. \\
- \textit{Fear}. Subject creatures are frightened while they remain in this area. \\
- \textit{Protection from Energy}. Affected creatures receive resistance to a type of damage of your choice (except slash, piercing, or slashing damage) as long as they remain in the area. \\
- \textit{Inviolate Rest}. Dead bodies buried in the area cannot be transformed into undead. \\
- \textit{Silence}. No sound can emanate from within the area, and no sound can enter it. \\
- \textit{Energy Vulnerability}. Subject creatures receive vulnerability to a type of damage of your choice (except strike, piercing, or slashing damage) as long as they remain in the area.

\medskip \textbf{Sanctuary}\index[Incantesimi]{Sanctuary} \\
\textbf{School}: Abjuration \\
\textbf{Level}: 1, Common \\
\textbf{Casting Time}: 1 Immediate Action \\
\textbf{Range}: 9 meters \\
\textbf{Components}: V, S, M (a small silver mirror) \\
\textbf{Duration}: 1 minute \\
Protect a creature in range from attacks. Until the spell ends, any creature that targets the creature protected with a damaging attack or spell must first make a Will saving throw. If the saving throw fails, the attacker must choose a new target or lose the attack or spell. This spell does not protect the protected creature from area effects, such as the bursting of a fireball. If the protected creature makes an attack or casts a spell that affects enemy creatures, the spell ends.

\medskip \textbf{Private Shrine}\index[Incantesimi]{Private Shrine} \\
\textbf{School}: Abjuration \\
\textbf{Level}: 4, Very Rare \\
\textbf{Launch Time}: 10 minutes \\
\textbf{Range}: 36 meters \\
\textbf{Components}: V, S, M (a thin sheet of lead, a piece of opaque glass, a cotton ball or cloth, and powdered chrysolite) \\
\textbf{Duration}: 24 hours \\
Protect an area with magic. The area is a cube that can be as small up to 1 meter of edge or large as 30 meters of edge. The spell acts until the duration expires or until you use an action to break it. \\
When you cast the spell, you decide what kind of protection it provides by choosing one or more of the following properties: \\

- Sound cannot pass through the perimeter of the protected area. \\
- The perimeter of the protected area appears dark and foggy, preventing you from seeing through it (even in darkvision). \\
- Sensors created by divination spells cannot appear within the protected area or pass through its perimeter barrier. \\
- Creatures in the area cannot be the target of divination spells. \\
- Nothing can teleport into or out of the protected area. \\
- Inside the protected area, planar travel is forbidden. \\

Casting this spell on the same spot every day for a year makes the effect permanent. \\
\textbf{For each Magic Critical Success obtained} in the Magic Check you can increase the size of the cube by 10 meters edge or increase the duration by 12 hours.

\medskip \textbf{Cast Curse}\index[Incantesimi]{Cast Curse} \\
\textbf{School}: Necromancy \\
\textbf{Level}: 3, Common \\
\textbf{Casting Time}: 2 Actions \\
\textbf{Range}: Contact \\
\textbf{Components}: V, S \\
\textbf{Duration}: 1 minute \\
A creature you are in contact with must succeed at a Will saving throw or be cursed for the duration of the spell. When casting this spell, choose the nature of the curse from the following options: \\

- Choose a characteristic score. While cursed, the target has -1d6 on ability checks and saving throws possibly based on that ability score. \\
- While cursed, the target has -1d6 attack rolls against you. \\
- While cursed, the target must make a Will save at the start of each of his rounds. If he fails, he wastes that round's action by doing nothing. \\
- While the target is cursed, your attacks and spells deal an additional 1d8 Void damage against them. \\

The remove curse spell (see description) ends this effect. At the Storyteller's discretion, you can choose a curse with a different effect, but it shouldn't be more powerful than the ones described above. The Storyteller holds the final judgment on the effect of a curse. \\
\textbf{If you get a critic} the duration of the curse is one day. If you get 3 crit, the duration is permanent.

\medskip \textbf{Pickle}\index[Incantesimi]{Pickle} \\
\textbf{School}: Transmutation \\
\textbf{Level}: 2, Common \\
\textbf{Casting Time}: 2 Actions \\
\textbf{Range}: 18 meters \\
\textbf{Components}: V \\
\textbf{Duration}: Instantaneous \\
Choose an object that is within range and that you can see. The object can be a door, box, handcuffs, lock, or other object that has a common or magical method of preventing access. \\
A target that is closed by a common lock or that is locked or barred is opened, unlocked or released. If the object has several locks, only one of them is opened. \\
If you choose a target that is kept locked with an arcane lock that spell is suppressed for 10 minutes, during which time the target can be opened as normal. When you cast this spell, a loud knock, audible up to 90 meters away, emanates from the target object. \\
\textbf{For each Magic Critical Success you roll} in the Magic Check you can open another padlock / lock within range.

\medskip \textbf{Meteor Swarm}\index[Incantesimi]{Meteor Swarm} \hypertarget{sciamedimeteore}{} \\
\textbf{School}: Fire, Earth \\
\textbf{Level}: 9, Legendary \\
\textbf{Casting Time}: 2 Actions \\
\textbf{Range}: 0.9 miles \\
\textbf{Components}: V, S \\
\textbf{Duration}: Instantaneous \\
Glowing orbs of fire crash to the ground at four different points within range that you can see. Each creature, in a 2-meter radius sphere centered on your chosen spot, must make a Reflex saving throw. The sphere propagates around the corners. A creature takes 20d6 points of fire damage and 20d6 points of hit damage if it fails its saving throw, or one-half.
this damage if it exceeds it. A creature in the area of more than one blast of fire is affected only once. \\
\textbf{Saving Throw Success / Critical Failure}: On critical failure the damage doubles, on critical success the damage is further halved \\
\textbf{Every 3 crit rolls} in the Magic Check choose another point of impact.

\medskip \textbf{Carve Stone}\index[Incantesimi]{Carve Stone} \\
\textbf{School}: Earth \\
\textbf{Level}: 4, Common \\
\textbf{Casting Time}: 2 Actions \\
\textbf{Range}: Contact \\
\textbf{Components}: V, S, M (malleable clay, which must be worked to obtain a vague shape of the stone object) \\
\textbf{Duration}: Instantaneous \\
Carve a Medium or smaller stone object or a section of stone no thicker than 1 meter in any direction you are in contact with in any shape that suits your purposes. \\
So, for example, you could carve a large stone into a weapon, idol or coffin, or create a small passage through the wall, as long as the wall is less than 1 meter thick. You could also model a stone door or its frame to seal the door. The object you create can have up to two hinges and a latch, but it is impossible to create more complex mechanisms.

\medskip \textbf{Discover the Path}\index[Incantesimi]{Discover the Path} \\
\textbf{School}: Divination \\
\textbf{Level}: 6, Uncommon \\
\textbf{Casting Time}: 1 minute \\
\textbf{Duration}: Self \\
\textbf{Components}: V, S, M (divination tools - some ivory Staff, bones, cards, teeth or engraved runes - worth at least 100 gp and an item from the place you wish to find)\\
\textbf{Duration}: 1 day \\
This spell allows you to find the shortest and most direct physical route to a specific fixed place that you are familiar with and is on the same plane of existence. If you indicate a destination on another plane of existence, a moving destination (such as a moving fortress), or a non-specific destination (such as "a green dragon's lair"), the spell fails. \\
For the duration of the spell, as long as you are in the same plane of existence as the destination, you will know how far it is and in which direction it is. As you travel towards it, whenever you are presented with the choice of different routes, you will automatically determine which is the shortest and most direct (but not necessarily the safest) route to reach your destination.\\
\textbf{For each crit} obtained in the Trial of Magic the spell lasts 8 hours longer.

\medskip \textbf{Discover Traps}\index[Incantesimi]{Discover Traps} \\
\textbf{School}: Divination \\
\textbf{Level}: 2, Common \\
\textbf{Casting Time}: 2 Actions \\
\textbf{Range}: 36 meters \\
\textbf{Components}: V, S \\
\textbf{Duration}: 1 hour \\
For the duration of the spell, feel the presence of any ranged traps that are in your line of sight. A trap, for the purposes of this spell, includes anything that is capable of inflicting a sudden or unexpected effect that you may consider harmful or undesirable, and which was expressly intended as such by its creator. As a result, the spell would sense an area under the alarm spell, an interdict glyph, or a mechanical trap door, but would not reveal a natural floor weakness, unstable ceiling, or hidden pit. \\
The trap is highlighted in your sight with a purple signal.

\medskip \textbf{Secret Chest}\index[Incantesimi]{Secret Chest} \\
\textbf{School}: Summon \\
\textbf{Level}: 4, Rare \\
\textbf{Casting Time}: 2 Actions \\
\textbf{Range}: Contact \\
\textbf{Components}: V, S, M (a crafted chest, 1 meter x 50cm x 50cm, constructed of rare materials worth at least 5,000 gp, and a Tiny replica of it made of the same materials and valued at at least 50) \\
\textbf{Duration}: Instantaneous \\
Hide a chest and all its contents on the Ethereal Plane. When you cast this spell you must be in contact with the chest and the miniature replica that serves as the material component. The chest can hold up to 0.25 cubic meters of non-living material (1 x meter x 50cm x 50cm). While the chest remains on the Ethereal Plane, you can use an action to make contact with the replica and recall the chest. It will reappear in an unoccupied space on the ground within 1 meter of you. You can send the chest back to the Ethereal Plane, using an action and making contact with both the chest and the replica. \\
After 60 days, there is a cumulative 5 \% per day that the spell's effect ends. \\
The effect ends if the spell is cast again, if the replica of the chest is destroyed, or if you decide to end the spell with an action. If the spell ends and the chest is on the Ethereal Plane, it is irretrievably lost.

\medskip \textbf{Illusory Writing}\index[Incantesimi]{Illusory Writing} \\
\textbf{School}: Illusion \\
\textbf{Level}: 1, Common \\
\textbf{Casting Time}: 1 minute \\
\textbf{Range}: Contact \\
\textbf{Components}: S, M (a lead-based ink worth at least 10 gp, which the spell consumes) \\
\textbf{Duration}: 10 days \\
Write on a parchment, piece of paper, or some other writing material and imbue it with a powerful illusion that lasts for the duration of the spell. \\
For you and any creature you indicate when casting the spell, the writing appears normal, with your handwriting, and conveys whatever meaning you wanted to convey when you wrote the text. For all the others, the writing appears as if it were written in an unknown or magical writing, which is incomprehensible. Alternatively, you can make the writing look like a totally different message, in a different spelling and language, although it must be a language you know. \\
Should the spell be dispelled, both the original writing and the illusion vanish. A truth-seeing creature can read the hidden message.

\medskip \textbf{Scrutinize}\index[Incantesimi]{Scrutinize} \\
\textbf{School}: Divination \\
\textbf{Level}: 5, Rare \\
\textbf{Casting Time}: 10 minutes \\
\textbf{Duration}: Self \\
\textbf{Components}: V, S, M (a focus worth at least 1000 gp, such as a crystal ball, silver mirror, or source filled with Holy Water) \\
\textbf{Duration}: Concentration, maximum 10 minutes \\
You can see and hear a particular creature of your choice that is on the same plane of existence as you. The target must make a Will save, modified by how well you know the target and your physical connection to it. If the target knows you are casting the spell, he can voluntarily fail the saving throw, should he wish to be observed by.
you.

\medskip

\begin{tabular}{ll}
\toprule
\textbf{Knowledge} & \textbf{Form to the TS} \\
You've heard of it & + 5 \\
You met the target & + 0 \\
Know the target well & -5 \\
\end{tabular}

\begin{tabular}{ll}
\toprule
\textbf{Connection} & \textbf{Mod. TS} \\
Description or image & -2 \\
Property or garment & -4 \\
Body part (hair ...) & - 10 \\
\end{tabular}

\medskip

If the saving throw is successful, the target ignores the spell's effects, and you won't be able to use this spell against him again until 24 hours have passed. \\
If the saving throw fails, the spell creates an invisible sensor within 10 feet of the target. Through the sensor you can hear and see as if you were on the spot. The sensor moves with the target, staying within 10 feet of him for the duration of the spell. A creature that can see invisible objects sees the sensor as a luminous sphere about the size of a fist. \\
Instead of targeting a creature, you can target a location you've seen in the past as the spell's target. When you choose this option, the sensor appears in that place but does not move.

\medskip \textbf{Shield}\index[Incantesimi]{Shield} \\
\textbf{School}: Abjuration \\
\textbf{Level}: 1, Common \\
\textbf{Casting Time}: 1 Reaction, which you perform when hit by an attack or target of the magic missile spell \\
\textbf{Duration}: Self \\
\textbf{Components}: V, S \\
\textbf{Duration}: 1 round \\
An invisible barrier of magical force appears to protect you. Until the start of your next round, you have a +5 bonus to Defense including the trigger attack, and take no damage from Magic Bolt and Hidden Blast.

\medskip \textbf{Shield of Faith}\index[Incantesimi]{Shield of Faith} \\
\textbf{School}: Abjuration \\
\textbf{Level}: 1, Common \\
\textbf{Casting Time}: 1 Immediate Action \\
\textbf{Range}: 18 meters \\
\textbf{Components}: V, S, M (a small parchment with a fragment of sacred text written on it) \\
\textbf{Duration}: 10 minutes \\
A sparkling field appears surrounding a creature at range of your choice, granting it a +2 defense bonus for the duration of the spell.

\medskip \textbf{Fire Shield}\index[Incantesimi]{Fire Shield} \\
\textbf{School}: Fire, Water \\
\textbf{Level}: 4, Uncommon \\
\textbf{Casting Time}: 2 Actions \\
\textbf{Duration}: Self \\
\textbf{Components}: V, S, M (some phosphorus or a firefly) \\
\textbf{Duration}: 10 minutes \\
Thin, vaporous flames envelop your body for the duration of the spell, emitting intense light within a 10-foot radius and dim light for an additional 10-feet. You can end the spell early, using an action to break it. \\
The flames give you either a hot shield or a cold shield, of your choice. The hot shield gives you resistance to cold damage, while the cold shield gives you resistance to heat damage. \\
Also, whenever a creature within 1 meter of you hits you with a melee attack, the shield erupts flames. The attacker takes 2d8 points of fire damage from a hot shield, or 2d8 points of cold damage from a cold shield.

\medskip \textbf{Darkvision}\index[Incantesimi]{Darkvision} \\
\textbf{School}: Transmutation \\
\textbf{Level}: 2, Common \\
\textbf{Casting Time}: 2 Actions \\
\textbf{Range}: Contact \\
\textbf{Components}: V, S, M (or a pinch of carrot or dry agate) \\
\textbf{Duration}: 1 hour \\
A consenting creature you are in contact with gains the ability to see in the dark. For the duration of the spell, that creature has darkvision up to a range of 30 feet.
\textbf{For each Magic Critical Success you roll} in the Magic Check you double the duration.


\medskip \textbf{Faithful Hound}\index[Incantesimi]{Faithful Hound} \\
\textbf{School}: Summon \\
\textbf{Level}: 4, Rare \\
\textbf{Casting Time}: 2 Actions \\
\textbf{Range}: 9 meters \\
\textbf{Components}: V, S, M (a tiny silver whistle, and a piece of bone, and a thread) \\
\textbf{Duration}: 8 hours \\
You can summon a ghost watchdog to an unoccupied space at range and that you can see, where it will remain for the duration of the spell, until it is dismissed with an action, or until it moves more than 100 feet away from you. \\
The hound is invisible to all creatures except you and cannot be harmed. When a Small or larger creature approaches within 30 feet of it without first speaking the password you specified when you cast the spell, the hound starts barking loudly. The hound sees invisible creatures and can see into the Ethereal Plane. It ignores illusions. At the beginning of each of your rounds, the hound attempts to bite a creature within 1 meter of it and make it hostile to you. The hound's attack bonus is equal to your ability modifier for spells + CM. If it hits, it deals 2d8 points of piercing damage.

\medskip \textbf{Seem}\index[Incantesimi]{Seem} \\
\textbf{School}: Illusion \\
\textbf{Level}: 5, Uncommon \\
\textbf{Casting Time}: 2 Actions \\
\textbf{Range}: 9 meters \\
\textbf{Components}: V, S \\
\textbf{Duration}: 8 hours \\
This spell allows you to change the appearance of any number of creatures within range that you can see. Give each target a new illusory look. An unwilling creature can make a Will saving throw and, if successful, ignores the spell. \\
The spell camouflages physical appearance as well as clothing, armor, weapons and equipment. You can make each creature look 30 centimeters shorter or taller, look thin, fat, or somewhere in between. You cannot change the shape of the target's body, and therefore you must choose a shape that has the same basic distribution of limbs. \\
For everything else, the illusion is limited only by your imagination. The spell lasts for its duration, unless you use an action to break it first. The changes made by this spell are unable to withstand physical inspection. For example, if you use this spell to add a hat to a creature's clothing, objects go through the hat, and anyone who touches it would feel nothing and would end up touching the creature's head and hair.
If you use this spell to appear thinner than you are, a person's hand trying to touch you would bounce off of you, while the sight of it would appear to stop in midair. A creature can use 2 Actions to inspect a target and make an Awareness check against the spell's saving throw DC if it takes 3 Actions it has a + 1d6 bonus. If he succeeds, he realizes that the target is disguised.

\medskip \textbf{Demiplane}\index[Incantesimi]{Demiplane} \\
\textbf{School}: Summon \\
\textbf{Level}: 8, Rare \\
\textbf{Casting Time}: 2 Actions \\
\textbf{Range}: 18 meters \\
\textbf{Components}: S \\
\textbf{Duration}: 1 hour \\
You create a shadow door on a flat surface that is within range and that you can see. The door is large enough for a Medium creature to pass smoothly. When opened, the door leads to a demiplane that appears as an empty room 30 feet in each dimension, made of wood and stone. When the spell ends, the door disappears, and any creatures or objects within the demiplane are trapped there, while the door also disappears from the other side. \\
Each time you cast this spell, you create a new demiplane, or allow the shadow gate to connect to a demiplane created by a previous casting of the spell, or increase a known demiplane created by you previously by another 30 feet in each dimension. \\
Also, if you know the nature and contents of a demiplane created by another creature casting this spell, you can have the shadow gate connect to that demiplane instead.

\medskip \textbf{Arcane Lock}\index[Incantesimi]{Arcane Lock} \\
\textbf{School}: Abjuration \\
\textbf{Level}: 2, Common \\
\textbf{Casting Time}: 2 Actions \\
\textbf{Range}: Contact \\
\textbf{Components}: V, S, M (gold dust worth at least 25 gp, which is consumed by the spell) \\
\textbf{Duration}: Until dissolved \\
You cast the spell on contact with a locked door, window, portal, chest, or other entrance, and it becomes locked for the duration. You and the creatures you pointed to, when you cast this spell, can open the item normally. You can also set up a password which, when spoken within 1 meter of the object, suppresses the spell for 1 minute. Otherwise the opening is impassable until it is destroyed or the spell is dispelled or suppressed. Casting pick on the item suppresses arcane lock for 10 minutes. \\
While subject to this spell, the item is more difficult to destroy or force open; the DC to break it or pick a lock on it increases by 10. \\
\textbf{For each Magic Critical Success you roll} in the Magic Check you can affect another close.

\medskip \textbf{Invisible Servant}\index[Incantesimi]{Invisible Servant} \\
\textbf{School}: Summon \\
\textbf{Level}: 1, Common \\
\textbf{Casting Time}: 2 Actions \\
\textbf{Range}: 18 meters \\
\textbf{Components}: V, S, M (a piece of string and a piece of wood) \\
\textbf{Duration}: 1 hour \\
This spell creates an almost invisible force only bounded by a slight aura (the color of your choice) that performs simple tasks at your command, until the spell ends. The minion forms in an unoccupied space on the terrain, within range. He has Defense 10, 1 hit point, Strength 0, and cannot attack. If it drops to 0 hit points, the spell ends. \\
As an immediate action, during each of your rounds, you can mentally command the minion to move up to 4 meters and interact with an object. The servant can perform simple tasks like a human servant, such as gathering things, cleaning, mending, folding clothes, lighting fires, serving food, and pouring wine. Once the command is given, the minion will perform the task to the best of his ability until he completes it, and then he will wait for your next command. \\
If you command the minion to perform a task that will cause it to move more than 60 feet away from you, the spell ends.

\medskip \textbf{Freezing Sphere}\index[Incantesimi]{Freezing Sphere} \\
\textbf{School}: Water \\
\textbf{Level}: 6, Rare \\
\textbf{Casting Time}: 2 Actions \\
\textbf{Range}: 90 meters \\
\textbf{Components}: V, S, M (a small crystal ball) \\
\textbf{Duration}: Instantaneous \\
An icy orb of cold energy departs from your fingertips to a point of your choice at range, where it explodes into a sphere of 60 feet in radius. Each creature in the area must make a Fortitude saving throw. On a failed save, a creature takes 10d6 points of cold damage. If he succeeds, he takes half this damage. \\
If the globe hits a body of water or a liquid composed mainly of water (excluding water-based creatures, however), it freezes the liquid to a depth of 15 centimeters in a square area of 9 meters on each side. Ice lasts 1 minute. Creatures that were swimming on the surface of the frozen water get trapped in the ice. A trapped creature can use two actions to make a new saving throw in order to break free. \\
If you wish, after completing the spell, you can restrain yourself from shooting the orb. A small globe, about the size of a sling stone, cold to the touch, appears in your hand. At any time, you, or a creature you've given the orb to, can throw the orb (up to a range of 40 feet). This will shatter on impact, with the same effect as normal casting the spell. You can also place the globe on the ground without it shattering. After 1 minute, if the globe has not already been shattered, it will explode. \\
\textbf{For each magical critical success rolled} on the Magic Check the damage is increased by 1d6. \\
\textbf{Saving Throw Success / Critical Failure}: On critical failure the damage doubles, on critical success the damage is further halved

\medskip \textbf{Elastic Sphere}\index[Incantesimi]{Elastic Sphere} \\
\textbf{School}: Invocation \\
\textbf{Level}: 4, Rare \\
\textbf{Casting Time}: 2 Actions \\
\textbf{Range}: 90 meters \\
\textbf{Components}: V, S, M (a hemispherical piece of clear crystal and a corresponding hemispherical piece of gum arabic) \\
\textbf{Duration}: Concentration, maximum 1 minute \\
A sphere of light energy envelops a creature or object of Large or less than range. An unwilling creature must make a Reflex saving throw. If it fails, the creature is enveloped by the spell for its duration. \\
Nothing (no physical objects, no energy, no other spell effects) can pass through this barrier, in or out, although a creature within the sphere can breathe without problem. The sphere is immune to all damage, and a creature inside it can't be damaged by attacks or effects originating from outside, nor can a creature inside the sphere damage anything outside. The sphere is weightless and just large enough to hold the creature or object within it. A shrouded creature can use 1 Action to push against the walls of the sphere and then roll it up to half the creature's speed. Likewise, the orb can be picked up and moved by other creatures. \\
A disintegrate spell that targets the orb destroys it without harming anything inside.

\medskip \textbf{Flame Sphere}\index[Incantesimi]{Flame Sphere} \\
\textbf{School}: Fire \\
\textbf{Level}: 2, Common \\
\textbf{Casting Time}: 2 Actions \\
\textbf{Range}: 18 meters \\
\textbf{Components}: V, S, M (a little tallow, a pinch of sulfur and a handful of powdered iron) \\
\textbf{Duration}: 1 minute \\
For the duration of the spell, a sphere of 1 meter in diameter appears in a range space, chosen by you. Any creature that ends its round within 1 meter of the sphere must make a Reflex saving throw. The creature takes 2d6 points of fire damage on a failed save, or half that damage if it succeeds. \\
With one action you can move the sphere 9 meters. If you slam the sphere into a creature, the creature must make a saving throw against the sphere's damage, and the sphere will stop moving for that round.
When you move the sphere, you can move it over barriers up to 1 meter high, and blast it into spaces up to 3 meters wide. The sphere ignites flammable objects not worn or carried, and radiates a bright light within a radius of 6 meters and a dim light for an additional 6 meters. \\
While you have this spell active you are distracted from casting other spells. \\
\textbf{For each magical critical success rolled} on the Magic Check the damage is increased by 1d6.

\medskip \textbf{Blur}\index[Incantesimi]{Blur} \\
\textbf{School}: Illusion \\
\textbf{Level}: 2, Common \\
\textbf{Casting Time}: 2 Actions \\
\textbf{Duration}: Self \\
\textbf{Components}: V \\
\textbf{Duration}: 1 minute \\
Your body becomes blurry, indistinct and shaky to anyone who sees you. For the duration of the spell, all creatures have -1d6 attack rolls against you. Attackers who do not rely on sight are immune to this effect, for example if they have blind sight or are able to distinguish illusions, as if by seeing the truth.

\medskip \textbf{Penetrating Gaze}\index[Incantesimi]{Penetrating Gaze} \\
\textbf{School}: Necromancy \\
\textbf{Level}: 6, Very Rare \\
\textbf{Casting Time}: 2 Actions \\
\textbf{Duration}: Self \\
\textbf{Components}: V, S \\
\textbf{Duration}: Concentration, maximum 1 minute \\
For the duration of the spell, your eyes transform into a black void infused with terrible power. A creature of your choice within 60 feet of you that you can see must succeed at a Will saving throw or suffer one of the following effects of your choice for the duration. During each of your rounds, until the spell ends, you can use two Actions to target another creature, but you can't target a creature that made a saving throw against this piercing gaze cast again.\\

- \textit{Asleep}. The target falls unconscious. It awakens if it takes any amount of damage or if another creature uses 2 Actions to shake it from sleep. \\
- \textit{Sick}. The target has -1d6 on attack rolls and ability checks. At the end of each of her rounds, she can make another Will saving throw. If it exceeds it, the effect ends. \\
- \textit{Impanicato}. The target is scared of you. During each of its rounds, the frightened creature must make two move actions and move away from you in the shortest and safest way possible, unless it has room to move. If the target moves to a location at least 60 feet away from you where they can't see you, this effect ends.

\medskip \textbf{Silence}\index[Incantesimi]{Silence} \\
\textbf{School}: Illusion \\
\textbf{Level}: 2, Common \\
\textbf{Casting Time}: 2 Actions \\
\textbf{Range}: 36 meters \\
\textbf{Components}: V, S \\
\textbf{Duration}: 10 minutes \\
For the duration of the spell, no sound can be created within or traversing a sphere of 20 feet in radius centered on a range point of your choice. Any creature or object that is completely within the sphere is immune to sonic damage, and creatures that are completely within it are deafened. It is impossible to cast a spell that includes a verbal component while inside it. \\
\textbf{For each Magic Critical Success obtained} in the Magic Check the duration is doubled.

\medskip \textbf{Symbol}\index[Incantesimi]{Symbol}
\textbf{School}: Abjuration \\
\textbf{Level}: 7, Uncommon \\
\textbf{Casting Time}: 2 Actions \\
\textbf{Range}: Contact \\
\textbf{Components}: V, S, M (mercury, phosphorus and diamond and powdered opal with a total value of at least 1000 gp, which the spell consumes) \\
\textbf{Duration}: Until dissolved or activated \\
When you cast this spell, you inscribe a harmful glyph on a surface (such as a section of floor, wall, or table) or within an object that can be closed to hide the glyph (such as a book, scroll, or chest. ). If you choose a surface, the glyph can cover a surface area no greater than 3 meters in diameter. If you choose an object, that object must stay in place; if the object is moved more than 10 feet from where the spell was cast, the glyph is broken, and the spell ends without being activated. \\
The glyph is nearly invisible and can be found on a Survival check against the save DC of your spells. \\
You decide what activates the glyph when the spell is cast. \\
For glyphs inscribed on a surface, typical activation involves touching or standing over the glyph, removing another object covering the glyph, approaching some distance from the glyph, or manipulating the object on which the glyph is inscribed. glyph. \\
For glyphs inscribed on an object, typical activation involves opening the object, approaching some distance from the object, or seeing or reading the glyph. \\
You can better define the activation so that the spell activates only under certain circumstances or according to certain physical characteristics (such as height or weight) or species of creature (for example, the protection could act against hags or shapeshifters). You can also set up conditions to prevent the glyph from being triggered, such as pronouncing a password. \\
When writing the glyph you choose one of the following options as its effect. Once activated, the glyph glows, filling a 60-foot sphere of dim light for 10 minutes, after which the spell ends. Each creature in the sphere becomes the target of its effect when the glyph activates, as does a creature that first enters the sphere during a round or ends its round there. \\

- \textit{Dementia}. Each target must make a Will saving throw. On a failed save, the target becomes insane for 1 minute. A demented creature cannot perform actions, does not understand what others are saying to it, cannot read, and speaks only in gibberish. The Storyteller controls its movements, which are erratic. \\
- \textit{Discordia}. Each target must make a Fortitude saving throw. If it fails, the target starts bickering and arguing with another creature for 1 minute. During this time, he is unable to make any significant communications and has -1d6 on attack rolls and ability checks. Ache. Each target must make a Fortitude saving throw. If he fails, the target becomes incapacitated due to the excruciating pain. \\
- \textit{Death}. Each target must make a Fortitude save, taking 10d10 Void damage if it fails, or half that damage if it succeeds. \\
- \textit{Fear}. Each target must make a Will saving throw and, if it fails, be startled for 1 minute. While frightened, the target throws whatever it was holding and must move at least 30 feet away from the glyph during each of its rounds if able. \\
- \textit{Distrust}. Each target must make a Will saving throw. If the save fails, the target is overcome with despair for 1 minute. During this time, it cannot attack or target any creature with harmful abilities, spells, or other magical effects. \\
- \textit{Sleep}. Each target must make a Will save, and fall unconscious for 10 minutes if failed. A creature awakens if it takes damage or if someone uses an action to awaken it. \\
- \textit{Stun}. Each target must make a Will save, and be stunned for 1 minute if they fail.

\medskip \textbf{Simulacrum}\index[Incantesimi]{Simulacrum} \\
\textbf{School}: Illusion \\
\textbf{Level}: 7, Rare \\
\textbf{Casting Time}: 12 hours \\
\textbf{Range}: Contact \\
\textbf{Components}: V, S, M (plenty of snow or ice to create a full-size copy of the duplicated creature; some hair, nails, or other piece of that creature's body to place in the snow or ice; and a ruby powder worth 1,500 gp, scattered over the duplicate and consumed by the spell) \\
\textbf{Duration}: Until dissolved \\
You model an illusory duplicate of a beast or humanoid that remains in range for the entire time the spell is cast. The duplicate is a creature, partly real and made of ice or snow, that can perform actions and interact like a normal creature. It appears to be identical to the original, but has half that creature's maximum hit points and comes unequipped. Otherwise, the illusion uses all the stats of the creature it duplicates. \\
The simulacrum is friendly towards you and the creatures you indicate. It obeys your commands, moving and acting according to your wishes and acting during your combat round. The simulacrum lacks the ability to learn or become more powerful, and therefore never level or ability gains, nor can it regain spent spell slots. \\
If the simulacrum is damaged, you can repair it in an alchemical laboratory, using rare herbs and minerals worth 100 gp per hit point recovered. The simulacrum remains until it drops to 0 hit points, at which point it transforms back into snow and instantly melts. If you cast this spell again, any duplicates you created with this currently active spell are immediately destroyed.

\medskip \textbf{Dream}\index[Incantesimi]{Dream} \\
\textbf{School}: Illusion \\
\textbf{Level}: 5, Uncommon \\
\textbf{Casting Time}: 2 Actions \\
\textbf{Range}: Special \\
\textbf{Components}: V, S, M (a handful of sand, a tip of ink, and a writing pen taken from a sleeping bird) \\
\textbf{Duration}: 8 hours \\
This spell shapes a creature's dreams. Choose a creature known to you as the target of the spell. The target must be on the same plane of existence as you. Creatures that do not sleep cannot be subjected to this spell. You or a consenting creature you are in contact with enter a trance state, acting as a messenger. While in a trance, the messenger is aware of his surroundings, but cannot perform actions or move. \\
For the duration of the spell, if the target is asleep, the messenger appears in the target's dreams and can converse with the target as long as the target is asleep. The messenger can also shape the dream environment, creating terrains, objects and other images. The messenger can emerge from the trance at any time, prematurely ending the spell's effect. Upon awakening, the target remembers his dream perfectly. If the target is awake when you cast the spell, the messenger learns of it and can end the trance (and the spell) or wait for the target to fall asleep. At that point the messenger can appear in the target's dreams. \\
You can make the messenger appear to the target with a monstrous and terrifying appearance. If you do, the messenger can deliver a message of up to ten words and then the target must make a Will saving throw. On a failed saving throw, the fearsome monstrosity's echoes generate a nightmare for the target's sleep duration that prevents them from gaining any benefit from that rest. Additionally, when the target awakens, it takes 3d6 damage. \\
If you have a lock of hair, clipped nails, or similar portion of the target's body, he will make his saving throw at -1d6.

\medskip \textbf{Nap}\index[Incantesimi]{Nap} \\
\textbf{School}: Alteration \\
\textbf{Level}: 2, Legendary \\
\textbf{Casting Time}: 1 round \\
\textbf{Range}: 6 meters \\
\textbf{Components}: V, S, M (a feather, a piece of white cotton) \\
\textbf{Duration}: 1 minute \\
This spell allows you to rest up to 1 creature per spell-like proficiency / 4 for 1 hour. The creature must be consenting.

This rest hour is equivalent to 8 hours of rest when it comes to recovering Magic Points and Hit Points. It is not possible to use the spell's benefits more than once in 36 hours. \\
\textbf{For each magical critical success rolled} on the Magic Check you affect 1 more creature.

\medskip \textbf{Sleep}\index[Incantesimi]{Sleep} \\
\textbf{School}: Enchantment \\
\textbf{Level}: 1, Common \\
\textbf{Launch Time}: 2 Actions \\
\textbf{Range}: 27 meters \\
\textbf{Components}: V, S, M (a pinch of sand, rose petals or a cricket) \\
\textbf{Duration}: 1 minute \\
This spell places creatures in a magical torpor. Roll 5d8; the total is the number of creature hit points the spell can affect. Creatures within 20 feet of your chosen range are affected in ascending order of hit points (ignoring fainted creatures). \\
Starting with the creature with the lowest number of current hit points, each creature subject to this spell is unconscious until the spell ends, the sleeper takes damage, or someone uses an action to shake or slap the sleeping person. . Subtract each creature's hit points from the total before considering the creature with the next lowest hit point value. A creature's hit points must be equal to or less than the remaining total for the effect to affect it. Undead and creatures that cannot be charmed are unaffected by this spell. \\
\textbf{For each magical critical success rolled} on the Magic Check you affect an additional 2d8 hit points.

\medskip \textbf{Arcane Sword}\index[Incantesimi]{Arcane Sword} \\
\textbf{School}: Invocation \\
\textbf{Level}: 7, Rare \\
\textbf{Casting Time}: 2 Actions \\
\textbf{Range}: 18 meters \\
\textbf{Components}: V, S, M (a miniature platinum sword with copper and zinc handle and pommel, worth 250 gp) \\
\textbf{Duration}: Concentration, maximum 1 minute \\
For the duration of the spell, you create a plane of force in the form of a floating sword at range. When the sword appears, you make a CM + spell modifier melee attack against a target of your choice within 1 meter of the sword. If you hit, the target takes 3d10 points of force damage. Until the spell ends, you can use an action each of your rounds to move the sword 20 feet to a spot you can see and repeat this attack against the same or a different target.

\medskip \textbf{Color Spray}\index[Incantesimi]{Color Spray} \\
\textbf{School}: Illusion \\
\textbf{Level}: 1, Common \\
\textbf{Casting Time}: 2 Actions \\
\textbf{Range}: Personnel (4 meter cone) \\
\textbf{Components}: V, S, M (a pinch of powder or sand that is colored red, yellow and blue) \\
\textbf{Duration}: 1 round \\
A burst of colorful and dazzling lights emanates from your hand. Roll 6d10; the total is the amount of creature hit points this spell affects. Creatures, in a 4-meter cone originating from you, are subjected in ascending order of their current hit points (ignoring creatures that are unconscious and creatures that cannot see). \\
Starting with the creature that has the fewest current hit points, each creature subject to this spell remains blinded until the spell ends. Subtract each creature's hit points from the total before moving on to the creature with the next lowest hit point total. A creature's hit points must be equal to or less than the remaining total for the spell to affect it. \\
\textbf{For each magical critical success rolled} on the Magic Check, roll an additional 2d10 hit points.

\medskip \textbf{Prismatic Spray}\index[Incantesimi]{Prismatic Spray} \\
\textbf{School}: Invocation \\
\textbf{Level}: 7, Rare \\
\textbf{Casting Time}: 2 Actions \\
\textbf{Range}: Staff (18 meter cone) \\
\textbf{Components}: V, S \\
\textbf{Duration}: Instantaneous \\
Eight multicolored rays of light emanate from your hand. Each ray is a different color and has a different power and purpose. Each creature in a 60-foot cone must make a Reflex saving throw. For each target, roll a d8 to determine the color of the beam that hit it. \\

- \textit{1. Red}. The target takes 10d6 points of fire damage on a failed save, or half that damage if it succeeds. \\
- \textit{2. Orange}. The target takes 10d6 points of acid damage on a failed save, or half that damage if it succeeds. \\
- \textit{3. Yellow}. The target takes 10d6 points of lightning damage on a failed save, or half that damage if it succeeds. \\
- \textit{4. Green}. The target takes 10d6 points of poison damage on a failed save, or half that damage if a successful one. \\
- \textit{5. Blue}. The target takes 10d6 cold damage on a failed save, or half that damage if it succeeds. \\
- \textit{6. Indigo}. If the saving throw fails, the target is in the way. He then must make a Fortitude saving throw at the start of each of his rounds. On a successful save three times, the spell ends. If he fails his saving throw three times, he is permanently turned to stone and becomes a victim of the petrified condition. Successes and failures do not have to be consecutive; keep track of both until the target has three of the same type. \\
- \textit{7. Violet}. If the saving throw fails, the target is blinded. It must then make a Will saving throw at the start of your next round. On a successful saving throw, the blindness ends. If the saving throw fails, the creature is transported to another plane of existence of the Storyteller's choice and is no longer blinded (usually, a creature not on its home plane is exiled to it, while other creatures they are usually carried into the Astral or Ethereal planes). \\
- \textit{8. Special}. The target is hit by two beams. Roll two more times, retiring the 8s.

\medskip \textbf{Poisonous Spray} \index{Trick - Poisonous Spray} \\
\textbf{School}: Animals and Plants \\
\textbf{Level}: 0, Uncommon \\
\textbf{Casting Time}: 1 Action \\
\textbf{Range}: 3 meters \\
\textbf{Components}: V, S \\
\textbf{Duration}: Instantaneous \\
You reach out to a creature in range that you can see, and project a cloud of poison gas from your palm. The creature must succeed on a Fortitude save or take 1d12 points of poison damage. \\
The damage of the spell increases by 1d8 when you reach CM 5, CM 11 and CM 17, but it costs 2 Actions to cast it boosted and 2 Magic Points, it is also necessary to have taken Adept of Magic in this Spell List a number of times equal to the upgrades that you want to apply. \\
\textbf{For every two magical critical successes rolled} on the Magic Check you affect another creature within range.

\medskip \textbf{Shocking Grasp} \index{Trick - Shocking Grasp} \\
\textbf{School}: Aria \\
\textbf{Level}: 0, Common \\
\textbf{Casting Time}: 1 Action \\
\textbf{Range}: Contact \\
\textbf{Components}: V, S \\
\textbf{Duration}: Instantaneous \\
Lightning flashes from your hands that shock a creature you try to make contact with. Make a spell melee attack against the target. You have + 1d6 on the attack roll if the target is wearing armor made of metal. If you hit, the target takes 1d8 points of lightning damage, and cannot react until the start of its next round. \\
The damage of the spell increases by 1d8 when you reach CM 5, CM 11 and CM 17, but it costs 2 Actions to cast it boosted and 2 Magic Points, it is also necessary to have taken Adept of Magic in this Spell List a number of times equal to the upgrades that you want to apply. \\
\textbf{For two for each magical critical success rolled} in the Magic Check the damage is increased by 1d8

\medskip \textbf{Suggestion}\index[Incantesimi]{Suggestion} \\
\textbf{School}: Enchantment \\
\textbf{Level}: 2, Common \\
\textbf{Casting Time}: 2 Actions \\
\textbf{Range}: 9 meters \\
\textbf{Components}: V, M (a snake's tongue and a piece of honeycomb or a drop of sweet oil) \\
\textbf{Duration}: 8 hours \\
Suggest an activity course (limited to a sentence or two) and magically influence a creature in range that you can see and hear and understand, chosen by you. Creatures that cannot be charmed are immune to this effect. The suggestion must be pronounced so that the course of action sounds reasonable. Asking a creature to stab itself, throw itself on a spear, set itself on fire, or do some other manifestly harmful act automatically negates the spell's effects. \\
The target must make a Will saving throw. If it fails the saving throw, it follows the course of action you describe to the best of its ability. The suggested course of action can continue for the duration of the spell. If the suggested activity can be completed in a shorter time, the spell ends when the subject finishes doing what was asked. \\
You can also specify conditions that will trigger a special activity for the duration of the spell. For example, you might suggest that a knight surrender his war horse to the first beggar you meet. If the condition is not met before the spell ends, the activity will not be carried out. If you or any of your companions damage the target, the spell ends.

\medskip \textbf{Mass Suggestion}\index[Incantesimi]{Mass Suggestion} \\
\textbf{School}: Enchantment \\
\textbf{Level}: 6, Uncommon \\
\textbf{Casting Time}: 2 Actions \\
\textbf{Range}: 18 meters \\
\textbf{Components}: V, M (the tongue of a snake and a piece of honeycomb or a drop of sweet oil) \\
\textbf{Duration}: 24 hours \\
Suggest an activity course (limited to one or two sentences) and magically influence up to twelve creatures at range that you can see and hear and understand, chosen by you. Creatures that cannot be charmed are immune to this effect. The suggestion must be pronounced so that the course of action sounds reasonable. Asking a creature to stab itself, throw itself on a spear, set itself on fire, or do some other manifestly harmful act automatically negates the spell's effects. \\
Each target must make a Will saving throw. If it fails the saving throw, it follows the course of action you describe to the best of its ability. The suggested course of action can continue for the duration of the spell. If the suggested activity can be completed in a shorter time, the spell ends when the subject finishes doing what was asked. \\
You can also specify conditions that will trigger a special activity for the duration of the spell. For example, you might suggest that a group of soldiers give all their money to the first beggar they meet. If the condition is not met before the spell ends, the activity will not be carried out. If you or any of your companions damage a creature subject to this spell, the spell ends for that creature. \\
\textbf{For each Magic Critical Success obtained} in the Magic Check, add one day to the duration.

\medskip \textbf{Thaumaturgy} \index{Trick - Thaumaturgy} \\
\textbf{School}: Universal \\
\textbf{Level}: 0, Uncommon \\
\textbf{Casting Time}: 2 Actions \\
\textbf{Range}: 9 meters \\
\textbf{Components}: V \\
\textbf{Duration}: Maximum 1 minute \\
You manifest a minor trick in range, a sign of supernatural power. You create one of the following magical effects at range: \\

- Your voice sounds three times louder than normal for 1 minute. \\
- Allow the flames to flicker, intensify, fade or change color for 1 minute. \\
- Cause harmless tremors on the ground for 1 minute. \\
- You create an instant noise, such as a thunderclap, the cry of a crow, or a disturbing whisper, originating from a range point of your choice. \\
- Make an unlocked door or window open or slam shut. \\
- Change the look of your eyes for 1 minute. \\

If you cast this spell multiple times, you can keep up to three effects active for one minute at a time, and you can interrupt these effects with an action. \\
\textbf{For each Magic Critical Success you roll} in the Magic Check you can manifest one additional magical effect.

\medskip \textbf{Telekinesis}\index[Incantesimi]{Telekinesis} \\
\textbf{School}: Transmutation \\
\textbf{Level}: 5, Uncommon \\
\textbf{Casting Time}: 2 Actions \\
\textbf{Range}: 18 meters \\
\textbf{Components}: V, S \\
\textbf{Duration}: Concentration, maximum 10 minutes \\
Gain the ability to move or manipulate creatures or objects through thought. When you cast this spell, and as 2 Actions during each round, you can exert your will on a creature or object within range that you can see, causing the appropriate effect from the following. You can always act round after round on the same target, or choose a new one each time. If you switch targets, the previous target is no longer subject to the spell.
\textit{Creature}. You can attempt to move a Huge or smaller creature. Make an ability check using your spellcasting ability contested by a creature's Strength check. If you win the contest, move the creature 30 feet in any direction, including upward, but without exceeding the spell's range. Until the end of your next round, the creature is hampered by your telekinetic grip. A creature raised high, remains suspended in midair. \\
In subsequent rounds, you can use 2 Actions to attempt to maintain your telekinetic grip on the creature by repeating the contest.
\textit{Object}. You can attempt to move an object weighing up to 500 kilos. If the item is not being worn or carried, you automatically move it 30 feet in any direction, but without exceeding the spell's range. \\
If the item is worn or carried by a creature, you must make an ability check with your spellcasting characteristic contested by the creature's Strength check. If you win the contest, you drag the object away from that creature and move it 30 feet in any direction, without exceeding the spell's range. \\
You can exert precise control over objects via your telekinetic grip, allowing you to manipulate a simple tool, open a door or container, insert or retrieve an object from an open container, or pour material into a vial.

\medskip \textbf{Teleport}\index[Incantesimi]{Teleport} \\
\textbf{School}: Summon \\
\textbf{Level}: 7, Common \\
\textbf{Casting Time}: 2 Actions \\
\textbf{Range}: 3 meters \\
\textbf{Components}: V \\
\textbf{Duration}: Instantaneous \\
This spell instantly teleports you and eight other consenting creatures (or a single object) within range and that you can see, chosen by you, to a destination of your choice. If the target is an object, it must fit into a 10-foot-edged cube, and cannot be held or carried by an unwilling creature. \\
The destination you choose must be known to you, and must be on the same plane of existence as you are on. Your familiarity with the destination determines if you can get there. \\
The DM rolls a d100 and consults the table.
\end{multicols}
\medskip
\begin{tabular}{lllll}
\toprule
d100 & Error & Similar Area & Off Target & On Target \\
Permanent circle & - & - & - & 01-100 \\
Associated Object & - & - & - & 01-100 \\
Very Familiar & 01-05 & 06-13 & 14-24 & 25-100 \\
Seen by chance & 01-33 & 34-43 & 44-53 & 54-100 \\
Seen once & 01-43 & 44-53 & 54-73 & 74-100 \\
Description & 01-43 & 44-53 & 54-73 & 74-100 \\
False Destination & 01-50 & 51-100 & - & - \\
\end{tabular}
\medskip
\begin{multicols}{2}

\textit{Permanent circle} indicates a permanent teleportation circle whose sequence of seals you know. \\
\textit{Associated Item} indicates that you have an item taken in the last six months from your intended destination, such as a wizard's library book, royal suite linens, or a piece of marble from a lich's secret tomb. \\
\textit{Very familiar} is a place you've been to very often, a place you've studied carefully, or a place you can see when you cast the spell. \\
\textit{Seen casually} is a place you have seen more than once but are not very familiar with. \\
\textit{Seen once} is a place that you have only seen once, perhaps through magic. \\ \textit{Description} is a place whose position and appearance you only know from someone else's description, perhaps a map. \\
\textit{False Destination} is a place that doesn't exist. Maybe you've been trying to peer into an enemy's hideout but have seen an illusion instead, or you're trying to teleport to a familiar place that no longer exists. \\
\textit{On Target}. You and your party (or the target item) appear where you want. \\
\textit{Out of Target}. You and your party (or the target object) appear at a random distance from the target in a random direction. Distance off target is 1d10 x 1d10 percent of the distance traveled. For example, if you tried to travel 180 kilometers, land off target and roll 5 and 3 on two d10s, then you would be 15 \% off target, or 27 kilometers. The Storyteller determines the direction off target randomly, rolling a d8 and designating 1 as north, 2 as northeast, 3 as east, and so on following the compass directions. If you are teleporting to a coastal town and wind up 27km offshore at sea, you could be in trouble! \\
\textit{Similar Area}. You and your party (or the target object) end up in a different area that is visually or thematically similar to the target area. For example, if you are headed to your personal laboratory, you may end up in another charmer's laboratory or an alchemy shop that owns many of your laboratory's tools and instruments. Typically, you appear at the closest similar location, but since the spell has no range limit, you could end up pretty much anywhere on the same level. \\
\textit{Error}. The unpredictable magic of the spell causes a difficult journey. Each teleported creature (or the target object) takes 3d10 points of force damage, and the Storyteller re-rolls on the track to see where they end up (multiple errors can occur, dealing damage each
time).

\medskip \textbf{Firestorm}\index[Incantesimi]{Firestorm} \\
\textbf{School}: Fire \\
\textbf{Level}: 7, Rare \\
\textbf{Casting Time}: 2 Actions \\
\textbf{Range}: 45 meters \\
\textbf{Components}: V, S \\
\textbf{Duration}: Instantaneous \\
A storm composed of roaring flames appears at a point in range, chosen by you. The storm area consists of up to ten 10-foot-edged cubes, which you can arrange however you like. Each cube must have at least one face adjacent to that of another cube. Each creature in the area must make a Reflex saving throw. If it fails, it takes 7d10 points of fire damage, or half that damage if it succeeds. The fire damages objects in the area and ignites flammable objects that are not worn or carried. If desired, plant life in the area remains unaffected by the effects of this spell. \\
\textbf{For each Magic Critical Success you roll} in the Magic Check you increase the area of a cube by 10 feet of edge. \\
\textbf{Saving Throw Success / Critical Failure}: On critical failure the damage doubles, on critical success the damage is further halved

\medskip \textbf{Ice Storm}\index[Incantesimi]{Ice Storm} \\
\textbf{School}: Water, Air \\
\textbf{Level}: 4, Uncommon \\
\textbf{Casting Time}: 2 Actions \\
\textbf{Range}: 90 meters \\
\textbf{Components}: V, S, M (a pinch of dust and a few drops of water) \\
\textbf{Duration}: Instantaneous \\
An ice hailstorm hits the ground in a 6 meter radius and 12 meter high cylinder centered on a range point. Each creature in the cylinder must make a Reflex saving throw. The creature takes 2d8 points of hit damage and 4d6 points of cold damage on a failed save, or half if it succeeds. Hail turns the storm's area of effect into difficult terrain until the end of your next round. \\
\textbf{For each magical critical success rolled} on the Magic Check the damage is increased by 1d8. \\
\textbf{Saving Throw Success / Critical Failure}: On critical failure the damage doubles, on critical success the damage is further halved

\medskip \textbf{Sleet Storm}\index[Incantesimi]{Sleet Storm} \\
\textbf{School}: Water \\
\textbf{Level}: 3, Very Rare \\
\textbf{Casting Time}: 2 Actions \\
\textbf{Range}: 45 meters \\
\textbf{Components}: V, S, M (a pinch of dust and a few drops of water) \\
\textbf{Duration}: 1 minute \\
Until the spell ends, freezing rain and sleet fall into a cylinder 6 meters high and with a radius of 12 meters centered in a point of your choice at range. The area is dim as the exposed flames are extinguished. The ground in the area is covered in slippery ice, making it difficult terrain. When a creature enters the spell's area for the first time during a round or begins its round there, it must make a Reflex saving throw. If he fails, he falls prone. If a creature in the spell's area is focusing, it must succeed at a Fortitude saving throw against the spell's saving throw DC or lose concentration.

\medskip \textbf{Black Tentacles}\index[Incantesimi]{Black Tentacles} \\
\textbf{School}: Summon \\
\textbf{Level}: 4, Uncommon \\
\textbf{Casting Time}: 2 Actions \\
\textbf{Range}: 27 meters \\
\textbf{Components}: V, S, M (a piece of tentacle of a giant octopus or a giant squid) \\
\textbf{Duration}: 1 minute \\
Slimy ebony tentacles fill a 20-foot square on the ground, at range and you can see. For the duration of the spell, these tentacles transform the area into difficult terrain. \\
When a creature enters the affected area for the first time in a round or begins its round here, it must succeed on a Reflex saving throw or take 3d6 points of slash damage and remain \hyperlink{intralciato}{intralciata} by the tentacles until the spell ends. A creature that begins its round in the area and is already hampered by the tentacles takes 3d6 points of slash damage. A creature hampered by the tentacles can use 2 Actions to make a new saving throw to be free that round.

\medskip \textbf{Earthquake}\index[Incantesimi]{Earthquake} \\
\textbf{School}: Earth \\
\textbf{Level}: 8, Very Rare \\
\textbf{Casting Time}: 2 Actions \\
\textbf{Range}: 150 meters \\
\textbf{Components}: V, S, M (a pinch of soil, a piece of stone and a lump of clay) \\
\textbf{Duration}: Concentration, maximum 1 minute \\
You cause a seismic disturbance in a point on the ground at range and that you can see. For the duration, an intense tremor shakes the ground in a 30-meter radius circle centered on that point and shakes the creatures and structures in that area that are in contact with the ground. The ground in the area becomes difficult ground. Any creature on the ground that is concentrating must make a Fortitude saving throw. If he fails, his concentration is shattered. \\
When you cast this spell and at the end of each round you have spent focusing on it, each creature in the area that is on the ground must make a Reflex saving throw. If it fails, the creature falls prone. \\
This spell has additional effects depending on the type of terrain in the area, at the Storyteller's discretion. Cracks. At the beginning of the round following the one in which you cast the spell, fissures open throughout the spell area. A total of 1d6 slits open at locations chosen by the Storyteller. Each of them is 1d10 x 3 meters deep, 3 meters wide, and extends from one side of the spell area to the other. A creature standing where a rift opens must succeed on a Reflex saving throw or fall into it. A creature that succeeds at the saving throw moves to the edge of the rift as it opens. \\
A crack that opens under a structure causes it to collapse immediately (see below). Structures. The tremor deals 50 slam damage to any structures in contact with the ground in the area when you cast the spell and at the end of each of your turns until the spell ends. If a structure drops to 0 hit points, it collapses and could damage nearby creatures. A creature half the height of the structure or less from the structure must make a Reflex saving throw. If it fails, the creature takes 5d6 points of hit damage, falls prone, and is engulfed in rubble. He must then take an action and succeed in a DC 20 Dexterity (Athletics) check to break free. The Storyteller can change the DC up or down, depending on the nature of the rubble. On a successful save, the creature takes only half the damage and neither falls nor is buried.

\medskip \textbf{Illusory Terrain}\index[Incantesimi]{Illusory Terrain} \\
\textbf{School}: Illusion \\
\textbf{Level}: 4, Uncommon \\
\textbf{Casting Time}: 10 minutes \\
\textbf{Range}: 90 meters \\
\textbf{Components}: V, S, M (a stone, a sprig and a piece of green plant) \\
\textbf{Duration}: 24 hours \\
Make a piece of natural terrain at range, in a 45-meter-edged cube, appear, resonate, and smell like some other type of natural terrain. As a result, open fields or a road can be transformed into a swamp, hills, crevasse, or some other type of difficult or impassable terrain. A pond can be transformed into a grassy clearing, a precipice in a gentle slope, a cliff strewn with rocks into a wide and smooth road. Built structures, equipment, and creatures within the area do not change in appearance. \\
The tactile peculiarities of the terrain are unchanged, so that creatures entering the area are likely to reveal the illusion. If the difference isn't obvious on contact, a creature cautiously examining the illusion can attempt an Intelligence (investigate) check against your spell's saving throw DC to doubt it. A creature who recognizes the illusion for what it is perceives it as a vague image superimposed on the ground.

\medskip \textbf{Cold Touch} \index{Trick - Cold Touch} \\
\textbf{School}: Necromancy \\
\textbf{Level}: 0, Common \\
\textbf{Casting Time}: 1 Action \\
\textbf{Range}: 36 meters \\
\textbf{Components}: V, S \\
\textbf{Duration}: 1 round \\
You create a ghostly skeletal hand in the space of a creature at range. Make a ranged spell attack against the creature, to attack it with the chill of death. If you hit, the target takes 1d8 Void damage, and can't recover any hit points until the start of your next round. Until then, the hand will remain locked on the target. If you hit an undead target, it will also have -1d6 attack rolls against you until the end of its next round. \\
The damage of the spell increases by 1d8 when you reach CM 5, CM 11 and CM 17, but it costs 2 Actions to cast it boosted and 2 Magic Points, it is also necessary to have taken Adept of Magic in this Spell List a number of times equal to the upgrades that you want to apply. \\
\textbf{For every two magical critical successes rolled} in the Magic Check you create an additional skeletal hand that must attack a different creature within range.

\medskip \textbf{Vampiric Touch}\index[Incantesimi]{Vampiric Touch} \\
\textbf{School}: Necromancy \\
\textbf{Level}: 3, Common \\
\textbf{Casting Time}: 2 Actions \\
\textbf{Duration}: Self \\
\textbf{Components}: V, S \\
\textbf{Duration}: 1 minute \\
Contact with your shadow-shrouded hand can drain the life force of others to heal your wounds. Make a spell melee attack against a creature in range. If you hit, the target takes 3d6 Void damage, and you regain hit points equal to half the Void damage you dealt. Until the spell ends, you can make this attack again each round as your attack action. \\
While you have this spell active you are considered Distracted from casting other spells. \\
\textbf{For each Magic Critical Success rolled} on the Magic Check the damage is increased by 1d6.

\medskip \textbf{Hypnotic Plot}\index[Incantesimi]{Hypnotic Plot} \\
\textbf{School}: Illusion \\
\textbf{Level}: 3, Common \\
\textbf{Casting Time}: 2 Actions \\
\textbf{Range}: 36 meters \\
\textbf{Components}: S, M (a glowing stick of incense or a crystal vial filled with phosphorescent material) \\
\textbf{Duration}: 1 minute \\
You create a twisted texture of colors that moves in the air within a cube of 9 meters of edge. The plot appears for a moment and then fades. Each creature in the area that sees the plot must make a Will saving throw. If the saving throw fails, a creature is fascinated for the duration. While fascinated by this spell, the creature is incapacitated and has speed 0. The spell ends for the affected creature if it takes damage or if someone takes an action to shake it from its daze.

\medskip \textbf{Transformation}\index[Incantesimi]{Transformation} \\
\textbf{School}: Transmutation \\
\textbf{Level}: 9, Rare \\
\textbf{Casting Time}: 2 Actions \\
\textbf{Duration}: Self \\
\textbf{Components}: V, S, M (a jade circle worth at least 1,500 gp, which you must place on your head before casting the spell) \\
\textbf{Duration}: 1 hour \\
For the duration, you assume the form of a different creature. The new form can be that of any creature whose challenge rating is equal to or less than your CM. The creature cannot be a construct or an undead, and you must have seen it at least once. You turn into an average specimen of that creature, one with no specific abilities. You can remain in the assumed form until the spell ends. You automatically transform back if you fall unconscious, drop to 0 hit points, or die. Your game stats are replaced by the stats of the chosen creature, except for your Traits, and your Intelligence, Wisdom, and Charisma scores. You retain all of your skill proficiencies and saving throws, in addition to getting those of the creature. If the creature has the same abilities as you and the bonus listed in its stats is higher than yours, use the creature's bonus instead of yours. You cannot use any additional actions or lair actions of the new form. \\
When you transform, you take on the creature's hit points and hit dice. When you return to your normal form, you return to the number of hit points you had before transforming. However, if you transform back because you have been reduced to 0 hit points, all excess damage is restored to your original form. Unless the excess damage reduces your normal form to 0 hit points, you won't fall unconscious. \\
You retain all the benefits of any Skills you possessed, races, or other sources and can use them if the new form is physically capable of using them. However, you cannot use any of your special senses, such as darkvision, unless the new form also possesses the same sense. You can only speak if the creature is normally able to speak. \\
When you transform you choose whether your equipment falls to the ground in your space, merges with the new shape or is worn by it. The equipment worn works as normal, but it is up to the Storyteller to decide whether it is comfortable for the new form to wear such a piece of equipment, based on the size and dimensions of the creature. Your equipment does not change size or adapt to the new shape, and any equipment that the new shape cannot wear must be dropped to the ground or merged with the new shape. Equipment that melts is ineffective. \\
During the spell's duration, you can use two actions to take a different form following the same restrictions and rules as the original form, with one exception: if your new form has more hit points than your current form, your hit points remain at their current level. .

\medskip \textbf{Restser's Furious Transformation}\index[Incantesimi]{Restser's Furious Transformation} \\
\textbf{School}: Transmutation \\
\textbf{Level}: 6, Very Rare \\
\textbf{Casting Time}: 2 Actions \\
\textbf{Duration}: Self \\
\textbf{Components}: V, S, M (20cc of alcoholic beverage that is consumed by casting the spell, a magical weapon) \\
\textbf{Duration}: 1 round per Magical Proficiency \\
This spell allows a spellcaster to channel his magical energies to transform himself into a powerful fighter.

Until the end of the spell's duration, the caster's weapon proficiency becomes equal to his magical proficiency.

Based on the magical weapon held in hand at the time of the spell, you become competent in the Weapon List in which that weapon belongs, if the weapon is present in more than one list, the caster will choose the list. The caster acquires the abilities of that list of weapons as if he had chosen it a number of times equal to half his points in magical proficiency.

The caster gains 4 temporary hit points per point of magical proficiency possessed.
The unmodified score of the physical characteristics (Strength, Dexterity and Constitution) if less than 2 become 2.

For the duration of the spell, the caster is no longer able to cast spells.

\medskip \textbf{Tree Translation}\index[Incantesimi]{Tree Translation} \\
\textbf{School}: Animals and Plants \\
\textbf{Level}: 5, Rare \\
\textbf{Casting Time}: 2 Actions \\
\textbf{Duration}: Self \\
\textbf{Components}: V, S \\
\textbf{Duration}: 1 minute maximum \\
Gain the ability to enter a tree and move from within it into another tree of the same species within 150 meters. Both trees must be alive and at least the same size as you. You must use 1 meter of movement to enter the tree. Instantly learn the location of all other trees of the same species within 150 meters, and as part of the movement it takes to enter the tree, you can either pass into one of the other trees or exit the tree you entered. You reappear at a point of your choice within 1 meter of the target tree, using 1 more move action. If you have no movement left to use, you reappear within 1 meter of the tree you entered. \\
For the duration of the spell, you can use this transport ability once per round. You must finish each round outside of a tree.

\medskip \textbf{Plant Transport}\index[Incantesimi]{Plant Transport} \\
\textbf{School}: Animals and Plants \\
\textbf{Level}: 6, Very Rare \\
\textbf{Casting Time}: 2 Actions \\
\textbf{Range}: 3 meters \\
\textbf{Components}: V, S \\
\textbf{Duration}: 1 round \\
This spell creates a magical bond between a large or larger inanimate plant at range and another plant, at any distance, on the same plane of existence. You must have seen or come into contact with the target plant at least once. For the duration of the spell, any creature can enter the target plant and exit the target plant using 1 move action.

\medskip \textbf{Rope Trick}\index[Incantesimi]{Rope Trick} \\
\textbf{School}: Transmutation \\
\textbf{Level}: 2, Common \\
\textbf{Casting Time}: 1 minute \\
\textbf{Range}: Contact \\
\textbf{Components}: V, S, M (powdered grain extract and a parchment lace) \\
\textbf{Duration}: 1 hour \\
You come into contact with a piece of rope up to 18 meters long. One end of the rope rises into the air until the rope hangs perpendicular to the ground. At the opposite end of the rope, an invisible entrance opens into an extradimensional space which remains until the spell ends \\
Extradimensional space can be reached by climbing to the top of the rope (Climb check DC 15). The space can hold up to 2 Medium or smaller creatures. The rope can be dragged into space, making it disappear from the sight of those outside it. \\
Attacks and spells cannot cross the entrance into or out of extradimensional space, but those within it can see outside as if seeing through a 1 x 1 meter window centered on the rope. The detect magic spell allows you to see the opening. Anything in extradimensional space falls out of it when the spell ends. \\
\textbf{For each Magic Critical Success rolled} in the Magic Check the duration is doubled.

\medskip \textbf{Anointed}\index[Incantesimi]{Anointed} \\
\textbf{School}: Animals and Plants \\
\textbf{Level}: 1, Common \\
\textbf{Casting Time}: 2 Actions \\
\textbf{Range}: 18 meters \\
\textbf{Components}: V, S, M (a piece of pork rind or butter or greased topetto) \\
\textbf{Duration}: 1 minute \\
Slippery grease covers the ground in a 10-foot square, centered on a point at range, and turns it into difficult terrain for the duration of the spell \\
When fat appears, each target standing in the area must make a Reflex saving throw or fall prone. A creature that enters the area or ends its round there must succeed on a Reflex saving throw or fall prone.

\medskip \textbf{See Invisibility}\index[Incantesimi]{See Invisibility} \\
\textbf{School}: Divination \\
\textbf{Level}: 2, Common \\
\textbf{Casting Time}: 2 Actions \\
\textbf{Duration}: Self \\
\textbf{Components}: V, S, M (a pinch of talc and a handful of silver powder) \\
\textbf{Duration}: 1 hour \\
For the duration of the spell, you see invisible creatures and objects as if they were visible, and you can also see in the Ethereal Plane. Creatures and ethereal objects appear ghostly and transparent to you.

\medskip \textbf{Speed}\index[Incantesimi]{Speed} \\
\textbf{School}: Transmutation \\
\textbf{Level}: 3, Uncommon \\
\textbf{Casting Time}: 2 Actions \\
\textbf{Range}: 9 meters \\
\textbf{Components}: V, S, M (a grating of licorice root) \\
\textbf{Duration}: 1 minute \\
You change the flow of time by speeding it up around up to 4 creatures in a 6m edge cube at range. Until the spell ends, targets gain a +2 bonus to Defense, +2 Initiative, +2 to Reflex saving throws, and an additional Action during each of its rounds. \\
This action can only be used to perform an Attack Action (without a multi-attack penalty), Move, or Use an Item. \\
This spell counters and is countered by \hyperlink{lentezza}{Lentezza}. \\
When the spell ends, the targets cannot move or take Actions until their next round, while they are in a sudden drowsiness. \\
\textbf{For each magical critical success you roll} in the Magic Check you can affect another creature.

\medskip \textbf{Vigilance and Interdiction}\index[Incantesimi]{Vigilance and Interdiction} \\
\textbf{School}: Abjuration \\
\textbf{Level}: 6, Uncommon \\
\textbf{Casting Time}: 10 minutes \\
\textbf{Range}: Contact \\
\textbf{Components}: V, S, M (burnt incense, a small measure of sulfur and oil, a tied string, a small amount of earthen colossus blood, and a small silver rod worth at least 10 mo) \\
\textbf{Duration}: 24 hours \\
Create an interdiction that protects up to 225 square meters of floor (a square area of 15 meters per side, or one hundred squares of 1 meter per side or twenty-five squares of 3 meters per side). The forbidden area can be up to 6 meters high, and shaped as you like. You can interdict different floors of a stronghold by dividing the area between them, as long as you can walk continuously in each adjacent area, while casting the spell \\
When you cast this spell, you can specify individuals who ignore any or all of this spell's effects. You can also specify a password that, when spoken aloud, makes the speaker immune to these effects. \\
Vigilance and interdiction creates the following effects within the forbidden area. \\
\textit{Corridors}. The fog fills all the forbidden corridors, making them heavily obscured. Also, at each intersection or fork in the passage that offers a choice of direction, there is a 50 \% chance that a creature, other than you, will believe that it is going in the opposite direction to the one it chose. \\
\textit{Ports}. All doors in the forbidden area are magically closed, as if sealed by the arcane lock spell. Additionally, you can cover up to ten doors with an illusion (equivalent to the illusion object function of the minor illusion spell) to make them look like simple wall sections. \\
\textit{Scale}. Cobwebs cover all stairs in the forbidden area from top to bottom, as in the spider web spell. These threads grow back in 10 minutes if they are burned or torn while vigilance and interdiction remains active. \\
Other Spells in Effect. You may place one of the following magical effects of your choice within the forbidden area of the building \\

- Place dancing lights in four corridors. You can indicate a simple program that the lights will repeat for the duration of surveillance and interdiction. \\
- Place magic mouth in two places. \\
- Place stinking cloud in two places. The vapors appear in the place indicated by you; they return within 10 minutes if blown by the wind while surveillance and interdiction is still active. \\
- Place a steady gust of wind in a corridor or room. \\
- Place a suggestion in a place. Select an area 1 meter square, and any creature that enters or passes through that area is mentally given the suggestion. \\

The entire forbidden area radiates magic. A dispel magic spell cast against a specific effect removes only that effect if successful. You can create a permanently guarded and forbidden structure by casting this spell on it every day for a year. \\
\textbf{If you hit three crit} the duration is permanent.

\medskip \textbf{Fortitude}\index[Incantesimi]{Fortitude} \\
\textbf{School}: Healing \\
\textbf{Level}: 4, Rare \\
\textbf{Casting Time}: 2 Actions \\
\textbf{Range}: Contact meters \\
\textbf{Components}: V, S, M (water, salt, sugar) \\
\textbf{Duration}: 1 round per Magical Proficiency \\
The creature affected by this spell regains a level of fatigue, gains 3d6 temporary hit points. He can focus his energies to take one Attack Action without a multi-attack penalty or take an extra Move Action.


\medskip \textbf{Binding of Interdiction}\index[Incantesimi]{Binding of Interdiction} \\
\textbf{School}: Abjuration \\
\textbf{Level}: 2, Common \\
\textbf{Casting Time}: 2 Actions \\
\textbf{Range}: Contact \\
\textbf{Components}: V, S, M (a pair of platinum rings worth 50 gp each, which you and the target must wear for the duration) \\
\textbf{Duration}: 1 hour \\
You cast the spell on contact with a creature you want to protect. You create a mystical connection between you and the target until the spell ends. As long as the target is within 60 feet of you, he gains a +1 bonus to Defense and saving throws and has resistance to all damage. Also, each time the target takes damage, you take the same amount. The spell ends if you drop to 0 hit points or you and the target move more than 60 feet away. It ends even if you cast it again on the same creature it's already in effect on. You can interrupt the spell with an action.

\medskip \textbf{True Seeing}\index[Incantesimi]{True Seeing} \\
\textbf{School}: Divination \\
\textbf{Level}: 6, Rare \\
\textbf{Casting Time}: 2 Actions \\
\textbf{Range}: Contact \\
\textbf{Components}: V, S, M (an eye ointment that costs 25 gp; made of mushroom powder, saffron and grease; is consumed by the spell) \\
\textbf{Duration}: 1 hour \\
You cast the spell on contact with a consenting creature. The target receives the ability to see things as they really are. For the duration of the spell, the creature has true vision, notices secret doors hidden by magic, and can see into the Ethereal Plane, up to a range of 36 meters.

\medskip \textbf{False Life}\index[Incantesimi]{False Life} \\
\textbf{School}: Necromancy \\
\textbf{Level}: 1, Common \\
\textbf{Casting Time}: 2 Actions \\
\textbf{Duration}: Self \\
\textbf{Components}: V, S, M (a small amount of alcohol or distilled spirit) \\
\textbf{Duration}: 1 hour \\
For empowering yourself with a necromantic semblance of vitality, you gain 1d4 + 4 temporary hit points for the duration. \\
\textbf{For each magical critical success rolled} in the Magic Check, gain 5 temporary hit points.

\medskip \textbf{Flying}\index[Incantesimi]{Flying} \\
\textbf{School}: Air \\
\textbf{Level}: 3, Common \\
\textbf{Casting Time}: 2 Actions \\
\textbf{Range}: Contact \\
\textbf{Components}: V, S, M (a feather from the wing of any bird) \\
\textbf{Duration}: 10 minutes \\
You cast the spell on contact with a consenting creature. For the duration of the spell, the target gains 60 feet flight speed. When the spell ends, if it is still in the air, the target falls, unless it is able to stop the descent. \\
Casting a spell while flying is more complex, you are Distracted if you fail a Fly check at DC 11. \\
\textbf{For each magical critical success rolled} in the Magic Check, you can target an additional creature or double the duration.

\medskip \textbf{Mind Blank}\index[Incantesimi]{Mind Blank} \\
\textbf{School}: Abjuration \\
\textbf{Level}: 8, Uncommon \\
\textbf{Casting Time}: 2 Actions \\
\textbf{Range}: Contact \\
\textbf{Components}: V, S \\
\textbf{Duration}: 24 hours \\
Until the spell ends, a consenting creature you are in contact with during the cast is immune to any effects that would sense its emotions or read its thoughts, divination spells, and the fascinated condition. the spell also negates wish spells and other spells or effects of similar potency used for
to influence the target's mind or to obtain information about it. \\
\textbf{For each Magic Critical Success rolled} on the Magic Check the duration is doubled. If you get three crit, the duration is permanent.

\medskip \textbf{Area of Truth}\index[Incantesimi]{Area of Truth} \\
\textbf{School}: Enchantment \\
\textbf{Level}: 2, Uncommon \\
\textbf{Casting Time}: 2 Actions \\
\textbf{Range}: 18 meters \\
\textbf{Components}: V, S \\
\textbf{Duration}: 10 minutes \\
You create a magical zone that protects against scams in a sphere of 4 meters in radius centered on a range point of your choice. Until the spell ends, a creature that enters the spell's area for the first time during a round, or begins its round within it, must make a Will saving throw. If it fails the saving throw, the creature cannot deliberately lie while it is within range of the spell. You know if a creature has made or failed its saving throw. A creature subject to the spell is aware of it and can therefore avoid answering questions that it would normally answer with a lie. This creature can give elusive answers as long as it stays within the confines of truth.

\end{multicols}

%\vspace{2cm}
%\begin{center}
%	\includegraphics[width=0.4\linewidth]{immagini/Bocca_della_Verita.png}
%	\medskip
%	\textit{La Bocca della Verita', Chiesa di Santa Maria in Cosmedin, Roma}
%\end{center}


\pagebreak

\subsection{Ancient and lost spells}

The spells present here have been lost in history and only legends refer to their existence. \\
These spells not only have the Legendary Rarity but only the most learned have heard of them. Very often these are spells that were contrary to the will of some Patron who took steps to eliminate them from history and knowledge.

\begin{multicols}{2}

\medskip \textbf{Planar Ally}\index[Incantesimi]{Planar Ally} \\
\textbf{School}: Summon \\
\textbf{Level}: 6, Legendary \\
\textbf{Casting Time}: 10 minutes \\
\textbf{Range}: 18 meters \\
\textbf{Components}: V, S \\
\textbf{Duration}: Instantaneous \\
You beg an otherworldly entity to help you. The being must be known to you: a god, a primordial, a prince of demons, or some other creature of great power. That entity sends a celestial, elemental, or demon loyal to it to aid you by spawning the creature in an unoccupied space at range. If you know the name of a specific creature, you can say its name when you cast this spell to request that creature's help, although you may still receive another (at the Storyteller's discretion). \\
When the creature appears, it is under no obligation to act in any particular way. You can ask the creature to perform a service in exchange for a reward, but it doesn't have to satisfy you. The required task could be easy ("fly us over the chasm" or "help us fight this battle") or complex ("spy on our enemies" or "protect us as we explore the dungeon"). You must be able to communicate with the creature to negotiate its services. Reward can take many forms. A celestial might ask for a sizable donation of gold or magical items to an allied temple, while a demon might require a human sacrifice or the gift of a treasure. Some creatures may exchange their services for a mission that you will need to undertake on their own. As a general rule, a task that can be measured in minutes requires a reward of 100 gp per minute. A task measured in hours requires 1000 gp per hour. A task measured in days (10 days maximum) requires 10,000 gp per day. The Storyteller can modify these rewards based on the circumstances in which the spell is cast. If the task aligns with the creature's morale, the payment request could be halved or even canceled. Non-dangerous tasks usually only ask for half of what is suggested as payment, while very dangerous tasks may require larger donations. These creatures rarely accept tasks that seem suicidal. \\
After the creature completes the task, or when the agreed service period is over, the creature will return to its home plane after reporting to you, if appropriate to the task performed and if possible. If you are unable to agree on a price for the creature's services, the creature will immediately return to its home plane. A creature recruited to join your party is treated as a member of it, and receives a full share of the experience point rewards.

\medskip \textbf{Moonlight}\index[Incantesimi]{Moonlight} \\
\textbf{School}: Invocation \\
\textbf{Level}: 2, Legendary \\
\textbf{Casting Time}: 2 Actions \\
\textbf{Range}: 36 meters \\
\textbf{Components}: V, S, M (several beautiful night suits and a piece of opalescent plush) \\
\textbf{Duration}: Concentration, maximum 1 minute \\
A silvery beam of pale light shines in a 1 meter radius cylinder, 12 meters high centered at a range point. Until the spell ends, a dim light fills the cylinder. \\
When a creature enters the spell's area for the first time during a round or begins its round here, it is engulfed in ghostly flames that cause terrible pain, and must make a Fortitude saving throw. On a failed save, it takes 2d10 points of light damage, or half that damage if it succeeds. A shapeshifter makes a -1d6 saving throw. If it fails, it immediately reverts to its original form and cannot assume a different form until it comes out of the spell's light. \\
During each of your rounds after casting the spell, you can use an action to move the
beam of 18 meters in any direction. \\
\textbf{For each magical critical success rolled} on the Magic Check the damage is increased by 1d10.

\medskip \textbf{Contact Other Plans}\index[Incantesimi]{Contact Other Plans} \\
\textbf{School}: Divination \\
\textbf{Level}: 5, Legendary \\
\textbf{Casting Time}: 1 minute \\
\textbf{Duration}: Self \\
\textbf{Components}: V \\
\textbf{Duration}: 1 minute \\
You mentally contact a demigod, the spirit of a long-deceased sage, or some other mysterious entity from another plane. Contacting extraplanar intelligence can tire or even break your mind. When you cast this spell, make a DC 15 Will save. If you fail, you take 6d6 points of damage and remain demented until dawn the next day. While you are demented, you cannot perform actions, you cannot understand what other creatures are saying, you cannot read, and you only talk ranting. The greater restoration spell can end this effect. If you succeed on the saving throw, you can ask the entity up to five questions. You must ask the questions before the spell ends. The Storyteller will answer each question with a word: "yes", "no", "maybe", "never", "irrelevant" or "confused" (if the entity does not know the answer to the question). If a one-word answer might be misleading, the Storyteller might give a short sentence as an answer instead.

\medskip \textbf{Summon Celestials}\index[Incantesimi]{Summon Celestials} \\
\textbf{School}: Summon \\
\textbf{Level}: 7, Legendary \\
\textbf{Casting Time}: 1 minute \\
\textbf{Range}: 27 meters \\
\textbf{Components}: V, S \\
\textbf{Duration}: 10 minutes \\
You summon a celestial of challenge rating 4 or lower, which appears in an unoccupied space at range and which you can see. The celestial disappears when it drops to 0 hit points or the spell ends. The celestial is friendly towards you and your companions for the duration of the spell. Roll initiative for the celestial, which acts during its own round. He obeys any verbal command that is given to him (without needing you to perform any action), as long as he does not violate his Traits. If you do not give commands to the celestial, it will defend itself against hostile creatures, but it will take no further action. \\
\textbf{For each magical critical success rolled} in the Magic Check, increase the summoned creature's CR by one.

\medskip \textbf{Summon Woodland Creatures}\index[Incantesimi]{Summon Woodland Creatures} \\
\textbf{School}: Summon \\
\textbf{Level}: 4, Legendary \\
\textbf{Casting Time}: 2 Actions \\
\textbf{Range}: 18 meters \\
\textbf{Components}: V, S, M (a summoned creature holly berry) \\
\textbf{Duration}: 1 hour \\
Summon fairy spirits that appear in unoccupied spaces at range and that you can see. Choose one of the following options to determine what appears: \\

- A fairy of challenge grade 2 or lower

- Two fairies of challenge grade 1 or lower

- Four fairies of 1/2 or lower degree of challenge

- Eight fairies of degree of challenge 1/4 or lower

\medskip
A summoned creature disappears when it drops to 0 hit points or when the spell ends. Summoned creatures are friendly towards you and your companions. Roll initiative for creatures summoned as a group, which acts during its own round. They obey any verbal command that is given to them (without needing you to perform any action). If you do not give commands to the fairies, they will defend themselves from hostile creatures, but take no further action. \\
\textbf{For each Magic Critical Success rolled} in the Magic Check, two more creatures will appear.

\medskip \textbf{Summon Fairy}\index[Incantesimi]{Summon Fairy} \\
\textbf{School}: Summon \\
\textbf{Level}: 6, Legendary \\
\textbf{Casting Time}: 1 minute \\
\textbf{Range}: 27 meters \\
\textbf{Components}: V, S \\
\textbf{Duration}: 1 hour \\
You summon a fairy spirit of challenge grade 6 or lower, or a fairy spirit that takes the form of a beast of challenge grade 6 or lower. It appears in an unoccupied space at range that you can see. The fairy creature disappears when it drops to 0 hit points or when the spell ends. \\
The fairy creature is friendly towards you and your companions. Roll initiative for the fairy creature, which acts during its own turns. It obeys any verbal command that is given to it (without needing you to perform any action), as long as it does not violate its Traits. If you do not give commands, it will defend itself against hostile creatures, but take no further action. \\
\textbf{For each Magic Critical Success you roll} in the Magic Check, increase the summoned creature's CR by 1.

\medskip \textbf{Guardian of Faith}\index[Incantesimi]{Guardian of Faith} \\
\textbf{School}: Summon \\
\textbf{Level}: 4, Legendary \\
\textbf{Casting Time}: 2 Actions \\
\textbf{Range}: 9 meters \\
\textbf{Components}: V \\
\textbf{Duration}: 8 hours \\
A large ghostly guardian appears for the duration and floats in an unoccupied space at range that you can see, chosen by you. The guardian occupies that space and is indistinguishable except for a glowing sword and a shield with your Patron's symbol. \\
Any creature hostile to you that enters a space within 10 feet of the guardian for the first time during a round must make a Reflex saving throw. The creature takes 20 light / void damage if it fails its saving throw, or half that damage if it succeeds. The guardian vanishes after dealing a total of 60 damage.

\medskip \textbf{Spiritual Guardians}\index[Incantesimi]{Spiritual Guardians} \\
\textbf{School}: Summon \\
\textbf{Level}: 3, Legendary \\
\textbf{Casting Time}: 2 Actions \\
\textbf{Range}: Personnel (4 meters radius) \\
\textbf{Components}: V, S, M (a sacred symbol) \\
\textbf{Duration}: Concentration, maximum 10 minutes \\
Calls of spirits to protect you. For the duration of the spell, they will float around you at a distance of 4 meters. If you are good or neutral, the ghostly form will be angelic or fairy (your choice). If you are evil, they will have a demon appearance. When you cast this spell, you can designate any number of creatures that are immune to it. A subject creature's speed is halved within the area, and when a creature enters the area for the first time during a round or begins its round there, it must make a Will saving throw. If you fail the saving throw, you take 3d8 points of Light damage (if you are good or neutral) or 3d8 points of Void damage (if you are evil), or half that damage if you succeed. \\
\textbf{For each magical critical success rolled} on the Magic Check the damage is increased by 1d8 \\

\medskip \textbf{Planar Link}\index[Incantesimi]{Planar Link} \\
\textbf{School}: Abjuration \\
\textbf{Level}: 5, Legendary \\
\textbf{Casting Time}: 1 hour \\
\textbf{Range}: 18 meters \\
\textbf{Components}: V, S, M (a jewel worth at least 1000 gp, which the spell consumes) \\
\textbf{Duration}: 24 hours \\
With this spell, you try to bind a celestial, elemental, fairy or demon to your service. The creature must remain in range for the entire casting of the spell. (Usually, the creature is first summoned to the center of an inverted magic circle to keep it trapped while this spell is cast.) Upon completion of the cast, the target must make a Will saving throw. If he fails his saving throw, he is bound to your service for the duration. If the creature was summoned or created by another spell, that spell's duration is extended to match this spell's duration. A bound creature must carry out your instructions to the best of its ability. You could command the creature to accompany you on an adventure, to protect a place, or to deliver a message. The creature obeys your instructions to the letter, but if it is hostile to you, it will try to twist your words for its own ends. If the creature fully fulfills your instructions before the spell ends, if you are on the same plane of existence, it will return to you to let you know. If you are on different planes of existence, it will return to the place where you bound it and will remain there until the spell ends. \\
\textbf{For each magical critical success rolled} on the Magic Check, you double the creature's permanence.

\medskip \textbf{Hunter's Mark}\index[Incantesimi]{Hunter's Mark} \\
\textbf{School}: Divination \\
\textbf{Level}: 1, Legendary \\
\textbf{Casting Time}: 2 Actions \\
\textbf{Range}: 27 meters \\
\textbf{Components}: V \\
\textbf{Duration}: Concentration, 1 hour maximum \\
Choose a creature in range that you can see. The creature is mystically marked as your prey. Until the spell ends, deal an additional 1d6 damage to the target each time you hit it with a weapon attack, and you have + 1d6 on Awareness or Survival checks to find it. \\
If the target drops to 0 hit points before the spell ends, you can use an immediate action during your next round to mark a new creature. \\
\textbf{For each Magic Critical Success you roll} in the Magic Check, you can stay focused on the spell for another hour.

\medskip \textbf{Portal}\index[Incantesimi]{Portal} \\
\textbf{School}: Summon \\
\textbf{Level}: 9, Legendary \\
\textbf{Casting Time}: 2 Actions \\
\textbf{Range}: 18 meters \\
\textbf{Components}: V, S, M (a diamond worth at least 5000 gp) \\
\textbf{Duration}: Concentration, 1 minute maximum \\
You summon in an unoccupied space at range that you can see a portal linked to a specific place on a different plane of existence. The portal is a circular opening created by you, 1 to 6 meters in diameter. You can orient the portal in any direction you wish. The portal remains for the duration. \\
The portal has a front and a back on both floors where it appears. The journey through the portal is only possible by moving from the front. Anything that does is instantly transported to the other plane, appearing in the unoccupied space closest to the portal. \\
Gods and other planar rulers can prevent portals created by spells from opening in their presence or anywhere in their domains. When you cast this spell, you can say the name of a specific creature (alias, title or nickname don't work). If that creature is on a different plane than yours, the portal opens near the named creature and draws the creature through it, to the nearest unoccupied space on your side of the portal. You have no special power over the creature, and it is free to act as the Storyteller sees fit. It may leave, attack you or help you.

\medskip \textbf{Resurrection}\index[Incantesimi]{Resurrection} \\
\textbf{School}: Necromancy \\
\textbf{Level}: 7, Legendary \\
\textbf{Casting Time}: 1 hour \\
\textbf{Range}: Contact \\
\textbf{Components}: V, S, M (a diamond worth at least 1000 gp, which the spell consumes) \\
\textbf{Duration}: Instantaneous \\
You cast the spell on contact with a creature that has been dead for no more than a century, that has not died of old age and is not undead. If his soul is free and willing, the target will return to life with all his hit points. \\
This spell neutralizes all poisons and cures normal diseases that plagued the creature when it died. However, it does not remove magical diseases, curses and the like; if these effects are not removed before the spell is cast, they will affect the target upon reviving. \\
This spell closes all mortal wounds and restores any missing body parts. Returning from the dead is an ordeal. The target takes a -4 penalty on all attack rolls, saving throws, and ability checks. Each time the target ends a night's rest the penalty is reduced by 1 until it disappears. \\
Casting this spell to revive a creature that has been dead for a year or more wears you out. Until the end of a night's rest, you will no longer be able to cast spells and you will have -1d6 on all attack rolls, ability checks, and saving throws. \\
The creature brought back to life must make a Fortitude saving throw at DC 13 or the trauma suffered does not revive. \\
\textbf{This spell should not be available. Only a Patron can bring back to life.}

\medskip \textbf{Pure Resurrection}\index[Incantesimi]{Pure Resurrection} \\
\textbf{School}: Necromancy \\
\textbf{Level}: 9, Legendary \\
\textbf{Casting Time}: 1 hour \\
\textbf{Range}: Contact \\
\textbf{Components}: V, S, M (some Holy Water and diamonds worth 25,000 gp, which the spell consumes) \\
\textbf{Duration}: Instantaneous \\
You cast the spell on contact with a creature that has been dead for no more than 200 years and has died for any reason but not of old age. If his soul is free and willing, the creature will come back to life with all of its hit points. \\
This spell closes all wounds, neutralizes any poison, heals all diseases, and removes any curse that plagued the creature when it died. The spell replaces damaged organs and limbs. \\
The spell can also grant a new body if the original no longer exists, in which case you must say the creature's name. The creature will then reappear in an unoccupied space of your choice, within 10 feet of you. \\
\textbf{This spell should not be available. Only a Patron can bring back to life.}

\medskip \textbf{Saving the Dying} \index{Trick - Saving the Dying} \\
\textbf{School}: Animals and Plants \\
\textbf{Level}: 0, Legendary \\
\textbf{Casting Time}: 1 round \\
\textbf{Range}: Contact \\
\textbf{Components}: V, S, M (an offering to your Patron of at least 5 gp, which the spell consumes) \\
\textbf{Duration}: Instantaneous \\
A 0 hit point creature you are in contact with returns to 1 HP. The spell has no effect on undead or constructs. \\
\textbf{For each magical critical success rolled} on the Magic Check you heal the creature for 1d4 hit points.

\medskip \textbf{Planar Shift}\index[Incantesimi]{Planar Shift} \\
\textbf{School}: Summon \\
\textbf{Level}: 7, Legendary \\
\textbf{Casting Time}: 2 Actions \\
\textbf{Range}: Contact \\
\textbf{Components}: V, S, M (a forked metal rod worth at least 250 gp, tuned to a specific plane of existence) \\
\textbf{Duration}: Instantaneous \\
You and up to eight other consenting creatures, shaking hands to form a circle, are transported to a different plane of existence. You can specify a target destination in general terms, and you will reappear in or near that destination, at the Storyteller's discretion. \\
Alternatively, if you know the sequence of seals of a teleportation circle to another plane of existence, the spell can lead you to that circle. If the teleportation circle is too small to hold all the creatures you carry with you, they will reappear in the unoccupied space as close to the circle as possible. \\
You can use this spell to banish an unwilling creature to another plane. Choose a creature in range and make a spell-based melee attack against it. If you hit, the creature must make a Will saving throw. If the creature fails its saving throw, it is transported to a random location on the plane of existence you specify. A creature thus carried will have to find its own way back to your current plane of existence.

\medskip \textbf{Storm of Vengeance}\index[Incantesimi]{Storm of Vengeance} \\
\textbf{School}: Air, Water \\
\textbf{Level}: 9, Legendary \\
\textbf{Casting Time}: 2 Actions \\
\textbf{Range}: View \\
\textbf{Components}: V, S \\
\textbf{Duration}: Concentration, maximum 1 minute \\
A seething storm cloud forms, centered in a spot you can see and spreads out within a radius of 110 meters. The area is illuminated by lightning, thunder echoes and strong winds sweep it. When the cloud appears, each creature below it (that is, no more than 1,500 meters below the cloud) must make a Fortitude saving throw. On a failed save, the creature takes 2d6 points of sonic damage and is stunned for 5 minutes. \\
Each round you keep your focus on this spell, the storm during your round produces further effects. \\
\textit{Round 2}. Acid rain falls from the cloud. Each creature and object under the cloud takes 1d6 points of acid damage. \\
\textit{Round 3}. Call six lightning bolts from the cloud to strike six creatures or objects of your choice, which are under the cloud. A specific creature or object cannot be struck by more than one bolt of lightning. An affected creature must make a Reflex saving throw. The creature takes 10d6 points of lightning damage if it fails its saving throw, or half that damage if it succeeds. \\
\textit{Round 4}. The cloud produces a thick hailstorm. Each creature under the cloud takes 2d6 hit damage. \\
\textit{Round 5-10}. Gusts of wind and freezing rain hit the area under the cloud. The area becomes difficult terrain and is in dim light. Each creature in the area takes 1d6 points of cold damage. In the area it becomes impossible to make attacks with ranged weapons. Wind and rain are considered a serious distraction for the purpose of maintaining concentration on spells. \\ Finally, gusts of strong wind (ranging from 30 to 75 kilometers per hour) automatically disperse fog, mist and similar phenomena in the area, whether natural or magical.

\medskip \textbf{Find Familiar}\index[Incantesimi]{Find Familiar} \\
\textbf{School}: Animals and Plants \\
\textbf{Level}: 1, Legendary \\
\textbf{Casting Time}: 1 hour \\
\textbf{Range}: 3 meters \\
\textbf{Components}: V, S, M (10 gp of charcoal, incense and herbs to be consumed by the fire in a brass brazier) \\
\textbf{Duration}: Instantaneous \\
Get the service of a familiar, a spirit that takes an animal form of your choice: seahorse, raven, weasel, hawk, cat, crab, owl, lizard, fish (fizz), octopus, bat, spider, frog (toad) , rat or poisonous snake. Appearing in a ranged, unoccupied space, the familiar has the stats of the chosen form, although it is a celestial, fairy, or demon type (your choice) instead of a beast. Your familiar acts independently of you, but always obeys your commands. In combat, he rolls his initiative and acts during his own round. A familiar cannot attack, but it can perform other actions as normal.
You cannot have more than one familiar at a time. \\
\textbf{Familiar Skill Check} for the familiar abilities. You must have the familiar skill.


\end{multicols}

\vfill

\begin{center}
\includegraphics[keepaspectratio, width = 0.70 \textwidth]{immagini/Goetic_circle_from_The_Lesser_Key_of_Solomon.png}

\textit{The Circle of Solomon and Triangle of Solomon from The Goetia: The Lesser Key of Solomon the King, The Book of Evil Spirits by LW De Laurence}
\end{center}

\pagebreak



\subsection{List of Spells divided by Magic List} \hypertarget{elencoscuole}{}

This is the list of spells divided by List of Magic, next to each spell the level of the magic is indicated. Next to each List of Spells is the linked Characteristic to establish the maximum throwable level.

\begin{multicols}{3}

\bigskip

{\small

\flushleft{\textbf{Water List - Characteristic Dexterity}}

Energy Weapon (1) \\
Walk on Water (3) \\
Cone of Cold (5) \\
Check Water (4) \\
Check Weather (8) \\
Create or Destroy Water (1) \\
Cure Light Wounds (1) \\
Summon Elemental (5) \\
Summon Minor Elementals (4) \\
Acid Arrow (2) \\
Wall of Ice (6) \\
Cloud of Fog (1) \\
Stinking Cloud (3) \\
Cloud of Death (5) \\
Ray of Frost (0) \\
Breathe Underwater (3) \\
Remove Poison (3) \\
Fire Shield (4) \\
Freezing Sphere (6) \\
Ice Storm (4) \\
Sleet Storm (3) \\

\medskip \textbf{Air List - Characteristic Charisma}

\flushleft{Arma Energetica (1)}
Vital bubble (4) \\
Falling Feather (1) \\
Walking in the Wind (6) \\
Chain of Lightning (6) \\
Check Weather (8) \\
Summon Elemental (5) \\
Summon Minor Elementals (4) \\
Gust of Wind (2) \\
Lightning (3) \\
Call the Lightning Bolt (3) \\
Levitation (2) \\
Wall of Wind (3) \\
Prismatic Wall (9) \\
Cloud of Fog (1) \\
Stinking Cloud (3) \\
Cloud of Death (5) \\
Shimmer powder (2) \\
Thundering Wave (1) \\
Breathe Underwater (3) \\
Skip (1) \\
Lightning Grasp (0) \\
Ice Storm (4) \\
Flying (3) \\

\medskip \textbf{Fire List - Characteristic Intelligence}

\flushleft{Arma Energetica (1)} \\
Flame Strike (5) \\
Fire Bolt (1) \\
Summon Elemental (5) \\
Summon Minor Elementals (4) \\
Kyrin's Fire of Acorns (3) \\
Burning Blade (2) \\
Flamethrower (2) \\
Burning Hands (1) \\
Wall of Fire (4) \\
Incendiary Cloud (8) \\
Fireball (3) \\
Delayed Fireball (7) \\
Pyro Expert (2) \\
Shimmer powder (2) \\
Produce Flame (0) \\
Searing Ray (2) \\
Heat the Metal (2) \\
Meteor Swarm (9) \\
Fire Shield (4) \\
Burning Sphere (2) \\
Firestorm (7) \\

\medskip \textbf{List of the Earth - Characteristic Constitution}

\flushleft{Arma Energetica (1)\\}
Flesh to Stone - Stone to Flesh (6) \\
Summon Elemental (5) \\
Summon Minor Elementals (4) \\
Merge into Stone (3) \\
Move Terrain (6) \\
Stone Wall (5) \\
Bulkheads (5) \\
Pass Without Trace (2) \\
Stoneskin (4) \\
Repair (0) \\
Meteor Swarm (9) \\
Sculpting Stone (4) \\
Turn Stone to Mud (5) \\
Earthquake (8) \\

\medskip \textbf{Abjuration - Characteristic Intelligence}

\flushleft{Allarme(1)\\}
Anti-Detection (3) \\
Magic Armor (1) \\
Sacred Aura (8) \\
Vital bubble (4) \\
Anti-Magic Field (8) \\
Magic Circle (3) \\
Counterspell (3) \\
Dissolve Good and Evil (5) \\
Dispel Magic (3) \\
Dispel magic advanced (5)\\
Exile (4) \\
Lighthouse of Hope (3) \\
Glyph of Interdiction (3) \\
Globe of Invulnerability (6) \\
Imprisoning (9) \\
Death Ban (4) \\
Freedom of Movement (4) \\
Prohibition (6) \\
Poison Protection (2) \\
Protection from Good and Evil (1) \\
Energy Protection (3) \\
Minor Energy Protection (1) \\
Resistance (0) \\
Remove Curse (3) \\
Sanctuary (1) \\
Private Shrine (4) \\
Shield (1) \\
Shield of Faith (1) \\
Arcane Lock (2) \\
Symbol (7) \\
Supervision and Interdiction (6) \\
Constraint of Interdiction (2) \\
Mental Void (8) \\

\medskip \textbf{Animals and Plants - Characteristic Wisdom}

\flushleft{Amicizia con gli Animali (10)\\}
Animal Messenger (2) \\
Beneficial Berries (2) \\
Growth of Spikes (2) \\
Plant Growth (3) \\
Dominate Beasts (4) \\
Summon Animals (3) \\
Summon Mount (2) \\
Animal Forms (8) \\
Kyrin's Gragnola of Acorns (2) \\
Kyrin's Fire of Acorns (3) \\
Kyrin's Chestnut Gragnola (5) \\
Anti-Life Shell (5) \\
Giant Bug (4) \\
Entangle (1) \\
Locate Animals and Plants (2) \\
Locate Creature (4) \\
Metamorphosis (4) \\
Pure Metamorphosis (9) \\
Spider Movements (2) \\
Wall of Thorns (6) \\
Talking to Animals (1) \\
Talk to the Plants (3) \\
Pass Without Trace (2) \\
Bark Leather (2) \\
Plague of Insects (5) \\
Purifying Food and Drinks (1) \\
Spider web (2) \\
Enchanted Club (0) \\
Reincarnation (5) \\
Awakening (5) \\
Poisonous Spray (0) \\
Tree translation (5) \\
Plant Transport (6) \\
Find Familiar (1) \\
Greasy (1) \\

\medskip \textbf{Enchantment - Characteristic Charisma}

\flushleft{Anatema (1)\\}
Dislike / Sympathy (8) \\
Cruel Prank (0) \\
Calm Emotions (2) \\
Charm on People (1) \\
Command (1) \\
Compulsion (4) \\
Confusion (4) \\
Constriction (5) \\
Irresistible Dance (8) \\
Finger (0) \\
Dominate Beasts (4) \\
Dominate Monsters (8) \\
Dominating People (5) \\
Heroism (1) \\
Enrapture (2) \\
Change Memory (5) \\
Power Word Stun (8) \\
Word of Power to Kill (9) \\
Mental Regression (8) \\
Unstoppable Laughter (1) \\
Nap (2) \\
Sleep (1) \\
Suggestion (2) \\
Mass Suggestion (6) \\
Area of Truth (2) \\

\medskip \textbf{Healing - Characteristic Wisdom}

\flushleft{Aiuto (2)\\}
Cure Critical Wounds (5) \\
Mass Wound Cure (variable) \\
Cure Light Wounds (1) \\
Cure Wounds Series (3) \\
Healing (6) \\
Mass Healing (9) \\
Word Healing (1) \\
Mass Word Healing (3) \\
Prayer of Healing (2) \\
Regeneration (7) \\
Remove Disease (4) \\
Remove Poison (3) \\
Rebirth (3) \\
Lower Restoration (2) \\
Superior Catering (5) \\
Strength (4) \\

\medskip \textbf{Divination - Characteristic Wisdom}

\flushleft{Chiaroveggenza (3)\\}
Accurate Strike (0) \\
Understanding of Writings (2) \\
Understanding of Languages (1) \\
Communion (5) \\
Communion with Nature (5) \\
Knowledge of Legends (5) \\
Divination (6) \\
Guide (1) \\
Identification of Thoughts (2) \\
Identification of Good and Evil (1) \\
Detection of Diseases and Poisons (2) \\
Telepathic Bond (5) \\
Languages (3) \\
Locate Object (2) \\
Arcane Eye (4) \\
Omen (2) \\
Forecast (9) \\
Discover the route (6) \\
Discover Traps (2) \\
Scrutinizing (5) \\
See Invisibility (2) \\
True Seeing (6) \\

\medskip \textbf{Summon - Characteristic Intelligence}

\flushleft{Alleato Planare (6)\\}
Teleportation Circle (5) \\
Create Beer (0) \\
Creating Food and Water (3) \\
Invisible Cook (1) \\
Desire (9) \\
Floating Disc (1) \\
Summon Celestials (7) \\
Summon Woodland Creatures (4) \\
Instant Summons (6) \\
Acid Splash (0) \\
Terrify Infernal (1) \\
Labyrinth (8) \\
Magic Hand (0) \\
Word of the Withdrawal (6) \\
Veiled Step (2) \\
Dimensional Gate (4) \\
Portal (9) \\
Wonderful Palace (7) \\
Secret Chest (4) \\
Faithful Hound (4) \\
Half plane (8) \\
Invisible Servant (1) \\
Planar Displacement (7) \\
Teleportation (7) \\
Black tentacles (4) \\

\medskip \textbf{Illusion - Characteristic Intelligence}

\flushleft{Allucinazione Mortale (4)\\}
Arcanist Magical Aura (2) \\
Magic Mouth (2) \\
Disguise Yourself (1) \\
Creation (5) \\
Phantom Steed (3) \\
Fatal (9) \\
Mislead (5) \\
Programmed Illusion (6) \\
Major Image (3) \\
Projected Image (7) \\
Silent Image (1) \\
Mirror Image (2) \\
Invisibility (2) \\
Greater Invisibility (4) \\
Arcane Mirage (7) \\
Fear (3) \\
Illusory Writing (1) \\
Seem (5) \\
Blur (2) \\
Silence (2) \\
Simulacrum (7) \\
Dream (5) \\
Colored Spray (1) \\
Illusory Ground (4) \\
Hypnotic Plot (3) \\

\medskip \textbf{Invocation - Characteristic Constitution}

\flushleft{Spiritual Weapon (2)\\}
Solar Flare (6) \\
Banquet of Heroes (6) \\
Barrier of Blades (6) \\
Greater Blessing (2) \\
Supreme Blessing (3) \\
Hut (3) \\
Circle of Death (6) \\
Contingency (6) \\
Tracer Bolt (1) \\
Hidden Blast (1) \\
Solar Explosion (8) \\
Divine Favor (1) \\
Shatter (2) \\
Force Cage (8) \\
Send (3) \\
Daylight (3) \\
Dancing Lights (0) \\
Luminescence (1) \\
Arcane Hand (5) \\
Wall of Force (5) \\
Darkness (1) \\
Divine Word (7) \\
Marking Punishment (2) \\
Sanctify (5) \\
Elastic Ball (4) \\
Arcane Sword (7) \\
Prismatic Spray (7) \\

\medskip \textbf{Necromancy - Characteristic Constitution}

\flushleft{Aiuto (2)\\}
Animate Dead (3) \\
Blindness / Deafness (2) \\
Advanced Blindness / Deafness (3) \\
Clone (8) \\
Contagion (5) \\
Create Undead (6) \\
Finger of Death (6) \\
Wounding (6) \\
Magic Jar (6) \\
Wither (4) \\
Inflict Wounds (2) \\
Talk to the Dead (3) \\
Astral Projection (9) \\
Fatigue Radius (2) \\
Resurrection (7) \\
Pure Resurrection (9) \\
Raise Dead (5) \\
Rebirth (3) \\
Inviolate Rest (2) \\
Cast Curse (3) \\
Penetrating Gaze (6) \\
Cold Touch (0) \\
Vampiric Touch (3) \\
False Life (1) \\

\medskip \textbf{Transmutation - Characteristic Dexterity}

\flushleft{Alterare Sé Stesso (13)\\}
Animate Objects (5) \\
Magic Weapon (2) \\
Enhanced Feature (2) \\
Conceal (7) \\
Disintegration (6) \\
Fabricate (4) \\
Stopping Time (9) \\
Ethereal Form (7) \\
Gaseous form (3) \\
Zoom in / out (2) \\
Intermittence (3) \\
Inversion of Gravity (7) \\
Slowness (3) \\
Talkative (8) \\
Message (0) \\
Quick Step (1) \\
Quick Withdrawal (1) \\
Break in (2) \\
Darkvision (2) \\
Telekinesis (5) \\
Transformation (9) \\
Restser's Furious Transformation (6) \\
Rope Trick (2) \\
Speed (3) \\

\medskip \textbf{Universal - Any Characteristic}

\flushleft{Artificio Druidico (0)\\}
Bless Water (2) \\
Blessing (1) \\
Magic Bolt (1) \\
Everlasting Flame (2) \\
Sacred Flame (0) \\
Identify (1) \\
Minor Illusion (1) \\
Detection of the Magical (1) \\
Magic Reading (1) \\
Light (1) \\
Magic Mark (0) \\
Prestidigitation (0) \\
Thaumaturgy (0) \\

\subsection{List of Spells by rarity} \index{List of Spells by rarity}

Spells in order of \textbf{Rarity} and Level

\medskip \textbf{Common} \medskip

Alarm (1) \\
Altering Yourself (2) \\
Anathema (1) \\
Animal Messenger (2 \\
Animate Dead (3) \\
Animate Objects (5) \\
Magic Weapon (2) \\
Spirit Weapon (2) \\
Sacred Aura (8) \\
Beneficial Berries (2) \\
Barrier of Blades (6) \\
Cruel Prank (0) \\
Bless (2) \\
Blessing (1) \\
Block Monsters (5) \\
Block Person (2) \\
Magic Mouth (2) \\
Falling Feather (1) \\
Calm Emotions (2) \\
Walk on Water (3) \\
Disguise Yourself (1) \\
Enhanced Feature (2) \\
Blindness / Deafness (2) \\
Magic Circle (3) \\
Charm on People (1) \\
Clairvoyance (3) \\
Accurate Strike (0) \\
Flame Strike (5) \\
Command (1) \\
Understanding of Languages (1) \\
Confusion (4) \\
Cone of Cold (5) \\
Knowledge of Legends (5) \\
Counterspell (3) \\
Check Water (4) \\
Creating Food and Water (3) \\
Create or Destroy Water (1) \\
Growth of Spuntoni (2) \\
Invisible Cook (1) \\
Cure Light Wounds (1) \\
Fire Bolt (1) \\
Magic Bolt (1) \\
Hidden Blast (1) \\
Phantom Steed (3) \\
Floating Disc (1) \\
Dispel Magic (3) \\
Finger of Death (6) \\
Dominate Monsters (8) \\
Dominating People (5) \\
Exile (4) \\
Enrapture (2) \\
Summon Mount (2) \\
Fabricate (4) \\
Sacred Flame (0) \\
Acid Splash (0) \\
Gust of Wind (2) \\
Melting into the Stone (3) \\
Shatter (2) \\
Acid Arrow (2) \\
Lightning (3) \\
Glyph of Interdiction (3) \\
Globe of Invulnerability (6) \\
Help (0) \\
Identify (1) \\
Minor Illusion (0) \\
Major Image (3) \\
Silent Image (1) \\
Mirror Image (2) \\
Identification of Good and Evil (1) \\
Detection of the Magical (1) \\
Inflict Wounds, 1, Common \\
Zoom in / out (2) \\
Entangle (1) \\
Send (3) \\
Invisibility (2) \\
Call the Lightning Bolt (3) \\
Burning Blade (2) \\
Magic Reading (1) \\
Levitation (2) \\
Freedom of Movement (4) \\
Languages (3) \\
Locate Creature (4) \\
Locate Object (2) \\
Daylight (3) \\
Light (1) \\
Burning Hands (1) \\
Magic Mark (0) \\
Message (0) \\
Metamorphosis (4) \\
Wall of Force (5) \\
Wall of Fire (4) \\
Wall of Ice (6) \\
Stone Wall (5) \\
Cloud of Fog (1) \\
Arcane Eye (4) \\
Thundering Wave (1) \\
Darkness (1) \\
Fireball (3) \\
Talking to Animals (1) \\
Pass Without Traces (2) \\
Bark Leather (2) \\
Dimensional Door (4) \\
Prayer of Healing (2) \\
Omen (2) \\
Prestidigitation (0) \\
Produce Flame (0) \\
Protection from Good and Evil (1) \\
Energy Protection (3) \\
Marked Punishment (2) \\
Purifying Food and Drinks (1) \\
Fatigue Radius (2) \\
Searing Ray (2) \\
Spider web (2) \\
Resistance (0) \\
Breathe Underwater (3) \\
Remove Disease (4) \\
Remove Curse (3) \\
Remove Poison (3) \\
Repair (0) \\
Lower Restoration (2) \\
Jumping (1) \\
Sanctuary (1) \\
Cast Curse (3) \\
Pick (2) \\
Sculpting Stone (4) \\
Discover Traps (2) \\
Illusory Writing (1) \\
Shield (1) \\
Shield of Faith (1) \\
Darkvision (2) \\
Arcane Lock, 2 \\
Invisible Servant (1) \\
Burning Sphere (2) \\
Blur (2) \\
Silence (2) \\
Sleep (1) \\
Colored Spray (1) \\
Lightning Grasp (0) \\
Suggestion (2) \\
Teleport (7) \\
Cold Touch (0) \\
Vampiric Touch (3) \\
Hypnotic Plot (3) \\
Rope Trick (2) \\
Greasy (1) \\
See Invisibility (2) \\
Constraint of Interdiction (2) \\
False Life (1) \\
Flying (3) \\

\medskip \textbf{Uncommon} \medskip

Help (2) \\
Deadly Hallucination (4) \\
Animal Friendship (1) \\
Anti-Detection (3) \\
Magic Armor (1) \\
Druidic Artifice (0) \\
Arcanist Magical Aura (2) \\
Solar Flare (6) \\
Banquet of Heroes (6) \\
Greater Blessing (2) \\
Advanced Person Blocker (4) \\
Vital bubble (4) \\
Walking in the Wind \\
Hut (3) \\
Meat in Stone (6) \\
Advanced Blindness / Deafness (3) \\
Teleportation Circle (5) \\
Clone (8) \\
Understanding of Writings (2) \\
Compulsion (4) \\
Contagion (5) \\
Contingency (6) \\
Create Undead (6) \\
Plant Growth (3) \\
Cure Critical Wounds (5) \\
Cure Wounds Series (3) \\
Tracer Bolt (1) \\
Desire (9) \\
Disintegration (6) \\
Finger (0) \\
Heroism (1) \\
Summon Animals (3) \\
Summon Minor Elementals (4) \\
Divine Favor (1) \\
Wounding (6) \\
Gaseous form (3) \\
Mislead (5) \\
Kyrin's Acorn Shaw (2) \\
Anti-Life Shell (5) \\
Programmed Illusion (6) \\
Projected Image (7) \\
Wither (4) \\
Detection of Diseases and Poisons (1) \\
Giant Bug (4) \\
Death Ban (4) \\
Intermittence (3) \\
Terrify Infernal (1) \\
Greater Invisibility (4) \\
Locate (2) \\
Dancing Lights (1) \\
Luminescence (1) \\
Arcane Hand (5) \\
Slowness (3) \\
Spider Movements (2) \\
Move Terrain (6) \\
Wall of Thorns (6) \\
Wall of Wind (3) \\
Stinking Cloud (3) \\
Power Word Stun (8) \\
Word Healing (1) \\
Bulkheads (5) \\
Veiled Step (2) \\
Fear (3) \\
Stoneskin (4) \\
Stone in Mud (5) \\
Pyro Expert (2) \\
Shimmer powder (2) \\
Forecast (9) \\
Prohibition (6) \\
Poison Protection (2) \\
Inviolate Rest (2) \\
Unstoppable Laughter (1) \\
Heat the Metal (2) \\
Superior Catering (5) \\
Quick Withdrawal (1) \\
Discover the route (6) \\
Fire Shield (4) \\
Seem (5) \\
Symbol (7) \\
Dream (5) \\
Poisonous Spray (0) \\
Mass Suggestion (6) \\
Thaumaturgy (0) \\
Telekinesis (5) \\
Ice Storm (4) \\
Black tentacles (4) \\
Illusory Ground (4) \\
Speed (3) \\
Supervision and Interdiction (6) \\
Mental Void (8) \\
Area of Truth (2) \\

\medskip \textbf{Rare} \medskip

Dislike / Sympathy (8) \\
Supreme Blessing (3) \\
Anti-Magic Field (8) \\
Chain of Lightning (6) \\
Conceal (7) \\
Communion (5) \\
Constriction (5) \\
Create Beer (0) \\
Creation (5) \\
Dispel magic Advanced (5)\\
Dissolve Good and Evil (5) \\
Divination (6) \\
Solar Explosion (8) \\
Summon Elemental (5) \\
Instant Summons (6) \\
Lighthouse of Hope (3) \\
Fatal (9) \\
Stopping Time (9) \\
Ethereal Form (7) \\
Animal Forms (8) \\
Force Cage (6) \\
Kyrin's Flaming Acorn Rainfall (3) \\
Healing (6) \\
Imprisoning (9) \\
Identification of Thoughts (2) \\
Inversion of Gravity (7) \\
Labyrinth (8) \\
Flamethrower (2) \\
Telepathic Bond (5) \\
Talkative (8) \\
Pure Metamorphosis (9) \\
Arcane Mirage (7) \\
Prismatic Wall (9) \\
Incendiary Cloud (8) \\
Cloud of Death (5) \\
Delayed Fireball (7) \\
Talk to the Dead (3) \\
Talking to the Plants (3) \\
Word of Power to Kill (9) \\
Word of the Withdrawal (6) \\
Mass Word Healing (3) \\
Plague of Insects (5) \\
Stone in Flesh, (6) \\
Minor Energy Protection (1) \\
Mental Regression (8) \\
Reincarnation (5) \\
Awakening (5) \\
Sanctify (5) \\
Secret Chest (4) \\
Scrutinizing (5) \\
Faithful Hound (4) \\
Half plane (8) \\
Freezing Sphere (6) \\
Elastic Ball (4) \\
Simulacrum (7) \\
Arcane Sword (7) \\
Prismatic Spray (7) \\
Firestorm (7) \\
Transformation (9) \\
Tree translation (5) \\
Strength (4) \\
True Seeing (6) \\

\medskip \textbf{Very Rare} \medskip

Energy Weapon (2) \\
Circle of Death (6) \\
Communion with Nature (5) \\
Check Weather (8) \\
CTRLC + CTRLV (Copy Paste) (1) \\
Dominate Beasts (4) \\
Stone Mud (5) \\
Magic Jar (6) \\
Kyrin's Chestnut Gragnola (5) \\
Change Memory (5) \\
Divine Word (7) \\
Quick Step (1) \\
Astral Projection (9) \\
Rebirth (3) \\
Private Shrine (4) \\
Penetrating Gaze (6) \\
Sleet Storm (3) \\
Earthquake (8) \\
Restser's Furious Transformation (6) \\
Plant Transport (6) \\

\medskip \textbf{Legendary} \medskip

Planar Ally (6) \\
Lunar Flare (2) \\
Contact Other Plans (5) \\
Irresistible Dance (8) \\
Summon Celestials (7) \\
Summon Woodland Creatures (4) \\
Summon Fairy (6) \\
Everlasting Flame (2) \\
Spiritual Guardians (3) \\
Guardian of the Faith (4) \\
Mass Healing (9) \\
Planar Bond (5) \\
Hunter's Mark (1) \\
Portal (9) \\
Wonderful Palace (7) \\
Resurrection (7) \\
Pure Resurrection (9) \\
Raise Dead (5) \\
Regeneration (7) \\
Saving the Dying (0) \\
Meteor Swarm (9) \\
Nap (2) \\
Planar Shift (7) \\
Storm of Vengeance (9) \\
Find Familiar (1) \\

}
\end{multicols}

\pagebreak

\section{Advantages} \index{Advantages} \hypertarget{vantaggi}{} \label{vantaggiinizio}

\begin{changemargin}{0.3cm}{0.3cm} \begin{enfasi}{I love being a superhero! The working hours are bad, the pay is non-existent ... but at least I don't run the risk of being fired!  (PK)
} \end{enfasi} \end{changemargin} \medskip

\begin{multicols}{2}

\lettrine[lines = 2, lhang = 0.33, loversize = 0.25, findent = 1.5em]{E}{very} character can have, and it is not mandatory to have, Advantages. These have to be interesting, enjoyable, fun and above all playable.

Each Advantage has a cost, to be paid at each level. As mentioned, it should not be mandatory to take an advantage, nor should you take advantages just because they make you strong. The purpose of an Advantage is to amaze and have fun.

Having an advantage means being different, being a freak, having that detail that makes you special and unique, but not always the strongest, most powerful or invincible. An advantage is not just a skill, it is an opportunity for role play. The player is invited to be creative in choosing the advantages and also in creating new ones, the cost is then decided with the Storyteller. And it is always and in any case the Storyteller who has the last word on the chosen Benefits.

Several advantages do not have a concrete and immediate practical effect but are enriching the background, the story of the character, introducing opportunities for play and fun. When choosing the advantages, and consequently the disadvantages, it is not like going to stock up on super powers and extraordinary abilities, but of peculiarities, delusions, specialties that the character possesses and that once again make him different, unique, only yours. .

Therefore advantages and disadvantages must also and above all be played and interpreted.

The Storyteller could also insert advantages and disadvantages peculiar to the characterization of the character such as immunity to diseases, healing touches, extrasensory abilities, abilities that modify the relationship with a familiar ... Always be scrupulous in the analysis and evaluation of the benefits, remembering that there there must also be an adequate score given by Disadvantages.


\begin{itemize}

\item
Advantages with{*} and all those with a cost of 20 or more are at the discretion of the Storyteller in being admitted to the choice.

\item
The advantages are chosen at the first level, any advantage taken at subsequent levels must be agreed with the Storyteller.

\item
The cost points of an Advantage are paid with the points earned by the Disadvantages.

\item
The bonuses given to the skills are specific on the check when indicated in brackets.

\item Unless otherwise noted, it costs an Action to activate an Advantage (if the effect is not permanent).

\end{itemize}


\begin{changemargin}{0.3cm}{0.3cm} \begin{enfasi}{
From a big advantage comes a big disadvantage! (cit. "From a great power comes with great responsibility ", Amazing Fantasy 15, Stan Lee)
} \end{enfasi} \end{changemargin} \medskip


\subsection{List of Advantages} \index{List of Advantages}

\textbf{Wings of providence} \index{Wings of providence} 20: you have wings, the choice of shape and color is up to you, they usually fit on the shoulder blades and make you fly. Unless otherwise agreed, the in-flight movement is equal to the racial ground movement.

\textbf{Ambidextrous} \index{Ambidextrous} 10: You can use both hands indifferently. The penalties on checks where two hands are used decrease by 2

\textbf{Friend of animals} \index{Friend of animals} 5: +2 to the checks to manage animals (including wild ones)

\textbf{Anfibio} \index{Anfibio} 20: you can breathe both underwater and air

\textbf{Arcobaleno} \index{Rainbow} 10: you are an artist. Your fingers spontaneously produce color

\textbf{Aura of Courage} \index{Aura of Courage} 15: Around you, within 3 meters of distance, instill courage. +2 Saving Throw vs natural or magical effects of fear.

\textbf{Claws} \index{Claws} 5: Remember to tick off the claws every now and then. 1d4 damage per attack. Natural attacks with the second hand take the damage bonus given by Strength.

%\begin{center}
%\includegraphics[width=0.3\linewidth]{immagini/claw.png}
%\end{center}


\textbf{Drinking is good} \index{Drinking is good} 5: Prerequisite: The liver does not count. Your body metabolizes alcohol very effectively. A liter of beer gives you 1d4 hit points, a bottle of liquor 1d8 hit points. If bad quality, no .. You can still get drunk.

\textbf{Drinking is very good} \index{Drinking is very good} 10: Prerequisite: The liver does not count. Your body metabolizes alcohol very effectively. A liter of beer gives you 2d4 hit points, a bottle of liquor 2d8 hit points. If of bad quality, no. You can't get drunk on natural liquids.

\textbf{Cat Fall} \index{Cat Fall} 5: +2 to Dexterity checks on falls and +2 to Move Silently.

\textbf{Chameleon} \index{Chameleon} 10-20: Your skin can change color. Time needed 1 minute / 1 round.

\textbf{Change shape} \index{Change shape} 40: as a spell alter self. It can be used every 10 minutes.

\textbf{Walking on air} \index{Walking on air} 30: not too controlled. Anything other than walking requires a Dexterity check or falling prone (but not on the ground).

\textbf{Walking on water} \index{Walking on water} 30: but don't give yourself airs ..

\textbf{Magnetic} \index{Magnetic} 5-10: give off light whenever you want. thankfully not literally. +2 to Charisma checks.

\textbf{Reduced consumption} \index{Reduced consumption} 5: Drink and eat half of a normal man. You are under weight.

\textbf{Metabolism Control} \index{Metabolism Control} 10 - the name alone is great! Reduce Bleed damage by 2 at end of each round.

You regain hit points as if you had double your Constitution score.

\textbf{Effective Healing} \index{Effective Healing} 10: + 1d6 hit points healed whenever you use a Heal spell on yourself or others.

\textbf{Daredevil} \index{Daredevil} 10: You like to jump into the fray, especially if you are in danger. +2 attack / defense rolls while surrounded by three or more opponents.

\textbf{Teeth} \index{Teeth} 5: Your bite hurts, 1d6. Brush your teeth every now and then ..

\textbf{Universal digestion} \index{Universal digestion} 5: As long as it doesn't hurt you eat, +2 Fortitude vs Poison saving throw. Immune to natural stomach ailments.

\textbf{Absolute Direction} \index{Absolute Direction} 5: you always know where magnetic north is. You have + 1d6 on orientation checks.

\textbf{Hard to subjugate} \index{Hard to subjugate} 10: +2 Will save on Charm school spells

\textbf{Hard to Kill} \index{Hard to Kill} 5: You don't faint at 0 hit points, but at -LV / 2 hit points. Die at 15 + Constitution x 3 hit points.

\textbf{Empathy with plants} \index{Empathy with plants} 10: I understand the suffering of crushed grass.

\textbf{Empathy} 5: +2 on Perceiving Emotions checks.

\textbf{Animal Empathy} \index{Animal Empathy} 10: + 1d6 on trials to handle animals (even wild animals).

\textbf{Spiritual empathy} \index{Spiritual empathy} 5: you don't talk to spirits, but you feel their emotions.

\textbf{Hermaphroditus} \index{Hermaphroditus} 10: lgbtE !.

\textbf{Forged in steel} \index{Forged in steel} 5: Through painful operations your skin has been coated with metal plates. Your base Defense is 13.

\textbf{Shadow shape} \index{Shadow shape} 30: Consider being able to turn yourself into a shadow for 1 hour per level. You can only move in shady areas.

\textbf{Lucky} \index{Lucky} 5: 2 times per day you can roll a 1 on the die to 6, to be declared before the die roll.

\textbf{Accelerated healing} \index{Accelerated healing}: 5 every morning you regain twice as many hit points as you would normally regain. Combines with Metabolism Control. \index{Metabolism Control}.

\textbf{Healer} \index{Healer} 5: you know where to put your hands. + 1d6 on First Aid checks

\textbf{The liver is not counted} \index{The liver is not counted} 10: you can drink a lot and don't get drunk

\textbf{Illuminated} \index{Illuminated} 10-20: light up .. literally. Emit light in a 3/6 meter radius. You can control (20) the emission or not (10).

\textbf{Immune} \index{Immune} 5-20: to what?

\textbf{Invisible} \index{Invisible} 40: your body is invisible. All time. And it's not magic ...

\textbf{Wrath} \index{Wrath} 5: you are capable of getting mad. +2 to melee damage and -1 Attack and Defense Roll. Each other 5 points +2 damage -1 Attack and Defense Roll, max 20 points. Duration 4 (even non-consecutive) rounds every 5 points. It is activated with an Action.

\textbf{My shadow is my friend} \index{My shadow is my friend} 10: You can place your shadow wherever you want within 3 meters. Your shadow can manipulate objects like an invisible minion. You are assumed to be able to cast touch spells via your shadow (which must be present) within 10 feet.

\begin{center}
\includegraphics[width = 0.35 \linewidth]{immagini/shadow.png}
\end{center}

\textbf{Bonds of Fury} 15 \index{Bonds of Fury}: You can summon ethereal snares that threaten your enemies. 3 times per day at the cost of 1 Action all opponents within 30 feet around you are affected by the Entangle spell until the next round. Saving Throw vs Reflex DC 10 + 1/2 LV + Charisma) to break free.

\textbf{Slow and Firm} 5: \index{Slow and Firm} You are exceptionally stable on your feet. You can't be moved or lifted except by a creature 2 sizes larger.

\textbf{Universal language} \index{Universal language} 10. Your language skills are impressive. After two days in contact with a new language you are able to speak it correctly. After 3 days away from the environment you forget your language. You earn +2 on language-based checks.

\textbf{Explosive Magic} \index{Explosive Magic} 10: Your Invocation spells that cause damage have one more die of damage (when a die has to be rolled ..).

\textbf{Hands of the Fairy} \index{Hands of the Fairy} 10: + 1d6 check of Fairy Hands and Escape Artist involving hands. You can get 16 as you get a 10 on relative checks.

\textbf{Hand Webbed Foot} \index{Hand Webbed Foot} 5: + 1d6 on swim checks.

\textbf{Early riser} \index{Early riser} 5-10-15: you just need to sleep 6/5/4 hours a night to be fully rested.

\textbf{Medium} \index{Medium} 10-20: sometimes you want it, other times they look for you.

\textbf{Photographic memory} \index{Photographic memory} 20-50: fortunately it is not permanent (50). + 2d6 checks to remember details (Knowledge and Awareness).

\textbf{Hairy nose} \index{Hairy nose} 5: Your nostrils filter the toxins in the air you breathe. +2 to relative checks. Your nose is in size… not small.

\textbf{You don't sleep} \index{You don't sleep} 20{*}: and I don't know how you do it ..

\textbf{You don't get old} \index{You don't get old} 20{*}: you don't get old (but they can kill you anyway).

\textbf{You don't eat, drink} \index{You don't eat, drink} 20: and I don't know how you do it ..

\textbf{You don't breathe} \index{You don't breathe} 20: and I don't know how you do it ..

\textbf{The smell of blood} \index{The smell of blood} 10: The smell of blood is a powerful drug
Prerequisites: You cannot have "Liver Doesn't Matter". You gain +1 to attack rolls and +1 to damage for each enemy you killed with your weapon in the round. This bonus cannot exceed + 4 / + 4. The bonus remains active until the round following the last kill. Creatures with less than 3 LVs than you do not count.

\textbf{Oracle} \index{Oracle} 20: for some it is a curse. The use must always be agreed with the Storyteller.

\textbf{Great View} \index{Great View} 5: You have a great view (12/10). +2 to relative Awareness checks using sight.

\textbf{Excellent smell and taste} \index{Excellent smell and taste} 5: you have an excellent taste and smell. +2 to Awareness checks using smell or taste.

You gain + 1d6 on checks to recognize a potion or natural poison.

\textbf{Excellent tact} \index{Excellent tact} 5: You have excellent tact. you can read with your fingers. You are able to find a hidden door by tapping the wall.

\begin{center}
\includegraphics[width = 0.9 \linewidth]{immagini/braille2.png}
\end{center}


\textbf{Excellent hearing} \index{Excellent hearing} 5: You have excellent hearing. +2 to Awareness checks involving hearing.

\textbf{Talking with animals} \index{Talking with animals} 20: choose a family (sheep, marsupials, cords ...).

\textbf{Talking with plants} \index{Talking with plants} 20: I've always wanted to talk to courgettes ..

\textbf{Blind Perception} (Blind Sight): \index{Blind Sight} \index{Blind Sight} 30: You can perceive anything with your senses within 60 feet, from smell, to heat. You can "see" through and up to 18 meters, 5 cm of stone, 10 cm of wood, 0.5 cm of metal.

\textbf{Perfect balance} 5: \index{Perfect balance} +2 to the relative Acrobatics checks.

\textbf{Fast feet} \index{Fast feet} 10/20: Your movement increases by 1/3 meters.

\textbf{Green thumb} \index{Green thumb} 5: + 1d6 on Profession checks (Herbalism, Gardener ..).

\textbf{Iron lungs} \index{Iron lungs} 5: you can hold your breath 2 * Constitution minutes (minimum 2 minutes).

\textbf{Precognition} 30{*} \index{Precognition}: Like the Foresight spell.

\textbf{Recovery} \index{Recovery} 10: Your body spontaneously produces caffeine. You take half the time to recover from the fatigued condition.

\textbf{Endurance} \index{Endurance} 5-10: + 1 / + 2 Reflex or Fortitude or Will saving throw.

\textbf{Damage resistance} \index{Damage resistance} 10: -1 damage. -1 additional damage for every 5 additional points. It establishes the type of resistance (cut, blow, perforation, energy ..).

\textbf{Magic Resistance} \index{Magic Resistance} 20: You have an RM 10.

\textbf{Resistance to fire / cold / electricity} \index{Resistance to ...} 5-10: Ignore the first 3/6 points of damage per round.

\textbf{Rebuild} \index{Rebuild} 30: losing a hand has never been a problem ..

\textbf{Regeneration} \index{Regeneration} 30: + 1 Wound points per Turn (do not regenerate limbs).

\textbf{Fast regeneration} 40: + 1 Wound points per round (do not regenerate limbs). Die if they destroy your body (or there's nothing left but ash).

\textbf{Shrink} 30: You can decrease up to two sizes. Duration up to 8 hours.

\hypertarget{rinoceronte}{} \textbf{Rhino} 10: Your charge is destructive. It is considered that nothing under the strength of iron bars (hardness 15) can stop your charge. Behind you you leave a trail of destruction. +2 to attack rolls in charge and + 1d6 damage.

\textbf{Mind Shield} \index{Mind Shield} 5: +2 Saving Throw on mental controls and influences.

\textbf{Protected senses} \index{Protected senses} 5: +2 Saving Throw against sounds / lights / vapors or spells that affect and through your senses.

\textbf{Common sense} \index{Common sense} 5: if you are about to make a bad impression, a bell warns you.

\textbf{Fashion sense} \index{Fashion sense} 5: you always know how to dress well, even if only with a rag.

\textbf{Sense of vibrations} \index{Sense of vibrations} \index{Telluric sense} (Telluric sense) 30: everything makes the earth tremble a little, or almost, a radius of 18 meters around you.

\textbf{Sense of time} \index{Sense of time} 5: you always know what time it is, day or night.

\textbf{Spider Sense} \index{Spider Sense} 15: no a radioactive man did not bite you, but you are extremely sensitive to dangers. +2 initiative, you can't be surprised.

\textbf{Fearless} \index{Fearless} 10: You are immune to fear, magical or not.

\textbf{Silent} \index{Silent} 5: + 1d6 on Awareness checks (move silently).

\textbf{Spine} \index{Spine} 5: and you are ugly too. 1d4 damage.

\textbf{Super Dog Tag} \index{Super Dog Tag} 5: Reduce Bleed damage by 1 at the end of each round.

\textbf{Talent for languages} \index{Talent for languages} 5: Learn two languages by investing 1 point in Linguistic Knowledge.

\textbf{Wild Talent}: \index{Wild Talent} Let's talk.

\textbf{Cold Touch} \index{Cold Touch} 10: By touching a dead man (within 1 day per level) you can see and hear what happened in his last round of life.

\textbf{Troll} \index{Troll} 60: Regenerate 5 hit points per round even if hit points are negative. You also regenerate limbs. You can only be "killed" by fire or acid. A condition could still keep you at negative hit points (e.g. submerged underwater).

\textbf{Subsonic hearing} \index{Subsonic hearing} 10: Hear frequencies inaudible to humans (like a dog)

\textbf{See the invisible} \index{See the invisible} 15: Better X-ray vision .. drool ..

\textbf{Understanding the truth} \index{Understanding the truth} 5: The truth has a sound of its own. + 1d6 to Feel Emotion checks.

\textbf{Demonic Sight} \index{Demonic Sight} 15: See in total darkness, even magical, up to 60 feet.

\textbf{Perimeter View} \index{Perimeter View} 5: Sole? +2 to Awareness checks on the side.

\textbf{Telescopic Vision} \index{Telescopic Vision} 10: + 1d6 on Awareness and Vision checks but from a distance.

\textbf{Persuasive voice} \index{Persuasive voice} 5: +2 to Charisma checks that use the voice,

\textbf{Subsonic Voice} \index{Subsonic Voice} 10: Make sounds inaudible to humans. Dogs hate you.

\end{multicols}

\pagebreak

\section{Disadvantages} \index{Disadvantages}

\begin{changemargin}{0.3cm}{0.3cm} \begin{enfasi}{If you have to be a cripple, better be a rich cripple. (Tyrion Lannister)} \end{enfasi} \end{changemargin} \medskip

\begin{multicols}{2}

\lettrine[lines = 2, lhang = 0.33, loversize = 0.25, findent = 1.5em]{O}{ne} disadvantage characterizes the character, defines limits and fears. Each character must have at least 1 role disadvantage and this does not give him bonus points.

The points taken with the psycho / physical disadvantages are used to cover the points spent with the advantages. Obviously the Evil Storyteller also likes more disadvantages ...

\textbf{Each player must play his disadvantages otherwise he does not gain experience points and will be denied the use of the advantages.}

A disadvantage can be "undone" in the course of the character's story and there must be an adventure that justifies it all. As always, the Storyteller has the final say on any choice of advantages and disadvantages.

\bigskip

Tips
\begin{itemize}
\item
Take disadvantages that are fun to play, even if they will get you in trouble.
\item
Take disadvantages that are interesting to play with other players even if it will get them in trouble
\item
Take disadvantages that have to do with the character
\item
Take disadvantages that you will not regret
\end{itemize}

\textbf{Be careful}:

\begin{itemize}
\item
Avoid disadvantages that are difficult to play or because they are completely disconnected from the system or totally useless or severely harmful to others. If you want to be an extreme pacifist, consider the character and the group well.
\item
Don't take any disadvantages that may make you ashamed to act
\item
Do not take disadvantages that have nothing to do with the character (in perfect contradiction to what has already been said ...)
\item
Don't take silly disadvantages (like fear of turning right, of elevators ...)
\item
If you take a severe disadvantage, play it well, the Storyteller will be able to reward you
\end{itemize}

\subsection{Role Disadvantages and Psycho / Physical Disadvantages} \hypertarget{svantaggi}{} \label{svantaggidiruolo}

The disadvantages are divided into two categories, \textbf{Role Disadvantages} and \textbf{Psycho / physical disadvantages}.

The \textbf{Role Disadvantages} are small flaws, tics, big and small problems that serve to give a more "human" depth to the character. They have a deliberately ambiguous and joking description, choose them carefully and discuss with the Storyteller how you intend to interpret this disadvantage.

The player is invited to create new role disadvantages. These disadvantages do not grant a bonus or malus nor do they give points to take advantage. \textit{But they are fun!}

\bigskip

The \textbf{Psycho / physical disadvantages} instead have more impact in the game, in everyday life giving concrete disadvantages. These disadvantages provide the points with which to "pay" for the advantages. At the bottom you will find a list of Phobias.

\end{multicols}

\pagebreak

\subsubsection{Disadvantages of Role} \index{Disadvantages of Role}


\begin{multicols}{2}


\textbf{Alcoholism}: \index{Alcoholism} you like to drink, and a lot .. but when do you stop?

\textbf{Trendy} \index{Trendy}: Yours probably, even with new clothes you never dress well. The combination of colors is always an eyesore.

\textbf{Friend of animals}: \index{Friend of animals} understood as fleas, ticks, lice, bedbugs .. flies. You have a zoo on you.

\textbf{Attract animals}: \index{Attract animals} you don't know why but you are always surrounded by cats, dogs, bunnies, ladybirds ..

\textbf{Lure Trouble} \index{Lure Trouble}: It's not my fault the dragon swerved to come poo here ..

\textbf{Banana}: \index{Banana} the one you try to get in your hair, but can't. Your hair doesn't get along well with you.

\textbf{Low pain threshold}: \index{Low pain threshold} scratched me, help! I'm dying!!!

\textbf{Pimples}: \index{Pimples} full, your face is pockmarked and these disgusting yellow pimples keep forming.

\textbf{Ciuccione:} \index{Ciuccione} you don't do it often, but when you are most nervous you take out the old wooden pacifier .. (or failing that, your thumb is always good).

\textbf{Coward}: \index{Coward} it is better to run away, sorry, let's first gather all the information before attacking.

\textbf{Cogito ergo sum}: \index{Cogito ergo sum} you have a tendency to talk to yourself, but aloud even if there are people around and even if they are not friendly.

\textbf{Gullible}: \index{Gullible} come on? really ? and at what height did the donkey fly?

\textbf{Hamster}: \index{Hamster} understood as memory. You can't match names to faces.

\textbf{Rotten teeth}: \index{Rotten teeth} probably the toothbrush you use does not have real boar bristles ...

\textbf{Nose fingers}: \index{Nose fingers} I hope they are at least good.

\textbf{Diva}: \index{Diva} or at least you think you are. Do not miss an opportunity to show off your non-existent singing, comic, aesthetic skills ... with everyone's big laughs.

\textbf{Common face}: \index{Common face} what's your name? I think I've seen you before ...

\textbf{Gallant}: \index{Gallant} bordering on maniacal, in every gesture you are formal, appropriate and cordial.

\textbf{Killer}: \index{Killer} no, you are not a killer. However, you always have cold hands and feet.

\textbf{Scaring animals}: \index{Scaring animals} can also be handy, were it not for horses running away and bears attacking ...

\textbf{Unable to have fun}: \index{Unable to have fun} so? it's your problem, not mine.

\textbf{English:} \index{English} understood as humor. Nobody ever gets your lines.

\textbf{Glutton}: \index{Glutton} CIOMP !. Never skimp, it could be the last meal!

\textbf{Meteora}: \index{Meteora} you suffer from compulsive and noisy bloating, not to mention the unpleasant smell.

\textbf{Megalomaniac}: \index{Megalomaniac} let's involve the armies of the seven kingdoms and enter the dungeon!

\textbf{Mint:} \index{Mentina} if you only ate garlic and onion your breath would be less smelly

\textbf{Musichiere}: \index{Musichiere} with his mouth. You whistle constantly, on every occasion that you are lost in thought or very tense .. you start whistling.

\textbf{Not empathic}: \index{Not empathic} why is the baby whose bear I just set fire to cry?

\textbf{Obsession}: \index{Obsession} again, again, again. Another tube of skin cream !!

\textbf{Parcel}: \index{Parcel} yours. You always have a hand over there. Maybe the pants are tight? and no, I'm not shaking your hand.

\textbf{Bad font}: \index{Bad font} it's okay to be gruff .. but do you always have to make it obvious?

\textbf{Piebald}: \index{Piebald} no, not the cow or your mare but your armpit. Sweat profusely, whether it's hot or cold ... or nervous.

\textbf{Mental stiffness}: \index{Mental stiffness} no, I don't understand, the map says to go right. I don't care if there isn't a right.

\textbf{Know-how} \index{Know-how}: the right answer is yours alone. There is no doubt… for you.

\textbf{Nosebleed}: \index{Nosebleeds} it happens, and always as soon as you see a woman / man (depending on your taste) that you like.

\textbf{Sciarpina}: \index{Sciarpina} you must always have a certain type of garment on and visible, otherwise you don't go out of the cave.

\textbf{Secret}: \index{Secret} I have a secret, so much secret that I don't know if I do either ...

\textbf{Following Chaos}: \index{Following Chaos} is stronger than you, you can never obey any law or authority in charge.

\textbf{Following the Law}: \index{Following the Law} is stronger than you, no matter what the law is, you don't violate it.

\textbf{Tattooed:} \index{Tattooed} tattooing is the way of life. You have at least 30 \% of your body already tattooed and you don't miss opportunities to get new tattoos.

\textbf{Topi}: \index{Topi} you are a TOPI!

\textbf{Nails}: \index{Nails} You are a compulsive nail eater, your fingertips bleed sometimes.

\textbf{Last word} \index{Last word}: he is stronger than you, you must have the last word in every speech.

\end{multicols}

\pagebreak

\subsection{Psycho / physical disadvantages} \index{Psycho / physical disadvantages}


\begin{multicols}{2}

\textbf{Albino} \index{Albino}

You are White, as if it were milk. You can't tan and can't stand the light, your skin is delicate.

\textbf{13}: Besides being extremely recognizable you have the following disadvantages: Myopia and Photosensitivity and Sensitive Skin.

\textbf{Allergy} \index{Allergy}

You have some form of allergy. I hope not serious. Make sure you always have a potion of poison remover with you.

\textbf{5:} In the presence of a specific allergen the character sneezes loudly until the allergen is removed, -1 on all checks. (eg. Allergic to Beer)

\textbf{10}: Character suffers from coughing fits, hyperlacrimation, dizziness, -2 on all checks. Fortitude save DC 10 so as not to suffocate. The shot should be repeated every 20 rounds until you have moved away from the allergen.

\textbf{15:} The character suffers from violent coughing fits, nausea, cold sweats, palpitation. -1d6 on all checks, a DC 15 Fortitude save is required or faint. Throws are repeated every 5 rounds until the allergen is removed.

\textbf{20}: The character falls prey to a respiratory crisis, and is unable to take any action other than throwing up, gagging in vomit and trying to survive. Failing a DC 25 Fortitude save causes the character to die in inhuman spasms. The shot must be repeated every round until the allergen is removed.

Note: allergens that are too rare do not count.

\textbf{Hallucinations} \index{Hallucinations}

there is something wrong with your head, every now and then a spark is triggered.

\textbf{10}: The character sees and hears things that are not there. Each day you roll a 1d6.
If they roll a 1 or 2, nothing happens.
With 3,4 or 5 there will be one or two hallucinatory episodes with methods and times at the discretion of the Storyteller.
On a 6 the character will be the victim of hideous and disgusting visions (or the other way around) with a duration of 1d4 hours.

\textbf{Amnesia} \index{Amnesia}

\textbf{10}: You have forgotten your past and with that the memory of friends, enemies, goals. There is no way to recover lost memories.

\textbf{Ascetic} \index{Ascetic}

10, the rule says so. You will not take more than 10 items with you.

\textbf{20}: You cannot have more than 10 items with you, magic or normal or coins or weapons. Luckily, clothes don't matter.

\textbf{Stutterer} \index{Stutterer}

You can speak, but badly.

\textbf{5:} You have an annoying tendency to stutter just when you have something important to say. In these critical situations, only sketchy sounds come out of your lips.

\textbf{Bad Font} \index{Bad Font}

Manners are never an option.

\textbf{5}: You have never learned the art of diplomacy, and you hate being contradicted or insulted. This does not mean that you pass to the de facto ways, but that in the face of an insult or a frank criticism you tend to silence your interlocutor with really unpleasant expressions. You have a -2 on Charisma-based checks

\textbf{Spendaccione} \index{Spendaccione}

\textbf{10}: You must spend half of your mission earnings on futile pleasures (eating expensive food, drinking fine wine and spirits, luxurious clothing, no weapons or magical items)

\textbf{15}: You must spend all your mission earnings on futile pleasures (eat expensive food, drink fine wine and spirits, fancy clothes, no weapons or magic items)

\textbf{Charitable} \index{Charitable}

\textbf{10}: You must donate half of your mission earnings to charity

\textbf{15}: Cannot hold more than 10 gp in cash

\textbf{Blindness} \index{Blindness}

\textbf{10}: You are blind, impaired lateral vision, problems in understanding the distance of things.
Skill such as Awareness and Attack rolls to hit with thrown weapons have a -4. Defense worsens by 2.

\textbf{20}: you are blind. You do not see. all enemies are invisible.

\textbf{Cleptomania} \index{Cleptomania}

\textbf{5}: You feel an irresistible urge to appropriate "interesting" items from time to time. If you haven't stolen at least one item in a day, you won't be able to use Fate Points for that day.

\textbf{Code of Ethics / Vote} \index{Code of Ethics} \index{Vote}

You made a vow, a promise, an oath that conditions your actions.

5-10: Establish the rules well, in black and white, and be clear with the Storyteller

\textbf{Compulsive} \index{Compulsive}

There are certain behaviors, which are necessary for you, which you absolutely cannot do without (e.g. walking avoiding stains on the ground or passing only on them, removing the weapon only in a certain way, etc.)
These behaviors must be declared and made explicit when choosing the disadvantage.

\textbf{5-10}: when you are prey to compulsive behavior you have a -2 on Awareness checks / you are always the last to act regardless of the initiative rolled or marching order.

\textbf{Color blindness} \index{Color blindness}

You are blind to colors, a sunset will be something sad seen in gray

\textbf{5}: you do not have color awareness (achromatopsia). See everything in grayscale.

\textbf{Deformity} \index{Deformity}

Not everyone is born beautiful or straight. There are also those who are born crooked and ugly.

\textbf{5}: Minor malformation, affects either Strength or Dexterity or Constitution. Remove 1 point from this stat.

\textbf{10}: Two characteristics of your choice cannot exceed 2 points except magically. You have half the movement.

\textbf{20}: Severe malformation. Three traits of your choice cannot exceed 1 point except magically. You have half the movement

\textbf{Depression} \index{Depression}

Every day is a bad day and nothing will make it better

\textbf{8}: You adore the Blues but unfortunately you have lost the joy of living, the enthusiasm, the hope.

Nothing seems to matter, you just drag yourself wearily from one day to the next. -2 at each proficiency check

\textbf{Dependence} \index{Dependence}

\textbf{10}: You have an addiction, it can be alcohol, drugs, women ... If you don't consume an adequate dose every day (the Storyteller will be able to tell you enough) you take a -2 on all saving throws. After 3 days of abstinence you also become Depressed

\textbf{Dyslexia} \index{Dyslexia}

jk j0j zo mdbbdfd

\textbf{10}: You are unable to read and write. You are unable to correctly associate sounds with letters and shapes with sounds

\textbf{Compulsive Dishonesty} \index{Compulsive Dishonesty}

You lie, it's stronger than you.

\textbf{5}: The character is led by his own insecurity to lie always and in any case. Whenever the character is forced to admit his responsibilities or in any case to speak against his own interest, or in any situation in which he feels "examined", he will invent rather fanciful stories also endangering friends and relatives.

\textbf{Chronic Pain} \index{Chronic Pain}

oh how bad. Enchanter do you use a cure on me today too?

\textbf{10}: You recover no hit points except magically

\textbf{Hemophilia} \index{Hemophilia}

you tend to bleed all the time, even in the least opportune moments

\textbf{8}: PATCH !!! (each attack you take automatically stacks +1 Bleed)

\textbf{Epilepsy} \index{Epilepsy}

always and only in the least opportune moments

\textbf{15}: Whenever you roll a 3 on a saving throw or an attack roll, you fall to the ground for 1d6 rounds in convulsions, the attack or saving throw is considered to have failed. You are considered helpless.

\textbf{Fetishism} \index{Fetishism}

If you don't smell a woman's foot, you get depressed.

\textbf{5}: The character is irresistibly attracted to an object, body, category ... Any day he is away from his source of pleasure, consider himself to have fallen into Depression.

\textbf{Memories} \index{Memories}

Hey are you there? why are you paralyzed? and these things when did you learn them?

\textbf{5}: On each proficiency check, roll a d4. With 1-2 you do the normal check, with 3 you check with a -2, with 4 you check with a +2

\textbf{Phobias} \index{Phobias}

\textbf{Miscellaneous}: The character is terrified of an object, a category of people or living beings, a situation. In the presence of the triggering cause, the character falls prey to a panic attack: his only desire is to escape as far as possible from the source of his terror, by any means; whoever blocks his path is to be considered an enemy. If the character is unable to escape, he falls into a catatonic state until the root cause is eliminated. See the table of possible phobias at the bottom

\textbf{Photosensitivity} \index{Photosensitivity}

The light, even if light, bothers you.

\textbf{5}: The character has a -1 on any roll where the brightness is at least daylight

\textbf{10}: Character has a -2 on any roll where the brightness is at least that of a lantern

\textbf{20}: The character has a -3 on any shot where the brightness is at least that of a flashlight.

The character is so sensitive that it is impossible for him to move freely in places that are directly or not illuminated, he will prefer to move and travel at night.

\textbf{Dormouse} \index{Dormouse}

you like to sleep and a lot. Ronf

\textbf{5}: +2 for every 2 hours beyond 8, otherwise you are fatigued.

\textbf{Clumsiness} \index{Clumsiness}

\textbf{10}: The Dexterity score cannot exceed 2. You have a -2 on all checks that require Dexterity (deactivate gadgets, empty pockets, climb, initiative ...)

\textbf{Hygienist} \index{Hygienist}

I ran out of soap. I FINISHED THE SOAP! .. I do not touch that sword, even if it shines with sacred light and flies in mid-air until it is disinfected!

\textbf{5}: you have the urge to clean yourself constantly and clean everything you have to touch.

\textbf{Unconsciousness} \index{Unconsciousness}

\textbf{5}: You're not afraid of anything. Literally. If you have to do something, the most direct and immediate plan is the best choice. You can't come up with plans that last more than a minute. Take +1 to Initiative and -1 to Attack Roll.

\textbf{Undecided} \index{Undecided}

Let's not do it, wait for tomorrow .. maybe it's better!

\textbf{10}: You never act first. -1d6 on initiative checks

\textbf{Recurring Nightmares} \index{Recurring Nightmares}

\textbf{8}: The character cannot sleep well. Each night roll a d4. With 1 the character sleeps normally, 2 or 3 the character sleeps a restless sleep and wakes up fatigued, with 4 you wake up in the middle of the night screaming, in the morning you are exhausted.

\textbf{Open Book} \index{Open Book}

yes, I know, I can shut up, you have already understood everything.

\textbf{5}: it's not that you can't lie it's that you have -1d6 on Deceive checks

\textbf{Migraine} \index{Migraine}

It is never a good day. You suffer from constant and fierce headaches.

\textbf{15}: The character suffers from violent headaches. Each day the character rolls a d4: on a 1 the character has no effect, on a 2 or 3 he suffers a -1 penalty on all checks, on a 4 the penalty becomes -2.

\textbf{Cursed} \index{Cursed}

You are cursed. A dark fate has stained your soul

\textbf{5-10}: You carry a curse. Discuss with the Storyteller

\textbf{Myopia} \index{Myopia}

He hopes to find some glasses

\textbf{5}: You see little. You have a -2 on all proficiency checks with ranged weapons and Awareness checks beyond 12 meters.

\textbf{15}: You see very little there. You have a -1d6 proficiency check with ranged weapons and Awareness checks beyond 30 feet.

\textbf{Mute} \index{Mute}

You can't talk and worst of all you can't even shame the guy who's stepping on your foot

\textbf{10}: You are unable to make sounds. You do not speak or rather no one hears you. Get a -1d6 on Charisma and speaking skills checks

\textbf{Dyscalculia} \index{Dyscalculia}

1 + 1 =?

\textbf{10}: the character has a disorder that prevents him from mastering the concept of numbering. Not only is he unable to perform the simplest operations, he is not even able to understand major / minor concepts, or quantitative information of any kind.
Beware of the rest they give you ...

\textbf{Obesity} \index{Obesity}

You are definitely out of shape, and from time to time.

\textbf{10}: Dexterity can't be above 2. You have a -1d6 on Dexterity checks and reflex saving throws. You gain +2 on Fortitude saving throws

\textbf{Smell / Defective Taste} \index{Smell / Defective Taste}

Nose, palate, burnt tongue, abuse of chili or wasabi ... there can be many causes

\textbf{5}: -2 two checks using taste or smell. You do not feel flavors and smells if not extreme.

\textbf{Compulsive Honesty} \index{Compulsive Honesty}

\textbf{7}: You can't lie, the very idea of telling a lie makes you nervous. Take a -1d6 to Deceive. If cornered, the character will confess everything regardless of the importance of the information in his possession.

\textbf{Crystal Bones} \index{Crystal Bones}

It would be called osteogenesis imperfecta but for you it's just continuous pain.

\textbf{5}: Character has brittle bones. Any damage from a slash weapon causes 2 more hit points of damage

\textbf{10}: Character has brittle bones. Any damage from a slash weapon causes 5 more hit points of damage

\textbf{Monco} \index{Monco}

You are maimed, the choice is yours.

\textbf{7}: You are missing the secondary hand

\textbf{13}: You are missing the primary hand. -2 to all throws involving the use of the hand.

\textbf{Paranoid} \index{Paranoid}

You are paranoid and boring.

\textbf{5}: You always behave stealthily, even if there is no real need for it, thus arousing suspicion in the people around you.

Each opposing Awareness check has an additional -5 difficulty, and a critical failure indicates that the target has something vital to hide.

\textbf{Sensitive Skin} \index{Sensitive Skin}

You don't love the sun, or at least your skin doesn't love it.

\textbf{5}: Your character burns easily, prolonged exposure without adequate protection leads to painful and unsightly burns and discomfort.

\textbf{10}: You are extremely sensitive to ultraviolet rays. Each fire or light damage causes 2 additional damage.

\textbf{Lazy} \index{Lazy}

you are slow and listless

\textbf{5}: -2 to the initiative

\textbf{Loud} \index{Loud}

You don't do it on purpose, but there is always some noise around you. A smacking sword, a yawn, a burp, a noisy shoe ...

\textbf{5}: You have a -2 on move silently checks

\textbf{Weak Blood} \index{Weak Blood}

\textbf{10}: The character's immune system is definitely worth it. -2 to Fortitude saving throws

\textbf{Carelessness} \index{Carelessness}

Ops .. I didn't notice!

\textbf{10}: You tend not to pay attention to what's going on around you, unless you have very good reasons to be alert, or are actively looking for something take a -1d6 to Awareness

\textbf{Schizophrenia} \index{Schizophrenia}

It wasn't me, but the other one!

\textbf{4}: You have more personality, or maybe the other is convinced.

The character has at least one second personality (max 6).
Each additional Personality to manage, beyond the first, grants +1 to the cost.
So having 3 personalities brings the disadvantage to 6 points

1d6 is rolled each day. With 6, during the day the second (or third) personality comes to light.

\textbf{Unlucky} \index{Unlucky}

things don't just happen, you have to know how to look for them too

\textbf{5}: You ignore the first critic you make (TC or TS) in the day

\textbf{7}: You ignore the first three critics you make (TC or TS) in the day

\textbf{Manic Depressive Syndrome} \index{Manic Depressive Syndrome} \index{Depression}

Today is Friday !!! It's Friday!!!

\textbf{7}: The character alternates between states of euphoria and moments of gloomy despair. 1d4 is rolled each day. On a roll of 1 the character has a "normal" mood. With 2 or 3 he is considered to be in Depression, with 4 he is in a state of joyful elation (see Unconsciousness) and bravado.

\textbf{Awe} \index{Awe}

I apologize

\textbf{10}: The character is very insecure and tends to blindly trust others, especially if they are charismatic. Take a -2 on Intimidate and Entertain checks
Take a -2 on charm and domination saving throws

\textbf{Light Sleep} \index{Light Sleep} \index{fatigued}

Every noise disturbs you, you can never sleep well

\textbf{5}: If you sleep in an area with natural / human noises (forest / city) you cannot rest well. You are fatigued in the morning. You can avoid this by using earplugs, which force a -1d6 on your Hearing Awareness checks to wake you up.

\textbf{Deafness} \index{Deafness}

Silence has a sound of its own, says whoever hears us, for you it's just a heartbreaking, mute scream.

\textbf{10}: You can't hear us. You cannot do Awareness checks that require the use of hearing. You can't listen to people talking. But you can lip read if you know how to do it.

\textbf{Vertigo} \index{Vertigo}

The discomforts manifest themselves when the character is aware of the height. Only for walking in an elevated position has no penalty

\textbf{5}: At heights over 20 meters you tend to get stuck. Take a -2 on all checks

\textbf{7}: At heights over 10 meters you tend to get stuck. Take a -3 on all checks

\textbf{10}: At heights over 6 meters you tend to get stuck. Take a -1d6 on all checks

\textbf{Reduced night vision} \index{Reduced night vision}

Your eyes don't work well in low brightness.

\textbf{5}: When the brightness is equal to or less than that of a torch (or dim) the character has a -2 to attack rolls.

\textbf{Shyness} \index{Shyness}

\textbf{5}: You are shy and reserved.

You have a -2 on Charisma-based checks

\textbf{Lame} \index{Lame}

you are limping

\textbf{5}: your movement is reduced by 2 meters (9 to 7, 6 to 4)

\textbf{7}: your movement is halved (9 to 4, 6 to 3)

\textbf{10}: you are significantly crippled. -2 on Dexterity checks, your movement is halved

\bigskip

\end{multicols}

\textbf{Table of Phobias (5-15 points)} \index{Phobias} \index{Table of Phobias}

\begin{tabular}{ll}
\textbf{Name Phobia} & \textbf{Description} \\
\toprule
Blennophobia & Fear Of Slimy Things \\
Keraunophobia & Fear Of Thunder \\
Hypochondria & Fear Of Disease \\
Claustrophobia & Fear Of Closed Places \\
Coimetrophobia & Cemetery Fear \\
Hedonophobia & Fear Of Being Physically Pleased \\
Eisoptrophobia & Fear of Mirrors \\
Glossophobia & Fear Of Public Speaking \\
Monophobia & Fear Of Being Alone \\
Necrophobia & Fear Of Dead Bodies \\
Nictophobia & Fear Of Darkness \\
Acrophobia & Fear of Heights \\
Agoraphobia & Fear of Open Spaces \\
Rupophobia & Fear Of Dirty And Unhygienic. Feel the need to clean \\
Afephobia & Fear Of Touching And Being Touched \\
Asymmetrophobia & Fear Of Unsymmetrical Things \\
Gymnophobia & Fear Of Nudity \\
Hemophobic & Blood Fear \\
Traumatophobia & Fear Of Injury \\
Sciophobia & Fear of Shadows \\
\end{tabular}


\pagebreak

\section{Optional - Iconic skills} \index{Optional - Iconic skills} \hypertarget{abilitaiconiche}{} \label{abilitaiconiche}


\begin{changemargin}{0.3cm}{0.3cm} \begin{enfasi}{
Life has gotten immeasurably better since I've been forced to stop taking it seriously. (Daniel Day Lewis)
} \end{enfasi} \end{changemargin} \medskip


\begin{multicols}{2}

%{\small

These skills represent the pinnacle of a character, not intended as the capstone skills of the 20th level, but as skills related to the way of playing, to the type of character that has been created and grown. These skills should only be given to characters who have been brought from the first to at least the 12th level, it is a recognition to the player.

They are optional skills because they are strong, peculiar, unique and "broken". The Storyteller should give them at the end of a long and unique campaign when the characters are now legend. Each character can 'have only one iconic ability, one ability' that distinguishes heroes, capable of actions at the limit and beyond the human. Players are encouraged to create new Iconic Abilities based on character development. \\

{\Large{\textbf{Light against Darkness}}} \index{A light against darkness}

\textbf{Suggested requirements: Patron Ljust, Sumkjr}

Once a day, emit sacred light around you for 60 minutes which has the effect of the Protection from Good and Evil spell against Devotees and Followers not of your Patron. You can channel light once per day, and all Followers or Devotees of other Patrons creatures within 10 meters of you must make a Fortitude save DC 10 + sum Traits in common with the Patron + Wisdom or be stunned for 2d6 rounds. \\

{\Large{\textbf{The Blacksmith}}} \index{The Blacksmith}

\textbf{Requirements: ability to work metal}

Your skills in working with weapons and armor are legendary.
Any armor you make cluttered and weighs like a lower tier, weapons do one tier higher die damage. \\

{\Large{\textbf{The Oracle of War}}} \index{The Oracle of War}

\textbf{Requirements: Master melee combatant}

Any weapon in your hands is lethal. The weapon's die doubles as the damage caused by the Force doubles. Eg a longsword does 2d8 damage and if you have Strength +3 the total damage becomes 2d8 + 6 \\

{\Large{\textbf{The fearless hero}}} \index{The fearless hero}

\textbf{Requirements: courageous and resolute}

Once per fight you can ignore (for 2d4 rounds) the conditions of Blinded,
Frightened, Fainted, Shocked, Nauseated, Confused, Fatigued and Exhausted as a Reaction. \\

{\Large{\textbf{Mindmaster}}} \index{Mindmaster}

\textbf{Requirements: an adventurous life managed with intelligence and cool blood}

It replaces the bonus given by Intelligence or Wisdom with the bonus of physical characteristics in the checks. \\

{\Large{\textbf{On a pale horse}}} \index{On a pale horse}

\textbf{Requirements: do not fear death, do not be reckless}

You are the closest thing to death your enemies will ever see.
When you kill an enemy all opponents (who may have seen the scene) within 10m of range must make a Will save DC 10 + Weapon Proficiency + Charisma or be affected as by the Fear spell. The skill is usable 3 times per day. \\

{\Large{\textbf{The Magic Fury}}} \index{The Magic Fury}

\textbf{Requirements: A life dedicated to explosive magic}

You are capable of unleashing hell with magic. The difficulty (DC) of each of your spells increases by 4, when you make a Magic Check you roll 1d6 more. \\

{\Large{\textbf{The Shadow}}} \index{The Shadow}

\textbf{Requirements: A life dedicated to hiding and surprising enemies}

Three times a day you can swap your position with that of an opponent within 30 meters as long as you are both in a gray area. Will save DC equal to Hide in Shadows check. \\

{\Large{\textbf{The Mother}}} \index{The Mother}

\textbf{Requirements: spent more time in animal form than own}

It has the innate ability to leave the footprints of any animal, compatible with your size, even if you are not transformed. You can talk to any animal as if you are always under the effect of the Speak to Animals spell. \\

{\Large{\textbf{The Dead}}} \index{The Dead}

\textbf{Requirements: an everlasting life on the brink of death}

Three times a day when your hit points drop below 1, you recover 3d12 hit points with a Reaction Action. This ability can also be used when HP is negative or you should be directly dead. \\

{\Large{\textbf{The Hunter}}} \index{The Hunter}

\textbf{Requirements: a life dedicated to hunting and chasing}

Your Tracking Survival checks have a +10 bonus. The first hit that hits an opponent automatically gets 2 critical.
%}

\end{multicols}



\pagebreak

\section{Cosmology} \index{Cosmology} \hypertarget{cosmologia}{} \label{cosmologia}

\begin{changemargin}{0.3cm}{0.3cm} \begin{enfasi}{
It is easier to dominate those who believe in nothing (The Neverending Story, Kmorf)

Do you believe that there is only one God? You are right; even the demons believe it and they tremble! (Giacomo Il Giusto 2, 19. NdA Referring to his own Patron...)} \end{enfasi} \end{changemargin} \medskip

\begin{changemargin}{0.3cm}{0.3cm} \begin{Storytellere}
In OBSS, gods are different from traditional RPG gods.

The deities, the Patrons, love to get their hands dirty, participate in the affairs of the creatures who adore them, for them it is a continuous challenge to have more believers, followers and people more similar, for traits, to them.

The Patrons were created as \href{https://www.treccani.it/vocabolario/parossismo/}{parossismo} of the human soul, where everything is an excess. As spirits released from Pandora's box they have the sole purpose of bringing their Traits to dominion by making them the most common and present among creatures, especially the most powerful.
\end{Storytellere} \end{changemargin}

\begin{multicols}{2}

\bigskip

\lettrine[lines = 2, lhang = 0.33, loversize = 0.25, findent = 1.5em]{I}{n} the beginning was the nothing which in itself contained the whole.

The Energy deriving from the most primordial impulses exploded in all its power and without any control.

Love, hate, fear, pain, joy, serenity ... everything was entangled in a dense and infinite skein whose core was forming.

In interminable times to follow these energies, emotions and impulses have begun to create three entities: Atmos, the one who is in charge of controlling the progress of time and space, coluli who assists and the scribe; Ljust the positive energy, heat, light, life and syntropy; Calicantus, negative energy, icy hatred, destruction, death and entropy.

While Atmos does not have a definable form Ljust and Calicante have manifested as two flame tongues of a single progenitor energy.

\textbf{Ljust} \index{Ljust} is the representation of what light and life always carry with them. It represents the purity of the feeling of love, the protection of life, respect for the other, the curiosity for the new, the desire to always improve, the strength to fight with courage and value for the common good. It is the vital drive for change, the chaos that transforms but does not destroy.

\textbf{Calicante} \index{Calicante} is the representation of darkness, hatred, anger and violence. Calicantus is revenge and cold destruction, there is no interest in any form of life rather it uses them, exploits them and only in such cases does it suffer their presence. Sadistically loves suffering. It is the entropy that annihilates and annihilates and finds pleasure in doing so.

\textbf{Atmos} \index{Atmos} is the witness, the one who marks the passage of time and transcribes every event of Yeru and among the Patrons of Genesis. An entity born of creation to prevent absolute destruction. He watches and transcribes what the Patrons of Genesis do, the gods to whom the task of generating creation has been assigned.

Together the two Patrons of Genesis gave life to everything we know. Calicante created Tiya \index{Tiya} and Ljust created Curyan \index{Curyan}, the two kingdoms that make up our world, Yeru.

They played with forms and energies creating two realms that mirror each other but opposed and distinct. Tiya and Curyan, like Calicante and Ljust, are part of a whole but, just like the Patrons of Genesis, they are also profoundly different and magically divided. In fact, there is a physical barrier formed by deadly perennial sea storms and even magical ones, which delimit the boundaries and keep them clearly divided.

But just as their two creators who totally separated and distant cannot be, cannot exist, so Tiya and Curyan are indeed divided but also in contact with each other through the Portals. Portals that are generated autonomously, without any control and prediction, due to the magical energy that presses, pushes and feeds itself in the "non-place" on the border of the two kingdoms and which is generated by the continuous emotions generated by the Patrons of Genesis.

It is these magical ways that allow you to move between Tiya and Curyan and travel to the "non-place" that is what is outside of Yeru.

Ljust and Calicante decided, strangely by mutual agreement, to generate a Patron who oversaw these fractures, who was able to perceive, open and block these Portals. Thus was created \textbf{Lynx}, the Guardian of the Portals.

Many try to pass from Tiya to Curyan to seek peace, serenity. Others try to cross the reverse boundary in search of adventure and power, some try the normal way, others through the Portals, many are lost forever in the "no place".

Lynx \index{Lynx} oversees the cosmic void, the portals which, with the alternation of chaos and order, of good and evil, of light and darkness, are increasingly creating fractures on the border between the two kingdoms. Lynx perceives them, "feels" them, knows where they are generating or extinguishing, with the passage of time some of these Portals have become stable and definitive, while others continue to generate themselves randomly and always in a totally unknown way remain active or run out . Traveling continuously in the non-place, Lynx closes the largest portals but for one that closes another it opens. Lynx stripped the Lists of Magic of many of the spells that affect the planes, to protect Yeru from outside creatures.

In carrying out this important role, Lynx collided with a strange creature, reptiloid, gigantic, winged, powerful, strong, wise and magical. A red Dragon, \index{Tahil} named Tahil.
The latter moved in the "non-place" with the utmost freedom, without any difficulty and approached Lynx. The Atmos diaries tell of how Lynx tried to stop him and talk to us, of how he was fiercely attacked, of the screams of the Patron Guardian that were heard echoing in both realms, of the sound almost like a guttural roar that tore the silence in the realms of Tiya and Curyan. Of the speech by Ljust and Calicante. The first to save Lynx and the second to discover, to know this fascinating new "weapon".

Lynx was saved. Ljust infused him with her healing spells and helped him regenerate. However, he left himself scarred in memory of the meeting.

Tahil guided many other Dragons on Yeru and these moved by their thirst for knowledge and power have also spread through the Portals present on both Tiya and Curyan.

Hordes of dragons of all colors have darkened the skies for decades. No nation was saved. Looting, raiding and violence were perpetrated equally in the two kingdoms. They were highly intelligent, crafty and violent, powerful beyond imaginable, and extremely evil. They had a robustness out of the ordinary. But above all, they did not fear the Patrons. They did not submit to them.

Atmos, concerned for the balance of creation, channeled the primordial and divine energies of the Patrons of Genesis by creating deities who could rival the dragons and could defend Yeru.

The first created by Atmos, with the help of Ljust, and the intervention of Calicant, was \textbf{Gradh} \index{Gradh}, Patron of Humanity (and of all sentient races), the one who would defended creation from dragons and other Patrons. Gradh embodies the dualism of the two Patrons of Genesis, Ljust's innate instinct for protection, defense and care and Calicante's instinct for revenge, violence and fury.

He boldly throws himself into battles, attacks the enemy without fear, protects the weakest, defends life but is not afraid to take the path of the most destructive revenge towards those who exploit and destroy lives for no reason. Gradh likes to "drop in" among people and live with them, like them. He does not feel totally at ease in the Pantheon with the other Patrons or among the common people, he is Human among the Patrons and Patron among the Humans. Passionate and kind is the Patron who most cares about the fate of Yeru and his races.

The tongues of divine energies were too intense, chaotic and pure for Atmos to govern them to shape the further Patrons alone. Using the raw power of the Patrons of Genesis he created other Patrons, each influenced differently by Calicanthus or Ljust. These Patrons were less perfect and divine than his intentions, more imperfect and "human" as they originated from the emotions, uncontrollable and pure of the Patrons of Genesis. These new Patrons shape wills, found kingdoms, command in the shadows like pawns the creatures who dare to ask for their favors.

Gradh sensed early on that the Dragons represented an element of further chaos, of further suffering and war. As Patron of Yeru and his creatures he felt the Dragons as alien, non-native creatures, not part of the plan of Genesis.

Wary by nature Gradh decided to propose to the Patrons of Genesis to make a pact with the Dragons.

Here is that just over 300 years ago, on 15 Prineva of 65 of the sixth cycle, on the unreachable island of Alantia that divides Tiya and Curyan, the flame of Ljust and Calicante, Lynx and Gradh departs while Tàhil, the dragon evil and immortal red and Dyenos \index{Dyenos}, the wise and good silver dragon on the other. Atmos everywhere kept track of events.

Gradh sought to enforce the ousting of the Dragons and Lynx the permanent closure of the Gates. Ljust tried to mediate by understanding that not all Dragons were evil and that they could have knowledge and new evolutionary drive to Yeru.

Calicantus pretended to agree with Ljust with the sole purpose of bringing more chaos and destruction through the Dragons.

Realizing that the outcome of the meeting was already decided Gradh and Lynx left the Plain of Solitude, leaving the Dragons and the Patrons of Genesis to formalize the partition of Yeru. It had been a resounding defeat for both of them, Gradh has since been even more wary, if not prejudiced, of all dragons.

Tahil became the First General of Calicante and the Dark Hatred created for him a secret and unattainable kingdom, a land for the Dragons. Dyenos swore allegiance and trust in Ljust and together they promised to rule Curyan to the best of their ability.

Lynx, now an external spectator, did not remain without doing anything, fearing the worst he created a Portal that would lead to a new Yeru, a different planet and out of the influence of Dragons and Patrons. His research takes him to an almost idyllic world rich in nature, animal life and without Patrons, as he liked. He named it Ker \index{Ker} in memory of an ancient beloved. There is very little information about this world, only a few high Lynx devotees know it and even fewer have visited it.

The fact remains that the portal of conjunction exists and is stable, where exactly it is is not known but its gift to Yeru has already done so, the Gnomes \index{Gnomes}.

In this apparent calm the Patrons perpetuate their interests, become the strongest the most important, those who have the most followers. The goal is one to have as many people as possible following their Traits.

If a Patron acts in the first person or indiscriminately, he knows that he will trigger the reaction of Gradh or the intervention of Atmos that will prevent him from an uncontrolled and massive use of his powers directly on the world. This rarely stops them and nature itself, creatures and plants, are often influenced by the will of the Patrons.

In Tiya, but sometimes also in Curyan, aberrations are born more and more often, ever new diseases, cursed lands where nothing can grow, not to mention madness that often involves those who should instead protect the citizens. It is a hard life for the common man who continually has to face drought or floods, animal deaths and an irregular if not absurd weather, hordes of creatures from nowhere who just want to exterminate everyone. At every step he must look around because he can never know who has sold his soul to a Patron to live one more day.

In Curyan we see the development of harmony and almost perfect coexistence between nature and superior races. There is pain, there is disease and death but all as a natural cycle of life as part of it that is protected, guided, helped. A rich and generous land and for this reason more and more victim of the machinations of the Patrons if they follow Calicante.

Everywhere the strongest enemies are the Dragons who make raids to bring destruction and death and sow fear and horror.

It's not always all idyllic, vast regions of Curyan are becoming incubators of dark and evil races, legions of the undead led by mighty necromancers crowd the borders, Dragons train their corrupt adepts, and dark black coils in the sky promise storm.

\subsection{Patrons} \index{Patrons} \hypertarget{Patroni}{} \label{Patroni}

\begin{changemargin}{0.3cm}{0.3cm} \begin{enfasi}{
Conan: Which gods do you pray to?

Subotai: I pray to the four winds and you?

Conan: I pray to Crom, but only rarely ... he doesn't listen. (Conan the Barbarian, 1982 film)

\medskip

Indeed, just as the body without the spirit is dead, so too is faith without the works she is dead. (Giacomo Il Giusto 2, 26. NdA Referring to scores of the Traits connected to the Patron ...)} \end{enfasi} \end{changemargin} \medskip

All creatures, even those who do not use magic, can feel the influence of these Powers, of these Patrons.

If a character by his way of being (playing) and behaving has at least one Trait in common with a Patron and indeed matures and strengthens these convictions, even if he has not sworn loyalty to a Patron he could still feel the influence of the Patron and receive some gifts from him.

A Patron is very happy if someone follows his dictates, traits, and gives those who do it small powers as recognition for the loyalty reserved to him, deliberately or not. The powers indicated under "Common Tracts" are cumulative. Unless otherwise indicated, the powers are usable 1 time per day and cost 2 Shares.
When a spell is indicated it is manifested without Magic Checks or Armor penalties.

Each \textbf{Patron has a preference for one or more energy forms}, if you are a Follower you can use that energy in your spells, if you are a Devotee you must, in the same way the preferred Magic Lists are indicated.

The forms of Energy are distinguished between positive, neutral and negative sources, they also serve you to better frame your Master, pardon the Patron you serve.

Add up the elements, if positive the Patron can be considered good, if at zero value the Patron is neutral, if at negative value the Patron is evil.

In the description of the Patron you will also find his manifestation, or what happens when a character acts in a particularly and significantly consonant manner with the Traits followed by the Patron. The effect is purely scenic and circumstantial but always impresses anyone who can observe it, and usually guarantees an advancement point in some section connected to the Patron saint.

There is also an indication of the Patron's favorite weapon. There are no advantages in using it, the choice is purely personal and left to the devotion of the character.

A spellcaster who relies on a Patron, at least 3 Traits in common, becomes a Devotee. If he has at least 2 Traits in common and relies on a Patron then he is said to be a Follower. The \textbf{Advantage} shown is for the Devout only.

It may also not follow any Patron even though they have more Traits in common.

\begin{changemargin}{0.3cm}{0.3cm} \begin{Storytellere}
The Storyteller can still grant being a Follower or Devotee even if the Traits do not match perfectly. At the request of the player and at his discretion he can evaluate the similarity of some Character Traits to those of the Patron and evaluate them suitable for being a Follower or Devotee. In these situations it is necessary to understand how the player frames the character and understand not only if the traits but also the feeling of the character is akin to the chosen Patron.
\end{Storytellere} \end{changemargin}

The skills acquired related to the traits in common are independent of being a devotee, follower or simply "atheist". \index{Advantages}

\bigskip

\textbf{Table of Elements - Energy} \index{Table of Elements}

\medskip

\begin{tabular}{lll}
\textbf{Positive} (+1) & \textbf{Neutrals} (0) & \textbf{Negatives} (-1) \\
\toprule
En. Positive & Fire & En. Negative \\
Light & Cold & Emptiness \\
& Sound & \\
& Electricity & \\
\end{tabular}

\begin{changemargin}{0.3cm}{0.3cm} \begin{tcolorbox}[title = Devotees and Followers]

Being a Devotee or a Follower is your choice, no one imposes it on you. You have to feel it as an opportunity for role play, as an enrichment of the character and not a constraint. Being devotees or followers does not mean being prone to the will of the Patron, on the contrary, it means being even more convinced of one's own traits, of one's personality. A Patron does not ask for prayers, but asks to be oneself.

\end{tcolorbox} \end{changemargin}

\subsubsection{Ljust} \label{ljust}

\index{Ljust}

The Lady of Light, the one who radiates warmth and love. Generator of the impulses of love, protection, kindness, joy and forgiveness. It embodies the protective aspect of a mother, the strength and audacity of a fighter, the passion of a young lover and the joy, the search for the new, the imagination of a child. Ljust embodies the beauty of life and every creature that contemplates it sees what for it is the greatest harmony and falls prone to its charm.

Ljust can only be chosen by a character with 4 Traits in common with her, basically one was born to be a Ljust Devotee. Over the ages Ljust decided to select, choose and reward the women who most showed innate and deep love for life, curiosity for the new, unshakable strength, dedication, trust, respect and care for others by giving them the powers and opportunity to study and grow as a Student of the Light. These Pupils must follow the 8 Steps rule. \\

- \textbf{Symbol:} A star surrounded by solar rays \\
- \textbf{Characteristic} (Devout): Wisdom or Charisma \\
- \textbf{Traits}: Courageous, Generous, Empathic, Protective, Instinctive, Nonconformist, Sensitive, Outgoing, Correct, Compassionate, Selfless. The Devotee of Ljust has 4 Traits in common with the Patron. \\
- \textbf{Manifestation}: golden light floods the caster. \\
- \textbf{Sum of traits in common at 5 points}: You can cast the spell Light as Reaction, 3 times per day \\
- \textbf{\textbf{Sum of traits in common at 10 points}}: You gain +2 on Fortitude saving throws \\
- \textbf{Sum of traits in common at 15 points}: an armor of light protects you, you gain +2 on all saving throws and defense, the effect is permanent. \\
- \textbf{Sum of traits in common at 20 points}: You can cast the spell Solar Flare. 1 time a day. \\
- \textbf{Elements} (Follower / Devotee): Positive Energy, Light \\
- \textbf{Advantage} (Devout): Effective treatment \\
- \textbf{Privileged Magic Lists} (Follower / Devotee): Heal, Abjuration \\
- \textbf{Favorite Weapon}: Bastard Sword

\medskip

\textbf{The 8 Steps of the Pupils} \index{8 Steps of the Pupils} \index{Pupils}

The Students of the Light are a secret group of Devotees who, out of total affinity with Ljust, have embarked on the hard path of good and love. It is among the oldest groups founded in Yeru. The Pupils, 99 as a maximum number, but unfortunately often fewer in number, are Devotees of Ljust and must follow the 8 Steps of Light \\

1. Love and protect with all of yourself, with total and sincere dedication those around you. \\
2. Do not let your inaction cause suffering. \\
3. Yes, a point of comparison. Let your Light elevate the people around you and they can see in You are hope, serenity, calm, protection and security. \\
4. Use intelligence, cunning and wit. She is far-sighted and resolute in action. \\
5. Your work is for the common good. Make your Light always high and intense. \\
6. Seek no other Light than yours and that of your sisters. \\
7. Be bright but do not blind those around you. \\
8. Be the difference between despair and hope. \\

\medskip

The Pupils built a harmonious dance by transforming the steps of their Rule into dance.

There are also students of uncertain gender, rare but historically proven.

\subsubsection{Calicant} \index{Calicant} \label{calicante}

\begin{changemargin}{0.3cm}{0.3cm} \begin{enfasi}{
Superstition is the religion of weak spirits. (Edmund Burke)
} \end{enfasi} \end{changemargin}

It is dark, cold and angry. It embodies hatred, violence, destruction, revenge and perennial dissatisfaction. It collects the capricious and discontented personality of a child, the violent and sadistic boredom of a young man, the destructive force of a hurricane and the anger of a fighter who has nothing more to lose. Calicant only with presence makes you uncomfortable, makes you feel in danger, fascinates but with the weapons of fear and inconstancy.

Calicant can only be chosen by characters who have 4 Traits in common with him. His Devotees are the best assassins, his closest profession. Those who show the greatest contempt for the danger and life of others. His favorites are those who are feared, hated, those who are violent and cruel but deadly efficient and decisive in any combat situation. \\

- \textbf{Symbol}: A black whirlwind \\
- \textbf{Characteristic}: Strength or Dexterity \\
- \textbf{Traits}: Selfish, Vengeful, Superb, Wrathful, Passionate, Cynical, 				Competitive, Creative, Dishonorable, Anarchic, Brutal. The Devotee of Calicante has 4 Traits in common with the Patron. \\
- \textbf{Manifestation}: sword dripping with black blood \\
- \textbf{Sum of traits in common at 5 points} points: You can cast the spell Darkness. Once a day \\
- \textbf{Sum of traits in common at 10 points}: Your weapon cloaks itself in shadow. You gain a +2 attack roll and + 1d4 Void damage for 2d6 rounds, once per day. \\
- \textbf{Sum of traits in common at 15 points}: Create 4 Void darts, each dart does 2d6 damage, automatically hits within 60 feet. Once a day. \\
- \textbf{Sum of traits in common at 20 points}: You create a protective energy zone around you within 10 feet, halve all the damage you take, it is not possible to recover hit points in the area. Duration 10 consecutive minutes, once a day. \\
- \textbf{Elements}: Negative Energy, Vacuum \\
- \textbf{Benefits}: Mind Shield \\
- \textbf{Privileged Magic Lists}: Fire, Necromancy \\
- \textbf{Favorite Weapon}: Machete

\subsubsection{Atmos} \index{Atmos} \label{atmos}

The keeper of Time and the Clock Tower, as he started time and the creation of the new Patrons so will stop the challenge between them, the surviving Patrons will be judged, their works evaluated and Ljust or Calicante will benefit from it. As a challenge from a single copper coin new Patrons, new ideals will be created and we, little creatures, will see new civilizations and flourishing kingdoms born. The story is little known, only the few Devotees of Atmos, scribes and scholars of the library of Time, know the secret and the passage of time and the race, the others, ignorant, will live their time with a master surely guided by a Patron .

Atmos, the Patron of Time is the keeper of history and time, he is the one who keeps track of the thousand and more worlds that have been created.

It has the task of starting and stopping time. Atmos has the unique power reserved only for him to be able to banish a Patron from creation should this become too strong and threaten Calicante and Ljust. Atmos has used this power before. Atmos both for its totally neutral nature and for its role has never taken sides.

All Patrons fear Atmos for its power, the most terrible for them, that is their alienation, oblivion, forgetfulness, being distracted by time and challenge.

To be a Devotee of Atmos at the time of the rite it is necessary that the future Devotee possess at least four Traits in common with him, love history and knowledge.

Dressed in a soft brown habit and leather shoes, he moves among the endless shelves of the Library of Knowledge with always a strange time meter hanging from his waist. \\

- \textbf{Symbol:} A white paper with a pocket watch resting on it \\
- \textbf{Feature}: Intelligence or Wisdom \\
- \textbf{Traits}: \textbf{Atmos}: Observer, Detached, Prudent, Reflective, Integral, Anxious, Paranoid, Complaining, Wary, Foresighted, Apathetic. The Devotee of Calicante has 4 Traits in common with the Patron. \\
- \textbf{Manifestation}: the spell develops as in slow motion, it's just an illusory effect \\
- \textbf{Sum of the sections in common to 5 points}: Always know the exact date and time. \\
- \textbf{Sum of traits in common at 10 points}: You have an innate intuition for knowledge. You have + 1d6 on Knowledge checks \\
- \textbf{Sum of traits in common at 15 points}: You can cast the Globe of Invulnerability spell, 1 time per day. \\
- \textbf{Sum of traits in common at 20 points}: Whenever you have to make an Arcana check you can take 18 as you take 10 \\
- \textbf{Elements}: Sound, Cool \\
- \textbf{Advantage}: Sense of time \\
- \textbf{Privileged Magic Lists}: Divination, Abjuration \\
- \textbf{Favorite Weapon}: Light Mace

\subsubsection{Lynx} \index{Lynx} \label{lynx}

Patron of the Portals, it can only be chosen by characters who have at least 3 Traits in common. He is the first Patron generated by Ljust and Calicante, created to protect Yeru from external attacks.

Serious, cold eyes of a very clear blue is the Keeper of the Portals and of what is Beyond. Lethal guardian for those who try to pass them without permission, careful guide for those who ask for his help and his permission. He shields himself from his scars to keep everyone away. He is the lone controller of the world.

His Devotees are the travelers par excellence, those who guard and protect Yeru from what is alien, from what could disturb creation. \\

- \textbf{Symbol}: A portal to darkness \\
- \textbf{Characteristic}: Dexterity or Intelligence \\
- \textbf{Traits}: Solitary, Serious, Rigid, Controlled, Courageous, Insensitive, Stubborn, Determined, Intolerant, Introverted, Rational \\
- \textbf{Manifestation}: as if the landscape no longer had a horizon \\
- \textbf{Sum of traits in common at 5 points} points: Once per day you can perform one more Move Action \\
- \textbf{Sum of traits in common at 10 points}: Acquire one more move action per round \\
- \textbf{Sum of traits in common at 15 points}: You can cast the spell Exile, 1 time per day, DC 30. \\
- \textbf{Sum of Traits in common at 20 points}: You can teleport 500km per day (even multiple teleports or teleports as long as the total sum does not exceed 500km) \\
- \textbf{Elements}: Fire, Electricity \\
- \textbf{Advantage}: Slow and Still \\
- \textbf{Privileged Magic Lists}: Summon, Water \\
- \textbf{Favorite Weapon}: Short Sword

\subsubsection{Gradh} \index{Gradh} \label{gradh}

%\begin{changemargin}{0.3cm}{0.3cm}\begin{Storytellere}
%Agiamo nell'ombra per servire la luce. (Assassin's Creed II)
%\end{Storytellere}\end{changemargin}

The first Patron created by Atmos under the leadership of Ljust and the influence of Calicante.

Gradh embodies Ljust's innate instinct for protection, defense and care. Gradh is the most similar and deeply connected to Ljust that has been generated. He is balance, rationality and empathy.

Where there is defense, care and protection there is Gradh.

Gradh does not like to openly challenge Cattalm because he knows that he would do exactly his game, so with cunning he tries to lure him into his playground, where no life will be in danger and there he shows off his strategic and combat superiority.

But Calicante could not allow the creation of a Patron totally devoted to Ljust and thus infused in Gradh the coldness of revenge and the fury of anger. So then Gradh in the act of defending humanity, often must first of all protect it from himself.

Passionate and cold, he is perhaps the most human Patron of the current Pantheon. His warm and charismatic gaze which when he loves and protects is a reassuring chocolate color, can become cold and sharp with the shades of the cold frozen earth when he is prey to the fury of battle or revenge. Gradh likes to study the world around him and go unnoticed. Often he hides among the people and "lives" his human life. But he doesn't really let anyone approach him.
Gradh attracts to himself as easily as he pulls away from himself.

The Devotee of Gradh is proud and proud, indomitable and protective, and grieving, for however much he tries to punish evil it always continues to thrive. \\

- \textbf{Symbol}: A shield with two intertwined spirals engraved on it. \\
- \textbf{Feature}: Strength \\
- \textbf{Traits}: Indomitable, Protective, Vengeful, Courageous, Cold, Distrustful, Impetuous, Presumptuous, Gloomy, Reserved, Melancholic, Competitive \\
- \textbf{Manifestation}: two coils, one black as a shadow and one bright as a spark, encircle your weapon and intertwine \\
- \textbf{Sum of 5-point common traits} points: You can cast the cure serious wounds spell, but it deals 1d6 damage to you. 1 time per day \\
- \textbf{Sum of traits in common at 10 points}: For 10 consecutive minutes you have a + 1d6 saving throw on Reflex and Fortitude. Once a day. \\
- \textbf{Sum of traits in common at 15 points}: You emanate an aura that grants all your companions within 10 feet a +2 save. Once a day, for 30 consecutive minutes \\
- \textbf{Sum of traits in common at 20 points}: Explode your wrath in a Fireball. The damage is from negative energy. DC 25 Reflexes to halve. 2 times a day \\
- \textbf{Elements}: Positive Energy - Negative Energy \\
- \textbf{Advantage}: Senses protected \\
- \textbf{Privileged Magic Lists}: Abjuration, Invocation \\
- \textbf{Favorite Weapon}: Heavy mace

\subsubsection{Atherim} \index{Atherim} \label{atherim}

The Guardian Patron. Many see the generous bosom of Atherim as a sign of voluptuousness and passion. They let themselves be enchanted by her busty beauty and do not see the crystal eyes that scare those who dare to even think of approaching her.

Atherim is the keeper of dreams and hopes, the one to whom to entrust desires, like a mother. He is the Patron saint of children, secrets and midwives.

With a cheerful smile and a good soul, she will always be ready to help you make your dreams come true. And like a mother Atherim protects and keeps secrets and passions. Atherim is silent. She is the one who keeps Yeru's secrets inside her soul forever.

The Devotee of Atherim takes to heart those who have made a promise, punishes those who break them and those who reveal the secrets. Many Atherim Devotees are diplomats, notaries and midwives. \\

- \textbf{Symbol:} A gloved woman's hand holding an ampoule full of flows \\
- \textbf{Feature}: Wisdom \\
- \textbf{Traits}: Merry, Merry, Calm, Hardworking, Good, Silent, Kind, 	Patient, Shy, Emotional, Meek, Gullible \\
- \textbf{Manifestation}: a serene and reassuring silence falls around the enchanter \\
- \textbf{Sum of 5-point common traits}: You may add 1d6 to a saving throw after rolling it but before you know if it was successful or not. Once a day, as a Reaction. \\
- \textbf{Sum of traits in common at 10 points}: You gain 30 temporary hit points. Duration 1 hour, once a day, as immediate action. \\
- \textbf{Sum of traits in common at 15 points}: You can cast the Spell Zone of Truth 3 times per day \\
- \textbf{Sum of traits in common at 20 points}: Each potion you drink has double the duration or effect if immediate. \\
- \textbf{Elements}: Positive Energy, Electricity \\
- \textbf{Advantage}: Metabolism control \\
- \textbf{Privileged Magic Lists}: Enchantment \\
- \textbf{Favorite Weapon}: Dagger

\subsubsection{Belevon} \index{Belevon} \label{belevon}

It is the Patron who best embodies the lie and the fiction in order to have his own advantage. He only loves himself. He is a narcissist who surrounds himself only with people who indulge and flatter him. He abhors loneliness but at the same time hates being touched by someone.

He is always looking for new things, wonderful objects that he exchanges and reciprocates with other objects. He likes to argue and haggle, argue and take the sale to the end.

From the appearance of a young boy he perfectly embodies a dangerous rogue.

The Belevon Devotee is well described by the rich and curious merchant who never gives up an opportunity to trade new goods. He is not driven by greed or accumulation but by the Art of commerce and exchange. \\

- \textbf{Symbol}: A golden cage \\
- \textbf{Feature}: Intelligence \\
- \textbf{Traits}: Confusional, Narcissist, Chaste, Liar, Curious, Double agent, Inconstant, Clumsy, Reckless, Insolent, Envious \\
- \textbf{Manifestation}: as if the golden bars of a cage were woven around the caster \\
- \textbf{Sum of traits in common at 5 points} points: You can cast the Prestidigitation spell, 3 times per day. \\
- \textbf{Sum of traits in common at 10 points}: Gain the ability to cast the Greater Image spell once per day. \\
- \textbf{Sum of traits in common at 15 points}: You can cast the Death Hallucination spell. 1 time per day \\
- \textbf{Sum of the traits in common at 20 points}: By touching an object you get to know briefly the history of who created it. Once a day. It costs 3 Actions. \\
- \textbf{Elements}: Fire, Sound \\
- \textbf{Advantage}: Lucky \\
- \textbf{Privileged Magic Lists}: Illusion \\
- \textbf{Favorite Weapon}: Light pike

\subsubsection{Cattalm} \index{Cattalm} \label{cattalm}

%\begin{changemargin}{0.3cm}{0.3cm}\begin{enfasi}{
%Nessuno dà nulla per nulla. (anonimo)
%}\end{enfasi}\end{changemargin}

Generated directly by Calicanthus, as a response to Ljust's creation of Gradh, it is pure destruction, chaos and entropy. Cattalm has the sole purpose of destroying, bringing chaos and disease, earthquakes and floods.

Cattalm is among the few Patrons who dares to openly challenge Gradh and does so with joy because he knows that their battle will only bring further destruction. Cattalm accepts and invites to be his Devotee every creature capable of hatred, capable of destroying and wounding. Many of his Devotees are monstrous creatures or aberrations.

Cattalm, on the other hand, is among the most wonderful Patrons, with a bright white skin, soft feather wings and a light silver armor. Although his delicate features make him a beautiful being, however much he aspires to destruction.

Cattalm loves the chaos that manifests in the most violent ways with earthquakes, floods, tsunamis, diseases if not directly fiery rains. He almost never acts directly but lets chaos and destruction work for him.

Ljust could not fail to intervene in the creation of such an explicitly evil Patron and, secretly from Calicante, instilled in Cattalm love and protection for children. Cattalm destroys, poisons, weakens but not the children, not even indirectly, rather he activates himself to cancel the evil caused by his nature.

It has already happened that entire villages were flooded and all the little ones were found on the wooden roofs like barges.

Whenever calamity occurs, it is said that "Cattalm has stamped his foot". \\

- \textbf{Symbol}: A giant wave towering over the coast \\
- \textbf{Feature}: Strength \\
- \textbf{Traits}: Destructive, Anarchist, Meticulous, Sadistic, Provocative, Brutal, Fatalistic, Unmoved, Warlike, Calculating, Meticulous \\ \\
- \textbf{Manifestation}: the roar of thunder \\
- \textbf{Sum of traits in common at 5 points} points: Through your weapons you weaken the intended opponent. Following a critical hit, you can increase fatigue by one level. Once a day as a Reaction. \\
- \textbf{Sum of traits in common at 10 points}: Your touch rots food (up to 50kg) and water (a cube with a 10m edge). Once a day \\
- \textbf{Sum of traits in common at 15 points}: Your gaze blinds with anger. You cast the Confusion spell, but the only possible result is that the targets attack random subjects. DC 25. 1 time per day \\
- \textbf{Sum of traits in common at 20 points}: You cast the Cone of Cold spell, but the damage is from Void. DC 25. Once a day \\
- \textbf{Elements}: Negative Energy - Vacuum \\
- \textbf{Advantage}: Hard to kill \\
- \textbf{Privileged Magic Lists}: Fire \\
- \textbf{Favorite Weapon}: Large double ax

\subsubsection{Efrem} \index{Efrem} \label{efrem}

He is the Patron saint of those who make nature their home. It embodies in itself the purest aspects of nature itself, aggressive as only the most lethal felines can be; but also wild as the most hidden glades and rigorous as only nature can be.

Efrem sets out to defend Nature from man's contamination, from this pest species that destroys everything it encounters.

The Devotees of Ephrem are more related to the natural element. They manipulate mainly elemental magic and defend or attack themselves using animals and natural creatures as well. In rare cases by forcing even the Dragons to obey.

The Devotees of Ephrem have the supreme goal of protecting animals and plants, places and all that is natural and not artificial. Usually solitary and grumpy, he cannot understand the reason for the hatred that, from his point of view, the man unloads on Yeru.

A Devotee of Ephrem respects life as well as death, in the natural process which is evolution and the life cycle. Sometimes he decides to settle in a certain environment and elects it as his territory and as his home protects it. Other times he decides to be a wanderer and intervene throughout Yeru to protect his beloved flowers and animals. \\

- \textbf{Symbol}: A stirrup with a vine twisted around \\
- \textbf{Characteristic}: Constitution \\
- \textbf{Traits}: Indifferent, Loyal, Ironic, Pragmatic, Measured, Sober, 	Austere, Grumpy, Respectful, Solitary, Sincere \\
- \textbf{Manifestation}: coils of leaves wrap around the sword \\
- \textbf{Sum of traits in common to 5 points} points: Your touch makes nonmagical animals docile. Will save 20 to resist. 3 times a day. Cost of 2 Shares. \\
- \textbf{Sum of traits in common at 10 points}: You gain a + 1d6 on all Survival checks made in a natural environment. \\
- \textbf{Sum of traits in common at 15 points}: You can cast the Beneficial Berries spell 1 time per day. Each berry heals 1d6 hit points and removes disease or nonmagical poisons. \\
- \textbf{Sum of traits in common at 20 points}: Your touch is that of the master. You can tame magical creatures, but not Aberrations or Dragons you touch. Will save DC 30. Once per day. Cost of 2 Shares \\
- \textbf{Elements}: Electricity, Sound \\
- \textbf{Advantage}: Empathy with plants \\
- \textbf{Privileged Magic Lists}: Animals and Plants and an Elemental Magic List. \\
- \textbf{Favorite Weapon}: Staff

\subsubsection{Erondil} \index{Erondil} \label{erondil}

Patron of Earth and Air, Erondil is the Lord of the most concrete and rational elements. He who endowed with infinite power and rationality gives his Devotees the power of earth manipulation. The gift of creating gigantic constructions of millenary strength with simple "mud". He concludes his works with attention and precision.
Even with difficulty because if the final result does not satisfy him, he unleashes his lightning bolts to destroy it instantly. Perfectionist and insatiable, hardly anything is exactly as he imagined it.

Orderly and exuberant, he is the lord of storms, thunder and lightning, earthquakes and destruction. He loves to surround himself with the roar of thunder, the roar of the crumbling earth. He can be destructive to those who do not respect Yeru.
He has arms and chest covered with almost silvery tattoos that tell the legends of Terra and Aria. Erondil is the lord of thunder and earthquakes.

The Devotees of Erondil are the engineers of the impossible, every time one has to defy matter, gravity and the same reason a Devotee of Erondil will find bread for his teeth, he will find the right challenge for a Builder of the Impossible.\\

- \textbf{Symbol:} a sandcastle with a lightning bolt above \\
- \textbf{Feature}: Wisdom \\
- \textbf{Traits}: Incontentist, Perfectionist, Dreamer, Exuberant, Jealous, Destructive, Orderly, Superficial, Naive, Pragmatic, Rational \\
- \textbf{Manifestation}: storm sound and landslide roar \\
- \textbf{Sum of traits in common 5 points}: You no longer fear falls. You can cast the Fall Feather spell 3 times per day, only on yourself. \\
- \textbf{Sum of traits in common at 10 points}: Your touch shapes the stone. You can cast the Wall Pass spell 1 time per day. \\
- \textbf{Sum of traits in common at 15 points}: You can cast the spell Lightning from your hands. Reflex saving throw DC 30 to half. Cost of 2 Shares. \\
- \textbf{Sum of traits in common at 20 points}: You are able to create a very deep pit (1km) under your opponent (cut up to large). Reflex save 30 or fall. Once a day. After 1 minute the pit closes with whoever is inside. Cost of 2 Shares. \\
- \textbf{Elements}: Fire, Electricity \\
- \textbf{Advantage}: Universal digestion \\
- \textbf{Privileged Magic Lists}: Air, Earth \\
- \textbf{Favorite Weapon}: Warhammer

\subsubsection{Gaya} \index{Gaya} \label{gaya}

Patron of Water and Fire, in the depths of the earth, where water and lava meet, Gaya enjoys painting. He loves to surround himself with the streams of fire and water as if to create a dance among them. He loves the sounds of nature, the breaking of the waves on the rocks, the falling of raindrops on the cobbles, the hum of a crackling fire.

He paints by mixing hot and cold. The crystal clear and impetuous water to the intriguing and ardent fire. Jealous of beauty and the arts, she keeps all her works safe in an almost maniacal and protected order. As a true artist, she uses the elements to make the wonders of nature shine. Gaya is the painter of sunsets and storms.

Gaya Devotees are fickle and over the top artists. They are those who recreate the magic of sunrise or sunset or of the stormy sea in their works, they are those who put poetry and madness into normality. \\

- \textbf{Symbol:} a brush on the sky \\
- \textbf{Feature}: Intelligence \\
- \textbf{Traits}: Anarchist, Instinctive, Impetuous, Emotional, Touchy, Lunatic, Dreamer, Jealous, Fickle, Enthusiastic, Narcissistic \\
- \textbf{Manifestation}: coils of fire and water envelop the caster \\
- \textbf{Sum of traits in common to 5 points} points: You can create up to 5 liters of water or 1 liter of good quality liquor. Once a day. Cost of 2 Shares. \\
- \textbf{Sum of traits in common at 10 points}: Your metabolism can't stand the cold. Resist magical cold damage and are immune to natural damage. \\
- \textbf{Sum of traits in common at 15 points}: You can breathe underwater as you breathe air. Resist non-magical fire damage \\
- \textbf{Sum of traits in common at 20 points}: You generate a shower of fire. You cast the spell Flame Strike, DC 25 once per day. Resist magical fire damage. \\
- \textbf{Elements}: Electricity, Fire \\
- \textbf{Advantage}: Rainbow \\
- \textbf{Privileged Magic Lists}: Water, Fire \\
- \textbf{Favorite Weapon}: Trident

\bigskip

\textbf{Gaia} and \textbf{Erondil} are like the two sides of the same coin and oversee the elements, Gaia water and fire and Erondil Air and Earth; they act as a direct expression of the major Patrons, they are small manifestations of their immense power.

\subsubsection{Krondal} \index{Krondal} \label{krondal}

He is a powerful but reserved and reserved Patron. He keeps aloof, out of the game until he perceives the deprivation of liberty.

He can't see into the future, he can't know people but his formidable instinct makes him the most fearsome fighter you can ever meet. Brave to the point of recklessness, he acts fearlessly in battle. With a good spirit, Krondal enters the field in the most important moments, when it is not a situation that is decided but the future of life, of one's personal freedom.

Krondal, the blind fury, has a deep respect for freedom and is profoundly opposed to any slavery, racism or dictatorship.

A Devotee of Krondal is typically a bodyguard, protector, sheriff who knows and must decide for the sake of his country, whatever it takes.
A Krondal Devotee does not judge people or facts but adheres to his ethics of protection and freedom.

Under modest and threadbare clothes, but always clean, he hides a fighter's physique. \\

- \textbf{Symbol}: A sword held vertically in front of you \\
- \textbf{Characteristic}: Charisma \\
- \textbf{Traits}: Careful, Devout, Fair, Liberal, Conformist, Unsatisfactory, Bold, Reserved, Shy, Instinctive, Courageous, Reckless \\
- \textbf{Manifestation}: the cloak or robe of the Devotee becomes dirty with earth and blood \\
- \textbf{Sum of traits in common at 5 points} points: Curse your opponent. You cast Curse once per day. DC 20 to resist. \\
- \textbf{Sum of traits in common at 10 points}: You cannot be tied up or handcuffed. Twice a day you can only cast Freedom of Movement on yourself. \\
- \textbf{Sum of traits in common at 15 points}: Your presence blocks the sight of the opponents. Designate up to 6 creatures within 30 feet, they must make a Fortitude save at DC 30 or be blind only to you for 1d4 rounds. \\
- \textbf{Sum of traits in common at 20 points}: Your weapon is more effective against enemies. Each creature hit must make a DC 20 Will saving throw or be paralyzed for 3 rounds. Once the creature succeeds at the saving throw, it can no longer be affected for the next 24 hours. Once a day, activating the skill costs 1 Action and lasts 1 minute. \\
- \textbf{Elements}: Positive Energy, Fire \\
- \textbf{Advantage}: Magnetic \\
- \textbf{Privileged Magic Lists}: Abjuration \\
- \textbf{Favorite Weapon}: Long sword

\subsubsection{Ledyal} \index{Ledyal} \label{ledyal} \label{laydel}

He is the Patron saint without a precise face, without a voice if not a song. Changeable in the body and without a clear definition of its being. It manifests itself with a long fiery red cloak with a fabric made of a thousand butterflies. His touch is life and peace, he protects those who need his favors regardless of whether they ask for them or not. He longs for a world without suffering, with only happiness and harmony. Suspicious and deeply introverted, he does not believe those who agree with him. He has a heart full of life and goodness but he has no body to love with.

Ledyal also has a twin sister, or perhaps another personality. Or maybe they are the same Patron, no one has ever seen them together. The "twin" \textbf{Laydel} \index{Laydel} does not tolerate suffering, despises those who cause pain, fearlessly kills any creature that has sinned against an innocent, anyone who has caused suffering. \\

- \textbf{Symbol}: A butterfly dripping blood as it flies \\
- \textbf{Feature}: Wisdom (Ledyal) - Strength (Laydel) \\
- \textbf{Ledyal traits}: Introverted, Suspicious, Charitable, Kind, Shy, Loyal, 	Generous, Understanding, Passionate, Patient, Spontaneous 	\\
- \textbf{Laydel traits}: Integral, Suspicious, Charitable, Relentless, Susceptible, Passionate, 	Iracult, Anarchic, Rigid, Touchy, Ruthless \\
- \textbf{Manifestation}: as if a cloak of butterflies wrapped the Devotee \\
- \textbf{Sum of traits in common at 5 points} points: Your touch is life / attack. 3 times per day, you can touch a living creature and heal / cause it for 1d6 hit points. Cost 2 Actions (also includes Touch Action) \\
- \textbf{Sum of traits in common at 10 points}: Your touch is peace. You can cast the Sanctuary spell once per day. \\
- \textbf{Sum of traits in common at 15 points}: Your aura protects your companions. Within 20 feet, your teammates have +4 to Defense and +2 to saving throws. Duration 10 consecutive minutes, once a day. Cost of 2 Shares. \\
- \textbf{Sum of traits in common at 20 points}: Radiate a healing sphere around you. Each creature within 20 feet is healed for 60 hit points. Once a day. In the case of Laydel the effect is the opposite. Cost of 2 Shares \\
- \textbf{Elements}: Positive Energy, Electricity \\
- \textbf{Advantage}: Healer (Ledyal) or Fearless (Laydel) \\
- \textbf{Privileged Magic Lists}: Invocation or Heal \\
- \textbf{Favorite Weapon}: Truncheon / Spiked Chain

\subsubsection{Nethergal} \index{Nethergal} \label{nethergal}

The Patron Messenger. The letter from Nethergal flies on the feather of a goose. Quick, impetuous, direct, Nethergal is the messenger, the one to whom to entrust thoughts and writings. Sarcastic and talkative, she will inquire about your purposes, will ask you for information on the writings entrusted to her with explicit frankness and will always have something to say about the message to bring but will also be just as direct and precise in delivering it.

Nethergal is not just gossip and gossip, whatever text is written she knows, there is no written code or secret that she does not know.

The Devotee of Nethergal is a fine linguist, an expert in riddles and rebuses, a Devotee who, unlike Atmos, does not limit himself to guarding the writings but spreads their knowledge.

A Nethergal Devotee is a teacher, a college language teacher, a learned expert on a thousand subjects. \\

- \textbf{Symbol:} an iridescent white feather \\
- \textbf{Characteristic}: Dexterity \\
- \textbf{Traits}: Sarcastic, impetuous, immature, talkative, competitive, rash, sociable, 	impatient, bungler, blunt, curious \\
- \textbf{Manifestation}: cascade of feathers, a flying goose \\
- \textbf{Sum of the traits in common to 5 points} points: You can send a message of up to 144 characters to a subject that you can see within 50 meters without being heard / seen. Once an hour. Cost 1 Action. The subject must understand the language used. \\
- \textbf{Sum of traits in common at 10 points}: By placing your hand on a book you learn its contents as if you had read it. One book a week. You lose the knowledge gained in this way after a week. Time 1 Turn. The written language of the tome must be known. \\
- \textbf{Sum of traits in common at 15 points}: You can fly, as a spell of the same name, 1 hour per day. Cost 1 Reaction. \\
- \textbf{Sum of traits in common at 20 points}: You can cast the Zone of Truth spell 3 times per day. DC 30 to resist. Cost of 2 Shares. \\
- \textbf{Elements}: Electricity, Sound \\
- \textbf{Advantage}: Absolute Management \\
- \textbf{Privileged Magic Lists}: Transmutation, Air \\
- \textbf{Favorite Weapon}: Light Crossbow

\subsubsection{Nedraf} \index{Nedraf} \label{nedraf}

The Surviving Patron, the never tired old wolf who has gone through and fought a thousand battles. His flesh is wounded, his body covered with war scars and bruises but nothing will bring him down. Tenacity, passion, experience and a lot of anger make Nedraf not only an excellent fighter on any occasion but a connoisseur of the environment around him. Thanks to his impeccable training he knows how to make the most of the resources available. He knows how to urge men with passion at his orders.
Nedraf represents the one you would always like to be at your side in every battle.

Many captains of fortune and commanding officers are Nedraf devotees. The Devotee of Nedraf does not give up, does not give up, does not abandon his companions but this does not mean that he is reckless or irrational in his choices. \\

- \textbf{Symbol:} a strong hand, wrapped in a bloodstained bandage, wielding a sword \\
- \textbf{Characteristic}: Constitution \\
- \textbf{Traits}: Disciplined, Fighting, Tenacious, Aggressive, Planner, Mischievous, Honorable, Competitive, Angry, Rational, Determined \\
- \textbf{Manifestation}: the smell of blood and metal spreads in the air \\
- \textbf{Sum of traits in common at 5 points} points: You can wear light armor without penalty on the Magic Check \\
- \textbf{Sum of traits in common at 10 points}: Acquire a bonus point on a Weapon List. It may or may not be known \\
- \textbf{Sum of traits in common at 15 points}: You can wear medium armor without penalty on the Magic and Dexterity Check \\
- \textbf{Sum of traits in common at 20 points}: Acquire a bonus point on a Weapon List. It may or may not be known \\
- \textbf{Elements}: Positive energy, Sound \\
- \textbf{Advantage}: Accelerated healing \\
- \textbf{Privileged Magic Lists}: Enchantment, Earth \\
- \textbf{Favorite Weapon}: Two-handed Greatsword

\subsubsection{Nihar} \index{Nihar} \label{nihar}

He is the Patron saint of heroes by chance. Thoughtful and calm, he loves good wine and revelry. He is the one you would never choose as a comrade in arms because of his "common" appearance and his goliardic attitude. But then when it's time to be there, to fight, to make a difference, he amazes everyone and "resolves" the game.

He has the appearance of a small man, with luxurious and refined clothes and a cautious and cheerful expression. He always protects himself and at any cost, showing the world exactly what the world wants to see. He carefully checks the reality around him and even if it is always easier to see him with a glass in his hand, if you do not let yourself be deceived by appearances you will notice how his eyes never lose sight of the danger, the problem. Be careful, he doesn't trust anything or anyone. It has made its defects its strengths. \\

- \textbf{Symbol}: A dagger resting near a glass of wine \\
- \textbf{Feature}: Intelligence \\
- \textbf{Traits}: Selfless, Determined, Courteous, Attentive, Distrustful, Chaotic, Joker, Foresighted, Combative, Selfless, Ironic \\
- \textbf{Manifestation}: the sound of a toast or the uncorking of a bottle \\
- \textbf{Sum of traits in common to 5 points} points: You can turn water into wine. One liter a day. Cost 2 Actions. 2 times a day. \\
- \textbf{Sum of traits in common at 10 points}: One Immediate Action, you get a + 2d6 bonus to an Action in that round. 1 time a day. \\
- \textbf{Sum of traits in common at 15 points}: Your light weapon always deals critical damage when you hit. The bonus is always active. \\
- \textbf{Sum of traits in common at 20 points}: The dishes you prepare are very good. Anyone who is satisfied with a dish prepared by you recovers 2d6 hit points and is cured of poisons, even magical ones. Max 6 people per day. 0.5 hours of preparation per person. \\
- \textbf{Elements}: Positive Energy, Fire \\
- \textbf{Advantage}: Universal language \\
- \textbf{Privileged Magic Lists}: Enchantment, Divination \\
- \textbf{Favorite Weapon}: Short sword

\subsubsection{Orudjs} \index{Orudjs} \label{orudjs}

That is the Patron of illusion and fiction. He who only with the gift of the word, the gesticulation of the hands, the charismatic voice and the intriguing gaze is able to sell his every word as absolute truth. He loves theater for what it is for him, the representation of human falseness, being many people and in reality none. He loves politics and its intrigues. He pretends to listen to those around him but in reality he is not interested in the stories of others because his are always the best.

He is a coward without limits and the few truths he says, and they are very rare, are told by him only to save himself.

With a rather ordinary and almost predictable appearance as soon as he opens his mouth and begins his stories, he manages to attract the attention of the entire room. In fact, he possesses a warm and persuasive voice which, accompanied by the very good dialectic he possesses, enchants every listening ear.

Its Devotees are skilled actors and entertainers, undercover spies, diplomats or politicians. \\

- \textbf{Symbol}: A theatrical mask with only an open mouth and eyes \\
- \textbf{Characteristic}: Charisma \\
- \textbf{Traits}: Ironic, Cowardly, Knowledgeable, Sociable, Inconstant, Creative, 	Lunatic, Dishonest, Snobby, Liar, Insolent \\
- \textbf{Manifestation}: the sound of a deep and contagious laugh \\
- \textbf{Sum of traits in common to 5 points} points: Your speech is already legendary. +2 to Entertain checks. \\
- \textbf{Sum of traits in common at 10 points}: You can cast Silent Image 3 times per day. \\
- \textbf{Sum of traits in common at 15 points}: Your speech is already legendary. + 1d6 additional on Perform checks. You can cast Immagine Maggiore 1 time per day. \\
- \textbf{Sum of the traits in common at 20 points}: Your voice is persuasive. A creature you spot that listens to you for at least one minute must make a Will save DC 30 or be under the influence of dominate people. Once a day \\
- \textbf{Elements}: Electricity, Fire \\
- \textbf{Advantage}: Persuasive voice \\
- \textbf{Privileged Magic Lists}: Enchantment, Illusion \\
- \textbf{Favorite Weapon}: Rapier

\subsubsection{Orlaith} \index{Orlaith} \label{orlaith}

That is the Patron of Justice and Vengeance. He slavishly follows the laws and expects his subordinates to carry out the orders given without any discussion. He is moved by a gentle and good spirit that however keeps well hidden behind his direct and incisive, shameless and deadly actions. Orlaith is revenge that becomes law. He acts out of a sense of justice with his methods. His bearing and proud gaze attract him.

Devotees of Orlaith are often judges and executioners, people who have decided to bring justice everywhere, because Orlaith cannot stand still, there is always someone to judge and punish. \\

- \textbf{Symbol}: A hand stretched out over a closed book \\
- \textbf{Feature}: Strength \\
- \textbf{Traits}: Impartial, Righteous, Vengeful, Valiant, Outspoken, Expansive, Spontaneous, Enterprising, Disliking, Conformist, Traditionalist \\
- \textbf{Manifestation}: the image of a steelyard, unbalanced. \\
- \textbf{Sum of traits in common to 5 points} points: You call back 1 mastiff (normal) that obeys your commands. Duration 1 minute. Once a day. Cost of 2 Shares. \\
- \textbf{Sum of traits in common at 10 points}: A pair of handcuffs manifest around the creature's wrists (maximum large size) within 27 meters. Reflex saving throw DC 25 to cancel. Cost 2 Actions. Once a day. Force / Escape Artist DC 20 to break free. \\
- \textbf{Sum of traits in common at 15 points}: Your hearing is only for the truth. Around you for 3 meters, including you, the Zone of Truth is always active. \\
- \textbf{Sum of the traits in common at 20 points}: Create a ray of Light 27 meters long and a few centimeters wide. Each creature crossed takes 5d6 damage, DC 25 Reflexes for half. Once a day. Cost of 2 Shares. \\
- \textbf{Elements}: Light, Sound \\
- \textbf{Advantage}: Common sense \\
- \textbf{Privileged Magic Lists}: Illusion, Fire \\
- \textbf{Favorite Weapon}: Foot spear

\subsubsection{Rezh} \index{Rezh} \label{rezh}

The Patron who despises everything. Rezh loves, wants, touches, gazes only at his shiny and shiny coins. I am never enough, no wealth is ever enough for her. Rezh, the miser keeps everything to himself, knows no compassion, knows no charity, knows no sharing. Her hunger for money, for riches makes her prone to any baseness. He despises everything and everyone and judges everything and everyone by following only his own personal yardstick. In each coin there is a little bit of Rezh. Rezh's imprint can be seen in the oxidation of each coin.

If money buys freedom Rezh has to accumulate more and more if it will ever be enough.

Rezh's Devotees are usually chosen by her from the ranks of the greediest and wealthiest. Their aim is to accumulate wealth, more and more.
Rezh Devotees often become explorers, grave robbers, people always looking for an extra treasure and coin. \\

- \textbf{Symbol:} a pile of coins with a rat nearby \\
- \textbf{Feature}: Intelligence \\
- \textbf{Traits}: Greedy, Arrogant, Bad, Cold, Jealous, Habitual, 	Uncertain, Irritable, Careful, Unfair, Intolerant \\
- \textbf{Manifestation}: a sound of falling coins envelops the caster \\
- \textbf{Sum of traits in common to 5 points} points: You are an expert on coins and gems, no forger can deceive you. + 1d6 on related Awareness and Knowledge checks. \\
- \textbf{Sum of traits in common at 10 points}: You use gems as receptacles. You can download a spell of 3 level or lower into a gem, which must have a minimum value of 10th. The gem retains the spell for 6 hours. To activate the gem you use 2 actions and the spell it contains is cast. \\
- \textbf{Sum of traits in common at 15 points}: You can pull 1 gold coin out of your pocket whenever you want. Max 10 gp per day. Cost of 1 Action. \\
- \textbf{Sum of traits in common at 20 points}: Your armor is covered in golden glitter and gems. You gain +4 defense and + 1d6 Fortitude saving throw for 1 hour. Cost 1 Reaction, once per day. \\
- \textbf{Elements}: Vacuum, Electricity \\
- \textbf{Advantage}: Fairy Hands \\
- \textbf{Privileged Magic Lists}: Abjuration \\
- \textbf{Favorite Weapon}: Sickle

\subsubsection{Sumkjr} \index{Sumkjr} \label{sumkjr}

\begin{changemargin}{0.3cm}{0.3cm} \begin{enfasi}{
Anything that is not given is lost. (Dominique Lapierre)
} \end{enfasi} \end{changemargin}

Patron of the Arcanum of Light. Sumkjr is goodness, fairness, loyalty, justice, protection.

Sumkjr is the knight who protects the innocent, he is the sword "of Ljust" in the final battle. It defends the weak and soothes the wounds.

Sumkjr carries the Light of Ljust everywhere, no danger can ever stop Sumkjr from his continuous, infinite search for good.

A Sumkjr Devotee acts loyally and honorably, always pursuing the ultimate good, his being cannot be bent to evil, injustice, dishonor.

With courage and determination the Devotee faces every challenge but not only out of a sense of duty, but because he is deeply devoted to his destiny. Sumkjr knows that few people hold up this standard because unlike the Devotees of the Patroness of Genesis, his Devotees are not born to be such, but become so thanks to their deep and determined willpower.

For this reason, Ljust intervenes in their favor with the elaborate Rite of Renewal, thanks to which every year the deserving and repentant Devotee of having lost even just for a little the right direction, the Light, is made to recover every point. from the 7 Luminous Rules.

Sumkjr is a valiant soldier, the best friend of the righteous.

Calicante, taken by horror at the sight of such a Patron, deprived him of the ability to love and feel true feelings of affection. Carrying good for a Sumkjr Devotee is something normal as it is normal not to be able to be empathetic with those who suffer. The Devout knows what he must do and why, but he cannot be moved or loved in the face of the sufferings or caresses of a woman / man. \\

- \textbf{Symbol:} three drops of blood falling one behind the other \\
- \textbf{Characteristic}: Charisma \\
- \textbf{Traits}: Righteous, Curious, Good, Valiant, Candid, Messy, Idealist, Martyr, Protective, Humble, Stubborn \\
- \textbf{Manifestation}: the Devotee is wrapped in a golden brocade cloak \\
- \textbf{Sum of traits in common to 5 points} points: The touch of your sword is life. A creature touched with your weapon recovers 3d6 hit points. Once a day. Cost of 2 Shares. \\
- \textbf{Sum of traits in common at 10 points}: Your Will is stronger than metal. You gain +2 on Will saving throws \\
- \textbf{Sum of traits in common at 15 points}: You can cast the Cone of Cold spell, but the damage is from Electricity. DC 25 to halve. Once a day. Cost of 2 Shares. \\
- \textbf{Sum of traits in common at 20 points}: You sacrifice your life to bring a creature that has been dead for no more than 1 week to life. Once. Cost of 3 Shares. \\
- \textbf{Elements}: Positive Energy, Electricity \\
- \textbf{Advantage}: Aura of Courage \\
- \textbf{Privileged Magic Lists}: Heal \\
- \textbf{Favorite Weapon}: Bastard Sword

\textbf{The 7 Luminous Rules} \index{7 Luminous Rules} \\

The Seven Luminous Rules are a set of norms and behaviors held, in various capacities, by the Devotees who want to follow the path of the Light of Ljust.

The Devotees of Sumkjr must follow all 7 under penalty of loss of power (Trait points), other Devotees of other Patrons, always positive or at least neutral, follow only some of these dictates, as a rule not to fall into the arms of Calicant \\

1. Protect the weak and those who cannot defend themselves from abuse \\
2. Love life and protect it. \\
3. Fight against injustice and those who bring suffering and pain \\
4. Soothe wounds and pains. Calm souls and promote peace and harmony \\
5. Honesty and Loyalty are your foundations \\
6. You are a master of virtue. Let others take inspiration from your deeds \\
7. Be bright but don't blind others

\subsubsection{Shayalia} \index{Shayalia} \label{shayalia}

Patron of the Arcanum of Darkness. Shayalia is the dark soul of perdition, betrayal, the most sordid and sinful lust. He loves brothels. She likes the acrid smell of sweat, her skin shiny with oils and perfumes. The passions, the vendettas that consume them, the physical and moral destruction that is perpetrated in those places is his life.

Shayalia is a woman who enjoys herself, who experiences more physical pleasures. He lives on long and well-designed vendettas. Vengeful and amoral, she does not judge with human standards, her enjoyment is not even remotely understandable. Shayalia is the closest thing to Calicante that has been created. It is the passions, the drives, the humoral liquids that make her intoxicated.

Shayalia is the concubine who bewitches you and destroys you, drop by drop. Poisons are his weapons, human weaknesses his field.

Devotees of Shayalia are spies, bastard children, lovers of powerful lords who operate in the shadows.

Ljust disgusted by the vision of such a Patron instilled in Shayalia a love for nature, plants and animals. And so many of the most famous botanists, Herbalisms and zoologists are Shayalia devotees, perhaps the only things Shayalia can truly love. \\

\begin{changemargin}{0.3cm}{0.3cm} \begin{Storytellere}
While Efrem is the Patron of untouched nature, Shayalia embodies devotion and love for nature. The first supervises nature from above, the second descends and becomes one with it.
\end{Storytellere} \end{changemargin}

- \textbf{Symbol:} a wrinkled pillow stained with blood \\
- \textbf{Characteristic}: Charisma \\
- \textbf{Traits}: Lustful, aloof, cold, vengeful, arrogant, false, mischievous, rebellious, dishonest, dreamy, affable \\
- \textbf{Manifestation}: the Devotee is wrapped in a black velvet cloak \\
- \textbf{Sum of traits in common to 5 points} points: The time to prepare a potion is halved. Healing spells also affect animals and plants. \\
- \textbf{Sum of traits in common at 10 points}: Your touch is life for nature. Your healing spells have a maximal effect on natural animals and plants. \\
- \textbf{Sum of traits in common at 15 points}: From your palm you secrete poison. Your touch or melee weapon carries the poison. Saving Throw Fortitude DC 25 or -2 at Wisdom and Dexterity for 10 minutes, a subject who is poisoned cannot be poisoned again for 24 hours. Cost 1 Action. \\
- \textbf{Sum of traits in common at 20 points}: Your touch is life for nature. You can heal magical animals and plants. You are immune to natural poisons. + 1d6 Nature knowledge.
- \textbf{Elements}: Vacuum, Electricity \\
- \textbf{Advantage}: Animal Empathy \\
- \textbf{Privileged Magic Lists}: Illusion or Animals and Plants and an Elemental Magic List \\
- \textbf{Favorite Weapon}: Whip

\bigskip

\textbf{Sumkjr} and \textbf{Shayalia} are complementary in holding the elusive ranks of creatures. They act as a direct expression of the Patrons of Genesis.

\subsubsection{Sixiser} \index{Sixiser} \label{sixiser}

\begin{changemargin}{0.3cm}{0.3cm} \begin{enfasi}{
The force that opposes fate is actually a weakness. (Franz Kafka)
} \end{enfasi} \end{changemargin}


The Patron who is indifferent to the present as he is totally, compulsively obsessed with the future and his destiny. In the most remote corners of the known worlds it is said that Sixiser accumulates everything, indifferent to everything and everyone.

Terrified by the future he sees, by a hypothetical end of himself and altogether he lives a life of retreat, spiritual and physical. He voluntarily deprives himself of everything necessary. But at the same time it accumulates any object that crosses its path in the hope of a return.

He is paranoid and does not trust anyone. He uses his powers of divination to know and scrutinize everyone.

Sixiser's Devotees are often necromancers surrounded by undead and other silent and obedient creatures. Those who take refuge in search of solitude and study, those who instead aim to expand and govern entire cities and nations in order to feel more secure, are devoted to Sixiser. \\

- \textbf{Symbol}: A chest overflowing with everything that cannot be closed \\
- \textbf{Feature}: Wisdom \\
- \textbf{Traits}: Reserved, Indifferent, Accumulator, Paranoid, Liar, Insensitive, Fearful, Mean, 	Ambitious, Austere, Knowing \\
- \textbf{Manifestation}: two hands that surround, as if to hide, the head of the caster \\
- \textbf{Sum of the traits in common to 5 points} points: acquire twilight vision up to 18 meters, or 36 meters if already present. \\
- \textbf{Sum of traits in common at 10 points}: see in darkness even magical within 60 feet. Automatically detect non-magical traps within 3 meters of you. \\
- \textbf{Sum of traits in common at 15 points}: By touching an object you are able to understand all its magical and non-magical properties, even if it is cursed. 3 times a day. \\
- \textbf{Sum of traits in common at 20 points}: You are able to animate a creature that has been dead for no more than a day as an undead from 1 degree of Challenge (zombie / skeleton type depending on state). Once a day. Cost of 2 Shares. \\
- \textbf{Elements}: Electricity, Negative Energy \\
- \textbf{Advantage}: Reduced consumption \\
- \textbf{Privileged Magic Lists}: Necromancy \\
- \textbf{Favorite Weapon}: Falcione

\subsubsection{Tazher} \index{Tazher} \label{tazher}

The Patron of Shadows; he who is silent, kills you. You will never know why. You will never know how he looks but, if you suddenly feel freezing, Tazher is behind you ready to take your life.
Double agent with a bad soul, ask for his help only if you are willing to pay the price that he and he alone will decide.

He lives at night, he lives at night. Shadows are her friends and darkness her cloak. Deeply individualistic with a grumpy and touchy character, he has no friends, does not entertain relationships of any kind.

Tazher's Devotee is the thief, the killer, the bandit, anyone who lives for the darkness and their own gain. A Devotee of Tazher is extremely dangerous in combat. \\

- \textbf{Symbol}: The glint of the blade in the dark \\
- \textbf{Characteristic}: Dexterity \\
- \textbf{Traits}: Grumpy, Calculating, Perfectionist, Bad, Insensitive, Grumpy, Corrupt, Shy, Unfair, Imaginative, Hypocritical \\
- \textbf{Manifestation}: the Devotee's shadow comes to life by moving the weapon \\
- \textbf{Sum of 5-point common traits} points: You gain +2 on Move Silently and Hide checks. \\
- \textbf{Sum of traits in common at 10 points}: Once per day make one more attack (without penalty). An Immediate Action. \\
- \textbf{Sum of traits in common at 15 points}: as long as you walk over shadows or in the dark (dim light or darkness) you are invisible. You can still be detected with light or divination spells. \\
- \textbf{Sum of traits in common at 20 points}: Once per day each successful melee attack generates a critical. Cost 1 Reaction Action to be declared even after the attack roll but before knowing if the rolls were successful. Usable 3 times a day. \\
- \textbf{Elements}: Void, Ice \\
- \textbf{Advantage}: My shadow is my friend \\
- \textbf{Privileged Magic Lists}: Transmutation \\
- \textbf{Favorite Weapon}: Glaive up for auction

\subsubsection{Thaft} \index{Thaft} \label{thaft}

The Patron who accompanies in birth and death. Silent, he stands aside and observes the flow of human life. Almost humble in its simplicity, Thaft is everywhere. Silent witness of human life; in the moment in which a life slips away, Thaft attends, in the moment in which a life is born, Thaft is present.

Thaft also knows that one cannot always and only be an observer. Through his sacred and magical notebook he can decide and judge the life of men, because if a sword hurts, it is only Thaft who decides its death.

The Devotees of Thaft are the priests of the last journey, those who protect and watch over the souls and bodies of the dead. Deeply opposed to the use of the undead they pursue their destruction.

A Devotee of Thaft respects life as well as death and is not afraid of wreaking havoc for greater balance.

Thaft was shaped by Atmos. \\

- \textbf{Symbol}: An open book with a skull on it \\
- \textbf{Feature}: Wisdom \\
- \textbf{Traits}: Simple, Quiet, Meek, Confident, Disciplined, Optimistic, 	Modest, Worldly, Respectful, Outspoken, Indulgent \\
- \textbf{Manifestation}: we hear the cry of a newborn baby or the sigh of death \\
- \textbf{Sum of traits in common at 5 points} points: Your touch is lethal to the undead. Your touch deals 2d6 damage to an undead. Cost 2 Actions including touch. Up to 3 times a day. \\
- \textbf{Sum of traits in common at 10 points}: Your touch soothes. Once a day you can remove Blindness or Deafness. Cost of 2 Shares. \\
- \textbf{Sum of traits in common at 15 points}: An undead must make a DC 30 Fortitude saving throw or be destroyed if touched by your hand. Cost of 2 Shares. \\
- \textbf{Sum of traits in common at 20 points}: Kill the creature touched. Will save DC 30 or death. Once a week. Cost of 2 Shares. \\
- \textbf{Elements}: Sound, Electricity \\
- \textbf{Advantage}: Cold Touch \\
- \textbf{Privileged Magic Lists}: Necromancy, Animals and Plants \\
- \textbf{Favorite Weapon}: Bow

\subsubsection{Torbiorn} \index{Torbiorn} \label{torbion}

The Patron who best embodies the concept "is never enough". Tall, beautiful like a painting but, just like the latter, without warmth and life, Torbiorn borders on maniacal perfection in dressing, in posturing.

Nothing is ever enough for him. No one is ever at his height. And here he is, with arrogance and irony, he modifies everything that can be modified in order to appease this profound dissatisfaction. If the final result achieved does not satisfy him, and it happens very often, his cynicism takes over and destroys everything without caring about the suffering he is causing to those around him.

The Devotee of Torbiorn is the typical rich and listless aristocrat, the one who always seeks the easiest and least risky way.

Regardless of others, he enjoys exploiting the work of others and benefiting from them. \\

- \textbf{Symbol}: An opaque mirror \\
- \textbf{Characteristic}: Charisma \\
- \textbf{Traits}: Haughty, Anxious, Vain, Touchy, Prudent, Wrathful, Unruly, 	Licentious, Cocky, Weak, Deceitful \\
- \textbf{Manifestation}: splinters of mirror broken all around the Devotee like a whirlwind \\
- \textbf{Sum of the traits in common to 5 points} points: With a gesture you can refresh your clothes and yourself making them clean and fragrant. Cost 1 Action. 3 times a day. \\
- \textbf{Sum of traits in common at 10 points}: Your spit is poisonous. If the touch attack roll hits -2 Strength. Duration 1 minute. Three times a day. Cost of 1 Action. \\
- \textbf{Sum of traits in common at 15 points}: Staring the target in the eye forces him to stop. The target can no longer move its legs (or legs for movement). Will save DC 30. Once per day. Cost of 2 Shares. \\
- \textbf{Sum of the traits in common at 20 points}: Tendrils start from your fingers that sting up to 10 opponents. Each tendril up to 60 feet long deals 2d6 damage, Reflex save DC 25 to half. Cost of 2 Shares. \\
- \textbf{Elements}: Fire, Sound \\
- \textbf{Advantage}: Hard to subjugate \\
- \textbf{Privileged Magic Lists}: Transmutation \\
- \textbf{Favorite Weapon}: One-handed ax

\medskip

In agreement with the Storyteller, and adequately motivated, it is possible to change Skills and Magic Lists.


\subsubsection{Patron List - Trait} \index{Patron List - Trait} \hypertarget{tabellacollegamentoPatronotratto}{} \label{tabellacollegamentoPatronotratto}

{\footnotesize
%\begin{xltabular}{1\textwidth}{lllllll}
\textbf{Ljust}: Courageous, Generous, Empathetic, Protective, Instinctive, Nonconformist, 	Sensitive, Outgoing, Correct, Compassionate, Selfless \\
\textbf{Calicant}: Selfish, Vengeful, Superb, Wrathful, Passionate, Cynical, Competitive, Creative, Dishonorable, Anarchic, Brutal \\
\textbf{Atmos}: Observer, Detached, Prudent, Thoughtful, Integral, Anxious, Paranoid, Complaining, Distrustful, Foresightful, Apathetic \\
\textbf{Lynx}: Solitary, Serious, Rigid, Controlled, Courageous, Insensitive, Stubborn, Determined, Intolerant, Introverted, Rational \\
\textbf{Gradh}: Indomitable, Protective, Vengeful, Courageous, Cold, Distrustful, Impetuous, Presumptuous, Gloomy, Reserved, Melancholic, Competitive \\
\textbf{Atherim}: Cheerful, Calm, Hardworking, Good, Quiet, Kind, 	Patient, Shy, Emotional, Meek, Gullible \\
\textbf{Belevon}: Confusional, Narcissist, Chaste, Liar, Curious, Double agent, Inconstant, Clumsy, Reckless, Insolent, Envious \\
\textbf{Cattalm}: Destructive, Anarchist, Meticulous, Sadistic, Provocative, Brutal, Fatalistic, Unemotional, Warlike, Calculating, Meticulous \\
\textbf{Efrem}: Indifferent, Loyal, Ironic, Pragmatic, Measured, Sober, 	Austere, Grumpy, Respectful, Solitary, Sincere \\
\textbf{Erondil}: Incontentable, Perfectionist, Dreamer, Exuberant, Jealous, Destructive, Orderly, Superficial, Naive, Pragmatic, Rational 	\\
\textbf{Gaya}: Anarchist, Instinctive, Impetuous, Emotional, Touchy, Lunatic, Dreamer, Jealous, Fickle, Enthusiastic, Narcissistic \\
\textbf{Krondal}: Careful, Devout, Fair, Liberal, Conformist, Unsatisfactory, Bold 	, Reserved, Shy, Instinctive, Courageous, Rash \\
\textbf{Ledyal}: Introverted, Suspicious, Charitable, Forgiving, Shy, Loyal, Generous, Understanding, Passionate, Patient, Spontaneous \\
\textbf{Laydel}: Integral, Suspicious, Charitable, Relentless, Susceptible, Passionate, 	Wrathful, Anarchic, Rigid, Touchy, Ruthless 	\\
\textbf{Nethergal}: Sarcastic, impetuous, immature, talkative, competitive, reckless, sociable, impatient, bungler, blunt, curious \\
\textbf{Nedraf}: Disciplined, Fighting, Tenacious, Aggressive, Planner, Mischievous, Honorable, Competitive, Rabid, Rational, Determined 	\\
\textbf{Nihar}: Selfless, Determined, Courteous, Attentive, Distrustful, Chaotic, Joker, Foresighted, Combative, Selfless, Ironic \\
\textbf{Orudjs}: Ironic, Cowardly, Knowledgeable, Sociable, Inconstant, Creative, Lunatic, Dishonest, Snobby, Liar, Insolent \\
\textbf{Orlaith}: Impartial, Righteous, Vengeful, Valiant, Outspoken, Expansive, Spontaneous, Enterprising, Disagreeable, Conformist, Traditionalist \\
\textbf{Rezh}: Greedy, Arrogant, Bad, Cold, Jealous, Habitual, Uncertain, Irritable, Careful, Disloyal, Intolerant \\
\textbf{Sumkjr}: Righteous, Curious, Good, Valiant, Candid, Messy, Idealist, Martyr, Protective, Humble, Stubborn, \\
\textbf{Shayalia}: Lustful, aloof, cold, vengeful, arrogant, false, mischievous, rebellious, dishonest, dreamy, affable \\
\textbf{Sixiser}: Reserved, Indifferent, Accumulator, Paranoid, Liar, Insensitive, Fearful, Mean, 	Ambitious, Austere, Knowing, 	\\
\textbf{Tazher}: Grumpy, Calculating, Perfectionist, Bad, Insensitive, Grumpy, Corrupt, Shy, Unfair, Imaginative, Hypocritical 	\\
\textbf{Thaft}: Simple, Quiet, Meek, Confident, Disciplined, Optimistic, Modest, Worldly, Respectful, Outspoken, Indulgent 	\\
\textbf{Torbiorn}: Haughty, Anxious, Vanity, Touchy, Prudent, Wrathful, Unruly, 	Licentious, Bully, Weak, Deceitful \\

}

\end{multicols}


%\begin{center}
%\includegraphics[keepaspectratio,width=0.30\textwidth]{immagini/archetipijung.png}
%\textit{I 12 Archetipi di Jung}
%\end{center}

\pagebreak

\section{Equipment} \hypertarget{equipaggiamento}{} \label{equipaggiamento}

\subsection{Wealth and Money} \index{Wealth and Money}
\begin{changemargin}{0.3cm}{0.3cm} \begin{enfasi}{I am ready to go. I have a backpack! (Morgan Grimes, Chuck, TV series)
} \end{enfasi} \end{changemargin}

\begin{multicols}{2}

\label{ricchezza-e-denaro}

\lettrine[lines = 2, lhang = 0.33, loversize = 0.25, findent = 1.5em]{T}{he} common coins come in different denominations based on the relative value of the metal they are made of. The three most common types of coins are the gold coin (mo), the silver coin (ma) and the copper coin (mr).

A skilled (but not exceptional) craftsman can earn one gold coin per day. The gold coin is the standard unit of measure of wealth, although the coin itself is not widely used. When merchants discuss deals involving goods or services worth hundreds or thousands of gold coins, the transactions usually involve no exchange of cash.

Gold coin, on the other hand, is a measure of value, and the exchange is done in gold bars, letters of credit or commodities of value. A gold coin is worth ten silver coins, the most used type of coin among the population. A silver coin can cover a laborer's daily wages and buy a cruet of oil for a lamp or a bed for a night in a cheap inn.

A silver coin is worth ten copper coins, which are normally used by workers and beggars. A single copper coin can buy a candle, a torch or a piece of plaster.

Sometimes, however, unusual coins appear among the treasures, composed of other precious metals. The electrum coin (me) and the platinum coin (mp) come from forgotten empires and lost kingdoms, and when used in transactions they sometimes arouse suspicion and skepticism. An electrum coin is worth five silver coins while a platinum coin is worth ten gold coins.

A common coin weighs about ten grams, so that fifty coins weigh half a kilo.

A character who starts playing generally has enough gold coins to buy the basic elements: some weapons, second-hand armor (the least expensive one) and some miscellaneous equipment. As the character embarks on adventures and accumulates loot, he can afford better equipment and magical items. At the first level, characters have coins and equipment for a total of approximately 125 gp.

\subsubsection{Coins} \index{Coins}

The most common coin is the gold coin (mo). A gold coin is worth 10 silver coins (ma). Each silver coin is worth 10 copper coins (mr). In addition to copper, silver and gold coins there are also platinum (mp) coins, each worth 10 gold and electrum (me) coins worth 5 silver coins.

\medskip

\textbf{Table: Equivalence of Coins} \index{Equivalence Table of Coins}

\medskip

\begin{tabular}{llllll}

\textbf{Moneta} & \textbf{MR} & \textbf{MA} & \textbf{ME} & \textbf{MO} & \textbf{MP} \\
\toprule
Copper 		& 1 		& 1/10 	& 1/50 	& 1/100 	& 1/1000 \\
Silver 	& 10 & 1 	& 1/5 	& 1/10 	& 1/100 \\
Electrum 	& 50 	& 5 	& 1 	& 1/2 	& 1/10 \\
Gold 		& 100 & 10 	& 2 		& 1 	& 1/10 \\
Platinum 	& 1000 & 100 	& 20 	& 10 	& 1 \\
\end{tabular}

\medskip

Payments over 100 gold coins usually take place in 1.2.5 kilogram gold bars, equivalent to 100, 200 and 500 gold coins or better still in gems. In the case of even more conspicuous sums, it is possible that a letter of credit from some banking institution (but valid in very few and important cities) is issued.

\subsubsection{Wealth at the first level} \index{Wealth at the first level} \index{First level money}

Typically a character with 1-part weapon proficiency or magical proficiency with 125 gp that he can spend on basic equipment.

\subsubsection{Other Wealth - Exchange Goods} \index{Other Wealth}

Merchants usually trade goods even without the use of coins.
To get an idea of the commercial transactions, some exchange commodities are described in the table.

\medskip

\textbf{Table: Examples of other riches} \index{Table of examples of other riches}

\medskip


\begin{tabular}{ll}
\textbf{Cost} & \textbf{Object} \\
\toprule
1 mr & Wheat (0.5 kg) \\
2 mr & Flour (0.5 kg) or chicken (1) \\
1 ma & Iron (0.5 kg) \\
5 ma & Tobacco or copper (0.5 kg) \\
1 mo & Cinnamon (0.5 kg) or goat \\
2 mo & Ginger or pepper (0.5 kg) or sheep (1) \\
3 mo & Pork (1) \\
4 mo & Linen (1 m \textsuperscript{2} \\
5 mo & Salt or silver (0.5 kg) \\
10 gp & Silk (1m) or Cow (1) \\
15 gp & Saffron (0.5 kg) / ox (1) \\
30 gp & Cloves (1kg) \\
\end{tabular}

\medskip

See also the chapter on Dimensions in Motion and Transport.

\end{multicols}

\pagebreak

\section{Equipment - Weapons} \index{Equipment} \index{Weapons} \label{equipaggiamentoarmi}
\hypertarget{equipaggiamento.armi}{}

\label{equipaggiamento---armi}
\begin{changemargin}{0.3cm}{0.3cm} \begin{enfasi}{
This is my rifle. There are many like him, but this is mine. My shotgun is my best friend, it's my life. I have to master it how I dominate my life. Without me my rifle is nothing; without my rifle I'm nothing. I must know how to hit the target, I must shoot better than my enemy trying to kill me, I have to shoot I before he shoots me and I will. In the sight of God I swear by I believe this: my rifle and myself are the defenders of the homeland, we are the rulers of our enemies, we are the saviors of our own life and so be it, until there is no more enemy but only peace, amen. (Full Metal Jacket, 1987 Film)
\medskip

The really good sword is the one that remains in its scabbard. (Sanjuro)} \end{enfasi} \end{changemargin}

\medskip

I remember that using a Weapon without the proper proficiency requires a -1d6 to hit

The table presents the name of the weapon, its cost in gold coins, the damage and the type of damage (if from Slash, Botta or Spike), the range, the List of the belonging weapon and the special characteristics it can to have. \hyperref[sec: Special Actions in Combat]{See Special Actions in Combat section}, see also \hyperref[sec: Carrying-Capacity-and-Carrying-Bulk-Capacity]{Carrying and Carrying Capacity.}

\medskip

\textbf{Table: Weapons List} \index{Weapons List Table}

%\begin{tabularx}{lllll}
\begin{xltabular}{0.99\textwidth}{lllXl}
\textbf{Weapon} & \textbf{Cost} & \textbf{Size / Damage} & \textbf{Range, List, Special} & Weight (kg) \\
\toprule
Halberd & 10 & G / 1d10 P / T & \textbf{Lance}, \textbf{Rods}, Countercharge, Long Weapon, ED9 & 4 \\
Shortbow & 30 & M / 1d6 D & 15 yards, \textbf{Bow}, shooting & 1 \\
Short Composite Bow & notes * & M / Arrows & 20 meters, \textbf{Bow}, shooting & 1.5 \\
Longbow & 75 & G / Arrows & 20 meters, \textbf{Arco}, shooting & 2 \\
Long Composite Bow & notes * & G / Arrows & 36 meters, \textbf{Bow}, shooting & 2.5 \\
One-Handed Ax & 6 & M / 1d6 T & 6 meters, \textbf{Axes}, \textbf{Ranged Arms}, Versatile & 1 \\
Battle Ax & 10 & M / 1d10 T & \textbf{Axes} & 3 \\
Hammer Ax & 16 & M / 1d6 T / B & \textbf{Axes} & 3 \\
One-handed crossbow & 100 & M / Darts & 6 meters, \textbf{Crossbows}, shooting & 1 \\
Light Crossbow & 35 & P / Darts & 15 meters, \textbf{Crossbows}, \textbf{Simple Weapons}, shooting & 0.5 \\
Heavy Crossbow & 50 & G / Darts & 30 meters \textbf{Crossbows}, shooting & 3 \\
Staff & 3 & M / 1d6 B & \textbf{Simple Weapons}, Long Weapon, Versatile & 2 \\
Brandistocco & 10 & M / 2d4 P / T & \textbf{Lance}, Countercharge, Long Weapon & 3 \\
Spiked Chain & 25 & G / 2d4 P & \textbf{Spinning Balls}, Long Weapon & 4 \\
Estoc & 25 & G / 1d8 P & \textbf{Rods}, Long Weapon & 1.5 \\
Scythe & 18 & G / 2d4 P / T & \textbf{Weapons of Death}, Long Weapon & 3 \\
Sickle & 6 & P / 1d6 T & \textbf{Weapons of Death} & 1 \\
Falcione & 75 & M / 2d4 T & \textbf{Graceful Weapons}, \textbf{Lance}, ED7 & 2 \\
Pole Glaive & 12 & G / 1d10 P / T & \textbf{Lance}, Countercharge, Long Weapon, ED9 & 3 \\
Slingshot & - & P / 1d4 B & 10 meters, \textbf{Bows}, shooting & 0.5 \\
Scourge & 8 & M / 1d8 B & \textbf{Spinning Balls}, \textbf{Skull Breaker} & 3 \\
Double Scourge & 90 & M / 1d10 B & \textbf{Whirling Balls}, \textbf{Double Weapons} & 4 \\
Heavy Scourge & 15 & M / 1d10 B & \textbf{Whirling Balls}, \textbf{Dual Weapons} & 3 \\
Whip & 1 & M / 1d3 T & \textbf{Spinning Balls}, Long Weapon & 2 \\
Javelin & 1 & P / 1d6P & 12 meters, \textbf{Shafts}, \textbf{Missile weapons} \textbf{Simple weapons} & 1.5 \\
Great Double Ax & 25 & G / 1d12 T & \textbf{Axes}, \textbf{Dual Weapons}, Long Weapon & 4 \\
Big Club & 2 & M / 1d8 B & \textbf{Skull Breaker} & 2 \\
Spiked Gauntlet & 5 & P / 1d4 P & \textbf{Stunning Weapons} & 1 \\
Katana & 300 & M / 1d10 T & \textbf{Swords}, \textbf{Lethal Weapons}, ED9 & 1.5 \\
Lance & 10 & G / 1d8 P & \textbf{Lance}, Long Weapon, Countercharge & 3 \\
Infantry short spear & 1 & M / 1d6 P & 6 meters, \textbf{Missile weapons}, \textbf{Simple weapons}, \textbf{Shafts} & 1.5 \\
Foot Lance & 2 & M / 1d8 P & 6 meters, \textbf{Lance}, Long Weapon, Countercharge & 2 \\
Machete & 10 & M / 1d6 T & \textbf{Lethal Weapons} & 1 \\
Truncheon & 1 & P / 1d6 B & \textbf{Stunning weapons}, non-lethal & 0.5 \\
Warhammer & 5 & M / 1d8 B / P & 6 meters, \textbf{Skull Breaker} & 1.5 \\
Light Mace & 3 & P / 1d6 B / T & \textbf{Light Weapons}, \textbf{Skull Breaker}, \textbf{Simple Weapons} & 1 \\
Heavy Mace & 5 & M / 1d8 B / T & \textbf{Skull Breaker} & 2 \\
%\end{tabularx}
%\begin{xltabular}{0.99\textwidth}{lllXl}
%\textbf{Arma}&\textbf{Costo}&\textbf{Taglia/Danno} & \textbf{Gittata, Lista, Speciale} & Peso (kg)\\
%\toprule
Spiked Mace & 6 & M 1d8 B / P & \textbf{Skull Breaker}, \textbf{Simple Weapons} & 1 \\
Naginata & 8 & G / 1d10 T & \textbf{Lance}, Long Weapon, ED9 & 2 \\
Light Pike & 4 & M / 1d4 P & \textbf{Weapons of Death} & 1 \\
Heavy Pike & 8 & G / 1d6 P & \textbf{Weapons of Death}, Long Weapon & 3 \\
Dagger & 2 & P / 1d4 P & 6 meters, \textbf{Light weapons}, \textbf{Missile weapons}, \textbf{Simple weapons} & 0.5 \\
Punch / Bare Kick & Notes * & P / 1d4 B & Versatile & - \\
Club & 1 & P / 1d6 B & \textbf{Skull Breaker}, \textbf{Simple Weapons} & 0.5 \\
Scimitar & 15 & M / 1d6 T & \textbf{Light Weapons}, \textbf{Graceful Weapons}, Versatile & 1.5 \\
Two-Bladed Sword & 100 & G / 1d8 T & \textbf{Dual Weapons}, \textbf{Swords} & 3 \\
Bastard sword & 35 & M / 1d10 T & \textbf{Swords} & 2 \\
Short Sword & 10 & P / 1d6 P & \textbf{Light Weapons}, \textbf{Swords}, Versatile & 1 \\
Broadsword & 12 & M / 2d4 T & \textbf{Swords} & 1.5 \\
Longsword & 15 & M / 1d8 T & \textbf{Swords} & 1.5 \\
Two-Handed Greatsword & 50 & G / 2d6 T & \textbf{Swords} & 3 \\
Rapier & 20 & P / 1d6 P & \textbf{Light Weapons}, \textbf{Graceful Weapons}, Versatile & 1 \\
Trident & 15 & M / 1d6 P / T & 3 meters, \textbf{Rods}, \textbf{Missile weapons}, Long weapon, Countercharge & 2 \\
Urgrosh & 18 & M / 1d6 T / P & \textbf{Lance}, \textbf{Double Weapons} & 3 \\
\end{xltabular}


\bigskip

\textbf{Table: List of bullets - Bows - Crossbows - Slingshots} \index{Table List of projectiles - Bows - Crossbows - Slingshots}

\begin{tabular}{lccc}
\textbf{Bullet Name} & \textbf{Number of Shots / Cost (gp)} & \textbf{Damage / Type} & Weight (kg) \\
\toprule
Crossbow Bolts (one-handed, light) & 10/1 gp & 1d6 D & 0.1 \\
Crossbow Bolts (heavy) & 3/1 gp & 1d10 P & 0.3 \\
Hunting arrows (Shortbow, Longbow) & 20/1 gp & 1d6 HP & 0.1 \\
War arrows (longbow) & 10/1 gp & 1d8 HP & 0.2 \\
Marble Marbles (slingshots) & 15/1 gp & 1d4 B & 0.2 \\
Rock (slings) & - & 1d2 B & 0.2 \\
\end{tabular}

\medskip

\begin{multicols}{2}

A +1 weapon costs 1500 gp, +2 5000 gp. It is practically not possible to buy weapons with enchantments higher than +2, they must be "found".

An Arrow / Bolt / Magic Stone with a +1 bonus costs 25 gp, if +2 it costs 100 gp. Bullets with magic bonuses greater than +2 are virtually impossible to find.

\textbf{A projectile does not acquire magical properties because its caster is magical.}

\medskip

\textbf{Empty Fist}: \hyperlink{pugnovuoto}{vedi Lista d'Armi}

\bigskip

\textbf{Composite Arc} \index{Composite Arc}
A composite bow is a particularly strong and stiff bow that requires a certain minimum of Force to use.
A long composite bow has a fixed modifier, from +1 to +5. If the wearer has Strength greater than this modifier, he can apply a bonus equal to the composite bow modifier to the arrow damage.
A +3 composite bow used by a Strength 2 character can draw the bow not fully and therefore the arrow that leaves will have a +2 damage modifier.
A +1 composite bow used by a Strength 4 character will be shot fully but the damage modifier can be at most +1.

The cost of a composite bow depends on its modifier.
A composite bow with a +1 modifier costs 75 gp, +2 150 gp, +3 300 gp, +4 600 gp, +5 1500 gp. It is not possible to buy composite bows with bonuses greater than +3, they must be "found".

A composite shortbow has a maximum Strength modifier of +3.

\textbf{Crossbow} \index{Crossbows} \index{Crossbow reload}
A heavy crossbow requires two Actions to reload, allowing it to fire one bolt per round. A light or one-handed crossbow requires 1 action to reload.

\textbf{Range} \index{Range}
The distance indicated is that of a full attack roll. Each ranged weapon can strike within three times the indicated range.

If the target is within the indicated distance there is no penalty on hitting, if the target is between the first and second increase the penalty on hitting is -1d6. If the target is between the second and third increase, the penalty on hitting is -2d6.

A javelin thrown within 12 meters has no penalty, but thrown within 24 meters has a -1d6 to hit, at a distance between 24 and 36 meters a -2d6 to hit, beyond it cannot be thrown.

\medskip

\begin{center}
\includegraphics[width = 0.7 \linewidth]{immagini/bow2.png}
\end{center}

\medskip

A \textbf{Arrow or Bolt that hits is considered destroyed}, if missing it is considered to have a 50 \% (4-5-6 on a d6) probability that it is still intact.

A Magic Arrow / Bolt / Rock adds its bonuses to those of the caster to determine the attack roll and damage.

Remember that a normal projectile thrown by a magical caster does not become magical.

\medskip

The \textbf{Weapon Size} is indicated as P (small), M (medium), G (large). \hyperref[sec: Weapon too big]{See section Weapon too big}

A \textbf{larger weapon} \index{larger weapon} such as a Longsword forged for an Ogre increases its damage die by one category (1d4-1d6-1d8-1d10-2d6-2d8 -2d10 ..)


Weapons indicated a \textbf{Damage Type} \index{Damage Type}, ie T / B / P.
These letters indicate whether the damage is a Slash, Knock, or Puncture type. This trait can be important because certain creatures may be immune to or take less damage from a particular type of wound (eg a skeleton versus a penetrating weapon or a gelatinous cube versus a bladed weapon).

A weapon can be used to cause a different type of damage (slash to puncture or slash) by reducing the damage die by one category (eg Longsword to deal 1d6 slash damage).

\medskip


\textbf{Perfect Weapons} \index{Perfect Weapons}

A perfect weapon is a weapon created by a very skilled gunsmith who, although not magical, thanks to its perfect balance and sharpness, has a +1 to the attack roll.

A gunsmith must exceed the DC set for crafting the weapon by 10 to create a perfect weapon.

A perfect weapon costs twice as much as a normal weapon.

\textbf{Improvised Weapons} \index{Improvised Weapons}

Sometimes items that weren't meant to be weapons can have some combat effectiveness. Since they are not items intended for this use, the creature attacking with one takes a -1d6 penalty on the attack roll. A small improvised weapon (bottle) does 1d3 damage, medium size (a chair leg) to 1d6, large (a table leg) does 1d8 damage.

An improvised throwing weapon has a range of 10 feet.

\medskip

\textbf{Throwing weapons} \index{Throwing weapons}

A sword or in any case a weapon not made to be thrown can still be thrown at the opponent. The attack roll takes a -1d6 and the weapon does a lower damage category (the longsword does 1d6, a shortsword 1d4 ..). The range is 3 meters.

\medskip

\textbf{Using a Weapon without proper proficiency if it is not a Simple Weapon} Imposes a -1d6 attack roll.

\textbf{Example}: A small creature using a halberd in close combat has -1d6 because the weapon is large, -1d6 because it is not proficient, -1d6 because it uses the weapon in melee.

In this case, since the penalties are higher than 3d6, the character does not roll dice but only uses his Weapons Proficiency as a hit value.

\end{multicols}

\medskip

\begin{center}
\includegraphics[width = 0.6 \linewidth]{immagini/armiriempitivo3.png}
\end{center}


\pagebreak

\section{Equipment - Armor and Shields} \index{Armor} \index{Shields} \hypertarget{equipaggiamento.armature.scudi}{} \label{equipaggiamentoarmature}

\label{equipaggiamento---armature-e-scudi}

\begin{changemargin}{0.3cm}{0.3cm} \begin{enfasi}{
Armor (sf). Dress that you wear if your tailor is a blacksmith. (Ambrose Bierce)

Fantozzi armor: 4 winds weather vane as a plume, scary Viking helmet with zero visibility, subtracted bronze jockstrap to the statue of Pippin the Short and, at the feet, ironing irons a molten lead charcoal. Total weight of Fantozzi armor: 4 quintals, 32 kilos and 7 and a half pounds. (Superfantozzi, Film)} \end{enfasi} \end{changemargin} \medskip

\lettrine[lines = 2, lhang = 0.33, loversize = 0.25, findent = 1.5em]{A}{rmor} helps to be unaffected (raises Defense) and penalizes Magic Check and proficiency checks.

The Skills Penalty is the penalty that applies to skill checks affected by the weight and bulk of the armor. Different armor, specific or magical have different scores, this table serves as a guideline for the Storyteller.

\subsubsection{Armor Table} \index{Armor Table}

\label{tabella-armature}
\begin{tabular}{llllllll}
%\begin{xltabular}{0.95\textwidth}{lXXXXXXX}
\textbf{Armor} & \textbf{Cost (gp)} & \textbf{Defense} & \textbf{Malus Comp.} & \textbf{Type} & \textbf{Mov.} &\textbf{Magic Check} & \textbf{Weight (kg)} \\
\toprule
Padded & 5 & 1 & 0 & L & 0 & NO & 4 \\
Leather & 10 & 2 & 0 & L & 0 & SI & 5 \\
Reinforced leather & 25 & 3 & 0 & L & 0 & SI & 6.5 \\
Vest of Mail & 15 & 4 & -1 & M & 0 & + 1d6 & 10 \\
Scales & 50 & 5 & -1 & M & 0 & + 1d6 & 22.5 \\
Rings & 150 & 6 & -1 & M & 0 & + 1d6 & 20 \\
Bib & 200 & 6 & -2 & M & 0 & + 1d6 & 10 \\
Bands & 250 & 7 & -2 & P & 0 & + 2d6 & 30 \\
Half armor & 1200 & 8 & -2 & P & 1 & + 2d6 & 20 \\
da Campo & 1400 & 9 & -3 & P & 2 & + 2d6 & 25 \\
Complete & 1500 & 10 & -4 & P & 3 & + 2d6 & 32.5 \\
\end{tabular}

\medskip

\begin{multicols}{2}

\textbf{Cost}: This is for a medium size armor.

\textbf{Defense}: is the bonus given to Defense

\textbf{Malus Comp.}: Is the malus given to Competence checks given by the weight and bulk of the armor.

\textbf{Type}: indicates if the armature is \textbf{L} light, \textbf{M} edia or \textbf{P} esante

\textbf{Mov. (movement)}: is the reduction in meters of movement to be applied per Movement Action.

\textbf{Magic Check}: indicates whether the test is to be done (YES) or not (NO). If dice are indicated (+ 1d6, + 2d6 ..) it means that the Magic Check is to be done with the added dice marked. See Armor and Shields and Magic.

\textbf{Weight}: is the weight of the armor to be counted.

\textbf{Costs}: The cost of a +1 armor or shield is 2250th, +2 10000th. It is practically not possible to buy armor or shields or weapons with enchantments higher than +2, they must be "found".

\subsubsection{Armor, Shields and Magic} \index{Armor and Magic} \index{Shields and Magic} \hypertarget{armatureemagie}{}

All Armor and Shields, with the exception of Padded armor, force the caster to perform a Magic Check without considering any critical successes.

Light Armor and Shields make the Magic Check with no dice added, Medium ones with + 1d6, Heavy ones with + 2d6. In practice, the heavier the armor or shield, the more dice you roll, the more chances there are of a critical failure.

\medskip

\begin{changemargin}{0.3cm}{0.3cm} \begin{Storytellere} When counting the weight given by the armor and shield \textbf{worn} you must divide it by two (round up).

The weight marked for armor and shields is to be understood when it is "loaded in the backpack", or carried but not worn. \end{Storytellere} \end{changemargin}

\subsubsection{Armor Description}

\textbf{Light Armor}

Made of lightweight, flexible materials, lightweight armor favors agile adventurers as it offers protection without sacrificing mobility.

\textit{Padded}. Padded weaves consist of layers of fabric and padding sewn together.

\textit{Leather}. The bodice and shoulder pads of this armor are made of leather that has been hardened after being boiled in oil. The rest of the armor is composed of
softer and more flexible materials.

\textit{Reinforced Leather}. Made of tough yet flexible leather, the reinforced leather armor is embellished with rivets or spikes.

\medskip

\textbf{Medium Armor}

Medium armor offers more protection than light armor, but limits movement.

\textit{Giaco di Maglia}. Consisting of intertwined metal rings, a knitted jacket is worn over layers of clothing or leather. This type of armor offers modest protection to the upper body, while the sound of the rubbing rings is muffled by the other layers.

\textit{Scales}. This armor consists of a surplice and leggings (sometimes even a separate skirt) of leather covered by overlapping pieces of metal, similar to the scales of a fish. The armor is complete with gloves.

\textit{Rings}. This armor is leather armor with heavy rings sewn into it. The rings serve to reinforce the armor against sword and ax blows. The armor is complete with gloves.

\textit{Pectoral}. This armor consists of a metal bodice worn over a layer of leather. Although the armor leaves relatively bare arms and legs, the armor provides good protection for the character's vital organs, without taking up too much bulk.

\medskip

\textbf{Heavy Armor}

\textit{Bands}. This armor is made of metal strips sewn to a sturdy back of leather and iron mesh. The dimensions of the metal plates, interconnected to the metal bands, and the underlying layers of armor make it one of the most protective of the armors.

\textit{Half Armor}. Half plate armor consists of molded metal plates that cover much of the character's body. It does not include leg protectors other than simple greaves tied with leather laces.

\textit{from Campo}. Much like full armor but lighter in construction by sacrificing a little protection for greater flexibility and mobility.

\textit{Complete}. This armor consists of interlocking shaped metal plates that cover the entire body. A plate armor includes gloves, heavy leather boots, a visor helmet, and a thick layer of padding under the armor. Buckles and laces distribute the weight of the armor over the entire body.


\subsubsection{Basic rules for using reinforcement}

\textbf{Using Armor without proper proficiency} prevents the Dexterity bonus from being used and decreases the Defense bonus provided by 1.

\textbf{Using a shield without the proper proficiency} worsens the attack roll by 1 and decreases the defense bonus granted by 1.

\textbf{Sleeping in Armor}: If you sleep in medium or heavy armor, you are automatically \hyperlink{affaticato}{Affaticati} the next day.

Sleeping in light armor does not cause Fatigue.

Character's \textbf{movement allowance} will remain the same until banded armor then gradually decreases. The value indicated in the Mov. are the fewer meters the character does per move action.

For example a human in full armor has 6 meters movement, a dwarf 3 meters.

\textbf{Weight}: the indicated weight refers to the version for Medium-sized characters. Armor adapted for Small characters weigh half the weight, while Large armor weighs twice as much.

\textbf{Perfect Armor} \index{Perfect Armor}

A perfect armor is an armor created by a skilled blacksmith who, although not magical, thanks to its perfect balance and sharpening, has a +1 to Defense.

To craft a perfect weapon, a blacksmith must exceed the DC set for armor crafting by 10.

Perfect armor costs twice as much as normal armor. \\

\textbf{Magic armor} \index{Magic armor} \index{Magic shields}

A magical armor or magical shield not only protects better but is also lighter and akin to magic. A +2 armor lowers the Proficiency penalty by 1. A +3 armor / shield also removes 1 die from the Magic Check if added. A +4 armor lowers the Proficiency penalty by 2.


\begin{center}
\includegraphics[width = 0.9 \linewidth]{immagini/donnacavalierecavallo.png}
\end{center}

\subsubsection{The Shields}

The \textbf{Shields} \index{Shields} allow you to increase your Defense, the more massive and heavier the shield is, the more it protects, the more penalties on magical proficiency checks increase and the less it makes it easier to fight (hit penalty) .

Shields can be of Light, Medium, Heavy type.

\end{multicols}

\subsubsection{Shields Table} \index{Shields Table}

\label{tabella-scudi}

\begin{tabular}{lcccccc}
\textbf{Shields} & \textbf{Cost} & \textbf{Defense Bonus} & \textbf{Malus TC} & \textbf{Magic Check} & \textbf{Weight (kg)} &\textbf{Type} \\
\toprule
Buckler 		& 	5 gp 	& 1 	& 0 & SI 	& 1 & L \\
Light wooden shield & 	3 gp 	& 0 	& 0 & SI 	& 2 & L \\
Light Metal Shield & 	9 gp 	& 0 	& 0 & SI 	& 3 & L \\
Medium wood shield 	& 	5 gp 	& 2 	& 0 & + 1d6 & 3 & M \\
Medium metal shield 	& 	12 gp 	& 2 	& 0 & + 1d6 & 5 & M \\
Heavy wooden shield & 	7 gp 	& 3 & 1 & + 2d6 & 5 & P \\
Heavy metal shield & 	20 gp 	& 3 & 1 & + 2d6 & 7 & P \\
\end{tabular}

\medskip

\begin{multicols}{2}

\textbf{Defense Bonus}: This is the bonus that applies to Defense when taking up the shield.

\textbf{TC Penalty}: This is the penalty to attack roll when holding the shield.

\textbf{Weight (kg)}: is the weight expressed in kilograms. For the calculation of the size, a shield worn weighs half.

\textbf{Type}: indicates the size of the shield. \textbf{L} light, \textbf{M} edio, \textbf{P} clear.

A shield can be used as \textbf{improvised weapon}. The attack roll is penalized by -1d6 and a small shield does 2 damage (B / T), a medium shield does 1d4 damage (B / T), a heavy shield does 1d6 damage (B / T).

Using the shield as an improvised weapon does not apply its Defense bonus.

Wearing a shield occupies one hand / arm.

\subsubsection{Don and Remove Armor} \index{Don and Remove Armor}

Wearing and taking off armor is an operation that requires time and attention, doing it quickly does not help and indeed tends to worsen the protection given by the armor.

\end{multicols}

\textbf{Table: Times for putting on and taking off armor} \index{Time table for putting on and taking off armor}

\begin{tabular}{llll}
\textbf{Armor type} & \textbf{Put on} & \textbf{Put on quickly} & \textbf{Take off} \\
\toprule
Shield & 1 action & - & 1 action \\
Padded, Leather, Reinforced Leather & 1 minute & 3 rounds & - \\
Shirt Jacket & 1 minute & 5 rounds & 5 rounds \\
Scales, Rings, Breastplate, Bands & 4 minutes & 1 minute{*} & 1 minute \\
Half Armor, Field, Complete & 4 minutes{*}{*} & 4 minutes{*} & 1d4 + 1 minutes \\
\end{tabular}

\bigskip

\begin{multicols}{2}

{*} If someone helps, time is halved. A single character doing nothing else can help one or two adjacent characters. Two characters cannot help each other wear armor at the same time.

{*}{*} You need help to put on this armor. Without help, you can only put it on quickly.

\textbf{Put on armor quickly} implies a -1 penalty on Defense and an additional +1 penalty on proficiency checks.

\end{multicols}

\vfill

\begin{center}
\includegraphics[width = 0.3 \linewidth]{immagini/armaturacorpetto.png}
\end{center}

\pagebreak


\section{Goods and Services} \index{Goods} \index{Services}


\subsection{Wealth and Money} \index{Wealth and Money}


\begin{changemargin}{0.3cm}{0.3cm} \begin{enfasi}{
- Doc ... we just need a little bit of plutonium.

- Ah, I'm sure that in 85 plutonium is bought in the grocery store below home, but in '55 the matter is much more complicated! (Back to the Future, 1985 movie)
}
\end{enfasi} \end{changemargin} \medskip

\begin{multicols}{2}

\subsubsection{Selling Treasures}

\lettrine[lines = 2, lhang = 0.33, loversize = 0.25, findent = 1.5em]{I}{n} dungeons you explore will have ample opportunities to find treasures, gear, weapons, armor and more. Usually, you will be able to sell treasures and trinkets when you reach a town or other settlement, as long as you can find buyers and merchants interested in your loot.

\medskip

\textbf{Weapons, Armor and Other Equipment}

As a general rule, weapons, armor, and other unharmed equipment cost half as much when sold. The weapons and armor used by monsters are unlikely to be in prime condition for sale.

\medskip

\textbf{Magic Items}

Selling magical items is a problem. Finding someone who wants to buy a potion or parchment does not involve much difficulty, but most of the items are out of reach of anyone's pockets except the wealthiest nobles. Also, apart from a few common magical items, it is difficult to find magical items or spells for sale. The value of magic stirs the vile coin and should always be treated with consideration.

\medskip

\textbf{Gems, Jewels and Works of Art}

These items retain their full value on the market, and you can decide to exchange them for money or use them as currency in transactions. In the case of treasures of exceptional value, the Storyteller may request that you first be able to find a buyer in a large country or even a larger community.

\medskip

\textbf{Goods}

On border lands, most transactions are done through bartering. Like gems and objets d'art, commodities - iron ingots, sacks of salt, livestock, and so on - can be exchanged for currency at their full value.

\medskip

\end{multicols}

\vfill

\begin{center}
\includegraphics[width = 0.7 \linewidth]{immagini/jewelry-box-2931784_1280.png}
\end{center}

\pagebreak

\begin{multicols}{2}

\subsubsection{Adventure Gear}

This is a short, non-exhaustive list of equipment your characters may be interested in buying. The list is certainly not exhaustive or complete but it can provide you with pricing guidelines.

As a Storyteller, always use common sense in requests, carefully evaluate the type of request, the need for the object, the place where you buy it and how you buy it.

Depending on the type of companion, additional objects such as firearms or alchemicals may be available.

{\small
\begin{tabularx}{0.42\textwidth}{lll}
\textbf{Object} & \textbf{Cost} & \textbf{Weight} \\
\toprule
Abacus & 2 gp & 1 kg \\
Dress, Common & 5 ma & 1.5 kg \\
Dress, Costume & 5 gp & 2 kg \\
Dress, Prized & 15 gp & 3 kg \\
Suit, Traveler & 2 gp & 2 kg \\
Steel and Flint & 5 ma & 0.5 kg \\
Acid (vial) & 25 gp & 0.5 kg \\
Holy Water (ampoule) & 25 gp & 0.5 kg \\
Cruet or jug & 2 mr & 0,5 kg \\
Amulet & 5 gp & 0.5kg \\
Signet Ring & 2 gp & - \\
Antitoxin (vial) & 50 mo & \\
Portable Ram & 4 gp & 17.5 kg \\
Shaft (3 meters) & 5 mr & 3.5 kg \\
Climber Tools & 25 gp & 6 kg \\
Burglary Tools & 25 gp & 1kg \\
Yew Wood Stick & 10 gp & 0.5 kg \\
Wand & 10 gp & 0.5 kg \\
Bandolier & 1 gp & 0.5 kg \\
Barrel & 2 gp & 35 kg \\
Wooden staff & 5 gp & 2 kg \\
Stick & 5 gp & 2 kg \\
Merchant's Scale & 5 gp & 1.5 kg \\
Healer's Bag & 5 gp & 1.5 kg \\
Bag for Parts & 25 gp & 1 kg \\
Bag & 5 ma & 0.5 kg \\
Glass Bottle & 2 mo & 1 kg \\
Jug or Carafe & 2 mr & 2 kg \\
Campanella & 1 mo & - \\
Candela & 1 mr & - \\
Canister & 4 ma & 1 kg \\
Fishing rod & 1 mo & 2 kg \\
Scope & 1000 gp & 0.5 kg \\
Pulley and Hoist & 1 m or 2.5 kg \\
Paper (one sheet) & 2 ma & - \\
Chain (3 meters) & 5 gp & 5 kg \\
Waxed & 5 cm & 1kg \\
Rocker Nail & 5 mr & 0.1 kg \\
Hourglass & 25 gp & 0.5 kg \\
Blanket & 5 ma & 1.5 kg \\
Hemp Rope (15 meters) & 1 mo & 5 kg \\
Silk Rope (15 meters) & 10 gp & 2.5 kg \\
Whetstone for Sharpening & 1 mr & 0.5 kg \\
Crystal & 10 gp & 0.5 kg \\
Crossbow Bolt Case & 1 gp & 0.5 kg \\
Emblem & 5 gp & - \\
Quiver & 1 gp & 0.5 kg \\
Vial & 1 mo & - \\
Lure Whistle & 5 mr & - \\
\end{tabularx}

\begin{tabularx}{0.42\textwidth}{lll}
\textbf{Object} & \textbf{Cost} & \textbf{Weight} \\
\toprule
Chest & 5 gp & 12.5 kg \\
Arrows (20) & 1 gp & 0.5 kg \\
Alchemical Fire (ampoule) & 50 gp & 0.5 kg \\
Metal hook & 1 mo & 1.5 kg \\
Chalk (1 piece) & 1 mr & - \\
Bed & 1 mo & 3.5 kg \\
Globe & 20 mo & 1.5 kg \\
Ink (30 grams) & 10 gp & \\
Lamp & 5 ma & 0.5 kg \\
Shielded Lantern & 5 gp & 1 kg \\
Protruding Lens Lantern & 10 gp & 1 kg \\
Magnifying Glass & 100 gp & \\
Spellbook & 50 gp & 1.5 kg \\
Book & 25 gp & 2.5 kg \\
Handcuffs & 2 gp & 1 kg \\
Demolition Hammer & 2 gp & 8 kg \\
Hammer & 1 gp & 1.5 kg \\
Oil (ampoule) & 1 ma & 0.5 kg \\
Skinny & 2 m & 2.5 kg \\
Nib & 2 mr & \\
Parchment (one sheet) & 1 ma & \\
Miner's Pick & 2 gp & 5 kg \\
Crowbar & 2 gp & 2.5 kg \\
Map / Scroll Holder & 1 gp & 0.5 kg \\
Healing Potion & 50 gp & 0.25 kg \\
Perfume (vial) & 5 gp & - \\
Sprig of Mistletoe & 1 mo & - \\
Grapple & 2 gp & 2 kg \\
Rations (1 day) & 5 ma & 1 kg \\
Reliquary & 5 gp & 1 kg \\
Bag & 1 mr & 0.25 kg \\
Soap & 5 mr & - \\
Ladder (3 meters) & 1 m & 12.5 kg \\
Bucket & 5 mr & 1 kg \\
Lock & 10 gp & 0.5 kg \\
Metal Balls (100) & 3 gp & 1 kg \\
Small Metal Mirror & 5 gp & 0.25 kg \\
Spikes, Iron (10) & 1 gp & 2.5 kg \\
Cutter & 5 m or 12.5 kg \\
Tent, for Two Person & 2m or 10kg \\
Torch & 1 mr & 0.5 kg \\
Tribbles (bag of 20) & 1 gp & 1 kg \\
Spade or Shovel & 2 gp & 2.5 kg \\
Iron Pot & 2 gp & 5 kg \\
Poison, base (vial) & 100 gp & - \\
Rod & 10 gp & 1 kg \\
Vest & 1 mo & 2 kg \\
Backpack & 2 mo & 2.5 kg \\
\end{tabularx}
}
\medskip


\begin{changemargin}{0.3cm}{0.3cm} \begin{enfasi}{
Any sufficiently advanced technology is indistinguishable from magic. (Arthur C. Clarke, from Profiles of the Future)
} \end{enfasi} \end{changemargin}

\textbf{Acid}. With one action, you can spread the contents of this vial onto a creature within 1 meter of you or throw the vial up to 20 feet, smashing it on impact. Either way, make a ranged attack roll against the creature or object, treating the acid as an improvised weapon (-1d6 attack roll). If you hit, the target takes 2d6 points of acid damage.

\textbf{Sacred Water}. With one action, you can spread the contents of this vial onto a creature within 1 meter of you or throw the vial up to 20 feet, smashing it on impact. Either way, make a ranged attack roll against the creature or object, treating the holy water as an improvised weapon. If you hit, and the target is a fiend or undead, it takes 2d4 points of positive energy damage.

\textbf{Antitoxin}. A creature drinking from this vial of liquid gains + 1d6 on saving throws against the poison for 1 hour. It does not grant any bonuses to undead and constructs.

\textbf{Portable Aries}. You can use a portable ram to break down doors. As you do so, you gain a + 1d6 bonus on Strength checks. Another character can help you with the use of the ram, giving you +2 on the check.

\textbf{Fishing Tackle}. This kit includes a wooden rod, silk thread, wooden cutter, steel hooks, lead weight, velvet lures and a net.

\textbf{Bandolier}. This specialized belt for holding small items such as potions or scrolls is worn around the neck. Taking an object from it costs 1 Action, as if it were on the belt.

\textbf{Metal Marbles}. With one action, you can scatter a single bag of these tiny metal marbles to cover a 10-foot square flat area on each side. A creature crossing the covered area must succeed on a DC 12 Reflex saving throw or fall prone. A creature crossing the area at half speed does not have to make a saving throw.

\textbf{Merchant's Scales}. A merchant scale includes a small barbell, a plate, and an assortment of weights up to 1 pound. With it, you can measure the exact weight of small objects, such as precious metals or goods, to help you determine their value.

\begin{center}
\includegraphics[height = 0.5 \linewidth]{immagini/stadera.png}
\end{center}

\textbf{Exchange of Components}. A component bag is a small waterproof leather belt bag with compartments containing all the material components and other special items you need to cast your spells, except for those components that have a specific cost or are uncommon materials ( as indicated in the spell description).

\textbf{Bag}. A fabric or leather pouch can hold, among other things, up to 20 slingshot bullets or 50 blowgun needles. A pouch that is divided into compartments to hold spell components is called a component bag.

\textbf{Candela}. For 1 hour, a candle casts light within a radius of 1.5 meters and dim light for an additional 1.5 meters.

\textbf{Cerata}. It is a coat treated to be water repellent, allowing you to stay dry even in the rain.

\textbf{Telescope}. Objects observed through a telescope are magnified to double their size.

\textbf{Pulley and Hoist}. A series of levers connected by a cable and a hook for attaching to objects, pulley and hoist allow you to pull up to four times the
weight you can normally lift.

\textbf{Chain}. A chain has 15 hit points and hardness 6. It can be broken by making a Strength check with DC 24.

\textbf{Rope}. A rope, whether made of hemp or silk, has 2 hit points and can be broken by making a Strength check with DC 19.

\textbf{Quiver}. A quiver can hold up to 12 arrows.


\begin{center}
\includegraphics[width = 0.5 \linewidth]{immagini/forziere.png}
\end{center}


\textbf{Alchemical Fire}. This sticky fluid ignites when it comes in contact with air. With two actions, you can launch this ampoule up to 6 meters, smashing it on impact. Make a ranged attack roll against the creature or object, treating the alchemical fire like an improvised weapon. If you hit, the target takes 1d6 points of fire damage at the start of each of its rounds. A creature can end this damage by spending two Actions and making a DC 12 Dexterity check. If successful, the flames are extinguished.

\textbf{Healer's Kit}. This kit is a leather bag containing bandages, ointments and splints. The kit can be used ten times. Grants +2 first aid checks.

\textbf{Dining Kit}. This small tin box contains a bowl and simple cutlery. The two parts of the box can be detached, one side used as a cooking pot and the other as a plate or container

\textbf{Climber's Kit}. A climber's kit includes special nails, boot tips, gloves and a harness. You can anchor yourself using the climbing kit with an action; when you do, you cannot drop more than 7 meters from where you anchored, and you cannot climb more than 7 meters from where you anchored without first undoing the anchor.

\textbf{Lantern}. A lamp projects bright light in a 4 meter radius and dim light for an additional 9 meters. Once lit, it burns for 6 hours with an ampoule (0.5 liters) of oil.

\begin{center}
\includegraphics[width = 0.6 \linewidth]{immagini/lanterna.png}
\end{center}

\textbf{Protruding Lens Lantern}. A projecting lens lantern casts light into an 18 meter cone and dim light for an additional 18 meters. Once lit, it burns for 6 hours with an ampoule (0.5 liters) of oil.

\textbf{Screenable Lantern}. A shaded lantern casts light in a radius of 9 meters and dim light for an additional 9 meters. Once lit, it burns for 6 hours with an ampoule (0.5 liters) of oil. With one action, you can lower the shield, reducing the light to dim with a radius of 1 meter.

\textbf{Magnifying Glass}. This lens allows you to take a closer look at small objects. It is also a useful substitute for flint and flint in starting a fire. Starting a fire with the magnifying glass requires at least solar light, wood to light, and about 5 minutes for the wood to catch fire. A magnifying glass provides help (+ 1d6) in any check made to evaluate or analyze a small or very detailed object.

\textbf{Book}. A book can contain poems, historical accounts, information pertinent to a specific field of knowledge, diagrams and notes on gnome devices, or anything else that can be represented using text and drawings. A book containing spells is called a tome of magic.

\textbf{Tome of Spells}. Essential to spellcasters, a spellbook is a leather-bound tome with 100 blank fleece pages suitable for recording spells.

\textbf{Handcuffs}. These metal tools can imprison a Small or Medium creature. To get rid of the handcuffs you must pass a Dexterity check with DC 24. To break them you must pass a Strength check with DC 24. Each set of handcuffs comes with a key. Without the key, a creature can use Escape Artist or Disable Gadgets to unlock the lock on a DC 18 check. Handcuffs have 15 Hit Points and Hardness 2.

\textbf{Oil}. It is usually bought in a clay ampoule that holds 0.5 liters. With an action, you can sprinkle the oil in this cruet on a creature within 1 meter of you or throw it up to 20 feet, smashing it on impact. Either way, make a ranged attack roll against the creature or object, treating the oil like an improvised weapon. If you hit, the target is covered in oil. If the target takes any amount of fire damage before the oil dries (after 1 minute), the target takes an additional 5 fire damage from the burning oil. You can also pour a cruet of oil on the floor to cover a 1-meter square, as long as the surface is flat. If inflamed, the oil burns for 2 rounds and deals 5 Fire damage to any creature that enters the area or ends its round within it. A creature can only take this damage once per turn.

\textbf{Crow's foot}. Using a crowbar gives + 1d6 to Strength checks whenever the crowbar lever can be applied.

\textbf{Rations}. The rations consist of dry food suitable for long journeys, and include dried meat, dried fruit, biscuits and nuts.

\textbf{Box with the Tinder}. This small container contains stone, flint and bait (usually a dry rag soaked in oil) used to start a fire. Using it to light a torch (or any other easily ignitable object) requires two actions. Lighting any other fire takes 1 minute.

\textbf{Box for Maps or Scrolls}. This cylindrical leather box can hold up to ten pieces of paper or five sheets of parchment rolled up.

\textbf{Box for Crossbow Quadrelli}. This wooden box holds up to 12 bolts per crossbow.

\textbf{Lock}. A key is supplied with the lock. Without the key, a creature can pick this lock by making a Deactivate Devices check with DC 17. The Storyteller may decide that higher quality locks are available for larger amounts.

\begin{center}
\includegraphics[width = 0.6 \linewidth]{immagini/serratura.png}
\end{center}

\textbf{Sacred Symbol}. A sacred symbol is the depiction of a Patron. It could be an amulet depicting a Patron's symbol, the same symbol carefully engraved or woven onto an emblem or shield, or a tiny box containing a sacred relic.

\textbf{Tent}. A simple portable canvas shelter, a tent can hold two people.

\textbf{Torch}. A flashlight burns for 1 hour, providing light within a 6 meter radius and dim light for an additional 6 meters. If you make an attack roll with a burning torch and hit, you deal 14d damage plus 1 additional fire damage. A Torch used as a weapon is an improvised smash weapon that does 1d4 + 1d4 fire damage.

\textbf{Hunting Trap}. You use two actions to set this trap, formed by a serrated steel ring, which clicks when a creature steps on the metal plate in the center of it. The trap is attached by a heavy chain to an immobile object, such as a tree or spike driven into the ground. A creature that steps on the plate must succeed on a DC 15 Reflex save or take 1d4 points of piercing damage and stop movement. After that, until the creature breaks free from the trap, its movement is limited by the length of the chain (usually 90 centimeters long). A creature can use 2 actions to make a DC 15 Strength check, and if it succeeds, it frees or frees another creature in range. Each failed attempt deals 1 piercing damage to the trapped creature.

\textbf{Tribolo}. With one action, you can spread a single bag of these tiny tribbles to cover a square area of 1 meter on each side. A creature crossing the covered area must succeed at a DC 17 Reflex save or take 1 piercing damage. Until the creature regains at least 1 hit point, its walking speed has decreased by 10 feet. A creature crossing the area at half speed does not have to make a saving throw.

\begin{center}
\includegraphics[width = 0.6 \linewidth]{immagini/tribolo.png}
\end{center}

\textbf{Basic Poison}. You can use the poison in this vial to cover a slashing or piercing weapon or up to three pieces of ammo. Applying the poison requires action. A creature struck by a poisoned weapon or ammunition must succeed on a DC 12 Fortitude save or take 1d4 points of poison damage.
Once applied, the poison remains effective for 1 minute before drying out.

\subsubsection{Equipment}
The starting equipment a character receives at the first level depends on the profession and also includes a set of adventurer items. The contents of each bundle are listed below. If the character chooses to purchase his starter gear, he can purchase an endowment at the listed price, which is generally cheaper than buying the separate items individually.

\textbf{Adventurer's Pack (12 gp)}. Includes a backpack, crowbar, hammer, 10 climbing nails, 10 torches, flint and flintlock, 10 daily rations and a wineskin. The equipment also includes 15 meters of hemp rope tied to the backpack.

\begin{center}
\includegraphics[width = 0.7 \linewidth]{immagini/zaino.png}
\end{center}

\textbf{Hunter's Pack, 18 gp}: contains flint and flint, belt bag, 18m rope, bed, oilcloth, wineskin, iron pot, travel rations (5 days), torches (10) and a backpack.

\textbf{Diplomat's Endowment (39 gp)}. Includes a chest, 2 map and scroll cases, a fine dress, an ink bottle, a nib, a lamp, 2 oil ampoules, 5 sheets of paper, a perfume vial, seal wax and soap.

\textbf{Devotee's Pack (20 gp)}: Contains flint and flint, belt bag, Spell Parts Bag, candles (10), 18m rope, bed, iron pot, skin, travel rations (for 5 days), soap, a wooden sacred symbol, an inexpensive sacred text, torches (10) and a backpack.

\textbf{Scout's Equipment (10 gp)}. It includes a backpack, a bed, a mess tin, a flint and a flint, 10 torches, 10 daily rations and a wineskin. The equipment also includes 15 meters of hemp rope tied to the backpack.

\textbf{Cave Explorer Pack, 18 gp}: Contains a set of basic tools for exploring abandoned ruins and cities includes 2 candles, chalk, hammer and 4 Rock Climbing Nails, 18 meters of rope, a shield lantern with 5 ampoules of oil, 2 bags, 2 torches, travel rations (for 3 days)

\textbf{Entertainer's Pack (40 gp}). Includes backpack, bedding, 2 costumes, 5 candles, 5 daily rations, skin bottle and camouflage makeup.

\textbf{Burglar's Pack (16 gp)}. Includes a backpack, a bag with 1000 metal balls, 3 meters of string, a bell, 5 candles, a crowbar, a hammer, 10 climbing nails, a shielded lantern, 2 ampoules of oil, 5 daily rations, a flintlock and flint and a wineskin. The equipment also includes 15 meters of hemp rope tied to the backpack.

\textbf{Scholar's Endowment (40 gp)}. Includes backpack, study book, ink bottle, nib, 10 parchment sheets, sandbag and pocket knife.

\end{multicols}

\subsubsection{Capacity of Containers}

\begin{tabularx}{0.95\textwidth}{lX}
\textbf{Container} & \textbf{Capacity} \\
\toprule
Cruet or Mug & 0.5 liter \\
Barrel & 160 liquid liters, 4 cubes with 30 cm edge \\
Bag & 1 cube with 10 cm edge / 3 kg of equipment \\
Bottle & 1 liter of liquid \\
Jug or Carafe & 4 liquid liters \\
Basket & 2 cubes with 30 cm edge / 20 kg of equipment \\
Vial & 120ml liquid \\
Chest & 12 cubes with 30 cm edge / 150 kg of equipment \\
Bottle & 2 liquid liters \\
Bag & 1 cube with 30 cm edge / 15 kg of equipment \\
Bucket & 12 liters of liquids, 1 cube with edge of 25 cm \\
Iron Jar & 4 liquid liters \\
Backpack * & 1 cube with 30 cm edge / 15 kg of equipment \\
\end{tabularx}

\medskip

\begin{multicols}{2}

\subsubsection{Tools}

The list of tools presented helps the characters to perform the checks related to their professions.

Trials related to professions are usually related to Wisdom.

For example, a "Calligraphy" check is resolved with a Wisdom check, if the character has the right tools available ("\textit{Calligraph's Supply}") he gets a +2 bonus on the check.

If the character has to make a check on what is his profession this will be done with a bonus equal to half the level of the character, if he also has the tools available, he tries a further bonus of +2.

\end{multicols}

\medskip

\begin{tabularx}{0.95\textwidth}{llX | llX}
\textbf{Object} & \textbf{Cost} & \textbf{Weight} & \textbf{Object} & \textbf{Cost} & \textbf{Weight} \\
\toprule
Burglary / Forger Tools & 25 gp & 0.5 kg & Herbalist's Bag & 5 gp & 1.5 kg \\
Dice & 1 but & - & Deck of Cards & 5 but & - \\
Dragon Chess & 1 gp & 0.25 kg & Three Dragons in the Dark & 1 gp & \\
Poisoner's Substances & 50 gp & 1 kg & Alchemist's Supply & 50 gp & 4 kg \\
Calligrapher's Supply & 10 gp & 2.5kg & Shuffler's Supply & 20gp & 4.5kg \\
Shoemaker Tools & 5 gp & 2.5 kg & Cartographer Tools & 15 gp & 3 kg \\
Tanner's Tools & 5 gp & 2.5 kg & Builder's Tools & 10 gp & 4 kg \\
Blacksmith Tools & 20 gp & 4 kg & Carpenter's Tools & 8 gp & 3 kg \\
Jeweler's Tools & 25 gp & 1 kg & Carver's Tools & 1 gp & 2.5 kg \\
Inventor's Tools & 50 gp & 5 kg & Painter's Tools & 10 gp & 2.5 kg \\
Blower Tools & 30 gp & 2.5 kg & Weaver Tools & 1 gp & 2.5 kg \\
Potter's Tools & 10 gp & 1.5 kg & Chef's Tools & 1 gp & 4 kg \\
Navigator's Tools & 25 gp & 1 kg & Shawm & 2 gp & 0.5 kg \\
Bagpipe & 30 gp & 3 kg & Horn & 3 gp & 1 kg \\
Dulcimer & 25 gp & 5 kg & Flute & 2 gp & 0.5 kg \\
Pan Flute & 12 gp & 1 kg & Lyre & 30 gp & 1 kg \\
Lute & 35 gp & 1 kg & Drum & 6 gp & 1.5 kg \\
Purple & 30 gp & 0.5 kg & Camouflage Tricks & 25 gp & 1.5 kg \\
\end{tabularx}

\begin{multicols}{2}

\medskip

\subsubsection{Mounts and Vehicles}

A good mount can allow a character to quickly traverse wild territory, but its primary purpose is to carry equipment that would otherwise slow its master.

The "Mounts and Other Animals" table shows the basic speed and carrying capacity of each animal. An animal pulling a chariot, cart, wagon, carriage, or sled can move a weight equal to five times its carrying capacity, including the weight of the vehicle. If multiple animals pull the same vehicle, they can add their carrying capacity together.

There are other mounts in fantasy worlds than those listed in this section, but these are rare mounts that are not normally available for purchase, such as certain flying mounts (pegasi, griffins, hippogriffs and other similar animals) or even some mounts aquatic (such as giant seahorses).

To get hold of such a mount often requires stealing an egg and raising the creature in person, entering into a pact with a powerful entity, or negotiating with the mount itself.

\textbf{Harness}. A harness is armor designed to protect an animal's head, neck, chest, and body. Each type of armor listed in the "Armor" table in this chapter can be purchased as a harness. The cost is four times the equivalent armor made for humanoids, while the weight is double.

\begin{center}
\includegraphics[width = 0.7 \linewidth]{immagini/bardatura.png}

\textit{Full headband}
\end{center}


\textbf{Saddle}. A rider can clip onto a military saddle to remain in place on an active mount during the course of a battle. A military saddle gives advantage to the checks a character makes to stay in the saddle. An exotic saddle is required to ride an aquatic or flying creature.


\begin{center}
\includegraphics[height = 0.7 \linewidth]{immagini/sella2.png}
\end{center}


\textbf{Rowing Boats}. Barges and rowing boats are usually used on lakes and rivers. If a boat goes with the current, the speed of the current (typically 4.5 km per hour) is added to its speed. Generally it is not possible to row against the current if the current has a significant intensity, but it is possible to make these boats go up a stream by bringing them to shore and having them towed by one or more beasts of burden. A rowboat weighs 50 kg in case adventurers have to transport it by land.

\subsubsection{Mounts and Other Animals}


\begin{tabular}{lllll}
\toprule
\textbf{Mount} & \textbf{Cost} & \textbf{Movement} & \textbf{Load} & Km / h \\
& (\textbf{mo}) && kg & \\
Donkey or Mule & 8 & 12 m & 210 & 6km \\
Camel & 50 & 15m & 240 & 8km \\
Gallop Horse & 75 & 18m & 240 & 12km \\
Warhorse & 400 & 18m & 270 & 9km \\
Draft Horse & 50 & 12m & 270 & 6km \\
Elephant & 200 & 12m & 660 & 6km \\
Mastiff & 25 & 12m & 97.5 & 6km \\
Pony & 30 & 12m & 112.5 & 6km \\
\end{tabular}

\bigskip

\textbf{Harnesses and Shooting Vehicles} \\
\begin{tabularx}{0.45\textwidth}{llX}
\toprule
\textbf{Object} & \textbf{Cost} & \textbf{Weight} \\
Headband & x4 & x2 \\
Chariot & 250 gp & 50 kg \\
Saddlebags & 4 mo & 4 kg \\
Cart & 15 gp & 100 kg \\
Wagon & 35 gp & 200 kg \\
Carriage & 100 gp & 300 kg \\
Bit and Bridle & 2 gp & 0.5 kg \\
Nutrition (per day) & 5 mr & 5 kg \\
\end{tabularx}

\bigskip

\textbf{Saddle} \\
\begin{tabularx}{0.45\textwidth}{llX}
\toprule
\textbf{Object} & \textbf{Cost} & \textbf{Weight} \\
From Cargo & 5m or 7.5kg \\
Gallop & 10 gp & 12.5 kg \\
Exotic & 60 gp & 20 kg \\
Military & 20 gp & 15 kg \\
Sled & 20 gp & 150 kg \\
Stable (per day) & 5 months & \\
\end{tabularx}

\bigskip

\textbf{Boats} \\
\begin{tabularx}{0.45\textwidth}{llX}
\toprule
\textbf{Object} & \textbf{Cost} & \textbf{Speed} \\
Rowboat & 50 gp & 2.25 km per hour \\
Barcone & 3000 mo & 1.5 km per hour \\
Galea & 30,000 gp & 6 km per hour \\
Sailing Ship & 10,000 gp & 3km per hour \\
Warship & 25,000 gp & 3.75 km per hour \\
Longship & 10,000 gp & 4.5 km per hour \\
\end{tabularx}


\subsubsection{Expenses}
When not descending into the bowels of the earth, exploring ruins in search of lost treasures, or waging war on the forces of looming darkness, even adventurers must think about the most common needs. Even in a fantastic way, people have to meet basic needs like food, shelter and clothing. All of this comes at a cost, even if some lifestyles cost more than others.

\medskip

\textbf{Lifestyle Expenses}


Lifestyle Expenses are an easy way to account for the costs of living in a fantasy world. They cover accommodation, food, drink and all other essential needs of a character. These expenses also cover the maintenance cost of the character's equipment, to allow him to be ready when the next adventure call comes. At the beginning of each week or month (player's choice), each character chooses a lifestyle from the "Lifestyle Expenses" table and pays the required price to maintain that lifestyle. The prices listed are per day, so anyone wishing to calculate their living cost over a period of thirty days will have to multiply the indicated price by 30. A character can change their lifestyle from one period to another, depending on the funds a disposition, or he can maintain the same lifestyle throughout his career.

Lifestyle choice can have consequences. A character who maintains a rich lifestyle can make contacts with the rich and powerful more easily, but runs the risk of attracting some thieves. Similarly, a poor lifestyle can help him avoid criminals, but it is unlikely that it will allow him to make important contacts.

\bigskip

\textbf{Lifestyle Expenses}

\medskip

\begin{tabular}{ll}
Lifestyle & Price per Day \\
\toprule
Miserable & - \\
Shabby & 1 but \\
Poor & 2 but \\
Modesto & 1 mo \\
Affluent & 2 mo \\
Rich & 4 gp \\
Aristocrat & Minimum of 10 gp \\
\end{tabular}

\bigskip

\textbf{Miserable}. The character lives in inhumane conditions. He has no place he can call home and takes shelter where he can, sneaking into a barn, snuggling up in an old chest or relying on the good heart of someone luckier than him. A miserable lifestyle presents dangers in abundance. Violence, disease and hunger follow the character wherever he goes. The other wretches might set their sights on his armor, weapons, and adventurer gear, which are lucky by their parameters. Most people don't take the character into consideration at all.

\textbf{Shabby}. The character lives in a drafty barn, a mud-floored hut located just outside the village or in a flea-filled hostel in the worst neighborhood in town. It benefits from minimal shelter from the elements, but lives in a desperate and often violent environment, in places plagued by disease, hunger and misfortune. Most people don't take it into consideration at all and the law protects it little or nothing. Most people who lead this lifestyle are marked by some terrible misfortune: branded as exiles, suffering from a mental disorder or illness of some kind.

\textbf{Poor}. A poor lifestyle means having to get by without the comforts available in a stable community. Basic food and accommodation, poor quality clothing and unpredictable living conditions result in a lifestyle perhaps sufficient to survive, but certainly not very pleasant. The character sleeps in a hostel or in a common room on the first floor of a tavern. It benefits from a minimum of legal protection, but still has to contend with acts of violence, crime and disease. Unspecified laborers, junkies, beggars, thieves, mercenaries, and other disreputable figures tend to adopt this lifestyle.

\begin{center}
\includegraphics[width = 0.7 \linewidth]{immagini/mendicante.png}

\textit{Beggar - Francesco Londonio}
\end{center}

\textbf{Modest}. A modest lifestyle keeps a character out of the slums and allows him to take care of his equipment. The character lives in an old part of the city, has a rented room in a guesthouse, an inn or a temple. He is not hungry or thirsty and lives in a clean, albeit spartan, environment. Common individuals who lead modest lifestyles include soldiers with a family, laborers, students, priests, amateur charmers, and so on.

\begin{center}
\includegraphics[width = 0.8 \linewidth]{immagini/mercante.png}
\end{center}


\textbf{Affluent}. A character who is able to adopt a comfortable lifestyle can afford quality clothing and take care of their equipment without difficulty. He lives in a house on a well-known block or has a private room at a quality inn. Hang out with merchants, skilled craftsmen, and military officers.

\textbf{Rich}. A character who adopts a rich lifestyle lives in luxury, even if he has perhaps not achieved the social prestige associated with the old values of nobility and royal blood. He leads a lifestyle comparable to that of a highly successful merchant, an esteemed servant of a royal house, or the owner of some small business. Stay in a respectable abode, usually a spacious house in a respectable part of town or a comfortable apartment at a reputable inn. He is likely assisted by a small group of servants.


\begin{center}
\includegraphics[width = 0.6 \linewidth]{immagini/lucullo.png}

\textit{Lucius Licinius Lucullus. Rome, 117 BC, Naples 56 BC). Roman military and politician}
\end{center}


\textbf{Aristocrat}. The character lives comfortably and in abundance and frequents environments populated by the most powerful figures in the community. It has an excellent abode, perhaps a house in the most elegant district of the city or perhaps a series of rooms in the most renowned inn. He dines at the best restaurants, serves himself at the most skilled and fashionable tailors and can count on various servants who take care of his every need. He receives invitations to social events of the rich and powerful and spends his evenings in the company of politicians, guild leaders, high priests and nobles. It also has to contend with the deceptions and betrayals perpetrated at the highest levels. The greater his wealth, the greater the chances that he will be drawn into some political intrigue, sometimes as a pawn, sometimes as an active participant.


\subsubsection{Board and lodging} \index{Board and lodging}


The "Board and Lodging" table indicates the prices of individual dishes and a single overnight stay. These prices are included in a character's total lifestyle expenses.

\textbf{Room and Board}

\bigskip

\begin{tabular}{ll}
\textbf{Object} & \textbf{Cost} \\
\toprule
\textbf{Beer} & \\
Mug & 4 mr \\
Carafe (4 liters) & 2 ma \\
\textbf{Food} & \\
Banquet (per person) & 10 gp \\
Meat, 1 piece & 3 m \\
Cheese, 1 piece & 1 piece \\
Bread (loaf) & 2 mr \\
\textbf{Inn (daily}) & \\
Shabby & 7 mr \\
Poor & 1 but \\
Modest & 5 but \\
Agiata & 8 ma \\
Rich & 2 gp \\
Aristocratic & 4 mo \\
\textbf{Meal (daily)} & \\
Shabby & 3 mr \\
Poor & 6 mr \\
Modest & 3 but \\
Well-off & 5 but \\
Rich & 8 but \\
Aristocrat & 2 mo \\
\textbf{Wine} & \\
Ticket (bottle) & 10 gp \\
Common (carafe) & 2 ma \\
\end{tabular}

\subsubsection{Services}


Adventurers can pay non-player characters to help or act on their behalf in a variety of circumstances. Most of these wingmen are endowed with near-ordinary skills, while others have mastered an art or trade, and some have specialized in some adventuring skills.

Other common wingmen include the many inhabitants of a typical town or city that adventurers can hire to perform a specific task. For example, a spellcaster might pay a carpenter to have a prized casket (and its miniature replica) built for use in a spell.
A warrior might commission a blacksmith to forge a special sword.

\medskip

\textbf{Services}

\bigskip

\begin{tabular}{ll}
Service & Cost \\
\toprule
Carriage within a city & 1 mr for 1 km \\
Carriage between two villages & 5 mr for 1 km \\
Skilled Gregary & 2 gp per day \\
Inexperienced Gregary & 5 but per day \\
Messenger & 2 mr for 1.5 km \\
Passage by ship & 1 m for 1.5 km \\
Road or entrance fee & 1-5 mr \\
\end{tabular}


\subsubsection{Magical Services}

\textbf{Spell Level x Spell Level × 100 gp}

This is the cost of having a spell-wielding spellcaster. This cost assumes that you can go to the caster and ask him to manipulate a certain magic to your liking (usually takes at least 8 hours to prepare). If you want to take the caster somewhere to use magic, you need to negotiate with him, and the basic answer is "no".

\begin{center}
	\includegraphics[width = 0.8 \linewidth]{immagini/riempitivocavalieriapranzo.png}
\end{center}

If the spell a has dangerous consequences, the caster must receive certain proof that the character has the ability to pay and that he will not fail to do so should these consequences occur (provided that he agrees to cast the required spell, which is not true. 'at all safe). When it comes to spells that carry the character and the caster down a distance, the spell must be paid twice even if the character does not wish to go back with the caster.

Not all villages and towns have a spellcaster capable enough to manipulate magic. As a general rule, you need to travel to at least one small town to be sure enough to find a charmer. In a small country you might find a spellcaster capable of casting spells at level 2, in a large country those at level 3, a small town for those at level 5, in a large city for those at level 6, in a metropolis for those at level 6. those of level 8. Not even in a metropolis is it certain to find a spellcaster capable of casting spells of level 9 or higher.

\subsubsection{The Standard Backpack \texorpdfstring{\huge{\textregistered}}{\textregistered}} \index{Standard Backpack}

The Standard Backpack \textregistered \space is a list of objects that I have marked over time by adding everything that I needed during the adventures.
Take it as a starting point to understand what objects to have behind you, do not mark them all otherwise the Storyteller will seriously begin to look at the rules of encumbrance!

This is the content of the adventurer's backpack: belt, 3 candles, 6 torches, bait and flintlock, 7 dry rations, water bottle, rolled mattress, tarpaulin, tent, 18 meters rope, net, metal mirror, crowbar , compass, 3 lantern oil, ink, chalk, charcoal, hook, spade, fish hook, rags, metal cable, whistle, 6 potion vials, marble marbles, brass bell, 1 kg of flour in bag, 3 wedges, 12-meter metal chain, handcuffs, climbing nails, hammer, pulley, grappling hook.

\end{multicols}

\vfill

\begin{center}
\includegraphics[width = 0.6 \linewidth]{immagini/carrozza.png}
\end{center}


\pagebreak

\subsection{Special Materials} \index{Special Materials}

\begin{changemargin}{0.3cm}{0.3cm} \begin{enfasi}{
For this purpose, Captain De Medici had all the armor browned, however surprise the enemy even in the dark. (The craft of arms, Ermanno Olmi, 2001 film)} \end{enfasi} \end{changemargin} \medskip

\begin{multicols}{2}

Armor and weapons can be crafted from materials that possess inherent special qualities. If you build an armor or weapon with more than one special material, you only receive the benefits of the prevailing material. However, you can build a double weapon with each head made of a different special material.

\subsubsection{Living Steel} \index{Living Steel} \index{Living Steel Cost Table}

\label{acciaio-vivente}

\begin{tabularx}{0.45\textwidth}{Xl}
\textbf{Living Steel Item Type} & \textbf{Cost Modifier} \\
\toprule
Ammo & +40 gp per ammo \\
Weapon & +1000 gp \\
Light armor & +3000 gp \\
Medium armor & +8,000 gp \\
Heavy armor & +12,000 gp \\
Shield & +600 gp \\
Other items & 3000 gp / kg \\
\end{tabularx}

\medskip
A living steel tree is characterized by a particularly hard wood like steel. The origin of these trees remains a mystery to almost everyone. A living steel tree is an ordinary tree planted by a Devotee of Ephrem or Shayalia and given a blessing.

The living steel armor and shields are formally made of wood but have the same characteristics as adamantium. This particular wood is a favorite of those who fight and live for nature. It is not easy to identify a living steel tree for a non-expert and also for this reason it is extremely rare to find it raw, at most it is possible to find weapons or armor already made.

Living steel has 35 hit points for 1 inch thick and hardness 15.

\subsubsection{Adamantium} \index{Adamantium} \index{Adamantium Cost Table}

\label{adamantio}

\begin{tabularx}{0.45\textwidth}{Xl}
\textbf{Adamantium object type} & \textbf{Cost modifier} \\
\toprule
Ammo & +60 gp per ammo \\
Weapon & +1500 gp \\
Light armor & +5,000 gp \\
Medium armor & +10000 gp \\
Heavy armor & +15,000 gp \\
Shield & +1000 gp \\
Other items & 5,000 gp / kg \\
\end{tabularx}

\medskip
This very hard metal is found only in meteorites and contributes to the quality of a weapon or armor.

So the weapons and ammunition in adamantium have a +1 bonus on attack rolls, and the penalty given by the armor (Skills and Magic Checks Penalty) is decreased by 1 (or one die) compared to a normal armor of his. same kind. Objects without metal parts cannot be constructed with adamantium. An arrow can be adamantium, but a shod stick cannot.

Weapons and armor normally made of steel and constructed with adamantium have one-third more hit points than normal. Adamantium has 40 Hit Points for 1 inch thick and Hardness 20.

\subsubsection{Alchemical Silver} \index{Alchemical Silver} \index{Silver Weapon Cost Table}

\label{argento-alchemico}

\begin{tabularx}{0.45\textwidth}{Xl}
\textbf{Alchemical Silver Item Type} & \textbf{Cost Modifier} \\
\toprule
Ammo & +2 gp per ammo \\
Light weapon & +20 gp \\
Medium weapon & +90 gp \\
Heavy weapon & +180 gp \\
Shield & +100 gp \\
\end{tabularx}

The alchemical silvering process can only be applied to metallic weapons and does not work on special metals such as adamantium, cold iron and mithral.

A complex process involving metallurgy and alchemy can bind silver to a weapon made of steel so that it bypasses the damage reduction of creatures such as werewolves.

Alchemical silver has 10 hit points for every 1 inch of thickness and hardness 8.

\subsubsection{Cold Iron} \index{Cold Iron}

\label{ferro-freddo}

This iron is mined deep underground and is known for its effectiveness against demons and goblins. It is forged at a lower temperature to retain its delicate properties. Building weapons made of cold iron costs twice as much as their normal counterparts. Also, any magical upgrade costs an additional 2000 gp. This boost is applied the first time the item is boosted, not once per added quality.

Objects without metal parts cannot be made of cold iron. An arrow might be made of cold iron but a club not (except all metal). A double weapon that is only half made of cold iron increases its cost by 50 \%.

Cold iron has 30 hit points for 1 inch thick and hardness 10.


\subsubsection{Mithral} \index{Mithral} \index{Mithral Weapon Cost Table}

\label{mithral}

\begin{tabularx}{0.45\textwidth}{Xl}
\textbf{Object Type in Mithral} & \textbf{Modifier to Cost} \\
\toprule
Light armor & +1000 gp \\
Medium armor & +4,000 gp \\
Heavy armor & +9,000 gp \\
Shield & +1000 gp \\
Other items & +1000 gp / kg \\
\end{tabularx}

\bigskip

\begin{center}
\includegraphics[width = 0.9 \linewidth]{immagini/mithral.png}
\end{center}


Mithral is a very rare, shiny, silver-like metal, lighter than iron but just as hard. When worked like steel, it becomes a wonderful material to create armor with, and is occasionally used for other items as well. Most mithral armor is one class lighter than normal, and is easier for movement and other restrictions. Heavy armor is treated as medium armor, and medium armor is treated as light, but light armor remains light.

This decrease does not apply to the proficiency required to wear the armor in question (you must have Weapon Proficiency 3 to wear heavy mithral armor, although this is counted as an average for other factors). You must be proficient in the appropriate type of armor, otherwise you incur related penalties as normal.

The chances of a spell failing for mithral armor and shields decrease by 2 dice (remaining common required) and the skill check penalty decreases by 2 (to a minimum of 0), movement penalties decrease by 1 meter .

The mithral has 30 hit points for every 1 inch of thickness and hardness 15.

\subsubsection{Dragon Skin} \index{Dragon Skin}

\label{pelle-di-drago}

Armor makers can work dragon skins to produce armor or shields.
A dragon provides enough skin for a single skin armor for a creature one size smaller than the dragon. By selecting only the best scales and parts of skin, an armormaker can produce banded armor for a creature two sizes smaller, half armor for a creature three sizes smaller, and plate armor or full armor. for a creature of four sizes smaller.

\begin{center}
\includegraphics[width = 0.9 \linewidth]{immagini/dragonhide.png}
\end{center}


You cannot buy a Dragon skin armor or shield, it is always necessary to bring the raw material to the craftsman who will take care of building the armor.

In any case, there is always enough skin to produce a light or heavy shield in addition to the armor, as long as the dragon is Large or greater.
If the dragon hide comes from a Dragon that has immunity to an energy type, the armor is also immune to that energy type, although it does not grant any protection to the wearer. If the shield or armor is later granted the ability to protect the wearer from a specific energy type, the cost of this upgrade is reduced by 25 \%.

The failure chance of a dragon skin armor spell goes to zero (but the Magic Check is always required) while the skill penalty is decreased by 1 (down to a minimum of 0), the movement penalties are decreased by 1 meter.

Dragonhide armor costs four times as much as that type of armor, but doesn't take longer to build. Magic or medium or heavy armor must be found.

Dragon hide has 10 hit points for 1 inch thick and Hardness 10. Dragon skin is typically 1.25 to 1 inch thick.

\end{multicols}

\pagebreak

\section{Break through and Enter} \index{Break through} \index{Enter}

\begin{changemargin}{0.3cm}{0.3cm} \begin{enfasi}{
In the life of a man sooner or later there comes a day when, to go where he has to go, if there are no doors or windows, he has to break through the wall. (Bernard Malamud)

The crime of theft will be punished with the thieves' brand in full chest. In case of reiteration of the offense, the ears and then two fingers. (Twoslad, Rights and Duties Citizen's)

} \end{enfasi} \end{changemargin} \medskip

\begin{multicols}{2}

\label{sfondare-ed-entrare}

\lettrine[lines = 2, lhang = 0.33, loversize = 0.25, findent = 1.5em]{W}{hen} you try to split an object there are two choices: hit it with an object (weapon?) Or break it with brute force.

\smallskip

\textbf{Size matters ...}

Depending on the size of the object this can be more or less easy to hit. \\

\textbf{Table: Object Size and Defense - Hitting an Object} \index{Object Size and Defense Table - Hitting an Object}

\medskip

\begin{tabularx}{0.43\textwidth}{lll}
\textbf{Size} & \textbf{Mod. Defense} & \textbf{Dimensions} \\
\toprule
Colossal & -8 & 18m + \\
Gargantuan & -6 & 9-18m \\
Huge & -4 & 4-9m \\
Large & -2 & 2.4-4m \\
Average & +0 & 1.2-2.4m \\
Small & +2 & 60-120cm \\
Tiny & +4 & 30-60cm \\
Minute & +6 & 15-30cm \\
Extra small & +8 & 5-20cm \\
\end{tabularx}

\smallskip

\textbf{Defense Modifier}

Objects are easier to hit than creatures as they don't usually move, but many are tough enough to ignore damage with each hit. An object's Defense is equal to 10 + its Size modifier (see Table: Hitting an Object) + its Dexterity modifier (in case it ever has one).

If you use 3 Actions to aim you will automatically hit with a melee weapon and gain a + 2d6 bonus on hitting with a ranged weapon.

\subsection{Hardness}

\textbf{Table: Object Hardness and Hit Points} \index{Object Hardness and Hit Points Table}

\begin{tabularx}{0.43\textwidth}{lll}
\textbf{Substance} & \textbf{Hardness} & \textbf{PF for Thickness} \\
\toprule
Glass & 1 & 1 every 2.5 cm \\
Paper or fabric & 0 & 2 every 2.5 cm \\
Rope & 0 & 2 every 2.5 cm \\
Ice & 0 & 3 every 2.5cm \\
Leather or leather & 2 & 5 every 2,5 cm \\
Wood & 5 & 10 every 2.5 cm \\
Stone & 15 & 15 every 2.5 cm \\
Iron or steel & 10 & 10 every 2.5 cm \\
Mithral & 15 & 30 every 2,5 cm \\
Adamantium & 20 & 40 every 2.5 cm \\
\end{tabularx}

\subsection{Damaging objects}

\textbf{Energy Attacks}: Almost all objects have Damage Resistance to energy attacks (fire, electricity ...), divide the damage by 2 before applying Hardness. Certain types of energy can be particularly effective against certain objects, at the discretion of the Storyteller.

For example, fire could deal double the damage to scrolls, cloth, and other items that burn easily. Objects and creatures made of crystal or ceramic could take double damage (vulnerability) against a sonic attack.

Remember that Light damage is half Positive Energy \index{Light} and half Fire, while Void \index{Void} is half Cold and half Negative Energy, Negative or Positive Energy do not damage objects, only living creatures or not.

\textbf{Ranged Weapon Damage}: Objects take half the damage from a ranged weapon (except for Siege Engines and similar). Divide the damage by 2 before applying the item's Hardness.

\textbf{Ineffective Weapons}: Certain weapons simply cannot deal damage to certain objects. For example, a stab weapon cannot cut a rope.
Likewise, it is very difficult to break down a door or a stone wall with most melee weapons, unless they are specifically designed to do so, such as picks and hammers.

\textbf{Immunity}: Inanimate objects are immune to nonlethal damage and critical hits (but not damage explosion). Animated objects, if not regarded as creatures, also have these immunities.

\textbf{Vulnerability to Certain Attacks}: Certain attacks can be particularly effective against certain objects. In these cases, the attacks deal double damage and can ignore the object's Hardness.

\textbf{Damaged Objects}: A damaged object remains fully functional with the broken condition until its hit points reach 0, at which point it is considered destroyed. Damaged items (but not destroyed items) can be repaired by a Craftsman Profession and some Spells (see the Broken condition for more details).

\textbf{Saving Throw}: Unattended nonmagical items never make a saving throw. They are considered to have failed their saving throws, and are therefore always subject to the spell and other attacks that allow a saving throw to resist or negate the effect.

An object guarded by a character (whether he is holding it, touching it or wearing it) is saved if the character is saved.

\textbf{Magic items always have saving throws}. The bonus to Fortitude, Reflex, or Will saving throws of a magical item is equal to 2 + level x2 of the most powerful spell it hosts. If the item does not have a spell it is considered a bonus of +4 for every +1 bonus possessed. Magical items being guarded (worn) make the saving throw only if their owner fails his own. If an effect specifically affects the magic object and not the wearer then it is only the magic object that makes the saving throw.

\textbf{Animated Objects}: Animated objects count as creatures to determine their Defense (they are not considered inanimate objects).

\begin{center}
\includegraphics[width = 0.60 \linewidth]{immagini/portarinforzata2.png}

\textit{Reinforced door}
\end{center}

\subsection{Break Objects} \index{Break Objects}

\label{rompere-oggetti}

When trying to break something with brute force rather than dealing damage, you need to make a Strength check to see if you can.

Since Hardness does not affect DC to break the object, this value depends more on the way the object is constructed than on the material. See the table below for a list of the most common DCs related to breaking objects.

\textbf{Table: Hardness, Hit Points and DC on Strength for Breaking Objects} \index{Table Hardness, Hit Points and DC on Strength for Breaking Objects}

\medskip

\begin{tabular}{llll}
\textbf{Object} & \textbf{Dur.} & \textbf{PF} & \textbf{DC} \\
\toprule
Rope (2.5 cm diameter) & 0 & 2 & 23 \\
Forcing tied strings & 0 & 2 		& 15 \\
Handcuffs & 10 & 10 & 22 \\
Perfect Handcuffs & 10 & 10 & 24 \\
Chain & 10 & 5 & 26 \\
Small chest & 5 & 1 & 17 \\
Treasury & 5 & 15 & 23 \\
Simple wooden door & 5 & 10 & 13 \\
Good wooden door & 5 & 15 & 15 \\
Sturdy wooden door & 5 & 20 & 18 \\
Iron door (5cm thick) & 10 & 60 & 28 \\
Reinforced door & 10 & 10 & 25 \\
Bend iron bars & 10 & 10 & 24 \\
Stone wall (30cm thick) & 8 & 90 & 35 \\
Cut stone (90 cm thick) & 8 & 540 & 50 \\
\end{tabular}

\medskip

Creatures larger or smaller than Medium have bounty bonuses or penalties on the Strength check for breaking through a door:

\medskip

\textbf{Table: Strength check Modifiers According to Your Size} \index{Strength check Modifiers for Door Breakthrough Table}

\medskip

\begin{tabular}{ll|ll}
\textbf{Size} & \textbf{Mod.} & \textbf{Size} & \textbf{Mod.} \\
\toprule
Very small & -16 & Minute & -12 \\
Tiny & -8 & Small & -4 \\
Normal & + 0 & Large & +4 \\
Huge & + 8 & Gargantuan & +12 \\
Colossal & +16 && \\
\end{tabular}

\medskip

A \textbf{crowbar} \index{crowbar} or a \textbf{portable ram} \index{ram} increases the character's chance to break through a door by + 1d6.

\end{multicols}

\medskip


\begin{center}
	\includegraphics[width = 0.55 \linewidth]{immagini/kitladro.png}

	\textit{Rogue Kit}
\end{center}

\pagebreak

\section{Environment} \index{Environment}

\label{ambiente}
\begin{changemargin}{0.3cm}{0.3cm} \begin{enfasi}{
Nature is not cruel, it is just mercilessly indifferent. This is one of the hardest lessons a human being has to learn. (Richard Dawkins)

The main antidote to a bad environment is, of course, in replacing it with a good one. (Robert Baden-Powell)
} \end{enfasi} \end{changemargin} \medskip

\begin{multicols}{2}

\lettrine[lines = 2, lhang = 0.33, loversize = 0.25, findent = 1.5em]{F}{rom} lifeless deserts to dungeons full of traps, the environment helps to define the world, make it alive, dynamic and rich. It allows you to create an exciting and engaging gaming experience.

\subsection{Environmental Rules}

\label{regole-ambientali}

\subsubsection{Vision and Light} \index{Vision} \index{Light}

\label{sec:visione-e-luce}

In a natural environment, lighting can take on different shades and these shades help to understand how far a creature can see.

The gradations of light can be:
\begin{itemize}
\item
\textbf{Darkness} ': pitch dark, can be natural or magical
\item
\textbf{Dim light / Slightly obscured / Twilight}: low light, allows to recognize the shapes
\item
\textbf{Light}: intense light, a bright, covering, sunny light
\end{itemize}

The sources of illumination, or their absence, will determine how much light there is and how far away. The Light Sources Table indicates for the most common light sources the fully illuminated beam, the less illuminated one (Dim Light) and the duration.

\begin{changemargin}{0.3cm}{0.3cm} \begin{tcolorbox}[title = Note on light sources]
You may have noticed, or you will soon, that magical light sources work differently, very often they last much shorter or generate little light. This is due to the will of a Patron and as such only a Patron can cancel the effects (or the Storyteller!).
\end{tcolorbox} \end{changemargin}

\begin{changemargin}{0.3cm}{0.3cm} \begin{Storytellere} The different functioning of light sources wants to make exploration more gloomy, dark and difficult, especially in caves and areas without light sources. Enough groups that light brown every minute. Darkness helps the imagination and raises the level of tension. \end{Storytellere} \end{changemargin}

\medskip

\begin{center}
\includegraphics[width = 0.8 \linewidth]{immagini/oscurita.png}

\textit{Henry Justice Ford}
\end{center}

\bigskip

\textbf{Table: Light sources} \index{Table of light sources}

\medskip

\index{Dim Light}

\begin{tabular}{l|cc|c}
\textbf{Source of} & \multicolumn{2}{c}{\textbf{Radius in meters}} & \textbf{Duration} \\
\textbf{Light} & \textbf{Light} & \textbf{Dim Light} & \\
\toprule
Candela & 1 meter & - & 1 hour \\
Flashlight & 3 meters & 6 meters & 1 hour \\
Lantern & 6 meters & 12 meters & 6 hours \\
\multicolumn{4}{c}{\textbf{spells}} \\
Light 		& 3 meters & 6 meters & 1TxCM \\
Daylight & 9 meters & 18 meters & 1 hour \\
\end{tabular}

\medskip

The \textbf{Dim Light} \index{Dim Light} is the light beyond a light source. It is the passage in a 3-meter corridor if it is lit only by dim candles. It's a full moon night.
Generally speaking, a light source creates dim light in a radius twice as large as the normal light beam.

For a creature with normal vision, dim light provides light Cover, or +2 to Defense, and makes Awareness checks more difficult (+2 to DC).

\medskip

\textbf{Darkness} \index{Darkness}: is the most complete darkness without any light source. For creatures with normal vision, darkness is what is beyond dim light.
The \textbf{blind character} \index{Blind} or who fights in the dark (and cannot see in the dark) has -1d6 to Visual Awareness and all opponents are invisible.

\medskip

The \textbf{Light} \index{Light} is the light outdoors in the sun, but also if you hold a torch in your hand or in a corridor lit by lanterns. In its small size, even a candle provides light, but only enough to envelop ourselves.

\subsubsection{Types of Vision and Lighting}

\begin{itemize}
\item
A creature with \textbf{Normal Vision} \index{Normal Vision} sees up to distance, as a circular ray around the light source, indicated by Light. Beyond is Dim Light and beyond is Darkness.

\item
A creature with \textbf{Low Light Vision} \index{Low Light Vision} sees smoothly up to distance, as a circular ray around the light source, indicated in Dim Light, or indicated by the race if less, beyond is darkness.

\item
A creature with \textbf{Darkvision} \index{Darkvision} sees up to the distance indicated by its darkvision ability, whether there is light or not, it can't see beyond.
Darkvision is black and white vision.
\end{itemize}


\subsubsection{Dark} \index{Dark}

\label{buio}

Torches and lanterns can be suddenly extinguished by a gust of wind, magical light sources can be dissolved or thwarted, and some magical traps can create areas of impenetrable darkness.

In some cases, some characters or monsters may be able to see while others are Blinded. For the purposes of the rules that follow, a blinded creature is simply a creature that is unable to see its surroundings.

\subsubsection{Blinded} \index{Blinded} \index{Invisible}

\label{accecato}

Blinded creatures lose their ability to deal extra damage caused by, for example, the backstab ability (but not damage blast or critical hit).

Blinded creatures move at half speed \index{Move in the dark}. They must make a DC 12 Acrobatics check per move action to move at normal speed. If the check fails, they fall prone to the ground. Blinded creatures can't charge.

A creature that is blinded, or battling an invisible creature, \index{Invisible} can make a Consciousness check on difficulty 20 (or 10 + Move silently if the opponent doesn't want to be found) to locate the creature as long as it is within twice the melee distance from the character.

A blinded creature \index{Blinded} takes a -1d6 penalty on Awareness checks and -2 on Strength and Dexterity checks and automatically fails any vision-dependent Awareness checks.

Additionally, a blinded creature cannot use gaze spells and is immune to gaze spells.

See attack modifiers details in \hyperlink{invisibilita}{Invisibilità}.

\subsubsection{Falls} \index{Falls} \index{Fall} \hypertarget{cadute}{}

\label{cadute}

Falling creatures get hurt. Divide the fall height (in meters) by 3, round down, the resulting number is the d6 of damage suffered. Eg 16 meters drop is 16/3 = 5d6 of damage. Fall damage cannot exceed 20d6 damage, for every 3 dice over 20 add 6 damage (X / 3) d6 + (X / 3-20) * 6.

Creatures that take damage from a fall land in a prone position.

A successful Acrobatics check with DC 15 allows the character to halve the damage when falling from less than 30 feet.

Falls on soft surfaces (soft ground, mud, etc.) reduce damage by 1d6. This reduction applies before the damage reduction for using the Acrobatic skill.


\begin{center}
\includegraphics[width = 0.8 \linewidth]{immagini/oggetticadenti.png}

\textit{Henry Justice Ford}
\end{center}


A character cannot use spells while falling, unless the fall is greater than 150 meters or the spell is cast as a Reaction, Immediate Action, or 1 Action. Casting a spell while falling increases is considered to be Distracted.

\medskip

\textbf{Falling into the Water} \index{Falling into the Water}

Falls in the water are handled a little differently. As long as the water is at least 3 meters deep and the dive is from a height within 12 meters, no damage is taken.

You take 2d6 damage from a fall over 15 meters and 5d6 for falls over 15 meters.

Characters who voluntarily dive into the water take no damage if they pass an Acrobatics check or a DC 15 Swim check if the water is at least 20 feet deep. The DC of the check increases by 5 every 5 meters beyond 15 meters.

\subsubsection{Effects of Acid} \index{Acid}

\label{effetti-dellacido}

Corrosive acids deal 1d6 points of damage per round of exposure, except in the case of total immersion (such as in a vat of acid), which deals 10d6 points of damage per round. An acid attack, such as that of a flung bottle or the saliva / breath of a monster, should be considered as an exposure round.

The vapors produced by most acids are equivalent to inhaled poisons. Those who get very close to a large mass of acid must make a DC 13 Fortitude save or take 1 Constitution damage per round. This poison has no frequency, so a creature is safe if it strays from the acid.

Creatures immune to the caustic properties of acid may still drown if they are totally immersed in it (see Drowning).

\subsubsection{Effects of Smoking} \index{Smoke}

\label{effetti-del-fumo}

A character forced to breathe thick smoke must make a Fortitude saving throw each round (DC 15, +1 on each previous check) or spend the round coughing and choking. A character who continues to choke for 2 consecutive rounds takes 1d6 points of nonlethal damage for an additional round of exposure. Smoke obscures vision, providing Light Cover (+2 Defense) to characters within it.

\subsubsection{Hunger and Thirst} \index{Fame} \index{Thirst}

\label{fame-e-sete}

Characters may find themselves without water or food and without the means to obtain them. In normal climates, Medium-sized characters need at least 2 liters of liquid and 0.5 kg of decent food per day to avoid hunger, Small characters need half. In very hot climates, characters may need two to three times that amount of water to avoid dehydration.

Every day without food you must make a Fortitude save at difficulty 11 +1 per day without food, if you don't have to drink the difficulty increases to +3.

If you fail the saving throw, you take 1d4 damage and become increasingly fatigued. Penalties from fatigue remain until you eat and drink enough.

\subsubsection{Falling Objects} \index{Falling Objects}

\label{oggetti-cadenti}

Just as characters take damage from falls greater than 10 feet, so do they take damage if hit by falling objects.

Objects that fall on characters deal damage depending on their weight and the distance they fell from.

\textbf{Table: Damage from Falling Objects} determines the amount of damage an object deals based on its size. It is assumed that the object is made of a dense and heavy material, such as stone.
Items made of lighter materials may deal half or less damage, at the Storyteller's discretion. For example, a Huge boulder hitting a character deals 6d6 points of damage, while a wooden chariot might only deal 3d6 damage.

Also, if the object falls from a distance of less than 10 feet, it deals half the indicated damage. If an object falls more than 20 meters away, it deals double damage. The falling object takes the same amount of damage it deals.

\bigskip

\textbf{Table: Falling Object Damage} \index{Falling Object Damage Table}

\medskip

\begin{tabular}{ll}
\textbf{Object Size} & \textbf{Damage} \\
\toprule
Tiny or Smaller & 1d6 \\
Small & 2d6 \\
Media & 3d6 \\
Large & 4d6 \\
Huge & 6d6 \\
Gargantuan & 8d6 \\
Colossal & 10d6 \\
\end{tabular}

\bigskip

Dropping an object on a creature requires a ranged touch attack (Dexterity-based touch vs. attack defense). These attacks usually have a range of 10 feet. If an object falls on a creature (instead of being thrown), that creature must make a DC 15 Reflex saving throw to halve the damage if it is aware of the falling object. Falling objects that are part of a trap use the trap rules instead of the ones described here.

\subsubsection{Dangers of Water} \index{Dangers of Water} \index{Water}

\label{pericoli-dellacqua}

Any character can traverse relatively calm waters that are no deeper than his height, without the need for evidence. Likewise, Swimming in calm water requires a DC 10 check. Trained swimmers (at least 1 point in Swim) can take 10. Remember that armor or heavy equipment makes any attempt to swim more difficult. Getting around in water is considered difficult terrain.

In faster or more violent water with a successful Swim check or a DC 15 Strength check, characters are in no danger of going underwater. If they fail, they take 1d3 nonlethal damage per round (1d6 lethal damage if waters rush over rocks and valleys).

Very deep water is not only pitch black, making navigation very dangerous, but it deals even worse damage due to water pressure on the order of 1d6 points of damage per minute for every 100 feet that separate you from the surface. A successful Fortitude save (DC 15, +1 for each previous check) indicates that the immersed character takes no damage that minute. Very cold water deals 1d6 nonlethal damage per minute of exposure due to hypothermia.

\textbf{Drowning} \index{Drowning} \index{Drowning} \index{Choking} \hypertarget{trattenereilfiato}{}

Any character can hold his breath for a number of rounds equal to 6 rounds for his Constitution score, with a minimum of 3 rounds. For each Action performed the remaining duration decreases by 1 round, casting a spell consumes 3 rounds of air. After this period of time has elapsed, the character must make a DC 12 Fortitude save each round to continue holding his breath. Each round, the DC increases by 1.

\medskip
\begin{center}
\includegraphics[width = 0.8 \linewidth]{immagini/affogare.png}

\textit{Henry Justice Ford} \end{center}
\medskip

If the saving throw fails, the character immediately goes to 0 hit points and faints. From the next round he begins to lose 1 hit point per round until death (or revive!)

You can drown in substances other than water, such as sand, quicksand, very fine dust or a silo filled with spelled or simply by holding your breath.

\subsubsection{Dangers of the Heat} \index{Hot}

\label{pericoli-del-caldo}

A creature subjected to very high temperatures (above 40 ° C) must make a Fortitude saving throw every hour (DC 15, +1 on each previous check) or take 1d4 points of nonlethal damage. If he's wearing heavy clothing or any type of armor, he takes a -1d6 penalty on these saving throws. A character adds his Survival proficiency value and can give companions a bonus equal to half the value on the same saving throw. Unconscious characters begin taking lethal damage (1d4 damage per hour).

A character who suffers Non-Lethal Damage from exposure to heat is subject to heat stroke and is Fatigued. These penalties end when the character recovers nonlethal damage taken from the heat.

Infernal heat (air temperature above 60 ° C, fire, boiling water, lava) inflicts lethal damage. Breathing air in these temperatures deals 1d6 points of fire damage per minute (without saving throw).

Boiling water deals 1d6 points of burn damage, unless the character is completely submerged in it, in which case they would take 10d6 points of damage per round of exposure.


\begin{center}
	\includegraphics[height = 0.7 \linewidth]{immagini/desert.png}
\end{center}

\subsubsection{Catching Fire} \index{Catching Fire} \index{Fire}

\label{prendere-fuoco}

Characters exposed to boiling oil, campfires, non-instant magical fires can see their clothing, hair, or equipment catch fire. Spells specify whether they are capable of setting fire.

Characters in danger of catching fire can make a DC 15 Reflex saving throw to avoid this fate. If a character's clothing or hair catches fire, he immediately takes 1d6 points of damage. For each subsequent round, the burning character must make another Reflex saving throw. Failure means it takes another 1d6 points of damage that round. Success indicates that the fire is extinguished (that is, once it succeeds at the saving throw, it is no longer on fire).

A character who is on fire can automatically extinguish flames by jumping into enough water to put them out. If there is no large amount of water available, rolling on the ground or dampening the flame with cloaks or the like can grant the character another saving throw with a + 1d6 bonus.

Those who are unlucky enough to see their equipment or clothing catch fire must make a Reflex saving throw (DC 15) for each object. Flammable objects that fail the roll take the same amount of damage as the character.

\begin{center}
\includegraphics[width = 0.8 \linewidth]{immagini/fuocopericolo.png}
\end{center}

\medskip

\textbf{Effects of Lava} \index{Lava}

Lava or magma deals 2d6 damage per round of exposure, except in the case of total immersion (such as when a character falls into the crater of an active volcano), which deals 20d6 damage per round (plus any fall damage).

The damage from the magma continues for 1d3 rounds after the exposure ends, but this additional damage is only half of that dealt during the last round of actual contact. An Immunity or Fire Resistance also serves as a resistance or resistance to lava or magma. However, Immune or Fire Resistant creatures may drown if immersed in lava (see Drowning).


\subsubsection{Dangers of Cold} \index{Cold}

\label{pericoli-del-freddo}

Characters ill-dressed in cold climates (below 5 ° C) must make a Fortitude saving throw every hour (DC 15, +1 for each previous check) or take 1d6 points of nonlethal damage.
In extreme cold or exposure below -17 ° C, a face-up character must make a Fortitude saving throw every 10 minutes (DC 15, +1 for each previous check), taking 1d6 points of lethal damage for each failed saving throw. Characters wearing winter clothing only need to check for cold and exposure once per hour.

A character adds his Survival proficiency value to his saving throws and can give his companions a bonus equal to half the value on the same saving throw.

A character who takes nonlethal damage from cold or exposure is prone to chilblains or hypothermia (treat him as fatigued). These penalties end when the character recovers from nonlethal damage taken from cold and exposure.

Intolerable cold or exposed conditions (below -28 ° C) inflict 1d6 lethal damage per minute (with no saving throw) on players unless specifically protected.


\subsubsection{Effects of Ice} \index{Ice}

Characters walking on ice are like walking on difficult ground. Movement is halved, any Acrobatics checks have a +5 difficulty increase. Characters who have been in contact with ice for a long time may suffer extreme cold damage.

\begin{center}
\includegraphics[height = 0.6 \linewidth]{immagini/snowfall.png}
\end{center}

\subsubsection{Slow Choking} \index{Choking}

A Medium-sized character can breathe quietly for about 6 hours in a sealed chamber measuring 3 meters on each side. After this time, he takes 1d6 points of nonlethal damage every 15 minutes. Each additional Medium-sized character or any significant fire (a torch, for example) proportionally reduces the duration of the breathing air. Once unconscious from the accumulation of nonlethal damage, characters begin taking lethal damage at the same rate. Small characters consume half the air of Medium sized characters.

\subsection{Weather - Weather} \index{Weather}

\label{tempo-atmosferico---meteo}

Sometimes the weather can play an important role in the course of an adventure. The Table: Random Weather is a generic table that can be used to establish local atmospheric conditions. The terms of the table are defined below:

\end{multicols}

\medskip

\textbf{Table: Random Atmospheric Time} \index{Random Atmospheric Time Table}

\medskip

\begin{tabularx}{0.95\textwidth}{llXXX}
	\textbf{d \%} & \textbf{Weather} & \textbf{Cold climate} & \textbf{Temperate climate{*}} & \textbf{Desert} \\
	\toprule
	01-70 & Normal & Cold, calm & Normal for the season{*}{*} & Torrid, calm \\
	71-80 & Abnormal & Heat Wave (01-30) - Cold Wave (31-100) & Heat Wave (01-50) - Cold Wave (51-100) & Torrid, breezy \\
	81-90 & Inclement & Precipitation (snow) & Precipitation (normal for the season) & Torrid, breezy \\
	91-99 & Storm & Snowstorm & Lightning Storm - Snowstorm & Duststorm \\
	100 & Storm & Blizzard & Blizzard, Blizzard, Hurricane, Tornado & Downpour \\
\end{tabularx}

\medskip

* Temperate includes forests, hills, swamps, mountains, plains, and warm marine areas.

** Winter is cold, summer is hot, autumn and spring are moderate. Marshes are always slightly warmer in winter.

\begin{multicols}{2}

\textbf{Downpour}: Think of it as rain (see Precipitation below), but it offers cover like fog. It can cause flooding and usually lasts 2d4 hours.

\textbf{Warm}: The temperature is between 15 ° and 30 ° C during the day, and between 6 and 11 degrees lower at night.

\textbf{Calm}: Light wind (between 0 and 15 km / h).

\textbf{Cold}: Temperature between -17 ° and 5 ° C during the day, and between 6 to 11 degrees lower at night.

\textbf{Moderate}: Temperature between 5 and 15 ° C during the day, and between 6 and 11 degrees lower at night.

\textbf{Heat Wave}: Raises the temperature by 6 ° C.

\textbf{Cold Wave}: Lower the temperature by 6 ° C.

\textbf{Precipitation}: Roll a d100 to determine if the precipitation is fog (01-30), rain / snow (31-90), or sleet / hail (91-00). Snow and sleet only occur when the temperature is 0 ° C or lower. Most rainfall lasts 2d4 hours. Hail, on the other hand, only lasts 3d6 minutes but is usually accompanied by 1d4 hours of rain.

\textbf{Storm} (Lightning / Snow / Dust): The wind is very strong (45 to 75 km / h) and visibility is reduced by three quarters. Storms last 2d4-1 hours. See Storms, below, for more details.

\textbf{Storm} (Blizzard / Blizzard / Hurricane / Tornado): The wind speed is greater than 75 km / h (see Table: Wind Effects). Furthermore, blizzards are accompanied by heavy snowfall (1d3 \texttimes{} 30 cm), and hurricanes are accompanied by downpours. Blizzards last 1d6 hours, blizzards 1d3 days. Hurricanes can last up to a week, but the greatest impact for characters will occur in a period of time between 24 and 48 hours as the center of the storm moves into their zone. Tornadoes are very short-lived (1d6 \texttimes{} 10 minutes), and usually form as part of a lightning storm.

\textbf{Torrid}: Temperature between 30 ° and 43 ° C during the day and between 6 and 11 degrees lower at night.

\textbf{Breezy}: The wind speed is moderate to strong (15 to 45 km / h); see Table: Wind Effects.

\textbf{Rain, Snow, Sleet and Hail}

Bad weather frequently slows or blocks land transport and makes navigation virtually impossible. torrential downpours and blizzards obscure the view as much as a dense fog would.

Most precipitation manifests as rain, but in cold climates it can also manifest as snow, sleet, or hail. Precipitation of any kind, followed by a drop in temperature from above to below 0 ° C can produce ice.

\begin{center}
\includegraphics[width = 0.9 \linewidth]{immagini/Paesaggio-pioggia-Auvers.png}

\textit{Vincent van Gogh, Landscape in the rain at Auvers, 1890, oil on canvas, 50 x 100 cm}
\end{center}


\textbf{Heavy rain} \index{Heavy rain}

Rain cuts visibility in half, and imposes a -1d6 penalty on Awareness checks. It has the same effect as a very strong wind on flames, ranged weapon attacks, and Awareness checks as a very strong wind.

\textbf{Snow} \index{Snow}

As it falls, snow has the same effects as rain on visibility, ranged weapon attacks, and Awareness checks, and the terrain is considered difficult. A snowfall lasting one day leaves 3d6 * 2.5 centimeters of snow on the ground.

\textbf{Thick Snow}

A heavy snowfall has the same effects as a normal snowfall, but obscures visibility like fog (see Fog). A day of heavy snow leaves 2d4 x 30 centimeters of snow on the ground and the terrain is considered doubly difficult (movement / 4). Heavy snow accompanied by strong or very strong winds can result in snowdrifts 1d4 x 1.5 meters deep, especially on and around objects large enough to deflect the wind (a hut or large tent, for example).
There is a 10 \% chance that a heavy snowfall is accompanied by lightning (see Lightning Storm). Snow has the same effect as moderate wind on flames.

\textbf{Sleet}

It is basically frozen rain, which has the same effects as rain when it falls (except the probability of extinguishing protected flames is 75 \%) and those of snow once it has settled.

\textbf{Hail}

Hail does not reduce visibility, but the sound of falling hail makes Hearing-based Awareness checks more difficult (-1d6 penalty). Sometimes (5 \% probability) the hail can be so large that it inflicts 1 lethal damage (per storm) to anything in the open. Once deposited, hail has the same effect on movement as snow.

\subsubsection{Storms} \index{Storms}

\label{tempeste}

The combined effects of precipitation (or dust) and wind, which accompany all storms, reduce visibility by three-quarters, imposing a -8 penalty on all Awareness checks. Storms make attacks with ranged weapons impossible, except with siege weapons, which take a -1d6 penalty on attack rolls.
They automatically extinguish candles, torches or similar unprotected flames. Protected flames, such as those from lanterns, are shaken violently and have a 50 \% chance of extinction. See Table: Wind Effects for possible consequences on creatures caught outside without cover.

Storms are of three types.

\textbf{Dust Storm (Challenge Rank 3)}

these desert storms differ from other storms in that they have no precipitation. Conversely, dust storms carry grains of sand that obscure vision, suffocate unprotected flames and can even extinguish protected ones (50 \% probability). Many dust storms are accompanied by very strong winds and leave behind a deposit of 1d6 \texttimes{} 2.5 centimeters of sand.
There is also a 10 \% chance of encountering large dust storms with gusts of wind (see Table: Wind Effects). These violent dust storms inflict 1d3 nonlethal damage per round to anyone caught outdoors without cover and also pose the risk of suffocation (see Drowning, except that a character with a scarf or similar protection over their mouth and nose does not start. to suffocate except after a number of rounds equal to 10 \texttimes{} his Constitution score). Great dust storms settle behind (2d3-1) x 30cm of sand.

\textbf{Snowstorm}

in addition to the winds and precipitation common to other storms, snow storms deposit 1d6 \texttimes{} 2.5 centimeters of snow on the ground.

\textbf{Lightning Storm}

in addition to winds and precipitation (usually rain, but sometimes also hail), lightning storms are accompanied by electrical discharges that pose a danger to characters who are outdoors without shelter (especially if they are wearing metal armor). As a general rule, one can consider lightning per minute for a period of one hour in the heart of the storm. Each bolt deals electrical damage between 4d8 and 10d8. One in ten lightning storms is accompanied by a tornado.

\textbf{Violent Storms}

Very strong winds and torrential rainfall reduce visibility to zero, and make it impossible to make Awareness checks and make attacks with ranged weapons. Unprotected flames are automatically extinguished, and there is a 75 \% chance that this also occurs for protected ones. Creatures caught in these zones must make a Fortitude save or face effects depending on their size (see Table: Wind Effects). Violent storms are divided into the following four types.

\begin{center}
\includegraphics[width = 0.95 \linewidth]{immagini/Vincent_van_Gogh_tempesta.png}

\textit{Vincent van Gogh, Wheatfield under a stormy sky (Auvers-sur-Oise, July 1890)}

\end{center}


\textbf{Blizzard}: Although having little or no precipitation, blizzards can cause extensive damage due to the force of the wind.

\textbf{Blizzards}: The combination of strong winds, thick snow (usually 1d3 \texttimes{} 30 cm) and intense cold makes blizzards deadly for anyone not prepared for them.

\textbf{Hurricane}: In addition to very strong winds and heavy rain, hurricanes are followed by floods. Many activities on an adventure are impossible under these conditions.

\textbf{Tornado}: In addition to very strong winds, tornadoes can severely injure and kill those caught inside.

\subsubsection{Fog} \index{Fog}

\label{nebbia}

Whether in the form of a low-altitude cloud or a mist rising from the ground, the fog obstructs the view beyond a distance of 3 meters. Creatures further than 10 feet away have Light Cover (+2 Defense).

Fog makes the terrain difficult.

The fog could also be very dense in that case creatures farther than 10 feet have Medium Cover (+4 to Defense) and those within 1 meter still have light cover.

\subsubsection{Twenty} \index{Twenty}

\label{venti}

Winds can create swirls of sand or dust, fuel large fires, overturn small boats, and disperse gases or vapors. If they are strong enough they can even knock characters to the ground (see Table: Wind Effects), interfere with ranged attacks, or impose penalties on some Skill Checks.

\medskip

\textbf{Table: Wind Effects Wind Force} \index{Wind Effects Table Wind Force}

\medskip

\begin{tabular}{lll}
\textbf{Intensity} & \textbf{Speed} & \textbf{Ranged Attacks} \\
\toprule
Lightweight & 0-15km & \\
Moderate & 16.5-30km / h & \\
Strong & 31.5-45 & -2 \\
Very strong & 45.5-75km / h & -4 \\
Storm & 76.5-111km / h & impossible \\
Hurricane & 12-261km / h & impossible \\
Tornado & 262-450km / h & impossible \\
\end{tabular}

\bigskip

\textbf{Light Wind}

A gentle breeze, which has no practical effect on the game.

\textbf{Moderate Wind}

A sustained wind, which has a 50 \% chance of extinguishing any small, unprotected flame, such as that of a candle.

\textbf{Strong wind:} Gusts that automatically extinguish unprotected flames (candles, torches and the like). These flurries impose a -2 penalty on ranged attack rolls and Awareness checks.

\textbf{Very Strong Wind}

In addition to automatically extinguishing unprotected flames, winds of this intensity violently shake protected flames (such as those of a lantern) and have a 50 \% chance of extinguishing them. Attacks with ranged weapons and Awareness checks take a -1d6 penalty.

\textbf{Blizzard} \index{Blizzard}

Strong enough to knock down branches or even entire trees, blizzards automatically extinguish unprotected flames and have a 75 \% chance of extinguishing protected flames, such as lanterns. Ranged weapon attacks are impossible, and siege weapons also take a -1d6 penalty on attack rolls. Hearing-based Awareness checks take a -8 penalty for the howling of the wind.

\textbf{Hurricane} \index{Hurricane}

Extinguish all flames. Ranged attacks are impossible (except with siege weapons that take a -8 penalty on attack rolls). Hearing-based Awareness checks are also impossible, and all the characters can hear is the howling of the wind. Hurricanes are often capable of cutting down trees.

\textbf{Tornado (Challenge grade 10)} \index{Tornado}

Extinguish all flames. All ranged attacks are impossible (including those with siege weapons), as are Awareness checks based on hearing. Instead of being carried away (see Table: Wind Effects), characters who are in the immediate vicinity of a tornado and who fail a Fortitude save are sucked into the tornado.

Those who come into contact with the cone cloud are lifted off the ground and slammed for 1d10 rounds, taking 6d6 points of damage per round, before being ejected violently (with falling damage applied).

Although the rotational speed of a tornado can reach 450 km / h, the cone itself moves forward at an average of 45 km / h (about 75 meters per round). A tornado is capable of uprooting trees, destroying buildings, and causing other forms of similar devastation.

\end{multicols}

\vfill

\begin{center}
\includegraphics[keepaspectratio, width = 0.6 \textwidth]{immagini/blizzard.png}

\end{center}

\pagebreak

\section{Water Adventures} \index{Water Adventures}

\label{avventure-in-acqua}
\begin{changemargin}{0.3cm}{0.3cm} \begin{enfasi}{
He looked at the sea and realized how lonely he was now. (The old and the sea, Ernest Hemingway)
} \end{enfasi} \end{changemargin} \medskip

\begin{multicols}{2}

\lettrine[lines = 2, lhang = 0.33, loversize = 0.25, findent = 1.5em]{W}{ater} allows societies to exist, but it can also destroy them. Life could not exist without it. Commerce and travel are facilitated by its presence. Yet water can also kill, both by drowning people and by generating large-scale floods and tsunamis. Terrestrial life is dependent on water but at the same time fears it.

\textbf{Aquatic Adventures}

An aquatic adventure can take place wherever water represents the main element of the territory: such as swamps, rivers, lakes, ponds, oceans, the Water Plane and the like. Aquatic adventures, however, do not require characters to have the ability to breathe underwater; introducing Aquatic challenges for low-level adventurers brings a lot of tension and a feeling of danger to an adventure.

\textbf{Adapting to Aquatic Environments}

The rules for underwater combat apply to creatures that are not native to this dangerous environment, such as most PCs. For prolonged Aquatic adventures and particularly in-depth exploration, characters will need the use of magic to continue their adventures. spells of transformation or abjuration are of obvious use.

Pressure damage can be totally avoided by spells that offer resistance.

\subsection{Underwater Combat}
Creatures living on land have considerable difficulty fighting underwater. Water affects a creature's Defense, its attack rolls, damage, and movement.

\begin{itemize}
\item
A creature underwater loses its Dexterity bonus on Defense.
\item
A creature that is not under the \textit{Freedom of movement} or Swim speed spell makes attack rolls with a -1d6 and the opponent is considered to have slash and slash damage resistance.

Weapons such as Trident, Short Spear, Short Sword, Javelin have no penalty on hitting.
\item
If you do not have a Swim-type move, you can move in the middle of the move per Move Action (difficult terrain).
\end{itemize}
\medskip

\subsubsection{Ranged attacks underwater} \index{Attacks underwater}
Throwing weapons are ineffective underwater, even when thrown from the ground. Attacks with ranged weapons take a -2 penalty on attack rolls and -1 to damage for every 2 meters of water they pass through.

\subsubsection{Attacks from the mainland}
Those characters who swim, float, or cross the water on the surface, or wade a stretch where the water is at least chest high, enjoy medium coverage.

A completely submerged creature has full cover against land-based opponents.


\subsubsection{Magic effects in water}
Magical effects are unaffected, except those that require an attack roll (which are treated like all other effects) and fire effects.

\textbf{Fire}: Non-magical fire (including alchemist's fire) does not burn under water. Spells or magical effects of fire are ineffective underwater. A partially submerged creature has fire resistance.

\textbf{Casting spells underwater} \index{Casting spells underwater}

Casting spells while underwater can be difficult for someone who does not have the ability to breathe underwater.

A creature that is unable to breathe underwater takes three rounds of holding its breath to cast a spell underwater.

Some spells may work differently underwater, at the discretion of the Storyteller

Remember that a character can hold his breath for COS * 6, minimum 3 rounds, rounds before starting to drown and each Action consumes an additional round.

\subsection{To drown} \index{To drown} \index{To choke}

A character drowns when he is no longer able to hold his breath underwater, or when the swim check, given the penalties, has a value of less than 5.
The Swim check has difficulties based on the state of the water and the type of liquid which at least are considered "difficult terrain" in which you move. In case of calm waters the DC is 10, troubled waters DC 15, stormy waters DC 20. In case of failure of the check you do not move and you have a -1 to the next check, in case of critical failure the next check takes a -1d6.
A drowning character must make a DC 12 Fortitude save each round to hold his breath and each subsequent round the check increases by 1, each Action taken increases the next difficulty by 1. When the check fails the character goes to 0 points. Wounded, passed out, and from the next round loses 1 hit point per round.


\subsection{Nautical Adventures}

Water can provide the setting for a different and unique gaming experience: the nautical adventure. In such a scenario, the effects and dangers of underwater adventures are replaced by surface challenges, as characters and their opponents use ships and boats to navigate this environment. Nautical adventures usually resolve normally, with a combat aboard a ship similar to a land-based one. If combat occurs during a storm or in rough seas, consider the ship's deck as difficult terrain. Remember to consider the effects on Concentration checks for weather or roll.

\begin{center}

\includegraphics[width = 0.8 \linewidth]{immagini/avventure_acqua_grey.png}

\textit{The Mermaid and the Boy "1904 by HJFord}
\end{center}


\subsubsection{Quick Combat at Sea}

When ships fight, things change a bit. The following rules are not intended to accurately simulate all aspects of naval combat, but only provide you with quick and simple rules for unraveling such situations when they turn into a nautical adventure, whether it is a battle between two ships or between a ship and a ship. a sea monster.

{Preparation:} Determine which types of ships are involved in the combat (see Table: Ship Statistics). Use a large, empty battle grid to represent the waters in which battle takes place. A single square corresponds to 30 feet away. Picture each ship by placing tokens that take up the appropriate number of squares (toy ships are great tokens and can be found in model shops).

{Begin Combat:} When combat begins, let the characters (and important allied NPCs) roll Initiative normally; the ship moves and attacks based on the captain's initiative result. If one of the ships in battle uses sails to move, randomly determine which direction the wind is blowing by rolling 1d8 and following the guidelines for Spread Weapons missing the target.

{Movement:} Based on the captain's Initiative score, the ship may move at its base speed in a single round as if the Action matched that of the captain himself (or double his speed as the only action of the round), as long as it has its minimum full crew. The ship can increase or decrease its speed by 30 feet per round, until it reaches its maximum speed. Alternatively, the captain can change direction (at most one side of a square at a time) (2 Actions). A ship can change direction only at the beginning of the round.

{Attacks:} Members in excess of a ship's minimum crew requirement can be deployed to maneuver Siege Engines. Siege Engines attack based on the captain's Initiative score.


\begin{center}
\includegraphics[width = 0.9 \linewidth]{immagini/acquapericoli.png}
\end{center}

A ship can also attempt to ram a target if it hosts the minimum crew. To ram a target, the ship must move at least 30 feet and end up with the bow in a square adjacent to it.
Then, the ship's captain makes a Profession (sailor) check: if the result is equal to or greater than the target's Defense, the ship hits its target, dealing damage to it as indicated in Table: Ship Statistics and at the same time taking the minimal damage. A ship equipped with a spur deals an additional 3d6 damage to the target (the attacking ship takes no additional damage).

\textbf{Sinking} \index{Sinking}

A ship gains the sinking condition when its hit points drop to 0 or less. A sinking ship cannot move or attack and is considered sunk after 10 rounds. Every 25 damage a sinking ship takes reduces its sinking by 1 round. The fabricate spell allows you to repair a sinking ship if its hit points are reported above 0, in which case the ship loses the sinking condition. Non-magical repairs typically take too long to save a ship from sinking once it starts sinking.

\textbf{Statistics of a Ship}

In the real world, there is a huge variety of boats and ships, from small rafts to massive galleons. Representing this, Table: Ship Statistics ranks seven standard ship sizes and their respective statistics. Just as real-world cultures created and adapted different types of boats, so too could the races of fantasy worlds create their own bizarre ships.
Storytellers may use or modify these statistics to meet the needs of their creations and, in any case, describe such means of transport as they wish. All ships have the following traits.

\textbf{Type}: This is a general category listing the basic ship type.

\textbf{Defense}: The ship's Defense. To calculate a ship's effective Defense, add the captain's Profession (sailor) score to that ship's base Defense. Touch attacks against a ship ignore the captain's modifier. A ship is never Unprepared.

\textbf{Basic Save}: A ship's basic saving throw modifier (Fortitude, Reflexes, and Wisdom) have the same value. To determine a ship's effective saving throw modifiers, add the captain's Profession (sailor) modifier to this value.

\textbf{Maximum Speed}: The maximum speed of a ship in combat. An asterisk indicates that the ship has sails and can move at double speed if it moves in the same direction as the wind. A ship that only has sails can move only in windy conditions.

\textbf{Armaments}: The number of Siege Engines that can be equipped on the ship. A spur uses one of these slots, and a ship can only be equipped with a spur.

\begin{center}
\includegraphics[width = 0.8 \linewidth]{immagini/navenotte.png}
\end{center}


\textbf{Ramming}: The amount of damage a ship deals with a successful ramming attack (without a spur).

\textbf{Squares}: The number of squares the ship occupies on the combat grid.

\textbf{Crew}: The first number indicates the minimum crew the ship needs to function normally, excluding the arms crew. The second indicates the maximum number of the crew plus additional soldiers or passengers. A ship without its minimum crew can only move, change speed, change direction, or ram if its captain makes a DC 20 Profession (sailor) check.
A crew that exceeds the minimum number does not affect movement, but its members can replace fallen members or maneuver additional weapons.

\bigskip

\end{multicols}

\textbf{Table: Ship Statistics} \index{Ship Statistics Table}

\medskip

\begin{tabular}{lllllllll}
\textbf{Type} & \textbf{Defense} & \textbf{PF} & \textbf{TS base} & \textbf{Vel. (m / s)} & \textbf{Arma} & \textbf{Sperona} & \textbf{Quad}. & \textbf{Crew} \\
\toprule
Raft & 9 & 10 & + 0 & 4.5 & 0 & 1d6 & 2 & 1/4 \\
Rowboat & 9 & 20 & + 2 & 9 & 0 & 2d6 + 6 & 6 & 1/3 \\
Boat & 8 & 60 & + 4 & 9 & 1 & 2d6 + 6 & 12 & 4/15 + 100 \\
Longship & 6 & 75 & + 5 & 18 & 1 & 4d6 + 18 & 40 & 50/75 + 100 \\
Sailboat & 6 & 125 & + 6 & 18 & 2 & 3d6 + 12 & 20 & 20/50 + 120 \\
Warship & 2 & 175 & + 7 & 18 & 3 & 3d6 + 12 & 35 & 60/80 + 160 \\
Galea & 2 & 200 & + 8 & 27 & +4 & 6d6 + 24 & 60 & 200/250 + 200 \\
\end{tabular}

\pagebreak

\section{Adventures in the City} \index{City}

\label{avventure-in-citta}
\begin{changemargin}{0.3cm}{0.3cm} \begin{enfasi}{
God created the countryside, and man created the city. (William Cowper)
} \end{enfasi} \end{changemargin} \medskip

\begin{multicols}{2}

\lettrine[lines = 2, lhang = 0.33, loversize = 0.25, findent = 1.5em]{A}{t} first sight a city is very similar to a dungeon, as it is made up of walls, doors, rooms and corridors . Adventures set in the city differ from those set in dungeons for two main reasons. Characters have access to more resources and must take into account the presence of law enforcement.

\textbf{Access to Resources}: Unlike dungeons and the wilderness, characters can buy and sell Equipment very quickly in the city. A large city or metropolis probably has high-level NPCs and experts specializing in the darkest areas of knowledge, who can offer help and interpret clues. And when the characters are battered and bruised, they can always return to the comforts of their rooms in the inn.

The freedom to retreat and access to market goods means players have more control over the pace of a city adventure.

\textbf{Law Enforcement}: The other element of distinction between going on an adventure in a city and exploring a dungeon is that the dungeon is, almost by definition, a place without rules where the only law is that of the jungle: to kill or be killed.

A city, on the other hand, is backed by a code of laws, many of which have been explicitly devised to prevent the kind of behavior adventurers often indulge in: killing and looting. However, city laws recognize the seriousness of the threat monsters pose to city stability, and it is very rare that the ban on killing also applies to monsters such as aberrations or the Fiend.

Most evil humanoids, however, usually enjoy the same protection as all other citizens. Having a set of Evil Traits is not a crime (except perhaps in those cities where there is a severe theocracy, with the magical power necessary to enforce the law); only evil acts are considered a violation of the law.

\begin{center}
\includegraphics[width = 0.8 \linewidth]{immagini/cavalieri.png}
\end{center}


Even when adventurers encounter a malefactor committed to committing the most horrific crimes against the city's population, the law still frowns upon those who do justice for themselves by killing the malefactor or otherwise preventing them from being brought before a court of law. processed.

\textbf{Weapon and Spell Restrictions}

Each city has its own laws regarding the weapons that you can carry around in public and the restrictions on spells.

City laws may not affect all characters equally. A man of faith who moves with a weapon in tow is not hindered in any way by the law that requires a snare of weapons, but a spellcaster suffers a considerable reduction in his power if his Tome is confiscated at the gates of the city.

\textbf{Urban Elements}

Walls, doors, dim lighting, and rough terrain - in many ways, a city is similar to a dungeon. New elements suitable for a city setting are described below.

\textbf{Walls and Gates}

Many cities are defended by a circle of walls. Normal city walls are made of reinforced stone, 1.5 meters thick and 6 meters high. Such a wall is quite smooth, and a DC 30 Climb check is required to climb. The walls feature small battlements on one side to provide a parapet for the guards at the top, and the walking space on the walls is barely enough for a guard.

\textbf{The walls}

Unlike smaller cities, metropolises often also have internal walls, sometimes the old walls erected when the city was smaller, or walls that separate the various neighborhoods from each other. Sometimes these walls are as high and wide as the outer ones, but much more often they are the size of a large or small city.

\textbf{Watchtowers}: Some city walls have watchtowers that pop up at regular intervals. Few cities have enough guards to place on each watchtower, unless the city expects an attack from the outside. The towers offer an elevated view of the surrounding countryside as well as a bulwark of defense against enemy invaders.

\medskip

\begin{center}
\includegraphics[width = 0.85 \linewidth]{immagini/muraparigi.png}

\textit{Chronicles, Jean Froissart. Queen Isabella of France arrives in Paris, 15th century}
\end{center}

\medskip


The watchtowers are usually 3 meters higher than the wall they are part of, and their diameter is equal to 5 times the thickness of the walls. Loopholes for archers open on the upper floors of the tower, and the top is crenellated in the same way as the surrounding walls. In the smaller towers (about 7.5 meters in diameter, along a 1.5 meter thick wall) a simple ladder connects the inside of the tower to the roof. In the larger towers there are real stairs.

Access to the tower is protected by heavy wooden doors, with iron reinforcements and good locks (Disable Devices DC 25). Normally it is the captain of the guards who holds the access key to the tower, and a second copy is kept in the inner fortress or in the town barracks.

\textbf{Cancelli}: A typical entrance gate to the city consists of a guardhouse with two shutters and loopholes in the space between them. In villages and small towns, the main entrance is protected by double iron doors embedded in the city walls.

Gates usually remain open during the day and locked or barred at night. Generally, only one gate lets travelers in after dark, and is guarded by guards who will only open the doors for someone who looks honest, presents the appropriate documents, or bribes them with a sufficient amount (depending on the type of city and guards).

\textbf{Guards and Soldiers}

A city usually has full-time military personnel equal to 1 \% of its adult population, in addition to duty or conscript soldiers equal to 5 \% of the population. Full-time soldiers are city guards responsible for maintaining order in the city, with a role similar to that of the modern police, and (to a much lesser extent) for defending the city from outside assaults. Forced conscripts are called to arms in the event of an attack in the city.

A typical deployment of city guards is spread over three shifts of eight hours each, with 30 \% of its forces on duty during the day (from 8 to 16), 35 \% on duty in the evening (from 16 to 24) and 35 \% of service in the night shift (from 24 to 8). At any given moment, 80 \% of the guards on duty are on patrol in the streets, while the remaining 20 \% are assigned to various positions around the city, ready to react to any alarms. A similar guard post is present in at least every neighborhood in the city (a neighborhood is made up of several neighborhoods).

The majority of the city guards are made up of fighters, almost all 1st level. Officers are higher-level fighters, and possibly some charmer as well.

\textbf{Siege Engines} \index{Siege Engines}

Siege engines are large weapons, temporary structures or mechanisms traditionally used to besiege a castle or fortress.

\begin{center}
\includegraphics[width = 0.7 \linewidth]{immagini/armidaassedio.png}
\end{center}


\textbf{Heavy Catapult}: \index{Catapult} A heavy catapult is a giant siege engine capable of throwing boulders or other heavy objects with great force. Since the catapult's launch arc is very high, the contraption is capable of hitting areas beyond its line of sight. To fire a heavy catapult, the chief machine operator makes a special check with DC 15 using only his Attack Proficiency value with his Intelligence modifier, range penalty, and modifier for the lower section of the Table. : Siege Machines.

If the check is successful, the catapult boulder hits the melee zone the catapult was aimed at, dealing the indicated damage to any object or character in the zone. Characters who successfully make a DC 15 Reflex save take half damage. Once the boulder has hit the area, subsequent shots will hit the same area, unless the catapult is redirected or the wind changes direction or speed.

If a catapult boulder misses its target, use the weapons table on \hyperlink{spargimento}{spargimento} (pg. \pageref{attacchiarmidaspargimento}). The distance covered is 1d4x10 meters.

Charging a catapult requires a series of actions that take the whole round away. A DC 15 Strength check is required to lower the catapult arm; most catapults have wheels that allow up to two operators to use the Help Another action to assist the main operator of the pulley.

A DC 15 Profession (siege engineer) check allows you to lock the arm in place, and then another DC 15 Profession (siege engineer) check will be used to load the projectile onto the catapult. It takes four rounds to reload a heavy catapult (several catapult operators can perform these actions in the same round, so four people can reload a catapult in just 1 round). A heavy catapult occupies a space of 4.5 meters.


\textbf{Light Catapult}: This is a smaller and lighter version of the heavy catapult. It essentially functions like a heavy catapult, except that it takes a DC 10 Strength check to lock the arm in place, and only 2 rounds to redirect the catapult. A light catapult takes up a space of 3 meters.

\textbf{Ballista}: \index{Ballista} A ballista is basically a huge fixed heavy crossbow. Its size makes it difficult for most creatures to use, so an average creature takes a -1d6 penalty on attack rolls when using a ballista, and a small creature takes a -6 penalty. For a smaller than large creature, it takes 2 rounds to reload the ballista after firing.

A ballista occupies a space of 1.5 meters.

\textbf{Aries}: \index{Aries} This massive trunk is sometimes tied and suspended from a movable trellis that allows its manipulators to swing it with ever-increasing force against a target. As the only action of the round, the character closest to the ram's tip makes an attack roll against the building's Defense, applying a --1d6 penalty for lack of proficiency (proficiency in the use of this machine is not possible). In addition to the damage listed in Table: Siege Engines, up to nine other characters can push the ram and add their Strength modifiers to the ram damage, if they reserve an attack action to do so. You need at least one Huge creature or larger, 2 Large creatures, 4 Medium creatures, or 8 Small creatures to maneuver a ram (Tiny or smaller creatures cannot use a ram).

A ram is usually 30 feet long. In a battle, creatures wielding a ram must line up in two adjacent rows of equal length with the ram supported between the two rows.

\textbf{Siege Tower} \index{Siege Tower}: This machine is a huge wooden tower mounted on wheels or cylinders that can be pushed against a wall to allow the besiegers to climb the tower and thus arrive in top of the walls while benefiting from Cover. The wooden walls of the tower are usually about 30 cm thick.

A typical siege tower occupies a space of 4.5 meters. Creatures within it propel it at a speed of 10 feet (a siege tower cannot run). The eight creatures that push the tower to the ground floor enjoy full cover, those on the upper floors have improved cover and can shoot through loopholes for archers.

\end{multicols}

\medskip

\textbf{Table: Catapult Attack Modifiers} \index{Catapult Attack Modifiers Table}

\medskip

\begin{tabular}{p{0.57\textwidth} p{0.35 \textwidth}}
\textbf{Circumstance} & \textbf{Modifier} \\
\toprule
Line of sight does not reach target zone & -6 \\
Consecutive shot (operators can see where most recent misses landed) & +2 cumulative for previous miss (max +10) \\
Consecutive Shot (Operators cannot see where most recent misses have landed but an observer provides guidance) & +1 cumulative for previous miss (max +5)) \\
\end{tabular}

\bigskip

\textbf{Table: Siege Engines} \index{Siege Engines Table}

\medskip

\begin{tabular}{lllll}
\textbf{Machine} & \textbf{Cost (gp)} & \textbf{Damage} & \textbf{Range} & \textbf{Soldiers} \\
\toprule
Heavy Catapult & 800 & 6d6 & 60m & 4 \\
Light Catapult & 5500 & 4d6 & 45m & 2 \\
Ballista & 500 & 3d8 & 36m & 1 \\
Aries & 1000 & 3d6 & - & 10 \\
Siege Towers & 2000 & - & - & 20 \\
\end{tabular}

\begin{multicols}{2}

\bigskip

\textbf{City Streets} \index{City Streets}

Typical city streets are narrow and winding. Most of the city streets are 3 to 6 meters wide, while the alleys range from 3 meters wide to only 1.5 meters wide. If the paved floor is in good condition, it is possible to move normally, while roads in poor condition and badly damaged are considered equivalent to scattered debris, and increase the DC of the Acrobatics checks by 2.

Some cities do not have large avenues, especially those that have gradually grown from small settlements. Cities that have been designed around the table, or that have perhaps been consumed by a major fire that allowed authorities to build new roads on what were once populated areas, may have some larger roads crossing them. These main streets are 7.5 meters wide, allowing wagons to pass side by side, with 1.5 meter sidewalks on either side.

\textbf{Crowd}: The city streets are packed with people coming and going, busy with various daily chores. In most cases it is not necessary to include every 1st level peasant on the map when it comes to a fight on the main avenue of the city. Instead, it is sufficient to indicate which areas on the map are occupied by the crowd. If the crowd sees something dangerous, it will move away at a speed of 30 feet per round at Initiative 10 count. To make contact with the crowd, you must have melee distance. Crowd provides Full Cover.

\textbf{Leading the Crowd}: A Diplomacy check with DC 15 or Intimidate with DC 20 is required to get a crowd to move in a certain direction, and the crowd must be able to hear or see the character making the attempt. It takes a whole round to make the Diplomacy check, while it only takes one Action to make the Intimidate check.

If two or more characters attempt to push the crowd in two different directions, they make opposing Diplomacy or Intimidate checks to determine who the crowd will listen to. The crowd will ignore both if both check results are lower than the DCs indicated above.

\textbf{Roofs}: Climbing a roof usually requires climbing a wall, unless a character can reach a roof by jumping off a higher window, balcony or bridge. Flat roofs are common only in warm climate areas (accumulating snow can cause a flat roof to collapse) and are easy to run over. Moving to the top of a roof requires a DC 20 Acrobatics check. Moving horizontally on a pitched roof (moving parallel to its top, basically) requires a DC 15 Acrobatics check. Moving up and down a pitched roof requires a check of Acrobatics with DC 10.

Sooner or later a character will reach the end of the roof, and will have to make a long jump to move to the next roof or to go down to the ground. The distance from one roof to the next is usually 3 meters, but the roof on the other side could be 1.5 meters higher or lower, or the same height. The indications given for Acrobatics (the peak of height in a long jump is equal to one quarter of the horizontal distance) are used to determine if the character is capable of making a jump.

\textbf{Sewers}: To enter the sewers, characters usually have to open a grate (1 round) and jump 10 feet down. The sewers are built exactly like dungeons, with the difference that the floor is slippery or covered with water. Sewers are also similar to dungeons in terms of the creatures that can be encountered within them. Some cities were built on the ruins of older civilizations, so sewers could also lead to treasures and dangers belonging to a bygone era.

\textbf{City Buildings}
Most city buildings are divided into three categories. Many buildings in a city are two to five stories high and are built side by side to form long lines, interrupted only by the main or back streets. These terraced buildings usually house a shop on the ground floor, with offices or apartments on the upper floors. Inns, the wealthiest commercial enterprises, and the largest warehouses (plus any mills, tanneries, and other space-consuming businesses) typically are large freestanding buildings up to five stories high.
Finally, homes, shops, warehouses and smaller warehouses are simple one-story wooden buildings, especially in poorer neighborhoods.

\textbf{City Lighting}
If a city has large driveways, these will be illuminated by lanterns hanging at a height of about 2 meters on the sides of the buildings. These lanterns are placed at a distance of 9 meters from each other, so the lighting in these streets is practically continuous. The back streets and alleys are not illuminated; it is customary for citizens to pay a lantern to accompany them if they have to go out at night. The alleys can be dark places even during the day, thanks to the shadows of the surrounding taller buildings. A dark alley during the day is not dark enough to provide full but light coverage.

\end{multicols}

\medskip

\begin{center}
	\includegraphics[width = 0.58 \linewidth]{immagini/fognelondra.png}

	\textit{Map of the London sewers, 1880}
\end{center}


\pagebreak

\section{Adventures and Disasters} \index{Adventures} \index{Disasters}

\label{avventure-e-disastri}
\begin{changemargin}{0.3cm}{0.3cm} \begin{enfasi}{
First thing, no one is left behind. (anonymous)
} \end{enfasi} \end{changemargin} \medskip

\begin{multicols}{2}


\lettrine[lines = 2, lhang = 0.33, loversize = 0.25, findent = 1.5em]{T}{he natural disasters} are terrifying environmental dangers that bring death and devastation. The supernatural ones can be even more destructive, as they can forever disfigure a world. A disaster is more like an adventure than an encounter, and does not have a specific Degree of Challenge. Rather, each part of the disaster should be treated as a separate encounter designed with a degree of Challenge appropriate to the PCs.

Below are the rules for dealing with the effects of three different types of disasters, both natural and supernatural. Some disasters occur rapidly, such as earthquakes and tsunamis, while others proceed through numerous phases, such as forest fires, volcanoes, and undead uprisings. Adjust the adventure scheme to fit the disaster, to allow events to unfold over the course of a few minutes or several days depending on what you need.

\textbf{Volcanoes} \index{Volcanoes}

When the earth's crust breaks and ejects its molten heart, one of the most dramatic natural disasters takes place: the eruption of a volcano. Volcanic eruptions offer a wide range of options to the Storyteller, including lava, lava bombs, poison gas, and pyroclastic flows. Storytellers might also consider the idea of portending a dramatic volcanic eruption (or volcanic dragons) with pre-existing dangers, such as avalanches and minor earthquakes.

\textbf{Lava} \index{Lava}

Lava flows are generally associated with non-explosive eruptions and can be a permanent feature of active volcanoes. Lava flows are mostly slow and move at 4.5 meters per round (move action penalty 1), but the hottest ones are fast, reaching 12 meters per round (no move action penalty). Channeled lava, like in a lava tube, is very dangerous, as it moves at a speed of 36 meters per round (4 move actions per round) (a challenge with a grade of 6). Creatures reached by a lava flow must succeed on a DC 20 Reflex saving throw or are submerged in lava. Success indicates that they are in contact with Lava but not Immersed.

\textbf{Lava Bombs} (Challenge grade 2 or 8) \index{Lava Bombs}

Agglomerates of molten stone can be hurled many kilometers from an erupting volcano, cooling to solid stone before reaching the ground. A typical lava bomb hits a point designated by the Storyteller and explodes within a 20-foot radius. All creatures in the area must make a DC 15 Reflex save or take 4d6 points of damage. Creatures that have Cover or are able to cover themselves (such as with a shield) gain a +2 bonus on this roll. Sometimes very large lava bombs are formed that deal 12d6 points of damage. Normal lava bombs have Challenge 2 rank, large ones Challenge 5.

\textbf{Poisonous Gases} (Challenge grade 5) \index{Poisonous Gases}

One of the most insidious threats from a volcano is toxic gas, often unnoticed between fire and destruction. Different types of poisonous vapors arise from a volcanic eruption, some visible, some not. Poison gas deals 1d3 points of Constitution damage per round when inhaled (Fortitude DC 15 negates, DC increases by 1 for each previous saving throw), and visible gases also function as thick smoke. The poisonous gas clouds flow downward, and generally reach a height of 6 meters. Strong winds can deflect gas clouds, as can high barriers, provided the gas has somewhere else to go.

\textbf{Pyroplastic Flows} (Challenge grade 10)

Some volcanic eruptions create a devastating wave of burning ash, boiling gas and volcanic debris called a pyroclastic flow that can travel for miles. A pyroclastic flow is treated like an Avalanche traveling at 150 meters per round, combined with the effects of the poison gases noted above. Contact with the hot debris from the casting deals 2d6 points of fire damage per round, while any creature buried by the casting takes 10d6 points of damage per round.

\textbf{Tsunami} \index{Tsunami}

Tsunamis, sometimes attributed to tidal waves, are tremendous waves of water caused by underwater earthquakes, volcanic explosions, landslides or asteroid impacts. Tsunamis cannot be detected until they reach shallow water, when the body of water forms a large wave. Depending on the size of the tsunami and the slope of the coast, the wave can cover any distance, from hundreds of meters to over a kilometer on dry land, leaving a trail of destruction in its wake. The water then recedes, carrying away all sorts of debris and creatures to the high seas.

The exact devastation caused is subject to the Storyteller's discretion, but a typical tsunami knocks down or eradicates all temporary or poorly constructed structures in its path, destroying approximately 25 \% of well-constructed buildings (causing significant damage to those that remain ) and leaves the solid fortifications slightly damaged. At least 1/4 of the population living in the area (including animals and monsters) dies in the disaster, dragged into the sea, drowned on the beach or buried under rubble.

A creature can avoid being carried away by the sea on a DC 25 Swim check; otherwise it is dragged 6d6 x 3 meters from the shore. The waters after a tsunami are always considered rough or stormy, barring magical influences. A creature involved in the collapse of a building takes 6d6 points of damage (Reflex save DC 15 halves), or half if the structure is particularly small. there is a 50 \% chance that the creature will be buried (as in a Collapse), or that the tsunami will destroy the building, freeing the creature from the rubble.

\textbf{Undead Uprising} \index{Undead}

The result of an ancient curse or necromantic acts, one of the most terrifying supernatural disasters is the Undead uprising: the dead emerging from the grave to claim the living. This disaster can affect any area where the dead have been buried, not just towns and cities. More than one battlefield has seen a legion of withered undead fighters rise. Undead uprisings unfold in waves, with timing varying according to the main forces at play. Events can happen over the course of a few days, with the devastation of a city, or go on for weeks with the terrified population huddling behind locked doors and struggling to survive. During the day, life often reverts to a semblance of normalcy, as daylight temporarily suppresses the power of undeath.

\textbf{The Restless Dead}

In the first few nights of an Undead uprising, the recent dead revive like zombies. Those buried in consecrated earth do not revive, but bodies left unburied or in mass graves stagger out into the streets, wreaking havoc. Initially, only a few corpses are able to free themselves from their coffins and graves, but every evening, the number of living corpses increases. When dawn comes, the dead seek safety in their graves or other hidden places. Anyone caught by daylight agitates in confusion until they are destroyed or reach a refuge. At the Storyteller's discretion, corpses of non-humanoids may be resurrected as undead in the following nights.

\textbf{Awakening of the Skeletons}

As the uprising progresses, older and older corpses join the ranks of the undead. Skeletons bearing traces of long-rotted funeral robes dig a way out of cemeteries and crypts with their claws, and act with a malevolence and organization rarely found among their fellow men. The undead remain devoid of intelligence, but the magical power behind the raid gives them the efficiency and tactical acumen of an army of the living. Skeletons find weapons and armor to equip for battle. The elite of Champion Skeletons lead the troops, using magical items stolen from abandoned graves. Finally, Ghouls and Wights also roam the streets in the dark for prey, along with other lesser undead with free will.

\textbf{Lost Souls}

As the uprising gathers its forces, the restless souls of corpses long since reduced to dust are also awakening. Ghosts, Shadows, Wraiths and even Ghosts arise to hunt down the living. Some Ghosts might break free from the malevolent influence of the uprising and enterprising characters might glean valuable information from these restless spirits.

The infusion of negative energy fortifies undead within the raid area, granting the benefits of a blessing. Areas once consecrated are now treated as normal terrain, and can serve as new sources of corpses for undead armies; the sanctified ground remains inviolate.

As the undead grow stronger, the rising wave of negative energy draws the Plane of Shadows closer, fading or graying colors except during the brightest hours of the day. Even the undead most vulnerable to light can move with impunity from late afternoon to mid-morning.

\textbf{Necropolis}

The flow of negative energy is irreversible, the darkness finally reclaims the area, covering it with a perpetual shadow. The sanctified ground remains a rare sanctuary, but only until it is destroyed by the malevolent forces outside.

Heroes who died in battle return as fearsome undead generals. The few living survivors are enslaved as slaves. The area becomes a death city or construction begins if no city existed or survived. Undead with free will gather in this new sanctuary, and only the greatest heroes make it back from this withered area to the world of the living.

\end{multicols}

\vfill

\begin{center}
	\includegraphics[width = 0.45 \linewidth]{immagini/anubis.png}

	\textit{Representation \href{https://it.wikipedia.org/wiki/Anubi}{Anubi}}
\end{center}



\pagebreak

\section{Dungeon Adventures} \index{Dungeon}

\begin{changemargin}{0.3cm}{0.3cm} \begin{enfasi}{
Linux is user friendly. It's just very picky about who its friends are. (anonimo)

\medskip

The dungeon is tilted. The creatures are enraged because they fail to play marbles (Dungeon Keeper 2, Videogame, 1999)

} \end{enfasi} \end{changemargin} \medskip

\label{avventure-nei-dungeon}
\begin{multicols}{2}

\lettrine[lines = 2, lhang = 0.33, loversize = 0.25, findent = 1.5em]{O}{f} all the strange places an adventurer can explore, none are more deadly than a dungeon. These labyrinths, filled with deadly traps, hungry monsters and wondrous treasures, check every skill and ability of the characters. These rules apply to any type of dungeon, from a ship wreck to a vast complex of underground caves.

\begin{changemargin}{0.3cm}{0.3cm} \begin{Storytellere}
The dungeon, cave, tunnel, call it what you prefer is a cornerstone of the adventure!

A dungeon is a recipe made of humidity, stench, stale air, dirt, mud, remains of creatures, traps, treasures, traps (abound ...), monsters, enemies, monsters (abound!), Darkness, sinister noises, mushrooms , creaks, yelps, screams, moans .. but also of fear, tension, thrill terror \& horror, emphasis, ecstasy, pain, disappointment, joy and treasures!

Your dungeon is never just a cave. NEVER!
\end{Storytellere}
\end{changemargin}

Whether they are caverns, caverns, caves, lairs, caverns, the "Dungeons" often represent the focal point of adventure, of exploration.

Characters will spend a lot of time in these environments and the Storyteller must be prepared and ready for the environment they will encounter.

When preparing a cave it is necessary to think intelligently about the type of cave and the creatures that will be encountered.
Putting a group of lizards without thinking about what they eat, where they sleep, what kind of organization they have is dangerous, not to mention inserting a chimera.
Will his wings have atrophied because the cave is 3 meters high and 3 meters wide and it is hard to move? What did she feed on during this period? Better to use a gorgon that feeds on minerals ...

If designed with care and attention, a cave can become an excellent experience of encounters, situations and adventures.

\subsection{The subsoil} \index{The subsoil}

The natural conditions of the subsoil depend on various factors but there are certainly points in common.

- No lights to illuminate the spaces. There may be sporadic fluorescent mushrooms but nothing that can illuminate the whole environment

- Humid environment

- Usually cool room temperature, there are rarely caves with extreme temperatures in both heat and cold.

\subsubsection{Lighting} \index{Lighting}

There are no artificial or natural sources of light in a cave other than those introduced by sentient creatures. There may be groups of fungi, lichens, which dimly illuminate the ground where they grow but nothing else around.
Moreover, if torn from the ground they lose their bioluminescence after 2d4 turns.

Creatures that live in caves must have gotten used to darkness probably by developing some form of alternate vision, such as darkvision, telluric sense, or blind sight.

Anyone who is not seen has full cover and hiding in the shadows is done with a + 2d6 bonus.

Even the flashlight itself can provide limited relief as its beam of light is 10 feet plus 10 feet of dim light and lasts an hour before going out.

\subsubsection{Movement} \index{Movement}

If you do not have the means to see the terrain, it is considered difficult and holes, precipices and various obstacles can be very dangerous.

In total darkness and in a natural environment, a Dexterity check at DC 12 must be made every 30 meters or stumbling and taking 1 temporary damage.

\subsection{Types of caves}

Different types of caves can be identified:

\begin{itemize}


\item created by flowing water. In this case the tunnel can be quite chaotic in its unraveling due to the type of rocks that the water has encountered. There may still be underground rivers and lakes.

\item created by erosion. In this case the water is probably not there any more if not in part, the resulting caves can also be very large with salt of tens if not hundreds of meters in width.

\item may have been created by a volcano as lava flows. In this case the tunnel dug out of the rock is often linear and somewhat smooth, once the lava has congealed it has then crumbled over the millennia.

\item may be arctic caves, carved out of ice by water. In this case, take a good look at the surrounding environment and the freezing temperature.

\item may be man-made caves, built by creatures of various kinds.

\end{itemize}

\subsubsection{The four types of Dungeon} \index{The four types of Dungeon}

The four basic types of dungeons are defined by their current state. Many dungeons are variations of these basic types or combinations of multiple types. Occasionally, ancient dungeons are used repeatedly by new inhabitants for different purposes.

\textbf{Ruined Structure}: Once inhabited, this place is now abandoned (completely or in part) by its original creators and is occupied by other creatures. Many underground creatures go in search of underground and abandoned buildings in which to establish their lairs. Any traps that may have existed have probably already been removed or activated, but wandering beasts can be found.

\textbf{Facility Occupied}: This dungeon is still in use. Creatures (usually intelligent) still inhabit it, although they may not be the creators of the dungeon. An occupied structure could be a house, a fortress, a temple, an active mine, a prison, a headquarters.

This type of dungeon is less likely to have traps or wandering beasts, and more likely to have organized guards, both stationary and patrolling. Traps and wandering beasts that can be encountered are often under the control of the occupants. The occupied structures have furniture suitable for the inhabitants, as well as decorations, food reserves, and the possibility for the inhabitants to move.

\medskip
\begin{center}
	\includegraphics[width = 0.8 \linewidth]{immagini/avventure_dungeon.png}

	\textit{The Red Romance Book, Illustration by HJ Ford}
\end{center}
\medskip

The inhabitants can also have a communication system, and almost always control at least one access to the outside.

Some dungeons are partially occupied and partially empty or in ruins. In these cases, the occupants are usually not the original builders of the place, but rather a group of intelligent creatures who have established their base, lair or fortification within the abandoned dungeon.

\textbf{Safe Shelter}: When someone wants to protect something, they often bury it underground. Whether the object he wants to protect is a fabulous treasure, a forbidden artifact or the corpse of an important man, these valuables are placed inside a dungeon and surrounded by barriers, traps and guardians.

The safe shelter type dungeon is the one that will have more traps and fewer wandering beasts. It is normally built based on functionality rather than appearance, although it is sometimes decorated with statues and painted walls, especially for the graves of important characters.

\begin{center}
	\includegraphics[width = 0.8 \linewidth]{immagini/dungeon.png}
\end{center}

Sometimes, however, a treasure room or crypt is built to house living guardians. The problem with this strategy is that you have to keep creatures alive between one intrusion attempt and another. Magic is usually the best way to supply these creatures with food and water. Tomb and grave builders usually place undead and constructs, which need no sustenance or rest, to protect their dungeons. Magic traps can attack intruders by summoning monsters to the dungeon that disappear when they finish their task.

\textbf{Natural Cave Complex}: The underground caverns offer shelter to all kinds of creatures of the depths. Naturally created and connected by a system of labyrinthine passageways, these caverns lack any semblance of order, logic or decoration. With no intelligent power to build it, this type of dungeon is the least likely to have traps or doors.

Multiple varieties of fungi live in caves, sometimes growing to form huge mushroom and wasp forests, where underground predators roam in search of those who eat these plants. Some varieties of fungi produce a phosphorescent glow that can provide the natural cave complex with its own limited source of illumination. In other areas, the use of Daylight spells can provide enough light for green plant growth.

Often, a natural cave complex is linked to other types of dungeons, having been discovered when the man-made dungeon was built. A cave complex can connect two independent dungeons, sometimes producing a strange mixed environment. A natural cave complex coupled with another dungeon often offers a path that underground creatures can use to reach an artificial dungeon and populate it.

\subsection{Exploration} \index{Exploration} \index{Moving carefully}

Moving around a dungeon requires attention and a cool head. Rough floors, sinister noises, hatches and traps, lights that appear and disappear make it not easy to venture safely into these dangerous environments.

Characters will need to be alert, actively search for traps, look into the distance, and be cautious. All this means that the movement is halved if the characters \textit{take precautions} to avoid problems.

Describing what the character does to look for traps, passages .. \textit{problems} or requesting a check (Survival or Awareness) from DC 12 can give general indications about \textit{feeling} that something is wrong.

\subsection{Dungeon terrain} \index{Dungeon terrain}

The following rules cover the basic terrains that can be found in a dungeon.

\subsubsection{Walls} \index{Walls}

Sometimes, brick walls (stones stacked on top of each other usually, but not always, held together with lime) divide dungeons into corridors and rooms. Dungeon walls can also be carved out of bare rock, thus achieving a chiseled appearance, or they can be composed of smooth, plain stone as found in natural caves. Dungeon walls are difficult to damage or break through, but are usually easily scalable.

\end{multicols}
\textbf{Table: Walls} \index{Table Walls}
\medskip

\begin{tabularx}{0.95\textwidth}{XllllX}
\textbf{Wall Type} & \textbf{Thickness} & \textbf{Break Through} & \textbf{Hardness} & \textbf{Hit Points} & \textbf{DC to Climb} \\
\toprule
Bricks & 30 cm & 35 & 8 & 90 & 20 \\
Upper bricks & 30 cm & 35 & 8 & 120 & 25 \\
Reinforced bricks & 30 & 45 & 8 & 180 & 20 \\
Carved Stone & 90 & 50 & 8 & 540 & 25 \\
Raw stone & 150 cm & 65 & 8 & 900 & 25 \\
Iron & 7.5cm & 30 & 10 & 90 & 25 \\
Card & variable & 1 & - & 1 & 30 \\
Wood & 15 cm & 20 & 5 & 60 & 21 \\
\end{tabularx}

\medskip

\textbf{Table: Digging a tunnel} \index{Table Digging a tunnel}

\medskip

\begin{tabular}{llll}
\textbf{Miner} & \multicolumn{3}{c}{\textbf{Excavating Material - 1 minute}} \\
& \textbf{Ground} & \textbf{Stone} \textbf{soft} & \textbf{Hard stone} \\
\toprule
Human & 30cm & 15cm & 7cm \\
Gnome & 45cm & 30cm & 15cm \\
Dwarf / Ogre & 55cm & 45cm & 20cm \\
Stone Giant & 3m & 1.5m & 75cm \\
Xorn & 6m & 6m & 6m \\
Earth Elemental & 9m & 9m & 9m \\
\end{tabular}

\medskip

The indicated excavated distances are assumed to be obtained with spades or picks, otherwise reduce to one third.

\begin{multicols}{2}

\textbf{Brick Walls}: The most common type of wall for a dungeon, brick walls are usually at least 30 centimeters thick. Often these ancient walls have holes and cracks, inside which sludge and small creatures can nest, waiting there for their prey. Brick walls block out all but the loudest noises. A DC 20 Climb check is required to move along a brick wall.

\textbf{Superior Quality Brick Walls}: Sometimes brick walls are better constructed (smoother, with better interlocking stones and less damaged) and occasionally these higher quality walls are covered with mortar or stucco. These walls are often embellished with paintings, bas-reliefs or other decorations. Higher quality brick walls are no harder to damage than regular brick walls, but they are harder to Climb (DC 25).

\textbf{Reinforced walls} These are brick walls with iron bars on one or both sides, or inserted inside the wall itself to reinforce it. The hardness of the reinforced wall remains the same, but the hit points are doubled and the DC for the Strength check to break through it is increased by 10.

\textbf{Carved Stone Walls}: These walls are generally found in rooms or passages carved out of bare rock. The rough surface of a carved wall has tiny protrusions on which fungi can grow and cracks within which parasites, bats or underground snakes can live.

When such a wall has an "other side" (the wall separates two rooms in a dungeon), the wall is at least 90 centimeters thick; if it were thinner it would risk making everything collapse because it would not be able to support the weight of the stone vault. A DC 25 Climb check is required to scale a carved stone wall.

\textbf{Rough Stone Walls}: These surfaces are irregular and rarely flat. They are smooth to the touch but filled with tiny holes, hidden alcoves and ledges at various heights. They are usually wet or at least humid, as natural caves are usually the product of water infiltration. When such a wall is on the "other side", the wall is usually at least 150 centimeters thick.

a DC 15 Climb check is required to move along a rough stone wall.

\textbf{Iron Walls}: These walls are placed inside dungeons around important locations such as treasure halls.

\textbf{Walls of Paper}: The walls of paper are the opposite of those of iron, used as screens to block the view but nothing more.

\textbf{Wooden Walls}: Wooden walls are often found as recent additions to older dungeons, used to create animal enclosures, storerooms, or even just to divide a series of smaller rooms into a larger one.

\textbf{Magically Treated Walls}: These walls are stronger than average, with higher Hardness, with more Hit Points and a higher DC must be overcome to break through. Magic can usually double the wall's Hardness and Hit Points and add up to +20 to its DC to break through. A magically treated wall also gains a saving throw against spells that might affect it, with the saving throw bonus equal to 2 + half the caster level of the spell reinforcing the wall. Creating a magical wall requires the Craft Wonder Item feat and the cost of 1,500 gp for each 3-by-3-meter section.

\textbf{Louvered Walls}: Louvered walls can be constructed of any strong material, but are usually made of brick, carved stone, or wood. They allow defenders to shoot crossbow bolts or arrows at intruders while remaining behind the relative protection of a wall. Archers behind the loopholes enjoy superior cover that gives them a +8 bonus on Defense, a + 1d6 bonus on Reflex saving throws.

\subsubsection{Floors} \index{Floors}

As with walls, there are many types of dungeon floors.

\textbf{Paved}: Like brick walls, floors can consist of interlocking stones. They are usually filled with cracks and usually barely leveled. Mud and mold grow inside these crevices. In some cases the water flows in small drains through the stones or forms stagnant pools. Paving is the most common type of flooring in dungeons.

\textbf{Rough Paving}: Over time, some floors can become so uneven that they require a DC 10 Acrobatics check to run or Charge on their surface. Those who fail the check cannot move during that round. Such dangerous floors should actually be the exception and not the rule.

\textbf{Carved Stone Floor}: Rough and uneven, carved stone floors are usually covered with loose stones, gravel, dust and other debris. A DC 10 Acrobatics check is required to run or Charge on such a floor. Failure means that the character can still act, but cannot run or Charge in that round.

\textbf{Scarce Gravel}: Small and sparse debris is present on the ground. A pavement with scant cobblestones adds 2 to the Acrobatics check DC.

\textbf{Dense Crush}: The ground is covered with debris of all sizes. Rubble is considered difficult terrain. A dense crushed stone floor adds 5 to the DC of Acrobatics checks, and adds 2 to the DC of Awareness checks (Move Silently)

\textbf{Smooth Stone Floor}: Smooth, flawless and sometimes even polished floors are found only in dungeons created by skilled and careful builders.

\textbf{Natural Stone Floor}: The floor of a natural cave is as uneven as the walls. It is unlikely that these caves have large flat surfaces; their floors are more likely to be layered.

Some adjacent surfaces may vary in elevation by just 30 centimeters, so that moving from one point to another is no more difficult than climbing a step on a ladder, but in certain points the floor could go down or up several tens of centimeters, forcing the character on a Climb check to move from one surface to another.

Unless there is a path carved by time or well-trodden the terrain is considered difficult and therefore movement is halved, the DC of Acrobatics checks is increased by 5. Charging and running in these environments are impossible, except that on the routes in question.

\begin{center}
	\includegraphics[width = 1 \linewidth]{immagini/pavimento_grey.png}
\end{center}

\textbf{Slippery}: Water, ice, slime, or blood can make any floor described in this section more treacherous. Slippery floors increase the DC of Acrobatics checks by 5.

\textbf{Grate}: A grate often covers a pit or area below the main floor. The gratings are usually constructed of iron, but the larger ones could also be made of reinforced tree trunks. Many grates have hinges that allow access to the area below (these grates can be locked like a door), while others are fixed and created so that they cannot be moved. A typical 2.5cm thick iron grate has 25 Hit Points, Hardness 10, and DC 27 to break through or dislodge it.

\textbf{Ledges}: Ledges allow creatures to walk over an area below. They are often arranged around pits along underground rivers as balconies that surround a large room or provide a location from which archers can lurk to attack enemies from above.

Narrow ledges (less than 30 centimeters wide) require acrobatics checks from those moving over them. Failure implies that the moving character falls off the ledge.

Sometimes the ledges have a railing. In these cases the characters gain a +5 bonus on Acrobatics checks for moving along the ledge. A character near the railing has a +2 bonus on their opposed Strength check to avoid being pushed off the ledge.

The ledges can sometimes also be bordered by balustrades 60-90 centimeters high. Such walls provide Cover from attackers within 10 feet of the other side of the wall, provided the target is closer to the railing than the attacker.

Transparent Floors: Transparent floors made of reinforced glass or magical materials allow you to observe a dangerous environment from above. Transparent floors are usually placed on top of lava pits, arenas, monster lairs and torture chambers and can be used by defenders
to monitor an area.

\textbf{Sliding Floors}: A sliding floor is a type of trap door, created to be moved and reveal something underneath. Typically a sliding floor moves so slowly that anyone on it can avoid falling into the opening, as long as they have room to move. If such a floor flows so fast that there is a chance that a character will fall into the one beneath it (sharp spears, a hot oil tank, or a shark-infested pool) then it is a trap.

\textbf{Trap Floors}: These floors have been designed to suddenly become dangerous. With the application of the right amount of weight or the operation of a lever nearby, spikes burst from the floor, flames or puffs of steam depart from hidden holes, or the entire floor moves. These strange floors are usually found inside arenas, designed to make combat more exciting and deadly. This type of floor is built in the same way as a trap.

\subsection{The doors} \index{The doors}

\textbf{Doors} \index{Doors} The doors inside dungeons are more than just entrances or exits. Often they can be real encounters. Dungeon doors come in three basic types: wood, stone, and iron.

\end{multicols}

\textbf{Table: Doors} \index{Table Doors}

\medskip

\begin{tabular}{llllll}
\textbf{Door Type} & \textbf{Typical Thickness (cm)} & \textbf{Hardness} & \textbf{Hit Points} & \multicolumn{2}{c}{\textbf{DC to break through}}\\

&& & & Blocked & Key Locked \\
\toprule
Plain wood & 2.5 & 5 & 10 & 13 & 15 \\
Good Wood & 3.75 & 5 & 15 & 16 & 18 \\
Sturdy wood & 5 & 5 & 20 & 23 & 25 \\
Stone & 10 & 8 & 60 & 28 & 28 \\
Iron & 5 & 10 & 60 & 28 & 28 \\
Wooden gate & 7.5 & 5 & 30 & 25{*} & 25{*} \\
Iron gate & 5 & 10 & 60 & 25{*} & 25{*} \\
Lock & - & 15 & 30 & - & - \\
Hinges & - & 10 & 30 & - & - \\
\end{tabular}

\medskip

{*} DC to lift. Use the appropriate door entry to break through.

\begin{multicols}{2}

\bigskip

\textbf{Wooden Doors}: Constructed of thick nailed planks, sometimes reinforced with iron bars (also placed to prevent the deformation produced by the humidity of the dungeons), wooden ones are the most common type of door. Wooden doors vary in hardness - they can be simple, good, or sturdy. Simple doors (DC 15 to break through) are not designed to keep motivated assailants at bay.

Well-made doors (DC 18 to break through), while strong and tough, are still not designed to take a great deal of damage. The sturdy doors (DC 25 to break through) are iron clad and are fairly strong barriers against those who try to break through. Iron hinges support the door, and usually a circular ring in the center is used to open it. Sometimes, instead of a ring, a door has an iron bar on one or both sides that functions as a handle.

In inhabited dungeons these doors are usually well-kept (unblocked) and not locked, although important areas will likely be locked.

\textbf{Stone Doors}: Constructed from solid stone blocks, these heavy and unwieldy doors are often designed to rotate when opened, although dwarves and other skilled craftsmen are able to build hinges strong enough to support the weight of a stone door.

Secret doors hidden along a stone wall are usually made of stone. Otherwise, doors of this type are designed to become strong barriers that protect anything beyond them. As a result, they are often found locked or barred.

\textbf{Iron Doors}: Rusty but tough, iron doors in a dungeon are hinged like wooden doors. These doors are the strongest doors of the non-magical type. They are usually locked or barred.

\begin{center}
\includegraphics[width = 0.85 \linewidth]{immagini/porta_grey.png}
\end{center}

\textbf{Break Through}: Dungeon doors can be locked, trapped, reinforced, barred, magically sealed, or sometimes simply locked.

Everyone except weaker characters will be able to knock down a door with a heavy tool like a mallet, numerous spells and magical items can offer characters an easy way to get through a closed door.

\textbf{DC 10 or lower}: A door that anyone can break through.

\textbf{DC 11--15}: A door which a strong person would have to break through in one attempt, and which a person of average strength might have some hope of breaking down in one blow.

\textbf{DC 16--20}: A door that practically anyone could break through, given enough time.

\textbf{DC 21--25}: A door that only a strong or very strong person has a hope of breaking through, and probably not on the first try.

\textbf{DC 26 or higher}: A door that only a person of exceptional strength can have any hope of breaking through.

\textbf{Locks}: Dungeon doors are often locked and so the Disable Gadgets skill comes in handy. The locks are recessed on the edge opposite the hinges or straight in the center of the door. Locks usually control an iron or wooden bar that extends from the door into the wall that supports it.

Padlocks secure between two rings, one on the door and one on the wall. More complex locks, such as combination or puzzle locks, are usually built inside the door itself.

The DC for picking a lock with a Deactivate Gadgets check often falls between 15 and 30, although locks with higher or lower DCs exist. A door can have more than one lock, each of which must be opened separately. \index{Pick a door}. Breaking into a door without burglary tools carries a -1d6 penalty to the check.

Locks are often equipped with traps, usually poisoned needles that snap out to prick the thief's fingers.

\subsubsection{Breaking a lock} \index{Breaking a lock}

A special door might have a keyless lock, but one that requires you to guess the right combination of nearby levers or press the symbols on a panel in the correct order to be able to open it.

\textbf{Locked Doors}: Dungeons are often damp places, and in some cases doors get locked, particularly if they are made of wood. It is usually assumed that approximately 10 \% of wooden doors and 5 \% of other doors are locked. These values can be doubled (to 20 \% and 10 \% respectively) in the case of long-abandoned or neglected dungeons.

\textbf{Barred Doors}: When a character tries to break through a barred door, it is the quality of the bar that makes the difference, not the material of the door itself. Breaking through a door closed by a wooden bar requires a Strength check with DC 25, and the DC rises to 30 in the case of a metal bar.

Characters can attack the door and destroy it, leaving the bar hanging in the cleared passage. Using a crowbar to force a wedged / locked door grants a + 1d6 check. \index{Crowbar on door}

\textbf{Magic Seals}: Spells placed on a door can make it difficult to walk through a door.

A door on which a magic lock has been thrown is considered closed even if it does not physically have a lock. You need a spell that will crack or destroy spells or a successful Strength check to get through a locked door this way.

\textbf{Hinges}: Most doors have hinges. Obviously the sliding doors are not (these are rather equipped with grooves in the floor, which allow them to slide to the side with ease).

\textbf{Standard Hinges}: These hinges are made of metal and hold the door together with its support or wall. Remember that the door opens towards the side where the hinges are (so if the hinges are on the PCs' side, the door will open towards them; otherwise it will open towards the other direction).

Adventurers can remove the hinges one at a time by passing various Disable Gadgets checks (only if, of course, they are in front of the side of the door the hinges are on). Such an action has a DC of 20, as many of the hinges are rusted or stuck.

Breaking a hinge is difficult. Most have Hardness 10 and 30 Hit Points. The DC to break a hinge is the same as that used to break down the door

\textbf{Insertion Hinges}: These hinges are much more complex and are only found in areas of excellent construction. These hinges are built into the wall and allow the door to open in both directions. Characters cannot reach for the hinges to remove them unless they break through the door post or wall. Push-in hinges are usually found on stone doors, but are also sometimes seen on wooden or iron doors.

\begin{center}
\includegraphics[width = 0.9 \linewidth]{immagini/cardini.png}
\end{center}

\textbf{Pins}: Pins are not real hinges, but simple pegs that extend from the top and bottom of the door and slide into the holes in its holder, allowing it to turn. The advantages of pins are that they cannot be removed like hinges and are easy to make. The downside is that because the door pivots on its center of gravity (usually in the middle), nothing larger than half the width of the door can pass through it.

Pivoted doors are usually made of stone and often wide enough to overcome the disadvantage. Another solution is to place the pin towards one end and make the door thicker on that side and thinner on the other, so that it opens more or less like a normal door.

Secret doors within walls often rotate, as the lack of hinges makes it easier to conceal the presence of the door. The pins also allow objects such as a bookcase to be used as secret doors.

\textbf{Secret Doors}: Disguised as a common portion of wall (or floor or ceiling), a bookcase, a hearth, a fountain, a secret door leads to a secret passage or room.

Someone examining the area may find a secret door (if one exists) with a successful Awareness check (DC 20 for a common secret door and DC 30 for a very well hidden door).

Many secret doors require a special method to open, such as a hidden button or pressure plate. Secret doors can open like common doors, pivot on a pivot, slide, sink, rise, or even lower like a drawbridge.

A builder could place a secret door very low near the floor or very high on a wall, making it more difficult to find and use the door.

\begin{center}
\includegraphics[width = 0.6 \linewidth]{immagini/arcoserpenti.png}

\textit{Henry Justice Ford}
\end{center}


\textbf{Magical Doors} Enchanted by the original builder, a door can apostrophize the explorers telling them not to continue. It could be protected from damage, with higher Hardness or more hit points, as well as an improved saving throw bonus. A magical door may not lead to the space behind it, but actually be a portal to a very distant place or even to another plane of existence. Other magical doors may need a password or special keys to open.

\textbf{Shutters}: These special doors are made of rods of iron or thick reinforced wood that drop from a recess in the upper part of an arch. Sometimes a penstock has horizontal bars to form a grid, other times not. Raised usually with a winch or similar machinery, the gates can be lowered quickly, and the bars terminate in spikes to discourage anyone from walking underneath (or attempting to run through them as they descend). Once down, a shutter closes, unless it is so large that no normal person would be able to lift it. In any case, lifting a typical penstock requires a Force check with DC 25.

\textbf{Walls, Doors and Identification actions}

The walls of stone, iron, and iron doors are generally thick enough to block most Divinations. Wooden walls, wooden doors, and stone doors are generally not thick enough to do the same. However, a secret stone door built into a wall and as thick as the wall itself (at least 30 centimeters) will block most of these Actions.

\textbf{Stairs} The most traditional method of connecting different levels of a dungeon is via stairs. A character can climb or descend a ladder as part of his movement without penalty but cannot run. Increase the DC of any Acrobatics check made on a scale of 4. Some particularly steep stairs are treated as difficult terrain.

\subsection{Dungeon dangers} \index{Dungeon dangers}

In dungeons and caves, in addition to monsters, there are also other dangers including collapses, molds, fungi and more.

\subsubsection{Collapses and Settlements (Challenge grade 8)} \index{Collapses and Settlements}

Collapses and subsidence in tunnels are extremely dangerous. Not only do dungeon explorers run the risk of being crushed by tons of stone, but also, should they survive, of getting stuck under a pile of debris or being unable to reach an exit.

A collapse buries anyone in the middle of the buried area, and then the debris that rolls away will inflict damage on anyone in the peripheral areas of the buried area. A typical collapsed corridor might have a 3 meter radius buried zone and a melee radius slip zone at the end of the buried one.

A crumbling ceiling can be identified with an Engineering check with a DC 20 or a Mason Profession with a DC 20. A Dwarf can perform this check simply by walking within 10 feet of a crumbling ceiling.

A crumbling ceiling can collapse under the impact of a large force. A character can cause a collapse by destroying half of the pillars supporting the ceiling.

Characters in the buried zone take 8d6 points of damage, or half damage if they succeed on a DC 15 Reflex save. They are then buried. Characters in the scroll zone take 3d6 damage, or no damage if they make a DC 15 Reflex save. Characters in the scroll zone are also buried if they fail the saving throw.

Characters who are buried take 1d6 points of nonlethal damage for every minute they remain under the rubble. If a character in this condition falls unconscious, he must make a DC 15 Constitution check. If the character fails the check, he begins to take 1d6 lethal damage per minute until he is released or dies.

Characters who have not been buried can extract their companions from under the rubble. In 1 minute, using only their hands, a character can move an amount of rock and debris equal to five times their heavy load limit. The amount of loose rock that fills a scrum area weighs approximately 1 ton (1000 kg). Equipped with the proper tools, such as a pickaxe, crowbar, or shovel, a digger can take half the time it would take to do it by hand. A buried character could also be allowed to free himself by passing a Strength check with DC 25.

\subsubsection{Sludge, Mold and Mushroom} \index{Sludge, Mold and Mushroom}

In the humid and dark recesses of the dungeons, molds and fungi thrive, fear the columns of mold! As for spells and other special effects, all sludge, mold and mildew are considered plants. Like traps, dangerous muds and molds come with a Challenge rank, and characters earn Experience Points for encountering them.

A shiny organic slime covers anything that remains immersed in the darkness and humidity of dungeons for too long. This type of sludge, while it may be repellent, is not dangerous. Mold and mildew abound in dark, cold and humid places. While some are as harmless as normal dungeon sludge, others are somewhat dangerous. Edible mushrooms, vesus, yeasts, molds, and other types of fibrous, bulbous fungi or entire stretches of fungal spores can be found in most dungeons. They are usually harmless and are often edible as well (although most are unappealing or have a strange taste).

\textbf{Strident Boleto} \index{Strident Boleto}: These human-sized purple mushrooms emit a piercing sound that lasts 1d3 rounds whenever there is movement or a source of light within 10 feet. This cry makes it impossible to hear any other sounds or noises within melee range. The sound draws nearby creatures who are willing to investigate. Some creatures that live near screeching boletes have learned that noise very often means food.

\begin{center}
	\includegraphics[width = 0.9 \linewidth]{immagini/funghi.png}

	\textit{They are bright in the dark, believe me! and fried are even better!}
\end{center}

\textbf{Green Slush} \index{Green Slush} (Challenge Rank 4): This dungeon hazard is a treacherous variety of ordinary slush. Green sludge devours meat and organic materials that come into contact with it, and is even capable of dissolving metals. Bright green, wet and sticky, it spreads out in patches on walls, floors and ceilings and reproduces by consuming organic material. It drops off walls and ceilings when it detects movement (and possible nourishment) beneath it.

Green sludge deals 1 Constitution damage for each round it devours meat. On the first round of contact, the slime can be removed from a creature (with the likely destruction of the object used to remove it), but after the first round it must be frozen, burned, or cut (dealing damage to its victim as well) to be removed. . Anything that deals fire or cold damage, sunlight, or a disease-removing spell destroys a patch of green sludge. In the case of wood or metal, green slime deals 2d6 points of damage per round, ignoring the hardness of the metal but not the hardness of the wood. It does not damage the stone.

\textbf{Phosphorescent Mushroom} \index{Phosphorescent Mushroom}: This strange underground mushroom gives off a faint purplish glow that illuminates caves and underground passages like a candle. Rare spots of this fungus light up like a torch.

\textbf{Yellow Mold} \index{Yellow Mold} (Challenge rank 6): If disturbed, within 10 feet it releases a cloud of poisonous spores. Anyone within 10 feet of the mold must succeed at a DC 15 Fortitude save or take 1d3 points of Constitution damage. Another DC 15 Fortitude save is required once per round for the next 5 rounds or to avoid taking another 1d3 points of Constitution damage. A successful saving throw blocks this effect. Fire destroys yellow mold, while sunlight renders it inert.

\textbf{Brown Mold} \index{Brown Mold} (Challenge Rank 2): Brown mold feeds on heat, drawing it from everything around it. It usually occurs in patches with a melee size diameter and the temperature around the mold is always cold within a radius of 3 meters. Living creatures within melee range of it take 3d6 points of nonlethal cold damage. If a fire source is brought into melee by the mold, it immediately doubles its size. Cold damage, such as that inflicted by a cone of cold, destroys it instantly.

\subsubsection{Example of Dungeon Traps} \index{Example of Dungeon Traps}

The name of the trap is indicated, the DC for the Survival check to find the trap and the indications for its use. \\

\textbf{Flooded room, DC 17}: If the characters do not notice the pressure plate on the floor it will seal the entrance door and the room will begin to fill with water.
The room fills with water in 10 rounds. A DC 15 Survival check, combined with a DC 12 Swim check, detects the plate that triggers the water leak.

\textbf{Crushing room, DC 15}: If the characters do not notice the pressure plate on the floor it will seal the entrance door and loud noises of screeching and gears will fill the room. The walls will begin to come closer together like the ceiling to the floor. If the characters do not find the hidden tile (DC 17) they will take 10d6 of crush damage. The trap is easier to detect than others because the walls are thicker making the room smaller.

\textbf{Crushing ceiling, DC 18}: If the characters do not notice the activation system (pressure plate, cable, interrupted light beam ...) a section of the ceiling of 3m x 3m will fall on the characters with damage of 3d6.

\textbf{Tunnel of cobwebs, DC 12}: This tunnel is evidently full of thick, dense, robust webs. If characters enter they are considered entangled. After 1d4 rounds of permanence an activator will generate a spark and set the webs on fire. Each round inside the tunnel takes 2d4 fire damage.

\textbf{Pit, DC 15}: The inattentive character will collapse a 3m x 3m section of floor onto a pit. This can be a simple pit (1d6 fall damage), with spikes (1d6 + 2d4), with acid (1d6 per round), with undead ...

\textbf{Garrotte, DC 14}: This trap can be very insidious. A magically sharpened wire is 1 meter above the ground, between one wall and the opposite one, and runs towards the players.
A DC 13 Athletics check is required or suffer 2d6 slash damage.

\textbf{Crushing post, DC 16}: This newly touched door swings on central hinges and whirls beats the character (or characters if a large door). Deals 1d6 hit damage and continues spinning for 1d6 rounds.

\textbf{Finger Cutter, DC 14}: This trap is very subtle. It has a hole of about 1 cm in diameter and 7 cm deep. Anything that touches the bottom will trigger the trap, causing 2d4 damage to the inserted finger / object. The blade could also be poisoned.

\end{multicols}

\pagebreak

\section{Dangers in Adventure} \index{Dangers in Adventure}


\begin{changemargin}{0.3cm}{0.3cm} \begin{enfasi}{
An adventure is a reasonable result. Two are better, three deserve to be passed down, and four ... no one will ever contest four adventures. (John Steinbeck)

The one who, even if he is safe, is on guard is less in danger. (Publilius Syrus)
} \end{enfasi} \end{changemargin} \medskip

\label{pericoli-in-avventura}

\begin{multicols}{2}

\lettrine[lines = 2, lhang = 0.33, loversize = 0.25, findent = 1.5em]{T}{he} world is full of dangers as well as dragons and ravenous filthies. The dangers are threats based on the peculiarities of the area that have a lot in common with traps, but which are usually part of the place rather than being built. The dangers fall into three main categories: environmental, living and magical.

Environmental hazards include landslides, fires and the like. Living dangers include creatures that, while not considered monsters, pose a threat to unwary adventurers, such as mud, fungus, and moss. Magical dangers are the most unpredictable and can be remnants of arcane experiments, strange dungeon radiation, or failed ancient spells.

\medskip

\begin{center}
\includegraphics[width = 0.8 \linewidth]{immagini/boscopericoli.png}
\end{center}

\textbf{Antidweomer (Challenge grade 6)} \index{Antidweomer}

Zones of magical entropy destroying spells, antidweomers form on the sites of great magical duels, through the destruction of powerful artifacts or from vortexes of mystical energy at the edge of antimagic zones. The sizes vary from small bubbles of just a few meters to large areas the size of a city.

A successful Arcana check with DC 20 reveals the proximity of an antidweomer with a tingle in the air. An active spell carried into an antidweomer may be dispelled, and any spells cast within it are subject to an immediate counter-spell. If a critical is rolled in the Magic Check, it passes the counter spell but generates no further effects.

If the spell fails, the release of magical energy deals 2d6 points of force damage in an explosion within 10 feet centered on the attempted spell; a DC 15 Reflex saving throw reduces this damage in half.

A spell manifested by an object always fails.

If multiple overlapping bursts hit the same target, only the most damaging one applies. A spell that has resisted a dissolution attempt is not affected again unless it exits and re-enters the antidweomer.

The more powerful antidweomers are even more destructive. Each +1 increase in Challenge rank increases the DC damage of the saving throw by 1d6.

\medskip
\textbf{Stale Air (Challenge grade 1 or 4)} \index{Stale Air}

An invisible danger, the gas pockets are a risk to miners, speleologists and adventurers who investigate caves. Non-flammable gases such as carbon dioxide or nitrogen are Challenge 1 grade and require a DC 25 Survival check to be noticed.

Creatures that breathe that air must make a Fortitude saving throw (DC 15 +1 on each previous roll) every hour or become fatigued. Once Fatigued, they begin to Choke Slowly. Creatures that hold their breath can avoid these effects.

Flammable vapors such as coal gas are much more dangerous (Challenge grade 4). This gas replaces the breathing air in the lungs, causing fatigue: in addition, any open flame or spark causes an explosion that deals 6d6 points of damage (Reflex save with DC 15 halves) to anyone in the cave or within 10 feet of an entrance. Fire burns oxygen in the air, making it unbreathable for 2d4 minutes. After an explosion, flammable gas generally takes several days to return to dangerous levels.

\medskip
\textbf{Parasites} \index{Parasites}

Parasites such as ear seekers or necrophagous larvae cause parasites, a type of Affliction similar to Diseases. Parasitosis can only be cured through specific treatments; no matter how many saving throws you make, the parasitosis continues to plague the target. Even if a Disease Remover (or similar effect) immediately kills a parasite, disease immunity offers no protection, as it is caused by parasites.

\medskip
\textbf{Earmuffs (Challenge rank 5)} \index{Earmuffs}

Ear seekers are tiny white worms that live in rotten wood or other organic debris. They can be seen with an Awareness check (DC 15). Otherwise, a living creature rummaging in their lair will inadvertently move onto one or more ear seekers, which then look for a warm area on the creature's body, favoring the ear canal, and lay 2d8 eggs there before dying.

The eggs hatch 4d6 hours later and the larvae devour the surrounding meat. When their host dies, the worms crawl out and look for a new one.

Remove Disease kills all ear seekers or eggs not yet hatched on a host. Some ear seekers prefer to live in corrupted wood, often hiding in dungeon doors. The small holes left by this variant are very difficult to notice (Awareness DC 20).

\medskip
\textbf{Ear Finder}

Type: Parasitosis

TS: Quench DC 15

Onset: 4d6 hours

Saving Throw Frequency: 1 per hour

Effects: 1d3 to Constitution on a failed saving throw

\medskip
\textbf{Mnemonic Crystals (Challenge grade 3)} \index{Mnemonic Crystals}

Mnemonic crystals are large (3-12 meters high) clusters of purple quartz crystals that radiate a strong alteration aura. To identify them, you need an Arcana check with DC 25.

Mnemonic crystals accumulate magical energy to grow and defend themselves, draining prepared spells from spellcasters who must make a DC 22 Will save each round while within 10 feet of the crystals.

If the roll fails, they lose 1 available spell slot (one less castable spell). By damaging or breaking crystals, absorbed spells are expelled with a burst of mental energy that deals 1d4 points of Wisdom damage to anyone within 20 feet of range.

Mnemonic crystals are very fragile (Hardness 0, 1 Hit Point).
In crystal-rich areas, creatures passing through must pass a DC 10 Acrobatics check to avoid walking on or smashing them.

\medskip
\textbf{Necrophagous Larvae (Challenge grade 4)} \index{Necrophagous Larvae}

Once occupied with a living body, the larvae burrow towards the host's heart, brain, and other key internal organs, eventually causing death.

In the first round of parasitosis, applying fire to the entrance hole can kill the larvae and save the host, but the host takes 1d6 points of fire damage.

Extracting them also works, but the longer the larvae stay in the host, the more damage this method does. Extracting the larvae requires a slashing weapon and a First Aid check with DC 20, dealing 1d6 points of damage for each round the host has been afflicted with parasites. If the First Aid check is successful, a larva is removed. Remove Disease kills all necrophagous larvae in a host.

\medskip
\textbf{Necrophagous Larvae}

Type: Parasitosis

TS: Quench DC 17

Onset: immediate

Frequency: 1 / round

Effects: 1d2 Constitution damage per larva

\medskip
\textbf{Magnetized Ore (Challenge grade 2)} \index{Magnetized Ore}

The strange energies of the underworld can charge stones and mineral veins with powerful magnetic fields, creating a danger to anyone wearing or wearing ferrous metals. All iron or steel things brought within 10 feet of the ore are pulled towards it.

\begin{center}
\includegraphics[width = 0.8 \linewidth]{immagini/neodimio.png}

\textit{Neodymium}
\end{center}


Small creatures are also dragged with 7.5kg of metal, Large creatures only with 30kg. For creatures of other sizes, the weight changes based on the Carrying Capacity rules. Creatures wearing metal armor take a penalty, whoever is hit is dragged up to 30 feet, takes 2d6 points of damage from the impact with the rock, and is considered grabbed. Releasing a hit object requires a DC 20-25 Strength check

\medskip
\textbf{Cursed Chicken (Challenge grade 3)} \index{Cursed Chicken}

The prolonged effects of ancient curses or the noxious energy that spreads from a submerged cursed magical object can turn a mere pool of water into a risky magical hazard. A cursed pool lures passersby into its depths through the illusion (save on Will with DC 16 for doubting) of a glittering treasure on the 10ft deep bottom. Any creature that reaches the treasure activates the curse.

A creature inside the pool must succeed on a DC 16 Will save or is struck by the curse, which distorts its perception of the pool. The water seems to thicken into a viscous sapropelite (NdA: also sapropel or fetid slime, used in geology to indicate a blackish, pasty and more or less constipated slush, originating from the deposition of remains of organisms in stagnant or slightly moved waters mixed with calcareous or siliceous shells of microorganisms and clayey substances), while the pool seems to reach a depth of 12 meters.

Pool Swim checks take a -10 penalty, speed is reduced to half normal due to these effects, and you are considered Distracted when casting spells.

A cursed pool radiates strong magic, and can be destroyed by Magic Destruction or Remove Curse.

\medskip
\textbf{Poison Oak (Challenge grade 1 or 3)} \index{Poison Oak}

Contact with a poison oak (Challenge grade 1) causes a painful itchy rash that makes the victim sick until the damage heals. Full body contact or inhalation of smoke from a burning poison oak could be fatal (Challenge rank 3). A check of Nature (or Herbalism) with DC 15 reveals the dangers inherent in the apparently harmless plant. This danger can also be used for similar noxious plants (poison ivy, poison sumac or stinging nettles, but the latter are not dangerous when they burn).

\textbf{Poison Oak}

Type: Poison, contact

TS: Temper DC 13

Onset: 1 hour

Effects: Dexterity damage 1d4, creature is sickened until damage heals

Heal: 1 TS


\subsection{Getting ready for the night} \index{Getting ready for the night} \index{Sleeping} \index{Guard duty}

Every adventurer must rest every now and then, he must do it carefully and being careful not to run into nasty and dangerous surprises.

Whenever a character ends a 24-hour period without sleeping for at least 8 hours, he must succeed at a DC 12 Fortitude save, otherwise he becomes fatigued.

Any further missed rest will make him even more fatigued by accumulating the relative penalties. If the character stays awake for several days, fighting sleep becomes more difficult. After the first 24 hours, the DC increases by 4 for each consecutive 24-hour period without sleep for 8 hours. The DC returns to 12 when the character completes a rest of at least 8 hours.

Sleeping in medium or heavy armor makes you Fatigued, unless you have the Skill \hyperlink{secondapelle}{Seconda pelle}

\subsubsection{Organizing Guard Shifts}

If the group is large, the guard shifts to keep watch and control the environment become shorter.

\medskip{}

\textbf{Table: Duration of guard shifts} \index{Table: Duration of guard shifts}

This table indicates the duration of the guard shifts and the total rest time of the group, assuming to rest at least 8 hours.

\medskip{}

\begin{tabularx}{0.45\textwidth}{XXX}
\textbf{Members} & \textbf{Duration} & \textbf{Duration} \\
\textbf{of the group} & \textbf{of the Turn} & \textbf{Total} \\
\textbf{2} 	& 8 h 	& 16 h \\
\textbf{3} 	& 4 h & 12 h \\
\textbf{4} 	& 2 h and 30 min. & 10 h and 30 min. \\
\textbf{5} 	& 2 h 	& 10 h \\
\textbf{6} 	& 1 h and 30 min. & 9 h and 30 min. \\
\end{tabularx}

\medskip{}

An abrupt noise grants a DC 15 Consciousness check, or equal to the opponent's Move Silently +5 check, to wake up.

\end{multicols}

\vfill

\begin{center}
	\includegraphics[width = 0.6 \linewidth]{immagini/ederavelenosa.png}
\end{center}

\pagebreak

\subsection{Adventures and Traps} \index{Traps}

\begin{changemargin}{0.3cm}{0.3cm} \begin{enfasi}{
Whoever sets the trap in the same place will not catch any iguana. (African proverb)
} \end{enfasi} \end{changemargin} \medskip


\begin{multicols}{2}

A trap can be found almost everywhere. Traps can be magical or mechanical in nature. Mechanical traps include pits, arrows, falling boulders, water-filled rooms, spinning blades, and anything else that depends on a mechanism to operate. Magic traps are magical trap devices or trap spells. Magic trap devices when activated generate the effects of a spell. Trap spells are spells like ward glyph and symbol that function like traps.

\textbf{Traps in the Game}
When adventurers come across a trap, you should know how the trap activates and what it does, as well as having an idea of how the characters can spot the trap and be able to disarm or avoid it.

\subsubsection{Activate a Trap}
Most traps are activated when a creature gets to a spot or touches something that the creator of the trap wanted to protect. Normal activation systems are pressure plates or false floor sections, pulling a cable, turning a handle and using the wrong key in the lock. Magic traps often activate when a creature enters an area or touches an object. Some magical traps (such as the disqualifying glyph spell) have more complex activation conditions, including the use of passwords to prevent the trap from being activated.

\subsubsection{Detect and Disable a Trap}
Usually, some elements of a trap are clearly visible on close inspection.

The trap description specifies the checks and DCs required to detect, disable, or both. A character actively seeking a trap can attempt a \textbf{Survival} check against the DC of the trap.

The Storyteller can also compare the DC to locate the trap against the characters' Survival score (at roll 8) in order to determine if a party member notices the trap. If adventurers notice the trap before activating it, they may attempt to disarm it, either permanently or long enough for it to pass.

The Storyteller may request a Deactivation Check. If you have no burglary tools \index{burglary tools} or adequate, you do the check with a -1d6 penalty. \index{Deactivate devices without tools} The Survival ability can also be used albeit with a -2d6 to deactivate a trap, lock ..., in this case the duration of the operation is equal to 1 Action per DC of the trap.

If you want to temporarily disable \index{Trap temporarily disable} a trap add 6 to the difficulty. This will disable the trap for 2d4 minutes.

Any character can attempt an Intelligence check (with Arcana scoring at least 1) to disarm a magic trap, in addition to any other checks listed in the trap description. Additionally, the spell \hyperlink{dissolvimagie}{Dissolvi Magie} has a chance to disable most magical traps.\index{Disable magical traps}

A magic trap can be disabled with a Disable Gadgets check as long as the Arcana value is at least 1/6 (rounding down to a minimum of 1) of the trap's DC.

In most cases, the description of the trap is clear enough for the Storyteller to judge whether a character's actions locate or foil the trap.

Use common sense, drawing on the description of the trap to determine what happens. No trap design could ever be able to anticipate every possible action that characters may attempt to take.

The Storyteller should allow a character to discover a trap without having to make proficiency checks if his action or description of what he does would clearly reveal the trap's presence.

Throwing traps can be a little more complicated. Let's take the case of a chest defended by a trap. If the chest is opened without pulling on the two handles placed on the sides, a mechanism placed inside it shoots a barrage of poisoned needles at anyone who is in front of it.

After inspecting the chest and making some checks, the characters are still not sure if it is equipped with traps. Rather than opening the chest, they point a shield in front of it and open it remotely using an iron rod. In this case, the trap is activated, but the barrage of needles is fired at the shield without harming anyone.

Traps are often designed with mechanisms that allow them to be disarmed or bypassed.

\subsubsection{Traps Effects}
The effects of traps can range from simple inconvenience to lethal. The description of a trap specifies what happens when it is activated.
A trap's attack bonus, the saving throw DC to resist its effects, and the damage it deals can vary based on how dangerous the trap is.

Use the DC Traps Saving Throws and Attack Bonuses chart and the Damage Severity per Level chart as suggestions for the three levels of trap severity.

\medskip

\begin{center}
\includegraphics[width = 0.9 \linewidth]{immagini/medusa.png}
\end{center}


\textbf{Table: Saving Throw DCs and Trap Attack Bonuses} \index{Saving Throw DCs and Trap Attack Bonuses}

\medskip

\begin{tabularx}{0.45\textwidth}{XXX}
Trap Danger & DC Saving Throw & Attack Bonus \\
\toprule
Mishap & 13-14 & + 4 to +6 \\
Dangerous & 16-20 & + 8 to +10 \\
Mortal & 21-26 & + 12 to +15 \\
\end{tabularx}

\medskip

\textbf{Table: Damage Severity by Level} \index{Damage Severity by Level Table}

\medskip

\begin{tabularx}{0.45\textwidth}{XlXX}
Character & Setback & Dangerous & Deadly Level \\
\toprule
1st-4th & 1d10 & 2d10 & 4d10 \\
5th -10th & 2d10 & 4d10 & 10d10 \\
11th-16th & 4d10 & 10d10 & 18d10 \\
17 ° -20 ° & 10d10 & 18d10 & 24d10 \\
\end{tabularx}

\medskip

\subsubsection{Complex Traps}
Complex traps function like normal traps, except that once activated they perform a series of actions each round.

A complex trap transforms the process of tackling a trap into something more like a combat encounter. When a complex trap is activated, roll its initiative.

The trap description includes an initiative bonus. During its round, the trap activates again, often making an action, be it an attack, an effect that changes over time, a dynamic challenge. Otherwise, the complex trap can be detected and disabled in the usual ways.

\subsubsection{Example Traps}
\textbf{Poisoned Needle}

Mechanical trap

A poisoned needle is hidden inside the lock of a chest, or other object that can be opened. Opening the chest without the proper key would trigger the needle, which dispenses a dose of poison.

When the trap is activated, the needle extends 7.5 centimeters from the lock. A creature at range takes 1 damage and 11 (2d10) poison damage, and must succeed on a DC 20 Fortitude save or take -1d6 on attack rolls and -1d6 on proficiency checks for 1 hour.

The character who passes a Survival check with DC 22 can deduce the presence of the trap from the modifications made to the lock to accommodate the needle. A successful Disarm Device check using burglary tools disarms the trap by removing the needle from the lock. A failed attempt to pick the lock triggers the trap. Claiming to wedge a stick in the lock is just as effective in disabling the trap.

\medskip

\textbf{Poison Darts}

Mechanical trap

When a creature steps on a hidden pressure plate, poison darts are fired from a spring mechanism or from pressurized tubes cunningly hidden within the surrounding walls. An area might have multiple pressure plates, each linked to its own set of darts.

The tiny holes in the walls are hidden by dust and cobwebs, or cunningly hidden among the bas-reliefs, murals or frescoes that adorn the room. The DC of the check to notice them (Survival) is 18.

The character who passes a Survival check with DC 18 can deduce the presence of the hidden pressure plate from the differences in the flooring it is made of compared to the rest of the floor.

Wedging an iron tip or other object under the pressure plate prevents the trap from being activated. Filling the holes with fabric or wax prevents the darts contained inside from escaping.

The trap is activated when more than 10 kilos of weight are placed on the pressure plate, thus firing four darts. Each bolt makes a ranged attack with a +10 attack bonus against a random target within 10 feet of the pressure plate (line of sight has no impact on this attack roll).

If there are no targets in the area, the dart hits nothing. A hit target takes 2 (1d4) points of piercing damage and must make a Fortitude save with a DC 18 and take 11 (2d10) poison damage if it fails, or half that damage if it succeeds.


\medskip

\textbf{Fosse}

Mechanical trap

We present four basic types of pits below.

\medskip

\textit{Simple Pit}

The simple pit is a hole dug in the ground. The hole is covered with a thick fabric anchored to the edge of the pit and camouflaged with earth and debris.
The DC for scoring the pit is 12. Anyone who steps on the cloth falls into the hole and pulls behind the cloth, taking damage based on the depth of the pit (usually 10 feet, but some pits are deeper).

\medskip

\textit{Hidden Pit}

This pit has a cover made of identical material to that of the surrounding floor.
Passing an Awareness check with DC 18 shows the absence of traces in the section of the floor that forms the cover of the pit.

A DC 18 Survival check must be passed to confirm that section of the floor actually covers a pit.

When a creature steps onto the cover, it swings open like a trap door, plunging the intruder into the pit below. The pit is usually between 3 and 6 meters deep, but it can be even deeper.

Once the pit has been located, an iron spike or similar object can be driven between the pit cover and the surrounding ground to prevent the cover from opening, making passage safe. The cover can also be magically held closed with an arcane lock spell or similar spell.

\medskip
\textit{Snap Pit}

This pit is identical to the hidden pit trap, with one key exception: the trap door that covers the pit hides a spring mechanism. After a creature falls into the pit, the cover slams shut to trap the victim inside.

A DC 20 Strength check must be passed to force open the cover. The cover can also be destroyed. A character inside the pit can also attempt to disable the spring mechanism from within by passing a Disable Gadgets on DC 18 check and using burglary tools, as long as they can reach and see the mechanism in question. In some cases, another mechanism causes the pit to reopen.

\medskip

\textit{Pit with Spikes}

The pit is a simple, hidden or snap-on pit, on the bottom of which there are wooden spikes or iron spikes. A creature that falls into the pit takes 11 (2d10) piercing damage from the spikes, in addition to falling damage.

More cruel versions of this trap are equipped with poison sprinkled on the spikes located at the bottom of the pit. In that case, anyone taking piercing damage from the spikes must also make a DC 16 Fortitude save and take 22 (4d10) poison damage if they fail, or half that damage if they succeed.


\medskip

\textbf{Falling Network}

Mechanical trap

This trap uses a wire to release a net hanging from the ceiling.

The cable is placed 7 centimeters from the ground and extends between two columns or trees. The net is hidden by cobwebs or foliage. The DC (Survival) to note the wire and network is 15. A successful Deactivate Device with DC 20 using burglary tools disables the wire.

A character without the burglary tools can still attempt the -1d6 check using a sharp weapon or tool. If the check fails, the trap is activated.

When the trap is activated, the net is released covering a square area of 3 meters on each side. All creatures in the area are trapped by the net and are hindered, while those that fail a Fortitude save with a DC 13 Strength modifier also fall prone.

A creature can use 2 Actions to make a DC 13 Strength check, releasing itself or another creature in range if it succeeds.

The net has Defense 10 and 20 hit points. dealing 5 slashing damage to the net destroys a 5-foot square section of it, freeing any creatures trapped in that section.

\medskip

\textbf{Rolling Sphere}

Mechanical trap

When 10 or more pounds are placed on the trap's pressure plate, a hidden hatch in the ceiling opens, releasing a 10-foot-diameter sphere made entirely of stone.

Upon successful completion of a DC 20 Survival check, a character will notice the trap door and pressure plate. If a floor examination is accompanied by a passed DC 20 Survival check, it will reveal the presence of the pressure plate through the difference in structure of the flooring that accommodates it. The same check performed while checking the ceiling will reveal the presence of a trap door. Wedging an iron spike or other object under the pressure plate will prevent the trap from activating.

Activating the sphere causes all creatures present to roll initiative. The sphere rolls initiative with a +8 bonus.

During its round, the ball moves 60 feet in a straight line. The sphere can move through a creature's space, and creatures can move through the space it occupies, considering it difficult terrain.

Whenever the sphere enters a creature's space or a creature enters its space while the sphere is rolling, the creature must succeed on a DC 15 Reflex save or take 55 (10d10) hit damage and fall prone.

The sphere stops when it hits a wall or similar barrier. It can't turn corners, but skilled dungeon builders incorporate slight curves and curvilinear turns into neighboring passages that allow the sphere to keep moving.

With 2 Actions, a creature within 1 meter of the sphere can attempt to slow it by making a Strength check with DC 20. If the check is successful, the speed of the sphere is reduced by 4 meters. If the speed of the sphere drops to 0, it stops moving and is no longer a threat.

\medskip

\textbf{Collapsing ceiling}

Mechanical trap

This trap uses a wire to collapse the supports holding an unstable section of the ceiling.

The cable is placed 7 centimeters from the ground and extends between the two supports. The DC (Survival) to note the wire is 13. A successful Deactivate Gadget check with DC 20 using burglary tools disables the wire.

A character without the burglary tools can still attempt the -1d6 check using a sharp weapon or tool. If the check fails, the trap is activated.

Anyone who inspects the supports can easily deduce that they are only supported. With an action, the character can drop a support and activate the trap.

The ceiling above the cable is in poor condition, and anyone who can see it can understand that it is in danger of collapsing. When the trap is activated, the unstable ceiling collapses. All creatures in the area under the unstable section must make a DC 20 Reflex save, taking 22 (4d10) hit damage if they fail or half that damage if they succeed. Once the trap is activated, the floor of the area is full of rubble and becomes difficult terrain.


\medskip

\textbf{Blow Fire Statue}

Magic trap

This trap is activated when an intruder steps on a hidden pressure plate, releasing a burst of magical flame from a nearby statue.

The DC (Survival) to notice the pressure plate or burn marks on the floor and walls is 20. A spell or other effect that can sense the presence of magic, such as detection of magic, reveals a magical aura of invocation around the statue.

The trap is activated when more than 10 kilos of weight are placed on the pressure plate, causing a 9-meter cone of fire to spring from the statue. All creatures in the cone must make a DC 17 Reflex save, taking 22 (4d10) fire damage if they fail or half that damage if they succeed.

Sticking an iron spike or other object under the pressure plate prevents the trap from activating. A check to Deactivate Gadgets at DC 20 (and you must have 3 in Arcana) deactivates the trap. A dispel magic (DC 17) cast on the statue destroys the trap.


\medskip

\textbf{Spell Trap}

Magic trap

The above traps can be equipped with a spell that activates with the trap.
The saving throws to resist the spell are the same as those for the spell cast.

\bigskip

\subsubsection{Other examples of traps}

Further traps are presented here for your delight.


\medskip

\textbf{Small legend}:

Degree of Challenge: indicates what the degree of challenge of the trap is

Type: if the trap is mechanical or magical

DC Survival: What is the check and difficulty to reveal the trap

DC Disabling Gadgets: What is the check and difficulty to disable the trap. If the check succeeds by 5 or more, the character can decide whether to reactivate the trap once it has been avoided, otherwise it is defused.

Activator: if activated by contact or distance

Reset: Whether the trap can be reset once it is triggered

Effect: What is the effect of the trap


\medskip


\textbf{Poison Bolt}

Degree of Challenge: 1

Type: mechanical

DC Survival: 20

DC Deactivate Gadgets: 20

Activator: contact

Restore: none

Effect: 12m ranged attack +10 (1d3 damage plus Lucos' fermented slime)


\textbf{Arrow}

Degree of Challenge: 1

Type: mechanical

DC Survival: 20

DC Deactivate Gadgets: 20

Activator: contact

Restore: none

Effect: Ranged attack 12 yards +15 (1d8 + 1 / × 3)


\textbf{Fossa}

Degree of Challenge: 1

Type: mechanical

DC Survival: 20

DC Deactivate Gadgets: 20

Activator: position

Reset: manual

Effect: 10-foot deep pit (2d6 fall damage)

TS: Reflexes DC 20 avoids

Target: multiple targets (all targets within 10 feet)


\textbf{Mowing Blade}

Degree of Challenge: 1

Type: mechanical

DC Survival: 20

DC Deactivate Gadgets: 20

Activator: position

Reset: manual

Effect: Melee attack +10 (1d8 + 1 / × 3)

Target: multiple targets (all targets in a line within 3 meters)


\textbf{Pit with Spikes}

Challenge degree: 2

Type: mechanical

DC Survival: 20

DC Deactivate Gadgets: 20

Activator: position

Reset: manual

Effect 10 ft deep pit (1d6 fall damage) + spikes (Melee attack +10, 1d4 spikes per target for 1d4 + 2 damage each)

TS: Reflexes DC 20 avoids

Target: multiple targets (all targets in a 10-foot square)


\textbf{Burning Hands}

Challenge degree: 2

Type: magical

DC Survival: 26

DC Disable Gadgets / Arcana: 26/4

Activator: proximity (Alarm)

Restore: none

Effect: 2d4 fire damage

TS: Reflexes DC 11 halves

Target: multiple targets (all targets in a 6m long cone and 3m trailing)


\textbf{Javelin}

Challenge degree: 2

Type: mechanical

DC Survival: 20

DC Deactivate Gadgets: 20

Activator: position

Restore: none

Effect: Ranged attack 12 meters +15 (1d6 + 6), within 6 meters range


\textbf{Acid Arrow}

Challenge degree: 3

Type: magical

DC Survival: 27

DC Disable Gadgets / Arcana: 27/4

Activator: proximity (Alarm)

Restore: none

Effect: Attack 16 meters range (2d4 acid damage for 4 rounds)


\textbf{Hidden Pit}

Challenge degree: 3

Type: mechanical

DC Survival: 25

DC Deactivate Gadgets: 20
Activator: position

Reset: manual

Effect: medium deep pit (3d6 fall damage)

TS: Reflexes DC 20 avoids

Target: multiple targets (all targets in a 10-foot square)


\textbf{Electric Arc}

Challenge grade: 4

Type: magical

DC Survival: 25

DC Disable Gadgets / Arcana: 20/3

Activator: contact

Restore: none

Effect: Electric arc, 4d6 points of electricity damage

TS: Reflexes DC 20 halves

Target: multiple targets (all targets in a line at a distance of 6 meters)


\textbf{Wall Scythe}

Challenge grade: 4

Type: mechanical

DC Survival: 20

DC Deactivate Gadgets: 20

Activator: position

Reset: automatic

Effect: Melee attack +20 (2d4 + 6)


\textbf{Block in Fall}

Challenge degree: 5

Type: mechanical

DC Survival: 20

DC Deactivate Gadgets: 20

Activator: position

Reset: manual

Effect: Melee attack +15 (6d6)

Target: multiple targets (all targets in a 10-foot square)

\textbf{Flame Strike}

Challenge degree: 6

Type: magical

DC Survival: 30

DC Disable Gadgets / Arcana: 30/5

Activator: proximity (Alarm)

Restore: none

Effect: 8d6 points of fire damage, range 10 feet

TS: Reflexes DC 17 halves

Target: multiple targets (all targets in a 3 meter radius cylinder)

\textbf{Poisoned Arrow}

Challenge degree: 6

Type: mechanical

DC Survival: 20

DC Deactivate Gadgets: 20

Activator: position

Restore: none

Effect: Ranged attack 60 feet +15 (1d6 plus poison × 3)


\textbf{Cold Fangs}

Challenge degree: 7

Type: mechanical

DC Survival: 25

DC Deactivate Gadgets: 20

Activator: position

Duration: 3 rounds

Restore: none

Effect: distance 3 meters (splash of frozen water, 3d6 cold damage)

TS: Reflexes DC 20 halves

Target: multiple targets (all targets in a 3x3x3 meter room)


\textbf{Gas Trap}

Challenge grade: 8

Type: mechanical

DC Survival: 25

DC Deactivate Gadgets: 20

Activator: position

Recovery: repairable

Effect: Poisonous gas

Target: multiple targets (all targets located in a room 3x3x3 meters)


\textbf{Volley of Arrows}

Challenge degree: 9

Type: mechanical

DC Survival: 25

DC Deactivate Gadgets: 25

Activator: visual (Arcane Eye)

Recovery: repairable

Effect: Ranged attack +20 (6d6)

Target: multiple targets (all targets in a 6m line)


\textbf{Hidden Pit with Spikes}

Challenge grade: 8

Type: mechanical

DC Survival: 25

DC Deactivate Gadgets: 20

Activator: position

Reset: manual

Effect 15m deep pit (5d6 fall damage) + spikes (Melee attack +15, 1d4 spikes per target for 1d6 + 5 damage each)

TS: Reflexes DC 20 avoids

Target: multiple targets (all targets in a cube with side 3x3x3 meters)


\textbf{Shocking Floor}

Challenge degree: 9

Type: magical

DC Survival: 26

DC Disable Gadgets / Arcana: 26/4

Activator: proximity (Alarm)

Duration: 1d6 rounds

Restore: none

Effect: Melee touch attack +9, 4d6 Electricity damage

Target: multiple targets (all targets in a 6x6x3 meter room)

\textbf{Energy Drain}

Challenge grade: 10

Type: magical

DC Survival: 34

DC Disable Gadgets / Arcana: 34/5

Activator: visual (True seeing)

Restore: none

Effect: Touch attack at range 60 feet +10, max hit points drop by 10d4 + fatigued.

TS: Quench DC 23 negates after 24 hours


\textbf{Room of Lame}

Challenge grade: 10

Type: mechanical

DC Survival: 25

DC Deactivate Gadgets: 20

Activator: position

Duration: 1d4 rounds

Recovery: repairable

Effect: Melee Attack +20 (3d8 + 3)

Target: multiple targets (all targets located in a 3x3x3 meter room)


\textbf{Ice Splinter Cone}

Challenge degree: 11

Type: magical

DC Survival: 30

DC Disable Gadgets / Arcana: 30/5

Activator: proximity (Alarm)

Restore: none

Effect: cone of ice spears, 15d6 cold damage

TS: Reflexes DC 17 halves

Target: multiple targets (all targets in a cone of 18 meters long and 6 meters at the end)


\textbf{Mortal Spear}

Challenge degree: 18

Type: mechanical

DC Survival: 30

DC Disable Devices: 30

Activator: visual

Reset: manual

Effect: Ranged attack 36 meters +20 (1d8 + 6 plus poison)


\textbf{Hell of Fire}

Challenge grade: 13

Type: magical

DC Survival: 31

DC Disable Gadgets / Arcana: 31/5

Activator: proximity (Alarm)

Restore: none

Effect: 60 Fire damage

TS: Reflexes DC 14 halves

Target: Multiple targets (all targets in a 6m radius explosion)


\textbf{Crushing Boulder}

Challenge grade: 15

Type: mechanical

DC Survival: 30

DC Deactivate Gadgets: 20

Activator: position

Reset: manual

Effect: Melee attack +15 (16d6)

Target: multiple targets (all targets in a 10-foot square)


\textbf{Enhanced Attack}

Challenge grade: 16

Type: magical

DC Survival: 33

DC Disable Devices: 33

Activator: visual (True seeing)

Restore: none

Effect: +9 contact at 60 feet, 30d6 damage, save: DC 19's Fortitude reduces to 5d6 damage


\textbf{Lightning Gallery}

Challenge grade: 17

Type: magical

DC Survival: 29

DC Deactivate Gadgets: 29

Activator: proximity (Alarm)

Duration: 1d6 rounds

Restore: none

Effect: 8d6 electricity damage)

TS: Reflexes DC 16 halves

Target: all targets in a corridor of 12x3x3 meters


\textbf{Poison Pit}

Challenge grade: 12

Type: mechanical

DC Survival: 25

DC Deactivate Gadgets: 20

Activator: position

Reset: manual

Effect 15m deep pit (5d6 fall damage) + spikes (melee attack +15, 1d4 spikes per target for 1d6 + 5 damage each plus poison)

TS: Reflexes DC 25 avoids

Target: multiple targets (all targets in a 3x3 meter square)


\textbf{Meteor Swarm}

Challenge grade: 19

Type: magical

DC Survival: 34

DC Disable Gadgets: 34

Activator: visual

Restore: none

Effect: 4 separate-target meteors, +9 contact at 27m range, 2d6 impact plus 6d6 fire damage

TS: Reflex DC 23 halves Fire damage

Target: multiple targets (four targets, two of which cannot be more than 12m away from each other)


\textbf{Destruction}

Challenge degree: 20

Magic type

DC Survival: 34

DC Disable Gadgets: 34

Activator: proximity (Alarm)

Restore: none

Effect: Death saving throw

TS: Fortitude DC 23 reduces damage to 5d12 otherwise 10d12

\end{multicols}

\medskip

\begin{changemargin}{0.3cm}{0.3cm} \begin{tcolorbox}[title = Tups and the trap]{\small
In this example I bring you the old school approach when it was assumed that there were traps. Nothing prevents the Storyteller from allowing Survival Checks or Disabling Gadgets. I can only say that this approach is more engaging though.

\medskip

\textit{Storyteller}: A 3 meter wide corridor leads north into the darkness.

\textit{Tups}: We move forward by feeling the floor with our 10-foot pole.

\textit{Storyteller}: The pole was left stuck in the encounter with the stone idol.
[\textit{If he had used the pole the trap would have been easily discovered}.]
Do you continue down the corridor?

\textit{Tups}: No, I'm suspicious. Can I see any cracks in the floor, perhaps square in shape?

\textit{Storyteller}: No, there are millions of cracks, you can't see a pit that clearly[\textit{Storyteller estimates that the pit is well camouflaged and Tups has poor lighting to see well}]

\textit{Tups}: Okay, I'll take my water bottle out of my backpack. I'm going to pour some water on the floor. Does it seem to squeeze into the floor somewhere or reveal some form of texture?

\textit{Storyteller}: Yes, the water seems to convey around a square shape, slightly raised on the floor.

\textit{Tups}: Does it look like a covered pit?

\textit{Storyteller}: Could be

\textit{Tups}: Can I turn it off?

\textit{Storyteller}: how?[\textit{The Storyteller deliberately does not give a check, but involves the player}]

\textit{Tups}: I put the crowbar on it so that the mechanism does not open the hatch[\textit{Tups does not ask to roll a die to figure out how to disarm or disarm it directly, it explains to the Storyteller how it does it and that's it }]

\textit{Storyteller}: Cross the area now safely and see that it opens onto a small room with two reinforced wooden doors ...}

\medskip

Freely inspired by \href{https://friendorfoe.com/d/Old%20School%20Primer.pdf}{ \textbf{Quick Primer for Old School Gaming}}

\end{tcolorbox} \end{changemargin}

\begin{changemargin}{0.3cm}{0.3cm} \begin{Storytellere}
A "visible / obvious" trap forces players to interact with it, make an effort to understand how it works, and strive to avoid or disable it. Avoid when you can only die roll-based resolutions (Seeking Traps / Disabling Traps), rather reward the player's even simple but creative ingenuity to avoid danger.
\end{Storytellere} \end{changemargin}

\pagebreak

\subsection{Optional - Reputation and Fame} \index{Reputation} \index{Fame} \index{Optional - Reputation and Fame}


\begin{changemargin}{0.3cm}{0.3cm} \begin{enfasi}{
Fame and honor sometimes go more easily to those who do not seek them. (Titus Livio)
} \end{enfasi} \end{changemargin} \medskip

\begin{multicols}{2}

While some heroes are content with the rewards of their exploits or hide behind a husk of humility, others try to live forever in the sagas and songs of their epic exploits. The story measures a hero's success with tales of triumph and daring, repeated for generations.

A hero who cannot recall his story to anyone soon falls into oblivion, along with his untold efforts. The story of the brave deeds becomes the yardstick by which a hero is measured, and sculpts both his identity and his reputation.

Reputation represents how the general public perceives the character positively or negatively. This perception precedes him, speaks for him in his absence and determines how he will be treated by those who have heard of him. Reputation implies different things for different types of characters, based on the social and cultural values of different regions. A character who embodies the qualities of a hero in one region might be considered a depraved or a dishonest in another. An icon widely revered and respected in the motherland could slip from fame to oblivion if it travels to a neighboring kingdom.

When using these reputation rules, the Storyteller must determine what reputation means to the campaign's players and NPCs. For example, a Viking-themed campaign might base reputation on looting.

If you manage to acquire a strong or notable reputation, you could be praised for your deeds and rewarded with resources beyond those obtainable by lesser-known individuals. Similarly, reputation can be used to influence people socially, politically or economically.

Fame rises and falls based on one's actions. Current Fame determines overall reputation, Sphere of Notoriety defines where reputation benefits can be applied.

\end{multicols}

\textbf{Table: how to acquire Fame points} \index{Table: how to acquire Fame points}

\medskip

\begin{tabularx}{0.95\textwidth}{lX}
\textbf{Events} & \textbf{Form Fame} \\
\toprule
\textbf{Positive Events} & \\
Acquire a notable treasure from a worthy opponent & +1 \\
Consecrate a temple to your Patron & + 1 \\
Craft a powerful magic item & + 12 \\
Increase by one Level & + 1 \\
Detect and disarm three or more appropriate CR traps in a row & + 1 \\
Make a noteworthy historical, scientific or magical discovery & + 1 \\
Owning a legendary item or artifact & + 14 \\
Receiving a medal or similar honor from a public figure & + 1 \\
Return a significant magic item or relic to its owner & + 1 \\
Plunder the stronghold of a powerful noble (enemy) & +1 \\
Single combat defeat an enemy with a CR higher than your level & + 15 \\
As a group, win a combat match with an APL +3 plus & + 1 \\
Defeat a public slanderer in combat & + 2 \\
Pass a Profession check with DC 30 or more to create a work or object & + 2 \\
Pass a public Intimidate check with DC 30 or more (must be witnesses) & + 2 \\
Pass a public Entertain check with DC 30 or more (must be witnesses) & + 2 \\
Complete an adventure with a difficulty appropriate to your level & + 3 \\
Obtain a formal title (lady, lord, knight, etc.) & + 3 \\
Defeat a key (campaign) rival in combat & +5 \\
\textbf{Negative Events} & \\
Being convicted of a petty crime & -1 \\
Accompanying an unbecoming person & -18 \\
Being convicted of a serious non-violent crime & -2 \\
Publicly escaping a match with a weaker opponent & -3 \\
Attacking innocent people & -5 \\
Being convicted of a violent serious crime & -5 \\
Publicly lose a match with a weaker opponent & -5 \\
Being convicted of murder & -8 \\
Being convicted of treason & -10 \\
\end{tabularx}

\bigskip

\begin{multicols}{2}

\subsubsection{Fame}

You start the game with a Renown equal to your character level + your Charisma modifier. Fame ranges from -100 to 100, with 0 representing a lack of notoriety.

Over the course of the campaign, words and deeds help build a reputation. While an adventurer performs many feats, not all of them are significant enough to warrant a change in Renown. If possible, the Storyteller should stick to those feats that directly affect the story or campaign, and not award points for secondary wins.

The significance of a specific feat should be at the discretion of the Storyteller, but Table: Events of Renown provides some examples. If Renown falls below 0, see Discredit and Infamy below.

\subsubsection{Sphere of Notoriety}

A character's reputation goes hand in hand with the story of his exploits. Although he is a great hero in his homeland, when he travels elsewhere he will soon find that his reputation dwindles and that sooner or later he will reach regions where he is completely unknown. The higher the reputation, the more the Affected Area extends.

Fame determines the maximum radius of the Sphere of Notoriety. The Sphere of Notoriety has a radius of 150 kilometers, and typically increases by another 150 kilometers when the Fame reaches 10, 20, 30, 40 and 55.

Increasing the Sphere of Notoriety is not always automatic, you can express an opinion on where your reputation is concentrated. For example, you might require your sphere to extend further south to a large city and ignore barbarian tribes to the east, or to extend inland to another country rather than out to the ocean.

While reputation may spread by accident, it generally does so deliberately, because wandering storytellers embellish a character's deeds stories to make them more entertaining, his allies amplify the most common feats, his enemies repeat gossip about him to hire others and fight it, or the character himself tells his story to smug listeners.

Where these stories are told determines where they will meet and creates a Sphere of Notoriety: a heroic sorceress could hire bards to brag about her magic in a nearby kingdom she plans to visit, while an antagonistic Barbarian might push south. the wounded survivors of his raids, to spread fear among his next victims.

The following actions and conditions affect the modifier on Charisma, Diplomacy, or Intimidate checks for the purpose of expanding the Sphere of Notoriety.

\medskip

\textbf{Table: Sphere of Notoriety Modifiers} \index{Table of Sphere of Notoriety Modifiers}

\end{multicols}

\medskip

\begin{tabularx}{0.95\textwidth}{Xl}
\textbf{Action} & \textbf{Modifier to Proof} \\
\toprule
Allies or minions spread stories of the PC's exploits before his arrival & + 5 \\
A bard spreads stories or chants of the PC's exploits before he arrives & + 1/2 Bard's Entertainment Score \\
You have contact with the NPCs in the settlement & +1 \\
You have enemies in the settlement & + 1 \\
Distance from own Sphere of Notoriety & -1 for 15 kilometers \\
The main language of the settlement is different from your own & -5 \\
\end{tabularx}

\begin{multicols}{2}

\subsubsection{The Level of Renown}

\begin{itemize}


\item The Fame score is what makes the character popular.

\item A fame score within 10 points will make him a local, small town hero.

\item A fame score between 10 and 20 points will make him a public figure, known to everyone in a small town or neighborhood hero in a big city.

\item A score between 20 and 30 points makes the character known to everyone even in a big city, his deeds are also known in the region, perhaps not in all the details.

\item A score between 30 and 40 is a real celebrity in his hometown, known by name even in neighboring towns and respected throughout the region.

\item Fame between 40 and 50 points makes the character a true respected eminence in the state.

\item A score over 55 points makes the character a legend whose deeds are handed down and magnified over the centuries to come.

\end{itemize}

\subsubsection{Discredit and Infamy}


If your Fame falls below 0, your reputation is based on infamy rather than Fame. Consider Fame as a positive number rather than a negative number for all rules related to Renown, Sphere of Notoriety, and Prestige Points (for example, an antagonist Fame of -20 equals a Hero Fame of 20 for your admirers.


In the event that an event could increase the Fame, you can choose to increase the Fame (bringing it closer to 0) or to decrease it (making it a greater negative number). For example, if a character's Renown is 20 and you publicly roll a 30 on a Profession check to create a sword (which is typically +2), you can increase your Renown to 18 or decrease it to 22.

Negative events that decrease Fame always count as negative (an antagonist attacking innocent people does not inspire sympathy in the public).

If you have negative Renown, non-evil NPCs will often have bad or hostile reactions (see Table: Negative Renown Reactions). Note that if you have a reputation as a powerful and dangerous person, NPCs may avoid the PC rather than confront him.


\end{multicols}

\textbf{Table: Reactions to Negative Fame} \index{Reactions to Negative Fame Table}

\medskip

\begin{tabularx}{0.95\textwidth}{lX}
\textbf{Fame} & \textbf{Reaction} \\
\toprule
-5 & Merchants, mercenaries, and innkeepers charge PG 10 \% additional to discourage them from doing business in their community. \\
-8 & Merchants, mercenaries, and innkeepers refuse to do business. The PC who enters a shop is immediately asked to leave. If he refuses, the owner calls the authorities or fellow citizens to throw him out. \\
-10 & As the PC approaches, the shops close their windows and bar their doors. Most citizens refuse to talk to him. Others urge him to leave immediately. If he stays longer than 24 hours or acts blatantly against the citizens, his Renown decreases by 5 and the citizens rally to chase the PC. \\
-15 & Burned by the PC's shameless audacity to show up in the community, an angry mob gathers. If the PC doesn't leave within a few minutes, the mob begins bombarding him with rotten fruit, branches, and stones. \\
-20 & An angry mob forms immediately after the PC enters the city. Unwilling to await a potentially corrupt trial, they try to capture and execute him for his crimes. \\
-25 & An authority figure has issued an arrest edict against the PC, including a reward for anyone who catches him. This is well known and many want to collect it. \\
-30 & An authority figure placed a price on the PC's head. This is well known and many want to collect it. \\

\end{tabularx}

\vfill

\begin{center}
\includegraphics[keepaspectratio, width = 0.55 \textwidth]{immagini/Eastern_Story_Teller_1878.png}

\textit{Legends are told. Travelers in the Middle East Archive, Wilhelm Gentz}
\end{center}

\pagebreak

\section{Poisons and Potions} \index{Poisons} \index{Potions}

\label{veleni-e-pozioni}


\begin{changemargin}{0.3cm}{0.3cm} \begin{enfasi}{
One day, a man was hit by a poisoned arrow. Friends and relatives, anxious, called a doctor. When they approached him for take the arrow, the man said to them, "Before I do that, I would like to know who pierced me with this arrow ... Was a slave, a king, or a Brahmin? Was it big? Small? What color was her skin? Where is it did he live? And how was the arrow built? What poison it was office worker? ...  While he was asking himself all these questions ... the poison made his own effect and the wounded man ended up dying. (Buddha)
} \end{enfasi} \end{changemargin} \medskip

\begin{multicols}{2}

\subsection{Poison Type and Potion}

\lettrine[lines = 2, lhang = 0.33, loversize = 0.25, findent = 1.5em]{P}{oisons} and potions can be distinguished based on how you come in contact with them.
Not all poisons are toxic when ingested or inhaled.

To identify a natural potion you need a Herbalism check at DC 12 + the rarity of the plant or in the case of Poisons the difficulty is equal to the saving throw of the same. It costs 1 Action every 10 DC or with Herbalism 6 or more it costs 1 Action every 15 DC and with 12 points it costs 1 Action every 20 DC. Potions unless otherwise described must be drunk (ingested).

\textbf{Contact}: are contracted when someone touches the poison with bare skin. Contact poisons usually have a 1 round onset time. A contact poison can be an ointment, balm, liquid of any density or even a powder if specific for contact and not inhalation.

\textbf{Ingestion}: Activate when a creature eats or drinks them. Ingested poisons usually have an onset time of 10 minutes.

\textbf{Wounding}: Mostly transferred by attacks from certain creatures and by weapons sprinkled with poison. Injuring poisons usually have an instant onset time.

\textbf{Inhalation}: Activate the moment a creature enters an area that contains such poisons. Many inhaled poisons fill a volume equal to a cube with a corner of 3x3x3 meters per dose. Creatures can attempt to hold their breath while inside the area to avoid inhaling the toxin.
A creature can hold its breath for 6 rounds for its Constitution score, with a minimum of 3 rounds, and each Action decreases the remaining time by 1 round.
After the time has elapsed, they must make a Fortitude save at difficulty 12 each round to avoid inhaling the gas. Each round you hold your breath increases the difficulty check by 1.
See also the rules for holding your breath and choking in \hyperlink{trattenereilfiato}{Ambiente}

\subsection{Onset and Effect} \index{Poison Onset} \index{Poison Activation Time}

Onset is how long it takes the poison or potion to take effect. If the onset time is 1 turn it means that due to the effects of the poison / potion and the saving throw is made after 10 minutes. If onset is not specified in the poison / potion table, it means that the effect is immediate after coming into contact with the poison.

The effect of a poison / potion is immediate after onset. Check the description of the poison to understand its effect. If the Fortitude save is successful, the poison has no effect and can be considered neutralized.

There are some cases in which the item Frequency is present, on these rare occasions the saving throw must be repeated every time the indicated Frequency passes, in case of failure of the saving throw the indicated damage is reapplied.

\begin{center}
\includegraphics[height = 0.7 \linewidth]{immagini/potion.png}
\end{center}

\begin{changemargin}{0.3cm}{0.3cm} \begin{Storytellere}
The poisons proposed here are some of the many present and possible. Use them as guidelines. If for your ethics and style you do not like poisons, especially the bad ones, I suggest you use the Generic Potions that you find at the end of the chapter. They are milder and less personal poisons, probably more easily usable by players as well.
\end{Storytellere} \end{changemargin}

\subsubsection{Poisoned} \index{Poisoned}

\textbf{First dose}: When exposed to a poison for the first time (during your own action or someone else's), you must make a saving throw within the onset to avoid being poisoned.

\textbf{Success}: Resists poison. There are no negative effects and no further saving throws are required.

\textbf{Failure}: You have been poisoned and immediately suffer the listed effect.

\textbf{Multiple doses}: If you are exposed to multiple doses of the same poison in the same round the saving throw difficulty increases by 1 per additional dose. \index{Velono more doses}

\textbf{At different times}: If you are exposed to the poison at different times, there will be a new saving throw each time and you will suffer any effects on schedule.

If you are exposed to different poisons, you must make a saving throw for each type of poison taken.

\medskip

\begin{changemargin}{0.3cm}{0.3cm} \begin{tcolorbox}[title = Poison?]
{Poison is a double-edged sword. As long as you use it it's fine but if they use it against you, maybe the same, it becomes a problem. There are also ethical aspects to using poisons, consider if your Traits allow you to use poisons and what types.
} \end{tcolorbox} \end{changemargin}

\subsection{Apply Poison} \index{Apply Poison}

Applying poison to a weapon or ammunition requires 3 Actions.

Whenever a character applies or prepares a poison for use, he must roll 3d6 + Intelligence and if he rolls a 4 or less he has made contact with the poison and must make a saving throw against the poison as normal. This does not consume the dose of poison.

Whenever a character attacks with a poisoned weapon, if he rolls a 4 or less on the attack roll, he exposes himself to the effects of the poison. This consumes the poison on the weapon.
One poison potion is enough to cover a medium weapon or 3 arrows in poison. The poison is consumed and remains active on the weapon until it hits.

A creature under the effects of a poison, whether already 'unleashed or not, has the poisoned condition.

\subsection{Create Natural Poisons} \index{Create Natural Poisons}

Natural poisons can be made using Herbalism. The DC to prepare a poison is equal to the DC of the Fortitude save which requires -5. If you buy the ingredients, the cost to prepare the potion is half of the indicated selling cost, if you search in kind the cost per production drops to a quarter. The time to prepare these potions / drugs is equal to DC / 2 in hours.

Rolling a 3 or 4 on the dice with the Herbalism check exposes you to poison during its preparation. If the DC Herbalism check is successful, 1d2 + 1 doses are prepared.

The following examples represent just some of the possible poisons. All costs are expressed in Gold Coins.

\begin{center}
\includegraphics[height = 0.6 \linewidth]{immagini/poison.png}
\end{center}

\begin{changemargin}{0.3cm}{0.3cm} \begin{Storytellere}
Poisons are part of the long tradition of problems and adversity in role-playing games. When you want to use a poison, think first of all why it is there, for whom it was to be used, for what purpose. Not all poisons have to kill, a skilled thief could also use poisons that stun or weaken the will of his right target just enough to get the safe opened.
\end{Storytellere} \end{changemargin}

\end{multicols}

\bigskip

\begin{center}
\includegraphics[width = 0.5 \linewidth]{immagini/funeralebarca.png}
\end{center}

\vspace{3cm}

\textbf{Table: Poisons} \index{Table Poisons}

\medskip

\begin{tabularx}{1\textwidth}{m{4.5cm} lllm{6.5cm} l} %{XlllXl}
\toprule
\textbf{Poison Name} & \textbf{Usage} & \textbf{TS} & \textbf{Ins.} & \textbf{Effect (damage)} & \textbf{MO} \\
\toprule
Barsar Purple Berry \index{Barsar Purple Berry} & I & 18 & 1 shift & Unable to rape for 3d8 hours & 40 \\
\toprule
Ditch Blue Berries \index{Ditch Blue Berries} & I & 21 & 1 turn & -1d3 Intelligence and Wisdom for 6 hours & 55 \\
\toprule
Lucos's fermented slime \index{Lucos's fermented slime} & F & 15 & - & 1d8 Hit Points & 25 \\
\toprule
Yellow Bark Ash \index{Yellow Bark Ash} & F & 15 & 6 rounds & Unconscious for 1d3 hours & 25 \\
\toprule
Purple Concentrate \index{Purple Concentrate} & F & 15 & & 2d6 Hit Points & 15 \\
\toprule
Daraka's Fingers \index{Daraka's Fingers} & F & 17 & - & -1d6 Strength, for 1 hour & 35 \\
\toprule
Pink pointed grass \index{Pink pointed grass} & I & 22 & 1 turn & -1d6 Dexterity, for 1 hour & 60 \\
\toprule
Purple Shrew Liver \index{Purple Shrew Liver} & I & 25 & 1 hour & 2d6 damage to Wis. and Int. Permanent & 75 \\
\toprule
Mucot's white bow \index{Mucot's white bow} & C & 20 & - & Sleeps for 2d12 hours & 20 \\
\toprule
Fumi di Curna \index{Fumi di Curna} & R & 18 & - & -1d3 Wisdom & 40 \\
\toprule
Blue Frost \index{Blue Frost} & F & 18 & & 3d6 Cold Hit Points & 25 \\
\toprule
Purple Shrew Fat \index{Purple Shrew Fat} & C & 13 & 1 round & 2d12 Hit Points & 15 \\
\toprule
Kreex Tongue \index{Kreex Tongue} & F & 20 & - & The wound is bleeding. +1 bleeding damage. 1 use in 24 hours. & 50 \\
\toprule
Red Mix \index{Red Mix} & F & 13 & - & -1d6 TC / TS for 10 minutes & 10 \\
\toprule
Yellow Moss \index{Yellow Moss} & I & 20 & 1 round & creature gains a size. -2 Int and Sag. Duration 10 minutes& 50 \\
\toprule
Dennar's Core \index{Dennar's Core} & I & 13 & 1 turn & -1d2 Strength, for 3 days & 15 \\
\toprule
Nabar oil \index{Nabar oil} & RF & 20 & - & Confused for 2d6 rounds & 50 \\
\toprule
Blue Toad Hide \index{Blue Toad Hide} & C & 22 & 1 minute & Paralyzed for 1d6 turns & 60 \\
\toprule
Omro's Rose Pollen \index{Omro's Rose Pollen} & I & 15 & - & -1d3 Constitution and Dexterity, for 1 hour & 25 \\
\toprule
Ragmor's Scent \index{Ragmor's Scent} & R & 16 & - & -1d3 Charisma, for 1 day & 30 \\
\toprule
Blood of Thrun \index{Blood of Thrun} & C & 26 & - & -1d3 Constitution & 80 \\
\toprule
Juice of Ythis \index{Juice of Ythis} & I & 14 & 1 turn & -1d2 Intelligence, for 1d & 20 \\
\toprule
Venom of Ottalm \index{Venom of Ottalm} & F & 20 & - & Death or -1d2 Permanent Constitution & 50 \\
\toprule
Blood Serpent Venom \index{Blood Serpent Venom} & F & 25 & - & Paralysis for 1d6 hours -1d4 Strength points for 7 days & 75 \\


\end{tabularx}

\medskip

\textbf{Application}: \textbf{I} (nmanagement), \textbf{F} (eriment), \textbf{C} (ontact), \textbf{R} (exhalation).

The saving throw is always Fortitude unless otherwise noted

The lost characteristic points are recovered at the rate of 1 per day if not permanent or otherwise indicated.

\subsubsection{Notes on poisons}

\textbf{Purple Shrew Liver}: poisoning recognizable by the typical bloodshot eyes

\begin{center}
\includegraphics[width = 0.20 \linewidth]{immagini/mandragola2.png}

\textit{Mandrake plant}
\end{center}

\subsection{Natural Potions} \index{Potions}

\begin{changemargin}{0.3cm}{0.3cm} \begin{enfasi}{
I believe that a leaf of grass is no less than a day's work accomplished by the stars. (Walt Whitman)
} \end{enfasi} \end{changemargin}

\begin{multicols}{2}

The time to prepare these potions / drugs is equal to DC / 2 in hours, while the difficulty of the Herbalism check is equal to DC -5. If you buy the ingredients, the cost to prepare the potion is half of the indicated selling cost, if you search in kind the cost per production drops to a quarter.

If the DC Herbalism check is successful, 1d2 + 1 potions (1 dose) are prepared.

You cannot benefit from more than one dose of natural potions (of each type) per day, unlike magical ones.

\end{multicols}

\medskip
{\small
\begin{xltabular}{0.95\textwidth}{llllXlc}
\textbf{Name} & \textbf{Usage} & \textbf{Ins.} & \textbf{DC} & \textbf{Effect} & \textbf{Loc.} & \textbf{Cost} \\
\toprule
Arduuar \index{Arduuar} & I & 1 round & 25 & Removes Poisons & SZ7 & 75 \\
\toprule
Arkasun \index{Arkasun} & C & 1 turn & 25 & Heals 1d6 hit points per turn for 3 turns & TM7 & 75 \\
\toprule
Arlan \index{Arlan} & C & 5 rounds & 15 & Heals 1d6 + 3 Hit Points & TT5 & 50 \\
\toprule
Arlandas \index{Arlandas} & R & 1 hour & 24 & Rinsalda fractures & CF5 & 200 \\
\toprule
Attarna \index{Attarna} & I & 1 turn & 20 & Grants a new disease saving throw with a + 1d6 & TF7 & 50 \\
\toprule
Berries of Ljust \index{Berries of Ljust} & I & 1 round & 16 & Taken in the evening you recover double the minimum hit points 4) & AZ6 & 10 \\
\toprule
Ljust's Kiss \index{Ljust's Kiss} & C & 1 round & 35 & Heals 100 Hit Points & HO8 & 500 \\
\toprule
Barannie \index{Barannie} & I & 1 minute & 15 & Removes nausea & MD6 & 3 \\
\toprule
Burthelas \index{Burthelas} & I & 1 turn & 32 & Regenerates the hands & HD7 & 410 \\
\toprule
Dagmathir Bark \index{Dagmathir Bark Powder} & R & 1 round & 25 & Removes a level of Fatigue & SS5 & 15 \\
\toprule
Aklent's Bark \index{Aklent's Bark} & I & 1 turn & 10 & The bark chewed for at least 10 rounds grants +1 saving throw vs poison for the next 24 hours for the next 24 hours & TM6 & 1 \\
\toprule
Culcoa \index{Culcoa} & C & 1 round & 16 & 2d6 recoveries from fire damage & TS7 & 15 \\
\toprule
Darsirion \index{Darsirion} & C & 1 round & 25 & Heals 1d4 Hit Points & CM4 & 5 \\
\toprule
Delrean Plus \index{Delrean Plus} & I & 1 round & 18 & Repel bugs for 3 days & CC6 & 5 \\
\toprule
Delrean \index{Delrean} & C & 1 round & 15 & Bug repellent for 1 day & CC6 & 2 \\
\toprule
Draaf \index{Draaf} & C & 1 round & 20 & Heals 1d8 Hit Points & SO6 & 50 \\
\toprule
Eldrin'tail \index{Eldrin'tail} & I & 1 round & 15 & Grants a new save on poisons & FH7 & 18 \\
\toprule
Illa Berry Extract \index{Burnt Illa Berry Extract} & I & 1 round & 15 & +2 Initiative, +2 Dexterity, -1d6 Will save, for 10 minutes & MS6 & 5 \\
\toprule
Gisenosa root extract \index{Gisenosa root extract} & I & 3 rounds & 15 & Cough and cold cure & MT6 & 3 \\
\toprule
Febfendi \index{Febfendi} & C & 1 turn & 25 & Regenerate ears & CF7 & 75 \\
\toprule
Garioe \index{Garioe} & I & 1 round & 25 & Heals 2d6 Hit Points & AZ7 & 95 \\
\toprule
Geffnull \index{Geffnull} & I & 5 rounds & 28 & Heals 3d8 + 3 Hit Points & EV8 & 150 \\
\toprule
Gusterbloon \index{Gusterbloon} & C & 1 round & 20 & The skin gets darker by granting + 1d6 to Hide checks & CM5 & 8 \\
\toprule
Gylvert \index{Gylvert} & I & 1 minute & 25 & Allows you to breathe underwater for 4 hours & MO7 & 3 \\
\toprule
Harfy \index{Harfy} & C & - I & 12 & -1 to bleeding & SS6 & 3 \\
\toprule
Harfindar \index{Harfindar} & I & 1 turn & 15 & Abort & SS7 & 3 \\
\toprule
Jojopo \index{Jojopo} & C & 1 round & 15 & 2d6 recoveries from cold damage & FM6 & 18 \\
\toprule
Kelventare \index{Kelventare} & I & 1d4 rounds & 28 & Recover 2d6 Hit Points & TT7 & 100 \\
\toprule
Klagul \index{Klagul} & C & 1 turn & 20 & Cleans teeth & SS4 & 2 \\
\toprule
Klandor \index{Klandor} & I & I & 15 & Removes paralysis. Increase Fatigue Level by 1 & HB6 & 18 \\
\toprule
Klynkyx \index{Klynkyx} & C & 6 turns & 15 & Drops all hair for 1d6 + 4 days & MO6 & 4 \\
\toprule
White Musk Yeast \index{White Musk Yeast} & I & 1 minute & 12 & Baked goods using this yeast cause uncontrollable and incredibly smelly bloating for 12 hours & CA3 & 1 \\
\toprule
Xabax's Red Tongue \index{Xabax's Red Tongue} & C & 1 turn & 20 & Heals 2d6 hit points but if there is disease or poison it removes it for 2d6 HP of damage & TA7 & 13 \\
\toprule
Melandrir \index{Melandrir} & I & 1 round & 15 & Grants a new disease saving throw with +5 & CF7 & 100 \\
\toprule
Mirenna \index{Mirenna} & I & 1 round & 20 & Heals 5 Hit Points & CM6 & 30 \\
\toprule
Blend 31 \index{Blend 31} & I & 1 turn & 20 & The mount is extremely tough. +6 gallop hours per day & SM6 & 15 \\
\toprule
Silver Moss \index{Silver Moss} & I & I & 25 & Removes magical diseases & MU8 & 250 \\
\toprule
Musekiss \index{Musekiss} & C & 1 hour & 30 & Regenerates lower limbs & TH9 & 550 \\
\toprule
Nazamuse \index{Nazamuse} & I & I & 30 & Removes poisons and natural diseases & EW9 & 175 \\
\toprule
Nelthalion \index{Nelthalion} & I & I & 15 & Makes you vomit & SR3 & 1 \\
\toprule
Petals of Lisbeth \index{Petals of Lisbeth} & I & 1 turn & 15 & + 2 Intelligence, -2 Dexterity for 10 minutes & MC6 & 20 \\
\toprule
Green Rose Pollen \index{Green Rose Pollen} & R & 3 turns & 25 & Recover 2d4 damage Intelligence and Wisdom & FA8 & 35 \\
\toprule
Kathaus dry root \index{Kathaus dry root} & R & 1 round & 20 & +2 Strength and Dexterity for 1 hour & FW6 & 50 \\
\toprule
Rewky \index{Rewky} 		& I & 1 turn & 25 & Heals 2d8 Hit Points & TD6 & 20 \\
\toprule
Siranmuse \index{Siranmuse} & I & 1 day & 30 & Regenerates internal organs & SS8 & 850 \\
\toprule
Ucsaboo \index{Ucsaboo} & C & 1 turn & 30 & Regenerate eyes & MO8 & 400 \\
\toprule
Urk's Egg \index{Urk's Egg} & I & 1 turn & 12 & 1 day of food & FH7 & 1 \\
\toprule
Outbox \index{Outbox} & R & 1 turn & 25 & Removes blindness & MO7 & 125 \\
\toprule
Wickalim \index{Wickalim} & I & 1 hour & 15 & Heals 2 Hit Points & TD3 & 5 \\
\toprule
Yaveth \index{Yaveth} & I & 1 turn & 20 & Heals 2d8 Hit Points & MO5 & 100 \\
\end{xltabular}}


\subsubsection{Notes on seedlings ...}

\textbf{Aklent's bark}: also called \textit{Skunk bush} for its pungent and characteristic odor.

\textbf{Gisenosa root extract}: thistle-like plant, extremely thorny. It tends to grow surrounded by \textit{Tribulus terrestris} or "foot kisser".

\textbf{Silver Musk}: very similar, for a non-expert, to White Musk. The berries are harvested.

\subsection{Where to find the plants}

Ex: Gusterbloon FT5. The first letter indicates the CLIMATE, the second indicates the ENVIRONMENT, the third indicates the RARITY. The rarity indicates the possibility, on a d10, to find the wanted herb / plant. Roll 1d10 and do more than the indicated number, clearly if there is a match between climate and environment.

\textbf{Table: Potions - Places correspondence} \index{Potions - Places correspondence table}

\medskip

\begin{tabular}{ll|ll|ll}
	\textbf{1 word} & \textbf{Climate} & \textbf{2' lett.} & \textbf{Environment} & \textbf{2 word} & \textbf{Environment} \\
	\toprule
	A & Arid & A & Alpine & B & Gorges \\
	C & Cold & C & Coniferous Forest & D & Deciduous Forest \\
	E & Perennial ice & F & River and stream embankments & G & Frozen fields \\
	F & Severe Cold & H & Dry Fields & J & Jungle, Rainy Forests \\
	H & Moist and warm & M & Mountain & N & Ocean, salty expanses \\
	M & Temperate & S & Short Grass & T & Tall Grass \\
	S & Semi Arid & U & Caves and Underground & V & Volcanic \\
	Cool T & Temperate & W & Landfills or Trash & Z & Desert \\
	X & Unknown & X & Unknown && \\
\end{tabular}

\subsection{Generic Potions} \index{Generic Potions} \index{Potions}

The Storyteller is free to use all of the potions and poisons listed above or use ready-to-use generic potions, which can be bought in almost any herbalist's or potion store.

The table shows the costs and effects of these potions. The onset is always immediate, the duration for the cures is immediate, for the others it is 1 hour (so the Remove Poison potion "immunizes" you for 1 hour against a poison). For potions that cause damage, the saving throw is to negate the effects.

\textbf{Table: Generic Potions} \index{Table of Generic Potions}

\medskip

\begin{tabularx}{0.95\textwidth}{lXcc}
	\textbf{Potion Name} & \textbf{Effect} & \textbf{Cost (gp)} & \textbf{Application} \\
	\toprule
	Heal & recover 1d8 + 1 hit points & 50 & ingestion \\
	Enhanced heal & recoveries 3d8 + 3 hit points & 125 & ingestion \\
	Weakened & -1d6 TC. TS DC 15 Quench & 34 & Ingestion \\
	Empowered weakener & -1d6 TC. TS DC 18 Fortitude & 50 & Injury \\
	Poison & take 2d4 + 2 damage. TS DC 15 Quench & 30 & Ingestion \\
	Empowered poison & take 2d4 + 2 damage. TS DC 18 Fortitude & 25 & Wounding \\
	Remove Poison & negate the onset of a poison if taken within activation, or grant a new save with + 1d6 & 75 & Ingestion \\
\end{tabularx}

These generic potions like natural potions only take effect the first time they are taken within 24 hours.


\subsection{Optional - Drugs} \index{Drugs} \index{Optional - Drugs} \hypertarget{droghe}{} \label{droghe}

\textbf{Table: Drug List} \index{Drug List Table}

\medskip

\begin{tabularx}{0.99\textwidth}{llllXrr}
\textbf{Name} & \textbf{Usage} & \textbf{Ins.} & \textbf{DC} & \textbf{Effect} & \textbf{Loc.} & \textbf{Cost} \\
\toprule
Fermented Luside Leaves \index{Fermented Luside Leaves} & I & 1 turn & 17 & Sensory hallucinations for 2d4 hours. +2 Charisma and Intelligence & SF7 & 5 \\
\toprule
Ferpillon \index{Ferpillon} & I & 1 round & 20 & Makes you sleep for 24 hours & SC5 & 50 \\
\toprule
Greasy Gray \index{Greasy Gray} & I & 1 round & 24 & Removes mental conditioning caused by spells of level 5 or less & AH9 & 80 \\
\toprule
Arpasur's Ash \index{Arpasur's Ash} & R & 1 round & 20 & Removes 2 fatigue levels & FT6 & 10 \\
\toprule
Purple Spider Dried Meat \index{Purple Spider Dried Meat} & I & 1 round & 24 & +4 Strength -4 Intelligence (minimum -3) for 1 turn & SH7 & 30 \\
\toprule
Alcoholic Melzaa Extract \index{Alcoholic Melzaa Extract} & I & 1 round & 20 & + 1d4 Strength, + 1d4 Dexterity. -1d6 Will saving throw. For 3 hours & AF6 & 25 \\
\toprule
Scented essence of Inut \index{Scented essence of Inut} & R & I & 15 & +2 Intelligence, for 1d8 hours & HB6 & 15 \\
\toprule
Julnnaus Pollen \index{Julnnaus Pollen} & R & I & 20 & +3 Constitution for 2 hours & FO6 & 25 \\
\toprule
Erain Flower Pollen \index{Erain Flower Pollen} & R & 1 round & 20 & +2 Strength and Intelligence and Dexterity. + 3d6 temporary hit points, for 1 hour & FT7 & 75 \\
\end{tabularx}

\begin{multicols}{2}

\medskip

\textbf{The use of drugs is completely optional, it is the Storyteller who decides their presence and availability also based on the sensitivity of the players}.

Drugs are addictive. After the effect ends within 24 hours, make a Will saving throw at difficulty 15 or take another dose, the next saving throw will have difficulty +1, and so on.

Whenever you take a new dose within 2 weeks of the first one, the non-addict saving throw increases by 1. Not taking a dose increases your Fatigue level by one.

It takes 7 successful saving throws in a row to end the addictive effect.


\end{multicols}


\pagebreak

\section{Movement and Transport} \index{Transport} \index{Movement}

\label{movimento-e-trasporto}

\begin{changemargin}{0.3cm}{0.3cm} \begin{enfasi}{
My left foot works great, but I still wouldn't be able to walk if there was no right foot! (Madagascar 3 - Wanted in Europe, Film)

When you can't run anymore, walk fast; when you can't anymore walk fast, walk; when you can no longer walk, use the stick; but never hold back. (Mother Teresa of Calcutta)
}\end{enfasi} \end{changemargin} \medskip


\begin{multicols}{2}

\lettrine[lines = 2, lhang = 0.33, loversize = 0.25, findent = 1.5em]{M}{ovement} can be distinguished according to which situation it applies.

\medskip

\begin{itemize}
\item Tactical, when fighting, accurate distances, grid and squares of 1 meter side are used
\item Local, for exploring an area, measured in meters per minute.
\item By land, to move from one place to another, measured in km per hour or per day.
\end{itemize}

\subsection{Types of Movement}

When moving in different movement situations (Tactical, Local Overland), creatures generally walk or run.

\textbf{Walking}: \index{Walking} Walking represents an unhurried but determined movement of approximately 4 km per hour for a human without encumbrance.

\textbf{Running} \index{Running}: It means moving about 12 km per hour for a human.

The running character has a 1d6 penalty on attack rolls and 4 on defense in the round he runs.
Running as a move action doubles your movement speed, not triples it. Only in non-combat situations does the race triple the movement (local movement, overland)

\subsection{Table: Movement and Distance and Speed: on Foot} \index{Movement on Foot} \index{Table of Movement and Distance and Speed: on Foot}

This table shows basic ground movement values in noncombat situations.

\bigskip

\begin{tabularx}{0.43\textwidth}{lccc}
\multirow{2} *{Type of movement} &
\multicolumn{3}{c}{Movement} \\
\cmidrule (lr){2-4} & 6m & 9m & 12m \\
\midrule
\multicolumn{4}{c}{\textbf{Tactical Movement)}} \\
Walking & 6m & 9m & 12m \\
Run (x2) & 12m & 18m & 24m \\
\multicolumn{4}{c}{\textbf{One minute (local)}} \\
Walking & 36m & 54m & 72m \\
Run (x3) & 108m & 162m & 216m \\
\multicolumn{4}{c}{\textbf{One hour (by land)}} \\
Walking & 3km & 4km & 6km \\
Run (x3) & 9km & 12km & 18km \\
\multicolumn{4}{c}{\textbf{One day (by land)}} \\
Walking & 24km & 32km & 54km \\
\end{tabularx}


\subsection{Tactical Movement} \index{Tactical Movement}

Tactical Movement is used during a fight.
Distances are measured in one-meter squares, movement is handled through Movement Actions.

A character can use 1 (Move) Action to move up to all of his movement. He can perform the Move Action several times in the round, up to 3 times, thus moving three times his movement.

He can also perform a Sprint Action \index{Sprint} or travel twice his movement in a single Action. However, he stumbles on penalties for those who run (-1d6 to hit, -4 Defense).

A character can perform up to 3 Dash Actions, i.e. it runs for the whole round thus following its movement * 6.

\subsubsection{Obstructed Movement} \index{Difficult Terrain}

Difficult terrain, obstacles or poor visibility can prevent movement. When movement is hindered you are moving at half speed, you need 2 Actions to cover your distance of 30 feet (if you are human without encumbrance ...). Or with a move action you only cover 4 meters.

If there is more than one particular condition, add all the applicable additional costs to each other, i.e. if the terrain is difficult and you crawl it means you are moving a quarter of your movement.

In some situations movement is so hampered \index{Movement almost impossible} that the distance that can be traveled per Action is minimal, in which case all 3 Actions can be used to move only 1 meter in any direction.

Do not apply this rule to cross impassable terrain or to move when it is not possible to do so in any way.

You cannot \textbf{Sprint} (Run) or \textbf{Charge} \index{Charge on Difficult Ground} \index{Sprint on Difficult Ground} smoothly through a \textbf{path that hinders movement} , or difficult terrain. The player may attempt an Acrobatics check at DC 20 to be able to charge or run, however only covering half the distance. The Acrobatics check is not required, even if you are running halfway, if you have the Advantage \hyperlink{rinoceronte}{Rinoceronte}.

\textbf{Moving prone} \index{Moving prone} \index{Crawling}, Swim or Crawl \index{Crawl} is considered difficult terrain.

The terrain where there are bodies of creatures is considered difficult \index{Moving on bodies}.

\subsubsection{Through enemies} \index{Through enemies} \index{Crossing occupied squares}

\index{Passing through enemies} \index{Movement through} A character may \textbf{cross} but not stand in \textbf{an occupied zone} by a teammate without being \hyperlink{ristretto}{\textbf{ristretti}} \index{Restricted} . If the movement brings it to \textbf{share} \index{Sharing the square} the same map square (assuming both medium-sized creatures) will have a -1d6 attack roll and a -4 defense roll.

To cross terrain where there is a hostile creature, you must make a saving throw or Reflex or Fortitude opposite (your choice) to that of the opponent you want to \textbf{cross} the terrain. If you fail the check, you go back to the nearest square other than the creature you wanted to cross, potentially sharing it with someone and the Action (movement and attempt to pass) is considered finished. If the enemy has the Ability \hyperlink{opportunista}{Opportunista} in addition to obstructing the passage he can perform an attack. It costs 1 Action.

\subsubsection{Swap seats} \index{Swap seats} \index{Swap seats}

A character in contact with another creature can use \textbf{an Action} to \textbf{swap places}, if the creature is hostile an opposing Fortitude saving throw is required to be able to switch. For each \textbf{size of difference}, whoever has the largest one gets + 1d6 bonus on check. \index{Swap seats}. It costs 1 Action.

\begin{changemargin}{0.3cm}{0.3cm} \begin{tcolorbox}[title = Tups in the tunnel] %box giocatore
Tups is with his companions in a narrow tunnel in single file. It is in fourth position.

Suddenly an enemy appears in front and Tups is the fastest to react, using a Move Action \textit{\textbf{crosses}} the 3 companions in front of him remaining \textbf{restricted} with the first in line.

You may decide to (among various options):

- stay restricted and attack

- push the partner into the previous square, making him squeeze with another partner

- push the partner to the next square! making him cross the enemy square

- go back to its initial picture

- try to cross the opponent, but if he fails only 1 other Action will remain and he will be restricted with the partner, damaging both

\end{tcolorbox} \end{changemargin}

%\begin{changemargin}{0.3cm}{0.3cm}\begin{Storytellere}   %box Storytellere
%Se volete un crudo realismo allora è terreno difficile attraversare anche zone dove ci sono creature amichevoli. \end{Storytellere}\end{changemargin}

\subsection{Local Movement} \index{Local Movement}

Characters exploring an area use local movement, measured in meters per minute.

In these situations it is not essential to measure the distance precisely but as soon as the situation becomes "problematic" or requires attention, the map is converted into tactical movement, checked and measured.

\medskip

\begin{itemize}
\item
Walking: A character can walk without any problems on a local scale for 8 hours a day.
\item
Run: A character can Run for a number of minutes equal to three times their Constitution score on a local scale without having to rest (at least one round).
\end{itemize}


\subsection{Land Movement} \index{Land Movement}

Characters traveling long distances use ground movement. Land movement is measured in hours or days. One day represents 8 hours of real travel time. For rowing boats, one day means rowing for 10 hours. For sailing ships it represents 24 hours of movement.

\textbf{Walking} \index{Walking}

You can walk 8 hours in one day of travel without any problems.

Walking longer can be exhausting (see Forced March, below).

\textbf{Go Fast} \index{Go Fast}

You can go fast (movement * 2) for 1 hour without any problems. Going fast for a second hour between two sleep cycles causes 1 Non-Lethal Damage, and each additional hour does double the damage taken in the previous hour. A character who takes nonlethal damage from fast paced is considered fatigued.

A Fatigued character cannot Run or Charge.

\textbf{Run} \index{Run}

It is not possible to Run for a long time. Attempts to run and rest in cycles work like Go Fast.

\textbf{Land} \index{Land}

The terrain you travel on affects how much distance you travel in an hour or day (see Table: Terrain and Ground Movement). A main road is a main, straight and paved road. A common road is usually a rough path. A trail is like a common road except that it allows you to travel in single file only and does not benefit a group traveling with vehicles. Free land is a wilderness area with no marked trails.

\bigskip

\textbf{Optional - Table: Terrain and Ground Movement} \index{Optional - Terrain and Ground Movement Table}

The table shows the multipliers for the distance traveled.

\medskip

\begin{tabularx}{0.45\textwidth}{XXXX}
\textbf{Ground} & \textbf{Main road} & \textbf{Common road} & \textbf{Path not beaten} \\
\toprule
Heath & x1 & x1 & x3 / 4 \\
Hill & x1 & x3 / 4 & x1 / 2 \\
Sandy Desert & x1 & x1 / 2 & x1 / 2 \\
Forest & x1 & x3 / 4 & x1 / 2 \\
Jungle & x1 & x3 / 4 & x1 / 4 \\
Mountain & x3 / 4 & x3 / 4 & x1 / 2 \\
Swamp & x1 & \texttimes 3/4 & \texttimes 1/2 \\
Plain & x1 & \texttimes 3/4 & \texttimes 1/2 \\
Frozen Tundra & x1 & \texttimes 3/4 & \texttimes 3/4 \\
\end{tabularx}

\bigskip

\textbf{Forced March} \index{Forced March}

On a normal day of walking, you can walk for 8 hours. The rest of the day is used to make and break camp, rest and eat.

If you walk more than 8 hours, you must make a Fortitude save at difficulty 11 +1 for each consecutive day of forced walking or you become Fatigued. The saving throw is made every 2 hours after 8 hours of walking.

The forced march can be held for a number of days equal to the Constitution value + 1 before incurring fatigue regardless of the outcome of the saving throw.


\textbf{Movement in the saddle} \index{Movement in the saddle}

A mount carrying a rider can move at a fast pace. However, the damage it takes is normal damage rather than non-lethal. She can also be forced into a forced march, but her Constitution checks automatically fail and again the damage she takes is normal damage. Mounts are also considered Fatigued when they take damage from fast gait or forced march.

\end{multicols}

%\medskip
%\begin{center}
%\includegraphics[height=0.3\linewidth]{immagini/carretto.png}
%\end{center}

\subsection{Table: Mounts and Vehicles} \index{Mounts} \index{Vehicles} \index{Mounts and Vehicles Table} \index{Horse moving}

\medskip

\label{tabella-cavalcature-e-veicoli} \index{Dog} \index{Pony} \index{Cart} \index{Raft} \index{Boat} \index{Ship}

\begin{tabularx}{0.95\textwidth}{lXX}
\textbf{Mount or Vehicle (weight carried)} & \textbf{Per hour} & \textbf{Per day} \\
\toprule
Gallop Dog & 6km & 48km \\
Gallop Dog (50.5-150 kg) * & 4.5km & 36km \\
Light Horse & 7.5km & 60km \\
Light Horse (115.5-345 kg) * & 5.25km & 42km \\
Heavy Horse & 7.5km & 60km \\
Heavy Horse (150.5-450 kg) * & 5.52km & 42km \\
Pony & 6km & 48km \\
Pony (75.5-225 kg) * & 4.5km & 36km \\
Cart or Wagon & 3km & 24km \\
\textbf{Boat} & & \\
\toprule
Raft or Barge (pole or trailer) & 0.75km & 7.5km \\
Barcone (in Remi) ** & 1.5km & 15km \\
Rowboat ** & 2.25km & 22.5km \\
Sailing Ship (sails) & 3km & 72km \\
Warship (sails and oars) & 3.75km & 90km \\
Long Ship (sails and oars) & 4.5km & 108km \\
Galea (oars and sails) & 6km & 144km \\
\end{tabularx}

\begin{multicols}{2}

\bigskip

* Quadrupeds, like horses, can carry higher loads than characters. See Transportation Capacities for more information.

A mount can only carry a creature smaller than its own size.

** Rafts, barges and barges are used on lakes and rivers. If they go with the current, add the speed of the current (usually 4.5 km / h) to the speed of the boat. In addition to being paddled for 10 hours, the boat can also be carried by the current for another 14 hours if someone can steer it, adding another 63km to the daily distance traveled. These boats cannot be rowed against a very strong current, but they can be pulled against the current by pack animals on the shore.

\textbf{Mount Harnesses} \index{Mount Harnesses} \index{Horse Armor}

A mount can be harnessed with armor. Generally light armor will grant a +2 Defense bonus, Medium armor will grant a +4 Defense bonus by reducing movement by 25 \%, Heavy armor will give +6 Defense by lowering movement by 33 \%.

\subsection{Escape and Pursuit} \index{Escape} \index{Pursuit}

In round-by-round movement it is impossible for a slow character to escape a fast character without some kind of help. Likewise, it's not a problem for a fast character to escape a slower one.

When the speed of the two characters involved is the same, there is a fairly simple method to resolve a chase: if one creature is chasing another and both are moving at the same speed, and the chase continues for at least a few rounds, it is necessary that pursuer and pursued make 3 consecutive saving throws on opposing Reflexes.

Whoever wins the challenge manages to make them lose their tracks or grab the fugitive.

In the case of a long chase where there is no chance to hide or lose track, make 3 opposing Fortitude saving throws to determine which side can keep the pace longer. Whoever gets the most successes manages to escape or is the pursuer who manages to reach it.

\subsection{Load and Transport Capacity: Overall Weight} \index{Load Capacity} \index{Overall Space}

\label{sec:capacita-di-carico-e-trasporto-ingombro}

\subsubsection{Weight and Overall dimensions} \index{Overall dimensions} \index{Weight}

Carrying treasures, dragon pieces, full armor not to mention disproportionate weapons or breaching rams, pulleys and hoist make movement difficult.

When you evaluate the weight carried, also think about the size!
Carrying a roll of 12 meters x 6 meters of silk is not a demanding physical activity, it will be a few kilos, but the bulk is such that it cannot allow additional load.

There can be light but extremely bulky objects (hollow trunks, silk carpets in fact ...) or small but very heavy (mercury spheres, gold-woven clothes), for all these objects the weight value must also be considered as a function of footprint.

The weight values of the objects are added together to give the total load carried.

\subsubsection{Load Capacity}

Consult the Load Capacity table to understand your load capacity.

\bigskip

\end{multicols}

\textbf{Table: Load capacity} \index{Table Load capacity}

\medskip

\begin{tabularx}{0.95\textwidth}{XXXXX}
\toprule
\multirow{2} *{Strength} & \multicolumn{3}{c}{Load Capacity} & \\
\cmidrule (lr){2-5} & Normal (kg) & Weighted (kg) & Maximum (kg) & Thrust (kg) \\
\cmidrule (lr){2-5} & & -3 Des, -1/3 Mov, & -6 Des, Mov = 1/3 & Mov = 1/4 \\
-3 & 0-5 & 5-7 & 7-15 & within 22 \\
-2 & 0-7 & 7-10 & 10-20 & within 25 \\
-1 & 0-10 & 10-15 & 15-30 & within 45 \\
0 & 0-15 & 15-22 & 22-45 & within 65 \\
+ 1 & 0-20 & 20-30 & 30-60 & within 90 \\
+ 2 & 0-35 & 35-50 & 50-100 & 1 within 50 \\
+ 3 & 0-60 & 60-90 & 90-180 & within 270 \\
+ 4 & 0-95 & 95-140 & 140-280 & within 420 \\
+ 5 & 0-140 & 140-220 & 220-440 & within 660 \\
\end{tabularx}

\begin{multicols}{2}

\subsubsection{Push, Drag or Lift}

You can carry, between backpack, armor, weapons and equipment, a weight equal to that indicated on \textbf{Normal} without penalty.

If you carry a weight that falls into \textbf{Weighted} you have a -3 on Dexterity checks and the movement is decreased by one third.

The weight indicated as \textbf{Maximum} means the maximum transportable. You have a -6 on Dexterity-based checks, your movement decreases to a third.

If the weight is over the Maximum but within \textbf{Thrust} you can lean it and push it (if feasible), in this case your speed will be reduced to a quarter.

An object if it has "\textbf{wheels}" on which to move is considered to have a weight equal to one quarter as far as the Thrust is concerned.

\textit{Remember that the armor and shield worn are half the weight of what is marked.}

In case more creatures push "severe" consider the full Strength of the strongest creature plus half the Strength of creatures with less power (minimum 1). Also clearly evaluate how many people can simultaneously push given the size of the object to be transported.

\subsubsection{Larger and Smaller Creatures}

\textbf{Larger bipedal creatures} can carry more Encumbrance based on size category

\textbf{Table: Transport Modifiers for Creatures by Size} \index{Transport Modifiers for Creatures by Size}

\medskip

\begin{tabularx}{0.45\textwidth}{ll}
\textbf{Creature Size} 	& 	\textbf{Compared to Normal Load} \\
\toprule
Very small & 1/16 \\
Minute & 1/8 \\
Lowercase & 1/4 \\
Small & 1/2 \\
Large & x2 \\
Huge & x4 \\
Gargantuan & x8 \\
Colossal & x16 \\
\end{tabularx}

\medskip

\subsubsection{Multi-legged creatures}

Creatures with 4 legs or more can carry heavier weights. Consult the table below and if necessary multiply the modifiers shown with those due to the size.

%\begin{center}
%\includegraphics[height=0.5\linewidth]{immagini/cavallo.png}
%\end{center}

\textbf{Table: Transport Modifiers for Multi-legged Creatures} \index{Transport Modifiers Table for Multi-legged Creatures}

\medskip

\begin{tabularx}{0.45\textwidth}{ll}
\textbf{Creature Legs} 	& 	\textbf{Compared to Normal Load} \\
\toprule
4 legs & x2 \\
6 legs & x2.5 \\
8 legs & x3 \\
12 legs & x4 \\
every other 2 legs & +0.5 \\
\end{tabularx}

\medskip

A horse being Large and quadrupedal, with Strength 3, can carry without problems up to a maximum of 60kg of base * 2 because Large * 2 because it is quadrupedal, therefore 240kg without incurring a penalty.

\subsection{Other Types of Movement}

\label{altri-tipi-di-movimento}

\subsubsection{Swimming} \index{Swimming}

A creature with a Swim speed can move through water at its indicated speed without making Swim checks. You gain a +8 bonus on any Swim check to perform a particular action or avoid a hazard. The creature can always choose to take 10 on a Swim check, even if distracted or in danger when swimming. Cannot get 10 only in stormy waters. Such a creature can use the action of running while swimming, provided that it is swimming in a straight line.

If you do not have the Swim type of movement moving in water is considered difficult "terrain", and therefore you are moving at half the speed indicated by movement. The Swim check is needed every time you have to move, if you fail you don't move and you get a -1 on the next check, if you fail critically the next check takes a -1d6. When the proof is less than 5 it begins to sink and drown.

\subsubsection{Scalar} \index{Scalar}

A creature with a Climb speed has a +8 bonus on all Climb checks. If the creature must make a Climb check to climb any wall or slope, it can always choose to take 10, even if in a hurry or threatened on the way up.

If a creature with a Climb speed attempts a quick climb (see above), it's as if it were making a Sprint Action and makes a single Climb check with a -5 penalty.

A creature retains its Defense Dexterity bonus (if any) while climbing, and opponents gain no special bonus for their attacks against it.

If you do not have the Climb movement type, it is considered difficult "terrain", and therefore you are moving at half the speed indicated by the movement.

\subsubsection{Excavate} \index{Excavate}

A creature with a Dig speed can tunnel through earth, but not through rock unless the descriptive text says otherwise. Creatures cannot charge or run while digging.

Most burrowing creatures don't leave tunnels that other creatures can use (either because the material they dig through fills the tunnel behind them or because they don't actually move material when they dig), see the individual creature description for details.

\subsubsection{Walk - Speed Over Ground}

Ground Speed is the normal speed for characters who do not climb, swim, or fly.

\subsubsection{Volare} \index{Volare}

Flying for a creature with this ability is like walking for a "terrestrial" creature. A creature with flight uses its actions to move but is unlikely to be affected by difficult terrain.

\medskip

\begin{center}
\includegraphics[width = 0.8 \linewidth]{immagini/grifonicastello.png}
\end{center}


\end{multicols}

\pagebreak

\section{Mastering} \index{Mastering} \index{Storyteller}

\label{masterizzare}


\subsection{The Storyteller}

\begin{changemargin}{0.3cm}{0.3cm} \begin{enfasi}{
Who commands the story is not the voice: it is the ear. (Italo Calvino)
} \end{enfasi} \end{changemargin} \medskip

\begin{multicols}{2}

\label{il-Storytellere}

\lettrine[lines = 2, lhang = 0.33, loversize = 0.25, findent = 1.5em]{W}{hile} player plays a character in an adventure, the Storyteller is the one who manages it. He certainly has a lot more work, but creating a whole world for your friends to explore can be very rewarding.

The role of the Storyteller is not easy but it grants enormous privileges. Seeing your friends play, have fun, "go crazy" behind doubts, riddles and situations you create gives a lot of fun and moments of true conviviality.

Your role is that of the great orchestrator, planner or even landscape architect if you prefer, with a few simple brush strokes you outline the structure and the players will then add details and situations.

Your \textit{work} is fundamental and very important, the goodness of the game session depends on you. Your aim is to entertain, engage but also terrify and engineer.

You are not the protagonist nor the adventure, but the players, your friends, don't steal the show but like a great dance be the conductor where the instruments are the possibilities offered by the OBSS, the music is the adventure and the dancers the players.

\begin{changemargin}{0.3cm}{0.3cm} \begin{Storytellere}
OBSS wants to help you and other players have fun. Always use common sense when applying a rule. Your aim is not to kill characters but to create worlds and campaigns that evolve around the characters, their actions and decisions. Incorporate the things that interest players, keep them involved, make them understand that the world is alive and they are a part of it. If you are good at your adventures, situations will echo in other sessions and off the table.
\end{Storytellere} \end{changemargin}

\subsection{Experience Points} \index{Experience Points} \index{PX}

\label{punti-esperienza}

In OBSS, leveling up is not constrained by the number of monsters faced or the treasures obtained, but by the difficulty factor of the encounters and how the players have played.

In OBSS the primary suggestion is to reward the players who are most committed to the group, those who have contributed most to the success of the adventure and the session. Experience points not only measure success but also participation in the game.

It is therefore possible to have characters with different experience points and potentially even different levels.

Alternatively, you can decide to pass the level whenever you deem it necessary for a good game and adventure, standardizing the experience of the characters.

Below you will find the rules for establishing how to build encounters, clashes and how to reward players / characters.

Take this experience point table

\subsubsection{Table: Experience Points (XP) / Level} \index{Experience Points / Level Table} \index{PX per Level}

\label{tabella-punti-esperienza-livello}

\begin{tabularx}{0.45\textwidth}{lX | lX}
\textbf{Level} & \textbf{Experience Points} & \textbf{Level} & \textbf{Experience Points} \\
\toprule
1 	& 	0-10 PX & 	11 & 310 PX \\
2 	& 	11 PX & 	12 & 355 PX \\
3 	& 	25 PX & 	13 & 400 PX \\
4 	& 	60 PX & 	14 & 450 PX \\
5 	& 	100 PX & 	15 & 500 PX \\
6 	& 	130 PX & 	16 & 555 PX \\
7 	& 	165 PX & 	17 & 610 PX \\
8 	& 	200 PX & 	18 & 670 PX \\
9 	& 	235 PX & 	19 & 730 PX \\
10 	& 	270 PX & 	20 & 795 PX \\
	& 		& 	20 + & +70 PX \\
\end{tabularx}

\begin{itemize}
\item
\textbf{For each match designated to challenge the group in a medium or difficult way, award 1 experience point.}
\item
\textbf{For each encounter designated to be potentially deadly, award 2 experience points.}
\item
\textbf{For each final encounter, the climax of the adventure, assign 3 experience points, these points more than for the fight "with the final Boss" must be awarded as merit for having completed a long adventure.}
\end{itemize}

In the Monstruary there are PXs for monsters (eg Challenge 13 (10,000 PX). Ignore them. Use the challenge rating. Or if you want an even more OSR system, use them exclusively. See the Experience Points per Level option table.

I would continue to suggest rewarding players also based on the fun provided, ideas, epic actions, treasures found.

\medskip

\textbf{Table: Experience Points per Level Option} \index{Experience Points per Level Option Table}

\begin{tabularx}{0.45\textwidth}{lX | lX}
\textbf{Level} & \textbf{Experience Points} & \textbf{Level} & \textbf{Experience Points} \\
\toprule
1 & 0 		& 11 & 89,600 \\
2 & 250 		& 12 & 121,000 \\
3 & 1,100 		& 13 & 159,000 \\
4 & 3,100 		& 14 & 207,000 \\
5 & 6,500 		& 15 & 266,000 \\
6 & 11,800 	& 16 & 341,000 \\
7 & 19,700 	& 17 & 433,000 \\
8 & 30,600 	& 18 & 550,000 \\
9 & 45.600 	& 19 & 697,000 \\
10 & 64,800 	& 20 & 883,000 \\
\end{tabularx}

The most attentive will have noticed that it is not the standard 5e table, the difference is desired and sought after.

\medskip

These points will be awarded to the group and therefore to all players, as long as they have at least tried to participate in the clashes / challenges.

If the group transforms an easy encounter (from 0 experience points) into a mortal encounter due to its "inability" or "bad luck", you do not have to give 2 experience points. Try to reward the team spirit, the energy spent and if possible the creativity in coming out alive despite everything.

When I say "encounter" you don't mean just a fight with monsters, an encounter means any role-playing event that challenges and checks players. This challenge can be a witty discussion with the noble who does not want to pay them at the end of a mission, to the challenge of a riddle, rebus, of well-placed traps.
A monster doesn't have to be killed to get experience points, you just need to defeat it, capture it, win in a different way. In case of retreat by the players or the enemy, grant half the experience points provided for the fight if there was at least an attempt to challenge.

\begin{center}
\includegraphics[width = 0.75 \linewidth]{immagini/deathbeowulf.png}

\textit{Henry Justice Ford}
\end{center}

\bigskip

Whenever the player or group: \index{Bonus PX} \index{Experience Bonus}
\begin{itemize}
\item
\textbf{reach the set objectives} (group or single reward);
\item
\textbf{play a great role-playing game} (reward to the player);
\item
\textbf{fully exploits and indeed is alternative in the use of one's Skills and abilities (without falling into the powerplayer)} (player reward);
\item
\textbf{solve problems in a creative, imaginative and functional way} (prize to the player);
\item
\textbf{have good cooperation and know how to interact as a group towards NPCs} (reward to the group or player);
\item
\textbf{discover or start adventure clues and creation of new plots} (prize to the player);
\item
\textbf{collect 5000 gold coins (or equivalent treasure)} (group reward);
\end{itemize}

I also suggest you evaluate these actions:

\begin{itemize}
	\item Clever and crafty use of a skill or object
	\item Ingenious (and alternative) use of spell
	\item An action that jeopardizes one's life for the group
	\item Actions carried out following the creed of one's Patron (for Followers or Devotees). These could also award trait points
	\item Helping a player in need
	\item Propose alternative plans and actions to what is foreseen
\end{itemize}

\bigskip

Reward the player (s) with 1 experience point. These points must be given per game session to those who have earned them as an individual or group. There is no need to give experience points at the end of the game session, keep track of them and inform the players when there is a moment of pause, of reflection on what has happened and done.

In this system it takes about 6/10 sessions to pass the level, potentially even less if the players prove to be good and interpret characters and situations in a brilliant way.

Make sure that each session can award 1-4 points at least. Build the session so that all players can participate and no one feels left out.

As far as possible, each session should include a role part, an exploration part, three combat parts (even many more than three), a rest part.

\bigskip

\begin{changemargin}{0.3cm}{0.3cm} \begin{Storytellere}
It may seem anachronistic when the sixth edition of the most famous RPG is already in development, returning to reward players based on the gold taken from monsters.

However, I can guarantee you that if your group is particularly "poor" in role-playing games or simply prefers a more combative style, knowing that the gold collected equals experience can make going on an adventure much more dynamic and exciting.

OBSS is based on the principles of the OSR and as such the exploration and combat phase has its own important and vital weight.
\end{Storytellere} \end{changemargin}

\subsection{Meetings} \index{Meetings}

\label{incontri}

\begin{changemargin}{0.3cm}{0.3cm} \begin{enfasi}{
What is life without hope? A throw of dice in the darkness, among the
delusions. Ambrogio Bazzero
}\end{enfasi} \end{changemargin}


An encounter is a time of tension and hope, fear and challenge. It is an opportunity to show and manifest one's skills and to work as a group.

A meeting is not an opportunity to show off your absolute power, both as a Storyteller and as a Player. The Storyteller will know \xcancel{punire} to educate the player who wants to be beyond the group and not part of it.

You will find in the following pages the instructions to create easy (0 experience points), medium and high (1 experience point), extraordinary (2 experience points) and epic (3 experience points) challenges.

In any case, it will always be you, the Storyteller, who will establish and know if a challenge is trying or not, if it is challenging and critical for the players and therefore evaluate both its impact as experience points and as difficulty.

A meeting is an event that confronts the characters with a specific problem they have to solve. Many are fighting with monsters or hostile NPCs, but there are other types: a corridor bristling with traps, a political interaction with a suspicious king, a dangerous passage over a rickety rope bridge, an awkward argument with a friendly NPC who he feels that a character has betrayed him, or anything that adds some drama to the game.

Brain teasers, interpretation challenges, and skill checks are the classic methods of meeting resolution, but the more complex encounters to build are the most common combat encounters.

A clash can also be born clearly unbalanced, it will be the forethought of the players to understand when to escape!

When planning a combat encounter, first decide what level of challenge you want the PCs to face, then follow the steps outlined below.

\textbf{Determine APL}: \index{APL} Determine the average level of the characters: this is the Average Party Level (APL for short). You should round this value to the nearest whole number (this is one of the few exceptions to the round down rule).

Note that this reference guide to creating an encounter assumes a group of four or five PCs. If your party has six or more players, add one to their average level. If your party contains three or fewer players, subtract one from their average level. For example, if your party consists of six players, two 4th level and four 5th level, the APL is 6th (28 total levels, divided by six players, rounding up and adding one to the final result) .

\begin{center}
\includegraphics[width = 0.7 \linewidth]{immagini/impegnativa.png}

\textit{Henry Justice Ford}
\end{center}

\textbf{Determine Challenge Rank}: Challenge Rank (or Challenge Rank, CR) is a convenience number used to indicate the relative risks presented by a monster, trap, danger, or other encounter: plus the degree of Challenge is higher, the more dangerous the encounter. Refer to Table: Determine Encounters to determine the Degree of Challenge your group should face, based on the difficulty of the challenge you want and the APL.

\medskip

\textbf{Table: Determine Encounters} \index{Table Determine Encounters}

\medskip

\begin{tabular}{ll}
\textbf{difficulty} & \textbf{Degree of Challenge (degree of Challenge)} \\
\toprule
Easy 		& APL \\
Media 		& APL +2 \\
High 		& APL +3 \\
Extraordinary 	& APL +4 \\
Deadly 		& APL +6 \\
Epic 		& APL +8 \\
\end{tabular}


\subsubsection{How many fights to face} \index{How many fights to face}

There is no single answer. It is your choice, the system finds its balance between 3 and 5 fights per day. Of course, they don't all have to be on High difficulty !. Clashes are ultimately a resource management to be used against an enemy. These resources are hit points, spells, potions, and scrolls or consumable items you possess. If you place an Extraordinary match as the first match, it is likely that the players will then decide to rest to recover their energy, otherwise you could opt to tire them slowly with medium matches and then try them with a greater difficulty. Finally, remember that a \textit{fight} does not have to be physical, but also traps, puzzles / riddles, alternative challenges ... anything that makes you consume resources and reason.

Always evaluate where they move and what is around, it will be natural to find the right number and types of fights and enemies.

\subsubsection{Building the Encounter} \index{Building the Encounter}

To build a match, first calculate the value of the APL.

To develop your encounter, add in creatures, traps, and hazards until you get to your scheduled APL.

Start by calculating the highest Challenge rank challenges in the encounter, completing the rest with smaller challenges.

For example, you want your party of six 7th-level characters to have a Medium challenge and face some Gargoyles (Challenge rank 2 each), Xorns (Challenge rank 5) and their leader, a Stone Giant (Rank of Challenge 7). Characters have APL 8 and Table: Determine Encounters states that a Medium challenge for an APL 8 is a Challenge 10 grade encounter.

Starting from an established Challenge rank (10), follow this chart to determine how many "monsters to include in the fight".

\medskip

\textbf{Table: Challenge rank weight for APL calculation} \index{Challenge rank weight table for APL calculation}

\medskip

\begin{tabularx}{0.45\textwidth}{XXX}
\textbf{Target challenge} & \textbf{Creature vs. APL target challenge} & \textbf{"Weight" per single creature} \\
\toprule
Challenge grade & -7/8 & 5 \\
& -6 & 10 \\
& -5 & 15 \\
& -4 & 20 \\
& -3 & 30 \\
& -2 & 50 \\
& -1 & 65 \\
& 0 & 80 \\
                      & +1 		& 90 \\
                      & +2 		& 100 \\
\end{tabularx}

\bigskip

\textbf{To reach the goal we have to add "the weights" until we reach 100, which is 100 \% of the challenge.}

In our example, a Stone Giant has a Challenge degree 7, which is a Challenge degree -3 compared to our Challenge degree 10 difficulty goal, the Xorn has a Challenge degree 5 or -5 compared to a Challenge degree 10, the Gargoyles have Challenge 2 grade or -8 compared to Challenge 8 grade.

An enemy with a Challenge -3 rating has a weight of 30, a -5 Challenge rating has a weight of 15, a -8 Challenge rating has a weight of 5.

To reach the goal of a Challenge 10 rank I will put 1 Challenge rank -3 (i.e. a stone giant), 3 Challenge rank -5 (i.e. three Xorns) and 5 Challenge rank -8 (i.e. five gargoyles). The Total will be 30 (a Giant of Stones) + 3 * 15 (three Xorns) +2 + 5 * 5 (gargoyle) = 30 + 45 + 25 = 100. Goal achieved!

Opponents with a Challenge rating of less than 8 compared to the APL are counted, "weigh", only if they are greater than 20 as a unit.

\subsubsection{Add the PNGs}

A creature that possesses levels, abilities, skills, which could be a character considers itself an NPC. These creatures can play a very important role and should not be used as mere "monsters". Give it thickness and you will create unforgettable characters.

\subsubsection{Ad Hoc changes to Challenge rank}

While you can change the monster's specific Challenge rank by advancing it, applying changes or levels, you can also adjust the difficulty of the encounter by applying ad hoc changes to the encounter or creature itself.

Described here are three additional ways you can alter the difficulty of the encounter.

\begin{center}
\includegraphics[width = 0.7 \linewidth]{immagini/tesoro2.png}
\end{center}

\subsubsection{Favorable terrain for the PCs}

An encounter with a monster that is not in its preferred element (such as a Yeti encountered in a lava filled cave, or a huge Dragon encountered in a very small room) gives the characters an advantage. Treat the encounter as having a lower Challenge rating than its Royal Challenge rating.

\subsubsection{Terrain unfavorable to the PCs}

Monsters are designed with the premise that they are encountered in their preferred terrain: encountering an Aboleth underwater does not increase the Challenge rank of the encounter, even if no character is able to breathe underwater.

If, on the other hand, the terrain has a more significant impact on the encounter (such as an encounter with a creature with Blind Sight in an area that suppresses all light sources), you can, if necessary, increase the challenge rating of the encounter. meeting was of a higher degree.

\subsubsection{NPC Equipment Changes}

You can increase or decrease the difficulty given by NPCs by modifying their Equipment. An NPC encountered without Equipment should have a reduced Challenge rank of 1 (provided that the loss of Equipment is really counterproductive to the NPC), while an NPC who has an Equipment equivalent to that of a character (as indicated on Table: Wealth of Characters per Level) has a Challenge rank 1 higher than its Royal Challenge rank.

Care should be taken to assign this extra equipment to NPCs, especially at higher levels, where you can consume the entire treasure of your adventure in one fell swoop!

\begin{changemargin}{0.3cm}{0.3cm} \begin{Storytellere}
But how many fights can you manage per session?

There is no exact answer (2/4 per session / day ???), a lot depends on the type of players you have and what game they play.
Stay focused on the adventure, do not think that it is better not to tire the characters if they do not hold up the next fight. You don't have to worry too much about the characters, that's up to the players. You have to think based on the environment where you are and who is around.

In any case, common sense always helps. Wear out the characters but make sure it's not a "Total party kill" every session.
\end{Storytellere} \end{changemargin}

\subsubsection{Assign PX}

Characters advance in level by defeating monsters, overcoming challenges, having fun, completing the adventure and grabbing treasures: in doing so they earn Experience Points (PX for short). You can award Experience Points as soon as a challenge is passed, but this could disrupt the flow of the game. It is easier to assign experience points at the end of a game session (or sessions) which allows the characters to reflect on what has happened. The player can use the time available between game sessions to update the card.

\subsubsection{Arranging Treasures}

As characters level up, the amount of treasure they carry and use also increases. In OBSS it is assumed that all characters of the same level have roughly the same amount of treasure and magical items. Since the primary income for a character comes from treasures and loot from adventures, it is important to moderate wealth and treasure in your adventures.

To help you arrange the treasures, the amount of magical items and loot that characters receive for their adventures is linked to the Challenge rank of the encounters they face: the higher the Challenge rank of the encounter, the greater the treasure awarded. .

\textbf{Table: Character Wealth by Level} by Level indicates the amount of treasure each character should have at a specific level. Note that this table is based on a standard game model.

Adventures with rare magic could only assign half this value, while more epic games could double it. It is assumed that some of the treasure is consumed in the course of an adventure (such as potions and scrolls) and that some of the lesser-used items are sold for half their value to purchase more useful equipment.

Table: Character Wealth by Level can also be used to allocate Equipment for characters starting after 1st level, such as a new character created to replace a dead one. Characters should not spend more than half of their total wealth on a single item.

For a balanced method, characters created after 1st level should spend 25 \% of their wealth on weapons, 25 \% on armor and protective items, 25 \% on other magical items, 15 \% for consumable items such as wands, scrolls and potions and 10 \% for normal equipment and coins. Different character types may spend their wealth differently than suggested; for example, arcane spellcasters might spend more on magical and consumable items than on weapons.

\textbf{Table: Treasure Values per Encounter} lists the amount of treasure each encounter should award based on the average character level and PX progression rate of the campaign. Easy encounters should award a treasure that is half the level of the PCs' average level. The most dangerous, difficult, and heroic encounters should award a treasure one, two, or three levels above the average PC level, respectively. If magic is rare in the game, halve these values. If the game is more epic, double these values.

\medskip

\textbf{Table: Treasury values per encounter} \index{Treasury} \index{Table of treasury values per encounter}

\begin{tabularx}{0.45\textwidth}{XX | XX}

\textbf{Medium Group Level} & \textbf{per Encounter (mo)} & \textbf{Medium Group Level} & \textbf{per Encounter (mo)} \\
\toprule
1             & 50        & 11            & 1500\\
2             & 100        & 12            & 2200\\
3             & 150        & 13            & 3000\\
4             & 200      & 14            & 3700\\
5             & 300        & 15            & 4500\\
6             & 400       & 16            & 6000\\
7             & 550       & 17            & 8000\\
8             & 750       & 18            & 11000\\
9             & 1000       & 19            & 15000\\
10            & 1200       & 20            & 20000\\
\end{tabularx}
\bigskip{}

Use this table to get an indication of the value of the "monster" treasure.

A Double Treasure \index{Double Treasure} means that a group of example 7 APLs in a Medium match will find the equivalent of approximately 1000 gp.

An accidental treasure indicates that there may be treasures, objects, coins only "accidentally" obtained, perhaps by killing other creatures. Stated differently, the creature does not collect the treasures, but it may still have accumulated something.

I suggest valuing an accidental treasure a quarter of a normal treasure. An accidental treasure lends itself very well to providing particular objects that can give directions or provide new adventure plots (what was this forged vial in adamantium for?) ...

\begin{changemargin}{0.3cm}{0.3cm} \begin{Storytellere}
How to distribute a treasure is an important matter. Treasures are not to be slammed in the face, much less hidden that they cannot be found.

A tip is to make sure that the treasures (and coins) found in the dungeons are distributed according to this criterion:

\smallskip
- a third will have monsters on them

\smallskip
- a third will be hidden behind secret passages or traps

\smallskip
- a third will be scattered around

\smallskip

This will inspire players to continue exploring, confront monsters, and actively search the dungeon.
\end{Storytellere} \end{changemargin}

\medskip

\textbf{Table: Character Wealth by Level} \index{Character Wealth by Level Table}

\bigskip

\begin{tabular}{ll|ll}
\textbf{Level} & \textbf{Wealth} & \textbf{Level} & \textbf{Wealth} \\
\toprule
1 & 100 & 11 & 40000 \\
2 & 250 & 12 & 60000 \\
3 & 500 & 13 & 90000 \\
4 & 1000 & 14 & 120000 \\
5 & 2000 & 15 & 150000 \\
6 & 3000 & 16 & 180000 \\
7 & 5000 & 17 & 220000 \\
8 & 9000 & 18 & 270000 \\
9 & 15000 & 19 & 350000 \\
10 & 25000 & 20 & 450000 \
\end{tabular}

\medskip

Encounters against NPCs usually reward three times more treasure than a monster, thanks to the NPC's Equipment. To compensate, make sure the characters go through a couple of extra encounters that award little in the way of treasure.

Animals, Plants, Constructs, Unintelligent Undead, Oozes and Traps are excellent "encounters with little treasure". Alternatively, if the characters face a number of creatures with little or no treasure, they should have the opportunity to obtain a number of items of more significant value in the near future to compensate for the imbalance. As a general rule, characters should not own any magical items worth more than half the character's total wealth, so check carefully before rewarding characters with very expensive items.

\subsubsection{Building a Loot} \index{Building a Loot}

It is often enough to tell your players that they found 5,000 gp in gems and 10,000 gp in jewels. But sometimes it is more interesting to give details. Giving a treasure a personality can not only help the likelihood of the game, but can sometimes trigger new adventures.

The information on the following pages can help you determine types of treasures randomly: many of the items have been given values, but you can assign them as you see fit. It is easier to place the more expensive items first: if you want you can also determine the magic items at random using the tables in Magic Items, to determine which items are present in the treasure.

Once you have consumed a sizable portion of the treasure's value, the remainder can simply consist of scattered coins and non-magical items with values defined according to your needs.

\textbf{Coins}: Coins in a hoard can be copper, silver, gold and platinum - silver and gold are the most common, but you can decide otherwise. For the coins and their exchange value go to the Equipment.

\textbf{Gems}: While you can assign any value to a gem, some may be worth more than others. Use the value categories below (and associated gemstones) as a guideline when assigning values to gemstones.

\textbf{Low Quality Gems} (10 gp): agate; azurite; blue quartz; hematite; lapis lazuli; malachite; obsidian; rhodochrosite; eye of the Tiger; turquoise; river pearl (irregular).

\textbf{Semi Precious Gems} (50 gp): heliotrope, carnelian; chalcedony; chrysoprase; citrine; jasper; lunaria; onyx; chrysolite; rock crystal (clear quartz); sardonic; sardonyx; rose quartz, smoky or star rose; zircon.

\textbf{Medium Quality Gemstones} (100 gp): amber; amethyst; chrysoberyl; coral; red or green-brown garnet; jade; jet; white, golden, pink or silver pearl; spinel red, brown-red or dark green; tourmaline.

\textbf{High Quality Precious Stones} (500 gp): Alexandrite; aquamarine; purple garnet; black Pearl; dark blue spinel; golden yellow topaz.

\textbf{Jewels} (1000 gp): emerald; white, black, or fire opal; blue sapphire; fiery yellow or vermilion corundum; blue or black star sapphire.

\textbf{Exceptional Jewels} (5000 gp or more): Crystalline brilliant green emerald, diamond, hyacinth, ruby, crystalline honey rat.

\textbf{Non-Magical Treasures} This category includes jewelry, fine clothing, wares, alchemical items, perfect items, and others.

Unlike gems, many of these objects have set values, but you can always increase the value of the object by decorating it with precious stones or particularly artistic invoices.

This cost increase grants no additional abilities: a gem-embellished Cold Iron scimitar worth 40,000 gp functions as a normal 330 gp Cold Iron scimitar. Below you will find numerous examples of non-magical treasures, with typical values.

\textbf{Refined Works of Art} (100 gp or more): Although some works of art are composed of precious materials, the value of most paintings, sculptures, literary works, fine clothing, and the like is workmanship with which they are made and in the skill of those who made them. Art objects are often bulky or difficult to move, and fragile, making recovery and transport an adventure in itself.

\textbf{Minor Jewelry} (50 gp): This category includes jewelry made from materials such as brass, bronze, copper, ivory, or exotic woods, sometimes embellished with very small or defective low-quality gemstones. Minor jewelry includes rings, bracelets and earrings.

\textbf{Normal Jewelry} (100-500 gp): Most jewelry is made of silver, gold, jade, or coral, and often decorated with semi-precious gemstones or medium-quality gemstones. Normal jewelry includes all types of minor jewelry plus bracelets, necklaces and brooches.

\textbf{Precious Jewelry} (500 gp or more): Precious jewelry is made of gold, mithral, platinum, or similar rare metals. Such items include ordinary types of jewelry plus scepters, pendants, and other large items.

\textbf{Well Made Tools} (100-300 gp): This category includes tools for Professions or skill: see Equipment for details and costs of these items.

\medskip

\begin{center}

\includegraphics[width = 0.9 \linewidth]{immagini/Hoxne_Hoard_1.png}

\textit{Reproduction of Hoxne's treasure}

\end{center}

\medskip

\begin{changemargin}{0.3cm}{0.3cm} \begin{Storytellere}
Never go overboard with treasures, especially magical ones. A treasure does not have to become a habit even more if it is something special and particular. One account can be coins, gems and "consumables" one account is the real treasures, the magical, special, unique ones.

Respecting the Law of the Prize does not mean lining the pockets of the characters, otherwise they will get bored risking their lives for new treasures and objects. When you do find a magical object, always think in perspective. It is true that it can be nice to see players happy with what they have found but then you will be forced the next adventure to give something even more powerful.

\end{Storytellere} \end{changemargin}

\medskip

\textbf{Common Items} (up to 1000 gp): There are many valuable items of an alchemical or common nature that can be used as treasure. Most alchemical items are wearable and reputable items, but others such as locks, sacred symbols, spyglasses, fine wines, or fine clothing can also make up interesting parts of a treasure. Commercial goods can also serve as a treasure: 5 kg of saffron, for example, is worth 150 gp.

\textbf{Treasure Maps and Information Objects} (variables): Objects such as treasure maps, ship and house deeds, lists of informants or guards, passwords, and the like can be fun. objects to find in a treasure: you can determine the value of these objects as you wish and they can be of double use as they can generate ideas for new adventures.

\textbf{Magic Items}

Of course, the discovery of a Magic Item is the real reward for any adventurer. Be careful about placing Magic Items in treasure - it is much more satisfying for many players to find a magical item than to buy it.

While you should generally place the items with careful consideration of their likely effects on your campaign, it can be fun to spawn the magical items in a random treasure chest. Be careful, however! It is easy, with a little luck (or bad luck) of the dice, to inflate your game with too much treasure or deprive it of it. The placement of random magical items should always be tempered by the Storyteller's common sense.

Spells are also real treasures and prizes like magic items. Carefully consider which ones can be found. Remember that a magical ability is not a copiable spell, only those present in tomes, scrolls and anything else specially created to be a spell receptacle is eligible for copying.



\subsection{Acting} \index{Acting}

\label{recitare}

A role-playing game is not a simple roll of the dice, it is a meeting of thoughts, opinions, challenges, struggles. It is a cathartic, liberating, evolutionary and instructive game.

It is right that there is combat, struggle, blood fear and action, in the same way there must be the possibility to play your characters with their disadvantages, advantages, powers and stories and even personal dramas.

The player must always impersonate the character, identify with and actively participate.

There may be side situations, handled quickly, which are done in the third person, yet whenever it becomes necessary to play then it must be true, done by the player by fully immersing himself in the character.

\medskip

\textbf{When a player interprets well and describes the action he is going to carry out in a participatory, engaging, inspired way, give him a reward, grant a +1 bonus to the action he is carrying out}

\medskip

Let the player know that thanks to his interpretation he has that bonus.

At the same time, there may be situations that turn out to be unpleasant for some players to handle and play. Be very careful in this case, going against the sensitivity of a player, of a friend, is not like going against the ethics or morals of a character. If you feel a sense of discomfort and embarrassment stop the game immediately and clarify the situation with the players and resume only when you have agreed on how to change the situation to prevent it from happening again.

\subsection{About OBSS and dice rolls} \index{About OBSS and dice rolls}

OBSS uses a peculiar dice-rolling system by mixing a 3d6 distribution with the potential of the exploding 6s. This system manages to guarantee a good variance and even if concentrating the results around the central values of the distribution, it leaves the upper limit open to particularly lucky shots.

If you want to have fun studying the corresponding curve I recommend the site \href{https://anydice.com/}{Anydice}. This is the pseudo code to insert:

{\small function: explode ROLLEDVALUE: n \{

if ROLLEDVALUE = 6 \{ result: 6 +[explode d6] \}

if ROLLEDVALUE = 1 \{ result: 0 \}

result: ROLLEDVALUE \}

3d output[explode d6]}

or click \href{https://anydice.com/program/2610e}{qui} for the code already entered.

\subsection{Adventures in OBSS} \hypertarget{OSR}{} \index{OSR} \index{Adventure in OBSS}

I suggest reading the article in full: \href{https://lithyscaphe.blogspot.com/p/principia-apocrypha.html} {Principia Apocrypha}

https://lithyscaphe.blogspot.com/p/principia-apocrypha.html the following is a summary I adapted and modified of the guidelines I follow when I burn OBSS.

OBSS follows the principles of \href{https://it.wikipedia.org/wiki/Old_School_Renaissance}{OSR} (wikipedia). Adventures in OBSS aim to be lethal, have a freely navigable world, a sketchy storyline, push problem-solving, and have a reward system focused on exploration, treasure and group participation. OBSS doesn't care too much about match balance and appreciates players' resourcefulness and captures their ideas by putting them into the adventure.

For me, the OSR is not tables of random encounters and chaotic randomization nor a specific regulation, it is rather the spirit of adventure, wonder, fear, glory, awe and challenge that develops in adventures. Don't be too linear, too predictable, add the right mix in adventures that make them always unique.

If you may not like the method, use what you like best, personally over the decades I have learned to appreciate and see appreciated the spontaneity and naturalness that the OSR hinges bring to the game.

\bigskip

\textbf{These are basic Storyteller rules that I suggest for conducting adventures.} \index{Guidelines for Storytellers} \index{OSR Principles}

\medskip

\begin{itemize}

\item
You are the Storyteller, the Rules yours, the World yours.

Do not be limited by the adventure, the system, the list of monsters, always feel free to modify and adapt according to the needs of the adventure and the group

\item
Remember to be fair and correct. Improvise, adapt as much as you want but be consistent. If you make a rule (or a change to a rule) follow it to the end.

At the same time, if you need a rule and can't find it, use common sense, it's definitely the right choice at that moment.

Respect the dice and the results obtained, as will happen to the players, particular results will happen to you too. Rightly so.

\item
You don't have to save the characters' ass. You are not their friend nor their enemy. Your role is to tell stories that arise from the stories of the characters, from their actions and inactions.

\item
Sketch the story, write the central parts or to read to the players but don't let yourself be dominated or constrained by what you expect. Often and willingly the players will amaze you, better to know where they move and what they have around in order to always react promptly.

It is the players who lead the adventure and you who unravel it.

\item
Appreciate chance and create different situations where players can choose different paths or weave new ones. It is your luck to have creative players who know how to surprise you.

\item
Don't force anyone to do something, let the players make mistakes, let them pay for their choices. You don't have to hinder them nor do you have to feed them in one direction. It requires a considerable amount of imagination and adaptability on your part, but adventure will surely benefit you.

\item
Characters are explorers, by definition. Focus on exploration, the more you explore the more situations you create, the more you create hooks in the adventure, the more you know other pngs the more there are areas to explore.

It makes us understand that treasures are experience, in a literal and practical sense. You never have to push them into a dungeon but their lust for experience and treasure.

\item
Have the players solve the problems and not the characters. Let the scenes roll, they're always better than a dice roll. Encourage the player to interact and ask for proof only as a last resort. You propose problems that do not necessarily have to be solved with a die roll but rather through multiple actions, even complex ones.

Rewards creative actions and courageous choices, first of all intelligence and the desire to find alternative and creative situations.

\item
Make the players ask you for information, confront the environment and each other. Encourage interaction with the outside world and only as a last resort allow a die roll.

\item
It is natural that the characters know what the players know, try to limit this exchange, for the benefit of everyone's skills and abilities.

\item
Big challenges and risks always give big rewards. Do not disappoint the players (except for the purpose of adventure) by denying them the right treasure or experience, the deeper they go deeper the more lethal the dangers will be the greater the reward (Reward Law).

\item
There must be no habits, customs. Don't create a standard.
Always try to surprise players with out of place (but make sense) monsters, anomalous traps, alternate environments. Different situations will stimulate players to solve each problem differently.

Prepare different solutions and accept different solutions. Put in the adventure problems and situations that together allow the solution, each room must not be an aseptic environment but contain clues and solutions for other problems even without a direct solution.

\item
Accept death. If a fight is always lethal, don't be afraid to injure or kill the characters. Make them think, study the enemy, understand which is the best approach; and finally amaze them. Characters must first outwit their enemies by cunning and planning if they are to survive.
If you protect the characters the game will lack tension and the players will solve all the problems with brute force.
Dungeons don't have to be environments to be emptied of monsters. The purpose of monsters is to limit and guide actions, to consume options.

If the players are always looking for a head-on fight, then give it to them, as required.

\item
Keep your attention high. It makes sure that the passage of time has consequences, if the players fear the passage of time they will make more daring or perhaps wrong choices. You keep the tension between the desire to explore and loot and the terror of staying still for too long.

\item
You are the source of information, the players process it, the characters use it.

Do not hide information that the characters need to know or already know, you will not have to be a professor but in the same way make sure that they are aware of what is around them.
At the same time you don't have to reveal everything right away, have them investigate, snoop around. Like an onion, the information they get will be hidden under layers of other, perhaps less important, information.

\item
Clues create situations. Let your specific and curious clues grab the attention of the players. Like a bait on a hook it attracts players into situations of doubt, where to investigate and understand what happens.

Do not fill the adventure with unnecessary details, leave room for the creativity and imagination of the players, however the details you provide must not only make sense but be necessary for the adventure.

\item
If players tend to forget the useful information given, try to take advantage of an NPC who has memory or invite them to take notes, there is no harm in being prepared.

\item
The adventure is never static nor the world where the characters move.
The world is as important if not more important than the adventure itself. Player actions can trigger global events. Always think about the consequences of gestures.

\item
If you use NPCs (non-player characters) do not make them be mere specks, make sure that the characters can become attached and consider the NPC one of the group on a par with all the others.

\item
Monsters don't have to be stupid. Make them talk, reason, run away .. they too want to live!

\item
Remember the Law of the Prize. Reward the daring, rewards who go deeper into caves, rewards those who survive.

\end{itemize}

\end{multicols}

\vfill

\begin{center}
\includegraphics[keepaspectratio, width = 0.50 \textwidth]{immagini/dungeonsample.png}

\textit{Detail of a dungeon}
\end{center}

\pagebreak

\section{Creating Magic Items} \index{Creating Magic Items}

\begin{changemargin}{0.3cm}{0.3cm} \begin{enfasi}{
To create is to live twice. (Albert Camus)
} \end{enfasi} \end{changemargin} \medskip

\begin{multicols}{2}

\label{creare-oggetti-magici}

\lettrine[lines = 2, lhang = 0.33, loversize = 0.25, findent = 1.5em]{F}{or} Creating Magic Items you must have the Magic Item Crafting Skills.

The costs listed here are those of production, the revenue can be at least around 20 \% of the production price.

Knowing the spell (or having it available via Scroll) that applies to the item is a requirement of any magical item you create.

\subsubsection{Creating Magic Rings} \index{Magic Rings}

To create a magic ring, a character needs a heat source. It also needs a supply of materials, of which the most obvious is a ring or pieces of ring to be assembled. The cost of materials is included in the cost of creating the ring.

The production cost of the ring is equal to level * level * 4000, a Ring with Invisibility costs 2 * 2 * 4000 = 16000 gp

\begin{center}
\includegraphics[width = 0.5 \linewidth]{immagini/onering2.png}

\textit{It goes without saying that Ring is ...}
\end{center}

A ring allows you to set a spell to make the effect always active.
The ring must have an intrinsic value equal to at least 500th * level of the spell it is to host.

A ring can accommodate a level 9 spell or if multiple spells the maximum level is 7.

It is also possible to insert an activation spell, in this case consult the costs of the Rods.

Forging a ring takes 1 day for every 1000 gp of the base price. In the event of multiple spells, the costs and times add up.

\medskip

\textbf{Item crafting feat required}: Create greater magic items.

\subsection{Create Magical Armor and Shields} \index{Create Magical Armor}

To create a magical armor or shield, a character needs a heat source and some tools to work iron, wood, or leather. It also needs a supply of materials, of which the most obvious is the armor / shield itself or the pieces of armor to assemble. An armor / shield that is to be enchanted must be of quality.

\begin{center}
\includegraphics[width = 0.5 \linewidth]{immagini/Rustning_Gustav_Vasa.png}

\textit{Armor for Gustav I of Sweden by Kunz Lochner, c. 1540 (Livrustkammaren)}
\end{center}

If the prerequisites for creating armor include spells, the caster must know those spells.

The production cost of +1 magical armor costs 2050 gp, +2 7500 gp, +3 12000 gp, +4 25000 gp, +5 45000 gp plus the price of the armor itself.

Casting a spell into an armor costs the same as creating a ring with that spell.

Crafting magical armor / shields takes one day for every 1000 gp of the base price value.

\medskip

\textbf{Item crafting feat required}: Craft magic items.

\subsection{Create Magical Weapons} \index{Create Magical Weapons}

To create a magical weapon, a character needs a heat source and some tools to work the iron or material the weapon is made of. It also needs a supply of materials, of which the most obvious is the weapon itself or the weapon pieces to be assembled. Only a quality weapon can be enchanted to become a magical weapon, and its cost must be added to the total enchantment cost to determine the final market value.

A magical weapon must have at least a +1 bonus to have any special ability or spell.

\medskip

\begin{center}
\includegraphics[width = 0.6 \linewidth]{immagini/exacaliburfuori.png}

\textit{The drawing of the sword from the stone, Henrietta Elizabeth Marshall's Our Island Story (1906)}
\end{center}

\medskip

If the prerequisites for crafting the weapon include spells, the caster must know those spells.

At the moment of creation, the caster must decide whether the weapon emanates light or not, as a byproduct of the magic infused into the weapon. This decision does not affect the price or creation time, but once the item is completed, the decision is final.

Crafting dual weapons is considered analogous to crafting two weapons in terms of cost, time, and special abilities.

The production cost of a +1 weapon is 1200 gp, +2 4000 gp, +3 11000 gp, +4 25000 gp, +5 45000 gp plus the price of the weapon (only relevant if it is made of some rare or valuable material ).

The production cost of an Arrow +1 is 20 gp, +2 75 gp, +3 325 gp. More powerful enchantments are extremely rare.

To instill a spell in a weapon has a cost as if you were going to create a ring with that spell, if continuous, otherwise if single use as a potion.

Crafting a magical weapon takes one day for every 1000 gp worth of the base price.

\medskip

\textbf{Item crafting feat required}: Craft Magic Items.

\subsection{Create Wands} \index{Create Wands}

Wand production cost is level * level * 800, a wand with invisibility costs 2 * 2 * 800 = 4200 gp

A wand is a magical object that holds a previously loaded spell within itself.

To recharge a wand, a spellcaster must cast the same spell and have the Craft Magic Item skill. The wand recovers a charge but the caster in addition to using Magic Points spends the equivalent of 100 * level gold coins on components.

A wand can contain a maximum spell level of 5.

To create a wand, a character needs a supply of materials, of which the most obvious is a wand or pieces of a wand to assemble. The wands are always fully charged (20 charges) at the act of creation.

The caster must know the spell he inserts into the wand.

\begin{center}
\includegraphics[width = 0.5 \linewidth]{immagini/wand.png}
\end{center}

Making a wand takes 1 day for every 1000 gp of the base price value.

\medskip

\textbf{Item crafting feat required}: Craft magic items.

\subsection{Create Staff} \index{Create Staff}

\textbf{Basic Club Costs}

\bigskip

The production cost of the Staff is equal to level * level * 1200, a Staff with Invisibility costs 2 * 2 * 1200 = 4800 gp

\bigskip

A Staff is a magical item where one or more spells are charged.

When a staff is activated it is possible to use one spell at a time.

To make a stick, a character needs a supply of materials, of which the most obvious is a stick or pieces of a stick to assemble.

The Staff are always fully charged, 10 charged, at the act of creation.

\begin{center}
\includegraphics[width = 0.3 \linewidth]{immagini/staff2.png}
\end{center}

A staff can contain a maximum spell level of 8, or in the case of several spells the maximum level is 6.

Making a staff takes 1 day for every 1000 gp of the base price.

\medskip

\textbf{Item crafting feat required}: Create greater magic items.

	\subsection{Create Scrolls} \index{Create Scrolls} \index{Scrolls} \index{Isy Scroll}
\index{Easy Scrolls} \index{Buying Spells}
\medskip

There are two types of magical scrolls, those that can be performed by everyone (called ISY SCROLL, or Easy) and those that require the magical ability to cast spells, or Magical Proficiency greater than or equal to 1.

Easy scrolls have a production cost of level * level * 160 gp.

Normal, not easy scrolls have a production cost of level * level * 80 gp.

The cost of the scroll should be evaluated based on the rarity and level of the spell. A very rare or high-level spell can easily cost you level * level * level * 80.

\begin{center}
\includegraphics[width = 0.4 \linewidth]{immagini/scroll3.png}
\end{center}

If a scroll includes multiple spells, the cost is equal to the sum of the various spells. There can be no normal scroll spells on an ISY SCROLL scroll and vice versa.

The caster must know the spells he places on the scroll. Preparing a scroll requires 30 minutes of work per current spell level.

An ISY scroll can contain spells of level 3 as a maximum, while a normal scroll can contain a maximum spell level 9, in case of more spells the maximum level is 8.

\medskip

To read a parchment you need: \\

\textbf{in case of ISY SCROLL scrolls}:

- to understand the content, an Intelligence (or Arcana if known) check at difficulty DC 10 is sufficient

- to be able to read and cast the spell of the parchment, an Intelligence (or Arcana if known) check at difficulty 12 is required. \\

\medskip

\textbf{in case of normal scrolls}: \\

- An Arcana check at difficulty 15 is required to understand its contents

- to be able to read and cast the spell of the parchment you need an Arcana check at difficulty 11 + Level of the spell and have access to the Magic List of the spell contained and that this is of a level equal to the maximum castable +1.

\medskip

The \textbf{casting time} of a spell from a scroll is equal to the casting time of the present spell.

\textbf{Required Item Crafting Talent}: Craft Magic Items

Skill used in creation: Arcana or Profession (scribe).

\textbf{Note}: A Tome of Magic is equivalent to a set of regular scrolls. A character in a desperate situation can read the spell page from the Tome of Magic and manifest the magic as though it were from a scroll. The pages containing the spell will be pulverized and the caster will have to find a source from which to copy the spell back to Tome. \index{Tome of Magic as scroll}

\begin{center}
\includegraphics[width = 0.6 \linewidth]{immagini/potion2.png}

\textit{A witch, raising her arm above a flaming cauldron, recites a spell; a young woman kneels in front of the cauldron. Mezzotint by J. Dixon after JH Mortimer, 1773}
\end{center}

\subsection{Making Potions} \index{Making Potions} \index{Potions}

A potion contains the infusion of only one spell, each potion is therefore disposable.

\medskip

The production cost of the Potion is equal to level * level * 80, a Potion with Invisibility costs 2 * 2 * 80 = 320 gp

\bigskip

To create a potion, a character needs a horizontal work surface and some containers for mixing liquids, as well as a heat source to boil the brew.

A Potion can contain a maximum spell level of 3.

All ingredients and materials for brewing a potion must be fresh and never used.

The caster must know the spell he inserts into the potion. The brewing time of a potion is 1 hour per spell level.

\medskip

\textbf{Required Item Crafting Talent}: Brew Potions.

\subsection{Create Rods} \index{Create Rods} \index{Rods}

A rod is a special wand that is capable of regenerating its charges. They are precious and very expensive items.

To create a rod, a character needs a supply of materials, of which the most obvious is a rod or pieces of a rod to assemble.

\medskip

Rod production cost is level * level * 3200, a rod with invisibility costs 2 * 2 * 3200 = 12800 gp

\bigskip

A rod is able to cast its spell 1 time per day.

Multiply the cost by 4 if he is able to cast it 2 times, multiply by 8 if he is able to cast it 3 times a day.

You can also cast the spell contained in the rod one more time a day, after which the rod is destroyed.

A rod can have a maximum spell level of 3.

The caster must know the spell he inserts into the rod.

Making a rod takes 1 day for every 1000 gp of the base price.

\textbf{Item crafting feat required}: Create greater magic items.

\subsection{Add New Skills} \index{Add New Skills}

Sometimes, lack of funds or time makes it impossible to craft the desired magic item, but luckily it is possible to upgrade or modify a created magical item. Only time, gold, and the various prerequisites required by the new ability you want to add to the magic item place restrictions on the kind of additional powers one can infuse.

The cost of adding additional abilities to an item is the same as if the item were not magical, minus the value of the original item. Thus, a +1 longsword can become a +2 vorpal longsword, and the cost of creation is equal to that of a +2 vorpal longsword minus the cost of a +1 longsword.

There are many factors to consider when determining the price of an invented magical item. The easiest way to decide the price is to compare the new item to an item that already has a price, and use that price as a guide.

\end{multicols}

\vfill

\begin{center}
	\includegraphics[width = 0.2 \linewidth]{immagini/Rod_of_asclepius.png}

	\textit{The staff of Asclepius is an ancient Greek symbol associated with medicine. It consists of a snake coiled around a rod.}
\end{center}


\pagebreak

\section{Magic Item Rules} \index{Magic Item Rules} \hypertarget{identificareom}{}

\begin{multicols}{2}

\lettrine[lines = 2, lhang = 0.33, loversize = 0.25, findent = 1.5em]{T}{hese} are indications on the use of magical items.

\label{oggetti-magici}
\begin{itemize}
\item
A character may \textbf{carry numerous (up to 12) magic items} on him but no more than 2 items (eg 1 magic ring and a bracelet) can be added to determine the Defense bonus. Armor and Shield are not counted in this count.
\item
The same principle applies to the bonus to \textbf{Saving Throws}, you can only add the bonuses from two items together.
\item
If the bonus is at \textbf{Features} only the one with the higher bonus counts.
\item
A character \textbf{cannot carry more than two magic rings} otherwise they resonate causing 1d6 damage (not reducible or magically curable) per round for each ring beyond one second.
\item
The Identify spell is required for \textbf{recognize a magical object} and its abilities. DC 25 1 Minute. Arcana scoring 6 costs 3 rounds, 12 costs 1 round, 18 costs 1 Action.
\item
A magical item that \textbf{manifests spells} makes no Magic Check. The saving throw it imposes, if not specified, is equal to 10 + level * 2 of the spell it manifests. \index{saving throw for object spells}
\item
\textbf{Activating Spell-Like Abilities}: Unless otherwise noted, activating an object's magical ability costs 2 Actions.
\item
A magic item that grants a \textbf{static bonus (or malus)} applies its value even if the item has not been identified, the Storyteller will silently apply this bonus to Defense, Attack Rolls, Saving Throws .. informing the player that he perceives how the object interacts with the situation.

\end{itemize}

\subsubsection{Arms}

\textbf{Weapons}: A weapon with a special ability must have at least a +1 bonus. Weapons cannot have the same special ability more than once.

The magic bonus of a \textbf{weapon can be identified} following two crit attacks on an attack roll or dedicating 1 hour of training, any talents or magical abilities remain concealed.

\subsubsection{Armor and Shields}

\textbf{Weapons}: An armor or shield with a special ability must have at least a +1 bonus. Armor and Shields cannot have the same special ability more than once.

\textbf{Armor}: Each +2 magic lowers the penalty on Malus skill by 1 and the penalty on Magic Checks by one die.

\textbf{Shields}: Each +2 magic removes a die on the Magic Check

\textbf{The cost of Weapons and Armor:} larger than Average is at least double (or quadruple depending on size). Small armor or Small weapons while requiring less material cost the same amount as medium weapons and armor.

\subsection{Size and Magic Items}

\label{taglia-e-oggetti-magici}

When a piece of clothing or a magical piece of jewelry is discovered, most of the time the size is not a problem: many magical clothes are easy to use for everyone or magically adapt to the wearer. As a rule, the size should not prevent characters of various physical types from using a magical item.

There may be rare exceptions, especially with items made for a specific race.

Weapons and armor found randomly have a 30 \% chance of being Small (01--30), 60 \% of being Medium (31--90), and 10 \% of being be another size.

\subsection{Magic Objects on the Body} \index{Magic Objects on the Body}

\label{oggetti-magici-sul-corpo}

Many magical items must be worn by a character who wishes to use them or benefit from their abilities. Up to 12 magical items can be worn at a time for a humanoid-shaped creature. Each of these items must be worn over a specific part of the body called a "slot".

A humanoid-shaped body can carry magical equipment consisting of one item of each of the following groups, tied to the part of the body on which the item is worn.

\textbf{Ring} (two at most): rings.

\textbf{Vesti}: breastplates, armor, tunics and robes

\textbf{Belt}: belts.

\textbf{Neck}: amulets, necklaces, medallions, scarabs, brooches, talismans and scarves

\textbf{Hands}: gauntlets and gauntlets.

\textbf{Eyes}: eyes, glasses and lenses.

\textbf{Feet}: shoes, boots and slippers.

\textbf{Wrist}: bracelets and bracelets.

\textbf{Shield}: shields.

\textbf{Shoulders}: hoods and cloaks.

\textbf{Head}: hats, diadems, helmets, masks, crowns, bands and phylacteries

\textbf{Chest}: shirts, jackets, sweaters and cloaks.

\medskip

Of course, a character can own as many items of the same type he wants. But additional magic items of the same type, in addition to those provided in the slots, will not work.

\subsection{Saving Throws Against Magical Item Powers} \index{Saving Throws}

\label{tiri-salvezza-contro-i-poteri-degli-oggetti-magici}

Magic items normally reproduce spells or other magical effects. For a saving throw against magic or a magical effect generated by a magical object, the DC is 10 + level of the manifested spell x2 unless otherwise noted.

\subsection{Damaging Magic Objects} \index{Damaging Magic Objects}

\label{danneggiare-gli-oggetti-magici}

A magical object does not need to make a saving throw unless it is unattended, the specific target of the effect, or its owner rolls a natural 3 on its saving throw.

Magic items are always entitled to a saving throw against something that could damage them, even when a normal item of the same type would have no chance of making a saving throw. Magic items always use the same bonus on saving throws, regardless of type (Fortitude, Reflex, or Will). A magic item's bonus on saving throws is equal to 2 + 2x the level of the most powerful spell it hosts (or +4 for each +1 they have). The only exceptions to this rule are intelligent magic items, which make Will saving throws based on their Wisdom score.

\subsection{Repair Magic Objects} \index{Repair Magic Objects}
\label{riparare-gli-oggetti-magici}

To repair a magical item, it takes materials and time, equal to half the time and cost to create it.

\subsection{Fills, Doses and Multiple Uses} \index{Fills} \index{Doses} \index{Multiple Uses}

\label{cariche-dosi-e-usi-multipli}

Many items, especially wands and staffs, have power limited by the number of charges they contain. Normally, items with charges never exceed the maximum of 20 charges (10 for Staff). If similar items are found as a random part of a treasure, roll a 5d6 and divide by 2 to determine the number of stacks left (rounding down, minimum 1). If an object has a maximum number of charges other than 20, it is rolled randomly to see how many charges are left.

The prices shown refer to items at their maximum charges (when an item is created, it always has the maximum charge). The value of an object depends on the number of residual charges, in the case of objects that can have a use even with few or without charges, the value remains higher.

\end{multicols}

\subsection{Acquire Magic Items} \index{Acquire Magic Items} \index{Acquire Magic Items Table}

\label{acquisire-oggetti-magici}

\bigskip

\begin{tabular}{lllll}
\textbf{Community Size} & \textbf{Base Value} & \textbf{Common} & \textbf{Uncommon} & \textbf{Rare} \\
\toprule
Settlement & 50th & 1d2 items && \\
Borgo & 200mo & 1d4 items && \\
Village & 500th & 1d6 items & 1d2 items & \\
Small town & 1000th & 1d4 items & 1d2 items & \\
Big Country & 2000th & 1d6 items & 1d4 items & 1d2 items \\
Small town & 4000th & 2d4 items & 1d6 items & 1d4 items \\
Big city & 8000th & 3d4 items & 2d4 items & 1d6 items \\
Metropolis & 16000mo &{*} & 3d4 items & 2d4 items \\
\end{tabular}

{*} Almost all minor magical items can be found in a metropolis.

\begin{multicols}{2}

\bigskip

Magic items are valuable, and most large cities have at least one or two suppliers of magical items, from a simple potion vendor to a blacksmith who specializes in forging magical swords. Of course, not every item in this manual is available in every city.

The following guidelines help Storytellers determine which items are available in a specific community. They assume a campaign with a medium level of magic. Some cities may deviate greatly from this baseline at the Storyteller's discretion. The Storyteller should keep a list of items available from each merchant and should occasionally replenish stocks with new acquisitions.

The number and types of magical items available in a community depend on its size. Each community has a base value linked to it (see Table: Available Magic Items).

there is a 75 \% chance that any item of that value or less can easily be found for sale in that community. Additionally, the community has a number of other items for sale. These objects are determined at random and are divided into categories (minor, medium or major).

After determining the number of items available in each category, consult the Random Generation of Magic Items chapter to determine the type of each item (potion, scroll, ring, weapon, etc.) before moving on to the specific tables to determine the item. exact. Re-roll whenever items don't fit the community base value.

If the use of magic in the campaign in which you play is rare, you need to halve the base value and the number of items in each community. In campaigns with extremely rare or no magic magic there may be no magic items for sale at all. Storytellers conducting this type of campaign should include changes to the challenges faced by characters due to the lack of magical items.

Campaigns with abundant magical items may have communities with double the established base value and random magical items available. Alternatively, all communities could be determined to be of a larger category in order to determine the magical items available. In a campaign with very common magic, all magical items can be purchased in a metropolis.

Nonmagical items and tools are typically available in a community of any size unless the item is very expensive, such as full armor, or made of an unusual material, such as an adamantium longsword. These items should follow the base value guideline for determining their availability, at the Storyteller's discretion.

\end{multicols}

\vfill

\begin{center}
\includegraphics[keepaspectratio, width = 0.90 \textwidth]{immagini/mappaparigi.png}

\textit{Ancient map of Paris}
\end{center}

\pagebreak


\section{Random Generation of Magic Objects} \index{Random Generation of Magic Objects}

\begin{changemargin}{0.3cm}{0.3cm} \begin{enfasi}{
Like any unrequited love, even that for things in the long run yes pay. (Adolfo Bioy Casares)
}
\end{enfasi} \end{changemargin} \medskip

\begin{multicols}{2}

The Storyteller in the preparation of the adventure can place the magical objects he prefers, if there is a need, or in pure OSR style rely on a random generation.

This approach is not always suggested, the results could upset the adventure if not the whole campaign! Usually if an "enemy" has a magical item there is a reason and this item will have a purpose. The fact remains that every now and then, rolling dice on the causal generation tables for magical treasures is very satisfying and fun!

First of all it is necessary to establish what type of object will be generated.

\medskip

\textbf{Table: Magic Item Type} \index{Magic Item Type Table}

\medskip

\begin{tabular}{lc}
\textbf{Type of magic item} & \textbf{1d20} \\
Magic Weapons & 1-3 \\
Armor and Shields & 4-6 \\
Amulets, Necklaces, Jewelry & 7-8 \\
Belts, Helmets, Boots and Gloves & 9-10 \\
Wand, Staff and Rods & 11 \\
Potions, Filters and Oils & 12-14 \\
Rings & 15 \\
Hats, Cloaks, Glasses, Tunics & 16-17 \\
Manuals and Tomes & 18 \\
Miscellaneous Magic Items & 19-20 \\
\end{tabular}

\subsubsection{Arms}

\textbf{Table: Weapons Generation} \index{Weapons Generation Table}

\medskip

\begin{tabularx}{0.45\textwidth}{lX}
\textbf{1d100} & \textbf{Magic Bonus} \\
1-50 & +1 \\
51-65 & -1 Cursed \\
66-72 & +2 \\
73-76 & +3 \\
77-79 & +4 \\
80 & +5 \\
81-87 & re-roll + Type 3 Weapons Special Ability \\
88-91 & re-roll + Type 2 Weapons Special Ability \\
92-94 & re-roll + Type 1 Weapons Special Ability \\
95-100 & -2 Cursed \\
\end{tabularx}

\medskip

When in \textit{Magic Bonus} it says \textit{reroll + Weapon Type Special Ability ...} it means that you must reroll the 1d100, ignoring other results above 80 and keep the magic bonus obtained, then you can roll on \textit{Weapon Type Special Ability Table ...} resulting.

\medskip

\textbf{Table: Weapon Type 1 Special Ability} \index{Weapon Type 1 Special Ability Table}

\medskip

\begin{tabular}{ll}
\textbf{1d100} & \textbf{Weapon Type 1 Special Ability} \\
1-8 & Accumulate Spells \\
9-16 & Anathema \\
17-21 & Dancing \\
22-27 & Defensive \\
28-34 & Destroyer of the Giants \\
35-41 & Destruction \\
42-47 & Light Energy \\
48-54 & Gloriosa \\
55-60 & Guardian \\
61-63 & Fortunata \\
65-70 & Thief of the Nine Lives \\
71-73 & Sacra \\
74-80 & Ghost Touch \\
81-86 & Vampira \\
87-92 & Speed \\
93-99 & Cursed Weapon \\
100 & Vorpal \\
\end{tabular}

\medskip

\begin{center}
\includegraphics[width = 0.55 \linewidth]{immagini/armatura-med.png}
\end{center}

\medskip

\textbf{Table: Type 2 Weapon Special Ability} \index{Weapon Type 2 Special Ability Table}

\medskip

\begin{tabular}{ll}
\textbf{1d100} & \textbf{Weapon Type 2 Special Ability} \\
1-8 & Conductive \\
9-16 & Courageous \\
17-23 & Cruel \\
24-30 & Duel \\
31-36 & Innate Fury \\
37-43 & Vital Impulse \\
44-60 & Immoral \\
60-61 & Lethal \\
62-67 & Perfida \\
67-72 & Pietosa \\
73-79 & Punitive \\
80-85 & Contemptuous \\
87-95 & Terror \\
95-100 & Titanica \\
\end{tabular}

\bigskip

\begin{center}
\includegraphics[width = 0.7 \linewidth]{immagini/shield1.png}
\end{center}

\textbf{Table: Type 3 Weapon Special Ability} \index{Weapon Type 3 Special Ability Table}

\medskip

\begin{tabular}{ll}
\textbf{1d100} & \textbf{Weapon Type 3 Special Ability} \\
1-4 & Adaptive \\
5-8 & Sharp \\
9-12 & Dragon Slayer \\
13-16 & Giant Killer \\
17-20 & Hunter \\
21-24 & Corrosive \\
25-28 & Designant \\
29-32 & Distance \\
33-36 & Extinguish Fire \\
37-40 & Fanaticizing \\
41-44 & Injury \\
45-48 & Dazzling \\
49-52 & Frosty \\
53-56 & Fiery \\
57-60 & Marina \\
61-65 & Masking \\
65-68 & Phantom Ammo \\
69-72 & Infinite Ammo \\
73-76 & Planar \\
77-80 & Prhensile \\
81-84 & Researcher \\
85-88 & Returning \\
89-92 & Tonante \\
93-96 & Transformant \\
97-99 & Trovacos \\
100 & Cursed Weapon \\
\end{tabular}

\medskip

\subsubsection{Armor and Shields}

\textbf{Table: Armor / Shields Generation} \index{Armor / Shields Generation Table}

\medskip

\begin{tabularx}{0.45\textwidth}{lX}
\textbf{1d100} & \textbf{Magic Bonus} \\
1-50 & +1 \\
51-65 & -1 Cursed \\
66-72 & +2 \\
73-76 & +3 \\
77-79 & +4 \\
80 & +5 \\
81-85 & re-roll + Armor / Shields Special Ability Type 2 \\
86-90 & re-roll + Armor / Shields Special Ability Type 1 \\
91-100 & -2 Cursed \\
\end{tabularx}

\medskip

When in \textit{Magic Bonus} it says \textit{reroll + Armor / Shield Type Special Ability ...} it means that you must reroll 1d100, ignoring other results above 80 and keep the magic bonus gained, then you can roll on the resulting \textit{Armor / Shield Type ... Special Ability Table}.

\textbf{Table: Armor / Shields Type 1 Special Ability} \index{Armor / Shields Type 1 Special Ability Table}

\begin{tabularx}{0.45\textwidth}{lX}
\textbf{1d100} & \textbf{Armor / Shields Special Ability Type 1} \\
1-5 & Aries \\
6-10 & Balanced \\
11-15 & Archer's Bracers \\
16-20 & Bracers of Defense \\
21-25 & Bracers of Major Defense \\
26-30 & Brilliant \\
31-35 & Determination \\
36-40 & Spell Defense \\
41-45 & Elegant \\
46-50 & Hospitable \\
51-55 & Poison Resistance \\
56-60 & Energy Resistance \\
61-65 & Greater Energy Resistance \\
66-70 & Wild \\
71-75 & Dragon Scales \\
76-80 & Animated Shield \\
81-85 & Bullet Attraction Shield \\
86-90 & Breath of the Dragon \\
91-95 & Ghost Touch \\
95-100 & Armor / Cursed Shield \\
\end{tabularx}

\medskip

\textbf{Table: Type 2 Armor / Shields Special Ability} \index{Armor / Shields Type 2 Special Ability Table}

\begin{tabularx}{0.45\textwidth}{lX}
\textbf{1d100} & \textbf{Armor / Shields Special Ability Type 2} \\
1-5 & Blinding \\
6-10 & Adamantium \\
11-15 & Amorphous \\
16-20 & Antihemorrhagic \\
21-25 & Wrangler \\
26-30 & Load \\
31-35 & Demonic Armor \\
36-40 & Denegant \\
41-45 & Sweatshirt \\
46-50 & Ethereal Form \\
51-55 & Invulnerability \\
56-60 & Untraceable \\
61-65 & Masking \\
66-70 & Mithral \\
71-75 & Shadow \\
76-80 & Perceptive \\
81-85 & Titanica \\
86-90 & Vulnerability \\
91-100 & Armor / Cursed Shield \\
\end{tabularx}

\medskip

When the cursed special ability is indicated, you must re-roll and reverse the weapon's magic bonuses, so a +2 armor or shield becomes a -2 armor or shield.

\subsubsection{Amulets, Necklaces and Jewels} \index{Amulets, Necklaces and Jewels Generation Table}


\begin{tabular}{ll} \\
\textbf{Object Type} & \textbf{1d8} \\
Amulets, Necklaces and Jewelery Type 1 & 1-6 \\
Amulets, Necklaces and Jewelery Type 2 & 7-8 \\
\end{tabular}

\medskip

\begin{tabularx}{0.45\textwidth}{lX}
\textbf{1d100} & \textbf{Amulets, Necklaces and Jewelry Type 1} \\
1-8 & Poison Amulet \\
8-12 & Gangrene Amulet \\
12-18 & Healing Amulet \\
19-26 & Amulet Against Possession \\
27-34 & Amulet of Inevitable Location \\
35 & Amulet of the Planes \\
36-42 & Detection and Location Protection Amulet \\
42-46 & Amulet of Physical Endurance \\
47-53 & Blast Headband \\
53-60 & Adaptation Series \\
61-70 & Strangulation Necklace \\
71-77 & Fireball Necklace \\
78-83 & Rosary Necklace \\
84-90 & Death Beetle \\
91-100 & Protection Beetle \\
\end{tabularx}

\medskip

\begin{tabularx}{0.45\textwidth}{lX}
\textbf{1d100} & \textbf{Amulets, Necklaces and Jewelry Type 1} \\
1-7 & Elemental Gem \\
8-13 & Gem of Brightness \\
9-16 & Sight Gem \\
17-26 & Monster Attracting Jewel \\
27-33 & Medallion of Thoughts \\
34-41 & Medallion of the Feather Fall \\
42-49 & Pearl of Wisdom \\
50-57 & Defense Pin \\
58-60 & Talisman of Pure Good \\
61-62 & Talisman of Extreme Evil \\
63-70 & Poison Protection Charm \\
71-78 & Health Charm \\
79-85 & Talisman of the Sphere \\
86-100 & Jewel without value
\end{tabularx}


\begin{center}
	\includegraphics[width = 0.8 \linewidth]{immagini/gauntlet.png} \\
\end{center}


\subsubsection{Belts, Helmets, Boots and Gloves} \index{Generation Table of Belts, Helmets, Boots and Gloves}

\begin{tabularx}{0.45\textwidth}{lX}
\textbf{1d100} & \textbf{Belts, Helmets, Boots and Gloves} \\
1-3 & Giants Belt \\
3-6 & Belt of the Dwarves \\
6-11 & Helm of Understanding Languages \\
12 & Helm of Luster \\
13-17 & Helmet of the Underwater Movement \\
18-22 & Helm of Telepathy \\
23-26 & Teleportation Helm \\
27-31 & Bullet Grab Gloves \\
31-35 & Gloves of Orc Power \\
36-41 & Swimming and Climbing Gloves \\
41-46 & Gloves of Dexterity \\
47-52 & Clumsy Gloves \\
53-58 & Spider Slippers \\
59-63 & Winged Boots \\
64-66 & Boots of Running and Jumping \\
67-77 & Elf Boots \\
78-83 & Winter Boots \\
84-90 & Boots of Levitation \\
91-95 & Boots of Speed \\
96-100 & Dancing Boots \\
\end{tabularx}

\subsubsection{Wands, Staff and Rods} \index{Generation Table of Wands, Staff and Rods}

Roll 1d8 to determine if a Wand or Staff or Rod is found.

\medskip

\begin{tabular}{ll} \\
\textbf{Object Type} & \textbf{1d8} \\
Wand & 1-4 \\
Staff & 5-7 \\
Rods & 8 \\
\end{tabular}

\medskip

\textbf{Table: Wands Generation} \index{Wands Generation Table}

\medskip

\begin{tabularx}{0.45\textwidth}{lX}
\textbf{1d100} & \textbf{Wand} \\
1-5 & Metal Search Wand \\
6-10 & Magic Bolt Wand \\
11-15 & Wand of Conveniences \\
16-20 & Lightning Wand \\
21-25 & Fire Wand \\
26-30 & Ice Wand \\
31-35 & Individ wand. of the Magic \\
36-38 & Individ wand. of Enemies \\
39-44 & Wand of Illusions \\
45-48 & Wand of Detection of Secret Doors \\
46-50 & Wand of Light \\
51 & War Wizard's Wand \\
52 & Wand of Metamorphosis \\
53 & Wand of Wonders \\
54 & Wand of Denial \\
55-60 & Fireball Wand \\
61-65 & Wand of Paralysis \\
66-70 & Wand of Fear \\
71-75 & Wand Discover Traps \\
76-80 & Wand of Secrets \\
81-85 & Web Wand \\
86-90 & Binding Wand \\
91-95 & Assisted Escape Wand \\
96-100 & Cursed Wand \\
\end{tabularx}

\medskip

\textbf{Table: Stick Generation} \index{Stick Generation Table}

\medskip

\begin{tabularx}{0.45\textwidth}{lX}
\textbf{1d100} & \textbf{Staff} \\
62 & Staff of the Archmage \\
63-65 & Withering Staff \\
66-67 & Staff of the Woods \\
68-70 & Staff of Charm \\
71-72 & Staff of Striking \\
73-74 & Staff of Fire \\
75-76 & Staff of Frost \\
77-78 & Healing Staff \\
79-80 & Swarming Insect Staff \\
81-82 & Python's Staff \\
83 & Staff of Power \\
84-86 & Staff of Thunder and Lightning \\
87 & Staff of Sorcery \\

\end{tabularx}

\medskip

\textbf{Table: Rod Generation} \index{Rod Generation Table}

\medskip

\begin{tabularx}{0.45\textwidth}{lX}
\textbf{1d100} & \textbf{Rod} \\
1-10 & Rod of Charm \\
11-20 & Rod of Absorption \\
21-30 & Immovable Rod \\
31-41 & Rod of Mighty Blow \\
42-50 & Rod of Sovereign Strength \\
51-60 & Rod of Readiness \\
61-70 & Security Rod \\
71-80 & Rod of Sovereignty \\
81-90 & Sprawling Rod \\
91-100 & Cursed Rod \\
\end{tabularx}



\subsubsection{Potions, Filters and Oils} \index{Potions, Filters and Oils Generation Table}

\begin{tabular}{ll}
\textbf{Potion} & \textbf{1d8} \\
Potion Types 1 & 1-4 \\
Potion Types 2 & 5-7 \\
Potion Types 3 & 8 \\
\end{tabular}



\medskip

\begin{tabular}{ll}
\textbf{1d100} & \textbf{Potion Type 1} \\
1-8 & Climbing Potion \\
9-15 & Growth Potion \\
16-23 & Potion of Heroism \\
24-29 & Gaseous Form Potion \\
30-35 & Giant Strength Potion \\
36-46 & Healing Potion \\
47-53 & Potion of Deception \\
54-64 & Invisibility Potion \\
65-74 & Levitation Potion \\
77-78 & Potion of Fortitude \\
79-84 & Underwater Breathing Potion \\
84-90 & Shrink Potion \\
91-95 & Potion of Speed \\
96-100 & Flight Potion \\
\end{tabular}

\medskip

\begin{tabular}{ll}
\textbf{1d100} & \textbf{Potion Type 2} \\
1-10 & Animal Clair Audience Potion \\
11-20 & Animal Clairvoyance Potion \\
21-28 & Animal Control Potion \\
29-33 & Dragon Control Potion \\
34-38 & Undead Control Potion \\
39-49 & Potion of People Control \\
50-55 & Plant Control Potion \\
56-66 & Potion of Invulnerability \\
67-77 & Thought Reading Potion \\
78-85 & Poison Potion \\
86-95 & Major Healing Potion \\
96-100 & Greater Poison Potion \\
\end{tabular}

\medskip

\begin{tabular}{ll}
\textbf{1d100} & \textbf{Potion Type 3} \\
1-13 & Love Filter \\
14-27 & Discoverers Filter \\
28-40 & Sharpness Oil \\
41-53 & Ethereal Form Oil \\
54-66 & Slippery Oil \\
67-79 & Animal Friendship Potion \\
80-85 & Potion of Longevity \\
86-95 & Potion of Metamorphosis \\
96-100 & Greater Poison Potion \\
\end{tabular}

\medskip

\begin{center}
\includegraphics[width = 0.8 \linewidth]{immagini/cupdrinking.png} \\

\textit{Drinking cup depicting scenes from the Odyssey, Athens 550–525 BC}
\end{center}

\subsubsection{Rings} \index{Rings Generation Table}

\begin{tabular}{ll}
\textbf{Ring} & \textbf{3d6} \\
Ring Type 1 & 3-16 \\
Ring Type 2 & 17-18 \\
\end{tabular}

\medskip

\begin{tabular}{ll}
\textbf{1d100} & \textbf{Rings Type 1} \\
1-5 & Spell Accumulator Ring \\
6-13 & Ring of Aries \\
14-21 & Feather Fall Ring \\
22-28 & Water Walk Ring \\
29-35 & Ring of Heat \\
36-41 & Ring of Weakness \\
42-47 & Circumvention Ring \\
48-50 & Ring of Influence on Animals \\
51-55 & Ring of Deception \\
56-61 & Action Freedom Ring \\
61-67 & Swimming Ring \\
68-77 & Protection Ring \\
76-84 & Ring of Resistance \\
85-93 & Ring of the Leap \\
93-100 & Telekinesis Ring \\
\end{tabular}

\medskip

\begin{center}
	\includegraphics[width = 0.8 \linewidth]{immagini/romanring.png}
\end{center}

\begin{tabular}{ll}
\textbf{1d100} & \textbf{Rings Type 2} \\
1-8 & Ring of People Control \\
9-17 & Plant Control Ring \\
18-23 & Ring of Water Elementals \\
24-29 & Ring of Air Elementals \\
31-36 & Ring of Fire Elementals \\
37-42 & Ring of Earth Elementals \\
43-48 & Djinni Ring of Summoning \\
49-56 & Ring Spell Repel \\
57-65 & Invisibility Ring \\
66-75 & Regeneration Ring \\
76-83 & Mind Shield Ring \\
84-90 & Ring of Shooting Stars \\
91-92 & Ring of the Three Wishes \\
92-96 & Ring of the Three Wishes sold out \\
97-100 & X-Ray Sight Ring \\

\end{tabular}


\subsubsection{Hats, Cloaks, Glasses, Tunics} \index{Hats, Cloaks, Glasses, Tunics Generation Table}

\begin{tabularx}{0.45\textwidth}{lX}
\textbf{1d100} & \textbf{Hats, Cloaks, Glasses, Tunics} \\

1-3 & Bandana of Intelligence \\
4-10 & Camouflage Hat \\
11-17 & Arachnid Cloak \\
18-23 & Cloak of the Charlatan \\
24-29 & Cloak of Distortion \\
30-40 & Cloak of the Elves \\
41-45 & Manta Cloak \\
46-50 & Bat Cloak \\
51-57 & Cloak of Protection \\
58-62 & Cloak of Spell Resistance \\
63-68 & Cloak of Venominess \\
69-72 & Eyes of Petrification \\
73-75 & Fascinating Eyes \\
76-77 & Eagle Eyes \\
78-80 & Eyes of Detail View \\
80-82 & Night Glasses \\
83-86 & Tunic of Camouflage \\
87 & Tunic of the Archmage \\
88 & Tunic of Shimmering Colors \\
89-91 & Robe of Undermining \\
92-94 & Robe of Eyes \\
95-99 & Tunic of Useful Items \\
100 & Tunic of the Stars \\
\end{tabularx}


\subsubsection{Manuals and Tomes} \index{Manuals and Tomes Generation Table}

\begin{tabularx}{0.45\textwidth}{lX}
\textbf{1d100} & \textbf{Manuals and Tomes} \\
1-9 & Handbook of Golems \\
10-24 & Good Health Handbook \\
25-40 & Speed of Action Manual \\
40-54 & Exercise Manual \\
55-69 & Volume of Authority and Influence \\
70-84 & Tome of Understanding \\
85-100 & Tome of Clear Thought \\
\end{tabularx}

\subsubsection{Miscellaneous Magic Items} \index{Miscellaneous Magic Items Generation Table}

Roll 1d10 to determine if a rare or legendary magical item is found or from the miscellaneous magical item lists

\medskip

\begin{tabular}{ll}
\textbf{Object Type} & \textbf{1d12} \\
Various Magic Items 1 & 1-3 \\
Various Magic Items 2 & 4-5 \\
Various Magic Items 3 & 6-7 \\
Various Magic Items 4 & 8-9 \\
Various Magic Items 5 & 10-12 \\
Rare and Legendary & 10 \\
\end{tabular}

\medskip


\subsubsection{Rare and Legendary} \index{Rare and Legendary Generation Table}

\medskip

\begin{tabularx}{0.45\textwidth}{lX}
\textbf{1d100} & \textbf{Magic Object} \\
1-3 & Wings of Flight \\
4-6 & Iron Vial \\
7-10 & Elemental Water Amphora \\
11-12 & Apparatus of the Crab \\
13-15 & Folding Boat \\
17-20 & Type III Conservative Bag \\
21-24 & Type IV Preservative Bag \\
25-28 & Bean Bag \\
29-30 & Efreeti bottle \\
31 & Potions Jug \\
32-33 & Invocation Candle \\
34-35 & Horn of Valhalla \\
36-39 & Phylactery of youth \\
40-42 & Instant Fortress \\
43-45 & Deck of Wonders \\
46-49 & Miniature of Wonderful Power \\
50-53 & Kill Ammunition \\
54-58 & Crystal Ball \\
59-62 & Scroll Against Elementals \\
63-65 & Scroll Against the Undead \\
66-70 & Sewer Pipe \\
71-75 & Wonder Pigments \\
76-83 & Cubic Portal \\
84-85 & Well of the Many Worlds \\
86-87 & Mirror of Mental Skill \\
88-89 & Life Trapping Mirror \\
90-91 & Sphere of Annihilation \\
92-94 & Elemental Air Thurible \\
95-96 & Portable Compartment \\
97-98 & Hooves of Speed \\
99-100 & Zephyr's Hooves \\
\end{tabularx}

\medskip

\subsubsection{Miscellaneous Magic Items 1} \index{Miscellaneous Magic Items Generation Table 1}

\begin{tabularx}{0.45\textwidth}{lX}
\textbf{1d100} & \textbf{Miscellaneous magic items 1} \\
1-8 & Purifying water \\
9-17 & Battaglio dell'Apertura \\
18-27 & Type I Preservative Bag \\
28-34 & Climbing Rope \\
35-43 & Efficient Quiver \\
44-52 & Locating Arrow \\
53-60 & Lantern of Revelation \\
61-70 & Pearl of Power \\
71-80 & Stone of Good Fortune \\
81-83 & Universal Solvent \\
84-94 & Restorative Ointment \\
95-100 & Practical Backpack \\
\end{tabularx}

\subsubsection{Miscellaneous Magic Items 2} \index{Miscellaneous Magic Items Generation Table 2}


\begin{tabularx}{0.45\textwidth}{lX}
\textbf{1d100} & \textbf{Miscellaneous magic items 2} \\
1-8 & Fire Elementals Brazier \\
9-17 & Brazier of Cursed Sleep \\
18-27 & Cold protection cube \\
28-34 & Censer of Air Elementals \\
35-43 & Entanglement Net \\
44-52 & Trap Net \\
53-60 & Animated Attack Broom \\
61-70 & Flying Broom \\
71-80 & Cursed Flight Broom \\
81-88 & Mirror of Duplication \\
89-90 & Flying Carpet \\
99-100 & Titan Hoe \\
\end{tabularx}

\medskip
\subsubsection{Miscellaneous Magic Items 3} \index{Miscellaneous Magic Items Generation Table 3}

\begin{tabularx}{0.45\textwidth}{lX}
\textbf{1d100} & \textbf{Miscellaneous magic items 3} \\
1-8 & Cancellation Exchange \\
9-18 & Jug of Infinite Water \\
19-26 & Dimensional Logs \\
27-35 & Supreme Glue \\
36-40 & Meditation incense \\
41-51 & Protective Scroll Against Magic \\
52-60 & Revealing Dust \\
61-70 & Vanishing Dust \\
71-82 & Sneeze Powder \\
83-90 & Arcane Stone \\
91-96 & Weight Stone \\
97-100 & Arcane Fan \\
\end{tabularx}

\begin{center}
\includegraphics[width = 0.8 \linewidth]{immagini/ancientdrum.png}
\end{center}


\subsubsection{Miscellaneous Magic Items 4} \index{Miscellaneous Magic Items Generation Table 4}

\begin{tabularx}{0.45\textwidth}{lX}
\textbf{1d100} & \textbf{Miscellaneous magic items 4} \\
1-5 & Vial of Curses \\
6-10 & Battaglio of Cannibalism \\
11-16 & Type II Preservative Bag \\
17-20 & Devouring Bag \\
21-25 & Steaming Bottle \\
26-31 & Portable Hole \\
32-37 & Healthy Air Necklace \\
38-43 & Tangle Cord \\
44-48 & Choke Cord \\
49-50 & Horn of Destruction \\
51-52 & Force Cube \\
53-58 & Iron Bands of the Binding \\
59-64 & Phylactery against non-motion \\
65-69 & Obsession Incense \\
70-71 & Deck of Illusions \\
72-76 & Hypnotic Crystal Ball \\
77-82 & Parchment against werewolves \\
83-84 & Stone of Earth Elementals \\
85-89 & Fife of Fright \\
90-92 & Arcane Feather \\
93-94 & Dryness Dust \\
95-96 & Panic Drums \\
97-98 & Stunning Drums \\
99-100 & Cursed Summoning Thurible \\
\end{tabularx}


\begin{center}
\includegraphics[width = 0.6 \linewidth]{immagini/ancientbraziers2.png}

\textit{Teotihuacano Old God vessels: Top - stone brazier in Natural History Museum of Los Angeles County}
\end{center}




\pagebreak

\section{Description of Magic Items} \index{Description of Magic Items}


Magic items are presented in alphabetical order by grouping categories. The description of a magic item gives the item name, its category, rarity, and its magical properties.

Although the costs are reported it is always good to grant magical items as rewards, honey, following a mission.

Basically a Common item, the only one that could easily be found in a big city, can cost 50 to 100 gp, an Uncommon between 150 and 500 gp, a Rare between 500 and 5,000 gp, a Very Rare up to 50,000 gp.

Items with a bonus over +2, or Legendaries, are never bought, it must be an epic adventure to find them.



\medskip

Spells are also magical items and as such, if the Storyteller permits, they can be purchased (horror! There is nothing nicer to find a new spell among the treasures of an adventure).

A spell costs level * level * level * 80 in gold

\bigskip

\subsection{Special Abilities of Magical Weapons}

\lettrine[lines = 2, lhang = 0.33, loversize = 0.25, findent = 1.5em]{W}{eapon} with a special ability must have at least +1 magic bonus.

Here are listed the magical abilities that an armor, shield or weapon can have in addition to the generic magic bonus (+ 1, + 2 ....). Use this list as guidelines and examples, same for prices, use them as an indication of rarity.

\index[Magic Items]{Magic Weapons! Magic Weapons} \subsubsection*{Magic Weapon}

\textit{Weapon (Any)} +1 1,800 gp, +2 6,000 gp, +3 17,000 gp, +4 45,000 gp, +5 80,000 gp

You have a bonus on attack rolls and damage rolls made with this weapon. The bonus is determined by the rarity of the weapon. Some magical weapons have additional properties, such as emitting light.

\subsubsection*{Accumulate Spells} \index[Magic Items]{Magical Weapons! Accumulate Spells}

A Spell Accumulator weapon allows a spellcaster to store a spell targeting up to level 3 in the weapon. The spell must have a standard casting time of 2 Actions. Whenever the weapon hits a creature and the creature takes damage, the wielder of the weapon with an immediate action can release the spell.

Once the spell is cast, a spellcaster can store any other targeted spell within it, always up to level 3.

The weapon magically reveals to the wielder the name of the spell currently contained. A randomly created Spell Accumulator weapon has a 50 \% chance that it already has a spell contained within it. This special ability can only be added to melee weapons.

An accumulate spell weapon emits a strong aura of the invocation school, plus the aura of the spell it contains.

\textbf{Details}: Aura Strong and variable invocation; Requirements to craft greater magic items, cost +3000 gp.

\subsubsection*{Adaptive} \index[OggettiMagici]{Magic Weapons! Adapt}

This ability can only be added to composite arcs. An Adaptive bow reacts to the strength of the wielder, acting like a bow with a Strength bonus equal to that of the wielder. The wielder can fire with a lower Strength bonus (and cause less damage) if desired.

\textbf{Details}: Aura Weak Transmutation; Requirements for Crafting Magical Items, Animal and Plant List; Cost +1500 gp.

\subsubsection*{Sharp} \index[OggettiMagici]{Magic Weapons! Sharp}

This critical ability allows you to count the number of 6 shot and increase it by 1. Only slashing or piercing melee weapons can be sharpened.

\textbf{Details}: Aura Moderate Transmutation; Requirements for Creating Greater Magical Items, Earth List; Cost +5,000 gp.

\index[Magic Items]{Magic Weapons! Dragon Slayer} \subsubsection*{Dragon Slayer}

When you hit a dragon with this weapon, the dragon takes an additional 3d6 points of damage of the weapon type. For the purposes of this weapon, "dragon" is any creature of the dragon type.

\textbf{Details}: Aura Moderate invocation; Requirements for Creating Greater Magical Items; Cost +8,000 gp.

\index[Magic Items]{Magic Weapons! Giant Slayer} \subsubsection*{Giant Slayer}

When you hit a giant with this weapon, the giant takes an additional 2d6 points of damage of the weapon type and must succeed on a DC 18 Fortitude save or fall prone. For the purposes of this weapon, "giant" is any creature of the giant type.

\textbf{Details}: Aura Moderate invocation; Requirements for Creating Greater Magical Items; Cost +8,000 gp.

\subsubsection*{Destroyer of the Giants} \index[OggettiMagici]{Magic Weapons! Destroyer of the Giants}

You must wear a \textit{belt of giants} (any variety) and \textit{gloves of orc power} to use this weapon.

While using the hammer your Strength score increases by 2 (to a maximum of 7).

When you get a critical attack roll made with this weapon against a giant, the giant must succeed on a DC 21 Fortitude save or die.

You can spend 1 charge and make a ranged weapon attack by throwing it as if it had a range of 20 feet. If the attack hits, the hammer produces audible thunder from up to 90 meters away. The target and all creatures within 30 feet of it must succeed on a DC 21 Fortitude save or be stunned until the end of your next round.

The hammer has 5 charges, and recovers 1 spent charge every day at dawn.

\subsubsection*{Anathema} \index[OggettiMagici]{Magic Weapons! Anathema}

Anathema weapon excels at attacking certain creatures. Against the favored enemy, his effective bonus becomes +2. The weapon also deals an additional + 2d6 damage against that enemy. To randomly determine the weapon's chosen enemy, the following table is used:

\medskip

\begin{tabular}{ll}
d \% & Favored Enemy \\
01-05 & Aberrations \\
06-09 & Beasts \\
10-16 & Constructs \\
17-22 & Draghi \\
23-27 & Fairies \\
28-60 & Humanoids (choose subtype) \\
61-70 & Magical Creatures \\
71-72 & Slime \\
73-88 & Fiend \\
89-90 & Plants \\
91-98 & Undead \\
99-100 & Insects \\
\end{tabular}

\medskip

\textbf{Details}: Aura Moderate Summon; Requirements for Creating Greater Magical Items, Abjuration List; Cost +3,000 gp.

\subsubsection*{Hunter} \index[Magic Items]{Magic Weapons! Hunter}

A Hunter's weapon helps the wielder locate and capture prey. When the weapon is held, the wielder gains the weapon's bonus on Survival checks made to track any creatures the weapon has damaged over the previous day. Deals + 1d6 damage to creatures tracked with Survival by the wielder over the previous day.

\textbf{Details}: Aura Moderate divination; Requirements to Craft Superior Magical Items, Locate Animals and Plants; Cost +3,000 gp.

\subsubsection*{Conductive} \index[OggettiMagici]{Magic Weapons! Conductive}

A Conductive weapon is capable of channeling the energy of a magical ability that requires a melee or ranged touch attack to hit its target.

When the wielder successfully makes an attack of the appropriate type, he may choose to spend two uses of his spell-like ability to channel it through the weapon, in order to strike the opponent, who suffers the effects of the weapon's attack and those. special ability (such as channeling energy, laying on of hands ...).

This weapon special ability can only be used once per round (even if it has multiple conductive weapons).

\textbf{Details}: Aura Moderate necromancy; Requirements to Craft Greater Magic Items, Magic Hand; Cost +3,000 gp.

\subsubsection*{Courageous} \index[Magical Objects]{Magical Weapons! Courageous}

This special ability can only be added to a melee weapon. A Brave weapon fortifies the wearer's courage and morale in battle. The wielder gains a bonus on saving throws versus fear equal to the weapon's bonus.

\textbf{Details}: Aura Weak enchantment; Requirements for Crafting Magical Items, Heroism, Fear; Cost +3,000 gp.

\subsubsection*{Corrosive} \index[OggettiMagici]{Magic Weapons! Corrosive}

On command, a Corrosive weapon becomes coated in a layer of acid that deals an additional 1d6 points of acid damage when it hits its target. The acid does not harm the wielder. The effect lasts until a new command is given.

\textbf{Details}: Aura Moderate invocation; Requirements to Craft Magic Items, Acid Arrow; Cost +3,000 gp.

\subsubsection*{Cruel} \index[OggettiMagici]{Magic Weapons! Cruel}

A Cruel weapon feeds on fear and suffering. When the wielder hits a Frightened creature with a cruel weapon, the weapon becomes sickened for 1 round. When the wielder uses the weapon to unconscious or kill a creature, she gains 5 temporary hit points that last for 10 minutes.

\textbf{Details}: Aura Faint Necromancy; Crafting Magical Item Requirements, Fear, Cost +3000 gp.

\subsubsection*{Dancing} \index[OggettiMagici]{Magic Weapons! Dancing}

As a standard action, a dancing weapon can be set free to fight alone. The weapon fights for 4 rounds using the Defense of the one who released it and then falls to the ground.

It always remains next to the person who freed it, even if it moves by physical or magical means. If the one who released it has a free hand he can retake the weapon he is attacking alone, as an immediate action, but once taken back, the sword will no longer be able to dance (attack alone) for 4 rounds.

This ability can only be added to melee weapons.

\textbf{Details}: Aura Strong Transmutation; Requirements for Creating Greater Magical Objects, Animating Objects; Cost +25,000 gp.

\subsubsection*{Designant} \index[OggettiMagici]{Magic Weapons! Designante}

This special ability can only be added to ranged weapons or ammo. Whenever a ranged weapon or ammo with this ability hits a creature, it magically designates the target. All allies gain a +2 bonus on attack rolls for 1 round. Multiple hits on the same target do not increase their bonuses or their duration.

\textbf{Details}: Aura Moderate enchantment; Requirements for Creating Higher Magical Objects, Light; Cost +6,000 gp.

\subsubsection*{Defensive} \index[Magic Items]{Magic Weapons! Defensive}

A Defensive weapon allows the wielder to transfer part or all of the weapon's bonus to their Defense as a cumulative bonus with any other bonuses. As an immediate action, the wielder can choose how to have the weapon bonus at the start of the round, before using it, and the Defense bonus lasts until the next round.

\textbf{Details}: Aura moderate abjuration; Requirements to Craft Greater Magic Items, Shield; Cost +3,000 gp.

\subsubsection*{Distance} \index[OggettiMagici]{Magic Weapons! Distance}

This special ability can only be added to projectiles. A Ranged bullet has twice the range given by the weapon it fires.

\textbf{Details}: Aura Moderate divination; Requirements for Crafting Magical Items, Clairvoyance; Cost +3,000 gp.

\subsubsection*{Destruction} \index[OggettiMagici]{Magic Weapons! Destruction}

A weapon of Destruction is the bane of all undead. Each undead creature hit in combat must succeed on a DC 14 Will save or be destroyed or take an additional 2d8 points of Light damage. A weapon of Destruction must be a melee shot weapon.

\textbf{Details}: Aura Strong Summon; Requirements for Creating Greater Magic Items, Healing; Cost +6,000 gp.

\subsubsection*{Duel} \index[OggettiMagici]{Magic Weapons! Duel}

This ability can only be granted to a melee weapon. A Duel weapon (which must be a weapon that can be used with the Accurate Weapon feat) grants the wielder a + 1d6 bonus on Initiative checks, as long as the weapon has been drawn and wielded when the Initiative.

\textbf{Details}: Aura Weak Transmutation; Requirements for Crafting Magical Items, Animal and Plant List; Cost +7,000 gp.

\subsubsection*{Light Energy} \index[OggettiMagici]{Magic Weapons! Light Energy}

This object looks like the hilt of a long sword, but without a blade. When you grab its handle, you can use two actions to cause a blade of pure luminescence to form, or make the blade inserted into the handle disappear.

As long as the sword exists, this magical longsword has the Versatile property. If you are proficient with short swords or long swords, you are also proficient with the sun blade.

You gain a +2 bonus on attack and damage rolls made with this weapon, which deals Light damage rather than slashing damage. When you hit an undead creature with it, the target takes an additional 1d8 points of Light damage.

The sword's luminous blade emits intense light within a 4.5 meter radius and dim light for an additional 4.5 meters. Light is sunlight. As long as the blade is active, you can use two actions to expand or reduce the radius of bright and dim light by 5 feet each, to a maximum of 30 feet or a minimum of 10 feet each.

\textbf{Details}: Aura Strong Transmutation; Requirements to Craft Wondrous Magical Items, Everlasting Flame, Sunburst; Cost +45,000 gp.

\subsubsection*{Extinguish Fire} \index[Magic Items]{Magic Weapons! Extinguish Fire}

This special ability can only be added to melee weapons. An Extinguish Fire weapon is capable of extinguishing a Medium or smaller non-magical fire. When used against a Fire creature, it deals an additional 1d6 points of damage. The wielder of an extinguish-fire weapon gains a +2 competence bonus on saving throws against fire-based effects, and the weapon itself is immune to fire damage.

\textbf{Details}: Aura Weak Transmutation; Requirements for Crafting Magical Items, List of Water; Cost +3,000 gp.

\subsubsection*{Fanatizing} \index[OggettiMagici]{Magic Weapons! Fanaticizing}

Cursed Weapon. This ability grants a +2 bonus to attacks, however, at the start of the battle, it causes the wearer to be seized with irrepressible anger. The character will attack the closest creature, enemy or friend, until there are none left within 60 yards.

\subsubsection*{Wounding} \index[Magic Items]{Magic Weapons! Wound}

This ability can only be added to melee weapons. A Wounding weapon deals 1 Bleed damage when it hits a creature. Multiple damage from this weapon increases Bleed damage up to a maximum of 5.
Bleeding creatures take bleed damage at the start of their round.

The bleeding can be stopped with a DC 15 First Aid check or with any spell that heals hit point damage. A critical hit does not multiply the bleed damage. Creatures immune to critical hits are immune to bleed damage dealt by this weapon.

\textbf{Details}: Moderate necromantic aura; Requirements for Creating Greater Magic Items, Contagion; Cost +6,000 gp.

\subsubsection*{Shocking} \index[OggettiMagici]{Magic Weapons! Shocking}

On command, a Shock weapon is engulfed in crackling electricity that deals an additional 1d6 points of electricity damage per hit. This electricity does not harm the wielder of the weapon. The effect is always active as long as the weapon is unsheathed.

\textbf{Details}: Aura Moderate invocation; Requirements to Craft Greater Magic Items, Lightning Bolt; Cost +3,000 gp.

\subsubsection*{Innate Fury} \index[OggettiMagici]{Magic Weapons! Innate Fury}

This special ability can only be added to melee weapons. A weapon of Innate Fury draws power from the anger and frustration the wielder feels when battling enemies who refuse to die. Whenever the wielder deals damage to an opponent with the weapon, his bonus increases by +1 when making attacks against that enemy (up to a maximum total bonus of +5). This added bonus wears off if the opponent dies, or if the wielder uses the weapon to attack a different creature, misses on the attack roll or 1 hour passes.

\textbf{Details}: Aura Moderate enchantment; Requirements for Creating Greater Magical Items, Heroism; Cost +4,000 gp.

\subsubsection*{Fortunata} \index[OggettiMagici]{Magic Weapons! Fortunata}

As long as you are wearing the sword, you also receive a +1 bonus on saving throws.

- \textit{Luck}. If you are wearing a sword, you can rely on its luck (requires no action) to re-roll an attack roll, ability check, or saving throw that doesn't satisfy you. You are forced to use the second result of the die. This property cannot be used again until the next sunrise.

- \textit{Desire}. While wielding it, you can use two actions to spend 1 charge and cast the wish spell with it. This property cannot be used again until the next sunrise. The sword has 1d4-1 stacks, and loses this property if it runs out of stacks.

\textbf{Details}: Aura Very strong invocation; Requirements for Crafting Mythic Magical Items, Wish; Cost +30000 gp.

\subsubsection*{Frosty} \index[Magical Objects]{Magical Weapons! Frosty}

On command, a Frosty weapon is engulfed in a terrible chill that deals 1d6 points of cold damage per hit. This cold does not harm the wielder. The effect is always active as long as the weapon is unsheathed.

\textbf{Details}: Aura Moderate invocation; Requirements for Creating Greater Magic Items, List of Water; Cost +3,000 gp.

\subsubsection*{Gloriosa} \index[Magic Items]{Magic Weapons! Gloriosa}

A Glorious weapon illuminates a Daylight spell with a dazzling light when drawn. The wielder cannot suppress this light, although it can be temporarily suppressed by any effect that can suppress Daylight.

When a Glorious weapon makes a Critical Hit, the target is Blinded until the start of the owner's next round (Fortitude DC 14 negates). Only a melee weapon can have the Glorious ability.

\textbf{Details}: Aura Moderate invocation; Requirements for Crafting Magical Items, Blindness / Deafness, Daylight; Cost +6,000 gp.

\subsubsection*{Guardian} \index[OggettiMagici]{Magic Weapons! Guardian}

This ability can only be added to melee weapons. A Guardian weapon allows the wielder to transfer part or all of the weapon's bonus to his saving throws as a stacking bonus with all others. As an immediate action, the wielder chooses how to distribute the weapon bonus at the start of his or her round before using the weapon. The bonus on all saving throws lasts until its next round. Only the weapon's own bonus can be sacrificed, no other bonuses from other effects can be used.

If a weapon has both Defensive and Guardian abilities, sacrificing a single point of the bonus improves Defense or saving throws, but not both.

\textbf{Details}: Aura moderate abjuration; Requirements to Craft Greater Magical Items, Resistance; Cost +3,000 gp.

\subsubsection*{Immoral} \index[OggettiMagici]{Magic Weapons! Immoral}

This ability can only be added to melee weapons. When an Immoral weapon strikes an opponent, it produces a bolt of Void that echoes between the wielder and his target. The energy deals an additional 2d6 damage to the opponent and 1d6 damage to the wielder.

\textbf{Details}: Aura Moderate invocation; Requirements to Craft Greater Magical Items, Debilitation; Cost +3,000 gp.

\subsubsection*{Life Impulse} \index[Magical Objects]{Magical Weapons! Life Impulse}

This special ability can only be added to melee weapons. A Vital Impulse weapon increases and sustains the wielder's life energy while in the midst of combat. The wielder gains a bonus on saving throws against necromancy effects (including ability damage, ability drain, and maximum hit point reductions due to undead powers) equal to the weapon's bonus. Additionally, whenever the wielder gains temporary hit points from any source, he adds the weapon's bonus to it.

\textbf{Details}: Aura Moderate Summon; Requirements to Craft Greater Magic Items, Cure Serious Wounds, Superior Restoration; Cost +6,000 gp.

\subsubsection*{Fiery} \index[Magic Items]{Magic Weapons! Fiery}

On command, a fiery weapon is engulfed in flames that inflict 1d6 points of fire damage per hit. This fire does not harm the wielder. The effect remains active until it is deactivated with another command.

\textbf{Details}: Aura Moderate invocation; Requirements to Craft Greater Magic Items, Fireball; Cost +3,000 gp.

\index[Magic Items]{Magic Weapons! Thief of the Nine Lives} \subsubsection*{Thief of the Nine Lives}

You get a +2 bonus on attack and damage rolls made with this magical weapon. If you get a critical hit against a creature that has fewer than 100 hit points, it must succeed on a DC 17 Fortitude save or be immediately killed, while the sword drains its life force from the body (constructs and undead are immune. to this property).

The sword has 1d8 + 1 stacks, and loses 1 stacks when a creature is killed. When the sword is no longer charged, it loses this property.

\textbf{Details}: Strong Necromantic Aura; Requirements to Craft Greater Magic Items, Fireball; Cost +25,000 gp.

\subsubsection*{Lethal} \index[OggettiMagici]{Magic Weapons! Lethal}

This special ability can only be added to melee weapons that normally deal nonlethal, stun damage. All damage from a lethal weapon is normal (lethal). On command, immediate action, the weapon suppresses this ability until the wielder orders him to reactivate it.

\textbf{Details}: Aura Faint Necromancy; Requirements to Craft Greater Magic Items, Cure Light Wounds (inverted); Cost +3,000 gp.



\subsubsection*{Marina} \index[OggettiMagici]{Magic Weapons! Marina}

This special ability can only be added to melee weapons. A marine weapon works calmly in aquatic environments. With the weapon in hand, the wielder gains a bonus on Swim checks equal to double the weapon's bonus.

In addition, the wielder does not suffer the normal penalties for attack rolls and for damage from being underwater, as if he were subject to a freedom of movement spell.

\textbf{Details}: Aura Moderate necromancy; Requirements to craft greater magic items, freedom of movement, cost +3000 gp.

\subsubsection*{Masking} \index[OggettiMagici]{Magic Weapons! Masking}

A weapon of disguise can be commanded to change its shape and appear like another object of similar size. The weapon retains all its properties (including weight) even when disguised, but it does not radiate magic. Only true seeing or other similar spells reveal the true nature of the transformed weapon. After a weapon of disguise is used to attack, this special ability is suppressed for 1 minute.

\textbf{Details}: Aura moderate illusion; Requirements to Craft Superior Magical Items, Magical Weapon, Disguise Self; Cost +2000 gp.

\subsubsection*{Phantom Ammo} \index[Magical Items]{Magical Weapons! Phantom Ammo}

This ability can only be granted to ammunition. An ammo with this weapon special ability dissolves 1 round after being thrown. In addition, if the Bullet hits a target, the wound it inflicts re-closes as the ammunition disintegrates. The Bullet deals damage normally, but leaves no visible trace of violence.

Price is for 50 Phantom Ammo.

\textbf{Details}: Aura Moderate Transmutation; Requirements to Create Higher Magical Objects, Disintegrate, Repair; Cost +1000.

\subsubsection*{Infinite Ammo} \index[OggettiMagici]{Magic Weapons! Infinite Ammo}

Only bows and crossbows can be made weapons from Infinite Ammo. Whenever a weapon with Infinite Ammo is nocked, a single nonmagical arrow or bolt is spontaneously created by its magic, so the wielder never needs to load the weapon with ammo.

If the wielder attempts to load the weapon with more ammunition, the arrow or bolt created vanishes immediately and the weapon can be loaded as normal. This ability does not reduce the amount of time it takes to load or fire the weapon. The arrow or bolt created vanishes if removed from the weapon; it persists only if thrown. Unlike regular bow or crossbow ammunition, these arrows and bolts are always destroyed when thrown.

\textbf{Details}: Aura Moderate Summon; Requirements for Creating Higher Magical Items, Creation; Cost +6,000 gp.

\index[OggettiMagici]{Magic Weapons! Wicked} \subsubsection*{Wicked}

When you roll a 17 or 18 on the attack roll with this magical weapon, the target takes an additional 7 damage of the weapon type.

\textbf{Details}: Aura Weak Summoning; Requirements for Crafting Magical Items, Causes Light Injuries; Cost +3,000 gp.

\subsubsection*{Pietosa} \index[OggettiMagici]{Magic Weapons! Pietosa}

All damage dealt by the weapon is temporary.

On command, the weapon suppresses this ability until ordered to reactivate it (allowing it to deal lethal damage).

\textbf{Details}: Aura Weak Summoning; Requirements for Crafting Magical Items, Cure Light Wounds; Cost +3,000 gp.

\subsubsection*{Planar} \index[OggettiMagici]{Magic Weapons! Planar}

A Planar weapon is effective against all types of extradimensional beings, being able to overcome their resistance to physical damage. When used to attack Outsiders, a Planar weapon ignores 5 points of their Damage Reduction or Resistances.

\textbf{Details}: Aura Moderate Summon; Requirements for Creating Greater Magical Items, Planar Shift; Cost +3,000 gp.

\subsubsection*{Prhensile} \index[OggettiMagici]{Magic Weapons! Prhensile}

This ability can only be granted to whips. A gripping whip can, as a move action, cling to an object as if it were a grapple. The Whip can then be used to climb surfaces or swing across a room or any outdoor area.

\textbf{Details}: Aura Moderate enchantment; Requirements to Craft Greater Magic Items, Rope Trick; Cost +2.500.

\index[Magic Items]{Magic Weapons! Mace of Retribution} \subsubsection*{Mace of Retribution}

You get an additional +3 to hit and damage when using this weapon to attack a construct.

When you gain a critical attack roll with this weapon, the target takes an additional 7 hit damage, or an additional 14 hit damage if it's a construct. If, after taking this damage, a construct has 25 or fewer hit points left, it is destroyed.

\textbf{Details}: Aura Strong invocation; Requirements for Creating Greater Magical Items; Cost +7,000 gp.

\subsubsection*{Seeker} \index[Magical Objects]{Magical Weapons! Seeker}

This ability can only be added to ranged weapons. A Seeker weapon turns towards its target, negating any chance of missing it that might apply, such as that due to Concealment. The wielder must still aim the weapon in the correct square. Arrows shot by mistake in an empty space, for example, do not veer to hit Invisible opponents, if anyone is nearby.

\textbf{Details}: Aura Strong Divination; Requirements for Creating Higher Magical Objects, True Seeing; Cost +3,000 gp.

\subsubsection*{Returning} \index[OggettiMagici]{Magic Weapons! Returning}

A Returning weapon can teleport into its owner's hands as an immediate action, even if it is in possession of another creature. This ability has a maximum range of 30 meters, and effects that block teleportation prevent the return of a Returning weapon. A Returning weapon must be in possession of a creature for at least 24 hours for this ability to work.

\textbf{Details}: Aura Moderate Summon; Requirements to Craft Greater Magic Items, Teleportation; Cost +3,000 gp.

\subsubsection*{Sacred} \index[OggettiMagici]{Magic Weapons! Sacred}

You get a +3 bonus on attack and damage rolls made with this magical weapon. When you hit a fiend or undead with it, that creature takes an additional 2d10 points of Light damage.

As you wield the unsheathed sword, it creates a 10-foot radius aura around you. You and all friendly creatures within the aura gain + 1d6 on saving throws against spells and other magical effects generated by followers or devotees of other Patrons. If you have Traits in common with Patron 13 or more, the aura radius increases to 30 feet.

\textbf{Details}: Aura Moderate invocation; Common traits 12; Requirements for Creating Greater Magical Items; Cost +6,000 gp.



\subsubsection*{Contemptuous} \index[Magical Items]{Magical Weapons! Contempt}

This special ability can only be added to melee weapons. A contemptuous weapon helps its wielder survive in desperate conditions. It remains in the hands of the wielder even if the latter is Frightened, Stunned or Unconscious. The wielder adds his bonus as a bonus to First Aid checks when he is unconscious or dying, and also adds the same to saving saves against spells that cause instant death.

\textbf{Details}: Aura Strong abjuration; Requirements to Create Higher Magical Items, Stabilize; Cost +6,000 gp.

\index[OggettiMagici]{Magic Weapons! Terror} \subsubsection*{Terror}

While wielding it, you can use two actions and spend 1 charge to unleash a wave of terror.
Each creature of your choice, within 30 feet of you, must succeed on a DC 17 Will save or be scared of you for 1 minute. While frightened in this way, a creature must spend its turns trying to move as far away from you as possible, and cannot consciously move into a space that is within 30 feet of you. It also cannot react. As his action, he can only use the Move action to Disengage. If it can't move anywhere, the creature can use the Total Defense Action.

At the end of each of its rounds, the creature can re-roll the saving throw, ending the effect for itself if you succeed. This magical weapon has 3 stacks, and recovers 1d3 stacks every day at dawn.

\textbf{Details}: Aura of moderate Charm; Requirements for Creating Greater Magical Items, Fear; Cost +8,000 gp.

\subsubsection*{Titanic} \index[OggettiMagici]{Magic Weapons! Titanic}

This weapon is 3m long and weighs almost 40kg, can only be used by a giant (or an enlarged character). When used as a weapon it has a +2 bonus on hit and inflicts 1d4x10 wounds. It can also be used to quickly plant poles as large as tree trunks and to pull up doors and gates with a few strokes.

\textbf{Details}: Aura Moderate Transmutation; Requirements to Create Greater Magic Items, Enlarge / Reduce; Cost +3,000 gp.

\subsubsection*{Ghost Touch} \index[OggettiMagici]{Magic Weapons! Ghost Touch}

A Ghost Touch weapon deals damage to Incorporeal creatures normally, regardless of its magic bonus and the creature's immunities.

\textbf{Details}: Aura Moderate Summon; Requirements for Creating Greater Magical Items, Planar Shift; Cost +3,000 gp.

\subsubsection*{Thundering} \index[OggettiMagici]{Magic Weapons! Thundering}

A thundering weapon creates a tremendous thunder-like din when it lands a critical hit. The sonic energy does not harm the wielder and deals an additional 1d8 sonic damage for each critical hit scored. remains deaf permanently.

\textbf{Details}: Aura Faint Necromancy; Requirements for Crafting Magical Items, Blindness / Deafness; Cost +3,000 gp.

\subsubsection*{Transformant} \index[OggettiMagici]{Magic Weapons! Transforming}

This ability can only be added to melee weapons. A transforming weapon alters its shape at the command of its wielder, becoming any other melee weapon of similar size.For example, a transforming longsword can take the form of any other Medium-handed melee weapon, such as a Scimitar, a Light Flail, or a Trident, but not a Light or Medium two-handed melee weapon (such as a Medium Short Sword or a Two-Handed Greatsword).

The weapon retains all of its abilities, including bonuses and weapon special abilities, except for those prohibited by its current new form. If left unattended, the weapon reverts to its original shape.

\textbf{Details}: Aura Moderate Transmutation; Requirements for Creating Greater Magical Items, Greater Creation; Cost +5,000 gp.

\subsubsection*{Thing Finder} \index[OggettiMagici]{Magic Weapons! Find Thing}
This ability allows the wielder of this weapon to cast the locate object spell once per day

\textbf{Details}: Aura Light divination; Requirements to Craft Magic Items, Locate Item; Cost +1000 gp.

\index[Magic Items]{Magic Weapons! Vampire} \subsubsection*{Vampire}

When you attack a creature with this magical weapon and gain a critical attack roll, the target, other than constructs and undead, takes an additional 10 Void damage and you gain 10 temporary hit points.

\textbf{Details}: Moderate necromantic aura; Requirements to Craft Greater Magic Items, Vampire Touch; Cost +8,000 gp.

\subsubsection*{Speed} \index[OggettiMagici]{Magic Weapons! Speed}

When making multiple attacks (2 Actions), the wielder of a Speed weapon can make one additional attack with the weapon. The additional attack does not have the penalties of multiple attacks. This ability cannot be combined with spells or similar effects.

\textbf{Details}: Aura Moderate Transmutation; Requirements to Craft Greater Magic Items, Speed; Cost +15,000 gp.

\index[Magic Items]{Magic Weapons! Vorpal} \subsubsection*{Vorpal}

Although it is a +1 magical weapon, it is considered a +5 magical weapon to assess immunity and bonus to attack rolls and damage. In addition, the weapon ignores resistance to cutting damage. When you attack a creature that has at least one head with this weapon and gain a critical attack roll, you cut off one of the creature's heads. The creature dies if it cannot survive without the loss of the head.

A creature is immune to this effect if it is immune to slashing damage, does not have or need a head, or the Storyteller decides the creature is too large for its head to be severed by this weapon.

Instead, such a creature takes an additional 6d8 slashing damage from the hit it takes.

\textbf{Details}: Aura Very strong invocation; Requirements for Crafting Mythic Magical Items; Cost +150000 gp, legendary.

\subsection{Special Abilities of Armor and Magic Shields}

Most magical armor and shields have only bonuses, but some have some of the special abilities described below. An armor or a shield with special abilities must have at least a +1 bonus.

\index{Armor / Magic Shield} \subsubsection*{Armor / Magic Shield}

\textit{Armor (any)} +1 2,500 gp, +2 10,000 gp, +3 18,000 gp, +4 35,000 gp, +5 80,000 gp

\textit{Shields (small, medium, heavy)}: +1 1,500 gp, +2 4,000 gp, +3 9,000 gp, +4 20,000 gp, +5 35,000 gp

While wielding this shield / armor, you have a Defense bonus determined by the shield / armor magic bonus. This bonus is in addition to the normal Defense bonus provided by the shield / armor.

\subsubsection*{Blinding} \index[Magic Items]{Armor and Shields! Blinding}

A shield with this enchantment emits a blinding light up to two times per day at the command of its wielder. Everyone within 20 feet of the shield, except the wielder, must make a DC 14 Reflex save or are blinded for 1d4 rounds.

\textbf{Details}: Aura Moderate invocation; Building Requirements Create Greater Magic Items, Daylight; Cost +3,000 gp.

\index[OggettiMagici]{Armor and Shields! Adamantium} \subsubsection*{Adamantium}

Armor (medium or heavy, but not leather), uncommon +700 gp above the base price of armor. While wearing it, any critical hits you take become a normal hit (but it doesn't protect against damage explosion).


\subsubsection*{Amorphous} \index[OggettiMagici]{Armor and Shields! Amorphous}

Once per day on command, the wearer of the armor (along with whatever equipment he wears) can take the form of a viscous liquid that is able to pass through any space in which thick mud could reasonably flow. While using this ability, your speed is reduced to 10 feet and you can only take move actions. You can assume this form for 1 minute or until you spend a move action to return to your natural form. Amorphous armor must be made primarily of leather, cloth, or other flexible, organic material.

\textbf{Details}: Aura Moderate Transmutation; Building Requirements Craft greater magic items, morph, cost +2,250 gp.


\subsubsection*{Antihemorrhagic} \index[Magic Items]{Armor and Shields! Antihemorrhagic}

Antihemorrhagic armor helps stop Bleeding from the wearer's wounds by automatically tightening like a tourniquet in the appropriate places while also magically reducing the extent of the injury.

Antihemorrhagic armor reduces damage to hit points by 1 per hit taken and if you are, you cannot take bleed damage.

\textbf{Details}: Aura Moderate Transmutation; Building Requirements Craft Greater Magic Items, Cure Critical Wounds, Lesser Restoration, or Stabilize; Cost +3,000 gp.

\subsubsection*{Aries} \index[OggettiMagici]{Armor and Shields! Aries}

These shields are very solid and often bear the emblem of a ram or bull. When the wearer of a ram shield makes a shield attack as part of a charge, the shield's Defense bonus applies to attack and damage rolls. This does not stack with any other upgrades that possess the shield. This ability is not applicable to light type shields.

\textbf{Details}: Aura Weak Invocation; Construction Requirements Create greater magic items, cost +3000 gp.


\subsubsection*{Wrangler} \index[Magic Items]{Armor and Shields! Wrangler}

The wearer of Brawler armor gains a +2 bonus on attack rolls and damage rolls for unarmed attacks. His unarmed strikes count as magical weapons for the purpose of overcoming damage reduction. The Wrangler ability can only be applied to light armor.

\textbf{Details}: Aura Weak Transmutation; Building Requirements Crafting Magical Items, Strength of the Bull; Cost +15,000 gp

\subsubsection*{Balanced} \index[Magic Items]{Armor and Shields! Balanced}

This armor repels anything that threatens to knock the wearer to the ground. The bearer gains a + 1d6 bonus against anyone who tries to push or knock him to the ground.

Plunging to the ground while wearing Balanced armor is a move action instead of an immediate action. The balanced ability can be applied to light or medium armor, but not to heavy armor or shields.

\textbf{Details}: Aura Weak Transmutation; Construction Requirements Craft Magic Items, Cost +3000 gp.

\index[Magic Items]{Armor and Shields! Archer's Bracers} \subsubsection*{Archer's Bracers}

While wearing these bracelets, you are proficient with the longbow and shortbow, and gain a +2 bonus on damage rolls for ranged attacks made with these weapons.

\textbf{Details}: Aura Weak Transmutation; Construction Requirements Craft Magic Items, Cost +3000 gp.

\index[Magic Items]{Armor and Shields! Bracers of Defense} \subsubsection*{Bracers of Defense}
\textit{Wonderful, rare object}

While wearing these bracelets, you have a +1, +2, +3, +4+, +5 bonus to your Defense if you don't wear any armor and don't use any shields.

\textbf{Details}: Aura Abjuration; Construction Requirements Create greater magic items, Cost +6,000 gp, 15,000 gp, 30,000 gp, 45,000 gp, 60,000 gp.

\index[Magic Items]{Armor and Shields! Bracers of the Major Defense} \subsubsection*{Bracers of the Major Defense}
\textit{Marvelous, legendary object}

These bracelets function like armor while not being such. You come wrapped in an invisible magical shield that grants you Defense 15, 17, 19, 21, 23. Defense can be increased with magic items that improve Defense, except armor and shields.

\textbf{Details}: Aura Abjuration; Building Requirements Craft greater magical items, Cost +12,000 gp, 24,000 gp, 36,000 gp, 50,000 gp, 75,000 gp


\subsubsection*{Brilliant} \index[OggettiMagici]{Armor and Shields! Brilliant}

Armor and shields with the Brilliant special ability radiate light like a torch when worn, which can be suppressed or reactivated on command. The appearance of the object is usually characterized by bright colors and a brilliant sheen even when not illuminated. Once per day, the wearer can command the armor or shield to glow with the intensity of a Daylight spell for 10 minutes or until instructed to dim it.

This armor must be cleaned at least once a week or it loses its powers for a week.

\textbf{Details}: Aura Moderate invocation; Building Requirements Craft Magic Items, Daylight; Cost +3,750 gp.

\subsubsection*{Load} \index[Magic Items]{Armor and Shields! Load}

A Load armor distributes the weight carried by the wearer more effectively, allowing him to carry more without suffering the effects of bulk. The wearer's bulk capacity is increased by 50 \%.

\textbf{Details}: Aura Weak Transmutation; Building Requirements Craft Magic Items, Passive Armor; Cost +2000 gp.

\index[Magic Items]{Armor and Shields! Demonic Armor} \subsubsection*{Demonic Armor}

While wearing the armor you can understand and speak the Abyssal. Additionally, the armor's clawed knobs transform unarmed strikes made with your hands into magical weapons that deal slashing damage, with a +1 bonus on attack and damage rolls and the d8 as a damage die.

\textbf{Details}: Aura Strong Summon; Building Requirements Create Greater Magic Items; Cost +5,000 gp.

\subsubsection*{Denegant} \index[OggettiMagici]{Armor and Shields! Denegative}

Once per day, when the wearer of the armor is the target of a Critical Hit or Explosion of damage made with a melee weapon, he can automatically negate this Critical and make it a normal attack. This ability can only be applied to heavy armor.

\textbf{Details}: Aura Strong abjuration; Building Requirements Create Greater Magic Items; Cost +25,000 gp.

\subsubsection*{Determination} \index[OggettiMagici]{Armor and Shields! Determination}

A shield or armor grants the ability to fight in seemingly impossible circumstances. Once per day, when the wielder reaches 0 or fewer hit points, the item automatically activates the cure serious wounds spell.

\textbf{Details}: Aura Moderate Summon; Building Requirements Crafting Greater Magic Items, Cure Serious Wounds; Cost +15,000 gp.

\index[Magic Items]{Armor and Shields! Spell Defense} \subsubsection*{Spell Defense}

You have + 1d6 to saving throws against spells and other magical effects.

\textbf{Details}: Aura Strong abjuration; Building Requirements Create Greater Magic Items; Cost +5,000 gp.

\index[OggettiMagici]{Armor and Shields! Elegant} \subsubsection*{Elegant}

You can use two actions to say the command word to get the armor to take on the appearance of an ordinary suit or some other type of armor. You decide the look, including color, style and accessories, but the armor / shield retains its normal bulk and weight. The illusory aspect lasts until you use this property again or take off your armor.

\textbf{Details}: Aura Moderate illusion; Building Requirements Create Greater Magic Items; Cost +3,000 gp.


\subsubsection*{Sweatshirt} \index[OggettiMagici]{Armor and Shields! Sweatshirt}

An armor with the sweatshirt ability counts against the penalties of wearing armor to light armor. The character is able to move almost without difficulty with this armor.

\textbf{Details}: Aura Strong Transmutation; Building Requirements Create Greater Magic Items; Cost +6,000 gp.

\subsubsection*{Ethereal Form} \index[Magical Objects]{Armor and Shields! Ethereal Form}

On command, this property allows the wearer to become Ethereal (as per the Ethereal Form spell) once per day. The character can remain Ethereal for as long as he wishes, but once back to normal, he can no longer become Ethereal for that day.

\textbf{Details}: Aura Strong Transmutation; Building Requirements Craft Wondrous Magic Items, Ethereal Form; Cost +24,500 gp.

\subsubsection*{Invulnerability} \index[Magic Items]{Armor and Shields! Invulnerability}

This armor grants the wearer 5 / magic damage reduction. Armor with Invulnerability emits a strong abjuration aura.

\textbf{Details}: Aura Strong abjuration; Building Requirements Craft Mythic Magic Items, Wish; Cost +15,000 gp.

\subsubsection*{Untraceable} \index[Magic Items]{Armor and Shields! Untraceable}

Untraceable armor lightens the wearer's footsteps and disguises its appearance. Survival checks to track the bearer take a --5 penalty, and the wearer of the armor gains a +5 bonus on Stealth checks. Only leather or hide armor can be untraceable.

\textbf{Details}: Aura Weak Transmutation; Building Requirements Craft Magic Items, Pass Without Trace; Cost +3,750 gp.

\subsubsection*{Masking} \index[OggettiMagici]{Armor and Shields! Masking}

On command, such armor changes its shape and looks like an ordinary set of clothes. The armor retains all its properties (including weight) even when it is masked. Only true seeing or other similar spells reveal the true nature of the transformed armor.

\textbf{Details}: Aura Moderate illusion; Building Requirements Create Superior Magic Items, Disguise Himself; Cost +1,350 gp.

\index[Magic Items]{Armor and Shields! Mithral} \subsubsection*{Mithral}

Medium or heavy armor, but not leather, uncommon +800 gp above base armor price. Mithral is a light and flexible metal. A vest of mail or a breastplate of mithral can be worn under normal clothing. Reduces weight class by 1 to determine penalties on Proficiency and Magic Checks.

\subsubsection*{Shadow} \index[OggettiMagici]{Armor and Shields! Shadow}

This armor makes the wearer blurry whenever they try to hide, providing a +5 bonus to their Hide in Shadows checks. The armor check penalty applies normally.

\textbf{Details}: Aura Faint illusion; Building Requirements Craft Magic Items, Invisibility, Silence; Cost +1,875 gp.

\subsubsection*{Hospitable} \index[OggettiMagici]{Armor and Shields! Hospitable}

An armor or shield with this special ability hides live animals within its iconography to keep them safe. The bearer with a command word magically stores an animal he is attached to, such as a familiar or mount. The stored animal appears as a symbol on the armor or shield, whether it is an imitation of the animal's appearance or a more symbolic and abstract representation.

While stored, the animal sleeps and gives no benefit (such as a familiar bonus to abilities) to the wearer. The size of the animals that can be stored depends on the type of armor or shield. Light or medium armor and light or heavy shields can store an animal up to the size of the wearer. Heavy armor or a tower shield can store an animal up to one size category higher than the wearer. A second command word releases the animal stored in the hospitable armor or shield. A freed animal immediately awakens, appears in a space adjacent to the bearer, and can take actions in the round it appears.

Since the stored animal sleeps rather than being in suspended animation (or even hibernation), it ages and becomes hungry at its normal rate while it is stored. A Hospitable armor or shield automatically releases a stored animal 24 hours after it is stored in it.

\textbf{Details}: Aura Moderate Summon; Building Requirements Create Superior Magic Items, Secret Chest; Cost +3,750 gp.


\subsubsection*{Perceptive} \index[OggettiMagici]{Armor and Shields! Perceptive}

Perceptive armor comes to the rescue when the wearer has been blinded, is in total darkness (if the wearer does not have darkvision or the ability to see in the dark), or is in magical darkness. When one of these conditions affects the wearer of the armor, Perceptive armor immediately grants Blind Sight within a 5-foot radius. As soon as the wearer sees again, the additional senses cease. The wearer of armor cannot achieve these abilities by closing their eyes.

\textbf{Details}: Aura Strong Divination; Building Requirements Create Wondrous Magic Items, True Seeing; Cost +15,000 gp.

\subsubsection*{Poison Resistance} \index[Magic Items]{Armor and Shields! Poison Resistance}

An armor or shield with this special ability grants the wearer a +3 bonus on saving throws against poison.

\textbf{Details}: Aura Weak Transmutation; Building Requirements Create Superior Magic Items, Remove Poison; Cost +1,125 gp.

\subsubsection*{Energy Resistance} \index[Magic Items]{Armor and Shields! Energy Resistance}

This type of armor or shield protects against one type of energy (Fire, Light, Sound, Electricity, Positive Energy, Negative Energy, Cold, Void) and is decorated with designs depicting the element from which it protects. The armor or shield absorbs the first 10 points of energy damage per attack that would normally be taken by the wearer.

\textbf{Details}: Aura Faint abjuration; Building Requirements Crafting Magic Items, Energy Protection; Cost +9,000 gp.

\subsubsection*{Greater Energy Resistance} \index[Magic Items]{Armor and Shields! Greater Energy Resistance}

This type of armor or shield protects against one type of energy (Fire, Light, Sound, Electricity, Positive Energy, Negative Energy, Cold, Void) and is decorated with designs depicting the element from which it protects. Armor or shield grants Resistance to the indicated energy.

\textbf{Details}: Aura moderate abjuration; Building Requirements Create Superior Magic Items, Energy Protection; Cost +21,000 gp.

\subsubsection*{Wild} \index[OggettiMagici]{Armor and Shields! Wild}

Armor with this special ability generally appears to be made of magically hardened animal skin. A wearer of armor or a shield with this ability retains Defense even while transformed into an animal (either by spell or skill).

Armor and shields with this ability usually bear leaf motifs. While the wearer is in wild form, the armor is not visible.

\textbf{Details}: Aura Moderate Transmutation; Building Requirements Create Greater Magic Items, Polymorph; Cost +15,000 gp.

\index[Magic Items]{Armor and Shields! Dragon Scale} \subsubsection*{Dragon Scale}

This armor or shield is made from the scales of some kind of dragon.

While wearing it, you have + 1d6 on saving throws against the dreadful presence and dragons' breath weapons, and you have resistance to a type of damage determined by the species of dragon that provided the scales.

Additionally, with two actions you can focus your senses to magically determine the distance and direction of the closest dragon within 45 kilometers that is of the same species as the armor. This special action can no longer be used until the next sunrise.

\textbf{Details}: Aura moderate abjuration; Building Requirements Create Greater Magic Items; Cost +8,000 gp.

\subsubsection*{Animated Shield} \index[OggettiMagici]{Armor and Shields! Animated Shield}

While holding this shield, with two actions you can say a command word and make it animate. The shield will float in the air within your space to protect you as if you were holding it, leaving your hand free.

The shield remains animated for 1 minute, until you use two actions to end its effect, are incapacitated or die, at which point the shield will drop to the ground or return to your hand if you have one free.

\textbf{Details}: Aura Strong Transmutation; Building Requirements Create Superior Magic Items, Animate Items; Cost +6,000 gp.

\index[Magic Items]{Armor and Shields! Shield of Bullet Attraction} \subsubsection*{Shield of Bullet Attraction}

While wielding this shield you apparently have resistance to damage from ranged weapon attacks.

\textit{Cursed version}.

Taking off your shield does not end the curse. Whenever a ranged weapon attack is made against a target within 10 feet of you, the curse causes you to become the target of the attack.

\textbf{Details}: Aura Strong Transmutation; Building Requirements Create Superior Magic Items, Animate Items; Cost +2000 gp.


\subsubsection*{Dragon's Breath} \index[Magic Items]{Armor and Shields! Dragon's Breath}

A shield with this special ability is usually made with a dragon's jaws wide open at the front. A shield with the Dragon's Breath special ability is tied to one type of energy (poison, electricity, cold, or fire). The shield recovers 1d4 stacks each sunrise and can hold up to 10.

On command, 2 Actions, the wearer can consume 1 to 5 charges of the shield to cause it to blow a breath into a 15-foot cone that deals 1d4 points of energy damage per charge consumed (Reflex DC 11 halves). This damage is of the same energy type tied to the shield. A shield cannot have more than one dragon breath ability.

\textbf{Details}: Aura Weak Invocation; Building Requirements Craft Magic Items, Burning Hands; Cost +2,500 gp.

\subsubsection*{Titanica} \index[OggettiMagici]{Armor and Shields! Titanica}

An armor with the Titan property is almost comically oversized, even if the effect is only exterior and the interior accommodates a creature as normal, without requiring modification. A creature wearing Titan armor is considered to be of a higher size category, including for the purpose of using items and weapons or being affected by size-dependent special attacks, such as Swallow and Trample.

\textbf{Details}: Aura Moderate Transmutation; Building Requirements Create Greater Magic Items, Enlarge; Cost +15,000 gp.

\subsubsection*{Ghost Touch} \index[Magic Items]{Armor and Shields! Ghost Touch}

This armor or shield appears almost transparent. The Defense value given by the armor counts against the attacks of corporeal and Incorporeal creatures. Additionally, the armor or shield can be picked up, moved, and worn by corporeal and Incorporeal creatures at any time. Incorporeal creatures gain the item's bonus against corporeal and corporeal attacks, and still retain the ability to pass through solid objects.

\textbf{Details}: Aura Strong Transmutation; Building Requirements Craft Wondrous Magic Items, Ethereal Form; Cost +15,000 gp.

\index[OggettiMagici]{Armor and Shields! Vulnerability} \subsubsection*{Vulnerability}

While wearing it, you have resistance to one of the following types of damage: bludgeoning, piercing, or slashing. The Storyteller chooses the type. Armor is cursed, while you are cursed, you have vulnerability to two of the three damage types associated with armor (other than what you have resistance to).

\textbf{Details}: Moderate necromantic aura; Building Requirements Craft Magic Items, Cast Curse; Cost +3,000 gp.


\subsection{Amulets, Necklaces and Jewels}

\index[OggettiMagici]{Magic Objects! Anti-poison Amulet} \subsubsection*{Anti-poison Amulet}
3000 gp, uncommon, this gem on a silver chain is black and shiny. The wearer has a + 1d6 saving throw against poison.

\index[Magical Objects]{Magical Objects! Amulet of Cancrena} \subsubsection*{Amulet of Cancrena}
this engraved gem hanging on a chain appears to be of little value. If a character keeps it with him for more than 1 day, he is struck by a terrible gangrene which causes him to permanently lose 1 point of Dexterity, Constitution and Charisma per week. The gem (and gangrene) can only be countered by remove curse and cure disease, followed by healing or wish. Gangrene can also be defeated by grinding a health amulet and sprinkling its dust on the afflicted character

\index[OggettiMagici]{Healing Amulet} \subsubsection*{Healing Amulet}
25,000 gp, very rare, this gem on a gold chain is red and shiny. The wearer recovers hit points twice as quickly as normal (even max hit points). The amulet prevents you from taking Bleed damage.

\index[OggettiMagici]{Amulet Against Possession} \subsubsection*{Amulet Against Possession}
32,000 gp, very rare, the owner of this copper amulet becomes immune to possession and domination spells.

\index[OggettiMagici]{Amulet of Unavoidable Location} \subsubsection*{Amulet of Unavoidable Location}
this cursed amulet has the appearance of an unobtainable amulet. On the contrary, it makes the wielder vulnerable to this type of magic. The probability of observing the owner and the duration of spells used for this purpose are doubled.

\index[OggettiMagici]{Amulet of the Planes} \subsubsection*{Amulet of the Planes}
160,000 gp, legendary, while wearing this amulet, you can use two actions to name a place you are familiar with that is on another plane of existence. Make an Intelligence check with DC 18. If the check is successful, you cast the planar shift spell with the amulet. If the check fails, you and each creature and object within 5 meters of you are transported to a random destination. Roll a 1d8. From 1 to 4, go to a random destination on the floor you named. From 5 to 8, you reach a randomly determined plane of existence.

\index[OggettiMagici]{Amulet of Protection from Detection and Location} \subsubsection*{Amulet of Protection from Detection and Location}
20,000 gp, rare, while wearing this amulet you are concealed from divination magic. You cannot be targeted by these spells or sensed by magical scrying sensors.

\index[OggettiMagici]{Amulet of Physical Resistance} \subsubsection*{Amulet of Physical Resistance}
8,000 gp, rare, not while wearing this amulet do you get +2 to Fortitude saving throws.

\index[OggettiMagici]{Circlet of the Explosion} \subsubsection*{Circlet of the Explosion}
1500 gp, uncommon, while wearing this headband, you can use two actions to cast the scorching ray spell through it. The headband cannot be used this way again until the next sunrise.


\index[OggettiMagici]{Adaptation Series} \subsubsection*{Adaptation Series}
1500 gp, uncommon, while wearing this necklace, you can breathe normally in any environment that has air, and you have + 1d6 on saving throws made against noxious gases and vapors.

\index[OggettiMagici]{Strangulation Necklace} \subsubsection*{Strangulation Necklace}
this necklace looks like a jewel of great value. When donned, it tightens lightly around the neck, dealing 6 damage per round. It cannot be removed in any way other than with a wish or remove curse, holding onto its victim's neck even after death. The necklace will only loosen when the victim has become a skeleton, ready to be picked up by an unsuspecting treasure hunter.

\index[OggettiMagici]{Fireball Necklace} \subsubsection*{Fireball Necklace}
depending on orbs present: 500 gp, 1000 gp, 1600 gp, 2300 gp, 3100 gp, 4000 gp, 4500 gp, 5000 gp, 5500 gp, 6000 gp, uncommon / rare / very rare: 1d6 + hang from this necklace 3 balls. You can use two actions to detach a sphere and throw it up to 60 feet away. When it reaches the end of its trajectory, the sphere detonates like a fireball spell (DC 18).

\index[OggettiMagici]{Rosary Necklace} \subsubsection*{Rosary Necklace}
3000 gp + variable, rare, this necklace has 1d4 + 2 magical spheres made of aquamarine, black pearl or topaz. It also possesses several non-magical orbs. If a magical sphere were removed from the necklace, that sphere would lose its magic.

There are six types of magical spheres. The Storyteller decides the type of each sphere in the necklace. A necklace can have more than one sphere of the same type. To use it, you must wear the necklace. Each sphere contains a spell that you can cast with two actions, with the spell's DC equal to 10 + 2x Level on a saving throw. Once a magical sphere's spell has been cast, you won't be able to use that sphere again until the next sunrise.

\medskip

\begin{tabularx}{0.45\textwidth}{llX}
\textbf{3d6} & \textbf{Sphere of ...} & \textbf{Enchantment} \\
\hline
3-5 & Blessing & blessing \\
6-11 & Care & cure serious wounds or inferior restorative \\
12-14 & Favor & Superior Restoration \\
15-16 & Punish & Marking Punishment \\
17 & Wind & walking in the wind \\
18 & Summon & Planar Ally \\
\end{tabularx}


\index[OggettiMagici]{Elemental Gem} \subsubsection*{Elemental Gem}
1200 gp, uncommon, this gem contains a spark of elemental energy. When you use two actions to shatter the gem, it summons an elemental as if you had cast the summon elemental spell, and the gem's magic wears off. The type of gem determines the elemental summoned by the spell.

\medskip

\begin{tabular}{ll}
\textbf{Gem} & \textbf{Summoned Elemental} \\
\hline
Red Corundum & Fire Elemental \\
Yellow Diamond & Earth Elemental \\
Emerald & Water Elemental \\
Blue Sapphire & Air Elemental \\
\end{tabular}

\medskip

\index[OggettiMagici]{Gem of Luminosity} \subsubsection*{Gem of Luminosity}
5000 gp, rare, this prism has 50 stacks. While holding it, you can use two actions to say one of the three command words to cause one of the following effects:

\begin{itemize}
\item
The first command word causes the gem to produce bright light within 30 feet and dim light for an additional 30 feet. The effect does not consume charges. Lasts until you use two actions to repeat the command word or until you use another gem function.

\item
The second command word spends 1 charge and causes the gem to cast a beam of bright light at a visible creature within 60 feet of you. The creature must succeed at a DC 17 Fortitude save or be blinded for 1 minute.

\item
The third command word spends 5 stacks and causes the gem to radiate a blinding light into a 30-foot cone originating from you. Each creature within the cone must make a saving throw as if struck by the beam created by the second command word.

\end{itemize}

\medskip

When all of the gem's stacks have been spent, the gem becomes a common jewel worth 50 gp.

\index[OggettiMagici]{Sight Gem} \subsubsection*{Sight Gem}
32000 gp, very rare, with two actions, you can say the command word of the gem and spend 1 charge. For the next 10 minutes, when you look through the gem you have true vision up to 36 meters away. The gem has 3 charges, and recovers 1 spent charge every day at dawn.

\index[OggettiMagici]{Monster Attracting Jewel} \subsubsection*{Monster Attracting Jewel}
this magical jewel is cursed, the wielder attracts wandering monsters with double the probability. Monsters will also chase him twice as likely if he escapes. The jewel cannot be abandoned and will immediately reappear on the owner's person every time he tries to get rid of it. Only Remove Curse will allow the owner to leave the jewel behind.

\index[OggettiMagici]{Medallion of the Feather Fall} \subsubsection*{Medallion of the Feather Fall}
400 gp, uncommon, this medallion automatically triggers the Falling Feather spell when the owner falls from a height of 2 meters or more.

\index[OggettiMagici]{Medallion of Thoughts} \subsubsection*{Medallion of Thoughts}
3000 gp, uncommon, while wearing this medallion, you can use two actions and spend 1 charge to cast the detect thoughts spell (saving throw DC 15) with it. The medallion has 3 charges, and recovers 1 expense charge every day at dawn.

\index[Magic Items]{Pearl of Wisdom} \subsubsection*{Pearl of Wisdom}
20,000 gp, rare, this magical pearl grants an extra point of Wisdom that keeps it with it for 4 weeks. After this time the pearl must always be worn in order not to lose its benefits. There is a 5 \% chance that a pearl will be cursed and have the opposite effect. In this case, after 4 weeks, the negative effect is permanent and can only be canceled by desire.

\index[OggettiMagici]{Death Beetle} \subsubsection*{Death Beetle}
this brooch in the shape of a scarab looks like a simple good luck charm. However, if held in hand for 1 round or carried for 1 turn, it transforms into a hideous carnivorous bug. Equipped with powerful jaws, the ravenous creature penetrates through leather and fabric, sinking into the poem and reaching the heart in 1 round. After killing its victim, the creature takes on the form of a pin. Only the heat that comes from contact with a living being can animate the monstrous insect, so putting the pin in a box or in a display case is a sufficient precaution to avoid any danger.

\index[OggettiMagici]{Protection Beetle} \subsubsection*{Protection Beetle}
36,000 gp, legendary, if you hold this scarab medallion in your hand for 1 round, an inscription will appear on it revealing its magical nature. While it's on you, it provides two benefits

- You have +2 to saving throws against spells.

- The scarab has 12 charges. If you fail a saving throw against a necromancy spell or noxious effect originating from an undead creature, you can use a Reaction Action to spend 1 charge and turn the failed saving throw into a success. The scarab is reduced to dust and is destroyed when its last charge is spent.

\index[Magic Items]{Pin of Defense} \subsubsection*{Pin of Defense}

7,500 gp, uncommon, the pin can absorb 101 damage from Strength spells, then lose its magical properties.

\index[OggettiMagici]{Talisman of pure Good} \subsubsection*{Talisman of pure Good}
50,000 gp, legendary, a Devotee of Gradh or Sumkjr in possession of this item can cause a chasm of flames to appear at the foot of a Devotee of Calicanthus or Shayalia within 30 yards. The victim is swallowed by the fire and falls screaming towards the center of the Earth. A pure good talisman has 6 charges and cannot be recharged. If a devotee of Calicanthus or Shayalia touches him, he takes 6d6 wounds. Any other Devotee or Follower suffers no effect. The Talisman pulsates with light within 36 meters of a Devotee or Follower of Calicanthus or Shayalia.

\index[OggettiMagici]{Talisman of extreme Evil} \subsubsection*{Talisman of extreme Evil}
50,000 gp, legendary, this talisman functions exactly like the talisman of pure good but with the Patrons reversed.

\index[Magic Items]{Charm of Poison Protection} \subsubsection*{Charm of Poison Protection}
5,000 gp, rare, while wearing this pendant, poisons have no effect on you. You are immune to the poisoned condition and have immunity to poison damage.

\index[OggettiMagici]{Talisman of Health} \subsubsection*{Talisman of Health}
5000 gp, rare, while wearing this pendant you are immune to the possibility of contracting any disease. If you are already infected with a disease, its effects are suspended as long as you wear this pendant.

\index[OggettiMagici]{Talisman of the Sphere} \subsubsection*{Talisman of the Sphere}
75,000 gp, legendary, when you make an Arcana check to control an orb of annihilation while wielding this talisman, you have a bonus of 5. Also, when you start the round with control of an orb of annihilation, you can use two actions to levitate it 3 meters plus an additional number of meters equal to 3 x your Intelligence value.

\subsection{Belts, Helmets, Boots and Gloves}

\index[OggettiMagici]{Belt of the Giants} \subsubsection*{Belt of the Giants}

10000/15000/20000/30000/45000 gp, rarity varies, while wearing this belt, your score matches the score given by the belt. If your Strength score is already equal to or greater than the belt score, the item has no effect on you.

There are four variants of this belt, each corresponding to a kind of real giants. The stone giant's belt and the frost giant's belt appear different, but have the same effect.

\medskip

\begin{tabular}{lll}
\textbf{Kind of Giant} & \textbf{Strength} & \textbf{Rarity} \\
\hline
\textbf{Hill} & 5 & Rare \\
\textbf{Frost / Stone} & 6 & Very Rare \\
\textbf{Fire} & 7 & Very Rare \\
\textbf{Clouds} & 8 & Legendary \\
\textbf{Storms} & 9 & Legendary \\
\end{tabular}

\index[Magic Items]{Belt of the Dwarves} \subsubsection*{Belt of the Dwarves}
86,000 gp, rare, while wearing this belt, you gain the following benefits:

- your Constitution score increases by 1, up to a maximum of 5.

- you have +2 to Charisma checks made to interact with dwarves.

Also, while you are wearing the belt you have a 50 \% chance every day at dawn to see yourself growing a full beard, if it can grow, or to see yours even thicker, if you already have one.

If you are not a dwarf, you get the following additional benefits when wearing this belt:

- you have +2 saving throws against poison and have resistance to poison damage. You have darkvision with a range of 60 feet. You can speak, read and write in Dwarven.

\index[OggettiMagici]{Helmet of Understanding Languages} \subsubsection*{Helmet of Understanding Languages}
600 gp, common, while wearing this helmet, you can use two actions to cast the understand languages spell at will.

\index[Magic Items]{Helm of Luster} \subsubsection*{Helm of Luster}
75,000 gp, legendary, this glowing helmet is set with 1d10 diamonds, 2d10 rubies, 3d10 fire opals, and 4d10 opals. Any gem extracted from the helmet is reduced to dust. When all gems are removed or destroyed, the helmet loses its magic. While wearing it you get the following benefits:

\medskip

\begin{itemize}
\item
You can use two actions to cast one of the following spells, using one of the helmet gems of the specified type as a component: daylight (opal), wall of fire (ruby), fireball (fire opal), or prismatic spray (diamond ). When the spell is cast the gem is destroyed and disappears from the helmet.

\item
As long as it has at least one diamond, the helmet emits light within a 30-foot radius when at least one undead is within this area. Any undead that begins its round inside the area takes 1d6 points of Light damage.

\item
As long as the helmet has at least one ruby, you have resistance to fire damage.
\end{itemize}

\medskip

As long as the helmet has at least one fire opal, you can use two actions and say a command word to cause a weapon you are holding to be engulfed in flames. The flames emit light within a radius of 3 meters and dim light for an additional 3 meters. The flames are harmless to you and the weapon. When you hit with an attack with the fired weapon, the target takes an additional 1d6 points of fire damage. The flames persist until you use two actions to say the command word again or until you drop or sheathe your weapon.

If you are wearing the helmet and take fire damage from a critical save against a spell, the helmet emits a beam of light through the remaining gems. Any creature within 60 feet of the helmet, other than you, must succeed on a DC 21 Reflex saving throw or be struck by the beam, taking Light damage equal to the number of gems in the helmet x 5. Then, the gems and the helmet are destroyed.

\index[OggettiMagici]{Helmet of Underwater Movement} \subsubsection*{Helmet of Underwater Movement}
4000 gp, rare, this helmet, usually made of fish skin, grants the ability to breathe underwater, movement Swim 60 feet, echolocation 60 feet. The power is usable for 6 hours a day and recharges at dawn.

\index[OggettiMagici]{Helm of Telepathy} \subsubsection*{Helm of Telepathy}
12,000 gp, rare, while wearing this helmet, you can use two actions to cast the detect thoughts spell (saving throw DC 13) with it. As long as you keep your focus on the spell, you can use two actions to send a telepathic message to the creature you're focused on. She can replicate (using two actions to do this) as long as you keep focusing on her.

While focusing on a creature with detect thoughts, you can use two actions to helm the suggestion spell (saving throw DC 13) on that creature. Once used, the suggestion property cannot be used again until the next sunrise.

\index[Magic Items]{Teleportation Helm} \subsubsection*{Teleportation Helm}
64,000 gp, rare, while wearing this helmet, you can use two actions and spend 1 charge to cast the teleport spell through it. The helmet has 3 charges, and recovers 1 every morning at dawn.

\index[Magic Items]{Bullet Grab Gloves} \subsubsection*{Bullet Grab Gloves}
3000 gp, uncommon, these quanta almost seem to merge with your skin when you wear them. When a ranged weapon attack hits you while wearing them, you can use a Reaction Action to reduce the damage by 1d10 + Dexterity, as long as you have a free hand. If you reduce the damage to 0 and the bullet is small enough to be held in your hand, you can grab it.

\index[OggettiMagici]{Gloves of Orcish Power} \subsubsection*{Gloves of Orcish Power}
9000 gp, rare, while wearing these knobs your Strength is 4. Gloves have no effect if your Strength is already 4 or higher.

\index[OggettiMagici]{Swimming and Climbing Gloves} \subsubsection*{Swimming and Climbing Gloves}
2000 gp, uncommon, while wearing both of these gloves, climbing and swimming cost you no additional movement. In addition, you have a + 1d6 bonus on Constitution and Wisdom checks made while climbing or swimming.

\index[Magic Items]{Gloves of Dexterity} \subsubsection*{Gloves of Dexterity}
12000 gp, rare, these gloves give the owner a minimum Dexterity of +2 and if he already has a score of +2 this increases by 1 (up to a maximum +4). In addition, the wielder gains + 1d6 in the Fairy Hands skill

\index[OggettiMagici]{Clumsy Gloves} \subsubsection*{Clumsy Gloves}
these gloves can be made of soft leather or heavy protective material suitable for use with armor. In the first case they appear to be gloves of dexterity. In the second case they appear to be gloves of orc power. At each try the gloves appear to perform the above functions as long as the wearer is not under attack or in a life or death situation. At that moment the curse is activated. The character becomes clumsy, with a 50 \% chance each round of dropping an object he is holding. Gloves reduce Dexterity by 2 points. Once the curse is active, the gloves can only be removed with a remove curse spell or wish.

\index[OggettiMagici]{Slippers of the Spider} \subsubsection*{Slippers of the Spider}
5000 gp, uncommon, while wearing these lightweight shoes, you can move up, down, and along vertical surfaces and upside down on the ceiling, leaving your hands free. You have a climbing speed equal to the walking speed. However, slippers do not allow you to move in this way on difficult terrain, such as walls covered by ice, oil, rubble ...

\index[OggettiMagici]{Winged Boots} \subsubsection*{Winged Boots}
15,000 gp, rare, while wearing these boots, you have a flight speed equal to your walking speed. You can use these boots to fly for up to 4 hours, all together or divided into short flights, each taking a minimum of 1 minute in duration. If the duration ends while you are flying, descend at a speed of 30 feet per round until you land. The boots recover 2 hours of flying ability every sunrise.

\index[OggettiMagici]{Boots of Running and Jumping} \subsubsection*{Boots of Running and Jumping}
5000 gp, uncommon, while wearing these boots, your walking speed becomes 30 feet, unless it is higher, and your speed is not reduced if you are cluttered or wearing heavy armor. In addition, you jump three times the normal distance, up to a maximum of 30 feet.

\index[Magic Items]{Boots of the Elves} \subsubsection*{Boots of the Elves}
3000 gp, uncommon, while wearing these boots, your footsteps make no sound no matter what surface you are walking through. You have + 1d6 on Move Silently checks.

\index[OggettiMagici]{Boots of Winter} \subsubsection*{Boots of Winter}
10,000 gp, rare, while wearing these boots you have resistance to cold damage, ignore difficult terrain produced by snow or ice. You can tolerate temperatures down to -45 ° C without the need for additional protection. If you wear warm clothes, you can tolerate temperatures as low as -75 ° C.

\index[OggettiMagici]{Boots of Levitation} \subsubsection*{Boots of Levitation}
5000 gp, rare, while wearing these boots, you can use two actions at will to cast the levitation spell on you.

\index[Magic Items]{Boots of Speed} \subsubsection*{Boots of Speed}
5000 gp, rare, while wearing these boots, you can use a bonus action to use only to move. You can end the effect whenever you want. The effect lasts until finished, for up to 10 minutes a day. Capacity recharges at dawn.

\index[OggettiMagici]{Dancing Boots} \subsubsection*{Dancing Boots}
these cursed boots work like other magic boots. However, when the character enters combat or attempts to escape from a potential fight, he is affected by an irresistible dance spell, with no chance of a saving throw. Dancing Boots can be removed with the Remove Curse or Wish spell.

\subsection{Wands, Rods and Staff}

\index[MagicObjects]{Metal Finding Wand} \subsubsection*{Metal Finding Wand}
500 gp, uncommon, when a charge is spent, the wand points in the direction of any metal mass of at least 100 kg within 20 feet. Whoever holds the wand has an intuitive perception of the type of metal identified.

\index[Magic Items]{Wand of Enchanted Darts} \subsubsection*{Wand of Enchanted Darts}
8000 gp, rare, while wielding this wand, you can use two actions to spend 1 or more of its charges to cast an enchanted bolt through it, such as the spell of the same name. Each charge generates 1 dart. The wand has 7 charges. The wand recovers 1d3 + 1 charges spent at dawn each day. However, if you spend the last charge of the wand, roll 1d6 if you get a 1 the wand grinds to dust and is destroyed.

\index[OggettiMagici]{Wand of Comfort} \subsubsection*{Wand of Comfort}
300 gp, common, The wielder can spend 1 charge to cast invisible minion or invisible cook or floating disc spells. The wand has 7 charges which are recovered at dawn.

\index[OggettiMagici]{Wand of Lightning} \subsubsection*{Wand of Lightning}
32,000 gp, rare, while holding this wand, you can use two actions to spend 1 charge to cast the lightning spell (saving throw DC 18) with it.
This wand has 7 charges. The wand recovers 1d3 + 1 charges spent at dawn each day. However, if you spend the last charge of the wand, roll 1d6 if you get a 1 the wand grinds to dust and is destroyed.

\index[OggettiMagici]{Wand of Fire} \subsubsection*{Wand of Fire}
18,000 gp, very rare, a fire wand produces various spells and consumes 1 charge + level of the manifested spell. The manifestable spells are: burning hands, expert pyro, fireball, wall of fire. As long as the wand is held in hand, every 1 on the fire damage dice it inflicts is treated as 2. The wand has 7 stacks and recovers 1 at dawn.

\index[OggettiMagici]{Ice Wand} \subsubsection*{Ice Wand}
15000 gp, very rare, a fire wand produces several spells and consumes 1 charge + level of the manifested spell. The manifestable spells are: frost ray, sleet storm, ice storm, cold cone. As long as the wand is held in hand, every 1 on the dice for cold damage it inflicts is treated as 2. The wand has 7 stacks and recovers 1 at dawn.

\index[OggettiMagici]{Magic Detection Wand} \subsubsection*{Magic Detection Wand}
1500 gp, uncommon, while holding this wand, with two actions you can spend 1 charge to cast the detect magic spell with it. This wand has 7 stacks, and recovers 1d3 stacks spent every morning at sunrise.

\index[OggettiMagici]{Enemy Detection Wand} \subsubsection*{Enemy Detection Wand}
4000 gp, rare, while wielding this wand, you can use two actions and spend 1 charge to speak its command word. For the next minute, know which direction the closest hostile creature is within 60 feet of you, but not the distance between you. The wand can sense the presence of hostile creatures that are ethereal, invisible, disguised, or hidden, as well as those in plain sight. The effect ends if you stop holding the wand. This wand has 7 charges. The wand recovers 2 charges spent at dawn each day. However, if you spend the last charge of the wand, roll 1d6 if you get a 1 the wand grinds to dust and is destroyed.

\index[OggettiMagici]{Wand of Illusions} \subsubsection*{Wand of Illusions}
3000 gp, rare, the wielder of this wand can cast Major Image (3), Silent Image (1), Mirror Image (2). Each spell costs a number of charges equal to level +1. While focusing on the effect, the character can only move at half speed. If he is hit he must succeed in a Magic Check or the illusion immediately vanishes.

\index[OggettiMagici]{Wand of Detection of Secret Doors} \subsubsection*{Wand of Detection of Secret Doors}
300 gp, uncommon, this wand points to the closest secret passage within 20 feet. The effect consumes one charge of the 7 available, every day at dawn all the charges are recovered.

\index[OggettiMagici]{Wand of Light} \subsubsection*{Wand of Light}
3500 gp, rare, a wand of light manifests several spells and consumes 1 charge + level of the spell manifested. The manifestable spells are: dancing lights, light, everlasting flame, daylight. Finally, by spending 5 charges, the wielder can create a beam of intense sunlight. The intense golden-yellow light has a range of 36 m and forms a sphere of light with a diameter of 12 m. Anyone in the area of effect is blinded and stunned for 1 round if they fail a DC 17 Fortitude save. The golden sphere has a devastating effect on the undead, inflicting 6d6 Light wounds with no chance of a saving throw. This wand has 7 charges. The wand recovers 2 charges spent at dawn each day. However, if you spend the last charge of the wand, roll 1d6 if you get a 1 the wand grinds to dust and is destroyed.

\index[Magic Items]{Wand of the War Wizard} \subsubsection*{Wand of the War Wizard}
1500/5500/25000 gp, uncommon (+1), rare (+2), or very rare (+3), while wielding this wand, you gain a bonus on attack rolls with spells determined by the rarity of the wand. Also, you ignore light cover when making a spell attack.

\index[OggettiMagici]{Wand of Metamorphosis} \subsubsection*{Wand of Metamorphosis}
32000 gp, very rare, while wielding this wand, you can use two actions to spend 1 charge to cast the morph spell with it (DC 18, Will saving throw). This wand has 3 charges. The wand recovers 1 charge spent at dawn each day. However, if you spend the last charge of the wand, roll 1d6 if you get a 1, the wand grinds to dust and is destroyed.

\index[MagicObjects]{Wand of Wonders} \subsubsection*{Wand of Wonders}
25000 gp, very rare, while wielding this wand, you can spend 1 charge with two actions and choose a target within 36 meters of you. The target can be a creature, object, or point in space. The Storyteller decides or randomly determines what will happen when you use the wand. Spells cast with the wand have a saving throw DC of 18. If the spell normally has a range in meters, the range becomes 36 meters if it doesn't already. If an effect covers an area, you must center the spell on the target and include it. If an effect acts on as many subjects as possible, the Storyteller randomly determines who is affected.

This wand has 7 charges. The wand recovers 1 charge every day at dawn. If you spend the last charge of the wand, roll 1d6 if you get 1 the wand is reduced to dust and is destroyed.

Each time you use the Wand of Wonders, roll a d100 and consult this table.

\end{multicols}

\begin{center}
	\includegraphics[width = 0.3 \linewidth]{immagini/bacchette.png}

\end{center}

\medskip

\begin{tabularx}{0.95\textwidth}{lX}
\textbf{d100} & \textbf{Contents} \\
\hline
01-05 & Slow launches. \\
06-10 & Fairy fire launches. \\
11-15 & You are stunned until the start of your next round, and you think something amazing has happened. \\
16-20 & Gust of wind launches. \\
21-25 & Launch identification of thoughts on the target of your choice. If your target isn't a creature, you take 1d6 points of damage instead. \\
26-30 & Throw stinking cloud. \\
31-33 & Heavy rain falls within a 60-foot radius centered on the target. The area becomes darkened slightly. The rain continues to fall until the start of your next round. \\
34-36 & An animal appears in the unoccupied space closest to the target. The animal is not under your control and acts as normal. Roll a d100 to determine which species of animal appears. 01-25, a rhino; 26-50, an elephant; 51-100, a rat. \\
37-46 & Lightning Throws. \\
47-49 & A cloud of 600 huge butterflies fills a 30-foot radius around the target. The area becomes heavily darkened. The butterflies remain for 10 minutes. \\
50-53 & Zoom in on the target as if you had cast the zoom in / zoom out spell. If the target can't be subject to the spell, or if it's not a creature, you become the target. \\
54-58 & Dark Throws. \\
59-62 & Thick grass sprouts within 60 feet around the target. If there is already grass, it grows tenfold and stays that way for 1 minute. \\
63-65 & An object of the Storyteller's choice disappears on the Ethereal Plane. The object must not be worn or carried, it must be within 36 meters of the target, and no larger than 3 meters in each dimension. \\
66-69 & You shrink as if casting the enlarge / reduce spell on yourself. \\
70-79 & Fireball Throws. \\
80-84 & You cast invisibility on yourself. \\
85-87 & Leaves grow on the target. If you have chosen a point in space as a target, leaves will sprout on the creature closest to that point. Unless plucked, the leaves will turn brown and drop off after 24 hours. \\
88-90 & A stream of 1d4 x 10 gems worth 1 gp each flows from the tip of the wand in a line 30 feet long and 5 feet wide. Each gem deals 1 bludgeoning damage, and their total damage is divided equally among all creatures on the line. \\
91-95 & A burst of twinkling and colorful lights extends from you within a 30-foot radius. You and all creatures in the area must succeed at a DC 15 Fortitude save or be blinded for 1 minute. A creature can re-roll its saving throw at the end of each of its rounds, ending the effect on itself if you succeed. \\
96-97 & Target's skin takes on a deep blue tint for 1d10 days. If you chose a point in space, the subject will be the closest creature to that point. \\
98-00 & If the target is a creature, it must make a DC 18 Fortitude save. If the target is not a creature, the target becomes you and you make the saving throw. If the saving throw fails by 5 or more, the target is petrified. If the saving throw fails less, the target is hampered and begins to turn to stone. While hampered in this way, the target must re-roll the saving throw at the end of each of its rounds, becoming petrified on failure or ending the effect on success. The target remains petrified until freed from stone in flesh or similar spells. \\
\end{tabularx}

\medskip

\begin{multicols}{2}

\index[OggettiMagici]{Wand of Denial} \subsubsection*{Wand of Denial}
35,000 gp, very rare, this wand negates spells or similar effects produced by magical items. The wielder points the wand at the object within 36 meters, and it emits a light gray beam that hits the target. The beam automatically negates the manifestation of spells or similar effects of level 3 or less. Each use of the wand costs 1 charge and can only be used once per round. This wand has 3 charges. The wand recovers 1 charge every day at dawn. If you spend the last charge of the wand, roll 1d6 if you get 1 the wand is reduced to dust and is destroyed.

\index[Magic Items]{Wand of Fireballs} \subsubsection*{Wand of Fireballs}
32,000 gp, rare, while holding this wand, you can use two actions to spend 1 charge to cast the fireball spell (saving throw DC 18) with it. This wand has 7 charges. The wand recovers 1 charge spent at dawn each day. However, if you spend the last charge of the wand, roll 1d6 if you get a 1 the wand grinds to dust and is destroyed.

\index[OggettiMagici]{Wand of Paralysis} \subsubsection*{Wand of Paralysis}
16,000 gp, rare, while wielding this wand, you can use two actions to expend 1 charge to cause a thin beam to fire from its tip to a visible creature within 60 feet of you. The target must succeed at a DC 17 Fortitude save or be paralyzed for 1 minute. At the end of each round the target can make a DC 15 Fortitude save, ending the effect on him if he succeeds. This wand has 7 charges. The wand recovers 1 expense charge at sunrise of each day. However, if you spend the last charge of the wand, roll 1d6 if you get a 1 the wand grinds to dust and is destroyed.

\index[OggettiMagici]{Wand of Fear} \subsubsection*{Wand of Fear}
13,000 gp, rare, this wand has 7 charges for the following properties. The wand recovers 1 expense charge at sunrise of each day. However, if you spend the last charge of the wand, roll 1 if you get 1 the wand grinds to dust and is destroyed.

\textbf{Command} While holding this wand, you can use two actions to spend 1 charge and command another creature to flee or crawl, as per the command spell (saving throw DC 18)

\textbf{Cone of Fear} While holding this wand, you can use two actions to expend 2 charges, causing the tip of the wand to emit light in a 60-foot cone. Each creature in the cone must succeed on a DC 18 Will save or be scared of you for 1 minute. While frightened in this way, a creature must spend its turns trying to move as far away from you as possible, and cannot voluntarily move within 30 feet of you.

It also cannot react. As its action, the creature can only use the Sprint action or try to free itself from an effect that prevents it from moving. If it can't move anywhere, the creature can use the Full Defense action. At the end of each of its rounds, the creature can re-roll the saving throw, ending the effect on itself if it succeeds. This wand has 7 charges. The wand recovers 1 expense charge at sunrise of each day. However, if you spend the last charge of the wand, roll 1d6 if you get a 1 the wand grinds to dust and is destroyed.

\index[OggettiMagici]{Discover Traps Wand} \subsubsection*{Discover Traps Wand}
400 gp, uncommon, this wand points to the nearest trap within 6 yards. The effect consumes one charge. This wand has 7 charges. The wand recovers all the charges spent at dawn each day.

\index[OggettiMagici]{Wand of Secrets} \subsubsection*{Wand of Secrets}
500 gp, uncommon, while holding this wand, you can use two actions to spend 1 charge and detect if secret door or trap is within 30 feet of you, the wand pulses and points to the one closest to you. The wand has 3 charges. The wand recovers all the charges spent at dawn each day.


\index[OggettiMagici]{Web Wand} \subsubsection*{Web Wand}
8000 gp, uncommon, while wielding it, you can use two actions to spend 1 charge to cast the web spell with it (saving throw DC 18). This wand has 7 charges. The wand recovers 1 charge spent at dawn each day. However, if you spend the last charge of the wand, roll 1d6 if you get a 1 the wand grinds to dust and is destroyed.

\index[OggettiMagici]{Wand of Binding} \subsubsection*{Wand of Binding}
10,000 gp, rare, this wand has 7 charges for the following properties. The wand recovers 1 expense charge at sunrise of each day. However, if you spend the last charge of the wand, roll 1d6 if you get a 1 the wand grinds to dust and is destroyed. While wielding this wand, you can use two actions and spend some of its charges to cast one of the following spells (saving throw DC 21):

\textbf{block monsters} (5 charges) or \textbf{block people} (2 charges).

\index[OggettiMagici]{Wand of Assisted Escape} \subsubsection*{Wand of Assisted Escape}
2000 gp, rare, while wielding this wand, you can use the Reaction action and spend 1 charge to get + 1d6 on the saving throws you make to avoid being paralyzed or entangled, or you can spend 1 charge to get + 1d6 on any check you make to escape an attempt to grab.

\index[OggettiMagici]{Staff of the Archmage} \subsubsection*{Staff of the Archmage}
125,000 gp, legendary, the archmage's staff is a very powerful version of the staff of sorcery. It provides the owner with various spells. The staff can be used to manifest spells: arcane lock, detect magic, zoom in / out, and light. These capabilities do not require the consumption of charges. In addition, the staff has the following abilities that cost 1 charge per use: dispel magic, lightning bolt, invisibility, wall of fire, fireball, wall pass, pyro expert, web, burglary, and ice storm. The following powerful abilities cost 2 charges per use: Summon Elemental, Planar Shift, Telekinesis. The wielder of the staff receives a +2 bonus on saving throws against spells. The staff can be recharged, but only by absorbing the magical energies thrown at the owner, who can absorb them in an amount equal to 1 charge per spell level. This operation is the only action possible in a round, and the staff cannot be used for other effects in the same round in which it absorbs energy. Each stick has a maximum number of charges possible, and it will only absorb charges up to its limit without incurring any deleterious effects. The wielder has no way of knowing this limit, or how many charges have been used, unless some magical method is used. If the stick absorbs excess energy, it explodes as in the case of a final blow, described below. An archmage's staff can be used for a final blow, which requires it to be broken by its owner. The break must not be accidental and must be declared. All charges stored in the stick are instantly released within 30 feet. All creatures within 10 feet suffer wounds equal to 10 times the number of stacks on the staff; between 3 m and 6 m the wounds are 6 times the number of charges; and between 6 m and 9 m the wounds are 4 times the number of charges. A DC 25 Fortitude save reduces the damage in half. The character who breaks the staff has a 50 \% chance of going to another plane of existence, otherwise the explosive release of magical energy destroys him. When all charges have been used up, the staff becomes a +2 staff. If the stacks are exhausted, it cannot be used for a final hit.

\index[Magic Items]{Staff of Withering} \subsubsection*{Staff of Withering}
3000 gp, rare, the staff can be wielded like a magical fighting staff. If you hit, it deals damage like a normal combat staff, and you can spend 1 charge to deal an additional 2d10 Void damage to the target. In addition, the target must succeed at a DC 18 Fortitude save or have -1d6 for 1 hour on any ability check or saving throw that requires Constitution. This staff has 3 stacks and recovers 1d3 stacks spent at midnight.

\index[Magic Items]{Staff of the Woods} \subsubsection*{Staff of the Woods}
44,000 gp, rare, the staff can be wielded as a magical fighting staff that grants a +2 bonus on attack and damage rolls made with it. When you wield it, you also have a +2 bonus on attack rolls with spells.
This staff has 10 charges for the following properties. Recover 2 charges spent every day at sunrise. If you spend the last charge of the stick, roll 1d6 if you get 1 the stick blackens, turns to ash, and is destroyed.

- \textit{Spells}. You can use two actions to spend 1 or more charges on the staff to cast one of the following spells using it, using your spell saving throw DC: animal friend (1 charge), locate animals and plants (1 charge), wall of thorns (6 charges), talking to animals (3 charges), leathery skin (2 charges) or awakening (5 charges). You can also use two actions to cast the spell pass without a trace with the stick
spend charges.

- \textit{Tree Shape}. You can use two actions to plant one end of the stick in fertile soil and spend 1 charge to turn the stick into a vigorous fruit tree. The tree is 18 meters high, with a trunk of 1.5 meters in diameter; at the top its branches extend for 6 meters. The tree looks like a normal tree but radiates a faint aura of transmutation magic if it is the target of the detect magic spell. While in contact with the tree and using another action to say its command word, return the staff to its normal shape. Any creature on the tree falls when it transforms back into a staff.

\index[OggettiMagici]{Staff of Charm} \subsubsection*{Staff of Charm}
12,000 gp, rare, while wielding this staff, you can use two actions to spend 1 charge to cast charm on people, command, or understand languages through it, using your DC of spell saving throws. The staff can be used as a magical fighting staff.

If you are holding the staff and fail a saving throw against an enchantment spell that targets only you and not an area, you can turn the failed saving throw into a success. You will no longer be able to use this staff property until dawn the next day.

If you succeed in a saving throw against an enchantment spell that targets only you, with or without the intervention of the staff, you can use a Reaction Action to spend 3 charges from the staff and turn the spell against the caster. as if the spell was cast by you.

The staff has 7 charges, and recovers 1 spent charge every day at dawn. If you spend the last charge, roll 1d6 if you get 1 the staff becomes a normal fighting staff.

\index[Magic Items]{Staff of Striking} \subsubsection*{Staff of Striking}
25,000 gp, very rare, this staff can be wielded as a magical fighting staff that grants a +3 bonus to attack and damage rolls made with it. When you hit with a melee attack using the staff, you can spend up to 3 of its stacks. For each charge spent, the target takes an additional 1d6 points of force damage. The staff has 10 charges, and recovers 2 charges spent every day at dawn. If you spend the last charge, roll 1d6 if you get 1 the staff becomes a normal fighting staff.

\index[Magic Items]{Staff of Fire} \subsubsection*{Staff of Fire}
16,000 gp, very rare, while wielding this staff, you have resistance to fire damage.
Additionally, you can use two actions to expend 1 or more of its charges to cast one of the following spells through it: burning hands (1 charge, DC 13), wall of fire (4 charges, DC 19), or fireball (3 charges , DC 17).

The staff has 10 charges, and recovers 2 spent charges each day at dawn. If you spend the last charge of the stick, roll 1d6 if you get 1 the stick blackens, turns to ash, and is destroyed.

\index[OggettiMagici]{Staff of Frost} \subsubsection*{Staff of Frost}
26000 gp, very rare, while wielding this staff, you have resistance to cold damage.
Additionally, you can use two actions to spend 1 or more of its charges to cast one of the following spells through it.

- \textit{Spells}: cone of cold (5 charges, DC 21), wall of ice (4 charges, DC 19), cloud of fog (1 charge, DC 13) or ice storm (4 charges, DC 19) ).

The staff has 10 charges, and recovers 2 spent charges each day at dawn. If you spend the last charge of the stick, roll 1d6 if you get 1 the stick turns into water and is destroyed.

\index[Magic Items]{Staff of Healing} \subsubsection*{Staff of Healing}
13,000 gp, rare, while wielding it, you can use two actions to spend 1 or more of its stacks to cast one of the following spells through it: heal light wounds (1 stack), restore lesser (2 stacks), remove disease (3 stacks) ). This staff has 10 charges, and recovers 1 charge spent every day at dawn. If you spend the last charge of the staff, roll 1d6 if you roll 1 the staff vanishes in a flash of light, lost forever.

\index[OggettiMagici]{Staff of the Swarming Bugs} \subsubsection*{Staff of the Swarming Bugs}
160,000 gp, rare, this staff has 10 stacks that you can use to use the properties described below and recovers 1 stacks every day at dawn. If you spend the last charge of the stick, roll 1d6 if you get 1 a swarm of insects consumes and destroys the stick, and then scatters.

- \textit{Spells}. While wielding this staff, you can use two actions to spend its charges and cast one of the following spells: giant bug (4 charges, DC 19) or bug plague (5 charges, DC 21).

- \textit{Cloud of Insects}. While wielding this staff, you can use two actions and spend 1 charge to cause a swarm of harmless insects to spread within a 30-foot radius around you. The bugs remain for 10 minutes, making the area heavily darkened for everyone but you. The swarm moves with you, staying centered on you. A wind of at least 15 kilometers per hour disperses the swarm and ends the effect.

\index[OggettiMagici]{Staff of the Python} \subsubsection*{Staff of the Python}
2000 gp, uncommon, you can use two actions to say the command word of the staff and hurl it on the ground up to 10 feet away. The staff becomes a giant constricting snake under your control and acts at its own initiative count. Using two actions to say the command word again, return the staff to its normal shape in the space previously occupied by the snake.

During your round, you can mentally command the snake as long as it is within 60 feet of you and you are not incapacitated. You decide what actions the snake will take and where it will move during its next round, or you can give it a generic command, such as attack your enemies or defend a place. If the snake is reduced to 0 hit points, it dies and reverts to its staff form. Then, the stick shatters and is destroyed. If the snake transforms back into stick form before losing all of its hit points, it regains all of its lost hit points.

\index[Magic Items]{Staff of Power} \subsubsection*{Staff of Power}
150,000 gp, legendary, this staff can be wielded as a magical fighting staff that grants a +2 bonus on attack and damage rolls made with it. While wielding it, you receive a +2 bonus on defense, saving throws, and attack rolls with spells. This staff has 20 charges for the following properties. Recover 1d8 + 1 stacks spent every day at dawn. If you spend the last charge of the stick, roll 1d6 if you get 1 or less the stick retains its +2 bonus on attack and damage rolls but loses all other properties.

- \textit{Power Strike}. When you hit with a melee attack using this staff, you can expend 1 charge to deal an additional 1d6 points of force damage to the target.

- \textit{Spells}. While wielding this staff, you can use two actions to spend 1 or more of its charges to cast one of the following spells with it: block monster (5 charges DC 21), cone of cold (5 charges, DC 21), globe of invulnerability ( 6 charges, DC 22), levitation (2 charges DC 15), wall of force (5 charges, DC 21), fireball (3 charges DC 17), magic missile (1 charge), beam of enfeeblement (1 DC charge 11) or lightning (3 DC 17 charges).

- \textit{Strike of Vengeance}. You can use two actions to snap the stick on your knee or against a solid surface, performing a vengeance strike. The staff is destroyed and releases its remaining magic in an explosion that expands to fill a 30-foot-radius sphere centered on it.

You have a 50 \% chance to instantly travel to a random plane of existence, thus avoiding the explosion. If you fail to avoid the effect, you take Force damage equal to 16 x the number of stacks on the staff. Every other creature in the area must make a DC 27 Reflex saving throw. If the saving throw fails, the creature takes an amount of damage based on the distance from the point of origin of the explosion, as shown on the following table.

If the saving throw is successful, the creature takes half this damage.

\medskip

\begin{tabularx}{0.45\textwidth}{Xl}
\hline
\textbf{Distance from the origin} & \textbf{Damage} \\
3 meters or less & 8 x charges in the stick \\
Up to 6 meters & 6 x charges in the stick \\
Up to 9 meters & 4 x charges in the stick \\
\end{tabularx}

\medskip

Note: The Staff of the Archimage and Power are similar, this is because they are prepared by two bitter enemies who wanted to create the most powerful Staff.

\index[Magic Items]{Staff of Thunder and Lightning} \subsubsection*{Staff of Thunder and Lightning}
10,000 gp, very rare, the staff can be wielded as a magical fighting staff which grants a +2 bonus to attack and damage rolls made with it. It also has the following properties. When one of these properties is used, it cannot be used again until the next sunrise.

- \textit{Lightning}. When you hit with a melee attack using the staff, you can cause the target to take an additional 2d6 points of lightning damage.

- \textit{Thunder}. When you strike with a melee attack using the staff, you can cause the staff to emit the sound of thunder, audible up to 90 meters away. The hit target must succeed on a DC 21 Fortitude save or be stunned until the end of your next round.

- \textit{Lightning Strike}. You can use two actions to make a lightning bolt leap from the tip of the stick into a line 1.5 meters wide and 36 meters long. Each creature on the line must make a DC 21 Reflex saving throw, taking 9d6 points of lightning damage if it fails, or half that damage if it succeeds.

- \textit{Rumble of Thunder}. You can use two actions to cause the staff to produce a deafening thunder rumble, audible up to 180 meters away. Each creature within 60 feet of you (excluding you) must make a DC 21 Fortitude save. If the saving throw fails, the creature takes 2d6 points of sonic damage and is deafened for 1 minute. On a successful save, she takes half damage and is not deafened.

- \textit{Thunder and Lightning}. You can use two Actions to use the Lightning Strike and Roar of Thunder properties together. Doing so does not consume the daily use of those properties, only the use of this one.

\index[Magic Items]{Staff of Sorcery} \subsubsection*{Staff of Sorcery}
85,000 gp, very rare, in combat, this staff functions as a +1 staff. Can be used to cast elemental summons, invisibility, bulkhead and cobweb. The cane can be used as a paralysis wand. Each of these powers requires a charge. It is possible to break the stick to produce a "final blow", the effect of which depends on the number of remaining charges. The staff explodes in a large sphere of flames, striking all creatures within 30 feet (including the owner of the staff) and inflicting 8 wounds per charge remaining, Fortitude save DC 27 for half.

\index[Magical Objects]{Rod of Charm} \subsubsection*{Rod of Charm}
28000 gp, rare, for 1 charge, the wielder can cast dominate beasts, with 2 dominate people charges, and with 3 dominate monsters charges.

\index[OggettiMagici]{Rod of Absorption} \subsubsection*{Rod of Absorption}
50,000 gp, very rare, while wielding this rod, you can use an Action to absorb a spell that targets only you and has no area of effect. The absorbed spell effect is canceled, and the spell's energy (not the spell itself) is absorbed by the rod. In the course of its existence the rod can absorb and contain up to a sum of 31 Levels of spells. Once the rod has absorbed 8 spells (max level 4), it will no longer be able to absorb. If you are the target of a spell that the rod cannot contain, the rod has no effect on the spell. When you pick up the rod, you know how many spells the rod has absorbed so far. If you are a spellcaster and you hold the rod, you can convert all the energy contained for 10 more Magic Points.

\index[OggettiMagici]{Immovable Rod} \subsubsection*{Immovable Rod}
5,000 gp, uncommon, this flat iron rod has a button on one end. You can use two actions to press the button, which causes the rod to magically remain fixed in place. As long as you or another creature use two actions to press the button again, the rod won't move, even if it were to defy gravity. The rod can support up to 4000 pounds of weight. A heavier weight causes the rod to deactivate and fall. A creature can use two actions to make a DC 30 Strength check, moving the rod 10 feet on success.

\index[Magic Items]{Rod of Mighty Strike} \subsubsection*{Rod of Mighty Strike}
30,000 gp, very rare, a mighty strike rod deals 1d8 + 3 wounds, and functions as a +3 magic light mace. When the rod is used against golems, it consumes 1 charge per hit, and inflicts 2d8 + 6 wounds. Note that when the rod is used as a weapon against a golem, a critical attack roll instantly annihilates it. In addition, this rod inflicts additional wounds on fiends and undead. When attacking these monsters, a critical attack roll consumes 1 charge, and the rod inflicts triple the number of wounds.

\index[Magic Items]{Rod of Sovereign Force} \subsubsection*{Rod of Sovereign Force}
50,000 gp, legendary, this rod has a flanged head, and functions as a magic club that grants a +3 bonus to attack and damage rolls made with it. The rod has properties associated with the six different buttons that are arranged along the handle. It also has three other properties described below.

\textbf{Six Buttons}. You can press one of the six buttons on the rod with two actions. The button's effect lasts until you press a different button or press the same button again, returning the rod to its normal shape.

- If you press \textit{button 1}, the rod becomes a tongue of fire weapon, and a fiery blade protrudes from the opposite end of the flanged head.

- If you press \textit{button 2}, the rod's flanged head folds up and two crescent blades pop out, transforming the rod into a magical battle ax that grants a +3 bonus on attack and damage rolls made with it.

- If you press the \textit{button 3}, the flanged head of the rod folds up, and a spearhead comes out of the end of the rod, while the handle extends up to 1.8 meters, turning the rod into a magic spear that grants a +3 bonus to attack and damage rolls made with it.

- If you press \textit{button 4}, the rod turns into a climbing pole up to 15 meters long, as requested by you. On hard surfaces such as granite, one spike on the bottom and three on the top hold the rod fixed in place. 7.5cm long horizontal bars unwind along the sides of the rod, 30cm apart, to form a ladder. The rod can support 2000 kilos. A heavier weight or lack of a solid anchor causes the rod to return to its normal shape.

- If you press \textit{button 5}, the rod transforms into a battering ram and grants its user a +10 bonus on Strength checks made to break through doors, barricades or other barriers.

- If you press \textit{button 6}, the rod assumes or remains in its normal form and indicates magnetic north (nothing happens if this rod function is used in areas without magnetic north). The rod also gives you an approximate knowledge of the depth underground and your height above sea level.

\textit{Draining Life}. When you hit a creature with a melee attack using the rod, you can force the target to make a DC 21 Fortitude save. If it fails, the target takes an additional 4d6 Void damage and is removed from its maximum hit points, and you recover a number of hit points equal to half the Void damage dealt. Once used, this property will no longer be used until dawn the next day.

\textbf{Paralyze}. When you hit a creature with a melee attack using the rod, you can force the target to make a DC 21 Fortitude save. If it fails, the target is paralyzed for 1 minute. The target can re-roll the saving throw at the end of each of his rounds, ending the effect on himself if he succeeds. Once used, this property cannot be used again until dawn the next day.

\textit{Terrify}. While wielding this rod, you can force any creature you see within 30 feet of you to make a DC 21 Will save. If it fails, the target is scared of you for 1 minute. The frightened target can re-roll the saving throw at the end of each of his rounds, ending the effect on himself if he succeeds. Once used, this property cannot be used again until dawn the next day.

This rod cannot be reloaded. When the charges run out, one remains

\index[OggettiMagici]{Rod of Readiness} \subsubsection*{Rod of Readiness}
25,000 gp, very rare, this flanged-headed rod has the following properties.

\textit{Readiness}. While wielding this rod, you have +2 on Wisdom checks and initiative rolls.

\textit{Spells}. While holding this rod, you can use two actions to cast one of the following spells with it: detect good and evil, detect magic, detect poison and disease, or see invisibility.

\textit{Protective Aura}. With two actions, you can plant the pointed end of the rod into the ground. At that point the rod head will radiate bright light within a radius of 18 meters and dim light for an additional 18 meters. Within this bright light, you and any creature friendly to you will gain a +1 bonus to Defense and saving throws, and you can sense the location of any hostile invisible creature that is also within the bright light. The rod head stops glowing and ends after 10 minutes, or when a creature uses two actions to pull the rod out of the ground. This property cannot be used again until dawn the next day.

\index[OggettiMagici]{Rod of Security} \subsubsection*{Rod of Security}
90000 gp, very rare, while wielding this rod, you can use two actions to activate it. As a result, the rod carries you and up to 199 other visible consenting creatures into a paradise located in extraplanar space. You will be the one to choose the shape of this paradise. It could be a peaceful garden, a pleasant clearing, a cheerful tavern, a huge palace, a tropical island, or a fantastic fair or anything else you can imagine. Whatever its nature, paradise contains enough food and drink to feed its visitors. Everything with which one can interact in extraplanar space can only exist inside it.

For every hour spent in this paradise, a visitor regains hit points as if he had rested for a night. Furthermore, as long as creatures remain in heaven they do not age, although time passes normally. Visitors can stay in paradise for up to 200 days divided by the number of creatures present (round down).

When time runs out or you use two actions to end it, all visitors reappear in the place they occupied when you activated the rod, or in the unoccupied space closest to that. The rod cannot be used again until ten days have passed.

\index[OggettiMagici]{Rod of Sovereignty} \subsubsection*{Rod of Sovereignty}
16,000 gp, rare, you can use two actions and present the rod and request obedience from each visible creature within 36 meters of you of your choice. Each target must succeed on a DC 17 Will save or be fascinated by you for 8 hours. While fascinated in this way, the creature regards you as a trusted leader. If it is harmed by you or your companions, or ordered to do something contrary to its nature, the target will stop being fascinated in this way. The rod cannot be used again before the next sunrise.

\index[OggettiMagici]{Tentacular Rod} \subsubsection*{Tentacular Rod}
5,000 gp, rare, this rod is a magical weapon that ends in three leather tentacles. While wielding the rod, you can use two actions to direct each tentacle to attack a visible creature within 15 feet of you. Each tentacle makes a melee attack roll with a +9 bonus. If you hit, the tentacle deals 1d6 hit damage. If you hit a target with all three tentacles, it must make a DC 15 Fortitude save. If it fails, the creature's speed is halved, it has -1d6 on Reflex saving throws, and for 1 minute it can't use. his reactions. Also, during each of his rounds, he may take two actions or two actions but not both. The target can re-roll the saving throw at the end of each of his rounds, ending the effect on himself if he succeeds.


\subsection{Potions - Oils}

\index[OggettiMagici]{Potion of Friendship with Animals} \subsubsection*{Potion of Friendship with Animals}

uncommon, 200 gp, when you drink this potion, you can cast the Animal Friendship spell (saving throw DC 15) at will for 1 hour.

\index[Magic Items]{Potion of Climbing} \subsubsection*{Potion of Climbing}

common, 250 gp, when you drink this potion, you gain climbing speed equal to your walking speed for 1 hour. During this time you have + 1d6 to the Endurance checks you make to make a climb.

\subsubsection*{Potion of animal Clairaudience} \index[OggettiMagici]{Potion of Animal Clairaudience}

uncommon, 500 gp, this potion grants its drinker the ability to hear sounds through the ears of an animal within 60 feet. A lead barrier blocks this effect.

\subsubsection*{Potion of Animal Clairvoyance} \index[Magic Items]{Potion of Animal Clairvoyance}
uncommon, 500 gp, this potion grants its drinker the ability to see through the eyes of an animal within 60 feet. A lead barrier blocks this effect.

\subsubsection*{Potion of Animal Control} \index[Magic Items]{Potion of Animal Control}
rare 1500 gp, anyone who drinks this stance is as good as casting Dominate Beasts


\subsubsection*{Dragon Control Potion} \index[Magic Items]{Dragon Control Potion}
legendary, 5,000 gp, this potion grants power equivalent to the dominate monster spell on a single type of dragon. It is possible to control a dragon within 60 feet for 5d4 rounds.

\subsubsection*{Potion of Undead Control} \index[Magic Items]{Potion of Undead Control}
2500 gp, rare, although undead are normally immune to this type of effect, this potion allows the drinker to affect 3d6 HD of undead (intelligent or not) as if using the charm spell. The duration of the effect is 5d4 rounds.

\subsubsection*{Potion of People Control} \index[Magic Items]{Potion of People Control}
500 gp, uncommon, once ingested, this potion grants the drinker a power similar to the charm spell.

\subsubsection*{Plant Control Potion} \index[OggettiMagici]{Plant Control Potion}
1500 gp, rare, the drinker of this potion is able to control all plants and plant creatures (including fungi) in a square area of 6x6m and within a distance of 27 meters. The effect lasts for 5d4 rounds. Plants obey according to their possibilities (for example, lianas can twist and thicken, causing slowness or obstruction of vision). It is possible to give orders to sentient plant creatures, but they are entitled to a DC 19 Will saving throw. As with other types of enchantment, a controlled creature cannot be ordered to harm itself.

\index[Magic Items]{Potion of Growth} \subsubsection*{Potion of Growth}
300 gp, uncommon, when you drink this potion, you gain the "enlarge" effect of the enlarge / reduce spell for 1d4 hours (no concentration required).

\index[Magic Items]{Potion of Heroism} \subsubsection*{Potion of Heroism}
200 gp, rare, when you drink this potion, you gain 10 temporary hit points that last 1 hour. For the same duration you are under the effect of the blessing spell (requires no concentration).

\index[Magic Items]{Gaseous Form Potion} \subsubsection*{Gaseous Form Potion}

1500 gp, rare, when you drink this potion, for 1 hour or until you end the effect with two actions, you gain the effect of the gaseous form spell (no concentration required).

\index[Magic Items]{Potion of Strength of the Giants} \subsubsection*{Potion of Strength of the Giants}
Rarity varies, cost varies, when you drink this potion, your Strength score changes for 1 hour. The type of giant determines the score (see the table below). The potion has no effect if your Strength score equals or exceeds the new score. The Frost Giant's Potion of Strength and the Stone Giant's Potion of Strength have the same effect.

- of the hills, Strength 5, Uncommon 500 gp

- stone or frost, Strength 6, Rare 1000 gp

- of fire, Strength 7, Rare 2000 gp

- of the clouds, Strength 8, Very Rare 5,000 gp

- of storms, Strength 9, Legendary 10,000 gp

\index[Magic Items]{Healing Potion} \subsubsection*{Healing Potion}
Rarity varies, cost varies, when you drink from this potion, you recover a number of hit points that vary depending on the rarity of the healing potion.

- Common, 2d4 + 2 hit points, 75 gp

- Major, hit points 4d4 + 4, 150 gp

\index[Magic Items]{Healing Potion} \subsubsection*{Major Healing Potion}
Rarity varies, cost varies, when you drink from this potion, you recover a number of hit points that vary depending on the rarity of the healing potion.

- Superior, hit points 8d4 + 8, 350 gp

- Ultimate, hit points 10d4 + 20, 1500 gp

\subsubsection*{Potion of Deception} \index[Magic Items]{Potion of Deception}

500 gp, rare, this potion has a very fitting name, convincing the drinker that they have ingested a potion of another type. For example, a fake "clairaudience potion" could make the drinker hear sounds that don't really exist. If several people taste this potion, there is a 90 \% chance that they will agree that it is of the same type.

\index[Magic Items]{Potion of Invisibility} \subsubsection*{Potion of Invisibility}
200 gp, very rare, when you drink this potion, you become invisible for 1 hour. While you are invisible, everything you carry or wear also remains invisible with you. The effect ends when you attack or cast a spell.

\subsubsection*{Potion of Invulnerability} \index[Magic Items]{Potion of Invulnerability}
800 gp, rare, a potion of invulnerability grants the drinker a +2 bonus on saving throws and a 2-point improvement in Defense.

\index[OggettiMagici]{Potion of Reading Thought} \subsubsection*{Potion of Reading Thought}
200 gp, rare, when you drink this potion, you gain the effect of the detect thoughts spell (saving throw DC 15).

\subsubsection*{Potion of Levitation} \index[Magic Items]{Potion of Levitation}
200 gp, uncommon, this potion has the same effect as the levitation spell.

\subsubsection*{Potion of Longevity} \index[Magic Items]{Potion of Longevity}
15,000 gp, legendary, this potion rejuvenates by 1d12 years. Regained youth not only cancels natural aging, but also aging caused by magical effects or creatures. There is a danger in using this potion, as each time you drink a potion of longevity, there is a 1 \% cumulative chance that all previously earned benefits with potions of this type will be nullified. It is not possible to consume a partial dose of this potion.

\subsubsection*{Potion of Morph} \index[OggettiMagici]{Potion of Morph}
2,500 gp, rare, this potion confers power similar to the morph spell.

\index[Magic Items]{Potion of Fortitude} \subsubsection*{Potion of Fortitude}
300 gp, uncommon, when you drink this potion, you gain resistance to one type of damage for 1 hour. The Storyteller chooses the type of damage or determines it randomly (Acid, Cold, Fire, Strength, Lightning, Void, Poison, Light, Sound)

\index[Magic Items]{Potion of Breathing Underwater} \subsubsection*{Potion of Breathing Underwater} \textit{Potion, uncommon} 200 gp

After drinking this potion, you can breathe underwater for 1 hour.

\index[Magic Items]{Shrink Potion} \subsubsection*{Shrink Potion}
300 gp, rare, when you drink this potion, you gain the "reduce" effect of the enlarge / reduce spell for 1d4 hours (no concentration required).

\index[Magic Items]{Potion of Poison} \subsubsection*{Potion of Poison}
250 gp, uncommon, this spirit resembles, smells, and tastes like a healing potion or other beneficial potion. However it is actually a poison disguised as illusion spells. The identify spell reveals its true nature.

If you drink it, you take 3d6 points of poison damage, and you must succeed on a DC 13 Fortitude save or be poisoned one more round and take 1d6 points of damage at the start of the next round.

\index[Magic Items]{Poison Potion} \subsubsection*{Greater Poison Potion}
450 gp, uncommon, this spirit resembles, smells, and tastes like a healing potion or other beneficial potion. However it is actually a poison disguised as illusion spells. If identified, the true nature is understood.

If you drink it, you take 5d6 points of poison damage, and you must succeed on a DC 18 Fortitude save or become poisoned. At the beginning of each of your rounds, while you are poisoned in this way, you take 2d6 points of poison damage. You can re-roll the saving throw at the end of each of your rounds. If the saving throw is successful, the poison damage taken in subsequent turns decreases by 1d6. The poison ceases its effects when the damage drops to 0d6.

\index[Magic Items]{Potion of Speed} \subsubsection*{Potion of Speed}
400 gp, very rare, when you drink this potion, you gain the effect of the speed spell for 1 minute (requires no concentration).

\index[OggettiMagici]{Potion of Flight} \subsubsection*{Potion of Flight}
500 gp, very rare, when you drink this potion, for 1 hour you gain flight speed equal to your normal walking speed and can float. If the potion ends while you are flying, you will fall unless you have some other method of staying in the air.

\index[OggettiMagici]{Filter of Love} \subsubsection*{Filter of Love}
120 gp, uncommon, you will be fascinated for 1 hour by the first creature you see within 10 minutes of drinking this filter. If the creature is of a species or genus that you are normally attracted to, as long as you are fascinated you will consider it your one and only great love.

\subsubsection*{Discoverers filter} \index[OggettiMagici]{Discoverers filter}
500 gp, rare, whoever drinks this potion can perceive treasures containing precious metals or gems within 72 meters, as long as they are worth at least 50 gold. The direction of the treasure can be perceived, but not its exact distance. No non-magical barrier can prevent you from perceiving treasures, except a lead plate.

\index[OggettiMagici]{Sharpness Oil} \subsubsection*{Sharpness Oil}
3200 gp, very rare, this oil can coat a slashing or piercing weapon or up to 5 slashing or piercing ammo. Applying the oil takes 1 minute. For 1 hour, the oil-coated weapon is magical and has a +3 bonus on attack and damage rolls.

\index[OggettiMagici]{Oil of Ethereal Form} \subsubsection*{Oil of Ethereal Form}
2,000 gp, rare, a dose of oil is enough to coat a creature of Medium or smaller size, and the equipment it wears and carries (an additional vial is required for each size category above Medium). Applying the oil takes 10 minutes. After that, the creature gains the effect of the ethereal form spell for 1 hour.

\index[OggettiMagici]{Slippery Oil} \subsubsection*{Slippery Oil}
500 gp, uncommon, oil can cover a creature of Medium or smaller size, along with all equipment it wears or carries (an additional vial is required for each size category above Medium). Applying the oil takes 10 minutes. The creature then gains the benefit of the freedom of movement spell for 8 hours. Alternatively, you can pour oil on the ground with two actions, doubling the effect of the anointed spell on that area for 8 hours.


\subsection{Rings}

\index[Magic Items]{Spell-Collecting Ring} \subsubsection*{Spell-Collecting Ring}
24,000 gp, rare, this ring stores spells cast on it, retaining them until the wearer uses them. The ring can accumulate up to 3 Spells for a total of 17 Magic Points with a maximum of 6 single Magic Points.

Any creature can cast an accumulated spell level 1 to 5 on the ring by touching it. The spell has a DC equal to 10 + 2 x Spell Level, any attack roll is made by the caster.

The cast caster must aim for the ring to absorb it. If the ring cannot contain the spell, the spell manifests normally. A spell cast via this ring is no longer contained within it, and frees up space for other spells.

\index[OggettiMagici]{Ring of Aries} \subsubsection*{Ring of Aries}
5000 gp, rare, while wearing this ring, you can use two actions to spend 1 to 3 charges to attack a visible creature within 60 feet of you.

The ring produces a ghostly ram's head and makes its attack roll with a +7 bonus. If it hits, for each charge spent, the target takes 2d10 points of force damage and is pushed 5 feet away from you.

Alternatively, you can spend 1 to 3 charges of the ring with two actions to attempt to break a visible object within 60 feet of you that is not being worn or carried. The ring makes a +5 Strength check for each charge spent.

This ring has 3 stacks, and recovers 1d3 stacks spent every morning at dawn.

\index[OggettiMagici]{Feather Falling Ring} \subsubsection*{Feather Falling Ring}
2000 gp, rare, if you fall more than 1 meter and wear this ring, the Feather Fall spell is activated

\index[OggettiMagici]{Waterwalk Ring} \subsubsection*{Waterwalk Ring}
1500 gp, uncommon, while wearing this ring, you can stand or move on any liquid surface as if it were solid ground.

\index[OggettiMagici]{Ring of Heat} \subsubsection*{Ring of Heat}
5000 gp, uncommon, while wearing this ring, you have resistance to cold damage. Plus, you and everything you wear and carry are immune to the effects of temperatures as low as -45 ° C.

\index[OggettiMagici]{Ring of Water Elementals} \subsubsection*{Ring of Water Elementals}
250,000 gp, legendary. this ring is connected to the Elemental Plane of Water. While wearing it, you have + 1d6 to attack rolls against elementals on the Elemental Plane of Water, and they have -1d6 to attack rolls made against you.

You can spend 2 charges of the ring to cast dominate monsters on a water elemental. Also, you can stand and walk on liquid surfaces as if they were solid ground. You can speak and understand Aquan.

If you help kill a water elemental while wearing the ring, you gain access to the following additional properties:

\medskip

\begin{itemize}
\item
You can breathe underwater and have speed equal to your walking speed again.
\item
You can cast the following spells through the ring, spending the required number of charges: create or destroy water (1 charge), control weather (3 charges), wall of ice (3 charges) or ice storm (2 charges).
The ring has 5 charges. Recover 1d4 + 1 stacks every day at dawn. Spells cast through the ring have a saving throw DC of 21.

\end{itemize}

\index[Magic Items]{Ring of Air Elementals} \subsubsection*{Ring of Air Elementals}
250,000 gp, legendary, this ring is linked to the Elemental Plane of Air. While wearing it, you have + 1d6 to attack rolls against elementals on the Elemental Plane of Air, and they have -1d6 to attack rolls made against you.

You can spend 2 charges of the ring to cast dominate monsters on an air elemental. Also, when you fall, you drop 60 feet per round and take no damage from the fall. You can speak and understand Auran.

If you help kill an air elemental while wearing the ring, you gain access to the following additional properties:

\medskip

\begin{itemize}
\item
You have resistance to lightning damage.
\item
You have flight speed equal to your walking speed and can float.
\item
You can cast the following spells through the ring, spending the required number of charges: chain of lightning (3 charges), gust of wind (2 charges) or wall of wind (1 charge).
\end{itemize}
\medskip

The ring has 5 charges. Recover 1d4 + 1 stacks every day at dawn.

Spells cast through the ring have a saving throw DC of 21.


\index[OggettiMagici]{Ring of Fire Elementals} \subsubsection*{Ring of Fire Elementals}
250,000 gp, legendary, this ring is linked to the Elemental Plane of Fire. While wearing it, you have + 1d6 attack rolls against elementals on the elemental plane of fire, and they have -1d6 attack rolls made against you.

You can spend 2 charges of the ring to cast dominate monsters on a fire elemental. Additionally, you have resistance to fire damage. You can speak and understand the Ignan.

If you help kill a fire elemental while wearing the ring, you gain access to the following additional properties:

\medskip

\begin{itemize}
\item
You have immunity to fire damage.
\item
You can cast the following spells through the ring, spending the required number of charges: burning hands (1 charge), wall of fire (3 charges) or fireball (2 charges).
\end{itemize}

\medskip

The ring has 5 charges. Recover 1d4 + 1 stacks every day at dawn.

Spells cast through the ring have a saving throw DC of 21.

\index[OggettiMagici]{Ring of Earth Elementals} \subsubsection*{Ring of Earth Elementals}
250,000 gp, legendary, this ring is linked to the Elemental Plane of Earth. While wearing it, you have + 1d6 to attack rolls against elementals on the elemental plane of Earth, and they have -1d6 to attack rolls made against you.

You can spend 2 charges of the ring to cast dominate monsters on an earth elemental. In addition, you can move on difficult terrain consisting of rubble, stones or dirt as if it were normal terrain. You can speak and understand Terran.

If you help kill an earth elemental while wearing the ring, you gain access to the following additional properties:

\medskip

\begin{itemize}
\item
You have resistance to acid damage.
\item
You can move through dirt or solid rock as if they were difficult terrain. If you finish your round, you are thrown out into the closest unoccupied space you last occupied.
\item
You can cast the following spells through the ring, spending the required number of charges: sculpt stone (2 charges), stone wall (3 charges), or skin of stone (1 charge).
\end{itemize}

\medskip

The ring has 5 charges. Recover 1d4 + 1 stacks every day at dawn.

Spells cast through the ring have a saving throw DC of 21.

\subsubsection*{Ring of People Control} \index[OggettiMagici]{Ring of People Control}
2,500 gp, rare, this ring grants the wearer the ability to use the charm spell once a day. The effect lasts until the controlling player ends it, 1 hour passes, or dispel magic is used.

\subsubsection*{Plant Control Ring} \index[OggettiMagici]{Plant Control Ring}
5000 gp, very rare, the wearer of this ring can control plants and plant creatures in a 3x3m square area within 60 feet. Even if a plant is immobile, it can move while under the effect of this ring. Control lasts as long as the person exercising it maintains total concentration, which prevents any other action.

\subsubsection*{Ring of Weakness} \index[OggettiMagici]{Ring of Weakness}
rare, once worn, this ring can only be removed to remove curse. Over the course of 6 rounds, the wearer's strength is reduced to -3.

\index[OggettiMagici]{Ring of the Three Wishes} \subsubsection*{Ring of the Three Wishes}
75,000 gp, legendary, while wearing this ring, you can use two actions to spend 1 of its 1d3 charges to cast the wish spell through it. The ring loses its magic when you use the last charge.

\index[OggettiMagici]{Circumvention Ring} \subsubsection*{Circumvention Ring}
5,000 gp, rare, while wearing this ring and failing a Reflex saving throw, you can use your Reaction action to spend 1 charge to succeed at the saving throw you just failed. This ring has 3 stacks, and recovers 1d3 stacks spent every morning at dawn.

\index[OggettiMagici]{Ring of Summoning the Djinni} \subsubsection*{Ring of Summoning the Djinni}
35,000 gp, legendary, while wearing this ring, you can utter its command word with two actions to summon a specific djinni of the Elemental Plane of Air. The djinni appears in an unoccupied space of your choice, within 36 meters of you. Stay as long as you stay focused (as if focusing on a spell), for up to 1 hour, or until it drops to 0 hit points. Then he returns to his home plane.

As long as it is summoned, the djinni is friendly towards you and your companions. He obeys any command you give him, no matter the language used. If you do not give him orders, the djinni will defend itself against attacks but will not take any other action.

After the djinni leaves, it can no longer be summoned before 24 hours have passed, and if the djinni dies the ring loses its magic.

\index[OggettiMagici]{Ring of Influence on Animals} \subsubsection*{Ring of Influence on Animals} \textit{Ring, rare} 4000 gp

While wearing this ring, you can use two actions to spend 1 of its charges to cast one of the following spells with it: friendship with animals (saving throw DC 15), talking to animals, fear (saving throw DC 15, takes only target beasts that have Intelligence -2 or less).

This ring has 3 stacks, and recovers 1d3 stacks spent every day at dawn.

\subsubsection*{Ring of Deception} \index[OggettiMagici]{Ring of Deception}
rare, the wearer of this cursed ring is convinced that it has a power chosen by the Storyteller or determined at random.

\index[OggettiMagici]{Invisibility Ring} \subsubsection*{Invisibility Ring}
10,000 gp, very rare, while wearing this ring, you can make yourself invisible with two actions. Everything you wear or carry becomes invisible with you. You remain invisible until the ring is removed, attack or cast a spell, or until you use two actions to become visible.

\index[OggettiMagici]{Ring of Freedom of Action} \subsubsection*{Ring of Freedom of Action}
20,000 gp, rare, while wearing this ring, hindering terrain costs you no additional movement. Additionally, magic can neither reduce your speed nor make you paralyzed or hindered.

\index[OggettiMagici]{Swimming Ring} \subsubsection*{Swimming Ring}
3000 gp, uncommon, while wearing this ring, you have swimming speed 12 meters.

\index[OggettiMagici]{Ring of Protection} \subsubsection*{Ring of Protection}
cost varies, rarity varies, while wearing this ring, you have a +1 (5,000 gp, rare), +2 (7,500 gp, rare), +3 (12,000 gp, very rare) bonus to Defense and saving throws.

\index[Magic Items]{Ring Spell Repel} \subsubsection*{Ring Spell Repel}
35,000 gp, legendary, while wearing this ring, you have + 1d6 on saving throws against any spell that targets only you and not an area of effect. Also, if you make a critical save success and the spell is level 6 or lower, the spell has no effect on you and instead targets the caster who cast the spell.

\index[OggettiMagici]{Ring of Regeneration} \subsubsection*{Ring of Regeneration}
12000 gp, very rare, while wearing this ring, you regain 1d6 hit points every 10 minutes, as long as you have at least 1 hit point left. If you lose a part of the body, the ring causes the missing part to grow back and return to its full functionality in 1d6 + 1 days, as long as you always have at least 1 hit point left throughout the period.

\index[OggettiMagici]{Ring of Resistance} \subsubsection*{Ring of Resistance}
6000 gp, rare, while wearing this ring, you have resistance to one type of damage. The gem set in the ring indicates the type of damage, which is chosen or randomly determined by the Storyteller.

\medskip

\begin{tabular}{lll}
\textbf{d10} & \textbf{Damage Type} & \textbf{Gem} \\

\hline
1 & Acid & Pearl \\
2 & Strength & Sapphire \\
3 & Cold & Tourmaline \\
4 & Lightning & Citrine \\
5 & Fire & Garnet \\
6 & Vuoto & Jet \\
7 & Positive Energy & Jade \\
8 & Light & Topaz \\
9 & Sound & Spinel \\
10 & Negative Energy & Amethyst \\
\end{tabular}

\medskip

\index[OggettiMagici]{Ring of the Jump} \subsubsection*{Ring of the Jump}
2500 gp, uncommon, while wearing this ring, you can cast the jump spell at will with two actions, but only you can be the target.

\index[Magic Items]{Mental Shield Ring} \subsubsection*{Mental Shield Ring}
16,000 gp, uncommon, while wearing this ring, you are immune to magic that allows other creatures to read your thoughts, determine if you are lying, know your traits, or learn what kind of creature you are. Creatures can only telepathically communicate with you if you allow them to.

You can use two actions to make the ring invisible until another action makes it visible again, until you remove it or die. If you die while wearing this ring, your soul is captured in it, unless you already harbor another soul. You can decide to stay in the ring or reach for the afterlife. As long as your soul remains in the ring, you can telepathically communicate with any creature wearing it. The wearer cannot prevent this form of telepathic communication.

\index[OggettiMagici]{Ring of Shooting Stars} \subsubsection*{Ring of Shooting Stars}
14,000 gp, very rare, while wearing this ring in dim light or in darkness, you can cast dancing lights and light at will through it. Casting either spell through the ring requires two actions. The ring has 6 charges for the following other properties.

The ring recovers 1d6 stacks spent each day at dawn.

\textit{Luminescence}. Spend 1 charge with two actions to cast the glow spell through the ring.

\textit{Ball of Lightning}. You can spend 2 charges with two actions to create one to four lightning balls 1 meter in diameter. The more spheres you create, the less powerful each sphere will be individually.
Each sphere appears in an unoccupied space visible within 36 meters of you. The sphere lasts as long as you focus on it (as if focusing on a spell), up to a maximum of 1 minute. Each sphere radiates dim light within a 30-foot radius. With two actions you can move each sphere up to 9 meters, but without exceeding 36 meters away from you. When a creature other than you is within 5 feet of a sphere, the sphere discharges lightning at that creature and then disappears. That creature must make a DC 18 Reflex saving throw. If the saving throw fails, the creature takes lightning damage based on the number of orbs you create (4 orbs, 2d4 damage; 3 orbs, 2d6 damage; 2 orbs, 5d4 damage; 1 sphere, 4d12 damage).

\textit{Falling Stars}. You can spend 1 to 3 charges with two actions. For each charge spent, cast a spark of light from the ring to a visible spot within 60 feet of you. Each creature in the 15-foot-wide cube originating from that point is covered in sparks and must make a DC 15 Dexterity saving throw, taking 5d4 points of fire damage if it fails, or half that damage if it succeeds. .

\index[OggettiMagici]{Telekinesis Ring} \subsubsection*{Telekinesis Ring}
80,000 gp, very rare, while wearing this ring, you can cast the telekinesis spell at will, but you can only target items that are not worn or carried.

\index[OggettiMagici]{Ring of X-Ray Sight} \subsubsection*{Ring of X-Ray Sight}
6,000 gp, rare, while wearing this ring, you can use two actions to speak its command word. When you do this, you can see through solid matter for 1 minute. This sight has a radius of 9 meters. To you, solid objects within the beam appear transparent and do not block light from passing through them.

This sight can penetrate 30cm of stone, 2.5cm of common metal, or up to 90cm of wood or earth. Thicker substances block vision, as does a thin sheet of lead. Whenever you use the ring again before ending a night's rest, you must succeed at a DC 18 Fortitude save or become fatigued.

\subsection{Hats, Cloaks, Glasses, Tunics}


\index[OggettiMagici]{Bandana of Intelligence} \subsubsection*{Bandana of Intelligence}
8000 gp, rare, while wearing this bandana your Intelligence is +4. The tie has no effect if you already have Intelligence is already +4 or higher.

\index[Magic Items]{Hat of Camouflage} \subsubsection*{Hat of Camouflage}
5000 gp, uncommon, while wearing this hat, you can use two actions to cast the disguise self spell at will. The spell ends when the hat is removed.

\index[OggettiMagici]{Arachnid Cloak} \subsubsection*{Arachnid Cloak}
8000 gp, very rare, while wearing this elegant black silk dress woven with silver thread, you get the following benefits:

\medskip

\begin{itemize}
\item
You have resistance to poison damage.
\item
You have climbing speed equal to your walking speed.
\item
You can move up, down and along vertical surfaces and upside down on ceilings, keeping your hands free.
\item
You cannot be caught by any sort of web and you move through the webs as if they were difficult terrain.
\item
You can use two actions to cast the web spell (saving throw DC 15). The web created by the spell fills twice its normal area. Once used, this Cloak property cannot be used again until the next sunrise.
\end{itemize}

\index[OggettiMagici]{Cloak of the Charlatan} \subsubsection*{Cloak of the Charlatan}
8,000 gp, rare, while wearing this cape that smells faintly of sulfur, you can use it to cast the dimension door spell with two actions. Ownership of this cape cannot be used again until dawn. When you disappear, you leave behind a cloud of smoke, and reappear at your destination within such a cloud of smoke. This smoke slightly obscures the space you left and the space where you reappear, and dissipates at the end of your next round. A light or stronger wind disperses the smoke.

\index[Magic Items]{Cloak of Distortion} \subsubsection*{Cloak of Distortion}
60,000 gp, rare, while wearing this cloak, it casts an illusion that makes you appear as if you are somewhere close to your real location, causing all creatures to have -1d6 attack rolls against you. If you take damage, the property ceases to function until the start of your next round. This property is suppressed while you are incapacitated, hindered or otherwise unable to move.

\index[Magic Items]{Cloak of the Elves} \subsubsection*{Cloak of the Elves}
5000 gp, uncommon, while wearing this cloak by pulling up the hood, Awareness checks made to notice you have -1d6, and you have + 1d6 on Dexterity checks made to hide. Pulling the hood up or down requires two actions.

\index[Magic Items]{Manta Cloak} \subsubsection*{Manta Cloak}
6000 gp, uncommon, while wearing this cloak with the hood pulled up, you can breathe underwater and have 60-foot swimming speeds. Pulling the cap up or down takes 1 action.

\index[Magic Items]{Bat Cloak} \subsubsection*{Bat Cloak}
6,000 gp, rare, while wearing this cloak, you have + 1d6 on Dexterity checks. In areas of dim light or darkness, you can grab the edges of the Cloak with both hands and use it to move at 12m flight speed. If you stop holding the edges of the Cloak while flying this way, you lose your flight speed. While wearing the Cloak in an area of dim light or darkness, you can use your action to morph on you, transforming yourself into a bat. When in bat form, keep your Intelligence, Wisdom, and Charisma scores. The Cloak cannot be used this way again until the next dawn.

\index[Magic Items]{Cloak of Protection} \subsubsection*{Cloak of Protection}
Rarity varies, cost varies, while wearing this cloak, you gain a bonus of +1 (uncommon, 3500 gp), + 2 (rare, 6000 gp), + 3 (very rare, 15000 gp) on Defense and saving throws.

\index[Magic Items]{Cloak of Spell Resistance} \subsubsection*{Cloak of Spell Resistance}
uncommon, 3000 gp, while wearing this cloak, you have +2 on saving throws against spells.


\subsubsection*{Cloak of Poisonousness} \index[Magic Items]{Cloak of Poisonousness}
rare, 4,000 gp, this cloak is usually made of wool, although it can also be leather. The garment can be manipulated safely, but it causes 5d6 points of poison damage as soon as worn. Each subsequent round, a DC 21 Fortitude save can be made to reduce the damage by 1d6 to a minimum of 1d6 damage remaining. The cloak can only be removed with a remove curse or wish spell.

\subsubsection*{Eyes of Petrification} \index[OggettiMagici]{Eyes of Petrification}
these two magical crystal lenses overlap the pupils of the eyes. When a creature puts these lenses on, it is immediately petrified without a saving throw. About a quarter of these objects (probability of 25 \%) allow the wearer to petrify them with their eyes, but in this case the victims are entitled to a saving throw. It is not possible to combine two types of magic lenses.

\index[OggettiMagici]{Fascinating Eyes} \subsubsection*{Fascinating Eyes}
3000 gp, uncommon, while wearing these crystal lenses in front of your eyes, you can spend 1 charge with two actions to cast the charm spell on people (saving throw DC 15) on a humanoid within 30 feet of you, as long as you and the target you can see. The lenses have 3 charges and recoup 1 charge of those spent every day at dawn.

\index[OggettiMagici]{Eagle's Eyes} \subsubsection*{Eagle's Eyes}
4500 gp, uncommon, while wearing these crystal lenses in front of your eyes, you have + 1d6 on sight-based Awareness checks. In clear conditions, you can distinguish details of even very distant creatures and objects as large as 50 centimeters.

\index[MagicObjects]{Eyes of Detail View} \subsubsection*{Eyes of Detail View}
2500 gp, uncommon, while wearing these crystal lenses in front of your eyes, you can see much better than normal up to a distance of 30 centimeters. You have + 1d6 on sight-based Awareness checks while searching an area or studying an object at close range.

\index[OggettiMagici]{Night Glasses} \subsubsection*{Night Glasses}
1500 gp, uncommon, while wearing these dark lenses, you have darkvision, with a range of 60 feet. If you already have darkvision, wearing these glasses increases its range by 60 feet.

\subsubsection*{Tunic of Mimicry} \index[OggettiMagici]{Tunic of Mimicry}
1500 gp, rare, when wearing this robe, a character immediately realizes its power. A camouflage tunic allows the character to blend in with the surrounding environment, whatever it is, and to hide. Has + 1d6 on Hide in Shadows checks. The wielder can assume the appearance of another humanoid at will, as per the Alter himself (Change Appearance) spell. In this case, the possessor's friends and those who know him very well are instinctively aware of his true identity.

\subsubsection*{Tunic of the Archmage} \index[OggettiMagici]{Tunic of the Archmage}
8000 gp, legendary, this seemingly normal suit can be yellow (01-45 on 1d100), gray (46-75), or black (76-00). Can only be worn by a spellcaster with Magical Proficiency 2 or higher. Grants the following bonuses:

- Defense 15

- +2 to saving throws against spells and magical items

\index[OggettiMagici]{Tunic of Shimmering Colors} \subsubsection*{Tunic of Shimmering Colors}
6000 gp, very rare, this robe has 3 stacks, and recovers 1d3 stacks spent every day at dawn. When wearing it, you can use two actions and spend 1 charge to make the garment produce a changing texture of dazzling colors until the end of your next round. During this time, the robe gives off bright light within a 30-foot radius and dim light for an additional 30 feet. Creatures that see you have -1d6 attack rolls against you. In addition, any creature under bright light that sees you when the robe's power is activated must succeed on a DC 17 Will save or be stunned until the effect ends.

\subsubsection*{Robe of Undermining} \index[Magic Items]{Robe of Undermining}
5,000 gp, rare, a robe of enfeeblement looks like a magical garment of another kind. As soon as a character wears it, his Strength and Intelligence drop to -3 and he loses the ability to cast spells. The robe can be removed easily, but to restore attributes, remove curse followed by healing.

\index[OggettiMagici]{Tunic of Eyes} \subsubsection*{Tunic of Eyes}
30,000 gp, rare, this robe is adorned with an eye design. While wearing it, you get the following benefits:

- The robe allows you to see in all directions and you have + 1d6 on sight-based Awareness checks.

- You have darkvision with a range of 36 meters.

- You can see invisible creatures and objects, as well as in the Ethereal Plane, up to a range of 36 meters.

The eyes of the robe cannot be closed or averted, and while wearing this robe it is never considered to be closed or averted.

The light spell cast on the robe or the daylight spell cast within 5 feet of the robe will blind you for 1 minute. At the end of each of your rounds, you can make a Fortitude save (DC 13 for light or DC 17 for daylight), ending the blinded condition if you pass it.

\index[OggettiMagici]{Tunic of Useful Objects} \subsubsection*{Tunic of Useful Objects}
300 gp, uncommon, while wearing this robe covered in patches of various shapes and colors, you can use two actions to detach one of the patches, making it the object or creature it represents. When the last patch is removed, the dressing gown becomes a regular garment. The robe has two of each of the following patches:

3-meter pole, Hemp rope (15 meters, rolled up), Lantern with protruding lens (full and lit), Dagger, Sack, Steel mirror.

Additionally, the robe has 4d4 other patches. The Storyteller chooses the patches or determines them at random, choosing from properties that are totally different from those already present.

Roll a d100 on the table below to discover the properties of the other 4d4 patches on the useful item robe.

\end{multicols}

\medskip
\begin{tabularx}{0.95\textwidth}{lX}
\textbf{d100} & \textbf{Effect} \\
\hline
01-08 & Bag with 100 gp. \\
09-15 & Silver chest (30cm long, 15cm wide and 15cm deep) worth 500 gp. \\
16-22 & Iron door (maximum 3 meters wide and high, barred from the side of your choice), which you can place on any opening within reach; it adapts to enter the opening, fixing itself and creating hinges. \\
23-30 & 10 gems worth 100 gp each. \\
31-44 & A wooden staircase (7.5 meters). \\
45-51 & A racehorse with saddlebags 52-59 Fossa (a 3 meter edge cube), which you can place on the ground within 3 meters of you. \\
60-68 & 4 healing potions. \\
69-75 & Rowing boat (3.5 meters long). \\
76-83 & Spell Scroll containing a spell level from 1st to 3rd. \\
84-90 & Two mastiffs. \\
91-96 & Window (60 x 120 cm, maximum depth 60 cm), which you can place on any vertical surface within reach. \\
97-100 & Portable Aries. \\
\end{tabularx}

\begin{multicols}{2}

\medskip

\index[OggettiMagici]{Robe of Stars} \subsubsection*{Robe of Stars}
60,000 gp, rare, while wearing this robe, you gain a +1 bonus on saving throws. Six stars, positioned on the top front of the dressing gown, are larger than the others. While wearing this robe, you can use two actions to draw one of the stars and use it to throw magic missile. Each day at sunset, the removed star reappears on the robe. While wearing the robe, you can use two actions to enter the Astral Plane along with everything you wear or carry. You will stay there until you use two actions to return to the floor you were on before. You reappear in the last space you occupied, or if that space is occupied, in the nearest unoccupied space.

\subsection{Manuals, Tomes, Books}


\index[OggettiMagici]{Golem Handbook} \subsubsection*{Golem Handbook}
10,000 gp, very rare, this tome contains the information and enchantments needed to build a particular type of golem. The Storyteller chooses the type of golem that can be built or determines it randomly. To decipher and use the manual you must have at least 10 magical proficiency. A creature that cannot use the golem manual and tries to read it takes 6d6 points of force damage.

\medskip

\begin{tabular}{llll}
3d6 & Golem & Time & Cost \\
\hline
3-4 & Clay & 30 days & 65,000 gp \\
5-16 & Meat & 60 days & 50,000 gp \\
17 & Iron & 120 days & 100,000 gp \\
18 & Stone & 90 days & 80,000 gp \\
\end{tabular}


\medskip

To create a golem, you must spend the time indicated above, working without interruption with the manual available and resting for no more than 8 hours a day. You must also pay the specified cost to purchase the necessary materials.

Once you finish creating the golem, the book is consumed by arcane flames. The golem comes alive when the ashes of the manual are scattered over it. It will be under your control and understands and obeys the orders uttered by you.

\subsubsection*{Handbook of Good Health} \index[OggettiMagici]{Handbook of Good Health}
15,000 gp, very rare, this tome contains instructions for strengthening the body and health. It takes 24 hours to read the book in a minimum of 3 days. His instructions will be followed for 4 weeks, after which the reader will permanently earn a Constitution point. Once read, the book loses its magical power and the reader will never be able to use one like it again.

\subsubsection*{Action Speed Manual} \index[OggettiMagici]{Action Speed Manual}
15,000 gp, very rare, this tome contains exercises for balance and coordination. It works like a good health textbook, but gains a Dexterity point.

\subsubsection*{Exercise Manual} \index[OggettiMagici]{Exercise Manual}
15,000 gp, very rare, this tome works exactly like the health manual, but gives the reader a Strength point.

\index[OggettiMagici]{Tome of Authority and Influence} \subsubsection*{Tome of Authority and Influence}
15,000 gp, very rare, this book contains guidance on how to influence and fascinate others, and its words are suffused with magic. If you spend 48 hours in a period of 6 days or less studying the contents of the book and practicing its directions, your Charisma score increases by 1. Then the manual loses its magic, to regain it after a century.

\index[OggettiMagici]{Tome of Understanding} \subsubsection*{Tome of Understanding}
15,000 gp, very rare, this book contains exercises in intuition and discernment, and its words are suffused with magic. If you spend 48 hours in a period of 6 days or less studying the contents of the book and practicing its directions, your Wisdom score increases by 1, and so does your maximum score for that trait. Then the manual loses its magic, to recover it after a century.

\index[OggettiMagici]{Tome of Clear Thought} \subsubsection*{Tome of Clear Thought}
15,000 gp, very rare, this book contains exercises in memory and logic, and its words are suffused with magic. If you spend 48 hours in a period of 6 days or less studying the contents of the book and practicing its directions, your Intelligence score increases by 1. Then the manual loses its magic, to regain it after a century.

\subsection{Various Magic Items}

\subsubsection*{Purifying water} \index[OggettiMagici]{Purifying water}
500 gp, rare, this sweet liquid can be used to purify water (including desalinating sea water) and to transform poisons, acids and other noxious liquids into a drinkable beverage. In addition, purifying water neutralizes the effectiveness of any other potion. This potion can transform up to 1000 cubic meters of almost any water-based liquid, but only 10 cubic meters of acid. The effects are permanent, and a purified liquid cannot be deteriorated or contaminated again for a period of 5d4 rounds.

\index[OggettiMagici]{Wings of Flight} \subsubsection*{Wings of Flight}
54,000 gp, legendary, while wearing this cloak, you can use two actions to utter its command word, transforming it into a pair of bat or bird wings that sprout from your back for 1 hour or until you repeat the command word with an action. The wings provide you with flight speeds of 60 feet. When they disappear, you will no longer be able to use them until dawn the next day hours.

\index[OggettiMagici]{Iron Vial} \subsubsection*{Iron Vial}
35,000 gp, legendary, this iron bottle has a brass stopper. You can use two actions to speak the vial's command word, targeting a visible creature within 60 feet of you. If the target is native to a plane of existence other than the one you are on, he must succeed on a DC 21 Will save or be trapped in the vial. If the target has already been trapped in the vial, they receive + 1d6 on their saving throw. Once trapped, the creature will remain in the vial until it is released. The ampoule can only hold one creature at a time. A creature trapped in the ampoule does not need to breathe, eat or sleep and does not age. You can use two actions to remove the bottle cap and free the creature it contains. The creature will be friendly towards you and your companions for 1 hour and will obey your commands for that duration. If you don't give her commands or give her one that would result in her death, she will defend herself but take no other action. At the end of the duration, the creature will act according to its normal behavior

The identify spell reveals that a creature is inside the vial, but the only way to determine what sort of creature it is is to open the vial. A newly discovered iron vial may already contain a creature chosen by the Storyteller or randomly determined.

\medskip

\begin{tabular}{ll}
\hline
d100 & Contains \\
1-50 & Blank \\
51-66 & Demon \\
67 & Angelo Deva \\
68-69 & Devil (upper) \\
70-73 & Diavolo (lower) \\
74-75 & Djinni Genius \\
76-77 & Genius Efreeti \\
78-83 & Elemental (any) \\
84-86 & Invisible Persecutor \\
87-90 & Night hag \\
91 & Angelo Planetar \\
92-95 & Salamander \\
96 & Angelo Solar \\
97-99 & Succubus / Nightmare \\
100 & Xorn \\
\end{tabular}
\medskip


\subsubsection*{Elemental water amphora} \index[OggettiMagici]{Elemental water amphora}
2500 gp, rare, this jar can be used to summon and control a water elemental in a similar way to the summon elemental spell. You must prepare the magic item and conduct a ritual for one turn before the actual summoning, which takes one round. After the elemental has been summoned, concentration must be maintained to be able to issue orders. The amphora can be used once a day.

\index[OggettiMagici]{Crab Apparatus} \subsubsection*{Crab Apparatus}
15,000 gp, legendary, this item appears as a sealed iron barrel of size Large and weighing 250 pounds. The barrel hides a latch, which can be found by passing an Intelligence check with a DC 25. Removing the latch opens a compartment at one end of the apparatus, which allows two Medium or smaller creatures to enter it. At the opposite end there are ten levers, each in a neutral position, capable of moving up or down. When certain levers are employed, the apparatus transforms and resembles a giant lobster.

The apparatus is a Large item with the following stats.

Defense - 20, Hit Points - 200, Speed - 30 ', Swim 30' (or 0m both if legs and tail are not extended)

Damage Immunity: Poison

To be used as a vehicle, the apparatus requires a pilot. When the appliance door is closed, the compartment is watertight, and does not allow air or water to enter. The compartments hold enough air for 10 hours, divided by the number of creatures inside. The apparatus floats in water and can even go underwater to a depth of 270 meters. Below this threshold, the apparatus takes 2d6 points of blow damage per minute from pressure. A creature inside the compartment can use two actions to move up or down by up to two levers. After each use, the lever returns to its neutral position. Each lever, from left to right, works as shown in the following table.

1: Extends the legs and tail, allowing the apparatus to walk and swim. It retracts legs and tail, reducing the speed of the apparatus to 0 and making it unable to benefit from speed bonuses.

2: Opens the front door. Closes the front door.

3: Opens the side doors (two on each side). Closes the side portholes (two on each side).

4: Extends two claws from the front of the apparatus. He portrays the claws.

5: Make a melee weapon attack with each pincer extended: +8 attack roll, range 5 feet, one target. Hits: 7 (2d6) hit damage. Make a melee weapon attack with each pincer extended: +8 to attack roll, range 5 feet, one target. Strikes: The target is grabbed (DC 18 to escape).

6: The apparatus walks or swims forward. The apparatus walks or swims backwards.

7: The apparatus turns 90 degrees to the left. The apparatus turns 90 degrees to the right.

8: Front slits emit intense light within a radius of 9 meters and dim light for further meters. He turns off the lights.

9: The apparatus sinks 6 meters in liquids. The apparatus rises 6 meters from liquids.

10: Unlocks and opens the tailgate. Closes and seals the tailgate.


\subsubsection*{Vial of Curses} \index[OggettiMagici]{Vial of Curses}
800 gp, rare, this item looks like a cruet, bottle, carafe, container, flask, or pitcher. It may contain a liquid or give off smoke. When the vial is first uncorked, all creatures within 30 feet are cursed.

\index[OggettiMagici]{Folding Boat} \subsubsection*{Folding Boat}
12000 gp, rare, this item looks like a wooden box measuring 30cm long, 15cm wide and 15cm deep. It weighs 2 kilos and floats. It can be opened to store items inside. This object has three command words, each of which requires two actions to be spoken. A command word causes the box to unfold into a boat 3 meters long, 1.2 meters wide and 50 centimeters deep. The boat has a pair of oars, an anchor, a mast and a sail. The boat can hold up to four Medium-sized creatures.

The second command word causes the box to unfold into a ship 7.2 meters long, 2.5 meters wide and 2 meters deep. The ship has a deck, rowing lines, five sets of oars, a rudder, an anchor, a cabin and a mast with a square sail. The ship can hold fifteen Medium-sized creatures.

The third command word causes the collapsible boat to fold back into the box, as long as no creatures are aboard. Any objects on board that cannot fit into the box remain outside the box as it folds. Any item on board that can fit into the box will fit into it.

\subsubsection*{Drowning basin}: This cursed basin has the appearance of an elemental water amphora. However, instead of summoning an elemental, it unleashes a globe of water that envelops the character's head. He drowns in 2d4 rounds unless he succeeds in a saving throw against spells. Water is "sticky" and can only be removed by magic (scattering of magic or destroying water).

\index[Magical Objects]{Battaglio dell'Apertura} \subsubsection*{Battaglio dell'Apertura}
1500 gp, rare, this hollow metal tube measures approximately 30 centimeters in length and weighs 0.5 kilos. You can beat it with two actions, by pointing it at an object within 36 meters that can be opened, such as a door or a lock. The clapper makes a clear sound, and an object lock or snare opens unless the sound is prevented from reaching the object. If there are no locks or laces left to open, the item opens by itself.

The clapper can be used ten times. After the tithe, it splits and becomes unusable.

\subsubsection*{Battaglia del Cannibalismo} \index[OggettiMagici]{Battaglia del Cannibalismo}
this object looks like an opening clapper. It functions as such for the first round of use (and has 1d4x10 stacks for this purpose). However, on the second rattle all creatures within 60 feet must succeed in a DC 21 Will save or fall prey to ravenous hunger, attacking the nearest humanoid to kill and devour it. On alternate rounds, a new saving throw is allowed. If no humanoids are present, the affected creatures will attack the other creatures present.

\index[OggettiMagici]{Preservative Bag} \subsubsection*{Preservative Bag}

There are different types of Preservative Bags and they all have in common the ability to hold much more than they should given their size.

Preservative Bags are divided into 4 types (Type I, II, III; IV) depending on the storage capacity they have.

If the pouch is overloaded, punctured or torn, the pouch breaks and is destroyed and its contents scattered across the Astral Plane. If the bag is turned inside out, its contents are expelled, unharmed, but the bag must be put back in the right direction before it can be used again. Breathing creatures placed in the bag can survive for a number of minutes equal to 10 divided by the number of creatures (minimum 1 minute), after which they will begin to suffocate.

Placing a holding bag within extradimensional space generated by a practical backpack, portable hole, or similar object destroys both objects and opens a portal to the Astral Plane. The portal originates at the point where one object has been placed inside the other. Any creature within 10 feet of the portal is sucked into it and reappears in a random place on the Astral Plane, then the portal closes. The portal is one-way and cannot be reopened.


Some spellcasters prefer to make Preservative Chests, which work in the same way as the Preservative Bags.

\index[OggettiMagici]{Preservative Bag Type I} \subsubsection*{Preservative Bag Type I}
500 gp, uncommon, this is the smallest model of the holding bags. Apparently it is a small bag of 20 cm in diameter with a mouth about the same wide.
It is not possible to bring in objects that have a width greater than 20 cm and a length greater than 50cm.
The maximum capacity is 20 kg.

\index[OggettiMagici]{Preservative Bag Type II} \subsubsection*{Preservative Bag Type II}
1000 gp, uncommon, this is the average model of holding bags. Apparently it is a bag of 40 cm in diameter with a mouth about the same wide.
It is not possible to bring in objects that have a width greater than 40 cm and a length greater than 100cm.
The maximum capacity is 100 kg.

\index[OggettiMagici]{Preservative Bag Type III} \subsubsection*{Preservative Bag Type III}
1500 gp, rare, apparently a sack 80 cm in diameter with a mouth about as wide.
It is not possible to bring in objects that have a width greater than 80 cm and a length greater than 150cm. The maximum capacity is 200 kg.

\index[OggettiMagici]{Preservative Bag Type IV} \subsubsection*{Preservative Bag Type IV}
5000 gp, very rare, apparently a sack 120 cm in diameter with a mouth about the same wide.
It is not possible to bring in objects that have a width greater than 120 cm and a length greater than 200cm. The maximum capacity is 300 kg.

\index[OggettiMagici]{Devouring Bag} \subsubsection*{Devouring Bag}
2000 gp, rare, the bag appears as a holding bag. If the stock exchange is turned over, its properties stop working. An extradimensional creature attached to the bag can sense anything placed inside it. The animal or vegetable matter placed entirely inside the bag is devoured and is lost forever. When a part of a living creature is placed in the bag, there is a 50 \% chance that the creature will be dragged into the bag. A creature inside the bag can use two actions to try to escape by making a DC 18 Strength check.

Another creature can use two actions to grab the creature inside the bag and pull it out, making a DC 20 Strength check (and provided it isn't dragged into the bag itself). Any creature that starts its round inside the bag is devoured, its body destroyed.

Inside the bag can be placed inanimate objects, up to 27 dm3 of material. However, once a day, the bag swallows any object placed inside it and spits it out into another plane of existence. The Storyteller determines the moment and the plan. If the bag were torn apart or torn, it is destroyed, and whatever it contains would be transported to a random place on the Astral Plane.

\index[OggettiMagici]{Bottle of Efreeti} \subsubsection*{Bottle of Efreeti}
15000 gp, very rare, this painted brass bottle weighs 500 grams. When you use two actions to remove the cap, a cloud of thick smoke escapes from the bottle. At the end of your round, the smoke dissipates in a harmless flash of fire, and an efreeti appears in an unoccupied space within 30 feet of you. The first time the bottle is opened, the Storyteller randomly determines what happens.

\medskip

\begin{tabularx}{0.45\textwidth}{lX}
\textbf{3d6} & \textbf{Effect} \\
\hline
3-5 & Efreeti attacks you. After fighting for 5 rounds, the efreeti disappears and the bottle loses its magic. \\
6-16 & The ephreeti obeys you for 1 hour, acting on your commands. Then it goes back into the bottle, and a new cap can hold it. The cap cannot be removed for 24 hours. The next two times the bottle is opened, the same effect occurs again. If the bottle is opened a fourth time, the ephreeti escapes and disappears, and the bottle loses its magic. \\
17-18 & The ephreeti can cast the wish spell in your favor three times. Disappears when granting the final wish or after 1 hour when the bottle loses its magic.
\end{tabularx}


\index[OggettiMagici]{Bag of Beans} \subsubsection*{Bag of Beans}
5,000 gp, rare, 3d4 dried beans are inside this bag. The bag weighs 250 grams plus 125 grams for each bean it contains.

If you dump the contents of the bag onto the ground, the beans explode within a 10-foot radius. Each creature in the area, including you, must make a DC 18 Reflex saving throw, taking 5d4 points of fire damage if it fails, or half that damage if it succeeds.

The fire ignites flammable objects in the area that are not worn or carried. If you remove the bean from the bag, plant it in the ground or sand, and water it, the bean will take effect 1 minute later, starting from where it was planted on the ground. The Storyteller chooses the effect or determines it randomly.

\end{multicols}

\vfill

\begin{center}
	\includegraphics[width = 0.6 \linewidth]{immagini/borsetta.png}

	\textit{Preservative bag, classic model, Type II}
\end{center}


\medskip

\begin{tabularx}{0.95\textwidth}{lX}
\textbf{d100} & \textbf{Effect} \\
\hline
01 & 5d4 mushrooms appear. If a creature eats a mushroom, roll a dice. If the result is odd, he must succeed on a DC 15 Fortitude save or take 5d6 points of poison damage and be poisoned for 1 hour. If the result is a tie, he gains 5d6 temporary hit points for 1 hour. \\
02-10 & A geyser erupts and spits water, beer, juice, tea, vinegar, wine, or oil (at Storyteller's discretion) 30 feet into the air for 1d12 rounds. \\
11-20 & A tree man appears. There is a 50 \% chance that the tree man is chaotic evil and will attack you. \\
21-30 & An animated stone statue with your features rises from the ground. It will begin to verbally threaten you. If you were to leave and other people came to the scene, the statue would describe you as the most dangerous of criminals, and would urge them to seek out and attack you. If you are on the same plane of existence as the statue, it will always know where you are. After 24 hours the statue will become inanimate. \\
31-40 & A campfire that produces blue flames rises from the ground and burns for 24 hours (or until extinguished). \\
41-50 & Spit 1d6 + 6 howler mushrooms. \\
51-60 & 1d4 + 8 fuchsia toads appear. Whenever a toad is touched, it transforms into a Large or smaller monster of the Storyteller's choice. The monster stays for 1 minute and then disappears in a puff of fuchsia smoke. 61-70 A bulette comes out of the ground and attacks. \\
71-80 & A fruit tree grows. Has 1d10 + 20 fruit. 1d8 of these function as a randomly determined magical potion, while one of them serves as an ingested poison of the type determined by the Storyteller. The tree vanishes after 1 hour. The harvested fruits, on the other hand, remain, and retain their magic for 30 days. \\
81-90 & A nest appears with 1d4 + 3 eggs. Any creature that eats an egg must make a DC 28 Fortitude saving throw. If the saving throw is successful, the
creature permanently increases its lowest ability score by 1, randomly choosing on a tie. If the saving throw fails, the creature takes 10d6 points of force damage from a magical explosion within it. \\
91-99 & A pyramid with a square base of 18 meters emerges from the ground. Inside is a sarcophagus which contains a sovereign mummy. The pyramid is regarded as the lair of the sovereign mummy, and its sarcophagus contains a treasure of the Storyteller's choice. \\
100 & A huge beanstalk grows in place, to a height of the Storyteller's choice. The top leads wherever the Storyteller wants, be it a cloud giant's castle or another plane of existence.
\end{tabularx}

\begin{multicols}{2}

\medskip

\index[OggettiMagici]{Smoking Bottle} \subsubsection*{Smoking Bottle}
1,200 gp, uncommon, smoke is continually escaping from the mouth of this brass bottle, held by its lead stopper. The bottle weighs 500 grams. When you use two actions to remove the cap, a cloud of thick smoke spreads within a 60-foot radius around the bottle. The cloud area is heavily obscured. For each minute that the bottle is open and inside the cloud, the radius increases by 3 meters until it reaches the maximum radius of 36 meters.

The cloud persists as long as the bottle remains open. Closing the bottle requires you to speak its command word with two actions. Once the bottle is closed, the cloud disperses after 10 minutes. A moderate wind (15 to 30 km / h) can disperse the smoke in 1 minute, and a strong wind (more than 30 km / h) can disperse it in 1 round.

\subsubsection*{Exchange of Cancellation} \index[OggettiMagici]{Exchange of Cancellation}
9000 gp, rare, this magical bag functions as a holding bag for 1d6 days. After this period, all the material inside it or new material added is subject to a transformation dependent on its nature. Precious stones become useless stones, and precious metals are transformed into metals of lesser value such as lead. Magical items lose their power without a saving throw, and transform into mundane items of their own type. Only extremely powerful magical items are possibly immune to this effect.

\index[Magic Items]{Fire Elemental Brazier} \subsubsection*{Fire Elemental Command Brazier}
8000 gp, rare, while the fire burns inside this brass brazier, you can use two actions to say the brazier command word and summon a fire elemental, as if you had cast the summon elemental spell. The brazier cannot be used like this again until the next sunrise.

The brazier weighs 2.5 kilos.

\subsubsection*{Brazier of Cursed Sleep} \index[OggettiMagici]{Brazier of Cursed Sleep}
this brazier has the appearance of, and functions as, a fire element command brazier. However, when activated, the smoke builds up to a 10-foot radius around the brazier, causing anyone in the area to fall into cursed sleep unless successful on a DC 21 Will save. A fire elemental appears. normally, but it is hostile and attacks all creatures present. Creatures subject to cursed sleep sleep indefinitely until killed, unless Remove Curse is used.

\index[Magic Items]{Jug of Infinite Water} \subsubsection*{Jug of Infinite Water} 12000 gp 12000 gp, uncommon, this corked vial makes a liquid sound when stirred, as if it contained water. The jug weighs 1 kilo. You can use two actions to remove the cap and say one of the three command words, at which point an amount of fresh water or salt water (your choice) will pour out of the vial, until the start of your next round. Choose one of the following options:

\medskip

\begin{itemize}
\item
"Stream" produces 4 liters of water.
\item
"Fontana" produces 20 liters of water.
\item
"Geyser" produces 150 liters of water which are projected from a geyser 9 meters long and 30 centimeters wide. With two actions, while holding the pitcher, you can target a visible creature within 30 feet of you from the geyser.

The target must succeed on a DC 15 Fortitude save or take 1d4 points of slash damage and fall prone. Instead of a creature, you can target an object that isn't worn or carried and that weighs no more than 100 pounds. The object is flipped or pushed 4.5 meters away from you.
\end{itemize}

\subsubsection*{Potions Jug} \index[Magic Items]{Potions Jug}
18,000 gp, legendary, this blue ceramic jug has a solid gold stopper. The pitcher contains 1d4 + 1 magical potions, each of which can be poured every 2 days. The specific potions are determined at random, remain the same over time and must always be poured in the same order. Not all of them are necessarily beneficial.

\index[OggettiMagici]{Portable Hole} \subsubsection*{Portable Hole}
10,000 gp, rare, this elegant, silky-soft black fabric folds to the size of a handkerchief. It unfolds in a circular layer of 1 meter in diameter. You can use 1 round to unfold a Pocket Hole and place it on or against a solid surface, on which the Pocket Hole makes a 10ft deep hole. Any creature small enough can use the Portable Hole to traverse the wall or surface it rests on as long as it is less than 10 feet deep.

You can use 1 round to close a Portable Hole by taking the edges of the fabric and folding it back. Folding the fabric closes the hole, and any creatures or objects within it are ejected with a 50 \% chance of exiting one way or the other.

Placing a portable hole within extradimensional space created by a holding bag, portable compartment, utility backpack, or similar object instantly destroys both objects and opens a portal to the Astral Plane. The portal originates from the point where one object was placed inside the other. Any creature within 10 feet of the portal is sucked into it and deposited in a random location on the Astral Plane. Then the portal closes. The portal is one-way and cannot be reopened.

\index[OggettiMagici]{Invocation Candle} \subsubsection*{Invocation Candle}
8000 gp, very rare, this long thin candle is dedicated to a Patron and shares his traits. The traits of the candle can be identified through a 1 hour ritual of flanking the candle.

The Storyteller chooses the Patron and the Traits associated with it or determines it randomly.

The magic of the candle is activated when the candle is lit with two actions. After burning for 4 hours, the candle is destroyed. You can decide to turn it off early to reuse it later. Deduct the time left for the candle to go out in 1 minute increments to determine how long the candle has burned.

When lit, the candle radiates dim light within a 30-foot radius. Any creature within the devoted light or follower of the candle's light makes attack rolls, saving throws, and skill checks with + 1d6.

Alternatively, when you light the candle for the first time, you can cast the portal spell. Doing so destroys the candle.

\index[OggettiMagici]{Dimensional Shackles} \subsubsection*{Dimensional Shafts}
4000 gp, rare, you can use 2 Actions to place these handcuffs on an incapacitated creature. Handcuffs fit any creature from Small to Large. In addition to serving as common handcuffs, shackles prevent a creature bound with them from using any method of extradimensional movement, including teleporting or traveling to different planes of existence. However, they do not prevent a creature from passing through an interdimensional portal.

You and any creature you indicate when using the logs can use two actions to remove them. Once every 30 days, the bound creature can make a DC 40 Strength check. If it succeeds, the creature breaks free and destroys the shackles.

\index[OggettiMagici]{Supreme Glue} \subsubsection*{Supreme Glue}
400 gp, uncommon, this milky white viscous substance can form a permanent adhesive bond between any two objects. It must be contained in a jar or ampoule that has been coated inside with slipperiness oil. When found, its container holds 1d6 + 1 per 30 grams. 30 grams of glue can cover a square area of 30 centimeters on each side. The glue takes 1 minute to set. Once the glue is fixed, the bond created can only be broken by the universal solvent or oil of the ethereal form, or by the wish spell.

\subsubsection*{Necklace of Healthy Air} \index[OggettiMagici]{Necklace of Healthy Air}
2500 gp, uncommon, this necklace is a chain with a platinum locket. The magic of the necklace surrounds the wearer with a bubble of pure air, making him immune to the effects of vapors and gases. The bubble allows you to survive in an airless environment for a week.

\index[OggettiMagici]{Climbing Rope} \subsubsection*{Climbing Rope}
2000 gp, uncommon, this 18 meter long silk rope weighs 1.5 kilos and can hold up to 1,500 kilos. If you grab one end of the string and use two actions to say the command word, the string comes alive. With two actions you can command the other end to move to a destination of your choice. That end moves 10 feet during your round when it receives your first command, and 10 feet during each subsequent round until it reaches its destination, to its maximum length, or until you tell it to stop. You can also tell the rope to tighten or unhook from an object, knot or unwind, or rewind to be carried. If you tell the rope to tie a knot, large knots will appear at 30cm intervals along the rope. While knotted, the rope decreases to a length of 15 meters and grants + 1d6 to checks made to climb it.

The rope has Defense 20, Hardness 3, and 20 Hit Points. He regains 1 hit point every 5 minutes until he has at least 1 hit point. If the rope drops to 0 hit points, it is destroyed.

\index[Magic Items]{Rope of Entanglement} \subsubsection*{Rope of Entanglement}
4000 gp, rare, this rope is 9 meters long and weighs 1.5 kilos. If you hold one end of the rope and use two actions to say its command word, the other end will snap forward to entangle a visible creature within 20 feet of you. The target must succeed on a DC 18 Reflex saving throw or be hindered. You can release the creature by using two actions to say a second command word. A target trapped by the rope can use two actions to make a Strength or Escape Artist check with a DC 18 (target's choice). If it passes, the creature is no longer hampered by the rope.

The rope has Defense 20 and 20 hit points. He regains 1 hit point every 5 minutes until he has at least 1 hit point. if the rope drops to 0 hit points, it is destroyed.

\subsubsection*{Choke Cord} \index[OggettiMagici]{Choke Cord}
rare, this magic rope, although normal in appearance, can become animated and attack anyone who tries to use it, tightening around the neck and trying to strangle its victim. The choke rope is long enough to strangle 1d4 creatures in a 10-foot radius, inflicting 2d6 wounds per round to each of them. A DC 19 Reflex saving throw is required to avoid being caught. The rope has Defense 22 and 25 Hit Points, but only those who are not strangled can attack it. Victims cannot free themselves in any way, nor cast spells.

\index[OggettiMagici]{Horn of Destruction} \subsubsection*{Horn of Destruction}
750 gp, rare, you can use two actions to say the command word of the horn and then sound it, emitting a thundering blast in a 30-foot cone and audible up to 180 meters away. Each creature within the cone must make a DC 18 Fortitude save. If the saving throw fails, the creature takes 5d6 points of sonic damage and is stunned for 1 minute. If the saving throw is successful, the creature takes half the damage and is not deafened. Creatures and objects made of glass or crystal have -1d6 on their saving throw and take 10d6 points of sonic damage rather than 5d6.

Each use of horn magic has a 20 \% chance to detonate it. The explosion deals 10d6 points of fire damage to the sounder and destroys the horn.

\index[OggettiMagici]{Horn of Valhalla} \subsubsection*{Horn of Valhalla}
6000 gp, rare, you can use two Actions to blow this horn. As an answer, the warrior spirits of Asgard appear within 60 feet of you. These spirits use berserker stats. They return to Asgard after 1 hour or when they drop to 0 hit points. Once used, the horn cannot be used again for 7 days.


\index[OggettiMagici]{Cube of Force} \subsubsection*{Cube of Force}
16,000 gp, rare, this cube has an edge of one inch. Each face has a unique brand that can be pressed. The cube starts with 36 charges, and recovers 3d6 charges spent every day at dawn. You can use two Actions to press one of the cube's faces, spending a number of charges based on the cube's face.

Each face has a different effect. If the cube no longer remains charged, nothing happens. Otherwise, a barrier of invisible force rises, forming a cube with an edge of 4.5 meters. The barrier is centered on you, moves with you, and lasts for 1 minute, until you use two actions to press the sixth face of the cube, or the cube runs out of stacks. You can change the barrier effect by pressing a different face of the cube and spending the required number of charges, resetting its duration.

If your movement causes the barrier to come into contact with a solid object that cannot pass through the cube, as long as the barrier remains you will not be able to approach the object.

\medskip

The cube loses stacks when the barrier is targeted by certain spells or comes into contact with certain spells or magical item effects, as indicated in the following table.

\medskip

\begin{tabular}{ll}
\textbf{Spell or Item} & \textbf{Lost Charges} \\
\hline
Magic Bolt (5 hits) & 1 \\
Disintegration & 1d12 \\
Wall of Fire & 1d4 \\
Bulkhead & 1d6 \\
Prismatic Spray & 3d6 \\
\end{tabular}

\medskip

\begin{tabularx}{0.45\textwidth}{llX}
\textbf{Face} & \textbf{Charges} & \textbf{Effect} \\
\hline
1 & 1 & Gas, wind and fog cannot penetrate the barrier \\
2 & 2 & Nonliving matter cannot cross the barrier. Walls, floors and ceilings can pass through it at your discretion. \\
3 & 3 & Living matter cannot cross the barrier. \\
4 & 4 & The spell's effects cannot cross the barrier. \\
5 & 5 & Nothing can cross the barrier. Walls, floors and ceilings can pass through it at your discretion. \\
6 & 0 & The barrier is deactivated. \\
\end{tabularx}


\subsubsection*{Cold protection cube} \index[OggettiMagici]{Cold protection cube}
2500 gp, rare, this cubic charm is activated and deactivated by pressing a face (immediate action). When activated, it emanates a cubic protective field with the edge of 3 m (similar to that of a cube of strength but with a different effect). The temperature inside the protective field is maintained at 21 ° C. The field absorbs all attacks of cold, negating them completely. If it negates more than 50 cold damage in a round (from either a single attack or multiple attacks), however, the magic field collapses and cannot be reactivated for one hour. If the field negates more than 100 cold wounds in a round, the cube is destroyed.

\index[OggettiMagici]{Iron Bands of the Binding} \subsubsection*{Iron Bands of the Binding}
5,000 gp, rare, this rusted iron sphere measures 7.5 centimeters in diameter and weighs 500 grams. You can use two actions to say a command word and hurl the orb at a visible creature of Huge size or smaller within 60 feet of you. The sphere moves in the air, opening up in a network of metal bands. Make a ranged attack roll, if you hit, the target is in the way until you take two actions to say a command word and free it. Doing so, or missing the attack, causes the bands to contract and return to being a sphere.

A creature, including one that is entangled, can use two actions to make a DC 25 Strength check to break the iron bands. If successful, the object is destroyed, and the entangled creature is free. If the check fails, any further attempts made by the creature automatically fail until 24 hours have elapsed. Once the bands have been used, they will no longer be able to be used until the next sunrise.

\index[OggettiMagici]{Efficient Quiver} \subsubsection*{Efficient Quiver}
2,500 gp, rare, each of the quiver's three compartments is connected to an extradimensional space that allows it to carry numerous objects never weighing more than 1 kilo.

The smaller compartment can hold up to 60 arrows, bolts or similar objects. The middle compartment can hold up to 18 javelins or similar items. The longest compartment can hold up to 6 long items, such as bows, fighting Staff or spears. You can take out any item contained in the quiver as if you were taking it from a normal quiver or scabbard.

\subsubsection*{Phylactery against the undead} \index[OggettiMagici]{Phylactery against the undead}
1000 gp, rare, this sacred item allows you to use the Turn undead skill with a +2 bonus to the sum of Traits shared with the Patron.

\subsubsection*{Phylactery of youth} \index[OggettiMagici]{Phylactery of youth}
10,000 gp, legendary, the parchment strip of this phylactery is usually encased in a metal tube to be worn around the neck. When a character wears it, his natural aging rate drops to 75 \%, while any magical aging is reduced by half.

\index[OggettiMagici]{Instant Fortress} \subsubsection*{Instant Fortress}
75000 gp, very rare, you can use two actions to place this one-inch-edged metal cube on the ground and speak its command word. The cube rapidly grows into a fortress that will remain as long as you use two actions to say the command word that dismisses it, which only works when the fortress is empty.

The fortress is a square tower, 6 meters on each side and 9 meters high, with loopholes on all sides and battlements at the top. Its interior is divided into two floors, with a staircase running along one wall to connect them. The staircase ends with a trap door that opens onto the roof. When activated, the tower has a small door on the side facing you. The door opens only at your command, which you can say with two actions. It is immune to the knock spell and similar spells, such as that of the opening clapper.

Each creature in the area where the fortress appears must make a DC 17 Reflex save, taking 10d10 hit damage if it fails, or half that damage if it succeeds. In both cases, the creature is pushed into a space outside the fortress but in close proximity to it. Objects in the area that are not worn or carried take the same damage and are pushed automatically.

The tower is made of adamantium, and its magic prevents it from being overturned. The roof, door, and walls have 100 hit points each, immunity to damage from non-magical weapons except siege weapons, and resistance to all other damage.

Only the wish spell can repair the fortress. Each cast of wish causes the roof, door, or wall to recover 50 hit points.


\subsubsection*{Locating Arrow} \index[OggettiMagici]{Locating Arrow}
400 gp, uncommon, this arrow can be used up to 8 times per day. It is thrown into the air, and when it lands it points to a desired direction or place. Possible indications include the nearest exit or entrance, stairs, passageways, caves and similar areas.

\index[Magic Items]{Censer of the Air Elementals Command} \subsubsection*{Censer of the Air Elementals Command}
8000 gp, rare, while the incense burns inside this censer, you can use two actions to say the brazier command word and summon an air elemental, as if you had cast the summon elemental spell. The censer cannot be used this way again until the next dawn. This censer 15cm wide and 30cm high resembles a goblet with a decorated cover. It weighs 0.5 pounds.

\subsubsection*{Incense of meditation} \index[OggettiMagici]{Incense of meditation}
5,000 gp, rare, this sweet-scented block of incense is indistinguishable from ordinary incense until ignited. When it burns, its distinctive fragrance and pearly smoke are recognizable with an Arcana check at DC 13. After a spellcaster has spent 8 hours reviewing the Tome and meditating near a lit block, he will acquire the ability to cast his spells with maximum effect and maximum duration possible, spells that require a saving throw will impose an additional -1 penalty. Each block of incense burns for 8 hours and the effect persists for another 8 hours. There are usually 2d4 blocks of incense in the same case.

\index[OggettiMagici]{Lantern of Revelation} \subsubsection*{Lantern of Revelation}
5,000 gp, uncommon, while lit, this lantern burns for 6 hours with 1 flask of oil, radiating bright light within a 30-foot radius and dim light for an additional 30 feet. Invisible creatures and objects are made visible while under the bright light of the lantern.

It can use two actions to lower the cover, reducing the light to dim with a radius of 1.5 meters.

\subsubsection*{Incense of Obsession} \index[OggettiMagici]{Incense of Obsession}
Rare, very similar to meditation incense, this incense also gives the user the impression of its effect, but will be in confusion for 24 hours if they fail a DC 23 Will save.

\index[Magic Items]{Deck of Illusions} \subsubsection*{Deck of Illusions}
6,500 gp, uncommon, this box contains a set of parchment cards. A full deck contains 34 cards, each featuring a different creature. The creatures depicted are left to the Storyteller's discretion. Usually the decks found around are free of 3d6-3 cards.

The deck spell works only if the cards are drawn at random (you can use a modified deck of normal playing cards to simulate the deck of illusions). You can use two actions to draw a card from the deck and throw it to a spot on the ground 30 feet away from you.

The illusion of one or more creatures forms on top of the cast card and remains until dispelled. The illusory creature looks real, the appropriate size, and acts as if it were a real creature, except that it cannot do damage. As long as you are within 100 feet of the illusory creature and can see it, you can use two actions to magically move it to any point within 30 feet of the card. Any physical interaction with the illusory creature reveals it as an illusion, as objects pass through it. Someone who uses two actions to visually inspect the creature identifies it as illusory by passing an Intelligence check with DC 17. The creature will then appear transparent to her.
The illusion persists until the card is moved or the illusion is dissolved. When the illusion ends, the image on the card disappears, and that card can no longer be used.

\end{multicols}

\medskip

\begin{center}
	\includegraphics[width = 0.55 \linewidth]{immagini/Incenso.png}

\end{center}

\begin{tabular}{ll|ll}
\textbf{Playing Card} & \textbf{Illusion} & \textbf{Playing Card} & \textbf{Illusion} \\
\hline
Ace of Hearts & Red Dragon & Ace of Diamonds & Beholder \\
King of hearts & Knight and four guards & King of diamonds & Archmage and apprentice mage \\
Queen of Hearts & Succubus or Nightmare & Queen of Diamonds & Night Hag \\
Jack of Hearts & Druid & Jack of Diamonds & Assassin \\
Ten of Heart & Giant of Clouds & Ten of Diamonds & Giant of Fire \\
Nine of Hearts & Ettin & Nine of Diamonds & Oni \\
Eight of Hearts & Bugbear & Eight of Diamonds & Gnoll \\
Two of Hearts & Goblins & Two of Diamonds & Kobold \\
Ace of Spades & Lich & Ace of Clubs & Iron Golem \\
King of spades & Priest and two acolytes & King of clubs & Captain bandit and three bandits \\
Queen of Spades & Medusa & Queen of Clubs & Erinyes \\
Jack of Spades & Veteran & Jack of Clubs & Berserker \\
Ten of Spades & Frost Giant & Ten of Clubs & Hill Giant \\
Nine of Spades & Trolls & Nine of Clubs & Ogre \\
Eight of Spades & Hobgoblin & Eight of Clubs & Ogre \\
Two of Spades & Goblins & Two of Clubs & Kobold \\
Joker (2) & You (deck owner) && \\
\end{tabular}

\begin{multicols}{2}

\medskip

\index[MagicObjects]{Deck of Wonders} \subsubsection*{Deck of Wonders}
100,000 gp, legendary, usually found in a pouch or box containing cards made of ivory or fleece. Most of these decks (75 \%) only have thirteen cards, while the remaining decks have twenty-two.

Before drawing a card, you must declare how many cards you intend to draw and then draw them randomly (you can use a modified deck of playing cards to simulate the deck). Any cards drawn in excess of this number have no effect. Otherwise, as soon as you draw a card from the deck, its spell takes effect.

You must draw each card within 1 hour of the previous draw. If you don't draw the chosen number of cards, the remaining number of cards will come out of the deck spontaneously and take effect at the same time. Once a card is drawn, it will vanish from existence. Unless the card is the Fool or the Fool, the card reappears in the deck, making it possible to draw the same card twice.

\medskip

\end{multicols}

\begin{tabularx}{0.95\textwidth}{lX | lX}
\textbf{Playing Card} & \textbf{Card} & \textbf{Playing Card} & \textbf{Card} \\
\hline
Ace of Diamonds & Vizier * & Ace of Hearts & Fate * \\
King of diamonds & Sun & King of hearts & Throne \\
Queen of Diamonds & Moon & Queen of Hearts & Key \\
Jack of Diamonds & Star & Jack of Hearts & Knight \\
Two of Diamonds & Comet * & Two of Hearts & Gemma * \\
Ace of clubs & Spurs * & Ace of spades & Dungeon * \\
King of Clubs & The Void & King of Spades & Ruin \\
Queen of clubs & Flames & Queen of spades & Euryale \\
Jack of Clubs & Skull & Jack of Hearts & Rogue \\
Two of Clubs & Idiot & Two of Spades & Hanging * \\
Jolly & Matto * & Jolly & Buffone \\
\end{tabularx}

\begin{multicols}{2}

\medskip

* Only in 22-card deck

\textit{Hanging} (only in packs of 22). Your mind is upset, and you change 2 Traits

\textit{Buffoon}. You get 35 PX or you can draw two additional cards in addition to your declared draws.

\textit{Knight}. Obtain the services of a level 4 WP warrior who appears in a space of your choice within 30 feet of you. The warrior is of the same kind as you and will serve you loyally until death, believing that it was fate that brought him to your service. The character is controlled by you.

\textit{Key}. A rare, very rare, or legendary magical weapon with which you are proficient appears in your hands. The Storyteller determines what kind of weapon it is.

\textit{Comet} (only in packs of 22). If you single-handedly defeat the next monster or hostile group you encounter, you will gain enough experience points to earn a level. Otherwise, this card will have no effect.

\textit{Euryial}. You are cursed by the card and take a -2 penalty on all saving throws as long as you are cursed this way. Only a Patron or the magic of the Fate card can end this curse.

\textit{Fate} (only in decks of 22). The structure of reality dissolves and reforms, allowing you to avoid or cancel an event as if it never happened. You can use this card's spell as soon as you draw it or wait any other time until your death.

\textit{Flames}. A mighty devil becomes your enemy. The devil will try to ruin and haunt your existence, savoring your suffering until the moment he tries to kill you. This enmity will last until your or the devil's death.

\textit{Rogue}. A non-player character of the Storyteller's choice becomes hostile towards you. The identity of the new enemy is unknown until the NPC or someone else reveals it. Nothing short of a divine wish or intervention can put an end to the NPC's hostility towards you.

\textit{Gem} (only in packs of 22). Twenty-five jewels worth 2,000 gp each or fifty gems worth 1,000 gp each appear before your feet.

\textit{Idiot} (only in decks of 22). Permanently reduce your Intelligence score by 2 (down to a minimum score of 3). You may draw an additional card before your other declared draws.

\textit{Moon}. You gain the ability to cast the wish spell 1d3 times.

\textit{Fool} (only in decks of 22). You lose 10000 PX, discard this card, and draw from the deck again, counting both draws as just one of your draws. If losing that number of XP would cause you to lose a level, you will instead be left with just enough XP to maintain your level.

\textit{Ruin}. You lose all riches you have with you, apart from other magical items. Activities, buildings, and lands you own are lost in the least reality-altering way. Any document proving that you are the owner of something that you have lost due to this card disappears.

\textit{Sun}. You get 35 XP, and a wonderful item (determined by the Storyteller) appears in your hands.

\textit{Dungeon} (only in deck of 22). You disappear and are buried in a state of suspended animation within an extradimensional sphere. Everything you were wearing or carrying remains in the space you occupied when you disappeared. You will remain imprisoned until you are found and removed from the sphere. You cannot be located by any divination spell, but the wish spell can reveal the location of your prison. No further cards are drawn.

\textit{Spurs} (only in packs of 22). Any magical item you wear or carry is disintegrated. Artifacts in your possession are not disintegrated, but they vanish.

\textit{Stella}. Increase your ability score by 1. Your score can exceed 5, but it cannot exceed 7.
\textit{Skull}. You summon an avatara of death (a ghostly humanoid skeleton wrapped in a tattered black robe, wielding a ghostly scythe). It appears in a space of the Storyteller's choice within 10 feet of you and attacks you, warning everyone else that you must win the battle alone. The avatar fights until you die or until it drops to 0 hit points, at which point it wears off. If someone tries to help you, they will summon their avatar of death. A creature slain by a death avatar cannot be brought back to life.

\textit{Throne}. Gain + 1d6 in Diplomacy. Also, you get the ownership right on a small rock somewhere in the world. However, the keep is currently occupied by monsters, which you will need to hunt before you can claim it as your own.

\textit{Vizier} (only in packs of 22). At any time of your choosing, within one year of drawing this card, you can ask, meditating, an answer to your question and receive a truthful answer to it. Aside from providing information, the answer can help you solve a complex problem or dilemma. In other words, knowledge is provided along with wisdom on how to use it.

\textit{Blank}. This black card is a sign of disaster. Your soul is kidnapped by the body and imprisoned inside an object in a location of the Storyteller's choice. One or more powerful creatures protect this place. As long as your soul is so trapped, your body is incapacitated. The wish spell cannot restore your soul, but it can reveal the whereabouts of the object that contains it. No more cards are drawn.

\textit{Avatar of Death}

Mean undead, neutral evil

\textbf{Strength} +3

\textbf{Dexterity} '+3

\textbf{Intelligence} +3

\textbf{Wisdom} +3

\textbf{Charisma} +3

\textbf{Defense} 20

\textbf{Hit Points} half of its summoner's hit points

\textbf{Movement}: Speed 18m, flight 18m (float)

\textbf{Immunity to Damage}: Void, poison

\textbf{Condition Immunity}: fascinated, poisoned, paralyzed, petrified, frightened, passed out

\textbf{Sensi}: darkvision 18 m, true vision 18 m

\textbf{Languages}: all languages known to its summoner

\textbf{Challenge} (0 PX)

\textbf{Incorporeal Movement}. The avatar can traverse creatures and objects as if they were difficult terrain. It takes 5 (1d10) force damage if it ends its round inside an object.

\textbf{Immunity to Turning}. The avatar is immune to effects that turn undead out.

\textbf{Actions}

\textbf{Reaper Scythe}. The avatar sinks its ghostly scythe into a creature within 5 feet of it, dealing 7 (1d8 + 3) piercing damage plus 4 (1d8) negative Energy damage.

\index[OggettiMagici]{Miniature of Wonderful Power} \subsubsection*{Miniature of Wonderful Power}
Variable rarity, variable cost, a miniature of marvelous power is a statuette of a beast, small enough to fit in your pocket. If you use two actions to say a command word and throw the figure anywhere on the ground within 60 feet of you, the figure becomes a living creature. If the space in which the creature appears is occupied by another creature or object, or if there is not enough space for the creature, the figure does not transform.

The creature is friendly towards you and your companions. She understands your languages and obeys your orders. If you don't give it orders, the creature defends itself but takes no other actions. See the Bestiary for the creature's other stats.

The creature remains for the duration specified for each figure. At the end of the duration, the creature reverts to its miniature form. It transforms early if it drops to 0 hit points or if you use two actions to say the command word again while tapping it. After the creature reverts to being a figure, its properties can no longer be used until a certain amount of time has elapsed, as specified in the figure's description.

\textit{Onyx Dog} (Rare, 500 gp). This onyx figurine depicts a dog. It can become a Mastiff for up to 6 hours. The hound has Intelligence -2 and can speak Common. It also has darkvision with a range of 60 feet and can see invisible creatures and objects within that range. Once used, it cannot be used again for 7 days.

\textit{Ivory goat (Rare. 1000 gp)}. These ivory goat figurines are always made in sets of three. Each goat has a unique appearance and functions differently from the others. Their properties are as follows:

The dread billy goat can grow into a giant billy goat for up to 3 hours. The billy goat cannot attack, but you can remove its horns and use them as weapons. One horn becomes a +1 knight's spear while the other becomes a +2 longsword.

Removing a horn takes two actions, and the weapons disappear and the horns reappear when the goat reverts to its miniature form. Additionally, the goat radiates an aura of terror with a 30-foot radius as long as you ride it. Any creature hostile to you that begins its round within the aura must succeed at a DC 17 Will save or stay
scared of the goat for 1 minute, or until the goat reverts to miniature form. The frightened creature can re-roll the saving throw at the end of each of its rounds, ending the effect if it succeeds. Once a creature is successful at its saving throw against this effect, it is immune to the goat's aura for the next 24 hours. Once used, the miniature cannot be used again for 15 days.

The labor billy goat can grow into a giant billy goat for up to 3 hours. Once used, it cannot be used again for 30 days.
The Traveling Billy Goat can become a Big Billy Goat with the same stats as a Racehorse. It has 24 charges, and each hour or portion of it spent in beast form costs 1 charge. As long as it has charges, you can use it as much as you like. Once the stacks are finished, it reverts to being a miniature and cannot be used again until 7 days have passed, once it has recovered all its stacks.

\textit{Silver Raven} (Uncommon, 300 gp). This silver figurine depicts a raven. It can become a crow for up to 6 hours. Once used, it cannot be used again before 2 days have passed. While in raven form, the miniature allows you to cast the animal messenger spell on it at will.

\textit{Obsidian Steed} (Very Rare, 1000 gp). This smooth obsidian figurine becomes a nightmare for up to 24 hours. The nightmare only fights to defend itself. Once used, it cannot be used again for 5 days.

\textit{Marble Elephant} (Rare, 1500 gp). This marble figurine is approximately 10 centimeters high and wide. It can become an elephant for up to 24 hours. Once used, it cannot be used again for 7 days.

\textit{Bronze Griffin} (Rare, 1250 gp). This bronze statuette depicts a rampant griffin. It can become a griffin for up to 6 hours. Once used, it cannot be used again for 5 days.

\textit{Serpentine Owl} (Rare, 400 gp). This serpentine owl figurine can grow into a giant owl for up to 8 hours. Once used, it cannot be used again before 2 days have passed. If you are on the same plane of existence, the owl can telepathically communicate with you at any distance.

\textit{Golden Lions} (Rare, 800 gp). These golden lion figurines are always created in pairs. You can use one or both thumbnails at the same time. Each can become a lion for up to 1 hour. Once one of the lions has been used, it cannot be used again for 7 days.

cd \index[OggettiMagici]{Kill Ammo} \subsubsection*{Kill Ammo}
700 gp, very rare, if a creature of the type, race, or group that the arrow of kill is associated with takes damage from the arrow, the creature must make a Fortitude save at DC 21, taking an additional 6d10 piercing damage if it fails, or half of that damage if successful.

Once the kill arrow has dealt additional damage to the creature, it becomes a nonmagical arrow.

\index[OggettiMagici]{Crystal Ball} \subsubsection*{Crystal Ball}
50,000 gp, very rare or legendary, a typical crystal ball is approximately six inches in diameter. While touching it, you can cast the search spell (saving throw DC 21) with it. The following variant crystal balls are legendary items and have additional properties.

\textit{Crystal Ball of Reading Thought}. This crystal ball is approximately 12cm in diameter. While touching it, you can cast the search spell (saving throw DC 21) with it. You can use two actions to cast the detect thoughts spell (saving throw DC 21) while searching through this crystal ball, targeting creatures that you can see and are within 30 feet of the spell's sensor. You don't have to focus on this thought pinpointing to hold it for its duration, which ends when it ends scrutinizing.

\textit{Crystal Ball of Telepathy}. This crystal ball is approximately 12cm in diameter. While touching it, you can cast the search spell (saving throw DC 21) with it. As you peer through this crystal ball, you can telepathically communicate with creatures you can see and are within 30 feet of the spell's sensor. You can also use two actions to cast the suggestion spell (saving throw DC 21) on one of these creatures via the sensor. You don't have to focus on this suggestion to keep it for its duration, which ends if it ends scrutinizing. Once used, the crystal ball's suggestion power cannot be used again until the next dawn.

\textit{True Seeing Crystal Ball}. This crystal ball is approximately 12cm in diameter. While touching it, you can cast the search spell (saving throw DC 21) with it. As you peer with this crystal ball, you have true vision with a 36-meter radius centered on the spell's sensor.

\subsubsection*{Hypnotic Crystal Ball} \index[OggettiMagici]{Hypnotic Crystal Ball}
rare, this cursed item is indistinguishable from a normal crystal ball. However, anyone who attempts to use the device is fascinated for 1d6 turns, and a telepathic suggestion is implanted in his mind if he fails a DC 27 Will save. The user of the device believes he has seen the desired creature or scene, but in reality is under the influence of a powerful spellcaster, or even a power or being from another plane of existence. With each further use the user falls more and more under the influence of the controller, as a servo or as a tool. The user is always unaware of being subjugated.

\index[Magic Items]{Scroll of Spells} \subsubsection*{Scroll of Spells}
variable rarity, see scroll creation costs, a scroll of spells shows the words of a single spell, written in a mystical code.

To read a parchment you need:

\textbf{in case of ISY SCROLL parchments}:

An Arcana check at DC 10 difficulty is sufficient to understand the content

An Intelligence (or Arcana if known) check on difficulty 12 is required in order to read and cast the scroll's spell.

\textbf{in case of normal scrolls}:

To understand its contents, an Arcana check at difficulty 15 is required

An Arcana check at difficulty 20 is required in order to read and cast the scroll spell.

Casting the spell by reading it from a scroll takes the normal spell casting time. Once the spell is cast, the words on the scroll vanish, and the scroll is reduced to dust. If the cast is interrupted, the scroll does not dissolve.

\subsubsection*{Protective Scroll Against Elementals} \index[Magical Objects]{Protective Scroll Against Elementals}
800 gp, rare, protects against all elementals for 20 rounds, granting +4 defense and saving throws against attacks or effects produced by elementals.

\subsubsection*{Scroll Against Werewolves} \index[Magical Objects]{Protective Scroll Against Werewolves}
700 gp, uncommon, protects against all werewolves for 20 rounds, granting +4 defense and saving throws against werewolf attacks or effects.

\subsubsection*{Scroll Against the Undead} \index[Magic Items]{Protective Scroll Against the Undead}
900 gp, uncommon, protects against all undead for 20 rounds, granting +4 Defense and saving throws against attacks or effects produced by the undead.

\subsubsection*{Scroll Against Magic} \index[OggettiMagici]{Protective Scroll Against Magic}
1500 gp, rare, the scroll casts an Anti-Magic Field spell.

\index[OggettiMagici]{Pearl of Power} \subsubsection*{Pearl of Power}
6000 gp, uncommon, while you have the pearl with you, you can use two actions to recover 2d4 Magic Points. Once used, the pearl cannot be used again until the next sunrise. There are more powerful variants that make you recover more points.

\subsubsection*{Weight Stone} \index[OggettiMagici]{Weight Stone}
this object looks like a smooth and shiny black stone. When the wearer is involved in a fight or flight, he suddenly suffers the effects of the slow spell. Once taken, the stone cannot be thrown away normally, as after a short time it magically reappears on the owner's person. To get rid of the stone permanently, you need the Remove Curse spell.

\index[OggettiMagici]{Arcane Stone} \subsubsection*{Arcane Stone} \index{Ioun Stone}
variable cost, variable rarity, there are different types of arcane stone, each type a specific combination of shapes and colors.

When you use two actions to throw one of these stones into the air, the stone starts orbiting your head at a distance of 1d3 x 30 centimeters and grants you a benefit.
After that, another creature will have to use two actions to grab or harness the stone and separate it from you, making a defense attack roll 24 or a successful DC 31 Dexterity check. You can use two actions to grab and set aside. the stone, ending its effect.

A stone has Defense 24, 10 hit points and resistance to all damage. As it orbits around your head it is considered a worn object.

\textit{Dexterity} (very rare, 3000 gp). As it orbits your head, your Dexterity score increases by 1, to a maximum of 5.

\textit{Absorb} (very rare, 6000 gp). As it orbits your head, you can use your Action to cancel a level 4 or lower spell cast by a visible creature that targets only you. Once the stone has cleared 5 Spells, it runs out and turns dull gray, losing its magic.

\textit{Authority} (very rare, 3,000 gp). As it orbits your head, your Charisma score increases by 1, to a maximum of 5.

\textit{Awareness} (rare, 12,000 gp). As it orbits around your head you can't be surprised.

\textit{Strength} (very rare, 3000 gp). As it orbits your head, your Strength score increases by 1, to a maximum of 5.

\textit{Intelligence} (very rare, 3000 gp). As it orbits your head, your Intelligence score increases by 1, up to a maximum of 5.

\textit{Insight} (very rare, 3000 gp). As it orbits your head, your Wisdom score increases by 2, to a maximum of 5.

\textit{Protection} (rare, 10,000 gp). As it orbits your head you get a +1 Defense bonus.

\textit{Sustenance} (rare, 3500 gp). As it orbits around your head, you don't need to eat or drink.

\index[OggettiMagici]{Stone of Good Fortune} \subsubsection*{Stone of Good Fortune}
4500 gp, uncommon, while the stone is with you, you gain a +1 bonus on ability checks and saving throws.

\index[OggettiMagici]{Stone of Control of Earth Elementals} \subsubsection*{Stone of Control of Earth Elementals}
8000 gp, rare, if the stone touches the ground, you can use two actions to say the command word and summon an earth elemental, as if you had cast the summon elemental spell. The stone cannot be used in this way again until the next dawn. The stone weighs 2.5 kilos.

\index[OggettiMagici]{Sewer Pipe} \subsubsection*{Sewer Pipe}
2000 gp, uncommon, you have to be proficient with wind instruments to use this pipe. While using this pipe, normal rats and giant rats are indifferent to you and will not attack you unless you threaten or harm them. If you fiddle with two actions, you can use two actions to spend 1 to 3 charges, summoning a swarm of rats for each charge expended, as long as there are enough rats within 750 meters of you to summon this way (at the discretion of the Storyteller ). If there aren't enough rats to swarm, the charge is wasted. The recalled swarms move to the music via the shortest possible route, but are in no other way under your control. The fife has 3 charges and recovers 1d3 charges spent every day at dawn.

Whenever a swarm of rats not under the control of another creature approaches within 30 feet of you while you are piping, you may make a Charisma check contested by the swarm's Wisdom check. If you lose the contest, the swarm behaves as normal and cannot be distracted by the fife music again for the next 24 hours. If you win the contest, the swarm is attracted to the music of the fife and becomes friendly towards you and your teammates as long as you keep playing the fife with two actions each round. A friendly swarm obeys your commands. If you don't give orders to a friendly swarm, it will defend itself but take no further action.

If a friendly swarm at the start of the round cannot hear the fife music, your control over that swarm ends, and the swarm behaves as it normally would and cannot be attracted to the fife music again for the next 24 hours.

\index[Magical Objects]{Fife of Fright} \subsubsection*{Fife of Fright}
6000 gp, uncommon, you must be proficient with wind instruments to use this pipe. You can use two actions to play it and spend 1 charge to create an enchanting and spooky sound. Any creature within 30 feet of you that hears you ringing must succeed on a DC 17 Will save or be scared of you for 1 minute. If you wish, any creatures in the area that are not hostile to you can automatically succeed at their saving throw. A creature that fails the saving throw can re-roll it at the end of its round, ending the effect on itself if it succeeds. A creature that successfully saves is immune to this pipe's effect for 24 hours. The fife has 3 charges and recovers 1d3 charges spent every day at dawn.

\index[MagicObjects]{Pigments of Wonders} \subsubsection*{Pigments of Wonders}
400 gp, very rare, usually found in 1d4 jars inside elegant wooden boxes together with a brush (total weight 500 grams), these pigments allow you to create three-dimensional objects, painting them in two dimensions. Paint flows from the brush to form the desired object as you focus on the image

Each pot of paint is sufficient to cover 90 square meters of an area, allowing you to create inanimate objects and characteristics of the terrain (doors, pits, flowers, trees, cells, rooms or weapons) that occupy a total of 270 cubic meters. It takes 10 minutes to cover 90 paintings.

When you complete the painting, the painted terrain object or feature becomes a real object, not a magical one. So, painting a door on a wall creates a real door that can be opened to access what lies beyond it. Painting a pit on the floor creates a real pit, the depth of which is counted in the total area of objects you can create.

Nothing created from pigments can be worth more than 25 gp. If you paint an object of higher value (a diamond or a pile of gold), the object will appear authentic, but careful examination will reveal that it is made of plaster, bone, or some other worthless material.

If you paint a form of energy, such as fire or lightning, the energy appears but dissipates as soon as you complete the painting, doing no harm to anything.

\index[OggettiMagici]{Arcane Feather} \subsubsection*{Arcane Feather}
variable cost, variable rarity, this tiny object resembles a feather. There are several types of arcane feathers, each featuring a single disposable effect. The Storyteller chooses the type of arcane feather.

\textit{Tree}. You must be outdoors in order to use this arcane feather. You can use two actions to lean it against an unoccupied space on the ground. The feather vanishes and a non-magical oak tree emerges in its place. The tree is 18 meters high and has a trunk of 1.5 meters in diameter. At the top, its branches extend for up to 6 meters. 50 gp

\textit{More}. You can use two actions to prop the arcane feather against a boat or ship. For the next 24 hours, the vessel cannot be moved in any way. Touching the vessel with the arcane feather again ends this effect. When the effect ends, the feather wears off. 50 gp

\textit{Whip}. You can use two actions to throw the arcane feather at a point within 10 feet of you. The feather vanishes and a floating whip appears in its place. You can then use two actions to make a melee spell attack against a creature within 10 feet of the whip, with a +9 attack bonus. If you hit, the target takes 1d6 + 5 force damage. During your round, with two actions, you can direct the whip to fly up to 20 feet and repeat the attack against a creature within 10 feet of it. The whip expires after 1 hour, when you use two actions to dismiss it, or when you are incapacitated or die. 250 gp

\textit{Swan Ship}. You can use two actions to place the arcane feather on a body of water that is at least 60 feet in diameter. The feather vanishes and in its place appears a boat 15 meters long and 6 meters wide in the shape of a swan. The boat moves by itself and moves in the water at the speed of 9 kilometers per hour. You can use two actions while on board to command her to move or turn 90 degrees. The boat can carry up to thirty-two Medium or smaller creatures. A Large creature counts as four Medium creatures, while a Huge creature counts as nine Medium creatures. The boat vanishes after 24 hours. You can dismiss the boat with two actions. 3000 gp

\textit{Bird}. You can use two actions to throw the Arcane Feather 5 feet into the air. The feather vanishes and a huge multicolored bird takes its place. The bird has the stats of a Roc, but obeys simple commands and cannot attack. It can carry up to 250 kilos while flying at its maximum speed (24 kilometers per hour for up to 216 kilometers per day, with an hour of rest every 3 hours of flight), or 500 kilos of weight at half speed. The bird vanishes after flying the maximum possible distance in one day or if it drops to 0 hit points. You can dismiss the bird with two actions. 3000 gp

\textit{Fan}. If you are on a boat or ship, you can use two actions to throw the arcane feather up to 10 feet into the air. The feather vanishes and a giant fan appears in its place. The fan floats and creates a wind strong enough to inflate the ship's sails, increasing its speed by 7.5 kilometers per hour for 8 hours. You can dismiss the fan with two actions. 250 gp

\index[OggettiMagici]{Dust of Aridity} \subsubsection*{Dust of Aridity}
120 gp, rare, this small pack contains 1d6 + 4 pinches of dust. You can use two actions to sprinkle a pinch of dust on the water The dust transforms a 4-foot-edged cube of water into a ball of dust the size of a marble, which floats or settles where it was thrown the dust. The weight of the ball is negligible.

Anyone can use two actions to smash the ball against a hard surface, causing the ball to break and release the absorbed water from the powder. Doing so depletes the magic of the ball.

An elemental composed primarily of water and exposed to a pinch of this dust must make a Fortitude save with a DC 15, taking 10d6 Void damage if it fails, or half that damage if successful.

\subsubsection*{Revealing Dust} \index[OggettiMagici]{Revealing Dust}
500 gp, uncommon, this fine dust looks like a very light metal dust. A handful of this substance sprayed into the air covers all objects within a 10-foot radius, making everything visible. When sprayed through a blowgun, the powder fills a cone 6 meters long and 1.5 meters wide at the end. The dust nullifies the effects of illusory power, distorting cloak, elven cloak, and the special abilities of creatures such as unstable molossians and warping panthers; the effect lasts for 2d10 turns. The telltale powder is usually stored in small silk bags or hollow tubes made of bone; normally 5d10 doses of powder are found.

\index[Magic Items]{Dust of Disappearance} \subsubsection*{Dust of Disappearance}
700 gp, rare, found in small pouches, this powder looks like very fine sand. There is enough in a bag for one use. When you use two actions to throw dust into the air, you and each creature and object within 10 feet of you become invisible for 2d4 minutes. The duration is the same for all subjects, and when the magic takes effect the dust is consumed. If a creature under the effect of the dust attacks or casts a spell, invisibility ends only for that creature.

\index[Magic Items]{Sneeze and Choke Powder} \subsubsection*{Sneeze and Choke Powder}
480 gp, uncommon, found in small containers, this powder looks like fine sand. It appears similar to vanishing dust, and the identify spell reveals it as such. There is enough for one use. When you use two actions to throw a handful of dust into the air, you and any creatures that need to breathe and are within 30 feet of you must succeed on a DC 17 Fortitude save or stop breathing, and start sneezing in. uncontrollable way. A creature afflicted in this way is incapacitated and suffocates. As long as it is conscious, the creature can re-roll the saving throw at the end of each of its rounds, ending the effect if you succeed. Even the lesser restoration spell can end the effect afflicting the creature.

\index[OggettiMagici]{Cubic Portal} \subsubsection*{Cubic Portal}
40,000 gp, legendary, this three-inch-edged cube radiates palpable magical energy. The six faces of the cube are each connected to a different plane of existence, one of which is the Material Plane. The other faces are connected to planes determined by the Storyteller.

You can use two actions to press one face of the cube to cast the portal spell through it, opening a passage to the plane connected to that face. Alternatively, if you use two actions to press a face twice, you can cast the planar shift spell (saving throw DC 17) via the cube and transport its targets to the plane connected to that face. The cube has 3 charges. Each use of the cube expends 1 charge. The cube recovers 1 charge spent every day at dawn.

\index[OggettiMagici]{Well of Many Worlds} \subsubsection*{Well of Many Worlds}
75,000 gp, legendary, this elegant, silky-soft black fabric is wrapped to the size of a handkerchief. It unfolds into a circular sheet of 1.8 meters in diameter. You can use two actions to unfold and place the Well of Many Worlds on a solid surface, upon which it creates a two-way portal to another world or plane of existence. Whenever the object opens a portal, the Storyteller decides where it leads. You can use two actions to close an open portal by grabbing the edges of the fabric and folding them back. Once a well of the many worlds has opened a portal, it will not be able to do so again until 1d8 hours have passed.

\subsubsection*{Entanglement Net} \index[OggettiMagici]{Entanglement Net}
800 gp, rare, this 3m-wide square net can be thrown at an opponent to get in the way. The net is very strong and it takes the strength of a giant (For 5) to tear it with your bare hands. The net also resists cuts, and must be hit with extreme precision (Defense 25, PF 30) for it to yield. The net can also be hung or placed on the ground as a trap, which will magically activate at the command of the owner.

\subsubsection*{Entrapment Net} \index[MagicObjects]{Entrapment Net}
900 gp, rare, this net can only be used underwater, but it functions exactly like a tangle net on the surface, floating if it takes up to 30 feet to grab an opponent.

\subsubsection*{Animated Attack Broom} \index[OggettiMagici]{Animated Attack Broom}
this object is indistinguishable in appearance from a normal broom. In all checks it is identical to a flying broom, up to a height of 6 meters. When this happens the broom performs a pirouette and drops its pilot onto the head from a height of 1d4 + 5 x 30cm (no fall damage is dealt as the distance is less than 3m). The broom then attacks the victim, hitting her in the face with the brush and beating her with the handle. The broom makes two attacks per round with each end (two brush attacks and two handle attacks for a total of four attacks). The brush blinds the victim for 1 round when it hits. The handle inflicts 1d3 wounds. The broom has Defense 13, 18 hit points, and has +4 to attack rolls.

\index[OggettiMagici]{Flying Broom} \subsubsection*{Flying Broom}
8000 gp, uncommon, this wooden broom, weighing approximately 1.5 kilos, functions like a normal broom until you sit on it and say the command word. It begins to float beneath you and can be ridden through the air. It has a flight speed of 15 meters. It can carry up to 200 kilos, but its flight speed becomes 9 meters if it were to carry more than 100 kilos. When you land, the broom stops floating.

By saying the command word, naming the place and if you are familiar with it, you can send the broom yourself to a place up to 1.5 kilometers from you. The broom will return to you when you say another command word, as long as it is still within 1.5 kilometers of you.

\subsubsection*{Broom of Cursed Flight} \index[OggettiMagici]{Broom of Cursed Flight}
this magic broom looks like a flying broom. However, when activated, it flies up to 15m high or up to the ceiling (whichever is lower) and then stops working, causing the rider to plummet. After that the broom falls to the ground and loses its magical power.

\index[OggettiMagici]{Sphere of Annihilation} \subsubsection*{Sphere of Annihilation}
250,000 gp, legendary, this 50-centimeter-diameter black sphere is actually a hole in the structure of the multiverse, floating in space and stabilized by the magical field surrounding it.

The sphere annihilates all the matter that passes through and all the matter that passes through it. The only exception is artifacts. Unless the artifact is susceptible to damage from the sphere of annihilation, it can traverse the sphere without problem. Anything else that touches the sphere and is not completely enveloped and annihilated by it takes 4d10 points of strength damage per round.

The sphere remains motionless until someone controls it. If you are within 60 feet of an uncontrolled sphere, you can take two actions to make an Arcana check with DC 30. If you pass it, the sphere will levitate in a direction of your choice, for a number of meters equal to 1.5 x the Intelligence (minimum 1.5 meters). If you fail, the sphere moves 10 feet towards you. A creature whose space the sphere enters must succeed on a DC 15 Reflex saving throw or be touched by it, taking 4d10 points of force damage.

If you attempt to control an orb that's under another creature's control, you make a contested arcane check against the other creature's arcana. The winner of the contest gains control of the sphere and can levitate it as normal.

If the sphere makes contact with a planar portal, such as the one created by the portal spell, or an extradimensional space, such as that inside a portable hole, the Storyteller randomly determines what happens, using the following table.

\medskip

\begin{tabularx}{0.45\textwidth}{lX}
\textbf{3d6} & \textbf{Result} \\
\hline
3-10 & The sphere is destroyed \\
12-16 & The sphere moves through the portal or into extradimensional space. \\
17-18 & A space rift sends every creature and object within 54 meters of the sphere, including the sphere, into a plane of random existence. \\
\end{tabularx}

\medskip

\index[OggettiMagici]{Universal Solvent} \subsubsection*{Universal Solvent}
300 gp, legendary, this tube contains a white liquid with a strong smell of alcohol. You can use two actions to pour its contents onto a surface within reach. The liquid instantly dissolves 1000cm x cm of adhesive it comes in contact with, including Supreme Glue.

\subsubsection*{Mirror of Mental Ability} \index[OggettiMagici]{Mirror of Mental Ability}
15000 gp, very rare, this item looks like an ordinary mirror five feet high and two feet wide. On command, the owner can use it in the following ways:

- I would read thoughts of a person reflected on its surface with telepathy (without the need to understand an unknown language).

- Seeing other places as with a crystal ball, with the possibility of seeing in other planes, as long as they are sufficiently familiar to the observer.

- Create a portal to visit other places. The owner must first visualize the place, then physically enter the mirror, alone or with the companions of his choice. The mirror will create an invisible portal on the other side, through which the owner, or whoever can locate it, can pass through.

- Once a week, the mirror can accurately answer a question about a person reflected on its surface (an effect similar to the knowledge of legends spell.

\subsubsection*{Mirror of Duplication} \index[OggettiMagici]{Mirror of Duplication}
legendary, this mirror is a little more than a meter high and a little less wide. When a creature reflects on the mirror surface, its reflection (a duplicate identical in all respects) comes out to attack the original. The duplicate has all the equipment and powers of the original, including magic. The duplicate disappears immediately, along with all its objects, upon his or the original's death.

\index[OggettiMagici]{Mirror Traps Life} \subsubsection*{Mirror Traps Life}
18,000 gp, rare, when this 120 cm tall mirror is viewed indirectly, its surface shows a vague image of the creature. The mirror weighs 25 pounds, has Defense 11, 10 hit points, and is vulnerable to hit damage. It shatters and is destroyed when reduced to 0 hit points.

If the mirror is hanging from a vertical surface and you are within 5 feet of it, you can use two actions to say its command word and activate it. It will remain active until you say the command word again.

Any creature other than you that sees its reflection in the activated mirror while within 30 feet of it must make a DC 17 Will save or be trapped, along with anything it wears or carries, in one of twelve. extra-dimensional mirror cells. This saving throw receives + 1d6 if the creature knows the nature of the mirror and the constructs automatically succeed at the saving throw.

An extradimensional cell is an infinite space filled with a dense haze that reduces visibility to 10 feet. Creatures trapped in mirror cells don't age, and they don't need to eat, drink, or sleep. A creature trapped inside a cell can escape from it using magic that allows planes to travel. Otherwise, the creature is confined to the cell until it is released.

If the mirror traps a creature but its twelve extradimensional cells are already occupied, the mirror frees one of the randomly trapped creatures to house the new prisoner. The freed creature appears in an unoccupied space in view of the mirror but facing away from it. If the mirror is broken, all creatures it contains are released and reappear in an unoccupied space near it.

While within 5 feet of the mirror, you can use two actions to say the name of one of the creatures trapped inside it or call up a particular cell number. The creature named or contained in the named cell appears as an image on the mirror surface. After that, you and the named creature can communicate normally.

In a similar way, you can use two actions to say a second command word and free one of the creatures trapped in the mirror. The freed creature appears, along with all its properties, in the unoccupied space closest to the mirror and facing away from it.

\subsubsection*{Panic Drums} \index[OggettiMagici]{Panic Drums}
1500 gp, uncommon, these drums are similar to timpani (small, easily transportable percussion instruments). They are found in pairs and are inconspicuous in appearance. If both are played, all creatures within 72m (except those within a 3m circle centered on the drums) are assaulted by Fear and flee for 30 rounds at full speed. A Will save is allowed at DC 21 to save herself from the effects.

\subsubsection*{Drums of Stunning} \index[Magic Items]{Drums of Stunning}
rare, these two paired drums resemble panic drums; when both are played, all creatures within 3 yards must succeed on a DC 21 Fortitude save to be stunned for 2d4 rounds. All creatures within 21 yards are immediately deafened. Greater restoration, healing, regeneration, or similar spells can cure deafness.

\index[OggettiMagici]{Flying Carpet} \subsubsection*{Flying Carpet}
15000 gp, very rare, you can say the carpet command word with two actions to make the carpet float and fly. It moves according to the directions given to it by voice, as long as you are within 30 feet of it.

There are four sizes of flying carpet. The Storyteller chooses the size of the carpet or determines it randomly.

\medskip

\begin{tabular}{llll}
d100 & Size & Capacity & Speed by Flight \\
01-20 & 90 cm x 1.5 m & 100 kg & 24 meters \\
21-55 & 1.2 mx 1.8 m & 200 kg & 18 meters \\
56-80 & 1.5mx 2.1m & 300kg & 12m \\
81-100 & 1.8 mx 2.7 m & 400 kg & 9 meters \\
\end{tabular}

\medskip

The carpet can carry up to twice the weight indicated on the chart, but flies at half speed if it carries more than its load capacity.


\subsubsection*{Elemental Air Thurible} \index[OggettiMagici]{Elemental Air Thurible}:
1500 gp, rare, this censer can be used to summon and control an air elemental in a similar way to the summon elemental spell. You must prepare the magic item and conduct a ritual for one turn before the actual summoning, which takes one round. After the elemental has been summoned, concentration must be maintained to be able to issue orders.

\subsubsection*{Censer of Cursed Summon} \index[OggettiMagici]{Censer of Cursed Summon}
rare, this censer looks like, and appears to function like, an elemental censer of air. However, once lit it is impossible to turn it off for 1d4 rounds. In each round, an air elemental emerges and attacks all nearby creatures.

\index[OggettiMagici]{Restorative Ointment} \subsubsection*{Restorative Ointment}
5,000 gp, uncommon, this glass jar, 7.5 centimeters in diameter, contains 1d4 + 1 doses of a thick mixture. The jar and its contents weigh 250 grams. With two actions, a dose of ointment can be swallowed or applied to the skin. The receiving creature recovers 2d8 + 2 hit points, stops being poisoned, and is cured of any disease.

\index[OggettiMagici]{Portable Compartment} \subsubsection*{Portable Compartment}
10,000 gp, rare, this elegant, silky-soft black fabric folds to the size of a handkerchief. It unfolds in a circular layer of 1.8 meters in diameter. You can use two actions to unfold a Portable Compartment and place it on or against a solid surface, on which the Portable Compartment creates an extra-dimensional hole 3 meters deep. The cylindrical space inside the hole is on a different plane, and therefore cannot be used to open passages. Any creature inside an open Portable Compartment can climb out of it.

You can use two actions to close a Portable Compartment by taking the edges of the fabric and folding it. Folding the fabric closes the Compartment, and any creatures or objects inside it remain in extradimensional space. No matter what it contains, the Compartment weighs nothing.

If the Compartment is folded back, a creature within the Compartment's dimensional space can use two actions to make a DC 10 Strength check. If the check succeeds, the creature breaks free and reappears within 5 feet of the Compartment or the creature carrying it. A breathing creature can survive inside a closed portable hole for up to 10 minutes, after which time it will begin to suffocate.

Placing a Portable Compartment within extradimensional space created by a storage bag, utility backpack, or similar object instantly destroys both objects and opens a portal to the Astral Plane. The portal originates from the point where one object was placed inside the other. Any creature within 10 feet of the portal is sucked into it and deposited in a random location on the Astral Plane. Then the portal closes. The portal is one-way and cannot be reopened.

\index[OggettiMagici]{Arcane Fan} \subsubsection*{Arcane Fan}
1500 gp, uncommon, while holding this fan, you can use two actions to cast the gust of wind spell (saving throw DC 15) with it. Once used, the fan
it should not be used again until the next sunrise. Whenever used before then, there is a 20 \% cumulative chance that it will not work and break into useless shreds devoid of magic.


\index[OggettiMagici]{Practical Backpack} \subsubsection*{Practical Backpack}
7000 gp, rare, this pack has one center and two side pouches, each of which is actually an extradimensional space. Each side bag can contain 10 kilos of material, which does not exceed a volume of 60 dm3

The large central bag can hold up to 240 dm3 or 40 kilos of material. The backpack always weighs 2.5 kilos, whatever its contents.

Placing an object inside the backpack follows the normal rules for interacting with objects. Retrieving an item from the backpack requires the use of two actions. When you search for an item in your backpack, it will magically always be on top of the stack of items it contains.

The backpack has some limitations. If overloaded, or a sharp object cuts or tears it, the backpack shatters and is destroyed. If the backpack is destroyed, what it contained is lost forever, although an artifact will always reappear somewhere in the multiverse. If the backpack is turned inside out, what it contains is ejected, without harming it, and the backpack must be put back in the right direction before it can be used again. If a breathing creature is placed inside the backpack, it can survive for up to 10 minutes before starting to suffocate.

Placing the backpack within extradimensional space created by a holding bag, portable hole, or similar object immediately destroys both objects and opens a portal to the Astral Plane. The portal originates from the point where the objects have been placed one inside the other. Any creature within 10 feet of the portal is sucked through it and dragged to a random location on the Astral Plane. Then the portal closes. The portal is one-way and cannot be reopened.

\subsubsection*{Hoe of the Titans} \index[OggettiMagici]{Hoe of the Titans}
2000 gp, uncommon, this oversized tool is 3m long and 120kg heavy, and can only be used by a giant (or zoomed-in character) to move large amounts of dirt and build embankments (one 3m cube per turn) . The hoe can also be used to split stone very quickly. If used as a weapon it has a +3 bonus on hit and inflicts 5d6 wounds.

\index[Magic Items]{Hooves of Speed} \subsubsection*{Hooves of Speed}
5,000 gp, rare, these iron clogs come in sets of four. When all four hooves are attached to a horse or similar creature, they increase that creature's walking speed by 30 feet.

\index[OggettiMagici]{Hooves of the Zephyr} \subsubsection*{Hooves of the Zephyr}
1500 gp, very rare, these iron clogs come in sets of four. When all four hooves are attached to a horse or similar creature, they allow that creature to move normally, as it floats about 10 centimeters above the ground. This effect means that the creature can cross or pass over non-solid or unstable surfaces, such as water or lava. The creature leaves no traces and ignores difficult terrain. Additionally, the creature can move at its normal speed for up to 12 hours per day without suffering fatigue from forced marching.

\end{multicols}

\pagebreak

\section{Cursed Objects} \index{Cursed Objects}

\begin{changemargin}{0.3cm}{0.3cm} \begin{enfasi}{
When an ungodly curses his opponent, he curses himself. (Sirach)

If you curse one person there will be two pits. (Japanese proverb)
}\end{enfasi} \end{changemargin} \medskip

\begin{multicols}{2}

\label{oggetti-maledetti}

\lettrine[lines = 2, lhang = 0.33, loversize = 0.25, findent = 1.5em]{G}{li} cursed items are magical items with a potentially negative influence on the character. Sometimes they tend to confuse evil with good, forcing their owner to make difficult choices.

Cursed objects are almost never made intentionally, but rather are the result of unsuccessful work, of inexperienced craftsmen or a lack of proper components.

The Storyteller can ask for an Arcana check with a DC equal to 10 + days taken to build the magic item in case of particularly complex objects or there have been problematic situations in the creation and when the check fails by 10 or more, roll on the table. to determine the type of curse the object possesses.

A curse can occur as a result of the extreme negative or emotional influences affecting an object.

\medskip

\textbf{Common Item Curses}

\medskip

\begin{tabular}{ll}
\textbf{\%} & \textbf{Curse} \\
\toprule
01-15 & Deception \\
16-40 & Effect or Opposite Target \\
41-50 & Discontinued Operation \\
51-65 & Requirement \\
66-90 & Inconvenience \\
91-100 & Completely different effect \\
\end{tabular}

\medskip

Cursed items are \hypertarget{oggettimaledettiid}{identificati} like any other magical item with one exception: unless the Arcana check to identify the item exceeds 30 or the Identify spell is cast on a Magic Check and gains a magic critical (2 times 6) the curse is not detected. If the check is below 30 or without magic crit, all that is revealed is the original purpose of the magic item.

If the object is known to be cursed, the nature of the curse can be determined by using DC \hyperlink{identificareom}{standard} to identify the object.

\begin{center}
\includegraphics[width = 0.75 \linewidth]{immagini/vasobasano.png}

\textit{Vase of Basano. This vase was made in the second half of the 15th century and is made of silver.}
\end{center}


\begin{changemargin}{0.3cm}{0.3cm} \begin{Storytellere}
A curse is always a particular \textit{inconvenience}, which is not used at random. Think carefully about the cursed objects that you will have the characters find because they will ask you for a lot of information and you have to be ready.

There is no need for the curse to be excessive and limiting, it can very well be ridiculous or particular, so let it be characterizing. The character must not feel (unless you want to) doomed forever, take the opportunity to build new adventures and team spirit.
\end{Storytellere} \end{changemargin}


\subsection{Remove Cursed Objects} \index{Remove Cursed Objects}

While some cursed items may simply be put down, others exert a strong compulsion on the owner to keep them with him, at any cost. Others reappear even if abandoned or it is impossible to throw them away.

These objects can only be removed after the Remove Curse spell is cast on the character or object.

If the object was cursed with the Cast Curse spell, or otherwise the Storyteller decides that the object has a specific curse for that character then the Magical Proficiency required to remove the curse must be greater than or equal to that of the one who 'he threw.
Eventually he can cast Remove Curse with a Magic Check and for each critical success he adds 6 to his Magical Proficiency to see if he can remove the curse.

If the check is successful, the item can be removed in the next round, but the curse remains and hits again if the item is used / worn again.

Each cursed object has its own method of being destroyed, from being thrown into an active volcano, to being struck by the hammer of the Thunder God (or Patron ...) or devoured by a colossal worm of the sands if not hit by the breath of a red dragon and a white dragon at the same time ...

\subsection{Common Effects of Cursed Items}

The most common effects of cursed items are as follows. Storytellers can invent new special effects for specific cursed items.

\subsubsection{Deception}

Those who use the object continue to believe that it is what it seems at first sight, but in reality it has no power, other than to deceive. The user is mentally led to believe that it works, and cannot be convinced otherwise except with the use of Remove Curse

\begin{center}
\includegraphics[width = 0.70 \linewidth]{immagini/mirror.png}

\textit{The mirror in The Myrtles Plantation.}
\end{center}

\subsubsection{Effect or Opposite Target}

These cursed objects tend to have functional defects that in some cases generate effects diametrically opposite to those desired by their creator, while in other cases they tend to affect those who use them instead of someone else.

But the most interesting thing is that these objects may not even be a disadvantage for their owners. The category of magic items with opposite effects also includes weapons that inflict penalties on attack and damage rolls, rather than bonuses.

Since a character shouldn't immediately know what a magic item's bonus is, they shouldn't even know the nature of its curse. Once it learns, however, the item can be abandoned unless there is some magical effect on it that forces its owner to keep and use it. In these cases, you will need the Remove Curse Spell to get rid of the item.

Some particularly strong curses, at the Storyteller's discretion, can be removed by Remove Curses cast by a very experienced caster (check value of Magical Proficiency).

\subsection{Discontinued operation}

Discontinuous objects work exactly as they should when they work. Determine if the object is Unreliable, Conditional or Uncontrollable.

\medskip
\subsubsection{Unreliable}

Each time the object is activated, there is a 5 \% chance that it will not work.

\subsubsection{Conditioned}

This item only works in certain situations. To determine what they are, choose an activation condition or consult the table below.

\subsubsection{Uncontrollable}

An uncontrollable object tends to activate randomly. Rolling d \% every day. On a result of 01--05 the object spontaneously activates at a certain time of day.

\medskip

\begin{tabularx}{0.45\textwidth}{lX}
\textbf{\%} & \textbf{Situation} \\
\toprule
01-03 & Temperature below zero \\
04-05 & Temperature above zero \\
06-10 & During the day \\
11-15 & During the night \\
16-20 & Exposed to sunlight \\
21-25 & In the absence of sunlight \\
26-34 & Underwater \\
35-37 & Out of the water \\
38-45 & Underground \\
46-55 & On the surface \\
56-60 & Within 10 feet of a random creature type \\
61-64 & Within 10 feet of a random race or creature type \\
65-72 & Within 10 feet of a spellcaster \\
73-80 & Within 3 meters of a Follower or Devotee of a specific Patron \\
81-85 & In the hands of a non-caster character \\
86-90 & In the hands of a charmer character \\
91-95 & In the hands of a creature with particular Trait \\
96 & In the hands of a creature of a particular sex \\
97-99 & On non-sacred days or during particular astronomical recurrences \\
100 & A more than 150 km from a given place \\
\end{tabularx}

\subsection{Requirement}

Some items have much harder requirements for them to work. To make the object in question work, one of the following conditions may need to be met:

\begin{itemize}
\item The character must eat twice as much as normal.
\item Character must sleep twice as much as normal.
\item The character must complete at least one specific mission.
\item Character must sacrifice (destroy) 100 gp worth of valuable items or materials per day.
\item The character must swear loyalty to a particular noble or his family.
\item The character must abandon all other magical items.
\item Character must be a Follower or Devotee of a specific Patron
\item The character must have a minimum number of ranks in a particular skill.
\item The character must sacrifice some of his life energy (1 permanent Constitution point) the first time he uses the item.
\item The object must be purified with the sacred water of a specific Patron each day.
\item The object must be soaked in at least half a liter of blood (animal or humanoid) per day.
\item Item must be used to kill one living creature per day.
\item The item must be used at least once per day, or it stops working for its current owner.
\item When wielded, the item must draw blood (weapons only). It cannot be put aside or exchanged for another item until it has scored a hit.
\end{itemize}

\medskip

\begin{center}
\includegraphics[width = 0.8 \linewidth]{immagini/donnalemb.png}

\textit{Woman of Lemb or Statue of the Goddess of Death, 3500 BC}
\end{center}

\medskip

The requirements depend so much on the convenience of the item that they should never be determined at random. A smart object with a requirement often dictates its own requirement due to its personality.

If the requirement is not met, the object stops working. If it is met, the item usually runs for a full day before having to meet the requirement again (although some requirements only need to be met once, others once a month, and still others continuously).

\subsection{Inconvenience}

Items that have drawbacks usually have positive effects on the user, but they also have negative aspects. While drawbacks sometimes only come to light when items are used (or held in the hand, in the case of items such as weapons), they usually remain present until the character gets rid of the item in question.

Unless otherwise noted, the drawbacks remain active for as long as the object remains in the character's possession. The DC of the saving throw to avoid these effects is equal to 10 + DC of the curse (if you have not set the difficulty set the saving throw, usually Will, to DC 25)

\end{multicols}

\medskip

\begin{changemargin}{0.3cm}{0.3cm} \begin{Storytellere} The list is an example to be able to randomly generate effects on the owner of the object. Get inspired and be creative! However, do not let a curse make it impossible to play the character rather it must be experienced as an opportunity to try, do something different. Never throw a cursed object at random in the heap of treasures, always think about what will happen and what consequences will be generated. A cursed object always requires a high level of attention and planning on the part of the Storyteller \end{Storytellere} \end{changemargin}

\bigskip

\textbf{Table: Effects of Cursed Magic Items} \index{Table of Effects of Cursed Magic Items}

\medskip

\begin{tabular}{ll}
\textbf{\%} & \textbf{Inconvenient} \\
\toprule
01-02 & The character's hair grows 2.5 cm per hour. \\
02-04 & Character's nails grow 1cm every 8 hours \\
05-06 & Character height decreases by 5d10 cm \\
07-09 & Character height increases by 5d10cm \\
10-11 & The temperature around the object is 5 ° C colder than normal. \\
12-13 & The temperature around the object is 20 ° C colder than normal. \\
14-15 & The temperature around the object is 5 ° C warmer than normal. \\
16-17 & The temperature around the object is 20 ° C warmer than normal. \\
18-20 & Character's hair color changes. \\
21-23 & Character's skin color changes. \\
24 & Character's hair color changes every hour \\
25 & Character's skin color changes every hour \\
26 & Horns like a ram grow on the character's head \\
27 & A stage of antlers like a moose grows on the character's head \\
28-29 & The character now bears a distinctive mark (a tattoo, a strange glow, etc.). \\
30-32 & Character's gender changes. \\
33-34 & Character's race or species change. \\
35 & II PC is struck by a randomly determined Disease, which cannot be cured. \\
36-39 & The object constantly emits unpleasant sounds (moans, curses, insults ...). \\
40 & The object has a ridiculous appearance (bright colors, shape, glows with a pink halo, etc.). \\
41 & A small blue unicorn, visible only by magic, always flies \\
& around the Character giving useless advice and making silly jokes. \\
42 & Every day you get a sudden desire and ability to crochet for at least 1 hour. \\
43-45 & The character becomes extremely possessive of the object. \\
46-49 & Character has an uncontrollable fear of losing the item or being damaged. \\
50 & One Trait is changed \\
51 & The character's metabolism changes and becomes exclusively carnivorous \\
52 & The character's metabolism changes and becomes exclusively vegetarian \\
53-54 & Character must attack the closest creature to him (5 \% chance each day). \\
55-57 & Character is stunned for 1d4 rounds each time the item has served its purpose \\
58-60 & The character goes deaf \\
61-64 & Maximum hit points drop by 10 (remaining with a minimum of 1). \\
65 & Maximum hit points drop by 20 (remaining with a minimum of 1). \\
66-70 & The PC must make a Will save each day with an Int modifier \\ & or take 1 Intelligence damage. \\
71-75 & The PC must make a Will save each day or take 1 damage to Wisdom. \\
76-80 & The PC must make a Will save each day with a Charisma modifier \\ & or take 1 damage to Charisma. \\
81-85 & The PC must make a Fortitude save every day or take 1 point of Strength damage. \\
86-90 & The PC must make a Fortitude save each day or take 1 damage to Dexterity. \\
91-95 & The PC must make a Fortitude save every day or take 1 point of Constitution damage. \\
96 & Character is transformed into a specific creature (5 \% chance each day). \\
97 & Character can no longer use magic items or spells above level 5 \\
98 & Character can no longer use magic items or Spells with difficulty over 3 \\
99 & Character can no longer use Spells \\
100 & Double Roll \\
\end{tabular}

\pagebreak

\section{Yeru} \index{Yeru} \index{Atilantis}

\begin{changemargin}{0cm}{0.5cm} \begin{enfasi}{
So the Earth is really round. But I didn't imagine it was blue. Because the men who live on such a beautiful planet do nothing else what to fight with each other? (Nadia - The mystery of the blue stone)

The planet does not belong to us, we belong to it. We are of passage, he stays. (Pierre Rabhi)

}\end{enfasi} \end{changemargin} \medskip

\begin{multicols}{2}

\label{yeru}

\lettrine[lines = 2, lhang = 0.33, loversize = 0.25, findent = 1.5em]{Y}{eru} is the reference planet of OBSS. A planet split both physically and magically.

Two stars Sparka and Andhakara revolve around Yeru. \index{Sparka} \index{Andhakara}

Sparka is of a warm golden color she is the one who brings warmth and light, around her Yeru makes a complete turn in 336 days of 24 hours each.

Sparka only ever lights up the northern hemisphere of Yeru, called Curyan \index{Curyan}.

Andhakara always and only illuminates the southern hemisphere of Yeru, Tiya, and is instead a cold blue star, lifeless, she is the one who brings energetic storms and strange natural occurrences. Bring a cold twilight.

If the 14 (06-20) daylight hours see Sparka and Andhakara protagonists in their dance in the sky; the 10 night hours see the two moons of Yeru named Idam and Kenatu as total protagonists.

The inhabitants of Yeru call them their moons even if in reality they are not really just moons but real inhabited planets.

The two moons are large and imposing on the night sky, Idam of a reddish gray color and Kenata of a warm pearly gray command the tides and influence navigation with their presence.

Yeru has a peculiar and unique distribution of the lands, fruit of the whim of the Gods of Genesis (Ljust and Calicante), you can imagine it as a mirror system on the equator.

The lands do not join the equator, leaving about 200 km of open sea.

The emerged lands that make up the northern and southern hemisphere are almost symmetrical and symbiotic to each other. The shape and subdivision of the large islands are very similar to each other. But from the climatic point of view there are profound differences.

The border open sea area is wild and inscrutable. The deepest and most powerful storms continually discharge their energy and even magic cannot penetrate. In the eye of this perennial and gigantic maelstrom is the civilized and powerful Alantia, considered by many to be a legendary island and the cradle of civilization.

Many areas of Yeru are still unmapped and unexplored, primeval chaos rules these areas and everything becomes possible.

There are few cities that exceed 50,000 inhabitants. Each state has a capital which, due to Yeru's mocking fate, is very often destroyed or disappears. The law is often absent and only that of the strongest is in force.


Wide lands unravel where ancient remains of disappeared civilizations are a refuge for new inhabitants. Layer upon layers of historical civilizations beneath your feet with treasures, secrets, caves and protectors.

Curyan is governed by the force of life, this region experiences a sort of perennial hot season with temperature gradations and atmospheric phenomena that vary depending on the latitude.

Areas with a hot and humid climate intersect with others with dry heat and no rainfall; there are phenomena such as sandstorms in desert areas and severe and devastating tropical storms in lush central bays.
There are areas pleasantly warm and others cooled by fresh breezes from the northern glaciers.

Tiya, on the other hand, is a semi withered hemisphere, the light that arrives is barely enough to allow agriculture and farmed animals have a pale and emaciated appearance.

The richest area is the one closest to the equator where the cold light of Andhakara is only slightly attenuated and makes room for a few rays of Sparka.
In this narrow belt, agriculture is more flourishing and there are fewer devastating meteorological phenomena.

It is the hemisphere where the law of the strongest is in force, where people struggle to live and there are few states that have an effective protection system.

The sea that embraces the equatorial is strong and tumultuous, very few boats venture from one continent to another, this means that exchanges between Tiya and Curyan are almost nil by sea, only very few captains, and secretly, dare to cross the maelstrom.

\subsection{Adventures in Yeru} \index{Adventures in Yeru}

The "problem" for adventurers and explorers is the extreme diversification and mutability not only of the environment but also of the cultures and civilizations that can be encountered.
Since something very important changes every thousand years it is possible that entire islands disappear or appear with civilizations once forgotten, entire cities are suddenly submerged by water or vegetation, or even worse, entire empires become belligerents of the undead (yes, it happened this too, along with a couple of zombie apocalypse that lasted several centuries).

You are never sure what you will find in Yeru !.
Moreover, the worst events are those that affect the rich and vital Curyan while more positive events take Tiya.

Yeru can never be said to be explored, the same area can change from one day to the next because a Patron has decided so. Curious, ineffable, fickle, they are able to build the adventure of life in a clap of hands just to enjoy the show.

Whether you are from Curyan or Tiya your life will not be easy, nothing will be given to you. The young people of both continents escape from misery, abuse, violence to embark on a new life even richer in misery, abuse and violence but which at least is only theirs, the result of their own choices.

When you decided to embark on this new career, or you were torn from the previous one, you knew that it would not be easy, that Yeru himself, through his Patrons, would do everything to defeat and humiliate you, but the Law of the Prize is superior even to Patrons and you would have had your prize even if what remained an unknown.

\medskip

\begin{changemargin}{0.3cm}{0.3cm} \begin{Storytellere}
Use the setting you prefer! Yeru is an example of a chaotic and slightly anarchic world dominated by the ever-changing moods of capricious deities.
Personally I prefer a less high fantasy setting, but any setting from the Forgotten Realms to Golarion to Mystara will do. You are the Storyteller, you are the world, you are the one projecting light and darkness!

The first suggestion I give you is to know \textbf{well} the setting, the more you know, the easier you will be able to adapt to the situation that will happen to you.

\end{Storytellere} \end{changemargin}



\subsection{Notable places of Yeru}

\subsubsection{Kranguran Desert}

In this immense desert gigantic monsters are hidden. Some hidden in the sand like huge dinosaurs use the telluric sense to hunt their prey.

Every creature in this desert is gigantic, monstrous and disproportionate in appearance, as if it were born from someone's nightmare.

The vegetation itself in the few oases present is huge and hypertrophic.

\subsubsection{City of Knandir}

This rich, prosperous and populous ancient city was destroyed overnight by a gigantic cataclysm.

It is said that Cattalm's will was so pervasive that all the buildings were destroyed or severely damaged. Not satisfied with the work, I condemn the city to be out of phase with reality, making it disappear in the eyes of all the others.

The few surviving inhabitants perished in excruciating suffering, condemned not to be able to go out, not to have anything to eat or drink.

The city was cursed and in the few days of the year in which it is possible to reach it, every person who sets foot there to plunder the immense treasures it contains seems condemned to never go out, victim of the curse or of the numerous ghosts, spirits and undead of the previous inhabitants. .

Moreover, the city never appears in the same place but moves following a pattern that is not well understood. The Lost Scroll of Knandir is said to explain its whereabouts.

\subsubsection{The Silent Sea}

There is a particular sea area, between three major islands and containing several minor islands, where any sound is muted. A sound that is generated in those waters, and not on dry land, is silenced. 2 distinct floating cities have arisen dedicated to ancient psionic traditions.

\subsubsection{The tower of the blue gorillas}

The origin of this ancient and magical building is now forgotten, it is said that it was created to challenge a Patron, probably Gradh. The tower, with a square base of 20 meters on each side, is apparently 7 floors high. On each floor, whose map seems to be constantly changing, blue gorillas appear, utterly brutal and with the intention of killing whoever is in the tower. Once the last gorilla on the floor is defeated, the door leading to the stairs to the next floor opens and the characters can go up. With each floor the gorillas become stronger, more resilient and more intelligent. It is known that already on the 4th floor they also acquire magical powers. Characters entered can leave whenever they want, if they die inside the tower they will automatically be teleported outside, but you live at 1 hit point and extremely tired, without the most precious object they had on at the time of death. The current record was reaching the 7th floor. Will new heroes make it to the end (???) of the tower, and what rewards will there be for those who survive?


\end{multicols}

\pagebreak

\subsection{Portals} \index{Portals}

\begin{changemargin}{0cm}{0.5cm} \begin{enfasi}{
Never open doors to those who open them even without yours permit. (Stanislaw Jerzy Lec)
} \end{enfasi} \end{changemargin} \medskip

\begin{multicols}{2}

\label{i-portali}

In a world where sea transfers do not work except between island and island of the same hemisphere as well as the teleportation spell, the ability to use portals to transfer goods and people has taken hold significantly.

This proliferation of small, large, lasting or instant tunnels has caused a rift in the dimensional fabric of Yeru, generating in turn a proliferation of more or less large and lasting spontaneous tunnels.

And these Portals are the cause of many problems both in Tiya and in Curyan as they not only link the two hemispheres but connect all of Yeru to other worlds (or so it is thought since few have returned to report it ..).

There are known and stable portals, until now, that connect Tiya to Curyan, almost all under the control, not to say inside the castle, of royalty or powerful.

There are areas where portals are opened more frequently but the destination is not always certain.

Then there are the dragon portals. Dragons are not native to Yeru but have been drawn to these magical gates, causing havoc and terror to Tiya and Curyan.

The dragons have well understood the nature of Yeru and with their fine intelligence and innate ability to shape magic they have built their portals by bringing hundreds of dragons to come. All evil.

Yes, there are no "good" dragons on Yeru except with a few exceptions.

They have always tried to destroy the dragon portals, with sacrifice and blood. Many have been destroyed, others have been generated. It is an endless war, the only one that can unite the peoples and destinies of the two hemispheres.

\end{multicols}

\subsection{Draghi} \index{Draghi}
\label{draghi}

\begin{changemargin}{0cm}{0.5cm} \begin{enfasi}{
Oh goddamn may Lynx close all portals for you\\
Oh murderers may Sumkjir exterminate you\\
O devastators may Nedraf break your bones!\\
(curses against the Dragons)
} \end{enfasi} \end{changemargin} \medskip

\begin{multicols}{2}

The Dragons are not natives of Yeru but arrived just under 300 years ago bringing with them death and destruction for both Curyan and Tiya.

Tàhil, powerful red dragon, using the magic of interdimensional travel wanted to find new treasures and lands to subjugate unfortunately for him he ventured too far into empty space, where even the light and the stars do not arrive but it flexes and comes back.

In these non-places he was subjugated, dominated, his mind made up by beings beyond human understanding, pure madness and chaos.

Dead, rebuilt, destroyed, reassembled, annihilated, reassembled countless times of its original being nothing remained, only lucid aggressive madness.

When these beings passed to a new game Tàhil created a new magical portal, no longer having a perception or memory of home this led him to Yeru, a different and ambiguous world where he immediately clashed with a Patron, Lynx.

His body had been remade, reforged, reassembled of the same non-matter, his mind a whirlwind of pure chaotic power.

The Patron, perhaps a divinity for that world, proved to be an easy opponent and enjoyed scarifying the body of that weak entity.

Now reached the final blow the Patron opened a portal under him and let himself fall inside closing it immediately.

Tàhil immediately understood the magic used and the potential of the world, how his magic was even more effective.
He opened hundreds of portals calling into that world every abomination he had ever imagined and ... dragons, many, thousands, all the dark dragons he could ever dream of.

The invasion of Yeru had begun. Tàhil immediately reaffirmed his power and leadership by killing dozens and dozens of dragons of all colors (red, green, blue, white, black and purple) in a single day.
The other dragons submitted to his will, remaining chained to the will of Tàhil.

All of Yeru was ravaged for over 60 years by dragons while few perished at the hands of the world's adventurers.

With the utmost effort Gradh managed to involve the Patrons of Genesis and only in this way did Tàhil accept a meeting.

The story is known, Calicanthus saw in Tahil a different and supremely powerful weapon to bring destruction and entropy to Yeru, Ljust saw in Deynos the help and knowledge of those arcane creatures.
An alliance and a fake truce were made.

Calicant deprived Tàhil of some of the madness, but not all of it, he loved the chaos and death that he could bring in so many ways.

Ljust extended Deynos' powers to transform captured dark dragons into good dragons.

Exactly, good dragons.

Up to that moment the one and only dragon that had proved genuinely good was Deynos, who by pure chance, if not by mistake, had used a portal created by Tàhil before it closed.

No other good dragon has ever come to Yeru, the very few present are those who have been captured and transformed by Deynos.

Tàhil has become a wild card, a weapon of pure aggression thrown into the world, to raise the level of Yeru's chaos, revenge, violence.
Dragons, when not commanded by Tàhil, go around Tiya and Curyan to destroy, amass wealth and create legends.

A dragon is a practically unchallenged creature, the most powerful ballistae can hurt them but their breath is certain death.

Whenever Tàhil decides that it is necessary to strengthen his troops, he opens new portals bringing in new dragons to subdue and dominate.

It is an unequal struggle for the poor Yeruites, every time a dragon dies with extreme sacrifice, two others arrive. Where is the lair of Tàhil is a mystery, hidden by the very power of Calicantus that not even Ljust has been able to grasp.

The hope is that sooner or later a legendary group of heroes will find his lair and kill Yeru's greatest enemy.

Note: unfortunately it is not true that dragons are only 300 years old, in reality the last victory of the millennium has made us forget that these creatures have been on Yeru for many millennia and from the same they sow destruction, chaos and death.

\subsubsection{The Colors of Dragons}

Each Dragon has its own typical and peculiar characteristics.
All dragons on Yeru obey Tahil, blindly and without resistance, at least until they are transferred to Yeru via one of his portal.

Whenever Tahil wants to summon a dragon, he opens a portal and a dragon comes out of it, of random color. As soon as Tàhil's magic of domination arrives, he subdues the dragon who can no longer rebel against his orders.

The dragon leads the life he "likes", usually this contemplates the destruction of some cities and hundreds of deaths, only when recalled by Tàhil it stops its activities to fly where required. Or Tahil can send a mental order without the dragon needing to move.
In the last 300 years all dragons have been summoned only once, otherwise only the eldest and most powerful are summoned to act as Tàhil's lieutenants in the territory.


For the Dragons Tàhil is their divinity.

\subsubsection{Black Dragon} \index{Black Dragon}

Black Dragons are violent and aggressive, live in swamps and marshes and generally rule as undisputed masters.

Black Dragons are menacing creatures that have large forward curved horns.
The head connects to a relatively short neck and a large, muscular lizard body.

They have very small wings that are located on the sides, but still manage to fly thanks to magic.
They have webbed legs to allow them to swim more easily in the marshy areas where they live.

Black Dragons tend to make their lairs in the center of the swamp or bog.
They consider that territory their own and no one can get wet without suffering their wrath.

A black dragon's lair can be a gigantic pile of logs but also a submerged underground cave, if not the bottom of a lake.
Being able to breathe underwater, they don't worry about where to build their home.

Their home is always protected by traps and their evil followers who bring them food, possibly live.

The environment where a black dragon lives suffers its effects, acid vapors, destruction, corruption are immediately perceptible.

The Black Dragon represent the traits of selfishness and violence by hating all life, including the black dragons themselves.

Black dragons have + 1d6 on Magic Checks with the School of Necromancy and are immune to acid.


\subsubsection{Blue Dragon} \index{Blue Dragon}

Blue Dragons dwell in the clouds, flying (and levitating) in storms.

Blue Dragons have a snake-like, elongated and binding appearance, with long back horns.

A Blue Dragon's face is less rippled and remains smooth.
They are the only dragons that have no wings while flying better than any other dragon.

Their magical but natural ability to fly combined with the fact of feeding on electricity makes them purely flying creatures that almost never come down to earth (and never touch the ground considering it impure and dirty!), Prefer to stay among the clouds, especially among the darker ones and full of energy to feed

The lair of the Blue Dragon is usually among the highest peaks of the mountains possibly high enough to reach the clouds. This is never covered and often resembles giant nests.

The Blue Dragons can assimilate meat but not vegetables, they do not derive nourishment from what they eat having a purely electric metabolism.

They are social dragons, who love to be around their fellows and are very protective of their offspring.
Usually you never find a nest alone, but entire plateaus dominated by dozens of dragons.

They don't get along well with purple dragons they despise for choosing to have given up flying to live underground.

Blue Dragons master electricity and are immune to damage (magical or otherwise).


\subsubsection{Green Dragon} \index{Green Dragon}

Green Dragons love forests and unspoiled nature where they consider themselves the undisputed masters and kings.

Powerful green dragons have rounded heads and pronounced backward ears, horns are short and not pointed.
The claws and jaws are devastating, powerful and capable of slicing anything.
The nose is wide and the nostrils open as if it were to blow at any moment.

The breath of green dragons is poison, so it can kill living creatures but not plants.

A green dragon's lair is always near a water source, possibly in the most lush and unspoiled part of the forest.

A green dragon does not like to fly and prefers to jump by crushing with its weight and tearing with its claws.

Among the many dragons, the green one is perhaps the one that will make adventurers talk if they show respect and fear of its royalty.

Green Dragons master poisons and are immune to both magical and natural poisons.

\subsubsection{White Dragon} \index{White Dragon}

White Dragons are among the wildest and most "beast" of all dragons.
They love cold, icy places, finding refuge in colder valleys like icy mountain peaks and icy steppes.

The White Dragons have a wild aspect almost always showing their teeth and the claws are extracted to move nimbly on the frozen ground.
They have no movement penalty on these terrains.

They use their natural camouflage to attack and capture prey, they are excellent hunters, very intelligent in exploiting the environment.

Little inclined to magic, however, they know how to blow ice splinters much more frequently than other dragons. It is immune to attacks based on cold and ice.

Their lairs are frozen caverns in the mountains or carved out of the most massive glaciers.

White Dragons master ice and are immune to both magical and natural ice.


\subsubsection{Purple Dragon} \index{Purple Dragon}

Purple Dragons live underground and have adapted perfectly to underground life.
Able to see in the dark as if it were broad daylight, endowed with Telluric Sense, they lost the ability to fly but acquired the ability to dig with the same speed as if they were running.

A Purple Dragon is very territorial and stability a perimeter (about 5 km radius) creates, if not already present, an intricate series of tunnels and caverns for its servants.

A Purple Dragon is very protective of his creatures, those who bring him food and offer them treasures.

With a serpentine appearance, they have fine teeth and enormous claws that continuously grow.

He is strong and courageous, arrogant but not brazen. He is not afraid to fight if he thinks he will win. Always take the battle underground where it can create pits to plunge enemies or escape if necessary.

A Purple Dragon blows a powerful sonic attack that often creates cave collapses, collapses that are completely indifferent to him. It is immune to sound attacks.


\subsubsection{Yellow Dragon} \index{Yellow Dragon}

The Yellow Dragons have scales of various shades of yellow that with the growth take to resemble more and more the color of the sands where they live, from light yellow to brick ocher.

They are very intelligent but being solitary by nature they have no interest in communicating with other races.

They live in deserts where they often ambush their prey by hiding at the bottom of large holes dug in the sand.
As soon as they perceive a movement above them they go out and devour any creature.
They have a passion for dwarf meat which they find tasty even when dry.

The Yellow Dragon, although intelligent, is a death machine and hardly ever comes to terms, only if he is in serious danger.

A Yellow Dragon has a searing breath, though not quite fire. It is immune to Fire attacks.

\subsubsection{Red Dragon} \index{Red Dragon}

The Red Dragon believes himself to be the King of Dragons because of his physical power and the breath capable of melting the stone.

Red Dragons are the largest dragons in both build and wingspan.
Often the scales, of a dark red almost blood, have sharp and elongated edges.

Red Dragons prefer warm mountains and if possible directly inside a volcano.

They fight using their size, wings, bite claws .. in short, everything they are and have at their disposal. A Red Dragon always fights to the death does not retreat, run away or give up a challenge, the pride they are proud of does not allow them to show themselves weak.

A Red Dragon is immune to natural and magical fire.


%Tàhil rosso
%Dyenos argento
%Curyan vita
%Tiya scuro


\end{multicols}

\subsection{The Calendar} \index{Calendar}

\begin{changemargin}{0cm}{0.5cm} \begin{enfasi}{
I have often ended up on a calendar. But never for a date precise. (Marilyn Monroe)

It all began on the thirteenth hour of the thirteenth day of thirteenth month ... We were there to discuss the misprints of the calendars purchased by the school. (The Simpsons)

} \end{enfasi} \end{changemargin} \medskip


\begin{multicols}{2}

\label{il-calendario}

Based on the Kenatu cycle it presents 12 months by 28 days.

These are the names of the months starting from what is defined as the beginning of the year
\bigskip

1 °) Ianas

2 °) Prineva

3 °) Marc

4 °) Epral

5 °) Meea

6 °) Vernam

7 °) Ilai

8 °) Arkast

9 °) Cester

10 °) Koper

11 °) Narava

12 °) Kartan

\bigskip
The week is in turn divided into 7 named days

1 °) Kalint (or Sparka's day, usually a holiday)

2 °) Iratam

3 °) Munrat

4 °) Arai

5 °) Venran

6 °) Kittam

7 °) Viltar

The day is divided into 24 hours

\end{multicols}

\subsection{The thousand-year cycles} \index{The thousand-year cycles}

\begin{changemargin}{0cm}{0.5cm} \begin{enfasi}{

Then I saw an angel coming down from heaven with the key of the Abyss and a large chain in hand.

He grabbed the dragon, the ancient serpent - that is, the devil, satan - and it chained for a thousand years;

he threw him into the abyss, shut him up and sealed the door above him, so that he might no longer seduce the nations, until the thousand is fulfilled years. After this it will have to be dissolved for some time. (Apocalypse 20: 1-3, apostle John)
} \end{enfasi} \end{changemargin} \medskip


\begin{multicols}{2}

The myth says that every thousand years the Yeru dies to be reborn again, more beautiful than before.

It is not quite like that but it comes very close.

Few scholars of Atmos know that every thousand years the Patrons recognized and from which many derive their powers disappear and leave their place, after exactly 1 year to new Patrons.

Suddenly the spells stop working, only magic items that can absorb and retain magic work (such as a Potion, Armor or Weapon if not a Ring or Staff that has charges, but not items that automatically reload like the Rods), even Devotees or Followers no longer have access to any Magic Lists.

With some exceptions. The Patrons of Genesis, Atmos and Lynx and the Conqueror are the only ones who remain constant and do not change and only their Devotees and Followers can continue to use the known Lists of Magic.

Starting from the sixth month, the old Followers and Devotees begin to hear voices, to dream of new faces and new Patrons.

Each new Patron, according to the Traits he commands, approaches a believer and tries to convince him to accept him as the new Patron.

This Follower / Devotee must have at least two Traits in common with the new Patron to be his Follower and at least 3 to remain Devoted to him (as always).

Spellcasters will only be able to use spells at the end of the year, regardless of whether they follow a Patron or not.

It is an extremely turbulent and agitated period where wars and revenge break out taking advantage of the absence of magic. For many it is a period of pure hatred and violence where the lowest instincts are vented knowing then that they will not be judged by any divinity.

The truth is that every thousand years the Patrons of the Genesis judge their creatures, the Patrons, evaluating who did better and who did worse. It is a challenge between Calicante and Ljust to those who, through the Patrons, have obtained the most Followers and Devotees.

The Patron who more than all has shown himself capable of maintaining and conquering people will remain even in the following millennium, this will be the Winner and his believers will sing of his glory and power for another thousand years.

Intoxicated by the victory, the Patron of Genesis will express a desire that the other must try to respect as much as possible. Obviously he / she herself could manifest it but the satisfaction of making the other do something he or she hates is superior to everything.

And this is why something happens every thousand years, in addition to the birth of new Patrons. It can be a continent, sea ... moon, animals ... something massive changes for all yeruits. It is a time of global upheaval.

Only the highest Devotees of Atmos know this truth as they know that the Patrons of Genesis after the victory lie together for six months generating the new Patrons.

\bigskip

\begin{changemargin}{0.3cm}{0.3cm} \begin{Storytellere}
Consider well when to start your campaigns, based on the duration and the year you could run into these events.
Use the change of Patrons in your favor and benefit of adventure, let magic play some "rest", help players with more magical characters to recover.
\end{Storytellere} \end{changemargin}

\end{multicols}

\pagebreak

\section{The Planes}

\begin{multicols}{2}

\lettrine[lines = 2, lhang = 0.33, loversize = 0.25, findent = 1.5em]{E}{ven} if infinite adventures await you in the game, there are other worlds besides this, other continents, other planets, other galaxies . However, even beyond the existence of innumerable planets there are other worlds, completely different dimensions from reality, known as planes of existence. Except for rare connecting points that allow you to travel between them, each plane is a universe in itself with its own natural laws.

Although the number of planes is limited only by the imagination, they can all be reduced to five general types: the Material Plane, the Transition Planes, the Inner Planes, the Outer Planes, and the innumerable demiplanes.


\begin{changemargin}{0.3cm}{0.3cm} \begin{Storytellere} %box Storytellere
Consistent with the environment, the Planes should not be reachable. Lynx does not allow it. Yeru was born as a closed and isolated planet although this did not stop the Dragons and other fiends from arriving. The Storyteller decides Yeru's level of isolation.
\end{Storytellere} \end{changemargin}


\subsection{What is a Plan?}

The planes of existence are actually different universes, intertwined links. Except for rare connecting points, each plane is effectively a universe for its own sake, with its own natural laws. The floors are divided into a series of general types: the Material Plane, the Transition Planes, the Inner Planes, the Outer Planes and the Semi-planes.

\textit{Material Plane}: The Material Plane tends to be the closest to earth and to function using the same natural rules. Its "size" depends on the campaign: it can only conform to the actual game world, or encompass an entire universe of planets, moons, stars and galaxies. The Material Plane is the ground plane for the game.

\textit{Planes of Transition}: These planes have an important element in common: they are all superimposed on the other planes and serve to travel between superimposed realities. These planes are strongly interconnected with the Material Plane, and can be accessed using numerous spells. They also have native inhabitants. Some Transition Plans are described below.

\begin{itemize}
\item
\textit{Astral Plane}: A silvery void connecting the Material Plane to the Inner and Outer Planes, the Astral Plane is the medium through which the souls of the departed reach the afterlife. A traveler in the Astral Plane sees the plane as an infinite void periodically punctuated by tiny flashes of physical reality detached from the innumerable superimposed planes. Powerful spellcasters use the Astral Plane for a short split second when teleporting, or they can use it to travel between planes with spells such as Astral Projection.

\item
\textit{Ethereal Plane}: The Ethereal Plane is a hazy and hidden dimension coexisting with the Material Plane and the Plane of Shadows. Travelers crossing the Ethereal Plane experience the real world as being insubstantial and can move between solid objects without being seen in the real world. Bizarre creatures inhabit the Ethereal Plane, as do ghosts and dreams, many of which can at times extend their influence into the real world in mysterious and terrifying ways. Powerful spellcasters use the Ethereal Plane with spells such as Ethereal Form, Intermittence.

\item
\textit{Plane of Shadows}: The mysterious and deadly Plane of Shadows is a gray, colorless version of the Material Plane. It overlaps the Material Plane but is smaller, and is in many ways a distorted and perverse "reflection" of the Material Plane, infused with negative energy (see Inner Planes) and inhabited by terrifying monsters such as shadows or even worse creatures. Powerful spellcasters use the Plane of Shadows to quickly travel immense distances on the Material Plane.
\end{itemize}

\medskip

\textit{Inner Planes}: These planes are the embodiments of the basic building blocks of the universe. They can be seen as "containing" the Material Plane, but they are not superimposed on it like the Transition Planes. They are composed of a single type of energy or element, superior to all others. The inhabitants of a specific Interior Floor themselves are composed of the floor element. Among the Inner Planes there are:


\medskip

\begin{itemize}
\item
Elemental Planes: The four classic Inner Planes are the Plane of Water, Plane of Air, Plane of Fire and Plane of Earth. From these planes come the creatures known as elementals, but they are also inhabited by other bizarre creatures, such as geniuses, xorns, mephits and invisible persecutors.

\item
Planes of Energy: There are two planes of energy, the Positive Energy Plan (where the life sparks come from) and the Negative Energy Plan (where the corruption of undeath comes from). The energy of both planes is infused into reality, and the flow of this energy flows through all creatures from birth to death. Devotees use the power of these two planes when channeling energy.

\end{itemize}

\medskip
\end{multicols}
\pagebreak

\vfill

\begin{center}
\includegraphics[width = 0.96 \linewidth]{immagini/mappaplanare3.png} \\
\medskip
\textit{Planar Map. Licensed by the author https://www.reddit.com/r/ImaginaryGolarion/comments/97rog0/pathfinderhardwar}
\end{center}
\pagebreak
\begin{multicols}{2}

\textit{Outer Planes}: Beyond the mortal realms, beyond the elements of reality, are the Outer Planes. Vast beyond imagination it is to them that the souls of the dead reach and it is here that the deities dwell. Each of them has its own set of Traits, which represent a particular moral or ethical aspect, and their inhabitants tend to behave following these Traits.

The Outer Planes are also the final resting place of spirits from the Material Plane, whether they are destined for calm introspection or eternal damnation. The inhabitants of the Outer Planes form the mythologies of civilizations, including angels and demons, titans and devils, and countless other incarnations of the possible. Each game world should have different Outer Planes that conform to specific themes and needs, but classic Outer Planes include Heaven (legal and good traits), Abyss (anarchic and evil traits), Hell (legal and bad) and Elysium (freedom and goodness). Powerful spellcasters can contact the Outer Planes for guidance and counsel with spells such as Communion and Contact Other Planes, or they can summon allies with Summon spells.

\textit{Half-planes}: This category is used to collect all other extra-dimensional spaces which function like planes but which have measurable and limited access and dimensions. The other types of planes are theoretically infinite in size, but a demiplane could be only a few hundred meters long. There are countless demiplanes adrift in reality, and while many are connected to the Astral Plane and the Ethereal Plane, others are completely cut off from the Transition Planes and can only be reached through well-hidden portals or dark magic.

\subsection{Multilayer tops}
Infinity can be divided into smaller infinities and planes into smaller planes related to each other. These layers are in effect separate planes of existence, and each layer can have its own particular characteristics. The layers are connected to each other through numerous planar portals, natural eddies, paths and shifting boundaries.

Accessing a multi-layered plane from another source usually occurs on a specific layer, the first tier of the plane, which can be either the top or the bottom, depending on the plane in question. Many fixed access points (such as portals and natural eddies) lead up to this layer, which becomes the gateway to the other layers of the plane. The spell \textit{Planar Shift} also deposits the caster on the first layer of the plane.

\subsection{Planar Interaction}
Two floors that are separated from each other do not overlap and do not connect directly to each other. They are like planets on different orbits. The only way to move from one floor to another is to go through a third floor, such as a Transition Plane.

\textit{Adjacent Floors}: Those planes that connect to each other at specific points are considered to be adjacent. Wherever they touch, there is a connection through which travelers can leave one reality and enter the other.

\textit{Coexisting Planes}: If it is possible to create a link between two planes at any point, the two planes coexist. These planes overlap each other completely. A coexisting plane can be reached from any point on the plane on which it is superimposed. When moving on a coexisting plane, one can often see or interact with the superimposed plane.

\subsection{Planar Features}
Each plane of existence has its own peculiarities; the natural laws of his universe. The planar features are divided into general areas. All plans have the following features.

\textit{Physical Characteristics}: They determine the physical and natural laws of the plane, including the functioning of gravity and time.

\textit{Elemental and Energy Characteristics}: The influence of elemental and energetic forces is determined by these traits.

\textit{Traits}: Just as characters can have traits, so many planes are tied to a particular moral or ethics.

\textit{Magical Characteristics}: Magic works differently from plane to plane; magical traits mark the boundary between what magic can and cannot do on every plane.

\textit{Physical Characteristics}

The two most important natural laws determined by physical traits concern the function of gravity and time. Other physical characteristics concern the size and shape of a plane and the way in which its nature can be altered.


\subsection{Severity}

The direction of gravitational attraction can be unusual, and may even change directions within the same plane.


\subsection{Time}

The pace at which time passes may vary in different planes, although it remains constant within any specific plan. Time is always subjective for the viewer. The same subjectivity applies to the various planes. Travelers may find they are gaining or wasting time moving between floors, but from their perspective, time passes naturally.


\textit{Normal Time}: Defines the passage of time on the Material Plane. One hour on a plane with normal time equals 1 hour on the Material Plane. Unless otherwise specified in the description of a plan, it is assumed that it is characterized by normal time.
\textit{Irregular Time}: Some floors are characterized by slowing down and speeding up time, whereby an individual may lose or gain time while moving between floors like this and others. For the inhabitants of such a plane, time passes naturally and the displacement goes unnoticed.


\textit{Flowing Time}: On some planes, the flow of time is considerably faster or slower. Someone may travel to another plane, spend a year there, and then return to the Material Plane to find that only 6 seconds have passed. Anything on the floor you returned to lived for just a few seconds longer. For the traveler and the objects, spells and effects working on him, that year of being away was completely real. When designing the functioning of time on planes with flowing time, think first of the time flow of the Material Plane, then of the flow present in the other plane.


\textit{Timelessness}: On floors with this characteristic, time passes but its effects are limited. How timelessness affects certain activities and conditions such as hunger, thirst, aging, the effects of poison, and healing varies from floor to floor. The danger of a timeless plan is that when an individual abandons that plan to go to another where time flows normally, conditions such as hunger and aging occur retroactively. If a plane has timelessness in relation to magic, any spells cast with a non-instant duration become permanent until dispelled.


\subsection{Elemental and Energy Characteristics}

Four basic elements and two types of energy combine to shape everything; the elements are water, air, fire and earth; the types of energy are positive and negative. The Material Plane reflects a balance of these elements and energies: it is possible to find them all in it. Each of the Inner Planes is dominated by an element or type of energy. Many planes have no elemental or energetic characteristics; these characteristics are specified in the description of a plan only if present.


\textit{Water Dominant}: Planes with this characteristic are mostly liquid. Visitors who cannot breathe underwater or who cannot reach an air pocket are likely to drown. Fire creatures are found extremely uncomfortable on water-dominant planes. Such creatures, made up of fire, take 1d10 points of damage each round.

\textit{Dominant Air}: Consisting essentially of open space, floors with this feature house just a few pieces of floating stone or other solid material. They usually have a breathable atmosphere, although such a plane could have clouds of acidic or toxic gas. Creatures of the Earth Subtype are uncomfortable on air dominant planes due to the scarce amount or absence of natural earth with which to come into contact. However, they do not suffer any actual damage.

\textit{Ruling Earth}: Planes with this characteristic are mostly solid. Travelers arriving there are at risk of suffocation unless they reach a cave or other ravine. Worse still, individuals who do not possess the Dig ability get trapped underground and have to dig their own way out (1 meter per round).
Creatures of the Air Subtype are uncomfortable on land-dominated planes as they consider them cramped and claustrophobic, but apart from having difficulty in movement they do not run into other drawbacks.

\textit{Ruling Fire}: Floors with this characteristic consist of flames that burn continuously without depleting their power source. Fire-dominant planes are extremely hostile to Material Plane creatures, and those without resistance or immunity to fire are quickly incinerated. Wood, paper, unprotected fabric and other flammable materials ignite almost instantly, as do those who wear unprotected and flammable clothing. In addition, individuals take 3d10 points of fire damage for each round they remain on a fire-dominant plane. Creatures of the Water subtype are found extremely uncomfortable on fire-dominant planes. Such creatures, made up of water, take double damage each round.

\textit{Dominant Negative Energy}: The planes with this characteristic are vast and empty recesses that suck the life essence of the travelers who pass through them. They tend to be desert and haunted floors, stripped of color and filled with winds that carry the faint moans of those who have died within them. There are two types of traits based on dominant negative energy: lower and higher dominant negative energy. In the former, living creatures take 1d6 points of damage per round. At 0 hit points or less, these are reduced to ashes.

The latter are even more dangerous. Each round, those inside must make a DC 25 Fortitude save, the maximum hit points drop by 6 if he dies this way he becomes a Wraith. The death ward spell protects the traveler from the damage and energy drain of a plane with dominant negative energy.

\textit{Dominant Positive Energy}: The abundance of life marks the planes that exhibit this characteristic. As with the dominant negative energy planes, the dominant positive energy planes can also be minor and higher.
A minor dominant positive energy plane is a tumultuous burst of life in all its forms. The colors are brighter, the fires warmer, the noises louder and the sensations more intense thanks to the positive energy diffused in the plane. All individuals in a dominant positive energy plane regenerate 2 hit points per round.

The higher dominant positive energy planes go even further. A creature on one of these planes must make a DC 15 Fortitude save to avoid being blinded by the surrounding glow for 10 rounds. Simply being on the plane grants 5 HP regeneration per round. In addition, those with their maximum hit points gain 5 additional temporary hit points per round. Those temporary hit points expire 3d6 rounds after the creature leaves the plane. However, a creature must make a DC 20 Fortitude save for each round that the amount of temporary hit points exceed its normal maximum hit points. Failing this saving throw causes the creature to explode in a blaze of energy, dying.


\subsection{Traits}

Some plans have a predisposition towards a specific set of Traits. The inhabitants of these planes mostly share this set of Traits or part of them, even powerful creatures such as gods. The set of Traits of a plan influences its social interactions. Characters who have different Traits than most villagers may have a hard time dealing with the natives and situations on the plane. Traits have multiple components. First of all there are the moral and ethical components. Secondly, there could be a specific indication as to whether this set of Traits manifests itself moderately or more markedly. Many floors have no Traits; the latter are specified in the description of a plan only if present

In principle, the elemental, astral and hetero planes have no trait.

\subsection{Magical Features}

The magical characteristics of a plane defines the magic on that plane relative to the Material Plane. In particular places on a plane (such as those under the direct control of the deities) a different magical feature may apply.


\textit{Normal Magic}: This magical trait implies that all spells and supernatural abilities act as described. Unless otherwise described, a plane is assumed to have the normal magic trait.

\textit{Dead Magic}: Marks planes where magic doesn't exist at all. A plane with the dead magic characteristic functions in all respects as an antimagic field. Divination spells cannot detect someone in a dead magic plane, nor can a spellcaster use the teleport spell to move in and out of it. The only exception to the "no magic" rule are permanent planar portals, which still function normally.

\textit{Enhanced Magic}: On planes with this magical characteristic, particular spells and magical abilities are easier to use or produce more powerful effects than they do in the Material Plane. Natives of a plane are aware of which spells and magical abilities are enhanced, but planar travelers may find out on their own. If a spell is empowered, it functions as if it had hit a critical spell in the Magic Check.

\textit{Hindered Magic}: Particular spells and magical abilities are more difficult to use on planes with this characteristic, often because the nature of the plane hinders them. To cast an obstructed spell, he must gain a critical in the Magic Check. If the check fails, the spell has no effect but is still wasted. If the check is successful, the spell takes effect normally.

\textit{Limited Magic}: Plans with this characteristic only allow the use of spells and magical abilities that meet particular requirements. Magic can be limited in its effects by certain schools or sub-schools, by effects with certain descriptors, or by effects of a given level (or by any combination of these aspects). Spells and spell-like abilities that don't meet the requirements simply have no effect.

\textit{Wild Magic}: On a plane with the wild magic trait, spells and magical abilities work in a totally different and sometimes dangerous way. There is a possibility that any spells or spell-like abilities used on a wild magic plane will have no effect. When the caster casts a spell, he must make two checks, if even one fails, something unusual happens; roll a d100 and consult the

\medskip

\end{multicols}

\pagebreak

\textbf{Table: Wild Magic Effects.} \index{Wild Magic Effects Table}

\medskip


\begin{tabularx}{0.95\textwidth}{lX}
d100 & Effect \\
\toprule
01-19 & The spell bounces off the caster with normal effect. If the spell cannot affect the caster, it has no effect. \\
20-23 & A circular pit with a diameter of 4.5 meters opens under the feet of the caster; its depth is 3 meters per spellcasting skill. \\
24-27 & The spell has no effect, but its target (s) are hit by a shower of small objects (anything from flowers to rancid fruit), which disappear as soon as they hit. The attack continues for 1 round. During this time, targets are blinded and must make Concentration checks (DC 15 + the spell's level) to cast spells. \\
28-31 & The spell hits a random target or area. Randomly choose a different target from those within the spell's range or hit the spell in a random location within that range. To randomly generate direction, roll 1d8 and count clockwise, starting from the south. To randomly generate range, roll 3d6. Multiply the result by 1 meter for short range spells, 6 meters for medium range spells, and 24 meters for long range ones. \\
32-35 & The spell functions normally, but any material components are not consumed. The spell does not disappear from the caster's mind (a spell slot or prepared spell can still be used). Likewise, an object does not lose stacks, and the effect does not affect the usage limit of an object or spell-like ability. \\
36-39 & The spell has no effect. Instead, someone (friend or foe) within 30 feet of the caster receives the effect of a heal spell. \\
40-43 & The spell has no effect. Instead, the effects of deep darkness and silence cover a 30-foot radius around the caster for 2d4 rounds. \\
44-47 & The spell has no effect. Instead, a reverse gravity effect covers a 30-foot radius around the caster for 1 round. \\
48-51 & The spell takes effect, but sparkling colors swirl around the caster for 1d4 rounds. Treat this area as a shimmering dust effect on a DC 10 saving throw + the level of the spell that generated that result. \\
52-59 & Nothing happens. The spell has no effect. Any material component is used. The spell or spell slot is used, an object loses stacks, and the effect affects the usage limit of an object or spell-like ability. \\
60-71 & Nothing happens. The spell has no effect. Any material components are not used. The spell does not disappear from the caster's mind (a spell slot or prepared spell can still be used). An object does not lose stacks, and the effect does not affect the usage limit of an object or spell-like ability. \\
72-98 & The spell takes effect normally. \\
99-100 & The spell has enhanced effect. Trial of Magic automatically generates a critical \\
\end{tabularx}

\addvspace{2cm}

\begin{multicols}{2}


\subsection{Material Plane}
The Material Plane is the hub of most cosmologies and defines what can be considered normal. This is the level on which most of the campaigns are focused. \\
The Material Plane has the following features: \\
\textit{Normal Severity} \\
\textit{Normal Time} \\
\textit{No Elemental or Energy Traits}: However, specific locations may have such traits. \\
\textit{Moderately Neutral}: Although in some places it may present high concentrations of evil or good, law or chaos. \\
Normal Magic \\

\subsection{Plane of Shadows}
The Plane of Shadows is a dimly lit dimension which at the same time coincides and coexists with the Material Plane. It overlaps the Material Plane as much as it does the Ethereal Plane, so the planar traveler can use the Plane of Shadows to cover great distances quickly. The Plane of Shadows also coincides with other planes. With the right spell, a character can use the Shadow Plane to visit other realities. The Plane of Shadows is a world in black and white: the environment is devoid of color. If it weren't for that, it would resemble the Material Plane. Despite the absence of light sources, some plants, animals and humanoids consider the Plane of Shadows to be their home.

The Shadow Plane has the following characteristics:

\textit{Imperfect geography}: Parts of the Plane of Shadows continually flow to other planes. Therefore, despite the presence of landmarks, creating a precise map of the plan is almost impossible. Additionally, some spells, such as Shadow of a Summon and Shadow of an Invocation, modify the basic structure of the Plane of Shadows. These spells within the Plane of Shadows are particularly useful for both explorers and natives.

\textit{Traits}: Unruly, Free, Superficial, Vengeful, Pessimistic

\textit{Empowered Magic}: Spells working with shadow are empowered on the Plane of Shadows. Despite the dark nature of the Plane of Shadows, spells that generate, use, or manipulate darkness are unaffected by the plane.

\textit{Hindered Magic}: Spells of light or that use or generate light or fire are hindered on the Plane of Shadows. Spells that generate light are less effective in general, since all light sources on this plane have a half range.


\subsection{Negative Energy Plan}
There is very little for an observer to see on the Negative Energy Plane. It is a dark and empty place, an infinite pit into which the traveler could fall until the plane has erased light and life. The Negative Energy Plane is the most hostile of the Inner Planes, the most indifferent and intolerant towards life. Only creatures immune to its sucking effects can survive here.

The Negative Energy Plan has the following characteristics:

\textit{Greater Dominant Negative Energy}: Creatures that are not undead take 10 HP of Void damage per round. Upon death, one becomes a Wraith.

In Lesser Dominant Negative Energy zones, creatures that are not undead take 2 HP of Void damage per round.

\textit{Empowered Magic}: Spells and spell-like abilities that use negative energy are empowered. Skills that use negative energy, such as channel negative energy, gain a +4 bonus to the DC on the saving throw for resisting the ability.

\textit{Hindered Magic}: Spells and magical abilities that use positive energy (including healing spells) are hindered. Characters on this plane must Critically pass the Magic Check to cast spells that heal or remove negative effects.


\subsection{Positive Energy Plan}
The Positive Energy Plane has no surface and is similar to the Plane of Air with its totally open space. However, every corner of this plane is brightly illuminated by an innate power. Such power is dangerous to the mortal forms, not predisposed to suffer it. Despite its beneficial effects, it is one of the most hostile Inner Planes. A character with no defenses will overflow with power as soon as positive energy is channeled onto him. But since its mortal form is unable to contain this power, it will be incinerated, like a speck of dust captured at the tip of a supernova. Visits to the Positive Energy Plan are short-lived, and even in this case travelers must be adequately protected.
The Positive Energy Plan has the following characteristics:

\textit{Greater Dominant Positive Energy}: Every 10 rounds, the effect of the Greater Restoration spell is affected. You regenerate 10 HP per round, once a maximum of 10 HP is taken in temporary rounds, when the temporary HP reaches double the maximum HP the creature explodes into colored energy.

Minor dominant positive energy zones are affected by the lesser restoration spell every 10 rounds. You regenerate 2 HP per round, once a maximum of 2 HP is taken in temporary rounds, when the temporary HP reaches double the maximum HP the creature bursts into colored energy.

\textit{Empowered Magic}: Spells and magical abilities that use positive energy are empowered. Skills that harness positive energy, such as channel positive energy, gain a +4 bonus to DC for resisting the ability.

\textit{Hindered Magic}: Spells and spell-like abilities that use negative energy (including spells inflict) are hindered.

\subsection{Elemental Plane of Water}
The Water Plane is a sea without a bottom or surface, a liquid environment illuminated by diffused light. It is one of the more hospitable Inner Planes, once the traveler overcomes the problem of breathing underwater. The infinite oceans of this plane range from freezing cold to incandescent heat and between salt water and fresh water. The permanent settlements of the plane spawn around pieces of wrecks suspended in this endless fluid, drifting with the tides.

The Water Plan has the following characteristics:

\textit{Dominant Water}

\textit{Empowered Magic}: Spells and magical abilities that use water are empowered.

\textit{Hindered Magic}: Spells and magical abilities that use or create fire, and spells that summon fire elementals or outsiders of the Fire subtype are hindered.


\subsection{Elemental Plane of Air}
The Plane of Air is an empty plane, consisting of sky in every direction. It is the most comfortable and liveable of the inner floors and is the home of all kinds of creatures of the air. Indeed, flying creatures gain a great advantage on this plane. While travelers may survive well here even without the ability to fly, they are still handicapped.
The Air Plane has the following characteristics:

\textit{Dominant Air}

\textit{Empowered Magic}: Spells and magical abilities that use, manipulate, or generate air are empowered.

\textit{Hindered Magic}: Spells and magical abilities that use or generate earth and spells that summon earth elementals or outsiders of the earth subtype are hindered.


\subsection{Elemental Plane of Fire}
On the Plane of Fire everything is illuminated. The soil is made up of nothing but vast and changing layers of compressed fire. The air is stirred by the heat of the continuous rains of fire and the most common liquid is magma. The oceans are made up of liquid flame and the mountains pour molten lava. Here the fire lasts without power or air, but the flammable elements introduced on the hob are consumed quickly.

The Plane of Fire has the following characteristics:

\textit{Dominant Fire}

\textit{Greater Dominant Fire}: Each round you take 10 HP of non-resistable fire damage if you are not immune to fire.

\textit{Lower Dominant Fire}: Each round takes 2 HP of fire damage.

\textit{Empowered Magic}: Spells and magical abilities that use, manipulate, or generate fire are empowered.

\textit{Hindered Magic}: Spells and magical abilities that use or generate water and spells that summon water elementals or outsiders of the Water subtype are hindered.


\subsection{Elemental Plane of Earth}
The Earth Plane is a solid place made up of earth and stone. An imprudent traveler could find himself buried by this vast solid mass: its pulverized remains will remain a warning for those who dare to follow him. Despite its solid and rigid nature, the Earth Plane has variable consistency, ranging from soft ground to veins of harder and more precious metal.

The Earth Plan has the following characteristics:

\textit{Ruling Earth}

\textit{Empowered Magic}: Spells and magical abilities that use, manipulate, or generate earth or stone are empowered.

\textit{Hindered Magic}: Spells and spell-like abilities that use or generate air and spells that summon air elementals or outsiders of the air subtype are hindered.


\subsection{Ethereal Plane}
The Ethereal Plane coexists with the Material Plane and often with other planes as well. The Material Plane itself is visible from the Ethereal Plane, but appears silent and indistinct; the colors blend together and the borders are blurred. Although it is possible to see the Material Plane from the Ethereal Plane, the latter is usually invisible to those on the Material Plane. Normally, creatures from the Ethereal Plane cannot attack creatures from the Material Plane, and vice versa. A traveler on the Ethereal Plane is invisible, incorporeal, and totally silent to someone on the Material Plane.

The Ethereal Plane has the following characteristics:

\textit{Absence of Gravity}

\textit{Normal Magic}: Spells function normally on the Ethereal Plane, even if they do not cross the Material Plane. The only exceptions are spells and magical abilities and abilities that affect ethereal entities.

No magical attacks go from the Ethereal Plane to the Material Plane, including power attacks.


\subsection{Astral Plane}
The Astral Plane is the space between inner and outer planes and borders on all planes. When a character passes through a portal or projects his spirit onto another plane of existence, he travels through the Astral Plane. Spells that allow instant movement across a plane also have a minor effect on the Astral Plane. The latter is a great endless expanse of clear silvery sky, both above and below. Occasional pieces of solid matter can be found here, but most of the Astral Plane is open and boundless space.

The Astral Plane has the following characteristics:

\textit{Timelessness}: Age, hunger, thirst, suffering (such as Illnesses, Curses and Poisons) and natural healing have no effect in the Astral Plane, although they resume functioning when the traveler leaves the floor.

\textit{Enhanced Magic}: All spells and magical abilities used in the Astral Plane have a speed of 1 Action. Already speeded up spells and spell-like abilities are unaffected, as are spells from magical items. Spells speeded up in this way are still prepared and cast at their original level.

\subsection{Other Plans}

The following are some of the best known Planes (internal or external), many others exist even if only a few daring ones have managed to reach them (spontaneously!).

\subsection{Abaddon}
A realm of wastes under a putrid sky, Abaddon is shrouded in a sickening black fog, and the oppressive twilight of an endless solar eclipse. The Styx poisoning was born in Abaddon, before entering the other planes like a twisted snake. Abaddon is one of the most hostile Outer Planes: kingdom of Daemons, filthy of pure evil indifferent to the conflict between law and chaos, who represent oblivion and destruction. Daemons ruled by four arcidaemons with god-like powers are feared as eaters of souls.

Abaddon has the following characteristics:

\textit{Traits}: Destructive, Relentless, Incontentable, Irrational, Wrathful, Sadistic

\textit{Empowered Magic}: Evil spells and magical abilities are empowered.

\textit{Hindered Magic}: Benevolent spells and spell-like abilities are hindered well.

\subsection{Abyss}
The Abyss, multi-layered plane, surrounds the Outer Sphere like the extended peel of an onion; it is made up of gigantic canyons and gorges that open up into the fabric of the Outer Floors and is bordered by the nefarious waters of the River Styx. The infinite strata of the Abyss, bordering all the Outer Planes, are connected to each other by constantly shifting paths. In the Abyss there are no rules, laws, order or hope. The Abyss represents the corruption of freedom, a nightmare realm of absolute horror where desire and suffering take on demonic form, a land of proliferation of countless races of Demons, among the most ancient beings of all creation.

The Abyss has the following characteristics:

\textit{Traits}: Anarchist, Vengeful, Touchy, Arrogant, Double agent

\textit{Strongly Chaotic and Strongly Evil}

\textit{Empowered Magic}: Chaotic or evil spells and magical abilities are empowered.

\textit{Hindered Magic}: Legal or good spells and magical abilities are hindered.


\subsection{Elysium}
A vast land of untouched and wild nature, Elysium is the plane of benevolent chaos, freedom and independence, personified in the Azata natives. In the Elysium, selfless cooperation and fierce competition collide violently, but such conflicts never overshadow the noble concepts of courage, creativity and good unhampered by rules or laws.

Elysium has the following features:

\textit{Traits}: Good, Charitable, Anarchist, Innovative, Competitive

\textit{Empowered Magic}: Chaotic or good spells and magical abilities are empowered.

\textit{Hindered Magic}: Legal or evil spells and magical abilities are hindered.


\subsection{Hell}
The nine layers of Hell form a structured labyrinth of premeditated evil where torment goes hand in hand with purification. Plane of iron cities, burning wastes, frozen glaciers and endless volcanic peaks, Hell is divided into nine concentric layers, each under the cruel rule of an archdevil. Torture, anguish and suffering are inevitable in Hell, but they are imparted methodically, not out of spite or whim, and support a planned design under the watchful eye of the disciplined ranks of Hell's Lesser Devils. The nine layers of Hell, from first to last, are Avernus, Dite, Erebus, Phlegethon, Stygia, Malebolgia, Cocitus, Caina and Nessus.

Hell has the following characteristics:

\textit{Traits}: Wicked, Disciplined, Wrathful, Sadistic, Arrogant

\textit{Strongly Lawful and Strongly Evil}

\textit{Empowered Magic}: Legal or evil spells and magical abilities are empowered.

\textit{Hindered Magic}: Chaotic or good spells and spell-like abilities are hindered.


\subsection{Limbo}
A vast ocean of unbridled chaos and untapped potential surrounds and adjoins each of the Outer Planes. This is Limbo: splendid, deadly and truly infinite. From its unexplored depths all other planes were born and in its rebellious bowels, in the end, all creation will return. Where the shapeless sea of Limbo bathes the coasts of other planes, its mass assumes a certain degree of stability, and it is in these borderlands that the journey is safer, even if still fraught with dangers deriving from the inhabitants corrupted by the chaos of the Limbo. Deeper into the plane, the native Proteans of Limbo plunge into the Primordial Chaos, creating and destroying the raw matter of chaos with incomprehensible transport. \
Limbo has the following characteristics:

\textit{Traits}: Anarchist, Incoherent, Instinctive, Irrational, Unruly

\textit{Irregular Time}

\textit{Wild Magic and Normal Magic}: On the few stable islands of Limbo, magic is more likely to be normal. In any other place the magic is wild.


\subsection{Nirvana}
Nirvana is an impartial paradise existing between the two extremes of Elysium and Paradise. Its stunning mountains, hills and dense forests meet the traveller's expectations of a pastoral paradise, but Nirvana also contains mysteries that lead to enlightenment. Nirvana is a sanctuary and a resting place for all who are seeking redemption or enlightenment. The native Agathions of Nirvana have willingly set aside their transcendence to guard the enigmas of the plane, while the celestials battle the forces of evil present between the planes.

Nirvana has the following characteristics:

\textit{Traits}: Good, Kind, Calm, Simple, Safe

\textit{Empowered Magic}: Good spells and magical abilities are empowered.

\textit{Hindered Magic}: Spells and evil spell-like abilities are hindered.


\subsection{Paradise}
The towering mountain of Heaven towers above the Outer Sphere. This orderly reign of honor and compassion is divided into seven layers. The slopes of Paradise are full of tidy and well-structured cities and neat and tidy gardens and orchards. Although they began their lives as mortals, the native Archons of Paradise see law and good as the two inseparable halves of the same supreme concept and take sides against the cosmic corruptions of chaos and evil.

Paradise has the following characteristics:

\textit{Traits}: Good, Stiff, Fighting, Practical. Sincere, Valiant

\textit{Empowered Magic}: Legal or good spells and magical abilities are empowered.

\textit{Hindered Magic}: Chaotic or evil spells and spell-like abilities are hindered.


\subsection{Purgatory}
Each soul passes through Purgatory to be judged before being sent to its ultimate destination. Vast cemeteries and wastelands fill its gloomy expanses, along with dusty and echoing courts serving the judgment of the dead. Purgatory is the abode of the Aeons, a race which embodies the dualistic nature of existence and whose members are constantly at war and at peace with each other and with themselves.

Purgatory has the following characteristics:

\textit{Timelessness}: Age, hunger, thirst, suffering (such as Illnesses, Curses and Poisons) and natural healing have no effect in Purgatory, although they resume their functioning when the traveler leaves the plane .

\textit{Empowered Magic}: Spells and magical abilities affecting death or rest are empowered.


\subsection{Utopia}
Utopia is a stronghold of order contrasted with the chaos of Limbo and the endless demonic hordes of the Abyss. A great city of eternal perfection, whose streets and buildings are models of architecture and aesthetics: everything is in order and nothing happens by chance. Although Utopia is not ruled by any race, Axiomites and Inevitable make it their home, constantly seeking to expand their perfect city.

Utopia has the following characteristics:

\textit{Traits}: Stiff, Disciplined, Serious, Direct, Cold

\textit{Empowered Magic}: Legal spells and magical abilities are empowered.

\textit{Hindered Magic}: Spells and chaotic spell-like abilities are hindered.


\end{multicols}

\pagebreak


\section{OBSS Monster}

\textbf{Monsters are Coming ...}

\begin{changemargin}{0.3cm}{0.3cm} \begin{enfasi}{
Those who struggle with monsters must be careful not to become, in so doing, a monster. And if you look into an abyss for a long time, the abyss will also search inside you. (Friedrich Nietzsche)

Monsters can only be defeated by their own kind. (Claymore)

The tragedy of monsters is that they are too big and powerful to be accepted by mankind. (Ishiro Honda)
} \end{enfasi} \end{changemargin} \medskip

\begin{multicols}{2}

\lettrine[lines = 2, lhang = 0.33, loversize = 0.25, findent = 1.5em]{W}{elcome} in a universe full of bad enemies violent sneaky clever petty gigantic .. and whatever else you want. Monsters are the cornerstone of any fantasy RPG.

Monsters are explained and presented here, certainly not all of them, much less exhaustive, use them to populate the adventures of your companions with nightmares.

\medskip

\begin{center}

\includegraphics[width = 0.8 \linewidth]{immagini/sangiorgioedrago.png}

\textit{Saint George and the dragon (c. 1460) by Paolo Uccello. National Gallery of London}
\end{center}

\subsection{Introduction}

An adventure is not just a set of monsters but of situations, places, surprises, in short, everything that can fascinate, involve, amaze, engage the players. But monsters are also needed. Hitting has a cathartic, liberating aspect.

Insert difficult and dangerous monsters in the adventure where needed but from time to time, rarely make the players feel powerful, let them face monsters that can solve in very few rounds. Describe the fight by emphasizing the hits, critics, pain and blood of monsters. Make it clear how powerful the characters can be.

Other times you make monsters scary because they are big, hungry, magical and evil, it is necessary that the players be afraid for their characters, that they never take victory for granted.

The confidence in describing the situation, a few lines, staring at the players. Involve the players and once they have your attention the characters will be more attentive too. Try to put monsters consistent with the environment, the adventure, the situation. Do not roll randomly on tables, a well-organized fight gives much more satisfaction that you randomly show that \textit{spawn}.

Don't reduce everything to an MMORG where the goal is just to kill everything and everyone, there can always be so many choices if you put in a little effort.

\begin{changemargin}{0.3cm}{0.3cm} \begin{tcolorbox}[title = Facing monsters]
{
Let this old man give you a couple of tips young adventurer!

- Not all enemies are defeated with the sword, many times a club is also needed!

- Sometimes weapons and brute force aren't enough. If you don't have companions who can cast spells make sure you always have a chance to start a fire.

- Escape. It is always a viable option if you have a way and see that the situation does not bode well.

- Organized! do not enter the dungeon with your head down and never stop except when you are dead! Rest, explore, check the environment and when you are safe and feel better go on! your enemies also organize themselves and rest in the meantime, be careful!

- Sometimes you can even talk to enemies, they also don't want to die all the time.

- If you have to kill do it with nastiness and speed. Don't waste time and optimize your shots, save your energy and immediately prepare for another fight.

} \end{tcolorbox} \end{changemargin}

\subsection{Editing Creatures}

Despite the versatile collection of monsters in this manual, you may still find yourself embarrassed when it comes to finding the perfect creature for your adventure. Feel free to modify existing creatures and turn them into something more useful to you, perhaps borrowing a trait or two from a different monster.

Keep in mind that modifying a monster could change its degree of challenge.

\subsection{Size}

A monster can be Tiny, Small, Medium, Large, Huge, or Gargantuan and Colossal. The Size Categories table shows how much space a creature of a specific size occupies in combat.

If not indicated the range of a creature depends on the size and the weapon used (think of a gigantic greatsword wielded by a titan ..)


\end{multicols}

\textbf{Table: Size Categories, Occupied Squares and Range} \index{Size, Occupied Squares and Range Categories Table} \index{Range for Creatures} \index{Squares for Creatures} \index{Size and squares} \index{Creatures per square}

\medskip

\begin{tabularx}{0.95\textwidth}{llXl}
\toprule
\textbf{Size} & \textbf{Space} & \textbf{Example} & \textbf{Range} \\
Tiny & 25 x 25 cm & Cat, sprite & 0m \\
Small & 0.5 x 0.5m & Goblin, Dog, Gnome & 0m \\
Medium & 1 x 1m & Orc, Human, Elf, Dwarf, Nibals & 1m \\
Large & 3 x 3m & Ogre & 2m \\
Huge & 5 x 5m & Giant, Ent & 3m \\
Gargantuan & 6 x 6m & Kraken, Dragon & 4m \\
Colossal & 12 x 12m & Elder Dragon, Tarrasque & 6m \\
\end{tabularx}

\medskip

\begin{multicols}{2}

\subsection{Type}

The type of a monster refers to its basic nature. Certain spells, magical items, abilities, and other game effects interact in special ways with creatures of a specific type. For example, a \textit{dragon slayer arrow} deals extra damage not only to dragons but also to all other dragon-type creatures, such as turtle dragons and wyverns.

The game includes the following types of monsters:

\medskip \textbf{Aberrations}, totally alien creatures. Many of them possess innate magical abilities that draw upon the creature's alien mind rather than the mystical forces of the world. Classic examples of aberrations are abolets, observers, mind flayers, and the batrachians of chaos.

\medskip \textbf{Beasts}, non-humanoid creatures that are a natural component of a fantasy world. Some possess magical powers, but most are devoid of Intelligence and have no form of society or language. Classic examples of beasts are all common animal species, dinosaurs and giant versions of animals.

\medskip \textbf{Celestials}, creatures native to the Higher Planes. Many of them are servants of the gods, employed as messengers or agents in the mortal world and for the planes. \\
Celestials are of a good nature, classic examples of celestials are angels, couatl and pegasi.

\medskip \textbf{Constructs}, are created and not born. Some are programmed by their creators to follow a simple set of instructions, while others are sentient and able to think for themselves. Golems are the most representative constructs.

\medskip \textbf{Dragons}, are large reptilian creatures of ancient origin and enormous power. True dragons, including good metallic dragons and evil chromatic dragons, are highly intelligent and possess innate magical abilities. This category also includes creatures distantly related to true dragons, but less powerful, less intelligent and less magical, such as wyverns and pseudodrags.

\medskip \textbf{Elementals}, are creatures native to the elemental planes. Some creatures of this type are little more than animated masses of the respective element, and include creatures simply called elementals. Other creatures possess biological forms infused with elemental energy. The races of geniuses, including djinn and efreet, form the most important civilizations of the elemental planes. Other elemental creatures are the azers, the invisible persecutors and the oddities of water.

\medskip \textbf{Fairies}, are magical creatures closely related to the forces of nature. They live in hidden clearings and foggy forests. Examples of fairies are dryads, pixies, fairies and satyrs and La Topi.

\medskip \textbf{Giants}, they tower over humans and their fellows. They are human in shape, although some have multiple heads (ettins) or deformities (fomori). The six variants of true giants are hill giant, stone giant, frost giant, fire giant, cloud giant, storm giant. Besides these, ogres and trolls are also giants.

\medskip \textbf{Fiend}, perverse creatures native to the Lower Planes. Some are in the service of gods, but many more work under the orders of archdevils and demon princes. Evil priests and spellcasters sometimes summon the Fiend into the material world to carry out their will. If an evil celestial is a rarity, a good fiend is practically inconceivable. Fiend include demons, devils, hellhounds, and rakshasas.

\medskip \textbf{Slime}, are gelatinous creatures that hardly have a fixed shape. They live mainly underground, settling in caves and undergrounds, feeding on waste, carcasses or creatures unfortunate enough to run into them. Black protoplasm and gelatinous cubes are among the most recognizable oozes.

\medskip \textbf{Monstrosity}, are monsters in the strictest sense of the term frightening creatures that are neither common nor truly natural, and almost never benign. Some are the result of magical experiments gone wrong (like the Owlbear), while others are the product of terrible curses (including the minotaur). They escape any categorization, and somehow serve as an all-encompassing category for those creatures that do not correspond to any other type of monster.

\begin{center}
\includegraphics[width = 0.7 \linewidth]{immagini/sanmichelesatana.png} \\
\textit{Saint Michael defeats Satan. Raphael and helpers (1518). Louvre Museum}
\end{center}

\medskip \textbf{Undead}, are once living creatures brought to a horrible state of undead through the practice of necromantic magic or some blasphemous curse. The undead include walking corpses, such as vampires and zombies, or incorporeal spirits, such as ghosts and specters.

\medskip \textbf{Plants}, in this context we are dealing with plant creatures, not normal flora. Most of them are mobile, and some are carnivorous. The most classic example of plants are walking mounds and ents. Fungoid creatures such as gas spores and myconids also fall into this category.

\medskip \textbf{Humanoids}, are the primary population of the game worlds, civilized and savage, comprising humans and a wide range of other species. They possess a language and culture, little or no innate magical ability (although many humanoids can learn spells), and a bipedal form. The most common races of humanoids are the ones best suited as player characters: humans, dwarves, elves and nibals, various. Almost as numerous, but more brutal and savage, and nearly all evil, are the goblinoid races (goblins, hobgoblins, and bugbears), orcs, gnolls, lizards and kobolds. \\

\medskip

These categories can in turn be grouped into types of Creatures:
\smallskip
\begin{itemize}
\item
The \textbf{Natural Creatures}: they are Insects, Reptiles, Beasts, Humanoids, Plants, Aquatic creatures, Monstrusities, Oozes
\item
The \textbf{Magical Creatures} are: Fiend, Fairies, Spirits, Undead, Giants, Celestials, Constructs, Aberrations (anything alien or unnatural) and Dragons.

If a Natural Creature has magical powers then it also considers itself as a Magical Creature.
\end{itemize}


\medskip \textbf{Labels}

A monster can have one or more labels indicated in parentheses, following its type. For example, an ogre has the type \textit{humanoid (ogre)}. The labels in parentheses provide additional categorizations for certain creatures. Labels don't have their own specific rules, but some game elements, such as magic items, can refer to them. For example, a spear that is particularly effective against demons would work against any monster that carries the demon label.

\subsection{Traits}

The monsters do not present the detailed list of Traits, you will only find the indication on the axes of Chaos, Law, Good and Evil. Remember that these are indications, exceptions can happen especially in the more intelligent species.
Certain creatures are \textbf{misaligned}, meaning they have no moral or ethical conduct.

\subsection{Defense}

A monster wearing armor or carrying a shield has a Defense that takes into account armor, shield, and Dexterity. Otherwise, a monster's Defense is based on its Dexterity value and natural armor if it has it. If a monster has natural armor, wears armor, or carries a shield, it is indicated in parentheses after its Defense value.

If the monster is \textbf{taken by surprise} subtract from the Defense the Dexterity and Shield value if present.

If the monster is hit with a \textbf{touch effect} (Defense touch) subtract the value of the armor and shield from the Defense.

\subsection{Hit Points}

Usually when it drops to 0 hit points, a monster dies or is destroyed.

A monster's hit points are presented both as a set of dice and as an average value. For example, a monster with 2d8 hit points has an average of 9 hit points (2 x 4.5).

It will happen that the players ask you \textbf{\textit{how is the monster}}, I suggest you never go into detail by saying how many HP he has in all or he has lost, but stay in these ranks: Not injured (HP full), Wounded (30 \% HP suffered), Seriously Injured (at least 50 \% HP suffered), or give a generic description of the state. \index{How is the monster} \index{Asking Monster HP}

A monster's size determines the die used to calculate its hit points, as shown on the Hit Dice by Size chart.

\subsection{Hit Dice per Monster Size} \index{Hit Dice per Monster Size}

\medskip
\begin{tabular}{lll}
\toprule
Cut & Nut Life & PF for Nut \\
Lowercase & d4 & 2.5 \\
Small & d6 & 3.5 \\
Average & d8 & 4.5 \\
Large & d10 & 5.5 \\
Huge & d12 & 6.5 \\
Gargantuan & d20 & 10.5 \\
Colossal & 2d12 & 12 \\
\end{tabular}
\medskip

A monster's Constitution value also affects the number of hit points it has. Its Constitution value is multiplied by the number of Hit Dice it has and the result is added to its Hit Points. For example, a monster that has a Constitution of 1 and 2d8 Hit Dice, and therefore will have 2d8 + 2 Hit Points (average 10).

\subsection{Movement}

A monster's Movement tells you how far it can move during its round per Move Action

All creatures have a walk movement, simply called monster movement. Creatures that have no form of land movement have walking speed 0 meters.

Some creatures have one or more of the following additional movement modes.

\medskip \textbf{Swimming}

A monster that has swimming speed does not have to spend extra movement to swim (it is not difficult terrain)

\medskip \textbf{Climbing}

A monster that possesses Climb Speed can use all or part of its movement to move across vertical surfaces. The monster does not have to spend extra movement (x4) to climb.

\medskip \textbf{Burrow}

A monster that has Dig Speed can use its Dig Speed to traverse sand, dirt, mud, etc. A monster cannot dig through solid rock unless it has a special trait that allows it.

\medskip \textbf{Flight}

A monster that has flight speed can use all or part of its movement to fly. Some monsters have the ability of \textbf{float}, which makes them difficult to take down. The monster stops floating when it dies.

\begin{center}
	\includegraphics[width = 0.7 \linewidth]{immagini/roc.png} \\
	\textit{Henry Justice Ford}
\end{center}

\subsection{Characteristic Scores}

Each monster has six ability scores (Strength, Dexterity, Constitution, Intelligence, Wisdom, Charisma)

\subsection{Skills}

The item Skills is reserved for those monsters who are capable of one or more skills. For example, a monster that is very alert and stealthy might have bonuses on Wisdom (Awareness) and Dexterity checks.

Other modifiers can also be applied, for example, a monster may have a larger bonus than expected to account for its great skill.

\subsection{Vulnerability, Resistance and Immunity} \index{Weapon Equivalence} \index{Magic Fists}
Some creatures possess vulnerabilities, resistances, or immunity to a certain type of damage. Particular creatures are even resistant or immune to non-magical attacks (a magical attack is an attack made via a spell, magical item, or other source of magic).

It is also possible that a specific minimum magic bonus is indicated in order to damage the creature.

Additionally, certain creatures are immune to certain conditions. If a monster is immune to a game effect that isn't considered damage or condition, it has a special trait instead.

The table below indicates which magical enchantment of the weapon is required to overcome the indicated immunity. The minimum level of natural attack (Weapon Proficiency) is also indicated in case of hitting with kicks and punches.

In the case of a character with a Weapon List \textbf{Empty Fist} you check how many times the list has been taken.

\medskip

\textbf{Table: Magic Weapons Equivalence} \index{Magic Weapons Equivalence Table} \hypertarget{equivalenzemagiche}{} \label{equivalenzaarmimagiche}

\medskip

\begin{tabular}{lp{0.055\textwidth} p{0.06 \textwidth} p{0.07 \textwidth}}
\toprule
\textbf{Immunity} & \textbf{Magic Weapon} & \textbf{Nat Attack} & \textbf{Empty Fist} \\
+1	   	& +1 & 3 & 2 \\
+2 		& +2 & 6 & 4 \\
Cold Iron 	& +1 & 4 & 2 \\
Silver 		& +1 & 4 & 2 \\
Adamantium & +2 & 6 & 4 \\
+3 		& +3 & 12 & 6 \\
+4	 	& +4 & 16 & 8 \\
+5	 	& +5 & - & 8 \\
\end{tabular}

\subsection{Sensi}

The Senses entry lists any special senses the monster possesses. The special senses are described below. If Senses is not present, the creature has standard senses (vision, smell, taste, touch ...)

\subsubsection{Telluric perception}

A monster with telluric perception can locate and find the origins of vibrations within a specific radius, as long as the monster and the source of the vibration are in contact with the same ground or substance. Telluric perception cannot be used to detect flying or incorporeal creatures. Many burrowing creatures, such as ankheg and earthen giants, possess this special sense.

\subsubsection{Low-light vision or darkvision}

A creature with low-light vision can see in the faintest of lights, but not in complete darkness unlike those with darkvision. Many creatures that live underground have this special sense. See chapter \hyperlink{visioneeluce}{Caratteristiche Speciali}.

\subsubsection{True seeing}

A true seeing monster can, up to a specific range, see through normal and magical darkness, see invisible creatures and objects, automatically detect illusions and make saving throws against them, and perceive the original shape of a shapeshifter. or a creature transformed by magic. Additionally, the creature can see into the Ethereal Plane up to the same range.

\subsubsection{Blind Sight}

A blind-sighted creature can perceive its surroundings, without relying on sight, up to a specific range.

Creatures without eyes such as grimlocks and oozes and creatures with enhanced echolocation or senses, such as bats and dragons, possess this sense.

If a monster is naturally blind, this is noted in parentheses, indicating that the range of its blind sight also defines the maximum range of its perception.

\begin{center}
\includegraphics[width = 0.7 \linewidth]{immagini/ciclope.png}

\textit{Henry Justice Ford}
\end{center}

\subsection{Languages}

The languages a monster can speak are listed in alphabetical order. Sometimes a monster can understand a language but not speak it, and this is indicated in this entry. If a monster does not have the well-known Languages it means that it does not know any languages other than its own language (if applicable).

\subsection{Telepathy}

Telepathy is an ability that allows a monster to mentally communicate with another creature within a specified range. The contacted creature does not need to speak the same language as the monster to communicate in this way. A creature without telepathy can receive and respond to telepathic messages but cannot initiate or end a telepathic conversation.

A telepathic monster does not need to see the contacted creature and can end telepathic contact at any time. Contact is broken as soon as the two creatures are no longer within range or if the telepathic monster contacts another creature in range. A telepathic monster can initiate or end a telepathic conversation without having to use an action, but while the monster is incapacitated, it cannot initiate telepathic contact, and any ongoing contact is terminated. To initiate telepathic communication, the target must have at least been identified.

A creature in the area of a \textit{anti-magic field} or any other place where the spell fails can send or receive telepathic messages.

\subsection{Challenge}

A monster's \textbf{degree of challenge} (CR) tells you how great the threat it poses. A properly equipped and rested company of four adventurers must be able to defeat a monster with a challenge rating equal to its average level without taking losses. For example, a company of four 3rd level characters should consider a challenge 3 monster a worthy challenge, but not lethal.

Monsters that are significantly weaker than 1st-level characters have a challenge rating of less than 1. Monsters with a challenge rating of 0 do not present problems except in large numbers; those with no real attacks are not worth experience points.

Some monsters present a greater challenge than even a company of 20th level can handle. These monsters have a challenge rating of 21 or higher and are designed specifically to check characters' abilities.

\subsection{Special Traits}

Special traits (which appear after a monster's challenge rating but before any action or reaction) are features that are likely to play a role in a combat encounter and require explanation.

\begin{center}
	\includegraphics[width = 0.7 \linewidth]{immagini/lich2.png}

\textit{Lich - Battle of Wesnoth}

\end{center}

\subsection{Spells}

A monster with the Spells privilege is capable of casting Spells.

A monster can cast a spell from its spell list without making the Magic Check and without the ability to make critical rolls or not. DC is 10 + spell x2 + Intelligence or Wisdom whichever is best or indicated. A monster with spells cannot convert the Magic Points of higher-level spells into lower-level spells, except if it has a Magical Proficiency value (eg Lich, Mummy, Naga ...).

\subsection{Innate Spells}

A monster with the innate ability to cast spells has the Spells special trait.
A monster's innate spells cannot be exchanged for other spells.

A monster never voluntarily makes a Magic Check unless it has an Intelligence greater than 1.

\subsection{Actions}

Monsters also act according to the 3 Action pattern available per round. Skills and abilities can be marked that allow him to perform a higher number of Actions.

When a monster performs its actions, it can choose from the options in the Actions section of its stat block or use one of the actions available to all creatures, such as Dash or Hide.

\subsubsection{Melee and Ranged Attacks}

The most common action a monster will perform in combat will be a melee or ranged attack. They can be spell attacks or weapon attacks, where the weapon can be an artifact or a natural weapon, such as claws or a spiked tail.

\textit{\textbf{Creature vs Target}.} The target of a melee or ranged attack is usually a creature or target.

\textbf{Range}: The range shown is the distance \textbf{within} how many meters the creature can hit the opponent. Even if the range is greater than that of the opponent, advantages such as Long Weapon (+2 to TC) are not considered with natural attacks. A creature with 0 range must be on you to hit you, extremely small creatures usually have 0 range.

\textit{\textbf{Strikes.}} Any damage dealt or other effect that occurs as a result of an attack that strikes the target is described in the notation `` \textit{Strikes} ''. You can choose to take the medium damage or roll the dice; for this reason both the average damage and a dice formula are presented.

I suggest that the Critical Roll is still applicable for enemies while the Blast of Damage is to be used if you want a more lethal campaign.

\textbf{\textit{Missing}.} If an attack has an effect produced by a miss, that information is provided by the notation `` \textit{Missing} ''.

\textit{\textbf{Damage.}} If a monster wields manufactured weapons, it deals appropriate damage to the weapon. Larger monsters usually wield larger weapons that deal extra damage when they strike. If they use this type of weapon, the damage is already marked, otherwise if they pick up or use an unexpected weapon, double the dice of the weapon if the creature is Large, triple them if Huge and quadruple them if Gargantuan if they use weapons of their size.

A creature has -1d6 on attack rolls for a weapon built for a size larger than its size. The Storyteller may decide that weapons two or more sizes larger than the attacker's are completely impossible to use.

A creature with at least Challenge Rank 6 at the Storyteller's discretion can make an attack of opportunity (see \hyperlink{opportunista}{Opportunista})

\subsubsection{Multiattack}

A creature that can make multiple attacks during its round has the Multiattack ability. The Multi-Attack Action consumes 2 Actions even if it carries more than 2 attacks.

\subsubsection{Grabbing Rules for Monsters}

Many monsters have a special attack that allows them to quickly grab prey. When a monster hits with such an attack, it does not need to make another ability check to determine if the grab is successful, unless the attack says otherwise.

A creature grabbed by the monster can use an action to try to escape it. To do so, he must succeed in an opposed Strength check (Fortitude save with Strength bonus) against the escape DC in the monster's stat block. If an escape DC is not provided, assume the DC equals the monster's Fortitude + Strength saving throw bonus.

\begin{center}
	\includegraphics[width = 0.55 \linewidth]{immagini/polpo.png}

	\textit{Alphonse de Neuville - Hetzel edition of 20000 Lieues Sous les Mers}
\end{center}

\subsubsection{Ammunition}

A monster carries enough ammo with it to make its ranged attacks. You can assume that a monster has 2d4 shells for a thrown weapon attack (javelins, boulders ...), and 2d10 shells for a projectile weapon such as a bow or crossbow.

\subsubsection{Reactions}

Whether a monster can do something special with its reactions is listed here. If a creature has no special reactions, this section is absent.

\subsubsection{Limited Use}

Some special abilities have restrictions on the number of times they can be used.

\textbf{\textit{X / Day}.} The notation `` X / Day '' indicates a special ability that can be used X times before dawn breaks to recover the uses consumed. For example, `` 1 / Day '' indicates a special ability that can be used once before the monster has to wait for the new dawn.

\textit{\textbf{Cooldown XY.}} The notation `` Cooldown XY '' indicates that the monster can use a special ability once and that the ability has a random chance to recharge each subsequent round of combat . At the beginning of each monster round, roll a d6. If the result is one of the numbers in the reload record, the monster recovers the use of the special ability. The ability also recharges at the dawn of a new day.

%\begin{center}
%	\includegraphics[width=0.6\linewidth]{immagini/cupido.png}
%
%	\textit{Eros con il suo arco. Musei Capitolini}
%\end{center}

For example, "5-6 Cooldown" means that a monster can use its special ability once. Then, at the beginning of the monster's round, regain the use of the ability if he rolls a 5 or 6 on a d6.

\subsection{Equipment}

The stat block refers to the equipment, besides the weapons or armor used by the monster. A creature that normally wears clothing, such as a humanoid, is assumed to be appropriately dressed.

You can equip monsters with additional equipment as you like, using chapter \hyperlink{equipaggiamento}{Equipaggiamento} as a source of inspiration, and you decide how much of the monster's equipment is recoverable after the creature is killed or whether any part of its equipment is still usable. For example, a dented armor made for a monster is unlikely to be usable by someone else. If a spellcasting monster needs material components to cast its spells, assume it has the material components to cast spells in its stat block.

\subsection{Additional Actions}

Certain creatures can perform special actions outside of their own round, and some can extend their power to the environment, causing extraordinary magical effects to occur in their vicinity.

A creature with additional actions can perform a number of special actions - called additional actions - outside of its round. Only one additional action can be used at a time and only at the end of another creature's round. A creature with additional actions recovers the additional actions it used at the beginning of its round. She is not required to use her additional actions and cannot use the additional actions while she is incapacitated or otherwise unable to perform actions. If surprised, she cannot use them until after her first combat round.

If a creature takes the form of a creature with additional actions, perhaps through a spell, it doesn't get the additional actions, the lair actions.

\subsubsection{A Creature's Lair}

A creature with additional actions may have a section describing its lair and the special effects it can create there while it is there, either by its own will or simply by being there. This section only applies to legendary creatures who spend a lot of time in their lairs and are highly likely to be encountered there.

\subsubsection{Stocks from Tana}

If a creature with additional actions has a lair action, it can use it to harness its lair's environmental magic. On initiative count 10, losing ties, the creature can use one of its lair action options. He cannot do this while he is incapacitated or otherwise unable to perform actions. If surprised, she cannot use them until after her first combat round.

\subsubsection{Types of Treasures}

Each type of creature can prefer a different type of treasure (intended as objects, coins, gems ...). These are just tips on how to build the monster treasure.

\medskip

\begin{itemize}

\item \textbf{Aberration}
Many aberrations have little regard for treasures, possessing only what they take from the remains of their previous victims. Others are cunning opponents who use various magical items and treasures to enhance their abilities.

\item \textbf{Animal}
Animals do not care at all for treasures, instead leaving coins and items with the remnants of their meals. For those with treasure, the treasure is typically found in their lairs, scattered among the bones and other scraps.

\item \textbf{Magical Beast}
Caring little for values, most magical beasts are solely looking for their next meal. The hiding places of these creatures are often littered with precious trinkets and magical items.

\item \textbf{Construct}
The treasure alone carried by the constructs is generally part of the construct itself, such as a weapon or magical item. Constructs, however, are typically used to guard treasures or magical items of greater value.

\item \textbf{Dragon}
Known for their prized treasures, dragons often mull over piles of coins, gems, magical items, and other expensive items.

\item \textbf{External}
Outsiders are among the most varied types of creatures and as a result they may really have any kind of treasure on them or hidden in their lair. The Storyteller should consider the individual creature when determining the type of treasure that best suits that exterior.

\item \textbf{Elf}
Above all else, goblins value beautiful and magical items. They have little regard for the instruments of exchange and trade used by more civilized races, such as coins and values.

\item \textbf{Slime}
Oozes do not conceive of things as treasures and leave whatever they find in their search for the next meal. Any treasure they may carry is completely accidental.

\item \textbf{Undead}
The treasures carried by the undead vary depending on whether or not they are intelligent creatures. Undead devoid of intellect typically have only the meager values they brought with them to life, rarely truly usable as treasure, while intelligent ones harness a wide range of magical items to destroy the living.

\item \textbf{Parasite}
Like other mindless creatures, parasites do not crave treasure, although these creatures sometimes find themselves infesting areas where values are kept.

\item \textbf{Humanoid}
Creatures of this type are very diverse, but even the most primitive humanoids use magical equipment and items to some extent. In larger groups, such as communities, humanoids often possess a large amount of treasure that they collectively guard.

\item \textbf{Vegetable}
Like animals, plant creatures don't care about treasures, and anything that might be found where they grow is simply the undigested remains of a previous victim.

\end{itemize}

\end{multicols}

\pagebreak

\subsection{The Monsters}


\begin{changemargin}{0.3cm}{0.3cm} \begin{Storytellere}
The creatures presented here are meant to be a full-bodied example of the enemies your friends may encounter. Attention, it is not said that they are all enemies or necessarily that they have negative intentions.

More civilized creatures will have their own individual ethical and moral conduct, even within the same group of "enemies" there are those who could be more "enemies" or simply indifferent.

Take advantage of the peculiarities and uniqueness of creatures to create meetings that are not predictable and challenging from a tactical point of view. Not to be taken for granted but not even absurd in the choices, there must always be consistency in choosing creatures.
\end{Storytellere} \end{changemargin}


\bigskip

\begin{changemargin}{0cm}{0.5cm} \begin{enfasi}{

Amon Goth: Control is power. This is the power.

Oskar Schindler: Is that why they fear us?

Amon Goth: We have the power to kill. This is why they fear us.

Oskar Schindler: They fear us because we have the power to kill arbitrarily. A man commits a crime, he had to think about it, we do it kill and feel at peace. Or we kill it ourselves and feel it even better. This is not the power though! This is justice, it is one something other than power. Power is when we have every justification to kill and we don't.

Amon Goth: Is This Power?

Oskar Schindler: The emperors had that. A man steals something, he is brought before the emperor and falls to the ground trembling, she begs for mercy. He is aware that he is about to leave. And the emperor forgives him instead. That man, undeserving, leaves him free.

(Schindler's list, Film, 1993)
} \end{enfasi} \end{changemargin} \medskip


\begin{changemargin}{0cm}{0.5cm} \begin{enfasi}{
I am the monster that men who breathe would want to kill. I'm Dracula. (Bram Stoker's Dracula)
} \end{enfasi} \end{changemargin} \medskip


\bigskip

\begin{multicols}{2}

\medskip\index[Mostruario]{Aboleth} \textbf{Aboleth}

\textit{Large aberration, lawful evil}

\textbf{STRENGTH} +5

\textbf{DEXTERITY} -1

\textbf{CONSTITUTION} +2

\textbf{INTELLIGENCE} +4

\textbf{WISDOM} +2

\textbf{CHARISMA} +4

\textbf{Initiative} +4 - \textbf{Defense} 22

\textbf{Hit Points} 135 (18d10 + 36)

\textbf{Movement} 3m, swim 12m

\textbf{Saving Throws} Fortitude +8, Reflex +5, Will +11

\textbf{Skills} Awareness +10, History +12

\textbf{Senses} darkvision 36 m

\textbf{Languages} Language of the Depths, telepathy 36 m

\textbf{Challenge} 10 (5.900 PX)

\textit{\textbf{Amphibian.}} The aboleth can breathe air and water.

\textit{\textbf{Mucus Cloud.}} While underwater, the aboleth is enveloped in mutant mucus. A creature that comes into contact with the aboleth, or hits it with a melee attack while within 1 meter of it, must make a DC 14 Fortitude save. If it fails, the creature is sick for 1d4 hours. . The sick creature can only breathe underwater.

\textit{\textbf{Telepathic Probe.}} If a creature communicates telepathically with the aboleth, and the aboleth can see it, the aboleth learns its greatest desires.

\textbf{Actions}

\textit{\textbf{Multiattack.}} The aboleth makes three attacks with its tentacles

\textit{\textbf{Tentacle.} Melee Weapon Attack}: +9 to hit, 3m range, one target.

\textit{Strikes:} 12 (2d6 + 5) hit damage. If the target is a creature, it must succeed on a DC 14 Fortitude saving throw or become ill. The disease has no effect for 1 minute and can be removed by any magic that cures disease. After 1 minute, the diseased creature's skin becomes transparent and slimy, the creature cannot recover hit points unless it is underwater, and the disease can only be removed by \textit{heal} or another cure disease spell level 3 or higher. When the creature is outside a body of water, it takes 6 (1d12) acid damage every 10 minutes unless its skin is wet before these 10 minutes have passed.

\textit{\textbf{Tail.} Melee Weapon Attack}: +9 to hit, 3m range, one target.

\textit{Hits:} 15 (3d6 + 5) hit damage.

\textit{\textbf{Enslave (3 / Day).}} The aboleth targets a creature it can see within 30 feet of it. The target must succeed on a DC 14 Will saving throw or be magically fascinated by the aboleth until the aboleth dies or the two are on different planes of existence. The fascinated target is under the control of aboleth and cannot react. The aboleth and the target can telepathically communicate with each other at any distance.

Whenever the fascinated target takes damage, it can re-roll the saving throw. If successful, the effect ends. No more than once every 24 hours, he can re-roll the saving throw when he is at least 1.5 kilometers away from the aboleth.

\textbf{Additional Actions}

The aboleth can perform 3 additional Actions, chosen from the following options. He can only use one legendary option at a time, and only at the end of another creature's turn. The aboleth recovers any additional Actions spent at the start of its round.

\textbf{Spot.} The aboleth makes a Wisdom (Consciousness) check.

\textbf{Psychic Drain (Costs 2 Actions).} A creature fascinated by the aboleth takes 10 (3d6) damage and the aboleth recovers a number of hit points equal to the damage taken by the creature.

\textbf{Tail Sweep.} The aboleth makes a tail attack.

\textbf{Ecology} \\
Environment: Any Aquatic \\
Organization: Solitary, pair, brood (3-6) or pack (7-19) \\
\textbf{Treasure}: Double \\
\textbf{Description} \\
As their primitive appearance suggests, aboleth hermaphrodites are among the oldest life forms in the world. Already ancient when the gods began to take an interest in the Material Plane, the aboleths have always lived far from other mortals: they are alien, cold and always busy weaving planes. They once ruled the world in a vast empire, and today they see other life forms as food or slaves ... sometimes both together. They despise the gods, as they believe they are the true masters of creation, an aboleth is 7 meters long and weighs about 3.2 tons. In the darkest depths of the sea, the aboleths still inhabit their grotesque cities, gigantic and nauseating. They are served by countless slaves taken from every nation, both land and sea, and the land ones are doubly enslaved by their masters and their mucus, which allows them to breathe underwater, the aboleths encountered alone are usually explorers from these hidden cities, looking for new slaves.

\subsection{Angels}

\medskip\index[Mostruario]{Angelo Deva} \textbf{Angelo Deva}

\textit{Medium celestial, legal good}

\textbf{STRENGTH} +4

\textbf{DEXTERITY} +4

\textbf{CONSTITUTION} +4

\textbf{INTELLIGENCE} +3

\textbf{WISDOM} +5

\textbf{CHARISMA} +5

\textbf{Initiative} +4 - \textbf{Defense} 22

\textbf{Hit Points} 136 (16d8 + 64)

\textbf{Movement} 9 m, flight 27 m

\textbf{Saving Throws} Fortitude +16, Reflex +13, Will +11

\textbf{Skills} Perceiving Emotions +9, Awareness +9

\textbf{Damage Resistance} from Light; non-magical weapon
\textbf{}
\textbf{Condition Immunity} fascinated, fatigued, scared


\textbf{Senses} darkvision 36 m

\textbf{Languages} all, telepathy 36 m

\textbf{Challenge} 10 (5.900 PX)

\textit{\textbf{Angelic Weapons.}} The deva's weapon attacks are magical. When the deva strikes with any weapon, the weapon deals an additional 4d8 points of Light damage (already included in the attack).

\textit{\textbf{Innate Spells.}} The deva's innate spellcasting characteristic is Charisma (DC 17 on saving throws for spells). The deva can innately cast the following spells, using only the verbal components:

At will: \textit{identification of good and evil}

1 / day: \textit{communion, raise dead}

\textit{\textbf{Resistance to Magic.}} The deva has + 1d6 on saving throws against spells and other magical effects.

\textbf{Actions}

\textit{\textbf{Multiattack.}} The deva makes two melee attacks.

\textit{\textbf{Mace.} Melee Weapon Attack}: +19 to hit, 1m range, one target.

\textit{Hits:} 7 (1d6 + 4) slash damage plus 18 (4d8) Light damage.

\textit{\textbf{Healing Touch (3 / Day).}} The deva makes contact with another creature. The target magically recovers 20 (4d8 + 2) hit points and is free from any blindness, disease, curse, deafness, or poison.

\textit{\textbf{Change Form.}} The deva can magically transform into a humanoid or beast whose degree of challenge is equal to or less than his own, or return to his true form. At death it returns to its true form. Any equipment he is wearing or carrying is absorbed or carried in the new form (at the choice of the deva).

In the new form, the deva retains her game stats and ability to speak, but her Defense, movement methods, Strength, Dexterity, and special senses are replaced by those of the new form, and she gains any stats or abilities (Additional Actions and den shares) owned by its new form and not its original.

\textbf{Ecology}
Environment: Any plan with good traits \\
Organization: Solitary, pair, or squadron (3-6) \\
\textbf{Treasure}: Double (+1 Flaming Greatsword, other treasure) \\
\textbf{Description} \\
Movanic devas make up the infantry ranks of the celestial armies, although they spend most of their time patrolling the Positive, Negative, and Material Plane. On the Positive Plane they guard the wandering good souls, and this time brings them into conflict with the Jyoti. On the Negative Plane they fight the undead, the Sceaduinar and other strange beings that hunt in the ravenous void. Their rare times on the Material Plane are usually meant to bring aid to mighty mortals when great danger threatens to bring an entire kingdom into the hands of evil.

\medskip\index[Mostruario]{Angelo Planetar} \textbf{Angelo Planetar}

\textit{Large celestial, legal good}

\textbf{STRENGTH} +7

\textbf{DEXTERITY} +5

\textbf{CONSTITUTION} +7

\textbf{INTELLIGENCE} +4

\textbf{WISDOM} +6

\textbf{CHARISMA} +7

\textbf{Initiative} +5 - \textbf{Defense} 27

\textbf{Hit Points} 200 (16d10 + 112)

\textbf{Movement} 12 m, flight 36 m

\textbf{Saving Throws} Fortitude +19, Reflex +11, Will +19

\textbf{Skills} Awareness +11

\textbf{Damage Resistance} from Light;

\textbf{Condition Immunity} fascinated, fatigue, scared, weapons +1

\textbf{Senses} vision of the true 36 m

\textbf{Languages} all, telepathy 36 m

\textbf{Challenge} 16 (15000 PX)

\textit{\textbf{Angelic Weapons.}} The planetar's weapon attacks are magical. When striking with any weapon, the weapon deals an additional 5d8 points of Light damage (already included in the attack).

\textit{\textbf{Divine Awareness.}} The planetar recognizes lies immediately.

\textit{\textbf{Innate Spells.}} The planetarium's innate spellcasting characteristic is Charisma (DC 20 on saving throws for spells). The planetarium can innately cast the following spells, without the need for material components:

At will: \textit{identification of good and evil}, \textit{invisibility} (personal only)

3 / day: \textit{barrier of blades, fire strike, dispel good and good} \textit{evil, raise dead}

1 / day: \textit{communion, weather control, insect plague}

\textit{\textbf{Resistance to Magic.}} The planetar has + 1d6 on saving throws against spells and other magical effects.

\textbf{Actions}

\textit{\textbf{Multiattack.}} The planetar makes two melee attacks.

\textit{\textbf{Greatsword.} Melee Weapon Attack}: +26 to hit, 1m range, one target.

\textit{Strikes:} 21 (4d6 + 7) slashing damage plus 22 (5d8) Light damage.

\textit{\textbf{Healing Touch (4 / Day).}} The planetar makes contact with another creature. The target magically recovers 30 (6d8 + 3) hit points and is free from any blindness, disease, curse, deafness, or poison.

\textbf{Ecology}
Environment: Any plane with good traits \\
Organization: Solitary or couple \\
\textbf{Treasure}: Double (Holy Greatsword +3) \\
\textbf{Description} \\
Planetars are the generals of the celestial armies aimed at the destruction of evil. A planetar is typically 2.7 meters tall and weighs around 250 kg. They are excellent diplomats, but against the Fiend they prefer a war rather than negotiate a peace.


\medskip\index[Mostruario]{Angelo Solar} \textbf{Angelo Solar}

\textit{Large celestial, legal good}

\textbf{STRENGTH} +8

\textbf{DEXTERITY} +6

\textbf{CONSTITUTION} +8

\textbf{INTELLIGENCE} +7

\textbf{WISDOM} +7

\textbf{CHARISMA} +10

\textbf{Initiative} +7 - \textbf{Defense} 31

\textbf{Hit Points} 243 (18d10 + 144)

\textbf{Movement} 15m, flight 45m

\textbf{Saving Throws} Fortitude +25, Reflex +14, Will +23

\textbf{Skills} Awareness +14

\textbf{Damage Resistance} from Light;

\textbf{Immunity to Damage} Void, poison, weapons +2

\textbf{Condition Immunity} fascinated, poisoned, fatigue, frightened, weapon +2

\textbf{Senses} vision of the true 36 m

\textbf{Languages} all, telepathy 36 m

\textbf{Challenge} 21 (33000 PX)

\textit{\textbf{Angelic Weapons.}} The solar's weapon attacks are magical. When striking with any weapon, the weapon deals an additional 6d8 points of Light damage (already included in the attack).

\textit{\textbf{Divine Awareness.}} The solar immediately recognizes lies.

\textit{\textbf{Innate Spells.}} The solar's innate spellcasting characteristic is Charisma (DC 25 on saving throws for spells). The solar can innately cast the following spells, without the need for material components:

At will: \textit{identification of good and evil}, \textit{invisibility} (personal only)

3 / day: \textit{blade barrier, fire strike, dispel good and evil, resurrection}

1 / day: \textit{communion, check weather}

\textit{\textbf{Magic Resistance.}} The solar has + 1d6 on saving throws against spells and other magical effects.

\textbf{Actions}

\textit{\textbf{Multi Attack.}} The solar makes two attacks with its broadsword.

\textit{\textbf{Greatsword.} Melee Weapon Attack}: +30 to hit, 1m range, one target.

\textit{Strikes:} 22 (4d6 + 8) slashing damage plus 27 (6d8) Light damage.

\textit{\textbf{Longbow of Kill.} Ranged weapon attack}: +30 to hit, range 45m, one target.

\textit{Strikes:} 15 (2d8 + 6) piercing damage plus 27 (6d8) Light damage. If the target is a creature with 100 hit points or less, it must succeed on a DC 15 Fortitude saving throw or die.

\textit{\textbf{Flying Sword.}} The solar releases its broadsword to magically float in an unoccupied space within 1 meter of it. If the solar can see the sword, with a bonus action he can mentally order it to fly up to 15 meters and make an attack on a target or return to the solar hand. If the floating sword is the target of an effect, it is treated as if it were wielded by the solar. If the solar dies, the floating sword falls to the ground.

\textit{\textbf{Healing Touch (4 / Day).}} The solar makes contact with another creature. The target magically recovers 40 (8d8 + 4) hit points and is free from any blindness, disease, curse, deafness, or poison.

The solar can perform 3 additional actions, chosen from the following options. He can only use one Additional Action at a time, and only at the end of another creature's round. The solar recovers the additional actions spent at the beginning of its round.

\textbf{Glowing Explosion (Costs 2 Actions).} The solar emits divine magical energy. Each creature of his choice within 10 feet must make a DC 30 Reflex saving throw, taking 14 (4d6) fire damage plus 14 (4d6) Light damage on a failed save, or half if it fails. succeeds.

\textbf{Blinding Gaze (Costs 3 Actions).} The solar targets a creature within 30 feet that it can see. If the target can see the solar, the target must succeed on a DC 18 Fortitude save or be blinded until a spell such as \textit{lesser refresh} removes the blindness.

\textbf{Teleport.} The solar magically teleports up to 36 meters away, along with all equipment it is wearing or carrying, into an unoccupied space that it can see.

\textbf{Ecology} \\
Environment: Any plane with good traits \\
Organization: Solitary or couple \\
\textbf{Treasure}: Double (Full Armor +5, Dancing Greatsword +5, Composite Longbow +5) \\
\textbf{Description} \\
Solars are the most powerful of angels, usually the right hand of a deity or champions of causes that benefit an entire world or plane. A solar usually has an almost human appearance, although some of them resemble other humanoid races and some have even more unusual shapes. A solar is about 2.7 meters tall, weighs about 250 kg and has a deep and imperious voice, impossible to ignore. Most of them have silvery or golden skin.

Blessed with a more powerful array of magical abilities, solars are terrifying foes capable of killing the most powerful evil creatures on their own. Among the celestials they are regarded as the most excellent track seekers, and the best among them, it is said, are capable of following day-old tracks left by a Pit Devil across the Astral Plane. Some of them take the mantle of monster slayers and hunt down powerful fiends and undead such as devourers, night hags, night shadows and pit devils, even making forays into evil planes and the Negative Energy Plane. to destroy these creatures at the source, before they can harm mortals. Some of the oldest solar have accomplished their mission, and are reputed to be slayers of now extinct creatures.

Solars accept the role of guardians, usually of supernatural concepts or objects or creatures of great importance. On one world, a group of solar protects the sun's energy conduits against attempts to extinguish it and bring eternal darkness by evil races such as the Elves. On another, seven solar watch over seven mystical chains that keep the gods of evil imprisoned in a demiplane. On yet another, a solar with a flaming sword protects the Earthly Paradise, preventing all creatures from entering.

In worlds where gods can take physical form, solars are sent to become prophets and gurus (often in the guise of mortals), thus laying the foundations of cults that will become great religions. In worlds oppressed by evil, solars are the clandestine priests who bring hope to the oppressed or who allow themselves to be martyred so that their essence can explode in the surrounding regions and grow in the hearts of future heroes.

While not a deity, the solar power approaches that of the demigods, and they often act as counselors for younger or weaker deities. In some polytheistic faiths, mortals venerate one or more solars as aspects or servants on a par with true deities (however never without the approval of the deity in question) or consider the more famous solar as children, spouses, or lovers of true deities (what which, depending on the divinity, could correspond to the truth).

Unlike other angels, most solar are created as direct servants of the gods, amalgamating good souls and pure divine energy, but increasingly these powerful angels are being created through the "promotion" of lesser angels as devas and planetars. It is rarely the case that particularly powerful and pure souls directly ascend to solar status. The most ancient of them predate the creation of mortals, and are among the earliest creations of the gods. These solar are champions among their own kind, and have little or no interaction with mortals, focusing instead on abstract concepts such as gravity, entropy, dark matter and primal evil.

Solars who spend a lot of time on the Material Plane, especially those who take the form of mortals, are sometimes a source of aasimar or half-celestial bloodlines in human families, sometimes due to a romance, sometimes simply. for the closeness of mortals to their celestial emanations. They rarely have direct descendants, and when this happens it is always a mortal mother who carries the child: even though solars can appear of any sex, the gods have not given them the chance to give birth to a child. This is why solars tend to seek a mortal lover. Other solars have little regard for one of their kind giving a child to a mortal, so solar fathers tend to avoid contact with their offspring to avoid bringing shame upon themselves. Solars, however, tend to control their children from afar and, in times of difficulty, to help them, albeit in mysterious and discreet ways.

All angels respect the solar power and wisdom, and although they tend to work alone, they sometimes command planetar-led armies and serve as generals for great raids against the legions of Hell or the hordes of the Abyss.

\medskip\index[Mostruario]{Ankheg} \textbf{Ankheg}

\textit{Large monstrosity, misaligned}

\textbf{STRENGTH} +3

\textbf{DEXTERITY} +0

\textbf{CONSTITUTION} +1

\textbf{INTELLIGENCE} -5

\textbf{WISDOM} +1

\textbf{CHARISMA} -2

\textbf{Initiative} +0 - \textbf{Defense} 15, 12 while prone

\textbf{Hit Points} 39 (6d10 + 6)

\textbf{Movement} 9 m, Burrow 3 m

\textbf{Senses} darkvision 18 m, telluric perception 18 m,

\textbf{Languages} -

\textbf{Challenge} 2 (450 PX)

\textbf{Actions}

\textit{\textbf{Bite.} Melee Weapon Attack}: +5 to hit, 1m range, one target.

\textit{Strikes:} 10 (2d6 + 3) slashing damage plus 3 (1d6) acid damage. If the target is a Large or smaller creature, it is grabbed (DC 13 to flee). Until the grab is finished, the ankheg can only bite the grabbed creature and has + 1d6 to attack rolls against it.

\textit{\textbf{Acid Spray (Refill 6).}} The ankheg spits acid in a line 30 feet long and 1 meter wide, as long as it's not grabbing any creatures. Each creature on that line must make a DC 13 Reflex saving throw, and take 10 (3d6) acid damage on a failed save, or half that damage on a successful one.

\textbf{Ecology} \\
Environment: Temperate or warm plains \\
Organization: Solitary, pair or nest (3-6) \\
\textbf{Treasure}: Accidental \\
\textbf{Description} \\
Ankhegs are an all too common plague on rural areas. This horse-sized burrowing monster typically avoids more populous inhabited areas, but its fondness for the flesh of livestock and humans keeps them away from uninhabited areas. Their favorite habitat is rural countryside, as loose topsoil makes it very easy for them to dig around. There are tales of larger ankheg living in remote deserts, feeding on Scorpions and Camels, and rarely coming into contact with civilization (a desert ankheg is a Huge advanced ankheg).

In combat, ankheg prefer to attack with their bite. Against multiple opponents, an ankheg grabs one of the targets and attempts to retreat underground. A creature dragged underground can breathe, albeit with difficulty (even the ankheg has to do so, so the tunnels are quite porous), but is often eaten alive before its companions can save it.

Ankheg dig with their legs and jaws, moving lightning fast through dirt, sand and gravel (not rock). A digging ankheg often stops to build tunnels, sprinkling the walls with thick oral secretion. If an ankheg wants to build a tunnel while digging, he must move at half his dig speed. A typical ankheg tunnel is 3 meters high and 3 meters wide, vaguely circular in shape and 18 to 45 meters long ([1d10 + 5] × 10). Ankheg groups share the same territory and create complex networks of tunnels under the countryside, sometimes creating chasms where too many of them dig at the same time.

Although ankhegs resemble immense insects, they are more intelligent and, with a little time and a good trainer, can become pets or cargo. The fact that even "domesticated" ankheg tend to spit acid when frightened or surprised makes them unsafe in more civilized regions, but among savage races, such as Hobgoblins, Troglodytes and especially Orcs they are popular as guardians or even pets. living room. An ankheg can reach a length of 3 meters and weigh around 400 kg.

\medskip\index[Mostruario]{Harpy} \textbf{Harpy}

\textit{Medium monstrosity, chaotic evil}

\textbf{STRENGTH} +1

\textbf{DEXTERITY} +1

\textbf{CONSTITUTION} +1

\textbf{INTELLIGENCE} -2

\textbf{WISDOM} +0

\textbf{CHARISMA} +1

\textbf{Initiative} +1 - \textbf{Defense} 12

\textbf{Hit Points} 38 (7d8 + 7)

\textbf{Movement} 6m, flight 12m

\textbf{Languages} Common

\textbf{Challenge} 1 (200 PX)

\textbf{Actions}

\textit{\textbf{Multiattack.}} The armor makes two attacks: one with the claws and one with the club.

\textit{\textbf{Claws.} Melee Weapon Attack}: +3 to hit, 1m range, one target.

\textit{Strikes:} 5 (2d4 + 1) slashing damage, 1 bleed damage.

\textit{\textbf{Club.} Melee Weapon Attack}: +3 to hit, 1m range, one target.

\textit{Strikes:} 3 (1d4 + 1) hit damage.

\textit{\textbf{Song Charm.}} The harpy sings a magical melody. Any humanoid and giant within 90 meters of the harpy who can hear the song must succeed on a DC 11 Will saving throw or be fascinated until the song ends. The harpy must take a bonus action during her next round to continue singing. You can stop singing at any time. The singing ends if the harpy is incapacitated.

While fascinated by the harpy, one target is incapacitated and ignores the songs of other harpies. If the fascinated target is more than 1 meter from the harpy, the target must move during their round to make their way to the harpy using the most direct route. Before moving into dangerous terrain, such as lava or a well, and before taking damage from any source other than the harpy, the target can re-roll the saving throw. A creature can re-roll its saving throw at the end of each of its rounds. If the saving throw is successful, the effect ends for that target.

A target that succeeds at the saving throw is immune to that harpy's song for the next 24 hours.

\textbf{Ecology} \\
Environment: Temperate Marshes \\
Organization: Solitary, pair or flock (3-12) \\
\textbf{Treasure}: Standard (Leather Armor, Spiked Mace, and other treasure) \\
\textbf{Description} \\
Often viewed as evil and corrupt creatures, harpies know how others think and act. This perceptive ability gives them an advantage in finding their favorite meals. Although wild creatures easily fall prey to the enchanting song, these evil birdwomen prefer meals topped with complex sentient thoughts. Easy prey makes the meal boring.

Although ultimately wild and with no remorse for their actions, several harpies live among humanoid societies and enjoy exploiting creatures they deem potential meals.

Harpies tend to wear stolen trinkets and trinkets from their victims, because they love to take pleasure in the bright ornaments of men. Up close, these creatures exude the stench of their devoured victims and seldom let creatures not yet bewitched get too close so they don't smell the blood and rot on their feathers. For this reason, many harpies sprinkle themselves with perfumes and aromatic oils.

Harpies are markedly different depending on the region they live in. Some resemble a mixture of vultures and women, while others carry the regal traits of hawks and falcons on their feathers. Rare broods of harpies, in isolated and tropical places of the world, also have colorful feathers like parrots.

\medskip\index[Mostruario]{Azer} \textbf{Azer}

\textit{medium Elemental, legal neutral}

\textbf{STRENGTH} +3

\textbf{DEXTERITY} +1

\textbf{CONSTITUTION} +2

\textbf{INTELLIGENCE} +1

\textbf{WISDOM} +1

\textbf{CHARISMA} +0

\textbf{Initiative} +1 - \textbf{Defense} 18 (natural armor, shield)

\textbf{Hit Points} 39 (6d8 + 12)

\textbf{Movement} 9 m

\textbf{Saving Throws} Fortitude +2, Reflex +1, Will +1

\textbf{Immunity to Damage} fire, poison

\textbf{Condition Immunity} poisoned

\textbf{Languages} Ignan

\textbf{Challenge} 2 (450 PX)

\textit{\textbf{Heated Weapons.}} When the azer strikes with a metal melee weapon, it deals 3 (1d6) additional Fire damage (already included in the attack).

\textit{\textbf{Heated Body.}} A creature that contacts the azer or hits it with a melee attack while within 1 meter of it takes 5 (1d10) points of fire damage.

\textit{\textbf{Living Fire.}} An azer does not need food, drink or sleep.

\textit{\textbf{Illumination.}} The azer radiates bright light within a 10-foot radius and dim light for an additional 10-feet.

\textbf{Actions}

\textit{\textbf{Warhammer.} Melee Weapon Attack}: +6 to hit, 1m range, one target.

\textit{Hits:} 7 (1d8 + 3) slash damage, or 8 (1d10 + 3) slash damage when used with two hands to make a melee attack, plus 3 (1d6) fire damage.

\textbf{Ecology} \\
Environment any terrain (Plane of Fire) \\
Organization: Solitary, couple, group (3-6), team (11-20 plus 2 3rd level sergeants and 1 3rd-6th level leader) or clan (30-100 plus 50 \% of non fighters plus 1 3rd level sergeant every 20 adults, 5 5th level lieutenants and 3 7th level captains) \\
\textbf{Treasure}: Standard (Perfect Scaled Armor, Perfect Warhammer, Light Hammer, other treasure) \\
\textbf{Description} \\
A proud and industrious Race from the Plane of Fire, the Azer work in their strongholds of bronze and brass, always ready to fight their long and seething war against the Efreet. The Azer people live in a society where every member knows their place. Born with specific duties, usually related to the activities of their father or mother, the Azer are dedicated to these occupations throughout their lives. A caste system further keeps Azer society in line. Nobles, who reign without accountability, wear ornate brass kilts as a symbol of their caste, while those of merchants and shop owners are made of sturdy bronze. Copper kilts are worn by the working caste, made up of servants, craftsmen and laborers.

Capable of channeling heat through metal weapons and tools, Azer hardly ever uses non-metallic weapons, and prefer melee over ranged attacks. They are used to taking prisoners, taking them back to their fortresses and forcing them to work for them for a year and a day.

More than half a million Azer live in the legendary City of Brass. Most of these unfortunate Azer live a life of slavery under the Efreet. The Azer subjugated to this Slavery continue to perform their duties without question, preferring to wait for their contracts to be concluded or hoping that their masters will die or be defeated. Dedication to order burns intensely in this Race, to the point where some of the Azer Slaves act as supervisors over their own people. Outside of the Brass City, the Azer are free to live their lives, often in other Planar metropolises, crafting items, selling wares and running taverns.

To the untrained eye, the Azer are strikingly alike. They are 1.2 meters tall but weigh 100 kg.

\medskip\index[Mostruario]{Banshee} \textbf{Banshee}

\textit{Medium undead, chaotic evil}

\textbf{STRENGTH} -5

\textbf{DEXTERITY} +5

\textbf{CONSTITUTION} +0

\textbf{INTELLIGENCE} +1

\textbf{WISDOM} +1

\textbf{CHARISMA} +4

\textbf{Initiative} +5 - \textbf{Defense} 15

\textbf{Hit Points} 58 (13d8)

\textbf{Movement} 0m, flight 18m (floats)

\textbf{Saving Throws}: Fortitude +4, Reflex +9, Will +5

\textbf{Resistance to Damage} acid, lightning, fire, sound; from magic weapon +1

\textbf{Immunity to Damage} from Void, Poison, Cold

\textbf{Condition Immunity} fascinated, grabbed, poisoned, entangled, paralyzed, petrified, prone, fatigue

\textbf{Senses} darkvision 18 m

\textbf{Languages} elven, common

\textbf{Challenge} 4 (1.100 PX)

\textit{\textbf{Detection of Life}}. The Banshee senses the presence of creatures other than undead and constructs within a 5 kilometer radius. He knows the general direction in which they are located, but not their precise location.

\textit{\textbf{Incorporeal Movement}}. The Banshee can move through other creatures and objects as if they were difficult terrain. It takes 5 (1d10) force damage if it ends its turn inside an object.

\textit{\textbf{Undead Nature.}} The Banshee does not need air, food, drink or sleep.

\textit{\textbf{Sensitivity to Light}}. While out in the open, the Banshee has -1d6 on attack rolls, as well as sight-based Wisdom (Awareness) checks.

\textbf{Actions}

\textit{\textbf{Corrupting Touch}}. Attack vs Melee Touch Defense: +6 attack roll, range 1m, one target.

\textit{Hit}: 12 (3d6 +2) void damage.

\textit{\textbf{Terrifying Face}}. Any non-undead creature within 60 feet of the Banshee who is able to see it must succeed on a DC 15 Charisma modifier Will save, otherwise it is frightened for 1 minute. A frightened target can re-roll the saving throw at the end of each of its turns, taking -1d6 if the Banshee is within line of sight; if he succeeds, the effect ends for him. If a target succeeds on its saving throw or the effect ends for it, that target is immune to the Banshee's terrifying face for the next 24 hours.

\textit{\textbf{Lament (1 / Day)}}. The Banshee lets out a fatal moan as long as it is not exposed to sunlight. This lament has no effect on constructs and undead. Any other creature within 30 feet of her that can hear her must make a DC 15 Fortitude save; if it fails, it drops to 0 hit points, while if it succeeds, it takes 35 (10d6) psychic damage.

\textbf{Ecology} \\
Environment: Any \\
Organization: Solitary \\
\textbf{Treasure}: None \\
\textbf{Description} \\
The Banshee is the enraged spirit of an elf who has betrayed loved ones or has been betrayed herself. Mad with pain, the Banshee pours her revenge on every living creature (innocent or guilty) with her fearsome touch and deadly screams.

\medskip\index[Mostruario]{Basilisk} \textbf{Basilisk}

\textit{Medium monstrosity, misaligned}

\textbf{STRENGTH} +3

\textbf{DEXTERITY} -1

\textbf{CONSTITUTION} +2

\textbf{INTELLIGENCE} -4

\textbf{WISDOM} -1

\textbf{CHARISMA} -2

\textbf{Initiative} -1 - \textbf{Defense} 17

\textbf{Hit Points} 52 (8d8 + 16)

\textbf{Movement} 6 m

\textbf{Senses} darkvision 18 m

\textbf{Languages} -

\textbf{Challenge} 3 (700 PX)

\textit{\textbf{Petrifying Gaze.}} If a creature begins its round within 30 feet of the basilisk and the two can see each other, if not incapacitated the basilisk can force the creature to make a DC 12 Fortitude save. If the creature fails its saving throw, it magically begins to turn to stone and is hampered. The creature must re-roll the saving throw at the end of its next round. If successful, the effect ends. If it fails, the creature is petrified until it is released from the spell \textit{restore} \textit{superior} or other spell.

A creature that is not surprised can look away to avoid the saving throw at the start of its round. In that case, he won't be able to see the basilisk until the start of his next round, when he can look away again. Should he look at the basilisk in the meantime, he should immediately make the saving throw.

If the basilisk is within 30 feet of its bright light reflection and sees it, it mistakes it for a rival and becomes the target of its gaze.

\textbf{Actions}

\textit{\textbf{Bite.} Melee Weapon Attack}: +7 to hit, 1m range, one target.

\textit{Strikes:} 10 (2d6 + 3) piercing damage plus 7 (2d6) poison damage.

\textbf{Ecology} \\
Environment: Any \\
Organization: Solitary, pair or colony (3-6) \\
\textbf{Treasure}: Accidental \\
\textbf{Description} \\
The basilisk, often referred to as the "King of Snakes" is an eight-legged reptile with an aggressive disposition that has the ability to turn creatures to stone with its gaze. Legend has it that, like the Cockatrice, the first basilisks were born from eggs laid by snakes and hatched by roosters, but very little in the physiology of the basilisk leaves room for this theory.

Basilisks live in almost all dry environments, from forest to desert, and their skin tends to mirror the environment around them: a desert basilisk can be bronze or brown, while a forest-dwelling one can be green in color. switched on. They tend to use caves, burrows or other sheltered areas as a refuge. These shelters are often marked by statues depicting people and animals in natural poses, which are nothing more than the petrified remains of the unfortunate who came across a basilisk.

Basilisks have the ability to consume petrified creatures; the acid produced by their stomach dissolves and extracts nutrients from the stone, although the process is slow and inefficient, making them sluggish and inert. As a result, basilisks rarely attack or hunt prey that avoids their gaze, relying on their Stealth and element of surprise in order not to run out of food. When they are not expecting the small mammals, birds or reptiles that are part of their diet, basilisks spend time sleeping in burrows. Those who are brave enough to capture basilisks or hide a treasure near them find that these beings can act as keepers or watchdogs.

An adult basilisk is almost 4 meters long, half of which is occupied by the long tail, and weighs 135 kilos. Some breeds have small curved horns on their noses or small crests of bony stings above their heads that resemble a crown. Although they are typically solitary creatures that only come together to mate and lay eggs, in particularly dangerous areas they can congregate in small groups to protect themselves and attack intruders en masse.

For reasons unknown, weasels and ferrets are immune to the basilisk's gaze, and sometimes sneak into burrows while the adult is hunting to feed on its young. Some legends say that the blood of a basilisk can transform ordinary stones into another material, but they are probably witnesses who have misinterpreted the magical restoration of creatures or parts of the body previously petrified.

\medskip\index[Mostruario]{Behir} \textbf{Behir}

\textit{Huge monstrosity, neutral evil}

\textbf{STRENGTH} +6

\textbf{DEXTERITY} +3

\textbf{CONSTITUTION} +4

\textbf{INTELLIGENCE} -2

\textbf{WISDOM} +2

\textbf{CHARISMA} +1

\textbf{Initiative} +3 - \textbf{Defense} 23

\textbf{Hit Points} 168 (16d12 + 64)

\textbf{Movement} 15 m, climb 12 m

\textbf{Skills} Move Silently / Hide +7, Awareness +6

\textbf{Immunity to Damage} lightning

\textbf{Senses} darkvision 27 m

\textbf{Languages} Draconic

\textbf{Challenge} 11 (7.200 PX)

\textbf{Actions}

\textit{\textbf{Multiattack.}} The behir makes two attacks: one bite and one crush.

\textit{\textbf{Bite.} Melee Weapon Attack}: +16 to hit, 3m range, one target.

\textit{Strikes:} 22 (3d10 + 6) piercing damage.

\textit{\textbf{Construct.} Melee Weapon Attack}: +16 to hit, range 1m, a Large or smaller creature.

\textit{Strikes:} 17 (2d10 + 6) slash damage plus 17 (2d10 + 6) slashing damage. Target is grabbed (DC 16 to flee) If the behir isn't already crushing another creature, the target is grabbed and hampered until the grab is finished.

\textit{\textbf{Swallow.}} The behir makes a bite attack against a Medium or smaller target it is grabbing. If the attack hits, the target is engulfed, and the grab ends. The engulfed target is blinded and entangled, has full cover against attacks and other effects outside the behir, and takes 21 (6d6) acid damage at the start of each behir's turn. The behir can only swallow one creature at a time.

If the behir takes 30 or more damage in a single turn from a creature it swallowed, it must succeed on a DC 14 Fortitude save at the end of that turn or vomit the creature, which falls prone to a space within 10 feet of the behir. If the behir dies, a swallowed creature is no longer hampered by it and can exit the corpse using 5 meters of movement, coming out prone.

\textit{\textbf{Breath of Lightning (Recharge 5-6).}} The behir exhales lightning in a line 6 meters long and 1 meter wide. Each creature on that line must make a DC 16 Reflex saving throw and take 66 (12d10) lightning damage on a failed save, or half that damage if it succeeds.

\textbf{Ecology} \\
Environment: Hills and Warm Deserts \\
Organization: Solitary or couple \\
\textbf{Treasure}: Double \\
\textbf{Description} \\
Instinctive and eager, the behir spends much of its time crawling through the sandy hills and desert rocks that form its territory, hunting down any creatures that dare to enter its territory. Its six pairs of sturdy, clawed legs remain folded at its sides most of the time, and only reach out in combat to grab enemies, to gallop, or to climb the slopes of sheer cliffs, the lairs of these. creatures.

On average, the behir is 12 meters long and weighs around 1800 kg. In addition to the two prominent horns on the head, many have decorative spines at regular intervals along the spine.

While being territorial and bestial in its fury, the behir is neither stupid nor necessarily evil even though, due to its egotism and tendency to claim everything in existence as its own, it often comes into conflict with other races. As such, a behir can be bribed or persuaded by intrepid negotiators willing to approach him. In these cases, a behir's tendency to attack first and then reason (or not reason at all) means that anyone trying to find an agreement must have valid reasons and immediately impress the behir with a tempting offer.

It is often said that behir are somehow related to blue dragons, but the true nature of this bond remains a mystery. Many dragons deny any Bond and dislike behirs for their lack of intelligence - an affront that infuriates behirs who are already impulsive in themselves. Because of this, many behir hold a grudge against dragons and are ready to attack any dragon that enters their territory.

\medskip\index[Mostruario]{Explosive Cockroach} \textbf{Explosive Cockroach}

\textit{Small Elemental, neutral}

\textbf{STRENGTH} +1

\textbf{DEXTERITY} +2

\textbf{CONSTITUTION} +1

\textbf{INTELLIGENCE} -5

\textbf{WISDOM} -1

\textbf{CHARISMA} -2

\textbf{Initiative} +2 - \textbf{Defense} 14

\textbf{Hit Points} 45 (8d8 + 9)

\textbf{Movement} 4m, jump 9m, dig 2m

\textbf{Saving Throws} Fortitude +5, Reflex +6, Will +3

\textbf{Damage Resistances} from nonmagical weapon strike

\textbf{Immunity to Damage} from fire

\textbf{Condition Immunity} fatigue, frightened

\textbf{Senses} blind sight 5 m

\textbf{Languages} -

\textbf{Challenge} 2 (450 PX)

\textit{detection of fire}: the Explosive Cockroach can perceive fires within 100 meters of distance, as long as they are equal to or greater than a torch

\textit{Digging}: Explosive cockroach can burrow into solid ground midway through its movement.

\textbf{Actions}

\textit{\textbf{Multi-Attack.}} The Blast Cockroach can make 1 charge attack or emit a mush of fire.

\textit{\textbf{Charge.}} Melee attack: +6 aa hit, 1 meter range, one target.

\textit{Hits:} 12 (3d6 + 3) hit damage. The creature must make a DC 11 Fortitude save or fall prone.

\textit{\textbf{Mash of Fire}} Ranged attack: +7 to hit, 3m range, 2 squares. The Explosive Cockroach regurgitates a sticky and flammable liquid in the air. Reload 1 / 3-6.

\textit{Strikes:} 18 (4d6 + 6) fire damage. Reflex save DC 13 for half.

\textit{\textbf{Death:}} When the Explosive Cockroach dies the jelly inside in contact with the air explodes all around, within 1 meter around the cockroach the flames cause 12 (4d6) of damage, Reflex save DC 15 for half.

\textbf{Ecology} \\
Environment: hot caves \\
Organization: Solitary, nest (8-64) \\
\textbf{Treasure}: Diamond 1d4x1d50m \\
\textbf{Description} \\
Explosive Cockroaches are native creatures between the elemental plane of fire and earth. Usually attracted to environments rich in flames, stone or at least heat and earth.
With a shape proportionate to those of a common cockroach if not greasy about 40 cm and thinking about 4 kg, it is a creature completely devoid of intellect acting only by pure instinct.
They are now common in caves near volcanoes or red dragon lairs having become accustomed to living on Yeru.

In the nest where they live there is at least one queen who commands the cockroaches, much bigger and stronger. Explosive Cockroaches feed on charcoal, burnt wood, burnt carcasses. They are extremely greedy for diamonds which once burned are real delicacies.

\medskip\index[Mostruario]{B.O.C} \textbf{B.O.C}

\textit{Large monstrosity, lawful evil}

\textbf{STRENGTH} +4

\textbf{DEXTERITY} +3

\textbf{CONSTITUTION} +2

\textbf{INTELLIGENCE} -2

\textbf{WISDOM} +1

\textbf{CHARISMA} -1

\textbf{Initiative} +2 - \textbf{Defense} 17

\textbf{Hit Points} 42 (8d8 + 10)

\textbf{Movement} 13 m

\textbf{Skills} Move silently / Hide +8, Awareness +6

\textbf{Resistance} +4 saving throws on illusion list spells

\textbf{Senses} darkvision 20 m, low-light vision 18 m

\textbf{Languages} common, can only understand it

\textbf{Challenge} 4 (1.100 PX)

\textbf{Actions}

\textit{\textbf{Multiattack.}} The B.O.C makes two claw attacks and one bite attack, or makes two tentacle attacks

\textit{\textbf{Claws.} Melee weapon attack}: +6 hit, 3 yards range, one target, 1 bleed damage.

\textit{Strikes:} 7 (1d6 + 4) slash damage.

\textit{\textbf{Bite.} Melee Weapon Attack}: For each claw that hit the B.O.C it gets +2 to hit with the bite. +8 to hit, 3m range, one target.

\textit{Strikes:} 10 (1d8 + 6) slash damage.

\textit{\textbf{Tentacles.} Melee Weapon Attack}: Each tentacle can hit up to 20 feet away and each can hit a different target, +6 on hit.

\textit{Hits:} 6 (1d4 + 4) hit damage

\textit{\textbf{Deflect the light.}} The B.O.C is constantly affected by an effect that alters its position, each attack roll has -1d6. This penalty is lifted if the B.O.C can be attacked without using sight to locate it.

The B.O.C constantly bends the light around itself, appearing almost a meter off its real position. This ability is not affected by normal visions, only true seeing, blind sight or telluric sense can correctly perceive the B.O.C.

\textbf{Ecology} \\
Environment: Hills and forests \\
Organization: Solitary, pair or pack (2d4) \\
\textbf{Treasure}: Accidental \\
\textbf{Description} \\
The Black Ops Cat better known by B.O.C is a large predatory feline, obviously black in color. Fierce, insatiable, it kills for the sake of hunting. Usually acts in packs and is extremely loyal to the group.

\medskip\index[Mostruario]{Bugbear} \textbf{Bugbear}

\textit{Medium humanoid (goblinoid), chaotic evil}

\textbf{STRENGTH} +2

\textbf{DEXTERITY} +2

\textbf{CONSTITUTION} +1

\textbf{INTELLIGENCE} -1

\textbf{WISDOM} +0

\textbf{CHARISMA} -1

\textbf{Initiative} +2 - \textbf{Defense} 17

\textbf{Hit Points} 27 (5d8 + 5)

\textbf{Movement} 9 m

\textbf{Skills} Move Silently / Hide +6, Survival +2

\textbf{Senses} darkvision 18 m

\textbf{Languages} Common, Goblin

\textbf{Challenge} 1 (200 PX)

\textit{\textbf{Surprise Attack.}} If the bugbear surprises a creature and hits it with an attack during the first round of combat, the target takes an additional 7 (2d6) damage
from the attack.

\textit{\textbf{Brutus.}} A melee weapon deals an additional die of damage when the bugbear strikes with it (already included in the attack).

\textbf{Actions}

\textit{\textbf{Spiked Mace.} Melee Weapon Attack}: +4 hit, 1 m range, one target.

\textit{Strikes:} 11 (2d8 + 2) piercing damage.

\textit{\textbf{Javelin.} Melee or Ranged Weapon Attack}: +4 hit, 1 m range, 12m range, one target.

\textit{Strikes:} 9 (2d6 + 2) piercing damage in melee or 5 (1d6 + 2) piercing damage at range.

\textbf{Ecology} \\
Environment temperate mountains \\
Organization: Solitaire, pair, group (3-6) or warband (7-12 plus 2 1st level Warriors and 1 3rd-5th level captain) \\
\textbf{Treasure}: NPC gear (Leather Armor, Light Wooden Shield, Spiked Mace, 3 Javelins, other treasure) \\
\textbf{Description} \\
The bugbear is the greatest exponent of the Goblinoid race, a heavy-moving brute that outstrips most humans by at least one head. They are loners who prefer to live and kill alone rather than in tribes, although it is not unusual to find a small band of Bugbear collaborating or living with a tribe of Goblins or Hobgoblins acting as elite guards or executioners.

Bugbears don't form large settlements like goblins or nations like hobgoblins; they prefer something smaller and more chaotic that leaves them free to do what they like (kill and torture) on a more personal level. Humans are the bugbear's favorite prey, and most of them count human flesh as a staple of their diets. Macabre trophies such as ears and fingers are common decorations among bugbears.

Bugbears, when they turn to religion, favor the deities of murder and violence, with various demon lords a favorite. A typical bugbear stands 2.1 meters tall and weighs 200 kg.

\medskip\index[Mostruario]{Bulette} \textbf{Bulette}

\textit{Large monstrosity, misaligned}

\textbf{STRENGTH} +4

\textbf{DEXTERITY} +0

\textbf{CONSTITUTION} +5

\textbf{INTELLIGENCE} -4

\textbf{WISDOM} +0

\textbf{CHARISMA} -3

\textbf{Initiative} +0 - \textbf{Defense} 20

\textbf{Hit Points} 94 (9d10 + 45)

\textbf{Movement} 12 m, Burrow 12 m

\textbf{Skills} Awareness +6

\textbf{Senses} darkvision 18 m, telluric perception 18 m

\textbf{Languages} -

\textbf{Challenge} 5 (1.800 PX)

\textit{\textbf{Jump from Fermo.}} A bulette can jump up to 30 feet long and up to 5 meters high with or without a run-up.

\textbf{Actions}

\textit{\textbf{Bite.} Melee Weapon Attack}: +11 to hit, 1m range, one target.

\textit{Strikes:} 30 (4d12 + 4) piercing damage.

\textit{\textbf{Lethal Leap.}} If the bulette can jump at least 4 meters as part of her movement, she can then use this action to land on her feet in a space that contains one or more creatures. Each of these creatures must succeed on a DC 16 Fortitude or Reflex save (target's choice) or be thrown prone and take 14 (3d6 + 4) slash damage plus 14 (3d6 + 4) slashing damage. If the saving throw is successful, the creature takes only half the damage, is not thrown prone, and is pushed 1 meter out of the bulette's space into an unoccupied space of the creature's choice. If there are no unoccupied spaces at range, the creature falls prone in the bulette's space.

\textbf{Ecology} \\
Environment: Temperate Hills \\
Organization: Solitary or couple \\
\textbf{Treasure}: None \\
\textbf{Description} \\
The creation of an unknown wizard from the past, the bulette has now become a ferocious hill predator. Digging rapidly under the ground, it cuts through the surface with its dorsal fin leaving a distinctive trail behind. The bulette leaps out, freeing itself from stones and dirt, to tear its prey to pieces without remorse, thus giving rise to its nickname of "land shark".

Bulettes are known to have bad temper, and they attack creatures much larger than themselves without fear. Solitary beasts except for the occasional reproductive pair, spend most of their time patrolling their territories, which can exceed 4 km2, hunting and punishing intruders with a fury capable of shaking the slopes of the hills.

Bulettes are perfect machines for devouring and destroying bones, armor, and even magical items with their mighty jaws and bubbling stomach acid. Failing anything else, a bulette may nibble on common items, but for some reason does not voluntarily eat elf meat, perhaps a sign of elven magic being involved in their creation, or dwarves, although it can wreak havoc on members of both. Stingray. Halflings, on the other hand, are among these beasts' favorite foods, and there are no sensible halflings who venture into the territory of a light-hearted bulette.

The bulette is a cunning fighter, surprising enemies with impressive agility. One of his favorite tactics is to charge and pounce on prey by attacking with his razor-sharp claws. The flesh behind the beast's dorsal crest is said to be particularly tender, and that those who want or are able to wait for the fin to be raised in the excitement of fighting or mating may attempt to deliver a fatal blow there, even if the most of those who have faced a land shark agree that the best way to win a fight with a bulette is to avoid it altogether.

\medskip\index[Mostruario]{Black Knight} \textbf{Black Knight}

\textit{Medium undead, chaotic evil}

\textbf{STRENGTH} +5

\textbf{DEXTERITY} +1

\textbf{CONSTITUTION} +5

\textbf{INTELLIGENCE} +1

\textbf{WISDOM} +2

\textbf{CHARISMA} +3

\textbf{Initiative} +3 - \textbf{Defense} 28

\textbf{Hit Points} 171 (18d8 + 90)

\textbf{Movement} 9 meters

\textbf{Saving Throws}: Fortitude +22, Reflex +18, Will +20

\textbf{skill} Intimidate +12, Religion +8, Knowledge Planes +8, Arcane Knowledge +5

\textbf{Damage Resistances} cold, lightning

\textbf{Immunity to Damage} Void, Poison; weapons +1

\textbf{Condition Immunity} fascinated, poisoned, paralyzed, fatigue, frightened

\textbf{Senses} Darkvision 36 m

\textbf{Languages} Common, Abyssal

\textbf{Challenge} 18 (20000 PX)

\textit{\textbf{Spells.}} The Black Knight has CM 7. Her spellcasting characteristic is Charisma, +3 to hit with spell attacks. The Black Knight knows the following spells:

level 1 (4 slots): \textit{Command, magic missile, burning hands, shield}

level 2 (3 slots): \textit{lock person, magic weapon}

level 3 (3 slots): \textit{counterspell, dispel magic, fireball}

level 4 (3 slots): \textit{exile, Marking Smite (with 1 automatic magic critical, Void damage)}

\textit{\textbf{Undead Nature.}} The Black Knight does not need air, food, drink or sleep.

\textit{\textbf{Legendary Resistance (1 / Day).}} If the Black Knight fails a saving throw, he may choose to succeed instead.

\textit{\textbf{Resistance to turning.}} The Black Knight has + 1d6 on saving throws against effects that turn undead.

\textbf{Actions}

\textit{\textbf{Multiattack.} 3 longsword attacks +3}: +27 to hit, 1m range, up to three different creatures.

\textit{Strikes:} 13 (1d10 + 5 + 3) slash damage + Marking Smite (2d6 void)

\textbf{Ecology} \\
Environment: Any \\
Organization: Solitary \\
\textbf{Treasure}: Longsword +3 or full armor +3, the rest of the equipment disappears with the death of the Black Knight. \\

\textbf{Description}
Damned to the depths of his soul, the Black Knight is the antithesis of the Knight of Sumkjr, indeed it often arises from the corruption of a Knight of Sumkjr. Fearsome, crafty, tactical opponent, he loves to behave and reason, maliciously, like a person who is still alive. His tactic is to cast Marching Smite before starting the fight and then cast Fireball as soon as possible.

\medskip\index[Mostruario]{Centaur} \textbf{Centaur}

\textit{Large monstrosity, good neutral}

\textbf{STRENGTH} +4

\textbf{DEXTERITY} +2

\textbf{CONSTITUTION} +2

\textbf{INTELLIGENCE} -1

\textbf{WISDOM} +1

\textbf{CHARISMA} +0

\textbf{Initiative} +2 - \textbf{Defense} 13

\textbf{Hit Points} 45 (6d10 + 12)

\textbf{Movement} 15 m

\textbf{Skills} Acrobatics +6, Awareness +3, Survival +3

\textbf{Languages} Elvish, Sylvan

\textbf{Challenge} 2 (450 PX)

\textit{\textbf{Charge.}} If the centaur moves at least 30 feet towards the target and hits with a pike attack during the same turn, the target takes an additional 10 (3d6) piercing damage.

\textbf{Actions}

\textit{\textbf{Multiattack.}} The centaur makes two attacks: one with his pike and one with his hooves or two with his longbow.

\textit{\textbf{Pike.} Melee Weapon Attack}: +6 to hit, 3m range, one target.

\textit{Strikes:} 9 (1d10 + 4) piercing damage.

\textit{\textbf{Hooves.} Melee Weapon Attack}: +6 to hit, 1m range, one target.

\textit{Strikes:} 11 (2d6 + 4) hit damage.

\textit{\textbf{Longbow.} Ranged weapon attack}: +4 to hit, range 45m, one target.

\textit{Strikes:} 6 (1d8 + 2) piercing damage.

\textbf{Ecology} \\
Environment: Plains and temperate forests \\
Organization: Solitaire, couple, gang (3-10), tribe (11-30 plus 3 3rd level hunters and 1 6th level leader) \\
\textbf{Treasure}: Standard (Plate Armor, Heavy Metal Shield, Longsword, Spear, other treasure) \\
\textbf{Description} \\
Legendary hunters and skilled warriors, centaurs are partly men and partly horses. Generally located on the fringes of civilization, this stoic population varies enormously in appearance: usually the color of the skin is very tanned but similar to that of humans in the neighboring regions, while the lower part of the body has the hues of local equines. They have dark-colored hair and eyes and rather pronounced facial features, while their total size depends on the size of the horse whose lower body they have. So even though an average centaur stands 2.1 meters tall and weighs more than 1000 kg, there are multiple regional variations, from slender lowland runners to massive mountain hunters.

Centaurs live on average about 60 years. Distant from other races and in conflict with others of their kind, Centaurs are an ancient race that slowly begins to accept the modern world. Although the majority of centaurs still live in tribes wandering across vast plains or on the fringes of mystical forests, some have abandoned the isolationist ways of their ancestors to settle in cosmopolitan cities. Often these free spirits are considered outcasts and are despised by their tribes, and therefore the decision to abandon them is a heavy choice. In some cases, however, entire tribes led by progressive leaders have begun to trade or forge alliances with other communities of humanoids, especially Elves, sometimes Gnomes, and more rarely Humans or Dwarves. Many races remain cautious of centaurs, though, mostly due to legends portraying them as territorial and ferocious creatures and the periodic violent confrontations they have with stubborn settlers and expanding countries.

\medskip\index[Mostruario]{Chimera} \textbf{Chimera}

\textit{Large monstrosity, chaotic evil}

\textbf{STRENGTH} +4

\textbf{DEXTERITY} +0

\textbf{CONSTITUTION} +4

\textbf{INTELLIGENCE} -4

\textbf{WISDOM} +2

\textbf{CHARISMA} +0

\textbf{Initiative} +0 - \textbf{Defense} 17

\textbf{Hit Points} 114 (12d10 + 48)

\textbf{Movement} 9 m, flight 18 m

\textbf{Skills} Awareness +8

\textbf{Senses} darkvision 18 m

\textbf{Languages} understands the Draconic but cannot speak

\textbf{Challenge} 6 (2.300 PX)

\textbf{Actions}

\textit{\textbf{Multiattack.}} The chimera makes three attacks: one with its bite, one with its horns and one with its claws. When the fiery breath is available, he can use the breath instead of the bite or horns.

\textit{\textbf{Claws.} Melee Weapon Attack}: +10 to hit, 1m range, one target.

\textit{Strikes:} 11 (2d6 + 4) slashing damage, 1 bleed damage.

\textit{\textbf{Horns.} Melee Weapon Attack}: +10 to hit, 1m range, one target.

\textit{Hits:} 10 (1d12 + 4) hit damage.

\textit{\textbf{Bite.} Melee Weapon Attack}: +10 to hit, 1m range, one target.

\textit{Strikes:} 11 (2d6 + 4) piercing damage.

\textit{\textbf{Fiery Breath (Cooldown 5-6).}} The dragon's head exhales fire into a 5 meter cone. Each creature in that area must make a DC 15 Reflex saving throw and take 31 (7d8) fire damage on a failed save, or half that damage on a successful one.

\textbf{Ecology} \\
Environment: Temperate Hills \\
Organization: Solitary, pair, pack (3-6) or flock (7-12) \\
\textbf{Treasure}: Standard \\
\textbf{Description} \\
Chimeras are monstrous creatures born of primordial evil. Hateful and ravenous, they hunt both on the ground and in the air. A chimera's dragon head can be any type of evil dragon, with the corresponding breath and wings generally having the same scales as the head. Chimeras speak in three overlapping voices, but they rarely do so, typically only to flatter a more powerful creature. A chimera is 1 meter high at the withers, reaching a length of 3 meters and a weight of 350 kg. \\
Chimeras prefer meat, but can survive on vegetables if necessary (although their mood deteriorates further when forced to do so). The fact that they fly means that they can choose their prey carefully, and they generally hunt large areas looking for the easy ones. They are too stupid and belligerent to acquire followers, although a tribe of kobolds can sometimes make offerings to them. On the contrary, they are intelligent and stubborn enough to be mediocre pets, and only a creature far more powerful than them can succeed in subduing them. They can form equal partnerships with respectful humanoids or similar creatures, and they also agree to be used as mounts. A pack of chimeras have a hierarchy similar to that of lions, with a dominant male commanding the group and most of the hunts performed by the females. A lone chimera can be a lone young male or a female with cubs nearby.


\medskip\index[Mostruario]{Chuul} \textbf{Chuul}

\textit{Large aberration, chaotic evil}

\textbf{STRENGTH} +4

\textbf{DEXTERITY} +0

\textbf{CONSTITUTION} +3

\textbf{INTELLIGENCE} -3

\textbf{WISDOM} +0

\textbf{CHARISMA} -3

\textbf{Initiative} +0 - \textbf{Defense} 18

\textbf{Hit Points} 93 (11d10 + 33)

\textbf{Movement} 9 m, swim 9 m

\textbf{Skills} Awareness +4

\textbf{Immunity to Damage} poison

\textbf{Condition Immunity} poisoned

\textbf{Senses} darkvision 18 m

\textbf{Languages} understands the Language of the Deeps but cannot speak

\textbf{Challenge} 4 (1.100 PX)

\textit{\textbf{Amphibian.}} The chuul can breathe air and water.

\textit{\textbf{Sense of Magic.}} The chuul senses magic within 36 meters of himself. This trait functions like the \textit{detect} \textit{magical} spell but itself is not magical.

\textbf{Actions}

\textit{\textbf{Multiattack.}} The chuul makes two attacks with its claws. If the chuul is grabbing a creature, it can also use its tentacles once.

\textit{\textbf{Chele.} Melee Weapon Attack}: +10 to hit, 3m range, one target.

\textit{Strikes:} 11 (2d6 + 4) hit damage. A target is grabbed (DC 14 to flee) if it is Large or smaller and the chuul is not already grabbing two other creatures.

\textit{\textbf{Tentacles.}} A creature grabbed by the chuul must succeed on a DC 13 Fortitude save or be poisoned for 1 minute. Until the poisoning ends, the target is paralyzed. The target can re-roll the saving throw at the end of each of its rounds, ending the effect for itself on success.

\textbf{Ecology} \\
Environment temperate swamps \\
Organization: Solitary, pair or pack (3-6) \\
\textbf{Treasure}: Standard \\
\textbf{Description} \\
Chuul are armored shellfish-like predators, always lurking beneath the surface of shallow ponds and bogs, which come out of their hiding place to grab their prey with their claws and then paralyze them with the tentacles of their mouths before eating them alive.

Chuul are excellent swimmers, but they prefer to attack land creatures or creatures accustomed to shallow water. Once they grab their victims, chuul often drag them into deep water. Lizards are a favorite prey of chuul, although pale dungeon-dwelling species of chuul prefer morlocks, duergars, unwary elves, and others unfortunate who get too close to their underground streams, with the exception of troglodytes whose taste i chuul find it particularly disgusting.

Chuul are surprisingly intelligent, and many engage in pointless speculation about their origins and motives. They speak a chirping, gurgling Commune dialect, but few of them are inclined to chat with those not of their kind, and if there is a Chuul society outside of the frenzied mating season, no one has yet discovered. On the contrary, the minds of the chuul seem devoted only to finding the perfect place to ambush to attack other intelligent creatures and how to decorate their elaborate lairs with trophies of their victims. Although the chuul seem uninterested in using tools, they have a compulsive need to collect those of their victims. A typical chuul is 2.4 meters tall and weighs 325 kg.

\medskip\index[Mostruario]{Kobold} \textbf{Kobold}

\textit{Small humanoid (kobold), lawful evil}

\textbf{STRENGTH} -2

\textbf{DEXTERITY} +2

\textbf{CONSTITUTION} -1

\textbf{INTELLIGENCE} -1

\textbf{WISDOM} -2

\textbf{CHARISMA} -1

\textbf{Initiative} +2 - \textbf{Defense} 13

\textbf{Hit Points} 5 (2d6 - 2)

\textbf{Movement} 9 m

\textbf{Senses} darkvision 18 m

\textbf{Languages} Common, Draconic

\textbf{Challenge} 1/8 (25 PX)

\textit{\textbf{Sensitivity to Light}}. While out in the open, the kobold has -1d6 on attack rolls, as well as sight-based Wisdom checks.

\textit{\textbf{Pack Tactics.}} The kobold has + 1d6 to attack rolls against a creature if at least one of the kobold's allies is within 1 meter of the creature and that ally is incapacitated.

\textbf{Actions}

\textit{\textbf{Dagger.} Melee Weapon Attack}: +4 to hit, 1m range, one target.

\textit{Strikes:} 4 (1d4 + 2) piercing damage.

\textit{\textbf{Slingshot.} Ranged weapon attack}: +4 to hit, range 9m, one target.

\textit{Strikes:} 4 (1d4 + 2) hit damage.

\textbf{Ecology} \\
Environment: Temperate or underground forests \\
Organization: solitary, group (2-4), nest (5-30 plus an equal number of non-combatants, 1 3rd level sergeant every 20 adults and 1 4th-6th level chief) or tribe (31-300 more than 35 \% of non-combatants, 1 3rd level sergeant every 20 adults, 2 4th level lieutenants, 1 6th-8th level chief and 5-16 Cruel Rats) \\
\textbf{Treasure}: NPC gear (leather armor, spear, slingshot, other treasure) \\
\textbf{Description} \\
Kobolds are creatures of darkness, most easily encountered in huge underground mazes or in the dark corners of forests where the sun never shines. Because of the physical similarity, kobolds loudly proclaim themselves heirs of the dragon bloodline and destined to rule the earth under the wing of their great divine cousins, but most dragons consider them little more than annoying insects. But, even as they proclaim divine lineage and the evidence of their destiny, the kobolds are aware of their weakness. Cowardly and scheming, they never fight openly if they can avoid it, setting ambushes and traps instead, holing up in their mazes behind a blanket of crude but ingenious pitfalls, or rolling over the enemy in vast howling hordes.

The hue of the kobolds also varies between the brothers of the same brood, ranging between the colors of chromatic dragons, with a predominance of red, and more rarely white, green, blue and black.

\medskip\index[Mostruario]{Cockatrice} \textbf{Cockatrice}

\textit{Small monstrosity, misaligned}

\textbf{STRENGTH} -2

\textbf{DEXTERITY} +1

\textbf{CONSTITUTION} +1

\textbf{INTELLIGENCE} -4

\textbf{WISDOM} +1

\textbf{CHARISMA} -3

\textbf{Initiative} +1 - \textbf{Defense} 12

\textbf{Hit Points} 27 (6d6 + 6)

\textbf{Movement} 6m, flight 12m

\textbf{Senses} darkvision 18 m

\textbf{Languages} -

\textbf{Challenge} 1/2 (100 PX)

\textbf{Actions}

\textit{\textbf{Bite.} Melee weapon attack}: +3 to hit, range 1 yards, a creature.

\textit{Strikes:} 3 (1d4 + 1) piercing damage, and the target must succeed on a DC 11 Fortitude save to avoid being magically petrified. If the saving throw fails, the creature begins to turn to stone and is hampered. At the end of the next turn, he must re-roll the saving throw. If successful, the effect ends. If it fails, the creature is petrified for 24 hours.

\textbf{Ecology} \\
Environment: Temperate Plains \\
Organization: Solitary, pair, squadron (3-5) or flock (6-12) \\
\textbf{Treasure}: None \\
\textbf{Description} \\
Stupid, malevolent and repulsive, cockatrices are shunned by other creatures for their ability to turn flesh to stone. Legends state that the first cockatrice emerged from an egg laid by a rooster and hatched by a toad. Whether this story is true or not, today's cockatrices breed among themselves in terrifying, dirty burrows dug haphazardly by at least a dozen clucking creatures. Males outnumber females in these flocks, and are distinguished only by barbels and crests. A typical cockatrice stands just over 60 centimeters tall and weighs 2.5 kg.

Although their diet consists mainly of petrified seeds and insects (which serve as both gastroliths and nourishment in the creature's stomach), cockatrices fiercely defend their territory from anything they deem a threat, and wandering males wandering in search of new places to build lairs sometimes lead them to involuntary contact with humans, with devastating results.

The cockatrice's strange ability to turn other creatures into stone is her best defense, and her lair is invariably filled with the remains of petrified enemies. Ironically, however, weasels and ferrets, the creatures most likely to end up in cockatrices' nests to eat their eggs, seem completely immune to this effect. For unknown reasons, cockatrices are both terrified and furious with common roosters, and there is an equal chance that they will flee or attack when a confrontation occurs.


\medskip\index[Mostruario]{Couatl} \textbf{Couatl}

\textit{Medium Celestial, legal good}

\textbf{STRENGTH} +3

\textbf{DEXTERITY} +5

\textbf{CONSTITUTION} +3

\textbf{INTELLIGENCE} +4

\textbf{WISDOM} +5

\textbf{CHARISMA} +4

\textbf{Initiative} +5 - \textbf{Defense} 21

\textbf{Hit Points} 97 (13d8 + 39)

\textbf{Movement} 9 m, flight 9 m

\textbf{Saving Throws} Fortitude +9, Reflex +13, Will +14

\textbf{Damage Resistances} from Light

\textbf{Immunity to Damage} from a non-magical weapon

\textbf{Senses} vision of the true 36 m

\textbf{Languages} all, telepathy 36 m

\textbf{Challenge} 4 (1.100 PX)

\textit{\textbf{Magical Weapons.}} The couatl's weapon attacks are magical.

\textit{\textbf{Innate Spells.}} The couatl's innate spellcasting trait is Charisma. The couatl can cast these spells innately, using only verbal components:

At will: \textit{identification of good and evil, identification of the magic, identification of thoughts}

3 / day each: \textit{blessing, create food and water, heal wounds,} \textit{protection from poisons, restore lower, sanctuary, shield} 1 / day each: \textit{restore higher, search, dream }

\textit{\textbf{Protected Mind.}} The couatl is immune to scrutiny and any effect that senses its emotions, reads its thoughts or identifies its position.

\textbf{Actions}

\textit{\textbf{Bite.} Melee weapon attack}: +8 to hit, range 1 yards, a creature.

\textit{Strikes:} 8 (1d6 + 5) piercing damage, and the target must succeed on a DC 13 Fortitude save or be poisoned for 24 hours. Until the poisoning ends, the target is unconscious. Another creature can take an action to awaken the target.

\textit{\textbf{Construct.} Melee Weapon Attack}: +6 to hit, range 10 feet, a Medium or smaller creature.

\textit{Strikes:} 10 (2d6 + 3) hit damage, and the target is grabbed (DC 15 to escape). Until the grapple is complete, the target is in the way, and the couatl cannot constrict another target.

\textit{\textbf{Shapeshift.}} The couatl can magically transform into a humanoid or beast whose challenge rating is equal to or less than its own, or revert to its true form. At death it returns to its true form. Any equipment he is wearing or carrying is absorbed or carried in the new form (couatl's choice).

In the new form, the couatl retains its game stats and ability to speak, but its Defense, movement methods, Strength, Dexterity, and other actions are replaced by those of the new form, and it gains any stats or abilities (Additional Actions and den shares) owned by its new form and not its original. If the new form has a bite attack, the couatl can use their bite in the new form.

\textbf{Ecology} \\
Environment: warm forests \\
Organization: Solitary, pair or flock (3-6) \\
\textbf{Treasure}: Standard \\
\textbf{Description} \\
Couatls are servants of lawful good deities, although some operate independently of any higher entity. Respected and admired for their wisdom and beauty, they seek to lead mortals to the right path and use their powers to fight evil, especially those known to travel the planes. Some couatls are seen as benevolent deities by isolated societies, and although couatls cringe at the thought of pretending to be a deity, they allow these misunderstandings to perpetuate as they allow them to lead these societies on paths of peace and cooperation with their neighbors. . A couatl is about 3.6 meters long, with a wingspan of about 5 meters and weighs 900 kg.

As native outsiders, couatls must eat. They prefer the same foods as real snakes, such as mammals and birds, although they are known to devour evil humanoids. Because they prefer to spend time pursuing their intent rather than hunting, they appreciate food offerings, especially small boars and birds. A couatl sometimes shows its appreciation to an adventurer or group that has done him a service by giving him 1d4 of his bright colored feathers. These free feathers, when used as an additional material component, allow a caster who casts planar ally to summon that specific couatl without paying the normal cost in gold or other values, provided the couatl approves the service requested by the caster.

\medskip \index[Mostruario]{Creeping Cumulus} \textbf{Creeping Cumulus}

\textit{Large plant, misaligned}

\textbf{STRENGTH} +4

\textbf{DEXTERITY} -1

\textbf{CONSTITUTION} +3

\textbf{INTELLIGENCE} -3

\textbf{WISDOM} +0

\textbf{CHARISMA} -3

\textbf{Initiative} -1 - \textbf{Defense} 18

\textbf{Hit Points} 136 (16d10 + 48)

\textbf{Movement} 6m, swim 6m

\textbf{Skills} Move Silently / Hide +2

\textbf{Damage Resistances} cold, fire

\textbf{Immunity to Damage} lightning

\textbf{Condition Immunity} blinded, deafened, fatigue

\textbf{Senses} blind sight 18 m (blind beyond this radius)

\textbf{Languages} -

\textbf{Challenge} 5 (1.800 PX)

\textit{\textbf{Lightning Absorption.}} Whenever the creeping mound takes lightning damage, it takes no damage and recovers a number of hit points equal to the lightning damage dealt.

\textbf{Actions}

\textit{\textbf{Multiattack.}} The crawling mound makes two slam attacks. If both attacks hit a Medium or smaller creature, the target is grabbed (DC 14 to escape) and the crawling mound uses Wrap on it.

\textit{\textbf{Slam.} Melee Weapon Attack}: +11 to hit, 1m range, one target.

\textit{Hits:} 13 (2d8 + 4) hit damage.

\textit{\textbf{Wrap.}} The creeping mound engulfs a Medium or smaller creature it has grabbed. The shrouded target is blinded, hindered, and unable to breathe, and must succeed on a DC 14 Fortitude save at the start of each mound's turn or take 13 (2d8 + 4) slash damage. If the pile moves, the enveloped target moves with it. The pile can only engulf one creature at a time.

\textbf{Ecology} \\
Environment: Temperate Forests or Swamps \\
Organization: Solitary \\
\textbf{Treasure}: Standard \\
\textbf{Description} \\
The creeping mounds, also called creeping only, look like decaying plant masses. They are intelligent carnivorous plants, with a penchant for elven flesh. The brain and sensory organs are found in the upper body. Creeping mounds usually have a circumference of 2.4 meters and are 1.8 to 2.7 meters high. They weigh around 1,900 kg.

Creeping mounds are strange creatures, more like a tangle of vermin than a single rooted plant. They are omnivores, capable of drawing sustenance from anything, clinging to trees to suck their sap, inserting roots into the soil to absorb simple nutrients or consuming the flesh and bones of prey.

Creeping mounds are incredibly stealthy in their natural environment. They blend in with the surrounding terrain and can wait motionless for days for potential prey to arrive. They can be practically anywhere and attack at any time without warning and regardless of whether or not there are survivors, as long as they have to eat.

Creeping mounds usually lead a nomadic and solitary existence in deep forests and fetid swamps but can also be found underground, in the midst of mushroom thickets. Disturbing rumors speak of clusters of creeping mounds that congregate around large mounds deep in jungles and swamps, often during violent lightning storms. The reason for this behavior is unknown, and many sages wonder if there is a dark and alien purpose behind it.

\medskip \index[Mostruario]{Death Reaver}\textbf{Death Reaver}

\textit{Large construct, undead, unaligned}

\textbf{STRENGTH} +5

\textbf{DEXTERITY} +0

\textbf{CONSTITUTION} +4

\textbf{INTELLIGENCE} -4

\textbf{WISDOM} -2

\textbf{CHARISMA} -5

\textbf{Initiative} +2 - \textbf{Defense} 21

\textbf{Hit Points} 105 (10d10 + 50)

\textbf{Movement} 9 m

\textbf{Saving Throws} Fortitude +15, Reflex +9, Will +7

\textbf{Awareness} +4

\textbf{Immunity to Damage} poison

\textbf{Condition Immunity} poisoned, fascinated, fatigued, paralyzed, petrified

\textbf{Senses} darkvision 30m

\textbf{Languages} understands all the languages of the creature but cannot speak

\textbf{Challenge} 6 (2300 PX)

\textit{\textbf{Undead Nature.}} The Death Reaver does not need air, food, drink, or sleep.

\textit{\textbf{Immutable Form.}} As a construct he cannot be affected by spells or effects that change his form.

\textit{\textbf{Container.}} The Death Reaver has an opening compartment with a door on the metal back that can hold up to 100kg of objects, from large to small.

\textit{\textbf{Air Resistance.}} The Death Reaver has an innate resistance to spells from the Air Magic List.

\textit{\textbf {Sensitive to Fire.}} The Death Reaver takes one less action the next round if it takes fire damage.

\textbf{Actions}

\textit{\textbf{MultiAttack.}} The Death Reaver attacks with two claw or attacks with one pincer and uses Paralyzing Eye.

\textbf{\textit{Claw.}} +9 on strike, range 1 meter

\textit{Hit}: 16 (2d10 + 5) bludgeoning hit damage

\textit{\textbf {Paralyzing eye}}: The affected creature, within 60 feet, must make a Fortitude save at DC 16 or be paralyzed for 2d4 rounds.

\textbf{Ecology} \\
Environment: Any, caves \\
Organization: 1-2 Death Reaver, 1d4 + 1 Guardians \\
\textbf{Treasure}: How much collected \\
\textbf{Description} \\
Death Reaver are undead constructed from pieces of various corpses and pieces of iron to resemble species of large armored crabs.
The back, completely metallic, acts as a container for the treasures that the Death Reaver finds, the claws, in a variable number between 6 and 8, are just over a meter long and have the characteristic of leaving each a different imprint being assembled from pieces of metal and verse bodies.

The large central eye, perhaps once belonging to a humanoid, allows the controller and builder of the Razziamorti to see and command it. The purpose of a Death Reaver is to explore, usually a system of caves or paths, in search of the remains of past raiders and adventurers to steal magical items and treasures.

Usually a Death Reaver is always accompanied by several guardians (other creatures under the controller's command) who help him "fix" any "resistances" still active.

\subsection{Demons}

\medskip\index[Mostruario]{Balor} \textbf{Balor}

\textit{Huge Fiend (demon), chaotic evil}

\textbf{STRENGTH} +8

\textbf{DEXTERITY} +2

\textbf{CONSTITUTION} +6

\textbf{INTELLIGENCE} +5

\textbf{WISDOM} +3

\textbf{CHARISMA} +6

\textbf{Initiative} +5 - \textbf{Defense} 29

\textbf{Hit Points} 262 (21d12 + 126)

\textbf{Movement} 12 m, flight 24 m

\textbf{Saving Throws} Fortitude +29, Reflex +17, Will +25

\textbf{Resistance to Damage} cold, lightning;

\textbf{Immunity to Damage} fire, poison, weapons +1

\textbf{Condition Immunity} poisoned

\textbf{Damage Vulnerability} cold iron

\textbf{Senses} vision of the true 36 m

\textbf{Languages} Abyssal, telepathy 36 m

\textbf{Challenge} 19 (22000 PX)

\textit{\textbf{Magical Weapons.}} The demon's weapon attacks are magical.

\textit{\textbf{Aura of Fire.}} At the beginning of each demon's turn, each creature within 1 meter of it takes 10 (3d6) points of fire damage, and flammable objects in the aura and that are not worn or carried catch fire. A creature that comes into contact with the demon or hits it with a melee attack while within 1 meter of it takes 10 (3d6) fire damage.

\textit{\textbf{Magic Resistance.}} The demon has + 1d6 on saving throws against spells and other magical effects.

\textit{\textbf{Mortal Spasm.}} When the demon dies, it explodes; each creature within 30 feet of it must make a DC 20 Reflex saving throw, taking 70 (20d6) fire damage on a failed save, or half that damage on a successful one. The explosion sets fire to flammable objects that are not worn or carried, and destroys the demon's weapons.

\textbf{Actions}

\textit{\textbf{Multiattack.}} The demon makes two attacks: one with the longsword and one with the whip.

\textit{\textbf{Whip.} Melee Weapon Attack}: +30 hit, range 9 m, one target.

\textit{Strikes:} 15 (2d6 + 8) slashing damage plus 10 (3d6) fire damage, and the target must succeed on a DC 20 Fortitude save or be dragged 7 meters towards the demon.

\textit{\textbf{Longsword.} Melee Weapon Attack}: +30 to hit, 3m range, one target.

\textit{Strikes:} 21 (3d8 + 8) slashing damage plus 13 (3d8) lightning damage. If the demon gets a critical hit, roll the damage three times instead of two.

\textit{\textbf{Teleport.}} The demon magically teleports, along with all equipment it wears or carries, to an unoccupied space that it can see within 36 meters.

\textbf{Ecology} \\
Environment Any (Abyss) \\
Organization: Solitary or warband (1 Balor and 2-5 Glabrezu) \\
\textbf{Treasure}: Standard (Unholy Longsword + 1, Flame Whip + 1, other treasure) \\
\textbf{Description} \\
When people whisper terrifying tales of demonic creatures, they mostly imagine an imposing figure of fire and flesh, a horned nightmare armed with a flaming whip and sword, flying through the night in search of its prey. The demon these people fear is the Balor, and this fear is fully justified, since few demons can match the mighty Balor in strength or brutality.

In the Abyss, the Balors are mostly in the service of the demon lords, acting as generals or captains (other than extremely powerful balors, known as balor lords). A balor usually commands vast legions of demons, and while it often allows these eager, drooling servants to fight its battles, it is anything but a coward. If the opportunity arises to join a fight, few balors choose to hold back.

A Balor stands 4.2 meters tall and weighs 2,250 kg. Only the cruellest mortal souls can fuel the creation of a balor: unlike other demons, it often takes numerous souls of powerful villains to give birth to a new balor.

\medskip\index[Mostruario]{Demogorgon} \textbf{Demogorgon}

\textit{Huge Fiend (demon prince), chaotic evil}

\textbf{STRENGTH} +9

\textbf{DEXTERITY} +2

\textbf{CONSTITUTION} +8

\textbf{INTELLIGENCE} +5

\textbf{WISDOM} +3

\textbf{CHARISMA} +7

\textbf{Initiative} +5 - \textbf{Defense} 35

\textbf{Hit Points} 468 (26d10 + 208)

\textbf{Movement} 15 meters, swim 9m

\textbf{Saving Throws}: Fortitude +34, Reflex +28, Will +29

\textbf{Skills} all +15

\textbf{Damage Resistances} cold, lightning, fire

\textbf{Immunity to Damage} Void, Poison; weapons +2

\textbf{Condition Immunity} fascinated, poisoned, paralyzed, fatigue, frightened

\textbf{Senses} Vision of the real 40 m

\textbf{Languages} all, telepathy 45 m

\textbf{Challenge} 26 (90000 PX)

\textit{\textbf{Spells.}} The Demogorgon has CM 20. Her spellcasting characteristic is Charisma, +7 to hit with spell attacks. The Demogorgon knows the following spells:

At will: Detection of the magic, Greater Image

level 3 (4 slots): \textit{dispel magic, fear, telekinesis}

level 4 (1 slot): \textit{projected image, mental regression}

\textit{\textbf{Demonic Nature.}} The Demogorgon does not need air, food, drink or sleep.

\textit{\textbf{Legendary Resistance (3 / Day).}} If the Demogorgon fails a saving throw, he may choose to succeed instead.

\textit{\textbf{Resistance to turning.}} The demogorgon has + 1d6 on saving throws against effects that turn undead.

\textit{\textbf{Two heads.}} Demogorgon has + 1d6 on saving throws against being blind, deaf, passed out

\textbf{Actions}

\textit{\textbf{Multiattack.} 2 attacks with tentacle}: +30, range 10 feet, one creature. All of Demogorgon's attacks are considered magical +2.

\textit{Hits:} 35 (4d12 +9) hit damage. The affected creature must make a Fortitude save at DC 36 or its maximum hit points drop by the same amount.

\textit{\textbf{Gaze}} Demogorgon stares at a creature he can see within 40 meters. The target must make a Will save at DC 23.

\textit{Gaze Effect:} Demogorgon chooses one of these effects or is random:

1. Powerful Gaze. The target is passed out until the next round or until the Demogorgon is out of line of sight

2. Hypnotic gaze. The target is dominated by the Demogorgon who establishes every action. This look requires the use of both heads of the Demogorgon.

3. Gaze of Madness. The target is under the influence of the Confusion spell, which persists, with no further saving throw, as long as the demogorgon is in sight. The Demogorgon does not have to stay focused for the effect to last.

\textbf{Additional Actions}

The Demogorgon can perform 3 additional actions, chosen from those below and one per round only at the end of another creature's round.

\textbf{Tail.} The Demogorgon attacks with its tail. +30 to hit, 5m range, one target. If it hits 31 hit points of blow damage plus + + 4d6 Void damage

\textbf{Gaze of Madness.} Demogorgon uses either the Mighty Gaze or the Gaze of Madness

\textbf{Ecology} \\
Environment abyss \\
Organization: Unique \\
\textbf{Treasure}: Triple \\

\textbf{Description}
Demogorgon is a huge demon, prince of the abyss and madness about 5 meters tall. It appears as a bipedal reptiloid with two baboon heads, the necks are long and serpentine like the sprawling arms. The Demogorgon's two heads are have distinct personalities that loathe each other. They often attempt to dominate each other and many of the stories involving the Demogorgon deal precisely with how one or the other Czech head to dominate the whole. There is a strong rivalry between the Demogorgon and Orcus.


\medskip\index[Mostruario]{Dretch} \textbf{Dretch}

\textit{Small Fiend (demon), chaotic evil}

\textbf{STRENGTH} +0

\textbf{DEXTERITY} +0

\textbf{CONSTITUTION} +1

\textbf{INTELLIGENCE} -3

\textbf{WISDOM} -1

\textbf{CHARISMA} -4

\textbf{Initiative} +0 - \textbf{Defense} 12

\textbf{Hit Points} 18 (4d6 + 4)

\textbf{Movement} 6 m

\textbf{Damage Resistances} cold, lightning, fire

\textbf{Immunity to Damage} poison

\textbf{Condition Immunity} poisoned

\textbf{Damage Vulnerability} cold iron

\textbf{Senses} darkvision 18 m

\textbf{Languages} Abyssal, telepathy 60 feet (only works with creatures that understand Abyssal)

\textbf{Challenge} 1/4 (50 PX)

\textbf{Actions}

\textit{\textbf{Multiattack.}} The demon makes two attacks: one with its bite and one with its claws.

\textit{\textbf{Claws.} Melee Weapon Attack}: +2 to hit, 1 m range, one target.

\textit{Strikes:} 5 (2d4) slashing damage.

\textit{\textbf{Bite.} Melee Weapon Attack}: +2 to hit, 1m range, one target.

\textit{Strikes:} 3 (1d6) piercing damage.

\textit{\textbf{Fetid Cloud (1 / Day).}} A disgusting green gas extends within a 10-foot radius of the demon. The gas propagates around the corners, and its area is slightly darkened. It remains for 1 minute or until it is blown away by a strong wind. Any creature that starts its round in that area must succeed on a DC 11 Fortitude save or be poisoned until the start of its next round. While poisoned in this way, the target can only take one action or one bonus action, but not both, during its round, and cannot take reactions.

\textbf{Ecology} \\
Environment Any (Abyss) \\
Organization: Solitary, couple, gang (3-5), group (6-12) or crowd (13 +) \\
\textbf{Treasure}: None \\
\textbf{Description} \\
Even the lowest demon of the Abyss is dangerous and has an urgent need to spread doom and dismay. The miserable dretch is as horrifying and fetid as it is cruel, even though it lacks the strength and power to satisfy its urge to brutalize others in its native realm. The purpose of the dretch's existence is to serve more powerful demons as expendable victims, and only a lucky few manage to survive long enough to evolve.

Dretches are the favorite targets of amateurs in abyssal summons. Relatively weak and easy to bully, dretches can often be forced into long periods of servitude using vague promises of opportunity to vent their frustrations and anger against weaker opponents. Yet the potential dretch summoner would do well to remember that these demons are just as cowardly and treacherous as other demons. A dretch facing a more powerful foe will be delighted to exchange any information they have in exchange for their miserable life.

Unlike most demons, the dretch's sloppy personality and disregard for prolonged physical labor rarely pays off. Advanced dretches are rare, but those who can find the strength within themselves to become more than they were at the time of their creation become the poor, cruel, bitter Abyss rulers who reign over parasites, broken souls, not intellectless deaths and other dretches. Their empires are limited to abandoned stretches of sewers beneath forgotten cities, unstable swamps shunned by the wisest minds, and other unwelcome corners of the Abyss that even demons find uncomfortable or repulsive. Yet to the lords of the dretches these kingdoms are their empires, and they defend them with pitiful tenacity.

A dretch is 1.2 meters tall and weighs 90 kg. Dretches are usually formed from the souls of wicked and indolent mortals: only a small fragment of soul is enough to give rise to such a horrifying birth. A single soul can often cause a small army of dretches to appear, and the sight of a horde of newborn dretches breaking free from the pulsating protomatter of the Abyss is both sickening and terrifying.

\medskip\index[Mostruario]{Glabrezu} \textbf{Glabrezu}

\textit{Large Fiend (demon), chaotic evil}

\textbf{STRENGTH} +5

\textbf{DEXTERITY} +2

\textbf{CONSTITUTION} +5

\textbf{INTELLIGENCE} +4

\textbf{WISDOM} +3

\textbf{CHARISMA} +3

\textbf{Initiative} +4 - \textbf{Defense} 22

\textbf{Hit Points} 157 (15d10 + 75)

\textbf{Movement} 12 m

\textbf{Saving Throws} Fortitude +18, Reflex +4, Will +11

\textbf{Resistance to Damage} cold, lightning, fire; non-magical weapon

\textbf{Immunity to Damage} poison

\textbf{Condition Immunity} poisoned

\textbf{Damage Vulnerability} cold iron

\textbf{Senses} vision of the true 36 m

\textbf{Languages} Abyssal, telepathy 36 m

\textbf{Challenge} 9 (5000 PX)

\textit{\textbf{Innate Spells.}} The demon's spellcasting trait is Intelligence. The demon can cast these spells innately, without the need for material components:

At will: \textit{dispel magic, detect magic, darkness}

1 / day each: \textit{confusion, word of power stun, fly}

\textit{\textbf{Magic Resistance.}} The demon has + 1d6 on saving throws against spells and other magical effects.

\textbf{Actions}

\textit{\textbf{Multiattack.}} The demon makes four attacks: two with its claws and two with its fists. Alternatively, he can make two attacks with his pincers and cast a spell.

\textit{\textbf{Chela.} Melee Weapon Attack}: +14 to hit, 3m range, one target.

\textit{Hits:} 16 (2d10 + 5) hit damage. If the target is a Medium or smaller creature, it is grabbed (DC 15 to escape). The glabrezu has two pincers, each of which can grab a target.

\textit{\textbf{Punch.} Melee Weapon Attack}: +14 hit, 1m range, one target.

\textit{Strikes:} 7 (2d4 + 2) hit damage.

\textbf{Ecology} \\
Environment Any (Abyss) \\
Organization: Solitary or squad (1 glabrezu, 1 succubus and 2-5 Vrock)
\textbf{Treasure}: Standard \\
\textbf{Description} \\
While the succubus is a demon who lures its prey by exploiting their carnal desires and needs, the glabrezu is a tempter of another kind. Fierce and bestial in form, the glabrezu is actually a master of deceit and lies. With her ability to hide her true form behind pleasing illusions, she uses her magic to grant the wishes of mortal humanoids as a reward for those who succumb to her deceptions and deceptions. A wish granted by a glabrezu satisfies the need of the one who expresses it in the most ruinous way possible, although these consequences may not immediately turn out to be such. A struggling blacksmith may crave fame and skill in his chosen profession, only to discover that his best Patron is a cruel and sadistic murderer who uses weapons to further his own destructive desires. A lonely man who expresses a desire for a mate might see his wish come true with an old flame of his returned to "life" in vampire form, and other examples of this kind. The glabrezu is very creative in fulfilling the wishes of a mortal.

A glabrezu is 5.4 meters tall and weighs just over 3000 kg. These treacherous demons originate from the souls of traitors, false and subversives: souls of mortals who, in life, swore falsehood or used betrayal and deception to ruin the lives of others.

\medskip\index[Mostruario]{Hezrou} \textbf{Hezrou}

\textit{Large Fiend (demon), chaotic evil}

\textbf{STRENGTH} +4

\textbf{DEXTERITY} +3

\textbf{CONSTITUTION} +5

\textbf{INTELLIGENCE} 5 (-2)

\textbf{WISDOM} +1

\textbf{CHARISMA} +1

\textbf{Initiative} +3 - \textbf{Defense} 20

\textbf{Hit Points} 136 (13d10 + 65)

\textbf{Movement} 9 m

\textbf{Saving Throws} Fortitude +16, Reflex +3, Will +9

\textbf{Resistance to Damage} cold, lightning, fire; non-magical weapon

\textbf{Immunity to Damage} poison

\textbf{Condition Immunity} poisoned

\textbf{Damage Vulnerability} cold iron

\textbf{Senses} darkvision 36 m

\textbf{Languages} Abyssal, telepathy 36 m

\textbf{Challenge} 8 (3.900 PX)

\textit{\textbf{Stench.}} Any creature that begins its round within 10 feet of the demon must succeed on a DC 14 Fortitude save or be poisoned until the start of its round. On a successful saving throw, the creature is immune to the croaking demon's stench for 24 hours.

\textit{\textbf{Magic Resistance.}} The demon has + 1d6 on saving throws against spells and other magical effects.

\textbf{Actions}

\textit{\textbf{Multiattack.}} The demon makes three attacks: one with its bite and two with its claws.

\textit{\textbf{Claw.} Melee Weapon Attack}: +11 to hit, 1 m range, one target.

\textit{Strikes:} 11 (2d6 + 4) slashing damage, 2 bleed damage.

\textit{\textbf{Bite.} Melee Weapon Attack}: +11 to hit, 1m range, one target.

\textit{Strikes:} 15 (2d10 + 4) piercing damage.

\textbf{Ecology} \\
Environment Any (Abyss) \\
Organization: Solitary or gang (2-4) \\
\textbf{Treasure}: Standard \\
\textbf{Description} \\
the hezrou lives in the vast swamps, marshes and streams of the Abyss, at ease both in water and on land. The presence of a hezrou has a detrimental effect on flora, causing knots and mutations, and surrounding waters, making them foul-smelling and brackish, peculiarities more easily identifiable in the Material Plane than in the Abyss. Prolonged exposure to this corruption causes horrendous transformations and deformities. Often entire isolated communities of deformed mutants owe their twisted appearance not so much to their depraved mores as to the proximity of a hezrou.

Though smart enough, a hezrou can honestly be said to waste their intellect. These beings prefer the simpler pleasures: sleep, the taste of torture, the bliss of eating living flesh or the joy of feeling something beautiful break and crumble in the grip of their fists. They do not often seek to build empires or lead cults, although few hezrou would refuse potential followers who come to offer themselves of their own free will.

These monstrous and bestial creatures are born from the souls of evil mortals who have poisoned themselves, their relatives or their environment, for example, drug addicts, murderers and alchemists who did not care how their experiments poisoned the natural world.

\medskip\index[Mostruario]{Marilith} \textbf{Marilith}

\textit{Large Fiend (demon), chaotic evil}

\textbf{STRENGTH} +4

\textbf{DEXTERITY} +5

\textbf{CONSTITUTION} +5

\textbf{INTELLIGENCE} +4

\textbf{WISDOM} +3

\textbf{CHARISMA} +5

\textbf{Initiative} +5 - \textbf{Defense} 26

\textbf{Hit Points} 189 (18d10 + 90)

\textbf{Movement} 12 m

\textbf{Saving Throws} Fortitude +25, Reflex +18, Will +13

\textbf{Damage Resistances} cold, lightning, fire

\textbf{Immunity to Damage} poison, weapons +1

\textbf{Condition Immunity} poisoned

\textbf{Damage Vulnerability} cold iron

\textbf{Senses} vision of the true 36 m

\textbf{Languages} Abyssal, telepathy 36 m

\textbf{Challenge} 16 (15000 PX)

\textit{\textbf{Magical Weapons.}} The demon's weapon attacks are magical.

\textit{\textbf{Reactive.}} The demon can perform a Reaction Action during each combat turn.

\textit{\textbf{Magic Resistance.}} The demon has + 1d6 on saving throws against spells and other magical effects.

\textbf{Actions}

\textit{\textbf{Multiattack.}} The demon makes seven attacks: six with long swords and one with a tail.

\textit{\textbf{Tail.} Melee weapon attack}: +18 to hit, range 10 feet, one creature.

\textit{Hits:} 15 (2d10 + 4) hit damage. If the target is Medium or smaller, it is grabbed (DC 19 to escape). Until the grapple is complete, the target is in the way, and the demon can automatically strike the target with its tail, but cannot make tail attacks against other targets.

\textit{\textbf{Long Sword.} Melee Weapon Attack}: +18 to hit, 1 m range, one target.

\textit{Strikes:} 13 (2d8 + 4) slashing damage.

\textbf{Reactions}

\textit{\textbf{Block.}} The demon adds 5 to its Defense against a melee attack that would hit it. To do this, the demon must be able to see his attacker and wield a melee weapon.

\textbf{Ecology} \\
Environment Any (Abyss) \\
Organization: Solitary, pair or platoon (1 marilith, 1-3 Glabrezu and 3-14 Babau) \\
\textbf{Treasure}: Double (6 Long Swords, other treasure) \\
\textbf{Description} \\
Rulers of demonic hordes and queens of abyssal nations, the fearsome mariliths serve the demon lords as rulers, advisers, and even lovers, yet their supremacy as strategists makes them particularly in demand as generals and army commanders. The mightiest mariliths serve no one and instead command ravenous demonic legions.

A marilith is 1.8 to 2.7 meters tall, 6 meters long from head to tip of tail, and weighs 2000 kg. Only the most arrogant and proud evil souls, usually those of cruel rulers, sadistic generals and particularly violent warlords, can cause the birth of a marilith.

\medskip\index[Mostruario]{Nalfeshnee} \textbf{Nalfeshnee}

\textit{Large Fiend (demon), chaotic evil}

\textbf{STRENGTH} +5

\textbf{DEXTERITY} +0

\textbf{CONSTITUTION} +6

\textbf{INTELLIGENCE} +4

\textbf{WISDOM} +1

\textbf{CHARISMA} +2

\textbf{Initiative} +4 - \textbf{Defense} 25

\textbf{Hit Points} 184 (16d10 + 96)

\textbf{Movement} 6m, flight 9m

\textbf{Saving Throws} Fortitude +22, Reflex +9, Will +21

\textbf{Resistance to Damage} cold, lightning, fire; non-magical weapon

\textbf{Immunity to Damage} poison

\textbf{Condition Immunity} poisoned

\textbf{Damage Vulnerability} cold iron

\textbf{Senses} darkvision 36 m

\textbf{Languages} Abyssal, telepathy 36 m

\textbf{Challenge} 13 (10000 PX)

\textit{\textbf{Magic Resistance.}} The demon has + 1d6 on saving throws against spells and other magical effects.

\textbf{Actions}

\textit{\textbf{Multiattack.}} The demon uses Horror Halo if possible. Then make three attacks: one with the bite and two with the claws.

\textit{\textbf{Claw.} Melee Weapon Attack}: +18 to hit, 3m range, one target.

\textit{Hits:} 15 (3d6 + 5) slashing damage, 2 Bleed damage.

\textit{\textbf{Bite.} Melee Weapon Attack}: +18 to hit, 1m range, one target.

\textit{Strikes:} 32 (5d10 + 5) piercing damage.

\textit{\textbf{Aureole of Horror (Cooldown 5-6).}} The demon emits a glittering, multicolored magical light. Any creature within 5 meters of the demon that can see the light must succeed on a DC 15 Will save or be frightened for 1 minute. A creature can re-roll the saving throw at the end of each of its rounds, ending the effect for itself if it succeeds. If the creature's saving throw succeeds or the effect ends, the creature is immune to the halo of
Horror of the moaning demon for the next 24 hours.

\textit{\textbf{Teleport.}} The demon teleports, along with all equipment it is wearing or carrying, to an unoccupied space that it can see up to 36 meters away.

\textbf{Ecology}
Environment Any (Abyss) \\
Organization: Solitaire or warband (1 nalfeshnee, 1 Hezrou and 2-5 Vrock) \\
\textbf{Treasure}: Standard \\
\textbf{Description} \\
Few demons understand the internal mechanics that govern the Abyss like nalfeshnees, and it is not uncommon for these demons to serve the Abyss itself rather than a demon lord. Some oversee the organic realms that spawn the new demons, while others guard places of particular importance in the hidden recesses of the plane. Often the realm of a nalfeshnee in the Abyss is superior in strength and size to the greatest of mortal realms, as these demons have a natural predisposition to rule and impose some sort of order on the chaos of the Abyss. Mortal summoners often recall them for their insane but unmatched intellect, carefully examining the arrangements made with these demons to avoid any hidden consequences and unwanted implications, as a nalfeshnee rarely accepts something that, in some twisted way, does not allow them to satisfying the needs and desires of the Abyss.

Nalfeshnees are 6 meters tall and weigh 4000 kg. They are created from the souls of wicked greedy or greedy mortals, especially those who have reigned over empires of slavery, theft, banditry, and other even more violent vices.

\medskip\index[Mostruario]{Orcus} \textbf{Orcus}

\textit{Huge fiend (demon prince), chaotic evil}

\textbf{STRENGTH} +8

\textbf{DEXTERITY} +2

\textbf{CONSTITUTION} +7

\textbf{INTELLIGENCE} +5

\textbf{WISDOM} +5

\textbf{CHARISMA} +7

\textbf{Initiative} +5 - \textbf{Defense} 30

\textbf{Hit Points} 390 (26d8 + 182)

\textbf{Movement} 15 meters, fly 15 meters

\textbf{Saving Throws}: Fortitude +33, Reflex +28, Will +31

\textbf{Skills} all +13

\textbf{Damage Resistances} cold, lightning, fire

\textbf{Immunity to Damage} Void, Poison; weapons +2

\textbf{Condition Immunity} fascinated, poisoned, paralyzed, fatigue, frightened

\textbf{Senses} Vision of the real 40 m

\textbf{Languages} all, telepathy 45 m

\textbf{Challenge} 26 (90000 PX)

\textit{\textbf{Spells.}} Orcus has CM 17. His spellcasting characteristic is Charisma, +7 to hit with spell attacks. Orcus knows the following spells:

At will: Detect the magical, Frosty Touch

level 3 (3 slots): \textit{dispel spells}

level 6 (3 slots): \textit{Create undead}

level 9 (1 slot): \textit{Stop time}

\textit{\textbf{Demonic Nature.}} Orcus does not need air, food, drink or sleep.

\textit{\textbf{Legendary Resistance (3 / Day).}} If the Orcus fails a saving throw, he may choose to succeed instead.

\textit{\textbf{Lord of the undead.}} Orcus can always decide the type of undead he creates and this remains under his control indefinitely, plus he can cast the spell no matter what conditions he is.

\textbf{Actions}

\textit{\textbf{Multiattack.} 2 wand attacks}: +30, range 10 feet, one creature. All of Orcus's attacks are considered magical +3.

\textit{Strikes:} 21 (3d8 + 8) slam damage + 13 (2d12) Void

\textit{\textbf{Tail}} Orcus strikes with his tail. +30, range 10 feet, one creature

\textit{Strikes:} 21 (3d8 + 8) slam damage + 18 (4d8) poison

\textbf{Additional Actions}

The Orcus can perform 3 additional actions, chosen from those below and one per round only at the end of another creature's round.

\textbf{Tail.} The Orcus attacks with its tail. +30 to hit, 5m range, one target. If it hits 21 (3d8 + 8) hit damage + 18 (4d8) poison

\textbf{Taste of Death.} Orcus casts the blasphemous Flame Strike spell with Void damage

\textbf{Ecology} \\
Environment abyss \\
Organization: Unique \\
\textbf{Treasure}: Triple \\

\textbf{Description}
Orcus is the Daemon Prince of the undead. He favors the companionship and service of the undead. He longs to see all life disappear and all turn into a gigantic necropolis of the undead under his command. Orcus has a goat-like head and legs, ram-like horns, a puffy body, bat-like wings, and a long tail.




\medskip\index[Mostruario]{Quasit} \textbf{Quasit}

\textit{Tiny Fiend (demon, shapeshifter), chaotic evil}

\textbf{STRENGTH} -3

\textbf{DEXTERITY} +3

\textbf{CONSTITUTION} +0

\textbf{INTELLIGENCE} -2

\textbf{WISDOM} +0

\textbf{CHARISMA} +0

\textbf{Initiative} +3 - \textbf{Defense} 14

\textbf{Hit Points} 7 (3d4)

\textbf{Movement} 12m (3m, fly 12m in bat form; 12m, climb 12m in centipede form; 12m, swim 12m in toad form)

\textbf{Skills} Move Silently / Hide +5

\textbf{Resistance to Damage} cold, lightning, fire; non-magical weapon

\textbf{Immunity to Damage} poison

\textbf{Condition Immunity}
poisoned

\textbf{Senses} darkvision 36 m

\textbf{Languages} Abyssal, Common

\textbf{Challenge} 1 (200 PX)

\textit{\textbf{Shapeshifter.}} The demon can use its action to transform into a bestial form from a bat, centipede, or toad, or to return to its true form. Its stats are the same in all forms, although attacks may vary for some of them. Any equipment he is wearing or carrying is not transformed. At death it returns to its true form.

\textit{\textbf{Magic Resistance.}} The demon has + 1d6 on saving throws against spells and other magical effects.

\textbf{Actions}

\textit{\textbf{Claws (Bite in Beast Form).} Melee Weapon Attack}: +4 to hit, 1m range, one target. \textit{Strikes:} 5 (1d4 + 3) piercing damage. If the target is a creature, it must succeed on a DC 10 Fortitude save or take 5 (2d4) poison damage and be poisoned for 1 minute. The creature can re-roll the saving throw at the end of each of its rounds, ending the effect if successful.

\textit{\textbf{Invisibility.}} The demon remains invisible until it attacks or ends its concentration. Anything the demon is carrying or wearing remains invisible as long as it remains in contact with the demon.

\textit{\textbf{Fright (1 / day).}} A creature chosen by the demon within 20 feet of it must succeed on a DC 10 Will save or be frightened for 1 minute. The target can re-roll the saving throw at the end of each of its rounds, at -1d6 if the demon is in line of sight, ending the effect prematurely if the saving throw is successful.

\textbf{Ecology} \\
Environment Any (Abyss) \\
Organization: Solitary or flock (2-12) \\
\textbf{Treasure}: Standard \\
\textbf{Description} \\
The quasit is perhaps the least powerful demon, but it is not one of the least respected: even quasits consider themselves superior to the hordes of Dretch, and true to their nature, Dretches lack the courage or drive to prove them wrong. A quasit's primary role in life is that of a familiar in the service of a charmer, but those quasits that escape this humiliating servitude acquire a will of their own and are much more dangerous. A typical quasit is 45cm tall and weighs only 4kg.

Unique among the demonic hordes, quasits are not born from the souls of wicked deceased mortals, but from living souls: when a spellcaster tries to call a quasit to himself as a familiar, his soul touches the Abyss and it reacts, creating from its matter a quasit connected to the soul of the caster and generating a powerful bond between the two.

The newly created quasits are born directly into the Material Plane, where they become familiars and, as long as they are subject to the will of their master, they hate and despise him, since they can feel the pulse of his soul and know they could aspire to something more. . A quasit is needed, yet he observes and watches in the expectation of errors that could cost his master his life, or rather, that allow him to revolt against his master. When a quasit's master dies, he can attempt to follow its soul into the Great Beyond, making a DC 15 Will save. his master's soul, in the form of a larva, rather than using it to create new demonic forms of life. In this way, a quasit can use its newly captured soul to bargain with more powerful inhabitants of the lower planes, and perhaps achieve an abject "promotion" that transforms it into a more powerful life form.

Rarely does a quasit decide to ignore the death of its master and remain on the Material Plane in search of other ways to have fun - usually by settling in an urban area where there are many individuals to torment.

\medskip\index[Mostruario]{Vrock} \textbf{Vrock}

\textit{Large Fiend (demon), chaotic evil}

\textbf{STRENGTH} +3

\textbf{DEXTERITY} +2

\textbf{CONSTITUTION} +4

\textbf{INTELLIGENCE} -1

\textbf{WISDOM} +1

\textbf{CHARISMA} -1

\textbf{Initiative} +2 - \textbf{Defense} 18

\textbf{Hit Points} 104 (11d10 + 44)

\textbf{Movement} 12 m, flight 18 m

\textbf{Saving Throws} Fortitude +13, Reflex +10, Will +6

\textbf{Resistance to Damage} cold, lightning, fire; non-magical weapon

\textbf{Immunity to Damage} poison

\textbf{Condition Immunity} poisoned

\textbf{Senses} darkvision 36 m

\textbf{Languages} Abyssal, telepathy 36 m

\textbf{Challenge} 6 (2.300 PX)

\textit{\textbf{Magic Resistance.}} The demon has + 1d6 on saving throws against spells and other magical effects.

\textbf{Actions}

\textit{\textbf{Multiattack.}} The demon makes two attacks: one with its beak and one with its spurs.
or
\textit{\textbf{Beak.} Melee Weapon Attack}: +12 to hit, 1m range, one target.

\textit{Strikes:} 10 (2d6 + 3) piercing damage.

\textit{\textbf{Spurs.} Melee Weapon Attack}: +12 to hit, 1m range, one target.

\textit{Strikes:} 14 (2d10 + 3) slashing damage.

\textit{\textbf{Spore (Refill 6).}} A cloud of toxic spores spreads in a 5 meter radius around the demon. The spores spread around the corners. Each creature in that area must succeed on a DC 14 Fortitude saving throw or be poisoned. While poisoned in this way, a target takes 5 (1d10) poison damage at the start of each of its rounds. The target can re-roll the saving throw at the end of each of its rounds, ending the effect if successful. Emptying a vial of holy water on the target also ends the effect.

\textit{\textbf{Stunning Scream (1 / Day).}} The demon lets out a horrifying scream. Any creature within 20 feet of it that can hear it, and not a demon, must succeed on a DC 14 Fortitude save or be stunned until the demon's next round ends.

\textbf{Ecology} \\
Environment Any (Abyss) \\
Organization: Solitary, pair or gang (3-10) \\
\textbf{Treasure}: Standard \\
\textbf{Description} \\
Profane champions of the Abyss, the vrocks embody all the anger, hatred and violence of this realm. As voracious and grotesquely opportunistic as the scavenger they resemble, vrocks delight in the shedding of blood, enjoying the sound and sensations of tearing the still-pulsing intestines from a living creature. \\
A typical vrock is 2.4 meters tall and weighs 200 kg. These creatures usually originate from the souls of evil mortals filled with hatred and anger, particularly those who were professional criminals, mercenaries, or murderers.



\medskip\index[Mostruario]{Nightmare Steed} \textbf{Nightmare Steed}

\textit{Large Fiend, neutral evil}

\textbf{STRENGTH} +4

\textbf{DEXTERITY} +2

\textbf{CONSTITUTION} +3

\textbf{INTELLIGENCE} +0

\textbf{WISDOM} +1

\textbf{CHARISMA} +2

\textbf{Initiative} +2 - \textbf{Defense} 15

\textbf{Hit Points} 68 (8d10 + 24)

\textbf{Movement} 18 m, flight 24 m

\textbf{Immunity to Damage} fire

\textbf{Languages} understands Abyssal, Common and Hell but cannot speak

\textbf{Challenge} 3 (700 PX)

\textit{\textbf{Grant Fire Resistance.}} The Nightmare steed can grant fire damage resistance to anyone who rides it.

\textit{\textbf{Illumination.}} The nightmare steed radiates bright light within a 10-foot radius and dim light for an additional 10-feet.

\textbf{Actions}

\textit{\textbf{Hooves.} Melee Weapon Attack}: +6 to hit, 3m range, one target.

\textit{Strikes:} 13 (2d8 + 4) slam damage plus 7 (2d6) fire damage.

\textit{\textbf{Ethereal Step.}} The nightmare steed and up to three consenting creatures within 3 meters of it can magically enter the Ethereal Plane from the Material Plane and vice versa.

\textbf{Ecology} \\
Environment: Any (Abaddon) \\
Organization: Solitary \\
\textbf{Treasure}: None \\
\textbf{Description} \\
Nightmares are fiery harbingers of death. They allow only the most evil creatures to ride them, and they are never just mounts, but they collaborate in the destruction wrought by their riders.

\subsection{Devils}

\begin{changemargin}{0.3cm}{0.3cm}\begin{enfasi}{Hell is empty and all the devils are here. (William Shakespeare, The Tempest)}\end{enfasi}\end{changemargin}\medskip

\medskip\index[Mostruario]{Bearded Devil} \textbf{Bearded Devil}

\textit{Medium fiend (devil), lawful evil}

\textbf{STRENGTH} +3

\textbf{DEXTERITY} +2

\textbf{CONSTITUTION} +2

\textbf{INTELLIGENCE} -1

\textbf{WISDOM} +0

\textbf{CHARISMA} +0

\textbf{Initiative} +2 - \textbf{Defense} 15

\textbf{Hit Points} 52 (8d8 + 16)

\textbf{Movement} 9 m

\textbf{Saving Throws} Fortitude +9, Reflex +7, Will +3

\textbf{Resistance to Damage} cold; a non-magical weapon or a non-silver weapon

\textbf{Immunity to Damage} fire, poison

\textbf{Condition Immunity} poisoned

\textbf{Senses} darkvision 36 m

\textbf{Languages} Hellish, telepathy 36 m

\textbf{Challenge} 3 (700 PX)

\textit{\textbf{Magic Resistance.}} The devil has + 1d6 on saving throws against spells and other magical effects.

\textit{\textbf{Resolute.}} The devil cannot be frightened as long as he can see an allied creature within 30 feet of him.

\textit{\textbf{Sight of the Devil.}} The devil's darkvision is not limited by magical darkness.

\textbf{Actions}

\textit{\textbf{Multiattack.}} The devil makes two attacks: one with a beard and one with a glaive.

\textit{\textbf{Beard.} Melee weapon attack}: +7 to hit, range 1 yards, a creature.

\textit{Strikes:} 6 (1d8 + 2) piercing damage, and the target must succeed on a DC 12 Fortitude save or be poisoned for 1 minute. While poisoned in this way, the target cannot recover hit points. The target can re-roll the saving throw at the end of each of its rounds, ending the effect if the saving throw is successful.

\textit{\textbf{Glaive.} Melee Weapon Attack}: +7 to hit, 3m range, one target.

\textit{Strikes:} 8 (1d10 + 3) slashing damage. If the target is a creature, excluding constructs and undead, it must succeed on a Fortitude save 12 or lose 5 (1d10) hit points at the start of each of its rounds to the infernal wound. Whenever the devil hits the wounded target with this attack, the damage dealt by the wound increases by 5 (1d10). Any creature can take an action to block the wound on a successful DC 12 Wisdom (First Aid) check. The wound also closes if the target receives healing magic.

\textbf{Ecology} \\
Environment: Any (Hell) \\
Organization: Solitary, pair, team (3-10) or troop (10-40) \\
\textbf{Treasure}: Standard (Falcione, other treasure) \\
\textbf{Description} \\
Chosen warriors of the legions of hell, the bearded devils, or barbazu, fight savagely in the name of their infernal lords and command brutal hordes of the damned in battle. They gather and train with their sickles forged in the underworld, between the vaults of the third circle of Hell, Erebus, but inevitably return to the first circle, Averno, to serve alongside the fearsome lord Barbatos.

Barbazu love to make charge attacks with their glauts and try to keep a distance of 10 feet between them and their opponents so that they can use their signature Pole Weapons to maximum effect. Against an opponent who has greater reach (or is able to avoid the devil's favorite tactic), they throw their glaws and rely on their claws and hideous beards. Standing bearded devils stand more than 1.8 meters tall (although the squatting position they hold in battle often makes them appear shorter) and weigh over 100kg.


\medskip\index[Mostruario]{Devil of Chains} \textbf{Devil of Chains}

\textit{Medium Fiend (devil), lawful evil}

\textbf{STRENGTH} +4

\textbf{DEXTERITY} +2

\textbf{CONSTITUTION} +4

\textbf{INTELLIGENCE} +0

\textbf{WISDOM} +1

\textbf{CHARISMA} +2

\textbf{Initiative} +2 - \textbf{Defense} 20

\textbf{Hit Points} 85 (10d8 + 40)

\textbf{Movement} 9 m

\textbf{Saving Throws} Fortitude +9, Reflex +4, Will +3

\textbf{Resistance to Damage} cold; non-magical weapons that are not silver

\textbf{Immunity to Damage} fire, poison

\textbf{Condition Immunity} poisoned

\textbf{Senses} darkvision 36 m

\textbf{Languages} Hellish, telepathy 36 m

\textbf{Challenge} 8 (3.900 PX)

\textit{\textbf{Magic Resistance.}} The devil has + 1d6 on saving throws against spells and other magical effects.

\textit{\textbf{Sight of the Devil.}} The devil's darkvision is not limited by magical darkness.

\textbf{Actions}

\textit{\textbf{Multiattack.}} The devil makes two attacks with the chain.

\textit{\textbf{Chain.} Melee Weapon Attack}: +16 to hit, 3m range, one target.

\textit{Strikes:} 11 (2d6 + 4) slashing damage. The target is grabbed (DC 14 to flee) if the devil isn't already grabbing another creature. Until the grapple is complete, the target is in the way and takes 7 (2d6) piercing damage at the start of each of its rounds.

\textit{\textbf{Animate Chains (Reload after 1 hour).}} Up to four chains that the devil can see and are within 60 feet of him produce sharp edges and animate under the devil's control, as long as those chains are neither worn nor carried by someone else.

Each animated chain is an object with 20 Defense, 20 hit points, resistance to piercing damage, and immunity to sonic damage. When the devil uses Multiattack during his round, he can use each animated chain to make one more chain attack. An animated chain can grab a creature on its own but can't make attacks while grabbing. An animated chain reverts to its inanimate state if it is reduced to 0 hit points or if the devil is incapacitated or dies.

\textbf{Reactions}

\textit{\textbf{Nervous Mask.}} When a creature the devil can see starts its round within 30 feet of the devil, the devil can create an illusion to resemble lost love or a bitter rival to that creature. If the creature can see the devil, it must succeed on a DC 14 Will saving throw or be frightened until the end of its round.

\textbf{Ecology} \\
Environment: Any \\
Organization: Solitary, pair, ring (3-6) or chain (7-20) \\
\textbf{Treasure}: Standard \\
\textbf{Description}
Often classified by laymen among the ranks of hellish devils, sadomasochists are not true devils. While some are known to live in Hell, they exist outside the hierarchies established by the underworld gods and his archdevils and can sometimes be found on other planes, most notably on the Plane of Shadows. Many suggest that they are natives of Hell that existed before the advent of the diabolical bloodline, although others speculate that they were brought to the plane by some sadistic power. Regardless of their origins, they roam the planes indulging their desire to cause and receive suffering, seeking pain through violent rapture and sadistic depravity.


\medskip\index[Mostruario]{Horned Devil} \textbf{Horned Devil}

\textit{Large Fiend (devil), lawful evil}

\textbf{STRENGTH} +6

\textbf{DEXTERITY} +3

\textbf{CONSTITUTION} +5

\textbf{INTELLIGENCE} +1

\textbf{WISDOM} +3

\textbf{CHARISMA} +3

\textbf{Initiative} +3 - \textbf{Defense} 23

\textbf{Hit Points} 178 (17d10 + 85)

\textbf{Movement} 6 m, flight 18 m

\textbf{Saving Throws} Fortitude +18, Reflex +17, Will +13

\textbf{Resistance to Damage} cold; punch, piercing and sharp or that are not silver plated

\textbf{Immunity to Damage} fire, poison, weapons +1

\textbf{Condition Immunity} poisoned

\textbf{Senses} darkvision 36 m

\textbf{Languages} Hellish, telepathy 36 m

\textbf{Challenge} 11 (7,200 PX)

\textit{\textbf{Magic Resistance.}} The devil has + 1d6 on saving throws against spells and other magical effects.

\textit{\textbf{Sight of the Devil.}} The devil's darkvision is not limited by magical darkness.

\textbf{Actions}

\textit{\textbf{Multiattack.}} The devil makes three melee attacks: two with his pitchfork and one with his tail. He can use Flame Throw in place of any melee attack.

\textit{\textbf{Tail.} Melee Weapon Attack}: +18 to hit, 3m range, one target.

\textit{Strikes:} 10 (1d8 + 6) piercing damage. If the target is a creature, excluding constructs and undead, it must succeed on a Fortitude save of 17 or lose 10 (3d6) hit points at the start of each of its rounds to the infernal wound. Whenever the devil wounds the target with this attack, the damage dealt by the wound increases by 10 (3d6). Any creature can take an action to block the wound by succeeding at a DC 12 Wisdom (First Aid) check. The wound also closes if the target receives healing magic.

\textit{\textbf{Pitchfork.} Melee Weapon Attack}: +18 hit, 3m range, one target.

\textit{Strikes:} 15 (2d8 + 6) piercing damage.

\textit{\textbf{Sting.} Melee Weapon Attack}: +18 to hit, 3m range, one target.

\textit{Strikes:} 13 (2d8 + 4) piercing damage plus 17 (5d6) poison damage, and the target must succeed on a DC 14 Fortitude save, or be poisoned for 1 minute. The target can re-roll the saving throw at the end of each of its rounds, ending the effect if so
succeeds.

\textit{\textbf{Cast Flame.} Ranged spell attack}: +7 to hit, range 45 yards, one target.

\textit{Strikes:} 14 (4d6) fire damage. If the target is a flammable object that is not worn or carried, it catches fire.

\textbf{Ecology} \\
Environment: Any (Hell) \\
Organization: Solitary, pair or flock (3-10) \\
\textbf{Treasure}: Standard (Unholy Spiked Chain + 1, other treasure) \\
\textbf{Description} \\
Among the deadliest warriors of the archdevils and skilled commanders of the lesser devils, the horned devils spread the rules of Hell wherever they pass. These major devils are trained, forged and reforged to be among the most relentless and obedient warriors in the multiverse. The horned devils of the troops of the infernal armies are known as cornugons, while the largest among them are called malebranche.

A typical horned devil reaches a remarkable height of 2.7 meters, has wings with an opening of 4.2 meters, and weighs 350 kg.


\medskip\index[Mostruario]{Devil of the Pit} \textbf{Devil of the Pit} \hypertarget{diavolodellafossa}{}

\textit{Large Fiend (devil), lawful evil}

\textbf{STRENGTH} +8

\textbf{DEXTERITY} +2

\textbf{CONSTITUTION} +7

\textbf{INTELLIGENCE} +6

\textbf{WISDOM} +4

\textbf{CHARISMA} +7

\textbf{Initiative} +6 - \textbf{Defense} 29

\textbf{Hit Points} 300 (24d10 + 168)

\textbf{Movement} 9 m, flight 18 m

\textbf{Saving Throws} Fortitude +24, Reflex +21, Will +18

\textbf{Resistance to Damage} cold; from blow, piercing and cutting of weapons that are not silver

\textbf{Immunity to Damage} fire, poison, weapons +2

\textbf{Condition Immunity} poisoned

\textbf{Senses} vision of the true 36 m

\textbf{Languages} Hellish, telepathy 36 m

\textbf{Challenge} 20 (25000 PX)

\textit{\textbf{Magical Weapon.}} The pit fiend's weapon attacks are magical.

\textit{\textbf{Aura of Fear.}} Any creature hostile to the devil that begins its round within 20 feet of it must make a DC 21 Will save, unless the devil is incapacitated. If the saving throw fails, the creature is frightened until the start of its next round. If the creature's saving throw succeeds, the creature is immune to the devil's aura of fear for the next 24 hours.

\textit{\textbf{Innate Spells.}} The pit devil spellcasting characteristic is Charisma. The pit devil can cast these spells innately, without the need for material components:

At will: \textit{detect magic, fireball}

3 / day each: \textit{block monsters, wall of fire}

\textit{\textbf{Magic Resistance.}} The devil has + 1d6 on saving throws against spells and other magical effects.

\textbf{Actions}

\textit{\textbf{Multiattack.}} The devil makes four attacks: one with his bite, one with his claw, one with his club and one with his tail.

\textit{\textbf{Claw.} Melee Weapon Attack}: +30 to hit, 3m range, one target.

\textit{Strikes:} 17 (2d8 + 8) slashing damage, 3 bleed damage (to a maximum of 10).

\textit{\textbf{Tail.} Melee Weapon Attack}: +30 to hit, 3m range, one target.

\textit{Hits:} 24 (3d10 + 8) hit damage.

\textit{\textbf{Mace.} Melee Weapon Attack}: +30 to hit, 3m range, one target.

\textit{Strikes:} 15 (2d6 + 8) hit damage plus 21 (6d6) fire damage.

\textit{\textbf{Bite.} Melee Weapon Attack}: +30 to hit, 1m range, one target.

\textit{Strikes:} 22 (4d6 + 8) piercing damage. The target must succeed on a DC 21 Fortitude save or be poisoned. While poisoned this way, the target cannot recover hit points, and takes 21 (6d6) poison damage at the start of each of its rounds. The poisoned target can re-roll the saving throw at the end of each of its rounds, ending the effect on itself.

\textbf{Ecology} \\
Environment: Any (Hell) \\
Organization: Solitary, couple or council (3-9) \\
\textbf{Treasure}: Double \\
\textbf{Description}
Rulers of hell realms, generals of the armies of Hell and advisers to archdevils, pit fiends are the personification of the terrifying and frightening pinnacle of the evil race.

Massive, indomitable physique and gifted with ingenious wicked intellects, these diabolical tyrants possess great autonomy whether in the service of the archdevils or in their sovereignty over hellish wastes of slaves or when engaged in subjugating mortal worlds. Solid muscles stretch over their gigantic bodies, armored with thick cutting plates capable of blocking almost any attack. Their fanged jaws the size of daggers and their bestial faces hide some of the most insidious minds in Hell.

Born in the depths of Nessus, the ninth and deepest circle of Hell, pit devils are created from the ranks of cornugons and gelugons only by archdevils and their dukes. Although many travel to the upper ranks and beyond Hell, in command of the legions of hell, most remain in Nessus, serving the courts of the mighty of Hell or in dark covenants with unmentionable purposes.

Pit devils are always over 4.2 meters tall, with a wingspan of over 6 meters and weighing more than 500 kg.

The pit devils are lords of fire and prefer areas lapped by flames. In Hell, this predisposition of theirs makes Averno, Dite, Malebolge, Nessus, and Phlegethon the groups that most easily host their temples-citadels engulfed in flames. Fanatics obsessed with diabolical superiority and the most iron obedience, pit devils, if left to act undisturbed, gather immense armies, raking the pits of Hell in search of the most depraved lemurs to transform them into real devils. Once they are certain that they have created the perfect legions, they turn their attention to the more vulnerable demiplanes and mortal worlds, looking forward to their conquest.

Servants of archdevils or other sole infernal warlords, pit devils vow to their cause, obeying the will of the nobles chosen by some dark Patron in the hope that, one day, they will win the favor of the Prince of Darkness or of Hell itself. While obedient to the hierarchies of their race, they are also strict in enforcing the rules and, if a pit devil finds himself serving an unworthy master, he would feel obliged to lay him down. Therefore, whether lords or servants, pit devils embody the will of the relentless laws of hell and make sure that only the most powerful devils can (or dare) thrive.

Only the mightiest of mortal spellcasters can or dare to summon a pit fiend. The reactions of this type of devil to summoning are swift and premeditated, usually characterized by an irrepressible fury at the idea that such an insignificant being could waste their immortal time. Whoever fails to face the burning anger is killed and his soul damned in Hell and placed at the service of the evoked devil. Whoever manages to control these greater devils also manages to intrigue them.

A pit fiend may respectfully serve a mortal lord for centuries, but his purpose always remains the same: to corrupt more and more of his soul, to secure his complete damnation, and, when he finally dies, to reclaim his soul and begin the process. to make him a totally corrupt lemur servant.

Pit devils are aware that they are immortal and are intelligent enough to have incredibly disciplined patience. Thus the earliest pit devils see in their legions the faces of the countless fools who once claimed to be their masters.


\medskip\index[Mostruario]{Ice Devil} \textbf{Ice Devil}

\textit{Large Fiend (devil), lawful evil}

\textbf{STRENGTH} +5

\textbf{DEXTERITY} +2

\textbf{CONSTITUTION} +4

\textbf{INTELLIGENCE} +4

\textbf{WISDOM} +2

\textbf{CHARISMA} +4

\textbf{Initiative} +4 - \textbf{Defense} 25

\textbf{Hit Points} 180 (19d10 + 76)

\textbf{Movement} 12 m

\textbf{Saving Throws} Fortitude +15, Reflex +14, Will +12

\textbf{Damage Resistances} slashing, piercing, and cutting of weapons other than silver

\textbf{Immunity to Damage} cold, fire, poison, weapons +1

\textbf{Condition Immunity} poisoned

\textbf{Senses} blind sight 60 ft., Darkvision 36 ft

\textbf{Languages} Hellish, telepathy 36 m

\textbf{Challenge} 14 (11,500 PX)

\textit{\textbf{Magic Resistance.}} The devil has + 1d6 on saving throws against spells and other magical effects.

\textit{\textbf{Sight of the Devil.}} The devil's darkvision is not limited by magical darkness.

\textbf{Actions}

\textit{\textbf{Multiattack.}} The devil makes three attacks: one with its bite, one with its claws and one with its tail. Alternatively, make two attacks: one with the tail and one with a spear.

\textit{\textbf{Claws.} Melee Weapon Attack}: +21 to hit, 1m range, one target.

\textit{Hits:} 10 (2d4 + 5) slashing damage plus 10 (3d6) cold damage, 1 bleed damage.

\textit{\textbf{Tail.} Melee Weapon Attack}: +21 to hit, 3m range, one target.

\textit{Hits:} 12 (2d6 + 5) slam damage plus 10 (3d6) cold damage.

\textit{\textbf{Ice Spear.} Melee Weapon Attack}: +21 to hit, 3m range, one target.

\textit{Strikes:} 14 (2d8 + 5) piercing damage plus 10 (3d6) cold damage. If the target is a creature, it must succeed at a DC 15 Fortitude save, or have a 10-foot speed reduced for 1 minute; during each of his rounds he can perform only one action or one bonus action, but not both; he cannot react. The target can re-roll the saving throw at the end of each of its rounds, ending the effect on itself if successful.

\textit{\textbf{Bite.} Melee Weapon Attack}: +10 to hit, 1m range, one target.

\textit{Strikes:} 12 (2d6 + 5) piercing damage plus 10 (3d6) cold damage.

\textit{\textbf{Wall of Ice (Recharge 6).}} The devil magically forms a wall of opaque ice on a solid surface that he can see within 60 feet of him. The wall is 30 centimeters thick and up to 9 meters wide for a maximum of 3 meters high, or a hemispherical dome with a maximum diameter of 6 meters. When the wall appears, every creature in its space is pushed out of it by the shortest path. The creature chooses which side of the wall to end up on, unless the creature is incapacitated. The creature then makes a DC 17 Reflex saving throw, taking 35 (10d6) cold damage if it fails, or half that damage if it succeeds.

The wall remains for 1 minute or until the devil is incapacitated or dies. The wall can be damaged and punctured; each 10-foot section has Defense 5, 30 hit points, vulnerability to fire damage, and immunity to acid, cold, Void, and poison damage. If a section is destroyed, it leaves a patina of freezing air in the space that previously occupied the wall. Whenever a creature ends up moving through this freezing air during a turn, consenting or not, it must make a DC 17 Fortitude save, taking 17 (5d6) cold damage if it fails, or half that damage if it fails. he succeeds. The freezing air dissipates as the rest of the wall fades.


\medskip\index[Mostruario]{Bone Devil} \textbf{Bone Devil}

\textit{Large Fiend (devil), lawful evil}

\textbf{STRENGTH} +4

\textbf{DEXTERITY} +3

\textbf{CONSTITUTION} +4

\textbf{INTELLIGENCE} +1

\textbf{WISDOM} +2

\textbf{CHARISMA} +3

\textbf{Initiative} +3 - \textbf{Defense} 24

\textbf{Hit Points} 142 (15d10 + 60)

\textbf{Movement} 12 m, flight 12 m

\textbf{Saving Throws} Fortitude +12, Reflex +12, Will +7

\textbf{Skills} Deceive +7, Perceive Emotions +6

\textbf{Resistance to Damage} cold; a non-magical weapon or a non-silver weapon

\textbf{Immunity to Damage} fire, poison

\textbf{Condition Immunity} poisoned

\textbf{Senses} darkvision 36 m

\textbf{Languages} Hellish, telepathy 36 m

\textbf{Challenge} 9 (5000 PX)

\textit{\textbf{Magic Resistance.}} The devil has + 1d6 on saving throws against spells and other magical effects.

\textit{\textbf{Sight of the Devil.}} The devil's darkvision is not limited by magical darkness.

\textbf{Actions}

\textit{\textbf{Multiattack.}} The devil makes three attacks: two with his claws and one with his sting, or one with his hooked weapon and one with his sting.

\textit{\textbf{Hooked Hooked Weapon.} Melee Weapon Attack}: +12 to hit, 3m range, one target.

\textit{Strikes:} 17 (2d12 + 4) piercing damage. If the target is a Huge or smaller creature, it is grabbed (DC 14 to escape). Until the grapple is complete, the devil cannot use his weapon attached to another target.

\textit{\textbf{Claw.} Melee Weapon Attack}: +12 to hit, 3 yards range, one target.

\textit{Strikes:} 8 (1d8 + 4) slashing damage, 1 bleed damage.

\textit{\textbf{Sting.} Melee Weapon Attack}: +12 to hit, 3m range, one target.

\textit{Strikes:} 13 (2d8 + 4) piercing damage plus 17 (5d6) poison damage, and the target must succeed on a DC 14 Fortitude save, or be poisoned for 1 minute. The target can re-roll the saving throw at the end of each of its rounds, ending the effect if successful.

\textbf{Ecology} \\
Environment: Any (Hell) \\
Organization: Solitary, team (2-3), council (4-10) or contingent (1-3 ice devils, 2-6 horned devils and 1-4 bone devils \\
\textbf{Treasure}: Standard (Frost Spear +1, other treasure) \\
\textbf{Description} \\
Enlightened strategists of the armies of Hell, the Ice Devil Insectoids are among the most ingenious and cruel minds in the legions of Hell. Known as a gelugon in the ranks of devils, an ice devil hides an icy heart stolen from a mortal in his chest, allowing him to make emotion-free decisions. Born in Cocito's frozen circle, the seventh circle of hell, most of the ice devils migrate to Caina, the eighth circle, where they plot to damn the world from frozen steel courts. Although they have the most alien and monstrous features of all devils, few others are accorded greater respect.

In combat, a gelugon sends his underlings forward, so that he can evaluate the tactics, strengths and weaknesses of the opponent in the rear, and provide them with support with his magical abilities, avoiding to capture them in the area of effect of his. spells: attitude not due to a sense of camaraderie, but to the cold and logical truth that his allies can survive longer in a fight if they are not exposed to friendly fire. Gelugons stand 3.6 meters tall and weigh approximately 350 kg.


\medskip\index[Mostruario]{Thorny Devil} \textbf{Thorny Devil}

\textit{Small Fiend (devil), lawful evil}

\textbf{STRENGTH} +0

\textbf{DEXTERITY} +2

\textbf{CONSTITUTION} +1

\textbf{INTELLIGENCE} +0

\textbf{WISDOM} +2

\textbf{CHARISMA} -1

\textbf{Initiative} +2 - \textbf{Defense} 14

\textbf{Hit Points} 22 (5d6 + 5)

\textbf{Movement} 6m, flight 12m

\textbf{Damage Resistances} cold; a non-magical weapon or a non-silver weapon

\textbf{Immunity to Damage} fire, poison

\textbf{Condition Immunity} poisoned

\textbf{Senses} darkvision 36 m

\textbf{Languages} Hellish, telepathy 36 m

\textbf{Challenge} 2 (450 PX)

\textit{\textbf{Magic Resistance.}} The devil has + 1d6 on saving throws against spells and other magical effects.

\textit{\textbf{Fly over.}} The devil does not provoke attacks of opportunity when flying out of range of an enemy.

\textit{\textbf{Spine Limited.}} The devil has twelve caudal spines. Used thorns grow back at midnight.

\textit{\textbf{Sight of the Devil.}} The devil's darkvision is not limited by magical darkness.

\textbf{Actions}

\textit{\textbf{Multiattack.}} The devil makes two attacks: one with his bite and one with his pitchfork or two with his tail spines.

\textit{\textbf{Pitchfork.} Melee Weapon Attack}: +2 to hit, 1m range, one target.

\textit{Strikes:} 3 (1d6) piercing damage.

\textit{\textbf{Bite.} Melee Weapon Attack}: +2 to hit, 1m range, one target.

\textit{Strikes:} 5 (2d4) slashing damage.

\textit{\textbf{Caudal Thorn.} Ranged weapon attack}: +4 to hit, range 6m, one target.

\textit{Strikes:} 4 (1d4 + 2) piercing damage plus 3 (1d6) fire damage.

\textbf{Ecology} \\
Environment: Any (Hell) \\
Organization: Solitary, pair, group (3-5) or platoon (6-11) \\
\textbf{Treasure}: Standard \\
\textbf{Description} \\
Sentinels of the vaults of Hell, jailers of the darkest souls and living weapons of the infernal forges, the hooked devils, known to the diabolists as hamatula, impose their fetters on the damned and guard the nefarious work of the greater devils. A hamatula likes to feel warm blood on their thorns and prefers to join the fray when offered the opportunity to fight.

Hamatula are collectors and organizers, and are the favorite allies of eager summoners, often bringing tempting treasures with them from the vaults of Hell or knowing the path to mortal riches. Left to act freely, these devils' hiding places often display the pierced trophies of old victims, hung like perverse collections of insects on bloody walls.

Most hook devils are 2.1 meters tall and weigh 150 kg, although their lean, muscular bodies look larger due to the ever-growing spikes that protrude from their bodies, sharp as blades.

\medskip\index[Mostruario]{Erinyes} \textbf{Erinyes}

\textit{Medium Fiend (devil), lawful evil}

\textbf{STRENGTH} +4

\textbf{DEXTERITY} +3

\textbf{CONSTITUTION} +4

\textbf{INTELLIGENCE} +2

\textbf{WISDOM} +2

\textbf{CHARISMA} +4

\textbf{Initiative} +3 - \textbf{Defense} 24 (plate armor)

\textbf{Hit Points} 153 (18d8 + 72)

\textbf{Movement} 9 m, flight 18 m

\textbf{Saving Throws} Fortitude +11, Reflex +12, Will +7

\textbf{Resistance to Damage} cold; a non-magical weapon or a non-silver weapon

\textbf{Immunity to Damage} fire, poison

\textbf{Condition Immunity} poisoned

\textbf{Senses} vision of the true 36 m

\textbf{Languages} Hellish, telepathy 36 m

\textbf{Challenge} 12 (8.400 PX)

\textit{\textbf{Diabolic Weapons.}} The erinni's weapon attacks are magical and deal 13 (3d8) additional poison damage when striking (already included in attacks).

\textit{\textbf{Magic Resistance.}} The erinni has + 1d6 on saving throws against spells and other magical effects.

\textbf{Actions}

\textit{\textbf{Multiattack.}} The erinni makes three attacks.

\textit{\textbf{Longsword.} Melee Weapon Attack}: +17 to hit, 1m range, one target.

\textit{Strikes:} 8 (1d8 + 4) slashing damage, or 9 (1d10 + 4) slashing damage when used with two hands, plus 13 (3d8) poison damage.

\textit{\textbf{Longbow.} Ranged weapon attack}: +17 to hit, range 45m, one target.

\textit{Strikes:} 7 (1d8 + 4) piercing damage plus 13 (3d8) poison damage, and the target must succeed on a DC 14 Fortitude save or be poisoned. The poison remains until removed by a spell \textit{inferior catering} or similar.

\textbf{Reactions}

\textit{\textbf{Block.}} The erinni adds 4 to its Defense against a melee attack that would hit it. To do this, the erinni must be able to see his attacker and wield a melee weapon.

\textbf{Ecology} \\
Environment: Any (Hell) \\
Organization: Solitaire or trio \\
\textbf{Treasure}: Triple (Fiery Composite Longbow + 1[Strength +5], rope, Longsword + 1) \\
\textbf{Description} \\
Known by many names, the Fallen, the Ashen Wings and the Furies, the devils known as erinyes insult their angelic form with their lust for vengeance and bloody justice. Executioners, not judges, the erinyes hover over the sharp eaves of Dis, the second cosmopolitan circle of Hell, always careful to seize every opportunity of battle, whether in defense of hell, at the whim of their diabolical lords or for the passionate call of capricious mortal summoners. All erinyes weave deadly living ropes with their own hair, which they use in battle to entangle and lift their enemies into the air, taunting and condemning them for their transgressions before letting them plummet from great heights.

Erinyes are beautiful and dark angels who deliberately enhance their sensuality with scars and bruises. Yet, despite their beauty, the Erinyes are not seducers: they lack the subtlety and patience required for this refined emotional art, as they prefer to solve their problems with acts of rapid and excruciating violence. Often an eriny will hold back her killing blow as she attempts to kill an enemy, only to prolong their suffering. Death is generally the only way to escape the attentions of an eriny, and the more powerful are adept at keeping their enemies alive but helpless, so as to prolong their torment, even going so far as to keep them alive with magic. The most powerful eriny torturers are said to have abilities that allow the suffering they inflict to persist even after the subject's death. Most erinyes stand just under 1.8 meters tall and weigh around 70 kg, and their black feathered wings are more than 3 meters wide.


\medskip\index[Mostruario]{Imp} \textbf{Imp}

\textit{Tiny fiend (devil, shapeshifter), lawful evil}

\textbf{STRENGTH} -2

\textbf{DEXTERITY} +3

\textbf{CONSTITUTION} +1

\textbf{INTELLIGENCE} +0

\textbf{WISDOM} +1

\textbf{CHARISMA} +2

\textbf{Initiative} +3 - \textbf{Defense} 14

\textbf{Hit Points} 10 (3d4 + 3)

\textbf{Movement} 6m, fly 12m (6m in rat form; 6m, fly 18m in crow form; 6m, climb 6m in spider form)

\textbf{Saving Throws} Fortitude +1, Reflex +6, Will +4

\textbf{Skills} Move Silently / Hide +5, Deceive +4, Feel Emotions +3

\textbf{Resistance to Damage} cold; a non-magical weapon or a non-silver weapon

\textbf{Immunity to Damage} fire, poison

\textbf{Condition Immunity} poisoned

\textbf{Senses} darkvision 36 m

\textbf{Languages} Infernal, Common

\textbf{Challenge} 1 (200 PX)

\textit{\textbf{Shapeshifter.}} The devil can use his action to transform into a bestial rat, raven, or spider form, or to return to his true form. Its stats are the same in all forms, although attacks may vary for some of them. Any equipment he is wearing or carrying is not transformed. At death it returns to its true form.

\textit{\textbf{Magic Resistance.}} The devil has + 1d6 on saving throws against spells and other magical effects.

\textit{\textbf{Sight of the Devil.}} The devil's darkvision is not limited by magical darkness.

\textbf{Actions}

\textit{\textbf{Sting (Bite in Beast Form).} Melee Weapon Attack}: +5 to hit, range 1m, a creature.

\textit{Strikes:} 5 (1d4 + 3) piercing damage, and the target must make a DC 11 Fortitude save, taking 10 (3d6) poison damage on a failed one, or half that damage if successful .

\textit{\textbf{Invisibility.}} The devil remains invisible until he attacks or ends his concentration. Whatever the devil is carrying or wearing remains invisible as long as he is in contact with the devil.

\textbf{Ecology} \\
Environment: Any (Hell) \\
Organization: Solitary, pair or flock (3-10) \\
\textbf{Treasure}: Standard \\
\textbf{Description} \\
Born directly from the pits of Hell, the imps are the least powerful devils, although these cruel and intrusive creatures play an important role in the corruption of mortal souls. Freed from the hierarchies and duties of the armies of hell, the imp delight in every opportunity to travel to the Material Plane and cunningly tempt mortals into increasingly depraved acts.

Voluntarily serving enchanters in the role of familiars, they play the part of faithful servants, often offering their masters astute advice and infernal insights. In reality, the imp work to send souls to Hell, making sure that their master's soul, along with many others, is doomed to damnation after death.

Imps vary greatly in appearance, across a broad spectrum of bestial and grotesque traits, although many of them are shaped like a winged, reddish-skinned humanoid with bulbous features. The typical imp is only 60cm tall, has a wingspan of 90cm and weighs 5kg.

One in a thousand imps is endowed with the ability to telepathically communicate with creatures within 15 meters and the power to change their form into that of a Small or Tiny animal, as if by the effect of a beast form II spell. These consular imps are highly prized by powerful devils, who send them as servants to their favorite followers or to corrupt mortal heroes. A consular imp can be summoned with the Improved Familiar feat, but only by a caster of 8th level or higher. Diabolists tell of other races of imp with similarly specialized abilities, but if these creatures really exist it is extremely rare.

Unlike other devils, imps often find themselves free and alone in the Material Plane, particularly after they have been summoned to serve as familiars and their masters have died (often, indirectly, due to the machinations of the imp itself). With no means to return home, these imps, free from all ties to arcane masters, can become dangerous nuisances or even lead small tribes of bloody humanoids, such as Goblins or Kobolds.


\medskip\index[Mostruario]{Lemur} \textbf{Lemur}

\textit{Medium Fiend (devil), lawful evil}

\textbf{STRENGTH} +0

\textbf{DEXTERITY} -3

\textbf{CONSTITUTION} +0

\textbf{INTELLIGENCE} -5

\textbf{WISDOM} +0

\textbf{CHARISMA} -4

\textbf{Initiative} -3 - \textbf{Defense} 8

\textbf{Hit Points} 13 (3d8)

\textbf{Movement} 5 meters

\textbf{Saving Throws} Fortitude +4, Reflex +3, Will +0

\textbf{Damage Resistances} cold

\textbf{Immunity to Damage} fire, poison

\textbf{Condition Immunity} fascinated, poisoned, frightened

\textbf{Senses} darkvision 36 m

\textbf{Languages} understands Hell but cannot speak

\textbf{Challenge} 0 (10 PX)

\textit{\textbf{Diabolic Rejuvenation.}} A lemur that dies in the Nine Hells returns to life with all its hit points in 1d10 days unless killed by a creature with good traits that has been executed on it. spell \textit{bless} or its remains come
sprinkled with sacred water.

\textit{\textbf{Sight of the Devil.}} The devil's darkvision is not limited by magical darkness.

\textbf{Actions}

\textit{\textbf{Punch.} Melee Weapon Attack}: +3 to hit, 1m range, one target.

\textit{Strikes:} 2 (1d4) hit damage.

\textbf{Ecology} \\
Environment: Any (Hell) \\
Organization: Solitary, pair, group (3-5), swarm (6-17) or host (10-40 or more) \\
\textbf{Treasure}: None \\
\textbf{Description} \\
The lowest of the devils, lemurs originate from the ranks of souls condemned to hell, shapeless masses of quivering flesh. The spark of instinct or memory that survives in their sleeping consciousness usually shapes their traits, which mimic those of their torturers or the tortured souls around them. Grotesque and useless, the features of a lemur reveal nothing of what it once was. Many sport several hideous faces or are nothing more than bubbling columns of cancerous flesh. Only their lumpy limbs, which constantly wiggle, appear to function properly, and are only used to destroy any non-hell life form that comes too close.

Moving lemurs consolidate into forms more than 1.2 meters tall and weighing more than 100 kg, although these disgusting devils, when resting, often have the indistinct appearance of masses of dissolved flesh with deformed features.

Although they are among the most revolting creatures in existence, lemurs play a vital role in the perverse ecology of Hell. When, at the end of its mortal existence, a soul is damned, either because it worships diabolical forces or because of a lack of faith in other divinities, it joins the masses of suffering souls who fill the plains of Avernus, the first circle of the 'Hell. Here the torments begin, as lesser devils push them along with other spirits, preparing them for the arduous journey to one of the deepest circles of hell, usually one suitable for the appropriate punishment for the crimes committed by the soul, or simply towards domination. of a devil who needs new slaves. Once in the realm of their damnation, souls face countless centuries of torment at the hands of devils, other evil beings, and the deadly machinations of Hell itself. As the mortal essence slowly goes mad, these creatures forget their lives, first becoming savage and finally little more than automatons driven by hatred and fear. After eons of this existence, the cruel process of Hell totally destroys the soul or, in the case of the most profane spirits, reconsecrates these forgotten beings in the form of lemurs, the most elemental life form of devils, senseless hordes of rotting flesh and diabolical. These repulsive beings gather in great masses, revolting waves of thousands upon thousands of these creatures.

The greater devils are able to recognize the most corrupt among them and, by means of mysterious tortures or thanks to the very powers of Hell, they reshape them into real devils, just born again and ready to serve obediently in the legions of the damned.


\subsection{Dinosaurs}

\medskip\index[Mostruario]{Plesiosaurus} \textbf{Plesiosaurus}

\textit{Large beast, misaligned}

\textbf{STRENGTH} +4

\textbf{DEXTERITY} +2

\textbf{CONSTITUTION} +3

\textbf{INTELLIGENCE} -4

\textbf{WISDOM} +1

\textbf{CHARISMA} -3

\textbf{Initiative} +2 - \textbf{Defense} 14

\textbf{Hit Points} 68 (8d10 + 24)

\textbf{Movement} 6m, swim 12m

\textbf{Saving Throws} Fortitude +18, Reflex +11, Will +9

\textbf{Skills} Move Silently / Hide +4, Awareness +3

\textbf{Languages} -

\textbf{Challenge} 2 (450 PX)

\textit{\textbf{Hold Your Breath.}} The plesiosaurus can hold its breath for 1 hour.

\textbf{Actions}

\textit{\textbf{Bite.} Melee Weapon Attack}: +6 to hit, 3m range, one target.

\textit{Strikes:} 14 (3d6 + 4) piercing damage.

\textbf{Ecology} \\
Environment: Warm Aquatic \\
Organization: Solitary, pair or pack (3-6) \\
\textbf{Treasure}: None \\
\textbf{Description} \\
The plesiosaurus is a long-necked aquatic reptile. Although technically not a dinosaur, this creature and its kin are often found hunting in lakes and oceans where dinosaurs are easy to find.


\medskip\index[Mostruario]{Tyrannosaurus} \textbf{Tyrannosaurus}

\textit{Huge beast, misaligned}

\textbf{STRENGTH} +7

\textbf{DEXTERITY} +0

\textbf{CONSTITUTION} +4

\textbf{INTELLIGENCE} -4

\textbf{WISDOM} +1

\textbf{CHARISMA} -1

\textbf{Initiative} +0 - \textbf{Defense} 17

\textbf{Hit Points} 136 (13d12 + 52)

\textbf{Movement} 15 m

\textbf{Saving Throws} Fortitude +15, Reflex +12, Will +10

\textbf{Skills} Awareness +4

\textbf{Languages} -

\textbf{Challenge} 8 (3.900 PX)

\textbf{Actions}

\textit{\textbf{Multiattack.}} The tyrannosaurus makes two attacks: one with its bite and one with its tail. Cannot make both attacks against the same target.

\textit{\textbf{Tail.} Melee Weapon Attack}: +14 to hit, 3m range, one target.

\textit{Hits:} 20 (3d8 + 7) hit damage.

\textit{\textbf{Bite.} Melee Weapon Attack}: +14 to hit, 3m range, one target.

\textit{Strikes:} 33 (4d12 + 7) piercing damage. If the target is a Medium or smaller creature, it is grabbed (DC 17 to escape). Until the grapple is complete, the target is in the way, and the tyrannosaurus cannot bite another target.

\textbf{Ecology} \\
Environment: Forests and Warm Plains \\
Organization: Solitary, pair or pack (3-6) \\
\textbf{Treasure}: None \\
\textbf{Description} \\
Tyrannosaurus is a primary predator measuring 12 meters in length and weighing 7000 kg.


\medskip\index[Mostruario]{Triceratops} \textbf{Triceratops}

\textit{Huge beast, misaligned}

\textbf{STRENGTH} +6

\textbf{DEXTERITY} -1

\textbf{CONSTITUTION} +3

\textbf{INTELLIGENCE} -4

\textbf{WISDOM} +0

\textbf{CHARISMA} -3

\textbf{Initiative} -1 - \textbf{Defense} 16

\textbf{Hit Points} 95 (10d12 + 30)

\textbf{Movement} 15 m

\textbf{Saving Throws} Fortitude +15, Reflex +8, Will +5

\textbf{Languages} -

\textbf{Challenge} 5 (1.800 PX)

\textit{\textbf{Overwhelming charge.}} If the triceratops moves at least 20 feet towards a creature and hits it with a gore attack during the same turn, the target must succeed on a DC 15 Fortitude save. or fall prone. If the target is prone, the triceratops can make a stomp attack against him as a bonus action.

\textbf{Actions}

\textit{\textbf{Gored.} Melee Weapon Attack}: +13 to hit, 1m range, one target.

\textit{Strikes:} 24 (3d10 + 6) piercing damage.

\textit{\textbf{Stomp.} Melee Weapon Attack}: +13 to hit, range 1m, a prone creature.

\textit{Strikes:} 22 (3d10 + 6) hit damage.

\textbf{Ecology} \\
Environment: Warm Plains \\
Organization: Solitary, pair or pack (5-8) \\
\textbf{Treasure}: None \\
\textbf{Description} \\
The triceratops is a short-tempered and stubborn herbivore. A typical triceratops is 9 meters long and weighs 10,000 kg.

\medskip\index[Mostruario]{Eat Brains} \textbf{Eat Brains}

\textit{Small aberration, chaotic evil}

\textbf{STRENGTH} +1

\textbf{DEXTERITY} +6

\textbf{CONSTITUTION} +5

\textbf{INTELLIGENCE} +3

\textbf{WISDOM} +0

\textbf{CHARISMA} +3

\textbf{Initiative} +10 - \textbf{Defense} 22

\textbf{Hit Points} 84 (8d8 + 48)

\textbf{Movement} 12 m

\textbf{Saving Throws} Fortitude +7, Reflex +8, Will +8

\textbf{Damage Resistance} non-magical weapons, cold, electricity

\textbf{Immunity to Damage} fire

\textbf{Condition Immunity} spells from the Illusion and Charm spell lists

\textbf{Senses} Blind Sight 18 m

\textbf{Languages} telepathy 50 m

\textbf{Challenge} 9 (3,900 PX)

\textit{\textbf{Eyes of Magic.}} The Brain Eater has Detect Magic always on.

\textit{\textbf{Innate Spells.}} The spellcaster trait of the Brain Eater is Charisma. The Brain Eater can innately cast the following spells, without the need for material components:

At will: \textit{Confusion (single target), Inflict Serious Wounds, Invisibility}

3 / day: \textit{cure moderate wounds, orb of invulnerability}

\textbf{Actions}

\textit{\textbf{Multiattack.}} The Brain Eater can make 4 attacks, one per claw

\textit{\textbf{Claw.} Melee weapon attack}: +9 to hit, range 1 yd, a creature.

\textit{Strikes:} 3 slash damage (1d4 + 1), 1 bleed damage.

\textbf{Special abilities}

\textit{\textbf{Theft of the body}}

By spending 3 Actions a Brain Eater can become tiny and crawl into the mouth / nose / ears of a defenseless or dead creature and reach the brain to feed on. This is an action that kills the creature. The Brain Eater takes control of the body and can use it at will, as if controlling the victim with a dominate monster spell. The Brain Eater has full access to all of the host's defensive and offensive abilities except spell-like abilities and spells (although the Brain Eater can still use its own spell-like abilities). A host body must not have been dead for more than 1 day for this ability to work, and even after being successfully occupied, the bodies decompose and become unusable within 7 days (unless this period is extended with the unbroken rest spell). As long as the Brain Eater occupies the body, it knows (and can speak) the languages known by the victim and the information on its identity and personality, but cannot possess the specific memories and knowledge. Damage dealt to the host body, which has double the original hit points, does not damage the brain eater and if the host body is destroyed the brain eater comes out and is stunned for 1 round.

\textbf{Ecology} \\
Environment any dungeon \\
Organization: Solitary, brood (2-6) or tribe (7-16) \\
\textbf{Treasure}: Double \\
\textbf{Description} \\
A Brain Eater is nothing more than a brain of about 50 cm with 4 powerful clawed legs.

Believed by some to be invaders from another dimension or planet, the sinister brain-eaters are certainly one of the cruelest races in the world. Unable to feel emotions or wallow in the sins of their own physical pleasure, brain-eaters are forced to steal bodies to satisfy their gluttony, lust and cruelty. There are stories that tell of entire underground cities of these creatures wearing bodies as if they were dressed to consume frightening orgies and grisly feasts. Lonely Brain Eaters often live in ruins or caves on the fringes of civilized regions in order to periodically raid the city to "acquire" a tempting new body.

Shayalia's garden is said to be full of Brain Eaters.

A Brain Eater is 90cm long and weighs around 30kg.

\medskip \textbf{Dobi} \\ \index[Mostruario]{Dobi}
\textit{Tiny fairy} \\
\textbf{Strength}: -3 \\
\textbf{Dexterity}: -1 \\
\textbf{Constitution}: +2 \\
\textbf{Intelligence}: -2 \\
\textbf{Wisdom}: +1 \\
\textbf{Charisma}: +3 \\
\textbf{Defense}: 12 - \textbf{Initiative}: +0 \\
\textbf{Hit Points}: 6 (1d8 + 2) \\
\textbf{Movement}: 3m, Swim 9m \\
\textbf{Saving Throws}: Fortitude +2, Reflex +0, Will +1 \\
\textbf{Sensi}: twilight vision 18 m \\
\textbf{Languages}: - \\
\textbf{Challenge} 0 (10 PX) \\
\textbf{Immunity}: to damage from non-magical weapons \\
\textbf{Resistance}: damage from cutting, perforation \\
\textit{\textbf{Dobi}} The Dobi sticks, to move it you need to be polite and ask it. \\
\textit{\textbf{Dobi Dobi Dobi}} When the Dobi takes more than 3 hit points of damage with a non-hit weapon it splits into two smaller Dobi each with the same amount of hit points remaining as the previous Dobi. \\
\smallskip \textbf{Actions} \\
\textit{\textbf{Dobi Dobi}} the Dobi casts a Calming Emotion aura like the spell of the same name but the saving throw is not granted. The Dobi can only affect one creature with its power at a time. \\
\textbf{Ecology} \\
Environment: Swamps \\
Organization: group \\
\textbf{Treasure}: Accidental \\
\textbf{Description} \\
{\small "... I moved the leaves of the marsh and saw on the ground a strange ball of fur, about ten centimeters in diameter, of light color. Intrigued, I picked it up, stroking its soft fur and scrutinized it carefully. It seemed to have no limbs or signs of having a muzzle with eyes, ears, mouth, but as soon as I stroked it the ball vibrated, emitting a squeak.

Finally I saw two black and lively eyes open in all that fur and then two round orecchiette sprouting, then two short but sturdy legs, suitable for jumping, resting on the ground and two others, always short but equipped with five fingers each, at half height. .

- Dobi! - replied the pet, expressing a kind of joy and enthusiasm. - Dobi dobi! -.

- So cute! - I exclaimed, stroking him. It was the cutest pet I had ever seen. - But now I'll put you down -.

"Dobi," replied the furball.
I brought my hand to the ground, but the animal did not move. I tried to peel it off my hand, but it stuck to the other. I took it with two fingers, pulling hard and quickly placed it on the ground, but immediately it jumped on my foot and remained attached to it. I had to cross the swamp with the dobi attached to my foot, not counting the other four I found clinging to the armor. "

From \textit{Journey to the first world} of \textbf{Tristan Cassandiel}}


\medskip\index[Mostruario]{Doppelganger} \textbf{Doppelganger}

\textit{Medium monstrosity (shapeshifter), neutral}

\textbf{STRENGTH} +0

\textbf{DEXTERITY} +4

\textbf{CONSTITUTION} +2

\textbf{INTELLIGENCE} +0

\textbf{WISDOM} +1

\textbf{CHARISMA} +2

\textbf{Initiative} +4 - \textbf{Defense} 16

\textbf{Hit Points} 52 (8d8 + 16)

\textbf{Movement} 9 m

\textbf{Saving Throws} Fortitude +4, Reflex +5, Will +6

\textbf{Skills} Deceive +6, Perceive Emotions +3

\textbf{Condition Immunity} fascinated

\textbf{Senses} darkvision 18 m

\textbf{Languages} Common

\textbf{Challenge} 3 (700 PX)

\textit{\textbf{Shapeshifter.}} The doppelganger can use its action to change its form to that of a Small or Medium humanoid it has seen, or to return to its true form. Its stats, aside from size, are the same in all forms. Any equipment he is wearing or carrying is not transformed. At death it returns to its true form.

\textit{\textbf{Lurking.}} In the first round of combat, the doppelganger has + 1d6 to attack rolls against any creature it took by surprise.

\textit{\textbf{Surprise Attack.}} If the doppelganger surprises a creature and hits it with an attack during the first round of combat, the target takes an additional 10 (3d6) damage from the attack.

\textbf{Actions}

\textit{\textbf{Multiattack.}} The doppelganger makes two melee attacks.

\textit{\textbf{Slam.} Melee Weapon Attack}: +6 to hit, 1m range, one target.

\textit{Strikes:} 7 (1d6 + 4) hit damage.

\textit{\textbf{Reading Thoughts.}} The doppelganger magically reads the surface thoughts of a creature within 60 feet of it. The effect can penetrate barriers, but 1 meter of wood or earth, 50 centimeters of stone, 5 centimeters of metal, or a thin sheet of lead blocks it. While the target is in range, the doppelganger's thoughts can continue to read, as long as the doppelganger's focus is not broken (like a spell's focus). While reading a target's mind, the doppelganger has + 1d6 on Wisdom and Charisma checks against the target.

\textbf{Ecology} \\
Environment: Any \\
Organization: Solitary, pair or gang (3-6) \\
\textbf{Treasure}: NPC gear \\
\textbf{Description} \\
Doppelgangers are strange beings who can take the form of those they meet. In its natural form, the creature more or less resembles a humanoid, but slender and frail, with lean limbs and not fully formed facial features. His complexion is pale, he is hairless and his eyes are white and blank.

Doppelgangers prefer to infiltrate societies where they can amass wealth and power, and see little prospect of founding cities with their fellow men. Younger doppelgangers check their skills on small tribes of orcs or goblins, then move into more complex societies such as dwarven, elven, and human communities. Rather than becoming targets by occupying leadership positions, they prefer to hold power from behind the throne, or use multiple identities to manipulate influential citizens or entire guilds.

Doppelgangers make excellent use of their natural mimicry for ambushing, baiting traps, and infiltrating humanoid societies. While they are not usually evil, they are only interested in themselves and view all others as toys to be manipulated and deceived. They love to invade human societies to satisfy their desires; some enjoy complex political games while others continually try to change race, gender and love partner. While not the norm, those doppelgangers who use their gifts for cruel and sadistic purposes are very famous, and these shape-shifters are primarily responsible for their race's sinister reputation. Certainly, a shape-shifting creature has an advantage when trying to avoid being captured for its crimes, and some particularly malevolent doppelgangers enjoy severing love affairs by staging betrayals.

Persistent rumors speak of even more powerful doppelgangers capable not only of changing their appearance, but also of making their own abilities, memories and even extraordinary and supernatural abilities of the creatures they choose to impersonate.


\subsection{Dragons}

Each Dragon has full access to all spells of a specific spell list depending on their color.

This access is guaranteed by Tàhil or Ljust depending on whether they are dragons loyal to one or the other.

\begin{itemize}
\item Each Dragon can cast spells up to a level equal to one quarter of its Challenge Rank, with minimal access to the first level.
\item Each Dragon has a number of Magic Points equal to 5 times its Challenge Rank
\item Each Dragon has a Magical Proficiency score equal to half its Challenge Rank
\end{itemize}

\medskip

\textbf{Table: Magic List for Dragons access} \index{Table: Magic List for Dragons access}

\medskip

\begin{tabular}{ll}
	\hline
\textbf{Dragon Color} 	& \textbf{Magic List Name} \\
White & Air \\\textbf{}
Blue & Air \\
Yellow & Fire, Summon \\
Black & Water, Necromancy \\
Purple & Earth \\
Red & Fire \\
Greenery & Animals and Plants \\
Silver & Transmutation, Illusion \\
Bronze & Abjuration \\
Gold & Heal, Summon \\
Brass & Divination \\
Copper & Invocation \\
\end{tabular}

\medskip

All dragons have access to the Universal magic list and have a preference for certain spells that are marked in their description.

\subsubsection{Chromatic Dragons}

\medskip\index[Mostruario]{Ancient White Dragon} \textbf{Ancient White Dragon}

\textit{Gargantuan dragon, chaotic evil}

\textbf{STRENGTH} +8

\textbf{DEXTERITY} +0

\textbf{CONSTITUTION} +8

\textbf{INTELLIGENCE} +0

\textbf{WISDOM} +1

\textbf{CHARISMA} +2

\textbf{Initiative} +0 - \textbf{Defense} 30

\textbf{Hit Points} 333 (18d20 + 144)

\textbf{Movement} 12m, swim 12m, dig 12m, fly 24m

\textbf{Saving Throws} Fortitude +19, Reflex +14, Will +16

\textbf{Skills} Move Silently / Hide +6, Awareness +13

\textbf{Damage Immunity} cold, weapons +1

\textbf{Senses} darkvision 36 m, blind sight 18 m

\textbf{Languages} Common, Draconic

\textbf{Challenge} 20 (25000 PX)

\textit{\textbf{Walk on Ice.}} The dragon can move and climb frozen surfaces without the need for ability checks. Also, hindering terrain made up of ice or snow does not cost him any additional movement.

\textit{\textbf{Legendary Resistance (3 / Day).}} If the dragon fails a saving throw, it may choose to succeed instead.

\textbf{Actions}

\textit{\textbf{Multiattack.}} The dragon can use its Frightening Presence. Then make three attacks: one with the bite and two with the claws.

\textit{\textbf{Claw.} Melee Weapon Attack}: +30 to hit, 3m range, one target.

\textit{Strikes:} 15 (2d6 + 8) slashing damage, 3 bleed damage (to a maximum of 10).

\textit{\textbf{Tail.} Melee Weapon Attack}: +30 to hit, 6m range, one target.

\textit{Strikes:} 17 (2d8 + 8) hit damage.

\textit{\textbf{Bite.} Melee Weapon Attack}: +30 to hit, 5m range, one target.

\textit{Strikes:} 19 (2d10 + 8) piercing damage plus 9 (2d8) cold damage.

\textit{\textbf{Frightening Presence.}} Any creature the dragon chooses, within 36 meters of it and aware of its presence, must succeed on a DC 16 Will saving throw or be startled for 1 minute. A creature can re-roll the saving throw at the end of each of its rounds, ending the effect if successful. If the creature's saving throw is successful or the effect ends, the creature is immune to the dragon's dreadful presence for the next 24 hours.

\textit{\textbf{Frost Breath (Cooldown 5-6).}} The dragon exhales a blast of ice into a 27 meter cone. Each creature in that area must make a DC 22 Fortitude save and take 72 (16d8) cold damage if it fails the saving throw, or half that damage if it succeeds.

\textbf{Additional Actions}

The dragon can perform 3 additional Actions, chosen from the following options. He can only use one legendary option at a time, and only at the end of another creature's turn. The dragon recovers any additional Actions spent at the start of their round.

\textbf{Wing Attack (Costs 2 Actions).} The dragon flaps its wings. Each creature within 5 yards of the dragon must succeed on a DC 22 Reflex saving throw or take 15 (2d6 + 8) hit damage and be thrown prone. The dragon can then fly up to half of its flight movement.

\textbf{Tail Attack.} The dragon makes a tail attack.

\textbf{Spot.} The dragon makes a Wisdom (Awareness) check.

\textbf{Ecology} \\
Environment: Cold Mountains \\
Organization: Solitary \\
\textbf{Treasure}: Triple \\
\textbf{Description} \\
Although many consider him the weakest and most beastly of the chromatic dragons, the white dragon makes up for his lack of cunning with sheer ferocity. White dragons live on remote, icy mountain peaks, and in the Arctic lowlands, making their lair in glistening caverns filled with ice and snow. They prefer their meals to be completely frozen. \\

\textbf{Spells} \index{White Dragon Spells} \\
This Dragon's favorite spells are: \\
- Fire Shield \\
- Ice Storm \\
- Sleet storm


\medskip\index[Mostruario]{Adult White Dragon} \textbf{Adult White Dragon}

\textit{Huge dragon, chaotic evil}

\textbf{STRENGTH} +6

\textbf{DEXTERITY} +0

\textbf{CONSTITUTION} +6

\textbf{INTELLIGENCE} -1

\textbf{WISDOM} +1

\textbf{CHARISMA} +1

\textbf{Initiative} +0 - \textbf{Defense} 25

\textbf{Hit Points} 200 (16d12 + 96)

\textbf{Movement} 12m, swim 12m, dig 9m, fly 24m

\textbf{Saving Throws} Fortitude +13, Reflex +9, Will +10

\textbf{Skills} Move Silently / Hide +5, Awareness +11

\textbf{Damage Immunity} cold

\textbf{Senses} darkvision 36 m, blind sight 18 m

\textbf{Languages} Common, Draconic

\textbf{Challenge} 13 (10000 PX)

\textit{\textbf{Walking on Ice.}} The dragon can move and climb frozen surfaces without the need for ability checks. Also, hindering terrain made up of ice or snow does not cost him any additional movement.

\textit{\textbf{Legendary Resistance (3 / Day).}} If the dragon fails a saving throw, it may choose to succeed instead.

\textbf{Actions}

\textit{\textbf{Multiattack.}} The dragon can use its Frightening Presence and then make three attacks: one with its bite and two with its claws.

\textit{\textbf{Claw.} Melee Weapon Attack}: +21 hit, 1 m range, one target, 1 bleed damage.

\textit{Strikes:} 13 (2d6 + 6) slashing damage.

\textit{\textbf{Tail.} Melee Weapon Attack}: +21 to hit, range 5 meters, one target.

\textit{Strikes:} 15 (2d8 + 6) hit damage.

\textit{\textbf{Bite.} Melee Weapon Attack}: +21 to hit, 3 yards range, one target.

\textit{Strikes:} 17 (2d10 + 6) piercing damage plus 4 (1d8) cold damage.

\textit{\textbf{Frightening Presence.}} Any creature the dragon chooses, within 36 meters of it and aware of its presence, must succeed on a DC 14 Will saving throw or be startled for 1 minute. A creature can re-roll the saving throw at the end of each of its rounds, ending the effect if successful. If the creature's saving throw is successful or the effect ends, the creature is immune to the dragon's dreadful presence for the next 24 hours.

\textit{\textbf{Frost Breath (Cooldown 5-6).}} The dragon exhales a blast of ice into a 60-foot cone. Each creature in that area must make a DC 19 Fortitude saving throw and take 54 (12d8) cold damage if it fails the saving throw, or half that damage if it succeeds.

\textbf{Additional Actions}

The dragon can perform 3 additional Actions, chosen from the following options. He can only use one legendary option at a time, and only at the end of another creature's turn. The dragon recovers any additional Actions spent at the start of their round.

\textbf{Wing Attack (Costs 2 Actions).} The dragon flaps its wings. Each creature within 10 feet of the dragon must succeed on a DC 19 Reflex saving throw or take 13 (2d6 + 6) hit damage and be thrown prone. The dragon can then fly up to half of its flight movement. \textbf{Tail Attack.} The dragon makes a tail attack
.
\textbf{Spot.} The dragon makes a Wisdom (Awareness) check.

\textbf{Ecology} \\
Environment: Cold Mountains \\
Organization: Solitary \\
\textbf{Treasure}: Triple \\
\textbf{Description} \\
Although many consider him the weakest and most beastly of the chromatic dragons, the white dragon makes up for his lack of cunning with sheer ferocity. White dragons live on remote, icy mountain peaks, and in the Arctic lowlands, making their lair in glistening caverns filled with ice and snow. They prefer their meals to be completely frozen. \\
\textbf{Spells} \index{White Dragon Spells} \\
This Dragon's favorite spells are: \\
- Fire Shield \\
- Ice Storm \\
- Sleet storm


\medskip\index[Mostruario]{Young White Dragon} \textbf{Young White Dragon}

\textit{Large dragon, chaotic evil}

\textbf{STRENGTH} +4

\textbf{DEXTERITY} +0

\textbf{CONSTITUTION} +4

\textbf{INTELLIGENCE} -2

\textbf{WISDOM} +0

\textbf{CHARISMA} +1

\textbf{Initiative} +0 - \textbf{Defense} 20

\textbf{Hit Points} 133 (14d10 + 56)

\textbf{Movement} 12m, swim 12m, dig 6m, fly 24m

\textbf{Saving Throws} Fortitude +8, Reflex +7, Will +5

\textbf{Skills} Move silently / Hide +3, Awareness +6

\textbf{Immunity to Damage} cold

\textbf{Senses} darkvision 36 m, blind sight 9 m

\textbf{Languages} Common, Draconic

\textbf{Challenge} 6 (2.300 PX)

\textit{\textbf{Walking on Ice.}} The dragon can move and climb frozen surfaces without the need for ability checks. Also, hindering terrain made up of ice or snow does not cost him any additional movement.

\textbf{Actions}

\textit{\textbf{Multiattack.}} The dragon can use its Frightening Presence. Then make three attacks: one with the bite and two with the claws.

\textit{\textbf{Claw.} Melee Weapon Attack}: +6 to hit, 1m range, one target.

\textit{Strikes:} 11 (2d6 + 4) slashing damage, 1 bleed damage.

\textit{\textbf{Bite.} Melee Weapon Attack}: +6 to hit, 3m range, one target.

\textit{Strikes:} 15 (2d10 + 4) piercing damage plus 4 (1d8) cold damage.

\textit{\textbf{Frost Breath (Cooldown 5-6).}} The dragon exhales a blast of ice into a 30-foot cone. Each creature in that area must make a DC 15 Fortitude save and take 45 (10d8) cold damage if it fails the saving throw, or half that damage if it succeeds.

\textbf{Ecology} \\
Environment: Cold Mountains \\
Organization: Solitary \\
\textbf{Treasure}: Triple \\
\textbf{Description} \\
Although many consider him the weakest and most beastly of the chromatic dragons, the white dragon makes up for his lack of cunning with sheer ferocity. White dragons live on remote, icy mountain peaks, and in the Arctic lowlands, making their lair in glistening caverns filled with ice and snow. They prefer their meals to be completely frozen. \\
\textbf{Spells} \index{White Dragon Spells} \\
This Dragon's favorite spells are: \\
- Fire Shield \\
- Ice Storm \\
- Sleet storm


\medskip\index[Mostruario]{Baby White Dragon} \textbf{Baby White Dragon}

\textit{Medium dragon, chaotic evil}

\textbf{STRENGTH} +2

\textbf{DEXTERITY} +0

\textbf{CONSTITUTION} +2

\textbf{INTELLIGENCE} -3

\textbf{WISDOM} +0

\textbf{CHARISMA} +0

\textbf{Initiative} +0 - \textbf{Defense} 17

\textbf{Hit Points} 32 (5d8 + 10)

\textbf{Movement} 9 m, swim 9 m, dig 5 meters, fly 18 m

\textbf{Saving Throws} Fortitude +2, Reflex +1, Will +1

\textbf{Skills} Move Silently / Hide +2, Awareness +4

\textbf{Immunity to Damage} cold

\textbf{Senses} darkvision 18 m, blind sight 3 m

\textbf{Languages} Draconic

\textbf{Challenge} 2 (450 PX)

\textbf{Actions}

\textit{\textbf{Bite.} Melee Weapon Attack}: +5 to hit, 3m range, one target.

\textit{Strikes:} 15 (2d10 + 4) piercing damage plus 4 (1d8) cold damage.

\textit{\textbf{Frost Breath (Cooldown 5-6).}} The dragon exhales a blast of ice into a 5 meter cone. Each creature in that area must make a DC 12 Fortitude saving throw and take 22 (5d8) cold damage if it fails the saving throw, or half that damage if it succeeds.

\textbf{Ecology} \\
Environment: Cold Mountains \\
Organization: Solitary \\
\textbf{Treasure}: Triple \\
\textbf{Description} \\
Although many consider him the weakest and most beastly of the chromatic dragons, the white dragon makes up for his lack of cunning with sheer ferocity. White dragons live on remote, icy mountain peaks, and in the Arctic lowlands, making their lair in glistening caverns filled with ice and snow. They prefer their meals to be completely frozen.

\medskip\index[Mostruario]{Ancient Blue Dragon} \textbf{Ancient Blue Dragon}

\textit{Gargantuan dragon, lawful evil}

\textbf{STRENGTH} +9

\textbf{DEXTERITY} +0

\textbf{CONSTITUTION} +8

\textbf{INTELLIGENCE} +4

\textbf{WISDOM} +3

\textbf{CHARISMA} +5

\textbf{Initiative} +4 - \textbf{Defense} 34

\textbf{Hit Points} 481 (26d20 + 208)

\textbf{Movement} 12 m, Burrow 12 m, flight 24 m

\textbf{Saving Throws} Fortitude +21, Reflex +13, Will +19

\textbf{Skills} Move Silently / Hide +7, Awareness +17

\textbf{Immunity to Damage} lightning, weapons +1

\textbf{Senses} darkvision 36 m, blind sight 18 m

\textbf{Languages} Common, Draconic

\textbf{Challenge} 23 (50000 PX)

\textit{\textbf{Legendary Resistance (3 / Day).}} If the dragon fails a saving throw, it may choose to succeed instead.

\textbf{Actions}

\textit{\textbf{Multiattack.}} The dragon can use its Frightening Presence. Then make three attacks: one with the bite and two with the claws.

\textit{\textbf{Claw.} Melee Weapon Attack}: +16 to hit,
range 3 m, one target.

\textit{Strikes:} 16 (2d6 + 9) slashing damage, 3 bleed damage (to a maximum of 10).

\textit{\textbf{Tail.} Melee Weapon Attack}: +30 hit, 6m range, one target.

\textit{Hits:} 18 (2d8 + 9) hit damage.

\textit{\textbf{Bite.} Melee Weapon Attack}: +30 to hit, 5m range, one target.

\textit{Strikes:} 20 (2d10 + 9) piercing damage plus 11 (2d10) lightning damage.

\textit{\textbf{Dreadful Presence.}} Any creature the dragon chooses, within 36 meters of it and aware of its presence, must succeed on a DC 20 Will saving throw or be startled for 1 minute. A creature can re-roll the saving throw at the end of each of its rounds, ending the effect if successful. If the creature's saving throw is successful or the effect ends, the creature is immune to the dragon's dreadful presence for the next 24 hours.

\textit{\textbf{Lightning Breath (Recharge 5-6).}} The dragon exhales lightning bolts in a line 36 meters long and 3 meters wide. Each creature on that line must make a DC 23 Reflex saving throw and take 88 (16d10) lightning damage on a failed save, or half that damage on a successful one.

\textbf{Additional Actions}

The dragon can perform 3 additional Actions, chosen from the following options. He can only use one legendary option at a time, and only at the end of another creature's turn. The dragon recovers any additional Actions spent at the start of their round.

\textbf{Wing Attack (Costs 2 Actions).} The dragon flaps its wings. Each creature within 5 yards of the dragon must succeed on a DC 24 Reflex saving throw or take 16 (2d6 + 9) hit damage and be thrown prone. The dragon can then fly up to half of its flight movement.

\textbf{Tail Attack.} The dragon makes a tail attack.

\textbf{Spot.} The dragon makes a Wisdom (Awareness) check.

\textbf{Spot.} The dragon makes a Wisdom (Awareness) check. \\
\textbf{Ecology} \\
Environment: Mountain peaks \\
Organization: Solitary \\
\textbf{Treasure}: Triple \\
\textbf{Description} \\
Blue dragons are intriguingly consumed and obsessively ordered. In combat, blue dragons prefer to take enemies by surprise if possible, and don't hesitate to retreat if things go wrong. They prefer to make their lair close to those they control, sometimes even within the confines of a city. \\
\textbf{Spells} \index{Blue Dragon Spells} \\
This Dragon's favorite spells are: \\
- Chain of lightning \\
- Force Cage \\
- Teleportation \\
- Ethereal form


\medskip\index[Mostruario]{Adult Blue Dragon} \textbf{Adult Blue Dragon}

\textit{Huge dragon, lawful evil}

\textbf{STRENGTH} +7

\textbf{DEXTERITY} +0

\textbf{CONSTITUTION} +6

\textbf{INTELLIGENCE} +3

\textbf{WISDOM} +2

\textbf{CHARISMA} +4

\textbf{Initiative} +3 - \textbf{Defense} 27

\textbf{Hit Points} 225 (18d12 + 108)

\textbf{Movement} 12 m, Burrow 12 m, flight 24 m

\textbf{Saving Throws} Fortitude +15, Reflex +10, Will +13

\textbf{Skills} Move Silently / Hide +5, Awareness +12

\textbf{Immunity to Damage} lightning

\textbf{Senses} darkvision 36 m, blind sight 18 m

\textbf{Languages} Common, Draconic

\textbf{Challenge} 16 (15000 PX)

\textit{\textbf{Legendary Resistance (3 / Day).}} If the dragon fails a saving throw, it may choose to succeed instead.

\textbf{Actions}

\textit{\textbf{Multiattack.}} The dragon can use its Frightening Presence. Then make three attacks: one with the bite and two with the claws.

\textit{\textbf{Claw.} Melee Weapon Attack}: +26 to hit, 1m range, one target.

\textit{Strikes:} 14 (2d6 + 7) slashing damage, 1 bleed damage.

\textit{\textbf{Tail.} Melee Weapon Attack}: +26 to hit, range 5 meters, one target.

\textit{Strikes:} 16 (2d8 + 7) hit damage.

\textit{\textbf{Bite.} Melee Weapon Attack}: +26 to hit, 3m range, one target.

\textit{Strikes:} 18 (2d10 + 7) piercing damage plus 5 (1d10) lightning damage.

\textit{\textbf{Dreadful Presence.}} Any creature the dragon chooses, within 36 meters of it and aware of its presence, must succeed on a DC 17 Will saving throw or be startled for 1 minute. A creature can re-roll the saving throw at the end of each of its rounds, ending the effect if successful. If the creature's saving throw is successful or the effect ends, the creature is immune to the dragon's dreadful presence for the next 24 hours.

\textit{\textbf{Lightning Breath (Cooldown 5-6).}} The dragon exhales lightning bolts in a line 27 meters long and 1 meter wide. Each creature on that line must make a DC 19 Reflex saving throw and take 66 (12d10) lightning damage on a failed save, or half that damage on a successful one.

\textbf{Additional Actions}

The dragon can perform 3 additional Actions, chosen from the following options. He can only use one legendary option at a time, and only at the end of another creature's turn. The dragon recovers any additional Actions spent at the start of their round.

\textbf{Wing Attack (Costs 2 Actions).} The dragon flaps its wings. Each creature within 10 feet of the dragon must succeed on a DC 20 Reflex saving throw or take 14 (2d6 + 7) hit damage and be thrown prone. The dragon can then fly up to half of its flight movement.

\textbf{Tail Attack.} The dragon makes a tail attack.

\textbf{Spot.} The dragon makes a Wisdom (Awareness) check.

\textbf{Ecology} \\
Environment: Mountain peaks \\
Organization: Solitary \\
\textbf{Treasure}: Triple \\
\textbf{Description} \\
Blue dragons are intriguingly consumed and obsessively ordered. In combat, blue dragons prefer to take enemies by surprise if possible, and don't hesitate to retreat if things go wrong. They prefer to make their lair close to those they control, sometimes even within the confines of a city. \\
\textbf{Spells} \index{Blue Dragon Spells} \\
This Dragon's favorite spells are: \\
- Chain of lightning \\
- Force Cage \\
- Teleportation \\
- Ethereal form


\medskip\index[Mostruario]{Young Blue Dragon} \textbf{Young Blue Dragon}

\textit{Huge dragon, lawful evil}

\textbf{STRENGTH} +5

\textbf{DEXTERITY} +0

\textbf{CONSTITUTION} +4

\textbf{INTELLIGENCE} +2

\textbf{WISDOM} +1

\textbf{CHARISMA} +3

\textbf{Initiative} +2 - \textbf{Defense} 23

\textbf{Hit Points} 152 (16d10 + 64)

\textbf{Movement} 12 m, Burrow 12 m, flight 24 m

\textbf{Saving Throws} Fortitude +10, Reflex +8, Will +8

\textbf{Skills} Move Silently / Hide +4, Awareness +9

\textbf{Immunity to Damage} lightning

\textbf{Senses} darkvision 36 m, blind sight 9 m

\textbf{Languages} Common, Draconic

\textbf{Challenge} 9 (5000 PX)

\textbf{Actions}

\textit{\textbf{Multiattack.}} The dragon can make three attacks: one with its bite and two with its claws.

\textit{\textbf{Claw.} Melee Weapon Attack}: +13 to hit, 1m range, one target.

\textit{Strikes:} 12 (2d6 + 5) slashing damage, 1 bleed damage.

\textit{\textbf{Bite.} Melee Weapon Attack}: +13 to hit, 3m range, one target.

\textit{Strikes:} 16 (2d10 + 5) piercing damage plus 5 (1d10) lightning damage.

\textit{\textbf{Lightning Breath (Cooldown 5-6).}} The dragon exhales lightning bolts in a line 18 meters long and 1 meter wide. Each creature on that line must make a DC 16 Reflex saving throw and take 55 (10d10) lightning damage on a failed save, or half that damage if it succeeds.

\textbf{Ecology} \\
Environment: Mountain peaks \\
Organization: Solitary \\
\textbf{Treasure}: Triple \\
\textbf{Description} \\
Blue dragons are intriguingly consumed and obsessively ordered. In combat, blue dragons prefer to take enemies by surprise if possible, and don't hesitate to retreat if things go wrong. They prefer to make their lair close to those they control, sometimes even within the confines of a city. \\
\textbf{Spells} \index{Blue Dragon Spells} \\
This Dragon's favorite spells are: \\
- Chain of lightning \\
- Force Cage \\
- Teleportation \\
- Ethereal form


\medskip\index[Mostruario]{Baby Blue Dragon} \textbf{Baby Blue Dragon}

\textit{Huge dragon, lawful evil}

\textbf{STRENGTH} +3

\textbf{DEXTERITY} +0

\textbf{CONSTITUTION} +2

\textbf{INTELLIGENCE} +1

\textbf{WISDOM} +0

\textbf{CHARISMA} +2

\textbf{Initiative} +1 - \textbf{Defense} 19

\textbf{Hit Points} 52 (8d8 + 16)

\textbf{Movement} 9 m, Burrow 5 meters, flight 18 m

\textbf{Saving Throws} Fortitude +4, Reflex +1, Will +1

\textbf{Skills} Move Silently / Hide +2, Awareness +4

\textbf{Immunity to Damage} lightning

\textbf{Senses} darkvision 18 m, blind sight 3 m

\textbf{Languages} Draconic

\textbf{Challenge} 3 (700 PX)

\textbf{Actions}

\textit{\textbf{Bite.} Melee Weapon Attack}: +5 to hit, 1m range, one target.

\textit{Strikes:} 8 (1d10 + 3) piercing damage plus 3 (1d6) lightning damage.

\textit{\textbf{Lightning Breath (Cooldown 5-6).}} The dragon exhales lightning bolts in a line 9 meters long and 1 meter wide. Each creature on that line must make a DC 12 Reflex saving throw and take 22 (4d10) lightning damage on a failed save, or half that damage on a successful one.

\textbf{Ecology} \\
Environment: Mountain peaks \\
Organization: Solitary \\
\textbf{Treasure}: Triple \\
\textbf{Description} \\
Blue dragons are intriguingly consumed and obsessively ordered. In combat, blue dragons prefer to take enemies by surprise if possible, and don't hesitate to retreat if things go wrong. They prefer to make their lair near those they control, sometimes even within the confines of a city.

\medskip \textbf{Ancient Yellow Dragon} \index[Mostruario]{Ancient Yellow Dragon} \\
\textit{Gargantuan dragon, neutral evil} \\
\textbf{Strength}: +10 \\
\textbf{Dexterity}: +1 \\
\textbf{Constitution}: +8 \\
\textbf{Intelligence}: +3 \\
\textbf{Wisdom}: +2 \\
\textbf{Charisma}: +4 \\
\textbf{Defense}: 27 (natural armor) - \textbf{Initiative}: +4 \\
\textbf{Hit Points}: 481 (26d20 + 208) \\
\textbf{Movement}: 12 m, Burrow 24 m, climb 24, flight 12 m \\
\textbf{Saving Throws}: Fortitude +21, Reflex +13, Will +19 \\
\textbf{Skills}: Crime +7, Awareness +17 \\
\textbf{Damage Immunity}: Lightning Bolt \\
\textbf{Senses}: Darkvision 36 m, blind sight 18 m \\
\textbf{Languages} Common, Draconic \\
\textbf{Challenge}: 23 (50000 PX) \smallskip \\
\textit{\textbf{Legendary Resistance (3 / Day).}} If the dragon fails a saving throw, it may choose to succeed instead. \\
\smallskip \textbf{Actions} \\
\textit{\textbf{Multiattack.}} The dragon can use its Frightening Presence. Then make three attacks: one with the bite and two with the claws. \\
\textit{\textbf{Claw.} Melee Weapon Attack}: +30 hit, 3m range, one target. \\
\textit{Strikes:} 16 (2d6 + 9) slashing damage, 3 bleed damage (to a maximum of 10). \\
\textit{\textbf{Tail.} Melee Weapon Attack}: +30 hit, 6m range, one target. \\
\textit{Strikes:} 18 (2d8 + 9) hit damage. \\
\textit{\textbf{Bite.} Melee Weapon Attack}: +30 hit, 5m range, one target. \\
\textit{Strikes:} 20 (2d10 + 9) piercing damage plus 11 (2d10) lightning damage. \\
\textit{\textbf{Dreadful Presence.}} Any creature the dragon chooses, within 36 meters of it and aware of its presence, must succeed on a DC 25 Will save or be frightened for 1 minute. A creature can re-roll the saving throw at the end of each of its rounds, ending the effect if successful. If the creature's saving throw is successful or the effect ends, the creature is immune to the dragon's fearful presence for the next 24 hours. \\
\textit{\textbf{Fire Breath (Cooldown 5-6).}} The dragon exhales hot air in a line 36 meters long and 3 meters wide. Each creature on that line must make a DC 30 Reflex saving throw and take 88 (16d10) fire damage on a failed save, or half that damage on a successful one. \\
\textbf{Additional Actions} \\
The dragon can perform 3 additional actions, chosen from the following options. He can only use one Additional Action at a time, and only at the end of another creature's round. The dragon recovers any Additional Actions spent at the start of their round. \\
\textbf{Wing Attack (Costs 2 Actions).} The dragon flaps its wings. Each creature within 5 yards of the dragon must succeed on a DC 31 Reflex saving throw or take 16 (2d6 + 9) hit damage and be thrown prone. The dragon can then fly up to half its flight speed. \\
\textbf{Tail Attack.} The dragon makes a tail attack. \\
\textbf{Spot.} The dragon makes a Wisdom (Awareness) check. \\
\textbf{Ecology} \\
Environment: Warm Deserts \\
Organization: Solitary \\
\textbf{Treasure}: Triple \\
\textbf{Description} \\
Yellow dragons are ravenous predators and unruly fighters. They love hunting and killing, they are consumed predators that instinctively attack anyone in their territory. The desert is their terrain where they dig rough traps to capture their poor victims.
A yellow dragon's breath is a heat wave (fire damage).
\\
\textbf{Spells} \index{Yellow Dragon Spells} \\
This Dragon's favorite spells are: \\
- Heating Metal \\
- Fireball \\
- Shield of Fire


\medskip\index[Mostruario]{Ancient Black Dragon} \textbf{Ancient Black Dragon}

\textit{Gargantuan dragon, chaotic evil}

\textbf{STRENGTH} +8

\textbf{DEXTERITY} +2

\textbf{CONSTITUTION} +7

\textbf{INTELLIGENCE} +3

\textbf{WISDOM} +2

\textbf{CHARISMA} +4

\textbf{Initiative} +3 - \textbf{Defense} 33

\textbf{Hit Points} 367 (21d20 + 147)

\textbf{Movement} 12m, climb 12m, flight 24m

\textbf{Saving Throws} Fortitude +20, Reflex +13, Will +18

\textbf{Skills} Move Silently / Hide +9, Awareness +16

\textbf{Immunity to Damage} acid, weapons +1

\textbf{Senses} darkvision 36 m, blind sight 18 m

\textbf{Languages} Common, Draconic

\textbf{Challenge} 21 (33000 PX)

\textit{\textbf{Amphibian.}} The dragon can breathe air and water.

\textit{\textbf{Legendary Resistance (3 / Day).}} If the dragon fails a saving throw, it may choose to succeed instead.

\textbf{Actions}

\textit{\textbf{Multiattack.}} The dragon can use its Frightening Presence. Then make three attacks: one with the bite and two with the claws.

\textit{\textbf{Claw.} Melee Weapon Attack}: +30 to hit, 3m range, one target.

\textit{Strikes:} 15 (2d6 + 8) slashing damage, 3 bleed damage (to a maximum of 10).

\textit{\textbf{Tail.} Melee Weapon Attack}: +30 hit, 6m range, one target.

\textit{Strikes:} 17 (2d8 + 8) hit damage.

\textit{\textbf{Bite.} Melee Weapon Attack}: +30 to hit, 5m range, one target.

\textit{Strikes:} 19 (2d10 + 8) piercing damage plus 9 (4d6) acid damage.

\textit{\textbf{Dreadful Presence.}} Any creature the dragon chooses, within 36 meters of it and aware of its presence, must succeed on a DC 19 Will saving throw or be frightened for 1 minute. A creature can re-roll the saving throw at the end of each of its rounds, ending the effect if successful. If the creature's saving throw is successful or the effect ends, the creature is immune to the dragon's dreadful presence for the next 24 hours.

\textit{\textbf{Acid Breath (Reload 5-6).}} The dragon exhales acid in a 27 meter line 3 meters wide. Each creature in that area must make a DC 22 Reflex saving throw and take 67 (15d8) acid damage if it fails the saving throw, or half that damage if it succeeds.

\textbf{Additional Actions}

The dragon can perform 3 additional Actions, chosen from the following options. He can only use one legendary option at a time, and only at the end of another creature's turn. The dragon recovers any additional Actions spent at the start of their round.

\textbf{Wing Attack (Costs 2 Actions).} The dragon flaps its wings. Each creature within 5 yards of the dragon must succeed on a DC 23 Reflex saving throw or take 15 (2d6 + 8) hit damage and be thrown prone. The dragon can then fly up to half of its flight movement.

\textbf{Tail Attack.} The dragon makes a tail attack.

\textbf{Spot.} The dragon makes a Wisdom (Awareness) check.

\textbf{Ecology} \\
Environment: Warm Swamps \\
Organization: Solitary \\
\textbf{Treasure}: Triple \\
\textbf{Description} \\
Lords of the darkest swamps and marshes, black dragons are the undisputed masters of their territory, which they dominate with cruelty and instilling terror in those who live nearby. Black dragons settle in the most remote areas of swamps, especially in caves at the bottom of fetid and dark pools. Inside, they amass their filthy treasures and sleep among roots and mud. Black dragons love slightly rotten food and often leave a meal to rot in a puddle for days before consuming it. Black dragons prefer treasures that do not decompose or degrade, hoarding treasures of coins, precious stones, jewelry, and other stone or metal objects. \\
\textbf{Spells} \index{Black Dragon Spells} \\
This Dragon's favorite spells are: \\
- Finger of death \\
- Disintegration \\
- Block Monsters


\medskip\index[Mostruario]{Adult Black Dragon} \textbf{Adult Black Dragon}

\textit{Huge dragon, chaotic evil}

\textbf{STRENGTH} +6

\textbf{DEXTERITY} +2

\textbf{CONSTITUTION} +5

\textbf{INTELLIGENCE} +2

\textbf{WISDOM} +1

\textbf{CHARISMA} +3

\textbf{Initiative} +2 - \textbf{Defense} 28

\textbf{Hit Points} 195 (17d12 + 85)

\textbf{Movement} 12m, climb 12m, flight 24m

\textbf{Saving Throws} Fortitude +14, Reflex +10, Will +12

\textbf{Skills} Move Silently / Hide +7, Awareness +11

\textbf{Immunity to Damage} acid

\textbf{Senses} darkvision 36 m, blind sight 18 m

\textbf{Languages} Common, Draconic

\textbf{Challenge} 17 (18000 PX)

\textit{\textbf{Amphibian.}} The dragon can breathe air and water.

\textit{\textbf{Legendary Resistance (3 / Day).}} If the dragon fails a saving throw, it may choose to succeed instead.

\textbf{Actions}

\textit{\textbf{Multiattack.}} The dragon can use its Frightening Presence. Then make three attacks: one with the bite and two with the claws.

\textit{\textbf{Claw.} Melee Weapon Attack}: +26 to hit, 1m range, one target.

\textit{Strikes:} 13 (2d6 + 6) slashing damage, 1 bleed damage.

\textit{\textbf{Tail.} Melee Weapon Attack}: +26 to hit, range 5 meters, one target.

\textit{Strikes:} 15 (2d8 + 6) hit damage.

\textit{\textbf{Bite.} Melee Weapon Attack}: +26 to hit, 3m range, one target.

\textit{Strikes:} 17 (2d10 + 6) piercing damage plus 4 (1d8) acid damage.

\textit{\textbf{Frightening presence.}} Any creature the dragon chooses, within 36 meters of it and aware of its presence, must succeed on a DC 16 Will saving throw or be frightened for 1 minute. A creature can re-roll the saving throw at the end of each of its rounds, ending the effect if successful. If the creature's saving throw is successful or the effect ends, the creature is immune to the dragon's dreadful presence for the next 24 hours.

\textit{\textbf{Acid Breath (Refill 5-6).}} The dragon exhales acid in a line of 60 feet wide. Each creature in that area must make a DC 18 Reflex saving throw and take 54 (12d8) acid damage if it fails its saving throw, or half that damage if it fails.
succeeds.

\textbf{Additional Actions}

The dragon can perform 3 additional Actions, chosen from the following options. He can only use one legendary option at a time, and only at the end of another creature's turn. The dragon recovers any additional Actions spent at the start of their round.

\textbf{Wing Attack (Costs 2 Actions).} The dragon flaps its wings. Each creature within 10 feet of the dragon must succeed on a DC 19 Reflex saving throw or take 13 (2d6 + 6) hit damage and be thrown prone. The dragon can then fly up to half of its flight movement.

\textbf{Tail Attack.} The dragon makes a tail attack.

\textbf{Spot.} The dragon makes a Wisdom (Awareness) check.

\textbf{Ecology} \\
Environment: Warm Swamps \\
Organization: Solitary \\
\textbf{Treasure}: Triple \\
\textbf{Description} \\
Lords of the darkest swamps and marshes, black dragons are the undisputed masters of their territory, which they dominate with cruelty and instilling terror in those who live nearby. Black dragons settle in the most remote areas of swamps, especially in caves at the bottom of fetid and dark pools. Inside, they amass their filthy treasures and sleep among roots and mud. Black dragons love slightly rotten food and often leave a meal to rot in a puddle for days before consuming it. Black dragons prefer treasures that do not decompose or degrade, hoarding treasures of coins, precious stones, jewelry, and other stone or metal objects. \\
\textbf{Spells} \index{Black Dragon Spells} \\
This Dragon's favorite spells are: \\
- Finger of death \\
- Disintegration \\
- Block Monsters


\medskip\index[Mostruario]{Young Black Dragon} \textbf{Young Black Dragon}

\textit{Large dragon, chaotic evil}

\textbf{STRENGTH} +4

\textbf{DEXTERITY} +2

\textbf{CONSTITUTION} +3

\textbf{INTELLIGENCE} +1

\textbf{WISDOM} +0

\textbf{CHARISMA} +2

\textbf{Initiative} +2 - \textbf{Defense} 22

\textbf{Hit Points} 127 (15d10 + 45)

\textbf{Movement} 12m, climb 12m, flight 24m

\textbf{Saving Throws} Fortitude +9, Reflex +8, Will +7

\textbf{Skills} Move Silently / Hide +5, Awareness +6

\textbf{Immunity to Damage} acid

\textbf{Senses} darkvision 36 m, blind sight 9 m

\textbf{Languages} Common, Draconic

\textbf{Challenge} 7 (2.900 PX)

\textit{\textbf{Amphibian.}} The dragon can breathe air and water.

\textbf{Actions}

\textit{\textbf{Multiattack.}} The dragon can make three attacks: one with its bite and two with its claws.

\textit{\textbf{Claw.} Melee Weapon Attack}: +9 to hit, 1m range, one target.

\textit{Strikes:} 11 (2d6 + 4) slashing damage, 1 bleed damage.

\textit{\textbf{Bite.} Melee Weapon Attack}: +9 to hit, 3m range, one target.

\textit{Strikes:} 11 (2d10 + 4) piercing damage plus 4 (1d8) acid damage.

\textit{\textbf{Acid Breath (Refill 5-6).}} The dragon exhales acid in a 30-foot-wide 1-meter line. Each creature in that area must make a DC 14 Reflex saving throw and take 49 (11d8) acid damage on a failed save, or half that damage on a successful one.

\textbf{Ecology} \\
Environment: Warm Swamps \\
Organization: Solitary \\
\textbf{Treasure}: Triple \\
\textbf{Description} \\
Lords of the darkest swamps and marshes, black dragons are the undisputed masters of their territory, which they dominate with cruelty and instilling terror in those who live nearby. Black dragons settle in the most remote areas of swamps, especially in caves at the bottom of fetid and dark pools. Inside, they amass their filthy treasures and sleep among roots and mud. Black dragons love slightly rotten food and often leave a meal to rot in a puddle for days before consuming it. Black dragons prefer treasures that do not decompose or degrade, hoarding treasures of coins, precious stones, jewelry, and other stone or metal objects. \\
\textbf{Spells} \index{Black Dragon Spells} \\
This Dragon's favorite spells are: \\
- Finger of death \\
- Disintegration \\
- Block Monsters


\medskip\index[Mostruario]{Black Dragon Pup} \textbf{Black Dragon Pup}

\textit{Medium dragon, chaotic evil}

\textbf{STRENGTH} +2

\textbf{DEXTERITY} +2

\textbf{CONSTITUTION} +1

\textbf{INTELLIGENCE} +0

\textbf{WISDOM} +0

\textbf{CHARISMA} +1

\textbf{Initiative} +2 - \textbf{Defense} 18

\textbf{Hit Points} 33 (6d8 + 6)

\textbf{Movement} 9m, climb 9m, flight 18m

\textbf{Saving Throws} Fortitude +2, Reflex +2, Will +0

\textbf{Skills} Move Silently / Hide +4, Awareness +4

\textbf{Immunity to Damage} acid

\textbf{Senses} darkvision 18 m, blind sight 3 m

\textbf{Languages} Draconic

\textbf{Challenge} 2 (450 PX)

\textit{\textbf{Amphibian.}} The dragon can breathe air and water.

\textbf{Actions}

\textit{\textbf{Bite.} Melee Weapon Attack}: +4 to hit, 1m range, one target.

\textit{Strikes:} 7 (1d10 + 2) piercing damage plus 2 (1d4) acid damage.

\textit{\textbf{Acid Breath (Refill 5-6).}} The dragon exhales acid in a 5 meter line 1 meter wide. Each creature in that area must make a DC 11 Reflex saving throw and take 22 (5d8) acid damage on a failed save, or half that damage on a successful one.

\textbf{Ecology} \\
Environment: Warm Swamps \\
Organization: Solitary \\
\textbf{Treasure}: Triple \\
\textbf{Description} \\
Lords of the darkest swamps and marshes, black dragons are the undisputed masters of their territory, which they dominate with cruelty and instilling terror in those who live nearby. Black dragons settle in the most remote areas of swamps, especially in caves at the bottom of fetid and dark pools. Inside, they amass their filthy treasures and sleep among roots and mud. Black dragons love slightly rotten food and often leave a meal to rot in a puddle for days before consuming it. Black dragons prefer treasures that don't decompose or degrade, hoarding treasures of coins, precious stones, jewelry, and other stone or metal objects.

\medskip\index[Mostruario]{Ancient Red Dragon} \textbf{Ancient Red Dragon}

\textit{Gargantuan dragon, chaotic evil}

\textbf{STRENGTH} +10

\textbf{DEXTERITY} +0

\textbf{CONSTITUTION} +9

\textbf{INTELLIGENCE} +4

\textbf{WISDOM} +2

\textbf{CHARISMA} +6

\textbf{Initiative} +4 - \textbf{Defense} 34

\textbf{Hit Points} 546 (28d20 + 252)

\textbf{Movement} 12m, climb 12m, flight 24m

\textbf{Saving Throws} Fortitude +22, Reflex +13, Will +21

\textbf{Skills} Move Silently / Hide +7, Awareness +16

\textbf{Immunity to Damage} fire, weapons +1

\textbf{Senses} darkvision 36 m, blind sight 18 m

\textbf{Languages} Common, Draconic

\textbf{Challenge} 24 (62000 PX)

\textit{\textbf{Legendary Resistance (3 / Day).}} If the dragon fails a saving throw, it may choose to succeed instead.

\textbf{Actions}

\textit{\textbf{Multiattack.}} The dragon can use its Frightening Presence and then make three attacks: one with its bite and two with its claws.

\textit{\textbf{Claw.} Melee Weapon Attack}: +30 to hit, 3m range, one target.

\textit{Strikes:} 17 (2d6 + 10) slashing damage, 3 bleed damage (to a maximum of 10).

\textit{\textbf{Tail.} Melee Weapon Attack}: +30 to hit, 6m range, one target.

\textit{Strikes:} 19 (2d8 + 10) hit damage.

\textit{\textbf{Bite.} Melee Weapon Attack}: +30 to hit, 5m range, one target.

\textit{Strikes:} 21 (2d10 + 10) piercing damage plus 14 (4d6) fire damage.

\textit{\textbf{Frightening presence.}} Any creature the dragon chooses, within 36 meters of it and aware of its presence, must succeed on a DC 21 Will saving throw or be frightened for 1 minute. A creature can re-roll the saving throw at the end of each of its rounds, ending the effect if successful. If the creature's saving throw is successful or the effect ends, the creature is immune to the dragon's dreadful presence for the next 24 hours.

\textit{\textbf{Fiery Breath (Cooldown 5-6).}} The dragon exhales fire into a 27 meter cone. Each creature in that area must make a DC 24 Reflex saving throw and take 91 (26d6) fire damage on a failed save, or half that damage on a successful one.

\textbf{Additional Actions}

The dragon can perform 3 additional Actions, chosen from the following options. He can only use one legendary option at a time, and only at the end of another creature's turn. The dragon recovers any additional Actions spent at the start of their round.

\textbf{Wing Attack (Costs 2 Actions).} The dragon flaps its wings. Each creature within 5 yards of the dragon must succeed on a DC 25 Reflex saving throw or take 17 (2d6 + 10) hit damage and be thrown prone. The dragon can then fly up to half of its flight movement.

\textbf{Tail Attack.} The dragon makes a tail attack.

\textbf{Spot.} The dragon makes a Wisdom (Awareness) check.

\textbf{Ecology} \\
Environment: Warm mountains \\
Organization: Solitary \\
\textbf{Treasure}: Triple \\
\textbf{Description} \\
Few creatures are more cruel and terrible than the mighty red dragon. Ruler of the chromatics, the terrible red dragon brings ruin and death to the lands threatened by his presence. \\
\textbf{Spells} \index{Red Dragon Spells} \\
This Dragon's favorite spells are: \\
- Fireball \\
- Incendiary Cloud \\
- Wall of fire


\medskip\index[Mostruario]{Adult Red Dragon} \textbf{Adult Red Dragon}

\textit{Huge dragon, chaotic evil}

\textbf{STRENGTH} +8

\textbf{DEXTERITY} +0

\textbf{CONSTITUTION} +7

\textbf{INTELLIGENCE} +3

\textbf{WISDOM} +1

\textbf{CHARISMA} +5

\textbf{Initiative} +3 - \textbf{Defense} 28

\textbf{Hit Points} 256 (19d12 + 133)

\textbf{Movement} 12m, climb 12m, flight 24m

\textbf{Saving Throws} Fortitude +16, Reflex +10, Will +15

\textbf{Skills} Move Silently / Hide +6, Awareness +13

\textbf{Immunity to Damage} fire

\textbf{Senses} darkvision 36 m, blind sight 18 m

\textbf{Languages} Common, Draconic

\textbf{Challenge} 17 (18000 PX)

\textit{\textbf{Legendary Resistance (3 / Day).}} If the dragon fails a saving throw, it may choose to succeed instead.

\textbf{Actions}

\textit{\textbf{Multiattack.}} The dragon can use its Frightening Presence and then make three attacks: one with its bite and two with its claws.

\textit{\textbf{Claw.} Melee Weapon Attack}: +28 to hit, 1m range, one target.

\textit{Strikes:} 15 (2d6 + 8) slashing damage, 1 bleed damage.

\textit{\textbf{Tail.} Melee Weapon Attack}: +28 to hit, 5m range, one target.

\textit{Strikes:} 17 (2d8 + 8) hit damage.

\textit{\textbf{Bite.} Melee Weapon Attack}: +28 to hit, 3m range, one target.

\textit{Strikes:} 19 (2d10 + 8) piercing damage plus 7 (2d6) damage from
fire.

\textit{\textbf{Frightening Presence.}} Any creature the dragon chooses, within 36 meters of it and aware of its presence, must succeed on a DC 19 Will saving throw or be startled for 1 minute. A creature can re-roll the saving throw at the end of each of its rounds, ending the effect if successful. If the creature's saving throw is successful or the effect ends, the creature is immune to the dragon's dreadful presence for the next 24 hours.

\textit{\textbf{Fiery Breath (Cooldown 5-6).}} The dragon exhales fire into a 60-foot cone. Each creature in that area must make a DC 21 Reflex saving throw and take 63 (18d6) fire damage on a failed save, or half that damage on a successful one.

\textbf{Additional Actions}

The dragon can perform 3 additional Actions, chosen from the following options. He can only use one legendary option at a time, and only at the end of another creature's turn. The dragon recovers any additional Actions spent at the start of their round.

\textbf{Wing Attack (Costs 2 Actions).} The dragon flaps its wings. Each creature within 10 feet of the dragon must succeed on a DC 22 Reflex saving throw or take 15 (2d6 + 8) hit damage and be thrown prone. The dragon can then fly up to half of its flight movement.

\textbf{Tail Attack.} The dragon makes a tail attack.

\textbf{Spot.} The dragon makes a Wisdom (Awareness) check.

\textbf{Ecology} \\
Environment: Hot mountains \\
Organization: Solitary \\
\textbf{Treasure}: Triple \\
\textbf{Description} \\
Few creatures are more cruel and terrible than the mighty red dragon. Ruler of the chromatics, the terrible red dragon brings ruin and death to the lands threatened by his presence. \\
\textbf{Spells} \index{Red Dragon Spells} \\
This Dragon's favorite spells are: \\
- Fireball \\
- Incendiary Cloud \\
- Wall of fire


\medskip\index[Mostruario]{Young Red Dragon} \textbf{Young Red Dragon}

\textit{Large dragon, chaotic evil}

\textbf{STRENGTH} +6

\textbf{DEXTERITY} +0

\textbf{CONSTITUTION} +5

\textbf{INTELLIGENCE} +2

\textbf{WISDOM} +0

\textbf{CHARISMA} +4

\textbf{Initiative} +2 - \textbf{Defense} 23

\textbf{Hit Points} 178 (17d10 + 85)

\textbf{Movement} 12m, climb 12m, flight 24m

\textbf{Saving Throws} Fortitude +11, Reflex +8, Will +10

\textbf{Skills} Move Silently / Hide +4, Awareness +8

\textbf{Immunity to Damage} fire

\textbf{Senses} darkvision 36 m, blind sight 9 m

\textbf{Languages} Common, Draconic

\textbf{Challenge} 10 (5.900 PX)

\textbf{Actions}

\textit{\textbf{Multiattack.}} The dragon can make three attacks: one with its bite and two with its claws.

\textit{\textbf{Claw.} Melee Weapon Attack}: +16 to hit, 1m range, one target.

\textit{Strikes:} 13 (2d6 + 6) slashing damage, 1 bleed damage.

\textit{\textbf{Bite.} Melee Weapon Attack}: +16 to hit, 3m range, one target.

\textit{Strikes:} 17 (2d10 + 6) piercing damage plus 3 (1d6) fire damage.

\textit{\textbf{Fiery Breath (Cooldown 5-6).}} The dragon exhales fire into a 30-foot cone. Each creature in that area must make a DC 17 Reflex saving throw and take 56 (16d6) fire damage on a failed save, or half that damage on a successful one.

\textbf{Ecology} \\
Environment: Warm mountains \\
Organization: Solitary \\
\textbf{Treasure}: Triple \\
\textbf{Description} \\
Few creatures are more cruel and terrible than the mighty red dragon. Ruler of the chromatics, the terrible red dragon brings ruin and death to the lands threatened by his presence. \\
\textbf{Spells} \index{Red Dragon Spells} \\
This Dragon's favorite spells are: \\
- Fireball \\
- Incendiary Cloud \\
- Wall of fire


\medskip\index[Mostruario]{Red Dragon Pup} \textbf{Red Dragon Pup}

\textit{Medium dragon, chaotic evil}

\textbf{STRENGTH} +4

\textbf{DEXTERITY} +0

\textbf{CONSTITUTION} +3

\textbf{INTELLIGENCE} +1

\textbf{WISDOM} +0

\textbf{CHARISMA} +2

\textbf{Initiative} +1 - \textbf{Defense} 19

\textbf{Hit Points} 75 (10d8 + 30)

\textbf{Movement} 9m, climb 9m, flight 18m

\textbf{Saving Throws} Fortitude +4, Reflex +3, Will +1

\textbf{Skills} Move Silently / Hide +2, Awareness +4

\textbf{Immunity to Damage} fire

\textbf{Senses} darkvision 18 m, blind sight 3 m

\textbf{Languages} Draconic

\textbf{Challenge} 4 (1,100 PX)

\textbf{Actions}

\textit{\textbf{Bite.} Melee Weapon Attack}: +8 to hit, 1m range, one target.

\textit{Strikes:} 9 (1d10 + 4) piercing damage plus 3 (1d6) fire damage.

\textit{\textbf{Fiery Breath (Cooldown 5-6).}} The dragon exhales fire into a 5 meter cone. Each creature in that area must make a DC 13 Reflex saving throw and take 24 (7d6) fire damage on a failed save, or half that damage on a successful one.

\textbf{Ecology} \\
Environment: Hot mountains \\
Organization: Solitary \\
\textbf{Treasure}: Triple \\
\textbf{Description} \\
Few creatures are more cruel and terrible than the mighty red dragon. Ruler of the chromatics, the terrifying red dragon brings doom and death to lands threatened by his presence.


\medskip\index[Mostruario]{Ancient Green Dragon} \textbf{Ancient Green Dragon}

\textit{Gargantuan dragon, lawful evil}

\textbf{STRENGTH} +8

\textbf{DEXTERITY} +1

\textbf{CONSTITUTION} +7

\textbf{INTELLIGENCE} +5

\textbf{WISDOM} +3

\textbf{CHARISMA} +4

\textbf{Initiative} +5 - \textbf{Defense} 32

\textbf{Hit Points} 385 (22d20 + 154)

\textbf{Movement} 12m, swim 12m, fly 24m

\textbf{Saving Throws} Fortitude +20, Reflex +12, Will +20

\textbf{Skills} Move Silently / Hide +8, Deceive +11, Feel Emotions +10, Awareness + 15

\textbf{Immunity to Damage} poison, weapons +1

\textbf{Condition Immunity}
poisoned

\textbf{Senses} darkvision 36 m, blind sight 18 m

\textbf{Languages} Common, Draconic

\textbf{Challenge} 22 (41000 PX)

\textit{\textbf{Amphibian.}} The dragon can breathe air and water.

\textit{\textbf{Legendary Resistance (3 / Day).}} If the dragon fails a saving throw, it may choose to succeed instead.

\textbf{Actions}

\textit{\textbf{Multiattack.}} The dragon can use its Frightening Presence. Then make three attacks: one with the bite and two with the claws.

\textit{\textbf{Claw.} Melee Weapon Attack}: +30 to hit, 3m range, one target.

\textit{Strikes:} 15 (2d6 + 8) slashing damage, 3 bleed damage (to a maximum of 10).

\textit{\textbf{Tail.} Melee Weapon Attack}: +30 hit, 6m range, one target.

\textit{Strikes:} 17 (2d8 + 8) hit damage.

\textit{\textbf{Bite.} Melee Weapon Attack}: +30 to hit, 5m range, one target.

\textit{Strikes:} 19 (2d10 + 8) piercing damage plus 10 (3d6) poison damage.

\textit{\textbf{Frightening presence.}} Any creature the dragon chooses, within 36 meters of it and aware of its presence, must succeed on a DC 19 Will saving throw or be frightened for 1 minute. A creature can re-roll the saving throw at the end of each of its rounds, ending the effect if successful. If the creature's saving throw is successful or the effect ends, the creature is immune to the dragon's dreadful presence for the next 24 hours.

\textit{\textbf{Poisonous Breath (Cooldown 5-6).}} The dragon exhales poisonous gas in a 27 meter cone. Each creature in that area must make a DC 22 Fortitude save and take 77 (22d6) poison damage if it fails the saving throw, or half that damage if it succeeds.

\textbf{Additional Actions}

The dragon can perform 3 additional Actions, chosen from the following options. He can only use one legendary option at a time, and only at the end of another creature's turn. The dragon recovers any additional Actions spent at the start of their round.

\textbf{Wing Attack (Costs 2 Actions).} The dragon flaps its wings. Each creature within 5 yards of the dragon must succeed on a DC 23 Reflex saving throw or take 15 (2d6 + 8) hit damage and be thrown prone. The dragon can then fly up to half of its flight movement.

\textbf{Tail Attack.} The dragon makes a tail attack.

\textbf{Spot.} The dragon makes a Wisdom (Awareness) check.

\textbf{Ecology} \\
Environment: Temperate Forests \\
Organization: Solitary \\
\textbf{Treasure}: Triple \\
\textbf{Description} \\
Green dragons live in the world's ancient forests, wandering in search of prey under giant leafy canopies. Of all the chromatic dragons, the green dragons are perhaps the most easily negotiated diplomatically. \\
\textbf{Spells} \index{Green Dragon Spells} \\
This Dragon's favorite spells are: \\
- Cloud of Death \\
- Illusory terrain \\
- Remove poison


\medskip\index[Mostruario]{Adult Green Dragon} \textbf{Adult Green Dragon}

\textit{Huge dragon, lawful evil}

\textbf{STRENGTH} +6

\textbf{DEXTERITY} +1

\textbf{CONSTITUTION} +5

\textbf{INTELLIGENCE} +4

\textbf{WISDOM} +2

\textbf{CHARISMA} +3

\textbf{Initiative} +4 - \textbf{Defense} 27

\textbf{Hit Points} 207 (18d12 + 90)

\textbf{Movement} 12m, swim 12m, fly 24m

\textbf{Saving Throws} Fortitude +14, Reflex +9, Will +14

\textbf{Skills} Move Silently / Hide +6, Deceive +8, Feel Emotions +7, Awareness +12

\textbf{Immunity to Damage} poison

\textbf{Condition Immunity} poisoned

\textbf{Senses} darkvision 36 m, blind sight 18 m

\textbf{Languages} Common, Draconic

\textbf{Challenge} 15 (13000 PX)

\textit{\textbf{Amphibian.}} The dragon can breathe air and water.

\textit{\textbf{Legendary Resistance (3 / Day).}} If the dragon fails a saving throw, it may choose to succeed instead.

\textbf{Actions}

\textit{\textbf{Multiattack.}} The dragon can use its Frightening Presence. Then make three attacks: one with the bite and two with the claws.

\textit{\textbf{Claw.} Melee Weapon Attack}: +23 to hit, 1m range, one target.

\textit{Strikes:} 13 (2d6 + 6) slashing damage, 1 bleed damage.

\textit{\textbf{Tail.} Melee Weapon Attack}: +23 to hit, range 5 meters, one target.

\textit{Hits:} 15 (2d8 + 6) hit damage.

\textit{\textbf{Bite.} Melee Weapon Attack}: +23 to hit, 3m range, one target.

\textit{Strikes:} 17 (2d10 + 6) piercing damage plus 7 (2d6) poison damage.

\textit{\textbf{Frightening Presence.}} Any creature the dragon chooses, within 36 meters of it and aware of its presence, must succeed on a DC 16 Will saving throw or be startled for 1 minute. A creature can re-roll the saving throw at the end of each of its rounds, ending the effect if successful. If the creature's saving throw is successful or the effect ends, the creature is immune to the dragon's dreadful presence for the next 24 hours.

\textit{\textbf{Poisonous Breath (Cooldown 5-6).}} The dragon exhales poisonous gas in a 60-foot cone. Each creature in that area must make a DC 18 Fortitude saving throw and take 56 (16d6) poison damage if it fails the saving throw, or half that damage if it succeeds.

\textbf{Additional Actions}

The dragon can perform 3 additional Actions, chosen from the following options. He can only use one legendary option at a time, and only at the end of another creature's turn. The dragon recovers any additional Actions spent at the start of their round.

\textbf{Wing Attack (Costs 2 Actions).} The dragon flaps its wings. Each creature within 10 feet of the dragon must succeed on a DC 19 Reflex saving throw or take 13 (2d6 + 6) hit damage and be thrown prone. The dragon can then fly up to half of its flight movement.

\textbf{Tail Attack.} The dragon makes a tail attack.

\textbf{Spot.} The dragon makes a Wisdom (Awareness) check.

\textbf{Ecology} \\
Environment: Temperate Forests \\
Organization: Solitary \\
\textbf{Treasure}: Triple \\
\textbf{Description} \\
Green dragons live in the world's ancient forests, wandering in search of prey under giant leafy canopies. Of all the chromatic dragons, green dragons are perhaps the most easily negotiated diplomatically. \\
\textbf{Spells} \index{Green Dragon Spells} \\
This Dragon's favorite spells are: \\
- Cloud of Death \\
- Illusory terrain \\
- Remove poison


\medskip\index[Mostruario]{Young Green Dragon} \textbf{Young Green Dragon}

\textit{Large dragon, lawful evil}

\textbf{STRENGTH} +4

\textbf{DEXTERITY} +1

\textbf{CONSTITUTION} +3

\textbf{INTELLIGENCE} +3

\textbf{WISDOM} +1

\textbf{CHARISMA} +2

\textbf{Initiative} +3 - \textbf{Defense} 22

\textbf{Hit Points} 136 (16d10 + 48)

\textbf{Movement} 12m, swim 12m, fly 24m

\textbf{Saving Throws} Fortitude +9, Reflex +7, Will +9

\textbf{Skills} Move Silently / Hide +4, Deceive +5, Awareness +7

\textbf{Immunity to Damage} poison

\textbf{Condition Immunity} poisoned

\textbf{Senses} darkvision 36 m, blind sight 9 m
\textbf{Languages} Common, Draconic

\textbf{Challenge} 8 (3,900 PX)

\textit{\textbf{Amphibian.}} The dragon can breathe air and water.

\textbf{Actions}

\textit{\textbf{Multiattack.}} The dragon can make three attacks: one with its bite and two with its claws.

\textit{\textbf{Claw.} Melee Weapon Attack}: +11 to hit, 1m range, one target.

\textit{Strikes:} 11 (2d6 + 4) slashing damage, 1 bleed damage.

\textit{\textbf{Bite.} Melee Weapon Attack}: +11 to hit, 3m range, one target.

\textit{Strikes:} 15 (2d10 + 4) piercing damage plus 7 (2d6) poison damage.

\textit{\textbf{Poisonous Breath (Cooldown 5-6).}} The dragon exhales poison gas into a 30-foot cone. Each creature in that area must make a DC 14 Fortitude saving throw and take 42 (12d6) poison damage if it fails the saving throw, or half that damage if it succeeds.

\textbf{Ecology} \\
Environment: Temperate Forests \\
Organization: Solitary \\
\textbf{Treasure}: Triple \\
\textbf{Description} \\
Green dragons live in the world's ancient forests, wandering in search of prey under giant leafy canopies. Of all the chromatic dragons, green dragons are perhaps the most easily negotiated diplomatically. \\
\textbf{Spells} \index{Green Dragon Spells} \\
This Dragon's favorite spells are: \\
- Cloud of Death \\
- Illusory terrain \\
- Remove poison

\medskip\index[Mostruario]{Baby Green Dragon} \textbf{Baby Green Dragon}

\textit{Medium Dragon, Lawful Evil}

\textbf{STRENGTH} +2

\textbf{DEXTERITY} +1

\textbf{CONSTITUTION} +1

\textbf{INTELLIGENCE} +2

\textbf{WISDOM} +0

\textbf{CHARISMA} +1

\textbf{Initiative} +2 - \textbf{Defense} 18

\textbf{Hit Points} 38 (7d8 + 7)

\textbf{Movement} 9m, swim 9m, fly 18m

\textbf{Saving Throws} Fortitude +3, Reflex +1, Will +0

\textbf{Skills} Move Silently / Hide +3, Awareness +4

\textbf{Immunity to Damage} poison

\textbf{Condition Immunity} poisoned

\textbf{Senses} darkvision 18 m, blind sight 3 m

\textbf{Languages} Draconic

\textbf{Challenge} 2 (450 PX)

\textit{\textbf{Amphibian.}} The dragon can breathe air and water.

\textbf{Actions}

\textit{\textbf{Bite.} Melee Weapon Attack}: +4 to hit, 1m range, one target.

\textit{Strikes:} 7 (1d10 + 2) piercing damage plus 3 (1d6) poison damage.

\textit{\textbf{Poisonous Breath (Cooldown 5-6).}} The dragon exhales poison gas into a 5 meter cone. Each creature in that area must make a DC 11 Fortitude save and take 21 (6d6) poison damage if it fails the saving throw, or half that damage if it succeeds.

\textbf{Ecology} \\
Environment: Temperate Forests \\
Organization: Solitary \\
\textbf{Treasure}: Triple \\
\textbf{Description} \\
Green dragons live in the world's ancient forests, wandering in search of prey under giant leafy canopies. Of all the chromatic dragons, the green dragons are perhaps the most easily negotiated diplomatically.


\subsubsection{Metallic Dragons}

\medskip\index[Mostruario]{Ancient Silver Dragon} \textbf{Ancient Silver Dragon}

\textit{Gargantuan dragon, lawful good}

\textbf{STRENGTH} +10

\textbf{DEXTERITY} +0

\textbf{CONSTITUTION} +9

\textbf{INTELLIGENCE} +4

\textbf{WISDOM} +2

\textbf{CHARISMA} +6

\textbf{Initiative} +4 - \textbf{Defense} 34

\textbf{Hit Points} 487 (25d20 + 225)

\textbf{Movement} 12 m, flight 24 m

\textbf{Saving Throws} Fortitude +21, Reflex +15, Will +23

\textbf{Skills} Arcane +11, Move silently / Hide +7, Awareness +16, Story +11

\textbf{Damage Immunity} cold, weapons +1

\textbf{Senses} darkvision 36 m, blind sight 18 m

\textbf{Languages} Common, Draconic

\textbf{Challenge} 23 (50000 PX)

\textit{\textbf{Legendary Resistance (3 / Day).}} If the dragon fails a saving throw, it may choose to succeed instead.

\textbf{Actions}

\textit{\textbf{Multiattack.}} The dragon can use its Frightening Presence. Then make three attacks: one with the bite and two with the
claws.

\textit{\textbf{Claw.} Melee Weapon Attack}: +30 to hit, 3m range, one target.

\textit{Strikes:} 17 (2d6 + 10) slashing damage, 3 bleed damage (to a maximum of 10).

\textit{\textbf{Tail.} Melee Weapon Attack}: +30 hit, 6m range, one target.

\textit{Strikes:} 19 (2d8 + 10) hit damage.

\textit{\textbf{Bite.} Melee Weapon Attack}: +30 to hit, 5m range, one target.

\textit{Strikes:} 21 (2d10 + 10) piercing damage.

\textit{\textbf{Dreadful Presence.}} Any creature the dragon chooses, within 36 meters of it and aware of its presence, must succeed on a DC 21 Will saving throw or be startled for 1 minute. A creature can re-roll the saving throw at the end of each of its rounds, ending the effect if successful. If the creature's saving throw is successful or the effect ends, the creature is immune to the dragon's dreadful presence for the next 24 hours.

\textit{\textbf{Breath Weapon (Cooldown 5-6).}} The dragon uses one of the following breath weapons:

\textit{Frost Breath.} The dragon exhales an icy blast into a 27-meter cone. Each creature in the area must make a DC 24 Fortitude save, taking 67 (15d8) cold damage if it fails the save, or half that damage if it succeeds.

\textit{Paralyzing Breath.} The dragon exhales paralyzing gas into a 24 meter cone. Each creature in the area must succeed on a Fortitude save 24 or be paralyzed for 1 minute. A creature can re-roll the saving throw at the end of each of its rounds, ending the effect for itself on success.

\textit{\textbf{Shape Shift.}} The dragon can magically transform into a humanoid or beast whose challenge rating is equal to or less than its own, or revert to its true form. At death it returns to its true form.

Any equipment he is wearing or carrying is absorbed or carried in the new form (dragon's choice).

In the new form, the dragon retains its traits, hit points, hit dice, ability to speak, skill, legendary Fortitude, lair actions, and intelligence, wisdom, and charisma scores, in addition to this action. Its stats and abilities are otherwise replaced by those of the new form, except for the new form's additional Actions.

\textbf{Additional Actions}

The dragon can perform 3 additional Actions, chosen from the following options. He can only use one legendary option at a time, and only at the end of another creature's turn. The dragon recovers any additional Actions spent at the start of their round.

\textbf{Wing Attack (Costs 2 Actions).} The dragon flaps its wings. Each creature within 5 yards of the dragon must succeed on a DC 25 Reflex saving throw or take 17 (2d6 + 10) hit damage and be thrown prone. The dragon can then fly up to half its flight speed.

\textbf{Tail Attack.} The dragon makes a tail attack.

\textbf{Spot.} The dragon makes a Wisdom (Awareness) check.

\textbf{Ecology} \\
Environment: Temperate Mountains \\
Organization: Solitary \\
\textbf{Treasure}: Triple \\
\textbf{Description} \\
Of all dragons, the silver dragons are the bravest, adhering to a chivalrous code that requires them to help the weak, defeat evil and behave honorably. \\
\textbf{Spells} \index{Silver Dragon Spells} \\
This Dragon's favorite spells are: \\
- Cold cone \\
- Ice Storm \\
- Creation \\
- Hut

\medskip\index[Mostruario]{Adult Silver Dragon} \textbf{Adult Silver Dragon}

\textit{Huge dragon, legal good}

\textbf{STRENGTH} +8

\textbf{DEXTERITY} +0

\textbf{CONSTITUTION} +7

\textbf{INTELLIGENCE} +3

\textbf{WISDOM} +1

\textbf{CHARISMA} +5

\textbf{Initiative} +3 - \textbf{Defense} 27

\textbf{Hit Points} 243 (18d12 + 126)

\textbf{Movement} 12 m, flight 24 m

\textbf{Saving Throws} Fortitude +15, Reflex +12, Will +17

\textbf{Skills} Arcane +8, Move silently / Hide +5, Awareness +11, Story +8

\textbf{Damage Immunity} cold

\textbf{Senses} darkvision 36 m, blind sight 18 m

\textbf{Languages} Common, Draconic

\textbf{Challenge} 16 (1500 PX)

\textit{\textbf{Legendary Resistance (3 / Day).}} If the dragon fails a saving throw, it may choose to succeed instead.

\textbf{Actions}

\textit{\textbf{Multiattack.}} The dragon can use its Frightening Presence. Then make three attacks: one with the bite and two with the claws.

\textit{\textbf{Claw.} Melee Weapon Attack}: +27 to hit, 1m range, one target.

\textit{Strikes:} 15 (2d6 + 8) slashing damage, 1 bleed damage.

\textit{\textbf{Tail.} Melee Weapon Attack}: +27 to hit, range 5 meters, one target.

\textit{Strikes:} 17 (2d8 + 8) hit damage.

\textit{\textbf{Bite.} Melee Weapon Attack}: +27 to hit, 3m range, one target.

\textit{Strikes:} 19 (2d10 + 8) piercing damage.

\textit{\textbf{Frightening presence.}} Any creature the dragon chooses, within 36 meters of it and aware of its presence, must succeed on a DC 18 Will saving throw or be startled for 1 minute. A creature can re-roll the saving throw at the end of each of its rounds, ending the effect if successful. If the creature's saving throw is successful or the effect ends, the creature is immune to the dragon's dreadful presence for the next 24 hours.

\textit{\textbf{Breath Weapon (Cooldown 5-6).}} The dragon uses one of the following breath weapons:

\textit{Frost Breath.} The dragon exhales an icy blast into a 60-foot cone. Each creature in the area must make a DC 20 Fortitude save, taking 58 (13d8) cold damage if it fails the save, or half that damage if it succeeds.

\textit{Paralyzing Breath.} The dragon exhales paralyzing gas into a 60-foot cone. Each creature in the area must succeed on a Fortitude save 20 or be paralyzed for 1 minute. A creature can re-roll the saving throw at the end of each of its rounds, ending the effect for itself on success.

\textit{\textbf{Shape Shift.}} The dragon can magically transform into a humanoid or beast whose challenge rating is equal to or less than its own, or revert to its true form. At death it returns to its true form. Any equipment he is wearing or carrying is absorbed or carried in the new form (dragon's choice).

In the new form, the dragon retains its traits, hit points, hit dice, ability to speak, skill, legendary Fortitude, lair actions, and intelligence, wisdom, and charisma scores, in addition to this action. Its stats and abilities are otherwise replaced by those of the new form, except for the new form's additional Actions.

\textbf{Additional Actions}

The dragon can perform 3 additional Actions, chosen from the following options. He can only use one legendary option at a time, and only at the end of another creature's turn. The dragon recovers any additional Actions spent at the start of their round.

\textbf{Wing Attack (Costs 2 Actions).} The dragon flaps its wings. Each creature within 10 feet of the dragon must succeed on a DC 21 Reflex saving throw or take 15 (2d6 + 8) hit damage and be thrown prone. The dragon can then fly up to half of its flight movement.

\textbf{Tail Attack.} The dragon makes a tail attack.

\textbf{Spot.} The dragon makes a Wisdom (Awareness) check.

\textbf{Ecology} \\
Environment: Temperate Mountains \\
Organization: Solitary \\
\textbf{Treasure}: Triple \\
\textbf{Description} \\
Of all dragons, the silver dragons are the bravest, adhering to a chivalrous code that requires them to help the weak, defeat evil, and behave honorably. \\
\textbf{Spells} \index{Silver Dragon Spells} \\
This Dragon's favorite spells are: \\
- Cold cone \\
- Ice Storm \\
- Creation \\
- Hut

\medskip\index[Mostruario]{Young Silver Dragon} \textbf{Young Silver Dragon}

\textit{Large dragon, legal good}

\textbf{STRENGTH} +6

\textbf{DEXTERITY} +0

\textbf{CONSTITUTION} +5

\textbf{INTELLIGENCE} +2

\textbf{WISDOM} +0

\textbf{CHARISMA} +4

\textbf{Initiative} +2 - \textbf{Defense} 23

\textbf{Hit Points} 168 (16d10 + 80)

\textbf{Movement} 12 m, flight 24 m

\textbf{Saving Throws} Fortitude +10, Reflex +8, Will +12

\textbf{Skills} Arcane +6, Move silently / Hide +4, Awareness +8, Story +6

\textbf{Immunity to Damage} cold

\textbf{Senses} darkvision 36 m, blind sight 9 m

\textbf{Languages} Common, Draconic

\textbf{Challenge} 9 (5000 PX)

\textbf{Actions}

\textit{\textbf{Multiattack.}} The dragon can make three attacks: one with its bite and two with its claws.

\textit{\textbf{Claw.} Melee Weapon Attack}: +15 to hit, 1m range, one target.

\textit{Strikes:} 13 (2d6 + 6) slashing damage, 1 bleed damage.

\textit{\textbf{Bite.} Melee Weapon Attack}: +15 to hit, 3m range, one target.

\textit{Strikes:} 17 (2d10 + 6) piercing damage.

\textit{\textbf{Breath Weapon (Cooldown 5-6).}} The dragon uses one of the following breath weapons:

\textit{Frost Breath.} The dragon exhales an icy blast into a 30-foot cone. Each creature in the area must make a DC 17 Fortitude save, taking 54 (12d8) cold damage if it fails the save, or half that damage if it succeeds.

\textit{Paralyzing Breath.} The dragon exhales paralyzing gas into a 30-foot cone. Each creature in the area must succeed on a Fortitude save of 17 or be paralyzed for 1 minute. A creature can re-roll the saving throw at the end of each of its rounds, ending the effect for itself on success.

\textbf{Ecology} \\
Environment: Temperate Mountains \\
Organization: Solitary \\
\textbf{Treasure}: Triple \\
\textbf{Description} \\
Of all dragons, the silver dragons are the bravest, adhering to a chivalrous code that requires them to help the weak, defeat evil and behave honorably. \\
\textbf{Spells} \index{Silver Dragon Spells} \\
This Dragon's favorite spells are: \\
- Cold cone \\
- Ice Storm \\
- Creation \\
- Hut

\medskip\index[Mostruario]{Silver Dragon Pup} \textbf{Silver Dragon Pup}

\textit{Medium dragon, legal good}

\textbf{STRENGTH} +4

\textbf{DEXTERITY} +0

\textbf{CONSTITUTION} +3

\textbf{INTELLIGENCE} +1

\textbf{WISDOM} +0

\textbf{CHARISMA} +2

\textbf{Initiative} +1 - \textbf{Defense} 18

\textbf{Hit Points} 45 (6d8 + 18)

\textbf{Movement} 9 m, flight 18 m

\textbf{Saving Throws} Fortitude +3, Reflex +3, Will +2

\textbf{Skills} Move Silently / Hide +2, Awareness +4

\textbf{Immunity to Damage} cold

\textbf{Senses} darkvision 18 m, blind sight 3 m

\textbf{Languages} Draconic

\textbf{Challenge} 2 (450 PX)

\textbf{Actions}

\textit{\textbf{Bite.} Melee Weapon Attack}: +6 to hit, 1m range, one target.

\textit{Strikes:} 9 (1d10 + 4) piercing damage.

\textit{\textbf{Breath Weapon (Cooldown 5-6).}} The dragon uses one of the following breath weapons:

\textit{Frost Breath.} The dragon exhales an icy blast into a 5 meter cone. Each creature in the area must make a 13 Fortitude save, taking 18 (4d8) cold damage if it fails the saving throw, or half that damage if it succeeds.

\textit{Paralyzing Breath.} The dragon exhales paralyzing gas into a 5 meter cone. Each creature in the area must succeed on a 13 Fortitude save or be paralyzed for 1 minute. A creature can re-roll the saving throw at the end of each of its rounds, ending the effect for itself on success.

\textbf{Ecology} \\
Environment: Temperate Mountains \\
Organization: Solitary \\
\textbf{Treasure}: Triple \\
\textbf{Description} \\
Of all dragons, the silver dragons are the bravest, and they abide by a chivalrous code that requires them to help the weak, defeat evil, and behave honorably.


\medskip\index[Mostruario]{Ancient Bronze Dragon} \textbf{Ancient Bronze Dragon}

\textit{Gargantuan dragon, chaotic good}

\textbf{STRENGTH} +9

\textbf{DEXTERITY} +0

\textbf{CONSTITUTION} +8

\textbf{INTELLIGENCE} +4

\textbf{WISDOM} +3

\textbf{CHARISMA} +5

\textbf{Initiative} +4 - \textbf{Defense} 33

\textbf{Hit Points} 444 (24d20 + 192)

\textbf{Movement} 12m, swim 12m, fly 24m

\textbf{Saving Throws} Fortitude +21, Reflex +13, Will +21

\textbf{Skills} Move Silently / Hide +7, Feel Emotions +10, Awareness +17

\textbf{Immunity to Damage} lightning, weapons +1

\textbf{Senses} darkvision 36 m, blind sight 18 m

\textbf{Languages} Common, Draconic

\textbf{Challenge} 22 (41000 PX)

\textit{\textbf{Amphibian.}} The dragon can breathe air and water.

\textit{\textbf{Legendary Resistance (3 / Day).}} If the dragon fails a saving throw, it may choose to succeed instead.

\textbf{Actions}

\textit{\textbf{Multiattack.}} The dragon can use its Frightening Presence. Then make three attacks: one with the bite and two with the claws.

\textit{\textbf{Claw.} Melee Weapon Attack}: +30 to hit, 3m range, one target.

\textit{Strikes:} 16 (2d6 + 9) slashing damage, 3 bleed damage (to a maximum of 10).

\textit{\textbf{Tail.} Melee Weapon Attack}: +30 to hit, 6m range, one target.

\textit{Strikes:} 18 (2d8 + 9) hit damage.

\textit{\textbf{Bite.} Melee Weapon Attack}: +30 to hit, 5m range, one target.

\textit{Strikes:} 20 (2d10 + 9) piercing damage.

\textit{\textbf{Frightening presence.}} Any creature the dragon chooses, within 36 meters of it and aware of its presence, must succeed on a DC 20 Will saving throw or be frightened for 1 minute. A creature can re-roll the saving throw at the end of each of its rounds, ending the effect if successful. If the creature's saving throw is successful or the effect ends, the creature is immune to the dragon's dreadful presence for the next 24 hours.

\textit{\textbf{Breath Weapon (Cooldown 5-6).}} The dragon uses one of the following breath weapons:

\textit{Lightning Breath.} The dragon exhales lightning in a line 36 meters long and 3 meters wide. Each creature on the line must make a DC 23 Reflex saving throw, taking 88 (16d10) lightning damage on a failed save, or half that damage on a successful one. \textit{Repulsive Breath.} The dragon exhales repulsive energy into a 30-foot cone. Each creature in that area must succeed at a DC 23 Fortitude save or it is 60 feet away from
dragon.

\textit{\textbf{Shape Change}} The dragon can magically transform into a humanoid or beast whose challenge rating is equal to or less than its own, or revert to its true form. At death it returns to its true form. Any equipment he is wearing or carrying is absorbed or carried in the new form (dragon's choice).

In the new form, the dragon retains its traits, hit points, hit dice, ability to speak, skill, legendary Fortitude, lair actions, and intelligence, wisdom, and charisma scores, in addition to this action. Its stats and abilities are otherwise replaced by those of the new form, except for the new form's additional Actions.

\textbf{Additional Actions}

The dragon can perform 3 additional Actions, chosen from the following options. He can only use one legendary option at a time, and only at the end of another creature's turn. The dragon recovers any additional Actions spent at the start of their round.

\textbf{Wing Attack (Costs 2 Actions).} The dragon flaps its wings. Each creature within 5 yards of the dragon must succeed on a DC 24 Reflex saving throw or take 16 (2d6 + 9) hit damage and be thrown prone. The dragon can then fly up to half of its flight movement.

\textbf{Tail Attack.} The dragon makes a tail attack.

\textbf{Spot.} The dragon makes a Wisdom (Awareness) check.

\textbf{Ecology} \\
Environment: Temperate Coastal Zones \\
Organization: Solitary \\
\textbf{Treasure}: Triple \\
\textbf{Description} \\
Bronze dragons are known to ally with travelers and adventurers if cause and reward are just and adequate \\
\textbf{Spells} \index{Bronze Dragon Spells} \\
This Dragon's favorite spells are: \\
- Check Atmospheric Weather \\
- Invoke the Lightning Bolt


\medskip\index[Mostruario]{Adult Bronze Dragon} \textbf{Adult Bronze Dragon}

\textit{Huge dragon, good chaotic}

\textbf{STRENGTH} +7

\textbf{DEXTERITY} +0

\textbf{CONSTITUTION} +6

\textbf{INTELLIGENCE} +3

\textbf{WISDOM} +2

\textbf{CHARISMA} +4

\textbf{Initiative} +3 - \textbf{Defense} 27

\textbf{Hit Points} 212 (17d12 + 102)

\textbf{Movement} 12m, swim 12m, fly 24m

\textbf{Saving Throws} Fortitude +15, Reflex +10, Will +15

\textbf{Skills} Move Silently / Hide +5, Feel Emotions +7, Awareness +12

\textbf{Immunity to Damage} lightning bolt

\textbf{Senses} darkvision 36 m, blind sight 18 m

\textbf{Languages} Common, Draconic

\textbf{Challenge} 15 (13000 PX)

\textit{\textbf{Amphibian.}} The dragon can breathe air and water.

\textit{\textbf{Legendary Resistance (3 / Day).}} If the dragon fails a saving throw, it may choose to succeed instead.

\textbf{Actions}

\textit{\textbf{Multiattack.}} The dragon can use its Frightening Presence and then make three attacks: one with its bite and two with its claws.

\textit{\textbf{Claw.} Melee Weapon Attack}: +24 to hit, 1m range, one target.

\textit{Strikes:} 14 (2d6 + 7) slashing damage, 1 bleed damage.

\textit{\textbf{Tail.} Melee Weapon Attack}: +24 to hit, range 5 meters, one target.

\textit{Strikes:} 16 (2d8 + 7) hit damage.

\textit{\textbf{Bite.} Melee Weapon Attack}: +24 to hit, 3m range, one target.

\textit{Strikes:} 18 (2d10 + 7) piercing damage.

\textit{\textbf{Frightening presence.}} Any creature the dragon chooses, within 36 meters of it and aware of its presence, must succeed on a DC 17 Will saving throw or be frightened for 1 minute. A creature can re-roll the saving throw at the end of each of its rounds, ending the effect if successful. If the creature's saving throw is successful or the effect ends, the creature is immune to the dragon's dreadful presence for the next 24 hours.

\textit{\textbf{Breath Weapon (Cooldown 5-6).}} The dragon uses one of the following breath weapons:

\textit{Lightning Breath.} The dragon exhales lightning bolts in a line 27 meters long and 1 meter wide. Each creature on the line must make a DC 19 Reflex saving throw, taking 66 (12d10) lightning damage on a failed save, or half that damage on a successful one. \textit{Repulsive Breath.} The dragon exhales repulsive energy into a 30-foot cone. Each creature in that area must succeed at a DC 19 Fortitude save or it is 60 feet away from the dragon.

\textit{\textbf{Shape Shift.}} The dragon can magically transform into a humanoid or beast whose challenge rating is equal to or less than its own, or revert to its true form. At death it returns to its true form. Any equipment he is wearing or carrying is absorbed or carried in the new form (dragon's choice).

In the new form, the dragon retains its traits, hit points, hit dice, ability to speak, skill, legendary Fortitude, lair actions, and intelligence, wisdom, and charisma scores, in addition to this action. Its stats and abilities are otherwise replaced by those of the new form, except for the new form's additional Actions.

\textbf{Additional Actions}

The dragon can perform 3 additional Actions, chosen from the following options. He can only use one legendary option at a time, and only at the end of another creature's turn. The dragon recovers any additional Actions spent at the start of their round.

\textbf{Wing Attack (Costs 2 Actions).} The dragon flaps its wings. Each creature within 10 feet of the dragon must succeed on a DC 20 Reflex saving throw or take 14 (2d6 + 7) hit damage and be thrown prone. The dragon can then fly up to half of its flight movement.

\textbf{Tail Attack.} The dragon makes a tail attack.

\textbf{Spot.} The dragon makes a Wisdom (Awareness) check.

\textbf{Ecology} \\
Environment: Temperate Coastal Zones \\
Organization: Solitary \\
\textbf{Treasure}: Triple \\
\textbf{Description} \\
Bronze dragons are known to ally with travelers and adventurers if cause and reward are just and adequate \\
\textbf{Spells} \index{Bronze Dragon Spells} \\
This Dragon's favorite spells are: \\
- Check Atmospheric Weather \\
- Invoke the Lightning Bolt


\medskip\index[Mostruario]{Young Bronze Dragon} \textbf{Young Bronze Dragon}

\textit{Large dragon, good chaotic}

\textbf{STRENGTH} +5

\textbf{DEXTERITY} +0

\textbf{CONSTITUTION} +4

\textbf{INTELLIGENCE} +2

\textbf{WISDOM} +1

\textbf{CHARISMA} +3

\textbf{Initiative} +2 - \textbf{Defense} 22

\textbf{Hit Points} 142 (15d10 + 60)

\textbf{Movement} 12m, swim 12m, fly 24m

\textbf{Saving Throws} Fortitude +10, Reflex +8, Will +10

\textbf{Skills} Move Silently / Hide +3, Feel Emotions +4, Awareness +7

\textbf{Immunity to Damage} lightning

\textbf{Senses} darkvision 36 m, blind sight 9 m

\textbf{Languages} Common, Draconic

\textbf{Challenge} 8 (3.900 PX)

\textit{\textbf{Amphibian.}} The dragon can breathe air and water.

\textbf{Actions}

\textit{\textbf{Multiattack.}} The dragon can make three attacks: one with its bite and two with its claws.

\textit{\textbf{Claw.} Melee Weapon Attack}: +12 to hit, 1m range, one target.

\textit{Strikes:} 12 (2d6 + 5) slashing damage, 1 bleed damage.

\textit{\textbf{Bite.} Melee Weapon Attack}: +12 to hit, 3m range, one target.

\textit{Strikes:} 16 (2d10 + 5) piercing damage.

\textit{\textbf{Breath Weapon (Cooldown 5-6).}} The dragon uses one of the following breath weapons:

\textit{Lightning Breath.} The dragon exhales lightning bolts in a line 18 meters long and 1 meter wide. Each creature on the line must make a DC 15 Reflex saving throw, taking 55 (10d10) lightning damage on a failed save, or half that damage on a successful one.

\textit{Repulsive Breath.} The dragon exhales repulsive energy into a 30-foot cone. Each creature in that area must succeed at a DC 15 Fortitude save or it is 12 meters away from the dragon.

\textbf{Ecology} \\
Environment: Temperate Coastal Zones \\
Organization: Solitary \\
\textbf{Treasure}: Triple \\
\textbf{Description} \\
Bronze dragons are known to ally with travelers and adventurers if cause and reward are just and adequate \\
\textbf{Spells} \index{Bronze Dragon Spells} \\
This Dragon's favorite spells are: \\
- Check Atmospheric Weather \\
- Invoke the Lightning Bolt


\medskip\index[Mostruario]{Bronze Baby Dragon} \textbf{Bronze Baby Dragon}

\textit{Medium dragon, chaotic good}

\textbf{STRENGTH} +3

\textbf{DEXTERITY} +0

\textbf{CONSTITUTION} +2

\textbf{INTELLIGENCE} +1

\textbf{WISDOM} +0

\textbf{CHARISMA} +2

\textbf{Initiative} +1 - \textbf{Defense} 18

\textbf{Hit Points} 32 (5d8 + 10)

\textbf{Movement} 9m, swim 9m, fly 18m

\textbf{Saving Throws} Fortitude +2, Reflex +1, Will +1

\textbf{Skills} Move Silently / Hide +2, Awareness +4

\textbf{Immunity to Damage} lightning

\textbf{Senses} darkvision 18 m, blind sight 3 m

\textbf{Languages} Draconic

\textbf{Challenge} 2 (450 PX)

\textit{\textbf{Amphibian.}} The dragon can breathe air and water.

\textbf{Actions}

\textit{\textbf{Bite.} Melee Weapon Attack}: +5 to hit,
1 m range, one target.

\textit{Strikes:} 8 (1d10 + 3) piercing damage.

\textit{\textbf{Breath Weapon (Cooldown 5-6).}} The dragon uses one of the following breath weapons:

\textit{Lightning Breath.} The dragon exhales lightning bolts in a line 12 meters long and 1 meter wide. Each creature on the line must make a DC 12 Reflex saving throw, taking 16 (3d10) lightning damage on a failed save, or half that damage on a successful one.

\textit{Repulsive Breath.} The dragon exhales repulsive energy into a 30-foot cone. Each creature in that area must succeed at a DC 12 Fortitude saving throw or it is 30 feet away from the dragon.

\textbf{Ecology} \\
Environment: Temperate Coastal Zones \\
Organization: Solitary \\
\textbf{Treasure}: Triple \\
\textbf{Description} \\
Bronze dragons are known to ally with travelers and adventurers if cause and reward are right and appropriate.

\medskip\index[Mostruario]{Ancient Gold Dragon} \textbf{Ancient Gold Dragon}

\textit{Gargantuan dragon, lawful good}

\textbf{STRENGTH} +10

\textbf{DEXTERITY} +2

\textbf{CONSTITUTION} +9

\textbf{INTELLIGENCE} +4

\textbf{WISDOM} +3

\textbf{CHARISMA} +9

\textbf{Initiative} +4 - \textbf{Defense} 34

\textbf{Hit Points} 546 (28d20 + 252)

\textbf{Movement} 12m, swim 12m, fly 24m

\textbf{Saving Throws} Fortitude +23, Reflex +14, Will +24

\textbf{Skills} Move Silently / Hide +9, Feel Emotions +10, Awareness +17, Deceive +16

\textbf{Immunity to Damage} fire, weapons +1

\textbf{Senses} darkvision 36 m, blind sight 18 m

\textbf{Languages} Common, Draconic

\textbf{Challenge} 24 (62000 PX)

\textit{\textbf{Amphibian.}} The dragon can breathe air and water.

\textit{\textbf{Legendary Resistance (3 / Day).}} If the dragon fails a saving throw, it may choose to succeed instead.

\textbf{Actions}

\textit{\textbf{Multiattack.}} The dragon can use its Frightening Presence. Then make three attacks: one with the bite and two with the claws.

\textit{\textbf{Claw.} Melee Weapon Attack}: +30 to hit, 3m range, one target.

\textit{Hits:} 17 (2d6 + 10) slashing damage, 3 bleed damage (to a maximum of 10).

\textit{\textbf{Tail.} Melee Weapon Attack}: +30 to hit, 6m range, one target.

\textit{Strikes:} 19 (2d8 + 10) hit damage.

\textit{\textbf{Bite.} Melee Weapon Attack}: +30 to hit, 5m range, one target.

\textit{Strikes:} 21 (2d10 + 10) piercing damage.

\textit{\textbf{Dreadful Presence.}} Any creature the dragon chooses, within 36 meters of it and aware of its presence, must succeed on a DC 24 Will saving throw or be startled for 1 minute. A creature can re-roll the saving throw at the end of each of its rounds, ending the effect if successful. If the creature's saving throw is successful or the effect ends, the creature is immune to the dragon's dreadful presence for the next 24 hours.

\textit{\textbf{Breath Weapon (Cooldown 5-6).}} The dragon uses one of the following breath weapons:

\textit{Fiery Breath.} The dragon exhales fire into a 27-meter cone. Each creature in the area must make a DC 24 Reflex saving throw, taking 71 (13d10) fire damage on a failed save, or half that damage on a successful one.

\textit{Weak Breath.} The dragon exhales gas into a 27-meter cone. Each creature in that area must succeed on a DC 24 Fortitude saving throw or have -1d6 on Strength-based attack rolls, Strength checks, and Fortitude saving throws for 1 minute. A creature can re-roll its saving throw at the end of each of its rounds, ending the effect on itself if it succeeds.

\textit{\textbf{Shape Shift.}} The dragon can magically transform into a humanoid or beast whose challenge rating is equal to or less than its own, or revert to its true form. At death it returns to its true form. Any equipment he is wearing or carrying is absorbed or carried in the new form (dragon's choice).

In the new form, the dragon retains its traits, hit points, hit dice, ability to speak, skill, legendary Fortitude, lair actions, and intelligence, wisdom, and charisma scores, in addition to this action. Its stats and abilities are otherwise replaced by those of the new form, except for the new form's additional Actions.

\textbf{Additional Actions}

The dragon can perform 3 additional Actions, chosen from the following options. He can only use one legendary option at a time, and only at the end of another creature's turn. The dragon recovers any additional Actions spent at the start of their round.

\textbf{Wing Attack (Costs 2 Actions).} The dragon flaps its wings. Each creature within 5 yards of the dragon must succeed on a DC 25 Reflex saving throw or take 17 (2d6 + 10) hit damage and be thrown prone. The dragon can then fly up to half of its flight movement.

\textbf{Tail Attack.} The dragon makes a tail attack.

\textbf{Spot.} The dragon makes a Wisdom (Awareness) check.

\textbf{Ecology} \\
Environment: Warm plains \\
Organization: Solitary \\
\textbf{Treasure}: Triple \\
\textbf{Description} \\
Golden dragons are the emblem of virtue. Other metallic dragons revere them as agents of divine powers and exemplary members of the dragon race, and often seek them for advice or help. \\
\textbf{Spells} \index{Gold Dragon Spells} \\
This Dragon's favorite spells are: \\
- Block Monsters \\
- Wall of Fire \\
- Dimensional Door

\medskip\index[Mostruario]{Adult Gold Dragon} \textbf{Adult Gold Dragon}

\textit{Huge dragon, legal good}

\textbf{STRENGTH} +8

\textbf{DEXTERITY} +2

\textbf{CONSTITUTION} +7

\textbf{INTELLIGENCE} +3

\textbf{WISDOM} +2

\textbf{CHARISMA} +7

\textbf{Initiative} +3 - \textbf{Defense} 28

\textbf{Hit Points} 256 (19d12 + 133)

\textbf{Movement} 12m, swim 12m, fly 24m

\textbf{Saving Throws} Fortitude +17, Reflex +11, Will +18

\textbf{Skills} Move Silently / Hide +8, Feel Emotions +8, Awareness +14, Deceive +13

\textbf{Immunity to Damage} fire

\textbf{Senses} darkvision 36 m, blind sight 18 m

\textbf{Languages} Common, Draconic

\textbf{Challenge} 17 (18000 PX)

\textit{\textbf{Amphibian.}} The dragon can breathe air and water.

\textit{\textbf{Legendary Resistance (3 / Day).}} If the dragon fails a saving throw, it may choose to succeed instead.

\textbf{Actions}

\textit{\textbf{Multiattack.}} The dragon can use its Frightening Presence. Then make three attacks: one with the bite and two with the claws.

\textit{\textbf{Claw.} Melee Weapon Attack}: +28 to hit, 1m range, one target.

\textit{Strikes:} 15 (2d6 + 8) slashing damage, 1 bleed damage.

\textit{\textbf{Tail.} Melee Weapon Attack}: +28 to hit, range 5 meters, one target.

\textit{Strikes:} 17 (2d8 + 8) hit damage.

\textit{\textbf{Bite.} Melee Weapon Attack}: +28 to hit, 3m range, one target.

\textit{Strikes:} 19 (2d10 + 8) piercing damage.

\textit{\textbf{Frightening presence.}} Any creature the dragon chooses, within 36 meters of it and aware of its presence, must succeed on a DC 21 Will saving throw or be frightened for 1 minute. A creature can re-roll the saving throw at the end of each of its rounds, ending the effect if successful. If the creature's saving throw is successful or the effect ends, the creature is immune to the dragon's dreadful presence for the next 24 hours.

\textit{\textbf{Breath Weapon (Cooldown 5-6).}} The dragon uses one of the following breath weapons:

\textit{Fiery Breath.} The dragon exhales fire into a 60-foot cone. Each creature in the area must make a DC 21 Reflex saving throw, taking 66 (12d10) fire damage on a failed save, or half that damage on a successful one.

\textit{Weak Breath.} The dragon exhales gas into a 60-foot cone. Each creature in that area must succeed on a DC 21 Fortitude save or have -1d6 on Strength-based attack rolls, Strength checks, and Fortitude saving throws for 1 minute. A creature can re-roll its saving throw at the end of each of its rounds, ending the effect on itself if it succeeds.

\textit{\textbf{Shapeshift.}} The dragon can magically transform into a humanoid or beast whose challenge rating is equal to or less than its own, or revert to its true form. At death it returns to its true form. Any equipment he is wearing or carrying is absorbed or carried in the new form (dragon's choice).

In the new form, the dragon retains its traits, hit points, hit dice, ability to speak, skill, legendary Fortitude, lair actions, and intelligence, wisdom, and charisma scores, in addition to this action. Its stats and abilities are otherwise replaced by those of the new form, except for the new form's additional Actions.

\textbf{Additional Actions}

The dragon can perform 3 additional Actions, chosen from the following options. He can only use one legendary option at a time, and only at the end of another creature's turn. The dragon recovers any additional Actions spent at the start of their round.

\textbf{Wing Attack (Costs 2 Actions).} The dragon flaps its wings. Each creature within 10 feet of the dragon must succeed on a DC 22 Reflex saving throw or take 15 (2d6 + 8) hit damage and be thrown prone. The dragon can then fly up to half of its flight movement.

\textbf{Tail Attack.} The dragon makes a tail attack.

\textbf{Spot.} The dragon makes a Wisdom (Awareness) check.

\textbf{Ecology} \\
Environment: Warm plains \\
Organization: Solitary \\
\textbf{Treasure}: Triple \\
\textbf{Description} \\
Golden dragons are the emblem of virtue. Other metallic dragons revere them as agents of divine powers and exemplary members of the dragon race, and often seek them for advice or help. \\
\textbf{Spells} \index{Gold Dragon Spells} \\
This Dragon's favorite spells are: \\
- Block Monsters \\
- Wall of Fire \\
- Dimensional Door

\medskip\index[Mostruario]{Young Gold Dragon} \textbf{Young Gold Dragon}

\textit{Large dragon, legal good}

\textbf{STRENGTH} +6

\textbf{DEXTERITY} +2

\textbf{CONSTITUTION} +5

\textbf{INTELLIGENCE} +3

\textbf{WISDOM} +1

\textbf{CHARISMA} +5

\textbf{Initiative} +3 - \textbf{Defense} 23

\textbf{Hit Points} 178 (17d10 + 85)

\textbf{Movement} 12m, swim 12m, fly 24m

\textbf{Saving Throws} Fortitude +12, Reflex +9, Will +13

\textbf{Skills} Move Silently / Hide +6, Feel Emotions +5, Awareness +9, Deceive +9

\textbf{Immunity to Damage} fire

\textbf{Senses} darkvision 36 m, blind sight 9 m

\textbf{Languages} Common, Draconic

\textbf{Challenge} 10 (5.900 PX)

\textit{\textbf{Amphibian.}} The dragon can breathe air and water.

\textbf{Actions}

\textit{\textbf{Multiattack.}} The dragon can make three attacks: one with its bite and two with its claws.

\textit{\textbf{Claw.} Melee Weapon Attack}: +16 to hit, 1m range, one target.

\textit{Strikes:} 13 (2d6 + 6) slashing damage, 1 bleed damage.

\textit{\textbf{Bite.} Melee Weapon Attack}: +16 to hit, 3m range, one target.

\textit{Strikes:} 17 (2d10 + 6) piercing damage.

\textit{\textbf{Breath Weapon (Cooldown 5-6).}} The dragon uses one of the following breath weapons:

\textit{Fiery Breath.} The dragon exhales fire into a 30-foot cone. Each creature in the area must make a DC 17 Reflex saving throw, taking 55 (10d10) fire damage on a failed save, or half that damage on a successful one.

\textit{Weak Breath.} The dragon exhales gas into a 30-foot cone. Each creature in that area must succeed on a DC 17 Fortitude saving throw or have -1d6 on Strength-based attack rolls, Strength checks, and Fortitude saving throws for 1 minute. A creature can re-roll its saving throw at the end of each of its rounds, ending the effect on itself if it succeeds.

\textbf{Ecology} \\
Environment: Warm plains \\
Organization: Solitary \\
\textbf{Treasure}: Triple \\
\textbf{Description} \\
Golden dragons are the emblem of virtue. Other metallic dragons revere them as agents of divine powers and exemplary members of the dragon race, and often seek them for advice or help. \\
\textbf{Spells} \index{Golden Dragon Spells} \\
This Dragon's favorite spells are: \\
- Block Monsters \\
- Wall of Fire \\
- Dimensional Door

\medskip\index[Mostruario]{Golden Baby Dragon} \textbf{Golden Baby Dragon}

\textit{Medium dragon, legal good}

\textbf{STRENGTH} +4

\textbf{DEXTERITY} +2

\textbf{CONSTITUTION} +3

\textbf{INTELLIGENCE} +2

\textbf{WISDOM} +0

\textbf{CHARISMA} +3

\textbf{Initiative} +2 - \textbf{Defense} 19

\textbf{Hit Points} 60 (8d8 + 24)

\textbf{Movement} 9m, swim 9m, fly 18m

\textbf{Saving Throws} Fortitude +3, Reflex +2, Will +1

\textbf{Skills} Move Silently / Hide +4, Awareness +4

\textbf{Immunity to Damage} fire

\textbf{Senses} darkvision 18 m, blind sight 3 m

\textbf{Languages} Draconic

\textbf{Challenge} 3 (700 PX)

\textit{\textbf{Amphibian.}} The dragon can breathe air and water.

\textbf{Actions}

\textit{\textbf{Bite.} Melee Weapon Attack}: +6 to hit, 1m range, one target.

\textit{Strikes:} 9 (1d10 + 4) piercing damage.

\textit{\textbf{Breath Weapon (Cooldown 5-6).}} The dragon uses one of the following breath weapons:

\textit{Fiery Breath.} The dragon exhales fire into a 5 meter cone. Each creature in the area must make a DC 13 Reflex saving throw, taking 22 (4d10) fire damage on a failed save, or half that damage on a successful one.

\textit{Weak Breath.} The dragon exhales gas into a 5 meter cone. Each creature in that area must succeed on a DC 13 Fortitude save or have -1d6 on Strength-based attack rolls, Strength checks, and Fortitude saving throws for 1 minute. A creature can re-roll its saving throw at the end of each of its rounds, ending the effect on itself if it succeeds.

\textbf{Ecology} \\
Environment: Warm plains \\
Organization: Solitary \\
\textbf{Treasure}: Triple \\
\textbf{Description} \\
Golden dragons are the emblem of virtue. Other metallic dragons revere them as agents of divine powers and exemplary members of the dragon race, and often seek them for advice or help.

\medskip\index[Mostruario]{Ancient Brass Dragon} \textbf{Ancient Brass Dragon}

\textit{Gargantuan dragon, chaotic good}

\textbf{STRENGTH} +8

\textbf{DEXTERITY} +0

\textbf{CONSTITUTION} +7

\textbf{INTELLIGENCE} +3

\textbf{WISDOM} +2

\textbf{CHARISMA} +4

\textbf{Initiative} +3 - \textbf{Defense} 30

\textbf{Hit Points} 297 (17d20 + 119)

\textbf{Movement} 12 m, Burrow 12 m, flight 24 m

\textbf{Saving Throws} Fortitude +20, Reflex +13, Will +18

\textbf{Skills} Move Silently / Hide +6, Awareness +14, Deceive +10, Story +9

\textbf{Immunity to Damage} fire, weapons +1

\textbf{Senses} darkvision 36 m, blind sight 18 m

\textbf{Languages} Common, Draconic

\textbf{Challenge} 20 (25000 PX)

\textit{\textbf{Legendary Resistance (3 / Day).}} If the dragon fails a saving throw, it may choose to succeed instead.

\textbf{Actions}

\textit{\textbf{Multiattack.}} The dragon can use its Frightening Presence. Then make three attacks: one with the bite and two with the claws.

\textit{\textbf{Claw.} Melee Weapon Attack}: +30 to hit, 3m range, one target.

\textit{Strikes:} 15 (2d6 + 8) slashing damage, 3 bleed damage (up to 10)

\textit{\textbf{Tail.} Melee Weapon Attack}: +30 hit, 6m range, one target.

\textit{Strikes:} 17 (2d8 + 8) hit damage.

\textit{\textbf{Bite.} Melee Weapon Attack}: +30 to hit, 5m range, one target.

\textit{Strikes:} 19 (2d10 + 8) piercing damage.

\textit{\textbf{Frightening presence.}} Any creature the dragon chooses, within 36 meters of it and aware of its presence, must succeed on a DC 18 Will saving throw or be startled for 1 minute. A creature can re-roll the saving throw at the end of each of its rounds, ending the effect if successful. If the creature's saving throw is successful or the effect ends, the creature is immune to the dragon's dreadful presence for the next 24 hours.

\textit{\textbf{Breath Weapon (Cooldown 5-6).}} The dragon uses one of the following breath weapons:

\textit{Fiery Breath.} The dragon exhales fire in a line 27 meters long and 3 meters wide. Each creature on the line must make a DC 21 Reflex saving throw, taking 56 (16d6) fire damage on a failed save, or half that damage on a successful one.

\textit{Soporific Breath.} The dragon exhales soporific gas into a 27 meter cone. Each creature in that area must succeed on a Fortitude save 21 or fall unconscious for 10 minutes. This effect
ends if the unconscious creature takes damage or someone takes an action to awaken it.

\textit{\textbf{Shape Shift.}} The dragon can magically transform into a humanoid or beast whose challenge rating is equal to or less than its own, or revert to its true form. At death it returns to its true form. Any equipment he is wearing or carrying is absorbed or carried in the new form (dragon's choice).

In the new form, the dragon retains its traits, hit points, hit dice, ability to speak, skill, legendary Fortitude, lair actions, and intelligence, wisdom, and charisma scores, in addition to this action. Its stats and abilities are otherwise replaced by those of the new form, except for the new form's additional Actions.

\textbf{Additional Actions}

The dragon can perform 3 additional Actions, chosen from the following options. He can only use one legendary option at a time, and only at the end of another creature's turn. The dragon recovers any additional Actions spent at the start of their round.

\textbf{Wing Attack (Costs 2 Actions).} The dragon flaps its wings. Each creature within 5 yards of the dragon must succeed on a DC 22 Reflex saving throw or take 15 (2d6 + 8) hit damage and be thrown prone. The dragon can then fly up to half of its flight movement.

\textbf{Tail Attack.} The dragon makes a tail attack.

\textbf{Spot.} The dragon makes a Wisdom (Awareness) check.

\textbf{Ecology} \\
Environment: Warm Deserts \\
Organization: Solitary \\
\textbf{Treasure}: Triple \\
\textbf{Description} \\
Excellent conversationalists, brass dragons prefer to talk rather than fight. Brass dragons make their lair near humanoid settlements, where they can hear the latest news and gossip. \\
\textbf{Spells} \index{Brass Dragon Spells} \\
This Dragon's favorite spells are: \\
- Send \\
- Hypnotic Plot \\
- Languages


\medskip\index[Mostruario]{Adult Brass Dragon} \textbf{Adult Brass Dragon}

\textit{Huge dragon, good chaotic}

\textbf{STRENGTH} +6

\textbf{DEXTERITY} +0

\textbf{CONSTITUTION} +5

\textbf{INTELLIGENCE} +2

\textbf{WISDOM} +1

\textbf{CHARISMA} +3

\textbf{Initiative} +2 - \textbf{Defense} 25

\textbf{Hit Points} 172 (15d12 + 75)

\textbf{Movement} 12 m, Burrow 9 m, flight 24 m

\textbf{Saving Throws} Fortitude +14, Reflex +10, Will +12

\textbf{Skills} Move Silently / Hide +5, Awareness +11, Deceive +8, Story +7

\textbf{Immunity to Damage} fire

\textbf{Senses} darkvision 36 m, blind sight 18 m

\textbf{Languages} Common, Draconic

\textbf{Challenge} 13 (10000 PX)

\textit{\textbf{Legendary Resistance (3 / Day).}} If the dragon fails a saving throw, it may choose to succeed instead.

\textbf{Actions}

\textit{\textbf{Multiattack.}} The dragon can use its Frightening Presence. Then make three attacks: one with the bite and two with the claws.

\textit{\textbf{Claw.} Melee Weapon Attack}: +20 to hit, 1m range, one target.

\textit{Strikes:} 13 (2d6 + 6) slashing damage, 1 bleed damage.

\textit{\textbf{Tail.} Melee Weapon Attack}: +20 to hit, 5m range, one target.

\textit{Strikes:} 15 (2d8 + 6) hit damage.

\textit{\textbf{Bite.} Melee Weapon Attack}: +20 to hit, 3m range, one target.

\textit{Strikes:} 17 (2d10 + 6) piercing damage.

\textit{\textbf{Frightening presence.}} Any creature the dragon chooses, within 36 meters of it and aware of its presence, must succeed on a DC 16 Will saving throw or be frightened for 1 minute. A creature can re-roll the saving throw at the end of each of its rounds, ending the effect if successful. If the creature's saving throw is successful or the effect ends, the creature is immune to the dragon's dreadful presence for the next 24 hours.

\textit{\textbf{Breath Weapon (Cooldown 5-6).}} The dragon uses one of the following breath weapons:

\textit{Fiery Breath.} The dragon exhales fire in a line 18 meters long and 1 meter wide. Each creature on the line must make a DC 18 Reflex saving throw, taking 45 (13d6) fire damage if it fails the saving throw, or half that damage if it succeeds.

\textit{Soporific Breath.} The dragon exhales sleep gas into a 60-foot cone. Each creature in that area must succeed at an 18 Fortitude save or fall unconscious for 10 minutes. This effect ends if the unconscious creature takes damage or someone takes an action to awaken it.

\textbf{Additional Actions}

The dragon can perform 3 additional Actions, chosen from the following options. He can only use one legendary option at a time, and only at the end of another creature's turn. The dragon recovers any additional Actions spent at the start of their round.

\textbf{Wing Attack (Costs 2 Actions).} The dragon flaps its wings. Each creature within 10 feet of the dragon must succeed on a DC 19 Reflex saving throw or take 13 (2d6 + 6) hit damage and be thrown prone. The dragon can then fly up to half of its flight movement.

\textbf{Tail Attack.} The dragon makes a tail attack.

\textbf{Spot.} The dragon makes a Wisdom (Awareness) check.

\textbf{Ecology} \\
Environment: Warm Deserts \\
Organization: Solitary \\
\textbf{Treasure}: Triple \\
\textbf{Description} \\
Excellent conversationalists, brass dragons prefer to talk rather than fight. Brass dragons make their lair near humanoid settlements, where they can hear the latest news and gossip. \\
\textbf{Spells} \index{Brass Dragon Spells} \\
This Dragon's favorite spells are: \\
- Send \\
- Hypnotic Plot \\
- Languages

\medskip\index[Mostruario]{Young Brass Dragon} \textbf{Young Brass Dragon}

\textit{Large dragon, good chaotic}

\textbf{STRENGTH} +4

\textbf{DEXTERITY} +0

\textbf{CONSTITUTION} +3

\textbf{INTELLIGENCE} +1

\textbf{WISDOM} +0

\textbf{CHARISMA} +2

\textbf{Initiative} +1 - \textbf{Defense} 20

\textbf{Hit Points} 110 (13d10 + 39)

\textbf{Movement} 12 m, Burrow 6 m, flight 24 m

\textbf{Saving Throws} Fortitude +9, Reflex +8, Will +7

\textbf{Skills} Move Silently / Hide +3, Awareness +6, Deceive +5

\textbf{Immunity to Damage} fire

\textbf{Senses} darkvision 36 m, blind sight 9 m

\textbf{Languages} Common, Draconic

\textbf{Challenge} 6 (2,300 PX)

\textbf{Actions}

\textit{\textbf{Multiattack.}} The dragon can make three attacks: one with its bite and two with its claws.

\textit{\textbf{Claw.} Melee Weapon Attack}: +7 to hit, 1m range, one target.

\textit{Strikes:} 11 (2d6 + 4) slashing damage, 1 bleed damage.

\textit{\textbf{Bite.} Melee Weapon Attack}: +7 to hit, 3m range, one target.

\textit{Strikes:} 15 (2d10 + 4) piercing damage.

\textit{\textbf{Breath Weapon (Cooldown 5-6).}} The dragon uses one of the following breath weapons:

\textit{Fiery Breath.} The dragon exhales fire in a line 12 meters long and 1 meter wide. Each creature on the line must make a DC 14 Reflex saving throw, taking 42 (12d6) fire damage on a failed save, or half that damage on a successful one. \textit{Soporific Breath.} The dragon exhales sleep gas into a 30-foot cone. Each creature in that area must succeed at a 14 Fortitude save or fall unconscious for 5 minutes. This effect ends if the unconscious creature takes damage or someone takes an action to awaken it.

\textbf{Ecology} \\
Environment: Warm Deserts \\
Organization: Solitary \\
\textbf{Treasure}: Triple \\
\textbf{Description} \\
Excellent conversationalists, brass dragons prefer to talk rather than fight. Brass dragons make their lair near humanoid settlements, where they can hear the latest news and gossip. \\
\textbf{Spells} \index{Brass Dragon Spells} \\
This Dragon's favorite spells are: \\
- Send \\
- Hypnotic Plot \\
- Languages

\medskip\index[Mostruario]{Baby Brass Dragon} \textbf{Baby Brass Dragon}

\textit{Medium dragon, chaotic good}

\textbf{STRENGTH} +2

\textbf{DEXTERITY} +0

\textbf{CONSTITUTION} +1

\textbf{INTELLIGENCE} +0

\textbf{WISDOM} +0

\textbf{CHARISMA} +1

\textbf{Initiative} +0 - \textbf{Defense} 17

\textbf{Hit Points} 16 (3d8 + 3)

\textbf{Movement} 9 m, Burrow 5 meters, flight 18 m

\textbf{Saving Throws} Fortitude +2, Reflex +0, Will +1

\textbf{Skills} Move Silently / Hide +2, Awareness +4

\textbf{Immunity to Damage} fire

\textbf{Senses} darkvision 18 m, blind sight 3 m

\textbf{Languages} Draconic

\textbf{Challenge} 1 (200 PX)

\textbf{Actions}

\textit{\textbf{Bite.} Melee Weapon Attack}: +4 hit, 1m range, one target.

\textit{Strikes:} 7 (1d10 + 2) piercing damage.

\textit{\textbf{Breath Weapon (Cooldown 5-6).}} The dragon uses one of the following breath weapons:

\textit{Fiery Breath.} The dragon exhales fire in a line 6 meters long and 1 meter wide. Each creature on the line must make a DC 11 Reflex saving throw, taking 14 (4d6) fire damage on a failed save, or half that damage on a successful one.

\textit{Soporific Breath.} The dragon exhales sleep gas into a 5 meter cone. Each creature in that area must succeed at a Fortitude save 11 or fall unconscious for 1 minute. This effect ends if the unconscious creature takes damage or someone takes an action to awaken it.

\textbf{Ecology} \\
Environment: Warm Deserts \\
Organization: Solitary \\
\textbf{Treasure}: Triple \\
\textbf{Description} \\
Excellent conversationalists, brass dragons prefer to talk rather than fight. Brass dragons make their lair near humanoid settlements, where they can hear the latest news and gossip.


\medskip\index[Mostruario]{Ancient Copper Dragon} \textbf{Ancient Copper Dragon}

\textit{Gargantuan dragon, chaotic good}

\textbf{STRENGTH} +8

\textbf{DEXTERITY} +1

\textbf{CONSTITUTION} +7

\textbf{INTELLIGENCE} +5

\textbf{WISDOM} +3

\textbf{CHARISMA} +4

\textbf{Initiative} +5 - \textbf{Defense} 33

\textbf{Hit Points} 350 (20d20 + 140)

\textbf{Movement} 12m, climb 12m, flight 24m

\textbf{Saving Throws} Fortitude +20, Reflex +13, Will +19

\textbf{Skills} Move Silently / Hide +8, Deceive +11, Awareness +17

\textbf{Immunity to Damage} acid, weapons +1

\textbf{Senses} darkvision 36 m, blind sight 18 m

\textbf{Languages} Common, Draconic

\textbf{Challenge} 21 (33000 PX)

\textit{\textbf{Legendary Resistance (3 / Day).}} If the dragon fails a saving throw, it may choose to succeed instead.

\textbf{Actions}

\textit{\textbf{Multiattack.}} The dragon can use its Frightening Presence. Then make three attacks: one with the bite and two with the claws.

\textit{\textbf{Claw.} Melee Weapon Attack}: +30 to hit, 3m range, one target.

\textit{Hits:} 15 (2d6 + 8) slashing damage, 3 bleed damage (to a maximum of 10).

\textit{\textbf{Tail.} Melee Weapon Attack}: +30 to hit, 6m range, one target.

\textit{Strikes:} 17 (2d8 + 8) hit damage.

\textit{\textbf{Bite.} Melee Weapon Attack}: +30 to hit, 5m range, one target.

\textit{Strikes:} 19 (2d10 + 8) piercing damage.

\textit{\textbf{Dreadful Presence.}} Any creature the dragon chooses, within 36 meters of it and aware of its presence, must succeed on a DC 19 Will saving throw or be startled for 1 minute. A creature can re-roll the saving throw at the end of each of its rounds, ending the effect if successful. If the creature's saving throw is successful or the effect ends, the creature is immune to the dragon's dreadful presence for the next 24 hours.

\textit{\textbf{Breath Weapon (Cooldown 5-6).}} The dragon uses one of the following breath weapons:

\textit{Acid Breath.} The dragon exhales acid in a line 27 meters long and 3 meters wide. Each creature on the line must make a DC 22 Reflex saving throw, taking 63 (14d8) acid damage on a failed save, or half that damage on a successful one.

\textit{Slow Breath.} The dragon exhales gas into a 27-meter cone. Each creature in that area must succeed at a DC 22 Fortitude save. If it fails the saving throw, the creature cannot use its reaction, has half speed, and cannot make more than one attack during its round. Also, the creature can use one action or one bonus action, but not both. These effects last for 1 minute. The creature can re-roll the saving throw at the end of each of its rounds, ending the effect on itself if it succeeds.

\textit{\textbf{Shape Shift.}} The dragon can magically transform into a humanoid or beast whose challenge rating is equal to or less than its own, or revert to its true form. At death it returns to its true form. Any equipment he is wearing or carrying is absorbed or carried in the new form (dragon's choice).

In the new form, the dragon retains its traits, hit points, hit dice, ability to speak, skill, legendary Fortitude, lair actions, and intelligence, wisdom, and charisma scores, in addition to this action. His stats and abilities

otherwise they are replaced by those of the new shape, except Additional Actions of the new shape.

\textbf{Additional Actions}

The dragon can perform 3 additional Actions, chosen from the following options. He can only use one legendary option at a time, and only at the end of another creature's turn. The dragon recovers any additional Actions spent at the start of their round.

\textbf{Wing Attack (Costs 2 Actions).} The dragon flaps its wings. Each creature within 5 yards of the dragon must succeed on a DC 23 Reflex saving throw or take 15 (2d6 + 8) hit damage and be thrown prone. The dragon can then fly up to half of its flight movement.

\textbf{Tail Attack.} The dragon makes a tail attack.

\textbf{Spot.} The dragon makes a Wisdom (Awareness) check.

\textbf{Ecology} \\
Environment: Warm Hills \\
Organization: Solitary \\
\textbf{Treasure}: Triple \\
\textbf{Description} \\
This capricious dragon in combat tries to thwart and frustrate his enemies. \\
\textbf{Spells} \index{Copper Dragon Spells} \\
This Dragon's favorite spells are: \\
- Metamorphosis \\
- Confusion \\
- Smelly cloud


\medskip\index[Mostruario]{Adult Copper Dragon} \textbf{Adult Copper Dragon}

\textit{Huge dragon, good chaotic}

\textbf{STRENGTH} +6

\textbf{DEXTERITY} +1

\textbf{CONSTITUTION} +5

\textbf{INTELLIGENCE} +4

\textbf{WISDOM} +2

\textbf{CHARISMA} +3

\textbf{Initiative} +4 - \textbf{Defense} 25

\textbf{Hit Points} 184 (16d12 + 80)

\textbf{Movement} 12m, climb 12m, flight 24m

\textbf{Saving Throws} Fortitude +14, Reflex +10, Will +13

\textbf{Skills} Move Silently / Hide +6, Deceive +8, Awareness +12

\textbf{Immunity to Damage} acid

\textbf{Senses} darkvision 36 m, blind sight 18 m

\textbf{Languages} Common, Draconic

\textbf{Challenge} 14 (11.500 PX)

\textit{\textbf{Legendary Resistance (3 / Day).}} If the dragon fails a saving throw, it may choose to succeed instead.

\textbf{Actions}

\textit{\textbf{Multiattack.}} The dragon can use its Frightening Presence. Then make three attacks: one with the bite and two with the claws.

\textit{\textbf{Claw.} Melee Weapon Attack}: +22 to hit, 1m range, one target.

\textit{Strikes:} 13 (2d6 + 6) slashing damage, 1 bleed damage.

\textit{\textbf{Tail.} Melee Weapon Attack}: +22 to hit, range 5 meters, one target.

\textit{Strikes:} 15 (2d8 + 6) hit damage.

\textit{\textbf{Bite.} Melee Weapon Attack}: +22 to hit, 3m range, one target.

\textit{Strikes:} 17 (2d10 + 6) piercing damage.

\textit{\textbf{Frightening Presence.}} Any creature the dragon chooses, within 36 meters of it and aware of its presence, must succeed on a DC 16 Will saving throw or be startled for 1 minute. A creature can re-roll the saving throw at the end of each of its rounds, ending the effect if successful. If the creature's saving throw is successful or the effect ends, the creature is immune to the dragon's dreadful presence for the next 24 hours.

\textit{\textbf{Breath Weapon (Cooldown 5-6).}} The dragon uses one of the following breath weapons:

\textit{Acid Breath.} The dragon exhales acid in a line 18 meters long and 1 meter wide. Each creature on the line must make a DC 18 Reflex saving throw, taking 54 (12d8) acid damage on a failed save, or half that damage on a successful one.

\textit{Slow Breath.} The dragon exhales gas into a 60-foot cone. Each creature in that area must succeed at a DC 18 Fortitude saving throw. If the saving throw fails, the creature cannot use its reaction, has half speed, and cannot make more than one attack during its round. Also, the creature can use one action or one bonus action, but not both. These effects last for 1 minute. The creature can re-roll the saving throw at the end of each of its rounds, ending the effect on itself if it succeeds.

\textbf{Additional Actions}

The dragon can perform 3 additional Actions, chosen from the following options. He can only use one legendary option at a time, and only at the end of another creature's turn. The dragon recovers any additional Actions spent at the start of their round.

\textbf{Wing Attack (Costs 2 Actions).} The dragon flaps its wings. Each creature within 10 feet of the dragon must succeed on a DC 19 Reflex saving throw or take 13 (2d6 + 6) hit damage and be thrown prone. The dragon can then fly up to half of its flight movement.

\textbf{Tail Attack.} The dragon makes a tail attack.

\textbf{Spot.} The dragon makes a Wisdom (Awareness) check.

\textbf{Ecology} \\
Environment: Warm Hills \\
Organization: Solitary \\
\textbf{Treasure}: Triple \\
\textbf{Description} \\
This capricious dragon in combat tries to thwart and frustrate his enemies. \\
\textbf{Spells} \index{Copper Dragon Spells} \\
This Dragon's favorite spells are: \\
- Metamorphosis \\
- Confusion \\
- Smelly cloud


\medskip\index[Mostruario]{Young Copper Dragon} \textbf{Young Copper Dragon}

\textit{Large dragon, good chaotic}

\textbf{STRENGTH} +4

\textbf{DEXTERITY} +1

\textbf{CONSTITUTION} +3

\textbf{INTELLIGENCE} +3

\textbf{WISDOM} +1

\textbf{CHARISMA} +2

\textbf{Initiative} +3 - \textbf{Defense} 21

\textbf{Hit Points} 119 (14d10 + 42)

\textbf{Movement} 12m, climb 12m, flight 24m

\textbf{Saving Throws} Fortitude +9, Reflex +8, Will +8

\textbf{Skills} Move Silently / Hide +4, Deceive +5, Awareness +7

\textbf{Immunity to Damage} acid

\textbf{Senses} darkvision 36 m, blind sight 9 m

\textbf{Languages} Common, Draconic

\textbf{Challenge} 7 (2.900 PX)

\textbf{Actions}

\textit{\textbf{Multiattack.}} The dragon can make three attacks: one with its bite and two with its claws.

\textit{\textbf{Claw.} Melee Weapon Attack}: +10 to hit, 1m range, one target.

\textit{Strikes:} 11 (2d6 + 4) slashing damage, 1 bleed damage.

\textit{\textbf{Bite.} Melee Weapon Attack}: +10 to hit, 3m range, one target.

\textit{Strikes:} 15 (2d10 + 4) piercing damage.

\textit{\textbf{Breath Weapon (Cooldown 5-6).}} The dragon uses one of the following breath weapons:

\textit{Acid Breath.} The dragon exhales acid in a line 12 meters long and 1 meter wide. Each creature on the line must make a DC 14 Reflex saving throw, taking 40 (9d8) acid damage on a failed save, or half that damage on a successful one.

\textit{Slow Breath.} The dragon exhales gas into a 30-foot cone. Each creature in that area must succeed at a DC 14 Fortitude save. If it fails the saving throw, the creature cannot use its reaction, has half speed, and cannot make more than one attack during its round. Also, the creature can use one action or one bonus action, but not both. These effects last for 1 minute. The creature can re-roll the saving throw at the end of each of its rounds, ending the effect on itself if it succeeds.

\textbf{Ecology} \\
Environment: Warm Hills \\
Organization: Solitary \\
\textbf{Treasure}: Triple \\
\textbf{Description} \\
This capricious dragon in combat tries to thwart and frustrate his enemies. \\
\textbf{Spells} \index{Copper Dragon Spells} \\
This Dragon's favorite spells are: \\
- Metamorphosis \\
- Confusion \\
- Smelly cloud

\medskip \textbf{Copper Pup Dragon}

\textit{Medium dragon, chaotic good}

\textbf{STRENGTH} +2

\textbf{DEXTERITY} +1

\textbf{CONSTITUTION} +1

\textbf{INTELLIGENCE} +2

\textbf{WISDOM} +0

\textbf{CHARISMA} +1

\textbf{Initiative} +2 - \textbf{Defense} 17

\textbf{Hit Points} 22 (4d8 + 4)

\textbf{Movement} 9 m, climb 9 m, flight 18 m

\textbf{Saving Throws} Fortitude +2, Reflex +2, Will +0

\textbf{Skills} Move Silently / Hide +3, Awareness +4

\textbf{Immunity to Damage} acid

\textbf{Senses} darkvision 18 m, blind sight 3 m

\textbf{Languages} Draconic

\textbf{Challenge} 1 (200 PX)

\textbf{Actions}

\textit{\textbf{Bite.} Melee Weapon Attack}: +4 to hit, 1m range, one target.

\textit{Strikes:} 7 (1d10 + 2) piercing damage.

\textit{\textbf{Breath Weapon (Cooldown 5-6).}} The dragon uses one of the following breath weapons:

\textit{Acid Breath.} The dragon exhales acid in a line 6 meters long and 1 meter wide. Each creature on the line must make a DC 11 Reflex saving throw, taking 18 (4d8) acid damage on a failed save, or half that damage on a successful one.

\textit{Slow Breath.} The dragon exhales gas into a 5 meter cone. Each creature in that area must succeed at a DC 11 Fortitude save. If it fails the save, the creature cannot use its reaction, has half speed, and cannot make more than one attack during its round. Also, the creature can use one action or one bonus action, but not both. These effects last for 1 minute. The creature can re-roll the saving throw at the end of each of its rounds, ending the effect on itself if it succeeds.

\textbf{Ecology} \\
Environment: Warm Hills \\
Organization: Solitary \\
\textbf{Treasure}: Triple \\
\textbf{Description} \\
This whimsical dragon tries to hinder and frustrate his enemies during combat.

\medskip\index[Mostruario]{Tàhil} \textbf{Tàhil}

\textit{mammoth dragon}

\textbf{STRENGTH} +10

\textbf{DEXTERITY} +0

\textbf{CONSTITUTION} +10

\textbf{INTELLIGENCE} +8

\textbf{WISDOM} +8

\textbf{CHARISMA} +9

\textbf{Initiative} +8 - \textbf{Defense} 40

\textbf{Hit Points} 615 (30d20 + 300)

\textbf{Movement} 20 meters, fly 20 meters

\textbf{Saving Throws}: Fortitude +40, Reflex +30, Will +38

\textbf{Skills} all +18

\textbf{Immunity to Damage} cold, lightning, fire, acid, poison, sound, weapons +3

\textbf{Condition Immunity} fascinated, poisoned, paralyzed, fatigue, frightened

\textbf{Senses} Darkvision 60 m, True vision 40 m

\textbf{Languages} Common, Draconic

\textbf{Challenge} 30 (155,000 PX)

\textbf{Immortal on Yeru.} When Tahil's body is killed on Yeru it reforms in 3d6 days in the lair made by Calicanthus.

\textit{\textbf{Spells.}} Tahil has CM 20. Her spellcasting characteristic is Charisma, +9 to hit with spell attacks. Tahil knows the following spells:

At will: Divine Word

\textit{\textbf{Divine Nature.}} Tahil does not need air, food, drink or sleep. Spells of 5 level or lower have no effect on Tahil unless he wills it.

\textit{\textbf{Master of Dragons.}} Every non-metallic Dragon on Yeru is faithful and obedient to the will of Tahil.

\textit{\textbf{Voice of the Master.}} Tahil can talk to any chromatic dragon in Yeru, regardless of distance.

\textit{\textbf{Master's Call.}} Tàhil opens a portal and 1d2 + 1 metallic dragons of random age and color emerge. Power is usable 1 time per day.

\textit{\textbf{Legendary Resistance (5 / Day).}} If the Tahil fails a saving throw, it may choose to succeed instead.

\textit{\textbf{Multiple heads.}} Tahil has + 1d6 on saving throws against being blind, deaf, passed out. Tahil can perform up to 6 Reactions per round.

\textit{\textbf{Regeneration.}} Tahil regenerates 30 hit points at the start of his round

\textbf{Actions}

\textit{\textbf{Multiattack.}} Tahil can use his Frightening Presence or make 3 attacks (2 with claws and one with tail) or just one with bite. Claw +30, range 5 meters. Tail +30 range 8 meters. Bite +30, range 6 meters. All of Tahil's attacks are considered magical +5.

\textit{Strikes:} Claw, 24 (4d6 +10, 5 bleed damage, to a maximum of 20) slash. Tail, 28 (4d8 +10) hit. Bite 48 (8d6 +10) on critical hit of the bite cuts the creature's body in half if a Fortitude save fails at DC 30.

\textit{\textbf{Dreadful Presence}} Any creature that can see Tahil and is within 80 meters must make a Will save at DC 26 or be startled for 1 minute. Each round the creature can make the saving throw, if successful, it is immune to Tàhil's frightening presence for the next 24 hours.

\textbf{Additional Actions}

The Tahil can perform 3 additional actions, chosen from those below and one per round only at the end of another creature's round. The actions depend on the chosen head.

\textbf{Claw attack.}: +19, range 20 feet, one target. If it hits 32 (4d10 + 10, 3 from bleed) slash damage plus 14 (4d6) damage from acid (black head) or electricity (blue head) or poison (green head) or fire (red head) or cold (White head) or by Fire (Yellow head) or by Sound (Purple head)

\textbf{Black Head.}: Costs 2 legendary actions, Tàhil blows Acid into a 40 meter cone. Reflex save DC 27 either take 68 (15d8) acid damage or halve.

\textbf{Blue Head.}: Costs 2 legendary actions, Tàhil blows Electricity into a 40 meter cone. Reflex save DC 27 or take 88 (16d10) Electricity damage or halve.

\textbf{Green Head.}: Costs 2 legendary actions, Tahil blows Poison into a 30 meter cone. Reflex save DC 27 or take 77 (22d6) damage from poison or halve.

\textbf{Red Head.}: Costs 2 legendary actions, Tahil blows Fire into a 30 meter cone. Reflex save DC 27 or take 91 (26d6) Fire damage or halve.

\textbf{White Head.}: Costs 2 legendary actions, Tàhil blows Ice into a 30 meter cone. Reflex save DC 27 or take 72 (16d8) ice damage or halve.

\textbf{Purple Head.}: Costs 2 legendary actions, Tahil blows Sound into a 30 meter cone. Reflex save DC 27 or take 90 (18d8) sonic damage or halve.

\textbf{Yellow Head.}: Costs 2 legendary actions, Tahil blows hot sand into a 60 meter cone. Reflex save DC 27 or take 72 (16d8) Fire damage or halve.


\textbf{Ecology} \\
Environment: Unknown \\
Organization: Unique \\
\textbf{Treasure}: Special \\

\textbf{Description}
Tahil is the incarnate Patron of Dragons. Nothing resists its fury, madness, rage and destruction. Tàhil is a mammoth creature with 7 dragon heads, each colored in a different way, each representing a color of a Dragon. See chapter on Cosmology for the details of its history.









\medskip\index[Mostruario]{Drider} \textbf{Drider}

\textit{Large monstrosity, chaotic evil}

\textbf{STRENGTH} +3

\textbf{DEXTERITY} +3

\textbf{CONSTITUTION} +4

\textbf{INTELLIGENCE} +1

\textbf{WISDOM} +2

\textbf{CHARISMA} +1

\textbf{Initiative} +3 - \textbf{Defense} 22

\textbf{Hit Points} 123 (13d10 + 52)

\textbf{Movement} 9 m, climb 9 m

\textbf{Saving Throws} Fortitude +7, Reflex +5, Will +9

\textbf{Skills} Move Silently / Hide +9, Awareness +5

\textbf{Senses} darkvision 36 m

\textbf{Languages} Elven, Language of the Depths

\textbf{Challenge} 6 (2,300 PX)

\textit{\textbf{Walk the Web.}} The drider ignores movement restrictions caused by webs.

\textit{\textbf{Fairy lineage.}} The drider has + 1d6 on saving throws to keep from being fascinated, and magic cannot put a drider to sleep.

\textit{\textbf{Innate Spells.}} The drider's innate spellcasting characteristic is Wisdom. The drider can innately cast the following spells, without the need for material components:

At will: \textit{dancing lights}

1 / Day: \textit{luminescence, darkness}

\textit{\textbf{Climbing like Spider.}} The drider can scale difficult surfaces, including standing upside down on the ceiling, without the need for a skill check.

\textbf{Actions}

\textit{\textbf{Multiattack.}} The drider makes three longsword or longbow attacks. It can replace one of these attacks with a bite attack.



\textit{\textbf{Bite.} Melee weapon attack}: +11 to hit, range 1 yards, a creature.

\textit{Strikes:} 2 (1d4) piercing damage plus 9 (2d8) poison damage.

\textit{\textbf{Longsword.} Melee Weapon Attack}: +1 to hit, 1 m range, one target.

\textit{Strikes:} 7 (1d8 + 3) slashing damage, or 8 (1d8 + 3) slashing damage when used with two hands.

\textit{\textbf{Longbow.} Ranged weapon attack}: +11 to hit, range 45m, one target.

\textit{Strikes:} 7 (1d8 + 3) piercing damage plus 4 (1d8) poison damage.

\textbf{Ecology} \\
Environment: Any dungeon \\
Organization: Solitary, pair or group (3-8) \\
\textbf{Treasure}: double (Perfect Heavy Mace, Perfect Composite Longbow[Strength +2] with 20 Arrows, other treasure) \\
\textbf{Description} \\
Created from the body of an elf, altered and mutated through special poisons and elixirs to take on the characteristics of a giant spider, the drider is a dangerous creature. \\
Driders are sexually dimorphic. The spider-like underside of a female drider's body is shiny and graceful, often resembling the body of a black widow, while the elf's upper torso maintains its alluring curves and pretty face (with the exception of the poisonous sharp fangs). . A male drider's lower body is stocky like a tarantula, while the upper body has a lean physique and supports a hideous spider-rather than elf-like face, complete with fanged jaws.


\medskip\index[Mostruario]{Dryad} \textbf{Dryad}

\textit{Fairy mean, neutral}

\textbf{STRENGTH} +0

\textbf{DEXTERITY} +1

\textbf{CONSTITUTION} +0

\textbf{INTELLIGENCE} +2

\textbf{WISDOM} +2

\textbf{CHARISMA} +4

\textbf{Initiative} +2 - \textbf{Defense} 12 (17 with \textit{bark skin})

\textbf{Hit Points} 22 (5d8)

\textbf{Damage Vulnerability} cold iron

\textbf{Movement} 9 m

\textbf{Saving Throws} Fortitude +5, Reflex +9, Will +7

\textbf{Skills} Move Silently / Hide +5, Awareness +4

\textbf{Senses} darkvision 18 m

\textbf{Languages} Elvish, Sylvan

\textbf{Challenge} 1 (200 PX)

\textit{\textbf{Tree Walk.}} Once during her round, the dryad can use 10 feet of movement to magically enter a living tree within her reach and emerge from another living tree within 60 feet of the first tree, reappearing in an unoccupied space within 1 meter of the second tree. Both trees must be Large or larger.

\textit{\textbf{Innate Spells.}} The dryad's innate spellcasting characteristic is Charisma (DC 14 on saving throws for spells). The dryad can innately cast the following spells, without the need for material components. At will:

\textit{art of the druid}

3 / day each: \textit{beneficial berries}, \textit{get in the way} 1 / day:
\textit{pass without traces, leathery skin, club} \textit{enchanted}

\textit{\textbf{Talking with Animals and Plants.}} The dryad can communicate with beasts and plants as if they spoke the same language.

\textit{\textbf{Magic Resistance.}} The dryad has + 1d6 on saving throws against spells and other magical effects.

\textbf{Actions}

\textit{\textbf{Club.} Melee Weapon Attack}: +2 to hit (+6 to hit with \textit{stick}), range 1m, one target.

\textit{Strikes:} 2 (1d4) slam damage, or 8 (1d8 + 4) slash damage with \textit{stick}

\textit{\textbf{Fairy Charm.}} The dryad can target a humanoid or beast within 30 feet of her that she can see. If the target can see the dryad, they must succeed on a DC 14 Will saving throw or be fascinated by the magic. Fascinated creatures consider the dryad a trusted friend to be listened to and protected. Although the target is not under the dryad's control, it will interpret the dryad's requests or actions as favorably as possible.

Whenever the dryad or its allies deal damage to the target, it can re-roll the saving throw, ending the effect on success. Otherwise, the effect lasts for 24 hours or until the dryad dies, is on a different plane of existence than the target, or ends the effect with a bonus action. If the target's saving throw is successful, the target will be immune to the dryad's fairy charm for the next 24 hours.

The dryad cannot keep more than one humanoid or three beasts fascinated at a time.

\textbf{Ecology} \\
Environment: Temperate Forests \\
Organization: Solitary, pair or grove (3-8) \\
\textbf{Treasure}: standard (Perfect Longbow with 20 Arrows, Dagger, other treasure) \\
\textbf{Description} \\
Dryads are tree-sprites who love secluded woodlands away from timber-needing humanoids. Dryads' primary concern is their own survival and that of their beloved forests, and they are known to magically force travelers to help them with tasks they cannot accomplish. They are friendly to non-evil druids and rangers, recognizing their empathy or respect for nature. \\
Dryads are benevolent guardians of trees, and while not violent by nature, they can block and thwart threats to their homes or turn enemies into allies. Some keep one or more bewitched humanoids in their territory to defend it or to deflect attackers. Disabled enemies are typically dragged to the edge of the forest by dryad allies and driven out, but evil or hostile ones are killed once the fight is over.

\medskip\index[Mostruario]{Duergar} \textbf{Duergar}

\textit{Medium humanoid (dwarf), lawful evil}

\textbf{STRENGTH} +2

\textbf{DEXTERITY} +0

\textbf{CONSTITUTION} +2

\textbf{INTELLIGENCE} +0

\textbf{WISDOM} +0

\textbf{CHARISMA} -1

\textbf{Initiative} +2 - \textbf{Defense} 17 (scale armor, shield)

\textbf{Hit Points} 26 (4d8 + 8)

\textbf{Movement} 8 m

\textbf{Saving Throws} Fortitude +4, Reflex +0, Will +1

\textbf{Resistance to Damage} poison

\textbf{Senses} darkvision 36 m

\textbf{Languages} Dwarven, Language of the Depths

\textbf{Challenge} 1 (200 PX)

\textit{\textbf{Duerga Resilience.}} The duergar has + 1d6 on saving throws against poisons, spells, and illusions, as well as resisting being fascinated or paralyzed.

\textit{\textbf{Sensitivity to Light}}. While in sunlight, the duergar has -1d6 on attack rolls, as well as sight-based Wisdom (Awareness) checks.

\textbf{Actions}

\textit{\textbf{Enlarge (Recharges after 1 hour).}} For 1 minute, the duergar magically increases in size, along with anything it is carrying or wearing. While maximized, the duergar is Large, doubles the damage dice of attacks with Force-based weapons (already included in attacks), and has + 1d6 on Strength checks and Strength saving throws. If the duergar does not have enough space to become Large, it gains the maximum size allowed by the space available.

\textit{\textbf{War Pick.} Melee Weapon Attack}: +4 to hit, 1m range, one target.

\textit{Strikes:} 6 (1d8 + 2) piercing damage, or 11 (2d8 + 2) piercing damage when zoomed.

\textit{\textbf{Javelin.} Melee or Ranged Weapon Attack}: +4 hit, 1 m range, 12m range, one target. \textit{Strikes:} 5 (1d6 + 2) piercing damage or 9 (2d6 + 2) damage
perforating when enlarged.

\textit{\textbf{Invisibility (Cooldown after 1 hour).}} The duergar magically becomes invisible for a maximum of one hour (as if holding concentration for a spell) or until it attacks, casts a spell, use Zoom in or his concentration is broken. All equipment the duergar wears or carries becomes invisible with him.

\textbf{Ecology} \\
Environment: Any underground \\
Organization: solitary, group (2-5), team (6-12 plus 3 3rd level sergeants and 1 3rd-8th level leader), or clan (13-80 plus 25 \% of non-children fighters plus 1 3rd level sergeant every 5 adults, 3-6 3rd-6th level lieutenants, and 1-4 9th level captains) \\
\textbf{Treasure}: NPC gear (Chainmail, Heavy Metal Shield, Warhammer, Light Crossbow with 20 Squares, 3d6 gp, other treasure) \\
\textbf{Description} \\
Distant relatives of the dwarves, darker and more deformed, the Duergar are bad tempered creatures who hate intruders in their subterranean realms, but never more than the Dwarves. They live in community in the depths of the subsoil. They have dull gray skin, as if it were dirty with dust or ash, but this natural shade allows them to blend in better in the underground areas. They are a Race of slavers, but while forcing non-Dwarf prisoners into grueling labor, they unhesitatingly slay captured Dwarves. In combat, Duergars fire crossbows, and then switch to the Warhammer a few rounds later. If outnumbered, or if there is adequate danger (and space), a Duergar will use its enlarge ability and attack.

\medskip \index[Mostruario]{Greater Water Elemental} \textbf{Greater Water Elemental}

\textit{Huge elemental, neutral}

\textbf{STRENGTH} +6

\textbf{DEXTERITY} +3

\textbf{CONSTITUTION} +5

\textbf{INTELLIGENCE} -2

\textbf{WISDOM} +1

\textbf{CHARISMA} +0

\textbf{Initiative} +4 - \textbf{Defense} 21

\textbf{Hit Points} 158

\textbf{Movement} 9 m, swim 33 m

\textbf{Saving Throws} Fortitude +12, Reflex +12, Will +6

\textbf{Resistance to Damage} acid; non-magical weapon

\textbf{Immunity to Damage} poison

\textbf{Condition Immunity} grabbed, poisoned, entangled, paralyzed, petrified, unconscious, prone, fatigue

\textbf{Senses} darkvision 18 m

\textbf{Languages} Aquan

\textbf{Challenge} 9

\textit{\textbf{Freezing.}} If the elemental takes cold damage, it freezes partially; his movement is reduced by 20 feet until the end of his next round.

\textit{\textbf{Shape of Water.}} The elemental can enter the space of a hostile creature and stop there. It can move through a narrow space up to 3 centimeters without having to squeeze.

\textit{\textbf{Elemental Nature.}} An elemental does not need air, food, drink or sleep.

\textbf{Actions}

\textit{\textbf{Multiattack.}} The elemental makes two slam attacks.

\textit{\textbf{Slam.} Melee Weapon Attack}: +19 to hit, 3m range, one target.

\textit{Strikes:} 26 (4d8 + 10) hit damage.

\textit{\textbf{Submerge (Cooldown 4-6).}} Each creature in the elemental space must make a DC 21 Fortitude saving throw. If it fails, the target takes 25 (5d8 + 5) damage from blow. If it is Huge or smaller, the target is also grabbed (DC 19 to escape). Until the grapple is complete, the target is hampered and cannot breathe unless they are able to breathe water. If the saving throw is successful, the target is pushed out of the elemental's space.

The elemental can grab a Huge creature or up to two Large or smaller at a time. At the start of each elemental's turn, each target grabbed takes 25 (4d8 + 5) slash damage. A creature within 10 feet of the elemental can drag a creature or object out of it, taking an action to attempt a DC 19 Strength check.

\textbf{Ecology} \\
Environment: Any (Plane of Water) \\
Organization: Solitary, pair or group (3-8) \\
\textbf{Treasure}: None \\
\textbf{Description} \\
Water elementals are patient, inflexible creatures made up of living water, fresh or salt. They prefer to cover their opponents with water or drag them in to gain an advantage. \\
Like other elementals, all water elementals have unique looks and shapes. Many are wave-like-looking creatures with a vaguely humanoid face and smaller waves on the sides that act as arms. Another common form is that of some aquatic creature, such as a shark or octopus, but made entirely of water. \\
A large water elemental is 30 feet tall and weighs 10000kg.

\medskip \index[Mostruario]{Water Elemental} \textbf{Water Elemental}

\textit{Large elemental, neutral}

\textbf{STRENGTH} +4

\textbf{DEXTERITY} +2

\textbf{CONSTITUTION} +4

\textbf{INTELLIGENCE} -3

\textbf{WISDOM} +0

\textbf{CHARISMA} -1

\textbf{Initiative} +4 - \textbf{Defense} 17

\textbf{Hit Points} 114 (12d10 + 48)

\textbf{Movement} 9 m, swim 27 m

\textbf{Saving Throws} Fortitude +9, Reflex +8, Will +2

\textbf{Resistance to Damage} acid; non-magical weapon

\textbf{Immunity to Damage} poison

\textbf{Condition Immunity} grabbed, poisoned, entangled, paralyzed, petrified, unconscious, prone, fatigue

\textbf{Senses} darkvision 18 m

\textbf{Languages} Aquan

\textbf{Challenge} 5 (1,800 PX)

\textit{\textbf{Freezing.}} If the elemental takes cold damage, it freezes partially; his movement is reduced by 20 feet until the end of his next round.

\textit{\textbf{Shape of Water.}} The elemental can enter the space of a hostile creature and stop there. It can move through a narrow space up to 3 centimeters without having to squeeze.

\textit{\textbf{Elemental Nature.}} An elemental does not need air,
food, drink, or sleep.

\textbf{Actions}

\textit{\textbf{Multiattack.}} The elemental makes two slam attacks.

\textit{\textbf{Slam.} Melee Weapon Attack}: +11 to hit, 1m range, one target.

\textit{Strikes:} 13 (2d8 + 4) hit damage.

\textit{\textbf{Submerge (Cooldown 4-6).}} Each creature in the elemental space must make a DC 15 Fortitude saving throw. If it fails, the target takes 13 (2d8 + 4) damage from blow. If it is Large or smaller, the target is also grabbed (DC 14 to flee). Until the grapple is complete, the target is hampered and cannot breathe unless they are able to breathe water. If the saving throw is successful, the target is pushed out of space
of the elemental.

The elemental can grab one Large creature or up to two Medium or smaller creatures at a time. At the start of each elemental's turn, each target grabbed takes 13 (2d8 + 4) slash damage. A creature within 1 meter of the elemental can drag a creature or object out of it, taking an action to attempt a DC 14 Strength check.

\textbf{Ecology} \\
Environment: Any (Plane of Water) \\
Organization: Solitary, pair or group (3-8) \\
\textbf{Treasure}: None \\
\textbf{Description} \\
Water elementals are patient, inflexible creatures made up of living water, fresh or salt. They prefer to cover their opponents with water or drag them in to gain an advantage. \\
Like other elementals, all water elementals have unique looks and shapes. Many are wave-like-looking creatures with a vaguely humanoid face and smaller waves on the sides that act as arms. Another common form is that of some aquatic creature, such as a shark or octopus, but made entirely of water. \\
A large water elemental is 4.8 meters tall and weighs 1125kg.


\medskip\index[Mostruario]{Lesser Water Elemental} \textbf{Lesser Water Elemental}

\textit{Middle elemental, neutral}

\textbf{STRENGTH} +2

\textbf{DEXTERITY} +1

\textbf{CONSTITUTION} +2

\textbf{INTELLIGENCE} -3

\textbf{WISDOM} +0

\textbf{CHARISMA} -1

\textbf{Initiative} +4 - \textbf{Defense} 15

\textbf{Hit Points} 16 (2d8 + 4)

\textbf{Movement} 9 m, swim 27 m

\textbf{Saving Throws} Fortitude +3, Reflex +4, Will +0

\textbf{Resistance to Damage} acid; non-magical weapon

\textbf{Immunity to Damage} poison

\textbf{Condition Immunity} grabbed, poisoned, entangled, paralyzed, petrified, unconscious, prone, fatigue

\textbf{Senses} darkvision 18 m

\textbf{Languages} Aquan

\textbf{Challenge} 2

\textit{\textbf{Freezing.}} If the elemental takes cold damage, it freezes partially; his movement is reduced by 20 feet until the end of his next round.

\textit{\textbf{Shape of Water.}} The elemental can enter the space of a hostile creature and stop there. It can move through a narrow space up to 3 centimeters without having to squeeze.

\textit{\textbf{Elemental Nature.}} An elemental does not need air, food, drink or sleep.

\textbf{Actions}

\textit{\textbf{Multiattack.}} The elemental makes two slam attacks.

\textit{\textbf{Slam.} Melee Weapon Attack}: +5 to hit, 1m range, one target.

\textit{Strikes:} 6 (1d6 + 2) hit damage.

\textit{\textbf{Submerge (Cooldown 4-6).}} Each creature in the elemental space must make a DC 13 Fortitude saving throw. If it fails, the target takes 8 (2d4 + 4) damage from blow. If it is Medium or smaller, the target is also grabbed (DC 12 to flee). Until the grapple is complete, the target is hampered and cannot breathe unless they are able to breathe water. If the saving throw is successful, the target is pushed out of the elemental's space.

The elemental can grab one Medium creature or up to two Small creatures at a time. At the start of each elemental's turn, each target grabbed takes 8 (2d4 + 4) slash damage. A creature within 1 meter of the elemental can drag a creature or object out of it, taking an action to attempt a DC 12 Strength check.

\textbf{Ecology} \\
Environment: Any (Plane of Water) \\
Organization: Solitary, pair or group (3-8) \\
\textbf{Treasure}: None \\
\textbf{Description} \\
Water elementals are patient, inflexible creatures made up of living water, fresh or salt. They prefer to cover their opponents with water or drag them in to gain an advantage. \\
Like other elementals, all water elementals have unique looks and shapes. Many are wave-like-looking creatures with a vaguely humanoid face and smaller waves on the sides that act as arms. Another common form is that of some aquatic creature, such as a shark or octopus, but made entirely of water. \\
A large water elemental is 1.8 meters tall and weighs 80kg.

\medskip\index[Mostruario]{Air Elemental} \textbf{Air Elemental}

\textit{Large elemental, neutral}

\textbf{STRENGTH} +2

\textbf{DEXTERITY} +5

\textbf{CONSTITUTION} +2

\textbf{INTELLIGENCE} -2

\textbf{WISDOM} +0

\textbf{CHARISMA} -2

\textbf{Initiative} +5 - \textbf{Defense} 18

\textbf{Hit Points} 90 (12d10 + 24)

\textbf{Movement} 0m, fly 27m (float)

\textbf{Saving Throws} Fortitude +9, Reflex +13, Will +2

\textbf{Resistance to Damage} lightning, sound; non-magical weapon

\textbf{Immunity to Damage} poison

\textbf{Condition Immunity} grabbed, poisoned, entangled, paralyzed, petrified, unconscious, prone, fatigue

\textbf{Senses} darkvision 18 m

\textbf{Languages} Auran

\textbf{Challenge} 5 (1,800 PX)

\textit{\textbf{Form of Air.}} The elemental can enter the space of a hostile creature and stop there. It can move through a narrow space up to 3 centimeters without having to squeeze.

\textit{\textbf{Elemental Nature.}} An elemental does not need air, food, drink, or sleep.

\textbf{Actions}

\textit{\textbf{Multiattack.}} The elemental makes two slam attacks.

\textit{\textbf{Slam.} Melee Weapon Attack}: +8 to hit, 1m range, one target.

\textit{Strikes:} 14 (2d8 + 5) hit damage.

\textit{\textbf{Whirlwind (Cooldown 4-6).}} Each creature in the elemental space must make a DC 13 Fortitude save. If it fails, the target takes 15 (3d8 + 2) damage from hit and is thrown 20 feet away from the elemental in a random direction and fall prone. If a thrown target hits an object, such as a wall or floor, it takes 3 (1d6) slam damage for every 10 feet it was thrown. If the target is cast at another creature, that creature must succeed on a DC 13 Reflex saving throw or take the same damage and fall prone.

If the saving throw is successful, the target takes half the hit damage and is not thrown away or prone.

\textbf{Ecology} \\
Environment Plane of Air \\
Organization: Solitary, pair or group (3-8) \\
\textbf{Treasure}: None \\
\textbf{Description} \\
Air elementals are swift flying creatures made of air. Primitive and territorial, they dislike being summoned or controlled by mortals, and prefer to spend their time on the Plane of Air, flying across the infinite sky. \\
Although all air elementals of the same size have the same stats, the exact appearance of each varies greatly from individual to individual: one may appear as an animated vortex of wind and smoke, while another as a smoke-like creature. a bird with sparkling eyes and wings of wind. \\
An air elemental prefers to attack creatures that fly, not only because of the advantages it has from its mastery of the air, but also because it hates touching the ground. An air elemental can move underwater, and although it runs no risk of drowning, it has no Swim degrees and loses much of its mobility and speed under water. \\
A Great Air elemental is 4.8m tall and weighs 2kg.

\medskip\index[Mostruario]{Fire Elemental} \textbf{Fire Elemental}

\textit{Large elemental, neutral}

\textbf{STRENGTH} +0

\textbf{DEXTERITY} +3

\textbf{CONSTITUTION} +3

\textbf{INTELLIGENCE} -2

\textbf{WISDOM} +0

\textbf{CHARISMA} -2

\textbf{Initiative} +3 - \textbf{Defense} 16

\textbf{Hit Points} 102 (12d10 + 36)

\textbf{Movement} 15 m

\textbf{Saving Throws} Fortitude +8, Reflex +11, Will +4

\textbf{Damage Resistances} from a non-magical weapon

\textbf{Immunity to Damage} fire, poison

\textbf{Condition Immunity} grabbed, poisoned, entangled, paralyzed, petrified, prone, unconscious, fatigue

\textbf{Senses} darkvision 18 m

\textbf{Languages} Ignan

\textbf{Challenge} 5 (1.800 PX)

\textit{\textbf{Form of Fire.}} The elemental can move through a space up to 3 centimeters wide without squeezing. A creature that contacts or hits the elemental with a melee attack while within 1 meter of it takes 5 (1d10) fire damage. Additionally, the elemental can enter a hostile creature's space and stop there. The first time it enters a creature's space on a turn, the creature takes 5 (1d10) fire damage and catches fire; until someone takes an action to put out the flames, the creature will take 5 (1d10) fire damage at the start of each of its rounds.

\textit{\textbf{Illumination.}} The elemental emits bright light within a 30-foot radius and dim light for an additional 30 feet.

\textit{\textbf{Elemental Nature.}} An elemental does not need air, food, drink, or sleep.

\textit{\textbf{Susceptibility to Water.}} The elemental suffers 1 cold damage for every 1 meter it moves in water or for every 4 liters of water sprayed on it.

\textbf{Actions}

\textit{\textbf{Multiattack.}} The elemental makes two touch attacks.

\textit{\textbf{Contact.} Melee Weapon Attack}: +7 to hit, 1m range, one target.

\textit{Strikes:} 10 (2d8 + 5) fire damage. If the target is a flammable creature or object, it catches fire. Until a creature takes an action to put out the flames, the creature will take 5 (1d10) fire damage at the start of each of its rounds.

\textbf{Ecology}
Environment any (Plane of Fire) \\
Organization: Solitary, pair or group (3-8) \\
\textbf{Treasure}: None \\
\textbf{Description} \\
Fire elementals are fast and cruel creatures made of living flames. They enjoy scaring those weaker than themselves, and they terrify any creature they can set fire to. A fire elemental cannot enter water or any non-flammable liquid. A body of water is an impenetrable barrier unless the elemental can climb over or jump over it, or it is covered with flammable material (such as a layer of oil). \\
Fire elementals have a variable aspect; they typically manifest in the form of serpentine coils made of smoke and flames, but some fire elementals take on more similar appearances to humans, demons, or other monsters to increase terror when they suddenly appear. The body of a fire elemental appears to be made of semi-stable flames or puffs of sparks, smoke or ash. \\

A large fire elemental is 4.8 meters tall.

\medskip\index[Mostruario]{Earth Elemental} \textbf{Earth Elemental}

\textit{Large elemental, neutral}

\textbf{STRENGTH} +5

\textbf{DEXTERITY} -1

\textbf{CONSTITUTION} +5

\textbf{INTELLIGENCE} -3

\textbf{WISDOM} +0

\textbf{CHARISMA} -3

\textbf{Initiative} -1 - \textbf{Defense} 20

\textbf{Hit Points} 126 (12d10 + 60)

\textbf{Movement} 9 m, Burrow 9 m

\textbf{Saving Throws} Fortitude +9, Reflex +1, Will +6

\textbf{Damage Vulnerability} sound

\textbf{Damage Resistances} from a non-magical weapon

\textbf{Immunity to Damage} poison

\textbf{Condition Immunity} poisoned, paralyzed, petrified, prone, unconscious, fatigue,

\textbf{Senses} telluric perception 18 m, darkvision 18 m

\textbf{Languages} Terran

\textbf{Challenge} 5 (1.800 PX)

\textit{\textbf{Siege Monster.}} The elemental deals double damage to objects and structures.

\textit{\textbf{Elemental Nature.}} An elemental does not need air, food, drink, or sleep.

\textit{\textbf{Earth Glide.}} The elemental can dig through nonmagical and unworked earth and stone. When it does, the elemental does not disturb the material it displaces.
\textbf{Actions}

\textit{\textbf{Multiattack.}} The elemental makes two slam attacks.

\textit{\textbf{Slam.} Melee Weapon Attack}: +12 to hit, 3m range, one target.

\textit{Strikes:} 14 (2d8 + 5) hit damage.

\textbf{Ecology}
Environment: Any (Plane of Earth) \\
Organization: Solitary, pair or group (3-8) \\
\textbf{Treasure}: None \\
\textbf{Description} \\
Earth elementals are slow, stubborn creatures made of stone or earth. When completely still they are indistinguishable from a pile of stones or a small hill. \\

When an earth elemental sets off heavily on the move, its outward appearance can vary, even though its stats remain identical to those of similarly sized peers. Earth elementals are mostly made of rock, earth, or crystal, with sparkling gems as eyes. The larger ones have the appearance of stone humanoids. Tufts of vegetation often grow on the soil which forms part of the body of an earth elemental. \\

A large earth elemental is 4.8 meters tall and weighs 3,000 kg.


\medskip\index[Mostruario]{Ettercap} \textbf{Ettercap}

\textit{Medium monstrosity, neutral evil}

\textbf{STRENGTH} +2

\textbf{DEXTERITY} +2

\textbf{CONSTITUTION} +1

\textbf{INTELLIGENCE} -2

\textbf{WISDOM} +1

\textbf{CHARISMA} 8 (-2)

\textbf{Initiative} +2 - \textbf{Defense} 14

\textbf{Hit Points} 44 (8d8 + 8)

\textbf{Movement} 9 m, climb 9 m

\textbf{Saving Throws} Fortitude +6, Reflex +4, Will +6

\textbf{Skills} Move Silently / Hide +4, Awareness +3, Survival +3

\textbf{Senses} darkvision 18 m

\textbf{Languages} -

\textbf{Challenge} 2 (450 PX)

\textit{\textbf{Walking on the Web.}} The electercap ignores movement restrictions caused by webs.

\textit{\textbf{Climbing like Spider.}} The Eertercap can scale difficult surfaces, including standing upside down on the ceiling, without the need for an ability check.

\textit{\textbf{Sense of the Web.}} While in contact with a web, the etercap knows the exact location of any other creature in contact with the same web.

\textbf{Actions}

\textit{\textbf{Multiattack.}} The etercap makes two attacks: one with its bite and one with its claws

\textit{\textbf{Claws.} Melee Weapon Attack}: +4 to hit, 1m range, one target.

\textit{Strikes:} 7 (2d4 + 2) slashing damage, 1 bleed damage.

\textit{\textbf{Bite.} Melee Weapon Attack}: +4 to hit, 1m range, one target.

\textit{Strikes:} 6 (1d8 + 2) piercing damage plus 4 (1d8) poison damage. The target must succeed on a DC 11 Fortitude save or be poisoned for 1 minute. The creature can re-roll the saving throw at the end of each of its rounds, ending the effect if it succeeds at the saving throw.

\textit{\textbf{Web (Cooldown 5-6).} Ranged weapon attack}: +4 to hit, range 9m, a Large or smaller creature. \textit{Strikes:} The creature is hampered by the web. As an action, the entangled creature can make a DC 11 Strength check, breaking free from the canvas if successful. The effect ends if the canvas is destroyed. The canvas has Defense 10, 5 hit points, vulnerability to fire damage, and immunity to hit and poison damage.

\textbf{Ecology} \\
Environment: Temperate Forests \\
Organization: solitary, pair or nest (3-6 plus 2-8 giant spiders) \\
\textbf{Treasure}: Standard \\
\textbf{Description} \\
Etcaps are usually 1.8 meters tall and weigh around 100 kg. They are solitary and rarely unite with others of their race except for mating. When grouped, they tend to attract various species of spiders, forming a strange combination of ettercaps and arachnids. \\
Etcaps are known for building cunning traps made of cobwebs and other natural materials, which they use to catch prey. They build web shelters, among the highest branches trees away from other terrestrial predators, and use monstrous spiders as lookouts and guardians. \\
Etcaps are not brave, but their traps often prevent the enemy from drawing their weapons. An ettercap attacks with claws and poisonous bites. He generally avoids melee with opponents who can still move and flees if they break free.


\medskip\index[Mostruario]{Ettin} \textbf{Ettin}

\textit{Large giant, chaotic evil}

\textbf{STRENGTH} +5

\textbf{DEXTERITY} -1

\textbf{CONSTITUTION} +3

\textbf{INTELLIGENCE} -2

\textbf{WISDOM} +0

\textbf{CHARISMA} -1

\textbf{Initiative} -1 - \textbf{Defense} 14

\textbf{Hit Points} 85 (10d10 + 30)

\textbf{Movement} 12 m

\textbf{Saving Throws} Fortitude +9, Reflex +2, Will +5

\textbf{Skills} Awareness +4

\textbf{Languages} Giant, Goblinoid

\textbf{Challenge} 4 (1,100 PX)

\textit{\textbf{Two Heads.}} The ettin has + 1d6 on Wisdom (Mindfulness) and saving throws checks against conditions blinded, fascinated, deafened, unconscious, frightened and stunned.

\textit{\textbf{Wake.}} When one of the two heads of theettin is asleep, the other is awake.

\textbf{Actions}

\textit{\textbf{Multiattack.}} The ettin makes two attacks: one with the battleaxe and one with the spiked club.

\textit{\textbf{Battle Ax.} Melee Weapon Attack}: +11 to hit, 1 m range, one target.

\textit{Strikes:} 14 (2d8 + 5) slashing damage.

\textit{\textbf{Spiked Mace.} Melee Weapon Attack}: +11 to hit, 1m range, one target.

\textit{Strikes:} 14 (2d8 + 5) piercing damage.

\textbf{Ecology} \\
Environment: Cold hills \\
Organization: Solitary, pair, group (3-6), troop (1-2 plus 1-2 Brown Bears, gang (3-6 plus 1-2 Brown Bears) or colony (3-6 plus 1-2 Brown Bears and 7-12 Orcs, or 9-16 Goblins) \\
\textbf{Treasure}: Standard (Leather Armor, 2 Light Flails, 4 Javelins, other treasure) \\
\textbf{Description} \\
Ettins, or two-headed giants, are malevolent and unpredictable nocturnal hunters. The two heads grant him unparalleled powers of perception, making him excellent guardians. \\
The ettins look like Hill Giants Stone Giants, but the fanged face betrays an orc lineage. They have pinkish brown skin and never wash unless forced to, which makes them so dirty and grimy that their skin looks thick and gray. \\
The adults stand 3.9 meters tall and weigh 2,600 kg. Ettins live about 75 years. \\
Ettins have no language of their own but speak a mixed lingo of Giant, Goblin and Orcs. Creatures that speak any of these languages can communicate with an ettin by making an Intelligence check with DC 15. The check is made once for each piece of information; if the other creature speaks two of these languages the DC is 10, while for someone who speaks all three it is 5. \\
Although ettins are not very intelligent, they are cunning warriors. They prefer to ambush their victims rather than engage them in combat, but once the battle has begun, an ettin fights furiously to the death of the enemy. \\
Ettins are solitary creatures, settling in the safety of rocky quarries and hollows, often surrounded by holes and ditches, and sometimes keeping cave bears as pets or guardians. \\
A particularly powerful ettin can attract a group of followers, especially Goblins or Orcs. However, these gatherings are more of an exception than anything else, and seldom last long, with ettin individualists going their way as opportunities for plunder and robbery diminish or if the leader is killed. \\
They generally form reproductive pairs to raise their offspring for only short periods before each resuming their own path. Young ettins mature quickly, reaching adult size within a year, thus being able to provide for themselves.

\medskip\index[Mostruario]{Ghost} \textbf{Ghost}

\textit{Medium undead, any trait}

\textbf{STRENGTH} -2

\textbf{DEXTERITY} +1

\textbf{CONSTITUTION} +0

\textbf{INTELLIGENCE} +0

\textbf{WISDOM} +1

\textbf{CHARISMA} +3

\textbf{Initiative} +1 - \textbf{Defense} 13

\textbf{Hit Points} 45 (10d8)

\textbf{Movement} 0m, flight 12m (floats)

\textbf{Saving Throws} Fortitude +7, Reflex +6, Will +7

\textbf{Resistance to Damage} acid, lightning, fire, sound; smash, piercing, cutting edge of non-magical attacks

\textbf{Immunity to Damage} cold, Void, poison

\textbf{Condition Immunity} fascinated, grabbed, poisoned, entangled, paralyzed, petrified, prone, fatigue, frightened

\textbf{Senses} darkvision 18 m

\textbf{Languages} any language known in life

\textbf{Challenge} 4 (1.100 PX)

\textit{\textbf{Incorporeal Movement.}} The ghost can traverse other creatures and objects as if they were hindering terrain. It takes 5 (1d10) force damage if it ends its round inside an object.

\textit{\textbf{Undead Nature.}} The ghost does not need air, food, drink or sleep.

\textit{\textbf{Ethereal Sight.}} The ghost can see 60 feet in the Ethereal Plane when on the Material Plane, and vice versa.

\textbf{Actions}

\textit{\textbf{Wither Touch.} Melee Weapon Attack}: +6 to hit, 1m range, one target.

\textit{Strikes:} 17 (4d6 + 3) Void damage. The target must make a DC 15 Fortitude save or become fatigued.

\textit{\textbf{Ethereality.}} The ghost enters the Ethereal Plane from the Material Plane, or vice versa. It is visible on the Material Plane while in the Ethereal Edge, and vice versa, but cannot interact with anything on the other plane.

\textit{\textbf{Possession (Cooldown 6).}} A humanoid, within 1 meter and visible to the ghost, must succeed on a DC 13 Will saving throw or be possessed by the ghost; the ghost then disappears, and the target is incapacitated and loses control of his body. The ghost now controls the body but does not deprive the target of its awareness. The ghost cannot be the target of attacks, spells, or other effects, except those that turn the undead, and retains its traits, intelligence, wisdom, charisma, and immunity to being fascinated and frightened. It otherwise uses the target's stats otherwise, but does not gain access to the target's knowledge and skills.

The possession lasts until the body drops to 0 hit points, the ghost ends it with a bonus action, or the ghost is chased or expelled by an effect such as the spell \textit{dispel good and evil}. When the possession ends, the ghost reappears in an unoccupied space within 1 meter of the body. The target is immune to this ghost's possession for 24 hours after a successful saving throw or at the end of the possession.

\textit{\textbf{Horrifying Face.}} Any creature that is not undead, within 60 feet of the ghost and that can see it, must succeed on a DC 13 Will saving throw or be startled for 1 minute. If the saving throw fails by 5 or more, the target also ages by 1d4 x 10 years. A frightened target can re-roll the saving throw at the end of each of their rounds, ending the effect for themselves if they succeed at the saving throw. If the target's saving throw is successful and the effect ends for him, the target is immune to the ghost's horrifying face for the next 24 hours. The aging effect can be reversed with the \textit{restore greater} spell, but only if performed within 24 of the aging effect.

\textbf{Ecology}
Environment: any \\
Organization: solitary \\
\textbf{Treasure}: NPC gear \\
\textbf{Description} \\
When a soul is not allowed rest due to some serious injustice, true or presumed, it sometimes comes back as a ghost. These beings are eternally distressed, devoid of substance and unable to put things right. Although ghosts can have any trait, many cling to the world of the living with a strong sense of hatred and anger, and as a result become evil; even a good creature after death can become a hateful and cruel ghost. \\

More than other monsters, the ghost must have a well-defined background. Why did this character become a ghost? What legends surround it? An encounter with a ghost should never happen accidentally - there are plenty of other incorporeal undead, like Wraith and Specters, for that. A proper encounter with a ghost should take place in a scene at the height of a long period of tension built with lesser servants or manifestations of undead spirits. The ghost example above depicts a human princess murdered by an unfaithful lover; after a confrontation, he bound her with chains and threw her into the castle well, where she drowned. The ghost's abilities were selected based on background, showing how a powerful antagonist can be created. Applying the archetype to creatures with own levels and therefore Skills or with significant racial abilities can create much more powerful ghosts. \\

When a ghost is created, it obtains "copies" of objects that it valued in life (provided the originals are not in the possession of other creatures). Equipment works normally for the ghost but passes through material objects or creatures. A weapon +1 or higher, however, can damage material creatures, but such attacks deal half the damage (50 \%) unless it is a phantom touch weapon. A ghost can only use shields and armor if they have the ghost touch ability. \\

The original objects are left behind, just like the physical remains of the ghost. If another creature holds the original, the incorporeal copy vanishes. This loss inevitably infuriates the ghost, which stops at nothing to return the object to its original place (and regain its use).


\medskip\index[Mostruario]{Bubbling Maw} \textbf{Bubbling Maw}

\textit{Medium aberration, neutral}

\textbf{STRENGTH} +0

\textbf{DEXTERITY} -1

\textbf{CONSTITUTION} +3

\textbf{INTELLIGENCE} -4

\textbf{WISDOM} +0

\textbf{CHARISMA} -2

\textbf{Initiative} -1 - \textbf{Defense} 10

\textbf{Hit Points} 67 (9d8 + 27)

\textbf{Movement} 3m, swim 3m

\textbf{Saving Throws} Fortitude +8, Reflex +4, Will +5

\textbf{Condition Immunity} prone

\textbf{Senses} darkvision 18 m

\textbf{Languages} -

\textbf{Challenge} 2 (450 PX)

\textit{\textbf{Gurgle.}} As long as the maw is able to see a creature and is not incapacitated, it utters incoherent sentences. Any creature that starts its round within 20 feet of its mouth and can hear its gurgling must make a DC 10 Will save. If it fails, the creature cannot react until the start of its next round and rolls a d8. to determine what it will do during its round. From 1 to 4, the creature does nothing. On a 5 or 6, the creature takes no action or bonus action and uses all of its movement to move in a randomly determined direction. On a 7 or 8, the creature makes a melee attack against a randomly determined creature within its range or does nothing if it can't make such an attack.

\textit{\textbf{Aberrant Terrain.}} Terrain within a 10-foot radius around the maw is considered hindering terrain. Any creature that starts its round in that area must succeed on a DC 10 Fortitude save or have its movement reduced to 0 until the start of its next round.

\textbf{Actions}

\textit{\textbf{Multiattack.}} The gurgling maw makes a bite attack and, if it can, Blinding Spit.

\textit{\textbf{Bite.} Melee weapon attack}: +3 to hit, range 1 yards, a creature.

\textit{Strikes:} 17 (5d6) piercing damage. If the target is Medium or smaller, he must succeed on a DC 10 Fortitude save or be thrown prone. If the target is killed by this damage, it is absorbed by the maw.

\textit{\textbf{Blinding Spit (Cooldown 5-6).}} The maw spits out a chemical globe at a visible point within 5 meters of it. The globe explodes on impact in a blinding flash of light. Any creature within 1 meter of the lightning bolt must succeed on a DC 13 Reflex saving throw or be blinded until the next round of the maw ends.

\textbf{Ecology} \\
Environment any dungeon \\
Organization: Solitary \\
\textbf{Treasure}: Standard \\
\textbf{Description} \\
Disgusting, sickening, and hungry - these are the only words that aptly describe the gurgling maw. Hideous beasts that lurk in caves, sewers and nightmares, the jaws have no other social, ecological, or religious sense other than their ability to drive those who hear them insane. Some scholars believe that the gurgling maw is a smaller variant of the much more dangerous shoggoth, while others theorize that it is a punishment by some powerful entity or deity inflicted on those who offended it.

\medskip\index[Mostruario]{Phoenix} \textbf{Phoenix}

\textit{Gargantuan celestial, Courageous, Protective, Good}

\textbf{STRENGTH} +8

\textbf{DEXTERITY} +6

\textbf{CONSTITUTION} +5

\textbf{INTELLIGENCE} +5

\textbf{WISDOM} +6

\textbf{CHARISMA} +6

\textbf{Initiative} +11 - \textbf{Defense} 28

\textbf{Hit Points} 210 (20d10 + 100)

\textbf{Damage Vulnerability} magical cold

\textbf{Movement} 9m, fly 27m (good)

\textbf{Saving Throws} Fortitude +17, Reflex +19, Will +14

\textbf{Immunity to Damage} Fire, Light, Poison, Weapons +1

\textbf{Condition Immunity} grabbed, poisoned, entangled, paralyzed, petrified, prone, unconscious, fatigue, bleeding

\textbf{Regeneration} a Phoenix regenerates 10 hit points at the start of each of its rounds

\textbf{Senses} Darkvision 60 ft., Low-light vision 60 ft

\textbf{Languages} Auran, Celestial, Common, Ignan

\textbf{Challenge} 15 (13000 PX)

\textit{\textbf{Awareness of Light.}} The Phoenix always has the following spells active \textit{Detect Magical, Detect Diseases and Poisons, See Invisibility}

\textit{\textbf{Innate Spells.}} The spellcasting trait of the Phoenix is Charisma. The Phoenix can innately cast the following spells, without the need for material components:

At will: \textit{Cure Critical Wounds, Dispel Magic, Everlasting Flame, Remove Curse, Polymorph (only in humanoids)}

3 / day: \textit{Mass cure critical wounds, heal, wall of fire, greater restoration, firestorm}

1 time: \textit{Resurrection} the Phoenix by sacrificing its life permanently can bring a creature back to life.

\textbf{Actions}

\textit{\textbf{Multiattack.}} The Phoenix can attack with two claws and bite

\textit{\textbf{Bite.} Melee weapon attack}: +23 to hit, range 6 yards, a creature.

\textit{Strikes:} 19 piercing damage (2d8 + 8 + 1d6 from Light)

\textit{\textbf{Claw.} Melee weapon attack}: +23 to hit, range 6 yards, a creature.

\textit{Strikes:} 17 slash damage (2d6 + 8 + 1d6 from Light)

\textbf{Special abilities}

\textit{\textbf{Rebirth}}

A slain Phoenix is reduced to a 3 cubic meter bonfire where a phoenix egg lies in the center. After 1d4 + 4 rounds this egg hatches and becomes a perfectly healthy phoenix. The only way to avoid rebirth is to remove the egg from the bonfire (20d6 light damage) or use a disintegrate spell on the egg.
A Phoenix can resurrect this way once a year, if it dies before this time has elapsed, death is final. Killing a Phoenix unleashes the wrath of the Pupils of Light and the Knights of Sumkjr.

\textit{\textbf{Wings of flame}}

The Phoenix can turn its feathers into flame as a free Reaction Action. These feathers deal 1d6 points of fire damage + 1d6 points of light damage to all creatures within 20 feet at the start of its round.

\textbf{Ecology} \\
Environment: Deserts and hot hills \\
Organization: Solitary \\
\textbf{Treasure}: Standard \\
\textbf{Description} \\
Legend has it that the Phoenicians are Ljust's pet birds, they are certainly majestic and beautiful creatures and emanate a Light similar to that of the Patroness of Genesis. The movement of their wings produces no noise while their voice is singing. The phoenix is a legendary bird of fire and light that usually lives in deserts. They are very intelligent and wise creatures and sometimes using their morph ability they travel to cities where they help those who fight against the dark.

\subsection{Mushrooms}

\medskip\index[Mostruario]{Screeching Mushroom} \textbf{Screeching Mushroom}

\textit{Average plan, misaligned}

\textbf{STRENGTH} -5

\textbf{DEXTERITY} -5

\textbf{CONSTITUTION} +0

\textbf{INTELLIGENCE} -5

\textbf{WISDOM} -4

\textbf{CHARISMA} -5

\textbf{Initiative} -5 - \textbf{Defense} 6

\textbf{Hit Points} 13 (3d8)

\textbf{Movement} 0 m

\textbf{Saving Throws}: Fortitude -3, Reflex +3, Will -4

\textbf{Condition Immunity} blinded, deafened, frightened

\textbf{Senses} blind sight 9 m (blind beyond this radius)

\textbf{Languages} -

\textbf{Challenge} 0 (10 PX)

\textit{\textbf{False Appearance.}} While the grating mushroom remains immobile, it is indistinguishable from a normal mushroom.

\textbf{Actions}

\textit{\textbf{Scream.}} When a bright light or creature is within 30 feet of the screeching mushroom, it emits an audible screech up to 90 meters away. The screeching mushroom continues to scream until the source of the disturbance has moved out of range and for 1d4 more turns thereafter, that is, until the hat is deflated.

\textbf{Ecology} \\
Environment: Any underground \\
Organization: Solitary, pair or spot (3-12) \\
\textbf{Treasure}: Accidental \\
\textbf{Description} \\
A grating mushroom is about 50 cm tall, with a broad brown cap. Once the scream is heard, the hat deflates.

It is told of Duergar cooks specialized in cooking these mushrooms in superfine dishes. The best also manage not to deflate the hat.

\medskip\index[Mostruario]{Violet Mushroom} \textbf{Violet Mushroom}

\textit{Average plan, misaligned}

\textbf{STRENGTH} -4

\textbf{DEXTERITY} -5

\textbf{CONSTITUTION} +0

\textbf{INTELLIGENCE} -5

\textbf{WISDOM} -4

\textbf{CHARISMA} -5

\textbf{Initiative} -5 - \textbf{Defense} 6

\textbf{Hit Points} 18 (4d8)

\textbf{Movement} 2 m

\textbf{Saving Throws}: Fortitude -3, Reflex -3, Will -3

\textbf{Condition Immunity} blinded, deafened, frightened

\textbf{Senses} blind sight 9 m (blind beyond this radius)

\textbf{Languages} -

\textbf{Challenge} 1/4 (50 PX)

\textit{\textbf{False Appearance.}} While the violet mushroom remains immobile, it is indistinguishable from a normal mushroom.

\textbf{Actions}

\textit{\textbf{Multiattack.}} The mushroom makes 1d4 attacks with Putrid Contact.

\textit{\textbf{Rotten Contact.} Melee Weapon Attack}: +2 to hit, 3m range, one target.

\textit{Strikes:} 4 (1d8) Void damage.

\textbf{Ecology} \\
Environment: Any dungeon \\
Organization: Solitary, pair or spot (3-12) \\
\textbf{Treasure}: Accidental \\
\textbf{Description} \\
Purple mushrooms are one of the best known and most feared of cave dangers. A traveler can often notice the marks left by the purple mushroom on those who live or hunt in the places where these carnivorous mushrooms lurk. These deep, hideous scars look like furrows carved into flesh - the signs of a close encounter with a purple mushroom.

A purple mushroom feeds on rotting organic matter, but unlike most mushrooms it is not a passive consumer. The tendrils of a purple mushroom can strike with unexpected rapidity and are coated with a virulent poison that causes the flesh to rot with sickening speed. This powerful poison, if neglected, can quickly rot an entire arm or leg, leaving behind only bones that will soon corrode as well.

Although purple mushrooms can move, they only do so to attack or hunt prey. A purple mushroom with a steady stream of rot to feed on is content to stay in one place. Many underground dwellers, particularly Troglodytes and Vegepigmies, use this behavior to their advantage and place multiple purple mushrooms in key junctions and entrances of their caves as guardians, making sure to provide them with enough corpses to prevent them from entering the shelter in search of safety. food.

Some species of Strident Boleto have a rather similar appearance to that of purple mushrooms, although they lack sprawling branching. It is not strange to find screeching boletes and purple mushrooms in the same tangle, especially in areas where other creatures grow these mushrooms as guardians.

A purple mushroom is 1.2 meters tall and weighs 25 kg.


\medskip\index[Mostruario]{Wisp} \textbf{Wisp}

\textit{Tiny undead, chaotic evil}

\textbf{STRENGTH} -5

\textbf{DEXTERITY} +9

\textbf{CONSTITUTION} +0

\textbf{INTELLIGENCE} +1

\textbf{WISDOM} +2

\textbf{CHARISMA} +0

\textbf{Initiative} +9 - \textbf{Defense} 20

\textbf{Hit Points} 22 (9d4)

\textbf{Movement} 0m, flight 15m (float)

\textbf{Saving Throws}: Fortitude +3, Reflex +12, Will +9

\textbf{Immunity to Damage} lightning, poison

\textbf{Resistance to Damage} acid, cold, fire, void, sound; weapons that are not magical

\textbf{Condition Immunity} grabbed, poisoned, entangled, paralyzed, unconscious, prone, fatigue

\textbf{Senses} darkvision 36 m

\textbf{Languages} the languages he knew in life

\textbf{Challenge} 2 (450 PX)

\textit{\textbf{Consume Life.}} As a bonus action, the will-o'-the-wisp can target a creature it can see within 1 meter of it that has 0 hit points and is still alive. The target must succeed on a DC 10 Fortitude saving throw against this spell or die. If the target dies, the will-o'-the-wisp heals 10 (3d6) hit points.

\textit{\textbf{Ephemeral.}} The will-o'-the-wisp cannot wear or carry anything.

\textit{\textbf{Variable Illumination.}} The will-o'-the-wisp gives off bright light in a radius of 1 to 6 meters and dim light for an additional number of meters equal to the chosen radius. Will-o'-the-wisp can modify this ray with a bonus action.

\textit{\textbf{Incorporeal Movement.}} Will-o'-the-wisp can move through other creatures and objects as if they were hindering terrain. It takes 5 (1d10) force damage if it ends its round inside an object.

\textit{\textbf{Undead Nature.}} The will-o'-the-wisp needs no air, food, or drink.

\textbf{Actions}

\textit{\textbf{Shake.} Melee spell attack}: +9 to hit, range 1 yd, a creature.

\textit{Strikes:} 9 (2d8) lightning damage.

\textit{\textbf{Invisibility.}} Will-o'-the-wisp and its light magically become invisible until it attacks or uses Consume Life, or until its concentration ends (as if focusing on a spell).

\textbf{Ecology}
Environment any swamp \\
Organization: Solitary, pair or sequence (3-4) \\
\textbf{Treasure}: Accidental \\
\textbf{Description} \\
Every hunter and farmer who lives near a swamp or swamp has named these spheres of dim light: jack lantern, candles of the dead, walking fires, pine lights, ghost lights, rush lights; but everyone knows that they are dangerous predators and false guides in the dark.

Evil creatures that feed on the strong psychic emanations of terrified creatures, the will-o'-the-wisps take pleasure in putting gullible travelers into dangerous situations. In the wilds, where they are very common, will-o'-the-wisps prefer simple tactics such as positioning themselves on rocks or quicksand where they can easily be mistaken for lanterns (especially if they can set the trap near real signal lanterns) to attract travelers. towards danger. On rare occasions, the will-o'-the-wisps in search of an easy life move to a city and settle near the gallows or follow, invisible, an army, so as to feed on the fear of dying men; why the vast majority choose to remain in swamps, where victims are in short supply, remains a mystery.

Wisps can only rely on their electric shock in dangerous situations, so they prefer to let other creatures or dangers deal with their victims while they float by and feast.

Will-o'-the-wisps can glow any color they wish, but they are more often yellow, white, green, or blue. They can also vary their brightness to create a design - many will-o'-the-wisps like to create shapes that vaguely resemble skulls in their luminescence to increase the terror in their victims. Their real bodies are barely visible globes of translucent spongy material of about 30 centimeters that weigh 1.5 kg and can emit light over their entire surface. The light of the will-o'-the-wisps glows roughly like a torch, and although they don't appear to use sound to communicate, they hear perfectly and can vibrate their bodies so rapidly that they mimic language.

Despite being vilified by most sentient creatures, will-o'-the-wisps are actually quite intelligent, although they think completely alien. Sometimes they organize themselves into groups called "sequences"; their society and their purposes remain completely unknown, as well as their origins, although they are sometimes known to make deals with those who offer them large numbers of suitably terrified victims.

Will-o'-the-wisps have no age and are in fact immortal, unless they die a violent death; the oldest will-o'-the-wisps can be excellent repositories of knowledge from the past, although getting one of these cruel creatures to cooperate can be quite tricky.


\medskip\index[Mostruario]{Flogger} \textbf{Flogger}

\textit{Large monstrosity, neutral evil}

\textbf{STRENGTH} +4

\textbf{DEXTERITY} -1

\textbf{CONSTITUTION} +3

\textbf{INTELLIGENCE} -2

\textbf{WISDOM} +3

\textbf{CHARISMA} -2

\textbf{Initiative} -1 - \textbf{Defense} 23

\textbf{Hit Points} 93 (11d10 + 33)

\textbf{Movement} 3 m, climb 3 m

\textbf{Saving Throws}: Fortitude +13, Reflex +5, Will +13

\textbf{Skills} Move Silently / Hide +5, Awareness +6

\textbf{Senses} darkvision 18 m

\textbf{Languages} -

\textbf{Challenge} 5 (1,800 PX)

\textit{\textbf{False Appearance.}} When the flogger remains motionless, it is indistinguishable from a normal rock formation, such as a stalagmite.

\textit{\textbf{Climbing like Spider.}} The flogger can scale difficult surfaces, including standing upside down on the ceiling, without the need for a skill check.

\textit{\textbf{Grasping Tendrils.}} The whiplash can have up to six tendrils at a time. Each tendril can be attacked (Defense 20; 10 hit points; immunity to poison damage). Destroying a tendril does no damage to the whip, which can produce a replacement tendril in its next round. A tendril can also be broken if a creature takes an action and makes a DC 15 Strength check against it.

\textbf{Actions}

\textit{\textbf{Multiattack.}} The flogger can make four attacks with his tendrils, use wrap, and make a bite attack.

\textit{\textbf{Bite.} Melee Weapon Attack}: +7 to hit, 1m range, one target.

\textit{Strikes:} 22 (4d8 + 4) piercing damage.

\textit{\textbf{Tendril.} Melee weapon attack}: +7 to hit, range 15 yards, a creature.

\textit{Strikes:} The target is grabbed (DC 15 to escape). Until the grapple is complete, the target is in the way and has -1d6 on Strength checks and Fortitude saving throws, while the whiplasher cannot use the same tendril against another target.

\textit{\textbf{Wrap.}} The flogger drags the creatures he has grasped 7 meters towards him.

\textbf{Ecology}
Environment any dungeon \\
Organization: Solitary, pair or group (3-6) \\
\textbf{Treasure}: Standard \\
\textbf{Description} \\
The flogger is an ambush hunter. Able to change the coloring and shape of its body, a hidden flogger looks like a stone or ice stalagmite (or in low-ceilinged places, a stone or ice column). In areas without these hiding stretches, a flogger can squeeze his body down to look like a boulder. The lashes it can evert are not of flesh but of a thick semi-liquid material similar to partially melted wax but with the strength of an iron chain and the ability to numb the flesh and weaken strength. The flogger can use these lashes with great skill and fly them up to 15 meters to steal objects that attract his attention.

Despite its alien and monstrous form, the flogger is one of the most intelligent inhabitants of the underground. They do not form large societies (although they often find themselves living together with other underground creatures such as the Brain Eaters, with whom they sometimes ally), but they often aggregate in small groups. Particularly interested in the philosophy of life and death, and the more subtle aspects of the world's most sinister and cruel religions, a flogger can talk or argue for hours with those he initially simply tried to eat. Some stories speak of particularly gifted orators and philosophers who have been kept for days or even years as pets or conversation companions by groups of floggers; Eventually, though, if they can't escape, the floggers' appetite ends up getting the better of their curious intelligence, especially in cases where these companion animals consistently outweigh the wit and patience of their guardians.
A flogger is 2.7 meters tall and weighs 1,100 kg.


\medskip\index[Mostruario]{Gargoyle} \textbf{Gargoyle}

\textit{Elemental mean, chaotic evil}

\textbf{STRENGTH} +2

\textbf{DEXTERITY} +0

\textbf{CONSTITUTION} +3

\textbf{INTELLIGENCE} -2

\textbf{WISDOM} +0

\textbf{CHARISMA} -2

\textbf{Initiative} +0 - \textbf{Defense} 16

\textbf{Hit Points} 52 (7d8 + 21)

\textbf{Movement} 9 m, flight 18 m

\textbf{Saving Throws}: Fortitude +4, Reflex +6, Will +4

\textbf{Damage Resistances} from non-magical or non-adamantium weapons

\textbf{Immunity to Damage} poison

\textbf{Condition Immunity} poisoned, petrified, fatigue

\textbf{Senses} darkvision 18 m

\textbf{Languages} Terran

\textbf{Challenge} 2 (450 PX)

\textit{\textbf{False Appearance.}} While the gargoyle remains immobile, it is indistinguishable from an inanimate statue.

\textit{\textbf{Elemental Nature.}} A gargoyle does not need air, food, drink, or sleep.

\textbf{Actions}

\textit{\textbf{Multiattack.}} The gargoyle makes two attacks: one with its bite and one with its claws.

\textit{\textbf{Claws.} Melee Weapon Attack}: +5 to hit, 1m range, one target.

\textit{Strikes:} 5 (1d6 + 2) slashing damage, 1 bleed damage.

\textit{\textbf{Bite.} Melee Weapon Attack}: +5 to hit, 1m range, one target.

\textit{Strikes:} 5 (1d6 + 2) piercing damage.

\textbf{Ecology}
Environment: Any \\
Organization: Solitary, pair or flock (3-12) \\
\textbf{Treasure}: Standard \\
\textbf{Description} \\
Gargoyles often appear to be stone winged statues, as they can remain motionless indefinitely and then surprise enemies. Gargoyles are prone to obsessive-compulsive behaviors, as diverse as their species is abundant. Books, stolen trinkets, weapons and trophies collected from fallen enemies are just a few examples of the types of items a gargoyle can collect to decorate his lair and territory.

Gargoyles tend to have a solitary lifestyle, although they sometimes form fearsome flocks called "wings" for protection and entertainment. Under certain conditions, a gargoyle tribe may even ally itself with other creatures, but even the most stable of these alliances can collapse for the smallest of reasons; gargoyles are just traitors, mean and vindictive.

Gargoyles are known to dwell in the heart of the larger cities, crouched among the stone decorations of cathedrals and buildings where they hide in plain sight by day and swoop down to feed on vagrants, beggars and other unfortunates at night.

The longer a gargoyle tribe dwells in an area of buildings or ruins, the more its members begin to resemble the area's architectural style. The changes a gargoyle's appearance undergoes are slow and subtle, but over the years they can become radical.

An unusual variant of the gargoyle does not live among buildings and ruins but under the waves of the sea. These creatures are known as kapoacinth; they have the same base stats as regular gargoyles, except they have the aquatic subtype and their wings grant them a swimming speed of 40 feet (but they are useless for flying). Kapoacinths inhabit shallow coastal regions where they can crawl out of the surf to hunt down the area's residents. They are more likely to form flocks, as kapoacinths prefer group life to solitary life.

\medskip\index[Mostruario]{GEC} \textbf{GEC}

\textit{large aberration, chaotic evil}

\textbf{STRENGTH} +6

\textbf{DEXTERITY} +1

\textbf{CONSTITUTION} +5

\textbf{INTELLIGENCE} +1

\textbf{WISDOM} +1

\textbf{CHARISMA} +1

\textbf{Initiative} +2 - \textbf{Defense} 20 (chitin)

\textbf{Hit Points} 95 (12d8 + 50)

\textbf{Movement} 9 m, dig 9 m

\textbf{Saving Throws}: Fortitude +8, Reflex +3, Will +6

\textbf{Resistance} +4 saving throws on spells on the charm and illusion list

\textbf{Skills} Awareness +10

\textbf{Senses} darkvision 18 m, telluric sense 18 m

\textbf{Languages} -

\textbf{Challenge} 10 (5900 PX)

\textbf{Actions}

\textit{\textbf{Multiattack.}} The GEC can attack with two claws or with its bite

\textbf{Claws}: Attack with natural melee weapon: +21 to hit, 3 yards range, one target.

\textit{Hits:} 15 (3d6 + 5) slashing damage, 1 bleed damage.

\textbf{Bite}: Attack with natural melee weapon: +21 to hit, 3m range, one target

\textit{Hits:} 16 (3d8 + 5) slashing damage, 1 bleed damage.

\textit{\textbf{Look.}} It is enough to look at the GEC to be affected by Confusion, as a spell of the same name. To resist, you must make a Will saving throw at DC 15. Each round you can re-roll the saving throw to resist the effect.

Fighting without looking at the GEC imposes -1d6 on the attack roll.

\textbf{Ecology} \\
Environment: Underground \\
Organization: solitary, group (2-4) \\
\textbf{Treasure}: Accidental \\
\textbf{Description} \\
The Great Chitinous Being, or GEC, is an insect with a vague humanoid appearance of almost 4 meters in height, powerful and equipped with two very strong and resistant claws capable of digging and shearing any material. 4 small, central and multi-faceted eyes emanate a dim iridescent luminescence that confuses the creatures that meet their gaze.

Probably the result of some transformation spell gone bad, the GEC are masters of the subsoil. Creatures with real intelligence love elf flesh and fight tactically and wisely.

\subsection{Geni}

\medskip\index[Mostruario]{Djinni} \textbf{Djinni}

\textit{Large elemental, good chaotic}

\textbf{STRENGTH} +5

\textbf{DEXTERITY} +2

\textbf{CONSTITUTION} +6

\textbf{INTELLIGENCE} +2

\textbf{WISDOM} +3

\textbf{CHARISMA} +5

\textbf{Initiative} +2 - \textbf{Defense} 23

\textbf{Hit Points} 161 (14d10 + 84)

\textbf{Movement} 9 m, flight 27 m

\textbf{Saving Throws} Fortitude +4, Reflex +9, Will +7

\textbf{Immunity to Damage} lightning, sound

\textbf{Senses} darkvision 36 m

\textbf{Languages} Auran

\textbf{Challenge} 11 (7,200 PX)

\textit{\textbf{Elemental Death.}} If the djinni dies, his body disintegrates in a warm breeze, leaving behind only the equipment the djinni was wearing or carrying.

\textit{\textbf{Innate Spells.}} The djinni's innate spellcasting characteristic is Charisma 17, +9 to hit with spell attacks). He can innately cast the following spells, without the need for material components:

At will: \textit{identification of good and evil, identification of the magical, thundering wave}

3 / day each: \textit{walking in the wind, creating food and water} (can create wine instead of water), \textit{languages}

1 / day each: \textit{creation}, \textit{summon elementals} (air elemental only), \textit{gas form, major image}, \textit{invisibility,} \textit{displacement planar}

\textbf{Actions}

\textit{\textbf{Multiattack.}} The djinni makes three attacks of
scimitar.

\textit{\textbf{Scimitar.} Melee Weapon Attack}: +15 to hit, 1m range, one target.

\textit{Strikes:} 12 (2d6 + 5) slashing damage plus 3 (1d6) lightning or sound damage (gin's choice).

\textit{\textbf{Create Whirlwind.}} A swirling cylinder of air 1 meter in radius and 9 meters high magically forms in a point visible to the djinni within 36 meters of it. The whirlwind remains as long as the djinni maintains concentration (as if concentrating on a spell). Any creature other than the djinni who enters the whirlwind must succeed at a DC 18 Fortitude saving throw or be hindered by it. The djinni can move the whirlwind up to 60 feet with one action, and creatures hampered by the whirlwind move with it. The whirlwind ends if the djinni loses sight of it.

A creature can use its action to free a creature entangled by the whirlwind, including itself, by succeeding on a DC 18 Strength check. If the check succeeds, the creature is no longer in the way and moves to the nearest space outside the whirlwind. .

\textbf{Ecology}
Environment: Any (Plane of Air) \\
Organization: Solitary, couple, company (3-6) or gang (7-10) \\
\textbf{Treasure}: Standard (Perfect Scimitar, other treasure) \\
\textbf{Description} \\
Djinn (singular djinni) are Genes from the Elemental Plane of Air. They are said to be made of clouds and have the strength of the most powerful storms. A Djinni is around 3 meters tall and weighs around 500 kg.

Djinn disdain physical combat, preferring to use their magical powers and aerial abilities against enemies. A Djinni defeated in Combat generally takes flight and becomes a whirlwind to harass those who pursue him. When they have no choice but to fight in melee, most Djinns prefer to wield Perfect Two-Handed Scimitars.

Towards the other Genes, the Djinn get along well with the Janni and the Marid. They are frequently at odds with the Shaitan, and are sworn enemies of the Efreeti, despising these ferocious Geniuses more than any other of the Genius Races. The conflict between the Efreeti and the Djinn is so legendary that many spellcasters attempt (with varying degrees of success) to secure the service of a Djinni by promising help in the cause against the hated enemies.


\medskip\index[Mostruario]{Efreeti} \textbf{Efreeti}

\textit{Large elemental, lawful evil}

\textbf{STRENGTH} +6

\textbf{DEXTERITY} +1

\textbf{CONSTITUTION} +7

\textbf{INTELLIGENCE} +3

\textbf{WISDOM} +2

\textbf{CHARISMA} +3

\textbf{Initiative} +3 - \textbf{Defense} 23

\textbf{Hit Points} 200 (16d10 + 112)

\textbf{Movement} 12 m, flight 18 m

\textbf{Saving Throws} Fortitude +7, Reflex +10, Will +9

\textbf{Immunity to Damage} fire

\textbf{Senses} darkvision 36 m

\textbf{Languages} Ignan

\textbf{Challenge} 11 (7,200 PX)

\textit{\textbf{Elemental Death.}} If the ephreeti dies, its body disintegrates in a flash of fire and a puff of smoke, leaving behind only the equipment the ephreeti was wearing or transporting.

\textit{\textbf{Innate Spells.}} The innate spellcasting characteristic of ephreeti is Charisma, +7 to hit with spell attacks). He can innately cast the following spells, without the need for material components:

At will: \textit{detect magic}

3 / day each: \textit{enlarge / reduce, languages}

1 / day each: \textit{summon elementals} (fire elemental only), \textit{gas form, major image}, \textit{invisibility, wall of fire, planar shift}

\textbf{Actions}

\textit{\textbf{Multiattack.}} The ephreeti makes two scimitar attacks or uses Flamethrower twice.

\textit{\textbf{Scimitar.} Melee Weapon Attack}: +21 to hit, 1m range, one target.

\textit{Strikes:} 13 (2d6 + 6) slashing damage plus 7 (2d6) fire damage.

\textit{\textbf{Throw Flame.} Ranged Weapon Attack}: +16 to hit, range 36 yards, one target.

\textit{Strikes:} 17 (5d6) fire damage.

\textbf{Ecology}
Environment: Any (Plane of Fire) \\
Organization: Solitary, couple, company (3-6) or gang (7-12) \\
\textbf{Treasure}: Standard (Perfect Falcione, other treasure) \\
\textbf{Description} \\
The Efreet (singular Efreeti) are Genes from the Plane of Fire. They are 3.6 meters tall and weigh around 1000 kg.

The Efreet have few allies among the other Genes: they hate the Djinni, and attack them on sight, they can't stand the Marids, and they see the Janni as weak and frail. The Efreet often cooperate well with the Shaitan, yet even these alliances are temporary.


\subsection{Ghoul}

\medskip\index[Mostruario]{Ghast} \textbf{Ghast}

\textit{Media undead, chaotic evil}

\textbf{STRENGTH} +3

\textbf{DEXTERITY} +3

\textbf{CONSTITUTION} +0

\textbf{INTELLIGENCE} +0

\textbf{WISDOM} +0

\textbf{CHARISMA} -1

\textbf{Initiative} +3 - \textbf{Defense} 14

\textbf{Hit Points} 36 (8d8)

\textbf{Movement} 9 m

\textbf{Saving Throws}: Fortitude +2, Reflex +2, Will +5

\textbf{Damage Resistances} from Void

\textbf{Immunity to Damage} poison

\textbf{Condition Immunity} fascinated, poisoned, fatigue

\textbf{Senses} darkvision 18 m

\textbf{Languages} Common

\textbf{Challenge} 2 (450 PX)

\textit{\textbf{Stench.}} Any creature that begins its round within 1 meter of the ghast must succeed at a DC 12 Fortitude save or be nauseated until the start of its next round. If the saving throw is successful, the creature is immune to the ghast stench for the next 24
hours.

\textit{\textbf{Rebellion to Turn.}} The ghast and all ghouls within 30 feet of it have + 1d6 on saving throws against effects that turn undead.

\textbf{Actions}

\textit{\textbf{Claws.} Melee Weapon Attack}: +6 to hit, 1m range, one target.

\textit{Strikes:} 10 (2d6 + 3) slashing damage. If the target is a creature other than an undead, it must succeed on a DC 12 Fortitude save or be paralyzed for 1 minute. The target can re-roll the saving throw at the end of each of its rounds, ending the effect if the saving throw is successful.

\textit{\textbf{Bite.} Melee weapon attack}: +6 to hit, range 1 yards, a creature.

\textit{Strikes:} 12 (2d8 + 3) piercing damage.

\textbf{Ecology} \\
Environment: Any terrain \\
Organization: Solitary, group (2-4) or pack (7-12) \\
\textbf{Treasure}: Standard \\
\textbf{Description} \\
Ghasts are Ghouls with a deeper connection to the void. A ghast's paralysis also affects Elves. Ghasts roam in packs or command groups of common ghouls. The stench of death and putrefaction that surrounds these creatures is overwhelming.


\medskip\index[Mostruario]{Ghoul} \textbf{Ghoul}

\textit{Media undead, chaotic evil}

\textbf{STRENGTH} +1

\textbf{DEXTERITY} +2

\textbf{CONSTITUTION} +0

\textbf{INTELLIGENCE} -2

\textbf{WISDOM} +0

\textbf{CHARISMA} -2

\textbf{Initiative} +2 - \textbf{Defense} 13

\textbf{Hit Points} 22 (5d8)

\textbf{Movement} 9 m

\textbf{Saving Throws}: Fortitude +1, Reflex +2, Will +4

\textbf{Immunity to Damage} poison

\textbf{Condition Immunity} fascinated, poisoned, fatigue

\textbf{Senses} darkvision 18 m

\textbf{Languages} Common

\textbf{Challenge} 1 (200 PX)

\textbf{Actions}

\textit{\textbf{Claws.} Melee Weapon Attack}: +4 to hit, 1m range, one target.

\textit{Strikes:} 7 (2d4 + 2) slashing damage, 1 bleed damage. If the target is a creature other than an elf or an undead, it must succeed on a DC 12 Fortitude save or be paralyzed for 1 minute. The target can re-roll the saving throw at the end of each of its rounds, ending the effect if the saving throw is successful.

\textit{\textbf{Bite.} Melee weapon attack}: +4 to hit, range 1 yards, a creature.

\textit{Strikes:} 9 (2d6 + 2) piercing damage.

\textbf{Ecology}
Environment any terrain \\
Organization: Solitary, group (2-4) or pack (7-12) \\
\textbf{Treasure}: Standard \\
\textbf{Description} \\
Ghouls are undead who frequent graveyards and eat corpses. Legends claim that the first ghouls were cannibal humans brought back from death by unnatural hunger, or humans who fed on the decaying remains of their fellows in life and who died (and then were reborn) from a horrendous disease; the true origin of these necrophagous undead is uncertain.

Ghouls lurk on the fringes of civilization (in or near cemeteries or city sewers) where they can find ample supplies of their favorite food. Although they prefer rotting bodies and often bury their victims to improve their flavor, they eat the dead fresh if they are hungry enough.


While many surface ghouls live primitive, rumors speak of deep underground ghoul cities ruled by priests who worship cruel ancient gods or strange hunger demon lords. These "civilized" ghouls are no less hideous in their eating habits, and indeed their concept of a well-laid banquet table is perhaps even more hideous than the idea of a fresh meal from a coffin.


\subsection{Giants}

\medskip\index[Mostruario]{Hill Giant} \textbf{Hill Giant}

\textit{Huge giant, chaotic evil}

\textbf{STRENGTH} +5

\textbf{DEXTERITY} -1

\textbf{CONSTITUTION} +4

\textbf{INTELLIGENCE} -3

\textbf{WISDOM} -1

\textbf{CHARISMA} -2

\textbf{Initiative} -1 - \textbf{Defense} 16

\textbf{Hit Points} 105 (10d12 + 40)

\textbf{Movement} 12 m

\textbf{Saving Throws}: Fortitude +11, Reflex +2, Will +3

\textbf{Skills} Awareness +2

\textbf{Languages} Giant

\textbf{Challenge} 5 (1.800 PX)

\textbf{Actions}

\textit{\textbf{Multiattack.}} The giant makes two attacks with the heavy club.

\textit{\textbf{Heavy Cudgel.} Melee Weapon Attack}: +12 to hit, 3m range, one target.

\textit{Hits:} 18 (3d8 + 5) hit damage.

\textit{\textbf{Rock.} Ranged weapon attack}: +6 to hit, range 18m, one target.

\textit{Hits:} 21 (3d10 + 5) hit damage.

\textbf{Ecology} \\
Environment: Temperate Hills \\
Organization: Solitary, group (2-5), gang (6-8), raiding group (9-12 plus 1d4 Cruel Wolves) or tribe (13-30 plus 35 \% non-combatant plus 1 fighting leader of 4 6th-6th level, 11-16 Cruel Wolves, 1-4 Ogres and 13-20 orc slaves) \\
\textbf{Treasure}: Standard (Leather Armor, Heavy Club, Other Treasure) \\
\textbf{Description} \\
Collina giants have skin that varies from light brown to reddish, brown or black hair, and eyes of the same color. They wear layers of crudely tanned skins with the fur still on. They rarely wash or repair their own clothing, and prefer to simply add new layers as the old ones wear out. Adults are about 3 meters tall and weigh approximately 550 kg. Collina giants can live up to 200 years, although they rarely reach this age.

Collina giants prefer to fight from the top of ledges and cliffs, from where they can hit opponents with rocks and boulders, thus limiting personal risk. They love to make overrun attacks against smaller creatures early in the fight, and only then do they take up positions and start swinging their massive clubs.

Collina giants are by nature nomads and prefer to travel from place to place to raid and plunder. Although they enjoy temperate climates more, they do not disdain traveling far from their favored environment if plundering is abundant and prosperous. They are, on the whole, very selfish creatures, rarely facing battles they are not sure they will win. Hill giants are known to push each other when confronted with fearsome opponents and don't hesitate to sacrifice a teammate to save their own skin. Wandering bands of Hill giants are rife in the temperate hills, and their constant aggression makes them one of the most feared dangers in this environment.


Solitary, non-evil Hill giants are very rare, but can sometimes be found in other humanoid societies, although they are almost never accepted in major cities or populated centers. They are at home as workers and soldiers in remote frontier towns, and often act as rudimentary diplomats to negotiate with the gangs of raiding Hill giants. Unfortunately, Collina giants who abandon their racial lifestyle for civilization are mocked and often killed on sight by their nomadic brothers. However, these "civilized" Hill giants can find their place in society and many have managed to live a peaceful and peaceful existence.


\medskip\index[Mostruario]{Fire Giant} \textbf{Fire Giant}

\textit{Huge giant, legal evil}

\textbf{STRENGTH} +7

\textbf{DEXTERITY} -1

\textbf{CONSTITUTION} +6

\textbf{INTELLIGENCE} +0

\textbf{WISDOM} +2

\textbf{CHARISMA} +1

\textbf{Initiative} +0 - \textbf{Defense} 27 (plate armor)

\textbf{Hit Points} 162 (13d12 + 78)

\textbf{Movement} 9 m

\textbf{Saving Throws}: Fortitude +14, Reflex +4, Will +9

\textbf{Skills} Acrobatics +11, Awareness +6

\textbf{Immunity to Damage} fire

\textbf{Languages} Giant

\textbf{Challenge} 9 (5000 PX)

\textbf{Actions}

\textit{\textbf{Multiattack.}} The giant makes two attacks with his broadsword.

\textit{\textbf{Greatsword.} Melee Weapon Attack}: +20 to hit, 3m range, one target.

\textit{Strikes:} 28 (6d6 + 7) slashing damage.

\textit{\textbf{Rock.} Ranged weapon attack}: +12 to hit, range 18m, one target.

\textit{Strikes:} 29 (4d10 + 7) hit damage.

\textbf{Ecology}
Environment: Hot mountains \\
Organization: Solitary, group (2-5), gang (6-12 plus a 35 \% non-combatant and 1 adept or Devotee of 1st-2nd level), group of raiders (6-12 plus 1 adept or 3rd-5th level wizard, 2-5 Hellhounds and 2-3 Trolls or Ettin) or tribe (20-30 plus 1 adept, mage or Devotee of 6th-7th level; 1 Warrior king or forest ranger of 8 9th-9th level; and 17-38 Hellhounds, 12-22 Trolls, 7-12 Ettin and 1-2 Young Red Dragons) \\
\textbf{Treasure}: Standard (Half Armor, Greatsword, other treasure) \\
\textbf{Description} \\
Fire giants are the most rigid and martial giants, always ready for war and brutally treating anyone they meet. Their rigid command structure requires soldiers, officers and even generals, and that all obey their king's orders without question.

Fire giants have bright orange hair that glows and sparkles as if on fire. An adult male stands between 3.6 and 4.8 meters tall, with a rib cage of around 2.7 meters, and weighs around 3,500 kg. Females are slightly shorter and slender. Fire giants can live up to 350 years.

Fire giants wear robes of sturdy fabrics or leather in orange, yellow, black, or red. Warriors wear burnished steel helmets and half armor and wield large greatswords that whirl across the battlefield. In large groups, the fire giants fight with brutal and efficient group tactics, and do not hesitate to sacrifice a few companions to ambush the enemy.

Fire giants prefer warm places - the warmer the better. They can be found in deserts, volcanoes, hot springs and in the depths of the earth near lava chimneys. They live in castles, fortified settlements or large caves, and the architecture of these places reflects their rigid and militaristic lifestyle, with officers living in better quarters than those of their subordinates.



\medskip\index[Mostruario]{Giant of Frost} \textbf{Giant of Frost}

\textit{Huge giant, neutral evil}

\textbf{STRENGTH} +6

\textbf{DEXTERITY} -1

\textbf{CONSTITUTION} +5

\textbf{INTELLIGENCE} -1

\textbf{WISDOM} +0

\textbf{CHARISMA} +1

\textbf{Initiative} -1 - \textbf{Defense} 19 (composite armor)

\textbf{Hit Points} 138 (12d12 + 60)

\textbf{Movement} 12 m

\textbf{Saving Throws} Fortitude +14, Reflex +3, Will +6

\textbf{Skills} Acrobatics +9, Awareness +3

\textbf{Immunity to Damage} cold

\textbf{Languages} Giant

\textbf{Challenge} 8 (3,900 PX)

\textbf{Actions}

\textit{\textbf{Multiattack.}} The giant makes two attacks with the double ax.

\textit{\textbf{Double-pronged Ax.} Melee Weapon Attack}: +18 to hit, 3m range, one target.

\textit{Strikes:} 25 (3d12 + 6) slashing damage.

\textit{\textbf{Rock.} Ranged weapon attack}: +11 to hit, range 18m, one target.

\textit{Hits:} 28 (4d10 + 6) hit damage.

\textbf{Ecology} \\
Environment cold mountains \\
Organization: Solitaire, gang (3-5), group (6-12 plus 35 \% non-combatants and 1 1st-2nd level wizard or Devotee), raider group (6-12 plus 35 \ % non-combatants, 1 Devotee or wizard of 3rd-5th level, 1-4 Winter Wolves and 2-3 Ogres) or tribe (21-30 plus 1 adept, wizard or Devotee of 6th-7th level; 1 jarl Barbarian or ranger 7th-9th level; and 15-36 Winter Wolves, 13-22 Ogres and 1-2 Young White Dragons) \\
\textbf{Treasure}: Standard (Mail Jack, Double-sided Ax, other treasure) \\
\textbf{Description} \\
A frost giant has blue or dirty yellow hair, and eyes typically the same color. They dress in skins and furs, adorning themselves with any jewelry they own. The fighting frost giants also wear mail jackets and metal helmets decorated with horns and feathers. An adult male stands 5 meters tall and weighs around 1,400 kg. Females are slightly shorter and leaner, but are otherwise identical to males. Frost giants can live up to 250 years.

Frost giants are much feared, as their lust for destruction and war and their contemptuous demeanor push them into ever greater displays of brutality. Frost giants start off by attacking from a distance, throwing rocks until they run out of ammo or the opponent approaches, then take on him with their huge axes. One of the favorite tactics is ambushing by hiding under the snow above an icy or snowy slope, where opponents will have a hard time reaching them, and then start causing an avalanche before heading into battle. Frost giants can hide very well in snowy environments and are masters of stealth in their domain.

Frost giants survive by hunting and raiding on their own, as they live in cold, desolate environments. Groups of frost giants are split almost equally between those living in makeshift settlements or abandoned castles and those who roam the freezing north, like nomads in search of loot and supplies. The leaders of the frost giants are called jarls and demand absolute obedience from their followers. At any time, a jarl can be challenged to battle for command of the tribe. These challenges typically end in the death of one of the contenders. A single jarl can often count on a dozen or more smaller tribes of frost giants as an extension of his own. In these cases, the leaders of the lesser tribes are known as captains or warlords.

Frost giants love to take prisoners and use them as both slaves and raw materials. Usually each group of frost giants keeps 1-2 humanoid slaves chained to a slave handler: the meanest and cruellest of the group after the jarl. They also have a certain fondness for monstrous pets: White Dragons and Winter Wolves are popular choices, but Remorhaz and Yetis can also be found in a frost giant's lair.

\medskip\index[Mostruario]{Giant of the Clouds} \textbf{Giant of the Clouds}

\textit{Huge giant, neutral good (50 \%) or neutral evil (50 \%)}

\textbf{STRENGTH} +8

\textbf{DEXTERITY} +0

\textbf{CONSTITUTION} +6

\textbf{INTELLIGENCE} +1

\textbf{WISDOM} +3

\textbf{CHARISMA} +3

\textbf{Initiative} +1 - \textbf{Defense} 19

\textbf{Hit Points} 200 (16d12 + 96)

\textbf{Movement} 12 m

\textbf{Saving Throws} Fortitude +16, Reflex +6, Will +10

\textbf{Skills} Perceiving Emotions +7, Awareness +7

\textbf{Languages} Common, Giant

\textbf{Challenge} 9 (5000 PX)

\textit{\textbf{Innate Spells.}} The giant's spellcasting trait is Charisma. The giant can cast these spells innately, without the need for material components:

At will: \textit{detection of magic, light, cloud of fog}

3 / day each: \textit{Fall Feather, misty pass, telekinesis}

1 / day each: \textit{check weather, gaseous form}

\textit{\textbf{Refined Smell.}} The giant has + 1d6 on smell-based Wisdom (Awareness) checks.

\textbf{Actions}

\textit{\textbf{Multiattack.}} The giant makes two attacks with the Spiked Mace.

\textit{\textbf{Spiked Mace.} Melee Weapon Attack}: +22 to hit, 3m range, one target.

\textit{Strikes:} 21 (3d8 + 8) piercing damage.

\textit{\textbf{Rock.} Ranged weapon attack}: +14 to hit, range 18m, one target.

\textit{Strikes:} 30 (4d10 + 8) hit damage.

\textbf{Ecology} \\
Environment: Temperate Mountains \\
Organization: Solitary, group (2-5), family (2-5 plus 35 \% non-combatants plus 1 wizard or Devotee of 4th-7th level and 2-5 Griffins) or tribe (6-20 plus 1 oracle magician or Devotee of 7th-12th level and 2-5 Griffins) \\
\textbf{Treasure}: Standard (Chain Jack, Spiked Mace, Other Treasure) \\
\textbf{Description} \\
The skin color of the cloud giants varies from milky white to dusty blue. Adult males stand around 5.4 meters tall and weigh approximately 2,500 kg. Females are slightly shorter and slender. Cloud giants can live up to 400 years, dress up in precious clothing and jewels. For many, appearance indicates status. The better the clothes and the finer the jewelry, the more important the wearer is. They also enjoy music, and the majority play one or more instruments (the harp is a favorite).

Cloud giants can have unusually varied traits; about half are good and half evil. Good cloud giants build roads connecting their settlements with humans' roads to promote trade. It is not unusual to see a good cloud giant walking among humans, for example, in a human city near a high mountain range. Evil cloud giants tend not to create stable settlements and instead prefer to live in crude shelters on high peaks, from which they descend only to plunder the villages of what they might need. These two philosophies often lead to the outbreak of violent and lasting wars between neighboring tribes.

There are many legends that speak of magical cities of cloud giants located in the clouds themselves, floating on the winds and circumnavigating the world. While cloud giants recognize that these are mostly fantasies, some claim to have seen them and have dedicated their entire existence to finding them.


\medskip\index[Mostruario]{Giant of Stone} \textbf{Giant of Stone}

\textit{Huge giant, neutral}

\textbf{STRENGTH} +6

\textbf{DEXTERITY} +2

\textbf{CONSTITUTION} +5

\textbf{INTELLIGENCE} +0

\textbf{WISDOM} +1

\textbf{CHARISMA} -1

\textbf{Initiative} +2 - \textbf{Defense} 21

\textbf{Hit Points} 126 (11d12 + 55)

\textbf{Movement} 12 m

\textbf{Saving Throws} Fortitude +12, Reflex +6, Will +7

\textbf{Skills} Acrobatics +12, Awareness +4

\textbf{Senses} darkvision 18 m

\textbf{Languages} Giant

\textbf{Challenge} 7 (2,900 PX)

\textit{\textbf{Stone cloaking.}} The giant has + 1d6 on Dexterity (Hide) checks made to hide on rocky terrain.

\textbf{Actions}

\textit{\textbf{Multiattack.}} The giant makes two attacks with the heavy club.

\textit{\textbf{Heavy Club.} Melee Weapon Attack}: +19 to hit, 5m range, one target.

\textit{Strikes:} 19 (3d8 + 6) hit damage.

\textit{\textbf{Rock.} Ranged weapon attack}: +15 to hit, range 18m, one target.

\textit{Hits:} 28 (4d10 + 6) hit damage. If the target is a creature, it must succeed on a DC 17 Fortitude saving throw or fall prone.

\textbf{Reactions}

\textit{\textbf{Grabbing Rocks.}} If a rock or similar object is thrown at the giant, the giant can, on a successful DC 10 Reflex save, grab the projectile and take no hit damage from it.

\textbf{Ecology}
Environment temperate mountains \\
Organization: Solitary, group (2-5), gang (4-8), hunting group (9-12 plus 1 Elder) or tribe (13-30 plus 35 \% non-combatants, 1-3 Elders and 4 -6 Cruel Bears) \\
\textbf{Treasure}: Standard (Heavy Club, other treasure) \\
\textbf{Description} \\
Stone giants prefer thick leather garments, dyed with shades of brown and gray to blend in with the stone around them. Adults are around 3.6 meters tall, weigh around 750 kg and can live up to 800 years.

Stone giants fight from a distance if possible, but if they can't avoid the fray they use giant stone clubs. One of the tactics favored by the stone giants is to stand still, blending in with the landscape, and then move forward by throwing rocks and surprising the enemies.

Stone giants prefer to live in huge caverns on rocky peaks. They rarely live more than a few days' travel from other bands of stone giants and raise shared herds of goats and other livestock.

Older stone giants tend to drift away from the tribe for a long time, either to live in solitude somewhere or try to fit into other humanoid civilizations. After decades of self-imposed exile, those who return are known as the Elder Rock Giant.


\medskip\index[Mostruario]{Giant of Storms} \textbf{Giant of Storms}

\textit{Huge giant, good chaotic}

\textbf{STRENGTH} +9

\textbf{DEXTERITY} +2

\textbf{CONSTITUTION} +5

\textbf{INTELLIGENCE} +3

\textbf{WISDOM} +4

\textbf{CHARISMA} +4

\textbf{Initiative} +3 - \textbf{Defense} 23 (scale armor)

\textbf{Hit Points} 230 (20d12 + 100)

\textbf{Movement} 15m, swim 15m

\textbf{Saving Throws} Fortitude +17, Reflex +8, Will +13

\textbf{Skills} Arcane +8, Acrobatics +14, Awareness +9, History +8

\textbf{Damage Resistances} cold

\textbf{Immunity to Damage} lightning bolt, sound

\textbf{Languages} Common, Giant

\textbf{Challenge} 13 (10000 PX)

\textit{\textbf{Amphibian.}} The giant can breathe air and water.

\textit{\textbf{Innate Spells.}} The giant's spellcasting trait is Charisma. The giant can cast these spells innately, without the need for material components:

At will: \textit{controlled fall, detection of magic,} \textit{levitation, light}

3 / day each: \textit{check weather, breathe} \textit{underwater}

\textbf{Actions}

\textit{\textbf{Multiattack.}} The giant makes two attacks with his broadsword.

\textit{\textbf{Greatsword.} Melee Weapon Attack}: +29 to hit, 3m range, one target.

\textit{Strikes:} 30 (6d6 + 9) slashing damage.

\textit{\textbf{Rock.} Ranged weapon attack}: +22 to hit, range 18m, one target.

\textit{Hits:} 35 (4d12 + 9) hit damage.

\textit{\textbf{Lightning Strike (Cooldown 5-6).}} The giant fires a magical bolt of lightning at a visible spot within 150 meters of himself. Each creature within 10 feet of that point must make a DC 17 Reflex saving throw, taking 54 (12d8) lightning damage if it fails, or half if it succeeds.

\textbf{Ecology} \\
Environment: Any heat \\
Organization: Solitary or family (2-5 plus 1 wizard or Devotee of 7-10th level, 1-2 Roc, 2-6 Griffins and 2-8 Sharks) \\
\textbf{Treasure}: Standard (Perfect Plate Chestpiece, Perfect Composite Longbow[Strength +9] with 20 Arrows, Perfect Greatsword, more treasure) \\
\textbf{Description} \\
Storm giants tend to have tanned complexions, although rare specimens have purple skin, dark purple or blue hair, and silver-gray or purple eyes. The color purple is considered a good omen among storm giants, and those who possess it tend to become bosses among their own people. Adults are normally 6.3 meters tall and weigh 6,000 kg. Storm giants can live up to 600 years.

When at rest, they prefer to wear short tunics and wide belts at the hips, sandals or bare feet and a headband. They wear a few simple but well-made jewels, the most common being anklets (preferred by barefoot giants), rings or diadems. But when they gear up for battle, they wear shimmering plate armor and wield huge greatswords and bows.

Storm giants tend to be solitary, preferring to live along remote coasts or on tropical islands. As their name suggests, they are prone to violent mood swings. Storm giants are easy to anger in the face of evil and can be brutal and dangerous enemies when insulted. In battle, they prefer to hurl a shower of arrows at their enemies, drawing their greatswords only after their opponents get close.

Storm giants live in beautiful towers, castles or walled settlements and love to cultivate the land. They have huge, well-tended gardens and manage hundreds of acres of crops per group. They often employ other humanoids, such as Elves or Humans, to support them in running their huge farms. An enclave of storm giants often takes responsibility for the safety of an entire island or coastline.

\medskip\index[Mostruario]{Gnoll} \textbf{Gnoll}

\textit{Medium humanoid (gnoll), chaotic evil}

\textbf{STRENGTH} +2

\textbf{DEXTERITY} +1

\textbf{CONSTITUTION} +0

\textbf{INTELLIGENCE} -2

\textbf{WISDOM} +0

\textbf{CHARISMA} -2

\textbf{Initiative} +1 - \textbf{Defense} 16 (leather armor, shield)

\textbf{Hit Points} 22 (5d8)

\textbf{Movement} 9 m

\textbf{Saving Throws}: Fortitude +4, Reflex +0, Will +0

\textbf{Senses} darkvision 18 m

\textbf{Languages} Gnoll

\textbf{Challenge} 1/2 (100 PX)

\textit{\textbf{Anger.}} When the gnoll reduces a creature to 0 hit points with a melee attack during its round, it can take a bonus action to move up to half its movement and make an attack of bite.

\textbf{Actions}

\textit{\textbf{Bite.} Melee weapon attack}: +4 to hit, range 1 yards, a creature.

\textit{Strikes:} 4 (1d4 + 2) piercing damage.

\textit{\textbf{Spear.} Melee or Ranged Weapon Attack}: +4 to hit, 1 m range, and 6 m range, one target.

\textit{Strikes:} 5 (1d6 + 2) piercing damage or 6 (1d8 + 2) piercing damage when used with two hands to make a melee attack.

\textit{\textbf{Longbow.} Ranged weapon attack}: +3 to hit, range 45m, one target.

\textit{Strikes:} 5 (1d8 + 1) piercing damage.

\textbf{Ecology} \\
Environment: Warm plains, deserts \\
Organization: Solitary, couple, hunting group (2-5 and 1-2 Hyenas), gang (10-100 adults plus 50 \% small non-combatants, 1 3rd level sergeant every 20 adults, 1 leader of 4 6th-6th level and 5-8 Hyenas) or tribe (20-200 plus 1 3rd level sergeant every 20 adults, 1 or 2 4th or 5th level lieutenants, 1 6th-8th level chief, 7-12 Hyenas and 4-7 Hyenodonts) \\
\textbf{Treasure}: NPC gear (Leather Armor, Heavy Wooden Shield, Spear, other treasure) \\
\textbf{Description} \\
Gnolls are a race of large, large humanoids who resemble hyenas not just in appearance; they show an evident affinity with these scavenger animals, so much so that they keep them as pets, and they reflect many of the behaviors of these animals.

Gnolls are skilled hunters, but they much prefer cleaning up or stealing a carcass rather than hunting prey. This laziness prompts them to procure slaves of any species available to force them to dig burrows, gather supplies and water, and even hunt for their gnoll masters.

Creatures other than hyenas or gnolls become food or slaves, depending on the temperament of the tribe. Even a dead or fallen companion becomes a fresh meal for a gnoll, who can honor a famous member of the tribe with a short prayer or cook one who died of a devastating disease entirely - otherwise, gnolls don't see a dead fellow much differently than any other. other creature. The more "civilized" gnolls do not eat their captives: instead, they keep them as slaves, to defend or improve their lair or to exchange them with other slave tribes or bands.

Gnolls have a taste for combat, but only when they are outnumbered. In other situations, they prefer to avoid combat except as a means of obtaining a carcass from another hunter, or as an ingenious ambush to take down a hefty meal. These hyena men see no value in bravery or heroism and instead prefer to flee once it is clear that victory is unattainable, arguing that it is better to run away with your tail between your legs than to lose it altogether.

During combat, gnolls use a strange combination of pack tactics and individual strategies. If a gnoll is sure to win, he tries to take down the weaker opponent rather than help his companions. If the gnolls are in trouble, they band together against a powerful opponent and attempt to eliminate him, hoping to force his allies to flee.

Gnoll leaders have ranger skills but it's not impossible to find gnolls devoted to some ravenous Patron as well. They hardly master magic effectively.


\medskip\index[Mostruario]{Gnome of the Deep (Svirfneblin)} \textbf{Gnome of the Deep (Svirfneblin)}

\textit{Small humanoid (gnome), neutral good}

\textbf{STRENGTH} +2

\textbf{DEXTERITY} +2

\textbf{CONSTITUTION} +2

\textbf{INTELLIGENCE} +1

\textbf{WISDOM} +0

\textbf{CHARISMA} -1

\textbf{Initiative} +2 - \textbf{Defense} 16 (shirt jacket)

\textbf{Hit Points} 16 (3d6 + 6)

\textbf{Movement} 6 m

\textbf{Saving Throws}: Fortitude +6, Reflex +6, Will +2

\textbf{Skills} Move Silently / Hide +4, Awareness +2

\textbf{Senses} darkvision 36 m

\textbf{Languages} Gnomics, Language of the Depths, Terran

\textbf{Challenge} 1/2 (100 PX)

\textit{\textbf{Gnome Cunning.}} The gnome has + 1d6 on saving throws against magic.

\textit{\textbf{Stone Camouflage.}} The gnome has + 1d6 on Dexterity (Hide) checks made to hide on rocky terrain.

\textit{\textbf{Innate Spells.}} The gnome's innate spellcasting trait is Intelligence. The gnome can cast these spells innately, without the need for components:

At will: \textit{anti-detection} (personal)

1 / day each: \textit{disguise oneself, blindness / deafness, blur}

\textbf{Actions}

\textit{\textbf{War Pick.} Melee Weapon Attack}: +4 to hit, 1m range, one target.

\textit{Strikes:} 6 (1d8 + 2) piercing damage.

\textit{\textbf{Poison Bolt.} Ranged weapon attack}: +4 to hit, range 9m, one target.

\textit{Strikes:} 4 (1d4 + 2) piercing damage, and the target must succeed on a DC 12 Fortitude save or be poisoned for 1 minute. The target can re-roll the saving throw at the end of each of its rounds, ending the effect on itself if successful.

\textbf{Ecology}
Environment: Any underground \\
Organization: Solitary, company (2-4), squad (5-20 plus 1 3rd-6th level leader and two 3rd level sergeants), or gang (30-50 plus 1 3rd level sergeant every 20 adults, 5 5th level lieutenants, 3 7th level captains, and 2-5 Medium Earth Elementals) \\
\textbf{Treasure}: NPC gear (Heavy Pickaxe, Light Crossbow with 10 Bolts, other treasure) \\
\textbf{Description} \\
The svirfneblin, or "depth gnomes", are a branch of the gnome race. They dwell underground, in hidden cities, safe from dark elves and other underground races. Their skin is the color of the rock, usually gray or brown. Males are bald and females have sparse gray hair. A svirfneblin's bond with the goblin realm is much stronger than that of surface gnomes; this makes svirfneblin strangely detached from their emotions or subject to sudden and violent emotional manifestations. The svirfneblin have long fought against the Duergar and cannot quite distinguish the Duergar from the Dwarves.

\medskip\index[Mostruario]{Globule} \textbf{Globule}

\textit{Small aberration, wicked one}

\textbf{STRENGTH} -2

\textbf{DEXTERITY} +2

\textbf{CONSTITUTION} +0

\textbf{INTELLIGENCE} +3

\textbf{WISDOM} +1

\textbf{CHARISMA} +3

\textbf{Initiative} +3 - \textbf{Defense} 15

\textbf{Hit Points} 30 (5d10 + 5)

\textbf{Movement} fly 18 m

\textbf{Saving Throws}: Fortitude +4, Reflex +6, Will +5

\textbf{Skills} -

\textbf{Senses} darkvision 36 m

\textbf{Languages} understands the common but does not speak it

\textbf{Challenge} 1 (200 PX)

\textbf{Vulnerability} Fire

\textbf{Immunity} Void, Cold

\textbf{Immunity to conditions} poison, prone

\textbf{I hate birds} the globule has + 1d6 to attack roll against birds. Attack birds and flying creatures first

\textbf{Unusual nature} the Globule is not breathing

\textbf{I hate water} the Globuli hates getting wet and every 5 liters of water sprayed on it takes 1d4 damage

\textbf{Actions}

\textit{\textbf{Tentacle}}. Melee attack, +5 on hit, 3m range, one target

\textit{\textbf{Strikes}} 5 (1d6 + 2) Void damage. The target must make a Fortitude save to DC 11 or increase fatigue by 1.

\textbf{\textit{Brilliance}} Once per day the globule becomes extremely bright, creatures within 20 feet around it must make a Fortitude save at DC 13 or go blind for 3 rounds.

\textbf{Ecology}
Environment: Any, desert, night \\
Organization: Solitaire, 2d4 groups \\
\textbf{Treasure}: None \\
\textbf{Description} \\
Globules are magical aberrations coming from some portal open to the Beyond. Creatures of cold and emptiness seem like little stars that yearn only to suck the life of the creatures they meet.
Intelligent and crafty, they prefer to attack while remaining in the air and sapping the opponent until he is exhausted. Once killed of a Globule, there remains only a small star-shaped creature with a large central eye, completely white.

\subsection{Golem}

\medskip\index[Mostruario]{Clay Golem} \textbf{Clay Golem}

\textit{Large construct, misaligned}

\textbf{STRENGTH} +5

\textbf{DEXTERITY} -1

\textbf{CONSTITUTION} +4

\textbf{INTELLIGENCE} -4

\textbf{WISDOM} -1

\textbf{CHARISMA} -5

\textbf{Initiative} -1 - \textbf{Defense} 19

\textbf{Hit Points} 133 (14d10 + 56)

\textbf{Movement} 6 m

\textbf{Saving Throws}: Fortitude +4, Reflex +3, Will +4

\textbf{Immunity to Damage} acid, poison; from a non-magical weapon or that are not adamantium

\textbf{Condition Immunity} fascinated, poisoned, paralyzed, petrified, fatigue, frightened

\textbf{Senses} darkvision 18 m

\textbf{Languages} understands the languages of its creator but cannot speak

\textbf{Challenge} 9 (5000 PX)

\textit{\textbf{Berserk.}} Each time the golem starts its round with 60 hit points or less, roll a d6. If you roll a 6, the golem goes berserk. During each of its rounds while berserk, the golem attacks the closest creature it can see. If there is no creature close enough to move and attack it, the golem attacks an object, with a preference for objects smaller than himself. Once the golem has gone berserk, it will continue to be berserk until it is destroyed or recovers all of its hit points.

\textit{\textbf{Magical Weapons.}} The golem's weapon attacks are magical.

\textit{\textbf{Acid Absorption.}} Whenever the golem is a victim of acid damage, it takes no damage but instead regains an equal number of hit points.

\textit{\textbf{Immutable Form.}} The golem is immune to any spells or effects that would alter its form.

\textit{\textbf{Nature of Construct.}} A golem does not need air, food, drink, or sleep.

\textit{\textbf{Magic Resistance.}} The golem has + 1d6 on saving throws against spells and other magical effects.

\textbf{Actions}

\textit{\textbf{Multiattack.}} The golem makes two slam attacks.

\textit{\textbf{Slam.} Melee Weapon Attack}: +18 to hit, 1m range, one target.

\textit{Hits:} 16 (2d10 + 5) hit damage.

\textit{\textbf{Speed (Cooldown 5-6).}} Until the end of his next round, the golem gains a +2 magic bonus to Defense, has + 1d6 on Reflex saving throws, and can use slam attacks as a bonus action.

\textbf{Ecology} \\
Environment: Any \\
Organization: Solitary or group (2-4) \\
\textbf{Treasure}: None \\
\textbf{Description} \\
Clay golems wear no clothing, except for a garment of treated leather or metal around the hips. On average, they are over 2.4 meters tall and weigh 300 kilos.

\textbf{Construction}
A clay golem can be sculpted from a single block of clay weighing at least 500 pounds, treated with rare oils and powders worth 1,500 gp.


\medskip\index[Mostruario]{Flesh Golem} \textbf{Flesh Golem}

\textit{Medium construct, neutral}

\textbf{STRENGTH} +4

\textbf{DEXTERITY} -1

\textbf{CONSTITUTION} +4

\textbf{INTELLIGENCE} -2

\textbf{WISDOM} +0

\textbf{CHARISMA} -3

\textbf{Initiative} -1 - \textbf{Defense} 12

\textbf{Hit Points} 93 (11d8 + 44)

\textbf{Movement} 9 m

\textbf{Saving Throws}: Fortitude +3, Reflex +2, Will +3

\textbf{Immunity to Damage} lightning, poison; from a non-magical weapon or that are not adamantium

\textbf{Condition Immunity} fascinated, poisoned, paralyzed, petrified, fatigue, frightened

\textbf{Senses} darkvision 18 m

\textbf{Languages} understands the languages of its creator but cannot
speak

\textbf{Challenge} 5 (1,800 PX)

\textit{\textbf{Berserk.}} Each time the golem starts his round with 40 hit points or less, roll a d6. If you roll a 6, the golem goes berserk. During each of its rounds while berserk, the golem attacks the closest creature it can see. If there is no creature close enough to move and attack it, the golem attacks an object, with a preference for objects smaller than himself. Once the golem has gone berserk, it will continue to be berserk until it is destroyed or recovers all of its hit points.

\textit{\textbf{Magical Weapons.}} The golem's weapon attacks are magical.

\textit{\textbf{Lightning Absorption.}} Whenever the golem is the victim of lightning damage, it takes no damage but instead regains an equal number of hit points.

\textit{\textbf{Aversion to Fire.}} If the golem takes fire damage, it has -1d6 on attack rolls and proficiency checks until the end of its next round.

\textit{\textbf{Immutable Form.}} The golem is immune to any spells or effects that would alter its form.

\textit{\textbf{Construct Nature.}} A golem does not need air, food, drink, or sleep.

\textit{\textbf{Magic Resistance.}} The golem has + 1d6 on saving throws against spells and other magical effects.

\textbf{Actions}

\textit{\textbf{Multiattack.}} The golem makes two slam attacks.

\textit{\textbf{Slam.} Melee Weapon Attack}: +11 to hit, 1m range, one target.

\textit{Strikes:} 13 (2d8 + 4) hit damage.

\textbf{Ecology} \\
Environment: Any \\
Organization: Solitary or group (2-4) \\
\textbf{Treasure}: None \\
\textbf{Description} \\
A flesh golem is a monstrous collection of humanoid anatomical parts stolen and stitched together. Its cadaverous flesh has a pale green or yellowish hue. A flesh golem wears any type of outfit its creator desires, usually just a well-worn pair of pants. It has no Equipment or weapons. A flesh golem is over 2.4 meters tall and weighs 250 kg.

A flesh golem does not speak, although it may emit a kind of hoarse growl. He walks and moves with a jerky gait, as if he is not in full control of his body.

While many flesh golems are devoid of reason, there are tales of exceptional golems who have somehow retained memories of their former lives. The head (and therefore the brain) of these flesh golems must be the right combination of freshness and (in the previous life) decision, but luck and chance also seem to be of absolute importance so that during their creation the intellect is preserved. . Certainly, those who build flesh golems prefer to have unintelligent slaves rather than willful slaves of their own, as a result intelligent flesh golems are rare.


\medskip\index[Mostruario]{Iron Golem} \textbf{Iron Golem}

\textit{Large construct, misaligned}

\textbf{STRENGTH} +7

\textbf{DEXTERITY} -1

\textbf{CONSTITUTION} +5

\textbf{INTELLIGENCE} -4

\textbf{WISDOM} +0

\textbf{CHARISMA} -5

\textbf{Initiative} -1 - \textbf{Defense} 28

\textbf{Hit Points} 210 (20d10 + 100)

\textbf{Movement} 9 m

\textbf{Saving Throws}: Fortitude +6, Reflex +5, Will +6

\textbf{Immunity to Damage} fire, poison; from a non-magical weapon or that are not adamantium

\textbf{Condition Immunity} fascinated, poisoned, paralyzed, petrified, fatigue, frightened

\textbf{Senses} darkvision 36 m

\textbf{Languages} understands the languages of its creator but cannot speak

\textbf{Challenge} 16 (15000 PX)

\textit{\textbf{Magical Weapons.}} The golem's weapon attacks are magical.

\textit{\textbf{Absorb Fire.}} Whenever the golem is the victim of fire damage, it takes no damage but instead regains an equal number of hit points.

\textit{\textbf{Immutable Form.}} The golem is immune to any spells or effects that would alter its form.

\textit{\textbf{Nature of Construct.}} A golem does not need air, food, drink, or sleep.

\textit{\textbf{Magic Resistance.}} The golem has + 1d6 on saving throws against spells and other magical effects.

\textbf{Actions}

\textit{\textbf{Multiattack.}} The golem makes two melee attacks.

\textit{\textbf{Slam.} Melee Weapon Attack}: +30 to hit, 1m range, one target.

\textit{Hits:} 20 (3d8 + 7) hit damage.

\textit{\textbf{Sword.} Melee Weapon Attack}: +30 to hit, 3m range, one target.

\textit{Strikes:} 23 (3d10 + 7) slashing damage.

\textit{\textbf{Poisonous Breath (Recharge 6).}} The golem exhales a poisonous gas into a 5 meter cone. Each creature in that area must make a DC 19 Fortitude saving throw, taking 45 (10d8) poison damage if it fails the saving throw, or half that damage if it succeeds.

\textbf{Ecology} \\
Environment: Any \\
Organization: Solitary or group (2-4) \\
\textbf{Treasure}: None \\
\textbf{Description} \\
An iron golem has a humanoid shaped body made of iron. The creator can give it any shape he wishes, but he almost always presents armor of some kind, be it ceremonial and precious or simple and usable. Compared to a stone golem it has a much more defined appearance. Iron golems sometimes carry a weapon with them, although more often than not they tend to prefer their slam attacks.

An iron golem is 3.6m tall and weighs approximately 2,500kg. An iron golem cannot speak or utter a voice. Furthermore, it does not emit any recognizable odors.

Although the practice of making iron golems gradually fell into disuse, venerable members of some great civilizations of the past considered the ability to forge iron golems of bewildering strength and size to be a source of pride. These golems (larger than or equal to Huge), in some remote corners of the world, still exist, and still mechanically carry out orders given to them by now vanished empires.

\textbf{Construction}
To build an iron golem, you need 2,500 kg of iron, cast with rare dyes worth 10,000 gp or more.


\medskip\index[Mostruario]{Stone Golem} \textbf{Stone Golem}

\textit{Large construct, misaligned}

\textbf{STRENGTH} +6

\textbf{DEXTERITY} -1

\textbf{CONSTITUTION} +5

\textbf{INTELLIGENCE} -4

\textbf{WISDOM} +0

\textbf{CHARISMA} -5

\textbf{Initiative} -1 - \textbf{Defense} 22

\textbf{Hit Points} 178 (17d10 + 85)

\textbf{Movement} 9 m

\textbf{Saving Throws}: Fortitude +4, Reflex +3, Will +4

\textbf{Immunity to Damage} poison; from a non-magical weapon or that are not adamantium

\textbf{Condition Immunity} fascinated, poisoned, paralyzed, petrified, fatigue, frightened

\textbf{Senses} darkvision 36 m

\textbf{Languages} understands the languages of its creator but cannot speak

\textbf{Challenge} 10 (5.900 PX)

\textit{\textbf{Magical Weapons.}} The golem's weapon attacks are magical.

\textit{\textbf{Immutable Form.}} The golem is immune to any spells or effects that would alter its form.

\textit{\textbf{Construct Nature.}} A golem does not need air, food, drink, or sleep.

\textit{\textbf{Magic Resistance.}} The golem has + 1d6 on saving throws against spells and other magical effects.

\textbf{Actions}

\textit{\textbf{Multiattack.}} The golem makes two slam attacks.

\textit{\textbf{Slam.} Melee Weapon Attack}: +21 to hit, 1m range, one target.

\textit{Hits:} 19 (3d8 + 6) hit damage.

\textit{\textbf{Slow (Cooldown 5-6).}} The golem targets one or more creatures within 10 feet of him that he can see. Each target must make a DC 17 Will saving throw against this spell. If the saving throw fails, the target cannot use reactions, has half speed, and cannot make more than one attack during its round. In addition, the target may take one action or one bonus action during its round, but not both. These effects last for 1 minute. The target can re-roll the saving throw at the end of each of its rounds, ending the effect for itself if successful.

\textbf{Ecology} \\
Environment: Any \\
Organization: Solitary or group (2-4) \\
\textbf{Treasure}: None \\
\textbf{Description} \\
A stone golem has a humanoid body made of stone, often stylized to suit its creator. For example, it can be sculpted to wear armor, with particular symbols carved into the breastplate, or have designs inlaid into the stone of its limbs. The head is often sculpted to look like a helmet or the head of some beast. Although he can be carved with a shield or a stone weapon such as a sword, these aesthetic choices do not affect his combat abilities.

As with most golems, a stone golem cannot speak and makes no sound other than that of the stone rubbing against the stone as it moves. A stone golem is 2.7 meters tall and weighs around 1000 kg.

There are numerous variants of Stone Golems, depending on the materials they are made of but also as expressions of elemental spirits, it is possible that an elemental spirit inhabits a "rock" (or gem) and defines its appearance and animates it as own body.

\textbf{Construction}
A stone golem's body is carved from a single block of hard stone, such as granite, weighing at least 1,500 kg. The stone must be of exceptional quality, costing 5,000 gp.


\medskip\index[Mostruario]{Gorgone} \textbf{Gorgone}

\textit{Large monstrosity, misaligned}

\textbf{STRENGTH} +5

\textbf{DEXTERITY} +0

\textbf{CONSTITUTION} +4

\textbf{INTELLIGENCE} -4

\textbf{WISDOM} +1

\textbf{CHARISMA} -2

\textbf{Initiative} +0 - \textbf{Defense} 22

\textbf{Hit Points} 114 (12d10 + 48)

\textbf{Movement} 12 m

\textbf{Saving Throws}: Fortitude +13, Reflex +6, Will +7

\textbf{Skills} Awareness +4

\textbf{Condition Immunity} Petrified

\textbf{Senses} darkvision 18 m

\textbf{Languages} -

\textbf{Challenge} 5 (1.800 PX)

\textit{\textbf{Overwhelming Charge.}} If the gorgon moves at least 20 feet in a straight line to the target and hits it with a gore attack during the same turn, the target must succeed on a Fortitude saving throw DC 16 or falling prone. If the target is prone, the gorgon can make a hoof attack against him as a bonus action.

\textbf{Actions}

\textit{\textbf{Gored.} Melee Weapon Attack}: +12 to hit, 1m range, one target.

\textit{Strikes:} 18 (2d12 + 5) piercing damage.

\textit{\textbf{Hooves.} Melee Weapon Attack}: +12 to hit, 1m range, one target.

\textit{Hits:} 16 (2d10 + 5) hit damage.

\textit{\textbf{Petrifying Breath (Recharge 5-6).}} The gorgon exhales a petrifying gas into a 30-foot cone. Each creature in that area must succeed on a DC 13 Fortitude saving throw. If the saving throw fails, the target begins to turn to stone and is hampered. The hampered target must re-roll the saving throw at the end of his next round. If successful, the target's effect ends. If it fails, the target is petrified until released from the spell \textit{greater restore} or similar spell.

\textbf{Ecology} \\
Environment Temperate Plains, Rocky Hills and Underground \\
Organization: Solitary, pair, pack (3-4) or herd (5-12) \\
\textbf{Treasure}: None \\
\textbf{Description} \\
Gorgons are magical and irascible creatures: although at first glance they may appear to be constructs, underneath the artificial-looking metal plates they are made of flesh and bone. Like aggressive bulls, they defy any unknown creature they encounter, often engulfing their opponent's corpse or shattering his petrified remains until the creature is no longer recognizable. Females are as dangerous as males, and the two sexes look the same. A typical gorgon is 1.8 meters tall and 2.4 meters long. It weighs about 2000 kg.

Gorgons derive their nourishment by consuming minerals, particularly the stone of their petrified victims, and every statue they create is completely devoured. They cannot digest metal or gems, so their dung (which looks like acrid-smelling gray dust) often contains small raw crystals and gold nuggets. Their aggressiveness towards all other creatures means that in their pastures there are few, if any, predators and prey. Each herd is led by a dominant bull; solitary gorgons are generally adolescent bulls turned away from the herd of the dominant bull.

Their flesh is tough and muscular (once the armor is removed), and for those who taste it it is quite nutritious. Many tribes of stone giants believe that eating gorgon meat increases their natural armor. Pulverized gorgon horns are worth 250 gp as an alternate material component for magic items and spells that affect Force or Stone.


\medskip\index[Mostruario]{Grick} \textbf{Grick}

\textit{Medium monstrosity, neutral}

\textbf{STRENGTH} +2

\textbf{DEXTERITY} +2

\textbf{CONSTITUTION} +0

\textbf{INTELLIGENCE} -4

\textbf{WISDOM} +2

\textbf{CHARISMA} -3

\textbf{Initiative} +2 - \textbf{Defense} 15

\textbf{Hit Points} 27 (6d8)

\textbf{Movement} 9 m, climb 9 m

\textbf{Saving Throws}: Fortitude +3, Reflex +3, Will +2

\textbf{Damage Resistance} non-magical weapon

\textbf{Senses} darkvision 18 m

\textbf{Languages} -

\textbf{Challenge} 2 (450 PX)

\textit{\textbf{Stone Camouflage.}} The grick has + 1d6 on Dexterity (Hide) checks made for hiding on rocky terrain.

\textbf{Actions}

\textit{\textbf{Multiattack.}} The grick makes an attack with its tentacles. If the attack hits, the grick can make a beak attack against the same target.

\textit{\textbf{Tentacles.} Melee Weapon Attack}: +4 to hit, 1m range, one target.

\textit{Strikes:} 9 (2d6 + 2) slashing damage.

\textit{\textbf{Beak.} Melee Weapon Attack}: +4 to hit, 1m range, one target.

\textit{Strikes:} 5 (1d6 + 2) piercing damage.

\textbf{Ecology}: \\
Environment any dungeon \\
Organization: Solitary or cluster (2-5) \\
\textbf{Treasure}: Accidental \\

\textbf{Description}
The grick worm is the terror of the caves and burrows it inhabits, waiting in ambush near busy tunnels or underground cities, to leap out of the dark and capture its prey. It is rare for such prey to be consumed on the spot. The grick prefers to take fresh food to its lair, a narrow passageway or to the high ledge of a cave, where it can consume it with small bites, in peace.

The origins of the grick are unknown. And while he has a rudimentary intelligence, he has no society to speak of, and most of the time you meet single specimens. On occasions when unfortunate travelers encounter more than one, groups of gricks don't seem to communicate or work with each other - each instead attacks individual targets and retreats with their loot as soon as they manage to take down an opponent. Capable predators, gricks also have a strange, weapon-resistant skin that makes them particularly dangerous. Many inexperienced adventurers perished under a grick's attack simply because they were unable to harm the creature with their non-magical weapons. Those familiar with gricks (especially Dwarves, Morlocks, and Troglodytes) know that the best strategy for dealing with them is to retreat and wait for more powerful or magical reinforcements.

Gricks rely on their dark color and ability to scale walls to keep themselves out of sight until they're ready to attack. On multiple occasions when food is scarce in a particular region, gricks make their way to the surface and roam the desert in search of prey, but these stays are almost always out of necessity, and eventually gricks soon find entrances to new underground burrows. . They prefer darkness and the comfort of a "roof" over their head, avoiding the open sky and doing a lot to stay covered by trees, low clouds or buildings.


\medskip\index[Mostruario]{Grifone} \textbf{Grifone}

\textit{Large monstrosity, misaligned}

\textbf{STRENGTH} +4

\textbf{DEXTERITY} +2

\textbf{CONSTITUTION} +3

\textbf{INTELLIGENCE} -4

\textbf{WISDOM} +1

\textbf{CHARISMA} -1

\textbf{Initiative} +2 - \textbf{Defense} 13

\textbf{Hit Points} 59 (7d10 + 21)

\textbf{Movement} 9 m, flight 24 m

\textbf{Saving Throws}: Fortitude +7, Reflex +6, Will +4

\textbf{Skills} Awareness +5

\textbf{Senses} darkvision 18 m

\textbf{Languages} -

\textbf{Challenge} 2 (450 PX)

\textit{\textbf{Sharpened Sight.}} The griffin has + 1d6 on sight-based Wisdom checks.

\textbf{Actions}

\textit{\textbf{Multiattack.}} The griffin makes two attacks: one with its beak and one with its claws.

\textit{\textbf{Claws.} Melee Weapon Attack}: +7 to hit, 1m range, one target.

\textit{Strikes:} 11 (2d6 + 4) slashing damage, 1 bleed damage.

\textit{\textbf{Beak.} Melee Weapon Attack}: +7 to hit, 1m range, one target.

\textit{Strikes:} 8 (1d8 + 4) piercing damage.

\textbf{Ecology} \\
Environment: Temperate Hills \\
Organization: Solitary, pair or pack (6-10) \\
\textbf{Treasure}: Accidental \\
\textbf{Description} \\
Griffins are powerful aerial predators, swooping from their towering nests to grab their prey with their beak and claws. Aggressive and territorial, they are not mere beasts, but shrewd fighters and loyal companions to those who earn their respect, fighting to the death to protect their friends and fellow men.

Weighing over 250 kg and 2.4 meters long, with a sharp beak to a crested tail, the griffin possesses an imposing profile that has long been used in heraldry and other iconography as a symbol of power, authority and justice. In reality, the griffon vulture is less interested in abstract concepts and more in hunting for food and defense. Although they can sometimes be trained or befriended to serve as mounts, griffins do not possess an innate affinity for humanoids, and often enter into bloody conflict with civilized races in an attempt to forage for their favorite food: horse meat. City folks may marvel at a trained griffin's stately style and 7-meter wingspan, but those farmers forced to share territory with its species know it pays to hurry indoors and secure their flocks when they hear the hunting cries of the beasts.

Griffins mate for life, and often seek revenge for years for killing a mate or a child. It was precisely for this innate stubbornness and fierce loyalty that brought them into domestic use as mounts and guardians of treasures. Despite the inherent danger, the trade in captured griffins and stolen eggs is fervent, with eggs worth up to 2,000 gp each and young alive up to 3,000. Characters who want a griffin as a mount, however, should know. that buying or violently taming intelligent creatures such as griffins is considered slavery by most good deities, and earning the spontaneous loyalty of a griffin is no easy task. Reaching a mutual agreement (or even friendship) is a much more elegant and safer way to secure a griffin as a mount.

Before it can be ridden in combat, a griffin must practice carrying the weight of its rider. To be well trained, a griffin must first have a friendly attitude towards its trainer (with a Handle Animal, Diplomacy, or Intimidate check). After that, 6 weeks of practice and a successful Handle Animal check with DC 20 is enough for the beast to be comfortable with the load, and due to their intelligence, trained griffins can be assumed to know all the tricks listed in the description of the beast. Handling Animals skills, and it is also possible that they will learn new commands by making simple requests in the Municipality.

Griffins can carry up to 150 kg as a light load, 300 kg as a medium load, and 450 kg as a heavy load. To ride a griffin you need an exotic saddle.


\medskip\index[Mostruario]{Grimlock} \textbf{Grimlock}

\textit{Medium humanoid (grimlock), neutral evil}

\textbf{STRENGTH} +3

\textbf{DEXTERITY} +1

\textbf{CONSTITUTION} +1

\textbf{INTELLIGENCE} -1

\textbf{WISDOM} -1

\textbf{CHARISMA} -2

\textbf{Initiative} +1 - \textbf{Defense} 12

\textbf{Hit Points} 11 (2d8 + 2)

\textbf{Movement} 9 m

\textbf{Saving Throws}: Fortitude +3, Reflex +1, Will +0

\textbf{Skills} Acrobatics +5, Move silently / Hide +3, Awareness +3

\textbf{Condition Immunity} blinded

\textbf{Senses} blind sight 9 m or 3 m if deaf (blind beyond this radius)

\textbf{Languages} Language of the Depths

\textbf{Challenge} 1/4 (50 PX)

\textit{\textbf{Stone Camouflage.}} The grimlock has + 1d6 on Dexterity (Hide) checks made for hiding on rocky terrain.

\textit{\textbf{Blind Senses.}} The grimlock cannot use blind sight while deaf and no longer sniff.

\textit{\textbf{Sharpened sense of smell and hearing.}} The grimlock has + 1d6 on Wisdom (Awareness) checks based on hearing or smell.

\textbf{Actions}

\textit{\textbf{Pointed Bone Club.} Melee Weapon Attack}: +5 to hit, 1 m range, one target.

\textit{Strikes:} 5 (1d4 + 3) hit damage plus 2 (1d4) piercing damage.

\textit{\textbf{Longbow.} Ranged weapon attack}: +3 to hit, range 45m, one target.

\textit{Strikes:} 5 (1d8 + 1) piercing damage.

\textbf{Ecology} \\
Grimlocks inhabit the abandoned settlements of other Races and are often found as slaves to other more organized creatures, such as the Duergar and Elves. They are believed to be an even more degenerate offshoot of the Morlocks, who travel to Sekamina to hunt Grimlocks for food and consider their meat a delicacy. \\
\textbf{Description} \\
Grimlocks are blind, savage humans who inhabit the realm of the deep dark lands, where they organize themselves into small tribal groups.

\medskip\index[Mostruario]{Guardian Protector} \textbf{Guardian Protector}

\textit{Large construct, misaligned}

\textbf{STRENGTH} +4

\textbf{DEXTERITY} -1

\textbf{CONSTITUTION} +4

\textbf{INTELLIGENCE} -2

\textbf{WISDOM} +0

\textbf{CHARISMA} -4

\textbf{Initiative} -1 - \textbf{Defense} 21

\textbf{Hit Points} 142 (15d10 + 60)

\textbf{Movement} 9 m

\textbf{Saving Throws}: Fortitude +6, Reflex +1, Will +2

\textbf{Immunity to Damage} poison

\textbf{Condition Immunity} fascinated, poisoned, paralyzed, fatigue, frightened

\textbf{Senses} darkvision 18 m, blind sight 3 m

\textbf{Languages} understands commands given in any language but cannot speak

\textbf{Challenge} 7 (2.900 PX)

\textit{\textbf{Accumulate Spells.}} A spellcaster wearing the guardian guardian amulet can cause the guardian to accumulate a spell of level 4 or lower. To do this, the caster must cast the spell on the guardian. The spell has no effect but is accumulated within the guardian. When commanded to do so by the wearer of the amulet or a predetermined situation arises from the caster, the guardian casts the accumulated spell with all the parameters set by the original caster, without the need for components. When the spell is cast or any new spell is accumulated, all previously accumulated spells are lost.

\textit{\textbf{Nature of Construct.}} The guardian does not need air, food, drink or sleep.

\textit{\textbf{Regeneration.}} The guardian guardian recovers 10 hit points at the start of his round if he still has at least 1 hit.

\textit{\textbf{Bound.}} The guardian protector is magically bound to an amulet. As long as the guardian and the amulet are on the same plane of existence, the wearer of the amulet can telepathically call the guardian to join him, and the guardian will know the distance and direction in which the amulet is. If the guardian is within 60 feet of the wearer, half of the wearer's damage (rounded down) is transferred to the guardian. If the amulet is destroyed, the guardian is incapacitated until a replacement amulet is created. The guardian amulet can be subject to direct attack if it is not worn or carried by anyone. Has Defense 10, 10 hit points, and immunity to poison damage. Making an amulet takes 1 week and costs 10,000 gp in parts.

\textbf{Actions}

\textit{\textbf{Multiattack.}} The golem makes two punch attacks.

\textit{\textbf{Punch.} Melee Weapon Attack}: +15 to hit, 1m range, one target.

\textit{Strikes:} 11 (2d6 + 4) hit damage.

\textbf{Reactions}

\textit{\textbf{Shield.}} When a creature attacks the wearer of the guardian's amulet, the guardian grants a +2 bonus to its Defense if within 1 meter of its controller.

\medskip\index[Mostruario]{Hobgoblin} \textbf{Hobgoblin}

\textit{Medium humanoid (goblinoid), lawful evil}

\textbf{STRENGTH} +1

\textbf{DEXTERITY} +1

\textbf{CONSTITUTION} +1

\textbf{INTELLIGENCE} +0

\textbf{WISDOM} +0

\textbf{CHARISMA} -1

\textbf{Initiative} +1 - \textbf{Defense} 19 (mail armor, shield)

\textbf{Hit Points} 11 (2d8 + 2)

\textbf{Movement} 9 m

\textbf{Saving Throws}: Fortitude +5, Reflex +2, Will +1

\textbf{Senses} darkvision 18 m

\textbf{Languages} Common, Goblin

\textbf{Challenge} 1/2 (100 PX)

\textit{\textbf{+ 1d6 Martial.}} Once per turn, the hobgoblin can deal an additional 7 (2d6) damage to a creature it hits with a weapon attack, if that creature is within 1 meter of an ally of the hobgoblin who is not incapacitated.

\textbf{Actions}

\textit{\textbf{Longsword.} Melee Weapon Attack}: +3 to hit, 1 m range, one target.

\textit{Strikes:} 5 (1d8 + 1) slashing damage or 6 (1d10 + 1) slashing damage when used with two hands.

\textit{\textbf{Longbow.} Ranged weapon attack}: +3 to hit, range 45m, one target.

\textit{Strikes:} 5 (1d8 + 1) piercing damage.

\textbf{Ecology} \\
Environment: Temperate Hills \\
Organization: Group (4-9), warband (10-24) or tribe (25+ plus 50 \% non-combatants, 1 3rd level sergeant for 20 adults, 1 or 2 4th or 5 lieutenants Level, 1 leader of 6th-8th level, 6-12 Leopards and 1-4 Ogre or 1-2 Troll) \\
\textbf{Treasure}: NPC gear (Studded Leather Armor, Light Metal Shield, Longsword, Longbow with 20 Arrows, other treasure) \\
\textbf{Description} \\
The Hobgoblins are militaristic and prolific, a combination that makes them very dangerous in some regions. They procreate quickly, replacing fallen members with new soldiers while keeping their numbers constant regardless of the fate of the war. Generally it takes little for them to declare war, but in most cases the reason is to capture new slaves: life as a slave in a Hobgoblin lair is brutal and short, and new slaves are always needed to replace those who die or are eaten. .

Of all the Goblinoid Races, the Hobgoblin is by far the most civilized.
They see the larger, loner Bugbears as tools to be hired and used where needed, usually for specific missions that involve murder and theft, and they look to the smaller species of Goblins with a mixture of shame and frustration. Hobgoblins admire the tenacity of Goblins, although the unpredictable nature and passion for fire of their tiny relatives make them unwelcome additions to Hobbgoblin tribes or settlements. However, most of the Hobgoblin tribes include a small group of Goblins, who normally hide in the worst corners of the settlement.

Many Hobgoblin tribes combine a love of war with a sharp intellect. The science of siege engines, alchemy and complex engineering feats fascinate most Hobgoblins, and those particularly gifted are treated like heroes and always gain high-ranking positions in the tribe. Slaves with refined minds are valued, so raids into Dwarven cities are commonplace.

Hobgoblins are known to be wary of Magic and despise it, especially Arcane. Their Shamans are regarded with a mixture of fear and respect, and are normally forced to live alone on the fringes of the tribe's lair. No one has ever heard of Hobgoblins practicing Arcane Magic or, as the Hobgoblins say, "Elf Magic". This is the cause of their hatred of Magic: Hobgoblins hate Elves.

A Hobgoblin is 1 meter tall and weighs 80 kg.


\medskip\index[Mostruario]{Hydra} \textbf{Hydra}

\textit{Huge monstrosity, misaligned}

\textbf{STRENGTH} +5

\textbf{DEXTERITY} +1

\textbf{CONSTITUTION} +5

\textbf{INTELLIGENCE} -4

\textbf{WISDOM} +0

\textbf{CHARISMA} -2

\textbf{Initiative} +1 - \textbf{Defense} 19

\textbf{Hit Points} 172 (15d12 + 75)

\textbf{Movement} 9 m, swim 9 m

\textbf{Saving Throws}: Fortitude +8, Reflex +7, Will +3

\textbf{Skills} Awareness +6

\textbf{Senses} darkvision 18 m

\textbf{Languages} -

\textbf{Challenge} 8 (3.900 PX)

\textit{\textbf{Multiple Heads.}} The hydra has five heads. As long as it has more than one head, the hydra has + 1d6 on saving throws against conditions blinded, fascinated, deafened, frightened, stunned, or passed out.

Whenever the hydra takes 25 or more damage in a single turn, one of its heads dies. If all the heads die, the hydra also dies.

At the end of its round, the hydra regrows two heads for each of its heads killed from its last turn, unless it took fire damage from its last turn. The hydra recovers 10 hit points for each head regrown this way.

\textit{\textbf{Reactive Heads.}} For each head possessed beyond the first, the hydra receives an extra Reaction Action that can only be used to make Awareness checks.

\textit{\textbf{Hold Your Breath.}} The hydra can hold its breath for 1 hour.

\textit{\textbf{Wake.}} While the hydra sleeps, at least one of its heads remains awake.

\textbf{Actions}

\textit{\textbf{Multiattack.}} The hydra makes as many bite attacks as there are heads.

\textit{\textbf{Bite.} Melee Weapon Attack}: +13 to hit, 3m range, one target.

\textit{Strikes:} 10 (1d10 + 5) piercing damage.

\textbf{Ecology} \\
Environment: Temperate Marshes \\
Organization: Solitary \\
\textbf{Treasure}: Standard \\
\textbf{Description} \\
Hydra is a multi-headed, but stupid dragon.


\medskip\index[Mostruario]{Hippogriff} \textbf{Hippogriff}

\textit{Large monstrosity, misaligned}

\textbf{STRENGTH} +3

\textbf{DEXTERITY} +1

\textbf{CONSTITUTION} +1

\textbf{INTELLIGENCE} -4

\textbf{WISDOM} +1

\textbf{CHARISMA} -1

\textbf{Initiative} +1 - \textbf{Defense} 12

\textbf{Hit Points} 19 (3d10 + 3)

\textbf{Movement} 12 m, flight 18 m

\textbf{Saving Throws}: Fortitude +5, Reflex +5, Will +2

\textbf{Skills} Awareness +5

\textbf{Languages} -

\textbf{Challenge} 1 (200 PX)

\textit{\textbf{Sharp Sight.}} The hippogriff has + 1d6 on sight-based Wisdom checks.

\textbf{Actions}

\textit{\textbf{Multiattack.}} The hippogriff makes two attacks: one with its beak and one with its claws.

\textit{\textbf{Claws.} Melee Weapon Attack}: +5 to hit, 1m range, one target.

\textit{Strikes:} 10 (2d6 + 3) slashing damage.

\textit{\textbf{Beak.} Melee Weapon Attack}: +5 to hit, 1m range, one target.

\textit{Strikes:} 8 (1d10 + 3) piercing damage.

\textbf{Ecology} \\
Environment: Temperate Hills or Plains \\
Organization: Solitary, pair or flock (7-12) \\
\textbf{Treasure}: None \\
\textbf{Description} \\
The hippogriff has the wings, front legs and head of a large bird of prey and the tail and body of a magnificent horse. Since horses are griffins' favorite food, scholars claim that a wizard with a sense of humor long ago created this unfortunate fusion of a horse and a hawk as a joke.

The feathers of the hippogriff have a color similar to that of a hawk or an eagle; however, some breeders have managed to produce specimens with completely white or charcoal feathers. A hippogriff's torso and posterior extremities are very often bay, hazel, or gray in color, with some coats showing dappled or even palomino coloring. A hippogriff is 3.3 meters long and weighs up to 680 kg.

The territorial hippogriffs fiercely protect their domain. Hippogriffs must also guard the skies for other predators, as they are a favorite food of griffins, wyverns, and young dragons. Hippogriffs nest in vast grassy grasslands, rugged hills and flowing grasslands. Exceptionally hardy hippogriffs make their homes within niches or canyon walls, from which they scour rocky deserts for coyotes, deer, and sometimes humanoids. Hippogriffs prefer mammals, however they graze grass after any meat meal to aid digestion. Their dietary habits can be dangerous for both farmers and their herds, so often the farming communities put rewards on them. The victims of these hunting parties are often embalmed, and frequently embalmed hippogriffs decorate border taverns and remote outposts.

Far easier to train than griffins and as intelligent as horses, hippogriffs are trained as riding animals by a select company of horse soldiers, who patrol the skies and swoop in on unwitting foes. Although they are magical beasts, if captured when young, hippogriffs can be trained with Train Animals as if they were animals. An adult hippogriff is much more difficult to train, and to do so you must follow the normal rules for training magical beasts using this skill. A hippogriff saddle must be made in such a way as not to hinder the movement of the creature's wings; these saddles are always exotic saddles.

Hippogriffs are oviparous: as a general rule, a hippogriff's nest contains only one egg at a time. A hippogriff egg is worth 200 gp, but a healthy young hippogriff is worth 500 gp. A fully trained hippogriff as a mount can see its value rise to 5,000 gp or more. A hippogriff can carry 90 kg as a light load, 180 kg as a medium load, and 270 kg as a heavy load.


\medskip\index[Mostruario]{Kraken} \textbf{Kraken}

\textit{Gargantuan monstrosity (titan), chaotic evil}

\textbf{STRENGTH} +10

\textbf{DEXTERITY} +0

\textbf{CONSTITUTION} +7

\textbf{INTELLIGENCE} +6

\textbf{WISDOM} +4

\textbf{CHARISMA} +5

\textbf{Initiative} +6 - \textbf{Defense} 30

\textbf{Hit Points} 472 (27d20 + 189)

\textbf{Movement} 6m, swim 18m

\textbf{Saving Throws}: Fortitude +21, Reflex +12, Will +11

\textbf{Immunity to Damage} lightning, weapons +1

\textbf{Condition Immunity} paralyzed, frightened

\textbf{Senses} vision of the true 36 m

\textbf{Languages} includes Abyssal, Celestial, Hellish and Druidic but cannot speak, telepathy 36 m

\textbf{Challenge} 23 (50000 PX)

\textit{\textbf{Amphibian.}} The kraken can breathe air and water.

\textit{\textbf{Freedom of Movement.}} The kraken ignores difficult terrain, and magical effects cannot reduce its speed or cause it to become hindered. Can spend 1 meter of movement to break free from nonmagical restrictions or being grabbed.

\textit{\textbf{Siege Monster.}} The kraken deals double damage to objects and structures.

\textbf{Actions}

\textit{\textbf{Multiattack.}} The kraken makes three tentacle attacks, each of which can be replaced by a use of Slash.

\textit{\textbf{Bite.} Melee Weapon Attack}: +30 hit, 6m range, one target.

\textit{Strikes:} 23 (3d8 + 10) piercing damage. If the target is a Large or smaller creature grabbed by the kraken, that creature is swallowed, and the grabbing ends. While engulfed, the creature is blinded and entangled, has full cover against attacks and other effects from outside the kraken, and takes 42 (12d6) acid damage at the start of each kraken's turn.

If the kraken takes 50 or more damage in a single turn from a creature inside it, the kraken must succeed on a DC 25 Fortitude save or vomit all swallowed creatures, which fall prone into a space within 10 feet of the kraken. If the kraken dies, a swallowed creature is no longer hampered by it and can escape the corpse using 5 meters of movement, coming out prone.

\textit{\textbf{Tentacle.} Melee Weapon Attack}: +30 hit, range 9 m, one target.

\textit{Strikes:} 20 (3d6 + 10) hit damage, and the target is grabbed (DC 18 to escape). Until the grapple is complete, the target is in the way. The kraken has ten tentacles, each of which can grab a target.

\textit{\textbf{Sling.}} A grasped object or creature grasped by the kraken, Large or smaller, is thrown 60 feet in a random direction and thrown prone. If the thrown target hits a solid surface, it takes 3 (1d6) slam damage for every 10 feet traveled. If the target is cast at another creature, that creature must succeed on a DC 18 Reflex saving throw or take the same damage and fall prone.

\textit{\textbf{Lightning Storm.}} The kraken magically creates three bolts of energy, each of which can hit a target within 100 feet and which the kraken can see. The target must make a DC 23 Reflex saving throw, and take 22 (4d10) lightning damage on a failed save, or half if successful.

\textbf{Additional Actions}

The kraken can perform 3 additional Actions, chosen from the following options. He can only use one legendary option at a time, and only at the end of another creature's turn. The kraken recovers any additional Actions spent at the start of its round.

\textbf{Tentacle Attack or Sling.} The kraken makes a tentacle attack or uses Sling.

\textbf{Cloud of Ink (Costs 3 Actions).} While underwater, the kraken expels a cloud of ink with a radius of 60 feet. The cloud spreads around corners, and that area is heavily obscured for all creatures except the kraken. Each creature other than the kraken that ends its round in the area must succeed at a Fortitude save 23, taking 16 (3d10) poison damage on a failed save, or half if it succeeds. A strong current disperses the cloud, which otherwise vanishes at the end of the next kraken round. \textbf{Lightning Storm (Costs 2 Actions).} The kraken uses Lightning Storm.

\textbf{Ecology} \\
Environment any ocean \\
Organization: Solitary \\
\textbf{Treasure}: Triple \\
\textbf{Description} \\
The legendary kraken is one of the sailors' greatest fears, because it is a creature the size of a whale, it can strike depths without being seen, it can command the winds and weather conditions necessary for the ship to move, and it possesses the cruel intellect of most of the most ruthless and creative criminals in the world. Some believe that krakens are divine punishment, while others believe them to be the true lords of the depths, who consider the air-breathing races to be nothing more than cattle.

Many legends have arisen regarding his understanding of Druidic language.

A kraken is nearly 30 meters long and weighs 2000 kg.


\medskip\index[Mostruario]{Lamia} \textbf{Lamia}

\textit{Large monstrosity, chaotic evil}

\textbf{STRENGTH} +3

\textbf{DEXTERITY} +1

\textbf{CONSTITUTION} +2

\textbf{INTELLIGENCE} +2

\textbf{WISDOM} +2

\textbf{CHARISMA} +3

\textbf{Initiative} +2 - \textbf{Defense} 15

\textbf{Hit Points} 97 (13d10 + 26)

\textbf{Movement} 9 m

\textbf{Saving Throws}: Fortitude +6, Reflex +9, Will +11

\textbf{Skills} Move Silently / Hide +3, Deceive +7, Feel Emotions +4,

\textbf{Senses} darkvision 18 m

\textbf{Languages} Abyssal, Common

\textbf{Challenge} 4 (1.100 PX)

\textit{\textbf{Innate Spells.}} The innate spellcasting characteristic of the lamia is Charisma - The lamia can innately cast the following spells, without the need for material components:

At will: \textit{disguise yourself} (any humanoid form) \textit{,} \textit{major image}

3 / Day each: \textit{charm on people, mirror image,}

\textit{scrutinize, suggestion}

1 / Day: \textit{restriction}

\textbf{Actions}

\textit{\textbf{Multiattack.}} The lamia makes two attacks: one with claws and one with dagger or Intoxicating Touch.

\textit{\textbf{Claws.} Melee Weapon Attack}: +9 to hit, 1m range, one target.

\textit{Hits:} 14 (2d10 + 3) slashing damage, 1 bleed damage.

\textit{\textbf{Dagger.} Melee Weapon Attack}: +9 to hit, 1m range, one target.

\textit{Strikes:} 5 (1d4 + 3) piercing damage.

\textit{\textbf{Intoxicating Touch.} Melee spell attack}: +5 to hit, range 1 yd, a creature.

\textit{Strikes:} The target is cursed for 1 hour by this spell. Until the curse ends, the target has -1d6 on Will saving throws and all proficiency checks.

\textbf{Ecology} \\
Environment: Temperate Deserts \\
Organization: Solitary, couple or sect (3-12) \\
\textbf{Treasure}: Double (Dagger + 1, other treasure) \\

\textbf{Description} \\
Hateful heirs of an ancient curse, lamias look like slender, attractive women from the waist up, while underneath they have the body of a mighty lion. Their humanoid features also bear the hallmarks of felines, their eyes are narrow and feral, and their teeth resemble the fangs of predators. A typical standing lamia is 1.8 meters tall, 2.4 meters long, and weighs more than 325 kg.

The lamias are drawn to ruined towers, abandoned cities and forgotten monuments that satisfy the crude aesthetic standards of these lethal hunters; especially those in arid or sterile areas. However, the lamias favor decrepit temples. They delight in seeing the temples of good gods in ruins and stray from their path to disrupt these thriving sacred places.

Lamias view the older females of their group as leaders, mothers and shamans, latching on to them with fanatical reverence. Although lamias shy away from most religions, seeing them as the source of the curse that has forced them into these bestial forms, the elderly lamias claim to hear the whispers of the wind scourging the desert and to know the cold whims of the stars, and they rely on these mystical springs to guide their people.

The lamias presented here are only the most common and least powerful exponents of this cursed race; others have serpentine, flying and even more perverse shapes.


\medskip\index[Mostruario]{Lich} \textbf{Lich}

\textit{Undead media, evil traits}

\textbf{STRENGTH} +0

\textbf{DEXTERITY} +3

\textbf{CONSTITUTION} +3

\textbf{INTELLIGENCE} +5

\textbf{WISDOM} +2

\textbf{CHARISMA} +3

\textbf{Initiative} +5 - \textbf{Defense} 28

\textbf{Hit Points} 135 (18d8 + 54)

\textbf{Movement} 9 m

\textbf{Saving Throws}: Fortitude +11, Reflex +12, Will +16

\textbf{Resistance to Damage} cold, lightning, from Void

\textbf{Immunity to Damage} poison; non-magical weapon

\textbf{Condition Immunity} fascinated, poisoned, paralyzed, fatigue, frightened

\textbf{Senses} vision of the true 36 m

\textbf{Languages} Common plus five other languages

\textbf{Challenge} 21 (33000 PX)

\textit{\textbf{Spells.}} The lich has CM 18. His spellcasting characteristic is Intelligence, +5 to hit with spell attacks). The lich knows the following spells:

Tricks (at will): \textit{magic hand, prestidigitation, ray} \textit{of frost}

level 1 (4 slots): \textit{magic missile, detect magic,} \textit{thundering wave, shield}

level 2 (3 slots): \textit{acid arrow, mirror image,} \textit{detection of thoughts, invisibility}

level 3 (3 slots): \textit{animate dead, counterspell, dispel} \textit{spells, fireball}

level 4 (3 slots): \textit{wither, dimensional gate}

level 5 (3 slots): \textit{death cloud, scan}

level 6 (1 slot): \textit{disintegration, globe of invulnerability}

level 7 (1 slot): \textit{finger of death, planar shift}

level 8 (1 slot): \textit{dominate monsters, word of power stun}

level 9 (1 slot): \textit{word of power to kill}

\textit{\textbf{Undead Nature.}} The lich does not need air, food, drink or sleep.

\textit{\textbf{Legendary Resistance (3 / Day).}} If the lich fails a saving throw, it may choose to succeed instead.

\textit{\textbf{Resistance to turning.}} The lich has + 1d6 on saving throws against turning undead effects.

\textit{\textbf{Reinvigoration.}} If it possesses a phylactery, the destroyed lich gains a new body in 1d10 days, regaining all of its hit points and returning to activity. The new body appears within 1 meter of the phylactery.

\textit{\textbf{Sacrifices of Souls.}} A lich must periodically feed his phylactery with souls to sustain the magic that maintains his body and consciousness. To do this, use the spell \textit{imprison}. Instead of choosing one of the normal spell options, the lich uses it to magically trap the target's body and soul within the phylactery. The phylactery must be on the same plane as the lich for this spell to work. A lich's phylactery can only contain one creature at a time, and \textit{dispel magic} cast as a level 9 spell on the phylactery frees any creature imprisoned within it. A creature imprisoned in phylactery for 24 hours is consumed and destroyed, after which nothing but divine intervention can bring it back to life.

A lich that forgets or fails to maintain its body with sacrificed souls begins to fall apart, and may eventually transform into a semilich.

\textbf{Actions}

\textit{\textbf{Paralyzing Touch.} Melee spell attack}: +18 to hit, range 1 yd, a creature.

\textit{Strikes:} 10 (3d6) cold damage. The target must succeed on a DC 18 Fortitude saving throw or be paralyzed for 1 minute. The target can re-roll the saving throw at the end of each of its rounds, ending the effect on itself if successful.

\textbf{Additional Actions}

The lich can perform 3 additional Actions, chosen from the following options. He can only use one legendary option at a time, and only at the end of another creature's turn. The lich recovers any additional Actions spent at the start of their round.

\textit{\textbf{Destroy Life (Costs 3 Actions).}} Any creature other than the undead within 20 feet of the lich must make a DC 18 Fortitude save against this spell, taking 21 (6d6) damage from Empty if it fails the saving throw, or half that damage if it succeeds. Creatures become Fatigued.

\textit{\textbf{Scary Gaze (Costs 2 Actions).}} The lich fixes its gaze on a visible creature within 10 feet of it. The target must succeed on a DC 18 Will saving throw against this spell or be frightened for 1 minute. The frightened target can re-roll the saving throw at the end of each of his rounds, ending the effect on himself if he succeeds. If the target's saving throw is successful or the effect ends for him, the target is immune to the lich's gaze for the next 24 hours.

\textit{\textbf{Paralyzing Touch (Costs 2 Actions).}} The lich uses his Paralyzing Touch.

\textit{\textbf{Trick.}} The lich launches a trick.

\textbf{Ecology} \\
Environment: Any \\
Organization: Solitary \\
\textbf{Treasure}: NPC gear (Ring of Protection +2, Sash of Lore +2 (Awareness), Boots of Levitation, Scroll of Dominate People, Scroll of Teleportation, Potion of Invisibility) \\

\textbf{Description}
Few creatures are more feared than liches. Apex of the necromantic arts, the lich is a spellcaster who has chosen to give up life and cheat death by becoming undead. While many of those who reach such heights of power would go to great lengths to achieve immortality, the idea of becoming a lich is abhorred by many creatures. The process involves extracting the life force of the caster and imprisoning it in a specially prepared phylactery; the caster surrenders his life, but remains trapped between life and death, and as long as his phylactery remains intact he can continue his research and work without fearing the passage of time.



\medskip\index[Mostruario]{Lizard} \textbf{Lizard}

\textit{Medium humanoid (lizard), neutral}

\textbf{STRENGTH} +2

\textbf{DEXTERITY} +0

\textbf{CONSTITUTION} +1

\textbf{INTELLIGENCE} -2

\textbf{WISDOM} +1

\textbf{CHARISMA} -2

\textbf{Initiative} +0 - \textbf{Defense} 16 (natural armor, shield)

\textbf{Hit Points} 22 (4d8 + 4)

\textbf{Movement} 9 m, swim 9 m

\textbf{Saving Throws}: Fortitude +4, Reflex +0, Will +0

\textbf{Skills} Move Silently / Hide +4, Awareness +3, Survival +5

\textbf{Languages} Draconic

\textbf{Challenge} 1/2 (100 PX)

\textit{\textbf{Hold Your Breath.}} The lizard can hold its breath for 15 minutes.

\textbf{Actions}

\textit{\textbf{Multiattack.}} The lizard makes two melee attacks, each with a different weapon.

\textit{\textbf{Javelin.} Melee or Ranged Weapon Attack}: +4 hit, 1 m range, 12m range, one target. \textit{Strikes:} 5 (1d6 + 2) piercing damage.

\textit{\textbf{Bite.} Melee Weapon Attack}: +4 hit, 1m range, one target.

\textit{Strikes:} 5 (1d6 + 2) piercing damage.

\textit{\textbf{Heavy Cudgel.} Melee Weapon Attack}: +4 to hit, 1m range, one target.

\textit{Strikes:} 5 (1d6 + 2) hit damage.

\textit{\textbf{Spiked Shield.} Melee Weapon Attack}: +4 to hit, 1m range, one target.

\textit{Strikes:} 5 (1d6 + 2) piercing damage.

\textbf{Ecology} \\
Environment: temperate swamps \\
Organization: solitary, couple, gang (3-12) or tribe (13-60) \\
\textbf{Treasure}: NPC gear (Heavy Wooden Shield, Spiked Club, 3 Javelins) \\

\textbf{Description} \\
Lizards are proud and powerful predatory reptiles that make their common homes in sparse villages in the recesses of swamps and marshes. With no interest in dryland colonization and satisfied with their simple weapons and rituals that have served them well for millennia, lizards are viewed by many of the other races as retrograde savages, but within their isolated communities they are actually a vital people rich in traditions and with an oral history that dates back to before man walked upright.

Most lizards are 1.8 to 2.1 meters tall and weigh 100 to 125 kg, and have massive muscles covered in gray, green or brown scales. Some stingrays have small dorsal crests or brightly colored collars, and all swim well by moving with quick movements of their mighty 1.2 meter long tail. Even though they are fully comfortable in the water, they hold their breath and return to their homes set on man-made hills to reproduce and sleep. Because their reptilian blood makes them slow in the cold, many lizards hunt and work during the day and retreat to their abodes at night to curl up with others of their tribe to share the warmth of large peat fires.

While generally neutral, the lizard's unfriendly behavior, their strenuous rejection of the "gifts" of civilization, and legendary ferocity in battle make them misjudged by most humanoids. These traits come for good reason, however, as their low reproduction rate is unmatched among warm-blooded humanoids, and if the tribes did not defend their swampy territories to their last breath they would soon find themselves overwhelmed by hordes of mammals. As for their propensity to eat the bodies of the dead both friend and foe, practical lizards are quick to point out that life is tough in the swamp, and nothing is to be wasted.

The lizards featured here live in swampy environments. Lizard tribes can live just as well in other environments, but for speed they gain Climb 5m instead of Swim.


\medskip\index[Mostruario]{Cursed immortal} \textbf{Cursed immortal}

\textit{Average aberration (human), basically insane}

\textbf{STRENGTH} +3

\textbf{DEXTERITY} +1

\textbf{CONSTITUTION} +2

\textbf{INTELLIGENCE} -1

\textbf{WISDOM} +1

\textbf{CHARISMA} -2

\textbf{Initiative} +3 - \textbf{Defense} 15

\textbf{Hit Points} 75 (12d8 + 21)

\textbf{Movement} 9 m

\textbf{Saving Throws} Fortitude +6, Reflex +5, Will +5

\textbf{Magic Resist} the immortal cursed has + 1d6 on each saving throw on spells

\textbf{Skills} Awareness +3, profession he had in life

\textbf{Immunity to Damage} cold, fire, emptiness

\textbf{Condition Immunity} fascinated, poisoned, petrified, frightened

\textbf{Unconscious} the Cursed One has no sense of taste or smell

\textbf{Languages} Common, dwarven, elven

\textbf{Challenge} 4 (1,100 PX)

\textit{\textbf{Immortal}} The Cursed Immortal regenerates 1 hit point per round, allowing him to regenerate limbs and come back to life. The only way to kill him is by dissolving him in magical acid. Remove Curse at DC 30 kills him instantly.

\textit{\textbf{Different nature}} The Immortal Cursed One does not eat, drink, sleep, age. He is not an undead

\textbf{Actions}

\textit{\textbf{Multiattack.}} The Immortal Accursed makes three longsword attacks.

\textit{\textbf{Sword.} Melee Weapon Attack}: +6 to hit, 1m range, one target.

\textit{Strikes:} 12 (1d10 + 7) slash damage.

\textbf{Ecology} \\
Environment: Any \\
Organization: Solitary \\
\textbf{Treasure}: NPC gear (Studded Leather Armor, 2 Daggers, Sword, other soro) \\
\textbf{Description} \\
The Cursed Immortal is a person who is often cursed by a Patron or a powerful spellcaster with the curse of the insane immortal life. The curse thunders the balance of the person and he finds himself wandering without a goal or a goal. Every so often they remember who they were and then they continue in search of who cursed them.
With the aim of being definitively killed, he throws himself into every fight hoping that the opponent will be able to kill him once and for all.


\subsection{Werewolves}

\medskip\index[Mostruario]{Werewolf Boar} \textbf{Werewolf Boar}

\textit{Medium humanoid (human, shapeshifter), neutral evil}

\textbf{STRENGTH} +3

\textbf{DEXTERITY} +0

\textbf{CONSTITUTION} +2

\textbf{INTELLIGENCE} +0

\textbf{WISDOM} +0

\textbf{CHARISMA} -1

\textbf{Initiative} +0 - \textbf{Defense} 12 in humanoid form, 13 in boar or hybrid form

\textbf{Hit Points} 78 (12d8 + 24)

\textbf{Movement} 9 m (12 m in boar form)

\textbf{Saving Throws}: Fortitude +7, Reflex +1, Will +4

\textbf{Skills} Awareness +2

\textbf{Immunity to Damage} from a non-magical or non-silver weapon

\textbf{Languages} Common (cannot speak in boar form)

\textbf{Challenge} 4 (1,100 PX)

\textit{\textbf{Charge (Boar Form or Hybrid Only).}} If the wereboar moves in a straight line at least 5 meters to a target and then hits it with its fangs during the same turn, the target it takes 7 (2d6) additional slashing damage. If the target is a creature, it must succeed on a DC 13 Fortitude saving throw or fall prone.

\textit{\textbf{Relentless (Cooldown after 1 hour).}} If the wereboar takes 14 damage or less that would reduce it to 0 hit points, it drops to 1 hit point instead.

\textit{\textbf{Shapeshifter.}} The wereboar can use its action to transform into a boar-humanoid hybrid or a boar, or to return to its true form, which is humanoid. Its stats, aside from Defense, are the same in all forms. Any equipment he is wearing or carrying is not transformed. At death it returns to its true form.

\textbf{Actions}

\textit{\textbf{Multiattack (Humanoid or Hybrid Form only).}} The wereboar makes two attacks, only one of which can be fanged.

\textit{\textbf{Mallet (Humanoid or Hybrid Form only).} Melee Weapon Attack}: +9 to hit, 1m range, one target. \textit{Strikes:} 10 (2d6 + 3) hit damage.

\textit{\textbf{Fangs (Boar or Hybrid Form only).} Melee Weapon Attack}: +9 to hit, 1m range, one target. \textit{Strikes:} 10 (2d6 + 3) slashing damage. If the target is a humanoid, he must succeed on a DC 12 Fortitude save or be cursed by the wereboar's lycanthropy.

\textbf{Ecology} \\
Environment any Forest or Plains \\
Organization: Solitary, couple, family (3-8) or troop (3-8 plus 1-4 Boars) \\
\textbf{Treasure}: NPC gear (Studded Leather Armor, 2 Daggers, other treasure) \\
\textbf{Description} \\
In their humanoid form, wereboars tend to be stocky, with upturned noses, shaggy fur, and prominent incisors. They have red, brown or black hair but some are also blond, white-haired or bald. They typically have hair on their upper lip, and males usually cannot grow a beard. Because they are stubborn and aggressive, they have small communities of their own kind and do not mix with non-werewolves - they usually live on small, absolutely normal looking farms. They tend to have large families and many children.


\medskip\index[Mostruario]{Werewolf} \textbf{Werewolf}

\textit{Medium humanoid (human, shapeshifter), chaotic evil}

\textbf{STRENGTH} +2

\textbf{DEXTERITY} +1

\textbf{CONSTITUTION} +2

\textbf{INTELLIGENCE} +0

\textbf{WISDOM} +0

\textbf{CHARISMA} +0

\textbf{Initiative} +1 - \textbf{Defense} 13 in humanoid form, 14 in wolf or hybrid form

\textbf{Hit Points} 58 (9d8 + 18)

\textbf{Movement} 9 m (12 m in wolf form)

\textbf{Saving Throws}: Fortitude +5, Reflex +1, Will +2

\textbf{Skills} Move Silently / Hide +3, Awareness +4

\textbf{Immunity to Damage} from a non-magical or non-silver weapon

\textbf{Languages} Common (cannot speak in wolf form)

\textbf{Challenge} 3 (700 PX)

\textit{\textbf{Shapeshifter.}} The werewolf can use his action to transform into a wolf-humanoid hybrid or wolf, or to return to his true form, which is humanoid. Its stats, aside from Defense, are the same in all forms. Any equipment he is wearing or carrying is not transformed. At death it returns to its true form.

\textit{\textbf{Hearing and refined smell.}} The werewolf has + 1d6 on Wisdom (Awareness) checks based on hearing or smell.

\textbf{Actions}

\textit{\textbf{Multiattack (Humanoid or Hybrid Form only).}} The werewolf makes two attacks: one with its bite and one with its claws or spear.

\textit{\textbf{Claws (Hybrid Form Only).} Melee Weapon Attack}: +6 to hit, range 1 yd, a creature. \textit{Strikes:} 7 (2d4 + 2) slashing damage.

\textit{\textbf{Spear (Humanoid form only).} Melee or Ranged Weapon Attack}: +4 to hit, range 1 m, range 6m, a creature.

\textit{Strikes:} 5 (1d6 + 2) piercing damage or 6 (1d8 + 2) piercing damage when used with two hands in a melee attack.

\textit{\textbf{Bite (Wolf or Hybrid form only).} Melee Weapon Attack}: +6 to hit, 1m range, one target.

\textit{Strikes:} 6 (1d8 + 2) piercing damage. If the target is a humanoid, he must succeed on a DC 12 Fortitude saving throw or be cursed by the werewolf's lycanthropy.

\textbf{Ecology} \\
Environment any terrain \\
Organization: Solitary, pair or pack (3-6) \\
\textbf{Treasure}: NPC gear (Chainmail, Longsword, Light Crossbow with 20 Squares, other treasure) \\
\textbf{Description} \\
In human form, werewolves resemble ordinary people, although some tend to have a feral appearance and unruly hair. Eyebrows growing together, index finger longer than the middle, and strange birthmarks on the palm are all commonly accepted signs that a person is actually a werewolf. Of course, these telltale signs aren't always accurate, because these physical traits also exist in normal people, but in areas where werewolves are a common problem, these traits can be considered overwhelming regardless.

\medskip\index[Mostruario]{Werebear} \textbf{Werebear}

\textit{Average humanoid (human, shapeshifter), neutral good}

\textbf{STRENGTH} +4

\textbf{DEXTERITY} +0

\textbf{CONSTITUTION} +3

\textbf{INTELLIGENCE} +0

\textbf{WISDOM} +1

\textbf{CHARISMA} +1

\textbf{Initiative} +0 - \textbf{Defense} 13 in humanoid form, 14

in bear or hybrid form

\textbf{Hit Points} 135 (18d8 + 54)

\textbf{Movement} 9m (12m, climb 9m in bear form or hybrid form)

\textbf{Saving Throws}: Fortitude +5, Reflex +6, Will +2

\textbf{Skills} Awareness +7

\textbf{Immunity to Damage} from a non-magical or non-silver weapon

\textbf{Languages} Common (cannot speak in bear form)

\textbf{Challenge} 5 (1.800 PX)

\textit{\textbf{Shapeshifter.}} The were bear can use its action to transform into a bear-humanoid hybrid or a bear, or to return to its true form, which is humanoid. Its stats, aside from Defense, are the same in all forms. Any equipment he is wearing or carrying is not transformed. At death it returns to its true form.

\textit{\textbf{Refined Smell.}} The were bear has + 1d6 on Wisdom (Awareness) checks based on smell.

\textbf{Actions}

\textit{\textbf{Multiattack.}} In bear form, the were bear makes two claw attacks. In humanoid form, make two double ax attacks. In hybrid form, it can attack like a bear or humanoid.

\textit{\textbf{Claw (Bear or Hybrid Form only).} Melee Weapon Attack}: +11 to hit, 1m range, one target. \textit{Strikes:} 13 (2d8 + 2) slashing damage.

\textit{\textbf{Double-headed Ax (Humanoid or Hybrid Form only).} Melee Weapon Attack}: +11 to hit, 1m range, one target. \textit{Strikes:} 10 (1d12 + 4) slashing damage.

\textit{\textbf{Bite (Bear or Hybrid Form only).} Melee Weapon Attack}: +11 to hit, 1m range, one target.

\textit{Strikes:} 15 (2d10 + 4) piercing damage. If the target is a humanoid, he must succeed on a DC 14 Fortitude saving throw or be cursed by the werewolf lycanthropy.


\textbf{Ecology} \\
Environment any forest \\
Organization: Solitary, couple, family (3-6) or troop (3-6 plus 1-4 Black or Gray bears) \\
\textbf{Treasure}: NPC gear (Mail Jack, Perfect Battle Ax, 2 Perfect Throwing Axes, other treasure) \\
\textbf{Description} \\
In their humanoid forms, werebears tend to be muscular and with broad shoulders, harsh features, and dark eyes. They have red, brown or black hair and look used to a life of hard work. Although the most benign of werewolves, they are shunned by most normal people, who fear their animal transformation. For the most part they live in isolated wooded areas or in small family units of their own species. They avoid confronting strangers, but they don't hesitate if they have to drive evil humanoids out of their territories.

\medskip\index[Mostruario]{Werewolfrat} \textbf{Werewolfrat}

\textit{Medium humanoid (human, shapeshifter), lawful evil}

\textbf{STRENGTH} +0

\textbf{DEXTERITY} +2

\textbf{CONSTITUTION} +1

\textbf{INTELLIGENCE} +0

\textbf{WISDOM} +0

\textbf{CHARISMA} -1

\textbf{Initiative} +2 - \textbf{Defense} 13

\textbf{Hit Points} 33 (6d8 + 6)

\textbf{Movement} 9 m

\textbf{Saving Throws}: Fortitude +2, Reflex +5, Will +3

\textbf{Skills} Move Silently / Hide +4, Awareness +2

\textbf{Immunity to Damage} from a non-magical or non-silver weapon

\textbf{Senses} darkvision 60 ft. (Only in rat form)

\textbf{Languages} Common (cannot speak in rat form)

\textbf{Challenge} 2 (450 PX)

\textit{\textbf{Shapeshifter.}} The wererat can use its action to transform into a rat-humanoid hybrid or rat, or to return to its true form, which is humanoid. Its stats, aside from Defense, are the same in all forms. Any equipment he is wearing or carrying is not transformed. At death it returns to its true form.

\textit{\textbf{Refined Smell.}} The wererat has + 1d6 on smell-based Wisdom (Awareness) checks.

\textbf{Actions}

\textit{\textbf{Multiattack (Humanoid or Hybrid Form only).}} The wererat makes two attacks, of which only one can be with the bite.

\textit{\textbf{Short Sword (Humanoid or Hybrid Form only).} Melee Weapon Attack}: +4 to hit, 1m range, one target. \textit{Strikes:} 5 (1d6 + 2) piercing damage.

\textit{\textbf{Hand Crossbow (Humanoid or Hybrid Form only).} Ranged Weapon Attack}: +4 to hit, range 9m, one target.

\textit{Strikes:} 5 (1d6 + 2) piercing damage.

\textit{\textbf{Bite (Rat or Hybrid Form only).} Melee Weapon Attack}: +4 to hit, 1m range, one target.

\textit{Strikes:} 4 (1d4 + 2) piercing damage. If the target is a humanoid, he must succeed on a DC 11 Fortitude save or be cursed by the wererat's lycanthropy.

\textbf{Ecology} \\
Environment: Any Urban \\
Organization: Solitary, pair, pack (5-10) or guild (11-30 plus 5-12 Cruel Rats) \\
\textbf{Treasure}: NPC Gear (Perfect Studded Leather Armor, Short Sword, Light Crossbow with 20 Bolts, other treasure) \\
\textbf{Description} \\
Natural werewolf rats are short, lean and muscular, with alert and lively eyes, and have nervous movements. Males often have a thin, shriveled mustache.

\medskip\index[Mostruario]{Werewolf Tiger} \textbf{Werewolf Tiger}

\textit{Medium humanoid (human, shapeshifter), neutral}

\textbf{STRENGTH} +3

\textbf{DEXTERITY} +2

\textbf{CONSTITUTION} +3

\textbf{INTELLIGENCE} +0

\textbf{WISDOM} +1

\textbf{CHARISMA} +0

\textbf{Initiative} +2 - \textbf{Defense} 14

\textbf{Hit Points} 120 (16d8 + 48)

\textbf{Movement} 9 m (12 m in tiger form)

\textbf{Saving Throws}: Fortitude +2, Reflex +7, Will +4

\textbf{Skills} Move Silently / Hide +4, Awareness +5

\textbf{Immunity to Damage} from non-magical weapons that are not silver

\textbf{Senses} darkvision 18 m

\textbf{Languages} Common (cannot speak in tiger form)

\textbf{Challenge} 4 (1.1100 PX)

\textit{\textbf{Leap.}} If the werewolf moves at least 5 meters in a straight line towards a creature and hits it with a claw attack during the same turn, the target must make a Fortitude save. DC 14 or falling prone. If the target is prone, the werewolf can make a bite attack against it as a bonus action.

\textit{\textbf{Shapeshifter.}} The werewolf tiger can use its action to transform into a tiger-humanoid hybrid or a tiger, or to return to its true form, which is humanoid. Its stats, aside from Defense, are the same in all forms. Any equipment he is wearing or carrying is not transformed. At death it returns to its true form.

\textit{\textbf{Smell and refined hearing.}} The werewolf has + 1d6 on Wisdom (Awareness) checks based on smell and hearing.

\textbf{Actions}

\textit{\textbf{Multiattack (Humanoid or Hybrid form only).}} In humanoid form, the werewolf makes two scimitar attacks or two longbow attacks. In hybrid form, it can attack as a humanoid or make two claw attacks.

\textit{\textbf{Claw (Tiger or Hybrid Form only).} Melee Weapon Attack}: +9 to hit, 1m range, one target. \textit{Strikes:} 7 (1d8 + 3) slashing damage, 1 bleed damage.

\textit{\textbf{Bite (Tiger or Hybrid Form only).} Melee Weapon Attack}: +9 to hit, 1m range, one target.

\textit{Strikes:} 8 (1d10 + 3) piercing damage. If the target is a humanoid, he must succeed on a DC 13 Fortitude saving throw or be cursed by the werewolf lycanthropy.

\textit{\textbf{Scimitar (Humanoid or Hybrid Form only).} Melee Weapon Attack}: +9 to hit, 1m range, one target. \textit{Strikes:} 6 (1d6 + 3) slashing damage.

\textit{\textbf{Longbow (Humanoid or Hybrid Form only).} Ranged Weapon Attack}: +8 to hit, range 45m, one target.

\textit{Strikes:} 6 (1d8 + 2) piercing damage.

\textbf{Ecology}
Environment: Any Plains or Swamps \\
Organization: Solitary or couple \\
\textbf{Treasure}: NPC gear (Studded Leather Armor, Short Sword, 2 Daggers, other treasure) \\
\textbf{Description} \\
Were tigers in humanoid form have large eyes, elongated noses, prominent cheekbones, and brown or red, or white, black or gray-blue hair. Their movements are careful and graceful, and the viewer might mistake them for an excellent bag cutter, a graceful dancer, or a skilled courtesan.


\medskip\index[Mostruario]{Manticore} \textbf{Manticore}

\textit{Large monstrosity, lawful evil}

\textbf{STRENGTH} +3

\textbf{DEXTERITY} +3

\textbf{CONSTITUTION} +3

\textbf{INTELLIGENCE} -2

\textbf{WISDOM} +1

\textbf{CHARISMA} -1

\textbf{Initiative} +3 - \textbf{Defense} 16

\textbf{Hit Points} 68 (8d10 + 24)

\textbf{Movement} 9 m, flight 15 m

\textbf{Saving Throws}: Fortitude +9, Reflex +7, Will +3

\textbf{Senses} darkvision 18 m

\textbf{Languages} Common

\textbf{Challenge} 3 (700 PX)

\textit{\textbf{Regrowing Tail Thorns.}} The manticore has twenty-four spines in its tail. The thorns used grow back at dawn.

\textbf{Actions}

\textit{\textbf{Multiattack.}} The manticore makes three attacks: one with the bite and two with the claws or three with the spines of the tail.

\textit{\textbf{Claw.} Melee Weapon Attack}: +7 to hit, 1m range, one target.

\textit{Strikes:} 6 (1d6 + 3) slashing damage, 1 bleed damage.

\textit{\textbf{Bite.} Melee Weapon Attack}: +7 to hit, 1m range, one target.

\textit{Strikes:} 7 (1d8 + 3) piercing damage.

\textit{\textbf{Tail Thorns.} Ranged weapon attack}: +7 to hit, range 30m, one target.

\textit{Strikes:} 7 (1d8 + 3) piercing damage.

\textbf{Ecology}
Environment: Hills and Hot Swamps \\
Organization: Solitary, pair or pack (3-6) \\
\textbf{Treasure}: Standard \\

\textbf{Description} \\
Manticores are ferocious predators that patrol large areas in search of fresh meat. A typical manticore is around 3 meters long and weighs around 500 kg. Some have a head similar to that of a human, usually bearded. Males and females are very similar.

Manticores eat any type of meat, even carrion, but they prefer human meat and rarely miss an opportunity to taste this delight. They are crafty and social enough to make deals with evil humanoids to form alliances or to force them to offer tribute, and many powerful creatures commission them to guard or control a place or area. They prefer to make their lairs in high places, such as the tops of the hills and the caves between the cliffs.

Although manticores are similar to magical creations, they have long been counted among the natural species. Curiously, manticores appear strangely fertile and can interbreed with numerous other similarly shaped species, including Lions, Tigers, Lamias, Sphinxes, and Chimeras.

\medskip\index[Mostruario]{Killing Shroud} \textbf{Killing Shroud}

\textit{Large aberration, neutral chaotic}

\textbf{STRENGTH} +3

\textbf{DEXTERITY} +2

\textbf{CONSTITUTION} +1

\textbf{INTELLIGENCE} +1

\textbf{WISDOM} +1

\textbf{CHARISMA} +2

\textbf{Initiative} +2 - \textbf{Defense} 18

\textbf{Hit Points} 78 (12d10 + 12)

\textbf{Movement} 3m, flight 12m

\textbf{Saving Throws}: Fortitude +6, Reflex +5, Will +7

\textbf{Skills} Move Silently / Hide +5

\textbf{Senses} darkvision 18 m

\textbf{Languages} Language of the Depths

\textbf{Challenge} 8 (3,900 PX)

\textit{\textbf{False Appearance.}} While the killer cloak remains motionless without exposing the lower body, it is indistinguishable from a black leather cloak.

\textit{\textbf{Sensitivity to Light}}. While in sunlight, the killer cloak has -1d6 on attack rolls, as well as sight-based Wisdom checks.

\textit{\textbf{Transfer Damage.}} While attached to a creature, the killer cloak takes only half of the damage dealt to it (round down), and the creature that is the victim of the killer cloak suffers the The other half.

\textbf{Actions}

\textit{\textbf{Multiattack.}} The assassin cloak makes two attacks:
one with the bite and one with the tail.

\textit{\textbf{Bite.} Melee Weapon Attack}: +12 to hit, range 1 yards, a creature.

\textit{Strikes:} 10 (2d6 + 3) piercing damage, and if the target is Large or smaller, the assassin's cloak sticks to it. If the assassin cloak has + 1d6 against the target, it sticks to its head and the target is blinded and unable to breathe as long as the assassin cloak sticks to it. While attached to the assassin's cloak, he can only make this attack against the target and has + 1d6 to attack roll. The Killer Cloak can detach by spending 1 meter of movement. A creature, including the target, can take its action to detach the assassin's cloak by succeeding at a DC 16 Strength check.


\textit{\textbf{Tail.} Melee Weapon Attack}: +12 to hit, range 10 feet, a creature.

\textit{Strikes:} 7 (1d8 + 3) slashing damage.

\textit{\textbf{Apparitions (Recharges after 1 hour).}} When not in bright light, the killer cloak creates three illusory duplicates of itself, which move with it and mimic its actions, swapping positions to make it impossible to understand what the real killer mantle is. If the original is in an area of bright light, the duplicates vanish.

Whenever a creature targets the assassin cloak with a noxious attack or spell while duplicates are still present, that creature randomly determines whether it targets the assassin cloak or one of the duplicates. A creature that cannot see or that relies on senses other than sight ignores this magical effect.

A duplicate has Defense and uses the assassin's cloak's saving throws. If an attack hits a duplicate, or if a duplicate fails a saving throw against an effect that deals damage, it wears off.

\textit{\textbf{Moan.}} Any creature within 60 feet of the assassin's cloak that can hear its moan and is not an aberration must succeed on a DC 13 Will save or be frightened to the end of the next round of the assassin cloak. If the creature's saving throw
succeeds, the creature is immune to the murderous cloak's moan for the next 24 hours.

\textbf{Ecology}
Environment: Underground \\
Organization: Solitary, pair, host (3-6) or flock (7-12) \\
\textbf{Treasure}: Standard \\
\textbf{Description} \\
Like hideously evil flying mantas, killer cloaks are mysterious and paranoid creatures. A typical specimen has a wingspan of 2.4 meters and weighs 50 kg.

Their motives are mysterious and confusing, and they are wary of even their own kind. The odd shape allows them to be mistaken for cloaks, tapestries, or other common items, and some stories tell of killer cloaks allying themselves with other creatures, being carried on their backs and helping to protect their allies for inscrutable reasons. Some specimens are priests of ancient gods, in command of cults of murderous cloaks and Skum intent on celebrating horrible rites with sinister purposes.


\medskip\index[Mostruario]{Dark Mantle} \textbf{Dark Mantle}

\textit{Small monstrosity, misaligned}

\textbf{STRENGTH} +3

\textbf{DEXTERITY} +1

\textbf{CONSTITUTION} +1

\textbf{INTELLIGENCE} -4

\textbf{WISDOM} +0

\textbf{CHARISMA} -3

\textbf{Initiative} +1 - \textbf{Defense} 12

\textbf{Hit Points} 22 (5d6 + 5)

\textbf{Movement} 3 m, flight 9 m

\textbf{Saving Throws}: Fortitude +5, Reflex +3, Will +0

\textbf{Skills} Move Silently / Hide +3

\textbf{Senses} blind sight 18 m

\textbf{Languages} -

\textbf{Challenge} 1/2 (100 PX)

\textit{\textbf{Echolocation.}} The dark mantle cannot use blind sight if deaf.

\textit{\textbf{False Appearance.}} While the dark mantle remains motionless, it is indistinguishable from a rock formation such as a stalactite or stalagmite.

\textbf{Actions}

\textit{\textbf{Smash.} Melee Weapon Attack}: +5 to hit, range 1 yards, one creature.

\textit{Strikes:} 6 (1d6 + 3) hit damage and the dark mantle sticks to the creature. If the target is Medium or smaller the dark mantle has + 1d6 to the attack roll, it sticks to the target's head, which is blinded and unable to breathe as long as the dark mantle sticks in this way.

While clinging to the target, the dark mantle cannot attack any other creatures except the target, but it does have + 1d6 on its attack rolls. The Dark Mantle's speed becomes 0 and it cannot benefit from any speed bonuses, moving with the target.

A creature can detach the dark mantle with an action and succeeding a DC 13 Strength check. During its round, the dark mantle can detach itself from the target by itself using 1 meter of movement.

\textit{\textbf{Aura of Darkness (1 / Day).}} A magical darkness with 5 meters of radius extends from the dark mantle, moving with it, and spreading beyond corners. The darkness persists as long as the dark mantle maintains concentration, maximum 10 minutes (as if concentrating on a spell). Darkvision cannot penetrate this darkness, nor can it be illuminated by any natural light. If any part of the darkness overlaps an area of light generated by a level 2 or lower spell, the spell creating the light is dispelled.

\textbf{Ecology}
Environment: Any (underground) \\
Organization: Solitary, pair or brood (3-12) \\
\textbf{Treasure}: None \\
\textbf{Description} \\
the sprawling opening of a dark mantle has a width of just under 1 m; when it is hung from the vault of a cave, masked by stalactite, its length varies between 60 and 90 cm. A typical specimen of dark mantle weighs 20 kg. The creature's head and body are usually the color of basalt or dark granite, but its membranous tentacles can change color to suit their surroundings.

Dark mantle climbers are not particularly skilled climbers, but they are able to hang themselves from a cave vault like bats, hooked by means of the hooks placed at the bottom of their tentacles, so that their dangling body is almost indistinguishable from a stalactite. From this hidden location, the creature waits for its prey to pass underneath it and, at this point, detaches itself by throwing itself towards it, bumping into the target and trying to wrap its membranous tentacles around it. If the dark mantle misses its prey, it climbs up and throws itself at the prey again, until the prey is defeated or the dark mantle is badly injured (in which case it flutters on the ceiling to hide, hoping that its "prey" will leave it. lose). This creature's innate ability to conceal its surroundings by means of magical darkness gives it an added advantage against opponents who need light to see.

Dark mantles prefer to live and hunt in caves and burrows closest to the surface, since these offer more frequent passage of prey for these monsters to hunt. They are not limited to these dark caves, however, and can sometimes be encountered in abandoned fortresses or even in the sewers of crowded cities. Any place where food abounds and has a ceiling to hang from is a possible lair for a dark mantle.

The life cycle of a dark mantle is rapid: young become adults within a few months and most die of old age after a few years. As a result, generations of dark mantles follow one another rapidly and over the years the evolution of these creatures is just as rapid. For this reason, the ecosystem of a cave can have important effects on the appearance, abilities and tactics of a dark mantle. In water caves, dark mantles capable of swimming may develop, while creatures inhabiting places prone to volcanism may develop specific resistance to fire. Other variants of dark mantle could have more resistant skins and instead of falling to crush the prey they could simply throw themselves trying to pierce it similarly to real stalactites. It is rumored that the darkest and deepest caves conceal dark mantles of incredible size, capable of simultaneously suffocating several human-sized targets in their enveloping embrace.


\medskip\index[Mostruario]{Medusa} \textbf{Medusa}

\textit{Medium monstrosity, lawful evil}

\textbf{STRENGTH} +0

\textbf{DEXTERITY} +2

\textbf{CONSTITUTION} +3

\textbf{INTELLIGENCE} +1

\textbf{WISDOM} +1

\textbf{CHARISMA} +2

\textbf{Initiative} +2 - \textbf{Defense} 18

\textbf{Hit Points} 127 (17d8 + 51)

\textbf{Movement} 9 m

\textbf{Saving Throws}: Fortitude +6, Reflex +8, Will +7

\textbf{Skills} Move Silently / Hide +5, Deceive +5, Feel Emotions +4, Awareness +4

\textbf{Senses} darkvision 18 m

\textbf{Languages} Common

\textbf{Challenge} 6 (2,300 PX)

\textit{\textbf{Petrifying Gaze.}} If a creature begins its round within 30 feet of a jellyfish whose eyes it can see, the jellyfish, if it is not incapacitated and can see the creature itself, he can force it to make a DC 14 Fortitude saving throw. If the creature fails its saving throw by 5 or more, it is instantly petrified, otherwise it magically begins to turn to stone and is hampered. The entangled creature must re-roll the saving throw at the end of its next round. If successful, the effect ends. If it fails, the creature is petrified until released by \textit{greater restore} or other spell.

A creature that is not surprised can look away to avoid the saving throw at the start of its round. In that case, he won't be able to see the jellyfish until the start of his next round, when he can look away again. Should he look at the jellyfish in the meantime, he should immediately make the saving throw.

If the jellyfish sees its reflection on a reflective surface within 30 feet of it in an area of bright light, it will suffer the effects of its own gaze due to its own curse.

\textbf{Actions}

\textit{\textbf{Multiattack.}} The jellyfish makes three attacks - one with serpentine hair and two with the short sword - or two ranged attacks with the longbow.



\textit{\textbf{Serpentine Hair.} Melee Weapon Attack}: +9 to hit, 1m range, one target.

\textit{Strikes:} 4 (1d4 + 2) piercing damage plus 14 (4d6) poison damage.

\textit{\textbf{Short Sword.} Melee Weapon Attack}: 9 to hit, 1m range, one target.

\textit{Strikes:} 5 (1d6 + 2) piercing damage.

\textit{\textbf{Longbow.} Ranged weapon attack}: +9 to hit, range 45m, one target.

\textit{Strikes:} 6 (1d8 + 2) piercing damage plus 7 (2d6) poison damage.

\textbf{Ecology} \\
Environment: Temperate and subterranean swamps \\
Organization: Solitary \\
\textbf{Treasure}: Double (Dagger, Perfect Longbow with 20 Arrows, other treasure) \\
\textbf{Description} \\
Jellyfish are human-like creatures with snakes instead of hair. From a distance of 30 feet or more, a jellyfish can easily pass for a beautiful woman if she is wearing something that covers her serpentine hair; when she wears clothing that hides her head and face, she can be mistaken for a human even at close range. Jellyfish use lies and disguises to conceal their face until opponents are close enough to use their petrifying gaze, although they like to play with their prey and can use flaming arrows to trap enemies from a distance. Some enjoy creating intricate decorations with their victims, using petrifaction to give their swampy hiding places a certain touch, but many jellyfish take care to hide the evidence of their previous encounters so that their new enemies don't notice their dangerous one. presence.

Accustomed to hiding, city jellyfish generally are thieves, while those in the wild often end up being forester. The jellyfish of the best known legends, however, are the ones that take spellcasting levels. Charismatic and intelligent, urban jellyfish are often involved in thieves' guilds and other aspects of the criminal underworld. Jellyfish can form alliances with blind creatures or intelligent undead, both of which are immune to their petrifying gaze. Enchanting jellyfish often serve as oracles or prophetesses, generally living in remote areas of legendary power or bad history. These oracle jellyfish take great delight in their role, and if presented with the right gifts and flattery, the secrets they offer can be truly helpful. Of course, the hiding places of these powerful creatures are decorated with the statues of those who have offended them, as a warning to use due caution during encounters.

All jellyfish are female. Rarely, a jellyfish decides to take a humanoid male as a mate, usually with the help of an Elixir of Love or some similar magic, and they always take care not to petrify their captive, unless they are bored of his company. .


\subsection{Mefiti}

\medskip\index[Mostruario]{Ice Mephitus} \textbf{Ice Mephitus}

\textit{Small elemental, neutral evil}

\textbf{STRENGTH} -2

\textbf{DEXTERITY} +1

\textbf{CONSTITUTION} +0

\textbf{INTELLIGENCE} -1

\textbf{WISDOM} +0

\textbf{CHARISMA} +1

\textbf{Initiative} +1 - \textbf{Defense} 12

\textbf{Hit Points} 21 (6d6)

\textbf{Movement} 9 m, flight 9 m

\textbf{Saving Throws}: Fortitude +2, Reflex +5, Will +3

\textbf{Skills} Move Silently / Hide +3, Awareness +2

\textbf{Damage Vulnerability} hit, fire

\textbf{Immunity to Damage} cold, poison

\textbf{Condition Immunity} poisoned

\textbf{Senses} darkvision 18 m

\textbf{Languages} Aquan, Auran

\textbf{Challenge} 1/2 (100 PX)

\textit{\textbf{False Appearance.}} While the mephitus remains immobile, it is indistinguishable from an ordinary piece of ice.

\textit{\textbf{Innate Spells (1 / Day).}} The mephitus can innately cast \textit{cloud of mist}, without the need for material components. His inherent charisma is Charisma.

\textit{\textbf{Elemental Nature.}} A mefito does not need food, drink or sleep.

\textit{\textbf{Deadly Blast.}} When the mephitus dies, it explodes in a blast of ice fragments. Each creature within 1 meter of it must make a DC 10 Reflex saving throw or take 4 (1d8) slashing damage on failure, or half that damage on case.
successfull.

\textbf{Actions}

\textit{\textbf{Claws.} Melee weapon attack}: +3 to hit, range 1 yards, a creature.

\textit{Strikes:} 3 (1d4 + 1) slashing damage plus 2 (1d4) cold damage.

\textit{\textbf{Frost Breath (Recharge 6).}} The mephitus exhales a 5 meter cone of cold air. Each creature in the area must make a DC 10 Reflex saving throw, taking 5 (2d4) cold damage on failure, or half that damage on success.

\textbf{Ecology} \\
Environment: Any (elemental plane of air) \\
Organization: Solitary, pair, group (3-6) or flock (7-12) \\
\textbf{Treasure}: Standard \\
\textbf{Description} \\
Mephits are the servants of powerful elemental creatures. The key locations and locations of the elemental planes are filled with mephits scrambling to perform an important duty or assignment.

Ice mephits are commonly found on the Plane of Air. These mephits are distant and cruel.


\medskip\index[Mostruario]{Mephitus of Magma} \textbf{Mephitus of Magma}

\textit{Small elemental, neutral evil}

\textbf{STRENGTH} -1

\textbf{DEXTERITY} +1

\textbf{CONSTITUTION} +1

\textbf{INTELLIGENCE} -2

\textbf{WISDOM} +0

\textbf{CHARISMA} +0

\textbf{Initiative} +1 - \textbf{Defense} 12

\textbf{Hit Points} 22 (5d6 + 5)

\textbf{Movement} 9 m, flight 9 m

\textbf{Saving Throws}: Fortitude +2, Reflex +5, Will +3

\textbf{Skills} Move Silently / Hide +3

\textbf{Damage Vulnerability} cold

\textbf{Immunity to Damage} fire, poison

\textbf{Condition Immunity} poisoned

\textbf{Senses} darkvision 18 m

\textbf{Languages} Ignan, Terran

\textbf{Challenge} 1/2 (100 PX)

\textit{\textbf{False Appearance.}} While the mephitus remains motionless, it is indistinguishable from an ordinary pool of magma.

\textit{\textbf{Innate Spells (1 / day).}} The mephitus can cast innately \textit{heat metal} (save DC of spell 10), without the need for material components. His inherent charisma is Charisma.

\textit{\textbf{Elemental Nature.}} A mefito does not need food, drink or sleep.

\textit{\textbf{Deadly Burst.}} When the mephitus dies, it explodes in a burst of lava. Each creature within 1 meter of it must make a DC 11 Reflex saving throw or take 7 (2d6) fire damage on failure, or half that damage on success.

\textbf{Actions}

\textit{\textbf{Claws.} Melee weapon attack}: +3 to hit, range 1 yards, a creature.

\textit{Strikes:} 3 (1d4 + 1) slashing damage plus 2 (1d4) fire damage.

\textit{\textbf{Fiery Breath (Recharge 6).}} The mephitus exhales a 5-meter cone of fire. Each creature in the area must make a DC 11 Reflex saving throw, taking 7 (2d6) fire damage on failure, or half that damage on success.

\textbf{Ecology} \\
Environment: Any (elemental plane of fire) \\
Organization: Solitary, pair, group (3-6) or flock (7-12) \\
\textbf{Treasure}: Standard \\
\textbf{Description} \\
Mephits are the servants of powerful elemental creatures. The key locations and locations of the elemental planes are filled with mephits scrambling to perform an important duty or assignment.

Magma mephits are commonly found on the Plane of Fire. These mephits are stupid brutes.


\medskip\index[Mostruario]{Mephito of Dust} \textbf{Mephito of Dust}

\textit{Small elemental, neutral evil}

\textbf{STRENGTH} -3

\textbf{DEXTERITY} +2

\textbf{CONSTITUTION} +0

\textbf{INTELLIGENCE} -1

\textbf{WISDOM} +0

\textbf{CHARISMA} +0

\textbf{Initiative} +2 - \textbf{Defense} 13

\textbf{Hit Points} 17 (5d6)

\textbf{Movement} 9 m, flight 9 m

\textbf{Saving Throws}: Fortitude +2, Reflex +5, Will +3

\textbf{Skills} Move Silently / Hide +4, Awareness +2

\textbf{Damage Vulnerability} fire

\textbf{Immunity to Damage} poison

\textbf{Condition Immunity} poisoned

\textbf{Senses} darkvision 18 m

\textbf{Languages} Auran, Terran

\textbf{Challenge} 1/2 (100 PX)

\textit{\textbf{Innate spells (1 / day).}} The mephitus can innately perform \textit{sleep} (save DC of spell 10), without the need for material components. His innate spellcasting ability is Charisma.

\textit{\textbf{Elemental Nature.}} A mefito does not need food, drink or sleep.

\textit{\textbf{Deadly Burst.}} When the mephitus dies, it explodes in a blast of dust. Any creature within 1 meter of it must succeed on a DC 10 Fortitude saving throw or be blinded for 1 minute. A blinded creature can re-roll its saving throw during each of its rounds, ending the effect on itself if it succeeds.

\textbf{Actions}

\textit{\textbf{Claws.} Melee Weapon Attack}: +4 to hit, range 1 yd, a creature.

\textit{Strikes:} 4 (1d4 + 2) slashing damage.

\textit{\textbf{Blinding Breath (Recharge 6).}} The mephitus exhales a 5-meter cone of blinding dust. Each creature in the area must succeed on a DC 10 Reflex saving throw or be blinded for 1 minute. A blinded creature can re-roll its saving throw during each of its rounds, ending the effect on itself if it succeeds.

\textbf{Ecology} \\
Environment: Any (elemental plane of air) \\
Organization: Solitary, pair, group (3-6) or flock (7-12) \\
\textbf{Treasure}: Standard \\
\textbf{Description} \\
Mephits are the servants of powerful elemental creatures. The key locations and locations of the elemental planes are filled with mephits scrambling to perform an important duty or assignment.

Dust mephits are commonly found on the Plane of Air. These mephits are irritating and persistent.

\medskip\index[Mostruario]{Mephito di Vapore} \textbf{Mephito di Vapore}

\textit{Small elemental, neutral evil}

\textbf{STRENGTH} -3

\textbf{DEXTERITY} +0

\textbf{CONSTITUTION} +0

\textbf{INTELLIGENCE} +0

\textbf{WISDOM} +0

\textbf{CHARISMA} +1

\textbf{Initiative} +0 - \textbf{Defense} 11

\textbf{Hit Points} 21 (6d6)

\textbf{Movement} 9 m, flight 9 m

\textbf{Saving Throws}: Fortitude +2, Reflex +5, Will +3

\textbf{Immunity to Damage} fire, poison

\textbf{Condition Immunity} poisoned

\textbf{Senses} darkvision 18 m

\textbf{Languages} Aquan, Ignan

\textbf{Challenge} 1/4 (50 PX)

\textit{\textbf{Innate Spells (1 / Day).}} The mephitus can innately perform \textit{blur}, without the need for material components. His innate spellcasting ability is Charisma.

\textit{\textbf{Elemental Nature.}} A mefito does not need food, drink or sleep.

\textit{\textbf{Deadly Burst.}} When the mephitus dies, it explodes in a cloud of vapor. Any creature within 1 meter of it must succeed on a DC 10 Reflex saving throw or take 4 (1d8) fire damage.

\textbf{Actions}

\textit{\textbf{Claws.} Melee Weapon Attack}: +2 to hit, range 1 yards, a creature.

\textit{Strikes:} 2 (1d4) slashing damage plus 2 (1d4) fire damage.

\textit{\textbf{Vapor Breath (Refill 6).}} The mephitus exhales a 5 meter cone of hot steam. Each creature in the area must make a DC 10 Reflex saving throw, taking 4 (1d8) points of fire damage on failure, or half that damage on success.

\textbf{Ecology} \\
Environment: Any (elemental plane of fire) \\
Organization: Solitary, pair, group (3-6) or flock (7-12) \\
\textbf{Treasure}: Standard \\
\textbf{Description} \\
Mephits are the servants of powerful elemental creatures. The key locations and locations of the elemental planes are filled with mephits scrambling to perform an important duty or assignment.

Vapor mephits are commonly found on the Plane of Fire. These mephits are insolent and contemptuous.



\subsection{Crone}

\medskip\index[Mostruario]{Sea Shrew} \textbf{Sea Shrew}

\textit{Fairy medium, chaotic evil}

\textbf{STRENGTH} +3

\textbf{DEXTERITY} +1

\textbf{CONSTITUTION} +3

\textbf{INTELLIGENCE} +1

\textbf{WISDOM} +1

\textbf{CHARISMA} +1

\textbf{Initiative} +1 - \textbf{Defense} 15

\textbf{Hit Points} 52 (7d8 + 21)

\textbf{Damage Vulnerability} cold iron

\textbf{Movement} 9m, swim 12m

\textbf{Saving Throws}: Fortitude +5, Reflex +7, Will +5

\textbf{Senses} darkvision 18 m

\textbf{Languages} Aquan, Common, Giant

\textbf{Challenge} 2 (450 PX)

\textit{\textbf{Amphibian.}} The hag can breathe air and water.

\textit{\textbf{Horrifying aspect.}} Any humanoid that starts its round within 30 feet of the hag and can see its true form must make a DC 11 Will save. If it fails the saving throw, the creature is scared for 1 minute. A creature can re-roll the saving throw at the end of each of its rounds, with -1d6 if the hag is in line of sight, and ending the effect if it succeeds at the saving throw. If the creature's saving throw succeeds or the effect ends on it, the creature is immune to the horrifying aspect for the next 24 hours.

Unless the target is surprised or the revelation of the hag's true form is sudden, the target can look away and avoid making the initial saving throw. Until the start of its next round, a creature that looks away
has -1d6 to attack rolls against the hag.

\textbf{Actions}

\textit{\textbf{Claws.} Melee Weapon Attack}: +5 to hit, 1m range, one target.

\textit{Strikes:} 10 (2d6 + 3) slashing damage, 1 bleed damage.

\textit{\textbf{Illusory Appearance.}} The hag covers herself and everything she is wearing or carrying in a magical illusion that gives her the appearance of a hideous creature about the same size and humanoid shape . The illusion ends if the hag takes a bonus action to end it or if she dies.

Changes made by this effect are unable to pass physical inspections. For example, the hag might appear as a creature without claws, but a person in contact with her hands would warn them. Otherwise, a creature must take an action to visually inspect the illusion and succeed on a DC 16 Intelligence check to understand that the hag has disguised itself.

\textit{\textbf{Death Look.}} The hag targets a scared creature visible within 30 feet of her. If the target can see the hag, they must succeed on a DC 11 Will saving throw against this spell or drop to 0 hit points.

\textbf{Ecology} \\
Environment: any aquatic \\
Organization: solitary or coven (3 hags of any species) \\
\textbf{Treasure}: standard \\
\textbf{Description} \\
These wicked and monstrous hags possess terrifying traits that few dare to gaze upon, take pleasure in the strife and death of sailors, and torment the seafarers with inescapable calamities. Sea hags always look terrifying, and despite their ravenous nature, they are generally emaciated creatures that seem on the verge of starvation. They are 1.8 meters tall and weigh 75 kg.

Sea hags prefer to live close to the shore where fishing boats and freighters are most common, and yet away from urban areas so that their actions do not attract too much attention from possible enemies, although it is not unusual for a brave sea hag or greedy settle in a port city or at the mouth of a deep river.

Sea hags form coves similar to those of other hags, but their aquatic nature generally prompts them to refrain from forming mixed coves. In the event that a Green Crone lives along the coast (often in a salt marsh or coastal swamp), a coven is made up of two sea hags who respect the Green Crone as their mother and leader. Most commonly, a coven of sea hags consists of a group of particularly friendly and close sea hags.


\medskip\index[Mostruario]{Night Hag} \textbf{Night Hag}

\textit{Medium fiend, neutral evil}

\textbf{STRENGTH} +4

\textbf{DEXTERITY} +2

\textbf{CONSTITUTION} +3

\textbf{INTELLIGENCE} +3

\textbf{WISDOM} +2

\textbf{CHARISMA} +3

\textbf{Initiative} +3 - \textbf{Defense} 20

\textbf{Hit Points} 112 (15d8 + 45)

\textbf{Movement} 9 m

\textbf{Saving Throws}: Fortitude +14, Reflex +8, Will +11

\textbf{Skills} Move Silently / Hide +6, Deceive +7, Feel Emotions +6, Awareness +6,

\textbf{Resistance to Damage} cold, fire; non-magical weapon or non-silver

\textbf{Senses} darkvision 36 m

\textbf{Languages} Abyssal, Common, Infernal, Druidic

\textbf{Challenge} 5 (1.800 PX)

\textit{\textbf{Innate spells.}} The hag's innate spellcasting characteristic is Charisma (DC 14 on saving throws for spells, +6 to hit with spell attacks). The hag can innately cast the following spells, without needing
material components.

At will: \textit{magic missile, detect magic} 2 / day each: \textit{ray of weakening, sleep, displacement} \textit{planar} (personal)

\textit{\textbf{Magic Resistance.}} The hag has + 1d6 on saving throws against spells and other magical effects.

\textbf{Actions}

\textit{\textbf{Claws (In Hag Form Only).} Melee Weapon Attack}: +10 to hit, range 1m, one target.

\textit{Strikes:} 13 (2d8 + 4) slashing damage, 1 bleed damage.

\textit{\textbf{Ethereal Form.}} The hag magically enters the Ethereal Plane from the Material Plane, and vice versa. To do this he must have a \textit{heart of stone}.

\textit{\textbf{Haunt Nightmares (1 / Day).}} While on the Ethereal Plane, the hag magically makes contact with a sleeping humanoid on the Material Plane. The spell \textit{protection from good and evil} cast on the target prevents this contact, as does \textit{magic circle}. As long as contact persists, the target suffers from horrible visions. If these visions last for at least 1 hour, the target gains no benefit from its rest, and its maximum hit points are reduced by 5 (1d10). If this effect reduces the target's maximum hit points to 0, the target dies, and if the target was evil, their soul becomes trapped in the hag's \textit{bag} \textit{souls}. The target's maximum hit point reduction remains until removed by the spell \textit{restore} \textit{superior} or similar spell.

\textit{\textbf{Shape Shift.}} The hag can magically transform into a Small or Medium-sized humanoid female, or revert to her true form. Its stats are the same in any form. Any equipment he was carrying or wearing is not transformed. Upon death, it returns to its true form.



\medskip\index[Mostruario]{Green Crone} \textbf{Green Crone}

\textit{Fairy mean, neutral evil}

\textbf{STRENGTH} +4

\textbf{DEXTERITY} +1

\textbf{CONSTITUTION} +3

\textbf{INTELLIGENCE} +1

\textbf{WISDOM} +2

\textbf{CHARISMA} +2

\textbf{Initiative} +1 - \textbf{Defense} 19

\textbf{Hit Points} 82 (11d8 + 33)

\textbf{Damage Vulnerability} cold iron

\textbf{Movement} 9 m

\textbf{Saving Throws}: Fortitude +6, Reflex +7, Will +7

\textbf{Skills} Arcane +3, Move silently / Hide +3, Deceive +4, Awareness +4

\textbf{Senses} darkvision 18 m

\textbf{Languages} Common, Draconic, Sylvan

\textbf{Challenge} 3 (700 PX)

\textit{\textbf{Amphibian.}} The hag can breathe air and water.

\textit{\textbf{Imitation.}} The hag can imitate animal sounds and humanoid voices. A creature that hears these noises can determine it is an imitation by succeeding a DC 14 Wisdom check.

\textit{\textbf{Innate Spells.}} The hag's innate spellcasting characteristic is Charisma (DC 12 on saving throws for spells). The hag can innately cast the following spells, without needing any material components.

At will: \textit{minor illusion, dancing lights, malignant mockery}

\textbf{Actions}

\textit{\textbf{Claws.} Melee Weapon Attack}: +6 to hit, 1m range, one target.

\textit{Strikes:} 13 (2d8 + 4) slashing damage, 1 bleed damage.

\textit{\textbf{Illusory Appearance.}} The hag covers herself and everything she is wearing or carrying in a magical illusion that gives her the appearance of another creature roughly the same size and shape humanoid. The illusion ends if the hag takes a bonus action to end it or if she dies.

Changes made by this effect are unable to pass physical inspections. For example, the hag might appear as a smooth-skinned creature, but contact would reveal its rough skin. Otherwise, a creature must take an action to visually inspect the illusion and succeed on a DC 20 Intelligence check to understand that it is a disguised hag.

\textit{\textbf{Invisible Passage.}} The hag can make herself invisible until she attacks or casts a spell, or until she finishes her concentration (as if she were concentrating on a spell). While invisible, it leaves no physical trace of its passage, so its traces can only be followed by magic. All equipment she is carrying or wearing becomes invisible with her.

\textbf{Ecology}
Environment: Temperate swamps \\
Organization: Solitary or coven (3 hags of any kind) \\
\textbf{Treasure}: Standard \\
\textbf{Description} \\
Terrifying wrinkled old women who haunt loathsome swamps and tangled forests, green hags harbor an intense hatred of all that is beautiful and pure. Making use of their various illusory abilities, these old women delight in killing the innocent, upsetting noble souls, and demeaning pure hearts. They love to use Disguise Himself to assume the forms of young and attractive girls so as to seduce and snatch young men from their loved ones and relatives, and to corrupt noble and honest citizens with all sorts of depravity and scandal. Some green hags prefer to reveal their true nature to their loved ones in a carefully crafted moment to drive man mad with horror and shame. Others prolong their flirtation and go to great lengths to completely ruin the lives of the men they seduce before showing them the truth. Finally, the luckiest of these unfortunates end up being devoured by their lover's green hag: for the unlucky, the final fate can be much worse, as the cruel fantasy of the green hag is immense. A typical green shrew is between 1.5 and 1.8 meters tall and weighs just under 80 kg.


\subsection{Slime}

\medskip\index[Mostruario]{Ameba Paglierina} \textbf{Ameba Paglierina}

\textit{Big slime, misaligned}

\textbf{STRENGTH} +2

\textbf{DEXTERITY} -2

\textbf{CONSTITUTION} +2

\textbf{INTELLIGENCE} -4

\textbf{WISDOM} -2

\textbf{CHARISMA} -5

\textbf{Initiative} +2 - \textbf{Defense} 9

\textbf{Hit Points} 45 (6d10 + 12)

\textbf{Movement} 3m, climb 3m

\textbf{Saving Throws}: Fortitude +8, Reflex -3, Will -3

\textbf{Resistance to Damage} acid

\textbf{Immunity to Damage} lightning bolt, cutting edge

\textbf{Condition Immunity} blinded, fascinated, deafened, prone, fatigue, frightened

\textbf{Senses} blind sight 18 m (blind beyond this radius)

\textbf{Languages} -

\textbf{Challenge} 2 (450 PX)

\textit{\textbf{Amorphous.}} The amoeba can move through a space up to 3 centimeters wide without having to squeeze.

\textit{\textbf{Nature of Slime.}} The amoeba does not need to sleep.

\textit{\textbf{Climbing like Spider.}} The amoeba can climb difficult surfaces, including standing upside down on the ceiling, without the need for a skill check.

\textbf{Actions}

\textit{\textbf{Pseudopod.} Melee Weapon Attack}: +4 to hit, 1m range, one target.

\textit{Strikes:} 9 (2d6 + 2) slam damage plus 3 (1d6) acid damage.

\textbf{Reactions}

\textit{\textbf{Division.}} When a Medium or larger amoeba takes damage from lightning or slashing, it splits into two new amoebas that have at least 10 hit points. Each new amoeba has a number of hit points equal to half the original amoeba, rounded down. The new amoebas are one size smaller than the original one.

\textbf{Ecology}
Environment dungeons or temperate swamps \\
Organization: Solitary \\
\textbf{Treasure}: None \\
\textbf{Description} \\
Ameba Paglierina are animated masses of protoplasm similar in color to a repellent amalgam of yellow, orange and brown. When at rest, their flat, pulsating body is about six inches tall and extends all the way around; in motion, they gather in a vaguely spherical shape and almost seem to move by rolling. Their malleable bodies allow them to pass through crevices and holes much smaller than the space they occupy. Creatures that live underground often seal all openings to defend against the Straw Ameba.

The highly specialized acid of the Amoeba Paglierina dissolves only the meat. This discovery has led many poisoners and alchemists to look for specimens to study them. From these experiments, several specific weapons designed to destroy bodies were born. It tells of the existence of a slow-acting poison that destroys the cells of living creatures one by one, whose secret is well kept by its creator.

Some notes in a forgotten tome speak of a funeral ritual used in distant places. Instead of burning the body, it was sealed in a stone sarcophagus with a Straw Ameba, which dissolved the body. Later, the gravediggers inserted the gelatin into an urn complete with a bronze plaque bearing the name of the deceased. This practice protects the buried objects with the dead (which is quickly reduced to a shiny skeleton) and the essence of the creature, which was believed to still live inside the jelly.

Amoeba Paglierina are about 15 centimeters high, have a diameter that can reach 3 meters and weigh about 1,300 kilos. In combat, they gather on themselves and produce long, moist pseudopods to strike and grab anything that moves.

Although the typical Straw Ameba has the statistics presented here, deep within the earth these predators can reach monstrous sizes. There is also talk of Ameba Paglierina who have developed other ways of capturing prey. For example, jellies that poison with touch and that expel clouds of toxic gas that burn the eyes and mouth, leaving you helpless but conscious as this protoplasmic beast slides over the bodies and feeds on them.


\medskip\index[Mostruario]{Gelatinous Cube} \textbf{Gelatinous Cube}

\textit{Big slime, misaligned}

\textbf{STRENGTH} +2

\textbf{DEXTERITY} -4

\textbf{CONSTITUTION} +5

\textbf{INTELLIGENCE} -5

\textbf{WISDOM} -2

\textbf{CHARISMA} -5

\textbf{Initiative} -4 - \textbf{Defense} 7

\textbf{Hit Points} 84 (8d10 + 40)

\textbf{Movement} 5 meters

\textbf{Saving Throws}: Fortitude +9, Reflex -4, Will -4

\textbf{Immunity to Damage} non-magical edged weapons

\textbf{Condition Immunity} blinded, fascinated, deafened, prone, fatigue, frightened

\textbf{Senses} blind sight 18 m (blind beyond this radius)

\textbf{Languages} -

\textbf{Challenge} 2 (450 PX)

\textit{\textbf{Slime Cube.}} The cube occupies its entire space. Other creatures can enter space, but they fall victim to the drowning cube and have -1d6 on their saving throw.

The creatures inside the cube are visible but enjoy full coverage.

A creature within 1 meter of the cube can take an action to pull a creature or object out of the cube. Doing so requires a successful DC 12 Strength check, and the creature making the attempt takes 10 (3d6) points of acid damage.

The cube can only contain one Large creature or a maximum of four Medium or smaller creatures at a time.

\textit{\textbf{Nature of Slime.}} The cube does not need to sleep.

\textit{\textbf{Transparent.}} Even when in plain sight, a DC 15 Wisdom (Awareness) check must be made to notice a cube that has not moved or attacked. A creature who tries to enter the cube space while unaware of its presence is surprised by the cube.

\textbf{Actions}

\textit{\textbf{Pseudopod.} Melee Weapon Attack}: +4 to hit, 1m range, one target.

\textit{Hits:} 10 (3d6) acid damage.

\textit{\textbf{Submerge.}} The cube moves up to its maximum movement. In doing so, it can enter the space of a Large or smaller creature. Whenever the cube enters a creature's space, the creature must make a DC 12 Reflex saving throw.

If the saving throw is successful, the creature can choose to be pushed back or to the side by 1 meter. A creature that chooses not to be pushed suffers the consequences of a failed saving throw.

If the saving throw fails, the cube enters the creature's space, which takes 10 (3d6) acid damage and is submerged. The submerged creature cannot breathe, is hampered, and takes 21 (6d6) acid damage at the start of the cube's turn. As the cube moves, the submerged creature moves with it.

A submerged creature can attempt to flee by taking an action to make a DC 12 Strength check. If successful, the creature escapes and enters the space of its choice within 1 meter of the cube.

\textbf{Ecology}
Environment: Any dungeon \\
Organization: Solitary \\
\textbf{Treasure}: Accidental \\
\textbf{Description} \\
Among the most unusual and peculiar predators of dungeons, the gelatinous cubes spend their existence wandering aimlessly through underground tunnels and dark caves, engulfing organic materials such as plants, waste, carrion and even living creatures. Matter that the cube cannot digest, such as metals and stone, fills the creature's volume with debris, and sometimes it can expel some of it from its body. Often the treasure and belongings of past victims remain inside the gelatinous cube: a ghostly image of their material remains.

Some sages believe these creatures evolved from Gray Oozes. Some beings use the gelatinous cubes as guardians of dungeons and underground fortifications, trapping these immense creatures in chests of solid metal and carrying them with powers or magic to their final guard post. They are particularly effective waste disposal mechanisms; a tribe can trap a gelatinous cube in a pit or other area it cannot climb using it as a dunghill or even a death trap, depending on the ingenuity of the creatures that captured it.

Gelatinous cubes typically have an edge of 3 meters and weigh more than 7,500 kg, although some underground explorers claim that larger specimens exist underground. In areas where food abounds, gelatinous cubes can live for hundreds, if not thousands, of years. However, if organic matter is lacking for more than 6 months, a gelatinous cube begins to wither, and its walls begin to leak, rapidly breaking down into liquid mucus until the entire body collapses and disappears completely.


\medskip\index[Mostruario]{Gray Slime} \textbf{Gray Slime}

\textit{Media slime, misaligned}

\textbf{STRENGTH} +1

\textbf{DEXTERITY} -2

\textbf{CONSTITUTION} +3

\textbf{INTELLIGENCE} -5

\textbf{WISDOM} -2

\textbf{CHARISMA} -4

\textbf{Initiative} -2 - \textbf{Defense} 9

\textbf{Hit Points} 22 (3d8 + 9)

\textbf{Movement} 3m, climb 3m

\textbf{Saving Throws}: Fortitude +9, Reflex -4, Will -4

\textbf{Resistance to Damage} acid, cold, fire

\textbf{Condition Immunity} blinded, fascinated, deafened, prone, fatigue, frightened

\textbf{Senses} blind sight 18 m (blind beyond this radius)

\textbf{Languages} -

\textbf{Challenge} 1/2 (100 PX)

\textit{\textbf{Amorphous.}} The slime can move through a space up to centimeters wide without having to squeeze.

\textit{\textbf{Corrode Metal.}} Any nonmagical weapon made of metal that hits the slime will corrode. After dealing damage, the weapon takes a permanent and cumulative -1 penalty on damage rolls. If the penalty goes to -5, the weapon is destroyed. Non-magical ammunition made of metal that hits slime is destroyed after dealing damage.

Slime can devour 5cm thick nonmagical metal in 1 round.

\textit{\textbf{False Appearance.}} When the slime remains motionless, it is indistinguishable from a puddle of oil or a wet stone.

\textit{\textbf{Nature of Slime.}} The slime does not need to sleep.

\textbf{Actions}

\textit{\textbf{Pseudopod.} Melee weapon attack}: +3 to hit, 1 m range, one target.

\textit{Hits:} 4 (1d6 + 1) crush damage plus 7 (2d6) acid damage, and if the target is wearing metal armor, it is partially dispelled and takes a permanent and cumulative penalty of - 1 to the Defense it offers. The armor is destroyed if the penalty reduces its Defense to 10.

\textbf{Ecology} \\
Environment: Cold and subterranean swamps \\
Organization: Solitary \\
\textbf{Treasure}: None \\
\textbf{Description} \\
Crawling through cold swamps and misty swamps or, sometimes in dungeons and caverns, gray oozes consume any organic matter they encounter. Although lacking in intelligence, gray slime is one of the creatures that gives a lot of problems for its transparency. While it cannot easily climb walls or swim, its habit of hiding in thick mud along swampy banks or standing still in harmless-looking pools on the gray floor of a dungeon make it very difficult to notice and avoid.

Some sages believe that gray oozes are the result of a failed alchemical experiment, while others theorize that the first gray oozes arose spontaneously from a pit of magical debris. Of course, these theories that do not regard them as living organisms, but the result of an unfortunate mixture of caustic fluids and magical residues, are derided by those who live in areas infested with these creatures, which have no history of magical pollution.


\medskip\index[Mostruario]{Black Protoplasm} \textbf{Black Protoplasm}

\textit{Big slime, misaligned}

\textbf{STRENGTH} +3

\textbf{DEXTERITY} -3

\textbf{CONSTITUTION} +3

\textbf{INTELLIGENCE} -5

\textbf{WISDOM} -2

\textbf{CHARISMA} -5

\textbf{Initiative} -3 - \textbf{Defense} 9

\textbf{Hit Points} 85 (10d10 + 30)

\textbf{Movement} 6 m, climb 6 m

\textbf{Saving Throws}: Fortitude +9, Reflex -2, Will -2

\textbf{Immunity to Damage} acid, cold, lightning, cutting

\textbf{Condition Immunity} blinded, fascinated, deafened, prone, fatigue, frightened

\textbf{Senses} blind sight 18 m (blind beyond this radius)

\textbf{Languages} -

\textbf{Challenge} 4 (1.100 PX)

\textit{\textbf{Amorphous.}} The black protoplasm can move through a space up to 3 centimeters wide without having to squeeze.

\textit{\textbf{Corrosive Form.}} A creature that contacts the black protoplasm or hits it with a melee attack while within 1 meter of it takes 4 (1d8) points of acid damage. Any non-magical weapon made of metal or wood that hits the black protoplasm will corrode. After dealing damage, the weapon takes a permanent and cumulative -1 penalty on damage rolls. If the penalty goes to -5, the weapon is destroyed. Non-magical ammunition made of metal or wood that hits the black protoplasm is destroyed after dealing the damage.

The black protoplasm can devour 5 centimeter thick wood or nonmagical metal in 1 round.

\textit{\textbf{Nature of Slime.}} The black protoplasm does not need to sleep.

\textit{\textbf{Climbing like Spider.}} Black protoplasm can scale difficult surfaces, including standing upside down on the ceiling, without the need to make a skill check.

\textbf{Actions}

\textit{\textbf{Pseudopod.} Melee Weapon Attack}: +7 to hit, 1m range, one target.

\textit{Strikes:} 6 (1d6 + 3) slash damage plus 18 (4d8) acid damage. Additionally, non-magical armor worn by the target is partially dispelled and takes a permanent and cumulative -1 penalty to the Defense it offers. The armor is destroyed if the penalty reduces its Defense to 10.

\textbf{Reactions}

\textit{\textbf{Division.}} When a medium-sized or larger black protoplasm takes lightning or slashing damage, it splits into two new black protoplasm of at least 10 hit points each. Each new black protoplasm has a number of hit points equal to half the original black protoplasm, rounded down. The new black protoplasm is one size smaller than the original one.

\textbf{Ecology} \\
Environment: Any underground \\
Organization: Solitary \\
\textbf{Treasure}: None \\
\textbf{Description} \\
Black protoplasms are the scavengers of the underworld, constantly looking for food. They can sense organic or metallic bodies within 60 feet and instinctively attack such objects or beings until they dissolve, or until the slime is killed. A black protoplasm reproduces by breaking off a piece of its own body and forming a new, smaller protoplasm that reaches adulthood within a month. Some of the most intelligent creatures in the underworld use black protoplasm to naturally dispose of garbage, creating stone quarries to house the protoplasm, then throwing organic waste or enemies into them.
The largest specimens of black protoplasm have been sighted in the deepest regions of the world: mammoth individuals possessing up to 30 HD. Colored protoplasms are also said to exist: some white living in the arctic areas, brown in the swamps and reddish in the desert.


\medskip\index[Mostruario]{Mimic} \textbf{Mimic}

\textit{Medium monstrosity (shapeshifter), neutral}

\textbf{STRENGTH} +3

\textbf{DEXTERITY} +1

\textbf{CONSTITUTION} +2

\textbf{INTELLIGENCE} -3

\textbf{WISDOM} +1

\textbf{CHARISMA} -1

\textbf{Initiative} +1 - \textbf{Defense} 13

\textbf{Hit Points} 58 (9d8 + 18)

\textbf{Movement} 5 meters

\textbf{Saving Throws}: Fortitude +5, Reflex +5, Will +6

\textbf{Skills} Move Silently / Hide +5

\textbf{Immunity to Damage} acid

\textbf{Condition Immunity} prone

\textbf{Senses} darkvision 18 m

\textbf{Languages} -

\textbf{Challenge} 2 (450 PX)

\textit{\textbf{Adherent (Object Shape Only).}} The mimic adheres to anything it comes into contact with. A Huge or smaller creature that the mimic adheres to is considered to be grabbed by it (DC 13 to escape). The trait checks carry out to escape from
this grab have -1d6.

\textit{\textbf{grabber.}} The mimic has + 1d6 on attack rolls against a creature it grabs.

\textit{\textbf{False Appearance (Object Shape Only).}} While the mimic remains motionless, it is indistinguishable from a common object.

\textit{\textbf{Shapeshifter.}} The mimic can use its action to transform into an object, or to return to its true amorphous form. Its stats are the same in any form. Any equipment he is wearing or carrying does not transform. At death it returns to its true aspect.

\textbf{Actions}

\textit{\textbf{Bite.} Melee Weapon Attack}: +6 to hit, 1m range, one target.

\textit{Strikes:} 7 (1d8 + 3) piercing damage plus 4 (1d8) acid damage.

\textit{\textbf{Pseudopod.} Melee Weapon Attack}: +6 to hit, 1m range, one target.

\textit{Strikes:} 7 (1d8 + 3) hit damage. If the mimic is in object form, the target is the victim of the Snug trait.

\textbf{Ecology}
Environment: Any \\
Organization: Solitary \\
\textbf{Treasure}: Accidental \\
\textbf{Description} \\
Mimics are believed to be the result of an alchemist's attempt to give life to an inanimate object through the application of a mystical reagent, the formula of which has been lost. Over the years, these strange but intelligent creatures have learned the ability to transform into simulacra of manufactured objects, particularly in places that are rarely visited by a small number of creatures, where they increase their chances of success with an attack on their victims.

While mimics are not inherently evil, some sages suggest that they attack humans and other intelligent creatures more for entertainment than for food. The desire to deceive others is part of their being, and their surprise attacks represent the culmination of this desire.

A typical mimic has a volume of 2 cubic meters (1m by 1m by 2m) and weighs around 450kg. Legends and stories speak of larger-sized mimics, with the ability to take the form of houses, ships or entire underground complexes that they garnish with treasures (both real and fake) to lure their unsuspecting food into them.


\medskip\index[Mostruario]{Minotaur} \textbf{Minotaur}

\textit{Large monstrosity, chaotic evil}

\textbf{STRENGTH} +4

\textbf{DEXTERITY} +0

\textbf{CONSTITUTION} +3

\textbf{INTELLIGENCE} -2

\textbf{WISDOM} +3

\textbf{CHARISMA} -1

\textbf{Initiative} +0 - \textbf{Defense} 16

\textbf{Hit Points} 76 (9d10 + 27)

\textbf{Movement} 12 m

\textbf{Saving Throws}: Fortitude +6, Reflex +5, Will +5

\textbf{Skills} Awareness +7

\textbf{Senses} darkvision 18 m

\textbf{Languages} Abyssal

\textbf{Challenge} 3 (700 PX)

\textit{\textbf{Charge.}} If the minotaur moves at least 10 feet toward a target and hits it with a gore attack during the same turn, the target takes an additional 9 (2d8) piercing damage. If the target is a creature, it must succeed on a DC 14 Fortitude save or be pushed away up to 10 feet away and fall prone.

\textit{\textbf{Careless.}} At the start of his round, the minotaur can get + 1d6 on all attack rolls for melee weapons made that turn, but attack rolls against it have + 1d6 until the start of his next round.

\textit{\textbf{Remembering Labyrinth.}} The minotaur can perfectly remember any route it has traveled.

\textbf{Actions}

\textit{\textbf{Double-headed Ax.} Melee Weapon Attack}: +8 to hit, 1m range, one target.

\textit{Strikes:} 17 (2d12 + 4) slashing damage.

\textit{\textbf{Gored.} Melee Weapon Attack}: +8 to hit, 1m range, one target.

\textit{Strikes:} 13 (2d8 + 4) piercing damage.

\textbf{Ecology} \\
Environment: Temperate Ruins and Underground \\
Organization: Solitary, couple or group (3-4) \\
\textbf{Treasure}: Standard (Double-sided Ax, other treasure) \\
\textbf{Description} \\
Nobody holds a grudge like a minotaur. Scorned by civilized races and born centuries ago by a divine curse, minotaurs have hunted, killed and devoured lower humanoids to punish true or alleged offenses for longer than they can remember. Most cultures have legends about how minotaurs were created by vengeful or offended deities who punished humans by deforming their likeness, stealing their beauty and intelligence, and equipping them with bull heads. Yet the majority of modern minotaurs despise these legends and do not believe they are the joke of some deity, but models of divine perfection created by the cruel and powerful demon lord Baphomet.

The traditional hiding places of minotaurs are labyrinths, both mazes built to confuse and confuse, and natural ones created by a tangle of caves or other underground passages. Thanks to their natural cunning, minotaurs use their labyrinthine hiding places to discourage unwary enemies who try to track them down or who simply stumble into their hiding places and get lost, slowly hunting down intruders who try in vain to find a way out. Only when desperation has clearly taken over does the minotaur hit its lost victims. When dealing with a group, minotaurs often let a creature escape, to spread its terrible tale and lure others, hoping to kill these beasts, into their labyrinths. Of course, to minotaurs, these would-be heroes make for delicious food.

Minotaurs can also be found in the service of a more powerful monster or evil creature, and they serve it as long as they can hunt and eat as they please. Generally this means guarding some powerful item or valuable location, but it can also mean working as a mercenary, hunting down the master's enemies.

Minotaurs are relatively straightforward fighters, using their horns to horribly gore the closest living creatures when they begin to fight.


\subsection{Mummies}

\medskip\index[Mostruario]{Mummy} \textbf{Mummy}

\textit{Media undead, lawful evil}

\textbf{STRENGTH} +3

\textbf{DEXTERITY} -1

\textbf{CONSTITUTION} +2

\textbf{INTELLIGENCE} -2

\textbf{WISDOM} +0

\textbf{CHARISMA} +1

\textbf{Initiative} -1 - \textbf{Defense} 13

\textbf{Hit Points} 58 (9d8 + 18)

\textbf{Movement} 6 m

\textbf{Saving Throws}: Fortitude +4, Reflex +2, Will +8

\textbf{Damage Vulnerability} fire

\textbf{Damage Resistances} from a non-magical weapon

\textbf{Immunity to Damage} Void, poison

\textbf{Condition Immunity} fascinated, poisoned, paralyzed, fatigue, frightened

\textbf{Senses} darkvision 18 m

\textbf{Languages} the languages he knew in life

\textbf{Challenge} 3 (700 PX)

\textit{\textbf{Undead Nature.}} A mummy does not need air, food, drink or sleep.

\textbf{Actions}

\textit{\textbf{Multiattack.}} The mummy can use her Terrifying Glance and make a rotten fist attack.

\textit{\textbf{Putrefying Punch.} Melee Weapon Attack}: +7 to hit, 1m range, one target.

\textit{Hits:} 10 (2d6 + 3) slash damage plus 10 (3d6) Void damage. If the target is a creature, it must succeed on a Fortitude save of 13 or be cursed by the mummy's rot. The cursed target cannot recover hit points, and his maximum hit points decrease by 10 (3d6) for every 24 hours the curse lasts. If the curse reduces the target's maximum hit points to 0, the target dies, and their body turns to dust. The curse lasts until removed by spell \textit{remove curse} or other spell.

\textit{\textbf{Dreadful Look.}} The mummy targets a creature she can see and is within 60 feet of her. If the target can see the mummy, they must succeed on a DC 11 Will saving throw against this spell or be frightened until the mummy's next round ends. If the target fails its saving throw by 5 or more, it is also paralyzed for the same duration. A target that succeeds at the saving throw is immune to the dreadful glance of all mummies (but not sovereign mummies) for the next 24 hours.

\medskip\index[Mostruario]{Sovereign Mummy} \textbf{Sovereign Mummy}

\textit{Media undead, lawful evil}

\textbf{STRENGTH} +4

\textbf{DEXTERITY} +0

\textbf{CONSTITUTION} +3

\textbf{INTELLIGENCE} +0

\textbf{WISDOM} +4

\textbf{CHARISMA} +3

\textbf{Initiative} +0 - \textbf{Defense} 25

\textbf{Hit Points} 97 (13d8 + 39)

\textbf{Movement} 6 m

\textbf{Saving Throws}: Fortitude +12, Reflex +6, Will +16

\textbf{Skills} Religion +5, History +5

\textbf{Damage Vulnerability} fire

\textbf{Immunity to Damage} from Void, poison; weapons +1

\textbf{Condition Immunity} fascinated, poisoned, paralyzed, fatigue, frightened

\textbf{Senses} darkvision 18 m

\textbf{Languages} the languages he knew in life

\textbf{Challenge} 15 (13000 PX)

\textit{\textbf{Heart of the Sovereign Mummy.}} As part of the ritual creating a sovereign mummy, the creature's heart and bowels are removed from the corpse and placed inside sealed containers. These containers are usually made of stone or ceramic, engraved or painted with religious hieroglyphs.

As long as his withered heart remains intact, the sovereign mummy cannot be permanently destroyed. When it drops to 0 hit points, the sovereign mummy is reduced to dust and reforms to full strength 24 hours later, reemerging from the dust near the sealed jar that holds its heart. To prevent a sovereign mummy from reforming and destroying it once and for all, its heart must be reduced to ashes. For this reason, the sovereign mummy usually keeps the heart and guts hidden inside a hidden tomb.

The heart of the sovereign mummy has Defense 5, 25 hit points, and immunity to all damage except fire.

\textit{\textbf{Spells.}} The mummy has CM 10. Her spellcasting characteristic is Wisdom, +9 to hit with spell attacks. The mummy has prepared the following spells: Tricks (at will): \textit{sacred flame, thaumaturgy}

level 1 (4 slots): \textit{command, tracer bolt, shield of faith}

level 2 (3 slots): \textit{spiritual weapon, block people, silence}

level 3 (3 slots): \textit{animate dead, dispel magic}

level 4 (3 slots): \textit{divination, guardian of faith}

level 5 (2 slots): \textit{contagion, insect plague}

level 6 (1 slot): \textit{wound}

\textit{\textbf{Undead Nature.}} A mummy does not need air, food, drink or sleep.

\textit{\textbf{Magic Resistance.}} The sovereign mummy has + 1d6 on saving throws against spells or other magical effects.

\textit{\textbf{Reinvigoration.}} A sovereign mummy forms a new body within 24 hours if its heart remains intact, recovering all hit points and being able to act again. The new body appears within 1 meter of the sovereign mummy's heart.

\textbf{Actions}

\textit{\textbf{Multiattack.}} The mummy can use her Terrifying Gaze and make a rotten fist attack.

\textit{\textbf{Putrefying Punch.} Melee Weapon Attack}: +22 to hit, 1m range, one target.

\textit{Hits:} 14 (3d6 + 4) slam damage plus 21 (6d6) Void damage. If the target is a creature, it must succeed on a Fortitude save 25 or be cursed by the mummy's rot. The cursed target cannot recover hit points, and his maximum hit points decrease by 10 (3d6) for every 24 hours the curse lasts. If the curse reduces the target's maximum hit points to 0, the target dies, and their body turns to dust. The curse lasts until removed by spell \textit{remove curse} or other spell.

\textit{\textbf{Dreadful Look.}} The mummy targets a creature she can see and is within 60 feet of her. If the target can see the mummy, they must succeed on a DC 16 Will saving throw against this spell or be frightened until the mummy's next round ends. If the target fails its saving throw by 5 or more, it is also paralyzed for the same duration. A target that succeeds at the saving throw is immune to the dreadful glance of all mummies (but not sovereign mummies) for the next 24 hours.

\textbf{Additional Actions}

The sovereign mummy can perform 3 additional Actions, chosen from the following options. He can only use one legendary option at a time, and only at the end of another creature's turn. The sovereign mummy recovers any additional Actions spent at the start of her round.

\textit{\textbf{Attack.}} The sovereign mummy makes a putrefying fist attack or uses its dreadful glance.

\textit{\textbf{Channel Negative Energy (Costs 2 Actions).}} The sovereign mummy can magically unleash negative energy. Creatures within 60 feet of the ruler mummy, including those behind barriers or corners, cannot recover hit points until the next ruler mummy round ends.

\textit{\textbf{Blasphemous Word (Costs 2 Actions).}} The sovereign mummy utters a blasphemous word. Any creature, excluding the undead, within 10 feet of the ruler mummy and that can hear this magical phrase must succeed on a DC 16 Fortitude save or be stunned until the ruler's next round ends.

\textit{\textbf{Blinding Dust.}} Blinding dust and sand magically swirl around the sovereign mummy. Any creature within 1 meter of the sovereign mummy must succeed on a DC 16 Fortitude saving throw or be blinded until the creature's next round ends.

\textit{\textbf{Sand Whirlwind (Costs 2 Actions).}} The sovereign mummy can magically transform into a whirlwind of sand, moving up to 60 feet, and then returning to its normal form. While in whirlwind form, the sovereign mummy is immune to all damage, and cannot be grabbed, petrified, thrown prone, hindered or stunned. The equipment worn or carried by the sovereign mummy remains in its possession.

\subsection{Naga}

\medskip\index[Mostruario]{Naga Guardian} \textbf{Naga Guardian}

\textit{Large monstrosity, legal good}

\textbf{STRENGTH} +4

\textbf{DEXTERITY} +4

\textbf{CONSTITUTION} +3

\textbf{INTELLIGENCE} +3

\textbf{WISDOM} +4

\textbf{CHARISMA} +4

\textbf{Initiative} +4 - \textbf{Defense} 23

\textbf{Hit Points} 127 (15d10 + 45)

\textbf{Movement} 12 m

\textbf{Saving Throws}: Fortitude +9, Reflex +12, Will +12

\textbf{Immunity to Damage} poison

\textbf{Condition Immunity} fascinated, poisoned

\textbf{Senses} darkvision 18 m

\textbf{Languages} Celestial, Common

\textbf{Challenge} 10 (5.900 PX)

\textit{\textbf{Spells.}} The naga has CM 11. Her spellcasting characteristic is Wisdom (+8 to hit with spell attacks), and she only needs the verbal components to cast her spells. The naga prepares the following spells:

Tricks (at will): \textit{sacred flame, repair, thaumaturgy}

level 1 (4 slots): \textit{command, heal wounds, shield of faith}

level 2 (3 slots): \textit{block people, calm emotions}

level 3 (3 slots): \textit{clairvoyance, cast curse}

level 4 (3 slots): \textit{exile, freedom of movement}

level 5 (2 slots): \textit{fire strike, constriction}

level 6 (1 slot): \textit{true seeing}

\textit{\textbf{Rejuvenation.}} If he dies, the naga comes back to life in 1d6 days and regains all of his hit points. Only the \textit{wish} spell can prevent this trait from working.

\textbf{Actions}

\textit{\textbf{Bite.} Melee weapon attack}: +14 to hit, range 10 feet, a creature.

\textit{Hits:} 8 (1d8 + 4) piercing damage, and the target must make a DC 15 Fortitude save, taking 45 (10d8) poison damage on a failed save, or half that damage if he succeeds.

\textit{\textbf{Spit Poison.} Ranged Weapon Attack}: +14 to hit, range 5m, a creature.

\textit{Strikes:} The target must make a DC 15 Fortitude save, taking 45 (10d8) poison damage on a failed save, or half that damage on a successful one.

\textbf{Ecology} \\
Environment: Temperate Plains \\
Organization: Solitary, pair or nest (3-6) \\
\textbf{Treasure}: Standard \\
\textbf{Description} \\
Though fierce in appearance, with brilliant scales, cobra-like hoods, and powerful serpentine bodies, guardian nagas serve as conscientious protectors of places of exceptional power and sacredness. Their scales often sport elaborate designs similar to those of exotic jungle snakes. A typical guardian naga reaches a length of 4.2 meters and an approximate weight of 175 kg.

While some guardian nagas adhere to exotic practices of ancient or forgotten deities, others are simply drawn to sites of outstanding natural beauty, such as temples on towering waterfalls, natural pinnacles and mountain peaks, guarding them with the utmost reverence and a sense of duty. Often these nagas join still active faiths, serving as protectors of shrines or ancient treasures. A pair of nagas may settle near a site they deem worthy of protection, hatching a brood and raising offspring there. When young people reach adulthood, they can choose to leave to find their home or stay and protect the area their parents are watching. Sometimes, a guardian naga who guards ruins or a temple is just the latest in a succession of sentries that have alternated over the centuries. These sentries often take the same name as their predecessors, appearing to be a single, exceptionally long-lived individual.


\medskip\index[Mostruario]{Spiritual Naga} \textbf{Spiritual Naga}

\textit{Large monstrosity, chaotic evil}

\textbf{STRENGTH} +4

\textbf{DEXTERITY} +3

\textbf{CONSTITUTION} +2

\textbf{INTELLIGENCE} +3

\textbf{WISDOM} +2

\textbf{CHARISMA} +3

\textbf{Initiative} +3 - \textbf{Defense} 19

\textbf{Hit Points} 75 (10d10 + 20)

\textbf{Movement} 12 m

\textbf{Saving Throws}: Fortitude +8, Reflex +10, Will +10

\textbf{Immunity to Damage} poison

\textbf{Condition Immunity} fascinated, poisoned

\textbf{Senses} darkvision 18 m

\textbf{Languages} Abyssal, Common

\textbf{Challenge} 8 (3.900 PX)

\textit{\textbf{Spells.}} The naga has CM 10. His spellcasting ability is Intelligence (+6 to hit with spell attacks), and he only needs the verbal components to cast his spells . The naga prepares the following spells:

Tricks (at will): \textit{minor illusion, magic hand, ray of} \textit{frost}

level 1 (4 slots): \textit{charm people, detect magic,} \textit{sleep}

level 2 (3 slots): \textit{block people, detect thoughts}

level 3 (3 slots): \textit{lightning, breathe underwater}

level 4 (3 slots): \textit{wither, dimensional gate}

level 5 (2 slots): \textit{dominate people}

\textit{\textbf{Rejuvenation.}} If he dies, the naga comes back to life in 1d6 days and recovers all of his hit points. Only the \textit{wish} spell can prevent this trait from working.

\textbf{Actions}

\textit{\textbf{Bite.} Melee Weapon Attack}: +12 to hit, range 10 feet, a creature.

\textit{Hits:} 7 (1d8 + 4) piercing damage, and the target must make a Fortitude save DC 13, taking 31 (7d8) poison damage on a failed save, or half that damage if he succeeds.

\subsection{Animated Objects}

\medskip\index[Mostruario]{Animated Armor} \textbf{Animated Armor}

\textit{Medium construct, misaligned}

\textbf{STRENGTH} +2

\textbf{DEXTERITY} +0

\textbf{CONSTITUTION} +1

\textbf{INTELLIGENCE} -5

\textbf{WISDOM} -4

\textbf{CHARISMA} -5

\textbf{Initiative} +0 - \textbf{Defense} 19

\textbf{Hit Points} 33 (6d8 + 6)

\textbf{Movement} 7 m

\textbf{Saving Throws}: Fortitude +2, Reflex +0, Will -4

\textbf{Immunity to Damage} poison

\textbf{Condition Immunity} blinded, fascinated, deafened, poisoned, paralyzed, petrified, fatigue, frightened

\textbf{Senses} blind sight 18 m (blind beyond this radius)

\textbf{Languages} -

\textbf{Challenge} 1 (200 PX)

\textit{\textbf{False Appearance.}} While the armor remains immobile, it is indistinguishable from normal armor.

\textit{\textbf{Susceptibility to Anti Magic.}} Armor is incapacitated if it is in the area of a \textit{anti-magic field}. If it is the target of \textit{dispel} \textit{spells}, the armor must succeed on a Fortitude save against the spell's saving throw DC or be unconscious for 1 minute.

\textbf{Actions}

\textit{\textbf{Multi Attack.}} The armor makes two melee attacks.

\textit{\textbf{Slam.} Melee Weapon Attack}: +4 hit, 1m range, one target.

\textit{Strikes:} 5 (1d6 + 2) hit damage.

\medskip\index[Mostruario]{Flying Sword} \textbf{Flying Sword}

\textit{Small construct, misaligned}

\textbf{STRENGTH} +1

\textbf{DEXTERITY} +2

\textbf{CONSTITUTION} +0

\textbf{INTELLIGENCE} -5

\textbf{WISDOM} -3

\textbf{CHARISMA} -5

\textbf{Initiative} +2 - \textbf{Defense} 18

\textbf{Hit Points} 17 (5d6)

\textbf{Movement} 0m, flight 15m (float)

\textbf{Saving Throws} Fortitude +1, Reflex +3, Will -4

\textbf{Immunity to Damage} poison

\textbf{Condition Immunity} blinded, fascinated, deafened, poisoned, paralyzed, petrified, frightened

\textbf{Senses} blind sight 18 m (blind beyond this radius)

\textbf{Languages} -

\textbf{Challenge} 1/4 (50 PX)

\textit{\textbf{False Appearance.}} While the weapon remains immobile and is not flying, it is indistinguishable from a normal sword.

\textit{\textbf{Susceptibility to Anti-Magic.}} The sword is incapacitated if it is in the area of a \textit{anti-magic field}. If it is the target of \textit{dispel} \textit{spells}, the sword must succeed on a Fortitude save against the spell's saving throw DC or be unconscious for 1 minute.

\textbf{Actions}

\textit{\textbf{Longsword.} Melee Weapon Attack}: +3 to hit, 1m range, one target.

\textit{Strikes:} 5 (1d8 + 1) slashing damage.


\medskip\index[Mostruario]{Carpet of Suffocation} \textbf{Carpet of Suffocation}

\textit{Large construct, misaligned}

\textbf{STRENGTH} +3

\textbf{DEXTERITY} +2

\textbf{CONSTITUTION} +0

\textbf{INTELLIGENCE} -5

\textbf{WISDOM} -4

\textbf{CHARISMA} -5

\textbf{Initiative} +2 - \textbf{Defense} 13

\textbf{Hit Points} 33 (6d10)

\textbf{Movement} 3 m

\textbf{Saving Throws}: Fortitude +4, Reflex +2, Will -4

\textbf{Immunity to Damage} poison

\textbf{Condition Immunity} blinded, fascinated, deafened, poisoned, paralyzed, petrified, frightened

\textbf{Senses} blind sight 18 m (blind beyond this radius)

\textbf{Languages} -

\textbf{Challenge} 2 (450 PX)

\textit{\textbf{False Appearance.}} While the carpet remains immobile, it is indistinguishable from a normal carpet.

\textit{\textbf{Susceptibility to Anti-Magic.}} The carpet is incapacitated while in the area of a \textit{anti-magic field}. If it is the target of \textit{dispel} \textit{spells}, the carpet must succeed on a Fortitude saving throw against the caster's saving throw DC or fall unconscious for 1 minute.

\textit{\textbf{Transfer Damage.}} While grabbing a creature, the rug takes only half of the damage dealt to it, and the creature grabbed by the rug takes the other half.

\textbf{Actions}

\textit{\textbf{Choke.} Melee Weapon Attack}: +5 to hit, range 1m, Medium or smaller creature.

\textit{Strikes:} The creature is grabbed (DC 13 to escape). Until the grapple is complete, the target is hampered, blinded, and in danger of suffocating, but the carpet cannot suffocate another target. Also, at the start of each target's turn, the target takes 10 (2d6 + 3) hit damage.

\medskip\index[Mostruario]{Ogre} \textbf{Ogre}

\textit{Large giant, chaotic evil}

\textbf{STRENGTH} +4

\textbf{DEXTERITY} -1

\textbf{CONSTITUTION} +3

\textbf{INTELLIGENCE} -3

\textbf{WISDOM} -2

\textbf{CHARISMA} -2

\textbf{Initiative} -1 - \textbf{Defense} 12 (leather armor)

\textbf{Hit Points} 59 (7d10 + 21)

\textbf{Movement} 12 m

\textbf{Saving Throws}: Fortitude +6, Reflex +0, Will +1

\textbf{Senses} darkvision 18 m

\textbf{Languages} Common, Giant

\textbf{Challenge} 2 (450 PX)

\textbf{Actions}

\textit{\textbf{Heavy Club.} Melee Weapon Attack}: +6 to hit, 1m range, one target.

\textit{Strikes:} 13 (2d8 + 4) hit damage.

\textit{\textbf{Javelin.} Melee or Ranged Weapon Attack}: +6 to hit, range 1 m and range 12m, one target.

\textit{Strikes:} 11 (2d6 + 4) piercing damage.

\textbf{Ecology} \\
Environment: Cold or temperate hills \\
Organization: Solitary, couple, group (3-4) or family (5-16) \\
\textbf{Treasure}: Standard (Leather Armor, Heavy Club, 4 Javelins, other) \\
\textbf{Description} \\
There are horrendous elements in the stories about ogres: brutality and ferocity, cannibalism and torture. Then rape, dismemberment, necrophilia, incest, mutilation and other examples of cruelty. Those who have never met ogres take these stories as a warning. Anyone who has survived such an encounter knows that stories are nothing compared to reality.

Ogres enjoy the suffering of others. If they do not have the smaller breeds available to crush in their fat hands or to violate in violent embraces, they have fun with each other. There is no taboo for ogres. One might think that, left to itself, a tribe of ogres would tear themselves apart and that only the strongest would survive: if there is one thing that ogres respect, however, it is the family.

The ogre tribes are known as families, and many of their deformities are caused by the common practice of incest. The chief of the tribe is often the father, but in some cases a female ogre is able to claim the title of mother. The ogre tribes quarrel with each other, which keeps them busy and prevents them from harassing their neighbors. From time to time, however, a particularly violent or feared patriarch emerges, capable of uniting multiple families under his command.

The regions inhabited by the ogres are sad and degraded places, as these giants live in squalor and do not feel the need to be in harmony with their surroundings. The boundary between civilized lands and those of the ogres is a place of despair inhabited by outcasts, where the Ogremanni live, deformed progeny that arise from the raids that ogres carry out in the lands of humans.

The games of the ogres are violent and cruel: the victims used as toys are lucky to die on the first day. The ogres 'cruel sense of humor is the only case in which they show creativity: the ogres' methods and tools of torture seem to have come out of nightmares.

The great strength and lack of imagination make them particularly suitable for heavy work, in mines, as blacksmiths or in logging. The most powerful giants (especially those of the Hills and Rocks) often subjugate ogre families to become their servants.

An adult ogre is about 3 meters tall and weighs around 325 kg.


\medskip\index[Mostruario]{Shadow} \textbf{Shadow}

\textit{Medium undead, chaotic evil}

\textbf{STRENGTH} -2

\textbf{DEXTERITY} +2

\textbf{CONSTITUTION} +1

\textbf{INTELLIGENCE} -2

\textbf{WISDOM} +0

\textbf{CHARISMA} -1

\textbf{Initiative} +2 - \textbf{Defense} 13

\textbf{Hit Points} 16 (3d8 + 3)

\textbf{Movement} 12 m

\textbf{Saving Throws}: Fortitude +3, Reflex +3, Will +4

\textbf{Skills} Move Silently / Hide +4 (+6 in dim light or dark)

\textbf{Damage Vulnerability} from Light

\textbf{Resistance to Damage} acid, cold, lightning, fire, sound; non-magical weapon

\textbf{Immunity to Damage} Void, poison

\textbf{Condition Immunity} grabbed, poisoned, entangled, paralyzed, petrified, prone, fatigue, frightened

\textbf{Senses} darkvision 18 m

\textbf{Languages} -

\textbf{Challenge} 1/2 (100 PX)

\textit{\textbf{Amorphous.}} The shadow can move through a narrow space up to 3 centimeters without squeezing.

\textit{\textbf{Weakness in the sunlight.}} While in the sunlight, the shadow has -1d6 on attack rolls, proficiency checks, and saving throws.

\textit{\textbf{Shadow Spirit.}} While in a zone of dim light the Shadow heals 5 hit points at the start of its round, if it is in a zone of darkness it heals 10 hit points at the beginning of his round and can become invisible using 1 Action. Shadow Spirit increases the Shadow Challenge Rank by 1.

\textit{\textbf{Shadow Stealth.}} When in dim light or darkness, the shadow can perform the Hide action as a bonus action.

\textit{\textbf{Undead Nature.}} A shadow needs no air, food, drink, or sleep.

\textbf{Actions}

\textit{\textbf{Force Drain.} Melee Weapon Attack}: +4 to hit, range 1 yd, a creature.

\textit{Strikes:} 9 (2d6 + 2) Void damage, and the target's Strength score is reduced by 1. The target dies if this reduces its Strength to -5. Otherwise, the reduction remains until the target rests for 8 hours.

If a non-evil humanoid dies from this attack, a new shadow will animate within 1d4 hours of its corpse.

\textbf{Ecology}
Environment: Any \\
Organization: Solitary, pair, group (3–6) or swarm (7–12) \\
\textbf{Treasure}: Standard \\

\textbf{Description} \\
The evil shadow moves along the border between the darkness of darkness and the harsh truth of light. The shadow prefers to haunt the ruins civilization leaves behind, where it hunts down living creatures foolish enough to stumble upon its territory. The shadow is a hideous undead, and as such it has no apparent purpose or motive other than sucking life force and vitality from living beings.


\medskip\index[Mostruario]{Homunculus} \textbf{Homunculus}

\textit{Tiny construct, neutral}

\textbf{STRENGTH} -3

\textbf{DEXTERITY} +2

\textbf{CONSTITUTION} +0

\textbf{INTELLIGENCE} +0

\textbf{WISDOM} +0

\textbf{CHARISMA} -2

\textbf{Initiative} +2 - \textbf{Defense} 14

\textbf{Hit Points} 5 (2d4)

\textbf{Movement} 6m, flight 12m

\textbf{Saving Throws}: Fortitude +0, Reflex +4, Will +1

\textbf{Immunity to Damage} poison

\textbf{Condition Immunity} fascinated, poisoned

\textbf{Senses} darkvision 18 m, blind sight 3 m

\textbf{Languages} understands the languages of its creator but cannot speak

\textbf{Challenge} 0 (10 PX)

\textit{\textbf{Telepathic Bond.}} While the homunculus is on the same plane of existence as his master, he can magically communicate to his master what he perceives, and the two can communicate telepathically.

\textbf{Actions}

\textit{\textbf{Bite.} Melee weapon attack}: +4 to hit, range 1 yards, a creature.

\textit{Strikes:} 1 piercing damage, and the target must succeed on a DC 10 Fortitude saving throw or be poisoned for 1 minute. If the saving throw is failed by 5 or more, the target is instead poisoned for 5 (1d10) minutes and while poisoned in this way is also unconscious.

\medskip\index[Mostruario]{Oni} \textbf{Oni}

\textit{Large giant, lawful evil}

\textbf{STRENGTH} +4

\textbf{DEXTERITY} +0

\textbf{CONSTITUTION} +3

\textbf{INTELLIGENCE} +2

\textbf{WISDOM} +1

\textbf{CHARISMA} +2

\textbf{Initiative} +2 - \textbf{Defense} 20 (chain mail)

\textbf{Hit Points} 110 (13d10 + 39)

\textbf{Movement} 9 m, flight 9 m

\textbf{Saving Throws}: Fortitude +7, Reflex +4, Will +6

\textbf{Skills} Arcane +5, Deceive +8, Awareness +4

\textbf{Senses} darkvision 18 m

\textbf{Languages} Common, Giant

\textbf{Challenge} 7 (2,900 PX)

\textit{\textbf{Magical Weapons.}} Weapon attacks from the oni are magical.

\textit{\textbf{Innate Spells.}} The spellcasting characteristic of the oni is Charisma. The oni can cast these spells innately, without the need for material components:

At will: \textit{invisibility, darkness}

1 / day: \textit{charm on people, cone of cold, gaseous form,}
\textit{sleep}

\textit{\textbf{Regeneration.}} If it has at least 1 hit point, the oni recovers 10 hit points at the start of its round.

\textbf{Actions}

\textit{\textbf{Multiattack.}} The oni makes two attacks, with claws or glaive.

\textit{\textbf{Claw (Oni Form only).} Melee Weapon Attack}: +11 to hit, 1m range, one target. \textit{Strikes:} 8 (1d8 + 4) slashing damage.

\textit{\textbf{Glaive.} Melee Weapon Attack}: +7 to hit, 3m range, one target.

\textit{Strikes:} 15 (2d10 + 4) slashing damage, or 9 (1d10 + 4) slashing damage in Small or Medium form.

\textit{\textbf{Shapeshift.}} The oni can magically transform into a Small or Medium humanoid, a Large giant, or revert to its true form. Aside from bounty, his stats are the same in each form. The only equipment that is transformed is the glaive, which shrinks so that it can also be wielded in humanoid form. If the oni dies, it reverts to its true shape, and the glaive reverts to its original size.

\medskip\index[Mostruario]{Orc} \textbf{Orc}

\textit{Medium humanoid (ogre), chaotic neutral}

\textbf{STRENGTH} +2

\textbf{DEXTERITY} +1

\textbf{CONSTITUTION} +2

\textbf{INTELLIGENCE} +0

\textbf{WISDOM} +0

\textbf{CHARISMA} +0

\textbf{Initiative} +2 - \textbf{Defense} 14 (leather armor)

\textbf{Hit Points} 12 (2d6 + 6)

\textbf{Movement} 9 m

\textbf{Saving Throws}: Fortitude +2, Reflex +2, Will +1

\textbf{Skills} Intimidate +1

\textbf{Senses} darkvision 18 m

\textbf{Languages} Common, Goblinoid

\textbf{Challenge} 1/2 (100 PX)

\textbf{Actions}

\textit{\textbf{Sword.} Melee Weapon Attack}: +4 to hit, 1m range, one target.

\textit{Strikes:} 8 (1d12 + 2) slashing damage.

\textit{\textbf{Javelin.} Melee or Ranged Weapon Attack}: +5 to hit, range 1 m and range 12m, one target. \textit{Strikes:} 6 (1d6 + 3) piercing damage.

\textbf{Ecology} \\
Environment: Temperate or underground hills and mountains \\
Organization: solitary, group (2-4), team (11-20 plus 2 3rd level sergeants and 1 3rd-6th level leader) or gang \\
\textbf{Treasure}: NPC gear (Studded Leather Armor, Sword, 4 Javelins, other treasure) \\
\textbf{Description} \\
Orcs are a breed created by Cattalm as an experiment with the aim of verifying whether a creature more intelligent but equally ferocious than the orcs could have been dominant.
The experiment was quite successful with orcs who founded kingdoms and conquered several regions. The chaotic push with the passage of time, acculturation, becoming permanent and the evolution of society has brought the orcs more and more out of the coils of Cattalm, even if it does not mean that many "barbaric" aspects have remained in traditional culture.
An adult male ogre stands 1.6 meters tall and weighs around 60 kg. Orcs and humans can mate, although this usually happens during raids, and not as a consensual union.

\medskip\index[Mostruario]{Orc} \textbf{Orc}

\textit{Medium humanoid (ogre), chaotic evil}

\textbf{STRENGTH} +3

\textbf{DEXTERITY} +1

\textbf{CONSTITUTION} +3

\textbf{INTELLIGENCE} -2

\textbf{WISDOM} +0

\textbf{CHARISMA} +0

\textbf{Initiative} +1 - \textbf{Defense} 14 (leather armor)

\textbf{Hit Points} 18 (3d8 + 6)

\textbf{Movement} 9 m

\textbf{Saving Throws}: Fortitude +3, Reflex +1, Will +2

\textbf{Skills} Intimidate +2

\textbf{Senses} darkvision 18 m

\textbf{Languages} Common, Goblinoid

\textbf{Challenge} 1 (100 PX)

\textit{\textbf{Aggressive.}} As a bonus action, the orc can move up to half its movement toward a hostile creature it can see.

\textbf{Actions}

\textit{\textbf{Double-headed Ax.} Melee Weapon Attack}: +5 to hit, 1m range, one target.

\textit{Strikes:} 9 (1d12 + 3) slashing damage.

\textit{\textbf{Javelin.} Melee or Ranged Weapon Attack}: +5 to hit, range 1 m and range 12m, one target. \textit{Strikes:} 6 (1d6 + 3) piercing damage.

\textbf{Ecology} \\
Environment: Temperate or underground hills and mountains \\
Organization: solitary, group (2-4), team (11-20 plus 2 3rd level sergeants and 1 3rd-6th level leader) or gang \\
\textbf{Treasure}: NPC gear (Studded Leather Armor, Glaive, 4 Javelins, other treasure) \\
\textbf{Description} \\
The main difference between orcs and civilized humanoids, besides their brute strength and inferior intelligence, is their character. As a culture, orcs are violent and aggressive, and the strong dominate the weak through fear and brutality. They take what they want by force and have no qualms about taking entire villages as slaves if they get the chance. They don't care about comforts, and their villages and fields tend to be dirty and precarious places, filled with drunken brawls, fighting arenas and other sadistic amusements. Deprived of the patience necessary to cultivate and capable of raising only the most robust and self-sufficient animals, orcs find it easier to take the fruit of their labor from others. They are arrogant and quick to get mad when challenged, but care about honor only as long as doing so benefits them.

An adult male ogre stands 2 meters tall and weighs approximately 115 kg. Orcs and humans can mate, although this usually happens during raids, and not as a consensual union. Many orc tribes raise half-orcs on purpose, as they are excellent strategists and chieftains.

Although the vulgate says that orcs were created by Cattalm to destroy and bring chaos, it is also true that very often they are the victim of prejudices and summary judgments. Not all orcs are the same and not only physically, single orcs if not entire tribes live their existence in a "normal, civilized" way and yet in no state of Yeru there are penalties for killing an orc.

\medskip\index[Mostruario]{Wall Climbing Horror} \textbf{Wall Climbing Horror}

\textit{Large monstrosity, misaligned}

\textbf{STRENGTH} +4

\textbf{DEXTERITY} +0

\textbf{CONSTITUTION} +2

\textbf{INTELLIGENCE} -2

\textbf{WISDOM} +1

\textbf{CHARISMA} -2

\textbf{Initiative} +1 - \textbf{Defense} 15

\textbf{Hit Points} 75 (10d10 + 25)

\textbf{Movement} 9 m, climb 9 m

\textbf{Saving Throws}: Fortitude +5, Reflex +3, Will +4

\textbf{Senses} darkvision 3 m, blind sight 18 m

\textbf{Languages} Wall Climbing Horror

\textbf{Challenge} 3 (700 PX)

\textit{\textbf{Radar Sense}} the Wall Climbing Horror cannot use blind sight if it is deafened.

\textbf{Actions}

\textit{\textbf{Multi Attack.}} The Wall Climb Horror makes two attacks with its hook claws.

\textit{\textbf{Claws.} Melee Weapon Attack}: +7 to hit, 1m range, one target.

\textit{Strikes:} 10 (2d6 + 4) piercing damage, 1 bleed damage.

\textbf{Ecology} \\
\textbf{Environment: Underground}
Organization: Solitary, pair or pack (3-8) \\
\textbf{Treasure}: Accidental \\
\textbf{Description} \\
The Wall Climb Horror is a ferocious underground predator, aggressively defending its hunting grounds. The subterranean caves in which these creatures reside thunder with the bangs and rustles of their hooks as these creatures climb rocky cliffs or cave walls.

A Wall Climbing Horror is a monstrous creature with a vulture-like head and the thorax of a huge scarab, protected by an exoskeleton studded with sharp bony protrusions. It draws its name as well as its hideous appearance from the fact that using its long and muscular limbs ending in deadly curved hooked claws it is climbing the walls.

\textit{Echoes in the Dark}. Wall Climbing Horrors communicate by striking their exoskeleton or surrounding rocky surfaces with their hooks. What appears to others as a random din is actually an elaborate language that only the Wall Climbing Horrors understand and whose echo spreads for miles and miles into the subsoil.

\textit{Herd of Predators}. Wall Climbing Horrors are omnivorous creatures: they feed on mushrooms, lichens, plants, and any creature they can catch. Thanks to the hooked limbs, horrors benefit from an excellent grip on rocky surfaces and use their climbing skills to ambush prey from above. They hunt in packs and work together to take on the biggest and most dangerous opponents. If a battle goes wrong, a Wall Climbing Horror quickly climbs a cave wall to escape.

\textit{Clan Solidali}. Hook horrors live in large family groups or clans. Each clan is ruled by the eldest female, who usually places her mate at the head of the clan hunters. Wall Climbing Horrors spawn in a central, well-defended area of the caves used as a lair.

\medskip\index[Mostruario]{Owlbear} \textbf{Owlbear}

\textit{Large beast, misaligned}

\textbf{STRENGTH} +5

\textbf{DEXTERITY} +1

\textbf{CONSTITUTION} +3

\textbf{INTELLIGENCE} -4

\textbf{WISDOM} +1

\textbf{CHARISMA} -2

\textbf{Initiative} +1 - \textbf{Defense} 15

\textbf{Hit Points} 59 (7d10 + 21)

\textbf{Movement} 12 m

\textbf{Saving Throws}: Fortitude +10, Reflex +5, Will +2

\textbf{Skills} Awareness +3

\textbf{Senses} darkvision 18 m

\textbf{Languages} -

\textbf{Challenge} 3 (700 PX)

\textit{\textbf{Sight and smell refined.}} The Owlbear has + 1d6 on Wisdom (Awareness) checks based on smell or sight.

\textbf{Actions}

\textit{\textbf{Multiattack.}} The Owlbear makes two attacks: one with its beak and one with its claws.

\textit{\textbf{Claws.} Melee Weapon Attack}: +9 to hit, 1m range, one target.

\textit{Strikes:} 14 (2d8 + 5) slashing damage.

\textit{\textbf{Beak.} Melee weapon attack}: +9 to hit, range 1 yards, a creature.

\textit{Strikes:} 10 (1d10 + 5) piercing damage.

\textbf{Ecology} \\
\textbf{Environment: Temperate Forests}
Organization: Solitary, pair or pack (3-8) \\
\textbf{Treasure}: Accidental \\
\textbf{Description} \\
The origins of the Owlbear are the subject of debate among scholars of monstrous creatures. Most of them agree that it was a wizard, in the past, who created the first example by combining a bear with a giant owl; perhaps as an experiment in some insane concept of the nature of life, but more likely because of its utter madness. Whatever the original purpose of a creation as insane as the Owlbear, the creature has begun to reproduce, and has become one of the best-known predators of the woodlands.

Owlbear are savage predators known for their bad temperament, aggression and ferocity. They tend to attack anything that moves in front of them, even if that doesn't show warlike intentions. Many scholars who have encountered these creatures in the wild have noticed that they always have bloodshot eyes that revolve around just before an attack. This is generally seen as a sign of insanity, suggesting that all Owlbears are born with a pathological need to fight and kill, but more realistic researchers believe it is due to the structure of their keen eyes.

The Owlbear live in the innermost and hidden areas of the woods, and prepare their burrows in intricate forests or in dark and deep caves. They can hunt both day and night, depending on the habits of the prey that populate the territories surrounding their lair.

Adult Owlbears live in pairs and hunt prey in packs, leaving the cubs in burrows. 1d6 cubs can usually be found in a lair, which can be worth up to 750 gp in city markets.

While it is nearly impossible to tame them due to their wild nature, Owlbears can be exploited as guardians of a specific territory, as long as they are left free to roam around to hunt. Professional trainers charge up to 2,000 gp to train a Owlbear to become a keeper who obeys simple commands (DC 23 for a baby owl, DC 30 for an adult Owlbear). \\

\textit{\textbf{Variant}}: \textbf{Polar Owlbear} \index{Polar Owlbear} \\
This bear-owl is found in arctic or snow-capped mountain regions. Unlike the normal Owlbear, it is more robust and strong. Has 85 HP, +10 hit, 21 claw damage, +1 Bleed, 15 beak damage. GS 4

\medskip\index[Mostruario]{Wise Owlbear} \textbf{Wise Owlbear}

\textit{Large monstrosity, neutral}

\textbf{STRENGTH} +3

\textbf{DEXTERITY} +1

\textbf{CONSTITUTION} +2

\textbf{INTELLIGENCE} +3

\textbf{WISDOM} +3

\textbf{CHARISMA} +1

\textbf{Initiative} +3 - \textbf{Defense} 15

\textbf{Hit Points} 45 (7d10 + 10)

\textbf{Movement} 12 m

\textbf{Saving Throws}: Fortitude +10, Reflex +5, Will +4

\textbf{Skills} Awareness +3

\textbf{Senses} darkvision 18 m

\textbf{Languages} understands and reads the following languages: Common, Celestial, Hellish, Dwarven, Elven, Orc, Giant

\textbf{Challenge} 3 (700 PX)

\textit{\textbf{Sight and smell refined.}} The wise Owlbear has + 1d6 on Wisdom (Awareness) checks based on smell or sight.

\textit{\textbf{Innate Spells.}} The spellcasting characteristic of the wise Owlbear is Intelligence. The wise Owlbear can innately cast the following spells, without the need for material components:

At will: \textit{Magic Hand}

\textbf{Actions}

\textit{\textbf{Multiattack.}} The wise Owlbear makes two attacks: one with its beak and one with its claws.

\textit{\textbf{Claws.} Melee Weapon Attack}: +7 to hit, 1m range, one target.

\textit{Strikes:} 14 (2d8 + 5) slashing damage.

\textit{\textbf{Beak.} Melee weapon attack}: +7 to hit, range 1 yards, a creature.

\textit{Strikes:} 10 (1d10 + 5) piercing damage.

\textbf{Ecology} \\
\textbf{Environment: Temperate Forests}
Organization: Solitary, pair or pack (3-8) \\
\textbf{Treasure}: Standard + 10 \% Manuals and Tomes \\
\textbf{Description} \\
The origins of the wise Owlbear are as mysterious as those of its unwise relative but enthusiasts of these creatures have them descended directly from Nethergal as a variant of the original Owlbear.
Usually the wise Owlbear loves to surround himself with books and adores the company of other wise men but does not disdain the tales of adventurers and the compelling ballads of storytellers. The wise Owlbear has a real talent for languages and although he cannot speak in a way that a man can understand, he can understand many languages spoken and written and in a few days he is able to learn new ones (such as Universal Language Advantage) both spoken and written. The wise bear is able to read any language or code if he has the opportunity to study it for 3 days.
Usually weaker and more fragile than the close relative they are however fearful beings in combat.
Preferably, a wise Owlbear does not attack except for defense and seeks an approach that is as tactical and useful as possible. A characteristic feature of the wise Owlbear is a scarf worn around the absent neck. Killing a wise Owlbear is an affront to the Devotees and Followers of Nethergal, it has also happened that the Patron himself took away the ability to communicate to those who are stained with brutality with his favorite creatures.

Training a wise Owlbear is much easier than a Owlbear but the creature's high intelligence will push it to have an equal relationship or rather as a familiar.

The Magic Hand spell is usually used to browse the more delicate tomes and to write, albeit very slowly.

\medskip\index[Mostruario]{Otyugh} \textbf{Otyugh}

\textit{Large aberration, neutral}

\textbf{STRENGTH} +3

\textbf{DEXTERITY} +0

\textbf{CONSTITUTION} +4

\textbf{INTELLIGENCE} +2

\textbf{WISDOM} +1

\textbf{CHARISMA} -2

\textbf{Initiative} +0 - \textbf{Defense} 17

\textbf{Hit Points} 114 (12d10 + 48)

\textbf{Movement} 9 m

\textbf{Saving Throws}: Fortitude +3, Reflex +2, Will +6

\textbf{Senses} darkvision 36 m

\textbf{Languages} Otyugh

\textbf{Challenge} 5 (1,800 PX)

\textit{\textbf{Limited Telepathy.}} The otyugh can magically transmit simple messages and images to any creature within 36 meters of it that can understand a language. This form of telepathy does not allow the receiving creature to respond telepathically.

\textbf{Actions}

\textit{\textbf{Multiattack.}} The otyugh makes three attacks: one with its bite and two with its tentacles.

\textit{\textbf{Bite.} Melee Weapon Attack}: +9 to hit, 1m range, one target.

\textit{Strikes:} 12 (2d8 + 3) piercing damage. If the target is a creature, it must succeed on a DC 15 Fortitude save against disease or be poisoned until the disease is cured. Every 24 hours thereafter, the target must re-roll the saving throw, reducing their maximum hit points by 5 (1d10) if they fail. If the saving throw is successful, the disease is over. The target dies if the disease reduces their maximum hit points to 0.

This reduction in the character's maximum hit points persists until the disease is cured.

\textit{\textbf{Tentacle.} Melee Weapon Attack}: +9 to hit, 3m range, one target.

\textit{Strikes:} 7 (1d8 + 3) hit damage plus 4 (1d8) piercing damage. If the target is Medium or smaller, it is grabbed (DC 13 to flee) and hampered until the grab is finished. The otyugh has two tentacles, each of which can grab a different target.

\textit{\textbf{Tentacle Slam.}} The otyugh slams creatures grabbed by its tentacles, into each other or onto the floor. Each creature must succeed on a DC 14 Fortitude save or take 10 (2d6 + 3) hit damage and be stunned until the end of the next round of the otyugh. If the saving throw is successful, the target takes half the hit damage and is not stunned.

\textbf{Ecology} \\
Environment any dungeon \\
Organization: Solitary, pair or group (3-4) \\
\textbf{Treasure}: Standard \\
\textbf{Description} \\
Otyughs are particularly filthy and hideous creatures that live in places sane people tend to avoid. Their burrows are found in sewers, cesspools, landfills and the most mephitic swamps: the dirtier a place, the more it attracts otyughs. They love the role of the scavenger, and roam underground caves in search of new treats among the garbage. Once found, they gorge themselves and bring back to their lair what they cannot consume at once. The otyughs spend a lot of time in their filthy burrows, which they fill with carrion and manure, which release mephitic scents.

Intelligent creatures living in subterranean areas near otyughs sometimes form alliances of convenience with them. They provide them with waste and raw meat to the otyughs, making them a real means of disposal. In return, the otyughs leave their benefactors alone, do not attack them, and can even act as guardians.

The thing that most breeds find terrifying about otyughs is not their diet or the smell of their burrows, but the fact that creatures with their tastes aren't just mindless scavengers. The otyughs are indeed surprisingly intelligent, and love to form alliances with those who supply them with more refined foods of manure and dirt. Most otyughs realize that other creatures find them revolting, but few who really care.

\medskip\index[Mostruario]{Panopticon} \textbf{Panopticon}

\textit{Large aberration, evil}

\textbf{STRENGTH} +0

\textbf{DEXTERITY} +1

\textbf{CONSTITUTION} +2

\textbf{INTELLIGENCE} +3

\textbf{WISDOM} +2

\textbf{CHARISMA} +2

\textbf{Initiative} +5 - \textbf{Defense} 26

\textbf{Hit Points} 82 (11d8 + 38)

\textbf{Movement} 1m, flight 10 meters (good)

\textbf{Saving Throws}: Fortitude +6, Reflex +7, Will +10

\textbf{Resistance}: acid, electricity

\textbf{Senses} darkvision 36m, true vision 18m

\textbf{Languages} telepathy 50 m

\textbf{Challenge} 12 (8.400 PX)

\textbf{Actions}

\textit{\textbf{Multiattack.}} The Panopticon can attack with two short tentacles.

\textit{\textbf{Tentacle.} Melee Weapon Attack}: +12 to hit, 1m range, one target.

\textit{Strikes:} 6 (1d6 + 3) piercing slash damage.

\textit{\textbf{He who sees everything}}. The Panopticon can activate one of its watched tentacles (2 Actions).

\textit{The one that freezes}: The eye points to a target within 60 feet, a ray of frost is activated on it. 8d8 cold damage, Reflex save DC 23 to avoid the hit completely.

\textit{The one that melts}: The eye points to a target within 30 feet, on which a beam is activated which has acid effects. 4d8 acid damage, Reflex save DC 23 to halve the damage.

\textit{The one that burns}: The eye points to a target within 60 feet, a fiery beam is activated on it. 8d8 fire damage, Reflex save DC 23 to avoid the hit completely.

\textit{The Paralyzed One}: The eye points to a target within 30 feet, upon which a beam is activated that paralyzes the creature. Will save DC 23 to completely avoid effects.

\textit{The one that slows down}: the eye points to a 30-foot cone. A slowing beam is projected onto affected creatures. Will save DC 23 to completely avoid effects. Duration 1 minute.

\textit{What confuses}: the eye points to a cone of 18 meters. A beam is projected on affected creatures causing confusion. Will save DC 23 to completely avoid effects. Duration 1 minute, each round you can make a new saving throw to recover from the effects.

\textit{The one that sleeps}: The eye points to a target within 36 meters, on which a beam is activated that puts the creature to sleep. Will save DC 23 to completely avoid effects.

\textit{The one who moves}; this eye can manifest the magic hand spell or telekinesis.

\textit{\textbf{One glance.}} The Panopticon activates the eye centering. The central eye can be used as a Reaction Action to cast counter spell on a spell it has seen cast.

\textbf{Ecology} \\
Environment any dungeon \\
Organization: Solitary, couple \\
\textbf{Treasure}: Triple \\
\textbf{Description} \\
Panopticons are xenophobic aberrations, balls of hard flying flesh with a large central eye, a large mouth and 7 tentacles about 1 meter long each with an eye (about 10 cm in diameter) of a different color.

Little is known of the origin of the panopticons, they are thought to be an evolutionary experiment by Calicante in an attempt to create a sentient and dominant race.

Unfortunately arrogance, pride, the desire to be the center of attention have wrecked these attempts at society and the panopticons have often dispersed underground.

The Panopticons have a very long life, in the order of a thousand years but they are also creatures that have more than doubled this limit. Panopticons increase in size with age and so do the number of eyes. The statistics reported here refer to an adult specimen of about 300 years.


\medskip\index[Mostruario]{Pegasus} \textbf{Pegasus}

\textit{Large celestial, chaotic good}

\textbf{STRENGTH} +4

\textbf{DEXTERITY} +2

\textbf{CONSTITUTION} +3

\textbf{INTELLIGENCE} +0

\textbf{WISDOM} +2

\textbf{CHARISMA} +1

\textbf{Initiative} +2 - \textbf{Defense} 13

\textbf{Hit Points} 59 (7d10 + 21)

\textbf{Movement} 18 m, flight 27 m

\textbf{Saving Throws} Fortitude +7, Reflex +6, Will +4

\textbf{Skills} Awareness +6

\textbf{Languages} includes Celestial, Common, Elven and Sylvan but cannot speak

\textbf{Challenge} 2 (450 PX)

\textbf{Actions}

\textit{\textbf{Hooves.} Melee Weapon Attack}: +6 to hit, 1m range, one target.

\textit{Strikes:} 11 (2d6 + 4) hit damage.

\textbf{Ecology}
Environment: Temperate and Warm Plains \\
Organization: Solitary, pair or pack (6-10) \\
\textbf{Treasure}: None \\
\textbf{Description} \\
The pegasus is a magnificent winged horse that sometimes serves the cause of good. While highly prized as flying mounts, pegasi are shy creatures that hardly make friends. A typical pegasus stands 1.8 meters high at the withers, weighs 750 kg and has a wingspan of 6 meters. Most pegasi are white, but sometimes some specimens have different colors.

The pegasus, despite appearances, is as intelligent as a human. Anyone who tries to train one to act as a mount will find that the pegasus is recalcitrant and even violent. A pegasus cannot speak, but he understands the Common and prefers the company of good creatures. The correct way to get a pegasus to be a mount is to befriend him with diplomacy, favors, and good deeds. A pegasus usually has an indifferent attitude towards good creatures, ill-disposed towards neutral ones and hostile towards evil ones. Before a pegasus can serve as a mount, it must be made friendly through a Diplomacy check or otherwise. Riding a pegasus requires an exotic saddle or bareback riding, as a normal saddle interferes with its wings. A pegasus can fight carrying a cavalier, but the cavalier cannot attack himself if he fails a Ride check. Trained pegasi have no fear of combat, and the cavalier does not need to make a Ride check to control it.

Pegasi lay eggs that are worth 1000 gp each on the market, while the young go up to 2000 gp each. Being intelligent and good creatures, selling eggs and young is essentially slavery: in good societies those who do it are despised or punished by the law.

Pegasi mature like horses. Professional trainers ask 1000 to train a pegasus, which will serve a good or neutral rider faithfully for life.

A light load for a pegasus is up to 150 kg; an average load is 150.5-300 kg; a heavy load is 300.5-450 kg.

In some pegasi the blood of an ancestor who was a heroic stallion is still strong. These champions have the lifespan of a human, the advanced archetype, perfect maneuverability, fire resistance 10, a +4 racial bonus on saving throws against poison, and immunity to petrify. Some manage to say a few words in Celestial or Common. They realize their superiority to other horses and pegasi, and do not have to be trained to fly with a rider, but only allow the greatest heroes to ride them.


\medskip\index[Mostruario]{Invisible Persecutor} \textbf{Invisible Persecutor}

\textit{Elemental mean, neutral}

\textbf{STRENGTH} +3

\textbf{DEXTERITY} +4

\textbf{CONSTITUTION} +2

\textbf{INTELLIGENCE} +0

\textbf{WISDOM} +2

\textbf{CHARISMA} +0

\textbf{Initiative} +4 - \textbf{Defense} 17

\textbf{Hit Points} 104 (16d8 + 32)

\textbf{Movement} 15m, flight 15m (floats)

\textbf{Saving Throws}: Fortitude +13, Reflex +11, Will +4

\textbf{Skills} Move Silently / Hide +10, Awareness +8

\textbf{Damage Resistances} from a non-magical weapon

\textbf{Immunity to Damage} poison

\textbf{Condition Immunity} grabbed, poisoned, entangled, paralyzed, petrified, unconscious, prone, fatigue

\textbf{Senses} darkvision 18 m

\textbf{Languages} Auran, understands the Common but does not speak it

\textbf{Challenge} 6 (2,300 PX)

\textit{\textbf{Infallible Hunter.}} The summoner assigns a prey to the persecutor. The persecutor knows the direction and distance at which the prey is as long as both are on the same plane of existence. The persecutor also knows the position of his summoner.

\textit{\textbf{Invisibility.}} The persecutor is invisible.

\textit{\textbf{Elemental Nature.}} An invisible persecutor does not need air, food, drink or sleep.

\textbf{Actions}

\textit{\textbf{Multiattack.}} The persecutor makes two slam attacks.



\textit{\textbf{Slam.} Melee Weapon Attack}: +12 to hit, 1m range, one target.

\textit{Strikes:} 10 (2d6 + 3) hit damage.

\medskip\index[Mostruario]{Pseudodrago} \textbf{Pseudodrago}

\textit{Tiny dragon, good neutral}

\textbf{STRENGTH} -2

\textbf{DEXTERITY} +2

\textbf{CONSTITUTION} +1

\textbf{INTELLIGENCE} +0

\textbf{WISDOM} +1

\textbf{CHARISMA} +0

\textbf{Initiative} +2 - \textbf{Defense} 14

\textbf{Hit Points} 7 (2d4 + 2)

\textbf{Movement} 5 meters, flight 18 m

\textbf{Saving Throws}: Fortitude +4, Reflex +5, Will +4

\textbf{Skills} Move Silently / Hide +4, Awareness +3

\textbf{Senses} darkvision 18 m, blind sight 3 m

\textbf{Languages} understands the Common and the Draconic but does not speak

\textbf{Challenge} 1/4 (50 PX)

\textit{\textbf{Resistance to Magic.}} The pseudo-dragon has + 1d6 on saving throws against spells and other magical effects.

\textit{\textbf{refined senses.}} The pseudodragon has + 1d6 on Wisdom (Awareness) checks based on sight, hearing, and smell.

\textit{\textbf{Limited Telepathy.}} The pseudodragon can communicate simple ideas, emotions, and images telepathically with any creature within 100 feet of it that can understand a language.

\textbf{Actions}

\textit{\textbf{Bite.} Melee Weapon Attack}: +4 hit, 1m range, one target.

\textit{Strikes:} 4 (1d4 + 2) piercing damage.

\textit{\textbf{Sting.} Melee weapon attack}: +4 to hit, range 1 yards, a creature.

\textit{Strikes:} 4 (1d4 + 2) piercing damage, and the target must succeed on a DC 11 Fortitude save or be poisoned for 1 hour. If the saving throw result is 6 or less, the target falls unconscious for the same duration, or until it takes damage or another creature uses an action to awaken it.

\textbf{Ecology} \\
Environment: Temperate Forests \\
Organization: Solitary, pair or nest (3-5) \\
\textbf{Treasure}: Standard \\
\textbf{Description} \\
Pseudodragons are small relatives of true dragons, playful and shy. They speak by chirping, hissing, growling and purring, but they can communicate telepathically with any intelligent creature. If approached peacefully with food offerings, they are willing to share information about what is on their territory, but threats and violence make them flee.

Pseudodragons are carnivores and eat insects, rodents, birds and snakes, although they eat eggs and love butter, cheese and fish. Sometimes they hunt on the ground like lizards or flying like predatory birds. As intelligent as most humanoids, they dislike being treated as pets, and prefer to be considered friends. They are wary of evil creatures, they can join spellcasters and Devotees as Familiars, and some have befriended Druids and rangers or collaborate with good dragons as sentries. Pseudodrags become Familiar only if they appreciate the caster's personality (and if the caster has the Familiar Skill and Charisma at least 1), but they can also bond with people whose company they enjoy. A pseudodragon might follow a character in this way for days, weeks, years, or even a lifetime, provided they are well fed and treated with affection.

Upon reaching adulthood, a pseudodragon's body is 30 centimeters long with a 60 centimeter tail, and weighs approximately 3.5 kg. A pseudodragon's eggs are as large as a chicken's, but leather-like in texture and spotted with brown, and females lay them in clusters of 2-5 each spring. A nest of pseudodragons (which make up a family group, and have not hatched from the same egg group) usually consists of a pair of adults and several near-adult pups.


\medskip\index[Mostruario]{Rakshasa} \textbf{Rakshasa}

\textit{Media fiend, lawful evil}

\textbf{STRENGTH} +2

\textbf{DEXTERITY} +3

\textbf{CONSTITUTION} +4

\textbf{INTELLIGENCE} +1

\textbf{WISDOM} +3

\textbf{CHARISMA} +5

\textbf{Initiative} +3 - \textbf{Defense} 23

\textbf{Hit Points} 110 (13d8 + 52)

\textbf{Movement} 12 m

\textbf{Saving Throws}: Fortitude +9, Reflex +12, Will +8

\textbf{Skills} Deceive +10, Perceive Emotions +8

\textbf{Vulnerability to Damage} piercing of magical weapons wielded by
good creature

\textbf{Immunity to Damage} hit, weapons +1

\textbf{Senses} darkvision 18 m

\textbf{Languages} Common, Hellish

\textbf{Challenge} 13 (10000 PX)

\textit{\textbf{Immunity to Limited Magic.}} The rakshasa is immune to affect or detection by spells of level 6 or lower unless he wishes to be affected. It has + 1d6 on saving throws against all other spells and magical effects.

\textit{\textbf{Innate Spells.}} The spellcasting characteristic of the rakshasa is Charisma (+10 to hit with spell attacks). The rakshasa can innately cast the following spells, without the need for material components:

At will: \textit{disguise oneself, minor illusion, individuation} \textit{of thoughts, magic hand}

3 / Day each: \textit{charm over people, major image,} \textit{detection of the magic, invisibility, suggestion} 1 / Day: \textit{dominate people, planar shift, vision of the} \textit{ true, fly}

\textbf{Actions}

\textit{\textbf{Multiattack.}} The rakshasa can make two claw attacks.

\textit{\textbf{Claw.} Melee Weapon Attack}: +13 to hit, 1m range, one target.

\textit{Strikes:} 9 (2d6 + 2) slashing damage, and if the target is a creature it is cursed. The magical curse takes effect whenever the target rests, filling the target's thoughts with horrible images and dreams. The cursed target receives no benefit from completing a rest. The curse lasts until removed by spell \textit{remove curse} or similar spell.

\textbf{Ecology}
Environment: Any \\
Organization: Solitary, couple or cult (3-12) \\
\textbf{Treasure}: Double (Dagger + 1, other treasure) \\
\textbf{Description} \\
The rakshasa is an evil spirit who disguises himself as a humanoid creature so that he can follow his prey in disguise. Personification of the taboos of most societies and capable of assuming the appearance of those he seeks to corrupt, a rakshasa performs a great many horrific deeds. If they were human, their blasphemy, cannibalism and even worse acts they perform would mark them as worthy criminals of the cruellest of hell.

When it has no other appearance, the rakshasa appears as a humanoid with the head of an animal. It often has the head of a large cat (such as tigers or panthers) or snake (such as cobras or vipers) and, although rarer, it can have the head of a gorilla, jackal, vulture, elephant, mantis, lizard, rhinoceros, wild boar and many still others. In many cases, the type of head possessed by a rakshasa says something about their personality: a tiger-headed rakshasa is stealthy and ravenous, while one with a boar's head can be greedy and cruel. These differences rarely affect rakshasa's base stats, although there are more powerful variants of the standard with multiple heads, more powerful magical powers, and weird and deadly additional special abilities.

Rakshasas despise religions; they recognize the power of the gods, but see themselves as the only beings worthy of reverence by the mortal races. Rakshasa devotees are therefore quite rare. Although rakshasas are external, they are also creatures of the Material Plane, and some believe that early rakshasas chose this exile over some other role offered to them by a long-forgotten god. While they are typically loners, it is not uncommon to find large families of rakshasas working together to bring about the downfall of a deadly civilization from within, over many generations.

A rakshasa is 1.8 meters tall and weighs 90 kg.

\medskip\index[Mostruario]{Remorhaz} \textbf{Remorhaz}

\textit{Huge monstrosity, misaligned}

\textbf{STRENGTH} +7

\textbf{DEXTERITY} +1

\textbf{CONSTITUTION} +5

\textbf{INTELLIGENCE} -3

\textbf{WISDOM} +0

\textbf{CHARISMA} -3

\textbf{Initiative} +1 - \textbf{Defense} 23

\textbf{Hit Points} 195 (17d12 + 85)

\textbf{Movement} 9 m, Burrow 6 m

\textbf{Saving Throws}: Fortitude +11, Reflex +7, Will +4

\textbf{Immunity to Damage} cold, fire

\textbf{Senses} darkvision 18 m, telluric sense 18 m

\textbf{Languages} -

\textbf{Challenge} 11 (7,200 PX)

\textit{\textbf{Heated Body.}} A creature that contacts the remorhaz or hits it with a melee attack while within 1 meter of it takes 10 (3d6) fire damage.

\textbf{Actions}

\textit{\textbf{Bite.} Melee Weapon Attack}: +18 to hit, 3m range, one target.

\textit{Strikes:} 40 (6d10 + 7) piercing damage plus 10 (3d6) fire damage. If the target is a creature, it is grabbed (DC 17 to escape). Until the grapple is complete, the target is in the way, and the remorhaz cannot bite another target.

\textit{\textbf{Swallow.}} The remorhaz makes a bite attack against a Medium or smaller target it is grabbing. If the attack hits, the creature takes bite damage and is swallowed, and the grab ends. The engulfed target is blinded and entangled, has full cover against attacks and other effects outside the remorhaz, and takes 21 (6d6) acid damage at the start of each remorhaz's turn.

If the remorhaz takes 30 or more damage in a single turn from a creature inside it, the remorhaz must succeed on a DC 15 Fortitude save at the end of that turn or vomit all engulfed creatures, which fall prone in a space within 3. meters from the remorhaz. If the remorhaz dies, a swallowed creature is no longer hampered by it and can exit the corpse using 5 meters of movement, coming out prone.

\textbf{Ecology} \\
Environment: Cold Deserts and Glaciers \\
Organization: Solitary \\
\textbf{Treasure}: None \\
\textbf{Description} \\
In a world of ice and snow, remorhazes are especially feared for the terrible fire that burns inside their bodies. This inner fire causes the plates along its back to become red-hot when the creature is particularly angry, excited, or panicked. Creatures that have adapted to arctic regions are often particularly vulnerable to fire, making the remorhaz's primary defense incredibly powerful and securing its role as a dangerous predator of frozen areas. Remorhazes live in extensive labyrinths carved into the heart of glaciers. These beasts use their heat to tunnel into the ice, tunnels whose smooth glass walls quickly refreeze along their wake creating numerous incredibly stable mazes.

Although the remorhaz has a lot in common with the smallest surface pests, this beast is surprisingly intelligent. Although unable to talk, the typical remorhaz understands the Giant well, and the tribes of giants often take advantage of this to form alliances with these beasts. The Frost Giants are particularly obsessed with them; these giants face the cruel and deadly burns that a remorhaz can inflict to become "friends of the worm" by obtaining a powerful weapon to use against their enemies, an assassin capable of digging through the floor of glacial fortifications to strike directly with the greatest weakness of a frost giant: fire. Other giants use these beasts as living forges, as their backs are hot enough to melt the metal.

A remorhaz is 7 meters long and weighs 5,000 kg.



\medskip\index[Mostruario]{Rustbug} \textbf{Rustbog}

\textit{Medium Monstrosity, misaligned}

\textbf{STRENGTH} +1

\textbf{DEXTERITY} +1

\textbf{CONSTITUTION} +1

\textbf{INTELLIGENCE} -4

\textbf{WISDOM} +1

\textbf{CHARISMA} -2

\textbf{Initiative} +1 - \textbf{Defense} 15

\textbf{Hit Points} 27 (5d8 + 5)

\textbf{Movement} 12 m

\textbf{Saving Throws}: Fortitude +2, Reflex +4, Will +5

\textbf{Senses} darkvision 18 m

\textbf{Languages} -

\textbf{Challenge} 1/2 (100 PX)

\textit{\textbf{Scent of Iron.}} The russet phage can identify, by smell, the exact position of ferrous metals within 30 feet.

\textit{\textbf{Rusting Metal.}} Any nonmagical weapon made of metal that hits the rust phage will corrode. Non-magical ammunition made of metal and affecting the rust phage is considered destroyed after dealing the damage.

\textbf{Actions}

\textit{\textbf{Bite.} Melee Weapon Attack}: +3 to hit, 1m range, one target.

\textit{Strikes:} 5 (1d8 + 1) piercing damage.

\textit{\textbf{Antennas.}} Rustbug corrodes nonmagical ferrous metal objects it can see and are within 1 meter. If the object is not worn or transported, contact with the rust phage destroys a cube with an edge of 30 centimeters. If the item is worn or carried by a creature, the creature can make a DC 11 Reflex saving throw to avoid contact with the rust phage.

If the item it contacts is a worn or carried metal armor or shield, they take a permanent and cumulative -2 penalty to the Defense they provide. Armor reduced to Defense 0 or shields that drop to a +0 bonus are destroyed. If the object it comes into contact with is a metal weapon held by someone, it will rust as described in the Rusting Metal trait.

\textbf{Ecology}
Environment any dungeon \\
Organization: Solitary, pair or nest (3-10) \\
\textbf{Treasure}: Accidental (no metal treasure) \\
\textbf{Description} \\
Of all the terrifying beasts an explorer can encounter underground, only the rusty beast has as its goal the one that the average adventurer values most: his treasure.

Typically 1 meter long and weighing at least 100 kg, the russet beast resembles a crustacean and would be scary enough even without the alien nutritional process from which it takes its name. Rustbugs eat metal objects, preferring those of iron and ferrous alloys such as steel but also devour mithral, adamantium and enchanted metals with equal ease. Any metal touched by the rust phage's delicate antennae or its armored skin corrodes and crumbles to dust in seconds, making it one of the most feared beasts by underground adventurers and mining Dwarves who must defend their forges and compete with them for the gold.

While rustbugs do not have an innate tendency for violence, their insatiable hunger prompts them to charge anything that comes near them with enough metal on, and any resistance is met with unexpected ferocity. It is not uncommon for rust bugs in metal-poor areas to follow fleeing victims for days using their metal sniffing ability, as long as they still have intact metal objects. \\
Fortunately, it is often possible to escape the attentions of a rust phage by throwing a dense metal object, such as a shield, at it and running in the opposite direction. Those who frequent the areas infested with rustbugs quickly learn to keep weapons of wood or stone close at hand.


\medskip\index[Mostruario]{Sahuagin} \textbf{Sahuagin}

\textit{Medium humanoid (sahuagin), lawful evil}

\textbf{STRENGTH} +1

\textbf{DEXTERITY} +0

\textbf{CONSTITUTION} +1

\textbf{INTELLIGENCE} +1

\textbf{WISDOM} +1

\textbf{CHARISMA} -1

\textbf{Initiative} +1 - \textbf{Defense} 13

\textbf{Hit Points} 22 (4d8 + 4)

\textbf{Movement} 9 m, swim 12 m

\textbf{Saving Throws}: Fortitude +4, Reflex +4, Will +4

\textbf{Skills} Awareness +5

\textbf{Senses} darkvision 36 m

\textbf{Languages} Sahuagin

\textbf{Challenge} 1/2 (100 PX)

\textit{\textbf{Limited Amphibian.}} The sahuagin can breathe air and water, but must remain submerged at least once every 4 hours to avoid suffocating.

\textit{\textbf{Bloody Frenzy.}} The sahuagin has + 1d6 to melee attack rolls against any creature not at its maximum hit points.

\textit{\textbf{Telepathy with Sharks}}. The sahuagin can magically command any shark within 36 meters of itself, using a limited form of telepathy.

\textbf{Actions}

\textit{\textbf{Multiattack.}} The sahuagin can make two melee attacks: one with its bite and one with its claws or spear.

\textit{\textbf{Claws.} Melee Weapon Attack}: +3 to hit, 1m range, one target.

\textit{Strikes:} 3 (1d4 + 1) slashing damage.

\textit{\textbf{Spear.} Melee or Ranged Weapon Attack}: +3 to hit, 1 m range and 6m range, one target.

\textit{Strikes:} 4 (1d6 + 1) piercing damage, or 5 (1d8 + 1) piercing damage if used with two hands to make a melee attack.

\textit{\textbf{Bite.} Melee Weapon Attack}: +3 to hit, 1m range, one target.

\textit{Strikes:} 3 (1d4 + 1) piercing damage.

\textbf{Ecology} \\
Environment: Temperate or Warm Oceans \\
Organization: Solitary, pair, squad (5-8), patrol (11-20 plus 1 3rd level lieutenant and 1-2 Sharks), gang (20-80 plus 100 \% non-combatants, 1 lieutenant of 3 1st level and 1 4th level captain every 20 adults, and 1-2 Sharks) or tribe (70-160 plus 100 \% non-combatants, 1 3rd level lieutenant every 20 adults, 1 4th level captain every 40 adults, 9 4th level guards, 1-4 3rd-6th level novices, 1 7th level priestess, 1 6th-8th level baron, and 5-8 Sharks)
\textbf{Treasure}: NPC gear (Trident, Heavy Crossbow with 10 Bolts, other treasure) \\
\textbf{Description} \\
Hungry and cruel, sahuagin are, unfortunately, among the most prosperous ocean races. Large cities have been built by this race in the dark depths of ocean trenches, and some fortresses rise near the coasts from where they launch continuous assaults on the air-breathing enemies who live near the shore. Proud and warlike, sahuagin seldom ally with others, and view other aquatic races, such as aboleth, seafaring, and the like as competitors. The only creatures that they seem to respect besides their own kind are sharks; in fact, in these relentless predators, the sahuagin see a lot of themselves. A sahuagin stands 2.1 meters tall and weighs around 125 kg.

Sahuagin are prone to genetic mutations, and when a mutant is born, they almost always rise to noble or command ranks in society. The most common sahuagin mutation is an extra pair of arms (granting two additional claw attacks or the ability to wield multiple weapons). Some speak of the rare malice, sahuagin who do not look like sharkmen but water elves, despite sharing the bloodlust and cruel nature of their fellows. The evil ones often serve as spies or assassins to the Sahuagin rulers, but there are stories of entire tribes made up of the evil ones in remote parts of the sea.


\medskip\index[Mostruario]{Salamandra} \textbf{Salamandra}

\textit{Large elemental, neutral evil}

\textbf{STRENGTH} +4

\textbf{DEXTERITY} +2

\textbf{CONSTITUTION} +2

\textbf{INTELLIGENCE} +0

\textbf{WISDOM} +0

\textbf{CHARISMA} +1

\textbf{Initiative} +2 - \textbf{Defense} 18

\textbf{Hit Points} 90 (12d10 + 24)

\textbf{Movement} 9 m

\textbf{Saving Throws}: Fortitude +10, Reflex +7, Will +6

\textbf{Damage Vulnerability} cold

\textbf{Damage Resistances} non-magical weapon

\textbf{Immunity to Damage} fire

\textbf{Senses} darkvision 18 m

\textbf{Languages} Ignan

\textbf{Challenge} 5 (1.800 PX)

\textit{\textbf{Heated Weapons.}} Any metallic melee weapon the salamander wields deals an additional 3 (1d6) Fire damage per hit (already included in the attack).

\textit{\textbf{Heated Body.}} A creature that contacts the salamander or hits it with a melee attack while within 1 meter of it takes 7 (2d6) fire damage.

\textbf{Actions}

\textit{\textbf{Multiattack.}} The salamander makes two attacks: one with its spear and one with its tail.

\textit{\textbf{Tail.} Melee Weapon Attack}: +10 to hit, 3m range, one target.

\textit{Strikes:} 11 (2d6 + 4) slam damage plus 7 (2d6) fire damage, and the target is grabbed (DC 14 to escape). Until the grapple is complete, the target is in the way, the salamander can automatically strike the target with its tail, and the salamander cannot make tail attacks against other targets.

\textit{\textbf{Spear.} Melee or Ranged Weapon Attack}: +9 hit, 1m range, 6m range, one target.

\textit{Strikes:} 11 (2d6 + 4) piercing damage, or 13 (2d8 +4) piercing damage when used with two hands to make a melee attack, plus 3 (1d6) fire damage.

\textbf{Ecology}
Environment any (Plane of Fire) \\
Organization: Solitary, pair or group (3-5) \\
\textbf{Treasure}: Standard (Spear, other non-flammable treasure) \\
\textbf{Description} \\
Salamanders are natives of the Plane of Fire, where their legions of fierce fighters are much feared by the rest of the Plane. Because many of the strongest Fire Elemental Races enslave Salamanders for their metallurgical prowess and fighting prowess, Salamanders hate Efreet and others with fervor.

Even if their hiding places exceed 250 degrees C in temperature, Salamanders can tolerate lower temperatures. They generally do so when forced, and are even more gruff and short-tempered than normal in these environments. Though hailing from the Plane of Fire, the Salamander Race identifies more with the Abyss, and has a great deal of respect for Demons (particularly those associated with fire, such as the Balor and certain fire-bound demon lords). It is therefore not unusual to encounter a large group of Salamanders in the Abyss.

Salamanders are often summoned to the Material Plane to serve as guardians or, more commonly, as makers of Armor, Weapons, and other metallurgical items, as their Skill in this field is legendary. Salamanders also infest those areas of the Material Plane where the boundary between this world and the Plane of Fire has become blurred, such as near and within Volcanoes.

Living in such extreme areas, Salamanders possess only treasures that resist high temperatures, such as Swords, Armor, Jewelery, Rods and other items that have a high melting point. Salamanders society is cruel and based on power and the ability to subjugate those inferior to them. Beings less than the Salamanders who cause trouble face a slow and painful death.



\medskip\index[Mostruario]{Satyr} \textbf{Satyr}

\textit{Fairy mean, chaotic neutral}

\textbf{STRENGTH} +1

\textbf{DEXTERITY} +3

\textbf{CONSTITUTION} +0

\textbf{INTELLIGENCE} +1

\textbf{WISDOM} +0

\textbf{CHARISMA} +2

\textbf{Initiative} +3 - \textbf{Defense} 15 (leather armor)

\textbf{Hit Points} 31 (7d8)

\textbf{Damage Vulnerability} cold iron

\textbf{Movement} 12 m

\textbf{Saving Throws}: Fortitude +4, Reflex +8, Will +8

\textbf{Skills} Move silently / Hide +5, Entertain +6, Awareness +2

\textbf{Languages} Common, Elvish, Sylvan

\textbf{Challenge} 1/2 (100 PX)

\textit{\textbf{Magic Resistance.}} The satyr has + 1d6 on saving throws against spells and other magical effects.

\textbf{Actions}

\textit{\textbf{Gored.} Melee Weapon Attack}: +3 to hit, 1m range, one target.

\textit{Strikes:} 6 (2d4 + 1) hit damage.

\textit{\textbf{Short Sword.} Melee Weapon Attack}: +5 to hit, 1m range, one target.

\textit{Strikes:} 6 (1d6 + 3) piercing damage.

\textit{\textbf{Short Bow.} Ranged Weapon Attack}: +5 to hit, range 24m, one target.

\textit{Strikes:} 6 (1d6 + 3) piercing damage.

\textbf{Ecology} \\
Environment: Temperate Forests \\
Organization: Solitaire, couple, gang (3-6) or party (7-11) \\
\textbf{Treasure}: Standard (Dagger, Short Bow plus 20 Arrows, perfect pan flute, other treasure) \\
\textbf{Description} \\
Satyrs, known in many regions as fauns, are debauched and hedonistic creatures of the deepest and most primeval parts of the forests. They adore wine, music and the pleasures of the flesh, are renowned as libertines and dudes who court unwary girls and shepherds and leave behind a trail of embarrassing explanations and unwanted pregnancies.

Although their bodies are almost always those of attractive and well-proportioned men, the seductive abilities of satyrs lie in their musical talent. With the help of his flute, a satyr is capable of weaving a wide range of melodic spells designed to charm others and induce them to indulge his whimsical wishes.

In addition to constantly flirting, satyrs often act as guardians of their forests, and those who manage to turn the faun's lust into anger will likely find themselves confronted with the most dangerous of the animals surrounding the faun. Furthermore, even though satyrs tend to put their enjoyment above the rights of others, they harbor no resentment against those they seduce.

The children born of these encounters are always pure blood satyrs and are generally taken away by their wild fathers soon after birth.


\medskip\index[Mostruario]{Skeleton} \textbf{Skeleton}

\textit{Media undead, lawful evil}

\textbf{STRENGTH} +0

\textbf{DEXTERITY} +2

\textbf{CONSTITUTION} +2

\textbf{INTELLIGENCE} -2

\textbf{WISDOM} -1

\textbf{CHARISMA} -3

\textbf{Initiative} +2 - \textbf{Defense} 14 (pieces of armor)

\textbf{Hit Points} 13 (2d8 + 4)

\textbf{Movement} 9 m

\textbf{Saving Throws}: Fortitude +0, Reflex +2, Will +2

\textbf{Damage Vulnerability} hit

\textbf{Damage Resistances} piercing and cutting

\textbf{Immunity to Damage} poison

\textbf{Condition Immunity} poisoned, fatigue

\textbf{Senses} darkvision 18 m

\textbf{Languages} understands all languages he spoke in life but cannot speak

\textbf{Challenge} 1/4 (50 PX)

\textit{\textbf{Undead Nature.}} The skeleton does not require air, food, drink or sleep.

\textbf{Actions}

\textit{\textbf{Short Sword.} Melee Weapon Attack}: +4 to hit, 1m range, one target.

\textit{Strikes:} 5 (1d6 + 2) piercing damage.

\textit{\textbf{Short Bow.} Ranged Weapon Attack}: +4 to hit, range 24m, one target.

\textit{Strikes:} 5 (1d6 + 2) piercing damage.

\textbf{Ecology} \\
Environment: Any \\
Organization: Any \\
\textbf{Treasure}: None (Broken Chain Jack, Broken Scimitar) \\
\textbf{Description} \\
Skeletons are animated dead bones, brought to life by sacrilegious magic. For the most part, skeletons are willless automatons, but they possess an evil cunning granted to them by the force that animates them: a cunning that allows them to carry weapons and wear armor.

\medskip\index[Mostruario]{Warhorse Skeleton} \textbf{Warhorse Skeleton}

\textit{Large undead, lawful evil}

\textbf{STRENGTH} +4

\textbf{DEXTERITY} +1

\textbf{CONSTITUTION} +2

\textbf{INTELLIGENCE} -4

\textbf{WISDOM} -1

\textbf{CHARISMA} -3

\textbf{Initiative} +1 - \textbf{Defense} 14 (pieces of harness)

\textbf{Hit Points} 22 (3d10 + 6)

\textbf{Movement} 18 m

\textbf{Saving Throws}: Fortitude +4, Reflex +3, Will +1

\textbf{Damage Vulnerability} hit

\textbf{Damage Resistances} piercing and cutting

\textbf{Immunity to Damage} poison

\textbf{Condition Immunity} poisoned, fatigue

\textbf{Senses} darkvision 18 m

\textbf{Languages} -

\textbf{Challenge} 1/2 (100 PX)

\textit{\textbf{Undead Nature.}} The skeleton does not require air, food, drink or sleep.

\textbf{Actions}

\textit{\textbf{Hooves.} Melee Weapon Attack}: +6 to hit, 1m range, one target.

\textit{Strikes:} 11 (2d6 + 4) hit damage.

\medskip\index[Mostruario]{Minotaur Skeleton} \textbf{Minotaur Skeleton}

\textit{Large undead, lawful evil}

\textbf{STRENGTH} +4

\textbf{DEXTERITY} +0

\textbf{CONSTITUTION} +2

\textbf{INTELLIGENCE} -2

\textbf{WISDOM} -1

\textbf{CHARISMA} -3

\textbf{Initiative} +0 - \textbf{Defense} 13

\textbf{Hit Points} 67 (9d10 + 18)

\textbf{Movement} 12 m

\textbf{Saving Throws}: Fortitude +6, Reflex +3, Will +2

\textbf{Damage Vulnerability} hit

\textbf{Immunity to Damage} poison

\textbf{Damage Resistances} piercing and cutting

\textbf{Condition Immunity} poisoned, fatigue

\textbf{Senses} darkvision 18 m

\textbf{Languages} understands the Abyssal but cannot speak

\textbf{Challenge} 2 (450 PX)

\textit{\textbf{Charge.}} If the minotaur skeleton moves at least 10 feet in a straight line to the target and then hits it with a gore attack during the same turn, the target suffers 9 (2d8) additional piercing damage. If the target is a creature, it must succeed on a DC 14 Fortitude save or be pushed 10 feet back and fall prone.

\textit{\textbf{Undead Nature.}} The skeleton does not require air, food, drink or sleep.

\textbf{Actions}

\textit{\textbf{Double-headed Ax.} Melee Weapon Attack}: +6 to hit, 1m range, one target.

\textit{Strikes:} 17 (2d12 + 4) slashing damage.

\textit{\textbf{Gored.} Melee Weapon Attack}: +6 to hit, 1m range, one target.

\textit{Strikes:} 13 (2d8 + 4) piercing damage.

\medskip\index[Mostruario]{Hellhound} \textbf{Hellhound}

\textit{Media fiend, lawful evil}

\textbf{STRENGTH} +3

\textbf{DEXTERITY} +1

\textbf{CONSTITUTION} +2

\textbf{INTELLIGENCE} -2

\textbf{WISDOM} +1

\textbf{CHARISMA} -2

\textbf{Initiative} +1 - \textbf{Defense} 17

\textbf{Hit Points} 45 (7d8 + 14)

\textbf{Movement} 15 m

\textbf{Saving Throws}: Fortitude +6, Reflex +5, Will +1

\textbf{Skills} Awareness +5

\textbf{Immunity to Damage} fire

\textbf{Senses} darkvision 18 m

\textbf{Languages} understands Hell but cannot speak

\textbf{Challenge} 3 (700 PX)

\textit{\textbf{Hearing and refined smell.}} The hound has + 1d6 on hearing or smell based Wisdom (Awareness) checks.

\textit{\textbf{Pack Tactics.}} The hound has + 1d6 to hit rolls against a creature if at least one of the hound's allies is within 1 meter of the creature and that ally is not incapacitated.

\textbf{Actions}

\textit{\textbf{Bite.} Melee Weapon Attack}: +7 to hit, 1m range, one target.

\textit{Strikes:} 7 (1d6 + 3) piercing damage plus 7 (2d6) fire damage.

\textit{\textbf{Fiery Breath (Cooldown 5-6).}} The hound exhales fire into a 5 meter cone. Each creature in that area must make a DC 12 Reflex saving throw, and take 21 (6d6) fire damage on a failed save, or half that damage on a successful one.



\subsection{Sphinxes}

\medskip\index[Mostruario]{Androsphinx} \textbf{Androsphinx}

\textit{Large monstrosity, lawful neutral}

\textbf{STRENGTH} +6

\textbf{DEXTERITY} +0

\textbf{CONSTITUTION} +5

\textbf{INTELLIGENCE} +3

\textbf{WISDOM} +4

\textbf{CHARISMA} +6

\textbf{Initiative} +3 - \textbf{Defense} 26

\textbf{Hit Points} 199 (19d10 + 95)

\textbf{Movement} 12 m, flight 18 m

\textbf{Saving Throws}: Fortitude +12, Reflex +8, Will +7

\textbf{Skills} Arcane +9, Awareness +10, Religion +15

\textbf{Immunity to Damage} non-magical weapon

\textbf{Condition Immunity} fascinated, scared

\textbf{Senses} vision of the true 36 m

\textbf{Languages} Common, Sphinx

\textbf{Challenge} 17 (18000 PX)

\textit{\textbf{Magical Weapons.}} The sphinx's weapon attacks are magical.

\textit{\textbf{Inscrutable.}} The sphinx is immune to any effects that can sense its emotions or read its thoughts, as well as any divination spells you reject. Wisdom (Perceiving Emotions) checks to discern the sphinx's intentions or sincerity have -1d6.

\textit{\textbf{Spells.}} The sphinx has CM 12.
Her spellcasting characteristic is Wisdom (spell saving throw DC 18, +10 to hit with spell attacks). It does not need material components to cast its spells. The sphinx holds the following spells prepared:

Tricks (at will): \textit{sacred flame, save the dying,} \textit{thaumaturgy}

level 1 (4 slots): \textit{command, detect magic,} \textit{detect evil and good}

level 2 (3 slots): \textit{lower refresh, zone of truth}

level 3 (3 slots): \textit{dispel spells, languages}

level 4 (3 slots): \textit{exile, freedom of movement}

level 5 (2 slots): \textit{fire strike, restore superior}

level 6 (1 slot): \textit{banquet of heroes}

\textbf{Actions}

\textit{\textbf{Multiattack.}} The sphinx can make two claw attacks.

\textit{\textbf{Claw.} Melee Weapon Attack}: +17 to hit, 1 m range, one target.

\textit{Strikes:} 17 (2d6 + 10) slashing damage, 1 bleed damage.

\textit{\textbf{Roar (3 / Day).}} The sphinx emits a magical roar. Whenever it roars before a new dawn, the loudest roar and effect is different, as detailed below. Any creature within 150 meters of the sphinx and capable of hearing its roar must make a saving throw.

\textbf{First Roar.} Any creature that fails a DC 18 Will save is frightened for 1 minute. A frightened creature can re-roll the saving throw at the end of each of its rounds, ending the effect for itself if successful.

\textbf{Second Roar.} Any creature that fails a DC 18 Will save is deafened and frightened for 1 minute. A frightened creature is paralyzed and can re-roll the saving throw at the end of each of its rounds, ending the effect for itself if successful.

\textbf{Third Roar.} Each creature makes a DC 18 Fortitude saving throw. Anyone who fails the saving throw takes 44 (8d10) sonic damage and is knocked prone. If the saving throw is successful, the creature takes half this damage and is not thrown prone.

\textbf{Additional Actions}

The sphinx can perform 3 additional Actions, chosen from the following options. He can only use one legendary option at a time, and only at the end of another creature's turn. The sphinx recovers any additional Actions spent at the start of its round.

\textbf{Claw Attack.} The Sphinx makes a claw attack.

\textbf{Casting a Spell (Costs 3 Actions).} The sphinx casts a spell from the list of prepared spells, using one spell slot as normal.

\textbf{Teleport (Costs 2 Actions).} The sphinx magically teleports, along with all equipment it is wearing or carrying, to an unoccupied space it can see, up to 36 meters away.


\textbf{Ecology} \\
Environment: Hills or Warm Deserts \\
Organization: Solitary \\
\textbf{Treasure}: Standard \\
\textbf{Description} \\
Androsphinxes, the most powerful of the common sphinxes, believe they represent all that is worthy and noble in their species and pose as if the weight of the whole world rests on their good example. They look upon the Cryosphinxes with paternalistic disdain, the Hieracosphinxes with undisguised disgust and the Gynosphinxes as the only other sphinxes worthy of their time.

The androsphinxes flaunt a grumpy and resentful facade towards foreigners. They make no effort to hide their annoyance when they are upset. They also tend to be jealous of their territory, although less so than other sphinxes. They almost inevitably issue thunderous warnings and proclamations before attacking, and they almost always respect a plea to negotiate. Androsphinxes trade information and conversation, not treasures, in exchange for safe passage.

Androsphinxes are 3.6 meters tall and weigh 500 kg.


\medskip\index[Mostruario]{Ginosfinge} \textbf{Ginosfinge}

\textit{Large monstrosity, lawful neutral}

\textbf{STRENGTH} +4

\textbf{DEXTERITY} +2

\textbf{CONSTITUTION} +3

\textbf{INTELLIGENCE} +4

\textbf{WISDOM} +4

\textbf{CHARISMA} +4

\textbf{Initiative} +4 - \textbf{Defense} 23

\textbf{Hit Points} 136 (16d10 + 48)

\textbf{Movement} 12 m, flight 18 m

\textbf{Saving Throws}: Fortitude +11, Reflex +9, Will +10

\textbf{Skills} Arcane +14, Awareness +9, Religion +9, History +14

\textbf{Damage Resistances} from a non-magical weapon

\textbf{Condition Immunity} fascinated, frightened

\textbf{Senses} vision of the true 36 m

\textbf{Languages} Common, Sphinx

\textbf{Challenge} 11 (7,200 PX)

\textit{\textbf{Magical Weapons.}} The sphinx's weapon attacks are magical.

\textit{\textbf{Inscrutable.}} The sphinx is immune to any effects that can sense its emotions or read its thoughts, as well as any divination spells you reject. Wisdom (Perceiving Deceptions) checks to discern the sphinx's intentions or sincerity have -1d6.

\textit{\textbf{Spells.}} The sphinx has CM of 9. His spellcasting ability is Intelligence (spell saving throw DC 17, +9 to hit with spell attacks). It does not need material components to cast its spells. The sphinx holds the following spells prepared: Tricks (at will): \textit{minor illusion, magic hand,} \textit{prestidigitation}

level 1 (4 slots): \textit{identify, detect magic, shield}

level 2 (3 slots): \textit{locate object, darkness, suggestion}

level 3 (3 slots): \textit{dispel spells, languages, remove curse}

level 4 (3 slots): \textit{exile, greater invisibility}

level 5 (2 slots): \textit{knowledge of legends}

\textbf{Actions}

\textit{\textbf{Multiattack.}} The sphinx can make two claw attacks.

\textit{\textbf{Claw.} Melee Weapon Attack}: +11 to hit, 1 m range, one target.

\textit{Strikes:} 13 (2d8 + 4) slashing damage, 1 bleed damage.

\textbf{Additional Actions}

The sphinx can perform 3 additional Actions, chosen from the following options. He can only use one legendary option at a time, and only at the end of another creature's turn. The sphinx recovers any additional Actions spent at the start of its round.

\textbf{Claw Attack.} The Sphinx makes a claw attack.

\textbf{Casting a Spell (Costs 3 Actions).} The sphinx casts one spell from the list of prepared spells, using one spell slot as normal.

\textbf{Teleport (Costs 2 Actions).} The sphinx magically teleports, along with all equipment it is wearing or carrying, to an unoccupied space it can see, up to 36 meters away.

\textbf{Ecology}
Environment: Deserts and hot hills \\
Organization: Solitary, couple or cult (3-6) \\
\textbf{Treasure}: Double \\
\textbf{Description} \\
While there are different types of sphinxes, the one that scholars refer to as Gynosphinx (a name that many sphinxes find offensive) is a wise and majestic creature but at the same time terrifying when angry. Less moralistic than their male counterparts (the Androsphinxes, creatures totally different from the one presented here), the sphinxes are prudent and methodical when making decisions, and are proud of their cold logic and their impartiality. They have little patience with the lower variants of sphinxes, considering them little more than animals. Sphinxes love puzzles and tricky riddles, and treasure unusual facts and arcane dilemmas far more than gold or gems.

While not great scholars in the traditional sense, the sphinxes' great appreciation for puzzles leads them to research a wide variety of subjects, often making them a valuable source of information, especially when they make use of their magical abilities. They are usually happy to have contact with other races, and regularly offer material goods in exchange for new and interesting information or riddles. They are excellent guardians of temples, tombs and other important places, as long as they are properly entertained. Sphinxes attach great importance to kindness, but they can be temperamental: they may selflessly decide to share their latest puzzles with travelers but don't think twice about devouring them if they don't pay enough attention or provide any clues to their resolution.

A typical sphinx is 3 meters long and weighs around 400 kg. Although their wings can hold them in the air for long periods of time, they are poor fliers, and prefer to land before they start fighting, attacking with their powerful claws. Despite being extremely territorial, sphinxes tend to warn intruders several times before attacking.

\medskip\index[Mostruario]{Hissing} \textbf{Hissing}

\textit{Large monstrosity, chaotic}

\textbf{STRENGTH} +2

\textbf{DEXTERITY} +1

\textbf{CONSTITUTION} +1

\textbf{INTELLIGENCE} -3

\textbf{WISDOM} +0

\textbf{CHARISMA} -2

\textbf{Initiative} +1 - \textbf{Defense} 14

\textbf{Hit Points} 32 (5d10 + 5)

\textbf{Movement} 6m, climb 6m

\textbf{Saving Throws}: Fortitude +3, Reflex +3, Will +2

\textbf{Skills} Move Silently +4, Awareness +3

\textbf{Senses} darkvision 18 m

\textbf{Fine Senses}: Hissing has + 1d6 on Awareness checks based on hearing or smell

\textbf{Languages}: -

\textbf{Challenge} 2 (450 PX)

\textbf{Actions}

\textit{\textbf{Multi Attack.}} The Hisser can perform two claw attacks or one tail strike.

\textit{\textbf{Claw.} Melee Weapon Attack}: +4 to hit, 1m range, one target.

\textit{Strikes:} 6 (1d8 + 2) slashing damage.

\textit{\textbf{Tail Whip}}: The Hissing wags its long tail and hits a target.

\textit{Strikes:} 11 (2d8 + 2) slash damage and 7 (2d6) slash damage, range 4 meters. In case of critical, any armor or shield is damaged by lowering the opponent's Defense by 1. Damage to armor is not considered permanent.

\textbf{Reactions}

\textit{\textbf{Landing}}: When the Hisser is attacked by a creature within range of his tail, it is lashed, forcing the attacker, after the resolution of his attack, to make a Fortitude / Reflex saving throw at DC 12 or take 7 (2d6) hit damage and fall prone. If the saving throw is successful, I take only half damage and is not prone.

\textbf{Ecology} \\
Environment: Caves \\
Organization: Solitary, pair or nest (2-4) \\
\textbf{Treasure}: random \\

\textbf{Description}

The Hissers, so called because of the noise that their tail makes when waving, is a very particular creature. At first sight it resembles a crocodile, about 5 meters long, 4 of which tail but has 8 legs and a short and flattened muzzle. The extremely sturdy tail ends with a kind of hook that the Hisser uses to hit, kill and grab enemies as if it were an additional leg.

Dark gray, brown in color, they prefer to hide in the dark and attack when hungry or to defend their territory. They try to keep their distance in combat and if seriously injured they escape by climbing the walls.


\medskip\index[Mostruario]{Sprite} \textbf{Sprite}

\textit{Tiny fairy, neutral good}

\textbf{STRENGTH} -4

\textbf{DEXTERITY} +4

\textbf{CONSTITUTION} +0

\textbf{INTELLIGENCE} +2

\textbf{WISDOM} +1

\textbf{CHARISMA} +0

\textbf{Initiative} +4 - \textbf{Defense} 16 (leather armor)

\textbf{Hit Points} 2 (1d4)

\textbf{Damage Vulnerability} cold iron

\textbf{Movement} 3m, flight 12m

\textbf{Saving Throws}: Fortitude +0, Reflex +5, Will +2

\textbf{Skills} Move silently / Hide +8 (check is -1d6 if the sprite is flying), Awareness +3

\textbf{Languages} Common, Elven, Sylvan

\textbf{Challenge} 1/4 (50 PX)

\textbf{Actions}

\textit{\textbf{Long Sword.} Melee Weapon Attack}: +2 to hit,

1 m range, one target.

\textit{Strikes:} 1 slashing damage.

\textit{\textbf{Short Bow.} Ranged weapon attack}: +6 to hit, range 12 yards, one target.

\textit{Strikes:} 1 piercing damage. If the target is a creature, it must succeed on a DC 10 Fortitude saving throw or be poisoned for 1 minute. If the result of this saving throw is 5 or less, the target falls unconscious for the same duration, or until it takes damage or another creature uses an action to awaken it.

\textit{\textbf{Invisibility.}} The sprite remains invisible until it attacks or ends its concentration. Anything the sprite is carrying or wearing remains invisible as long as it is in contact with the sprite.

\textit{\textbf{Sight of the Heart.}} The sprite makes contact with a creature and learns its current emotional state. If the target fails a DC 10 Fortitude saving throw, the sprite also learns the creature's traits. Celestial, fiendish, and undead automatically fail this saving throw.

\textbf{Description} \\
Sprites gather in groups in the depths of wooded regions, united in cause to protect nature. Whole tribes of sprites have declared themselves protectors of a particular person, place or creature of particular importance in their lands, even if the being does not want or need any protection.

A sprite's body is naturally luminous, although the creature can vary the color and intensity of the light emitted from its body at will. Immediately after his death, a sprite's body dissolves into a shimmering mist. Sprites are the smallest of the goblins, just over 22 centimeters tall and weighing rarely more than 1 kg.

In many ways sprites are more primitive than most sprites. They appreciate the company of their own kind, but tend to be wary of other goblins and assume that any humanoid or creature they have not expressly chosen to protect is going to harm them. Even animals are usually considered dangerous by them. The reason for this distrust is largely due to the tiny size of these creatures, which makes them easy prey for predators. Therefore, a sprite's initial reaction to danger is to run away: it typically uses its magical abilities to slow or distract pursuers, and later relies on its flying speed and size to be able to escape.

Although sprites themselves have an uneducated and wild nature, they have a healthy curiosity for all things with innate magic. They are particularly drawn to places of great latent magical power, such as the ruins of ancient temples. This curiosity also makes them unusually suited to the role of familiars. A 5th-level neutral chaotic spellcaster can gain a sprite as a familiar if he has the Familiar skill.


\medskip\index[Mostruario]{Striga (Stygian Bird)} \textbf{Striga (Stygian Bird)}

\textit{Tiny beast, misaligned}

\textbf{STRENGTH} -3

\textbf{DEXTERITY} +3

\textbf{CONSTITUTION} +0

\textbf{INTELLIGENCE} -4

\textbf{WISDOM} -1

\textbf{CHARISMA} -2

\textbf{Initiative} +3 - \textbf{Defense} 15

\textbf{Hit Points} 2 (1d4)

\textbf{Movement} 3m, flight 12m

\textbf{Saving Throws}: Fortitude +2, Reflex +6, Will +1

\textbf{Senses} darkvision 18 m

\textbf{Languages} -

\textbf{Challenge} 1/8 (25 PX)

\textbf{Actions}

\textit{\textbf{Blood Drain.} Melee Weapon Attack}: +5 to hit, 1 yd range, one creature.

\textit{Strikes:} 5 (1d4 + 3) piercing damage and the striga attaches itself to the target. While attacked, the striga does not attack. Instead, at the start of each striga's turn, the target loses 5 (1d4 + 3) hit points due to blood loss.

The striga can detach by spending 1 meter of movement. It does this automatically after draining 10 hit points from the target or upon death of the target. A creature, including the target, can use its action to detach the striga.

\textbf{Ecology}
Environment: Temperate and warm swamps \\
Organization: Solitary, colony (2-4), flock (5-8), cloud (9-14) or swarm (15-40) \\
\textbf{Treasure}: None \\
\textbf{Description} \\
Striga are dangerous bloodsuckers that infest swamps and prey on wild animals, livestock and unsuspecting travelers. While individually weak, swarms of these creatures are capable of draining a man in minutes, leaving only a desiccated corpse behind.

More like mammals than insects, striga soar with their four wings of flesh, seeking warm-blooded prey. They often hide near pools of drinkable water waiting for travelers to let their guard down and then attack them and drink their fill, sticking their trunks into their exposed veins. After feeding, they fly away to hide in the mud and reeds to lay their eggs and rest until hunger drives them to hunt again.

Striga are usually around 30 centimeters long, with a wingspan of around twice as much, and weigh less than 0.5 kg. They are rusty red or reddish brown, and have a dirty yellow belly, but those that have not fed properly are pale pink.

\medskip\index[Mostruario]{Succubus} \textbf{Succubus}

\textit{Medium fiend (shapeshifter), neutral evil}

\textbf{STRENGTH} -1

\textbf{DEXTERITY} +3

\textbf{CONSTITUTION} +1

\textbf{INTELLIGENCE} +2

\textbf{WISDOM} +1

\textbf{CHARISMA} +5

\textbf{Initiative} +3 - \textbf{Defense} 17

\textbf{Hit Points} 66 (12d8 + 12)

\textbf{Movement} 9 m, flight 18 m

\textbf{Saving Throws}: Fortitude +7, Reflex +9, Will +10

\textbf{Skills} Stealth 5, Perceive Emotions +5, Awareness +5, Deceive +9

\textbf{Resistance to Damage} cold, lightning, fire, poison; non-magical weapon

\textbf{Senses} darkvision 18 m

\textbf{Languages} Abyssal, Common, Hellish, telepathy 18 m

\textbf{Challenge} 4 (1.100 PX)

\textit{\textbf{Telepathic Bond.}} The fiend ignores the range restrictions of his telepathy when communicating with a creature he has charmed. The two are not even forced to be on the same plane of existence.

\textit{\textbf{Shapeshifter.}} The fiend can use its action to transform into a Small or Medium humanoid, or to return to its true form. Without wings, the fiend loses flight speed. Aside from size and speed, his stats are the same in all forms. Any equipment he is wearing or carrying is not transformed. At death it returns to its true form.

\textbf{Actions}

\textit{\textbf{Claw (Fiendish Form only).} Melee Weapon Attack}: +6 to hit, 1m range, one target.

\textit{Strikes:} 6 (1d6 + 3) slashing damage.

\textit{\textbf{Charm.}} A humanoid visible to the fiend within 30 feet of it must succeed on a DC 15 Will saving throw or be magically fascinated for 1 day. The fascinated target obeys the foul's verbal or telepathic commands. If the target takes damage or receives a suicide command, it can re-roll the saving throw, ending the effect if successful. If the target succeeds at the saving throw against the effect, or if the effect ends, the target is immune to the fiend's fascination for the next 24 hours.

The fiend can only have one target fascinated at a time. If another fascinates, the effect on the previous target ends.

\textit{\textbf{Sucking Kiss.}} The fiend kisses a charmed creature or a willing creature. The target must make a DC 14 Fortitude saving throw against this spell, taking 32 (5d10 + 5) damage if it fails, or half that damage if it succeeds. The target's maximum hit points are reduced by an amount equal to the damage taken. This reduction lasts until dawn breaks. The target dies if this effect reduces its maximum hit points to 0.

\textit{\textbf{Ethereal Form.}} The fiend magically enters the Ethereal Plane from the Material Plane, and vice versa.

\textbf{Ecology} \\
Environment Any (Abyss) \\
Organization: Solitary, pair or harem (3-12) \\
\textbf{Treasure}: double \\
\textbf{Description} \\
Among the demonic hordes a succubus can often reach very high levels of power, using its manipulations and its sensual charm, and many demonic wars rage due to the devious machinations of these creatures. A succubus originates from the souls of particularly lustful and greedy wicked mortals.


\medskip\index[Mostruario]{Tarrasque} \textbf{Tarrasque}

\textit{Colossal monstrosity (titan), misaligned}

\textbf{STRENGTH} +10

\textbf{DEXTERITY} +0

\textbf{CONSTITUTION} +10

\textbf{INTELLIGENCE} -2

\textbf{WISDOM} +0

\textbf{CHARISMA} +0

\textbf{Initiative} +0 - \textbf{Defense} 35

\textbf{Hit Points} 676 (33d20 + 330)

\textbf{Movement} 12 m

\textbf{Saving Throws}: Fortitude +31, Reflex +22, Will +12

\textbf{Immunity to Damage} fire, poison; weapons +2

\textbf{Condition Immunity} fascinated, poisoned, paralyzed, frightened

\textbf{Senses} blind sight 36 m

\textbf{Languages} -

\textbf{Challenge} 30 (155000 PX)

\textit{\textbf{Reflective Carapace.}} Whenever the Tarrasque is the target of a \textit{magic missile} spell, line spell, or spell that requires a range attack roll, roll a d6. From 1 to 5, the Tarrasque ignores it. On a 6, the Tarrasque ignores it, and the effect is reflected back against the caster as if it originated from the Tarrasque, turning the caster into the target.

\textit{\textbf{Siege Monster.}} The Tarrasque deals double damage to objects and structures.

\textit{\textbf{Legendary Resistance (3 / Day).}} If the Tarrasque fails a saving throw, it may choose to succeed instead.

\textit{\textbf{Resistance to Magic.}} The Tarrasque has + 1d6 on saving throws against spells or other magical effects.

\textbf{Actions}

\textit{\textbf{Multiattack.}} The Tarrasque can use its Frightening Presence. Then he makes five attacks: one with the bite, two with the claws, one with the horns, and one with the tail. Instead of the bite he can use Swallow.

\textit{\textbf{Claw.} Melee Weapon Attack}: +30 to hit, 5 meters range, one target.

\textit{Strikes:} 28 (4d8 + 10) slashing damage, 3 bleed damage.

\textit{\textbf{Tail.} Melee Weapon Attack}: +30 to hit, 6m range, one target.

\textit{Strikes:} 24 (4d6 + 10) hit damage. If the target is a creature, it must succeed on a DC 20 Fortitude saving throw or fall prone.

\textit{\textbf{Horns.} Melee Weapon Attack}: +30 to hit, 3m range, one target.

\textit{Strikes:} 32 (4d10 + 10) piercing damage.

\textit{\textbf{Bite.} Melee Weapon Attack}: +30 to hit, 3m range, one target.

\textit{Strikes:} 36 (4d12 + 10) piercing damage. If the target is a creature, it is grabbed (DC 20 to escape). Until the grapple is complete, the target is in the way, and the Tarrasque cannot use the bite against another target.

\textit{\textbf{Swallow.}} The Tarrasque makes a bite attack against a Large or smaller target it is grabbing. If the attack hits, the target is engulfed, and the grab ends. The engulfed target is blinded and entangled, has full cover against attacks and other effects outside the Tarrasque, and takes 56 (16d6) acid damage at the start of each Tarrasque turn.

If the Tarrasque takes 60 or more damage in a single turn from a creature within it, the Tarrasque must succeed on a DC 30 Fortitude save at the end of that turn or vomit all engulfed creatures, which fall prone in a space within 3. meters from Tarrasque. If the Tarrasque dies, a swallowed creature is no longer hampered by it and can exit the corpse using 30 feet of movement, coming out prone.

\textit{\textbf{Dreadful Presence.}} Any creature chosen by the Tarrasque, within 36 meters of it and aware of its presence, must succeed on a DC 17 Will saving throw or be startled for 1 minute. A creature can re-roll the saving throw at the end of each of its rounds, at -1d6 if the Tarrasque is in line of sight, ending the effect for itself if it succeeds. If the creature's saving throw is successful or the effect ends, the creature is immune to the Tarrasque's dreadful presence for the next 24 hours.

\textbf{Additional Actions}

The Tarrasque can perform 3 additional Actions, chosen from the following options. He can only use one legendary option at a time, and only at the end of another creature's turn. The tarrasque recovers any additional actions spent at the start of their round.

\textbf{Attack.} The Tarrasque makes a claw or tail attack. \textbf{Chew (Costs 2 Actions).} The Tarrasque makes a bite attack or uses Swallow.

\textbf{Move.} The Tarrasque moves up to halfway through its movement.

\textbf{Ecology} \\
Environment: Any \\
Organization: Solitary \\
\textbf{Treasure}: None \\
\textbf{Description} \\
The legendary Tarrasque is among the most destructive monsters in the world. Fortunately, he spends most of his time in some kind of deep hibernation in an unknown cave in a remote corner of the world. When he awakens, however, entire kingdoms die.

While not particularly intelligent, the Tarrasque is intelligent enough to understand some words in the language of the Depths (even though they cannot speak). Likewise, the fury is not uncontrolled - it focuses on the creature that harmed it the most, and it's hard to distract it with deception.

\medskip\index[Mostruario]{Flaming Skull} \textbf{Flaming Skull}

\textit{Small undead, Evil traits}

\textbf{STRENGTH} +0

\textbf{DEXTERITY} +1

\textbf{CONSTITUTION} +1

\textbf{INTELLIGENCE} +1

\textbf{WISDOM} +0

\textbf{CHARISMA} +0

\textbf{Initiative} +1 - \textbf{Defense} 13

\textbf{Hit Points} 7 (1d8 + 3)

\textbf{Movement} flight 10 m

\textbf{Saving Throws}: Fortitude +1, Reflex +2, Will +1

\textbf{Damage Resistances} from Void

\textbf{Immunity to Damage} fire, poison, non-magical weapon

\textbf{Condition Immunity} fascinated, poisoned, paralyzed, fatigue, frightened

\textbf{Senses} darkvision 18 m

\textbf{Challenge} 2 (200 PX)

\textit{\textbf{Spells.}} A Flaming Skull can cast the following spells innately.

at Will: \textit{Produce Flame}

1x per day: \textit{Kyrin's Flaming Acorn Hurdle}

\textit{\textbf{Undead Nature.}} The Flaming Skull does not need air, food, drink or sleep.

\textbf{Ecology} \\
Environment: Any \\
Organization: Solitary, pair, patrol (2d4) \\
\textbf{Treasure}: none \\

\textbf{Description}

Flaming Skulls are created from the corpses of spellcasters specializing in the Fire Magic List, via a variant of the Raise Dead spell.

Used as keepers and torches they often represent a first line of defense in dungeons.

\medskip\index[Mostruario]{Dragon Turtle} \textbf{Dragon Turtle}

\textit{Gargantuan dragon, neutral}

\textbf{STRENGTH} +7

\textbf{DEXTERITY} +0

\textbf{CONSTITUTION} +5

\textbf{INTELLIGENCE} +0

\textbf{WISDOM} +1

\textbf{CHARISMA} +1

\textbf{Initiative} +0 - \textbf{Defense} 29

\textbf{Hit Points} 341 (22d20 + 110)

\textbf{Movement} 6m, swim 12m

\textbf{Saving Throws} Fortitude +12, Reflex +8, Will +9

\textbf{Senses} darkvision 18 m

\textbf{Languages} Aquan, Draconic

\textbf{Challenge} 17 (18000 PX)

\textit{\textbf{Amphibian.}} The dragon tortoise can breathe air and water.

\textbf{Actions}

\textit{\textbf{Multiattack.}} The dragon can make three attacks: one with its bite and two with its claws. It can make one tail attack instead of two claw attacks.

\textit{\textbf{Claw.} Melee Weapon Attack}: +26 to hit, 3m range, one target.

\textit{Strikes:} 16 (2d8 + 7) slashing damage.

\textit{\textbf{Tail.} Melee Weapon Attack}: +26 to hit, range 5 meters, one target.

\textit{Hits:} 26 (3d12 + 7) hit damage. If the target is a creature, it must succeed on a DC 20 Fortitude save or be pushed 10 feet away from the dragon turtle and fall prone.

\textit{\textbf{Bite.} Melee Weapon Attack}: +26 to hit, range 5 meters, one target.

\textit{Strikes:} 26 (3d12 + 7) piercing damage.

\textit{\textbf{Breath of Steam (Refill 5-6).}} The dragon tortoise exhales hot steam in a 60-foot cone. Each creature in that area must make a DC 18 Fortitude saving throw and take 52 (15d6) fire damage if it fails the saving throw, or half that damage if it succeeds. Being underwater does not provide resistance against this type of damage.

\textbf{Ecology}
Environment: Temperate aquatic \\
Organization: Solitary \\
\textbf{Treasure}: Double \\
\textbf{Description} \\
Dragon tortoises inhabit fresh and salt waters, where they stand among the greatest dangers to sailors and those traveling by ship across the world's sea routes. Experienced sailors know what the area's dragon tortoises want and frequently make offers of gold and magic to ensure safe passage or avoid the area entirely. For its part, a dragon tortoise appreciates and even expects such tolls and gifts, and a dragon tortoise that expects gifts but is ignored is indeed a dangerous enemy.

The color of a dragon turtle's shell varies from individual to individual. Some have dull brown and rusty red shells, while others have deep blue-green carapaces with silvery reflections on the rocky tips. The color of the head, tail and legs is slightly paler than the shell and includes golden streaks along the crest and spines.

Dragon tortoises claim huge offshore territories, which include regions that often exceed 75 square km. Here, these dangerous beasts overturn ships that do not respect their territories, adding sunken wrecks and their precious cargoes to their hiding places. Dragon tortoises generally make their lairs in deep caverns accessible only through water, and often decorate them not only with the riches stolen from ships they have sunk, but also with the wrecks of these hapless boats. Their territorial nature and their predilection for this type of burrow put them in direct conflict with other underwater races such as Marinids and Sahuagin.

Large fish, such as tuna, sturgeon and even sharks are among the favorite foods of dragon tortoises, but being omnivorous, they sometimes also feed on large underwater fields of seaweed. They certainly do not disdain to supplement their diet with the passengers of sinking ships, even if this practice is not due to either wickedness or cruelty. Dragon tortoises have shells that are 5 meters in diameter, with limbs extending a few meters beyond, and measure 7 meters from the tip of their nose to the end of their mighty tail.

\medskip \textbf{Topi, La} \\ \index[Mostruario]{Topi, La}
\textit{Tiny fairy} \\
\textbf{Strength}: -1 \\
\textbf{Dexterity}: +4 \\
\textbf{Constitution}: +0 \\
\textbf{Intelligence}: +6 \\
\textbf{Wisdom}: +2 \\
\textbf{Charisma}: +6 \\
\textbf{Defense}: 17 - \textbf{Initiative}: +15 \\
\textbf{Hit Points}: 4 (1d10 - 1) \\
\textbf{Movement}: 6 m \\
\textbf{Saving Throws}: Fortitude +20, Reflex +30, Will +20 \\
\textbf{Senses}: Telluric sense 30, Darkvision 30 m, True vision 30 m \\
\textbf{Languages}: all \\
\textbf{Challenge} 0 (10 PX) \\
\textbf{Immunity}: to damage from weapons with magic bonus less than +6 \\
\textbf{Immunity}: whatever effect does not please Topi \\
\textbf{Immunity}: to any magic the Topi does not want to be affected \\
\textit{\textbf{It's The Topi}} The Topi has + 3d6 (or +18) every time she has to roll dice or count a value.
Any attack made by the Topi is considered magical +6 and is not resistable. \\
\smallskip \textbf{Actions} \\
\textit{\textbf{Little Face}} each creature of Topi's choice, within 30 meters, suffers a Little Face. The creature is knocked 2d6 meters away and takes 3d6 damage \\
\textit{\textbf{Rat Bite} Melee Weapon Attack}: +26 to hit, 1m range, one target. \\
\textit{Strikes:} 6 piercing damage. \\
\textit{\textbf{Scratch} up to 8 Melee Weapon Attacks}: Automatically hits, range 1m, up to 4 targets. \\
\textit{Strikes:} 1 piercing damage. \\
\textbf{Ecology} \\
Environment: Everywhere \\
Organization: Solitary \\
\textbf{Treasure}: Special \\
\textbf{Description} \\
She might be mistaken for a little white mouse, but La Topi is so much more. Smart, intelligent, beautiful, she loves going to the market and buying handbags.


\medskip \textbf{Dark Torch} \\ \index[Mostruario]{Dark Torch}
\textit{Mean, undead, evil} \\
\textbf{Strength}: +3 \\
\textbf{Dexterity}: +1 \\
\textbf{Constitution}: +2 \\
\textbf{Intelligence}: +0 \\
\textbf{Wisdom}: -1 \\
\textbf{Charisma}: -2 \\
\textbf{Defense}: 17 - \textbf{Initiative}: +2 \\
\textbf{Hit Points}: 75 (12d10 +20) \\
\textbf{Movement}: 6 m \\
\textbf{Saving Throws}: Fortitude +9, Reflex +8, Will +7 \\
\textbf{Senses}: Darkvision, sees in magical darkness \\
\textbf{Damage Resistances} from Void; from a non-magical weapon or a non-silver weapon \\
\textbf{Immunity to Damage} poison \\
\textbf{Condition Immunity} poisoned, fatigue \\
\textbf{Vulnerability} Light \\
\textbf{Invisible in the dark} a Dark Torch is completely invisible as long as it is in the dark \\
\textbf{Languages}: Understands Comume, but does not speak \\
\textit{\textbf{Undead Nature.}} Dark Torch does not need air, food, drink or sleep. \\
\textbf{Challenge} 4 (1100 PX) \\
\textit{\textbf{Sensitivity to Light}}. While out in the open, Dark Torch has -1d6 on attack rolls \\
\textbf{Multiattack} \\
\textit{\textbf{Attack}} Dark Torch attacks twice with his torch or executes Cone of Sadness \\
\textit{\textbf{Torch}} Melee attack, +8 on hit \\
\textit{\textbf{Strikes}} 7 (1d6 + 3) hit damage, cast Darkness spell on target hit, duration until Dark Torch is destroyed \\
\textit{\textbf{Cone of Sadness}} cone of 6 meters. Affected creatures must make a DC 14 Will saving throw or fall into sad despair that gives -1d6 to attack roll, -2 to melee damage. \\
\textbf{Ecology} \\
Environment: Dungeon \\
Organization Solitaire, group 2d4 \\
\textbf{Treasure}: Special \\
\textbf{Description} \\
A Dark Torch was an adventurer, like you, who died in terror after the last torch went out. A Dark Torch is an undead, usually humanoid, vaguely indefinite-looking, wielding a torch that emanates pure darkness. Its purpose is to kill new adventurers by enveloping them in eternal darkness.

Usually the Dark Torch hides in the darkness waiting to touch the opponent and wrap him in his curse. A creature killed by a dark torch is revived as a dark torch after 1d3 days.

When a Dark Torch is destroyed it leaves its torch on the ground. This torch of pure darkness can cast the dark touch spell three times per day, out of the hands of a dark torch if exposed to sunlight it destroys itself in 2d4 rounds.


\medskip\index[Mostruario]{Troll} \textbf{Troll}

\textit{Large giant, chaotic evil}

\textbf{STRENGTH} +4

\textbf{DEXTERITY} +1

\textbf{CONSTITUTION} +5

\textbf{INTELLIGENCE} -2

\textbf{WISDOM} -1

\textbf{CHARISMA} -2

\textbf{Initiative} +1 - \textbf{Defense} 18

\textbf{Hit Points} 84 (8d10 + 40)

\textbf{Movement} 9 m

\textbf{Saving Throws}: Fortitude +11, Reflex +4, Will +3

\textbf{Skills} Awareness +2

\textbf{Senses} darkvision 18 m

\textbf{Languages} Giant

\textbf{Challenge} 5 (1,800 PX)

\textit{\textbf{Refined Smell.}} The troll has + 1d6 on Wisdom (Awareness) checks based on smell.

\textit{\textbf{Regeneration.}} The troll recovers 10 hit points at the start of his round. If the troll takes acid or fire damage, this trait does not function at the start of the troll's next round. The troll only dies if it starts its round at -5 hit points and cannot regenerate.

\textbf{Actions}

\textit{\textbf{Multiattack.}} The troll can make three attacks: one with its bite and two with its claws.

\textit{\textbf{Claw.} Melee Weapon Attack}: +11 to hit, 1m range, one target.

\textit{Strikes:} 11 (2d6 + 4) slashing damage, 1 bleed damage.

\textit{\textbf{Bite.} Melee Weapon Attack}: +11 to hit, 1m range, one target.

\textit{Strikes:} 7 (1d6 + 4) piercing damage.

\textbf{Ecology} \\
Environment cold mountains \\
Organization: Solitary or gang (2-4) \\
\textbf{Treasure}: Standard \\
\textbf{Description} \\
Trolls possess sharp claws and incredible regenerative abilities that allow them to heal almost any wound. They are hunchbacked, ugly but very strong: combined with their claws, their strength allows them to tear the flesh with their bare hands. Trolls stand about 4 meters tall, but their posture makes them appear shorter. An adult troll weighs approximately 500 kg.

A troll's appetite and regenerative abilities make it an indomitable fighter, who charges head-on at the nearest living creature and attacks with all its fury. Only fire causes a troll to hesitate, but even what is a mortal danger to him does not stop his advance. Anyone who faces trolls knows they must locate and burn any part of him after a fight, because even from the smallest shred of his body, a complete troll can be reborn over time. Fortunately, only the larger parts of a troll, such as the limbs, grow back this way.

Despite their ferocity, trolls are extraordinarily tender and kind to their young. Female trolls work in groups, spending a lot of time teaching pups how to hunt and defend themselves before sending them out to find their own territory. A male troll lives a solitary existence, briefly meeting females only to mate. All trolls spend their time looking for food, as they have to consume huge quantities every day or starve. For this, most trolls create their own hunting grounds which are often defended by fighting with rivals. Such fights are usually non-lethal, but trolls know their weaknesses well, using them to kill the opponent in lean times.

\medskip\index[Mostruario]{Water Man} \textbf{Water Man}

\textit{Medium  humanoid (water man), neutral}

\textbf{STRENGTH} +0

\textbf{DEXTERITY} +1

\textbf{CONSTITUTION} +1

\textbf{INTELLIGENCE} +0

\textbf{WISDOM} +0

\textbf{CHARISMA} +1

\textbf{Initiative} +1 - \textbf{Defense} 12

\textbf{Hit Points} 11 (2d8 + 2)

\textbf{Movement} 3m, swim 12m

\textbf{Saving Throws}: Fortitude +3, Reflex +1, Will -1; +2 against Enchantment

\textbf{Skills} Awareness +2

\textbf{Languages} Aquan, Common

\textbf{Challenge} 1/8 (25 PX)

\textit{\textbf{Amphibian.}} Aquatic man can breathe air and water.

\textbf{Actions}

\textit{\textbf{Spear.} Melee or Ranged Weapon Attack}: +2 to hit, 1 m range, 6m range, one target.

\textit{Strikes:} 3 (1d6) piercing damage, or 4 (1d8) piercing damage if used with two hands to make a melee attack.

\textbf{Ecology} \\
Environment: Temperate oceans \\
Organization: Solitary, patrol (2-6), gang (6-10 plus a 3rd level lieutenant, company (11-60 plus 3 3rd level lieutenants, 2 5th level commanders, 1 7th level commodore and 3-12 Calamari \\
\textbf{Treasure}: NPC gear (Trident, Light Crossbow with 10 Bolts, other treasure) \\
\textbf{Description} \\
Physically, Fishmen resemble their ancestors, with expressive foreheads, pale skin, dark hair, and purple eyes. They have three thin gills on their necks, but they can pass as Human for short periods if they wish.

\medskip\index[Mostruario]{Tree Man (Treant)} \textbf{Tree Man (Treant)}

\textit{Huge plant, good chaotic}

\textbf{STRENGTH} +6

\textbf{DEXTERITY} -1

\textbf{CONSTITUTION} +5

\textbf{INTELLIGENCE} +1

\textbf{WISDOM} +3

\textbf{CHARISMA} +1

\textbf{Initiative} +1 - \textbf{Defense} 21

\textbf{Hit Points} 138 (12d12 + 60)

\textbf{Movement} 9 m

\textbf{Saving Throws}: Fortitude +13, Reflex +3, Will +9

\textbf{Damage Resistances} hit, piercing

\textbf{Damage Vulnerability} fire

\textbf{Languages} Common, Druidic, Elven, Sylvan

\textbf{Challenge} 9 (5000 PX)

\textit{\textbf{False Appearance.}} While the tree man remains motionless, he is indistinguishable from a normal tree.

\textit{\textbf{Siege Monster.}} Tree Man deals double damage to objects and structures.

\textbf{Actions}

\textit{\textbf{Multiattack.}} The Tree Man makes two slam attacks.

\textit{\textbf{Slam.} Melee Weapon Attack}: +16 to hit, 1m range, one target.

\textit{Hits:} 16 (3d6 + 6) hit damage.

\textit{\textbf{Rock.} Ranged weapon attack}: +16 to hit, range 18m, one target.

\textit{Hits:} 28 (4d10 + 6) hit damage.

\textit{\textbf{Animate Trees (1 / Day).}} Treeman magically animates one or two visible trees within 60 feet of him. These trees have the same stats as the ent, except they have an Intelligence and Charisma score -3, cannot speak, and only have the Slam attack option. An animated tree acts as a tree man's ally. The tree stays for 1 day or until it dies; until the tree man dies or is more than 100 feet away from the tree, or until the tree man takes a bonus action to turn him back into an inanimate tree. Then the tree will take root if possible.

\textbf{Ecology} \\
Environment any forest \\
Organization: Solitary or scrub (2-7) \\
\textbf{Treasure}: Standard \\
\textbf{Description} \\
Treants are forest guardians and tree ambassadors. As ancient as the forests themselves, they see themselves as parents and shepherds rather than gardeners: they are slow and methodical, but terrifying when forced to fight to defend their flock. While they rarely seek the company of short-lived races and have an innate distrust of change, they show tolerance to those who wish to learn from their long, slow monologues, especially those in whose eyes they see a desire to protect the wilderness. Against those who threaten their forests, especially lumberjacks who gather wood or those who would like to clear a forest to build a road or fort, the rage of the treants is unleashed swiftly and devastatingly. They are able to demolish what others build - a trait that helps them during their excesses of fury.

Treants are mainly solitary creatures, and a single individual is often responsible for an entire forest, but sometimes they gather in groups called thickets to exchange the latest news and reproduce.

In times of grave danger, all the groves of a region unite for a meeting lasting months called council, but such events are very rare, and millennia also pass between councils.

A typical treant is 9 meters tall, with a trunk diameter of 60 centimeters, and weighs around 2,250 kg. Treants resemble the most common trees in the territories where they live.

\medskip\index[Mostruario]{Magma Man (Magmin)} \textbf{Magma Man (Magmin)}

\textit{Small elemental, chaotic neutral}

\textbf{STRENGTH} -2

\textbf{DEXTERITY} +2

\textbf{CONSTITUTION} +1

\textbf{INTELLIGENCE} -1

\textbf{WISDOM} +0

\textbf{CHARISMA} +0

\textbf{Initiative} +2 - \textbf{Defense} 15

\textbf{Hit Points} 9 (2d6 + 2)

\textbf{Movement} 9 m

\textbf{Saving Throws}: Fortitude +6, Reflex +4, Will +3

\textbf{Damage Resistances} from a non-magical weapon

\textbf{Immunity to Damage} fire

\textbf{Senses} darkvision 18 m

\textbf{Languages} Ignan

\textbf{Challenge} 1/2 (100 PX)

\textit{\textbf{Incendiary Illumination.}} As a bonus action, Magma Man can turn his flames on or off. While the flame is lit, magma man radiates bright light within a 10-foot radius and dim light for an additional 10-foot.

\textit{\textbf{Deadly Burst.}} When the magma man dies, he explodes in a burst of fire and magma. Each creature within 10 feet of it must make a DC 11 Reflex saving throw, taking 7 (2d6) fire damage on a failed save, or half that damage on a successful one. Flammable objects that are not worn or carried and found in the area catch fire.

\textbf{Actions}

\textit{\textbf{Touch.} Melee Weapon Attack}: +4 to hit, 1m range, one target.

\textit{Strikes:} 7 (2d6) fire damage. If the target is a flammable creature or object, it catches fire. Until a creature takes an action to extinguish the flame, the creature takes 3 (1d6) points of fire damage at the end of each of its rounds.

\textbf{Ecology} \\
Environment any terrain (Plane of Fire) \\
Organization: Solitary or gang (2-8) \\
\textbf{Treasure}: Standard \\
\textbf{Description} \\
Although magmin populate the Plane of Fire, they sometimes slip into elemental cracks in the Material Plane. These cracks usually form in places of high heat, such as volcanoes or underground rivers of magma, or in places of strong and unpredictable magic. The latter scenario usually ends in more complex events, as magmins tend to inadvertently set fire to nearby flammable objects.

While not brave, these little outsiders are nevertheless fearsome enemies of creatures with no resistance to their intense heat. Their touch incinerates clothing, and creatures that strike their bodies with steel run the risk of reducing their weapons to dross. The best defense of the magmin at home on the Plane of Fire is their numbers. The settlements, dotted with lakes of magma and sprinkling geysers of molten rock, are teeming with incredible numbers of these creatures.

Magmins are paranoid and wary. Always frightened by the larger inhabitants of the Plane of Fire, magmin overwhelm intruders with thousands of questions, asking where they go, where they come from, and what they do near their precious magma lakes. If the travelers' answers are not satisfactory, the magmins try to get rid of the creatures as quickly as possible. Anyone who refuses to leave risks being thrown into a lake of liquid rock.

Magmins take great pride in how they care for their magma lakes. Each lake has a different purpose: to bathe, to cook meals or to relax. Magmins insert minerals and salts into these lakes to suit their purpose. Cooking lakes (sometimes called "killer lakes" by foreigners) are warmer than others, and recreational lakes are usually darker than bathing lakes.

At maturity, magmins are 1.2 meters tall, their dense composition makes them weigh 150 kg.

\medskip\index[Mostruario]{Unicorn} \textbf{Unicorn}

\textit{Large celestial, legal good}

\textbf{STRENGTH} +4

\textbf{DEXTERITY} +2

\textbf{CONSTITUTION} +2

\textbf{INTELLIGENCE} +0

\textbf{WISDOM} +3

\textbf{CHARISMA} +3

\textbf{Initiative} +2 - \textbf{Defense} 15

\textbf{Hit Points} 67 (9d10 + 18)

\textbf{Movement} 15 m

\textbf{Saving Throws}: Fortitude +7, Reflex +7, Will +6; +2 resistance against the Void, Negative Energy

\textbf{Immunity to Damage} poison

\textbf{Condition Immunity} fascinated, poisoned, paralyzed

\textbf{Senses} darkvision 18 m

\textbf{Languages} Celestial, Elvish, Sylvan, telepathy 18 m

\textbf{Challenge} 5 (1.800 PX)

\textit{\textbf{Magical Weapons.}} The unicorn's weapon attacks are magical.

\textit{\textbf{Charge.}} If the unicorn moves at least 20 feet in a straight line to the target and hits it with a horn attack during the same turn, the target takes 9 (2d8) piercing damage additional. If the target is a creature, it must succeed on a DC 15 Fortitude save or fall prone.

\textit{\textbf{Innate Spells.}} The unicorn's innate spellcasting characteristic is Charisma (DC 14 on saving throws for spells). The unicorn can innately cast the following spells, without the need for components:

At will: \textit{art of the druid, identification of good and evil,} \textit{go without traces}

1 / day each: \textit{calm emotions, dissolve good and evil,} \textit{get in the way}

\textit{\textbf{Magic Resistance.}} The unicorn has + 1d6 on saving throws against spells and other magical effects.

\textbf{Actions}

\textit{\textbf{Multiattack.}} The unicorn makes two attacks: one with its hooves and one with its horn.

\textit{\textbf{Horn.} Melee Weapon Attack}: +10 to hit, 1m range, one target.

\textit{Strikes:} 8 (1d8 + 4) piercing damage.

\textit{\textbf{Hooves.} Melee Weapon Attack}: +10 to hit, 1m range, one target.

\textit{Strikes:} 11 (2d6 + 4) hit damage.

\textit{\textbf{Teleport (1 / day).}} The unicorn can magically teleport itself and up to three other consenting creatures visible within 1 meter of it, along with any equipment they are wearing or carrying , in a place familiar to the unicorn, which is no more than 1.5 kilometers away.

\textit{\textbf{Healing Touch (3 / Day).}} The unicorn contacts another creature via its horn. The target magically recovers 11 (2d8 + 2) hit points. Additionally, contact removes all disease and neutralizes all poisons affecting the target.

\textbf{Additional Actions}

The unicorn can perform 3 additional Actions, chosen from the following options. He can only use one legendary option at a time, and only at the end of another creature's turn. The unicorn recovers any additional actions spent at the start of its round.

\textbf{Self-healing (Costs 3 Actions).} The unicorn magically recovers 11 (2d8 + 2) hit points.

\textbf{Shimmer Shield (Costs 2 Actions).} The unicorn creates a shimmering magical field surrounding him or another creature visible to him within 60 feet. The target gets a +2 Defense bonus until the unicorn's next round ends.

\textbf{Hoofs.} The unicorn makes a hoof attack.

\textbf{Ecology} \\
Environment: Temperate Forests \\
Organization: Solitary, pair or blessing (3-6) \\
\textbf{Treasure}: None \\
\textbf{Description} \\
Unicorns are fierce, intelligent sylvan creatures who prefer to remain isolated, appearing only to defend their abodes from evil. They avoid all creatures except good goblins, good humanoid women, and the animals native to their forest, but they might team up with other good creatures against common enemies. A typical unicorn is 2.4 meters long, 1 meter high at the withers and weighs 600 kg.

Pairs of unicorns remain together throughout their lives and dwell in particular clearings or within the forests they defend. They allow good and neutral creatures to cross, hunt, or inhabit them, but evil creatures or creatures that would disturb their ecosystem, such as hunting for fun or cutting down their trees to sell their timber, are quickly removed or killed. On rare occasions, unicorns whose partner has been killed take young women of rare virtue as surrogates, allowing them to ride them and become their lifelong guardians. If the woman becomes attached to someone else, such as a child or a lover, the bond with the unicorn melts lovingly, generating the legend that unicorns become friends only with virgins.

A unicorn's horn is the source of its powers, and in order to use its magical abilities on other creatures, they must touch them with it. Evil creatures place great value on unicorn horns as reagents for healing potions and dark rituals - a powdered unicorn horn is worth 800 gp when used to craft a magical healing item.

\subsection{Vampires}

\medskip\index[Mostruario]{Vampire} \textbf{Vampire}

\textit{Medium  undead (shapeshifter), lawful evil}

\textbf{STRENGTH} +4

\textbf{DEXTERITY} +4

\textbf{CONSTITUTION} +4

\textbf{INTELLIGENCE} +3

\textbf{WISDOM} +2

\textbf{CHARISMA} +4

\textbf{Initiative} +4 - \textbf{Defense} 23

\textbf{Hit Points} 144 (17d8 + 68)

\textbf{Movement} 9 m

\textbf{Saving Throws}: Fortitude +13, Reflex +11, Will +12

\textbf{Skills} Move Silently / Hide +9, Awareness +17

\textbf{Immunity to Damage} from Void; non-magical weapon

\textbf{Senses} darkvision 36 m

\textbf{Languages} the languages he knew in life

\textbf{Challenge} 13 (10000 PX)

\textit{\textbf{Shapeshifter.}} If the vampire is not in sunlight or immersed in running water, he can use his action to transform into a Tiny bat, a Cloud of Medium haze, or to return to the its true form.

While in bat form, the vampire cannot speak, his walking speed is 1 meter and he has 9 meters flight speed. Its stats, aside from size and speed, are unchanged. Whatever equipment he is wearing transforms with it, but what he was carrying is knocked to the ground. At death it returns to its true form.

While in haze form, the vampire cannot perform actions, speak, or manipulate objects. It is weightless, has 20-foot flight speed, can float, and can enter the space of a hostile creature and stop there. Also, if air passes through a space, the mist can do the same loosely, but it cannot pass through the water. It has + 1d6 to Fortitude and Reflex saving throws, and is immune to all nonmagical damage except damage taken from the light of the
Sun.

\textit{\textbf{Weaknesses of the Vampire.}} The vampire has the following defects:

\textit{Damaged by Running Water.} The vampire takes 20 points of acid damage if he ends his round in running water.

\textit{Hypersensitivity to Light.} The vampire takes 20 Light damage when she begins her round in sunlight. While out in the open, he has -1d6 on attack rolls and proficiency checks.

\textit{Stake in the Heart.} If a piercing weapon made of wood is driven into the vampire's heart while the vampire is incapacitated in its resting place, the vampire is paralyzed until the stake is removed.

\textit{Prohibition.} The vampire cannot enter a dwelling without an invitation from its occupants.

\textit{\textbf{Escape into the Mist.}} When it drops to 0 hit points outside of its resting place, the vampire transforms into a cloud of haze (as per the shapeshifter trait) instead of falling uncharted. senses, as long as it is not exposed to sunlight or running water. If it cannot transform, it is destroyed.

While at 0 hit points in this form, he cannot revert to his vampire form, and must reach his resting place within 2 hours or be destroyed. Once he reaches his resting place, he reverts to his vampire form. He will then remain paralyzed until he has recovered at least 1 hit point. After spending at least 1 hour in his resting place at 0 hit points, the vampire will recover 1 hit point.

\textit{\textbf{Undead Nature.}} The vampire does not need air.

\textit{\textbf{Legendary Resistance (3 / Day).}} If the vampire fails a saving throw, he may choose to succeed instead.

\textit{\textbf{Regeneration.}} The vampire recovers 20 hit points at the start of his round if he has at least 1 hit point and is not exposed to sunlight or running water. If the vampire takes damage from light or damage from holy water, this trait does not function at the start of the vampire's next round.

\textit{\textbf{Climbing as a Spider.}} The vampire can scale difficult surfaces, including standing upside down on the ceiling, without making a skill check.

\textbf{Actions}

\textit{\textbf{Multiattack.}} The vampire can make two attacks, but only one of them can be a bite attack.

\textit{\textbf{Unarmed Strike (Vampire Form Only).} Melee Weapon Attack}: +18 to hit, range 1 yd, a creature.

\textit{Strikes:} 8 (1d8 + 4) hit damage. Instead of dealing damage, the vampire can grab the target (DC to flee 18).

\textit{\textbf{Bite (In Bat or Vampire Form Only).} Melee Weapon Attack}: +18 to hit, range 1 yd, a willing creature or a creature grabbed by vampire, incapacitated or hindered.

\textit{Strikes:} 7 (1d6 + 4) piercing damage plus 10 (3d6) Void damage. The target's maximum hit points are reduced by an amount equal to the Void damage taken, and the vampire recovers a number of hit points equal to that amount, Fortitude save DC 23 to resist the loss of maximum hit points. The target becomes Fatigued. This reduction lasts until the new dawn. The target dies if this effect reduces its maximum hit points to 0. A humanoid killed this way and then buried in the ground revives the following night as vampire spawn under the vampire's control.

\textit{\textbf{Fascinating.}} The vampire targets a humanoid within 30 feet that he can see. If the target can see the vampire, they must make a DC 17 Will saving throw against this spell or be fascinated by it. The fascinated target considers the vampire a trusted friend to listen to and protect. Although the target is not under the vampire's control, it takes the vampire's requests and actions as favorably as possible, and is a willing target of the vampire's bite attack.

Whenever the vampire or the vampire's companions do something harmful to the target, he can re-roll the saving throw, ending the effect on himself if he succeeds. Otherwise, the effect persists for 24 hours or until the vampire is destroyed, is on a different plane of existence than the target, or takes a bonus action to end the effect.

\textit{\textbf{Children of the Night (1 / Day).}} The vampire magically summons 2d4 swarms of bats or rats, as long as the sun has not risen. While outside, the vampire can summon 3d6 wolves instead. The summoned creatures arrive in 1d4 rounds, acting as allies of the vampire and obeying his commands. The beasts remain for 1 hour, until the vampire dies, or until he dismisses them with a bonus action.

\textbf{Additional Actions}

The vampire can perform 3 additional Actions, chosen from the following options. He can only use one legendary option at a time, and only at the end of another creature's turn. The vampire recovers the additional Actions he spent at the beginning of his round.

\textbf{Unarmed Strike.} The vampire makes an unarmed strike.

\textbf{Bite (Costs 2 Actions).} The vampire makes a bite attack.

\textbf{Move.} The vampire moves its own motion without provoking attacks of opportunity.

\textbf{Ecology}
Environment: Any \\
Organization: Solitary or family (vampire plus 2-8 offspring) \\
\textbf{Treasure}: NPC gear (Ring of Protection +2, Sash of Seduction +4, Cloak of Resistance +3) \\
\textbf{Description} \\
Vampires are undead humanoid creatures that feed on the blood of the living. They look very similar to when they were alive, often becoming more attractive, although some appear tough and feral instead.


\medskip\index[Mostruario]{Vampiric offspring} \textbf{Vampiric offspring}

\textit{Medium  undead, neutral evil}

\textbf{Initiative} +0 - \textbf{Defense} 18

\textbf{Hit Points} 82 (11d8 + 33)

\textbf{Movement} 9 m

\textbf{Saving Throws} Fortitude +3, Reflex +2, Will +5

\textbf{STRENGTH} +3

\textbf{DEXTERITY} +3

\textbf{CONSTITUTION} +3

\textbf{INTELLIGENCE} +0

\textbf{WISDOM} +0

\textbf{CHARISMA} +1

\textbf{Skills} Move Silently / Hide +6, Awareness +3

\textbf{Damage Resistances} from Void; non-magical weapon

\textbf{Senses} darkvision 18 m

\textbf{Languages} the languages he knew in life

\textbf{Challenge} 6 (1.800 PX)

\textit{\textbf{Weaknesses of the Vampiric Spawn.}} The Vampiric Spawn has the following defects:

\textit{Damaged by Running Water.} The Vampiric Spawn takes 20 points of acid damage if it ends its round in running water.

\textit{Hypersensitivity to Light.} The Vampiric Spawn takes 20 Light damage when it begins its round in sunlight. While out in the open, he has -1d6 on attack rolls and proficiency checks.

\textit{Stake in the Heart.} The Vampiric Spawn is destroyed if a wooden piercing weapon is driven into his heart while he is incapacitated within his resting place.

\textit{Prohibition.} Vampiric Spawn cannot enter a dwelling without an invitation from its occupants.

\textit{\textbf{Undead Nature.}} Vampiric Spawn does not need air.

\textit{\textbf{Regeneration.}} The Vampiric Spawn recovers 10 hit points at the start of its round if it has at least 1 hit point and is not exposed to sunlight or running water. If the vampiric spawn takes damage from light or damage from holy water, this trait does not function at the start of the vampire's next round.

\textit{\textbf{Climbing like Spider.}} Vampiric Spawn can scale difficult surfaces, including standing upside down on the ceiling, without the need for a skill check.

\textbf{Actions}

\textit{\textbf{Multiattack.}} Vampire spawn can make two attacks, but only one of them can be a bite attack.

\textit{\textbf{Claws.} Melee Weapon Attack}: +9 to hit, range 1 yards, a creature.

\textit{Strikes:} 8 (2d4 + 3) slashing damage. Instead of dealing damage, the vampire can grab the target (DC to flee 13).

\textit{\textbf{Bite.} Melee Weapon Attack}: +9 to hit, range 1 yd, a creature grabbed by vampire, incapacitated or hampered.

\textit{Strikes:} 6 (1d6 + 3) piercing damage plus 7 (2d6) Void damage. The target's maximum hit points are reduced by an amount equal to the Void damage taken, and the vampire recovers a number of hit points equal to that amount, Fortitude save DC 16 to resist the loss of maximum hit points. This reduction lasts until the new dawn. The target dies if this effect reduces its maximum hit points to 0. The creature becomes fatigued.

\textbf{Ecology} \\
Environment: Any \\
Organization: Solitary, couple, group (3-6) or mob (7-12) \\
\textbf{Treasure}: Standard \\
\textbf{Description} \\
A Vampire may decide to create a vampire spawn from a victim instead of making them a full vampire only when he uses his ability to create spawn on a humanoid creature. This decision must be made as a free action as soon as a vampire kills an appropriate creature using the bite.

\medskip\index[Mostruario]{Worms of the flesh} \textbf{Worms of the flesh}

\textit{tiny monstrosity, misaligned}

\textbf{STRENGTH} -4

\textbf{DEXTERITY} +0

\textbf{CONSTITUTION} -2

\textbf{INTELLIGENCE} -4

\textbf{WISDOM} 0

\textbf{CHARISMA} -4

\textbf{Initiative} +0 - \textbf{Defense} 11

\textbf{Hit Points} 1 (1d6 -2)

\textbf{Movement} 1 m

\textbf{Saving Throws}: Fortitude -1, Reflex +0, Will -4

\textbf{Senses} telluric view 3 m

\textbf{Languages} -

\textbf{Challenge} 1 (200 PX)

\textbf{Actions}

\textit{\textbf{Infesting flesh.}} These tiny creatures penetrate exposed flesh without making an attack roll as long as the flesh is exposed in contact with them.

\textit{\textbf{Strikes.}} Within 2d4 rounds, the worms (3d6 creatures) of the flesh burrow into the tissue and head towards the heart. Worm infestation deals 1 hit point of damage per round while digging. Once at the heart each round the character must make a DC 14 Fortitude save, with a cumulative penalty of -1 per round. Once the saving throw fails the character dies.

\textit{\textbf{Eradicate the Worms of the flesh.}} The only way is to use a live flame (a torch does 1d6 damage per application or a spell like burning hands) on the part where the worms are burrowing. Each fire application can eliminate 3d6 worms. After 2d4 rounds the worms are too deep and it is useless to apply fire, only a cure disease spell can completely eradicate the infestation.

\textbf{Ecology} \\
Environment: rotten trees, rotten flesh \\
Organization: 3d6 groups \\
\textbf{Treasure}: None \\
\textbf{Description} \\

Meatworms are among the most feared parasites by adventurers. They are found in moist mounds of rotten leaves or trunks, in rotting corpses, in murky waters. Pale, slimy, with very sharp teeth, just over 4 millimeters long, they penetrate the exposed flesh very easily and perceive the heartbeat where they are headed. As they dig into the flesh they can be felt and even seen crawling under the skin.


\medskip\index[Mostruario]{Purple Worm} \textbf{Purple Worm}

\textit{Gargantuan monstrosity, misaligned}

\textbf{STRENGTH} +9

\textbf{DEXTERITY} -2

\textbf{CONSTITUTION} +6

\textbf{INTELLIGENCE} -5

\textbf{WISDOM} -1

\textbf{CHARISMA} -3

\textbf{Initiative} -2 - \textbf{Defense} 26

\textbf{Hit Points} 247 (15d20 + 90)

\textbf{Movement} 15 m, Burrow 9 m

\textbf{Saving Throws}: Fortitude +17, Reflex +8, Will +4

\textbf{Senses} blind sight 9 m, telluric sense 18 m

\textbf{Languages} -

\textbf{Challenge} 15 (13000 PX)

\textit{\textbf{Tunnel Digger.}} The worm can dig through solid rock at half the speed of digging and leaves a 10-foot-diameter tunnel behind it.

\textbf{Actions}

\textit{\textbf{Multiattack.}} The worm makes two attacks: one with its bite and one with its sting.

\textit{\textbf{Bite.} Melee Weapon Attack}: +30 to hit, 3m range, one target.

\textit{Strikes:} 22 (3d8 + 9) piercing damage. If the target is a large creature, it must succeed on a DC 19 Reflex saving throw or be swallowed by the worm. While engulfed, the creature is blinded and entangled, has full cover against attacks and other effects from outside the worm, and takes 21 (6d6) points of acid damage at the start of each worm's turn.

If the worm takes 30 or more damage in a single turn from a creature inside it, the worm must succeed on a DC 21 Fortitude save at the end of its round or vomit all swallowed creatures, which fall prone in a space within 3. meters from the worm. If the worm dies, a swallowed creature is no longer hampered by it and can escape the corpse using 20 feet of movement, coming out prone.

\textit{\textbf{Sting.} Melee weapon attack}: +9 to hit, range 10 feet, a creature.

\textit{Strikes:} 19 (3d6 + 9) piercing damage, and the target must make a Fortitude saving throw DC 19, taking 42 (12d6) poison damage on a failed save, or half that damage if he succeeds.

\textbf{Ecology} \\
Environment: Any underground \\
Organization: Solitary \\
\textbf{Treasure}: Accidental \\
\textbf{Description} \\
Purple worms are giant necrophages that inhabit the deepest regions of the world, eating whatever organic material they encounter. They are known to swallow their prey whole. It is not unusual to hear of a group of adventurers disappearing inside the ravenous jaws of a purple worm, screaming in terror as its members disappeared one at a time.

As they search for living creatures to devour them, the purple worms also swallow a huge amount of earth and minerals by burrowing underground. A purple worm's innards can contain a considerable number of gems and other objects that can withstand the corrosive acid within its esophagus. In areas rich in precious minerals, such as those close to dwarven mines, the natural tunnels created by the excavation of purple worms are often filled with a considerable number of raw gold nuggets.

A purple worm generally claims a large underground cave as its lair, and even if it returns to rest and digest food, it spends most of its time hunting for prey, burrowing through endless darkness, or sliding down pre-existing tunnels. constant search for food to satisfy his immense hunger. Though almost devoid of intellect, purple worms are rarely dumb. They are popular as guardians among those who can magically control them or have a room in their lair large enough to house them.


\medskip\index[Mostruario]{Tentacled Creeping Worm} \textbf{Tentacled Creeping Worm}

\textit{Large monstrosity, misaligned}

\textbf{STRENGTH} +4

\textbf{DEXTERITY} +1

\textbf{CONSTITUTION} +3

\textbf{INTELLIGENCE} -4

\textbf{WISDOM} 1

\textbf{CHARISMA} -3

\textbf{Initiative} +2 - \textbf{Defense} 17

\textbf{Hit Points} 55 (7d10 + 31)

\textbf{Movement} 9 m, climb 9 m

\textbf{Saving Throws}: Fortitude +5, Reflex +4, Will +7

\textbf{Senses} darkvision 18 m

\textbf{Languages} -

\textbf{Challenge} 4 (1000 PX)

\textit{\textbf{Climbing like Spider.}} The Tentacled Crawling Worm can scale difficult surfaces, including standing upside down on the ceiling, without the need to make a skill check.

\textbf{Actions}

\textit{\textbf{Multiattack.}} The Tentacled Crawling Worm makes 3 attacks, one with its bite and two with its tentacles.

\textit{\textbf{Bite.} Melee Weapon Attack}: +8 to hit, 1m range, one target.

\textit{Strikes:} 10 (2d8 + 6) piercing damage.

\textit{\textbf{Tentacle.} Melee weapon attack}: +7 to hit, range 10 feet, a creature.

\textit{Strikes:} 1 hit damage. The target must make a DC 18 Fortitude saving throw or be paralyzed until the end of the next round.

\textbf{Ecology} \\
Environment: Any dungeon \\
Organization: Solitaire, pair, tribe (8-12 + 3d6 small) \\
\textbf{Treasure}: Accidental \\
\textbf{Description} \\

A typical Tentacled Crawling Worm is an annelid nearly 4 meters long and weighs around 400 kilograms. Dark in color (of various shades from blue to green to brown) it is a large worm with a powerful mouth and long and light tentacles along the entire head.

The Tentacled Creeping Worm even if equipped with short "legs" does not walk but crawls secreting a sticky mucus that allows it to climb even on surfaces in any orientation.

They are ravenous creatures that never miss an opportunity to hunt and devour or store corpses to sow their eggs. They love Nibali's meat and feed on any living creature (often rats given the typical sewer environment).

The origins of the Tentacled Crawling Worms are quite speculative, some speculate that a charmer tried, failing critically, to transform into a Purple Worm, others firmly believe that the gardens of Shayalia needed more fertilization and so the Patroness transformed ordinary earthworms into these terrifying creatures to devour and digest the buried corpses.

\medskip\index[Mostruario]{Viverna} \textbf{Viverna}

\textit{Large dragon, misaligned}

\textbf{STRENGTH} +4

\textbf{DEXTERITY} +0

\textbf{CONSTITUTION} +3

\textbf{INTELLIGENCE} -3

\textbf{WISDOM} +1

\textbf{CHARISMA} -2

\textbf{Initiative} +0 - \textbf{Defense} 16

\textbf{Hit Points} 110 (13d10 + 39)

\textbf{Movement} 6m, flight 24m

\textbf{Saving Throws}: Fortitude +9, Reflex +6, Will +8

\textbf{Skills} Awareness +4

\textbf{Senses} darkvision 18 m

\textbf{Languages} -

\textbf{Challenge} 6 (2.300 PX)

\textbf{Actions}

\textit{\textbf{Multiattack.}} The wyvern can make two attacks: one with its bite and one with its sting. While flying, it can use its claws in place of one of the other attacks.

\textit{\textbf{Claws.} Melee Weapon Attack}: +13 to hit, 1m range, one target.

\textit{Strikes:} 13 (2d8 + 4) slashing damage, 1 bleed damage.

\textit{\textbf{Bite.} Melee weapon attack}: +13 to hit, range 10 feet, a creature.

\textit{Strikes:} 11 (2d6 + 4) piercing damage.

\textit{\textbf{Sting.} Melee weapon attack}: +13 to hit, range 10 feet, a creature.

\textit{Strikes:} 11 (2d6 + 4) piercing damage. The target must make a DC 15 Fortitude save, and take 24 (7d6) poison damage if it fails, or half that damage if it succeeds.

\textbf{Ecology} \\
Environment: Temperate or warm hills \\
Organization: Solitary, pair or flock (3-6) \\
\textbf{Treasure}: standard \\
\textbf{Description} \\
Wyverns are brutal and violent reptiles related to dragons. They are always aggressive and impatient and prefer to achieve their goals using force. For this reason, dragons look upon living with superiority, considering these distant relatives of theirs as primitive savages devoid of style and intelligence.

In most cases, this generalization is spot on. Although certainly not of animal intellect and capable of speech, most of the Wyverns do not care about diplomacy, preferring to fight first and discuss later, only if they find themselves in front of an opponent they cannot defeat or from which they cannot escape.

Wyverns are territorial creatures. While occasionally hunting larger prey in larger groups, they are solitary creatures whose hunting territory extends from 160 to 320 square km. It is known that wyverns often fight each other to the death over disputes over a territory rich in prey.

While constantly hungry and prone to attack, a wyvern can be made friendly through a careful combination of flattery, intimidation, food, and treasure, to make a powerful ally. Monstrous Giants and Humanoids often serve as guardians as guardians, and some tribes of Boggard and Lizardmen use them as mounts, although such arrangements often turn out to be quite expensive in terms of food and gold, as few wyverns agree to serve creatures for long. similar as mounts.

A wyvern is about 4.8 meters long, and the tail alone accounts for about half the length. A wyvern weighs an average of 1000 kg.


\medskip\index[Mostruario]{Wight} \textbf{Wight}

\textit{Medium  undead, neutral evil}

\textbf{STRENGTH} +2

\textbf{DEXTERITY} +2

\textbf{CONSTITUTION} +3

\textbf{INTELLIGENCE} +0

\textbf{WISDOM} +1

\textbf{CHARISMA} +2

\textbf{Initiative} +2 - \textbf{Defense} 16 (studded armor)

\textbf{Hit Points} 45 (6d8 + 18)

\textbf{Movement} 9 m

\textbf{Saving Throws}: Fortitude +3, Reflex +2, Will +5

\textbf{Skills} Move Silently / Hide +4, Awareness +3

\textbf{Damage Resistances} from Void; a non-magical weapon or a non-silver weapon

\textbf{Immunity to Damage} poison

\textbf{Condition Immunity} poisoned, fatigue

\textbf{Senses} darkvision 18 m

\textbf{Languages} the languages he knew in life

\textbf{Challenge} 3 (700 PX)

\textit{\textbf{Undead Nature.}} The wight does not need air, food, drink or sleep.

\textit{\textbf{Sensitivity to Light}}. While in sunlight, the wight has -1d6 on attack rolls, as well as sight-based Wisdom checks.

\textbf{Actions}

\textit{\textbf{Multiattack.}} The wight can make two longsword attacks or two longbow attacks. He can use Drain Life in place of one of his longsword attacks.

\textit{\textbf{Drain Life.} Melee Weapon Attack}: +5 to hit, range 1 yd, a creature.

\textit{Strikes:} 5 (1d6 + 2) Void damage. The target must succeed on a DC 13 Fortitude save or have its maximum hit points reduced by an amount equal to the damage taken. The target becomes Fatigued. This reduction lasts until the new dawn breaks. The target dies if the effect reduces its maximum hit points to 0.

A humanoid killed by this attack revives 24 hours later as a zombie under the wight's control, unless the humanoid is first revived or the body is destroyed. The wight cannot control more than twelve zombies at a time.

\textit{\textbf{Longsword.} Melee Weapon Attack}: +5 to hit, 1m range, one target.

\textit{Strikes:} 6 (1d8 + 2) slashing damage or 7 (1d10 + 2) slashing damage when used with two hands.

\textit{\textbf{Longbow.} Ranged weapon attack}: +5 to hit, range 45m, one target.

\textit{Strikes:} 6 (1d8 + 2) piercing damage.

\textbf{Ecology} \\
Environment: any \\
Organization: Solitary, pair, group (3-6) or pack (7-12) \\
\textbf{Treasure}: Standard \\
\textbf{Description} \\
Wights are humanoids resurrected as undead due to necromancy, a violent death, or an extremely malevolent personality. In some cases, a wight arises when an undead spirit permanently binds itself to a corpse, often that of a warrior. They are hardly recognizable by those who knew them in life: their flesh is corrupted by wickedness and undeath, their eyes burn with hatred and their teeth become those of a beast. In a sense, a wight is the connecting link between ghouls and ghosts: a deformed corpse that sucks life energy with its touch.

Being undead, wights do not need to breathe, so they can sometimes be found underwater, although they are not particularly skilled swimmers unless they originated from swimming creatures such as water elves and seafarers. Underwater wights prefer low ceiling caverns where their poor swimming skills are not a limitation.

\medskip\index[Mostruario]{Wraith} \textbf{Wraith}

\textit{Medium  undead, neutral evil}

\textbf{STRENGTH} -2

\textbf{DEXTERITY} +3

\textbf{CONSTITUTION} +3

\textbf{INTELLIGENCE} +1

\textbf{WISDOM} +2

\textbf{CHARISMA} +2

\textbf{Initiative} +3 - \textbf{Defense} 16

\textbf{Hit Points} 67 (9d8 + 27)

\textbf{Movement} 0m, flight 18m (float)

\textbf{Saving Throws}: Fortitude +6, Reflex +4, Will +6

\textbf{Resistance to Damage} acid, cold, lightning, fire, sound; a non-magical weapon or a non-silver weapon

\textbf{Immunity to Damage} Void, poison

\textbf{Condition Immunity} fascinated, grabbed, poisoned, entangled, paralyzed, petrified, prone, fatigue

\textbf{Senses} darkvision 18 m

\textbf{Languages} the languages he knew in life

\textbf{Challenge} 5 (1,800 PX)

\textit{\textbf{Incorporeal Movement.}} The wraith can pass through creatures and objects as if they were hindering terrain. It takes 5 (1d10) force damage if it ends its round inside an object.

\textit{\textbf{Undead Nature.}} The wraith needs no air, food, drink, or sleep.

\textit{\textbf{Sensitivity to Light}}. While in the open, the wraith has -1d6 on attack rolls, as well as sight-based Wisdom (Awareness) checks.

\textbf{Actions}

\textit{\textbf{Drain Life.} Melee Weapon Attack}: +7 to hit, range 1 yd, a creature.

\textit{Strikes:} 21 (4d8 + 3) Void damage. The target must succeed on a DC 15 Fortitude save or have its maximum hit points reduced by an amount equal to the damage taken. The target becomes Fatigued. This reduction lasts until the new dawn breaks. The target dies if the effect reduces its maximum hit points to 0.

\textit{\textbf{Create Ghost.}} The wraith targets a humanoid within 10 feet of it who has been dead no more than 1 minute and from violent causes. The target's spirit animates as a ghost in the space of his corpse and in the nearest unoccupied space. The specter is under the control of the wraith. The wraith cannot hold more than seven ghosts under its control at a time.

\textbf{Ecology} \\
Environment: Any \\
Organization: Solitary, pair, group (3-6) or pack (7-12) \\
\textbf{Treasure}: None \\
\textbf{Description} \\
Wraiths are creatures born of evil and darkness. They detest light and living creatures, having lost most of the connection with their previous life.


\medskip\index[Mostruario]{Xorn} \textbf{Xorn}

\textit{medium Elemental, neutral}

\textbf{STRENGTH} +3

\textbf{DEXTERITY} +0

\textbf{CONSTITUTION} +6

\textbf{INTELLIGENCE} +0

\textbf{WISDOM} +0

\textbf{CHARISMA} +0

\textbf{Initiative} +0 - \textbf{Defense} 22

\textbf{Hit Points} 73 (7d8 + 42)

\textbf{Movement} 6 m, Burrow 6 m

\textbf{Saving Throws}: Fortitude +8, Reflex +2, Will +5

\textbf{Skills} Move Silently / Hide +3, Awareness +6

\textbf{Damage Resistances} piercing and cutting of non-magical or non-adamantium weapons

\textbf{Senses} darkvision 18 m, telluric sense 18 m

\textbf{Languages} Terran

\textbf{Challenge} 5 (1,800 PX)

\textit{\textbf{Stone cloaking.}} The xorn has + 1d6 on Dexterity (Hide) checks made to hide on rocky terrain.

\textit{\textbf{Scroll the Earth.}} The xorn can dig through nonmagical and unworked earth and stone. When it does, the xorn does not disturb the material it moves.

\textit{\textbf{Sense of Treasure.}} The xorn can precisely detect, by smell, the location of metals and precious stones, such as coins and gems, within 60 feet of it.

\textbf{Actions}

\textit{\textbf{Multiattack.}} The xorn makes three claw attacks and one bite attack.

\textit{\textbf{Claw.} Melee Weapon Attack}: +9 to hit, 1m range, one target.

\textit{Strikes:} 6 (1d6 + 3) slashing damage, 1 bleed damage.

\textit{\textbf{Bite.} Melee Weapon Attack}: +9 to hit, 1m range, one target.

\textit{Strikes:} 13 (3d6 + 3) piercing damage.

\textbf{Ecology} \\
Environment: Any (Plane of the Earth) \\
Organization: Solitary, pair or group (3-6) \\
\textbf{Treasure}: Standard (only precious metals, gems and jewels and magical gems) \\
\textbf{Description}
Strange creatures as wide as they are tall, xorns have little interest in the natives of the Material Plane, were it not for the gems and precious metals they may have with them. Hidden beneath the surface of the ground for what might seem a very long time to a human, a xorn can wait months, even years, for its ideal prey, and then attack those who carry their favorite food, such as a particular gem or a particular food. type of silver. Adventurers who venture into xorn-inhabited regions often bring small nuggets of minerals or inexpensive gems and crystals with them to use as a tribute. Although its value is usually directly proportional to its flavor and the palatability it may have, most xorns are quite greedy, and prefer quantity over quality.

The treasure a xorn carries with him or hides in his lair consists of a snack that he has saved for the next day. Offering a particularly delicious (and expensive) jewel or precious metal to a xorn can cement a temporary alliance. Since xorns can traverse rock with ease, they are excellent guides in underground regions.

The xorns are not very religious, but those among them who find the faith are usually Druids (although it is rare, if not unlikely, that xorns have Animal Companions, as they cannot follow them into the rock, and choose Earth dominion instead. ). Bards and xorn Devotees are not unknown: Bards usually choose Entertain (chant), and Devotees invariably have the Elemental Bloodline (earth).


\medskip\index[Mostruario]{Zombies} \textbf{Zombies}

\textit{Medium  undead, neutral evil}

\textbf{STRENGTH} +1

\textbf{DEXTERITY} -2

\textbf{CONSTITUTION} +3

\textbf{INTELLIGENCE} -4

\textbf{WISDOM} -2

\textbf{CHARISMA} -3

\textbf{Initiative} -2 - \textbf{Defense} 9

\textbf{Hit Points} 22 (3d8 + 9)

\textbf{Movement} 6 m

\textbf{Saving Throws} Fortitude +0, Reflex +0, Will +3

\textbf{Immunity to Damage} poison

\textbf{Condition Immunity} poisoned

\textbf{Senses} darkvision 18 m

\textbf{Languages} understands all languages he spoke in life but cannot speak

\textbf{Challenge} 1/4 (50 PX)

\textit{\textbf{Undead Nature.}} The zombie does not need air, food, drink or sleep.

\textit{\textbf{Fortitude of the undead.}} If the damage reduces the zombie to 0 hit points, the zombie must make a DC 5 Fortitude saving throw + the damage taken, unless the damage is from Light or a critical hit. If successful, the zombie drops to 1 hit point instead.

\textbf{Actions}

\textit{\textbf{Slam.} Melee Weapon Attack}: +3 to hit, 1m range, one target.

\textit{Strikes:} 4 (1d6 + 1) hit damage.

\textbf{Ecology} \\
Environment: Any \\
Organization: Any \\
\textbf{Treasure}: None \\
\textbf{Description} \\
Zombies are the animated corpses of dead creatures, forced to move by necromantic spells such as Animate Dead. While zombies encountered are usually slow and sturdy, others have different traits, which allow them to spread disease or move faster.

Zombies are mindless automatons and can only follow orders. Left to their own devices, they wait motionless or move in search of living creatures to slaughter and devour. Zombies attack to destruction, regardless of their safety.

Although they are able to follow orders, zombies are often set free with orders to kill all living creatures. They are often encountered in flocks that infest the lands frequented by the living, in search of prey. Most zombies are created through Animate Dead. Such zombies are always standard, unless the creator also casts Speed or Remove Paralysis to create Quick Zombies or Contagion to create Infected Zombies.


\medskip\index[Mostruario]{Zombie Ogre} \textbf{Zombie Ogre}

\textit{Large undead, neutral evil}

\textbf{STRENGTH} +4

\textbf{DEXTERITY} -2

\textbf{CONSTITUTION} +4

\textbf{INTELLIGENCE} -4

\textbf{WISDOM} -2

\textbf{CHARISMA} -3

\textbf{Initiative} -2 - \textbf{Defense} 9

\textbf{Hit Points} 85 (9d10 + 36)

\textbf{Movement} 9 m

\textbf{Saving Throws}: Fortitude +6, Reflex +0, Will +3

\textbf{Immunity to Damage} poison

\textbf{Condition Immunity} poisoned

\textbf{Senses} darkvision 18 m

\textbf{Languages} understands Common and Giant but cannot speak

\textbf{Challenge} 2 (450 PX)

\textit{\textbf{Undead Nature.}} The zombie does not need air, food, drink or sleep.

\textit{\textbf{Fortitude of the undead.}} If the damage reduces the zombie to 0 hit points, the zombie must make a DC 5 Fortitude saving throw + the damage taken, unless the damage is from Light or a critical hit. If successful, the zombie drops to 1 hit point instead.

\textbf{Actions}

\textit{\textbf{Spiked Mace.} Melee Weapon Attack}: +6 to hit, 1m range, one target.

\textit{Strikes:} 13 (2d8 + 4) hit damage.


\subsection{Appendix A: Various Creatures}

This appendix contains the statistics of various animals, parasites and
other creatures. The statistics are organized in alphabetical order.

\medskip \textbf{Awakened Tree} \index[Mostruario]{Awakened Tree}

The awakened tree is a normal tree provided by ability magic
sentient and mobility.

\textit{Huge plant, misaligned}

\textbf{STRENGTH} +4

\textbf{DEXTERITY} -2

\textbf{CONSTITUTION} +2

\textbf{INTELLIGENCE} +0

\textbf{WISDOM} +0

\textbf{CHARISMA} -2

\textbf{Initiative} +0 - \textbf{Defense} 14

\textbf{Hit Points} 59 (7d12 + 14)

\textbf{Movement} 6 m

\textbf{Saving Throws}: Fortitude +6, Reflex -1, Will +1

\textbf{Damage Vulnerability} fire

\textbf{Damage Resistances} hit, piercing

\textbf{Languages} a language known by its creator

\textbf{Challenge} 2 (450 PX)

\textit{\textbf{False Appearance.}} While the tree remains immobile, it is indistinguishable from a normal tree.

\textbf{Actions}

\textit{\textbf{Slam.} Melee Weapon Attack}: +6 to hit, 3m range, one target.

\textit{Hits:} 14 (3d6 + 4) hit damage.

\medskip \textbf{Moose} \index[Mostruario]{Moose}

\textit{Large beast, misaligned}

\textbf{STRENGTH} +3

\textbf{DEXTERITY} +0

\textbf{CONSTITUTION} +1

\textbf{INTELLIGENCE} -4

\textbf{WISDOM} +0

\textbf{CHARISMA} -2

\textbf{Initiative} +0 - \textbf{Defense} 11

\textbf{Hit Points} 13 (2d10 + 2)

\textbf{Movement} 15 m

\textbf{Saving Throws}: Fortitude +4, Reflex +1, Will +0

\textbf{Languages} -

\textbf{Challenge} 1/4 (50 PX)

\textit{\textbf{Charge.}} If the moose moves at least 20 feet toward the target and hits it with a beak attack during the same turn, the target takes an additional 7 (2d6) hit damage . If the target is a creature, it must succeed at a Fortitude saving throw
DC 13 or falling prone.

\textbf{Actions}

\textit{\textbf{Rostrum.} Melee Weapon Attack}: +5 to hit, 1m range, one target.

\textit{Strikes:} 6 (1d6 + 3) hit damage.

\textit{\textbf{Hoofs.} Melee Weapon Attack}: +5 to hit, range 1 yards, a prone creature.

\textit{Strikes:} 8 (2d4 + 3) hit damage.

\medskip \textbf{Giant Elk} \index[Mostruario]{Giant Elk}

\textit{Huge beast, misaligned}

\textbf{STRENGTH} +4

\textbf{DEXTERITY} +3

\textbf{CONSTITUTION} +2

\textbf{INTELLIGENCE} -2

\textbf{WISDOM} +2

\textbf{CHARISMA} +0

\textbf{Initiative} +3 - \textbf{Defense} 15

\textbf{Hit Points} 42 (5d12 + 10)

\textbf{Movement} 18 m

\textbf{Saving Throws}: Fortitude +8, Reflex +7, Will +2

\textbf{Skills} Awareness +4

\textbf{Languages} Giant Elk, includes the Common, the Elvish and the

Silvano but cannot speak to them

\textbf{Challenge} 2 (450 PX)

\textit{\textbf{Charge.}} If the moose moves at least 20 feet toward the target and hits it with a beak attack during the same turn, the target takes an additional 7 (2d6) hit damage . If the target is a creature, it must succeed on a DC 14 Fortitude saving throw or fall prone.

\textbf{Actions}

\textit{\textbf{Rostrum.} Melee Weapon Attack}: +6 to hit, 3m range, one target.

\textit{Strikes:} 11 (2d6 + 4) piercing damage.

\textit{\textbf{Hooves.} Melee Weapon Attack}: +6 to hit, range 1 yards, a prone creature.

\textit{Strikes:} 22 (4d4 + 4) hit damage.

\medskip \textbf{Aquila} \index[Mostruario]{Aquila}

\textit{Small beast, misaligned}

\textbf{STRENGTH} -2

\textbf{DEXTERITY} +2

\textbf{CONSTITUTION} +0

\textbf{INTELLIGENCE} -4

\textbf{WISDOM} +2

\textbf{CHARISMA} -2

\textbf{Initiative} +2 - \textbf{Defense} 13

\textbf{Hit Points} 3 (1d6)

\textbf{Movement} 3m, flight 18m

\textbf{Saving Throws}: Fortitude +3, Reflex +4, Will +2

\textbf{Skills} Awareness +4

\textbf{Languages} -

\textbf{Challenge} 0 (10 PX)

\textit{\textbf{Sharpened Sight.}} The eagle has + 1d6 on sight-based Wisdom checks.

\textbf{Actions}

\textit{\textbf{Spurs.} Melee Weapon Attack}: +4 to hit, 1m range, one target.

\textit{Strikes:} 4 (1d4 + 2) slashing damage.

\medskip \textbf{Giant Eagle} \index[Mostruario]{Giant Eagle}

The giant eagle is a noble creature that speaks its own language and understands that of other races.

\textit{Large beast, good neutral}

\textbf{STRENGTH} +3

\textbf{DEXTERITY} +3

\textbf{CONSTITUTION} +1

\textbf{INTELLIGENCE} -1

\textbf{WISDOM} +2

\textbf{CHARISMA} +0

\textbf{Initiative} +3 - \textbf{Defense} 14

\textbf{Hit Points} 26 (4d10 + 4)

\textbf{Movement} 3m, flight 24m

\textbf{Saving Throws}: Fortitude +5, Reflex +7, Will +3

\textbf{Skills} Awareness +4

\textbf{Languages} Giant Eagle, understands the Common and the Auran but cannot speak them

\textbf{Challenge} 1 (200 PX)

\textit{\textbf{Sharpened Sight.}} The eagle has + 1d6 on sight-based Wisdom checks.

\textbf{Actions}

\textit{\textbf{Multiattack.}} The eagle makes two attacks: one with its beak and one with its spurs.

\textit{\textbf{Beak.} Melee Weapon Attack}: +5 to hit, 1m range, one target.

\textit{Strikes:} 6 (1d6 + 3) piercing damage.

\textit{\textbf{Spurs.} Melee Weapon Attack}: +5 to hit, 1m range, one target.

\textit{Strikes:} 10 (2d6 + 3) slashing damage.

\medskip \textbf{Vulture} \index[Mostruario]{Vulture}

\textit{Medium beast, misaligned}

\textbf{STRENGTH} -2

\textbf{DEXTERITY} +0

\textbf{CONSTITUTION} +1

\textbf{INTELLIGENCE} -4

\textbf{WISDOM} +1

\textbf{CHARISMA} -3

\textbf{Initiative} +0 - \textbf{Defense} 11

\textbf{Hit Points} 5 (1d8 + 1)

\textbf{Movement} 3m, flight 15m

\textbf{Saving Throws}: Fortitude +6, Reflex +3, Will +1; +4 against diseases

\textbf{Skills} Awareness +3

\textbf{Languages} -

\textbf{Challenge} 0 (10 PX)

\textit{\textbf{Sight and smell refined.}} The vulture has + 1d6 on Wisdom (Awareness) checks based on smell or sight.

\textit{\textbf{Pack Tactics.}} The vulture has + 1d6 to the attack roll against a creature if at least one of the vulture's allies is within 1 meter of the creature and that ally is not incapacitated.

\textbf{Actions}

\textit{\textbf{Beak.} Melee Weapon Attack}: +2 to hit, range 1m, one target.

\textit{Strikes:} 2 (1d4) piercing damage.

\medskip \textbf{Giant Vulture} \index[Mostruario]{Giant Vulture}

The giant vulture possesses superior intelligence and a malign attitude.

\textit{Large beast, neutral evil}

\textbf{STRENGTH} +2

\textbf{DEXTERITY} +0

\textbf{CONSTITUTION} +2

\textbf{INTELLIGENCE} -2

\textbf{WISDOM} +1

\textbf{CHARISMA} -2

\textbf{Initiative} +0 - \textbf{Defense} 11

\textbf{Hit Points} 22 (3d10 + 6)

\textbf{Movement} 3 m, flight 18 m

\textbf{Saving Throws}: Fortitude +10, Reflex +6, Will +3; +4 against diseases

\textbf{Skills} Awareness +3

\textbf{Languages} understands the Common but cannot speak

\textbf{Challenge} 1 (200 PX)

\textit{\textbf{Sight and Sight Refined.}} The vulture has + 1d6 on Wisdom (Awareness) checks based on smell or sight.

\textit{\textbf{Pack Tactics.}} The vulture has + 1d6 to the attack roll against a creature if at least one of the vulture's allies is within 1 meter of the creature and that ally is not incapacitated.

\textbf{Actions}

\textit{\textbf{Multiattack.}} The vulture makes two attacks: one with its beak and one with its spurs.

\textit{\textbf{Beak.} Melee Weapon Attack}: +4 hit, 1m range, one target.

\textit{Strikes:} 7 (2d4 + 2) piercing damage.

\textit{\textbf{Spurs.} Melee Weapon Attack}: +4 hit, 1m range, one target.

\textit{Strikes:} 9 (2d6 + 2) slashing damage.

\medskip \textbf{Baboon} \index[Mostruario]{Baboon}

\textit{Small beast, misaligned}

\textbf{STRENGTH} -1

\textbf{DEXTERITY} +2

\textbf{CONSTITUTION} +0

\textbf{INTELLIGENCE} -3

\textbf{WISDOM} +1

\textbf{CHARISMA} -2

\textbf{Initiative} +2 - \textbf{Defense} 13

\textbf{Hit Points} 3 (1d6)

\textbf{Movement} 9 m, climb 9 m

\textbf{Saving Throws}: Fortitude +3, Reflex +4, Will +1

\textbf{Languages} -

\textbf{Challenge} 0 (10 PX)

\textit{\textbf{Herd Tactics.}} The baboon has + 1d6 to the attack roll against a creature if at least one of the baboon's allies is within 1 meter of the creature and that ally is not incapacitated.

\textbf{Actions}

\textit{\textbf{Bite.} Melee Weapon Attack}: +1 to hit, 1m range, one target.

\textit{Strikes:} 1 (1d4 - 1) piercing damage.


\medskip \textbf{Killer Whale (Orca)} \index[Mostruario]{Orca}

\textit{Huge beast, misaligned}

\textbf{STRENGTH} +4

\textbf{DEXTERITY} +0

\textbf{CONSTITUTION} +1

\textbf{INTELLIGENCE} -4

\textbf{WISDOM} +1

\textbf{CHARISMA} -2

\textbf{Initiative} +0 - \textbf{Defense} 14

\textbf{Hit Points} 90 (12d12 + 12)

\textbf{Movement} 0 m, swim 18 m

\textbf{Saving Throws}: Fortitude +9, Reflex +8, Will +5

\textbf{Skills} Awareness +3

\textbf{Senses} blind sight 36 m

\textbf{Languages} -

\textbf{Challenge} 3 (700 PX)

\textit{\textbf{Ecolocation.}} The whale cannot use blind sight when deaf.

\textit{\textbf{Hold Breath.}} The whale can hold its breath for 30 minutes

\textit{\textbf{refined hearing.}} The whale has + 1d6 on hearing-based Wisdom (Awareness) checks.

\textbf{Actions}

\textit{\textbf{Bite.} Melee Weapon Attack}: +6 to hit, 1m range, one target.

\textit{Strikes:} 21 (5d6 + 4) piercing damage.

\medskip \textbf{Ax Beak} \index[Mostruario]{Ax Beak}

The ax-beak is a large, slender bird with no wings but with powerful legs, a wedge-shaped beak, and a bad temper.

\textit{Large beast, misaligned}

\textbf{STRENGTH} +2

\textbf{DEXTERITY} +1

\textbf{CONSTITUTION} +1

\textbf{INTELLIGENCE} -4

\textbf{WISDOM} +0

\textbf{CHARISMA} -3

\textbf{Initiative} +1 - \textbf{Defense} 12

\textbf{Hit Points} 19 (3d10 + 3)

\textbf{Movement} 15 m

\textbf{Saving Throws}: Fortitude +3, Reflex +1, Will +1

\textbf{Languages} -

\textbf{Challenge} 1/4 (50 PX)

\textbf{Actions}

\textit{\textbf{Beak.} Melee Weapon Attack}: +4 hit, 1m range, one target.

\textit{Strikes:} 6 (1d8 + 2) slashing damage.

\medskip \textbf{Camel} \index[Mostruario]{Camel}

\textit{Large beast, misaligned}

\textbf{STRENGTH} +3

\textbf{DEXTERITY} -1

\textbf{CONSTITUTION} +2

\textbf{INTELLIGENCE} -4

\textbf{WISDOM} -1

\textbf{CHARISMA} -3

\textbf{Initiative} -1 - \textbf{Defense} 10

\textbf{Hit Points} 15 (2d10 + 4)

\textbf{Movement} 15 m

\textbf{Saving Throws}: Fortitude +5, Reflex +6, Will +0

\textbf{Languages} -

\textbf{Challenge} 1/8 (25 PX)

\textbf{Actions}

\textit{\textbf{Bite.} Melee Weapon Attack}: +5 to hit, 1m range, one target.

\textit{Strikes:} 2 (1d4) hit damage.

\medskip \textbf{Dog of Death} \index[Mostruario]{Dog of Death}

The death dog is a hideous two-headed hound that roams the plains, deserts and dungeons.

\textit{Medium monstrosity, neutral evil}

\textbf{STRENGTH} +2

\textbf{DEXTERITY} +2

\textbf{CONSTITUTION} +2

\textbf{INTELLIGENCE} -4

\textbf{WISDOM} +1

\textbf{CHARISMA} -2

\textbf{Initiative} +2 - \textbf{Defense} 13

\textbf{Hit Points} 39 (6d8 + 12)

\textbf{Movement} 12 m

\textbf{Saving Throws}: Fortitude +4, Reflex +5, Will +2

\textbf{Skills} Move Silently / Hide +4, Awareness +5

\textbf{Senses} vision in the dark 36 m

\textbf{Languages} -

\textbf{Challenge} 1 (200 PX)

\textit{\textbf{Two-headed.}} The dog has + 1d6 on Wisdom (Awareness) checks and saving throws against conditions blinded, fascinated, deafened, frightened, stunned, or passed out.

\textbf{Actions}

\textit{\textbf{Multiattack.}} The dog makes two bite attacks.

\textit{\textbf{Bite.} Melee Weapon Attack}: +4 hit, 1m range, one target.

\textit{Strikes:} 5 (1d6 + 2) piercing damage. If the target is a creature, he must succeed on a DC 12 Fortitude saving throw against the disease or be poisoned until the disease is cured. After every 24 hours, the creature must re-roll its saving throw, reducing its maximum hit points by 5 (1d10) on failure. This reduction lasts until the disease is treated. The creature dies if the disease reduces its maximum hit points to 0.

\medskip \textbf{Intermittent Dog} \index[Mostruario]{Intermittent Dog}

The intermittent dog derives its name from its ability to get in and out of reality, a talent it uses to attack and avoid being attacked.

\textit{Fairy average, legal good}

\textbf{STRENGTH} +1

\textbf{DEXTERITY} +3

\textbf{CONSTITUTION} +1

\textbf{INTELLIGENCE} +0

\textbf{WISDOM} +1

\textbf{CHARISMA} +0

\textbf{Initiative} +3 - \textbf{Defense} 14

\textbf{Hit Points} 22 (4d8 + 4)

\textbf{Damage Vulnerability} cold iron

\textbf{Movement} 12 m

\textbf{Saving Throws}: Fortitude +5, Reflex +5, Will +4

\textbf{Skills} Move Silently / Hide +5, Awareness +3

\textbf{Languages} Intermittent Dog, understands the Sylvan but cannot speak him

\textbf{Challenge} 1/4 (50 PX)

\textit{\textbf{Hearing and refined smell.}} The dog has + 1d6 on Wisdom (Awareness) checks based on hearing or smell.

\textbf{Actions}

\textit{\textbf{Bite.} Melee Weapon Attack}: +3 to hit, 1m range, one target.

\textit{Strikes:} 4 (1d6 + 1) piercing damage.

\textit{\textbf{Teleport (Cooldown 4-6).}} The dog magically teleports, along with whatever it is wearing or carrying, up to 40 feet into an unoccupied space it can see. Before or after the teleport, the dog can make a bite attack.

\medskip \textbf{Goat} \index[Mostruario]{Goat}

\textit{Medium beast, misaligned}

\textbf{STRENGTH} +1

\textbf{DEXTERITY} +0

\textbf{CONSTITUTION} +0

\textbf{INTELLIGENCE} -4

\textbf{WISDOM} +0

\textbf{CHARISMA} -3

\textbf{Initiative} +0 - \textbf{Defense} 11

\textbf{Hit Points} 4 (1d8)

\textbf{Movement} 12 m

\textbf{Saving Throws}: Fortitude +1, Reflex +1, Will +0

\textbf{Languages} -

\textbf{Challenge} 0 (10 PX)

\textit{\textbf{Charge.}} If the goat moves at least 20 feet toward the target and hits with a beak attack during the same turn, the target takes an additional 2 (1d4) hit damage. If the target is a creature, it must succeed at a DC 10 Fortitude saving throw
or fall prone.

\textit{\textbf{Firm feet.}} The goat has + 1d6 on Fortitude and Reflex saving throws made against effects that would cause it to fall prone.

\textbf{Actions}

\textit{\textbf{Rostrum.} Melee Weapon Attack}: +3 to hit, range 1m, one target.

\textit{Strikes:} 3 (1d4 + 1) hit damage.

\medskip \textbf{Giant Goat} \index[Mostruario]{Giant Goat}

\textit{Large beast, misaligned}

\textbf{STRENGTH} +3

\textbf{DEXTERITY} +0

\textbf{CONSTITUTION} +1

\textbf{INTELLIGENCE} -4

\textbf{WISDOM} +1

\textbf{CHARISMA} -2

\textbf{Initiative} +0 - \textbf{Defense} 12

\textbf{Hit Points} 19 (3d10 + 3)

\textbf{Movement} 12 m

\textbf{Saving Throws}: Fortitude +4, Reflex +1, Will +1

\textbf{Languages} -

\textbf{Challenge} 1/2 (100 PX)

\textit{\textbf{Charge.}} If the goat moves at least 20 feet toward the target and hits with a beak attack during the same turn, the target takes an additional 5 (2d4) hit damage. If the target is a creature, it must succeed on a DC 13 Fortitude saving throw or fall prone.

\textit{\textbf{Firm feet.}} The goat has + 1d6 on Fortitude and Reflex saving throws made against effects that would cause it to fall prone.

\textbf{Actions}

\textit{\textbf{Rostrum.} Melee Weapon Attack}: +5 to hit, range 1m, one target.

\textit{Strikes:} 8 (2d4 + 3) hit damage.

\medskip \textbf{Horse Racing} \index[Mostruario]{Horse Racing}

\textit{Large beast, misaligned}

\textbf{STRENGTH} +3

\textbf{DEXTERITY} +0

\textbf{CONSTITUTION} +1

\textbf{INTELLIGENCE} -4

\textbf{WISDOM} +0

\textbf{CHARISMA} -2

\textbf{Initiative} +0 - \textbf{Defense} 11

\textbf{Hit Points} 13 (2d10 + 2)

\textbf{Movement} 18 m

\textbf{Saving Throws}: Fortitude +3, Reflex +1, Will +1

\textbf{Languages} -

\textbf{Challenge} 1/4 (50 PX)

\textbf{Actions}

\textit{\textbf{Hooves.} Melee Weapon Attack}: +5 to hit, 1m range, one target.

\textit{Strikes:} 8 (2d4 + 3) hit damage.

\medskip \textbf{Warhorse} \index[Mostruario]{Warhorse}

\textit{Large beast, misaligned}

\textbf{STRENGTH} +4

\textbf{DEXTERITY} +1

\textbf{CONSTITUTION} +1

\textbf{INTELLIGENCE} -4

\textbf{WISDOM} +1

\textbf{CHARISMA} -2

\textbf{Initiative} +1 - \textbf{Defense} 12 (plus possible harness)

\textbf{Hit Points} 19 (3d10 + 3)

\textbf{Movement} 18 m

\textbf{Saving Throws}: Fortitude +4, Reflex +2, Will +1

\textbf{Languages} -

\textbf{Challenge} 1/2 (100 PX)

\textit{\textbf{Overwhelming charge.}} If the horse moves at least 20 feet toward the target and hits it with a hoof attack during the same turn, the target must succeed on a DC 14 Fortitude save. or fall prone. If the target is prone, the horse can make another hoof attack against him as a bonus action.

\textbf{Actions}

\textit{\textbf{Hooves.} Melee Weapon Attack}: +6 to hit, 1m range, one target.

\textit{Strikes:} 11 (2d6 + 4) hit damage.

\medskip \textbf{draft horse} \index[Mostruario]{draft horse}

\textit{Large beast, misaligned}

\textbf{STRENGTH} +4

\textbf{DEXTERITY} +0

\textbf{CONSTITUTION} +1

\textbf{INTELLIGENCE} -4

\textbf{WISDOM} +0

\textbf{CHARISMA} -2

\textbf{Initiative} +0 - \textbf{Defense} 11

\textbf{Hit Points} 19 (3d10 + 3)

\textbf{Movement} 12 m

\textbf{Saving Throws}: Fortitude +5, Reflex +1, Will +2

\textbf{Languages} -

\textbf{Challenge} 1/4 (50 PX)

\textbf{Actions}

\textit{\textbf{Hooves.} Melee Weapon Attack}: +6 to hit, 1m range, one target.

\textit{Strikes:} 9 (2d4 + 4) hit damage.

\medskip \textbf{Giant Sea Horse} \index[Mostruario]{Giant Sea Horse}

The giant sea horse is often used as a mount by aquatic humanoids.

\textit{Large beast, misaligned}

\textbf{STRENGTH} +1

\textbf{DEXTERITY} +2

\textbf{CONSTITUTION} +0

\textbf{INTELLIGENCE} -4

\textbf{WISDOM} +1

\textbf{CHARISMA} -3

\textbf{Initiative} +2 - \textbf{Defense} 14

\textbf{Hit Points} 16 (3d10)

\textbf{Movement} 0m, swim 12m

\textbf{Saving Throws}: Fortitude +2, Reflex +3, Will +1

\textbf{Languages} -

\textbf{Challenge} 1/2 (100 PX)

\textit{\textbf{Charge.}} If the seahorse moves at least 20 feet toward the target and hits with a beak attack during the same turn, the target takes an additional 7 (2d6) hit damage. If the target is a creature, it must succeed on a DC 11 Fortitude save or fall prone.

\textit{\textbf{Breathing Water.}} The sea horse can only breathe underwater.

\textbf{Actions}

\textit{\textbf{Rostrum.} Melee Weapon Attack}: +3 to hit, range 1m, one target.

\textit{Strikes:} 4 (1d6 + 1) hit damage.

\medskip \textbf{Giant Centipede} \index{Giant Centipede}

\textit{Small beast, misaligned}

\textbf{STRENGTH} -3

\textbf{DEXTERITY} +2

\textbf{CONSTITUTION} +1

\textbf{INTELLIGENCE} -5

\textbf{WISDOM} -2

\textbf{CHARISMA} -4

\textbf{Initiative} +2 - \textbf{Defense} 14

\textbf{Hit Points} 4 (1d6 + 1)

\textbf{Movement} 9 m, climb 9 m

\textbf{Saving Throws}: Fortitude -2, Reflex +3, Will -2

\textbf{Senses} blind sight 9 m

\textbf{Languages} -

\textbf{Challenge} 1/4 (50 PX)

\textbf{Actions}

\textit{\textbf{Bite.} Melee Weapon Attack}: +4 to hit, range 1 yd, a creature.

\textit{Hits:} 4 (1d4 + 2) piercing damage and the target must succeed on a DC 11 Fortitude save or take 10 (3d6) poison damage. If the poison damage reduces the target to 0 hit points, the target is stable but remains poisoned for 1 hour, even after recovering hit points, and while poisoned in this way, he remains paralyzed.

\medskip \textbf{Deer} \index[Mostruario]{Deer}

\textit{Medium beast, misaligned}

\textbf{STRENGTH} +0

\textbf{DEXTERITY} +3

\textbf{CONSTITUTION} +0

\textbf{INTELLIGENCE} -4

\textbf{WISDOM} +2

\textbf{CHARISMA} -3

\textbf{Initiative} +3 - \textbf{Defense} 14

\textbf{Hit Points} 4 (1d8)

\textbf{Movement} 15 m

\textbf{Saving Throws}: Fortitude +2, Reflex +3, Will +2

\textbf{Languages} -

\textbf{Challenge} 0 (10 PX)

\textbf{Actions}

\textit{\textbf{Bite.} Melee Weapon Attack}: +2 to hit, 1m range, one target.

\textit{Strikes:} 2 (1d4) piercing damage.

\medskip \textbf{Boar} \index[Mostruario]{Boar}

\textit{Medium beast, misaligned}

\textbf{STRENGTH} +1

\textbf{DEXTERITY} +0

\textbf{CONSTITUTION} +1

\textbf{INTELLIGENCE} -4

\textbf{WISDOM} -1

\textbf{CHARISMA} -3

\textbf{Initiative} +0 - \textbf{Defense} 12

\textbf{Hit Points} 11 (2d8 + 2)

\textbf{Movement} 12 m

\textbf{Saving Throws}: Fortitude +2, Reflex +1, Will -1

\textbf{Languages} -

\textbf{Challenge} 1/4 (50 PX)

\textit{\textbf{Charge.}} If the boar moves at least 20 feet toward the target and hits with a fang attack during the same turn, the target takes an additional 3 (1d6) slashing damage. If the target is a creature, it must succeed at a Fortitude saving throw
DC 11 or falling prone.

\textit{\textbf{Relentless (Cooldown after 1 hour).}} If the boar takes 7 damage or less that would reduce it to 0 hit points, it drops to 1 hit point instead.

\textbf{Actions}

\textit{\textbf{Fang.} Melee Weapon Attack}: +3 to hit, 1m range, one target.

\textit{Strikes:} 4 (1d6 + 1) slashing damage.

\medskip \textbf{Giant Boar} \index[Mostruario]{Giant Boar}

\textit{Large beast, misaligned}

\textbf{STRENGTH} +3

\textbf{DEXTERITY} +0

\textbf{CONSTITUTION} +3

\textbf{INTELLIGENCE} -4

\textbf{WISDOM} -2

\textbf{CHARISMA} -3

\textbf{Initiative} +0 - \textbf{Defense} 13

\textbf{Hit Points} 42 (5d10 + 15)

\textbf{Movement} 12 m

\textbf{Saving Throws}: Fortitude +4, Reflex +2, Will +0

\textbf{Languages} -

\textbf{Challenge} 2 (450 PX)

\textit{\textbf{Charge.}} If the boar moves at least 20 feet toward the target and hits with a fang attack during the same turn, the target takes an additional 7 (2d6) slashing damage. If the target is a creature, it must succeed on a DC 13 Fortitude saving throw or fall prone.

\textit{\textbf{Relentless (Cooldown after 1 hour).}} If the boar takes 10 damage or less that would reduce it to 0 hit points, it drops to 1 hit point instead.

\textbf{Actions}

\textit{\textbf{Fang.} Melee Weapon Attack}: +5 to hit, 1m range, one target.

\textit{Strikes:} 10 (2d6 + 3) slashing damage.

\medskip \textbf{Crocodile} \index[Mostruario]{Crocodile}

\textit{Large beast, misaligned}

\textbf{STRENGTH} +2

\textbf{DEXTERITY} +0

\textbf{CONSTITUTION} +1

\textbf{INTELLIGENCE} -4

\textbf{WISDOM} +0

\textbf{CHARISMA} -3

\textbf{Initiative} +0 - \textbf{Defense} 13

\textbf{Hit Points} 19 (3d10 + 3)

\textbf{Movement} 6m, swim 9m

\textbf{Saving Throws}: Fortitude +6, Reflex +4, Will +2

\textbf{Skills} Move Silently / Hide +2

\textbf{Languages} -

\textbf{Challenge} 1/2 (100 PX)

\textit{\textbf{Hold Your Breath.}} The crocodile can hold its breath for 15 minutes.

\textbf{Actions}

\textit{\textbf{Bite.} Melee Weapon Attack}: +4 to hit, range 1 yards, a creature.

\textit{Strikes:} 7 (1d10 + 2) piercing damage, and the target is grabbed (DC 12 to flee). Until the grab is complete, the target is in the way, and the crocodile cannot bite another target.

\medskip \textbf{Giant Crocodile} \index[Mostruario]{Giant Crocodile}

\textit{Huge beast, misaligned}

\textbf{STRENGTH} +5

\textbf{DEXTERITY} -1

\textbf{CONSTITUTION} +3

\textbf{INTELLIGENCE} -4

\textbf{WISDOM} +0

\textbf{CHARISMA} -2

\textbf{Initiative} -1 - \textbf{Defense} 15

\textbf{Hit Points} 85 (9d12 + 27)

\textbf{Movement} 9m, swim 15m

\textbf{Saving Throws}: Fortitude +15, Reflex +8, Will +8

\textbf{Skills} Move Silently / Hide +5

\textbf{Languages} -

\textbf{Challenge} 5 (1.800 PX)

\textit{\textbf{Hold Your Breath.}} The crocodile can hold its breath for 30 minutes.

\textbf{Actions}

\textit{\textbf{Multiattack.}} The crocodile makes two attacks: one with its bite and one with its tail.

\textit{\textbf{Tail.} Melee Weapon Attack}: +8 to hit, 3m range, target not grasped by crocodile.

\textit{Hits:} 14 (2d8 + 5) hit damage. If the target is a creature, it must succeed on a DC 16 Fortitude saving throw or fall prone.

\textit{\textbf{Bite.} Melee Weapon Attack}: +8 to hit, 1m range, one target.

\textit{Strikes:} 21 (3d10 + 5) piercing damage, and the target is grabbed (DC 16 to escape). Until the grab is complete, the target is in the way, and the crocodile cannot bite another target.

\medskip \textbf{Raven} \index[Mostruario]{Raven}

\textit{Tiny beast, misaligned}

\textbf{STRENGTH} -4

\textbf{DEXTERITY} +2

\textbf{CONSTITUTION} -1

\textbf{INTELLIGENCE} -4

\textbf{WISDOM} +1

\textbf{CHARISMA} -2

\textbf{Initiative} +2 - \textbf{Defense} 13

\textbf{Hit Points} 1 (1d4 - 1)

\textbf{Movement} 3m, flight 15m

\textbf{Saving Throws}: Fortitude +1, Reflex +4, Will +2

\textbf{Skills} Awareness +3

\textbf{Languages} -

\textbf{Challenge} 0 (10 PX)

\textit{\textbf{Imitation.}} The crow can imitate simple sounds it has heard, such as a person's whisper, a baby's cry, or an animal's cry. A creature that hears the sound can identify it as imitation by making a DC 10 Wisdom (Survival) check.

\textbf{Actions}

\textit{\textbf{Beak.} Melee Weapon Attack}: +4 hit, 1m range, one target.

\textit{Strikes:} 1 piercing damage.

\medskip \textbf{Weasel} \index[Mostruario]{Weasel}

\textit{Tiny beast, misaligned}

\textbf{STRENGTH} -4

\textbf{DEXTERITY} +3

\textbf{CONSTITUTION} -1

\textbf{INTELLIGENCE} -4

\textbf{WISDOM} +1

\textbf{CHARISMA} -4

\textbf{Initiative} +3 - \textbf{Defense} 14

\textbf{Hit Points} 1 (1d4 - 1)

\textbf{Movement} 9 m

\textbf{Saving Throws}: Fortitude +2, Reflex +4, Will +1

\textbf{Skills} Move Silently / Hide +5, Awareness +3

\textbf{Languages} -

\textbf{Challenge} 0 (10 PX)

\textit{\textbf{Hearing and Smell refined.}} The weasel has + 1d6 on hearing or smell based Wisdom (Awareness) checks.

\textbf{Actions}

\textit{\textbf{Bite.} Melee Weapon Attack}: +5 to hit, 1m range, one target.

\textit{Strikes:} 1 piercing damage.

\medskip \textbf{Giant Weasel} \index[Mostruario]{Giant Weasel}

\textit{Medium beast, misaligned}

\textbf{STRENGTH} +0

\textbf{DEXTERITY} +3

\textbf{CONSTITUTION} +0

\textbf{INTELLIGENCE} -3

\textbf{WISDOM} +1

\textbf{CHARISMA} -3

\textbf{Initiative} +3 - \textbf{Defense} 14

\textbf{Hit Points} 9 (2d8)

\textbf{Movement} 12 m

\textbf{Saving Throws}: Fortitude +6, Reflex +7, Will +2

\textbf{Skills} Move Silently / Hide +5, Awareness +3

\textbf{Senses} vision in the dark 18 m

\textbf{Languages} -

\textbf{Challenge} 1/8 (25 PX)

\textit{\textbf{Hearing and Smell refined.}} The weasel has + 1d6 on Wisdom (Awareness) checks based on hearing or smell.

\textbf{Actions}

\textit{\textbf{Bite.} Melee Weapon Attack}: +5 to hit, 1m range, one target.

\textit{Strikes:} 5 (1d4 + 3) piercing damage.

\medskip \textbf{Elephant} \index[Mostruario]{Elephant}

\textit{Huge beast, misaligned}

\textbf{STRENGTH} +6

\textbf{DEXTERITY} -1

\textbf{CONSTITUTION} +3

\textbf{INTELLIGENCE} -4

\textbf{WISDOM} +0

\textbf{CHARISMA} -2

\textbf{Initiative} -1 - \textbf{Defense} 14

\textbf{Hit Points} 76 (8d12 + 24)

\textbf{Movement} 12 m

\textbf{Saving Throws}: Fortitude +13, Reflex +7, Will +6

\textbf{Languages} -

\textbf{Challenge} 4 (1000 PX)
\textit{\textbf{Overwhelming charge.}} If the elephant moves at least 20 feet toward a creature and hits it with a gored attack during the same turn, the target must succeed on a DC Fortitude save. 12 or fall prone. If the target is prone, the elephant can make a stomp attack against him as a bonus action.

\textbf{Actions}

\textit{\textbf{Gored.} Melee Weapon Attack}: +8 to hit, 1m range, one target.

\textit{Strikes:} 19 (3d8 + 6) piercing damage.

\textit{\textbf{Stomp.} Melee Weapon Attack}: +8 hit, 1m range, prone target.

\textit{Strikes:} 22 (3d10 + 6) hit damage.

\medskip \textbf{Hawk} \index[Mostruario]{Hawk}

\textit{Tiny beast, misaligned}

\textbf{STRENGTH} -3

\textbf{DEXTERITY} +3

\textbf{CONSTITUTION} -1

\textbf{INTELLIGENCE} -4

\textbf{WISDOM} +2

\textbf{CHARISMA} -2

\textbf{Initiative} +3 - \textbf{Defense} 14

\textbf{Hit Points} 1 (1d4 - 1)

\textbf{Movement} 3m, flight 18m

\textbf{Saving Throws}: Fortitude +2, Reflex +5, Will +2

\textbf{Skills} Awareness +4

\textbf{Languages} -

\textbf{Challenge} 0 (10 PX)

\textit{\textbf{refined sight.}} The hawk has + 1d6 on sight-based Wisdom checks.

\textbf{Actions}

\textit{\textbf{Spurs.} Melee Weapon Attack}: +5 hit, 1m range, one target.

\textit{Strikes:} 1 slashing damage.

\medskip \textbf{Blood Hawk} \index[Mostruario]{Blood Hawk}

Owing its name to its crimson feathers and aggressive nature, the blood hawk fearlessly attacks using its pointed beak.

\textit{Small beast, misaligned}

\textbf{STRENGTH} -2

\textbf{DEXTERITY} +2

\textbf{CONSTITUTION} +0

\textbf{INTELLIGENCE} -4

\textbf{WISDOM} +2

\textbf{CHARISMA} -3

\textbf{Initiative} +2 - \textbf{Defense} 13

\textbf{Hit Points} 7 (2d6)

\textbf{Movement} 3m, flight 18m

\textbf{Saving Throws}: Fortitude +3, Reflex +6, Will +3

\textbf{Skills} Awareness +4

\textbf{Languages} -

\textbf{Challenge} 1/8 (25 PX)

\textit{\textbf{Pack Tactics.}} The hawk has + 1d6 to attack rolls against a creature if at least one of the hawk's allies is within 1 meter of the creature and that ally is not incapacitated.

\textit{\textbf{refined sight.}} The hawk has + 1d6 on sight-based Wisdom checks.

\textbf{Actions}

\textit{\textbf{Beak.} Melee Weapon Attack}: +4 hit, 1m range, one target.

\textit{Strikes:} 4 (1d4 + 2) piercing damage.

\medskip \textbf{Pirana} \index[Mostruario]{Pirana}

The pirana is a sharp-toothed carnivorous fish.

\textit{Tiny beast, misaligned}

\textbf{STRENGTH} -4

\textbf{DEXTERITY} +3

\textbf{CONSTITUTION} -1

\textbf{INTELLIGENCE} -5

\textbf{WISDOM} -2

\textbf{CHARISMA} -4

\textbf{Initiative} +3 - \textbf{Defense} 14

\textbf{Hit Points} 1 (1d4 - 1)

\textbf{Movement} 0m, swim 12m

\textbf{Saving Throws}: Fortitude -4, Reflex +3, Will -2

\textbf{Senses} vision in the dark 18 m

\textbf{Languages} -

\textbf{Challenge} 0 (10 PX)

\textit{\textbf{Bloody Frenzy.}} The pirana has + 1d6 to melee attack rolls against any creature below maximum hit points.

\textit{\textbf{Breathing Water.}} The pirana can only breathe underwater.

\textbf{Actions}

\textit{\textbf{Bite.} Melee Weapon Attack}: +5 to hit, 1m range, one target.

\textit{Strikes:} 1 piercing damage.

\medskip \textbf{Cat} \index[Mostruario]{Cat}

\textit{Tiny beast, misaligned}

\textbf{STRENGTH} -4

\textbf{DEXTERITY} +2

\textbf{CONSTITUTION} +0

\textbf{INTELLIGENCE} -4

\textbf{WISDOM} +1

\textbf{CHARISMA} -2

\textbf{Initiative} +2 - \textbf{Defense} 13

\textbf{Hit Points} 2 (1d4)

\textbf{Movement} 12 m, climb 9 m

\textbf{Saving Throws}: Fortitude +1, Reflex +4, Will +1

\textbf{Skills} Move Silently / Hide +4, Awareness +3

\textbf{Languages} -

\textbf{Challenge} 0 (10 PX)

\textit{\textbf{Refined Smell.}} The cat has + 1d6 on Wisdom (Awareness) checks based on smell.

\textbf{Actions}

\textit{\textbf{Claws.} Melee Weapon Attack}: +0 hit, 1m range, one target.

\textit{Strikes:} 1 slashing damage.

\medskip \textbf{Giant Crab} \index[Mostruario]{Giant Crab}

\textit{Medium beast, misaligned}

\textbf{STRENGTH} +1

\textbf{DEXTERITY} +2

\textbf{CONSTITUTION} +0

\textbf{INTELLIGENCE} -5

\textbf{WISDOM} -1

\textbf{CHARISMA} -4

\textbf{Initiative} +2 - \textbf{Defense} 16

\textbf{Hit Points} 13 (3d8)

\textbf{Movement} 9 m, swim 9 m

\textbf{Saving Throws}: Fortitude +5, Reflex +2, Will +1

\textbf{Skills} Move Silently / Hide +4

\textbf{Senses} blind sight 9 m

\textbf{Languages} -

\textbf{Challenge} 1/8 (25 PX)

\textit{\textbf{Amphibian.}} The crab can breathe air and water.

\textbf{Actions}

\textit{\textbf{Claw (Chela).} Melee Weapon Attack}: +3 hit, 1m range, one target.

\textit{Strikes:} 4 (1d6 + 1) hit damage and the target is grabbed (DC 11 to escape). The crab has two claws, each of which can only grab one target.

\medskip \textbf{Owl} \index[Mostruario]{Owl}

\textit{Tiny beast, misaligned}

\textbf{STRENGTH} -4

\textbf{DEXTERITY} +1

\textbf{CONSTITUTION} -1

\textbf{INTELLIGENCE} -4

\textbf{WISDOM} +1

\textbf{CHARISMA} -2

\textbf{Initiative} +1 - \textbf{Defense} 12

\textbf{Hit Points} 1 (1d4 - 1)

\textbf{Movement} 1m, flight 18m

\textbf{Saving Throws}: Fortitude +2, Reflex +5, Will +2

\textbf{Skills} Move Silently / Hide +3, Awareness +3

\textbf{Senses} vision in the dark 36 m

\textbf{Languages} -

\textbf{Challenge} 0 (10 PX)

\textit{\textbf{Fly over.}} The owl does not provoke attacks of opportunity when flying out of range of an enemy.

\textit{\textbf{Sharpened hearing and sight.}} The owl has + 1d6 on Wisdom (Awareness) checks based on hearing or sight.

\textbf{Actions}

\textit{\textbf{Spurs.} Melee Weapon Attack}: +3 to hit, 1m range, one target.

\textit{Strikes:} 1 slashing damage.

\medskip \textbf{Giant Owl} \index[Mostruario]{Giant Owl}

Giant owls are intelligent creatures that protect the sylvan kingdoms.

\textit{Large beast, neutral}

\textbf{STRENGTH} +1

\textbf{DEXTERITY} +2

\textbf{CONSTITUTION} +1

\textbf{INTELLIGENCE} -1

\textbf{WISDOM} +1

\textbf{CHARISMA} +0

\textbf{Initiative} +2 - \textbf{Defense} 13

\textbf{Hit Points} 19 (3d10 + 3)

\textbf{Movement} 1m, flight 18m

\textbf{Saving Throws}: Fortitude +1, Reflex +4, Will +1

\textbf{Skills} Move Silently / Hide +4, Awareness +5

\textbf{Senses} vision in the dark 36 m

\textbf{Languages} Giant Owl, includes Common, Elvish and Sylvan but cannot speak them

\textbf{Challenge} 1/4 (50 PX)

\textit{\textbf{Fly over.}} The owl does not provoke attacks of opportunity when flying out of range of an enemy.

\textit{\textbf{Sharpened hearing and sight.}} The owl has + 1d6 on Wisdom (Awareness) checks based on hearing or sight.

\textbf{Actions}

\textit{\textbf{Spurs.} Melee Weapon Attack}: +3 hit, 1m range, one target.

\textit{Strikes:} 8 (2d6 + 1) piercing damage.

\medskip \textbf{Hyena} \index[Mostruario]{Hyena}

\textit{Medium beast, misaligned}

\textbf{STRENGTH} +0

\textbf{DEXTERITY} +1

\textbf{CONSTITUTION} +1

\textbf{INTELLIGENCE} -4

\textbf{WISDOM} +1

\textbf{CHARISMA} -3

\textbf{Initiative} +1 - \textbf{Defense} 12

\textbf{Hit Points} 5 (1d8 + 1)

\textbf{Movement} 15 m

\textbf{Saving Throws}: Fortitude +5, Reflex +5, Will +1

\textbf{Skills} Awareness +3

\textbf{Languages} -

\textbf{Challenge} 0 (10 PX)

\textit{\textbf{Pack Tactics.}} The hyena has + 1d6 to attack rolls against a creature if at least one of the hyena's allies is within 1 meter of the creature and that ally is not incapacitated.

\textbf{Actions}

\textit{\textbf{Bite.} Melee Weapon Attack}: +2 to hit, 1m range, one target.

\textit{Strikes:} 3 (1d6) piercing damage.

\medskip \textbf{Giant Hyena} \index[Mostruario]{Giant Hyena}

\textit{Large beast, misaligned}

\textbf{STRENGTH} +3

\textbf{DEXTERITY} +2

\textbf{CONSTITUTION} +2

\textbf{INTELLIGENCE} -4

\textbf{WISDOM} +1

\textbf{CHARISMA} -2

\textbf{Initiative} +2 - \textbf{Defense} 13

\textbf{Hit Points} 45 (6d10 + 12)

\textbf{Movement} 15 m

\textbf{Saving Throws}: Fortitude +6, Reflex +6, Will +2

\textbf{Skills} Awareness +3

\textbf{Languages} -

\textbf{Challenge} 1 (200 PX)

\textit{\textbf{Anger.}} When the hyena reduces a creature to 0 hit points with a melee attack during its round, the hyena can take a bonus action to move up to half its movement and take a bite attack.

\textbf{Actions}

\textit{\textbf{Bite.} Melee Weapon Attack}: +5 hit, 1m range, one target.

\textit{Strikes:} 10 (2d6 + 3) piercing damage.

\medskip \textbf{Lion} \index[Mostruario]{Lion}

\textit{Large beast, misaligned}

\textbf{STRENGTH} +3

\textbf{DEXTERITY} +2

\textbf{CONSTITUTION} +1

\textbf{INTELLIGENCE} -4

\textbf{WISDOM} +1

\textbf{CHARISMA} -1

\textbf{Initiative} +2 - \textbf{Defense} 13

\textbf{Hit Points} 26 (4d10 + 4)

\textbf{Movement} 15 m

\textbf{Saving Throws}: Fortitude +6, Reflex +7, Will +2

\textbf{Skills} Move Silently / Hide +6, Awareness +3

\textbf{Languages} -

\textbf{Challenge} 1 (200 PX)

\textit{\textbf{Leap.}} If the lion moves at least 20 feet toward a creature and hits it with a claw attack during the same turn, the target must succeed on a DC 13 Fortitude save or fall prone. If the target is prone, the lion can make a
bite attack as a bonus action.

\textit{\textbf{Refined Smell.}} The lion has + 1d6 on Wisdom (Awareness) checks based on smell.

\textit{\textbf{Run Jump.}} With a 3 meter run up, the lion can jump up to 7 meters long.

\textit{\textbf{Pack Tactics.}} The lion has + 1d6 to attack rolls against a creature if at least one of the lion's allies is within 1 meter of the creature and that ally is incapacitated.

\textbf{Actions}

\textit{\textbf{Claw.} Melee Weapon Attack}: +5 to hit, 1m range, one target.

\textit{Strikes:} 6 (1d6 + 3) slashing damage, 1 bleed damage.

\textit{\textbf{Bite.} Melee Weapon Attack}: +5 to hit, 1m range, one target.

\textit{Strikes:} 7 (1d8 + 3) piercing damage.

\medskip \textbf{Lizard} \index[Mostruario]{Lizard}

\textit{Tiny beast, misaligned}

\textbf{STRENGTH} -4

\textbf{DEXTERITY} +0

\textbf{CONSTITUTION} +0

\textbf{INTELLIGENCE} -5

\textbf{WISDOM} -1

\textbf{CHARISMA} -4

\textbf{Initiative} +0 - \textbf{Defense} 11

\textbf{Hit Points} 2 (1d4)

\textbf{Movement} 6m, climb 6m

\textbf{Saving Throws}: Fortitude +1, Reflex +4, Will +1

\textbf{Senses} vision in the dark 9 m

\textbf{Languages} -

\textbf{Challenge} 0 (10 PX)

\textit{\textbf{Climbing like Spider.}} The lizard can climb difficult surfaces, including standing upside down on the ceiling, without the need to make a skill check.

\textbf{Actions}

\textit{\textbf{Bite.} Melee Weapon Attack}: +0 hit, 1m range, one target.

\textit{Strikes:} 1 piercing damage.

\medskip \textbf{Giant Lizard} \index[Mostruario]{Giant Lizard}

Giant lizards are fearsome predators and are often used as mounts or draft animals by reptilian humanoids and underground residents.

\textit{Large beast, misaligned}

\textbf{STRENGTH} +2

\textbf{DEXTERITY} +1

\textbf{CONSTITUTION} +1

\textbf{INTELLIGENCE} -4

\textbf{WISDOM} +0

\textbf{CHARISMA} -3

\textbf{Initiative} +1 - \textbf{Defense} 13

\textbf{Hit Points} 19 (3d10 + 3)

\textbf{Movement} 9 m, climb 9 m

\textbf{Saving Throws}: Fortitude +11, Reflex +8, Will +4

\textbf{Senses} vision in the dark 9 m

\textbf{Languages} -

\textbf{Challenge} 1/4 (50 PX)

\textbf{Actions}

\textit{\textbf{Bite.} Melee Weapon Attack}: +4 hit, 1m range, one target.

\textit{Strikes:} 6 (1d8 + 2) piercing damage.

\textbf{VARIANT}

Some giant lizards possess one or both of the following traits.

\textit{\textbf{Climbing like Spider.}} The lizard can climb difficult surfaces, including standing upside down on the ceiling, without the need for a skill check.

\textit{\textbf{Hold Breath.}} The lizard can hold its breath for 15 minutes. (A lizard with this trait also possesses a 30-foot swimming speed.)

\medskip \textbf{Wolf} \index[Mostruario]{Wolf}

\textit{Medium beast, misaligned}

\textbf{STRENGTH} +1

\textbf{DEXTERITY} +2

\textbf{CONSTITUTION} +1

\textbf{INTELLIGENCE} -4

\textbf{WISDOM} +1

\textbf{CHARISMA} -2

\textbf{Initiative} +2 - \textbf{Defense} 14

\textbf{Hit Points} 11 (2d8 + 2)

\textbf{Movement} 12 m

\textbf{Saving Throws}: Fortitude +5, Reflex +5, Will +1

\textbf{Skills} Move Silently / Hide +4, Awareness +3

\textbf{Languages} -

\textbf{Challenge} 1/4 (50 PX)

\textit{\textbf{Hearing and refined smell.}} The wolf has + 1d6 on Wisdom (Awareness) checks based on hearing or smell.

\textit{\textbf{Pack Tactics.}} The wolf has + 1d6 to attack rolls against a creature if at least one of the wolf's allies is within 1 meter of the creature and that ally is not incapacitated.

\textbf{Actions}

\textit{\textbf{Bite.} Melee Weapon Attack}: +4 hit, 1m range, one target.

\textit{Strikes:} 7 (2d4 + 2) piercing damage. If the target is a creature, it must succeed on a DC 11 Fortitude saving throw or fall prone.

\medskip \textbf{Dinolupo (Direwolf)} \index[Mostruario]{Dinolupo (Direwolf}

\textit{Large beast, misaligned}

\textbf{STRENGTH} +3

\textbf{DEXTERITY} +2

\textbf{CONSTITUTION} +2

\textbf{INTELLIGENCE} -2

\textbf{WISDOM} +1

\textbf{CHARISMA} -2

\textbf{Initiative} +2 - \textbf{Defense} 15

\textbf{Hit Points} 37 (5d10 + 10)

\textbf{Movement} 15 m

\textbf{Saving Throws}: Fortitude +7, Reflex +6, Will +2

\textbf{Skills} Move Silently / Hide +4, Awareness +3

\textbf{Languages} -

\textbf{Challenge} 1 (200 PX)

\textit{\textbf{Hearing and refined smell.}} The wolf has + 1d6 on Wisdom (Awareness) checks based on hearing or smell.

\textit{\textbf{Pack Tactics.}} The wolf has + 1d6 to attack rolls against a creature if at least one of the wolf's allies is within 1 meter of the creature and that ally is not incapacitated.

\textbf{Actions}

\textit{\textbf{Bite.} Melee Weapon Attack}: +5 to hit, 1m range, one target.

\textit{Strikes:} 10 (2d6 + 3) piercing damage. If the target is a creature, it must succeed on a DC 13 Fortitude saving throw or fall prone.

\medskip \textbf{Winter Wolf} \index[Mostruario]{Winter Wolf}

Winter wolves inhabit arctic regions and are evil, intelligent creatures with snow-white coats and ice-colored eyes.

\textit{Large monstrosity, neutral evil}

\textbf{STRENGTH} +4

\textbf{DEXTERITY} +1

\textbf{CONSTITUTION} +2

\textbf{INTELLIGENCE} -2

\textbf{WISDOM} +1

\textbf{CHARISMA} -1

\textbf{Initiative} +1 - \textbf{Defense} 15

\textbf{Hit Points} 75 (10d10 + 20)

\textbf{Movement} 15 m

\textbf{Saving Throws}: Fortitude +9, Reflex +6, Will +3

\textbf{Skills} Move Silently / Hide +3, Awareness +5

\textbf{Immunity to Damage} cold

\textbf{Languages} Common, Giant, Winter Wolf

\textbf{Challenge} 3 (700 PX)

\textit{\textbf{Snow camouflage.}} The wolf has + 1d6 on Dexterity (Hide) checks made to hide on snowy ground.

\textit{\textbf{Hearing and refined smell.}} The wolf has + 1d6 on Wisdom (Awareness) checks based on hearing or smell.

\textit{\textbf{Pack Tactics.}} The wolf has + 1d6 to attack rolls against a creature if at least one of the wolf's allies is within 1 meter of the creature and that ally is not incapacitated.

\textbf{Actions}

\textit{\textbf{Bite.} Melee Weapon Attack}: +6 to hit, 1m range, one target.

\textit{Strikes:} 11 (2d6 + 4) piercing damage. If the target is a creature, it must succeed on a DC 14 Fortitude saving throw or fall prone.

\textit{\textbf{Frost Breath (Cooldown 5-6).}} The wolf exhales a blast of icy wind in a 5 meter cone. Each creature in that area must make a DC 12 Reflex saving throw, and take 18 (4d8) cold damage if it fails the saving throw, or half that damage if it succeeds.

\medskip \textbf{Mammoth} \index[Mostruario]{Mammoth}

The mammoth is an elephant-like creature with thick fur and long tusks.

\textit{Huge beast, misaligned}

\textbf{STRENGTH} +7

\textbf{DEXTERITY} -1

\textbf{CONSTITUTION} +5

\textbf{INTELLIGENCE} -4

\textbf{WISDOM} +0

\textbf{CHARISMA} -2

\textbf{Initiative} -1 - \textbf{Defense} 16

\textbf{Hit Points} 126 (11d12 + 55)

\textbf{Movement} 12 m

\textbf{Saving Throws}: Fortitude +14, Reflex +10, Will +7

\textbf{Languages} -

\textbf{Challenge} 6 (2.300 PX)

\textit{\textbf{Overwhelming Charge.}} If the mammoth moves at least 20 feet toward a creature and hits it with a gore attack during the same turn, the target must succeed on a DC 18 Fortitude save. or fall prone. If the target is prone, the mammoth can make a stomp attack against it as a bonus action.

\textbf{Actions}

\textit{\textbf{Gored.} Melee Weapon Attack}: +10 to hit, 3m range, one target.

\textit{Strikes:} 25 (4d8 + 7) piercing damage.

\textit{\textbf{Stomp.} Melee Weapon Attack}: +10 to hit, range 1m, a prone creature.

\textit{Strikes:} 29 (4d10 + 7) hit damage.

\medskip \textbf{Mastiff} \index[Mostruario]{Mastiff}

\textbf{I} Mastiffs are impressive hounds prized by humanoids for their reality and refined senses.

\textit{Medium beast, misaligned}

\textbf{STRENGTH} +1

\textbf{DEXTERITY} +2

\textbf{CONSTITUTION} +1

\textbf{INTELLIGENCE} -4

\textbf{WISDOM} +1

\textbf{CHARISMA} -2

\textbf{Initiative} +2 - \textbf{Defense} 13

\textbf{Hit Points} 5 (1d8 + 1)

\textbf{Movement} 12 m

\textbf{Saving Throws}: Fortitude +3, Reflex +3, Will +1

\textbf{Skills} Awareness +3, Survival (Follow Trail) +3

\textbf{Languages} -

\textbf{Challenge} 1/8 (25 PX)

\textit{\textbf{Hearing and refined smell.}} The hound has + 1d6 on hearing or smell based Wisdom (Awareness) checks.

\textbf{Actions}

\textit{\textbf{Bite.} Melee Weapon Attack}: +3 to hit, 1m range, one target.

\textit{Strikes:} 4 (1d6 + 1) piercing damage. If the target is a creature, it must succeed on a DC 11 Fortitude saving throw or fall prone.

\medskip \textbf{Mule} \index[Mostruario]{Mule}

\textit{Medium beast, misaligned}

\textbf{STRENGTH} +2

\textbf{DEXTERITY} +0

\textbf{CONSTITUTION} +1

\textbf{INTELLIGENCE} -4

\textbf{WISDOM} +0

\textbf{CHARISMA} -3

\textbf{Initiative} +0 - \textbf{Defense} 11

\textbf{Hit Points} 11 (2d8 + 2)

\textbf{Movement} 12 m

\textbf{Saving Throws}: Fortitude +3, Reflex +1, Will +1

\textbf{Languages} -

\textbf{Challenge} 1/8 (25 PX)

\textit{\textbf{Beast of Pack.}} The mule is considered a Large animal for purposes of determining its carrying capacity.

\textit{\textbf{Firm feet.}} The mule has + 1d6 on Fortitude and Reflex saving throws made against effects that would cause it to fall prone.

\textbf{Actions}

\textit{\textbf{Hooves.} Melee Weapon Attack}: +2 to hit, 1m range, one target.

\textit{Strikes:} 4 (1d4 + 2) hit damage.

\medskip \textbf{Brown Bear} \index[Mostruario]{Brown Bear}

\textit{Large beast, misaligned}

\textbf{STRENGTH} +4

\textbf{DEXTERITY} +0

\textbf{CONSTITUTION} +3

\textbf{INTELLIGENCE} -4

\textbf{WISDOM} +1

\textbf{CHARISMA} -2

\textbf{Initiative} +0 - \textbf{Defense} 12

\textbf{Hit Points} 34 (4d10 + 12)

\textbf{Movement} 12 m, climb 9 m

\textbf{Saving Throws}: Fortitude +6, Reflex +2, Will +3

\textbf{Skills} Awareness +3

\textbf{Languages} -

\textbf{Challenge} 1 (200 PX)

\textit{\textbf{Refined Smell.}} The bear has + 1d6 on Wisdom (Awareness) checks based on smell.

\textbf{Actions}

\textit{\textbf{Multiattack.}} The bear makes two attacks: one with its bite and one with its claws.

\textit{\textbf{Claws.} Melee Weapon Attack}: +5 to hit, 1m range, one target.

\textit{Strikes:} 11 (2d6 + 4) slashing damage.

\textit{\textbf{Bite.} Melee Weapon Attack}: +5 to hit, 1m range, one target.

\textit{Strikes:} 8 (1d8 + 4) piercing damage.

\medskip \textbf{Black Bear} \index[Mostruario]{Black Bear}

\textit{Medium beast, misaligned}

\textbf{STRENGTH} +2

\textbf{DEXTERITY} +0

\textbf{CONSTITUTION} +2

\textbf{INTELLIGENCE} -4

\textbf{WISDOM} +1

\textbf{CHARISMA} -2

\textbf{Initiative} +0 - \textbf{Defense} 12

\textbf{Hit Points} 19 (3d8 + 6)

\textbf{Movement} 12 m, climb 9 m

\textbf{Saving Throws}: Fortitude +4, Reflex +1, Will +1

\textbf{Skills} Awareness +3

\textbf{Languages} -

\textbf{Challenge} 1/2 (100 PX)

\textit{\textbf{Refined Smell.}} The bear has + 1d6 on Wisdom (Awareness) checks based on smell.

\textbf{Actions}

\textit{\textbf{Multiattack.}} The black bear makes two attacks: one with its bite and one with its claws.

\textit{\textbf{Claws.} Melee Weapon Attack}: +3 to hit, 1m range, one target.

\textit{Strikes:} 7 (2d4 + 2) slashing damage, 1 bleed damage.

\textit{\textbf{Bite.} Melee Weapon Attack}: +3 to hit, 1m range, one target.

\textit{Strikes:} 5 (1d6 + 2) piercing damage.

\medskip \textbf{Polar Bear} \index[Mostruario]{Polar Bear}

\textit{Large beast, misaligned}

\textbf{STRENGTH} +5

\textbf{DEXTERITY} +0

\textbf{CONSTITUTION} +3

\textbf{INTELLIGENCE} -4

\textbf{WISDOM} +1

\textbf{CHARISMA} -2

\textbf{Initiative} +0 - \textbf{Defense} 13

\textbf{Hit Points} 42 (5d10 + 15)

\textbf{Movement} 12 m, swim 9 m

\textbf{Saving Throws}: Fortitude +10, Reflex +7, Will +4

\textbf{Skills} Awareness +3

\textbf{Languages} -

\textbf{Challenge} 2 (450 PX)

\textit{\textbf{Refined Smell.}} The bear has + 1d6 on Wisdom (Awareness) checks based on smell.

\textbf{Actions}

\textit{\textbf{Multiattack.}} The bear makes two attacks: one with its bite and one with its claws.

\textit{\textbf{Claws.} Melee Weapon Attack}: +7 to hit, 1m range, one target.

\textit{Strikes:} 12 (2d6 + 5) slashing damage.

\textit{\textbf{Bite.} Melee Weapon Attack}: +7 to hit, 1m range, one target.

\textit{Strikes:} 9 (1d8 + 5) piercing damage.

\textbf{VARIANT: CAVE BEAR} \index[Mostruario]{Cave bear}

Some bears have adapted to life underground. They have the same stats as polar bears, but with 18m dark vision.

\medskip \textbf{Pantera} \index[Mostruario]{Pantera}

\textit{Medium beast, misaligned}

\textbf{STRENGTH} +2

\textbf{DEXTERITY} +2

\textbf{CONSTITUTION} +0

\textbf{INTELLIGENCE} -4

\textbf{WISDOM} +2

\textbf{CHARISMA} -2

\textbf{Initiative} +2 - \textbf{Defense} 13

\textbf{Hit Points} 13 (3d8)

\textbf{Movement} 15m, climb 12m

\textbf{Saving Throws}: Fortitude +3, Reflex +5, Will +3

\textbf{Skills} Move Silently / Hide +6, Awareness +4

\textbf{Languages} -

\textbf{Challenge} 1/4 (50 PX)

\textit{\textbf{Leap.}} If the panther moves at least 20 feet toward a creature and hits it with a claw attack during the same turn, the target must succeed on a DC 12 Fortitude saving throw or fall prone. If the target is prone, the panther can make a bite attack against it as a bonus action.

\textit{\textbf{Refined Smell.}} The panther has + 1d6 on Wisdom (Awareness) checks based on smell.

\textbf{Actions}

\textit{\textbf{Claw.} Melee Weapon Attack}: +4 hit, 1m range, one target.

\textit{Strikes:} 4 (1d4 + 2) slashing damage, 1 bleed damage.

\textit{\textbf{Bite.} Melee Weapon Attack}: +4 hit, 1m range, one target.

\textit{Strikes:} 5 (1d6 + 2) piercing damage.


\medskip \textbf{Pony} \index[Mostruario]{Pony}

\textit{Medium beast, misaligned}

\textbf{STRENGTH} +2

\textbf{DEXTERITY} +0

\textbf{CONSTITUTION} +1

\textbf{INTELLIGENCE} -4

\textbf{WISDOM} +0

\textbf{CHARISMA} -2

\textbf{Initiative} +0 - \textbf{Defense} 11

\textbf{Hit Points} 11 (2d8 + 2)

\textbf{Movement} 12 m

\textbf{Saving Throws}: Fortitude +5, Reflex +4, Will +0

\textbf{Languages} -

\textbf{Challenge} 1/8 (25 PX)

\textbf{Actions}

\textit{\textbf{Hooves.} Melee Weapon Attack}: +4 hit, 1m range, one target.

\textit{Strikes:} 7 (2d4 + 2) hit damage.

\medskip \textbf{Spider} \index[Mostruario]{Spider}

\textit{Tiny beast, misaligned}

\textbf{STRENGTH} 2 (-5)

\textbf{DEXTERITY} +2

\textbf{CONSTITUTION} -1

\textbf{INTELLIGENCE} -5

\textbf{WISDOM} +0

\textbf{CHARISMA} -4

\textbf{Initiative} +2 - \textbf{Defense} 13

\textbf{Hit Points} 1 (1d4 - 1)

\textbf{Movement} 6m, climb 6m

\textbf{Saving Throws}: Fortitude -4, Reflex +2, Will -4

\textbf{Skills} Move Silently / Hide +4

\textbf{Senses} vision in the dark 9 m

\textbf{Languages} -

\textbf{Challenge} 0 (10 PX)

\textit{\textbf{Walking on the Web.}} The spider ignores movement restrictions caused by webs.

\textit{\textbf{Climbing as a Spider.}} The spider can climb difficult surfaces, including standing upside down on the ceiling, without the need for a skill check.

\textit{\textbf{Sense of the Web.}} While in contact with a web, the spider knows the exact location of any other creature in contact with the same web.

\textbf{Actions}

\textit{\textbf{Bite.} Melee Weapon Attack}: +4 to hit, range 1 yd, a creature.

\textit{Strikes:} 1 point of piercing damage and the target must succeed on a 9 Fortitude saving throw or take 2 (1d4) points of poison damage.

\medskip \textbf{Phase Spider} \index[Mostruario]{Phase Spider}

The phase spider possesses the magical ability to enter and exit the Ethereal Plane. It appears to appear out of nowhere and quickly disappears after attacking.

\textit{Large monstrosity, misaligned}

\textbf{STRENGTH} +2

\textbf{DEXTERITY} +2

\textbf{CONSTITUTION} +1

\textbf{INTELLIGENCE} -2

\textbf{WISDOM} +0

\textbf{CHARISMA} -2

\textbf{Initiative} +2 - \textbf{Defense} 15

\textbf{Hit Points} 32 (5d10 + 5)

\textbf{Movement} 9 m, climb 9 m

\textbf{Saving Throws}: Fortitude +8, Reflex +8, Will +3

\textbf{Skills} Move Silently / Hide +6

\textbf{Senses} vision in the dark 18 m

\textbf{Languages} -

\textbf{Challenge} 3 (700 PX)

\textit{\textbf{Walk the Web.}} The spider ignores movement restrictions caused by webs.

\textit{\textbf{Aether Travel.}} As a bonus action, the spider can magically move from the Material Plane to the Ethereal Plane, or vice versa.

\textit{\textbf{Climbing as a Spider.}} The spider can climb difficult surfaces, including standing upside down on the ceiling, without the need to make a skill check.

\textbf{Actions}

\textit{\textbf{Bite.} Melee Weapon Attack}: +4 to hit, range 1 yards, a creature.

\textit{Strikes:} 7 (1d10 + 2) piercing damage and the target must make a DC 11 Fortitude saving throw, and take 18 (4d8) poison damage on a failed save, or half this damage if he succeeds. If the poison damage reduces the target to 0 hit points, the target is stable but poisoned for 1 hour, even after recovering hit points, and while poisoned in this way, the target is paralyzed.

\medskip \textbf{Giant Spider} \index[Mostruario]{Giant Spider}

\textit{Large beast, misaligned}

\textbf{STRENGTH} +2

\textbf{DEXTERITY} +3

\textbf{CONSTITUTION} +1

\textbf{INTELLIGENCE} -4

\textbf{WISDOM} +0

\textbf{CHARISMA} -3

\textbf{Initiative} +2 - \textbf{Defense} 15

\textbf{Hit Points} 26 (4d10 + 4)

\textbf{Movement} 9 m, climb 9 m

\textbf{Saving Throws}: Fortitude +4, Reflex +4, Will +1

\textbf{Skills} Move Silently / Hide +7

\textbf{Senses} blind sight 3 m, vision in the dark 18 m

\textbf{Languages} -

\textbf{Challenge} 1 (200 PX)

\textit{\textbf{Walk the Web.}} The spider ignores movement restrictions caused by webs.

\textit{\textbf{Climbing as a Spider.}} The spider can climb difficult surfaces, including standing upside down on the ceiling, without the need to make a skill check.

\textit{\textbf{Sense of the Web.}} While in contact with a web, the spider knows the exact location of any other creature in contact with the same web.

\textbf{Actions}

\textit{\textbf{Bite.} Melee Weapon Attack}: +5 to hit, range 1 yards, a creature.

\textit{Strikes:} 7 (1d8 + 3) piercing damage and the target must make a DC 11 Fortitude saving throw, and suffer 9

(2d8) poison damage on a failed save, or half that damage on a successful save. If the poison damage reduces the target to 0 hit points, the target is stable but poisoned for 1 hour, even after recovering hit points, and while poisoned in this way, the target is paralyzed.

\textit{\textbf{Web (Cooldown 5-6).} Ranged Weapon Attack}: +5 to hit, range 9m, a creature.

\textit{Strikes:} The target is hampered by the web. With an action, the entangled target can make a DC 12 Strength check and, if successful, break the web. The web can also be attacked and destroyed (Defense 10; HP 5; vulnerability to fire damage; immunity to hit and poison damage).

\medskip \textbf{Giant Wolf Spider} \index[Mostruario]{Giant Wolf Spider}

A giant wolf spider hunts for prey in open ground or hides in burrows or crevices in the ground to ambush.

\textit{Medium beast, misaligned}

\textbf{STRENGTH} +1

\textbf{DEXTERITY} +3

\textbf{CONSTITUTION} +1

\textbf{INTELLIGENCE} -4

\textbf{WISDOM} +1

\textbf{CHARISMA} -3

\textbf{Initiative} +3 - \textbf{Defense} 14

\textbf{Hit Points} 11 (2d8 + 2)

\textbf{Movement} 12 m, climb 12 m

\textbf{Saving Throws}: Fortitude +2, Reflex +4, Will +1

\textbf{Skills} Move Silently / Hide +7, Awareness +3

\textbf{Senses} blind sight 3 m, vision in the dark 18 m

\textbf{Languages} -

\textbf{Challenge} 1/4 (50 PX)

\textit{\textbf{Walk the Web.}} The spider ignores movement restrictions caused by webs.

\textit{\textbf{Climbing as a Spider.}} The spider can climb difficult surfaces, including standing upside down on the ceiling, without the need for a skill check.

\textit{\textbf{Sense of the Web.}} While in contact with a web, the spider knows the exact location of any other creature in contact with the same web.

\textbf{Actions}

\textit{\textbf{Bite.} Melee Weapon Attack}: +3 to hit, range 1 yards, a creature.

\textit{Strikes:} 4 (1d6 + 1) piercing damage and the target must make a DC 11 Fortitude saving throw, and take 7 (2d6) poison damage on a failed save, or half that damage if he succeeds. If the poison damage reduces the target to 0 hit points, the target is stable but poisoned for 1 hour, even after recovering hit points, and while poisoned in this way, the target is paralyzed.

\medskip \textbf{Frog} \index[Mostruario]{Frog}

\textit{Tiny beast, misaligned}

\textbf{STRENGTH} -5

\textbf{DEXTERITY} +1

\textbf{CONSTITUTION} -1

\textbf{INTELLIGENCE} -5

\textbf{WISDOM} -1

\textbf{CHARISMA} -4

\textbf{Initiative} +1 - \textbf{Defense} 12

\textbf{Hit Points} 1 (1d4 - 1)

\textbf{Movement} 6m, swim 6m

\textbf{Saving Throws}: Fortitude -4, Reflex +1, Will -2

\textbf{Skills} Move Silently / Hide +3, Awareness +1

\textbf{Senses} vision in the dark 9 m

\textbf{Languages} -

\textbf{Challenge} 0 (0 PX)

\textit{\textbf{Amphibian.}} The frog can breathe air and water.

\textit{\textbf{Leap from Stationary.}} A frog can jump up to 3 meters long and up to 1 meter high, with or without a run-up.

A \textbf{frog} is devoid of attacks. It feeds on small insects and usually lives near marshes, inside trees or underground.

\medskip \textbf{Giant Frog} \index[Mostruario]{Giant Frog}

\textit{Medium beast, misaligned}

\textbf{STRENGTH} +1

\textbf{DEXTERITY} +1

\textbf{CONSTITUTION} +0

\textbf{INTELLIGENCE} -4

\textbf{WISDOM} +0

\textbf{CHARISMA} -4

\textbf{Initiative} +1 - \textbf{Defense} 12

\textbf{Hit Points} 18 (4d8)

\textbf{Movement} 9 m, swim 9 m

\textbf{Saving Throws}: Fortitude +2, Reflex +2, Will +0

\textbf{Skills} Move Silently / Hide +3, Awareness +2

\textbf{Senses} vision in the dark 9 m

\textbf{Languages} -

\textbf{Challenge} 1/4 (50 PX)

\textit{\textbf{Amphibian.}} The frog can breathe air and water.

\textit{\textbf{Leap from Stationary.}} A frog can jump up to 20 feet long and up to 10 feet high, with or without a run-up.

\textbf{Actions}

\textit{\textbf{Bite.} Melee Weapon Attack}: +3 to hit, 1m range, one target.

\textit{Strikes:} 4 (1d6 + 1) piercing damage and the target is grabbed (DC 11 to escape). Until the grapple is complete, the target is in the way, and the frog cannot use the bite against another target.

\textit{\textbf{Swallow.}} The frog makes a bite attack against a Small or smaller target it is grabbing. If the attack hits, the target is engulfed, and the grab ends. The engulfed target is blinded and entangled, has full cover against attacks and other effects outside the frog, and takes 5 (2d4) acid damage at the start of each frog's turn. The frog can only swallow one target at a time. If the frog dies, a swallowed creature is no longer hampered by it and can exit the corpse using 1 meter of movement, coming out prone.

\medskip \textbf{Rat} \index[Mostruario]{Rat}

\textit{Tiny beast, misaligned}

\textbf{STRENGTH} -4

\textbf{DEXTERITY} +0

\textbf{CONSTITUTION} -1

\textbf{INTELLIGENCE} -4

\textbf{WISDOM} +0

\textbf{CHARISMA} -3

\textbf{Initiative} +0 - \textbf{Defense} 11

\textbf{Hit Points} 1 (1d4 - 1)

\textbf{Movement} 6 m

\textbf{Saving Throws}: Fortitude -4, Reflex +0, Will +0

\textbf{Senses} vision in the dark 9 m

\textbf{Languages} -

\textbf{Challenge} 0 (10 PX)

\textit{\textbf{Refined Smell.}} The rat has + 1d6 on Wisdom (Awareness) checks based on smell.

\textbf{Actions}

\textit{\textbf{Bite.} Melee Weapon Attack}: +0 hit, 1m range, one target.

\textit{Strikes:} 1 piercing damage.

\medskip \textbf{Giant Rat} \index[Mostruario]{Giant Rat}

\textit{Small beast, misaligned}

\textbf{STRENGTH} -2

\textbf{DEXTERITY} +2

\textbf{CONSTITUTION} +0

\textbf{INTELLIGENCE} -4

\textbf{WISDOM} +0

\textbf{CHARISMA} -3

\textbf{Initiative} +2 - \textbf{Defense} 13

\textbf{Hit Points} 7 (2d6)

\textbf{Movement} 9 m

\textbf{Saving Throws}: Fortitude +3, Reflex +5, Will +1

\textbf{Senses} vision in the dark 18 m

\textbf{Languages} -

\textbf{Challenge} 1/8 (25 PX)

\textit{\textbf{Refined Smell.}} The rat has + 1d6 on Wisdom (Awareness) checks based on smell.

\textit{\textbf{Pack Tactics.}} The rat has + 1d6 to the attack roll against a creature if at least one of the rat's allies is within 1 meter of the creature and that ally is not incapacitated.

\textbf{Actions}

\textit{\textbf{Bite.} Melee Weapon Attack}: +4 hit, 1m range, one target.

\textit{Strikes:} 4 (1d4 + 2) piercing damage.

\textbf{VARIANT: SICK GIANT RAT} \index[Mostruario]{Sick Giant Rat}

Some giant rats carry a terrible disease that they spread through their bite. A sick giant rat has a challenge rating of 1/8 (25 XP) and the following action instead of its normal bite attack.

\textit{\textbf{Bite.} Melee Weapon Attack}: +4 hit, 1m range, one target.

\textit{Strikes:} 4 (1d4 + 2) piercing damage. If the target is a creature, he must succeed on a DC 10 Fortitude saving throw or contract a disease. Until the disease is cured, the target cannot recover hit points except through magical methods, and the target's maximum hit points decrease by 3 (1d6) every 24 hours. If the target's maximum hit points drop to 0 as a result of the disease, the target dies.

\medskip \textbf{Rhinoceros} \index[Mostruario]{Rhinoceros}

\textit{Large beast, misaligned}

\textbf{STRENGTH} +5

\textbf{DEXTERITY} -1

\textbf{CONSTITUTION} +2

\textbf{INTELLIGENCE} -4

\textbf{WISDOM} +1

\textbf{CHARISMA} -2

\textbf{Initiative} -1 - \textbf{Defense} 12

\textbf{Hit Points} 45 (6d10 + 12)

\textbf{Movement} 12 m

\textbf{Saving Throws}: Fortitude +10, Reflex +4, Will +2

\textbf{Languages} -

\textbf{Challenge} 2 (450 PX)

\textit{\textbf{Charge.}} If the rhino moves at least 20 feet toward a target and hits it with a gore attack during the same turn, the target takes an additional 9 (2d8) hit damage. If the target is a creature, it must succeed on a DC 15 Fortitude save or fall prone.

\textbf{Actions}

\textit{\textbf{Gored.} Melee Weapon Attack}: +7 to hit, 1m range, one target.

\textit{Strikes:} 14 (2d8 + 5) hit damage.

\medskip \textbf{Giant Toad} \index[Mostruario]{Giant Toad}

\textit{Large beast, misaligned}

\textbf{STRENGTH} +2

\textbf{DEXTERITY} +1

\textbf{CONSTITUTION} +1

\textbf{INTELLIGENCE} -4

\textbf{WISDOM} +0

\textbf{CHARISMA} -4

\textbf{Initiative} +1 - \textbf{Defense} 12

\textbf{Hit Points} 39 (6d10 + 6)

\textbf{Movement} 6m, swim 12m

\textbf{Saving Throws}: Fortitude +6, Reflex +6, Will +0

\textbf{Senses} vision in the dark 9 m

\textbf{Languages} -

\textbf{Challenge} 1 (200 PX)

\textit{\textbf{Amphibian.}} The toad can breathe air and water.

\textit{\textbf{Leap from Stationary.}} A toad can jump up to 20 feet long and up to 10 feet high, with or without a run-up.

\textbf{Actions}

\textit{\textbf{Bite.} Melee Weapon Attack}: +4 hit, 1m range, one target.

\textit{Strikes:} 7 (1d10 + 2) piercing damage plus 5 (1d10) poison damage, and the target is grabbed (DC 13 to flee). Until the grab is complete, the target is in the way, and the toad cannot bite another target.

\textit{\textbf{Swallow.}} The toad makes a bite attack against a Medium or smaller target it is grabbing. If the attack hits, the target is engulfed, and the grab ends. The engulfed target is blinded and entangled, has full cover against attacks and other effects outside the frog, and takes 10 (3d6) acid damage at the start of each toad's turn. The toad can only swallow one target at a time.

If the toad dies, a swallowed creature is no longer hampered by it and can move out of the corpse using 1 meter of movement, coming out prone.

\medskip \textbf{Giant Fire Beetle} \index[Mostruario]{Giant Fire Beetle}

A giant fire beetle is a nocturnal creature that possesses a pair of light glands capable of emitting light for 1d6 days after the beetle's death.

\textit{Small beast, misaligned}

\textbf{STRENGTH} -1

\textbf{DEXTERITY} +0

\textbf{CONSTITUTION} +1

\textbf{INTELLIGENCE} -5

\textbf{WISDOM} -2

\textbf{CHARISMA} -4

\textbf{Initiative} +0 - \textbf{Defense} 14

\textbf{Hit Points} 4 (1d6 + 1)

\textbf{Movement} 9 m

\textbf{Saving Throws}: Fortitude +2, Reflex +0, Will +0

\textbf{Senses} blind sight 9 m

\textbf{Languages} -

\textbf{Challenge} 0 (10 PX)

\textit{\textbf{Illumination.}} The scarab radiates bright light within a 10-foot radius and dim light for an additional 10-feet.

\textbf{Actions}

\textit{\textbf{Bite.} Melee Weapon Attack}: +1 to hit, 1m range, one target.

\textit{Strikes:} 2 (1d6 - 1) slashing damage.

\medskip \textbf{Jackal} \index[Mostruario]{Jackal}

\textit{Small beast, misaligned}

\textbf{STRENGTH} -1

\textbf{DEXTERITY} +2

\textbf{CONSTITUTION} +0

\textbf{INTELLIGENCE} -4

\textbf{WISDOM} +1

\textbf{CHARISMA} -2

\textbf{Initiative} +2 - \textbf{Defense} 13

\textbf{Hit Points} 3 (1d6)

\textbf{Movement} 12 m

\textbf{Saving Throws}: Fortitude -1, Reflex +3, Will +1

\textbf{Skills} Awareness +3

\textbf{Languages} -

\textbf{Challenge} 0 (10 PX)

\textit{\textbf{Pack Tactics.}} The jackal has + 1d6 to attack rolls against a creature if at least one of the jackal's allies is within 1 meter of the creature and that ally is not incapacitated.

\textit{\textbf{Hearing and refined smell.}} The jackal has + 1d6 on hearing or smell based Wisdom (Awareness) checks.

\textbf{Actions}

\textit{\textbf{Bite.} Melee Weapon Attack}: +1 to hit, 1m range, one target.

\textit{Strikes:} 1 (1d4 - 1) piercing damage.

\medskip \textbf{Swarms} \index[Mostruario]{Swarms}

The swarms presented below are not normal or benign gatherings of small creatures. Instead, they form as a result of an external, often malignant influence. Druids are also unable to charm these swarms, and their aggression is almost unnatural.

\textbf{Centipede Swarm} \index[Mostruario]{Centipede Swarm}

\textit{Medium swarm of Tiny beasts, misaligned}

\textbf{STRENGTH} -4

\textbf{DEXTERITY} +1

\textbf{CONSTITUTION} +0

\textbf{INTELLIGENCE} -5

\textbf{WISDOM} -2

\textbf{CHARISMA} -5

\textbf{Initiative} +1 - \textbf{Defense} 13

\textbf{Hit Points} 22 (5d8)

\textbf{Movement} 6m, climb 6m

\textbf{Saving Throws}: Fortitude -1, Reflex +3, Will +1

\textbf{Damage Resistances} hit, piercing, cutting

\textbf{Condition Immunity} fascinated, grabbed, entangled, paralyzed, petrified, prone, frightened, stunned

\textbf{Senses} blind sight 3 m

\textbf{Languages} -

\textbf{Challenge} 1/2 (100 PX)

\textit{\textbf{Swarm.}} The swarm can occupy the space of another creature and vice versa, and the swarm can move through any opening large enough for a Tiny insect. The swarm cannot recover hit points or gain temporary hit points.

\textbf{Actions}

\textit{\textbf{Bites.} Melee Weapon Attack}: +3 to hit, 0m range, a target in swarm space.

\textit{Hits:} 10 (4d4) piercing damage, or 5 (2d4) piercing damage if the swarm has half or less of its hit points. A creature reduced to 0 hit points by a swarm of centipedes and stable remains poisoned for 1 hour, even after recovering hit points, and becomes paralyzed by the poison during this time.

\medskip \textbf{Swarm of Crows} \index[Mostruario]{Swarm of Crows}

\textit{Medium swarm of Tiny beasts, misaligned}

\textbf{STRENGTH} -2

\textbf{DEXTERITY} +2

\textbf{CONSTITUTION} -1

\textbf{INTELLIGENCE} -4

\textbf{WISDOM} +1

\textbf{CHARISMA} -2

\textbf{Initiative} +2 - \textbf{Defense} 13

\textbf{Hit Points} 24 (7d8 - 7)

\textbf{Movement} 3m, flight 15m

\textbf{Saving Throws}: Fortitude -1, Reflex +3, Will +2

\textbf{Skills} Awareness +5

\textbf{Damage Resistances} slashing, piercing, cutting

\textbf{Condition Immunity} fascinated, grabbed, hampered, paralyzed, petrified, prone, frightened, stunned

\textbf{Languages} -

\textbf{Challenge} 1/4 (50 PX)

\textit{\textbf{Swarm.}} The swarm can occupy the space of another creature and vice versa, and the swarm can move through any opening large enough for a Tiny Raven. The swarm cannot recover hit points or gain temporary hit points.

\textbf{Actions}

\textit{\textbf{Beaks.} Melee Weapon Attack}: +4 hit, 1m range, a target in swarm space.

\textit{Strikes:} 7 (2d6) piercing damage, or 3 (1d6) piercing damage if the swarm has half or less of its hit points.

\medskip \textbf{Swarm of Pirana} \index[Mostruario]{Swarm of Pirana}

\textit{Medium swarm of Tiny beasts, misaligned}

\textbf{STRENGTH} +1

\textbf{DEXTERITY} +3

\textbf{CONSTITUTION} -1

\textbf{INTELLIGENCE} -5

\textbf{WISDOM} -2

\textbf{CHARISMA} -4

\textbf{Initiative} +3 - \textbf{Defense} 14

\textbf{Hit Points} 28 (8d8 - 8)

\textbf{Movement} 0m, swim 12m

\textbf{Saving Throws}: Fortitude -3, Reflex +4, Will -1

\textbf{Damage Resistances} slashing, piercing, cutting

\textbf{Condition Immunity} fascinated, grabbed, hampered, paralyzed, petrified, prone, frightened, stunned

\textbf{Senses} vision in the dark 18 m

\textbf{Languages} -

\textbf{Challenge} 1 (200 PX)

\textit{\textbf{Blood Frenzy.}} The swarm has + 1d6 to melee attack rolls against any creature that is not at maximum hit points.

\textit{\textbf{Breathing Water.}} The swarm can only breathe underwater.

\textit{\textbf{Swarm.}} The swarm can occupy the space of another creature and vice versa, and the swarm can move through any opening large enough for a Tiny pirana. The swarm cannot recover hit points or gain temporary hit points.

\textbf{Actions}

\textit{\textbf{Bites.} Melee Weapon Attack}: +5 to hit, 0 yards range, a creature in swarm space.

\textit{Strikes:} 14 (4d6) piercing damage, or 7 (2d6) piercing damage if the swarm has half or less of its hit points.

\medskip \textbf{Swarm of Insects} \index[Mostruario]{Swarm of Insects}

\textit{Medium swarm of Tiny beasts, misaligned}

\textbf{STRENGTH} -4

\textbf{DEXTERITY} +1

\textbf{CONSTITUTION} +0

\textbf{INTELLIGENCE} -5

\textbf{WISDOM} -2

\textbf{CHARISMA} -5

\textbf{Initiative} +1 - \textbf{Defense} 13

\textbf{Hit Points} 22 (5d8)

\textbf{Movement} 6 m, climb 6 m

\textbf{Saving Throws}: Fortitude -3, Reflex +2, Will -1

\textbf{Damage Resistances} hit, piercing, cutting

\textbf{Condition Immunity} fascinated, grabbed, hampered, paralyzed, petrified, prone, frightened, stunned

\textbf{Senses} blind sight 3 m

\textbf{Languages} -

\textbf{Challenge} 1/2 (100 PX)

\textit{\textbf{Swarm.}} The swarm can occupy the space of another creature and vice versa, and the swarm can move through any opening large enough for a Tiny insect. The swarm cannot recover hit points or gain temporary hit points.

\textbf{Actions}

\textit{\textbf{Bites.} Melee Weapon Attack}: +3 to hit, 0m range, a target in swarm space.

\textit{Hits:} 10 (4d4) piercing damage, or 5 (2d4) piercing damage if the swarm has half or less of its hit points.

\medskip \textbf{Swarm of Bats} \index[Mostruario]{Swarm of Bats}

\textit{Medium swarm of Tiny beasts, misaligned}

\textbf{STRENGTH} -3

\textbf{DEXTERITY} +2

\textbf{CONSTITUTION} +0

\textbf{INTELLIGENCE} -4

\textbf{WISDOM} +1

\textbf{CHARISMA} -3

\textbf{Initiative} +2 - \textbf{Defense} 13

\textbf{Hit Points} 22 (5d8)

\textbf{Movement} 0 m, flight 9 m

\textbf{Saving Throws}: Fortitude -2, Reflex +4, Will +2

\textbf{Damage Resistances} slashing, piercing, cutting

\textbf{Condition Immunity} fascinated, grabbed, entangled, paralyzed, petrified, prone, frightened, stunned

\textbf{Senses} blind sight 18 m

\textbf{Languages} -

\textbf{Challenge} 1/4 (50 PX)

\textit{\textbf{Echolocation.}} The swarm cannot use blind sight if deaf.

\textit{\textbf{Swarm.}} The swarm can occupy the space of another creature and vice versa, and the swarm can move through any opening large enough for a Tiny bat. The swarm cannot recover hit points or gain temporary hit points.

\textit{\textbf{refined hearing.}} The swarm has + 1d6 on hearing-based Wisdom (Awareness) checks.

\textbf{Actions}

\textit{\textbf{Bites.} Melee Weapon Attack}: +4 to hit, 0 yards range, creature in swarm space.

\textit{Strikes:} 5 (2d4) piercing damage, or 2 (1d4) piercing damage if the swarm has half or less of its hit points.

\medskip \textbf{Swarm of Spiders} \index[Mostruario]{Swarm of Spiders}

\textit{Medium swarm of Tiny beasts, misaligned}

\textbf{STRENGTH} -4

\textbf{DEXTERITY} +1

\textbf{CONSTITUTION} +0

\textbf{INTELLIGENCE} -5

\textbf{WISDOM} -2

\textbf{CHARISMA} -5

\textbf{Initiative} +1 - \textbf{Defense} 13

\textbf{Hit Points} 22 (5d8)

\textbf{Movement} 6m, climb 6m

\textbf{Saving Throws}: Fortitude -3, Reflex +2, Will -1

\textbf{Damage Resistances} slashing, piercing, cutting

\textbf{Condition Immunity} fascinated, grabbed, hampered, paralyzed, petrified, prone, frightened, stunned

\textbf{Senses} blind sight 3 m

\textbf{Languages} -

\textbf{Challenge} 1/2 (100 PX)

\textit{\textbf{Walk on the Web.}} The swarm ignores movement restrictions caused by webs.

\textit{\textbf{Climbing like Spider.}} The swarm can scale difficult surfaces, including standing upside down on the ceiling, without the need to make a skill check.

\textit{\textbf{Sense of the Web.}} While in contact with a web, the swarm knows the exact location of any other creature in contact with the same web.

\textit{\textbf{Swarm.}} The swarm can occupy the space of another creature and vice versa, and the swarm can move through any opening large enough for a Tiny insect. The swarm cannot recover hit points or gain temporary hit points.

\textbf{Actions}

\textit{\textbf{Bites.} Melee Weapon Attack}: +3 to hit, 0m range, a target in swarm space.

\textit{Hits:} 10 (4d4) piercing damage, or 5 (2d4) piercing damage if the swarm has half or less of its hit points.

\medskip \textbf{Swarm of Rats} \index[Mostruario]{Swarm of Rats}

\textit{Medium swarm of Tiny beasts, misaligned}

\textbf{STRENGTH} -1

\textbf{DEXTERITY} +0

\textbf{CONSTITUTION} -1

\textbf{INTELLIGENCE} -4

\textbf{WISDOM} +0

\textbf{CHARISMA} -4

\textbf{Initiative} +0 - \textbf{Defense} 11

\textbf{Hit Points} 24 (7d8 - 7)

\textbf{Movement} 9 m

\textbf{Saving Throws}: Fortitude +0, Reflex +1, Will +1

\textbf{Damage Resistances} slashing, piercing, cutting

\textbf{Condition Immunity} fascinated, grabbed, entangled, paralyzed, petrified, prone, frightened, stunned

\textbf{Senses} vision in the dark 9 m

\textbf{Languages} -

\textbf{Challenge} 1/4 (50 PX)

\textit{\textbf{Refined Smell.}} The swarm has + 1d6 on Wisdom (Awareness) checks based on smell.

\textit{\textbf{Swarm.}} The swarm can occupy the space of another creature and vice versa, and the swarm can move through any opening large enough for a Tiny rat. The swarm cannot recover hit points or gain temporary hit points.

\textbf{Actions}

\textit{\textbf{Bites.} Melee Weapon Attack}: +2 to hit, 0 yards range, a target in swarm space.

\textit{Strikes:} 7 (2d6) piercing damage, or 3 (1d6) piercing damage if the swarm has half or less of its hit points.

\medskip \textbf{Scarab Swarm} \index{Scarab Swarm}

\textit{Medium swarm of Tiny beasts, misaligned}

\textbf{STRENGTH} -4

\textbf{DEXTERITY} +1

\textbf{CONSTITUTION} +0

\textbf{INTELLIGENCE} -5

\textbf{WISDOM} -2

\textbf{CHARISMA} -5

\textbf{Initiative} +1 - \textbf{Defense} 13

\textbf{Hit Points} 22 (5d8)

\textbf{Movement} 6 m, climb 6 m, Burrow 6 m

\textbf{Saving Throws}: Fortitude -3, Reflex +2, Will -1

\textbf{Damage Resistances} slashing, piercing, cutting

\textbf{Condition Immunity} fascinated, grabbed, entangled, paralyzed, petrified, prone, frightened, stunned

\textbf{Senses} blind sight 3 m

\textbf{Languages} -

\textbf{Challenge} 1/2 (100 PX)

\textit{\textbf{Swarm.}} The swarm can occupy the space of another creature and vice versa, and the swarm can move through any opening large enough for a Tiny insect. The swarm cannot recover hit points or gain temporary hit points.

\textbf{Actions}

\textit{\textbf{Bites.} Melee Weapon Attack}: +3 to hit, 0m range, a target in swarm space.

\textit{Hits:} 10 (4d4) piercing damage, or 5 (2d4) piercing damage if the swarm has half or less of its hit points.

\medskip \textbf{Swarm of Poisonous Snakes} \index{Swarm of Poisonous Snakes}

\textit{Medium swarm of Tiny beasts, misaligned}

\textbf{STRENGTH} -1

\textbf{DEXTERITY} +4

\textbf{CONSTITUTION} +0

\textbf{INTELLIGENCE} -5

\textbf{WISDOM} +0

\textbf{CHARISMA} -4

\textbf{Initiative} +4 - \textbf{Defense} 15

\textbf{Hit Points} 36 (8d8)

\textbf{Movement} 9 m, swim 9 m

\textbf{Saving Throws}: Fortitude +0, Reflex +5, Will +1

\textbf{Damage Resistances} slashing, piercing, cutting

\textbf{Condition Immunity} fascinated, grabbed, hampered, paralyzed, petrified, prone, frightened, stunned

\textbf{Senses} blind sight 3 m

\textbf{Languages} -

\textbf{Challenge} 2 (450 PX)

\textit{\textbf{Swarm.}} The swarm can occupy the space of another creature and vice versa, and the swarm can move through any opening large enough for a Tiny snake. The swarm cannot recover hit points or gain temporary hit points.

\textbf{Actions}

\textit{\textbf{Bites.} Melee Weapon Attack}: +6 to hit, 0 yards range, a creature in swarm space.

\textit{Strikes:} 7 (2d6) piercing damage, or 3 (1d6) piercing damage if the swarm has half or less of its hit points, and the target must make a DC 10 Fortitude saving throw, and suffer 14 (4d6) poison damage on a failed save, or half that damage on a successful save.

\medskip \textbf{Swarm of Wasps} \index{Swarm of Poisonous Snakes}

\textit{Medium swarm of Tiny beasts, misaligned}

\textbf{STRENGTH} -4

\textbf{DEXTERITY} +1

\textbf{CONSTITUTION} +0

\textbf{INTELLIGENCE} -5

\textbf{WISDOM} -2

\textbf{CHARISMA} -5

\textbf{Initiative} +1 - \textbf{Defense} 13

\textbf{Hit Points} 22 (5d8)

\textbf{Movement} 1m, flight 9m

\textbf{Saving Throws}: Fortitude -3, Reflex +2, Will -1

\textbf{Damage Resistances} hit, piercing, cutting

\textbf{Condition Immunity} fascinated, grabbed, entangled, paralyzed, petrified, prone, frightened, stunned

\textbf{Senses} blind sight 3 m

\textbf{Languages} -

\textbf{Challenge} 1/2 (100 PX)

\textit{\textbf{Swarm.}} The swarm can occupy the space of another creature and vice versa, and the swarm can move through any opening large enough for a Tiny insect. The swarm cannot recover hit points or gain temporary hit points.

\textbf{Actions}

\textit{\textbf{Bites.} Melee Weapon Attack}: +3 to hit, 0m range, a target in swarm space.

\textit{Hits:} 10 (4d4) piercing damage, or 5 (2d4) piercing damage if the swarm has half or less of its hit points.

\medskip \textbf{Monkey} \index[Mostruario]{Monkey}

\textit{Medium beast, misaligned}

\textbf{STRENGTH} +3

\textbf{DEXTERITY} +2

\textbf{CONSTITUTION} +2

\textbf{INTELLIGENCE} -2

\textbf{WISDOM} +1

\textbf{CHARISMA} -2

\textbf{Initiative} +2 - \textbf{Defense} 13

\textbf{Hit Points} 19 (3d8 + 6)

\textbf{Movement} 9 m, climb 9 m

\textbf{Saving Throws}: Fortitude +3, Reflex +3, Will +2

\textbf{Skills} Acrobatics +5, Awareness +3

\textbf{Languages} -

\textbf{Challenge} 1/2 (100 PX)

\textbf{Actions}

\textit{\textbf{Multiattack.}} The ape makes two fist attacks.

\textit{\textbf{Punch.} Melee Weapon Attack}: +5 to hit, 1m range, one target.

\textit{Strikes:} 6 (1d6 + 3) hit damage.

\textit{\textbf{Rock.} Ranged Weapon Attack}: +5 to hit, range 8m, one target.

\textit{Strikes:} 6 (1d6 + 3) hit damage.

\medskip \textbf{Giant Ape} \index[Mostruario]{Giant Ape}

\textit{Huge beast, misaligned}

\textbf{STRENGTH} +6

\textbf{DEXTERITY} +2

\textbf{CONSTITUTION} +4

\textbf{INTELLIGENCE} -2

\textbf{WISDOM} +1

\textbf{CHARISMA} -2

\textbf{Initiative} +2 - \textbf{Defense} 16

\textbf{Hit Points} 157 (15d12 + 60)

\textbf{Movement} 12 m, climb 12 m

\textbf{Saving Throws}: Fortitude +7, Reflex +6, Will +4

\textbf{Skills} Acrobatics +9, Awareness +4

\textbf{Languages} -

\textbf{Challenge} 7 (2.900 PX)

\textbf{Actions}

\textit{\textbf{Multiattack.}} The ape makes two fist attacks.

\textit{\textbf{Punch.} Melee Weapon Attack}: +9 to hit, 3m range, one target.

\textit{Hits:} 22 (3d10 + 6) hit damage.

\textit{\textbf{Rock.} Ranged Weapon Attack}: +9 to hit, range 15m, one target.

\textit{Strikes:} 30 (7d6 + 6) hit damage.

\medskip \textbf{Scorpio} \index[Mostruario]{Scorpio}

\textit{Tiny beast, misaligned}

\textbf{STRENGTH} -4

\textbf{DEXTERITY} +0

\textbf{CONSTITUTION} -1

\textbf{INTELLIGENCE} -5

\textbf{WISDOM} -1

\textbf{CHARISMA} -4

\textbf{Initiative} +0 - \textbf{Defense} 12

\textbf{Hit Points} 1 (1d4 - 1)

\textbf{Movement} 3 m

\textbf{Saving Throws}: Fortitude -3, Reflex +2, Will -1

\textbf{Senses} blind sight 3 m

\textbf{Languages} -

\textbf{Challenge} 0 (10 PX)

\textbf{Actions}

\textit{\textbf{Sting.} Melee Weapon Attack}: +2 to hit, range 1 yards, a creature.

\textit{Strikes:} 1 piercing damage and the target must make a Fortitude saving throw DC 9, and take 4 (1d8) poison damage on a failed save, or half that damage on a successful one.

\medskip \textbf{Giant Scorpion} \index[Mostruario]{Giant Scorpion}

\textit{Large beast, misaligned}

\textbf{STRENGTH} +2

\textbf{DEXTERITY} +1

\textbf{CONSTITUTION} +2

\textbf{INTELLIGENCE} -5

\textbf{WISDOM} -1

\textbf{CHARISMA} -4

\textbf{Initiative} +1 - \textbf{Defense} 17

\textbf{Hit Points} 52 (7d10 + 14)

\textbf{Movement} 12 m

\textbf{Saving Throws}: Fortitude +7, Reflex +1, Will +1

\textbf{Senses} blind sight 18 m

\textbf{Languages} -

\textbf{Challenge} 3 (700 PX)

\textbf{Actions}

\textit{\textbf{Multiattack.}} The scorpion makes three attacks: two with its claws and one with its sting.

\textit{\textbf{Claw.} Melee Weapon Attack}: +4 hit, 1m range, one target.

\textit{Strikes:} 6 (1d8 + 2) hit damage and the target is grabbed (DC 12 to escape). The scorpion has two claws, each of which can only grab one target.

\textit{\textbf{Sting.} Melee Weapon Attack}: +4 to hit, range 1 yards, a creature.

\textit{Hits:} 7 (1d10 + 2) piercing damage and the target must make a Fortitude saving throw DC 12, and take 22 (4d10) poison damage on a failed save, or half that damage if he succeeds.

\medskip \textbf{Constricting Serpent} \index[Mostruario]{Constricting Snake}

\textit{Large beast, misaligned}

\textbf{STRENGTH} +2

\textbf{DEXTERITY} +2

\textbf{CONSTITUTION} +1

\textbf{INTELLIGENCE} -5

\textbf{WISDOM} +0

\textbf{CHARISMA} -4

\textbf{Initiative} +2 - \textbf{Defense} 13

\textbf{Hit Points} 13 (2d10 + 2)

\textbf{Movement} 9 m, swim 9 m

\textbf{Saving Throws}: Fortitude +3, Reflex +2, Will +0

\textbf{Senses} blind sight 3 m

\textbf{Languages} -

\textbf{Challenge} 1/4 (50 PX)

\textbf{Actions}

\textit{\textbf{Bite.} Melee Weapon Attack}: +4 to hit, range 1 yards, a creature.

\textit{Strikes:} 5 (1d6 + 2) piercing damage.

\textit{\textbf{Construct.} Melee Weapon Attack}: +4 to hit, range 1 yards, a creature.

\textit{Strikes:} 6 (1d8 + 2) hit damage, and the target is grabbed (DC 14 to escape). Until the grab is complete, the creature is entangled, and the snake cannot constrict another target.

\medskip \textbf{Giant Constrictor Serpent} \index[Mostruario]{Giant Constrictor Serpent}

\textit{Huge beast, misaligned}

\textbf{STRENGTH} +4

\textbf{DEXTERITY} +2

\textbf{CONSTITUTION} +1

\textbf{INTELLIGENCE} -5

\textbf{WISDOM} +0

\textbf{CHARISMA} -4

\textbf{Initiative} +2 - \textbf{Defense} 13

\textbf{Hit Points} 60 (8d12 + 8)

\textbf{Movement} 9 m, swim 9 m

\textbf{Saving Throws}: Fortitude +3, Reflex +2, Will +0

\textbf{Skills} Awareness +2

\textbf{Senses} blind sight 3 m

\textbf{Languages} -

\textbf{Challenge} 2 (450 PX)

\textbf{Actions}

\textit{\textbf{Bite.} Melee Weapon Attack}: +6 to hit, range 10 feet, a creature.

\textit{Strikes:} 11 (2d6 + 4) piercing damage.

\textit{\textbf{Construct.} Melee Weapon Attack}: +6 to hit, range 1 yards, a creature.

\textit{Strikes:} 13 (2d8 + 4) hit damage, and the target is grabbed (DC 16 to escape). Until the grab is complete, the creature is entangled, and the snake cannot constrict another target.

\medskip \textbf{Poisonous Snake} \index[Mostruario]{Poisonous Snake}

\textit{Tiny beast, misaligned}

\textbf{STRENGTH} -4

\textbf{DEXTERITY} +3

\textbf{CONSTITUTION} +0

\textbf{INTELLIGENCE} -5

\textbf{WISDOM} +0

\textbf{CHARISMA} -4

\textbf{Initiative} +3 - \textbf{Defense} 14

\textbf{Hit Points} 2 (1d4)

\textbf{Movement} 9 m, swim 9 m

\textbf{Saving Throws}: Fortitude +1, Reflex +4, Will +1

\textbf{Senses} blind sight 3 m

\textbf{Languages} -

\textbf{Challenge} 1/8 (25 PX)

\textbf{Actions}

\textit{\textbf{Bite.} Melee Weapon Attack}: +5 to hit, 1m range, one target.

\textit{Strikes:} 1 piercing damage and the target must make a DC 10 Fortitude saving throw, and take 5 (2d4) poison damage on a failed save, or half that damage on a successful one.

\medskip \textbf{Giant Venomous Snake} \index[Mostruario]{Giant Venomous Snake}

\textit{Medium beast, misaligned}

\textbf{STRENGTH} +0

\textbf{DEXTERITY} +4

\textbf{CONSTITUTION} +1

\textbf{INTELLIGENCE} -4

\textbf{WISDOM} +0

\textbf{CHARISMA} -4

\textbf{Initiative} +4 - \textbf{Defense} 15

\textbf{Hit Points} 11 (2d8 + 2)

\textbf{Movement} 9 m, swim 9 m

\textbf{Saving Throws}: Fortitude +1, Reflex +5, Will +2

\textbf{Skills} Awareness +2

\textbf{Senses} blind sight 3 m

\textbf{Languages} -

\textbf{Challenge} 1/4 (50 PX)

\textbf{Actions}

\textit{\textbf{Bite.} Melee Weapon Attack}: +6 to hit, 3 yards range, one target.

\textit{Hits:} 6 (1d4 + 4) piercing damage and the target must make a DC 11 Fortitude save, and take 10 (3d6) poison damage on a failed save, or half that damage if he succeeds.

\medskip \textbf{Flying Serpent} \index[Mostruario]{Flying Snake}

A flying snake is a winged, brightly colored snake found in remote jungles.

\textit{Tiny beast, misaligned}

\textbf{STRENGTH} -3

\textbf{DEXTERITY} +4

\textbf{CONSTITUTION} +0

\textbf{INTELLIGENCE} -4

\textbf{WISDOM} +1

\textbf{CHARISMA} -3

\textbf{Initiative} +4 - \textbf{Defense} 15

\textbf{Hit Points} 5 (2d4)

\textbf{Movement} 9m, swim 9m, fly 18m

\textbf{Saving Throws}: Fortitude -2, Reflex +5, Will +1

\textbf{Senses} blind sight 3 m

\textbf{Languages} -

\textbf{Challenge} 1/8 (25 PX)

\textit{\textbf{Fly over.}} The snake does not provoke attacks of opportunity when flying out of range of an enemy.

\textbf{Actions}

\textit{\textbf{Bite.} Melee Weapon Attack}: +6 to hit, 1m range, one target.

\textit{Strikes:} 1 piercing damage plus 7 (3d4) poison damage.

\medskip \textbf{Hunter Shark} \index[Mostruario]{Hunter Shark}

A hunter shark is 4 to 6 meters long and usually hunts solo in deeper waters.

\textit{Large beast, misaligned}

\textbf{STRENGTH} +4

\textbf{DEXTERITY} +1

\textbf{CONSTITUTION} +2

\textbf{INTELLIGENCE} -5

\textbf{WISDOM} +0

\textbf{CHARISMA} -3

\textbf{Initiative} +1 - \textbf{Defense} 13

\textbf{Hit Points} 45 (6d10 + 12)

\textbf{Movement} 0m, swim 12m

\textbf{Saving Throws}: Fortitude +4, Reflex +2, Will +0

\textbf{Skills} Awareness +2

\textbf{Senses} blind sight 9 m

\textbf{Languages} -

\textbf{Challenge} 2 (450 PX)

\textit{\textbf{Bloody Frenzy.}} The shark has + 1d6 to melee attack rolls against any creature below maximum hit points.

\textit{\textbf{Breathing Water.}} The shark can only breathe underwater.

\textbf{Actions}

\textit{\textbf{Bite.} Melee Weapon Attack}: +6 to hit, 1m range, one target.

\textit{Strikes:} 13 (2d8 + 4) piercing damage.

\medskip \textbf{Coral Shark} \index[Mostruario]{Coral Shark}

Coral sharks are 2 to 3 meters long and live in shallower waters and along coral reefs.

\textit{Medium beast, misaligned}

\textbf{STRENGTH} +2

\textbf{DEXTERITY} +1

\textbf{CONSTITUTION} +1

\textbf{INTELLIGENCE} -5

\textbf{WISDOM} +0

\textbf{CHARISMA} -3

\textbf{Initiative} +1 - \textbf{Defense} 13

\textbf{Hit Points} 22 (4d8 + 4)

\textbf{Movement} 0m, swim 12m

\textbf{Saving Throws}: Fortitude +2, Reflex +2, Will +1

\textbf{Skills} Awareness +2

\textbf{Senses} blind sight 9 m

\textbf{Languages} -

\textbf{Challenge} 1/2 (100 PX)

\textit{\textbf{Breathing Water.}} The shark can only breathe underwater.

\textit{\textbf{Pack Tactics.}} The shark has + 1d6 to the attack roll against a creature if at least one of the shark's allies is within 1 meter of the creature and that ally is incapacitated.

\textbf{Actions}

\textit{\textbf{Bite.} Melee Weapon Attack}: +4 hit, 1m range, one target.

\textit{Strikes:} 6 (1d8 + 2) piercing damage.

\medskip \textbf{Giant Shark} \index[Mostruario]{Giant Shark}

The giant shark is 9 meters long and one meets it

normally only in the deepest oceans.

\textit{Huge beast, misaligned}

\textbf{STRENGTH} +6

\textbf{DEXTERITY} +0

\textbf{CONSTITUTION} +5

\textbf{INTELLIGENCE} -5

\textbf{WISDOM} +0

\textbf{CHARISMA} -3

\textbf{Initiative} +0 - \textbf{Defense} 16

\textbf{Hit Points} 126 (11d12 + 55)

\textbf{Movement} 0m, swim 15m

\textbf{Saving Throws}: Fortitude +7, Reflex +2, Will +1

\textbf{Skills} Awareness +3

\textbf{Senses} blind sight 18 m

\textbf{Languages} -

\textbf{Challenge} 5 (1,800 PX)

\textit{\textbf{Bloody Frenzy.}} The shark has + 1d6 on attack rolls

in melee against any creature that is not at full hit points.

\textit{\textbf{Breathing Water.}} The shark can only breathe underwater.

\textbf{Actions}

\textit{\textbf{Bite.} Melee Weapon Attack}: +9 to hit, 1m range, one target.

\textit{Strikes:} 22 (3d10 + 6) piercing damage.

\medskip \textbf{Striga} \index[Mostruario]{Striga}

This hideous monster looks like a cross between a large bat and an oversized mosquito. Its legs terminate in long pincers, and its long, needle-like proboscis cuts through the air as it tries to feed on the blood of living creatures.

\textit{Tiny beast, misaligned}

\textbf{STRENGTH} -3

\textbf{DEXTERITY} +3

\textbf{CONSTITUTION} +0

\textbf{INTELLIGENCE} -4

\textbf{WISDOM} -1

\textbf{CHARISMA} -2

\textbf{Initiative} +3 - \textbf{Defense} 15

\textbf{Hit Points} 2 (1d4)

\textbf{Movement} 3m, flight 12m

\textbf{Saving Throws}: Fortitude -3, Reflex +4, Will -1

\textbf{Senses} vision in the dark 18 m

\textbf{Languages} -

\textbf{Challenge} 1/8 (25 PX)

\textbf{Actions}

\textit{\textbf{Blood Drain.} Melee Weapon Attack}: +5 to hit, range 1 yd, a creature.

\textit{Strikes:} 5 (1d4 + 3) piercing damage and the striga attaches itself to the target. While attacked, the striga does not attack. Instead, at the start of each striga's turn, the target loses 5 (1d4 + 3) hit points due to blood loss.

The striga can detach by spending 1 meter of movement. It does this automatically after draining 10 hit points from the target or upon death of the target. A creature, including the target, can use its action to detach the striga.

\medskip \textbf{Rate} \index[Mostruario]{Rate}

\textit{Tiny beast, misaligned}

\textbf{STRENGTH} -3

\textbf{DEXTERITY} +0

\textbf{CONSTITUTION} +1

\textbf{INTELLIGENCE} -4

\textbf{WISDOM} +1

\textbf{CHARISMA} -3

\textbf{Initiative} +0 - \textbf{Defense} 11

\textbf{Hit Points} 3 (1d4 + 1)

\textbf{Movement} 6 m, Burrow 1 m

\textbf{Saving Throws}: Fortitude -3, Reflex +1, Will +1

\textbf{Senses} vision in the dark 9 m

\textbf{Languages} -

\textbf{Challenge} 0 (10 PX)

\textit{\textbf{Refined Smell.}} The rate has + 1d6 on Wisdom (Awareness) checks based on smell.

\textbf{Actions}

\textit{\textbf{Bite.} Melee Weapon Attack}: +2 to hit, 1m range, one target.

\textit{Strikes:} 1 piercing damage.

\medskip \textbf{Giant Badger} \index[Mostruario]{Giant Badger}

\textit{Medium beast, misaligned}

\textbf{STRENGTH} +1

\textbf{DEXTERITY} +0

\textbf{CONSTITUTION} +2

\textbf{INTELLIGENCE} -4

\textbf{WISDOM} +1

\textbf{CHARISMA} -3

\textbf{Initiative} +0 - \textbf{Defense} 11

\textbf{Hit Points} 13 (2d8 + 4)

\textbf{Movement} 9 m, Burrow 3 m

\textbf{Saving Throws}: Fortitude +2, Reflex +1, Will +2

\textbf{Senses} vision in the dark 9 m

\textbf{Languages} -

\textbf{Challenge} 1/4 (50 PX)

\textit{\textbf{Refined Smell.}} The rate has + 1d6 on Wisdom (Awareness) checks based on smell.

\textbf{Actions}

\textit{\textbf{Multiattack.}} The badger makes two attacks: one with its bite and one with its claws.

\textit{\textbf{Claws.} Melee Weapon Attack}: +3 to hit, 1m range, one target.

\textit{Strikes:} 6 (2d4 + 1) slashing damage.

\textit{\textbf{Bite.} Melee Weapon Attack}: +3 to hit, 1m range, one target.

\textit{Strikes:} 4 (1d6 + 1) piercing damage.

\medskip \textbf{Tigre} \index[Mostruario]{Tigre}

\textit{Large beast, misaligned}

\textbf{STRENGTH} +3

\textbf{DEXTERITY} +2

\textbf{CONSTITUTION} +2

\textbf{INTELLIGENCE} -4

\textbf{WISDOM} +1

\textbf{CHARISMA} -1

\textbf{Initiative} +2 - \textbf{Defense} 13

\textbf{Hit Points} 37 (5d10 + 10)

\textbf{Movement} 12 m

\textbf{Saving Throws}: Fortitude +4, Reflex +4, Will +2

\textbf{Skills} Move Silently / Hide +6, Awareness +3

\textbf{Senses} vision in the dark 18 m

\textbf{Languages} -

\textbf{Challenge} 1 (200 PX)

\textit{\textbf{Leap.}} If the tiger moves at least 20 feet toward a creature and hits it with a claw attack during the same turn, the target must succeed on a DC 13 Fortitude save or fall prone. If the target is prone, the tiger can make a bite attack against it as a bonus action.

\textit{\textbf{Refined Smell.}} The tiger has + 1d6 on Wisdom (Awareness) checks based on smell.

\textbf{Actions}

\textit{\textbf{Claw.} Melee Weapon Attack}: +5 to hit, 1m range, one target.

\textit{Strikes:} 7 (1d8 + 3) slashing damage, 1 bleed damage.

\textit{\textbf{Bite.} Melee Weapon Attack}: +5 to hit, 1m range, one target.

\textit{Strikes:} 8 (1d10 + 3) piercing damage.

\medskip \textbf{Saber-Toothed Tiger} \index[Mostruario]{Saber-Toothed Tiger}

\textit{Large beast, misaligned}

\textbf{STRENGTH} +4

\textbf{DEXTERITY} +2

\textbf{CONSTITUTION} +2

\textbf{INTELLIGENCE} -4

\textbf{WISDOM} +1

\textbf{CHARISMA} -1

\textbf{Initiative} +2 - \textbf{Defense} 13

\textbf{Hit Points} 52 (7d10 + 14)

\textbf{Movement} 12 m

\textbf{Saving Throws}: Fortitude +5, Reflex +3, Will +2

\textbf{Skills} Move silently / Hide +6, Awareness +3

\textbf{Languages} -

\textbf{Challenge} 2 (450 PX)

\textit{\textbf{Leap.}} If the tiger moves at least 20 feet toward a creature and hits it with a claw attack during the same turn, the target must succeed on a DC 14 Fortitude save or fall prone. If the target is prone, the tiger can make a bite attack against it as a bonus action.

\textit{\textbf{Refined Smell.}} The tiger has + 1d6 on Wisdom (Awareness) checks based on smell.

\textbf{Actions}

\textit{\textbf{Claw.} Melee Weapon Attack}: +6 to hit, 1m range, one target.

\textit{Strikes:} 12 (2d6 + 5) slashing damage, 1 bleed damage.

\textit{\textbf{Bite.} Melee Weapon Attack}: +6 to hit, 1m range, one target.

\textit{Strikes:} 10 (1d10 + 5) piercing damage.

\medskip \textbf{Giant Wasp} \index[Mostruario]{Giant Wasp}

\textit{Medium beast, misaligned}

\textbf{STRENGTH} +0

\textbf{DEXTERITY} +2

\textbf{CONSTITUTION} +0

\textbf{INTELLIGENCE} -5

\textbf{WISDOM} +0

\textbf{CHARISMA} -4

\textbf{Initiative} +2 - \textbf{Defense} 13

\textbf{Hit Points} 13 (3d8)

\textbf{Movement} 3m, flight 15m

\textbf{Saving Throws}: Fortitude +1, Reflex +3, Will +0

\textbf{Languages} -

\textbf{Challenge} 1/2 (100 PX)

\textbf{Actions}

\textit{\textbf{Sting.} Melee Weapon Attack}: +4 to hit, range 1 yards, a creature.

\textit{Strikes:} 5 (1d6 + 2) piercing damage and the target must make a DC 11 Fortitude saving throw, and take 10 (3d6) poison damage on a failed save, or half that damage if he succeeds. If the poison damage reduces the target to 0 hit points, the target is stable but poisoned for 1 hour, even after recovering hit points, and while poisoned in this way, the target is paralyzed.

\medskip \textbf{Worg} \index[Mostruario]{Worg}

Worgs are monstrous wolf-like predators who love to hunt and devour creatures weaker than themselves.

\textit{Large monstrosity, neutral evil}

\textbf{STRENGTH} +3

\textbf{DEXTERITY} +1

\textbf{CONSTITUTION} +1

\textbf{INTELLIGENCE} -2

\textbf{WISDOM} +0

\textbf{CHARISMA} -1

\textbf{Initiative} +1 - \textbf{Defense} 14

\textbf{Hit Points} 26 (4d10 + 4)

\textbf{Movement} 15 m

\textbf{Saving Throws}: Fortitude +3, Reflex +2, Will +2

\textbf{Skills} Awareness +4

\textbf{Senses} vision in the dark 18 m

\textbf{Languages} Goblin, Worg

\textbf{Challenge} 1/2 (100 PX)

\textit{\textbf{Hearing and refined smell.}} The worg has + 1d6 on hearing or smell based Wisdom (Awareness) checks.

\textbf{Actions}

\textit{\textbf{Bite.} Melee Weapon Attack}: +5 hit, 1m range, one target.

\textit{Strikes:} 10 (2d6 + 3) piercing damage. If the target is a creature, it must succeed on a DC 13 Fortitude saving throw or fall prone.

\subsection{Appendix B: Non-Player Characters} \index[Mostruario]{Non-Player Characters}

This appendix contains the statistics of various humanoid non-player characters (NPCs) that adventurers may encounter over the course of a campaign, from lowly commoners to powerful archmages. These stats can be used to represent human and non-human NPCs.

Customize the PNGs

There are many easy ways to customize the NPCs in this appendix for use in your home campaign.

\textit{\textbf{Changing Spells.}} One way to customize a spellcaster NPC is to replace one or more of his spells. You can replace any spell from the list of
NPC spells with a different spell of the same level. Changing spells in this way does not change the NPC's challenge rating.

\textbf{\textit{Change Weapons and Armor}.} You can upgrade or worsen the NPC's armor or add or change weapons. Defense and damage changes can change the NPC's challenge rating.

\textit{\textbf{Magic Items}}. The more powerful an NPC is, the more likely they are to possess one or more magical items. A wizard, for example, might have a magic wand or staff, as well as one or more potions and scrolls. Providing an NPC with a powerful magical item capable of dealing damage could change their degree of challenge.

Some example magic items are described later in this document.

\textbf{Fighters}

Fighters are individuals who make a living by putting their sword in the service of an individual or an ideal.

\medskip \textbf{Guard}

Guards include members of the city patrol, sentries of a citadel or fortified city, and the bodyguards of nobles and merchants.

\textit{Humanoid Medium (any race), any Trait}

\textbf{STRENGTH} +1

\textbf{DEXTERITY} +1

\textbf{CONSTITUTION} +1

\textbf{INTELLIGENCE} +0

\textbf{WISDOM} +0

\textbf{CHARISMA} +0

\textbf{Initiative} +1 - \textbf{Defense} 17 (mail jacket, shield)

\textbf{Hit Points} 11 (2d8 + 2)

\textbf{Movement} 9 m

\textbf{Saving Throws}: Fortitude +3, Reflex +1, Will +1

\textbf{Skills} Awareness +2

\textbf{Languages} any language (usually the Common)

\textbf{Challenge} 1/8 (25 PX)

\textbf{Actions}

\textit{\textbf{Spear.} Melee or Ranged Weapon Attack}: +3 to hit, 1 m range, 6m range, one target.

\textit{Strikes:} 4 (1d6 + 1) piercing damage or 5 (1d8 + 1) piercing damage when used with two hands to make a melee attack.

\medskip \textbf{Veteran}

Warriors survived for a long time, earning a great reputation as skilled and skilled fighters.

\textit{Humanoid Medium (any race), any Trait}

\textbf{STRENGTH} +3

\textbf{DEXTERITY} +1

\textbf{CONSTITUTION} +2

\textbf{INTELLIGENCE} +0

\textbf{WISDOM} +0

\textbf{CHARISMA} +0

\textbf{Initiative} +1 - \textbf{Defense} 19 (stripe armor)

\textbf{Hit Points} 58 (9d8 + 18)

\textbf{Movement} 9 m

\textbf{Saving Throws}: Fortitude +4, Reflex +2, Will +3

\textbf{Skills} Acrobatics +5, Awareness +2

\textbf{Languages} any language (usually the Common)

\textbf{Challenge} 3 (700 PX)

\textbf{Actions}

\textit{\textbf{Multiattack.}} The veteran makes two long sword attacks. If he has a short sword drawn, he can also make a short sword attack.

\textit{\textbf{Longsword.} Melee Weapon Attack}: +5 to hit, 1m range, one target.

\textit{Strikes:} 7 (1d8 + 3) slashing damage, or 8 (1d10 + 3) slashing damage when used with two hands.

\textit{\textbf{Short Sword.} Melee Weapon Attack}: +5 to hit, 1m range, one target.

\textit{Strikes:} 6 (1d6 + 3) piercing damage.

\textit{\textbf{Heavy Crossbow.} Ranged Weapon Attack}: +3 to hit, range 30m, one target. \textit{Strikes:} 6 (1d10 + 1) piercing damage.

\medskip \textbf{Knight}

Knights are fighters who swear allegiance to kings, religious orders, and noble causes. The knight's traits determine the extent to which he is willing to honor his oath.

\textit{Humanoid Medium (any race), any Trait}

\textbf{STRENGTH} +3

\textbf{DEXTERITY} +0

\textbf{CONSTITUTION} +2

\textbf{INTELLIGENCE} +0

\textbf{WISDOM} +0

\textbf{CHARISMA} +2

\textbf{Initiative} +0 - \textbf{Defense} 20 (plate armor)

\textbf{Hit Points} 52 (8d8 + 16)

\textbf{Movement} 9 m

\textbf{Saving Throws}: Fortitude +4, Reflex +1, Will +3

\textbf{Languages} any language (usually the Common)

\textbf{Challenge} 3 (700 PX)

\textit{\textbf{Brave.}} The cavalier has + 1d6 on saving throws against the frightened being.

\textbf{Actions}

\textit{\textbf{Multiattack.}} The cavalier makes two melee attacks.

\textit{\textbf{Big Sword.} Melee Weapon Attack}: +5 to hit, 1m range, one target.

\textit{Strikes:} 10 (2d6 + 3) slashing damage.

\textit{\textbf{Heavy Crossbow.} Ranged Weapon Attack}: +2 to hit, range 30m, one target.

\textit{Strikes:} 5 (1d10) piercing.

\textit{\textbf{Authority (Refill after 1 hour)}}. For 1 minute, the cavalier can utter a special command or warning whenever a non-hostile creature within 30 feet of him, and that he can see, makes an attack or saving throw. The creature can add a d4 to its roll as long as it can hear and understand the rider. A creature can only benefit from one Authority die at a time. This effect ends if the rider is incapacitated.

\textbf{Reactions}

\textit{\textbf{Parry.}} The cavalier can add 2 to his Defense against a melee attack that would hit him. To do this, the knight must see the attacker and be holding a melee weapon.

\medskip \textbf{Gladiator}

Trained to entertain crowds, they are among the most dangerous fighters around.

\textit{Medium humanoid (any race), any Trait}

\textbf{STRENGTH} +4

\textbf{DEXTERITY} +2

\textbf{CONSTITUTION} +3

\textbf{INTELLIGENCE} +0

\textbf{WISDOM} +1

\textbf{CHARISMA} +2

\textbf{Initiative} +2 - \textbf{Defense} 19 (studded leather armor, shield)

\textbf{Hit Points} 112 (15d8 + 45)

\textbf{Movement} 9 m

\textbf{Saving Throws}: Fortitude +5, Reflex +5, Will +3

\textbf{Skills} Acrobatics +10, Intimidation +5

\textbf{Languages} any language (usually the Common)

\textbf{Challenge} 5 (1,800 PX)

\textit{\textbf{Brutus.}} A melee weapon deals an additional die of damage

when a gladiator strikes with it (already included in the attack).

\textit{\textbf{Brave.}} The gladiator has + 1d6 on saving throws against the frightened being.

\textbf{Actions}

\textit{\textbf{Multiattack.}} The gladiator makes three melee attacks or two ranged attacks.

\textit{\textbf{Spear.} Melee or Ranged Weapon Attack}: +7 to hit, 1 m range, 6m range, one target.

\textit{Strikes:} 11 (2d6 + 4) piercing damage, or 13 (2d8 + 4) slashing damage when used with two hands.

\textit{\textbf{Shield Slash.} Melee Weapon Attack}: +7 to hit, 1m range, one target.

\textit{Strikes:} 9 (2d4 + 4) hit damage. If the target is a Medium-sized creature or smaller, it must succeed on a DC 15 Fortitude save or fall prone.

\textbf{Reactions}

\textit{\textbf{Parry.}} The gladiator adds 3 to his Defense against a melee attack that would hit him. To do this, the gladiator must see the attacker and wield a melee weapon.

\medskip \textbf{Citizens}

This category includes those individuals who are involved in running the world, carrying out the tasks necessary for the fields to be cultivated, the cities administered, the food grown and
new territories explored.

\medskip \textbf{Noble}

The nobles rule over the population, by virtue of a birthright or by accumulated wealth. Among these are also the courtiers who crowd the courts of the rich and powerful.

\textit{Medium humanoid (any race), any Trait}

\textbf{STRENGTH} +0

\textbf{DEXTERITY} +1

\textbf{CONSTITUTION} +0

\textbf{INTELLIGENCE} +1

\textbf{WISDOM} +2

\textbf{CHARISMA} +3

\textbf{Initiative} +1 - \textbf{Defense} 16 (bib)

\textbf{Hit Points} 9 (2d8)

\textbf{Movement} 9 m

\textbf{Saving Throws}: Fortitude +1, Reflex +1, Will +2

\textbf{Skills} Perceive Emotions +4, Deceive +5

\textbf{Languages} any two languages

\textbf{Challenge} 1/8 (25 PX)

\textbf{Actions}

\textit{\textbf{Rapier.} Melee Weapon Attack}: +3 to hit, 1m range, one target.

\textit{Strikes:} 5 (1d8 + 1) piercing damage.

\textbf{Reactions}

\textit{\textbf{Block.}} The noble adds 2 to his Defense against a melee attack that would hit him. To do this, the noble must see

the attacker and wield a melee weapon.

\medskip \textbf{Populate}

Commoners include peasants, servants, slaves, servants, pilgrims, merchants, artisans and hermits.

\textit{Humanoid Medium (any race), any Trait}

\textbf{STRENGTH} +0

\textbf{DEXTERITY} +0

\textbf{CONSTITUTION} +0

\textbf{INTELLIGENCE} +0

\textbf{WISDOM} +0

\textbf{CHARISMA} +0

\textbf{Initiative} +0 - \textbf{Defense} 11

\textbf{Hit Points} 4 (1d8)

\textbf{Movement} 9 m

\textbf{Saving Throws}: Fortitude +0, Reflex +0, Will +0

\textbf{Languages} any language (usually the Common)

\textbf{Challenge} 0 (10 PX)

\textbf{Actions}

\textit{\textbf{Club.} Melee Weapon Attack}: +2 to hit, 1m range, one target.

\textit{Strikes:} 2 (1d4) hit damage.

\medskip \textbf{Criminals}

Criminals are individuals who live on the fringes of legality, obtaining bread by carrying out activities that are often considered illicit and immoral.

\medskip \textbf{Thumper}

Thugs are ruthless criminals skilled at intimidating and perpetrating acts of violence. They work for money and have few qualms.

\textit{Medium humanoid (any race), any Trait}

\textbf{STRENGTH} +2

\textbf{DEXTERITY} +0

\textbf{CONSTITUTION} +2

\textbf{INTELLIGENCE} +0

\textbf{WISDOM} +0

\textbf{CHARISMA} +0

\textbf{Initiative} +0 - \textbf{Defense} 12 (leather armor)

\textbf{Hit Points} 32 (5d8 + 10)

\textbf{Movement} 9 m

\textbf{Saving Throws}: Fortitude +3, Reflex +1, Will +0

\textbf{Skills} Intimidation +2

\textbf{Languages} any language (usually the Common)

\textbf{Challenge} 1/2 (100 PX)

\textit{\textbf{Pack Tactics.}} The batter has + 1d6 to attack rolls against a creature if at least one of the batter's allies are within 1 meter of the creature and that ally is not
is incapacitated.

\textbf{Actions}

\textit{\textbf{Multi Attack.}} The thug makes two melee attacks.

\textit{\textbf{Mace.} Melee Weapon Attack}: +4 to hit, range 1 yards, a creature.

\textit{Strikes:} 5 (1d6 + 2) hit damage.

\textit{\textbf{Heavy Crossbow.} Ranged Weapon Attack}: +2 to hit, range 30m, one target. \textit{Strikes:} 5 (1d10) piercing damage.

\medskip \textbf{Bandit / Pirate}

Whether they are men of the street or of the sea (pirates) they earn their living by plundering others.

\textit{Medium humanoid (any race), any trait not legal}

\textbf{STRENGTH} +0

\textbf{DEXTERITY} +1

\textbf{CONSTITUTION} +1

\textbf{INTELLIGENCE} +0

\textbf{WISDOM} +0

\textbf{CHARISMA} +0

\textbf{Initiative} +1 - \textbf{Defense} 13 (leather armor)

\textbf{Hit Points} 11 (2d8 + 2)

\textbf{Movement} 9 m

\textbf{Saving Throws}: Fortitude +1, Reflex +2, Will +1

\textbf{Languages} any language (usually the Common)

\textbf{Challenge} 1/8 (25 PX)

\textbf{Actions}

\textit{\textbf{Scimitar.} Melee Weapon Attack}: +3 to hit, 1m range, one target.

\textit{Strikes:} 4 (1d6 + 1) slashing damage.

\textit{\textbf{Light Crossbow.} Ranged Weapon Attack}: +3 to hit, range 24m, one target. \textit{Strikes:} 5 (1d8 + 1) slashing damage.

\medskip \textbf{Spy}

A spy is an individual trained to find secrets on behalf of someone, or sometimes to resell them to the highest bidder.

\textit{Humanoid Medium (any race), any Trait}

\textbf{STRENGTH} +0

\textbf{DEXTERITY} +2

\textbf{CONSTITUTION} +0

\textbf{INTELLIGENCE} +1

\textbf{WISDOM} +2

\textbf{CHARISMA} +3

\textbf{Initiative} +2 - \textbf{Defense} 13

\textbf{Hit Points} 27 (6d8)

\textbf{Movement} 9 m

\textbf{Saving Throws}: Fortitude +2, Reflex +3, Will +3

\textbf{Skills} Awareness +6, Move Silently / Hide +4, Feel Emotions +4, Investigation +5, Deceive +5, Fairy Hands +4

\textbf{Languages} any two languages

\textbf{Challenge} 1 (200 PX)

\textit{\textbf{Sneak Attack (1 / Turn).}} The spy deals 7 (2d6) additional damage when hitting a target with a weapon attack and has + 1d6 attack roll, or when the target is within 1 meter of an assassin's ally who is not incapacitated and the assassin does not have -1d6 on the attack roll.

\textit{\textbf{Cunning Action.}} During each of its rounds, the spy can use a bonus action to take the Retreat, Hide, or Dash action.

\textbf{Actions}

\textit{\textbf{Multi Attack.}} The spy makes two melee attacks.

\textit{\textbf{Short Sword.} Melee Weapon Attack}: +4 hit, 1m range, one target.

\textit{Strikes:} 5 (1d6 + 2) piercing damage.

\textit{\textbf{Crossbow.} Ranged Weapon Attack}: +4 to hit, range 9m, one target. \textit{Strikes:} 5 (1d6 + 2) piercing damage.


\medskip \textbf{Bandit Captain / Pirate}

Whether he lives on land or in the sea, he is an individual with a great personality who manages to keep the rabble that responds to his orders in line.

\textit{Medium humanoid (any race), any trait not legal}

\textbf{STRENGTH} +2

\textbf{DEXTERITY} +3

\textbf{CONSTITUTION} +2

\textbf{INTELLIGENCE} +2

\textbf{WISDOM} +0

\textbf{CHARISMA} +2

\textbf{Initiative} +2 - \textbf{Defense} 16 (studded leather armor)

\textbf{Hit Points} 65 (10d8 + 8)

\textbf{Movement} 9 m

\textbf{Saving Throws}: Fortitude +5, Reflex +5, Will +3

\textbf{Skills} Acrobatics +4, Bluff +4

\textbf{Languages} any two languages

\textbf{Challenge} 2 (450 PX)

\textbf{Actions}

\textit{\textbf{Multiattack.}} The captain makes three melee attacks: two with the scimitar and one with the dagger. Or the captain makes two ranged attacks with daggers.

\textit{\textbf{Scimitar.} Melee Weapon Attack}: +5 to hit, 1m range, one target.

\textit{Strikes:} 6 (1d6 + 3) slashing damage.

\textit{\textbf{Dagger.} Melee or Ranged Weapon Attack}: +5 to hit, 1 m range, 6m range, one target. \textit{Strikes:} 5 (1d4 + 3) piercing damage.

\textbf{Reactions}

\textit{\textbf{Block.}} The captain adds 2 to his Defense against a melee attack that would hit him. To do this, the captain must see the attacker and wield a melee weapon.

\medskip \textbf{Assassin}

Loners or members of a guild, assassins are paid to eliminate, often quietly and discreetly, rivals and enemies of their employers.

\textit{Medium humanoid (any race), any trait not good}

\textbf{STRENGTH} +0

\textbf{DEXTERITY} +3

\textbf{CONSTITUTION} +2

\textbf{INTELLIGENCE} +1

\textbf{WISDOM} +0

\textbf{CHARISMA} +0

\textbf{Initiative} +3 - \textbf{Defense} 19 (studded leather armor)

\textbf{Hit Points} 78 (12d8 + 24)

\textbf{Movement} 9 m

\textbf{Saving Throws}: Fortitude +4, Reflex +6, Will +3

\textbf{Skills} Acrobatics +6, move silently / hide +9, awareness +3, deceit +3


\textbf{Languages} Thieves' jargon plus two other languages

\textbf{Challenge} 8 (3.900 PX)

\textit{\textbf{Assassinate.}} During his first turn, the assassin has + 1d6 to attack rolls against creatures that haven't taken a turn yet. Any hit that the killer lands against a surprised creature is a critical hit.

\textit{\textbf{Sneak Attack (1 / Turn).}} The assassin deals 14 (4d6) additional damage when hitting a target with a weapon attack and has + 1d6 attack roll, or when the target it is within 1 meter of an ally of the assassin who is not incapacitated and the assassin does not have -1d6 on the attack roll.

\textit{\textbf{Evasion.}} If the killer is the victim of an effect that allows a Reflex saving throw to be made to halve the damage, the killer takes no damage on the saving throw, and only the half fail.

\textbf{Actions}

\textit{\textbf{Multiattack.}} The assassin makes two attacks with short swords.

\textit{\textbf{Short Sword.} Melee Weapon Attack}: +6 to hit, 1m range, one target.

\textit{Strikes:} 6 (1d6 + 3) piercing damage, and the target must make a DC 15 Fortitude save, taking 24 (7d6) poison damage on a failed save, or half that damage if he succeeds.

\textit{\textbf{Light Crossbow.} Ranged Weapon Attack}: +6 to hit, range 24m, one target.

\textit{Strikes:} 7 (1d8 + 3) piercing damage, and the target must make a DC 15 Fortitude save, taking 24 (7d6) poison damage on a failed save, or half that damage if he succeeds.

\medskip \textbf{Magician}

The magician spends his life studying and practicing magic.

\textbf{VARIANT: FAMILIES}

Any spellcaster who can cast the spell \textit{find} \textit{familiar} is likely to have a familiar. The familiar can be one of the creatures described in the spell (see \textit{Core Rules}) or some other Tiny monster, such as a slithering claw, imp, pseudo dragon, or imp.

\medskip \textbf{Adventurer Wizard}

A novice wizard, who has successfully passed his early adventures and has begun to establish a reputation as a noble or notorious adventurer.

\textit{Medium  humanoid (any race), any evil}

\textbf{STRENGTH} -1

\textbf{DEXTERITY} +2

\textbf{CONSTITUTION} +0

\textbf{INTELLIGENCE} +3

\textbf{WISDOM} +1

\textbf{CHARISMA} +0

\textbf{Initiative} +3 - \textbf{Defense} 13

\textbf{Hit Points} 22 (5d8)

\textbf{Movement} 9 m

\textbf{Saving Throws}: Fortitude +0, Reflex +3, Will +2

\textbf{Skills} Arcane +5, History +5

\textbf{Languages} any four languages

\textbf{Challenge} 1 (200 PX)

\textit{\textbf{Spells.}} The mage has CM 4. His spellcasting ability is Intelligence (+5 to hit with spell attacks). The Wizard has prepared the following spells: Tricks (at will):

\textit{light, magic hand, dazzling grasp}

level 1 (4 slots): \textit{charm people, magic missile}

level 2 (3 slots): \textit{block person, veiled step}

\textbf{Actions}

\textit{\textbf{Staff.} Melee Weapon Attack}: +1 to hit, 1 m range, one target.

\textit{Strikes:} 3 (1d8 - 1) hit damage.

\medskip \textbf{Great Wizard}

A magician who has established a good reputation in the area and who attracts students from all over.

\textit{Medium humanoid (any race), any Trait}

\textbf{STRENGTH} -1

\textbf{DEXTERITY} +2

\textbf{CONSTITUTION} +0

\textbf{INTELLIGENCE} +3

\textbf{WISDOM} +1

\textbf{CHARISMA} +0

\textbf{Initiative} +3 - \textbf{Defense} 15 (18 with \textit{Mage armor})

\textbf{Hit Points} 40 (9d8)

\textbf{Movement} 9 m

\textbf{Saving Throws}: Fortitude +1, Reflex +4, Will +3

\textbf{Skills} Arcane +6, History +6

\textbf{Languages} any four languages

\textbf{Challenge} 6 (2,300 PX)

\textit{\textbf{Spells.}} The mage has CM of 9. His spellcasting ability is Intelligence (+6 to hit with spell attacks). The Magician has prepared the following spells:

Tricks (at will): \textit{flaming dart, light, magic hand,}
\textit{prestidigitation}

level 1 (4 slots): \textit{Mage armor, magic missile,}
\textit{detect magic, shield}

level 2 (3 slots): \textit{veiled step, suggestion}

level 3 (3 slots): \textit{counterspell, fireball, fly}

level 4 (3 slots): \textit{greater invisibility, ice storm}

level 5 (1 slot): \textit{cone of cold}

\textbf{Actions}

\textit{\textbf{Dagger.} Melee or Ranged Weapon Attack}: +5 to hit, 1 m range, 6m range, one target. \textit{Strikes:} 4 (1d4 + 2) piercing damage.

\medskip \textbf{Archmage}

A very powerful (and also very old) wizard who studies the secrets of the multiverse.

\textit{Medium humanoid (any race), any Trait}

\textbf{STRENGTH} +0

\textbf{DEXTERITY} +2

\textbf{CONSTITUTION} +1

\textbf{INTELLIGENCE} +5

\textbf{WISDOM} +2

\textbf{CHARISMA} +3

\textbf{Initiative} +5 - \textbf{Defense} 18 (21 with \textit{Mage armor})

\textbf{Hit Points} 99 (18d8 + 18)

\textbf{Movement} 9 m

\textbf{Saving Throws}: Fortitude +8, Reflex +10, Will +12

\textbf{Skills} Arcane +13, History +13

\textbf{Damage Resistances} spell damage; punching, piercing and non-magical cutting (from \textit{skin of stone})

\textbf{Languages} any six languages

\textbf{Challenge} 12 (8.400 PX)

\textit{\textbf{Spells.}} The mage has CM 18. His spellcasting ability is Intelligence (+9 to hit with spell attacks).

The archmage can perform \textit{disguise himself} and \textit{invisibility} at will and has prepared the following spells: Tricks (at will): \textit{flaming dart, light, magic hand,}
\textit{prestidigitation, dazzling grasp}

level 1 (4 slots): \textit{magic armor *, magic missile, identify, detect magic}

level 2 (3 slots): \textit{mirror image, identification of thoughts, veiled step}

level 3 (3 slots): \textit{counterspell, lightning bolt}

level 4 (3 slots): \textit{exile, skin of stone *, shield of fire}

level 5 (3 slots): \textit{cone of cold, wall of force, scrutinize}

level 6 (1 slot): \textit{globe of invulnerability}

level 7 (1 slot): \textit{teleport}

level 8 (1 slot): \textit{blank mind *}

level 9 (1 slot): \textit{stop time}

The archmage casts these{*} spells on himself before combat.

\textbf{Actions}

\textit{\textbf{Dagger.} Melee or Ranged Weapon Attack}: +6 to hit, 1 m range, 6m range, one target. \textit{Strikes:} 4 (1d4 + 2) piercing damage.


\medskip \textbf{Priests}

Priests are devotees of a deity or faith who take care of imparting divine teachings to their flock.

\medskip \textbf{Cultist}

Cultists swear allegiance to dark powers, and in their beliefs and practices they often show signs of insanity.

\textit{Medium humanoid (any race), any bad trait}

\textbf{STRENGTH} +0

\textbf{DEXTERITY} +1

\textbf{CONSTITUTION} +0

\textbf{INTELLIGENCE} +0

\textbf{WISDOM} +0

\textbf{CHARISMA} +0

\textbf{Initiative} + 0- \textbf{Defense} 13 (leather armor)

\textbf{Hit Points} 9 (2d8)

\textbf{Movement} 9 m

\textbf{Saving Throws}: Fortitude +1, Reflex +1, Will +2

\textbf{Skills} Bluff +2, Religion +2

\textbf{Languages} any language (usually the Common)

\textbf{Challenge} 1/8 (25 PX)

\textit{\textbf{Dark Devotion.}} The cultist has + 1d6 on saving throws against being fascinated or frightened.

\textbf{Actions}

\textit{\textbf{Scimitar.} Melee Weapon Attack}: +3 to hit, range 1 yards, a creature.

\textit{Strikes:} 4 (1d6 + 1) slashing damage.

\medskip \textbf{Acolyte}

Acolytes are lower-ranking members of the clergy, and usually answer to a higher-ranking priest. They perform various functions in a temple and are given the ability to cast minor spells by their deity.

\textit{Medium humanoid (any race), any Trait}

\textbf{STRENGTH} +0

\textbf{DEXTERITY} +0

\textbf{CONSTITUTION} +0

\textbf{INTELLIGENCE} +0

\textbf{WISDOM} +2

\textbf{CHARISMA} +0

\textbf{Initiative} +0 - \textbf{Defense} 11

\textbf{Hit Points} 9 (2d8)

\textbf{Movement} 9 m

\textbf{Saving Throws}: Fortitude +0, Reflex +0, Will +3

\textbf{Skills} First Aid +4, Religion +2

\textbf{Languages} any language (usually the Common)

\textbf{Challenge} 1/4 (50 PX)

\textit{\textbf{Spells.}} The acolyte has CM 1. Her spellcasting ability is Wisdom (+4 to hit with spell attacks). The acolyte has prepared the following spells: Tricks (at will): \textit{sacred flame, light, thaumaturgy} level 1 (3 slots): \textit{blessing}, \textit{heal wounds, sanctuary}

\medskip \textbf{Actions}

\textit{\textbf{Club.} Melee Weapon Attack}: +2 to hit, 1m range, one target.

\textit{Strikes:} 2 (1d4) hit damage.

\textbf{Fanatic of the Cult}

They are the leaders of a cult, who use their charisma and dogmas to influence the weak-willed.

\textit{Medium humanoid (any race), any bad trait}

\textbf{STRENGTH} +0

\textbf{DEXTERITY} +2

\textbf{CONSTITUTION} +1

\textbf{INTELLIGENCE} +0

\textbf{WISDOM} +1

\textbf{CHARISMA} +2

\textbf{Initiative} +2 - \textbf{Defense} 14 (leather armor)

\textbf{Hit Points} 33 (6d8 + 6)

\textbf{Movement} 9 m

\textbf{Saving Throws}: Fortitude +2, Reflex +2, Will +3

\textbf{Skills} Deceive +4, Bluff +4, Religion +2

\textbf{Languages} any language (usually the Common)

\textbf{Challenge} 2 (450 PX)

\textit{\textbf{Spells.}} The priest has CM 4. His spellcasting ability is Wisdom (+3 to hitting with spell attacks). The priest has prepared the following spells: Tricks (at will): \textit{sacred flame, light, thaumaturgy}

level 1 (4 slots): \textit{command, inflict wounds, shield of faith}

level 2 (3 slots): \textit{spirit weapon, person block}

\textit{\textbf{Dark Devotion.}} The cultist has + 1d6 on saving throws against being fascinated or frightened.

\textbf{Actions}

\textit{\textbf{Multiattack.}} The fanatic makes two melee attacks.

\textit{\textbf{Dagger.} Melee or Ranged Weapon Attack}: +4 to hit, range 1 m, range 6m, a creature. \textit{Strikes:} 4 (1d4 + 2) piercing damage.

\medskip \textbf{High Priest}

They are individuals in command of a temple or other sacred place and who have several acolytes at their disposal.

\textit{Medium humanoid (any race), any Trait}

\textbf{STRENGTH} +0

\textbf{DEXTERITY} +0

\textbf{CONSTITUTION} +1

\textbf{INTELLIGENCE} +1

\textbf{WISDOM} +3

\textbf{CHARISMA} +1

\textbf{Initiative} +1 - \textbf{Defense} 14 (shirt jacket)

\textbf{Hit Points} 27 (5d8 + 5)

\textbf{Movement} 7 m

\textbf{Saving Throws}: Fortitude +1, Reflex +1, Will +4

\textbf{Skills} First Aid +7, Deceive +3, Religion +4

\textbf{Languages} any two languages

\textbf{Challenge} 2 (450 PX)

\textit{\textbf{Divine Eminence.}} As a bonus action, the priest can spend one spell slot to cause his melee weapon attack to deal 10 (3d6) additional Light damage. The benefit lasts until the end of the turn.

\textit{\textbf{Spells.}} The priest has CM 5. His spellcasting ability is Wisdom (+5 to hit with spell attacks). The priest has prepared the following spells: Tricks (at will): \textit{sacred flame, light, thaumaturgy}

level 1 (4 slots): \textit{heal wounds, tracer bolt, sanctuary}

level 2 (3 slots): \textit{spirit weapon, lower restoring}

level 3 (2 slots): \textit{dispel magic}, \textit{spiritual guardians}

\textbf{Actions}

\textit{\textbf{Mace.} Melee Weapon Attack}: +2 to hit, 1m range, one target.

\textit{Strikes:} 3 (1d6) hit damage.


\medskip \textbf{Savages}

These individuals live on the fringes of civilization, sometimes rarely coming into contact with them. Uncomfortable within walls and in civilized lands, they find themselves in their environment when they can roam the wilderness.

\medskip \textbf{Berserker}

Coming from the wilderness, the unpredictable berserkers congregate in wartime companies and are always on the lookout for conflicts to fight in.

\textit{Medium humanoid (any race), any chaotic trait}

\textbf{STRENGTH} +3

\textbf{DEXTERITY} +1

\textbf{CONSTITUTION} +3

\textbf{INTELLIGENCE} -1

\textbf{WISDOM} +0

\textbf{CHARISMA} -1

\textbf{Initiative} +1 - \textbf{Defense} 14 (leather armor)

\textbf{Hit Points} 67 (9d8 + 27)

\textbf{Movement} 9 m

\textbf{Saving Throws}: Fortitude +4, Reflex +3, Will +2

\textbf{Languages} any language (usually the Common)

\textbf{Challenge} 2 (450 PX)

\textit{\textbf{Careless.}} At the start of his round, the berserker can get + 1d6 on all melee weapon attack rolls made that turn, but attack rolls against him have + 1d6 until the start of his next round.

\textbf{Actions}

\textit{\textbf{Big Ax.} Melee Weapon Attack}: +5 to hit, 1m range, one target.

\textit{Strikes:} 9 (1d12 + 3) slashing damage.

\textbf{Tribal Fighter}

They are the defenders of the tribes who live on the fringes of civilization.

\textit{Medium humanoid (any race), any Trait}

\textbf{STRENGTH} +1

\textbf{DEXTERITY} +0

\textbf{CONSTITUTION} +1

\textbf{INTELLIGENCE} -1

\textbf{WISDOM} +0

\textbf{CHARISMA} -1

\textbf{Initiative} +0 - \textbf{Defense} 13 (leather armor)

\textbf{Hit Points} 11 (2d8 + 2)

\textbf{Movement} 9 m

\textbf{Saving Throws}: Fortitude +2, Reflex +1, Will +1

\textbf{Languages} any language

\textbf{Challenge} 1/8 (25 PX)

\textit{\textbf{Pack Tactics.}} The tribal fighter has + 1d6 to attack rolls against a creature if at least one of the batter's allies are within 1 meter of the creature and that ally is incapacitated.

\textbf{Actions}

\textit{\textbf{Spear.} Melee or Ranged Weapon Attack}: +3 to hit, range 1 m and range 6m, one target.

\textit{Strikes:} 4 (1d6 + 1) piercing damage, or 5 (1d8 + 1) piercing damage if used with two hands to make a melee attack.

\medskip \textbf{Druid}

Druids protect the natural world from monsters and the advance of civilization. Some are tribal shamans who heal the sick, pray to animal spirits, and provide spiritual advice.

\textit{Medium humanoid (any race), any Trait}

\textbf{STRENGTH} +0

\textbf{DEXTERITY} +1

\textbf{CONSTITUTION} +1

\textbf{INTELLIGENCE} +1

\textbf{WISDOM} +2

\textbf{CHARISMA} +0

\textbf{Initiative} +1 - \textbf{Defense} 12 (17 with \textit{bark skin} *)

\textbf{Hit Points} 27 (5d8 + 5)

\textbf{Movement} 9 m

\textbf{Saving Throws}: Fortitude +1, Reflex +2, Will +3 \\

\textbf{Skills} First Aid +4, Nature +3, Awareness +4

\textbf{Languages} Druidic plus two other languages

\textbf{Challenge} 2 (450 PX)

\textit{\textbf{Spells.}} The priest has CM 4. His spellcasting ability is Wisdom (+4 to hitting with spell attacks). The priest has prepared the following spells: Tricks (at will): \textit{druid art, staff, producing flame}

level 1 (4 slots): \textit{get in the way, thundering wave, talk to them}
\textit{animals, quick step}

level 2 (3 slots): \textit{messenger animal, bark hide}

\textbf{Actions}

\textit{\textbf{Combat Staff.} Melee Weapon Attack}: +2 to hit (+4 to hit with \textit{stick *}), range 1 m, range 6m, one target.

\textit{Hits:} 3 (1d6) hit damage, or 6 (1d8 + 2) hit damage with \textit{stick} or when wielded with two hands.

\medskip \textbf{Explorer}

Skilled hunters and track hitters.

\textit{Humanoid Medium (any race), any Trait}

\textbf{STRENGTH} +0

\textbf{DEXTERITY} +2

\textbf{CONSTITUTION} +1

\textbf{INTELLIGENCE} +0

\textbf{WISDOM} +1

\textbf{CHARISMA} +0

\textbf{Initiative} +2 - \textbf{Defense} 14 (leather armor)

\textbf{Hit Points} 16 (3d8 + 3)

\textbf{Movement} 9 m

\textbf{Saving Throws}: Fortitude +1, Reflex +2, Will +3

\textbf{Skills} Move Silently / Hide +6, Nature +4, Awareness +5, Survival +5

\textbf{Languages} any language (usually Common)

\textbf{Challenge} 1/2 (100 PX)

\textit{\textbf{Sight and smell refined.}} The scout has + 1d6 on Wisdom (Awareness) checks based on smell or sight.

\textbf{Actions}

\textit{\textbf{Multiattack.}} The scout makes two melee attacks or two ranged attacks.

\textit{\textbf{Short Sword.} Melee Weapon Attack}: +4 hit, 1m range, one target.

\textit{Strikes:} 5 (1d6 + 2) piercing damage.

\textit{\textbf{Longbow.} Melee Weapon Attack}: +4 to hit, range 45m, one target.

\textit{Strikes:} 6 (1d8 + 2) piercing damage.


\end{multicols}

%{\scriptsize
%\printindex}
% \end{document}

\pagebreak

\subsection{Lista Mostri per Grado di Sfida}



\begin{multicols}{3}
{%\small
\flushleft{Aquila, Challenge 0 (10 PX) \\
	Vulture, Challenge 0 (10 PX) \\
	Baboon, Challenge 0 (10 PX) \\
	Billy goat, Challenge 0 (10 PX) \\
	Cervo, Challenge 0 (10 PX) \\
	Corvo, Challenge 0 (10 PX) \\
	Weasel, Challenge 0 (10 PX) \\
	Falco, Challenge 0 (10 PX) \\
	Screeching Mushroom, Challenge 0 (10 PX) \\
	Cat, Challenge 0 (10 PX) \\
	Owl, Challenge 0 (10 PX) \\
	Iena, Challenge 0 (10 PX) \\
	Lemur, Challenge 0 (10 PX) \\
	Lizard, Challenge 0 (10 PX) \\
	Homunculus, Challenge 0 (10 PX) \\
	Pirana, Challenge 0 (10 PX) \\
	Populate, Challenge 0 (10 PX) \\
	Spider, Challenge 0 (10 PX) \\
	Frog, Challenge 0 (10 PX)} \\
Rat, Challenge 0 (10 PX) \\
Giant Fire Beetle, Challenge 0 (10 PX) \\
Jackal, Challenge 0 (10 PX) \\
Scorpio, Challenge 0 (10 PX) \\
Badger, Challenge 0 (10 PX) \\
Mice, Challenge: 0 (10 PX) \\
Bandit / Pirate, Challenge 1/8 (25 PX) \\
Camel, Challenge 1/8 (25 PX) \\
Kobold, Challenge 1/8 (25 PX) \\
Cultist, Challenge 1/8 (25 PX) \\
Giant Weasel, Challenge 1/8 (25 PX) \\
Blood Hawk, Challenge 1/8 (25 PX) \\
Giant Crab, Challenge 1/8 (25 PX) \\
Guard, Challenge 1/8 (25 PX) \\
Mastiff, Challenge 1/8 (25 PX) \\
Mule, Challenge 1/8 (25 PX) \\
Noble, Challenge 1/8 (25 PX) \\
Pony, Challenge 1/8 (25 PX) \\
Giant Rat, Challenge 1/8 (25 PX) \\
Venomous Snake, Challenge 1/8 (25 PX) \\
Flying Snake, Challenge 1/8 (25 PX) \\
Striga, Challenge 1/8 (25 PX) \\
Striga (Stygian Bird), Challenge 1/8 (25 PX) \\
Aquatic Man, Challenge 1/8 (25 PX) \\
Acolyte, Challenge 1/4 (50 PX) \\
Elk, Challenge 1/4 (50 PX) \\
Ax Beak, Challenge 1/4 (50 PX) \\
Intermittent Dog, Challenge 1/4 (50 PX) \\
Racehorse, Challenge 1/4 (50 PX) \\
Draft Horse, Challenge 1/4 (50 PX) \\
Giant Centipede, Challenge 1/4 (50 PX) \\
Boar, Challenge 1/4 (50 PX) \\
Dretch, Challenge 1/4 (50 PX) \\
Violet Mushroom, Challenge 1/4 (50 PX) \\
Grimlock, Challenge 1/4 (50 PX) \\
Giant Owl, Challenge 1/4 (50 PX) \\
Giant Lizard, Challenge 1/4 (50 PX) \\
Wolf, Challenge 1/4 (50 PX) \\
Mefito di Vapore, Challenge 1/4 (50 PX) \\
Pantera, Challenge 1/4 (50 PX) \\
Pseudodrago, Challenge 1/4 (50 PX) \\
Giant Wolf Spider, Challenge 1/4 (50 PX) \\
Giant Frog, Challenge 1/4 (50 PX) \\
Skeleton, Challenge 1/4 (50 PX) \\
Swarm of Crows, Challenge 1/4 (50 PX) \\
Bat Swarm, Challenge 1/4 (50 PX) \\
Swarm of Rats, Challenge 1/4 (50 PX) \\
Constricting Snake, Challenge 1/4 (50 PX) \\
Giant Venomous Snake, Challenge 1/4 (50 PX) \\
Flying Sword, Challenge 1/4 (50 PX) \\
Sprite, Challenge 1/4 (50 PX) \\
Giant Badger, Challenge 1/4 (50 PX) \\
Zombies, Challenge 1/4 (50 PX) \\
Giant Billy Goat, Challenge 1/2 (100 PX) \\
Warhorse, Challenge 1/2 (100 PX) \\
Giant Seahorse, Challenge 1/2 (100 PX) \\
Crocodile, Challenge 1/2 (100 PX) \\
Cockatrice, Challenge 1/2 (100 PX) \\
Explorer, Challenge 1/2 (100 PX) \\
Gnoll, Challenge 1/2 (100 PX) \\
Depth Gnome (Svirfneblin), Challenge 1/2 (100 PX) \\
Hobgoblin, Challenge 1/2 (100 PX) \\
Lizard, Challenge 1/2 (100 PX) \\
Dark Mantle, Challenge 1/2 (100 PX) \\
Ice Mefito, Challenge 1/2 (100 PX) \\
Mefito di Magma, Challenge 1/2 (100 PX) \\
Mefito di Dust, Challenge 1/2 (100 PX) \\
Gray Slime, Challenge 1/2 (100 PX) \\
Shadow, Challenge 1/2 (100 PX) \\
Scoter, Challenge 1/2 (100 PX) \\
Black Bear, Challenge 1/2 (100 PX) \\
Thumper, Challenge 1/2 (100 PX) \\
Rustyfago, Challenge 1/2 (100 PX) \\
Sahuagin, Challenge 1/2 (100 PX) \\
Satyr, Challenge 1/2 (100 PX) \\
Warhorse Skeleton, Challenge 1/2 (100 PX) \\
Insect Swarm, Challenge 1/2 (100 PX) \\
Swarm of Spiders, Challenge 1/2 (100 PX) \\
Swarm of Beetles, Challenge 1/2 (100 PX) \\
Swarm of Wasps, Challenge 1/2 (100 PX) \\
Swarms, Challenge 1/2 (100 PX) \\
Monkey, Challenge 1/2 (100 PX) \\
Coral Shark, Challenge 1/2 (100 PX) \\
Magma Man (Magmin), Challenge 1/2 (100 PX) \\
Vespa Gigante, Challenge 1/2 (100 PX) \\
Worg, Challenge 1/2 (100 PX) \\
Giant Eagle, Challenge 1 (200 PX) \\
Animated Armor, Challenge 1 (200 PX) \\
Harpy, Challenge 1 (200 PX) \\
Giant Vulture, Challenge 1 (200 PX) \\
Bugbear, Challenge 1 (200 PX) \\
Death Dog, Challenge 1 (200 PX) \\
Dinolupo (Metalwolf), Challenge 1 (200 PX) \\
Baby Brass Dragon, Challenge 1 (200 PX) \\
Copper Dragon Pup, Challenge 1 (200 PX) \\
Driade, Challenge 1 (200 PX) \\
Duergar, Challenge 1 (200 PX) \\
Ghoul, Challenge 1 (200 PX) \\
Globule, Challenge 1 (200 PX) \\
Giant Hyena, Challenge 1 (200 PX) \\
Imp, Challenge 1 (200 PX) \\
Hippogriff, Challenge 1 (200 PX) \\
Leone, Challenge 1 (200 PX) \\
Adventurer Wizard, Challenge 1 (200 PX) \\
Ogre, Challenge 1 (100 PX) \\
Brown Bear, Challenge 1 (200 PX) \\
Quasit, Challenge 1 (200 PX) \\
Giant Spider, Challenge 1 (200 PX) \\
Giant Toad, Challenge 1 (200 PX) \\
Swarm of Pirana, Challenge 1 (200 PX) \\
Spy, Challenge 1 (200 PX) \\
Tiger, Challenge 1 (200 PX) \\
Awakened Tree, Challenge 2 (450 PX) \\
Giant Elk, Challenge 2 (450 PX) \\
Ameba Paglierina, Challenge 2 (450 PX) \\
Ankheg, Challenge 2 (450 PX) \\
Azer, Challenge 2 (450 PX) \\
Berserker, Challenge 2 (450 PX) \\
Explosive Cockroach, Challenge 2 (450 PX) \\
Bandit Captain / Pirate, Challenge 2 (450 PX) \\
Centaur, Challenge 2 (450 PX) \\
Giant Boar, Challenge 2 (450 PX) \\
Gelatinous Cube, Challenge 2 (450 PX) \\
Thorny Devil, Challenge 2 (450 PX) \\
White Dragon Pup, Challenge 2 (450 PX) \\
Silver Dragon Pup, Challenge 2 (450 PX) \\
Bronze Dragon Pup, Challenge 2 (450 PX) \\
Black Dragon Pup, Challenge 2 (450 PX) \\
Green Dragon Pup, Challenge 2 (450 PX) \\
Druid, Challenge 2 (450 PX) \\
Lesser Water Elemental, Challenge 2 (450 PX) \\
Ettercap, Challenge 2 (450 PX) \\
Bubbling Maw, Challenge 2 (450 PX) \\
Wisp, Challenge 2 (450 PX) \\
Gargoyle, Challenge 2 (450 PX) \\
Ghast, Challenge 2 (450 PX) \\
Grick, Challenge 2 (450 PX) \\
Griffin, Challenge 2 (450 PX) \\
Sea Hag, Challenge 2 (450 PX) \\
Mimic, Challenge 2 (450 PX) \\
Ogre, Challenge 2 (450 PX) \\
Polar Bear, Challenge 2 (450 PX) \\
Pegaso, Challenge 2 (450 PX) \\
Plesiosaurus, Challenge 2 (450 PX) \\
Werewolf, Challenge 2 (450 PX) \\
Rhino, Challenge 2 (450 PX) \\
Priest, Challenge 2 (450 PX) \\
Minotaur Skeleton, Challenge 2 (450 PX) \\
Swarm of Venomous Snakes, Challenge 2 (450 PX) \\
Giant Constrictor Snake, Challenge 2 (450 PX) \\
Hissing, Challenge 2 (450 PX) \\
Shark Hunter, Challenge 2 (450 PX) \\
Carpet of Suffocation, Challenge 2 (450 PX) \\
Flaming Skull, Challenge 2 (200 PX) \\
Saber-Tooth Tiger, Challenge 2 (450 PX) \\
Zombi Ogre, Challenge 2 (450 PX) \\
Killer Whale (Orca), Challenge 3 (700 PX) \\
Basilisk, Challenge 3 (700 PX) \\
Knight, Challenge 3 (700 PX) \\
Nightmare Steed, Challenge 3 (700 PX) \\
Bearded Devil, Challenge 3 (700 PX) \\
Doppelganger, Challenge 3 (700 PX) \\
Blue Dragon Pup, Challenge 3 (700 PX) \\
Golden Dragon Pup, Challenge 3 (700 PX) \\
Winter Wolf, Challenge 3 (700 PX) \\
Werewolf, Challenge 3 (700 PX) \\
Manticore, Challenge 3 (700 PX) \\
Green Crone, Challenge 3 (700 PX) \\
Minotaur, Challenge 3 (700 PX) \\
Mummy, Challenge 3 (700 PX) \\
Wall Climb Horror, Challenge 3 (700 PX) \\
Orsogufo, Challenge 3 (700 PX) \\
Wise Beardog, Challenge 3 (700 PX) \\
Spider Phase, Challenge 3 (700 PX) \\
Giant Scorpio, Challenge 3 (700 PX) \\
Hellhound, Challenge 3 (700 PX) \\
Veteran, Challenge 3 (700 PX) \\
Wight, Challenge 3 (700 PX) \\
B.O.C., Challenge 4 (1,100 PX) \\
Banshee, Challenge 4 (1,100 PX) \\
Chuul, Challenge 4 (1,100 PX) \\
Wereboar, Challenge 4 (1,100 PX) \\
Couatl, Challenge 4 (1,100 PX) \\
Red Dragon Pup, Challenge 4 (1,100 PX) \\
Elephant, Challenge 4 (1,100 PX) \\
Ettin, Challenge 4 (1,100 PX) \\
Ghost, Challenge 4 (1,100 PX) \\
Lamia, Challenge 4 (1.100 PX) \\
Cursed Immortal, Challenge 4 (1,100 PX) \\
Protoplasm Black, Challenge 4 (1.100 PX) \\
Succubus, Challenge 4 (1.100 PX) \\
Werewolf Tiger, Challenge 4 (1,100 PX) \\
Dark Torch, Challenge 4 (1,100 PX) \\
Tentacled Creeping Worm, Challenge 4 (1,100 PX) \\
Bulette, Challenge 5 (1,800 PX) \\
Giant Crocodile, Challenge 5 (1,800 PX) \\
Creeping Cumulus, Challenge 5 (1,800 PX) \\
Fire Elemental, Challenge 5 (1,800 PX) \\
Water Elemental, Challenge 5 (1,800 PX) \\
Elemental of Air, Challenge 5 (1,800 PX) \\
Earth Elemental, Challenge 5 (1,800 PX) \\
Flogger, Challenge 5 (1,800 PX) \\
Hill Giant, Challenge 5 (1,800 PX) \\
Gladiator, Challenge 5 (1,800 PX) \\
Flesh Golem, Challenge 5 (1,800 PX) \\
Gorgone, Challenge 5 (1.800 PX) \\
Night Crone, Challenge 5 (1,800 PX) \\
Werebear, Challenge 5 (1,800 PX) \\
Otyugh, Challenge 5 (1,800 PX) \\
Salamander, Challenge 5 (1,800 PX) \\
Giant Shark, Challenge 5 (1,800 PX) \\
Triceratops, Challenge 5 (1,800 PX) \\
Troll, Challenge 5 (1,800 PX) \\
Unicorn, Challenge 5 (1,800 PX) \\
Wraith, Challenge 5 (1,800 PX) \\
Xorn, Challenge 5 (1,800 PX) \\
Chimera, Challenge 6 (2,300 PX) \\
Young White Dragon, Challenge 6 (2,300 PX) \\
Young Brass Dragon, Challenge 6 (2,300 PX) \\
Drider, Challenge 6 (2,300 PX) \\
Great Magician, Challenge 6 (2,300 PX) \\
Mammut, Challenge 6 (2,300 PX) \\
Medusa, Challenge 6 (2,300 PX) \\
Invisible Persecutor, Challenge 6 (2,300 PX) \\
Vampiric Spawn, Challenge 6 (1,800 PX) \\
Wyvern, Challenge 6 (2,300 PX) \\
Vrock, Challenge 6 (2,300 PX) \\
Young Copper Dragon, Challenge 7 (2,900 PX) \\
Young Black Dragon, Challenge 7 (2,900 PX) \\
Stone Giant, Challenge 7 (2,900 PX) \\
Guardian Protector, Challenge 7 (2,900 PX) \\
Oni, Challenge 7 (2,900 PX) \\
Giant Monkey, Challenge 7 (2,900 PX) \\
Assassin, Challenge 8 (3,900 PX) \\
Devil of the Chains, Challenge 8 (3,900 PX) \\
Young Bronze Dragon, Challenge 8 (3,900 PX) \\
Young Green Dragon, Challenge 8 (3,900 PX) \\
Frost Giant, Challenge 8 (3,900 PX) \\
Hezrou, Challenge 8 (3,900 PX) \\
Idra, Challenge 8 (3.900 PX) \\
Killer Shroud, Challenge 8 (3,900 PX) \\
Spiritual Naga, Challenge 8 (3,900 PX) \\
Tyrannosaurus, Challenge 8 (3,900 PX)\\
Bone Devil, Challenge 9 (5000 PX) \\
Eater Brains, Challenge 9 (5000 PX) \\
Young Blue Dragon, Challenge 9 (5000 PX) \\
Young Silver Dragon, Challenge 9 (5000 PX) \\
Greater Water Elemental, Challenge 9 (5000 PX) \\
Fire Giant, Challenge 9 (5000 PX) \\
Cloud Giant, Challenge 9 (5000 PX) \\
Glabrezu, Challenge 9 (5000 PX) \\
Clay Golem, Challenge 9 (5000 PX) \\
Tree Man (Treant), Challenge 9 (5000 PX) \\
Aboleth, Challenge 10 (5.900 PX) \\
Angelo Deva, Challenge 10 (5.900 PX) \\
Young Gold Dragon, Challenge 10 (5.900 PX) \\
Young Red Dragon, Challenge 10 (5.900 PX) \\
G.E.C., Challenge 10 (5.900 PX) \\
Stone Golem, Challenge 10 (5,900 XP) \\
Naga Guardian, Challenge 10 (5,900 PX) \\
Behir, Challenge 11 (7,200 PX) \\
Horned Devil, Challenge 11 (7,200 PX) \\
Djinni, Challenge 11 (7,200 PX) \\
Efreeti, Challenge 11 (7,200 PX) \\
Ginosfinge, Challenge 11 (7,200 PX) \\
Remorhaz, Challenge 11 (7,200 PX) \\
Archmage, Challenge 12 (8,400 PX) \\
Erinyes, Challenge 12 (8,400 PX) \\
Panopticon, Challenge 12 (8,400 PX) \\
Adult White Dragon, Challenge 13 (10000 PX) \\
Adult Brass Dragon, Challenge 13 (10000 PX) \\
Storm Giant, Challenge 13 (10000 PX) \\
Nalfeshnee, Challenge 13 (10000 PX) \\
Rakshasa, Challenge 13 (10000 PX) \\
Vampire, Challenge 13 (10000 PX) \\
Ice Devil, Challenge 14 (11,500 PX) \\
Adult Copper Dragon, Challenge 14 (11,500 PX) \\
Adult Bronze Dragon, Challenge 15 (13000 PX) \\
Adult Green Dragon, Challenge 15 (13000 PX) \\
Phoenix, Challenge 15 (13000 PX) \\
Sovereign Mummy, Challenge 15 (13000 PX) \\
Purple Worm, Challenge 15 (13000 PX) \\
Angelo Planetar, Challenge 16 (15000 PX) \\
Adult Blue Dragon, Challenge 16 (15000 PX) \\
Adult Silver Dragon, Challenge 16 (15000 PX) \\
Iron Golem, Challenge 16 (15000 PX) \\
Marilith, Challenge 16 (15000 PX) \\
Androsfinge, Challenge 17 (18000 PX) \\
Adult Gold Dragon, Challenge 17 (18000 PX) \\
Adult Black Dragon, Challenge 17 (18000 PX) \\
Adult Red Dragon, Challenge 17 (18000 PX) \\
Dragon Turtle, Challenge 17 (18000 PX) \\
Black Knight, Challenge 18 (20000 PX) \\
Balor, Challenge 19 (22000 PX) \\
Pit Devil, Challenge 20 (25000 PX) \\
Ancient White Dragon, Challenge 20 (25000 PX) \\
Ancient Brass Dragon, Challenge 20 (25000 PX) \\
Angelo Solar, Challenge 21 (33000 PX) \\
Ancient Copper Dragon, Challenge 21 (33000 PX) \\
Ancient Black Dragon, Challenge 21 (33000 PX) \\
Lich, Challenge 21 (33000 PX) \\
Ancient Bronze Dragon, Challenge 22 (41000 PX) \\
Ancient Green Dragon, Challenge 22 (41000 PX) \\
Ancient Blue Dragon, Challenge 23 (50000 PX) \\
Ancient Silver Dragon, Challenge 23 (50000 PX) \\
Ancient Yellow Dragon, Challenge: 23 (50000 PX) \\
Kraken, Challenge 23 (50000 PX) \\
Ancient Gold Dragon, Challenge 24 (62000 PX) \\
Ancient Red Dragon, Challenge 24 (62000 PX) \\
Demogorgon, Challenge 26 (90000 PX) \\
Orcus, Challenge 26 (90000 PX) \\
Tàhil, Challenge 30 (155000 PX) \\
Tarrasque, Challenge 30 (155000 PX) \\

}

\end{multicols}

\pagebreak


\subsection{Monster Conversion}\index{Monster Conversion}

\bigskip

For adding new mosters on OBSS i suggest to convert from Pathfinder or 5ed. OBSS i loosely based on d0 so many numbers value's are the same even if gaming rule's are different.

\medskip

\textbf{Conversione da Pathfinder}

For example use Orc

https://www.d20pfsrd.com/bestiary/monster-listings/humanoids/orcs/orc/

skip narrative description and concentrate on numbers:

\bigskip

\textbf{Orc (grado di Sfida 1/3)} same value on OBSS

\textbf{XP 135} search on OBSS px table's

\textbf{Orc warrior 1} that's no mean on OBSS. We have no class

\textbf{CE Medium humanoid} Evil creature humanoid, medium sized

\textbf{Init +0} this is inititive. Check Dex or Int bonus

\textbf{Senses} darkvision 60 ft.; Perception -1: same value

\textbf{Weakness} light sensitivity : just use the penalty reported or search on disavantages for a like penalty

\textbf{AC} 13, touch 10, flat-footed 13 (+3 armor):  as Defense, same value

\textbf{Competenza Armi}: Weapon Profiency, if not indicated a Base Atk bonus just give +1 WP for CR

\textbf{Competenza Magica}: only useful if creature is a magic user, otherwise ignore

\textbf{hp} 6 (1d10+1) same walue on OBSS

\textbf{Fort} +3, Ref +0, Will -1 : same value on OBSS

\textbf{Speed} 30 ft. Same value on OBSS

\textbf{Melee} Falcione +5 (2d4+4/18--20) : use same value on OBSS both for to hit and damage

\textbf{Ranged} javelin +1 (1d6+3): as above. Just use it

\textbf{Str} 17, Dex 11, Con 12, Int 7, Wis 8, Cha 6. On OBSS we use only bonus value

\textbf{Base Atk} +1; CMB +4; CMD 14. First value is WP.

\textbf{Feats} Weapon Focus (Falcione) Arma Focalizzata. Just use the reported bonus

\textbf{Skills} Intimidate +2: as in OBSS

\textbf{Ferocity} (Ex): An orc remains conscious and can continue fighting even if its hit point total is below 0. It is still staggered and loses 1 hit point each round. A creature with ferocity still dies when its hit point total reaches a negative amount equal to its Constitution score. Just use the same it.


\bigskip

\textbf{Conversione dalla 5e} del famoso Gioco di Ruolo:

- Aumentate la Difesa ed i Tiri per Colpire di 1/2 del Grado di Sfida della creatura, arrotondate per eccesso.

- Per i Tiri Salvezza prendete come base il Grado di Sfida della creatura e poi applicate il modificatore di caratteristica relativo a Destrezza (Riflessi), Costituzione (Tempra), Volontà (Saggezza).


\pagebreak


\section{Conditions}\index{Conditions}

\begin{changemargin}{0cm}{0.5cm}\begin{enfasi}{
The greatness of man is in the decision to be stronger than his condition. (Albert Camus)
}
\end{enfasi}\end{changemargin}\medskip

%{\small
\begin{multicols}{2}

\label{condizioni}

\textbf{Blinded}: \index{Blinded} The creature is unable to see well due to excessive stimulation of the eyes. A dazzled creature takes a -1d6 penalty on attack rolls and Awareness checks based on sight.

\textbf{Blinded}: \index{Blinded} The character cannot see anything. he takes a --2 penalty to most Skills based on Strength and Dexterity.

All vision-based checks or activities (such as reading, or any vision-based Awareness checks) fail automatically. All opponents are considered to have invisibility towards the blinded character.

Blinded characters must make an Acrobatics check with DC 12 to move faster than their half speed. Creatures that fail this check fall prone. Characters who remain blinded for a long time may get used to some of these penalties and begin to overcome some, at the discretion of the Storyteller.

Whoever attacks a creature invisible to her has a -2d6 to attack roll, invisible creature attacking a creature that does not see it has + 1d6 to attack roll.

\textbf{Crouch}: \index{Crouch} A crouched character takes a -2 penalty on Defense and -1 on attack rolls, has a -2 on Dexterity checks.

\textbf{Fascinated}: \index{Fascinated} A fascinated creature is subjugated by a supernatural effect or a charm spell. The creature stands or sits quietly, taking no action other than paying attention to the source of the charm, as long as the effect lasts. The effect causes a -1d6 penalty on Proficiency Checks required as a reaction, such as Awareness checks.

Any potential threat, such as an approaching hostile creature, allows the fascinated creature a new saving throw against the charm effect. Any overt threat, such as someone drawing a weapon, casting a spell, or aiming a ranged weapon at the fascinated creature, automatically stops the effect.

An ally of the fascinated creature can shake it to free it from the effect by spending 2 Actions.

The charmer has + 1d6 on any ability checks for interacting socially with the creature.

\textbf{Fatigued} \index{Fatigued} \ hypertarget {fatigued}{}: A fatigued character cannot run or Charge and suffers a -1 penalty to Strength, Fortitude, and Dexterity. If he does anything that is normally fatiguing, he becomes Exhausted.

It takes 8 hours of total rest to remove the fatigued condition. If a character does not sleep at least 6 hours or sleeps in medium or heavy armor in the morning, he is fatigued.

\textbf{Grabbed} \index{Grabbed}: A grabbed character cannot move, he must use two Actions to break free (opposed Fortitude save). It loses the Dexterity bonus on Defense and Reflex saving throws.

She can attack with melee weapons if adequate (it is unlikely she will be able to use a broadsword, halberd .. a dagger or short sword is more likely).

\textbf{Friendly}: \index{Friendly} A friendly creature will not attack a character unless explicitly threatened.

\textbf{Drowning / Holding your breath}: \index{Drowning} \index{Holding your breath} Any character can hold his breath for a number of rounds equal to 6 rounds for his Constitution score, with a minimum of 3 rounds. For each Action performed, the remaining duration decreases by 1 round. After this period of time has elapsed, the character must make a DC 12 Fortitude save each round to continue holding his breath. Each round, the DC increases by 1.

\textbf{Deafened}: \index{Deafened} \index{Deaf} A deafened character cannot hear. He takes a --2 penalty on initiative checks, automatically fails all sound-based Awareness checks, and is considered Distracted when casting spells with at least verbal components.

Characters who are stunned for long periods of time can get used to these penalties and overcome some of them, at the discretion of the Storyteller.

\textbf{Poisoned} \index{Poisoned}: Any subject under the influence of a poison or potion is considered poisoned, regardless of whether it is already producing the effects or has yet to produce them given the time of onset.

\textbf{Charmato}: \index{Charmato} a charmed creature treats the player with a trusted friend and ally. If the creature is threatened or attacked, it can make a new Will save with a +2.

The charm effect does not allow the control of the target but this perceives your words in the most favorable way. You can also issue orders but must succeed on an Intimidate check against a Will saving throw.

A charmed target will do nothing dangerous to themselves (unless convinced) or other subjects
i whom he considers friends.

\textbf{Strike of Grace}: \index{Strike of Grace} As the only action in the round, a creature can use a melee weapon to deliver a final blow to a helpless character. He can also use a bow or crossbow, as long as it is adjacent to the target.

The attacker automatically hits and deals three critical hits. Creatures that are immune to critical hits cannot suffer a blow of grace.

\textbf{Confused}: \index{Confused} A confused creature is mentally darkened and cannot act normally. A confused creature cannot distinguish an ally from an enemy and treats everyone as enemies.

Allies who want to use a spell to benefit the confused creature must still touch it with a successful melee touch attack.

If a confused creature is attacked, it always attacks the last creature that attacked it, until that creature dies or goes out of sight.

Roll a die on the following table at the start of each confused creature round each round to see what the creature does that round.

\textbf{d100 Behavior:}

01-25 Acts normally

26-50 He does nothing but stammer incoherently

51-75 Inflicts 1d8 + Strength modifier with the weapon in hand

76-100 Attack nearest creature (for this purpose, a Familiar counts as part of the subject itself)

A confused creature that is unable to perform the indicated action will only stammer inconsistently. Attackers have no special advantage when attacking a confused creature. Any confused creature that is attacked automatically attacks its attacker in turn on its next round, as long as it remains confused when its round comes.

\textbf{Distracted} \index{Distracted}: If the caster comes is severely \textbf{Distracted}, impeded, disturbed, bleeding, is under attack while trying to cast a spell he must make a Magic Check.

If the check rolls a critical success the spell fails but suffers no consequences, if the check rolls a critical success it is not counted.

A failed distraction spell does not subtract Magic Points.

\textbf{Dominated}: \index{Dominated} one is able to control the actions of any humanoid creature through a telepathic bond with the subject's mind.

If you have a common language, you can generally force the subject to carry out commands within the limits of his ability. If you don't share any language, you can only issue basic commands like "come here", "go there", "fight" or "stay still". You are aware of what the subject is feeling but you are not receiving direct sensory perceptions from him, nor can you communicate with him telepathically.

Once an order is given to the dominated creature, it continues to attempt to execute it with the exclusion of all other activities except those necessary for daily survival (such as eating, sleeping, and so on). Thanks to this limited spectrum of activity, an Awareness check with DC 15 (instead of DC 25) can determine whether the subject's behavior has been affected by an enchantment effect.

By concentrating fully on the spell (2 Actions), one can receive sensory perceptions as interpreted by the subject's mind, even if the subject cannot communicate them anyway. You can't actually see through the subject's eyes, so it's not like you are there, but you can see what's going on.

Obviously, blatantly self-destructive orders are not carried out. Once control is established, the range within which it can be maintained is unlimited as long as both parties remain on the same level. There is no need to see the subject to check it. If you don't spend at least 1 minute focusing on the spell each day, the subject receives a new saving throw to break free of control.

\textbf{Sleeping} \index{Sleeping}: Whenever a character ends a 24-hour period without sleeping at least 8 hours, he must succeed at a DC 12 Fortitude save or he becomes fatigued. Any further missed rest will make him even more fatigued by accumulating the relative penalties.
If the character stays awake for several days, fighting sleep becomes more difficult. After the first 24 hours, the DC increases by 4 for each consecutive 24-hour period without sleep for 8 hours. The DC returns to 12 when the character completes a rest of at least 8 hours.

\textbf{Exhausted}: \index{Exhausted} \ hypertarget {exhausted}{} An exhausted character moves at half speed and takes a --2 penalty to Strength, Fortitude, and Dexterity. After 1 hour of complete rest (or Less Restoration), an Exhausted character becomes Fatigued. A Fatigued character becomes Exhausted by taking an action that they do notrmally it would tire him.

An Exhausted creature if affected by additional fatigue effects increases its fatigue level.

\medskip

\textbf{Table: Fatigue Levels} \index{Fatigue Levels Table}

\medskip

\begin{tabularx}{0.45\textwidth}{lcl}
	\textbf{Conditions}&\textbf{Malus}&\textbf{Recovery}\\
	&\textbf{For - Des - Cos}& \\
	\hline
	see \textbf{Fatigued} & 1 & 1h \\
	see \textbf{Exhausted} (1) & 1 & 1h \\
	Exhausted (2) & 1 & 8h \\
	Exhausted (3) & 2 & 24h \\
	Exhausted (4) & Death & - \\
\end{tabularx}

The penalties indicated are cumulative. After 8 hours of rest you go from Exhausted 2 to Exhausted 1. After 24 hours of rest you recover from Exhausted 3 to Exhausted 2.

\textbf{Disabled} \index{Disabled}: An incapacitated creature cannot perform actions or reactions.

\textbf{Unprepared}: \index{Unprepared} A character who has not yet acted in combat is unprepared, unable to react to the situation yet. An unprepared character loses his Dexterity bonus on Defense (if any).

\textbf{Fighting}: \index{Fighting} A grappling creature is held back by a creature, trap, or effect. Battling creatures can't move and take a --2 penalty to Dexterity. A grappling creature takes a --2 penalty on attack and defense rolls. Also, grappling creatures can't perform actions that require two hands to perform.

\textbf{Incorporeal}: \index{Incorporeal} Creatures of this type do not possess a physical body. Incorporeal creatures can only be hit by magical weapons with a +2 or greater bonus. Halve the effects of spells that are not specified to work on incorporeal creatures. Incorporeal creatures take full damage from other subjects and incorporeal effects, as well as all force effects.

An incorporeal creature can enter or pass through a corporeal object.

An incorporeal creature's attacks pass through (ignore) non-magical armor and shields, only natural Dexterity and indeed magical armor / shields offer resistance.

Incorporeal creatures can move and act normally in water as well as in air. Incorporeal creatures cannot fall and take falling damage.

Incorporeal creatures can't make trips to trip or grapple, nor can they be tripled or grabbed.

Incorporeal creatures have no weight, and do not trigger weight-activated traps.

An incorporeal creature always moves silently and cannot be felt with Awareness unless desired. He has no Strength score, and his Dexterity bonus applies to melee and ranged attacks

\textbf{Helpless}: \index{Helpless} \ hypertarget {dying}{} A character asleep, Unconscious, Dying, or for some other reason completely at the mercy of his opponents, is considered Defenseless.

A Helpless creature is incapacitated (see condition), cannot move or speak, and is unaware of what is happening around it.
The creature drops whatever it holds and falls prone. The creature automatically fails Fortitude and Reflex saving throws.
A helpless character is treated as having Dexterity 0 and no shield bonuses are considered. Melee attacks against a defenseless character gain a +2 bonus (equivalent to attacking a Prone character) in addition to the fact that he has no Dexterity bonus to Defense.

Ranged attacks do not gain any particular bonuses against helpless targets, other than the fact that there is no Defense bonus given by Dexterity.

Cannot perform actions.

\textbf{Entangled}: \index{entangled} \hypertarget{entangled}{} An entangled character has difficulty moving, but can still try to move, unless the ties that obstruct him are anchored to an immobile object or contested by an opposing force.

An entangled creature can move at half speed and cannot run or charge, and takes a --2 penalty on attack rolls and a --2 penalty on Dexterity checks.

A hampered character who tries to cast a spell is considered Distracted.

\textbf{Immobilized}: \index{Immobilized} An immobilized creature is strictly limited in movement and can only perform certain actions.

A pinned creature cannot move and is unprepared. A pinned creature can always attempt to break free, usually through an Escape Artist check or a Reflex saving throw.

He can perform verbal and mental actions, but can hardly use spells that have somatic or material components. An immobilized character who attempts to cast spells and use a spell-like ability is considered to be disturbed and must achieve a critical spell success to succeed in casting the spell.

If the subject is tied up, the evidence is against the Rope Use or Survival check of the tied person.

\textbf{Invisible}: \index{Invisibleand} Invisible creatures are not perceptible to sight.
Whoever attacks a creature invisible to her has a -1d6 attack roll, the invisible creature attacking an unseen creature has + 1d6 to attack roll.

\textbf{Dying} \index{Dying}: A dying character has -1 hit points and is considered helpless for penalties and is close to death. Each round he loses 1 hit point until he dies or is healed.

\textbf{Dead}: \index{Dead} \hypertarget{dead}{} The character's soul permanently leaves his body. Dead characters cannot benefit from normal or magical healing, and cannot be brought back to life by a spell. Only a Patron has enough power to bring the soul back into the body and bring the creature back to life. The Necromancy School has spells to revive a body as undead.

\textbf{Nauseated}: \index{Nauseated} Nauseated creatures suffer from stomach upset.
Nauseated creatures are unable to attack, use spells, focus on spells, or do anything else that requires attention. The only action such a character can take is a single move action per round.

\textbf{Paralyzed}: \index{Paralyzed} A paralyzed character is stuck in place and unable to move or act. He has effective Strength and Dexterity scores of -5, is Helpless, and can only perform mental actions.

A winged creature in flight, the moment it is paralyzed, can no longer flap its wings and falls.
A paralyzed swimmer can no longer swim and could drown.

A creature can traverse an area occupied by a paralyzed (or dead) creature, whether it is an ally or not, and is treated as hindering terrain.

\textbf{Skill point loss} \index{Skill point loss} \index{Skill point loss}: when Skill scores decrease remember to remove any 1 hit points per Constitution point lost per level, lower saving throws (Dexterity , Constitution, Wisdom), Attack Rolls (Strength and Dexterity), Defense (Defense). If not indicated as permanent, you recover 1 point in all of your Characteristic per day of rest.

\textbf{Petrified}: \index{Petrified} A petrified character has been turned to stone and is unconscious and helpless. If a petrified character cracks or breaks, but the broken pieces are joined to the body when it becomes flesh, the character is not injured or damaged. If the character's petrified body is incomplete when transformed back into flesh, the body remains incomplete and may have some permanent loss of hit points and / or other impairments.

\textbf{Fear, Scared}: \index{Fear} \index{Scared} Spells, magical items, and certain creatures can affect characters with fear. In many cases, the character must make a Will saving throw to resist the effects, and a failed roll indicates that the character is Frightened. A frightened creature has -1d6 on attack rolls, saving throws, and proficiency checks as long as the source of its fear is visible. A frightened creature cannot voluntarily approach the source of its fear.

\textbf{Prone} \index{Prone}: whoever is prone has -1d6 to attack and -4 to Defense. Getting up prone costs 2 Actions. You cannot go prone if you fly.
The only movement option for a prone creature is to crawl (difficult terrain), unless it gets up and ends its condition.
The character can perform an Acrobatics check, if it is equal to or greater than 13 he gets up with 1 Action, if he makes three 1s in the check he remains Prone and consumes all 3 Actions.
When the Acrobatics score reaches 6, getting up prone costs 1 Action, with an Acrobatics check at DC 15 costs a Reaction.

\textbf{Maximum Hit Points} \index{Maximum Hit Points}: A creature that takes an attack that lowers its maximum hit points must first decrease its current maximum hit points and then decrease its current hit points by the same amount if not already removed. If the maximum hit points reach 0, the creature is dead. The maximum hit points are recovered to the extent of 1 per Constitution value for 8 hours of rest.

\textbf{Damage Resistance} \index{Damage Resistance}: A creature that has Damage Resistance is considered to automatically halve damage from the specified source, eg. Damage Resistance: Sound. The Resistance to Damage can also be indicated with a numerical value, eg. Damage Resistance: Fire 10. In this case the protection works on the first 10 damage suffered, in case of an effect that grants a saving throw to halve, first the amount of protection is removed from the total, then the saving throw is made to halve residual damage.

\textbf{Stunned / Fainted}: \index{Stunned} \index{Fainted} is considered to be helpless.

\textbf{Hold}the breath: COS * 6 rounds. Each Action -1 round. Spell -3 rounds. Then DC Fortitude save 12 +1 per round.

\textbf{Restricted} \index{Restricted} \ hypertarget {Restricted}{}: Two medium or small creatures sharing the same map square are considered restricted. Both creatures take -1d6 to attack rolls and to Defense (-4) as long as they share the space. A creature can share the square with a creature of at least three times the size without penalty.

\textbf{Broken} \index{Broken}: The broken condition has the following effects, depending on the object:

- If the object is a weapon, all attacks made with the object take a --2 penalty on the attack roll and damage. Such weapons only get a critical hit with a natural 3 times 6 and only deal 1 time maximum damage in addition.

- If the item is armor or shield, the bonus it grants to Defense is halved, rounded down. Broken armor doubles the armor penalty on the Skill Check.

- If the item is a tool required for a Proficiency, all Proficiency checks made with it suffer a -2 penalty.

- If the item is a Wand or a Staff, use double the necessary charges each time it is used.

- If the item does not fit into any of the above categories, the broken condition has no effect on its use. Items with broken condition, regardless of type, are worth 25 \% of their normal cost. If the item is magical, it can only be repaired with the fabricate spell used by a spellcaster of a level equal to or higher than the one who created the item.

\textbf{Bleed Damage}\index{Bleeding}\index{Bleed} \hypertarget{Bleeding}{}: A creature that is taking bleeding damage takes the indicated amount of damage at the start of its round. Bleeding can be reduced by making a First Aid check with DC 12, 2 Actions.
For each Bleed value above 1 the difficulty increases by 2. Cost \textbf{2 Actions}. A 1 minute treatment guarantees 1 success, without trial. Each critical hit reduces the bleeding by one more point. Some bleeding effects cause ability damage or even ability drain.
Each natural or magical wound healing reduces Bleed damage by 1, if the subject's hit points are restored to their maximum value, bleeding ends. \index{Bleeding and healing}

Unless otherwise noted, bleed damage stacks with a maximum of 5 hit points per round.

\textbf{Vulnerability} \index{Vulnerability}: works in reverse of Resistance. The damage is doubled before any saving throw.



\end{multicols}
%}

\pagebreak




\section{Author}\index{Author}

\bigskip

\flushleft{

\textsc{Author and Creator}: Andres Zanzani - azanzani@gmail.com

\bigskip
\textsc{Co-author}: Roberta Giorgini - madgiorgini@yahoo.it

\bigskip
\textsc{Contributions}: Federica Angeli - angelifdc@gmail.com

\bigskip

Playtesting: Fabrizio Bonetti, Emanuele Pezzi, Nicola Ricottone, Marco Valmori, Edoardo Zanzani, Isotta Zanzani, Federica Angeli, Samuele Mazzotti, Simona Bissi, Alberto Dolcini, Carlo Dall'Ara, SicuramenteNonMirko, Dario Galassi.

\bigskip

A special thanks to all my family who put up with me and supported me in these desperate years!

\bigskip

Powered by \Large\LaTeX\ \normalfont\& \Large\textbf{GitHub}

\bigskip

Andres Zanzani}

\vspace{1.57cm}

\section{Scheda, Manuale e Schermo}\index{Scheda}\index{Manuale}\index{Schermo}

\label{scheda-e-manuale}
{\normalsize


You are invited to download from GitHub \textbf{Old Bell School System} freely and without restrictions other than those expressed by the license.
The main site is \href{https://github.com/buzzqw/TUS}{https://github.com/buzzqw/TUS}

* This link for \textbf{OBSS's Manual}: \href{https://github.com/buzzqw/TUS/blob/master/OBSS/OBSS.pdf}{OBSS.pdf}

or l'url https://github.com/buzzqw/TUS/blob/master/OBSS/OBSS.pdf

\medskip

* This is for \textbf{character sheet}:
\href{https://github.com/buzzqw/TUS/blob/master/OBSS/OBSS-scheda.pdf}{OBSS-Scheda.pdf}

or url  https://github.com/buzzqw/TUS/blob/master/OBSS/OBSS-scheda.pdf

\medskip

* This is for \textbf{Storyteller Screen's}:
\href{https://github.com/buzzqw/TUS/blob/master/OBSS/screen.pdf}{screen.pdf}

or url  https://github.com/buzzqw/TUS/blob/master/OBSS/screen.pdf

\medskip

This is \textbf{changelog} which is updated with each significant commit \href{https://github.com/buzzqw/TUS/blob/master/OBSS/changelog.md}{changelog.md}

or \url {https://github.com/buzzqw/TUS/blob/master/OBSS/changelog.md}

\medskip

In the Project Version section of GitHub you will find the Manual, Form and Screen packages. Beware that they are certainly out of date.

Any report or advice you want to give me is more than welcome! Open an issue on GitHub!

\vspace{1cm}

\section{Thanks}\index{Thanks}\index{EditoriFolli}

A huge thanks to \href{http://www.editorifolli.it/gdr/dnd5/srd5/}{EditoriFolli} (http://www.editorifolli.it/gdr/dnd5/srd5/) and to his collaborators for the translation work of the SRD of the 5e. Without their work this manual would not have been possible.

\bigskip

Thanks to \href{https://github.com/ThomasJockin/readexpro}{Readex} (https://github.com/ThomasJockin/readexpro) for the font used. Readex is a highly readable font even for people with reading difficulties.


The font \textit{Italic} and \textbf{\textit{Italic Bold}} are taken from \href{https://dejavu-fonts.github.io/}{Dejavu} (https://dejavu-fonts.github)

}

\medskip

\begin{changemargin}{0.3cm}{0.3cm}\begin{enfasi}{
"And enjoy the game." Players' Guide to Immortals. Frank Mentzer
}\end{enfasi}\end{changemargin}\medskip

\thispagestyle{plain}
\begin{center}
\includepdf[pages={1,2},scale=0.85]{OBSS-scheda.pdf}
\end{center}

%\thispagestyle{plain}
%\begin{center}
%\begin{tikzpicture}[remember picture,overlay]
%	\node[anchor=south west,inner sep=0pt] at ($(current page.south west)+(1cm,1cm)$) {
%		\includegraphics[scale=0.93]{OBSS-scheda-0.png}
%	};
%\end{tikzpicture}
%\end{center}
%\pagebreak
%\thispagestyle{plain}
%\begin{center}
%\begin{tikzpicture}[remember picture,overlay]
%	\node[anchor=south west,inner sep=0pt] at ($(current page.south west)+(1cm,1cm)$) {
%		\includegraphics[scale=0.93]{OBSS-scheda-1.png}
%	};
%\end{tikzpicture}
%\end{center}

\pagebreak


\section{Le mie Opzioni}\index{Le mie Opzioni}

\justify

I am also a Storyteller and although I have built OBSS around my preferences there are some choices that I personally would make differently. Many of these choices are present in the manual as Options and have not become "official" so as not to deviate too much from the game from the standard canons.

At my table I usually use the following Options:

\begin{itemize}

\item
\hyperlink{etadelpersonaggio}{Age of Creature} pag. \pageref{etadelpersonaggio}

\item
\hyperlink{successoparziale}{Partial Success}  pag. \pageref{successoparziale}

\item
\hyperlink{varianteiniziativa}{Initiative Variant}, if i got experienced players. Pag. \pageref{varianteiniziativa}

\item
\hyperlink{tirocriticovariante}{Variant Critical Shot} it is up to the player whether to use it or not at the time of character creation. Pag. \pageref{tirocriticovariante}

\item
\hyperlink{lunicaregola}{The One Rule} to be used with unexperienced players. Pag. \pageref{lunicaregola}

\item
\hyperlink{magiaspecialista}{Specialista} if I want to facilitate high-level spellcasters specialists. Possibly Supreme Magic, but not at the same time

\item
\hyperlink{abilitaiconiche}{Abilità Iconiche} on long campaign. Pag. \pageref{abilitaiconiche}

\item
 \hyperlink{droghe}{Droghe} only in the case of groups made up of mature and leading adults. Pag. \pageref{droghe}

\end{itemize}


\vfill
{\small
\begin{multicols}{2}

For me OBSS must be played in a frank way, without too many thoughts and bizarre projects. OBSS is not made to kill the characters but in the same way it does not facilitate their survival, everything is up to the Storyteller to decide how to play. The key to the game is the Storyteller, the style of the players and the interest of the group, OBSS wants to offer the framework, the tools, to play the adventure.

Try to emphasize the scenes, also be theatrical in the descriptions, remove the patina from the clean and politically correct game. It always remains your world, your table and your game, try to give that immersion that is often a bit lost in the most modern systems.
When there is a fight, make it so! You must hear the clash of weapons, the clash of armor, the ozone in the air caused by lightning, the crackling burns of fireballs. An adventurer in OBSS is not a hero because he defeats the bad guys (maybe he is also for this reason) but he is because he survived.

The first OBSS adventure is the equivalent of the initiation rite to challenge, fate and death. In the small village where a group of orcs has decided to raid the character who ends his first adventure and defeats the orcs will be remembered as a hero, one of the few who came back alive and not because he defeated his enemies. For the character it will be the spark for other adventures.

It will be the thirst for glory, wealth, power if not the cruel fate to drag the characters into new dangerous adventures. As I said before, the characters in OBSS are not the chosen heroes, in fact, they are much more likely to be scoundrels who want to survive and possibly get rich.

You create the group, and I don't mean just as a set of players, but as a set of characters as well. A group where people respect and trust each other (possibly ...). Build adventures that involve everyone, where everyone can contribute. There will also have to be more \textit{stitched} adventures around one character but that doesn't \textbf{must} exclude others from participating, in the broadest term of the word.
Take advantage of these adventures to introduce the characters to each other, nothing unites more than the fear of dying!

Once the group is made, and it may take some time, then take advantage of the personal stories, clues and hypotheses created by the players to shape situations and happenings. Like a millstone this will continue to create situations, adventures and new plots to follow.

There may be difficulties in creating the group, unfortunately it happens. Try to talk to the problem player. Try to figure out if it's his character who doesn't \textit{work} with the party or if it's the player who doesn't quite understand the mechanics of the party.

This is why I always suggest you do the so-called \textbf{Session Zero}, where as a Storyteller you will outline the cornerstones of the adventure, what you expect from the characters, what are the basic rules of morality to follow. . There is nothing worse than a bunch of loose characters where everyone wants to do something different and doesn't care about \textit{common goal}.

It is very important to understand what the players like, every person and group wants a certain style of play and it is correct to try to please them. If the group wants political adventures, romantic dramas try to make them find satisfaction in the while of the adventure. If, on the other hand, they prefer to fight more then do not skimp on fights as long as they are consistent with the adventure itself.

Make it clear that you have to function as a set of players and characters to be able to play at their best and all have fun. No player must be above the others, only the Storyteller has the last word.

Finally, always be correct, for better or for worse. There will be more unfortunate sessions and others where the dice will find the right path, where the brilliant idea will save the group. Do not be the Storyteller who saves \textbf{always and in any case} the characters, a help from time to time can be there especially in the most unfortunate session, but respect the choices of the characters and the outcome of the dice. Remember that players have Fate Points that they can use unlike poor monsters !.

And finally an obvious fact: \textbf{have fun}, make an effort to ensure that the session has that mixture of tension, fun and satisfaction. You are people who want to play, have fun and be together, never forget that.

\end{multicols}}

\pagebreak

\section{Licenza}\index{Licenza}

\medskip

\begin{center}
\textbf{Old Bell School System (OBSS)} è licenziato con la \href{http://media.wizards.com/2016/downloads/DND/SRD-OGL_V5.1.pdf}{OGL 5.1}
\end{center}

 \medskip

{\scriptsize\justify
Permission to copy, modify and distribute the files collectively known as the System Reference Document 5.1 ("SRD5") is granted solely through the use of the Open Gaming License, Version 1.0a. This material is being released using the Open Gaming License Version 1.0a and you should read and understand the terms of that License before using this material.

The text of the Open Gaming License itself is not Open Game Content. Instructions on using the License are provided within the License itself. The following items are designated Product Identity, as defined in Section 1(e) of the Open Game License Version 1.0a, and are subject to the Conditions set forth in Section 7 of the OGL, and are not Open Content: Dungeons \& Dragons, D\&D, Player's Handbook, Dungeon Master, Monster Manual, d20 System, Wizards of the Coast, d20 (when used as a trademark), Forgotten Realms, Faerûn, proper names (including those used in the names of Spells or items), places, Underdark, Red Wizard of Thay, the City of Union, Heroic Domains of Ysgard, EverChanging Chaos of Limbo, Windswept Depths of Pandemonium, Infinite Layers of the Abyss, Tarterian Depths of Carceri, Gray Waste of Hades, Bleak Eternity of Gehenna, Nine Hells of Baator, Infernal Battlefield of Acheron, Clockwork Nirvana of Mechanus, Peaceable Kingdoms of Arcadia, Seven Mounting Heavens of Celestia, Twin Paradises of Bytopia, Blessed Fields of Elysium, Wilderness of the Beastlands, Olympian Glades of Arborea, Concordant Domain of the Outlands, Sigil, Lady of Pain, Book of Exalted Deeds, Book of Vile Darkness, Beholder, gauth, Carrion Crawler, tanar'ri, baatezu, Displacer Beast, Githyanki, Githzerai, Mind Flayer, illithid, Umber Hulk, Yuan-ti.
All of the rest of the SRD5 is Open Game Content as described in Section 1(d) of the License. The terms of the Open Gaming License Version 1.0a are as follows:


\medskip

\textbf{OPEN GAME License Version 1.0a}

\medskip

The following text is the property of Wizards of the Coast, LLC. and is Copyright 2000 Wizards of the Coast, Inc ("Wizards"). All Rights Reserved.

\medskip

\textbf{1}. Definitions: (a)"Contributors" means the copyright and/or trademark owners who have contributed Open Game Content; (b)"Derivative Material" means copyrighted material including derivative works and translations (including into other computer languages), potation, modification, correction, addition, extension, upgrade, improvement, compilation, abridgment or other form in which an existing work may be recast, transformed or adapted; (c) "Distribute" means to reproduce, License, rent, lease, sell, broadcast, publicly display, transmit or otherwise distribute; (d)"Open Game Content" means the game mechanic and includes the methods, procedures, processes and routines to the extent such content does not embody the Product Identity and is an enhancement over the prior art and any additional content clearly identified as Open Game Content by the Contributor, and means any work covered by this License, including translations and derivative works under copyright law, but specifically excludes Product Identity. (e) "Product Identity" means product and product line names, logos and identifying marks including trade dress; artifacts; creatures characters; stories, storylines, plots, thematic elements, dialogue, incidents, language, artwork, symbols, designs, depictions, likenesses, formats, poses, concepts, themes and graphic, photographic and other visual or audio representations; names and descriptions of characters, Spells, enchantments, personalities, teams, personas, likenesses and Special abilities; places, locations, environments, creatures, Equipment, magical or supernatural Abilities or Effects, logos, symbols, or graphic designs; and any other trademark or registered trademark clearly identified as Product identity by the owner of the Product Identity, and which specifically excludes the OPEN Game Content; (f) "Trademark" means the logos, names, mark, sign, motto, designs that are used by a Contributor to Identify itself or its products or the associated products contributed to the Open Game License by the Contributor (g) "Use", "Used" or "Using" means to use, Distribute, copy, edit, format, modify, translate and otherwise create Derivative Material of Open Game Content. (h) "You" or "Your" means the licensee in terms of this agreement.

\medskip

\textbf{2}. The License: This License applies to any Open Game Content that contains a notice indicating that the Open Game Content may only be Used under and in terms of this License. You must affix such a notice to any Open Game Content that you Use. No terms may be added to or subtracted from this License except as described by the License itself. No other terms or Conditions may be applied to any Open Game Content distributed using this License.

\medskip

\textbf{3}.Offer and Acceptance: By Using the Open Game Content You indicate Your acceptance of the terms of this License.

\medskip

\textbf{4}. Grant and Consideration: In consideration for agreeing to use this License, the Contributors grant You a perpetual, worldwide, royalty-free, nonexclusive License with the exact terms of this License to Use, the Open Game Content.

\medskip

\textbf{5}.Representation of Authority to Contribute: If You are contributing original material as Open Game Content, You represent that Your Contributions are Your original Creation and/or You have sufficient rights to grant the rights conveyed by this License.

\medskip

\textbf{6}.Notice of License Copyright: You must update the COPYRIGHT NOTICE portion of this License to include the exact text of the COPYRIGHT NOTICE of any Open Game Content You are copying, modifying or distributing, and You must add the title, the copyright date, and the copyright holder's name to the COPYRIGHT NOTICE of any original Open Game Content you Distribute.

\medskip

\textbf{7}. Use of Product Identity: You agree not to Use any Product Identity, including as an indication as to compatibility, except as expressly licensed in another, independent Agreement with the owner of each element of that Product Identity. You agree not to indicate compatibility or co-adaptability with any Trademark or Registered Trademark in conjunction with a work containing Open Game Content except as expressly licensed in another, independent Agreement with the owner of such Trademark or Registered Trademark. The use of any Product Identity in Open Game Content does not constitute a Challenge to the ownership of that Product Identity. The owner of any Product Identity used in Open Game Content shall retain all rights, title and interest in and to that Product Identity.

\medskip

\textbf{8}. Identification: If you distribute Open Game Content You must clearly indicate which portions of the work that you are distributing are Open Game Content.

\medskip

\textbf{9}. Updating the License: Wizards or its designated Agents may publish updated versions of this License. You may use any authorized version of this License to copy, modify and distribute any Open Game Content originally distributed under any version of this License.

\medskip

\textbf{10}. Copy of this License: You MUST include a copy of this License with every copy of the Open Game Content You Distribute.

\medskip

\textbf{11}. Use of Contributor Credits: You may not market or advertise the Open Game Content using the name of any Contributor unless You have written permission from the Contributor to do so.

\medskip

\textbf{12}. Inability to Comply: If it is impossible for You to comply with any of the terms of this License with respect to some or all of the Open Game Content due to statute, judicial order, or governmental regulation then You may not Use any Open Game Material so affected.

\medskip

\textbf{13}. Termination: This License will terminate automatically if You fail to comply with all terms herein and fail to cure such breach within 30 days of becoming aware of the breach. All sublicenses shall survive the termination of this License.

\medskip

\textbf{14}. Reformation: If any provision of this License is held to be unenforceable, such provision shall be reformed only to the extent necessary to make it enforceable.

\medskip

\textbf{15}. COPYRIGHT NOTICE Open Game License v 1.0a Copyright 2000, Wizards of the Coast, LLC. System Reference Document 5.1 Copyright 2016, Wizards of the Coast, LLC.; Authors Mike Mearls, Jeremy Crawford, Chris Perkins, Rodney Thompson, Peter Lee, James Wyatt, Robert J. Schwalb, Bruce R. Cordell, Chris Sims, and Steve Townshend, based on original material by E. Gary Gygax and Dave Arneson.}

\vfill

For any use of OBSS or its parts or ideas, please notify us in advance.

{\normalsize The images included in this manual are in the public domain or are unlicensed. In case of inclusion of copyrighted images by mistake, I invite you to report them for removal.}

\pagebreak


{\scriptsize\printindex}

\TotalBox{OBSS}

{\scriptsize\printindex[Incantesimi]}

\TotalBox{Incantesimi}

{\scriptsize\printindex[OggettiMagici]}

\TotalBox{OggettiMagici}

{\scriptsize\printindex[Mostruario]}

\TotalBox{Mostruario}

\end{document}
