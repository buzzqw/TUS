\documentclass[a4paper,twoside,openany]{book}
\usepackage{quoting}
\usepackage{tcolorbox}
\usepackage{tikz}
\usetikzlibrary{shadows}
\usepackage{multicol}
\usepackage{tocloft}
\usepackage{lmodern}
\usepackage{caption}
\usepackage[utf8]{inputenc}
%\usepackage[utf8x]{inputenc}
\usepackage[T1]{fontenc}  %\usepackage[B1,T1]{fontenc}
\usepackage{setspace}
\usepackage[a4paper]{geometry}
\geometry{verbose,tmargin=2cm,bmargin=2cm,lmargin=2cm,rmargin=2cm}  %std
\setcounter{secnumdepth}{-1}
\usepackage{booktabs}
\usepackage{url}
\usepackage[english]{babel}
\usepackage{setspace}
\usepackage{graphicx}
\usepackage{amssymb}
\usepackage{makeidx}
%\usepackage[allfiguresdraft]{draftfigure}  %senza figure, deve rimanere alla riga 24
\usepackage{multirow}
\usepackage{titlesec}
\usepackage[unicode=true, bookmarks=true,
pdftitle={OBSS - Old Bell School System},pdfauthor={Andres Zanzani},
breaklinks=false,pdfborder={0 0 1},backref=section,colorlinks=false]
{hyperref}
\hypersetup{colorlinks=true,linkcolor=blue,pdfcreator={LaTeX}}
\usepackage{bookmark}
\usepackage{yfonts}
\usepackage{lettrine}
\usepackage{calligra}
\renewcommand{\LettrineFontHook}{\calligra}
\usepackage{accanthis}
\usepackage{auncial}
\usepackage{fontspec}
\usepackage{ragged2e}

\setmainfont[Path=../altro/fonts/,BoldItalicFont=DejaVuSerif-BoldItalic.ttf,ItalicFont=DejaVuSerif-Italic.ttf,BoldFont=ReadexPro-bold.ttf,Ligatures=TeX,Scale=0.94]{ReadexPro-Regular.ttf} %togliere questa e fontspec per usare pdflatex

\usepackage{wrapfig}
\usepackage{fancyhdr}
\usepackage{tcolorbox}
\tcbuselibrary{skins}
\tcbset{colback=brown!10, fonttitle=\scshape}
\usepackage{imakeidx}
\usepackage{cancel}

\def\CountIndexOccurrences#1{%
\expandafter\newcount\csname #1\endcsname%
\expandafter\newcount\csname #1\endcsname%
\def\indexentry##1##2{\expandafter\advance\csname #1\endcsname 1}%
\IfFileExists{#1.idx}{\input{#1.idx}}{}%
}
\CountIndexOccurrences{OBSS}
\CountIndexOccurrences{Spells}
\CountIndexOccurrences{Monsters}
\CountIndexOccurrences{MagicItem}
\def\TotalBox#1{\vfill%
\fbox{There are \expandafter\the\csname #1\endcsname\ entries in this index}\par}
\makeindex[columns=4, title=Index, intoc=true]
\makeindex[columns=4, name=Spells, title=Spells List, intoc=true]
\makeindex[columns=4, name=Monsters, title=Monsters List, intoc=true]
\makeindex[columns=4, name=MagicItem, title=Magic Item List, intoc=true]
\usetikzlibrary{shapes.misc,calc}
\definecolor{lightgray}{gray}{0.95}
\usetikzlibrary{shapes.misc,calc}
\definecolor{lightgray}{gray}{0.95}
\usepackage{fancyhdr}
\pagestyle{fancy}
\fancyhf{} 
\fancyhead[LE,RO]{\leftmark}
\fancyhead[RE,LO]{}
\fancyfoot[C]{\thepage}
\renewcommand{\sectionmark}[1]{\markboth{#1}{}}


\fancypagestyle{plain}{%
	%% Clear all headers and footers
	\fancyhf{}
	%% Right headers on odd pages
	\fancyhead[RO]{%
		\rotatebox{90}{
			\begin{tikzpicture}[overlay,remember picture]
				\node[fill=lightgray,text=black,
				font=\footnotesize,
				inner ysep=12pt, inner xsep=20pt,
				rounded rectangle,anchor=east,minimum width=7cm,
				xshift=-60mm,yshift=-23mm, text height=0.4cm]
				at ($ (current page.north east) + (-1cm,-0cm) + (-4*\thesection cm,0cm) $)
				{\sffamily\itshape\small\nouppercase{\leftmark}};
			\end{tikzpicture}
		}
	}
	%% Left headers on even pages
	\fancyhead[LE]{%
		\rotatebox{90}{
			\begin{tikzpicture}[overlay,remember picture]
				\node[fill=lightgray,text=black,
				font=\footnotesize,
				inner ysep=12pt, inner xsep=20pt,
				rounded rectangle,anchor=east,minimum width=7cm,
				xshift=-60mm,yshift=-2mm, text height=0.4cm]
				at ($ (current page.north west) + (1cm,0cm) + (-4*\thesection cm,0cm) $)
				{\sffamily\itshape\small\nouppercase{\leftmark}};
			\end{tikzpicture}
		}
	}
	\renewcommand{\headrulewidth}{0pt}
	\renewcommand{\footrulewidth}{0pt}
}
\pagestyle{plain}
\fancyfoot[C]{\thepage}
\renewcommand{\sectionmark}[1]{\markboth{#1}{}}
\usepackage{xltabular}
\usepackage{tabularx}
\usepackage{pdfpages}
\usepackage{hyperref}
\usepackage{tikz}
\usepackage[absolute,overlay]{textpos}
\usepackage{etoolbox}
\usepackage{soul}
\raggedbottom
\usepackage{array}
\newcolumntype{L}[1]{>{\raggedright\let\newline\\\arraybackslash\hspace{0pt}}m{#1}}
\newcolumntype{k}[1]{>{\centering\let\newline\\\arraybackslash\hspace{0pt}}m{#1}}
\newcolumntype{R}[1]{>{\raggedleft\let\newline\\\arraybackslash\hspace{0pt}}m{#1}}
\newcolumntype{D}[1]{>{\centering}m{#1}}
\newcolumntype{M}[1]{>{\centering\arraybackslash}m{#1}}
\titleformat{\section}{\filcenter\huge\bfseries\accanthis}{\thesection}{1em}\textsc{}
\titleformat{\subsection}{\Large\bfseries\accanthis}{\thesubsection}{1em}\textsc{}
\titleformat{\subsubsection}{\normalsize\bfseries\accanthis}{\thesubsubsection}{1em}\textsc{}
\def\changemargin#1#2{\list{}{\rightmargin#2\leftmargin#1}\item[]}
\let\endchangemargin=\endlist
\setcounter{tocdepth}{3}
\newtcolorbox{narrator}{
enhanced, % enable advanced settings
%left = 3mm,
%width=0.45\textwidth,
left = 9mm, % pushes text away from the left edge by 10mm
sharp corners, % disables rounded corners
rounded corners = southeast, % "round" the bottom right corner
arc is angular, % make the "round" corner an angle
arc = 3mm, % controls corner cut
boxrule=0.6pt, % sets box line thickness
underlay={%
\path[fill=black] ([yshift=3mm]interior.south east)--++(-0.4,-0.1)--++(0.1,-0.2); % triangle
\path[draw=black,shorten <=-0.05mm,shorten >=-0.05mm] ([yshift=3mm]interior.south east)--++(-0.4,-0.1)--++(0.1,-0.2); % triangle edge
\path[fill=gray!50!black,draw=none] (interior.south west) rectangle node[brown!10]{\Huge\bfseries ?!} ([xshift=8mm]interior.north west);
},
drop fuzzy shadow }

\newtcolorbox{emphasis}{
enhanced,
arc=5pt,
boxrule=0.3pt
}

\usepackage{zref-savepos,graphicx}
\newcommand{\filltopageendgraphics}[2][]{%\filltopageendgraphics[width=.5\linewidth]{image-a}
\par
\zsaveposy{top-\thepage}% Mark (baseline of) top of image
\vfill
\zsaveposy{bottom-\thepage}% Mark (baseline of) bottom of image
\smash{\includegraphics[keepaspectratio=true,height=\dimexpr\zposy{top-\thepage}sp-\zposy{bottom-\thepage}sp\relax,#1]{#2}}%
\par
}

\usepackage{enumitem} %oppure \setlength{\leftmargini}{1.25em} % default 2.5em
\usepackage[framemethod=TikZ]{mdframed}


\begin{document}
\def \versione {0.89a} \fontsize{10}{12}\selectfont

\thispagestyle{empty} \tikz[remember picture,overlay]  \node[opacity=1] at (current page.center){\includegraphics[width=21cm,height=\pdfpageheight]{copertina.png}}; \begin{textblock*}{20cm}(10cm,7cm)\Huge {Old Bell School System}\\ \end{textblock*} \begin{textblock*}{22cm}(13.5cm,8cm) \Large {\textbf{(OBSS)}}\\ \end{textblock*} \begin{textblock*}{13cm}(9cm,11cm) \Huge{\color{black} \calligra\Huge{Fantasy Adventure Game}} \end{textblock*} \newpage~\thispagestyle{empty}  \newpage~\thispagestyle{empty} %deve rimanere su riga 180



\newcommand\invisiblesection[1]{%
  \refstepcounter{section}%
  \addcontentsline{toc}{section}{\protect\numberline{\thesection}#1}%
  \sectionmark{#1}}


\bigskip
Dedicated to the only Woman ever loved, the one who accompanies me every day in my dreams\\


Never give up on your wishes, persist until they become real.

\vspace{\fill}
\begin{center}\textbf{\versione} - \today\end{center}
\thispagestyle{empty}


\newpage~\thispagestyle{empty}%%\newpage~\thispagestyle{empty}


\newcommand{\riga}{\rule{\textwidth}{0.4pt}}


{\Huge \begin{center} Old Bell School System \end{center}}

\bigskip

\begin{center}{\LARGE Player and Arbiter Manual}\\ \end{center}

{\large \begin{center} Fantasy Role Playing Guide and Rules \end{center}}

\begin{center} by \end{center}

{\LARGE \begin{center} Andres Zanzani \end{center}}

\vspace{2cm}


\begin{center}
\includegraphics[keepaspectratio,width=0.50\textwidth]{immagini/copertina_old_scratch.png}
\end{center}

\vfill

\begin{mdframed}[roundcorner=10pt]

\medskip

\textbf{Playtesting}: Fabrizio Bonetti, Emanuele Pezzi, Leonardo Pezzi, Nicola Ricottone, Marco Valmori, Edoardo Zanzani, Isotta Zanzani, Federica Angeli, Samuele Mazzotti, Simona Bissi, Lucia Dolcini, Carlo Dall'Ara, SicuramenteNonMirko, Dario Galassi, Stefano Mannino, Francesco Converso.

\medskip

\begin{flushleft}
\textbf{Terms of Use}: OBSS, Old Bell School System, is a registered trademark of Andres Zanzani (azanzani@gmail.com), licensed under Attribution-ShareAlike 4.0 International (CC BY-SA 4.0). See details in the License chapter.
\end{flushleft}

\vspace{0.5cm}

\begin{center}
\includegraphics[keepaspectratio,width=0.25\textwidth]{immagini/CC_BY-SA_icon.svg.png}
\end{center}

\medskip

\end{mdframed}

\pagebreak

\setcounter{page}{1}

\begin{multicols}{2}
{\small \tableofcontents{}}

\end{multicols}

\vfill

\begin{changemargin}{0.3cm}{0.3cm}\begin{tcolorbox}
"May you make all your Saving Throws!" Frank Mentzer, Spring 1985. Master Player's Book
\end{tcolorbox}\end{changemargin}

\pagebreak

\section{Introduction}

\begin{changemargin}{0.3cm}{0.3cm}\begin{tcolorbox}[enhanced,arc=5pt,boxrule=0.3pt]{You can find out more about a person in an hour of gameplay than in a year of conversation. (Plato)}\end{tcolorbox}\end{changemargin}\medskip

\begin{multicols}{2}

\lettrine[lines=2, lhang=0.33, loversize=0.25, findent=1.5em]{O}{BSS} is a tabletop role-playing game (RPG) in which players create characters who will embark on fantastic and amazing adventures. The Arbiter will guide the story based on the choices of the characters. As in a storytelling game, each character will actively contribute to the story with their choices, decisions, and actions.

As a player, you will soon realize that the world is not easy and things are not given away or simple. There will be more instances where people want to harm and rob you than when they want offer you a drink. Don't become too attached to your character, as only the strong survive and the mighty reign. Show who you are and what you want. Do you want to be the hero in shining armor? Go ahead, but remember that a cunning rat is more likely to survive.

In any case, you will be in charge of deciding everything about your character, from their appearance, to their name, to their abilities, and partly even what they possess. Will they want to be a charismatic pirate or a shy knight, a steppe barbarian or a sorcerer? Gold, honor, treasure, and plunder will punctuate your character's adventures with choices, battles, and revelry. Your story will be sung by minstrels for centuries to come.

If you are the Arbiter, you rule the world, the story, and the adventure. Your role is to illustrate the scenario in which the players move and make decisions. Will you lead them deep into the earth in search of the forgotten Tome of Atmos or challenge the great Dragons for the crown of Omniscience?

Your task is not easy, use your imagination, common sense, and the main rule: have fun. When you're in trouble, don't look for the precise rule, use your imagination and a pinch of sanity to try to amaze the players. The purpose is always and only one, to have fun together and grow, as players, as characters, and as friends.

%\begin{center}
%\includegraphics[keepaspectratio,width=0.90\linewidth]{immagini/dice.png}
%
%\textit{Tipico set di dadi da gioco di ruolo}
%\end{center}

In addition to this manual, you will also need some dice, the classics used in role-playing games. You will need the d4, d6, d8, d10, and d12. They indicate a 4-sided die, a 6-sided die (you must have at least 3 or 4 of these), an 8-sided die, a 10-sided die (usually sold in pairs to obtain a d100), and a 12-sided die. Whenever you are asked to roll a die, it will be written as XdZ, meaning roll X times a die with Z sides. For example, 4d6 means to roll a 6-sided die 4 times.

Miniatures may be necessary, but other objects like gift from snacks or chocolate eggs can also be used. Inside this manual, you will find everything you need, such as rules, to play. You will need imagination, friendship, dice, a few sheets of paper, and fun (sorry, chips and drinks are not included in the manual!)

Draw and use a map whenever the description and the situation require accurate details and precise positioning. Otherwise, close your eyes and use your imagination, let the narrator's voice accompany you and reconstruct the structures and locations in your mind.

Create and play the character that best suits you, that feels yours and makes you enjoy yourself. Don't look for combinations of skills and abilities that give you more power, as sooner or later the character will become boring. The more you play, the more experience your character will gain, and you will play them better too.

The Arbiter will take care of informing you of how much experience your character has gained based on how you played,how you collaborated and helped the group, how much you contributed to the fun. He will keep you engaged in dangerous, often deadly fights, he will put your character to the test and as a group you will, perhaps not always, be able to solve the intricate situations that the Arbiter has prepared. Remember that the Arbiter always has the last word in any discussion.

This manual is complete, i.e. inside you will find everything (apart from the dice!) to start playing!

You will also find many rules, yet many situations will have to be handled using the first rule: have fun. Common sense, experience and trust in the Arbiter will resolve any situation.

Whether you decide to be the Arbiter or you decide to play a character, you need to read the following chapters carefully.
It's important that you have a good understanding of the basic rules and, above all, know where to look for anything when you need it!

\textit{For the experts...}

At the core of OBSS lies dissatisfaction with playing the 5th edition of the famous role-playing game. The 5th edition flattens characters too much and, although the rule system is truly efficient, it does not allow for the diversification, even if often exaggerated, that you could have in Pathfinder.

I needed a middle ground, a game still based on the d20 but taking the best of what had already been created and adding what I liked from the countless role-playing games I had studied and played. Don't try to recreate the classes of the 5th edition or Pathfinder, you won't succeed, and OBSS doesn't want to lend itself to this task! In OBSS, classes do not exist, and characters gain depth and ability depending on what they learn to do. Skills are dictated by the chosen profession and do not reach exaggerated scores. Combat does not reach the epic complexity of Pathfinder or the flatness of the 5th edition, but instead tries to be quick and tactical, effective and spectacular. The Golden Rules and critical management give that little something extra that allows players to have fun every time a die is rolled.

Magic follows the standard canons of the 5th edition but is deeply revisited. Many spells have lost concentration, and the concept of spell enhancement using a higher-level slot no longer exists. The Golden Rules also apply to spells, allowing for more diversified outcomes, adding more tension to each spell.

The approach to alignment is completely changed, now becoming a fundamental aspect of character building; no longer two small letters (LG, CG...) but a choice based on character, morals, and ethics.
The divinities, pardon Patrons, have a more dirty and direct role, read them carefully, they are not the usual gods.
The monsters are based on the 5ed ones, modified to be tougher as there is no longer "bounded accuracy" so you have better results on the Attack Roll and on Saving Throws.

The License guarantees anyone to create and produce wonderful adventures and expansions for OBSS.

One last note. The system aims to be deadlier than 5ed, more wounds, more suffering, without the concept of short or long rest or hit point recovery based on hit dice. No more always and anyway heroes.

Don't worry, good role-playing and teamwork will always guarantee excellent results, even in spells! Participation and immersion are always key.

Finally, but I'm not saying it softly, OBSS refers to the OSR movement, that is, it would like to be played according to those principles. Read the chapter \hyperlink{OSR}{Mastering} (page \pageref{masterizzare}).

\medskip

\begin{center}
Happy reading and have fun!
\end{center}

\begin{flushright}
Andres Zanzani
\end{flushright}

\end{multicols}

\vfill

\begin{center}
\includegraphics[keepaspectratio,width=0.85\textwidth]{immagini/Dragon_by_Henry_Justice_Ford_grey.png}
	
\textit{The End of the Dragon - Henry Justice Ford}
\end{center}

\vspace{1cm}

\begin{changemargin}{0.3cm}{0.3cm}\begin{tcolorbox}
D\&D has misogynistic and racist traits in its origins which have been removed over time thanks to the many people of all genres and types who have played it.
OBSS wants to continue in the wake of an inclusive and free game. Each group is free to approach controversial topics as it sees fit but always with respect for each player and sensitivity. Don't let OBSS be a reason for arguments but for union and fraternal spirit, a game that unites and never divides.
\end{tcolorbox}\end{changemargin}


\pagebreak{}

\subsection{Common Terms}\label{Termini Comuni}

\begin{multicols}{2}

\lettrine[lines=2, lhang=0.33, loversize=0.25, findent=1.5em]{T}{his} is a list of a few terms\index{Common terms} that you will find repeated many times in the book.

\textbf{+1d6 or -1d6}: This is a bonus or penalty to a check. Add or subtract a die roll of 6 to the check. The maximum penalty brings the dice rolled to 0 and the maximum bonus to +3d6.

\textbf{Ability check}\index{Ability check}: is a proficiency check that uses the value of a characteristic as a bonus, such as Strength, Charisma...

\textbf{Action}: \index{Action}is what you do in a time interval. Everything a character does is measured in Actions. Fighting, casting spells, picking locks, drinking potions, moving... in each round you can do 3 Actions.

\textbf{Arbiter:}\index{Arbiter} is the person who leads the adventure, sets the rules, and controls the story elements. The duty of every Arbiter is to entertain, be correct and use common sense. The Arbiter has the final say in every matter.

\textbf{Attack Roll (AR)}:\index{Attack Roll} \index{AR}is a check of Attack (Weapons Proficiency) against Defence (Armour + shield + Skills + magic...). The attack roll can be melee (that is, for creatures close to your weapon, at melee range) or from a distance (for bows, crossbows, but also thrown daggers...).. Read the combat chapter carefully.

\textbf{Bonus}: \index{Bonus}any modification due to external factors, environmental, magical, circumstance or decided by the Arbiter is a bonus or penalty to be applied to the die roll or difficulty in the check.

\textbf{Statistics Scores / Characteristics}: \index{Statistics Scores} \index{Charateristics} Also abbreviated to Charateristics or Stats. Each character has 6 Characteristics: Strength (STR), Dexterity (DEX), Intelligence (INT), Wisdom (WIS), and Charisma (CH). The higher the value, the greater the valence or ability of the character in that specific area.

\textbf{Casting Spell under attack, threat, distraction..}:\index{Concentration Check}\index{Casting Spells under attack, threat, distraction..}\index{Distracted} when a spellcaster wants to use Magic but is disturbed, attacked, injured or otherwise distracted while casting a spell then he must make a Magic Test.

\textbf{Check/Check}: \index{Check}A check is a roll of 3d6 plus a score indicated by the Statistics and Proficiency involved, modifiers given by Skills and circumstances may apply.

\textbf{Class}: There are no classes in OBSS. Each character is built based on what he can do. So you won't find the word Class in the manual.

\textbf{Critical Roll}\index{Critical Roll}: when in the Attack Roll scores 6 several times. Every two 6s, even following a 6 reroll, you apply only the dice of the weapon.

\textbf{Critical Success/Critical Failure in the Saving Throw}\index{Magical Critical Success in the Saving Throw} \index{Critical Failure in the Saving Throw}: depending on the spell in case of a critical success in the saving throw (successful check and at least two 6s rolled) the effects are further halved while in the event of a critical failure (two 1s or a 1 and two 2s are rolled) the damage is suffered even more.

\textbf{Damage Reduction (DR)}: \index{Damage Reduction}\index{DR}Some creatures have innate resistance to damage and wounds. This resistance is denoted as DR.

\textbf{Damage Resistance (DR)}, \textbf{Resistance}: \index{Damage Resistance}\index{RD}: A creature might have resistance to one type of damage. In this case it is considered that you automatically halve the damage taken before applying any Saving Throws.

\begin{center}
	\includegraphics[width=0.4\textwidth]{immagini/cavaliere2.png}
\end{center}

\textbf{Defence}: \index{Defence}Defence means the total value obtained from 10 + Shield + Armour + Dexterity + various and any bonuses. It represents the difficulty in being hit, the higher it is the harder it is to be hit.

\textbf{Devoted}\index{Devoted}: A character who is bonded to a Patron has at least 3 Traits in common.

\textbf{Difficulty Class (DC)}:\index{Difficulty class} \index{DC}indicates how difficult it is to succeed in a task. It can be used to check skills (swimming..) as well as knowledge (poisons..). In spells, it is the difficulty of resisting the spell itself. Indicates what value to arrive at in order to pass and succeed in the check.

\textbf{Distance}:\index{Distance} Distance, when it comes to combat is measured in 1 meter squares.

\textbf{Enchanter, Mage, Wizard:} \index{Mage}\index{Wizard}means any user of magic in any capacity.

\textbf{Experience Points/XP}: \index{Experience Points} \index{PX} every time you solve difficulties, monsters, riddles, find treasure, play your character well and have fun, you gain experience. These points accumulated over time establish the level and therefore the abilities of the character.

\textbf{Explosion of 6}:\index{Explosion of 6} when you execute the Attack Roll, Saving Throw, Competence Check, Magic Test (read the specifics in the dedicated chapter) or in any case whenever it is indicated that the explosion of 6 is valid, it means that for each die rolled that results in a 6, the die must be marked and re-rolled. The result of the new roll is also added up and if you roll a 6 he continues to roll as long as you keep rolling a 6.

\textbf{Fate Points}:\index{Fate Points} \index{Beginner's Luck}Beginner's Luck are points available that the player can transform into d6s to add to Saving Throws or attack rolls or skill rolls. They are called Beginner's Luck because their number decreases as the character levels up.

\textbf{Feat}: \index{Feats}these are particular abilities that the character has learned to use. They are often similar to spell-like abilities, allowing particular actions and even subverting the rules at times. They are rare and are acquired at level-ups.

\textbf{Follower}\index{Follower}: A character who has bonded to a Patron with 2 Traits in common

\textbf{Hit Dice}\index{Hit Dice}: Hit dice are the levels of a creature. Basically, they are used to indicate how many Hit Points and level it has. Unless noted, a creature has 6 Hit Points per Hit Die.

\textbf{Hit Points (PF)}:\index{Hit Points} indicate the creature's life energy, stamina, luck in resisting wounds. As long as the creature has 1 hit point it will fight at its best, without problems (of course...it could even decide to run away rather than die...).

Each time you level up, you gain a certain number of Hit Points, established by the rules. Each wound is subtracted from this sum of energies and when you reach 0 (zero) Hit Points you pass out, unable to act. If you are further wounded or in any case your Hit Points drop to 10 + double your Constitution then you are dead.

\textbf{Initiative}: \index{Initiative}is a Dexterity or Intelligence check. Establishes the order of actions in combat. Whoever has the highest score goes first.

\textbf{Level}:\index{Level} Level indicates the proficiency and power achieved by the character. It can indicate when the enemy is strong.

\textbf{Magic Proficiency (MP)}: \index{Magic Proficiency}\index{MP}is your ability to use spells, the higher this value is, the more effective the spells will be.

\begin{center}
	\includegraphics[keepaspectratio,width=0.40\textwidth]{immagini/spiritomagia2.png}
\end{center}

\textbf{Magic Test}\index{Magic Test}: The Magic Test can be due for particular situations, for example when the character is injured or distracted, but it can also be requested by the player.

Magic Test allows the character to take their spell casting further and try to tap into and harness more magic.

Depending on the results it could get advantages or disadvantages.

\textbf{Magic Critical Success/Magic Critical Failure}\index{Magic Critical Success} \index{Magic Critical Failure}: in case the player passes the Magic Test with crits (two 1s or two 6s). Magical critical success leads to spectacular changes in the spell, vice versa bad things could happen to the caster.

\textbf{Melee}: \index{Melee}melee means contact, hand-to-hand, sword-to-sword combat, or when your character fights not with a weapon that has range (bow, crossbow, slingshot...) against an opponent.
Any creature you can reach with your non-ranged weapon is considered melee. A large enemy (or one with a long weapon) might be in melee with you but not vice versa.

\textbf{Movement}: \index{Movement}Movement represents the ability to move. A Move Action represents the movement of the character. The higher the movement value, the more meters a creature can move.

\begin{center}
	\includegraphics[width=0.40\textwidth]{immagini/merlin.png}
	
	\textit{Merlin dictating his prophecies to his scribe. Robert de Boron's Merlin en prose (written ca 1200)}
\end{center}

\textbf{One is bad luck}: \index{One bad luck}If you roll a 1 with the die, subtract 1 from the total roll. This does not mean that a 6 thrown becomes a 5, the explosion of the 6 remains.. only that you subtract 1 from the final result. Put differently, 1 equals 0.

\textbf{Patron}:\index{Patron} or divinity. The Patron is a superior being who can grant powers and guarantee benefits.

\textbf{PC, Character}: \index{Character}is the creature that is led, managed, rotated by the player.

\textbf{Penalty} \index{Penalty}: like the bonus, the penalties are values, numbers, which indicate unfavorable circumstances, penalizing spells or anything else that makes the check more difficult. Unfortunately, unlike Bonuses, penalties, unless otherwise specified, are always added together.

%\begin{wrapfigure}{c}{0.5\textwidth}
\begin{center}
	\includegraphics[width=0.35\textwidth]{immagini/Sakramentarz_tyniecki_02.png}
	
	\textit{Sakramentarz Tyniecki: Majuskuła "V".}
\end{center}

\textbf{PNG}: \index{NPC}non-player character. They are particular characters, important or not, that the Arbiter keeps to lead the adventure.

\textbf{Round}:\index{Round} Combat or actions are divided into rounds. A round represents a time unit of approximately 10 seconds. During the round, each creature has the opportunity to act on her initiative and perform up to 3 Actions.

\textbf{Rounding}: \index{Roundings}always down unless otherwise specified but with a minimun of one. Ex. 7/2 = 3, 9/4=2, 1/2=1

\textbf{Saving Throw (ST)}:\index{Saving Throw} \index{ST}When a creature is subjected to a particular effect, a Saving Throw is often allowed to mitigate or nullify the effects. The Saving Throw is an action that takes up no time or actions.

Saving Throws involve reflexes and dodging (Dexterity), resisting poisons/diseases or bodily changes (Constitution), or resisting mental attacks and effects that affect agency (Wisdom).

\textbf{Skill} \index{Skill}: Skill tells us what we know and its value indicates the degree of knowledge of a single skill. May it be studying a language, climbing it, noticing little things.

\textbf{Spell level}: Indicates the scale (1 to 9) of the spell's magical power.

\textbf{Trait}: \index{Trait}indicates a character component. Each character chooses 5 Traits to compose and build her personality.

\textbf{Turn}: \index{Turn}that's 10 minutes, or 60 rounds

\textbf{Weapon Proficiency (melee or ranged) (WP)} \index{Weapon Proficiency} is the ability to hit the opponent with melee (swords, maces...) or throw/ranged weapons (throwing daggers, bows, crossbows..)


\begin{changemargin}{0.3cm}{0.3cm}\begin{emphasis}{
D\&D (and \textit{OBSS}) game has neither losers nor winners, it has only players who love to exercise their imagination. The players and the DM (\textit{Arbiter}) share the creation of adventures in fantasy lands where heroes abound and magic actually works. In a sense, the game of D\&D has no rules, only hints of rules. No rule is infringed, particularly if a new or changed rule will encourage creativity and imagination. The important thing is to enjoy the adventure. (Tom Moldvay, 12/03/1980)
}\end{emphasis}\end{changemargin}

\medskip

In the Manual you will find different types of boxes, each one has a precise meaning

\medskip


\begin{changemargin}{0.3cm}{0.3cm}\begin{emphasis}{Example of a box containing a quote or motivational phrase}\end{emphasis}\end{changemargin}

\begin{changemargin}{0.3cm}{0.3cm}\begin{tcolorbox}[title = Information for the player]Box containing information and clarifications for the player.\end{tcolorbox}\end{changemargin}

\begin{changemargin}{0.3cm}{0.3cm}\begin{narrator}Box containing information and suggestions for the Arbiter\end{narrator}\end{changemargin}



\vfill

\begin{center}
\includegraphics[width=0.45\textwidth]{immagini/Jan_Steen2.png}

\textit{Jan Havicksz. steen}
\end{center}

\end{multicols}

\pagebreak

\section{Races}\index{Races}

\begin{changemargin}{0.3cm}{0.3cm}\begin{emphasis}{The real journey of discovery consists not in finding new territories, but in having other eyes, seeing the universe through the eyes of another, of hundreds of others: to observe the hundred universes that each of them observes, that each of them is. (Marcel Proust)

\medskip

It is not the most intelligent of species that survives; she's not even the strongest; the species that survives is the one that is able to adapt and adapt better to the changes in the environment in which it finds itself. (Leon C. Megginson)}\end{emphasis}\end{changemargin}\medskip


\begin{multicols}{2}

\lettrine[lines=2, lhang=0.33, loversize=0.25, findent=1.5em]{Y}{eru} is a multifaceted world rich in cultural, natural and human diversity.
Creatures make the planet vital and rich, each one nourishes, contributes, enriches the knowledge of all the others.


\subsection{Human}\index{Human}\label{umani}

Humans with their desire for discovery, power, glory and violence and reproductive ability are the master race.

The physical characteristics of humans are extremely varied, including skin color, clothing, cultural and food traditions, and lifestyles, which can be diverse and unique. These differences only enhance the human experience.

Leave the racism out of Yeru, there are enough wars already, there is no need to start new ones just because someone uses axes instead of swords.

Humans were the race created by Ljust and Calicante together so that with their chaotic, dynamic and adaptive nature, allow them to continually learn, grow, and improve.

\begin{center}
\includegraphics[height=0.35\textwidth]{immagini/uomovitruviano2.png}

\textit{Vitruvian man - Leonardo da Vinci}
\end{center}


\textbf{Racial Modifiers:} +1 to one characteristic

\textbf{Physical characteristics}: height 150-185 cm, 50-130 kg, life expectancy 65 years (50 + 2d10 years)

\textbf{Size:} Medium

\textbf{Movement}: 9m

\textbf{Languages}: Common

\textbf{Advantage}: +1 Feat at 1st level. The first point of WP or MP assigned is doubled.

\subsection{Elves}\index{Elves}\label{elfi}

The elves are a race created directly by Ljust to lead the world with the elegance, intelligence, and foresight of an immortal race.

After millennia of peace and prosperity throughout the world, and after natural and architectural wonders had spread in harmony, the creation of new races and their expansionist tendencies led the elves to reassess their mission.

Suddenly, they went from being ambassadors of beauty, culture, passion, and inspiration for the arts, to becoming more reclusive. The fear of losing all that was beautiful manifested in them.

Therefore, they decided to isolate themselves even further in order to preserve every form of art and beauty, from the written word to the arts, protected from those who could potentially diminish or defile them.

Their desire to preserve their creations became a relentless struggle against everything and everyone, against any creature that wanted to live and create something new. The evil deity Cattalm had so thoroughly poisoned their blood that they were unaware of it.

This led to a period of extremely heinous and violent centuries dominated by the elves' absolute determination to destroy everything by any means necessary. All nations and civilizations suffered greatly, both in terms of lives lost and a return to barbarism.

It took nearly 500 years and the destruction of entire nations and civilizations for all other creatures to unite against the elves in a final attempt to save themselves.

In what is now known as the Week of Hate, hundreds of thousands of creatures and almost all of the elves perished.

A sweeping purge followed, every elf was slain, and this continued for almost another century.

Just under 100 years ago, a new Elven Queen, Lycenea, undertook a journey to the court of every nation, facing the risk of lynching numerous times. She managed to secure a peace treaty that would protect the few remaining elves.

The surviving elves, although elderly, have not lost their hatred, their blood remains stained by Cattalm, and they continue to brood behind the treaty that they claim prevents them from fulfilling their true destiny.

Thankfully, not all of the new elves share this visceral hatred and would like to live a normal life in contact with other creatures, even though they are aware of how they are perceived and treated by others.

These are the young elves who do not only want to preserve beauty, but to live a nearly infinite life where it is a wonder in and of itself.

Elves are generally taller and leaner than humans, with light-colored eyes. They value the written word and magic, and are a rational race driven by a sharp mind, excellent senses, and an interest in the extraordinary and knowledge.

\begin{center}
\includegraphics[height=0.7\linewidth]{immagini/elf2-ai.png}
\end{center}

\textbf{Racial Modifiers:} +1 Intelligence, +1 Dexterity, -1 Charisma

\textbf{Physical characteristics}: height 165-195cm, 50-110kg, life expectancy 60d100+ years

\textbf{Size:} Medium

\textbf{Movement}: 9m

\textbf{Languages}: Elven

\textbf{Benefit}: low light vision 18m

\subsection{Dwarves}\index{Dwarves}\label{nani}

Dwarves are a stoic and stern race accustomed to the purest form of communism, without a true concept of property but rather a communal ownership of goods. The idea is that every dwarf works for the community and not for oneself.

Dwarves are a short and stocky race, reaching a maximum height of about 140 cm. They have a sturdy and compact build that gives them a massive appearance. Both males and females proudly wear long hair, and men often adorn their beards with various types of clips and intricate braids. Bald dwarves are also common, but never without a beard. Female dwarves do not have beards or excessive body hair. Gender is free and socialist.

Dwarves are guided by honor, tradition, and communism. They are often seen as gruff, but they have a strong sense of friendship, justice, and respect for those who work hard and dedicate themselves to the community and the group.

Dwarves are the race created by Erondil with the help of Atmos.

They judge the Elves harshly because the Elves have failed to fulfill and have even betrayed the dictates of Creation. Therefore, the dwarves feel the duty, burden, and honor of forging the created world and within it, the beauty and grandeur of Erondil.

\begin{center}
\includegraphics[height=0.6\linewidth]{immagini/nana2-ai.png}
\end{center}

\textbf{Racial Modifiers:} +1 Constitution, +1 Wisdom, -1 Dexterity

\textbf{Physical Characteristics}: Height 100-140 cm, 45-90 kg, life expectancy 450 years (400 + 1d100 years)

\textbf{Size:} Medium

\textbf{Movement}: 6m

\textbf{Languages}: Dwarven

\textbf{Advantage}: Low light vision 18 meters

\subsection{Gnomes}\index{Gnomes}\label{gnomi}

The Gnomes are a race of small, energetic beings who appeared around 1000 years ago. Their origin is unknown, but in a short time, they established populous and rich cities within virgin forests, thanks to their innate curiosity, tenacity, and inventiveness.

Gnomes have a deep connection to nature and a symbiotic relationship with it. They respect and preserve the environment and animals and build their perfectly functional cities within forests without destroying them but instead enriching them.

Many Gnomes are inventors and builders known for their imagination and ingenuity, often using their inventions to support their communities. The more curious Gnomes often leave their communities to embark on a life of adventure and discovery, bringing back new ideas to their community.

Gnomes need nature and suffer from sadness and apathy if forced to stay away from it. They get along with anyone who respects and protects nature.

\textbf{Racial Modifiers:} +1 Intelligence, +1 Charisma, -1 Strength

\textbf{Physical Characteristics}: Height 70-110 cm, 30-50 kg, life expectancy 650 years (600 + 1d100 years)

\textbf{Size:} Small

\textbf{Movement}: 6m

\textbf{Languages}: Gnomish, Sylvan

\textbf{Benefit}: 1 Druidic Artifice per day.


\subsection{Half-elf}\index{Half-elf}\label{mezzelfo}


Half-elves are often viewed with disdain by the elven community due to their mixed heritage. They are seen as the result of violence or, in rare cases, a secret romance between an elf and a human. As a result, half-elves are often seen as a betrayal of the elven race's beauty and purity.

However, half-elves possess a unique combination of physical features from both their elven and human heritage. They are taller than humans but shorter than elves and possess the slender build and striking features of the elves. Their skin color varies based on their human heritage, while their eyes can come in a range of exotic colors, from amber and purple to emerald green and dark blue.

Despite the negative views from the elves, half-elves have a deep understanding of loneliness and recognize that character is shaped more by life experiences than by race. They often lead lives that are rich in experience and can form meaningful connections with those who see beyond their mixed heritage.
\medskip

\begin{center}
\includegraphics[height=0.7\linewidth]{immagini/half-elf-ai.png}
\end{center}


\textbf{Racial Modifiers:} +1 to a Characteristic of your choice

\textbf{Physical characteristics}: height 160-185 cm, 50-100 kg, life expectancy 210 years (180 + 5d10 years)

\textbf{Size:} Medium

\textbf{Movement}: 9m

\textbf{Languages}: Common or Elven

\textbf{Benefit}: 9 meter low-light vision

\subsection{Half-orc}\label{mezzorco}\index{Half-orc}

Half-orcs are widely viewed by civilized cultures as abominations, born from perverted and violent acts. However, there are rare cases where half-orcs are conceived from romantic relationships. Regardless of their origin, half-orcs face a tough upbringing and often find themselves struggling to prove their worth and protect themselves.

Half-orcs average 6 feet tall, with a powerful physique and greenish or gray skin. In males the canines often grow quite long until they protrude from their mouths and these "fangs", combined with a sometimes broad forehead and somewhat pointed ears, give them that well-known "bestial" appearance. Despite these obvious orcish traits, half-orcs are as diverse as their human parents.

While half-orcs face discrimination and disrespect from both orc and human societies, many resort to crime as a means of survival. This is due to a lack of acceptance and opportunity in mainstream society, as well as the influence of their orc heritage, which is known for its chaotic and destructive tendencies.

Orcs were created by the deity Cattalm, with the help of Calicante. Their creators' chaotic and destructive tendencies are still present in the half-orc nature, but this does not define their entire character. 

It is important to remember that half-orcs, like all races, are unique individuals and should not be judged based on preconceived notions and prejudices.



\begin{center}
\includegraphics[height=0.7\linewidth]{immagini/half-orc2-ai.png}
\end{center}

\textbf{Racial Modifiers:} +2 Strength -1 Charisma

\textbf{Physical characteristics}: height 160-210 cm, 60 - 140 kg, life expectancy 70 years (50 + 5d10 years)

\textbf{Size:} Medium

\textbf{Movement}: 9m

\textbf{Languages}: Common or Orc

\textbf{Benefit}: 9 meter low-light vision


\subsection{Nibali}\index{Nibali}\label{nibali}


The Nibali are a magically created race meant to be slaves to the great enchanters of the north.

Legend has it that the dreadful ice wizards, starting from a pair of humans (after thousands had died horribly in previous experiments), succeeded in magically manipulating a race that was sturdier, stronger, more intelligent, and at the same time more docile and disciplined. A unique trait of this race was that every child born would be physically identical to either the father or the mother.

These events took place over 2000 years ago, and the eternal realm of evil collapsed under its own inability to adapt and understand new challenges (possibly with the intervention of some Patrons).

The Nibali continued to thrive and, benefiting from what the icy realm had left behind, created one of the most modern, democratic, and civilized civilizations in the world.

For many, the Nibali's extreme efficiency and dedication are odious, a yoke that leaves no room for personal freedoms. For the Nibali, it is simply a natural way to progress.

All Nibali are equal in terms of sex, but the fact that they cannot have children with other races does not make them a closed or racist people. On the contrary, absorbing the best of every culture makes them better and excellent diplomats. What truly distinguishes one Nibali from another is their hairstyle, tattoos, clothing... Respect for others and the Law are inseparably linked to their nature, yet nothing is freer than a Nibali.

For a Nibali, rules and laws should promote peace and freedom, be fair, and those who uphold them should be understanding and wise. For a Nibali, freedom is not doing whatever one wants, but the right to do what one must.

Male Nibali are bald with azure skin and violet eyes. Women have amber skin, chestnut hair with blonde highlights, and green eyes.


\textbf{Racial Modifiers:} +1 Constitution, +1 Intelligence, -1 Wisdom

\textbf{Physical characteristics}: height 183cm male, 170cm female, 50 - 120kg, life expectancy 130 years

\textbf{Size:} Medium

\textbf{Movement}: 9m

\textbf{Languages}: Common

\textbf{Benefit}: Reduced metabolism control. Each round you reduce the damage from Bleeding by 1


\subsection{Different}\index{Different}\label{diverso}\hypertarget{diverso}{}

Blessed or cursed, the "Different" are not like us. They are not the friends you expect. A Different is the result of a corrupt union. If the Patrons cannot directly act in Yeru, or at least that's what Gradh tries to avoid, they often use their powers to create a loyal lineage.

A Different is loyal to their Patron and cannot act otherwise. Fortunately, they are sterile with humans; otherwise, they would have already dominated the world.

A Different is stronger and more intelligent. Unfortunately, their frantic life is marked by a short lifespan. Typically, a Different does not exceed 50 years of age.

A Different is marked; somewhere on their body, there is a symbol, a birthmark, of their Patron. Almost all Different (15/18) have three or more concentric golden circles on their left wrist, which can indicate the Patron (or Patrons, in extremely rare cases) they are "children" of.

Different is an attribute that can be given to any race. The racial modifiers are replaced with those of the Different and the lifespan is halved (in the example below, referring to a Human). The original racial advantages remain valid, and the special advantage of the Different is added.

\begin{center}
\includegraphics[width=0.7\linewidth]{immagini/diverso.png}
\end{center}

\textbf{Racial Modifiers:} +1 to two Characteristics of your choice

\textbf{Physical Characteristics}: Height as parent breed, life expectancy halved of parent breed

\textbf{Size:} as original breed

\textbf{Movement}: As original race

\textbf{Languages}: As original race

\textbf{Special}: Must identify a Patron and have at least 3 common Traits. He can access the 5 Trait stack power even if he has fewer.


\subsection{Sornelian}\index{Sornelian}\index{Furryman}\label{sornelian}\hypertarget{sornelian}{}

The genesis of the Sornelians is due to \hyperlink{efrem}{Efrem} (page \pageref{efrem}) , his only known victory in the Thousand Years War. Efrem decided that nature should have a greater say in human affairs and decided that anthropomorphic creatures were created to balance the excessive power of humanoid creatures.

A Sornelian has a head similar to that of an anthropomorphic animal but the body is more similar to a humanoid biped. Depending on the animal, the Sornelian could also have fur, feathers, scales and claws. The size of a Sonerlian depends greatly on the original animal, ranging from small to large size. Size does not grant specific advantages. A Sornelian's anthropomorphic appearance is as varied as the animals they resemble.

\textbf{Racial modifiers:} +1 to a Characteristic of your choice

\textbf{Physical characteristics}: life expectancy depends on the longevity of the species, usually around 60+6d10 years.

\textbf{Size}:  it depends on the original species, from 50cm to 270cm, from small to large size.

\textbf{Speed}: 6 metres

\textbf{Languages}: Common. It gains +1d6 on checks to interact with animals of its bloodline.

\textbf{Special}: Upon creation, the player chooses 2 abilities among those listed that best characterize his Sornelian. Some example animals are indicated in parentheses. The player can confront the Arbiter to best create his Sornelian.

- \textit{Climber} (bear, cat, lizard, squirrel). You have hooked claws, sharp nails, or a serpentine tail. You have a climbing speed equal to your Speed.

- \textit{Predator} (bear, feline). Your natural attacks cause 1d6 of lethal damage, and you are considered proficient with them.

- \emph{Flying} (bat, eagle, owl, crow). You have vestigial wings. When you fall at least 10 feet, you can use a Reaction to glide and land safely, as the Feather Fall spell (page \pageref{featherfall}), without taking damage from the fall. When you make a Long or High Jump check, you roll an extra 1d6.


- \textit{Runner} (deer, dog, horse, velociraptor). When you take the Run Action, you make your Movement 3 times. Your Speed ​​is 30 feet.

- \textit{Swimmer} (crocodile, dolphin, frog, shark). You can hold your breath for up to an hour, you have a swimming speed equal to your Speed. You have 4 cold damage reduction.

- \textit{Night creature} (cat, lizard, bat, dolphin, owl). You have low-light vision 30 feet.

- \textit{Excellent Hearing/Smell} (dog, bat, owl). You have a +4 bonus on sense-based Awareness checks.

\begin{center}
\includegraphics[width=0.55\linewidth]{immagini/arpia-ai.png}

\textit{Harpy, bird woman, usually very angry...}

\end{center}

\subsection{Golian}\index{Golian}\index{Giant Men}\label{golian}\hypertarget{golian}{}

The Golians, like the Sornelians, descend from the will of \hyperlink{erondil}{Erondil}(page \pageref{erondil}) and \hyperlink{gaya}{Gaya}(page \pageref{gaya}) for the desire to have creatures that could represent the majestic giants, their little children.

The Golians, however similar to the giants, are not such and yet have the ability to grow rapidly and even if for a short time approach the height of the real giants.

Golians have physical features resembling the giants of their family lines. Some Golians have gray or almost marbled skin like stone giants, others shoot sparks by snapping their fingers like fire giants, still others have blue skin like sky giants.

\textbf{Racial modifiers:} +2 to Strength, -1 to a Characteristic of your choice

\textbf{Physical characteristics}: about 180/210cm tall. Life expectancy about 80 years (60+2d10)

\textbf{Size}: Medium

\textbf{Movement}: 9 meters

\textbf{Languages}: Common, Giant of their lineage.

\textbf{Special}: Each Golian descends from a line of giants and from this inherits peculiar powers. The indicated power is usable (MP+WP)/3, rounded up, per day.

- \textit{Cloud Giant}. One step in the sky. For the cost of two Actions you magically teleport up to 10 meters to an unoccupied space that you can see.

- \textit{Fire giant}. Burning embers. When you hit a target in melee, you can deal 1d10 fire damage to that target. Cost 1 Reaction.

- \textit{Frost Giant}. Deep frost. When you hit a target in melee, you can deal 1d6 cold damage and the creature's movement Movement decreases by 3 meters feet until the end of your next turn. Cost 1 Reaction.

- \textit{Hill Giant}. Angry blow. When you critically strike a creature of your size or smaller in melee, it falls prone. Cost 1 Action.

- \textit{Stone giant}. Stone Skin. When you take damage you can harden your skin to stone. Reduce damage taken by (WP or MP + Constitution)/2. Cost 1 Reaction.

- \textit{Storm Giant}. Resonance of Thunder. When you take melee damage, you can emit a shockwave that deals 1d10 sound damage to whom deal damage to you. Cost 1 Reaction.

\textbf{Big Size}: All Golians acquire, starting from MP+WP at least 5, the ability to enlarge and become of large size once a day. This transformation lasts 1 minute, Strength-based attack rolls and damage increase by 1d6, movement Movement increases by 1 meters per move action.

\textbf{Stable}. You are considered Large to withstand tests of being grabbed or pushed.



\begin{center}
\includegraphics[height=0.8\linewidth]{immagini/Herakles_Farnese_MAN_Napoli_Inv6001_n01.png}\\

\textit{Farnese Hercules, Naples National Archaeological Museum}

\end{center}


\subsection{Sulian}\index{Sulian}\label{sulian}\hypertarget{sulian}{}

The origin of the Sulians is not clear, some make them descend from elemental spirits, other voices, less insistent, say that they are children of\hyperlink{ledyal}{Ledyal or Laydel}(page \pageref{ledyal})  due to their changing appearance and personality.

The power, energy and vitality of the elements flow in the Sulian, whether it be a single type or multiple elements.

The Sulians are very similar to humans but the primordial energy that characterizes them can be seen flowing in their eyes and often in their skin.

\textbf{Racial Modifiers:} +1 to one Characteristics of your choice

\textbf{Physical characteristics}: about 150-190cm tall. Life expectancy about 180 years (160+2d10)

\textbf{Size}: Medium

\textbf{Movement}: 9 meters

\textbf{Languages}: Common. They can understand the elemental language of their bloodline but cannot speak it.

\textbf{Special}: Each Sulian descends from one or more elemental lines and from this they inherit unique powers and abilities. At the first point of WP or MP assigned and subsequently every 8 total points assigned (1,8,16...), the Sulian enhances his elemental bloodline and selects a power or unlocks another elemental line present in him to choose different powers.

The indicated power is usable (MP+WP)/3 per day.

-\textit{Primal Discharge}: the Sulian can at the cost of 1 Reaction when hit or strikes in melee discharge part of his elemental energy. The damage is equal to 2d6 times this power has been selected.

- \textit{Access to the Magic List}: through this power the Sulian can access an Elemental List. Each time he takes this power he spontaneously knows up to 3 spells on that list with a maximum spell level equal to the times he took this power on the same list -1 (the first time he only casts cantrips).
The Sulian does not make Magic Tests nor can he be considered distracted when he casts the spell. For any factors it is considered that the CM is equal to the sum of CM+AC and Adept of Magic has been taken a number equal to the times this power has been taken.

- \textit{Elemental Resistance}: through this power, the Sulian acquires Resistance to the chosen element.


\begin{center}
\includegraphics[height=0.9\linewidth]{immagini/Undine_Rising_from_the_Waters.png}\\

\textit{Undine Rising from the Waters, ca. 1880–1892, by Chauncey Bradley Ives (1810–1894), in the Yale University Art Gallery}

\end{center}

\end{multicols}

\
\vfill

\index{Races}\index{Races}
\begin{changemargin}{0.3cm}{0.3cm}\begin{tcolorbox}[title = Note on Benefits]
The player, in agreement with the Arbiter, can choose a different advantage from the one indicated as long as it is coherent with the story of the character.
\end{tcolorbox}\end{changemargin}

\begin{changemargin}{0.3cm}{0.3cm}\begin{tcolorbox}[title = Note on Breeds]
No description of a race can ever restrain or subdue a character. Every player is free to create their preferred race character (granted by the Arbiter) and describe them, frame them, feel them, and bring them to life as they please.
Do not limit yourselves to the descriptions provided here; they are merely starting points. Do not feel constrained in your choices because your race dictates this or that.
Let the most beautiful and complete characters be born.
Each character is alive and is an individual, and as such, they will always be different from one another. Each one is fantastic in their own unique way, defying any racial prejudice.\end{tcolorbox}\end{changemargin}

\begin{changemargin}{0.3cm}{0.3cm}\begin{tcolorbox}[title = Note on Gender]\index{Gender}
In case you're so narrow-minded, I repeat that there is no difference in ability or stats based on gender. Each player and player is invited to make the character of the gender (or not) that she prefers.

If the topic is a reason for you not to enjoy it, clarify it with the Arbiter, he will be able to orchestrate the adventure in a suitable way.
\end{tcolorbox}\end{changemargin}



\pagebreak

\section{Special Features}


\begin{changemargin}{0.3cm}{0.3cm}\begin{emphasis}{It's not enough to have eyes to see (anonymous)}\end{emphasis}\end{changemargin}\medskip


\begin{multicols}{2}


\lettrine[lines=2, lhang=0.33, loversize=0.25, findent=1.5em]{E}{ach} creature is special and unique yet there are beings even more unique and special due to their characteristics. These are the peculiarities of some of these.

\subsection{Twilight Vision}\index{Twilight Vision}\label{visionecrepuscolare}


What is seen as darkness by many, is viewed as twilight to those with \hypertarget{visioneeluce}{low-light vision}. This means they are able to see well as long as there is minimal light present.

Low-light vision includes color vision and a spellcaster with this ability can read a scroll as long as there is even the slightest source of light, such as a candle.

Characters with low-light vision are able to see outside on moonlit nights as if it were daylight. However, in complete darkness, low-light vision is useless and it remains pitch black and impenetrable.

\subsection{Darkvision}\index{Darkvision}\label{scurovisione}

Darkvision is the ability to see in complete darkness without the need for light sources, up to a specified range for each creature. It is a monochromatic vision, meaning that it only allows characters to see in shades of black and white and does not provide color distinction.

Darkvision does not make objects or creatures that are invisible or illusions visible, and it does not protect against gaze attacks. It operates independently of light, and therefore the presence of light does not affect it. However, performing tasks such as searching for traps or making visual-only awareness checks may incur a 1d6 penalty.

\subsection{Smell}\index{Smell}\label{fiuto}

This special quality allows a creature to use its sense of smell to detect hidden or approaching enemies and to follow trails. Creatures with scent can identify smells as humans can by what they see.

The creature can detect other creatures within 6 meters (20 meters if downwind and 3 meters if upwind) by their scent. Stronger odors, such as smoke, garbage, or decaying bodies, can be detected from twice the distance.

When a creature detects an odor, it does not reveal the exact location of its source. The creature can use an action to determine the direction from which the scent is coming. When within melee range of the source, it can pinpoint its location.

A creature with a nose can track using its sense of smell by making a Follow Trail check. The typical DC for a fresh trail is 10, regardless of the surface it is on. The DC may increase or decrease based on the strength of the trail, the number of creatures that made it, and the time since it was created. The DC increases by 2 for every hour that passes.

\begin{center}
\includegraphics[width=0.9\linewidth]{immagini/mostro.png}

\textit{John D. Batten}
\end{center}

This ability otherwise follows the rules of the Survival skill. Creatures that track by scent ignore the effects of tracked surfaces and poor visibility.

A creature with the scent ability identifies familiar scents just as a human might identify a familiar place. Water, and especially running water, negates the ability to track creatures.

Some strong odors can easily mask others. The presence of such an odor makes it impossible to accurately locate or identify a creature by Scent; the basic DC of the Survival perk to follow tracks in the presence of masking odors goes from 10 to 20.


\begin{center}
\includegraphics[width=0.9\linewidth]{immagini/argus2.png}

\textit{Argus Panoptes Guarding the Heifer (Io), Red Figure pitcher, c. 460 BC Museum of Fine Arts, Boston}
\end{center}


\subsection{Blindsight}\index{Blindsight}\label{vistacieca}

Using senses other than sight, such as vibration sensing, a keen nose, keen hearing, or sonar, a blindsighted creature moves and fights as well as a sighted creature.

Invisibility and darkness do not affect the ability, but the creature must have line of effect to detect a creature or object. A creature with cover still retains its defensive advantage.

The range of the ability is specified in the creature's description. Typically, the creature does not need to make Perception checks to detect creatures within the range of its blindsight. Unless otherwise noted, blindsight is always active and does not require any action to activate it. In some cases, blindsight may need to be activated as a reaction, which will be specified in the creature's description. If activation is required, the benefit applies only during the creature's turn. An ethereal creature is not visible to blindsight.

\subsection{Telluric Sense}\index{Telluric Sense}\label{sensotellurico}
A creature with tremorsense is sensitive to ground vibrations and can automatically detect any creature or object in contact with the ground within the specified range of its tremorsense.

Aquatic creatures with echolocation can sense the location of creatures in contact with water. The range of the ability is specified in the creature's descriptive text.

\end{multicols}

\vfill

\begin{center}
\includegraphics[height=0.6\linewidth]{immagini/grabroid.png}

\medskip

\textit{Grabroid. Also known as Grabbers. Tremors (Movie)}
\end{center}

\pagebreak

\section{The Characteristics}\index{Characteristics}


\begin{changemargin}{0.3cm}{0.3cm}\begin{emphasis}{Living is not breathing: it is acting, it is making use of the organs, senses, faculties, all those parts of ourselves for which we have the feeling of existence. (Jean-Jacques Rousseau)}\end{emphasis}\end{changemargin}\medskip


\begin{multicols}{2}

\lettrine[lines=2, lhang=0.33, loversize=0.25, findent=1.5em]{E}{ach} character has 6 Characteristics (also called Stats) which represent his basic attributes and constitute his potential talent and ability innate.

While it is not common for a character to make a check using only one of his or her ability scores, ability scores affect virtually every aspect of a character's abilities and skills.


\subsection{Characteristics Description}\label{decrizionedellecaratteristiche}

Ability score isn't everything in a character, much less a monster.

The more "instinctive" and aggressive monsters will certainly have negative Intelligence and Charisma scores, but they are not "stupid" for this, they simply act according to their natural patterns.

\subsubsection{Strength}\index{Strength}\label{forza}

\begin{changemargin}{0.3cm}{0.3cm}\begin{emphasis}{
Ah, it is excellent to have the strength of a giant, but to use it like a giant is tyranny! (William Shakespeare, Isabella: from "Measure for Measure", Act II, Scene II)
}\end{emphasis}\end{changemargin}


Strength measures the physical power, athleticism and limits of brute strength you can express. Strength applies in melee damage and for hand-pulled weapons.

A Strength check can be used for any attempt to lift, push, pull, or break something, to push your body into a space, or any other application of brute force.

A monster with Strength -4 is not close to dying, it simply has very little strength (imagine giving a Strength value to a mouse or a squirrel if not to a small spider..)

A character with a Strength score of -5 is dead.


\subsubsection{Dexterity}\index{Dexterity}\label{destrezza}

\begin{changemargin}{0.3cm}{0.3cm}\begin{emphasis}{
Tired barking. Strength means nothing in life. Knowing how to dodge is what counts. (Daniel Pennac)
}\end{emphasis}\end{changemargin}

Dexterity measures agility, reflexes, balance and coordination; determines Defence and Attack Rolls for Thrown Weapons.

A Dexterity check can be used for any attempt to move nimbly, dodge a blow, or avoid losing balance or pickpocketing.

A character with a Dexterity score of -5 is unable to move and is completely immobile (but not knocked unconscious).

\subsubsection{Constitution}\index{Constitution}\label{costituzione}

\begin{changemargin}{0.3cm}{0.3cm}\begin{emphasis}{
A little health now and then is the best remedy for the sick. (Friedrich Nietzsche)
}\end{emphasis}\end{changemargin}


The Constitution measures health, vigor and life force as well as resistance to stress.

A Constitution check can be used for your attempts to push yourself beyond the normal limits of your body and for endurance and durability checks.

A character with Constitution -5 is no longer in control of his body and is dead.

\subsubsection{Intelligence}\index{Intelligence}\label{intelligenza}

\begin{changemargin}{0.3cm}{0.3cm}\begin{emphasis}{
Strength without intelligence ruins under its own weight. (Horace)
}\end{emphasis}\end{changemargin}


Intelligence measures mental acuity, accuracy of recall, and ability to reason.
An Intelligence check comes into play when you need to rely on logic, education, memory, or deductive skills.

Your Intelligence (Arcana) checks measure your ability to recall information about spells, magical items, esoteric symbols, magical traditions, the planes of existence, and the inhabitants of those planes. Rummaging through ancient scrolls for a fragment of knowledge might require an Intelligence check.

A character with an Intelligence score of -5 is comatose.

\subsubsection{Wisdom}\index{Wisdom}\label{saggezza}

\begin{changemargin}{0.3cm}{0.3cm}\begin{emphasis}{
Strength does not come from physical capacity. It comes from an indomitable will. (Mahatma Gandhi)}\end{emphasis}\end{changemargin}


Wisdom reflects your attunement to the world around you and represents insight, intuition, willpower, and common sense.

A Wisdom check reflects an effort to interpret body language, understand someone's feelings, notice details of the environment, or heal an injured person.

A character with a Wisdom score of -5 is incapable of rational thought and is knocked unconscious.

\subsubsection{Charisma}\index{Charisma}\label{carisma}

\begin{changemargin}{0.3cm}{0.3cm}\begin{emphasis}{
Kogami, do you know what charisma is?

- The way I see it, it's an innate attitude, like that of a hero or a leader.

-[...] There are three elements that identify charisma: the innate disposition of heroes and prophets, the ability to instill well-being in others with mere presence and a culture that allows you to have a brilliant conversation on any topic. (Psycho Pass)
}\end{emphasis}\end{changemargin}


Charisma measures your ability to interact effectively with others. It includes factors such as confidence and eloquence, and can represent a charming or bossy personality.

A Charisma check may be required when trying to influence or entertain other people, when trying to impress or tell a lie, or when you have to navigate a complicated social situation.

Typical situations of using the \begin{changemargin}{0.3cm}{0.3cm}\begin{emphasis}{
Kogami, do you know what charisma is?

- The way I see it, it's an innate attitude, like that of a hero or a leader.

-[...] There are three elements that identify charisma: the innate disposition of heroes and prophets, the ability to instill well-being in others with mere presence and a culture that allows you to have a brilliant conversation on any topic. (psycho pass)
}\end{emphasis}\end{changemargin}
Charisma include trying to outwit a guard, cheat a merchant, make money gambling, pass as someone else through a disguise, allay someone's suspicions with false assurances, or keep a cool face while a blatant lie is told.

A character with a Charisma score of -5 is knocked unconscious.

\subsubsection{Read Characteristics scores}\index{Read Characteristics scores}\label{leggereipunteggidellecaratteristiche}

Each ability score typically ranges from 0 to 3, a good ability score is 1, 2 is great, 0 is "normal", 3 is rated "exceptional".

A score of -1 is judged weak, a -2 subnormal, a -3 severely problematic, a -4 almost leads to non-use of the characteristic, a -5 makes it appropriate that he stays in bed and that's it (if he isn't already in a coffin).

\subsubsection{Optional - Age of the character}\index{Optional - Age of the character}\hypertarget{etadelpersonaggio}{} \label{etadelpersonaggio}

Character age affects Physical and Mental Abilities.

\begin{tabular}{llllll}
Period & STR & DEX & CON & INT & WIS\\
\hline
Young & & & +1 & & -1 \\
\hline
Adult & & & -1 & & +1\\
\hline
Mature & & & -1 & & +1 \\
\hline
Elder & -2 & -1 & -1 & +1 & +1 \\
\hline
Venerable &-1 & -1 & -1 & -1 & +1 \\
\end{tabular}

\medskip

The indicated modifiers stack.

\subsection{Ability Scores} \hypertarget{assegnazione.punteggi.caratteristica}{}\label{assegnazionepunteggicaratteristica}

Characteristics scores play an important but not critical role. The player must understand that a "low" score does not mean having a bad character, but rather he will have more fun playing him by leveraging his skills, abilities and peculiar abilities, using ingenuity and wit. Multiple systems for pulling Characteristics are presented.

I personally suggest the \textbf{Basic Mode} approach. In OBSS the characters are not heroes, they are not the chosen ones who stand up as defenders of the planet. The characters are normal people often involved in spite of themselves in situations bordering on survival.

\begin{center}
\includegraphics[width=0.55\linewidth]{immagini/dice4.png}
\end{center}

The undoubted advantage of pulling the values in order of the characteristics is that it allows you to mess up the schemes and avoid builds done at the table.

It is probable that the results you hoped for will not come or that they will even come in features that did not interest you. That's okay. Change your mind, let yourself be inspired by the values obtained! Have fun with the new character, build something new and different, be amazed.

Ability rolls are made in order, so the first roll is for Strength, then for Dexterity, Constitution, Intelligence, Wisdom, and finally Charisma.

Finally, remember that OBSS is an RPG where character death occurs, even more often than in other RPGs. Create good, down-to-earth characters and let the adventure shape the details.

\textbf{\textit{Racial modifiers cannot raise or lower scores beyond +4/-4}}.

\subsubsection{Basic Mode}\index{Characteristics - Basic Mode}\label{modalitabase}

The player rolls 3d6 for each characteristic and in order, he can re-roll once a 1 rolled per triplet (3d6). He then rolls a seventh triplet which he can substitute for another triplet. For each characteristic rolled he checks the sum of the dice rolled with \textbf{Table: Characteristic Roll}.

\begin{changemargin}{0.3cm}{0.3cm}\begin{tcolorbox}[title = Let's Pull Tups Features]

\textbf{First triplet}: \cancel{1},1,4,3 total 8. Strength is -1

\textbf{Second}: 5,6,6 total 17. Dexterity is +2

\textbf{Third}: \cancel{1},2,1,4 total 7. Constitution is -1

\textbf{Fourth}: 6,6,6 total 18. Intelligence is +3

\textbf{Fifth}: 3,4,2 total 9. Wisdom is +0

\textbf{Sixth}: 3,4,4 total 11. Charisma is +0

\textbf{Seventh}: 3,5,2 total 10. Replacing Strength (-1 to +0)

As Human \hyperlink{diverso}{Different} (page \pageref{diverso}) , Tups gets +1 Constitution and +1 Intelligence

\end{tcolorbox}\end{changemargin}

\subsubsection{Optional mode (for cowards!)}\index{Optional - Characteristics Optional mode for cowards}\label{modalitapericodardi}

Each player distributes 5 points among the 6 Characteristics, each Characteristic must have a minimum score of -1 and a maximum of 2 before racial modifiers.

\subsubsection{Traditional Mode}\index{Characteristics - Traditional Mode}\label{modalitadellatradizione}

Each player rolls 4d6 6 times and adds up the best 3 rolls each time. The result obtained is checked against \textbf{Table: Characteristics Roll} and assigned in order.

\subsubsection{Table: Characteristics Roll}\index{Table Characteristics Roll}

The sum of dice rolled for Traits is compared to this table to determine the actual Stats values. \\

\begin{tabularx}{0.45\textwidth}{lX|lX}
\textbf{Val. pulled}& \textbf{Char.}&\textbf{Val. pulled}& \textbf{Chars}\\
\toprule
3 (or less)&-3&13-14-15&+1\\
4-5&-2&16-17&+2\\
6-7-8&-1&18 (or more)&+3\\
9-10-11-12&+0&&\\
\end{tabularx}


\textbf{Remember to apply racial modifiers!}

\subsection{Increase Stats}\label{aumentarelecaratteristiche}\hypertarget{aumentarelecaratteristiche}{}

Through the Skill \hyperlink{supremo}{Supreme} (page \pageref{supremo}) a Characteristic can be increased by one point, up to a maximum value of 4 + the racial bonus, or penalty, of the characteristic.

Magical items or spells are required to increase beyond this value. The increase in Characteristic has retroactive effect only for increases in Constitution, affecting the maximum Hit Points.

The increase in characteristic immediately applies the modifier to Saving Throws and rolls to attack and initiative, the increase in intelligence has repercussions in the next level in the number of skills acquired.

\begin{center}
\includegraphics[width=0.65\linewidth]{immagini/guerrieroispirato.png}

\textit{Brian Boru, High King of Ireland}
\end{center}

\end{multicols}

\vfill

\begin{changemargin}{0.3cm}{0.3cm}\begin{narrator}
Players will still complain about the rolled Stats, it's normal, especially more inexperienced players. Try to make him understand that he should not just look at the Characteristics but see the general whole of the character. Suggest Skills that can make up for the disadvantage.
\end{narrator}\end{changemargin}

\begin{changemargin}{0.3cm}{0.3cm}\begin{tcolorbox}[title = Low characteristics!]\index{Low characteristics}
Low stats are not character death! Instead, try to play so that you don't have to roll dice or try! Strive to be witty, intuitive, proactive, smart .. in short, everything that can help you resolve the situation without necessarily having to roll the dice. In OBSS the Arbiter rewards players who describe and get excited about what the character does with bonuses on Checks!
\end{tcolorbox}\end{changemargin}

\pagebreak

\begin{multicols}{2}

\section{Hit Points}\index{Hit Points}\index{PF}

\begin{changemargin}{0.3cm}{0.3cm}\begin{emphasis}{Anyone who doesn't value life doesn't deserve it. (Leonardo da Vinci)}\end{emphasis}\end{changemargin}


\lettrine[lines=2, lhang=0.33, loversize=0.25, findent=1.5em]{H}{it Points} represent the character's life energy but also skill, luck, the character's ability to resist and fight. As long as the character/opponent has at least 1 Hit Point (HP) he will fight and grapple to the best of his ability.

- Each character starts with 4 Hit Points at 1st level + Constitution score.

- At each level beyond 1st, gains 1d4 Hit Points + Constitution score.

Each point taken in Weapon Proficiency increases Hit Points taken by 3. Additional Skills can raise this score.

Mark the maximum Hit Points you have on the sheet and indicate the current value each time you lose or recover them. Always mark on the sheet what the amount of Hit Points is, after each hit or damage. The amount of Hit Points when the character is \textit{perfectly healthy} is also called Maximum Hit Points\\

\begin{changemargin}{0.3cm}{0.3cm}\begin{tcolorbox}[title = I'm about to die!]\index{I am about to die!}
ESCAPE! Retreat, hide, exit combat. There is no glory in being dead. Better a retreat than a TPK (Total Party Kill or death of the whole group).
\end{tcolorbox}\end{changemargin}

\medskip

\textbf{Hit Points are recovered in several ways}:\index{Recover Hit Point}

- for each night of rest (at least 8 hours) you recover the Constitution value in Hit Points * WP or MP (at player choice, with a minimum of 1 PF) \index{Recover HP sleeping}

- through healing spells (spells, potions .. or other magical items)

- competence \hyperlink{prontosoccorso}{First aid} (page \pageref{prontosoccorso}), through more or less long treatments

Hit Points can also be \textbf{temporary}\index{Temporary Hit Point} or temporarily added or subtracted from your current ones.

A spell that grants +10 temporary Hit Points will raise your current Hit Points by 10, so if you take 8 damage you will be left with 2 temporary Hit Points. If you were to take 13 damage in addition to losing all temporary Hit Points you will also suffer 3 normal Hit Points and further damage will be calculated on his real Hit Points.

- When you gain temporary Hit Points you must choose whether the effect replaces the previous one. The gain does not stack and cannot exceed half the maximum Hit Points

- At the end of the effect that grants temporary Hit Points, these disappear leaving the creature at its previous Hit Points.

- Unless otherwise stated, Temporary Hit Points disappear one hour after they were added.

- Temporary Hit Points are removed first when injured.

A weapon or effect that causes non-lethal damage means that it causes \hyperlink{recuperopuntiferitanonletali}{temporary wounds}\label{feritetemporanee}.

\section{Fate Points}\index{Fate Points}\index{Beginner's Luck}

\begin{changemargin}{0.3cm}{0.3cm}\begin{emphasis}{If fate is against us, so bad for it. (motto of the 1st Carabinieri Parachute Regiment "Tuscania")}\end{emphasis}\end{changemargin}


\lettrine[lines=2, lhang=0.33, loversize=0.25, findent=1.5em]{I}{n} not an easy world Beginner's Luck helps those without experience.
Each character has a number of Fate Points equal to (20 - Level)/5, with a minimum of 1. Fate Points are counted per game session. You recover a Fate Point every time you roll at least three 1s in a check. \index{Recover Fate Points}

At each session they are reset and recalculated, it follows that you do not accumulate Fate Points between one game session and another.

Calculation example:

A level 6 character has: 20-6 = 14/5 = 3 (round to nearest integer) Fate Points to use in the session.

A Fate Point is used as an Immediate or Reaction Action and a character can use a Fate Point to:

\medskip

- add 1d6 to a saving throw, attack roll, or proficiency check. To be declared before rolling the dice. The added die can explode according to the Golden Rules. 1 or more Fate Point\\
- reroll 1d6 (which maybe made 1...). 2 Fate Point\\
- recover 3 Hit Points if they are currently negative. 1 Fate Point\\
- negate a weapon critical roll immediately. 1 Fate Point\\

Using all the Fate Points available it is possible to retract a test, accepting only the new result obtained.


\subsection*{Optional - Chaos Points}\index{Optional - Chaos Points}

One way to add tension is to manage a pool of Fate Points shared between characters and opponents instead of the individual player's. A container, a glass, is placed in the center of the table, containing a number of d6s equal to the number of characters.

Each player is free to take one dice at a time and use them as if they were Fate Points.

The chaos is given by the fact that these data are then moved to another container that the Arbiter, always maximum one at a time per opponent, will use for \textit{his benefit from him}. Once the Arbiter has used the die it is placed back in the player container.

\end{multicols}

\pagebreak

\section{The Traits}\index{Traits}\hypertarget{tratti}{}\label{tratti}

\begin{changemargin}{0.3cm}{0.3cm}\begin{emphasis}{Whoever therefore knows how to do good and does not do it, commits a sin. (James the Just 4.17, Letter of James. NdA referring to the selected Traits)
\smallskip

It is a natural right to satiate one's soul with vengeance. (Attila)
\smallskip

Est Sularus Oth Mithas. ("My honor is my life", Oath of the Knights of Solamnia)}\end{emphasis}\end{changemargin}\medskip


\begin{multicols}{2}

\index{Traits}
\lettrine[lines=2, lhang=0.33, loversize=0.25, findent=1.5em]{O}{n} OBSS there is no clear distinction between good and evil, law and chaos, between what is right and what's wrong.

n OBSS, there are Traits, aspects, and nuances of character that contribute to the character’s background, help players role-play more effectively, and provide guidelines for interpreting the character they’ve created.

A Trait is a detail that better frames the character, outlining their main characteristics and granting them various nuances

\textbf{Each player chooses 5 Traits for his character at character creation.} These will be the \textit{moral, ethical and behavioral compasses} that will suggest the character in his actions and choices.

\smallskip

\begin{changemargin}{0.3cm}{0.3cm}\begin{tcolorbox}[title = Choosing Traits] %box giocatore
Traits are not the character, they do not block or fix it eternally over time. A character is always in constant evolution and so are their character, morals, behavior and desires. Don't be rigid but use the Traits to give you suggestions to inspire you.
\end{tcolorbox}\end{changemargin}

\smallskip

In OBSS, the Tratti (Traits) do not have a positive or negative connotation; they simply help frame the character and indicate which Patron is more interested in that character. They do not define whether you are good or bad; everyone has their own morality regardless of the Traits they possess. 

\textbf{Of the Traits chosen at the first level, identify one, this will start, always at the first level, with a value of 1, the others will start with a value of 0.}

Over time and adventures they can be earned or replaced (in concert between Arbiter and player based on how played) by other Traits. The higher a Trait value, the more present and permeating it is in the character's choices.

During the adventures, certain Traits may also be emphasised, i.e. the Arbiter, following particular scenes and roles, will be able to increase a character's Trait by one point, or a fraction of a point.

For example, following a particular choice and adventure climax, the Arbiter could grant everyone or someone a Courageous Trait or give a +1 to Courageous to those who already have this Trait. For Traits not taken, the base value in points of -1 is considered. that is, the first point is used to take the Stroke and the following ones to emphasize them.

While it is "relatively" easy to acquire new Traits it is extremely difficult to change existing ones. Talk about it with the Arbiter, he will be able to prepare situations and adventures that will help you understand how to evolve the character and possibly change the chosen Traits.

\textbf{Every particularly important action where the character has followed a Trait brings the character closer to the Patron (or Patrons) competent for that Trait}.

In the form you will find \textbf{check} to put next to the Traits, these are marked following suitable actions to increase the value of the Trait; once 10 checks have been reached, the Trait will increase by 1 point and a new ten will start again.

During the adventure, the Arbiter will tell you when to score or cancel partial points. \textbf{As a general rule, it is assumed that a character acquires at least one Trait point per level.}

As the value of the sum of the Traits common to the Patron increases, the character will be able to acquire powers, regardless of whether he is a believer (Follower or Devoted) of that Patron or not.

- At \textbf{5} points you can begin to feel the presence of a Patron linked to a Trait

- At \textbf{10} points the proximity of a Patron linked to a Trait is felt

- At \textbf{15} points you are linked to a Patron by a Trait

- At \textbf{20} points you are a Patron's Champion linked to a Trait.


It is not necessary to believe in a Patron to feel his closeness, it is simply one's own nature (one's Traits) that is similar to the Patron, whether one likes it or not.

Since a Patron's goal is to make his Traits dominant over others, having people of high status and power who are so like him will come in handy in the 1000 Year Judgment.

To identify the most similar Patron check your Trait with the highest value on \hyperlink{tabellacollegamentopatronotratto}{Patron List - Traits} (page \pageref{tabellacollegamentopatronotratto}) and identify the Patron with whom you share it, in case the Trait is shared by several Patrons check the other Traits and based on the similarity choose the Patron. Then check in \hyperlink{cosmologia}{Cosmology} (page \pageref{patroni}) the powers granted by the Patron. This check should be done at each increase of a Trait value.

The Arbiter is free to insert new Traits at his pleasure or requested by the players, it is suggested to attribute these new Traits also to the Patrons.

\pagebreak

\textbf{List of Traits}\index{Table Traits}

\medskip

\noindent\textbf{Greed}: Lack of will to spend material and non-material goods.\\

\textbf{Cynicism}: A skeptical and distrustful attitude towards the motivations and sincerity of others.\\

\textbf{Courage}: The ability to face fear, danger, or difficulty with determination.\\

\textbf{Selfishness}: Those who are only concerned with themselves, their well-being and their own profit, often to the detriment of others.\\

\textbf{Empathy}: The ability to understand and share the feelings of others.\\

\textbf{Generosity}: Willingness to give freely, often without expecting anything in return.\\

\textbf{Kindness}: Compassion, benevolence, and consideration for others.\\

\textbf{Justice}: Fairness, impartiality and respect for moral principles.\\

\textbf{Gratitude}: Recognition and appreciation for what one has received.\\

\textbf{Indifference}: Lack of interest, concern, or emotion.\\

\textbf{Indolence}: indifference to stimuli, boredom, laziness.\\

\textbf{Gluttony}: Desire to consume the best of material and non-material goods.\\

\textbf{Intransigence}: Refusal to compromise or change positions.\\

\textbf{Envy}: Feeling of jealousy or desire for what someone else has.\\

\textbf{Hypocrisy}: Pretending to have virtues or ideals that you don't really have.\\

\noindent\textbf{Lust}: Freeing oneself from one's passions without moral control.\\
\textbf{Malice}: Intentional desire to harm or cause suffering to others.\\

\textbf{Honesty}: Truth and integrity in words and actions.\\

\textbf{Patience}: Endurance and tolerance in the face of adversity and problems.\\

\textbf{Perseverance}: Persistence and determination in achieving goals.\\

\textbf{Responsibility}: Responsibility and reliability in the fulfillment of duties and obligations.\\

\textbf{Resentment}: Anger and indignation following perceived unfair treatment.\\

\textbf{Respect}: Esteem and consideration for others based on their value and ideas.\\

\textbf{Sincerity}: Genuine honesty and authenticity in communication and behavior.\\

\textbf{Pride}: Excessive self-esteem and one's merits, real or presumed.\\

\textbf{Humility}: The realistic vision of oneself and the ability to recognize one's limits.


\end{multicols}

%valutare le motivazioni, una tabella delle motivazioni

If a player doesn't roll the Character Traits he won't give the character experience points.

\vfill

\begin{center}
\includegraphics[height=0.35\linewidth]{immagini/troll.png}
\end{center}

\begin{changemargin}{0.3cm}{0.3cm}\begin{emphasis}{If a traveler doesn't bring back something to share, he's not a \textit{Hero}  but an imposter, an egotist without wisdom. (The Hero's Journey, Christopher Vogler)}\end{emphasis}\end{changemargin}

\pagebreak

-\section{Skills}\index{Skills}

\begin{changemargin}{0.3cm}{0.3cm}\begin{emphasis}{
Anyone who says something is impossible shouldn't disturb whoever is doing it.


\medskip
You don't really understand something until you can explain it to your grandmother. (Albert Einstein)}\end{emphasis}\end{changemargin}\medskip


\begin{multicols}{2}

\lettrine[lines=2, lhang=0.33, loversize=0.25, findent=1.5em]{S}{kills} represent what one knows and what one can do. The scores of skills represent how well the skill is known, and therefore the higher the value, the more expert one is.

\subsection{Core Skills}\index{Core Skills}\label{competenzebase}

\begin{changemargin}{0.3cm}{0.3cm}\begin{emphasis}{
%Anche se indubbiamente il desiderio di conoscere è naturale per tutti gli uomini, la voglia di imparare non è cosa da tutti; la maggior parte, anzi, assaggiato quanto lo studio sia fatica e provata la stanchezza sulla propria pelle, butta alla leggera la noce ancor prima di aver rotto il guscio per gustarne il gheriglio. (Richard de Bury)\medskip


Studying is for losers! (Lobo) }\end{emphasis}\end{changemargin}

Each character has an initial Profession, a life and work path that led him to learn certain skills or what he did (and if he wants to continue doing) before engaging in dangerous adventures.

Some Professions and their related skills are listed, the character acquires these skills with the score indicated in the table.

The initial Profession and the skills acquired must be marked on the sheet, obviously in agreement with the Arbiter it is possible to select different skills and also choose different professions!

\end{multicols}

\textbf{Table: List of Professions and related Skills}\index{Table List of Professions and related Skills}\index{Professions}

\medskip

\begin{tabularx}{0.95\textwidth}{lllll}
\textbf{\textbf{Profession}}& \textbf{1 point} & \textbf{2 points} & \textbf{2 points} & \textbf{3 points}\\ 
\toprule
\textbf{Acolyte}& Occult	& History or Geography 	& Arcana	& Religion\\
\textbf{Alchemist}	& Appraise	&Nature		& Herbalism	& Arcana\\
\textbf{Farmer}		& Survival	&Track	& Animal Handling	&Nature \\
\textbf{Mage's Apprentice}& History \& Geography	&Occult &Myths \& Legends &Arcana\\
\textbf{Lawyer}& Evaluate&Deceive&Sense Emotions&Diplomacy\\
\textbf{Librarian}& Nature\&Geography&Local Trad.&Religion\&Arcana&History\\
\textbf{Lumberjack}& Use Rope&Nature& Navigation&Survival\\
\textbf{Hunter}& Stealth&Track&Survival& Nature\\
\textbf{Caravaneer}&History or Geography&Appraise&Riding&Orientation\\
\textbf{Theatre}& Sense emotions& Languages&Entertain&Acrobatics\\
\textbf{Herbalist}& Myths&Geography&Nature&Herbalist\\
\textbf{Card Player}& Sense emotions&Evaluate&Entertain&Deceive\\
\textbf{Guard}& Sense Emotions&Knowledge Law&Ride&Intimidate\\
\textbf{Guide}& Myths&Dungeons&Nature&Geography\\
\textbf{Pickpocket} & Disable Device&Escape Artist&Stealth&Fairy Hands\\
\textbf{Thug}& Survival&Riding&Sense Emotions&Stealth\\
\textbf{Innkeeper}& First Aid&Appraise&Sense Emotions&Diplomacy\\
\textbf{Merchant}& Languages&Local Traditions&Evaluate&Deceive\\
\textbf{Miner}& Use Ropes&Evaluate&Orienteering&Dungeons\\
\textbf{Fisherman}& Orientation&Swimming&Use ropes&Nature\\
\textbf{Soldier}& Swim& Animal Handling&Jump&Ride\\
\textbf{Carteer}& Local Traditions & Orientation &  Animal Handling & Riding\\
\textbf{Medicine Man}& Myths&Nature&Herbal Medicine&First Aid\\
\textbf{Forest ranger}& Myths&Herbalist&Riding & Nature\\
\textbf{Farmer}& Survival &Herbalism& Animal Handling & Nature\\
\end{tabularx}

\bigskip

\begin{multicols}{2}

For each new profession that you will create associated with 4 skills taken from this list, one skill will start with a score of 1, two skills will start with a score of 2 and the more specific and professional one will start with a score of 3.

Obviously a profession is not expressed in just 4 skills but these are the ones that will come into use most during the adventures, the Arbiter will be helped by your profession to understand how your character will be able to resolve situations and how he will interact with the other characters.

Below is the \textbf{Skills list table} from which to choose for any new professions or customizations of the same.\\

\begin{changemargin}{0.3cm}{0.3cm}\begin{emphasis}{
Although undoubtedly the desire to know is natural for all men, the desire to learn is not something for everyone; indeed, the majority, having tasted how hard studying is and having experienced the weariness on their own skin, lightly throw away the nut even before having cracked the shell to taste the kernel. (Richard de Bury)\medskip
}\end{emphasis}\end{changemargin} \medskip

In agreement with the Arbiter it is also possible to change the order of the Skills making the character more capable in some skills rather than others.

\subsubsection{How the character's Background affects}\label{quintacompetenza}

At the creation of the character, the player can decide to take a +1 to an already known Competence or take a new Competence, linked to the character's history, with a score of 1. \\

The player \textbf{increases the score of a characteristic that relates to the Profession or background by 1} up to the maximum value of 4+racial modifications. It could be Intelligence to an Apprentice Mage, but if this is a hobby bodybuilder it could also be Strength.

\begin{changemargin}{0.3cm}{0.3cm}\begin{tcolorbox}[title = Profession ???]
Do not underestimate the choice of Profession!. Not everything can be resolved with axes or magic. Knowing how to untangle knots, follow tracks, recognize herbs or diseases make the character an expert, create a profession. You don't have to define the character only by the Skills he has but by what and how well he can do it. A character of low level but expert in survival will always be more useful than an experienced fighter when it comes to crossing a desert.\end{tcolorbox}\end{changemargin}

\end{multicols}

\medskip

\textbf{Table: List of Skills and related Characteristic of use}\index{Table List of Skills and related Characteristic of use}

\medskip

\begin{tabular}{lllll}
\textbf{Strength} & \textbf{Dexterity} & \textbf{Intelligence} & \textbf{Wisdom} & \textbf{Charisma}\\
\toprule
Climbing & Acrobatics & Arcana & Riding & Diplomacy \\
Intimidate& Escape Artist &Craft* &\textit{Awareness} & Entertain \\
Swimming & Fairy Hands & Knowledge* & Animal Handling & Deceive \\
Jumping & Stealth & Disabling Devices & Nature & Local Lore \\
&  & Herbalism &Orientation & \\
& Use Rope & Falsify & Sense Emotions & \\
& & Appraise &First Aid &\\
&& &Track &\\
&& &Survival &\\
&&& &\\
\end{tabular}

The \textbf{Knowledge} must be explained on which topic it concerns: Dungeons, Law, Languages, Plans, Occult, Architecture and Engineering, Nobility and Heraldry, Myths and Legends, Religion, History, Geography ...

\begin{multicols}{2}

At each \textbf{level after the first} distribute a number of points equal to half the Intelligence score +1,[(Int/2)+1], with a minimum of 1 point, among the skills already known or perfected in the adventure or learned from scratch.

\textbf{No Basic or Active skill can have a score higher than character level+2.}


\subsubsection{Awareness}\label{consapevolezza}\index{Awareness}

A perk that all characters have is \textbf{Awareness}, which is the ability to sense the environment around them. This proficiency has a fixed score equal to 1/3 of the character's level (rounded up).

Rather than using Awareness to seek information, players should ask questions, investigate, snoop, speculate, and debate rather than just asking for an Awareness roll to find something.

\subsubsection{Learn new skills, professions}\label{apprenderenuovecompetenze}

A character can learn a new skill by studying/practicing at least 4 hours a day for at least 4 months with a teacher who has a proficiency score equal to or higher than the character's target. After this time the player can award one point to the basic skill he applied for.

To learn a new profession, he must spend at least 6 months for 6 hours a day with those who practice that profession. After 6 months the character acquires the 4 skills of the profession. If he already has some of those skills, increase the score by 1 for each skill already possessed.

\subsubsection{Skills and their areas of use}\label{competenzeambitidiutilizzo}

The Competencies and their areas of use are briefly described. They are general guidelines on what to use the skills. The number of Actions needed to carry out the typical check is also indicated, obviously more complex uses require more time and Actions.

The Actions required for the check may vary according to the ability of the character and the complexity of the check.

In any case, always remember to carefully evaluate how the player claims to carry out the actions to understand their duration and effects. 

Skill with \textbf{*} get penalty given by \hyperlink{equipaggiamento.armature.scudi}{armor} donned (page \pageref{equipaggiamentoarmature}).\\


\textbf{Acrobatics* (DEX)}: This skill is used to maintain balance on narrow or precarious surfaces, to dive, roll, somersault, somersault, overcome obstacles as well as fall and not get hurt. 1 Action.

\textbf{Arcana (INT)}: With this proficiency, you are an expert in magic and spells, magic items and are able to identify the spells that are being cast. 1 Action.

\textbf{Climb* (STR)}: With this proficiency, you can climb vertical surfaces, from city walls to cliff faces. It is linked to the Move Action. With 8 points the movement is only halved.

\textbf{Craftsmanship (INT)}: You must specify the type of Craftsmanship you are proficient in. You are proficient, but not at the Profession level, in a form of craft.

\textbf{Escape Artist (DEX)}: With this skill, you can free yourself from shackles and handcuffs. 1 Action every 10 of DC. With 6 points time is 1 Action every 15 DCs, with 12 it is 1 Action every 20 DCs.

\textbf{Riding (WIS)}: With this proficiency, you can ride professionally and give commands to your mount. 1 Action.

\textbf{Awareness (WIS)}: to seek, notice, notice. It is something active. 2 Actions. Using 1 action imposes a -1d6 penalty on the check.

\textbf{Knowledge of Architectural and Engineering (INT)}: You are an experienced builder and can evaluate the structure of buildings. You can also recognize architectural styles and create interior and furniture projects. 1 Action.

\textbf{Knowledge of Dungeon (INT)}: With this proficiency you have knowledge of Aberrations, caves, dungeons, Oozes. 1 Action.

\textbf{Knowledge of Geography (INT)}: With this skill you have knowledge about climate, population, terrains, territories, nations and borders. 1 Action.

\textbf{Knowledge of Law (INT)}: With this proficiency you know the Law of a region. You are expert in knowing the rules and quibbles. You know how to cite cases and you know other hustlers and judges. 2 Actions.

\textbf{Knowledge of Languages  (INT)}: Each point in this skill allows you to learn a new written and spoken language. A good Language score helps you understand unfamiliar languages and make yourself understood. It is also used to understand complex texts. Variable cost.

\textbf{Knowledge of Myths and Legends (INT)}: You have a real passion for traditional and more remote myths and legends. Learn about locations, history and legendary creatures. 1 Action.

\textbf{Knowledge of Nobility and Heraldry (INT)}: Know noble lines, houses, rumors, coats of arms, personalities and major possessions and treasures. It also applies to famous and important people. 1 Action.

\textbf{Knowledge of Planes (INT)}: With this skill, you are an expert in Planes and their inhabitants. 1 Action.

\textbf{Knowledge of Occult (INT)}: With this proficiency, you are well versed in the occult, fiendish creatures. 1 Action.

\textbf{Knowledge of Religion (INT)}: With this skill you have knowledge of Patrons, mythology, Celestials, Undead, sacred symbols, ecclesiastical tradition, liturgical feasts and anniversaries. 1 Action.

\textbf{Knowledge of History (INT)}: With this competence you have knowledge of History such as wars, migrations, colonies, foundations of cities, important events.. 1 Action.

\textbf{Diplomacy (CAR)}: With this skill, you can resolve disputes, and gather valuable information and rumors from people. Competence is also used to negotiate effectively with the right etiquette and conduct suited to the disputed situation. Variable cost.

\textbf{Disable Devices (INT)}: With this skill you can disarm Traps and open locks, sabotage simple mechanical devices, such as catapults, wagon wheels or doors. 1 Action every 10 of DC. With 6 points time is 1 Action every 15 DCs, with 12 points it is 1 Action every 20 DCs.

\textbf{Herbalism (INT)}: With this proficiency you have knowledge of how to recognize and prepare potions and natural poisons. Scoring applies to potion brewing checks. Recognize Natural Potions 1 Action every 10 of DC. With 6 points time is 1 Action every 15 DCs, with 12 points it is 1 Action every 20 DCs.

\textbf{Falsify (INT)}: With this skill, you can forge works of art, maps, signatures... 1 Minute

\textbf{Animal Handling (WIS)}: With this skill, you can train and tame animals. 1 minute every 5 of DC. With 6 points the time is 1 minute every 10 DCs, with 12 it is 1 minute every 15 DCs.

\textbf{Intimidate (STR)}: Intimidate relies on the physical approach to convince the interested party. 2 Actions. With a score of 12 it costs 1 action.

\textbf{Deceive (CHA)}: The Deceive skill can be used to Deceive (thus telling lies) or Persuade (adapting the truth) in order to convince the interested party of his words. Variable cost.

\textbf{Entertaining (CHA)}: With this competence one is expert in an artistic expression, from singing to acting, from dancing to playing musical instruments. The form of entertainment must be specified. Variable cost.

\textbf{Fairy Hands* (DEX)}: With this skill, you can pickpocket, draw a concealed weapon, or perform other actions without being noticed. 1 Action.

\textbf{Stealth* (DEX)}: With this skill, you are able to move without making noise and hide. 1 Action.

\textbf{Nature (WIS)}: With this skill you have knowledge of Animals, Fairies, seasons and cycles, atmospheric weather, plants. 1 Action.

\textbf{Swim* (STR)}: With this skill, you are able to swim, even in stormy waters. Without competence, one knows how to float in placid waters. Tied to the Move Action.

\textbf{Orientation (WIS)}: With this proficiency you have a sense of direction and orientation making it impossible to get lost no matter what environment you are in. 2 Actions.

\textbf{Sense Emotions (WIS)}: With this skill you can tell if someone is lying or you can guess their true intentions. 1 Action.

\textbf{First Aid (WIS)}: With this skill, you can heal wounds and diseases. Variable cost.

\textbf{Jumping* (STR)}: With this perk, you are an expert and skilled jumper. 1 Action.

\textbf{Track (WIS)}: With this proficiency, you can follow tracks left in the environment. 1 Action every 10 of DC. With 6 points time is 1 Action every 15 DCs, with 12 points it is 1 Action every 20 DCs.

\textbf{Survival (WIS)}: With this proficiency, you can survive and navigate the wilderness. Proficiency is also used to actively search for traps and pits. 1 minute to search for traps in 3x3 meters, with a score of 6 it costs 3 rounds, with a score of 12 it costs 1 round, with a score of 18 it costs 1 Action.

\textbf{Local traditions (CHA)}: With this skill you have knowledge of the inhabitants (best known), customs, legends, laws, personalities, traditions. It is necessary to specify a geographic region where the knowledge is applicable. 1 Action.

\textbf{Use Rope (DEX)}: With this skill, you are proficient in ties and knots to secure and hold down objects or people. 2 Actions.

\textbf{Appraise (INT)}: With this proficiency, you can estimate the monetary value of an object. 1 Action every 5 of DC. With 6 points time is 1 Action every 10 DCs, with a score of 12 it is 1 Action every 20 DCs.

\medskip

\subsubsection{Optional - Don't use Basic Skills}\index{Optional - Don't use Basic Skills}\label{nonusarecompetenze}

Let the players choose their profession and do not score any base Skill values or scores.
Think with an open mind and understand, both you Arbiter and you Player, for each situation who has the profession and Skills that best suit the check.
If pertinent to the profession, the check resolves with a 3d6+Wisdom+1/2LV, if it is not pertinent the Arbiter will reduce the bonus given by the level, using the most appropriate Characteristic. Better yet, based on the description of how the check is carried out, the outcome will be decided.

\subsection{Active Skills}\index{Active Skills}\label{competenzeattive}

\textbf{The character takes 1 point, at each level, to distribute among the Active Skills or attribute it to the Basic Skills.}.

\medskip

The \textbf{Active Skills} are: Magic Proficiency, Weapon Proficiency, Saving Throws (Reflexes, Fortitude, Will).

\textbf{Magic Proficiency (MP)}: \index{MP}\index{Magic Proficiency} indicates the ability and proficiency in casting a spell.

\textbf{Weapon Proficiency (WP)}: \index{WP}\index{Weapon Proficiency} is the ability and prowess to fight with a melee or ranged weapon.

\textbf{Saving Throws} are increased by Feats choices.

Attributing the Active Skills point to \textbf{Basic Skills} means distributing 4 additional points on at least 3 Basic Skills as desired. \index{Increasing Basic Skills}


\begin{changemargin}{0.3cm}{0.3cm}\begin{emphasis}{There is only one way to train: the right one. (Carl Lewis)
\medskip


Wang Chi:  You Ready, Jack?

Jack Burton: I Was Born ready. 	(Big Trouble in Little China, Film 1986)
}\end{emphasis}\end{changemargin}


\subsubsection{Saving Throws}\index{Saving Throw}\label{tirisavellza}

The \textbf{Saving Throws} (abbreviated as ST) are used when the character is subjected to an effort, either of physical or mental endurance or exceptional agility. Saving Throw scores are modified by the Feats chosen. More physical Feats will tend to improve the character's endurance aspect, more athletic or attention Feats will increase reflexes, purely mental Feats will strengthen the character's will.

The \textbf{Fortitude save} indicates how well you are able to withstand physical suffering or attack against your vitality and health. The value of \textbf{Constitution} is added to the Fortitude saves.

The \textbf{Will save} indicates resistance against mind-affecting and other magical effects, which seeks to modify your free will in choices and actions. Add the value of \textbf{Wisdom} to Will Saving Throws.

The \textbf{Reflex save} indicates how agile and quick you are to avoid obstacles or magic. The value of \textbf{Dexterity} is added to Reflex Saving Throws.

\begin{changemargin}{0.3cm}{0.3cm}\begin{tcolorbox}[title = Non-standard Saving Throws]
It is possible that Saving Throws with different modifiers are required, i.e. a Fortitude save with a Strength modifier or a Will save with a Charisma modifier. The Arbiter will tell you when a different modifier applies.
\end{tcolorbox}\end{changemargin}

When a Saving Throw is asked, it means making a check on the required Active Proficiency, whether it be Will, Fortitude, or Reflexes.
The check will be performed by rolling 3d6 + the score in the Will, Reflex or Fortitude save + the value of the Characteristic connected to the Active Competence type (Wisdom, Dexterity or Constitution) + Feats + magical bonuses (objects that affect the Saving Throw) and various modifiers present.

If you roll \textbf{three times a 6 on the save} the same is successful, regardless of the final result.\index{Roll three time 6 on Saving Throw}

\subsubsection{Weapon Proficiency}\label{competenzaarmi}

The \textbf{Weapon Proficiency} (abbreviated as \textbf{WP}) indicates the ability and proficiency in using a weapon. Proficiency is directly reflected in checks to strike an opponent with weapons.

The \textbf{Attack roll for melee weapons}\index{Melee Weapon} is resolved with a Weapon Proficiency check (\textbf{WP}) + \textbf{Strength} + any Feats and magic bonuses opposed to the Defence of the opponent (Dexterity + Armour/shields/bonuses).

The \textbf{Attack roll with ranged weapons} \index{Ranged Weapon}(bows, crossbows, throwing knives, javelins, stones...) is resolved with a Weapon Proficiency check (\textbf{WP}) + \textbf{Dexterity} + any Feats and magical bonuses opposed to the opponent's Defence (Dexterity + armour/shields/bonuses).

When assigning a point to \textbf{WP} it must always be specified on which list of weapon it is taken, if it is not declared then it is like having taken it in the Simple Weapons group.
Check list \hyperlink{lista.armi}{Weapons for Uniform Type} (page \pageref{lista.armi}).\index{Uniform Type}

The character can decide to assign his point to a weapon list that already knows, thus improving his ability on these weapons or learn another list of weapon.

The player must consider that the better the skill he has with a list of weapon, the more easily he can take advantages in it, but he will know fewer weapons.

If the player has not assigned any points in \textbf{WP} he may use, without penalty to hit, only the weapons listed as Simple Weapons.

A character who uses a weapon present in the Weapon Lists that he knows or in Simple Weapons will always apply his Weapon Proficiency (WP) value to the Attack Roll, only when using an unknown weapon he will have penalties (-1d6).

The \textbf{Simple Weapons} are: Dagger, Light Mace, Club, Spiked Mace, Footman's Short Lance, Staff, Crossbow (Light), Javelin\index{Simple Weapons}

Using a \textbf{Weapon without the appropriate proficiency} in the group to which it belongs imposes a -1d6 on the attack roll.\index{Weapon without proficiency}

In order to use \textbf{Light Armour} or \textbf{Light Shield} it is necessary to have at least one point in Weapon Proficiency or Strenght at least -1.\index{Light Armour}\index{Light Shield}

In order to use \textbf{Medium Armour} and \textbf{Medium Shield} it is necessary to have at least 2 points in Weapon Proficiency.\index{Medium Armour}

With at least 3 points in Weapon Proficiency and 1 in Strength you can use without penalty \textbf{Heavy Armour} and \textbf{Heavy Shields}.\index{Heavy Armour}\index{Heavy Shields}

Using \textbf{Armour without the appropriate proficiency} prevents you from using the Dexterity value in Defence and the Armour's bonus to Defence is reduced by 1.\index{Armour without proficiency}

Using a \textbf{Shield without proper proficiency} worsens the attack roll by 1, and the shield grants a maximum Defence bonus of 1.\index{Shield without proficiency}

\subsubsection{Magic Proficiency}\label{competenzamagica}

The \textbf{Magic Proficiency} (abbreviated as \textbf{MP}) allows the character to be able to know more spells, more powerful, more effective and more easily.

A character with high \textbf{Magic Proficiency} can manipulate more spells and with better results.

The Magic Proficiency value establishes, together with the Adept of Magic Feat and the Characteristic score, the maximum level of spells that can be cast.

\begin{changemargin}{0.3cm}{0.3cm}\begin{tcolorbox}[title = Tups reaches 4th level!]

Tups has reached the 4th level! This is how he distributed the points of the Active Skills.

\textbf{1 level}: +1 Weapon Proficiency, Skills Devout Armour (+2 Will, +1 Reflex), Weapon Focus (+1 Reflex, +1 Fortitude)

\textbf{2 level}: +1 Magic Proficiency, Skills: Powerful Blows (+2 Fortitude)

\textbf{3rd level}: +1 Magic Proficiency, Skill: Faithful (+2 Will, +1 Fortitude)

\textbf{4th level}: +1 Weapon Proficiency, Skill: Prudent Spellcaster (+2 Reflexes, +1 Fortitude)

\textbf{\textit{Total}}: +2 WP, +2 MP, +5 Reflex saves, +5 Fortitude saves, +4 Will saves
\end{tcolorbox}\end{changemargin}

Each point attributed in the Basic Skills or Weapons Proficiency or Magic Proficiency allows you to take +1 bonus in the relative check (Attack Roll, Magic Test).

\subsubsection{Optional - Trade Active Skills for feat}\index{Optional - Trade Active Skills for feat}

The Arbiter can grant upon the player's request that the acquired point of Active Skill can be spent not to increase Magic or Weapon Proficiency, but to select a new Feat.


\end{multicols}

%\begin{center}
%\includegraphics[width=0.25\linewidth]{immagini/giavellottiragazzo4.png}
%\end{center}


\pagebreak

\section{Let's build the Character}\index{Character}

\begin{changemargin}{0.3cm}{0.3cm}\begin{emphasis}{
"Never forget who you are, for surely the world will not forget it. Make who you are into your strength, so it can never be your weakness. Make Armour out of it, and it can never be used against you." (Tyrion Lannister)
}\end{emphasis}\end{changemargin}\medskip


OBSS is a tough, dangerous, deadly system but also full of satisfaction. Your characters are not heroes, they are not chosen ones. They are unfortunate people who find themselves in businesses where perhaps they will survive and it will be at the expense of some comrades. You don't choose the adventure but it drags you impetuously into it. Be strong, courageous, witty but not reckless.

Survive and claim the Law of the Prize and you will see that as the levels go by you will acquire extraordinary skills and abilities!. \textit{Spes last goddess}!


\begin{multicols}{2}

\lettrine[lines=2, lhang=0.33, loversize=0.25, findent=1.5em]{F}{irst} of all prepare in front of you the card and a sheet where you can take notes and notes.

To create a character try answering these questions, they can help you imagine and shape it.

- Imagine what he looks like

- What is the main trait of the character

- What are his tics, ways of doing, habits

- What are its primary goals

- A curious thing, a funny thing, an embarrassing one and a typical expression of the character

- What he is good at, what he is committed to, what he is bad at

- The three main flaws and three strengths of the character

\begin{center}
\includegraphics[width=0.6\linewidth]{immagini/Leonidas_I_of_Sparta.png}

\textit{Leonidas of Sparta}
\end{center}

He grew up in a family, in a clan, a wanderer, on the street.. what brought him up and what choices did he make to get this far?

What is his fighting style and strategy type of him? Magic, Sword, from the rear.. inciting comrades.. running away...

And not least: what is the purpose of him? what made him leave the house, from his security .. from a "normal" life and embark on that of an adventurer?

To begin, read the chapter on Races and identify your character's.

Always remember that this is a cruel world, full of risks, traps and monsters, but also opportunities that can make you powerful or very rich.

Retrieve some d6 and roll!

Consult the chapter of \hyperlink{assegnazione.punteggi.caratteristica}{Characteristics} to understand how lucky you were (page \pageref{assegnazionepunteggicaratteristica}).

If you have Intelligence 2 or higher choose another \hyperlink{linguaggi}{language} (page \pageref{linguaggi}) spoken/written besides Common, and if you have 3 you can choose 2 more languages.

And if the Charatectistics values didn't come out as you expected then let the chaos lead you and create something different but equally fun and great.

Switch to Active Skills, here you have 1 point to distribute between Weapons Proficiency and Magical Proficiency.

Weapon Proficiency helps you strike better. Magic Proficiency is the only thing that allows you to use magic. Also remember that points in Weapon Proficiencies must be declared to which \hyperlink{lista.armi}{list of weapon} (page \pageref{lista.armi}) they have been applied.

If you have no points in Weapon Proficiency you can only use \hyperlink{armi.semplici}{Simple Weapons} (page \pageref{listweaponsinmultilists}) without incurring penalties on the attack roll.

Saving Throws improve as Feats are chosen, Saving Throw scores determine your survivability and ability to resist trauma and magic

Basic Skills are assigned based on the chosen Profession. Choose it with attention and care, in addition to the skills provided by the chosen Profession you can take a fifth skill from your background or increase the capacity in one. Your Profession determines what you can do, and remember that based on your background and chosen profession you increase a Characteristic by 1, up to a maximum of 4 + racial modifier.

Hit Points are equal to 1d4 + Constitution + 3 if you put 1 point in Weapon Proficiency (WP).

At this point choose the \hyperlink{tratti}{Traits} (page \pageref{tratti}). Do it carefully, you are building your character and the Traits outline the character with strong brushstrokes. Remember that they will be essential for the choice of \hyperlink{patroni}{Patrons} (page \pageref{patroni}).

In the Traits tab where there is a Patron, mark with an X the Traits that connect you to the Patron you have chosen (if you have chosen one).

Finally, remember that a "lonely" and "cruel" character sounds good in a story where he is the only protagonist, but here you play in a group, don't take Traits in obvious opposition to the others or in any case don't play as \textit{asshole}, otherwise the character will naturally be pushed away from the other players.

Choose the Role \hyperlink{svantaggi}{Disadvantage} (page \pageref{svantaggidiruolo}) and if you want also disadvantages and \hyperlink{vantaggi}{advantages} (page \pageref{vantaggi}). Remember to play it, otherwise it's not fun and you won't get experience points.

If you have points in Magic Proficiency remember that you must also have taken the Adept of Magic Feat otherwise you don't have access to any List of Magic and Spells.

At this point you must choose which Spells you know.
On your Tome of Magic you can record a number of spells equal to 2 + your Characteristic modifier for spells, of these spells he can know, i.e. cast, 2 + half the value of the Characteristic modifier for spells.

Go to \hyperlink{abilita}{Feats} (page \pageref{abilita}), at the first level you choose two, pay attention to the prerequisites and also to any Skills that your race grants you.

Choose the \hyperlink{equipaggiamento}{equipment} (page \pageref{equipaggiamento}), \hyperlink{equipaggiamento.armature.scudi}{ armor} (page \pageref{equipaggiamentoarmature}), \hyperlink{equipaggiamento.armi}{weapons} (page \pageref{equipaggiamentoarmi}), backpack, two torches, some food rations… a stuffed animal. . what seems indispensable to you for the adventure.

Then update the part of the sheet relating to Defence, noting what bonus the worn Armour and shield gives you.

Get into the part, treat yourself to play this amazing character. If you ever get tired of playing it and want to try something different, talk to the Arbiter, he will be able to advise you and suggest the best way.
Moreover, you have the advantage that classes do not exist in OBSS, the character grows, evolves and learns based on what you do and experience. You can prepare the "build" at the table but you will never be sure that your character will evolve as you thought. Let it live and grow!

\begin{center}
\includegraphics[width=0.9\linewidth]{immagini/Alexander_and_Bucephalus_-_Battle_of_Issus_mosaic.png}
\textit{Alexander the Great}
\end{center}

Finally, remember the Law of the Prize \index{Rules of the Prize}. Yeru is ferocious, often evil, even more he will want to kill you, yet for those who survive there is the Law of the Prize, a law that not even the Patrons can violate. The Law is quite simple in its basic concept "To whoever survives will go the treasures and the glory".

\subsection{Level Up}\index{Level}\index{Level Up}\label{avanzamentodilivello}

\begin{changemargin}{0.3cm}{0.3cm}\begin{emphasis}{
But there are things that cannot be understood with reflection, you have to live them. (The Neverending Story, Michael Ende)
}\end{emphasis}\end{changemargin}



Every time the Arbiter confirms that you have leveled up, you have to carry out various operations to update the character.

\textit{First take the board, pencil and eraser and dice (at least the d4).}

- Upgrade the Level by increasing it by 1

- Upgrade Experience Points

- Distribute 1 point between Weapon Proficiency and Magic Proficiency

- Increase Hit Points by 1d4+Constitution and add 3 if you gave 1 point in Weapon Proficiency

- If you have assigned a point in Weapon Proficiency, establish whether you take a new Weapon List or deepen your knowledge of an already learned list.

- Checked if you take a new Feat or improve an existing Feat, pay attention to the prerequisites. Remember that you get a new Feat at all levels except 5-10-15-20.

- Updated the Save Throw scheme based on the new Feat taken.

- Updated the part of the Attack rolls according to the new value of the Proficiency Weapons and Feat

- Distribute (Int/2)+1, with a minimum of 1 point, among the Basic Skills known or learned during adventures.

- Updated Fate Points score (20-level)/5

- Increase a Trait score as the Arbiter tells you. Check if you have reached enough points to acquire Trait-related powers.

- Check the maximum level of the spell that can be cast based on the new Magic Proficiency score, the Adept of Magic Feat and the Characteristic score.

- If you have increased your Magic Proficiency you learn 2 new spells or by sacrificing one you can learn two cantrips (0 level Spells) present in your Tome of magic. You can also replace the spells learned with others in the Tome.

- Updated the second part of the sheet according to the new Magic Proficiency score

As you may have noticed the scores of the Skills are low, you take a few points to distribute at a time.
As players you have the opportunity to prefer a specialized approach or "bet" on a few specific Skills or dilute the points on several skills to know a bit of everything and have no penalty in the checks (the check is done only with 1d6 + Characteristic if you have no points in Competency).

A suggestion is also to use Feats, and in particular Expert, which grants you a +2 bonus on Proficiency checks.

\begin{changemargin}{0.3cm}{0.3cm}\begin{narrator}
The \textbf{perceived} power level of characters in OBSS is lower than that of other RPGs. In OBSS the aim is to explore and understand this changing and crazy world. The weakness of the character is only a perception and in fact you will soon realize the true power of the character. Play as a group and you will survive because remember that this is a bad, spiteful and deadly world with selfish people.
\end{narrator}\end{changemargin}


%\subsection{Suggerimenti per divertirsi e sopravvivere nelle avventure di OBSS}\index{Linee guida per i giocatori}\label{suggerimentigiocatori}

\subsection{How to Survive and Have Fun}\index{Guidelines for players}\label{tips for players}


\begin{changemargin}{0.3cm}{0.3cm}\begin{emphasis}{
- We need a plan.

- Since when do heroes need plans? (Final Fantasy XII)

\medskip

I love successful plans! (Colonel John "Hannibal" Smith, A-Team)}\end{emphasis}\end{changemargin}\medskip


These are tips derived from OSR principles that help characters stay alive.

\begin{itemize}[leftmargin=*]

\item
Every fight is potentially lethal. Decide wisely and approach it carefully. Learn to escape, don't be afraid to survive.

\item
There is not everything in the sheet. A character sheet is the perimeter of the character but it does not define what he can or cannot do. Squeeze your brains and be creative, alternative, curious but not suicidal or reckless.

\item
Don't solve everything with a roll of dice. Ask the right questions, talk to your classmates and carefully describe what you intend to do. Remember that the Arbiter rewards accurate descriptions. Describing how something is done often avoids rolling the dice and then failing.

\item
Low characteristics are only low characteristics and not the character. Take advantage of the skills, the Abilities, make sure you have to roll as few dice as possible to solve the problem.

\item
Improvise, adapt and reach the purpose! (Tom Highway - Gunny, Film). Or like some of my players preferred "Improvise, \textbf{Deceive}, achieve the goal".

\item
Live your character to the fullest. Amplify his story brings his past into the present. Help your companions get to know you and the Arbiter build better stories around your stories.

\item
One thing that no one can ever take away from you is being heroic, intelligent, resolute, stubborn, stubborn but not stupid. Live the adventure to the full but never be afraid to survive.

\item
Describe in a realistic way what you do, you will help the Arbiter and the companions around you. It's definitely better than saying "I'll do an Awareness check". Exalted in describing the most important actions, the Arbiter will take them into account.

%\item
%E finché non potrai dire "\textit{Io sono cattivo, incazzato e stanco. Sono uno che mangia filo spinato, piscia napalm e riesce a mettere una palla in culo ad una pulce a 200 metri}." (Tom Highway - Gunny, Film) allora stai al tuo posto e non fare lo sbruffone, c'è sempre qualcuno più grosso ed arrabbiato di te.

\item

Always remember that the greater the danger, the greater the experience gained. The deeper the dungeon is, the more treasures and experience you gain!

\item
The aim is to have fun, entertain and savor the challenge. Don't create a character that is against other characters or always gives annoyance and problems. Balance your desire with the needs of the group, because always and \textbf{only as a group will you survive} and never only as an individual.

\item
Think before you act, but don't let others wait for you. Use the time between your rounds to plan how best to act.

\item
If you have difficulty understanding or imagining something, ask the Arbiter for more information and clarification, he will only be pleased.

\item Be proactive, try to overcome your shyness.

\item Embrace failure. Fail with style is much better than a boring victory.

\item Make your character always care about more than his life.

\item Don't be afraid to discuss other characters, but always make sure you don't get personal with the players.


\end{itemize}


\end{multicols}


\vfill

\begin{changemargin}{0.3cm}{0.3cm}\begin{emphasis}{
"The star that burns twice as bright burn half as long" (Anonymous)
}\end{emphasis}\end{changemargin}\medskip


\begin{center}
\includegraphics[width=0.4\linewidth]{immagini/threasure2.png}
\end{center}

\pagebreak

\section{Rules for Skills}\index{Rules for the Skills}\index{Skills}

\begin{changemargin}{0.3cm}{0.3cm}\begin{emphasis}{
The law needs to be brief, so that it is easier for the ill-practiced to remember it. (Seneca)}\end{emphasis}\end{changemargin}


\begin{multicols}{2}

\lettrine[lines=2, lhang=0.33, loversize=0.25, findent=1.5em]{T}{\textbf{he}} \textbf{tests (checks) for the Skills are performed by rolling 3d6, at the result of the dice is added the score of the competence (basic or active) and of the connected Characteristic and any magical and circumstance bonuses or penalty, the result obtained must be communicated to the Arbiter, who will compare it with the difficulty (DC) of the check }.

When you have to establish a difficulty, start thinking that the check must be related by a "normal" person. Don't think "if I had to do it the check would be impossible", "if Arsenio Lupine does the check, the check is very easy". Start from the assumption that the difficulty must encompass all the circumstantial elements.

Think if it is raining, there is little light, the character is running, he is injured, he does things in a hurry and also the complexity of the thing he has to do, jumping a 3 meter ditch is not like a 3 meter one in the dark, without shoes, in the rain and chased and with pockets full of coins...

Deciphering an ancient script may be a piece of cake for an expert linguist, but for a "normal person" who has no idea what he may have before the check is simply impossible. This "impossible" is your DC, the difficulty of the check.

And don't be scared if the characters fail the checks, it will make the adventure more interesting and allow the Arbiter to introduce facts, clues and new adventures.

%\medskip
%\begin{center}
%\includegraphics[width=0.8\linewidth]{immagini/master2.png}
%
%\textit{The Master of the Gamblers}
%\end{center}

\medskip

\textbf{When you have to make a check for a basic skill in which you are not prepared, i.e. you have no points, you must roll only 1d6 + score of the connected characteristic}.\index{Check skill without proficiency}


When written -1d6 it means that one die is rolled less (or two if it is -2d6), likewise if +1d6 it means to roll one more dice.

\begin{center}
\includegraphics[width=0.6\linewidth]{immagini/Foster_Bible_Pictures.png}

\textit{Bible Pictures and What They Teach Us. Experience Awareness, Stealth}
\end{center}


The table below serves to relate the difficulty to the minimum skill required to succeed with an average roll (a score of 10 when throwing 3d6). Use these indications to get an idea of the scales of difficulty.

The Arbiter won't tell you give me a check at difficulty 10, but he will say that the check doesn't present elements of particular difficulty.

%\begin{center}
%\includegraphics[width=0.9\linewidth]{immagini/difficulty.png}
%
%\textit{A City on a Rock, long attributed to Goya, is now thought to have been painted by 19th-century artist Eugenio Lucas Velázquez. Ottima prova di falsificazione}
%\end{center}
%\medskip

\medskip

\textbf{Table: Difficulty class}\index{Table Difficulty Class}

\medskip
\begin{tabularx}{0.45\textwidth}{lll}
\textbf{Diff.} & \textbf{Description} & \textbf{Level}\\
\textbf{DC}&\textbf{difficulty}& \textbf{Proficiency}\\
\toprule
5 & Extremely Easy & Null\\
10 & Easy & Mediocre\\
15 & Normal & Normal\\
20 & Difficult & Good\\
25 & Very Difficult & Very good\\
30 & Heroic & Excellent\\
35 & Near Impossible & Amazing\\
40 & Impossible & Epic\\

\end{tabularx}

\begin{changemargin}{0.3cm}{0.3cm}\begin{narrator}
Avoid asking for checks if the players declare HOW they perform the check, how and where they look for, what dialogue they concoct to intimidate the target. Evaluate carefully how the player describes what he does because this is already the check. It's not just for speeding up the game, it's for stimulating players to think whole-heartedly and immerse themselves in the character and environment. In addition to making the game more dynamic, all players will participate in the situation and collaborate by declaring what and how they act. Always use common sense and save dice rolls! Rolling a die means creating the possibility to fail!
\end{narrator}\end{changemargin}

\bigskip 

If you must roll an Characteristic check, you roll 3d6 and add the characteristic score and other modifiers. Communicate this result to the Arbiter who will compare it with the difficulty (DC).

\subsection{The Golden Rules}\index{The Golden Rules}

Unless otherwise specified, three basic rules \index{Basic Rules} called \textbf{Golden Rules}:\index{Golden Rules} apply to all skills checks (Basic, Active).

\begin{itemize}[leftmargin=*]
\item
The \textbf{6 explode}, i.e. if in the 3d6 check a die makes six, add the result and roll, and if it makes 6 again add the result and roll again and again..
\item
\textbf{1 bad luck}, if you roll a 1, subtract 1 from the sum of the dice rolled (and therefore the die that rolls one counts as zero)
\item
\textbf{Trust your luck}. For every 4 points between Skill (Basic or Active) and Characteristics that you give up adding in the check, roll one more d6 (To hit, Saving Throw, Skill checks). This value (4) cannot be subtracted from the points given by Feats or magic items.

\begin{center}\textbf{\textit{Corollary}}\end{center}\index{Golden Rules Corollary}

These notes count towards the initial roll of 3d6.

\item \textbf{Rolling 6 3 times is a success}, both on Proficiency Checks, Saving Throws and Attack Rolls regardless of the final result.\index{Rolling 6 three times}

\item \textbf{Rolling 3 times 1 is a failure}, both on Proficiency Checks, Saving Throws and Attack Rolls regardless of the final result. \index{Roll 1 three times}

\item \textbf{Rolling 6 twice} is an omen of good fortune (Critical Success) if you succeed in the \index{Critical Success} check

\item\textbf{Rolling 1 twice or once 1 and twice 2} is a harbinger of misfortune (Critical Failure) if you fail the \index{Critical Failure} check

\end{itemize}

Use \textbf{Golden Rules} to your advantage! Dare, try, risk when the situation doesn't allow for other solutions!

\begin{changemargin}{0.3cm}{0.3cm}\begin{tcolorbox}[title = It's not just the tab!]{
Don't necessarily look for the solution in the sheet. Use your ability to imagine, to resolve, to intuit to go out and resolve situations. The sheet represents only a small part of what your character can do.
}\end{tcolorbox}\end{changemargin}


\subsection{Pass or Fail the Check}\index{Pass of fail the check}\label{superareofallirelaprova}\index{Critical Success on check}

The check is passed when the 3d6 are rolled and the sum of the involved Skill and the Characteristic as well as the various modifier is equal or greater than the DC established by the Arbiter.

If the result is lower than the difficulty, the check is failed.

Whenever the check is \textbf{passed with a critical score}, i.e. the check is successful and at least two 6s have been rolled, the Arbiter can decide to give more information, grant a bonus to subsequent actions (+1).. any what can value how easily the check was passed.\index{Pass check with Critical Success}

Conversely, if the check fails \textbf{and two 1s were rolled or one 1 and two 2s} the Arbiter could describe how miserably the check failed and how the bad result affects the Action and the subsequent ones.

Think about how competent a character is in order to avoid any checks with a foregone conclusion.

\begin{changemargin}{0.3cm}{0.3cm}\begin{narrator}
If the test can be repeated until eventual success without problems or interruptions then do not have the test done, describe the attempts, the difficulties encountered and declare success.\\

This award clearly clashes with the suggestion to "reward" the HOW to do and not the WHAT to do. The Arbiter, also on the basis of the group he has to manage, will have to juggle well and seek a balance.\end{narrator}\end{changemargin}

\begin{changemargin}{0.3cm}{0.3cm}\begin{narrator}
I advise everyone to read the excellent article by Lorenzo Bertini \href{https://dietroschermo.wordpress.com/2022/03/10/elogio-del-10-e-del-20}{Elogio del 10 e del 20 } for a critical and intelligent examination of the success and failure of the tests.
\end{narrator}\end{changemargin}


\subsection{Awareness}\index{Awareness}\label{consapevolezza2}

Awareness is one of those skills that comes into play very often.

Let the players' questions and reasoning uncover the clues. An Awareness check can be made when there's something not immediately noticeable, something that must be actively searched for as it is not readily perceptible or intuitive, or something the players want to find but haven't asked the right question about.

\begin{changemargin}{0.3cm}{0.3cm}\begin{narrator}
Don't let checks dictate your game. Encourage the players to role-play, to interact, and based on their actions and words, determine if the check has been passed or not.

If they state, "I convince the guard to let us through," you may perform an Intimidate (or Diplomacy) check. If instead, they engage in a persuasive conversation, you can consider the check as having a positive outcome (or negative if they fail to persuade). Emphasize the HOW they did it, rather than just the WHAT they did.
\end{narrator}\end{changemargin}

\subsection{The Evidence}\index{Opposing Evidence}\label{opposing Evidence}\index{Comparing Evidence}


\subsubsection{Proficiency Tests pitted against an opponent}\index{Tests pitted against an opponent}

There are situations in which the character must perform an Opposite Test against an opponent, for example moving silently behind a guard, stealing from the merchant's pockets, intimidating the orc to give him directions, pushing an opponent...

In this case the character performs the indicated test whose \textbf{difficulty (DC) is equal to 10} + the Characteristic score + Proficiency or Saving Throw + contingent modifiers (bonuses/penalties). 

Whoever obtains the highest value wins, in case of a tie, the winner is the one with the highest value in the Competence, then in the Characteristic and finally the possible \emph{opponent}. \index{DC Static in contrasting tests}\index{Contrasting Tests}\\

\textbf{Some examples of Opposite Tests}

- Deceiving someone: Deceiving Vs Perceiving Emotions

- Dressing up to look like someone else: Entertainment Vs Awareness

- Creating a fake map: Falsifying vs Evaluating

- Hiding / Moving Silently: Competence Vs Awareness, as long as it is not seen

- Intimidate: Intimidate vs. Will save (with Charisma modifier)

- Stealing: Fairy Hands vs. Awareness, or Fairy Hands if possessed 

- Untying yourself from ropes: Using Ropes Vs Escape Artist

- Arm wrestling: Fortitude save (with Strength modifier)

\subsubsection{Opposing Characteristic Tests}

Whenever the \textbf{Test} or \textbf{Opposed Test} concerns a \textbf{Characteristic} and not a Competence, make the test (3d6+) by adding the most suitable Saving Throw to the most suitable Characteristic.

\medskip

\textbf{Table: Opposed Trials and Modifiers}\index{Table of Opposed Trials and Modifiers}\label{Table of Opposed Trials and Modifiers}

\begin{tabularx}{0.45\textwidth}{Xl}
	\toprule
	\textbf{Contrasted Test} & \textbf{TS} \\
	Strength & Temper \\
	Dexterity & Reflexes \\
	Constitution & Temper \\
	Intelligence, Wisdom, Charisma & Will \\
\end{tabularx}

\medskip

It is possible that Contrasting Tests may be requested with different modifiers indicated. Those shown in the table above are typical usage examples. It is possible to make a Strength Test by adding the Fortitude score to measure extreme resistance to fatigue.

\subsubsection{Unopposed, static tests}

If the Test is opposed to a \textit{static opponent}, i.e. not a creature equipped with Characteristics and Skills, but a lock, something to push... then the test is performed by comparing 3d6 + the Characteristic involved + the Active Skill ( TS) most suitable against the difficulty (\textbf{DC}) established by the Narrator.

\medskip

\begin{center}
	\includegraphics[width=0.6\linewidth]{immagini/Foster_Bible_Pictures.png}
	
	\emph{Bible Pictures and What They Teach Us}
\end{center}



\subsection{Pros and Cons} \index{Bonus}\index{Pro}\index{Penalty}\index{Disavantage}\label{vantaggi}

\begin{changemargin}{0.3cm}{0.3cm}\begin{emphasis}{Audentes fortuna iuvat ("Fortune favors the bold", Virgil)}\end{emphasis}\end{changemargin}

Depending on the circumstances there may be advantages or disadvantages in the test.

The modifier in \textbf{dynamic tests}\index{Dynamic tests} is to be used when the test is done by rolling 3d6, in this case you can add bonuses or penalties (-1, +2...) or even roll dice more or less (+1d6, -2d6), until no dice are rolled (with a 3d6 penalty)!.

If the accumulated penalty brings the dice of the check below zero, only the value of the Competence and Characteristic is counted.

We mean \textbf{fixed value tests} \index{Fixed value tests} when the value does not depend on the dice roll (e.g. Defense), in this case the score is raised/lowered by the indicated number.

Try to always stay between these advantage and disadvantage values, otherwise you can say that the check is directly successful or failed.

The player can still request to carry out the check even if the result is certain.

\medskip

\textbf{Table: Bonuses and penalty, Advantages and Disadvantages}:\index{Table Bonus and penalty, Advantages and Disavantages}

\medskip

\begin{tabular}{lll}
\multirow{2}*{\textbf{Advantage / Disadvantage}} & \multicolumn{2}{c}{\textbf{Evidence}}\\
\cmidrule(lr){2-3} & \textbf{Dynamics} & \textbf{Fixed} \\
\toprule
Light Bonus & +1& +1\\
Normal Bonus & +2 & +2\\
Strong Bonus & +1d6 & +4\\
Very strong bonus & +2d6 & +8\\
Slight Disadvantage & -1 & -1\\
Normal disadvantage & -2 & -2\\
Strong Disadvantage & -1d6 & -4\\
Very Strong Disadvantage & -2d6 & -8\\
\end{tabular}

\begin{changemargin}{0.3cm}{0.3cm}\begin{narrator}
The bonuses and penalties on the 3d6 roll have more \textit{effect} than on the d20 check. Try to always stay within the range of $\pm2$ and apply greater bonuses or penalty only in particular situations of effective and strong advantage or disadvantage.
\end{narrator}\end{changemargin}

\subsubsection{Time Factor}\index{Time Factor}\label{fattoretempo}

\textbf{If a character is not in difficulty or pressure}\index{No problem waiting}\index{Take 10} in carrying out the check he can take the 10 (+ skills + ability..), i.e. do not roll the dice and consider that he rolled 10 with the dice. The action takes 10 rounds. \label{prendere10}

\textbf{If the character has no pressing time limit}, ie he can dedicate at least 10 minutes to work on it (60 rounds) he can consider taking 15. Or as if he had made the check and rolled 15 with the 3d6. \label{prendere15}

\textbf{If time becomes a factor to be ignored}, i.e. the character has at least 1 hour to think and work and has no penalty or risk consider having rolled 18 (but there is no dice explosion or critical success even if the total is 18).\\ \label{prendere18}

If you want to take these values, ask the Narrator, he will tell you if based on the situation, urgency, danger of what surrounds you you can take the score. Breaking open a door in a dungeon asking for 10 requires extreme cold blood and recklessness. Take 10/14/18 should not be permitted on Knowledge check's.

\subsubsection{Help Another}\label{aiutarealtro}

\index{Helping another} You can help a friend in a trial by giving him support and suggestions. The helper must make the same check with a +1d6 bonus, if he succeeds he gets no effect but gives a +1 to the companion's check. If he critically succeeds (successful check and at least two 6s rolled) then the bonus is +2.

Multiple characters can help the same friend; bonuses of this type can be accumulated up to a bonus equal to a quarter of the difficulty to beat (eg +6 in the case of difficulty 25).

In the case of checks on Skills, the helper must have assigned at least one point to that Skill.

The Arbiter will evaluate the possibility of more than one character providing help considering spaces, ways and times (it is not easy to help someone thread through the eye of a needle).


\subsection{Check made by the Arbiter}\label{provefattedalnarratore}

Instead of doing checks for the players, be descriptive but don't mention the need for a check. If it's necessary to perform a check secretly, don't roll the dice but add 10 to the subject's characteristic and skill score or saving throw value, and compare the result with the check's difficulty.

\subsection{To roll or not to roll dice}\label{tirarenontiraredadi}

Don't have dice rolled for tests that have no chance of failing, for tests that don't have or generate \textbf{problems} if they fail or can be retried without problems. Have the dice rolled whenever the test results in a \textbf{spectacular} result, \textbf{failure} or triggers further scenes. Make the player enjoy success or fear critical failure.

\subsubsection{Optional - Partial Success}\index{Partial Success}\index{Risky check}\index{Optional - Partial Success}\hypertarget{successoparziale}{}\label{successoparziale}

The Arbiter can also decide to score a failed check as a partial success.

If the check fails by 1 it can be considered successful even if with a slight problem, if it fails by 2 it carries a serious problem if it fails by 4 it succeeds with a critical problem, if it fails by more than 4 the check is not succeeded. Applied to skills such as Knowledge, one can decide to provide information that is not complete or partly true and false, or even if it is a question of opening a lock, the lock pick in the lock could be broken!

It may also be the player who requests a \textbf{"Check with Risk}" in situations of particular tension and urgency in which the final result is more important than the risk taken. This request must be made before rolling the dice.


\subsection{Group Check}\label{provedigruppo}\hypertarget{provedigruppo}{}\index{Group Check}

There are situations in which the group has to make a proficiency check but the result must be unique, in this case if at least half of the group succeeds in the check it is successful for all.

\subsection{Examples of Competence Checks}\label{esempiprovecompetenza}\hypertarget{esempiprovecompetenze}{}\index{Example of Competence Checks}

\textbf{Atypical checks}\index{Atypical Checks}. The player is invited to find uses, solutions, approaches that go beyond the most obvious checks. Be creative and describe to the Arbiter what wonderful action you want to do and what results you hope to achieve! He will decide based on your description of the action what to try and how difficult it will be.

\medskip

To \textbf{recognize a magic object}\index{Recognize magic object} and its abilities, a check of \textbf{Arcana} at difficulty 20 is necessary to have general indications on the powers and areas of use, only with a result of at least 30 in the check, you can learn its details, magical bonuses and charges. \textbf{10 minutes}. With Arcana score 6 costs 5 minutes, with 12 it costs 1 minute, with Arcana 18 costs 1 Round.


\medskip

\textbf{Recognize a spell}\index{Recognize a spell} while it is being cast is a check of \textbf{Arcana} at DC equal to 11 + spell level. It costs one \textbf{Reaction}. If done in conjunction with casting a Counterspell, it costs no Reaction.

\medskip

To \textbf{recognize a monster}, a particular creature you make a Knowledge check. Check the chapter \hyperlink{riconoscereimostri}{Recognizing monsters} in the Monsterarium (page \pageref{riconoscereimostri})


\medskip

\textbf{Acrobatic}\index{Acrobatics} \textit{Penalties due to Armour}

A successful DC 15 Acrobatics check allows the character to halve the damage when falling from less than 10 meters (\textbf{Reaction}).

\medskip

\textbf{Climbing} \index{Climbing} \textit{Penalty due to Armour.}\label{Climbing}\label{Climb}

\medskip

Using a rope\index{Climbing with a rope}, climbing or clambering is equivalent to moving in \textbf{doubly difficult terrain}.

In case of failure of the check, the Action is consumed without moving. If it fails with a critical failure you lose your grip and fall, you can make a Reflex save at the same difficulty to grab onto something, if you also fail the Saving Throw you fall all the way. The indicated difficulties add up.\\

\begin{tabularx}{0.45\textwidth}{Xl}
\textbf{Example of Surface} & \textbf{DC}\\
\toprule
Movement only halved & -2d6\\
Slippery surface&+5\\
Raw wall with handholds, protruding bricks&+10\\
A tree, a knotless rope&+15\\
A smooth wall with handholds &+20\\
A wall with very few holds&+25\\
A smooth natural wall without holds&+30\\
You can lean against 2 opposite walls&-8\\
You can lean against 2 corner walls&-4\\
You can use a string&-8\\
\end{tabularx}

\medskip

To \textbf{identify a potion or natural poison}\index{Identify poison} \index{Identify Potion} a check of \textbf{Herbalism} is required at DC 12 + rarity factor of the plant, or the Saving Throw it grants in case of Poisons.

It costs 1 Action for every 10 of DC. With 6 in Herbalism the time is 1 Action every 15 of DC, with 12 points it is 1 Action every 20 DC. If you fail the check with a critical failure you have come into contact/ingested part of the potion and in case of poisons you suffer its effects.

\medskip

\textbf{Intimidate}\index{Intimidate}. The character uses \textbf{2 Actions} and performs an Opposed Test made with a Will save with a bonus given by Charisma.
If the saving throw fails, the opponent has -1 to attack rolls and -1 to defense until the end of his next round against the one who intimidated him. The opponent must have Intelligence equal to or greater than -3, if larger he gets +2 or -2 if smaller, for each size difference.

If the person attempting the Intimidate check fails critically, he suffers the same penalties as if he had been intimidated.

\medskip

\textbf{Tame an animal} is a check of \textbf{Handle Animals} at the animal's DC 12+2*CR. 1 minute every 3 of DC. With 6 points the time is 1 minute every 6 DCs, with 12 it is 1 minute every 10 DCs. The creature must have Intelligence -3 or higher.

\medskip

\textbf{Stealth} \index{Hide} \emph{Armor penalty.}\index{Move silently}

Stealth gathers the abilities to move silently, hide in shadows, go unseen, and all those actions that require not being seen or heard.

Trying to move silently costs no Actions, it is included in the Move Action used to move. However, the terrain is treated as difficult and if it already was, it becomes doubly difficult. Moving at full speed while trying not to make any noise imposes a 2d6 penalty on your Stealth check.

Although trying to hide is a seemingly simple activity, only those trained in Stealth have a greater chance of not being noticed.

Using \textbf{1 Action} you can try to hide from your opponents' sight. It is not possible to hide if the environment does not allow it, even though your test may be high you cannot hide if there is not something that can hide or conceal you. To hide behind a creature it must be at least 3 sizes larger than you (otherwise the creature only provides cover).

\medskip

\textbf{Swim}\index{Swim} \textit{Penalty due to Armour}

DC 10 in calm water, DC 15 in rough water, DC 20 in very rough water, DC 25 stormy. Check required to float or swim. Swimming in water is considered \textbf{difficult terrain}.

\medskip

Any check on \textbf{Profession} is made with 3d6+Wisdom+half the level.

\medskip

\textbf{First Aid}\hypertarget{prontosoccorso}{}\label{prontosoccorso}\index{First Aid}. A successful check (DC 15) restores 1d4 Hit Points \textbf{after a fight} or grants a +2 to a Fortitude save to resist a poison. To be done within 1 Turn of the end of the fight. Cost \textbf{2 minutes}. With score 6 it costs 1 minute. A score of 12 costs 3 rounds, a score of 18 costs 1 round.

A successful check (base DC 12) reduces damage from \hyperlink{sanguinamento}{\textbf{Bleeding}} by 1. For each Bleed value above 1 the difficulty increases by 2. Cost \textbf{2 Actions}. A 1 minute treatment guarantees 1 success, no trial. Each critically successful check reduces the bleeding by an additional point.

A successful check (DC base 13) to \textbf{take care for 8 hours} of a patient restores double Hit Points ((2*WP+Constitution+MP)*2 with a minimum of 4) and grants a new Fortitude save to vanquish natural diseases or poisons already in progress.
If carried out during the hours of rest, the caregiver will be Fatigued.

\medskip

\textbf{Jump}\index{Table: Jump} \textit{Penalty due to Armour.} \textbf{1 Action}\\
%\begin{tabular}{lc|lc}
%\textbf{Salto in Lungo} & \textbf{DC}&\textbf{Salto in Alto} & \textbf{DC}\\
%\multicolumn{2}{c}{Lunghezza} &\multicolumn{2}{c}{Altezza} \\
%\toprule
%1.5 m  & 5 & 0.02 m & 4\\
%3 m  & 10 &0.5 m  & 8\\
%5 m  & 15 & 1 m  & 12\\
%7 m  & 20 & 1.5 m  & 16\\
%+1,5 m & +5 &+0.5 m & +4\\
%\end{tabular}

The \textbf{long jump distance} is equal to 30cm per check result, rounded to the nearest integer. Eg if in the jump check I do 11, the jump will be 30cm*11=330cm=3 meters long, with 16 in the check it is 30cm*16=480cm=5m.

The \textbf{distance jumped up} is equal to 10cm per result achieved in the check.

In a \textbf{long jump} the highest point of the jump is equal to 1/3 of the length jumped. If you do a 3m long jump in mid-jump you are 1m up.

Descending from less than 1m does not use Actions. If you don't have at least 3 meters of run-up, you jump half. In long you jump to the maximum of your movement and up half.\\

Damage (page \pageref{cadute}): 1d6X fall height (every 3 meters). Acrobatics DC 15 1/2 damage (for falls within 9m).

\medskip

\textbf{Survival}\index{Survival}

\smallskip

\textbf{Chasing a creature}:

\begin{tabular}{ll}
Base Difficulty & DC 15\\
\toprule
If the ground is very soft& DC +5\\
If the ground is soft& DC +10\\
If the ground is stable& DC +15\\
If the ground is hard& DC +20\\
Every 3 creatures chased& DC -1\\
Depending on size& DC +-4\\
Every past 24 hours / Low Visibility&DC +4\\
Every hour of rain&DC +4\\
%Visibilità scarsa&DC +2\\
Try to hide the tracks&DC +5\\
\end{tabular}\\

Any modifiers given by concealment apply to the check.


Survival can be used instead of \textbf{Disable Device} with a -2d6 to disable traps or locks 1 Action per DC.

For every four points scored on the Survival check beyond 13 the character is able to \textbf{get food} for himself and one other person provided he is in an environment capable of sustaining life.

\medskip

The \textbf{Appraise}\index{Appraise} check is based on the rarity of the item, DC 12 + 2 common, 4 uncommon, 8 rare, 12 very rare, 16 legendary. \textbf{3 Actions}. With a score of 6 it costs 2 Actions, with a 12 it costs 1 Action.

\subsection{Languages}\index{Languages}\hypertarget{linguaggi}{}\label{linguaggi}

In Yeru, each culture has its own unique spoken and written language. Any character with an Intelligence score of -2 or higher can speak the language of their culture, and if their score is 0 or higher, they can write it as well. For each score greater than or equal to 2, the character can speak and write another language, which will be selected during character creation.

For each point in the Language Knowledge skill, the character can speak and write another language.

It's possible for a member of a race to have a first language that's different from the language of their race if their background justifies it (e.g., a Dwarf raised in a Goblin tribe).

Some languages* cannot be spoken except by creatures belonging to that specific species or cultural group.

Make use of the narrative richness that different languages can bring. Adventures in an environment where no one understands each other can lead to very interesting and unique situations.

\end{multicols}

\textbf{Table: List of Languages}\index{Table List languages}

\medskip

{\small \begin{tabular}{lll|lll}
\textbf{Cultural field}& \textbf{Spoken} & \textbf{Written}&\textbf{Cultural field}& \textbf{Spoken} & \textbf{Written}\\
\toprule
Human & Common& Common& Dwarven& Dwarven& Dwarven\\Elven& Elven & Elven & Gnomish& Gnomish & Gnomish \\Gnoll & Gnoll & Goblinoid & Giants& Giant & Giant\\
Ogre & Orc & Orc & Sentient Sea Creatures & Aquan & Elven\\
Sentient Birds& Ictum & Elven& Woodland Dwellers*& Silvanus& Silvanus\\
Druidic* & Druidic & -& Goblinoid& Goblinoid& Goblinoid\\
Celestial& Celestial & Celestial& Infernal & Infernal & Infernal\\
Abyssal & Abyssal& Abyssal& Dragons& Draconic & Draconic\\
Fire Elementals* & Ignan & - & Earth Elementals*& Tremun &-\\
Water Elementals* & Aquan & - & Air Elementals*& Ictum &-\\
Undead & Exspiram & - & Undercommon & Depth creat. & Depth creat.\\
Sign Language*& of Signs & Signs&&&\\
\end{tabular}}
\medskip

\textbf{Telepathy}\index{Telepathy} is a means of speaking with any creature that has speech and Intelligence greater than -3. There is no language constraint, telepathy acts as a universal translator.

\medskip

\begin{multicols}{2}


\begin{changemargin}{0.3cm}{0.3cm}\begin{tcolorbox}[title = Checks Checks and Checks!]
To be cynical, an RPG is all about checks, whether it's making a jump, hitting someone, avoiding a trap or a spell... You need to be cunning and clever. Checks can often be dodged or used to your advantage. Play with wit, use your imagination, be creative! The Arbiter will be pleased and you will be satisfied!!!
\end{tcolorbox}\end{changemargin}

\vfill

\begin{center}
\includegraphics[width=0.8\linewidth]{immagini/Pieter_Bruegel_the_Elder-The_Tower_of_Babel.png}

\textit{The Tower of Babel, Pieter Bruegel the Elder.\\
"... Therefore it was called Babel, because there the Lord confused the language of all the earth and from there the Lord scattered them over all the earth." (Genesis 11, 1-9.)}
\end{center}


\begin{changemargin}{0.3cm}{0.3cm}\begin{narrator}
Checks play a crucial role in the game and the way you handle them, like combat, shapes the type of game and adventure you have. It's important to listen to the player and understand their excitement, try to grasp their ultimate goals. 

An engaged and invested player will spread their enthusiasm to the rest of the group! Pay close attention to their suggestions, even if they seem unrealistic or crazy. You can always warn of potential dangers, but try not to dampen their excitement. 

If you can't find a suitable rule for a situation, use common sense, compare it to similar actions, get creative with describing the events, and be dramatic when needed. The positive spirit of the group will surely benefit.

When players tell you to make an Awareness check or say "I'm making a jump check", go along with their intentions but also try to involve them even more.

Remember, there isn't a rule for everything! Fun and common sense should always be prioritized!

\end{narrator}\end{changemargin}


\end{multicols}

\pagebreak

\section{Social Combat}\index{Social Combat}

\begin{changemargin}{0.3cm}{0.3cm}\begin{emphasis}{

In order to formulate dialectic clearly, it is necessary to consider it, regardless of objective truth (which is the object of logic), simply as the art of obtaining reason, which will certainly be all the easier if one is objectively right. (Arthur Schopenhauer)

}\end{emphasis}\end{changemargin}\medskip


\begin{multicols}{2}

By \textit{Social Combat}, we mean the attempt by characters to convince, force, or trick non-player characters (NPCs) or creatures under the control of the game master (referred to as the "Arbiter") to do or say things they don't want to do. This can occur when players try to bribe a guard, obtain information through diplomacy or intimidation, negotiate higher pay, deceive a merchant, or in any situation where the confrontation or clash is not resolved through weapons, but through words.

Social combat can encompass a variety of situations, but what ties them all together is the method used to achieve the desired outcome, which is to convince the opponent. The Arbiter can either evaluate the outcome based on the players' words or use rules to determine the winner of the confrontation, just like a fight. The choice of approach depends on the Arbiter's preference and experience with the system and role-playing in general.

When players use more or less aggressive methods, the opponent will resist accordingly. The player makes an Intimidate, Diplomacy, or Deceive check, and the opponent resists with a Will save and a Strength or Charisma bonus. The difficulty class (DC) for the players to beat can be set at 10 + Will + Strength (for Intimidate checks) or Charisma (for Diplomacy or Deceive checks).

Based on the NPC's level, the Arbiter will establish how many successes are needed to convince him. In general, one success plus one success for every 1 NPC levels  is required.

If successful, the \textit{combat} will be won and the information or what has been requested will be obtained. In case of critical success, i.e. beyond passing the test, at least two 6s have been rolled, two successes will be counted.

If the check fails, it can be attempted again with a -1 penalty, unless the consequences of the failure lead to a subsequent scene. If the check is a critical failure (i.e., the player rolls two 1s or one 1 and two 2s in addition to failure), not only does the check fail, but further attempts are not possible, and the opponent becomes less friendly. The Arbiter will determine the outcome based on the original request and scene.

If the check was for Intimidation, the target is likely to become hostile. If the check was for Deception, the target may feel deceived and lie or say nothing. If the check was for Diplomacy, the target may politely decline or remain silent.

The Arbiter must use the results of these checks, positive or negative, to advance the scene and add depth to the adventure. Even if the informant is unaware, information obtained through these checks may be incomplete or partial. This information will help both the players and the Arbiter better conduct the game. The Arbiter should not view giving information as a problem, as the players have earned it through their actions.


\end{multicols}

\vfill

\begin{center}
\includegraphics[width=0.55\linewidth]{immagini/Greuter_Socrates.png}

\textit{Johann Friedrich Greuter: Socrates and His Students, 17th century.}
\end{center}

\pagebreak

\section{Armed Combat}\index{Armed Combat}

\begin{changemargin}{0.3cm}{0.3cm}\begin{emphasis}{
Si vis pacem, para bellum ("If you want peace, prepare for war", anonimous)
\medskip

It doesn't matter how you fall, but if and how you get back up (anonymous)

\medskip
I'm not a hero. No and I never will be. I'm just a bad guy who gets paid to beat up worse guys than him. (Deadpool)

\medskip

An eye for an eye ... and the world goes blind (Mahatma Gandhi, NdA Traits of him abhorred violence!)}\end{emphasis}\end{changemargin} \medskip



\begin{multicols}{2}

\lettrine[lines=2, lhang=0.33, loversize=0.25, findent=1.5em]{T}{he} combat is one of the main phases of an adventure and is when the brave or fearful show off their mastery with weapons or spells.

\bigskip

The fight is divided into 2 phases:\index{Fighting}
\begin{itemize}[leftmargin=*]
\item verification of the initiative
\item action resolution (move, attack, miscellaneous action..)
\end{itemize}

\begin{center}
\includegraphics[width=0.8\linewidth]{immagini/Achildbookofwarriors.png}

\textit{A child's book of warriors (1907), William Canton}
\end{center}

\subsection{The Initiative}\index{Initiative}\label{iniziativa}

Initiative is a check (3d6) of Dexterity or Intelligence and any inherent Feats you may have.

The player chooses the Characteristic that he prefers. If Dexterity is chosen, reflexes will determine the character's reaction, while Intelligence will guide the ability to grasp the opponent's tactics and anticipate them.

Whoever has the highest initiative between players and enemies starts first and then the others act in descending order, declaring Actions and executing them. In the event of an equal number of Initiative scores, the player with the highest Characteristic score acts first, otherwise the clash will take place simultaneously. The initiative is valid for the entire battle and is withdrawn when the opponent changes.

\begin{changemargin}{0.3cm}{0.3cm}\begin{narrator} %box narratore
Try to make the fight flow naturally. Do not interrupt the flow of actions, but by describing the effects involve the players (and enemies) in the following actions. I recommend reading the article \href{https://theangrygm.com/manage-combat-like-a-dolphin/}{How to Manage Combat Like a \textit{xxx} Dolphin} to understand the method in detail.
\end{narrator}\end{changemargin}


\textbf{The Golden Rules also apply to the Initiative check.}

\subsubsection{Action Resolution}\index{Action Resolution}\label{risoluzionedelleazioni}

\begin{changemargin}{0.3cm}{0.3cm}\begin{emphasis}{
...the past is the prologue and the future is in your hands and mine. (Anthony, The Tempest, Shakespeare)}
\end{emphasis}\end{changemargin}\medskip


From fastest to slowest is the resolution of the Actions.

The Arbiter will ask the fastest player, the one with the highest initiative, to declare his Actions and act, he will then continue to ask and make the other players and enemies act.

In this way the choice of the action takes place when it is the player's round who will be able to act also on the basis of the actions and resolutions that have already taken place.


\begin{center}
\includegraphics[width=0.9\linewidth]{immagini/Arthur-Pyle_Two_Knights.png}
\textit{Howard Pyle, from the 1903 edition of The Story of King Arthur and His Knights}
\end{center}


\subsubsection{Time (Rounds, Minutes and Turns)}\index{Round}\label{iltempo}

\begin{changemargin}{0.3cm}{0.3cm}\begin{emphasis}{
"Hesitation is the death of advantage" (Magic, V.E. Schwab)}\end{emphasis}\end{changemargin}\medskip


A \textbf{round} lasts about 10 seconds, it's enough time to act, run, talk... fight. A Minute is therefore 6 rounds, and a Turn lasts 10 Minutes (or 60 rounds).

The rounds are used in combat scenes or where the tension must remain constantly high and each Action corresponds to an evolution of the situation.

%\begin{center}
%\includegraphics[width=0.75\linewidth]{immagini/hjford-fight.png}
%
%\textit{\\Henry Justice Ford, Fairy book - Fairytale illustration}
%\end{center}

\subsubsection{Time to reactivate Objects and Abilities}\index{Time to reactivate Objects and Abilities}\label{temporiattivazioneoggetti}

Unless otherwise specified, an item or Feat that permits a certain number of uses per day \textit{"e.g. once a day"} "recharges" at dawn following use.

\subsection{Actions in Round}\index{Actions in Round}\index{Actions}\label{azioninelround}

A character can perform 3 Actions per round, 1 Immediate Action, 1 Reaction Action.

If in the round of the initiative roll the character in the test itself rolled a \textbf{critical success}, twice 6, he will be able to use one more Reaction or Immediate Action, if he rolled at least two critical successes his great reactivity allows him to perform one additional Action.\index{Critical in the Initiative}. However, if he rolls a critical failure, the extreme slowness will prevent him from using the Reaction or Immediate Action, if he rolls at least three 1 he will take one less action.


These Actions can be performed in any order you prefer.

The table below lists the main actions that a character can take, they are just guidelines. The combat chapter includes other actions and their respective costs in terms of actions.

\textbf{An Action cannot be interrupted by another Action, but can be followed by a Reaction Action or an Immediate Action}. \index{Stop Actions}\index{Actions, Stop}

If a character wants to make more attacks while moving around the battlefield, he can, for example, use an Action to perform an attack, use a Move Action to move up to all of his available movement, and use a final attack Action to perform a last single attack, this second attack counts as a multiple attack with the related penalties.

It is possible to \textbf{delay} one or more Actions\index{Delay Actions} to wait for the scenes to unfold. The character who delays one of his Actions acts first among the subjects who act in that initiative value, in subsequent rounds he will continue to act in the new initiative order.

A player who declares that he is waiting for a certain situation to be able to act is equivalent to carrying out one or more \textbf{Prepared Actions}\index{Prepared Actions}. In this case the character (or enemy) acts after the triggering Action with his Actions but remains in his initiative order at the end of the round.

If a character has already performed all their actions, they can only act outside of their initiative through a reaction, if available. The reaction action always comes after the triggering action.

\bigskip

\textbf{Table: Actions per Round}\index{Table Actions per Round}


\begin{tabularx}{0.45\textwidth}{Xc}
\textbf{What to do} & \textbf{Actions}\\
\toprule
Perform an attack & 1\\
Perform two attacks & 2\\
Perform more than two attacks & 3\\
Cast a Spell* & 2\\
Perform a Move Action* & 1\\
Shot & 1\\
Standing up from prone & 2\\
Help someone & 2\\
Exchange dialogue with someone* & 3\\
\small{Exchange a few words with someone*} & 0\\
{\small Searching for something in the backpack} & 2\\
Use a hand held item & 1\\
{\small Take samothing from belt or ready} & 1\\
Drink a potion held at the belt & 1\\
Drawing/Sheathing Weapon & 1\\
Using a Magic Item & 2\\
Test a skill* & 1\\
{\small Breaking down a door with shoulder or kicks}&1\\
Forcing down a door with a crowbar &2\\
Hide & 1\\
Focus on a Spell & 1\\
Mount or dismount & 1\\
Action \textbf{I}mmediate - \textbf{R}eaction & I - R\\
Drinking a hand held potion & I\\
Throwing a hand held object & R\\
Fall prone & R\\
Recognize a Spell & R\\
\end{tabularx}

\medskip

Attack includes both the use of melee weapons and the use of thrown or projected weapons such as bows, crossbows, or throwing knives. In the case of thrown/projected weapons, each throw counts as one attack.

If the character performs an Attack Action and Casts a spell, he is considered Distracted, i.e. he must perform a Magic Test to cast the spell.

\textbf{Move Action*}: A Move Action is an Action dedicated to moving. You can move up to your full movement (9 meters for humans, 6 meters for dwarves…).

\textbf{Casting a Spell*}: Usually 2 Actions required. The spell description indicates the number of Actions required. In the Magic chapter there are  \hyperlink{piumagieround}{rules} (page \pageref{piumagieround}) for casting more spells per round.

\textbf{Having a conversation with someone*}: A conversation can be seconds or even minutes long. The Arbiter will judge how long this lasts.

\textbf{Exchange a few jokes with someone*}: As long as it's really a few jokes or a look doesn't consume Actions, if this becomes more articulate then use Actions. The goal is not to interrupt the flow of Actions with heavy dialogue but still allow interaction between characters.

\textbf{Execute check on a skill*}: if they yield a fraction of the round they cost 1 Action, otherwise 2 or more. Check in \hyperlink{esempiprovecompetenze}{Skill check example} the reported costs.

An Action "\textbf{Reaction (R)}" \index{Action Reaction}can be performed freely even outside its own round. This Action is usually due to special Feat or situations. Unless otherwise indicated, a Reaction Action occurs immediately after the cause that triggers it.

An Action "\textbf{Immediate (I)}" \index{Immediate Actions}may be performed freely in your round, before or after your Action. An Immediate Action is usually granted by particular Feat.

It is possible, if not specifically described in the Feats, to perform only one Immediate Action and one Reaction Action per round.

\medskip

This \textbf{list is not exhaustive}, take it as a guideline for weighing players' decisions and actions. One Action is about 3 seconds long activity.

The \textbf{order} in which Actions are performed is not important except by logical and physical correlation. The Move Action can be sandwiched between other Actions (move, attack/spells/other action, move).

A character could attack, move and attack again, this second attack would have the penalties described in multiple attacks.
\smallskip


\begin{center}
\includegraphics[width=0.8\linewidth]{immagini/Perseus_Fighting_Phineus_and_his_Companions.png}

\textit{Luca Giordano: Perseus turning Phineas and his Followers to Stone}
\end{center}

%\bigskip
%\begin{changemargin}{0.3cm}{0.3cm}\begin{narrator}Una creatura che ha una distanza di mischia (portata) superiore all'avversario si considera che abbia un bonus di \textbf{+2 al Tiro per Colpire} finché l'avversario non lo raggiunge in mischia.
%
%Questo bonus non si applica con le armi da lancio (archi, balestre, pugnali, asce... per quelle armi che usano la \textit{gittata}). \end{narrator}\end{changemargin}

\subsection{Optional - Initiative Variant}\index{Optional - Initiative Variant}\hypertarget{varianteiniziativa}{}\label{varianteiniziativa}

This variant of the initiative aims to stimulate the diversification of actions based on the situations that are faced from time to time.

Initiative is a value that is calculated round by round based on the Actions performed.

At the beginning of the round there is a declaration of the Actions that you want to undertake, starting from those who have the lowest Dexterity or Intelligence.

Up to 10 Action Points (AP) can be used per Round.

\textbf{Whoever uses the fewest Action Points acts first}

If AP used is equal, the person with the highest Dexterity or Intelligence acts first, if the opponent is equal, in the case of teammates we reach an agreement.


\end{multicols}


\begin{tabularx}{0.95\textwidth}{Xl|Xl}
	\textbf{What is done} & \textbf{PA}&\textbf{What is done} & \textbf{PA}\\
	\toprule
	Attacking with Light Weapon / Bare Hands & 3 &Attacking with Medium / Missile Weapon & 4\\
	Attack with Large Weapon & 5 & Use Magic Item & 6\\
	Move within 2m-3m (Move 6m-9m) & 1 &Sprint within 4m-6m (Move 6m-9m) & 1\\
	Rise from prone & 4& Mount or dismount & 4\\
	Take something in your backpack & 8& Use a hand-held object & 2\\
	Drinking a potion held on your belt & 4&Concentrating on a spell & 3\\
	Draw / Sheathe the weapon & 3& Take something from your belt or ready & 3\\
	Breaking down a door with your shoulder/kicking & 5&Forcing door with crowbar & 6\\
	Drinking a hand-held potion& 1&Throwing a hand-held object& R\\
	Falling prone& R&Recognize a Spell& R\\
\end{tabularx}

\begin{multicols}{2}

- Skills, providing Help, cost 3 Action Points per Action indicated.

- Exchanging dialogue can be free or cost AP depending on how much you talk.

- In case of Movement 6 m, with 1 AP you move up to 2m or 4m in Dash.

- The Speed ​​spell grants 4 more Action Points. These APs are not counted for the verification of the initiative of those who act first.

- The Slow spell subtracts 4 Action Points. These PAs are added to verify the initiative of who acts first. E.g. I use 4 Action Points, out of a maximum of 6 (10-4 for Slowness), to verify the initiative I used 4+4 points.

- Spells cost 3 AP per Component used (V,S,M), except when the casting time is longer than 1 round.


\bigskip

Moving around, deciding which weapon or spell to use determine not only when you act but the tactic you want to pursue. The collaboration between the characters becomes fundamental and does not slow down the flow of actions.

\end{multicols}

\subsection{Movement}\index{Movement}\label{movimento}

\begin{changemargin}{0.3cm}{0.3cm}\begin{emphasis}{"A slower mobile cannot be reached by a faster one; since what follows must arrive at the point which occupied what followed and where this is no more (when the latter arrives); thus the former always retains an advantage over the latter" (Zeno's Paradox)}
\end{emphasis}\end{changemargin}

\begin{multicols}{2}


The movement of a character is given by his size and race and by what he carries, by weights, dimensions but also spells and magical objects.

The Movement written in the character's race is an indication of how many meters per (Move) Action the character can make.

A creature or character might also decide to move faster than usual or run (Sprint Action).

The Sprint Action is a particular Move Action, it consists of running for that Action.
If you perform an Action of \textbf{Sprint} \index{Sprint}you double the meters traveled (2x9 meters for a human), for a dwarf (Movement 6m) it means traveling 12 meters in one Action.
It is also possible to perform several Sprint Actions (up to 3 in a round, i.e. run 6 times one's movement).

The character who takes a Sprint Action \index{Sprint Action} runs and has a 1d6 penalty on the attack roll and the Defence decreases by 4 for the entire round in which he has used the Sprint Action and is considered Distracted for casting spells .

You cannot move even 1 meter if you do not spend Move Actions.

These clarifications make sense and should be used when it comes to fighting and the displacement on the territory, map, is fundamental. During normal movements, while riding or walking free without danger, the normal management of movement is used.

In the case of diagonal movement\index{Diagonal movement}\index{Moving sideways} a distance of 1.5 meters per square is counted, in case of rounding on the last square it is done by defect, i.e. one goes back to the last got through.


\subsection{Distance}\index{Distance}\label{distanza}

\textbf{Touch distance} \index{Touch distance} \index{Touch}means a distance that allows touching the opponent, therefore no more than a meter for medium-sized creatures without long weapons or with reach. Touch distance is equal to melee distance when not using long weapons.

\textbf{Melee distance} \index{Melee distance} \index{Melee}means a distance that allows hand-to-hand combat (1 meter around the character, or 2 meters in the case of a long weapon). In monsters this distance is referred to as reach, for thrown weapons it is called range.

If not indicated in the opponent/monster, the melee/touch distance increases by 1 meter for each size above the medium.\index{Size and melee distance}

\begin{changemargin}{0.3cm}{0.3cm}\begin{tcolorbox}[title = Examples of Distance in Combat]
Eg for a spear-wielding creature, the melee distance is 2 meters because the weapon is long. For a gnome wielding a hammer, melee distance is 1 meter.
For a Hill Giant, melee distance (reach) is 3 meter. Attacks with bows, crossbows, or thrown weapons are referred to as Range.
\end{tcolorbox}\end{changemargin}

When we talk about "\textbf{square}" \index{Square} to indicate a distance or an influence we mean a map square of 1 meter x 1 meter.

\textbf{If moving into "difficult" terrain, half of the available movement is covered, so a human covers 4 meters per Move Action (each square crossed counts for two).}

At melee range, a medium-sized creature can have up to 8 medium-sized creatures.

\subsection*{Optional - Large and Small Creatures in Combat}\index{Optional - Large and Small Creatures in Combat}\label{creaturegrandipiccole}

The following table shows how many creatures that can surround a medium-sized creature depending on your size.

\medskip

\textbf{Table: Size, Scale of Creatures and number per square}\index{Table Size, Scale of Creatures and number per square}

\medskip

\begin{tabular}{lll}
\multirow{2}{*}\textbf{Size} &\multirow{2}{*}\textbf{Creatures}&\multirow{2}{*}\textbf{n. creatures}\\
{Creature}&{melee}&{in square}\\
\toprule
Fine & 100&16\\
Diminutive & 64&8\\
Tiny & 32&4\\
Small & 16&2\\
Medium & 8&1\\
\end{tabular}

\smallskip
These are the typical values of creatures for the listed size. There are frequent exceptions.

\end{multicols}

\begin{center}
\includegraphics[width=0.4\linewidth]{immagini/camminata.png}
\end{center}

\pagebreak

\subsection{Life and Death}\index{Death}\label{morire}

\begin{changemargin}{0.3cm}{0.3cm}\begin{emphasis}{He who does not know death does not know life. (Grand Hotel, film 1932)


\medskip

The worthy GM never purposely kills players' PCs. He presents opportunities for the rash and unthinking players to do that all on their own. (Gary Gygax)}\end{emphasis}\end{changemargin}\medskip

\begin{multicols}{2}

Weapon damage calculation is the sum of the weapon die, the character's Strength (or Dexterity if specified by a Feat), bonuses listed in the Weapon List, bonuses granted by Feats, bonuses specific to the weapon, and situational bonuses. \index{How to calculate weapon damage}\index{Weapon Damage}

When a character's Hit Points reach 0, they are considered unconscious and unable to take any action. Healing spells, potions, or a successful First Aid check (3 Actions, DC 12) can bring them back to consciousness with 1 hit point. After an hour, if their condition remains unchanged, they can make a Fortitude save (DC 15) to regain 1 hit point. If the save is failed, they drop to -1 Hit Points and become dying.

A dying character has negative Hit Points (-1 or less) and is unconscious and \hyperlink{morente}{close to death}. It will continue to lose one hit point per round until the value reaches double Constitution +10 and the character will die if not healed.

A Heal spell (spell or potion), of any level, will bring him to 1 Hit Point, subsequent cures will function normally.

A check of \hyperlink{prontosoccorso}{First aid}, 3 Actions, on difficulty 12 plus the value of negative Hit Points will bring the character to 0 Hit Points, i.e. knocked out. Each time the character drops below 0 Hit Points, the difficulty of the First Aid check increases by 2 and increases the level of Fatigue.

\begin{changemargin}{0.3cm}{0.3cm}\begin{tcolorbox}[title = Tups is dying]
Ex. Tups is badly injured and currently has -6 Hit Points, Jade decides to try to heal him (after moving him to a safer place). Jade attempts a First Aid check to at least stabilize his companion, her difficulty at the check is 12 + 6 or she must pass with DC 18 First Aid to bring him back to 0 Hit Points (unconscious)

A subsequent First Aid check at DC 12 will bring him to 1 Hit Points and a magical cure will cure him of the declared amount.
\end{tcolorbox}\end{changemargin}

A dying character who suffers further damage, enemies that affect the body or spells aimed at him or in an area, continues to subtract Hit Points to see if he ends up dying.

Mental type Conditions \index{Mental Conditions} such as Charmed, Confused, but not Dominated, end when the character becomes dying.

\begin{center}
\includegraphics[width=0.8\linewidth]{immagini/Nuremberg_chronicles.png}

\textit{The Dance of Death (1493) by Michael Wolgemut, Nuremberg Chronicle of Hartmann Schedel}
\end{center}

If an attack or spell takes the character directly to -(10+CON*2), the character dies\index{Immediate Death}\index{Massive Damage} with no possibility of being healed.

When a character returns to positive Hit Points after going negative he loses half his remaining Spell Points with a minimum of 10 and becomes further \hyperlink{affaticato}{Fatigued} (page \pageref{affaticato}).

When a character reaches negative Hit Points equal to 10+double his Constitution score it is \hyperlink{morto}{\textbf{dead}} [-(10+(CON*2))].

A character who arrives with 0 nonlethal Hit Points faints until they are back to 1.

\begin{changemargin}{0.3cm}{0.3cm}\begin{tcolorbox}[title = The death of the character]
Understand the reason for his death, the causes, and the mistakes he made. Evaluate the choices that led to this outcome. Each death is a personal loss, but also a valuable opportunity for growth and understanding. Treasure this experience and awareness, both personally and as a group. Instead of blaming each other, work together to understand what went wrong, with a mindset focused on improvement.
\end{tcolorbox}\end{changemargin}

Ex. If he has Constitution 2 he will die at -[10+2*2] = -14 Hit Points, if he has Constitution 0 he will die at -10 Hit Points, if he has Constitution -2 he will die at -[10+2*(-2)] = -6 Hit Points. In case of Constitution values  equal to or lower than -3 the character dies at -5 Hit Points.

If a character's non-lethal damage reaches negative Hit Points equal to 20+4*Constitution the character is dead. \hypertarget{puntiferitatemporaneimorte}{}

\begin{changemargin}{0.3cm}{0.3cm}\begin{narrator}
Depict the character's fall with emotive language to convey the pain and suffering they experienced. Highlight the dramatic elements, such as the impact of the fall, the spilling of blood, and the gasps of agony. However, if the players are easily affected, it may be best to tone down the graphic details to avoid causing distress.
\end{narrator}\end{changemargin}

A dead character cannot benefit from normal or magical healing, and cannot be brought back to life by a spell. Only a Patron has enough power to return the soul to the body and bring the creature back to life. The animate dead spell can reanimate a body, but as undead.

\subsubsection{Optional - Recover from 0 Hit Points} \index{Recover} \index{Unconscious}\index{Optional Recover from 0 Hit Points}\label{recuperozeropf}

\begin{changemargin}{0.3cm}{0.3cm}\begin{emphasis}{
The reports of my death are greatly exaggerated (Samuel Clemens)
}\end{emphasis}\end{changemargin}\medskip


\textbf{In case you want a less lethal system you can apply this optional rule.}

After reaching 0 Hit Points or less and becoming unconscious or dying, the character must make a Fortitude saving throw with a DC of 15 each round. 

If successful, they regain consciousness and have 1 hit point. If the saving throw fails, the DC increases by 1 for the next round. If the DC reaches 18 (after three consecutive failed saves), the character dies.

If the saving throw succeeds at any point, the character returns to 1 hit point.

\subsubsection{Characteristic points recovery}\index{Characteristic points recovery}\label{recuperopunticcaratteristica}

Any characteristic points lost are recovered at a rate of 1 point per day, unless marked as a permanent loss.

\subsubsection{Natural Hit point Recovery}\index{Natural Hit Point Recovery}\label{recuperopuntiferitanaturale}

Resting 8 hours, in 24 hors cycle, recovers the Constitution score + 2xWeapon Proficiency +Magic Proficiency per day in Hit Points, with a minimum of 1.

\subsubsection{Non-Lethal Hit Point Recovery}\index{Non-Lethal Hit Point Recovery}\index{Non-Lethal Hit Point}\label{recuperopuntiferitanonletali}\hypertarget{recuperopuntiferitanonletali}{}

Every hour you recover, with a minimum of 1 Hit Point, your Constitution value.

\subsubsection{Maximum Hit Points}\index{Recovering Maximum Hit Points}\index{Maximum Hit Points}\label{puntiferitamassimi}

Whenever the character takes damage that lowers his maximum Hit Points, he must subtract the indicated amount from his current Hit Points and also from his maximum Hit Points.

Every 8 hours of rest, in 24 hors cycle, you regain 1d4 + your Constitution score in maximum Hit Points, with a minimum of 1.

\end{multicols}

\vfill

\begin{center}

%\includegraphics[width=0.7\linewidth]{immagini/caravaggioSalomeLondon.png}

%\textit{Salomè con la testa del Battista è un dipinto di Caravaggio realizzato in olio su tela (91x106 cm) tra il 1607 e il 1610.\\ È conservato nella National Gallery di Londra.}
\includegraphics[width=0.45\linewidth]{immagini/giantdeath.png}

\textit{Henry Justice Ford}

\end{center}

\pagebreak

\subsection{Roll to Attack and Defence}\index{Roll to Attack and Defence}\index{Defence}\label{tiropercolpireedifesa}

\begin{changemargin}{0.3cm}{0.3cm}\begin{emphasis}{Always apply the right amount of force, never too much, never too little. (Kano Jigoro)}\end{emphasis}\end{changemargin}\medskip


\begin{multicols}{2}

The \textbf{Attack Roll} is a check against the opponent's Defence.

If the attacker uses:

\begin{itemize}[leftmargin=*]
\item \textbf{Melee or Touch Weapons}: the attacker must make a \textbf{Attack Roll (AR)}= 3d6 + Weapon Proficiency + Strength and any weapon Feats and magic bonuses and factors circumstantial (environment, curses..)

\item
\textbf{Range Weapons}: the attacker must make an Attack Roll (AR) = 3d6 + Weapon Proficiency + Dexterity + and any weapon Feats and magic bonuses and circumstantial factors (environment, curses..). Applies to bows, crossbows, drawn daggers, javelins...

\item
\textbf{Spell}: the attacker must make an Attack Roll (AR) = 3d6 + Weapon Proficiency + Strength (for melee spells) or + Dexterity (for ranged spells) any Feats and circumstantial modifiers.
\end{itemize}

The defender has a \textbf{Defence} equal to: 10 + Dexterity + Shield + Armour + any magic bonuses and Feats and circumstantial bonuses, for monsters the Defence value is already marked.
The player can decide to give up some bonus given by Weapon Proficiency in order to have a better Defence score. These points will not be available in the next attack (see Other actions and situations).

\begin{center}
\includegraphics[width=0.9\linewidth]{immagini/Coypel_Charles-Antoine_-_Fury_of_Achilles_-_1737.png}
\textit{Charles Antoine Coypel - Fury of Roland - 1737}
\end{center}

\subsection{Defence and Attack}\index{Defence}\index{Attack}\label{difesaeattacco}

\begin{changemargin}{0.3cm}{0.3cm}\begin{emphasis}{The Defence is always legitimate (anonymous victim)}\end{emphasis}\end{changemargin}\medskip

Each Attack Roll (3d6 + Proficiency with Weapons + Strength or Dexterity + any bonuses/penalties) compares the Defence or a value equal to 10 + Dexterity + Shield + Armour + any bonuses/penalties.

If the \textbf{Attack Roll}, i.e. the sum of the dice and modifiers, is equal to or greater than the Defence value the opponent has been hit and the damage of the wound will be established, given by the weapon + Strength score and others factors such as magic and Feats bonuses.

If the AR (Attack Roll) is lower than the Defence then the opponent will have parried, dodged, avoided.. The choice is left to the player (or Arbiter), once the attack is avoided no wounds are suffered.

There are situations that can benefit the Defence such as cover, hiding places, like pits, doors, companions that are much larger in size than your own. Consult the paragraphs related to \hyperlink{coperture}{Hideouts and Cover} to understand the advantage they can give.

There are occasions when it is not important to penetrate the Defence and hurt the opponent but simply touch the opponent.

Other times the opponent is surprised and cannot fully defend himself.

If it is \textbf{sufficient to touch the opponent} the attack roll has a +1d6 bonus since it is not necessary to deliver the blow but only to touch the opponent.\index{Touch the opponent}. In OBSS it is called Touch Attack in the manual.\index{Touch Attack}

If \textbf{the opponent is surprised} or does not expect the attack, the Defense will have a -4 penalty and also the Reflex saving throws. This is the value of the \textbf{Surprise Defense}.

\textbf{The Golden Rules also apply to shooting to hit. The d6 explode in case you roll 6 with the die, make 1 bad luck and trust of luck (i.e. subtract 4 points between Weapon Proficiency and Strength or Dexterity to add 1d6 to the Attack Roll, not from the bonuses given by Feats or magic items).}

If the modifiers and circumstances cause the damage inflicted to be negative, you will still do 1 damage.
This rule applies to weapon damage modifiers that cannot bring the total damage to be less than 1, if there are magical protections or damage reductions this can become zero and therefore you will not hurt the opponent (but if it becomes negative don't cure it!).

First remember that for every 6 rolled (in the 3d6 of the attack roll) you must roll another and keep rolling as long as you keep rolling a 6.

If you hit, \textbf{every two 6s rolled} (counting those of the Attack Roll and the subsequent ones resulting from the fact of having rolled a 6), the weapon does some extra damage or a Critical Roll. Re-roll the weapon damage die, with no magic or Strength or Feats for every two 6s rolled on the Attack Roll.

You can \textbf{subtract 4} or multiples from your attack to roll an extra d6. The choice has to be made in the most desperate situations where only luck can resolve the duel. The value is subtracted from the value of Weapon Proficiency and Strength or Dexterity, not from scores given by Feats or magical bonuses.

In case you roll a 1 in the Attack Roll, this lowers the total value by 1 (therefore 1 does not count) but does not affect whether you have scored a critical roll or not.

\textbf{The fact of rolling a Critical Roll is not a guarantee of having hit, you must always overcome the Defence}.

The basic rules of the Skills also apply to the Attack Roll. Defence is a fixed value and as such uses the modifiers for fixed value checks.

\begin{center}
\includegraphics[width=0.9\linewidth]{immagini/critico.png}

\textit{Henry Justice Ford}
\end{center}

\subsection{Roll 3 times 1}\index{Roll 2 times 1 or 2 times 2 and once 1}\label{tiraretrevolteuno}\index{Roll 3 times 1}

If you rolled 1's three times, you missed, regardless of the final result.

If you missed and rolled at least two 1s or one 1 and two 2s the Arbiter could decide bad things about your attack (you drop your weapon, you hit a friend, your weapon breaks, you get hurt, you fall, it appears a \hyperlink{diavolodellafossa}{Pit Fiend} to mock you...). 

\begin{changemargin}{0.3cm}{0.3cm}\begin{narrator}
OBSS aims to be enjoyable for players, promoting fun and allowing them to see the impact of their dice rolls and decisions. The Golden Rules and Burst Damage are designed to add excitement to the game and bring the results of the dice to life. Players, especially those with experience, will appreciate the realization that dice rolls are not just numbers, but opportunities to make a difference. Encourage players to describe their critical rolls and act out their triumphs!
\end{narrator}\end{changemargin}

\subsection{Roll 3 times 6}\index{Roll 3 times 6}\label{tiraretrevoltesei}

If in the first 3 Attack Rolls you make 6 three times you will hit the opponent regardless of the final result of the Attack Roll. In addition to having the certainty of having made a Critical Roll (see below) the Arbiter could decide to apply some additional descriptive (or effective) effect.

\subsection{Critical Roll}\index{Critical Roll}\index{Critical damage}\label{tirocritico}

You roll \textbf{additional weapon damage} (no magic or feats or strength bonuses, just the weapon die) for every two times you roll a 6 on the attack roll, this damage is also called \textbf{crit damage}. If you made two Critical Rolls it means that you have to roll 2 more weapon dice.


\begin{changemargin}{0.3cm}{0.3cm}\begin{tcolorbox}[title = Critical Shot Example]
I.e roll 6 4 5, roll over 6, roll over a 6, roll over 4: as damage you roll 2 times the damage of the weapon, once because you hit one because you rolled three times 6 (if you rolled a further 6 would have been Weapon + Strength + bonus/feats + 3{*}Weapon).
\end{tcolorbox}\end{changemargin}



\subsection{Optional - Critical Roll Variant}\index{Optional - Critical Roll Variant}\hypertarget{tirocriticovariante}{}\label{tirocriticovariante}

For players who may not prefer relying on chance, critical rolls can be based on the character's weapon proficiency instead. 

Another option is to consider an attack roll a critical hit if it exceeds the Defense score by a multiple of 6, regardless of the number of 6s rolled.

The decision to use this alternative critical hit method must be made at character creation and approved by the Arbiter. Changing it later requires giving up a Weapon Proficiency point and relearning how to fight differently.

\begin{changemargin}{0.3cm}{0.3cm}\begin{tcolorbox}[title = Critical Roll Variant]
Ex. Attack Roll 21, opponent's Defence is 13. I hit it with a margin of 8, i.e. I add 1 more weapon damage
If the attack roll had been 26, 2 critical damage or two weapon damage would have been added.
It is the Arbiter who says how many critics have been obtained.
\end{tcolorbox}\end{changemargin}

\subsection{Optional - Critical Roll Actions}\index{Optional - Critical Shot Actions}\label{OpzionaleAzioniTiro}

This Option allows for a fight less focused on damage but more on maneuvers and tactics. Combat is understood as a continuous exchange of actions and reactions that can also have repercussions in the following rounds

The player keeps track of the Critical Roll he roll in three rounds, and not applying damage, at a time, restarting the count at the end of the third round or when is empty.

Each round can scale one or more accumulated Critical Rolls to perform Critical Actions. The use of Critical Actions is against the opponent which the criticals have been performed. The list proposes more Critical Actions per sum of Critical Rolls used. You cannot have more than 3 Critical Throws accumulated. Activating these Critical Actions costs one Reaction.

\textbf{1 Critical Shot}: you inflict critical weapon damage; get +4 to hit until end of next round; +4 Defense until end of next round; the opponent has -4 to hit until the end of the next round.

\textbf{2 Critical Shot}: you inflict two critical weapon damage; the opponent misses you with the first useful melee attack; you can move and/or move the opponent one meter; reduce the damage of a melee attack by the amount of critical damage it causes; until the end of the next round the opponent has half movement.

\textbf{3 Critical Shot}: you can move your movement; the opponent loses the first Action he performs by the end of the next round; you and/or the opponent can move half of your movement, the opponent cannot move until the end of the next round.


These Critical Actions can be described as taking advantage of the opponent's distraction, throwing dirt in the eyes, forcing a weapon to move...

Player are invited to create and suggest new Critical Actions that must be approved by Arbiter.

This option is compatible and usable with also \textbf{Optional - Critical Roll Variant}.

\begin{center}
	\includegraphics[width=0.7\linewidth]{immagini/esplosionedanno.png}
	
	\textit{Henry Justice Ford}
\end{center}


\subsection{Burst Damage}\index{Burst Damage}\label{esplosionedeldanno}\index{Explosion Damage}

When the maximum value is rolled on the weapon die (e.g. rolling an 8 on a d8 for a long sword), the die is rerolled and the result is added to the original roll (only once).

For weapons with multiple dice (e.g. 2d4), the maximum value is considered the sum of both dice (e.g. 8). No Burst Damage is applied for weapons with a maximum damage of 6 or less

\smallskip


Some weapons have a different Burst Damage value. In the weapon table, if EDX is marked (e.g. ED9), the X value represents the minimum roll required to trigger the Burst Damage. For example, in the case of ED9, Burst Damage is triggered if the weapon damage die rolls 9 or higher. This is a feature of only a few highly lethal weapons.

Note that Burst Damage does not "explode" - even if the maximum value is rolled with the added damage die, it does not trigger another burst.

Damage rolls added due to a critical hit (rolling at least two 6s) do not benefit from Burst Damage. If the extra weapon die rolled as a result of the critical hit rolls the maximum, it is not rerolled or added to the damage. When rolling damage, make sure to declare which die is for the weapon and which is for the critical hit.

\subsection{Multiple Attacks}\index{Multiple Attacks}\label{attacchimultiplimischia}

With one Action, a character can make a single attack.

With two Actions, a character can make up to two attack rolls. If they want to make more than two attacks, they must use additional Actions.

Each individual arrow, dart, dagger, or ranged weapon thrown counts as one attack.

The first attack action has no penalty, while the second attack action has a -5 to hit penalty. Subsequent attack rolls will have an additional -5 to hit penalty, so a third attack will have -10, and a fourth attack will have -15, and so on.

If the cumulative hit penalty becomes greater than the attack roll, it is no longer possible to make further attacks.

For example, if a character has Weapon Proficiency 5, Strength 1, a +2 bonus from the Weapon List, a +1 to hit bonus from a Feat, +2 for flanking, and +1 for a magic weapon, the first attack roll will be 3d6+12, the second will be 3d6+7, and the third will be 3d6+2. A fourth attack cannot be made as the bonus to hit would become negative.

Any dynamic bonuses to hit, such as +1d6, only apply to the first attack roll and not to the calculation of the bonus for multiple attacks. In this example, the attack roll becomes 4d6+12 for the first attack and 3d6+7/+2 for subsequent attacks.

The player can choose to attack different targets with each attack. They can intersperse attacks with Move Actions as long as they have enough Actions.


\subsubsection{Optional - Multiple attacks variant}\index{Optional - Multiple attacks variant}\label{varianteattacchimultipli}

The player calculates their bonus to hit and makes a single attack roll. If the attack is successful, the player rolls for damage and for every multiple of 6 in their bonus to hit, they add an additional Critical Roll. This type of attack consumes two Actions and is the only attack that can be made in a round.

Note: This variant aims to speed up the game by requiring only one attack roll. However, it is not compatible with the Optional - Critical Roll Variant.

\begin{center}
\includegraphics[width=0.9\linewidth]{immagini/archer.png}

\textit{Scythian archers in ancient attic vase painting}
\end{center}


\subsection{Thrown Weapons}\index{Multiple attack with Thrown Weapons}\label{armidatiro}\index{Thrown Weapons}

Thrown weapons refer to all weapons with a range, including bows, crossbows, slingshots, as well as daggers, javelins, and spears if they are thrown. The damage bonus from Strength is automatically applied to slingshots, daggers, javelins and other weapons thrown by hand.

For composite bows, Strength modifies the damage, but crossbows do not benefit from this bonus. 

Dexterity only modifies attack rolls.

\textbf{Projectiles fired from magical bows, slings, crossbows are not considered magical.\\
However, in the case of magical projectiles, they add their magical bonus to both attack rolls and damage.}

Each throwing weapon has its range marked, indicating the maximum distance it can shoot without a penalty. The weapon can strike targets within three times the listed range.

If the target is within the indicated range, there is no penalty to hitting. If the target is between the first and second increments, the penalty to hit is -1d6. If the target is between the second and third increments, the penalty to hit is -2d6.

For example, a dagger thrown within 6 meters has no penalty, but if thrown between 6 and 12 meters, there is a -1d6 penalty to hit. If thrown between 12 and 18 meters, there is a -2d6 penalty to hit, and beyond that distance, it cannot be thrown.

%\begin{center}
%\includegraphics[width=0.75\linewidth]{immagini/fenice.png}
%
%\textit{Henry Justice Ford}
%\end{center}

\subsection{Light Weapons} \index{Light Weapons}\label{armileggere}

these weapons are light and indicated for \hyperlink{combattimentoaduemani}{two weapons fighting}.

\subsection{Versatile Weapons} \index{Versatile Weapons}\label{armiversatili}

on weapons with the Versatile feat you can decide whether to use Dexterity instead of Strength on attack rolls. Strength is always used to damage.

\subsection{Long Weapon} \index{Long weapons}\label{armalunga}

the long weapon grants the right to hit from a longer distance, i.e. 2 meters away. It grants a +2 bonus to the attack roll. This bonus remains valid until the opponent enters within melee range.

If the opponent also has a long weapon, do not consider the bonus (they are both within their own melee range).

\subsubsection{Using a long weapon on short distance} \index{Using a long weapon on short distance}\label{armalungabrevedistanza}

it is possible to use a long weapon in melee with an opponent with a short weapon or with a reach of less than 2 meters with a -4 on the attack roll, with the exception of the quaterstaff.


\begin{changemargin}{0.3cm}{0.3cm}\begin{tcolorbox}[title = Long Weapon Combat]
I.e. Tups, armed with a long sword, and a brigand, armed with a long spear, are facing each other. Tups has an initiative score of 15, while the brigand has 12.

Tups takes advantage of his agility and quickly closes in on the brigand, striking him with a powerful blow. The brigand, now in melee range with Tups, finds that his long weapon is a hindrance rather than a help.

As a reaction, the brigand uses an Action to move back two meters and attack Tups with a +2 bonus to hit, since Tups is now further away. As a third action, the brigand moves back another 9 meters and starts shouting curses at Tups..

At this point, Tups is 11 meters away from the brigand. Tups decides to charge, sacrificing his Defense for a bonus to hit. He charges the brigand hitting him and arriving at him and with one last action he decides to improve his Defense (Combat Mastery).

The very wounded brigand tries to hit him trusting that his difficulty in using a long weapon so close is balanced by the penalties given by Tups' run. Tups is hit and the brigand throws down his spear and draws a short dagger and goes on the defensive too.

\end{tcolorbox}\end{changemargin}

\subsection{Double Weapon} \index{Double Weapon}\label{armadippia}

a double weapon is a weapon that is dangerous from both ends. It can be used as a single weapon, or, incurring the penalties of combat with two weapons, as two weapons. Unless specified a double weapon used for two-weapon combat is equivalent to using two medium weapons.

\begin{center}
\includegraphics[width=0.9\linewidth]{immagini/twoweapon.png}
\end{center}

\subsection{Two-weapon combat}\index{Two Weapon Combat}\hypertarget{combattimentoaduemani}{}\label{combattimentoduemani}

Attacks made with the off-hand weapon are considered multiple attacks. The first attack, whether made with the primary or secondary weapon, will have the full bonus to hit. Subsequent attacks will have a -5 penalty to hit.

The damage bonus provided by Strength for the off-hand weapon is halved. If the secondary weapon isn't light the attack roll get an additional -3 penalty (es. 0,-8,-10,-18..).

Using the off-hand weapon can improve Defense by one point, but it cannot be used to make attacks.

\subsection{Charge} \index{Charge}\label{carica}

The opponent must be within 2 Movement Actions (typically 18 or 12 meters) and no closer than 3 meters. You must run until you are within melee range.

You receive a +1d6 bonus to your Attack Roll and a -4 penalty to Defense until the end of the round. The attack after the first takes a -10 penalty to hit, and subsequent attacks may have additional penalties of -15, -20, etc.

The Charge action takes 2 Actions for both movement and attack. No other penalties are incurred for running beyond the indicated penalties.

When using a Charge action with a long weapon, the attack still receives a +2 bonus to the attack roll, allowing it to strike from a distance and then engage the opponent in melee combat.

\begin{center}
\includegraphics[width=0.9\linewidth]{immagini/carica.png}

\textit{A Connecticut Yankee in King Arthur's Court / Samuel Clemens. New York : Charles L. Webster \& Co., 1889}
\end{center}

\subsubsection{Countercharge}\index{Countercharge}\label{controcarica}

an attack roll made with a weapon with the counter-charge feat when used against a charging opponent/mount inflicts a critical roll and strikes first, except where the opponent has a longer weapon or greater reach than the counter-charge preparer, in this case the attack is governed by initiative rolls.

\subsubsection{Ready a long/countercharge weapon against a charge} \index{Ready a long/countercharge weapon against a charge}\label{prepararearmalungacontrocarica}

It's a reaction.

\begin{center}
\includegraphics[width=0.9\linewidth]{immagini/pilum.png}

\textit{Roman soldiers armed with Pilum, ready for a counter charge.}
\end{center}

\subsubsection{Counter-Charging Weapon Charge} \index{Countercharge}\label{caricaarmadacontrocarica}

if the attack roll is successful when you use a weapon with the counter-charge feat to charge an opponent, it inflicts a critical roll.


\subsection{Attacks with Splash Weapons} \index{Splash Weapons}\index{Holy Water}\index{Burning Oil}\label{attacchiarmidaspargimento}\hypertarget{spargimento}{}

splash weapons are those that "spread" their contents where they fall, eg burning oil/holy water... A splash weapon has a range of 6 meters\index{Throw Splash Weapons}\index{Range of splash weapons}.

In case the attack misses (by at least 5) roll a d8 and consult this diagram to understand where the ball landed:

\medskip

\begin{tabularx}{0.30\textwidth}{ccc}
1& 2& 3\\
4 &\textbf{X}& 5\\
6 &7 &8\\
&\textbf{0}&\\
\end{tabularx}

\smallskip

\textbf{X} is considered the target of the thrown object. \textbf{0} the origin point of the launch.

If the roll misses by 5 or more, roll a 2d4 to determine along the direction indicated by the previous d8 how many meters away from the target it fell, i.e. count the meters from the target.

For example, with the d8 roll I get 5 and then rolling 2d6 I get 4, it means that the bottle fell to the right of the target 4 meters away.

It is also possible that the bottle was thrown on one's feet (eg you roll 7 and then 6.. you could have thrown it at a partner or behind you!).


\subsection{Unprepared -- Taken By Surprise}\index{Surprised}\index{Surprise}\label{coltidisorpresa}

If characters are taken by surprise, meaning they don't expect to be attacked, the first round should be treated as a surprise round.

During this round and for any attacks made during this round, their Defense and Reflex Saving Throw will suffer a -4 penalty. They won't be able to react, use Actions or Reactions unless explicitly allowed. From the next round, they will be able to declare initiative and act normally, and the same applies to opponents.

To evaluate whether a character is surprised, make a Reflex Saving Throw, comparing it with the Stealth check of the opponents. If the Saving Throw is lower, the character is surprised. If the character is at attention and expecting an ambush, grant +4 to the Saving throw.

When both characters and enemies are caught by surprise, to determine who is actually surprised, make a reflex save. Those who roll above 15 are not surprised.

\subsection{Magic in Combat}\index{Magic in Combat}\label{magiaincombattimento}

the spellcaster who casts a spell while in combat (has an opponent in melee or is targeted from a distance) is considered Distracted.

\subsection{Modifiers in attack or defense} \index{Modifiers in attack and defense}\label{particular attack-defense modifiers}

The best tip that can be given in managing the most chaotic combat situations is to think of them like a film, evaluate the cinematic nature of the situation.

It's not a question of miniatures, spaces, squares... it's a question of fun and visualization of the scene. Unorthodox solutions for unorthodox situations.

Give a bonus or penalty ($\pm 1-2$ unless otherwise indicated) whenever the player has an advantage or disadvantage and likewise to the opponent.\\

\end{multicols}

\begin{tabularx}{0.98\textwidth}{l|X|X}
\multicolumn{2}{c}{\textbf{Attacker}}&\multicolumn{1}{c}{\textbf{Defender}}\\
\textbf{Mod}.&\multicolumn{1}{c}{\textit{Situation}}&\multicolumn{1}{c}{\textit{Situation}}\\
\toprule
\textbf{-1} & Fatigued (1), Dim light & Fatigued (1) \\
\hline
\textbf{-2} Fatigued (2)&Fatigued (2), Dazzled, Entangled & Grabbed, You cast a spell while under attack \\
\hline
\textbf{-4} & Fatigued (3), Prone, Long Weapon at close range & Fatigued (3), Surprised, Prone, Kneeling, Sitting, Restricted, Stunned, Grasped by a wall, Blocked\\
\hline
\textbf{-1d6} & Restricted, Frightened, Thrown weapon against opponent in melee, Weapon unknown, Target invisible but detected, Grasped to a wall, Blocked & \\
\hline
%\textbf{+1} & & \\
%\hline
\textbf{+2} & Flank, pos. Overhead, Shoulder Attack, Long Weapon & Light Cover\\
\hline
\textbf{+4} & & Average coverage\\
\hline
\textbf{+1d6} & Invisible, Charge & \\
\hline
\textbf{+8} & & Full coverage\\

\end{tabularx}

\medskip

\begin{multicols}{2}

When you write -1d6 it means that you roll one die less (or two if it is -2d6), equally if it says +1d6 you roll one die at 6 more and add it.

When the penalty is to Defense, treat each -1d6 as a -4 to Defense.

\textbf{In principle in combat a light bonus is +1, medium +2, high +1d6 (or +4), a very high bonus is +2d6 (or +8), vice versa for penalties}.

\medskip

The bonuses are not added to each other but the one with the highest value is used. If an opponent is above the character, behind him and charging, he has a bonus to hit of +1d6, given by the charge.

The penalties are added to each other. If the character is surprised and prone he has -8 to Defense.


\medskip

Always remember the aim is to have fun, at the expense (for the Narrator) of some monsters, do not be rigid but dynamic and adapt to situations.

\subsection{Other actions and situations} \label{AltreAzioni}\index{Other actions and situations}

\subsubsection{Unarmed Attack} \index{Fist}\index{Kick} \index{Brawling}\label{attaccomaninude}

two weapons that no one will ever lack are one's punches and kicks.

If you haven't taken the Empty Fist weapon list, a punch or kick will do 1d2+STR of non-lethal damage. Only with the "Empty Fist" Weapon List does one become a martial artist.

\subsubsection{Stand up off prone}\index{Stand up off prone}\label{alzarsidaprono}

costs two Actions. The player can make an Acrobatics check if he rolls 13 or more it costs 1 Action to stand up. If you do a critical failure check, you cannot take any other actions that round and remain prone.

When the Acrobatics score reaches 6, getting up from prone costs 1 Action. With Acrobatics 8 it costs an Immediate Action.

When prone you can crawl\index{Crawl}\index{On all fours} or crawl. The terrain is considered difficult and you are still considered prone until you stand up.


\begin{center}
	\includegraphics[width=0.55\linewidth]{immagini/vantaggio.png}
	
	\textit{Henry Justice Ford}
\end{center}

\subsubsection{Help another}\index{Help another}\label{aiutare}

you can help a teammate attack or defend in melee combat by distracting or interfering with your opponent. You can make a melee attack (1 Action) against an opponent who has already engaged in battle with your ally.

A roll is made to hit against the opponent's Defence with a 1d6 bonus. If the attack hits, you do no damage, the companion gains a +1 bonus on attack rolls on your next attack (by the end of the following round) against that opponent or a +1 bonus to Defence against that opponent's next attack (your choice) within the next round. If the helper rolls a critical roll then the helper gets a +2 bonus.

Multiple characters can help the same ally; bonuses of this type are cumulative (maximum 4 on medium size), provided the opponent is surrounded.


\begin{center}
	\includegraphics[width=0.9\linewidth]{immagini/colpodigrazia.png}
	\textit{Beheading of St. John the Baptist. St. John's Co-Cathedral in Valletta (Malta). (Caravaggio)}
\end{center}

\subsubsection{Coup de Grace} \index{Coup de Grace}\label{colpodigrazia}

costs 3 Actions, you can use a melee weapon to deal a coup de grâce to a helpless (unconscious or trapped) target. You can also use a bow or crossbow, as long as you are adjacent to the target.

The attacker strikes automatically and inflicts three critical roll.


\subsubsection{Aimed Shots}\index{Aimed Shots}\label{tirimirati}\index{Aim at specific parts}

OBSS does not provide the ability to make aimed shots with any weapon or spell, unless specifically indicated.
When you hit the target you hit it generically, without the possibility of specifying whether to the head, leg or other, the same concept applies in the case of blows to objects, e.g. if you aim at a door hinge you hit the whole door. This does not prevent the Arbiter from evaluating appropriate consequences.

\subsubsection{Non-lethal damage}\index{Non-lethal damage}\label{dannononletale}

non-lethal damage is a form of damage caused by particular weapons or when the intentional purpose is to make the enemy unconscious and not kill him.

Non-lethal damage is treated like normal damage but must be marked separately on the sheet.

\subsubsection{Non-lethal damage with unsuitable weapon} \index{Non-lethal damage with unsuitable weapon}\label{dannononletalearmanonidonea}

if you wish to deal non-lethal damage with a weapon not predisposed to non-lethal damage, you have a -1d6 on the attack roll.

\subsubsection{Without Competence}\index{Without Competence}\label{senzacompetenza}

using a weapon without the appropriate proficiency, i.e. not having the Weapon List to which the weapon belongs, imposes a -1d6 on the attack roll. You can't use a weapon's Versatile ability if you don't know how to use it. A Simple Weapon can be used even without a specific knowledge.

\subsubsection{Throwing Weapons} \index{Throwing Weapons}\label{lanciarearmi}

a sword or other weapon not meant to be thrown, with no Range, can still be thrown at the opponent.

The attack roll takes a -1d6 and the weapon does a lower damage category (long sword does 1d6, short sword 1d4..). The throw range is 3 meters.

\subsubsection{Power Blows}\index{Power Blows}\label{colpipotenti}

the player can freely add +1 to the damage by subtracting 2 from the attack roll (Weapon Proficiency +1 requirement). You can't remove more than Weapon Proficiency/4 from your attack roll.

\subsubsection{Flank} \index{Flanking}\label{fiancheggiare}

if two characters are around the same target but not next to each other they get +2 to attack roll or Defence (their choice which bonus to take).

There can be at most 4 characters around a medium-sized creature that get the flanking bonus. The type of bonus is chosen round by round, if not declared it is worth +2 to the attack roll.

If, by pulling a hypothetical line that connects the two characters, it crosses the opponent's square completely, then there is a situation of flanking.

\bigskip

Flanking example\index{Flanking example}

\medskip

\begin{tabularx}{0.45\textwidth}{lll}
\toprule
A & G & D\\
B & \textbf{X} & E\\
C & H & F\\
\end{tabularx}

\bigskip

In this scheme the flanking is taken from the pairs: A-F, B-E, C-D, G-H

\bigskip

If the creature can face multiple creatures at the same time they will not enjoy the flanking bonus.

\subsubsection{Using a weapon with two hands} \index{Using a weapon with two hands}\label{usarearmaconduemani}

a one-handed weapon that can (but does not have to) be used two-handed increases the damage dice when used two-handed.

Ex. Longsword for a medium creature can cause 1d8 one-handed or 1d10 two-handed. A shortsword can't be wielded with two hands by a medium creature, but a small creature can.

If the weapon must be held in two hands because it is too big for one's size, this modifier is not considered. If EDX value is different from the weapon's maximum dice value then increases by 2 (Katana deals 2d6 damage with ED11) EDX value when weapon is used two handed.

\subsubsection{Combat Mastery} \index{Combat Mastery}\label{maestriacombattimento}

You use an Action to better prepare yourself for subsequent attacks from your opponents. Until the start of the next round you have +1 Defense.

The player can freely add +1 to Defense by subtracting 2 from the Attack Roll as long as he attacks in the round.

Conversely, he can take a -2 Defense to raise his Attack Roll by +1 and therefore improve his attack. This option is only usable if you make at least one attack.

You cannot remove/add more than Weapon Proficiency/4 to the Attack Roll/Defense, these are free actions.

You can instead use an Action to better prepare for subsequent attacks from your opponents. Until the end of your next round you have +2 Defense.



\subsubsection{Precise Hit} \index{Precise Hit}\label{precisehit}

the player, using 2 Actions, makes only one attack (and must not have made any in the round). On this single attack he gains a +2 bonus on attack roll.

\subsubsection{Taking Aim (sniper)} \index{Taking Aim (sniper)}\label{cecchino}

you spend 2 Actions per round aiming. You gain a bonus to hit of +1 on the first round, +2 on the second round, and finally +4 on the third and final round of Taking Aim. You cannot use Move Actions while Taking Aim.


\subsubsection{Using a thrown weapon while aiming at an opponent in combat} \index{Using a thrown weapon while aiming at an opponent in combat}\label{usarearmalancioinmischia}

it is not easy to aim correctly and not hit your partner, you have a -1d6 to attack. The bonus is cancelled if there is a difference of 2 or more sizes between the opponent and the companion. In case of Critical Miss the other creature get hit.

\subsubsection{Using a thrown weapon under threat} \index{Using a thrown weapon under threat}\label{usarearmalanciosottominaccia}

using a thrown weapon such as a bow, crossbow, or dagger (that you want to throw) while engaged in melee you have -1d6 to Attack Roll.

\subsubsection{Total defense} \index{Total defense}\label{total defense}

It costs 2 Actions, you cannot perform any weapon attacks or spell casting, if you move the terrain is difficult, you gain +4 to Defense.

\subsubsection{Disengage} \index{Disengage}\label{disengage}

costs 1 Action, you move 1 meter and do not provoke attacks of opportunity.\index{Taking a step}

\subsubsection{Fighting in the dark}

Fighting in low light conditions involves difficulties summarized in this diagram.

\medskip

\begin{tabular}{lll}
	\textbf{Vision} & \multicolumn{2}{c}{\textbf{Condition}}\\
	& Dim Light & Darkness\\
	\toprule
	Normal & -1 TH, Awarn. & Invisi. (page \pageref{invisibility})\\
	Twilight & Normal & Invisi. (page \pageref{invisibility})\\
\end{tabular}


\subsubsection{Weapon too big}\index{Weapon too big} \label{armatroppogrande}

The size indicated in the weapon table (see \hyperlink{dimensionediunarma}{Weapon size} ) refers to a medium creature. For a small creature, the size is considered to be one category larger, so a short sword that is small in the hand of a small creature is considered a medium-sized weapon. This does not change the damage caused or the type of damage.

Similarly, a large weapon, such as a two-handed greatsword, becomes a medium-sized weapon in the hands of a giant.

Normally, a creature can use a weapon up to its size with one hand or use two hands with a weapon one size larger.

Attacking with a \textbf{Weapon too big} \index{Weapon too big}relative to your size is problematic.

If the weapon is not among those "usable", for example a halberd (large weapon) for a creature of small size, the penalty on the attack roll is -1d6. Likewise, a small weapon cannot be used with two hands for a medium-sized creature.

In the weapon table, size is marked as P (small), M (medium), G (large), E (huge). A "larger" version of a weapon increases the weapon's damage by one category (1d4->1d6, 1d6->1d8, 1d8->1d10, 1d10->2d6, 2d6->2d8, 2d8->2d10, 2d10 ->3d6...)


\begin{center}
	\includegraphics[width=0.6\linewidth]{immagini/angelospadone.png}
\end{center}

\subsection{Optional - The Only Rule}\index{Optional - The Only Rule}\hypertarget{lunicaregola}{}\label{lunicaregola}

This option aims to simplify the management of any contested test, whether it be related to Basic or Active Skills.

When a creature/character has an advantage or disadvantage, roll an additional 1d6 on the check. If it has two advantages, roll 2d6, if it has three advantages, roll 3d6.
The check adds or subtracts, in case of bonus or penalty, the highest value of the dice rolled. These dice never explode.

A Disadvantage cancels an Advantage if present.

If the advantage/disadvantage is related to a static value (such as Defense) then this increases by 2 for each advantage/disadvantage accumulated.

\subsection{Optional Combat Maneuvers}\label{azioniopzionaliincombattimento}

These combat Actions are at the discretion of the Arbiter who may or may not grant them. Every Maneuvers counts as Attack Action.

\subsubsection{Disarm*}\index{Disarm}\label{disarmare}

roll an opposed check with Weapon Proficiency + Dexterity/Strength.

Who try to disarm fails and gets a critical failure, he loses the weapon. It costs 2 Actions.

%\begin{center}
%\includegraphics[width=0.9\linewidth]{immagini/alfieri37.png}
%\end{center}

\subsubsection{Feint*} \index{Feint}\label{finta}

roll opposed check of  Weapon Proficiency + Deceive (who feint) vs. Weapon Proficiency + Sense Emotion. If the check succeeds, the opponent got -2 penalty to Defence until the end of the next round.

If the person attempting the maneuver fails and gets a critical failure, it is he who get -2 to Defence until the end of the next round. It costs 1 Action.

\subsubsection{Pushing an opponent*} \index{Pushing an opponent}\label{spingereavversario}\hypertarget{spingereavversario}{}

it is an opposed Strength check (opposite Fortitude save using Strength bonus). Those of a larger size gain a +1d6 bonus per size difference.

The winner can push the opponent up to 0.5 meters in the direction he wants for success in the check (up to the maximum of your movement). Eg if you win the check of 7 you move the opponent up to 3 meters. It costs 2 Actions.

\subsubsection{Grabbing an opponent*}\index{Grabbing an opponent}\label{afferrareunavversario}

it is an Opposed Strength Test (Fortitude save with Strength bonus). Those with a larger size gain a bonus of +1d6 per size difference. If the person who succeeds in the maneuver obtains a critical success, the opponent is considered \hyperlink{blocked}{Blocked}.

It costs 2 Actions to do and hold and free yourself from the hold. It is considered that whoever grasps is also grasped and has at least one hand occupied in grasping.

The two contenders lose the Dexterity bonus to Defense and Reflex saving throws.

Moving a grabbed creature requires \hyperlink{pushopponent}{Push opponent}.

Each contestant can attack the other grabbed with a small weapon usable with one hand or with punches and kicks.


\subsubsection{Knocking down an opponent*} \index{Knocking down an opponent}\label{farecadereavversario}

it is an opposed Strength or Dexterity check, each contender choosing whichever he prefers.

Each makes a Fortitude (with a Strength modifier) or Reflex (with a Dexterity modifier) Saving Throw and compare the results.

For each additional leg/paw or Size of difference, you get a +1 bonus to the check.

If the person attempting the maneuver fails and gets a critical failure, he falls.

It costs 2 Actions.


\begin{changemargin}{0.3cm}{0.3cm}\begin{narrator} %box narratore
Checks such as \textbf{Disarm}, \textbf{Feint}, \textbf{Push}, \textbf{Grab}, \textbf{Drop} can be resolved by setting the difficulty of the check taker ( and not the player) to a fixed value equal to 10 + the relative modifiers.
\end{narrator}\end{changemargin}

\subsubsection*{Optional - Universal Maneuver Management}\index{Optional - Universal Maneuver Management}

In order to be able to neutrally handle any unforeseen maneuver or action in combat, an approach can be used that benefits both the player and the opponents.

You declare what type of action you want to do in combat, if the attack roll is successful then whoever suffers the action decides whether to suffer the desired effects of the action or suffer the damage of the attack. If the actioner scores a critical roll, then he enforces the action's chosen effect.

There are obvious limits to the type of action taken which at the Arbiter's discretion could be overcome by a certain number of critical rolls made.


\subsubsection{Change own size*}\index{Change own size}\label{modificatedimensioni}

if the character changes size \index{Change own size} his Defence changes accordingly

\bigskip

\begin{tabular}{ll|ll}
\textbf{Size} & \textbf{Defence}& \textbf{Size} & \textbf{Defence}\\
\toprule
Diminutive & +8 &Large & -1\\
Tiny & +4 & Huge & -2\\
Small & +2 & Gargantuan & -4\\
Small & +1 &Colossal & -8\\
Medium & +0&&\\
\end{tabular}


\subsection{Mounts}\index{Horse fighting}\index{Horse}\label{cavalcature}

\begin{changemargin}{0.3cm}{0.3cm}\begin{emphasis}{
- And you can find yourself another wife!

- Ah, yes. but the trouble is, she took away my gun and horse! Too bad, she was so beautiful, I was fond of her. I gave her a few whips, but she didn't notice.

- Who, your wife?

- No, my mare. Finding another wife is quick, but I can never find a mare like hers again. (Red Shadows, 1939 film)}\end{emphasis}\end{changemargin}\medskip


A mount also has its 3 Actions and are usually used to move around or to react and obey your commands.

A mount acts in your round, and you decide when it takes its Actions relative to yours. He doesn't roll initiative, he uses yours.

To make a mount move, attack, defend itself, you must use one of your Actions.

Attacks towards a character on horseback (or mount in general) unless otherwise declared aim at the rider and not at the horse.


\subsubsection{Situations and rules}\label{cavallosituazioniregole}

\begin{itemize}[leftmargin=*]
\item
Whenever the mount is hit the rider must make a Ride check at DC 15 or be unhorsed.

If the mount is a war mount, trained for combat, the check has difficulty 12.

\item
Fighting from an elevated position grants a +2 to your attack roll if your opponent is not at your height.

\item
Getting on or off your mount costs 1 Action if you have the Ride skill, otherwise 2 Actions.

\item
If a spell or situation abruptly moves your mount against your will, you must make a DC 15 Reflex save or a Ride check (DC 15) or be thrown from your horse.

\item
If you are thrown from your horse you fall prone and suffer 1d6 damage.
\end{itemize}


\subsubsection{Controlling a Mount}\label{controllocavalcatura}

While in the saddle, you have two choices:

\begin{itemize}[leftmargin=*]
\item you can command your mount
\item allow it to act on its own.
\end{itemize}

Particularly intelligent mounts tend to favor autonomy of action rather than being commanded.

You can control a mount only if it has been trained to accept a rider. Trained horses, mules, and similar creatures are assumed to have received such training.

A controlled mount's initiative changes to match that of its rider. It moves according to your directions and has only three Action options: Move, Attack, Disengage.
Each bonus and penalty reported by these 3 Actions applies only to the mount.

Making a mount perform two of the above Action costs the rider 1 Action.

If the mount is smart having a rider does not restrict the actions the mount can take and it moves and acts as it wishes. He may flee from combat, charge in and eat a badly wounded enemy, or otherwise act against your will.

\end{multicols}

\

%\vfill

%\begin{center}
%\includegraphics[width=0.55\linewidth]{immagini/napoleone.png}

%\textit{Jacques-Louis David, Bonaparte Crossing the Great St Bernard, 1801, Malmaison Castle}
%\end{center}


%\vfill

%\begin{center}
%\includegraphics[width=0.7\linewidth]{immagini/fauchard.png}
%\end{center}

\pagebreak

\section{Hideouts and cover} \index{Hideouts}\index{Cover}\hypertarget{coperture}{}

\begin{changemargin}{0.3cm}{0.3cm}\begin{emphasis} Where there is a lot of light, the shadow is darkest. (Johann Wolfgang von Goethe) \end{emphasis}\end{changemargin}\medskip

\begin{multicols}{2}

\lettrine[lines=2, lhang=0.33, loversize=0.25, findent=1.5em]{N}{ot} the adversary always reveals himself in front of us, often this can be hidden or even invisible.

It could be hidden behind a low wall or some barrels, if not behind a muscular and gigantic familiar.
What if he was behind us and we didn't even see him?

\subsection{The Coverage}\index{Coverage}\label{copertura}

If the target is known to exist but is cloaked in some way then it is said to have "cover".

\begin{itemize}[leftmargin=*]
\item
If the target has more than half (but not total) of the "visible" surface then the cover is defined as \textbf{light}, ie it has +2 to Defence. This may be the case of a creature behind another creature of the same size or 1 size larger.

This may be the case with an archer standing behind a 1m low wall.


\begin{center}
\includegraphics[width=0.9\linewidth]{immagini/hide.png}
\textit{British Soldiers Hiding From Boer Fire At The Battle Of Majuba Hill.}
\end{center}

\item
If the target has less than half (but at least one third) of the "visible" surface then the cover is defined as \textbf{medium}, ie it has +4 to Defence. This may be the case with a creature behind another creature 2 sizes larger.

This may be the case of an enemy armed with a crossbow who leans just enough to keep the crossbow leaning against the wall and shoot (shoulders, arms and head visible).

\item
If the target knows where it is but hides completely looking out only to check on the characters or shoot an arrow once in a while, behind a wall, window, door, table, a creature bigger than him (at least 3 sizes).. then the cover is defined as \textbf{complete}, ie it has +8 to Defence.

Clearly an opponent who doesn't know where he is cannot be hit normally...

\end{itemize}

Half the cover bonus also applies to Saving Throws against spells that have an area effect (e.g. fireballs exploding around…).

\subsection{Invisibility}\index{Invisibility} \hypertarget{invisibilita}{}\label{invisibility}

If an opponent is invisible or you don't know where he is, follow the rules of Invisibility.

\begin{center}
\includegraphics[width=0.8\linewidth]{immagini/brickwall.png}

\textit{is there anyone in front of this wall?}
\end{center}

Even if one is invisible, it does not mean that one cannot be perceived differently through other senses, such as smell, hearing or touch. Invisibility renders a creature undetectable by sight but does not itself render a creature imperceptible or immune to critical rolls or burst damage.

A blinded creature, fighting an invisible creature, or fighting in complete darkness without darkvision can make an Awarness check, 1 action, difficulty 20, or 2 actions, difficulty 15, to spot the creature if within 6 m from target.\index{Spot target}

Depending on the invisible creature's distance or what it does, there are different modifiers to the Awarness check to spot it.

\begin{changemargin}{0.3cm}{0.3cm}\begin{narrator}
Always allow the player to check by evaluating the appropriate modifiers.
\end{narrator}\end{changemargin}


\medskip

\textbf{Table: Modifiers to Awareness Check for Detect Invisible Creatures}\index{Table Modifiers to Awareness Check for Detect Invisible Creatures}

\medskip

\begin{tabularx}{0.45\textwidth}{ll}
\textbf{The Invisible Creature is...} & \textbf{Mod.}\\
\toprule
Moving & -4\\
Running or charging & -8\\
Using Stealth & check+10\\
Not moving and make no sound & +8\\
Every 1 meter beyond 6 m & +2\\
Light/Medium/Full Coverage & +4/8/12\\
\end{tabularx}

These modifiers are cumulative with each other.

If the invisible creature attacked in melee and didn't move it is considered \textbf{automatically spotted}.

If the check succeeds, the beholder has a sense that "something is there" but cannot see it or accurately target it with an attack.

Whoever attacks a creature \textbf{invisible to her but spotted} has a -1d6 on the attack roll, the creature that attacks the one who does not see her has +1d6 on the attack roll.

Attacking an undetected target means attaching a \textit{square} to the map case. Always allow attack rolls, whether there is an opponent in that square or not. If the target is in that square, modify its Defense by +8, obviously if the \textit{square} is empty the attack roll will not hit anyone.

\subsection{Notes on invisility}


If an invisible character picks up a visible item, the item remains visible. An invisible creature can pick up a small visible object and hide it on itself (putting it in a pocket or under the cloak, clenched in a fist) and effectively render it invisible.

Someone could sprinkle flour on an invisible object to at least keep track of its location (until the flour falls all the way or is blown away).

Invisible creatures leave footprints. Their tracks can be followed without any problems. Footprints in sand, mud, or other soft surfaces can give enemies indications of the invisible creature's location, making it detected.

An invisible creature in the water moves the liquid, revealing its location. The invisible creature is still difficult to hit and enjoys the benefits of medium cover (+4 to Defence).

A burning invisible torch still sheds light (as does an invisible object subject to light magic).

Invisible creatures cannot use gaze attacks. Invisibility does not affect being targeted by a divination spell.

\end{multicols}

%\vspace{4cm}

\vfill

\begin{center}
\includegraphics[keepaspectratio,width=0.75\textwidth]{immagini/impronteneve.png}

\textit{May help find an invisible wolf...}
\end{center}

\pagebreak


\section{List of Weapons by Homogeneous Type}\index{List of Weapons}\index{Homogeneous Type}\hypertarget{lista.armi}{}\label{lista.armi}

\begin{changemargin}{0.3cm}{0.3cm}\begin{emphasis}{Strength does not reside in a Sword, but in the arms of a brave man. (The Legend of Zelda: Twilight Princess)} \end{emphasis}\end{changemargin}\medskip

\begin{multicols}{2}

\lettrine[lines=2, lhang=0.33, loversize=0.25, findent=1.5em]{O}{gni} every time you assign a point to Weapon Proficiency you can decide whether to continue improving on an already known List of Weapons or learn a new one, if the use is not declared it is assigned to the Simple Weapons List.

On the sheet, note which List of Weapons you assign the Weapon Proficiency point to.

To reassign a Weapons Proficiency point to another list requires at least 4 hours of training for 4 months.

Using a weapon without the proper proficiency imposes a -1d6 on the attack roll.

All Weapon Lists grant, unless otherwise written, these cumulative benefits when the score in the Weapon List reaches the indicated value:


\begin{itemize}[leftmargin=*]\index{Common Bonuses Weapon List}

\item 6 points: If you face someone using a weapon on this list you are immediately able to understand their Weapon Proficiency ability.

\item 10 points: if you hit the same opponent with at least two attacks in the round, the second attack causes 1 critical damage if it has no generated

\item 14 points: if you hit the same opponent with at least two attacks in the round you can use an Free Action to move one meter.

\item 18 points: when you make an attack roll you also consider the 5 for the critical count (but do not reroll the die).

\item 20 points: when you make an attack roll you also consider the 5 for the critical count and reroll the die.

\end{itemize}

Points awarded in a Weapon List do not add to attack rolls! You must check the score in the Weapon List with any bonuses that the same list lists.

The bonuses indicated in the Weapon Lists apply only when fighting with the weapons indicated in the same list.

The benefits shown are cumulative unless otherwise indicated.

\subsection{Bows} \index{Bows} Long Bow, Short Bow, Long Composite Bow, Short Composite Bow\label{listaarmiarchi}



\begin{itemize}[leftmargin=*]

\item 4 points: Add the Strength value to the damage, even if the bow is not composite. On a short bow you can add up to +1 damage, on a long bow up to +2 damage.
\item 5 points: Reduce the penalty for shooting beyond standard range by 1d6.
\item 7 points: Your mastery of using the bow in combat is such that you suffer no penalty when shooting arrows at enemies in melee or light cover.
\item 9 points: the first shot you make shoots two arrows. The attack roll starts at a -5 penalty.
\item 11 points: you shoot an extra arrow with a -5 penalty on your attack roll, the penalty does not stack with the multi attack. (Attack Roll, TC-5, TC-5, TC -10...).
\item 16 points: The first arrow that hits in the round adds critical damage.

\end{itemize}

\begin{center}
\includegraphics[width=0.9\linewidth]{immagini/arma-arco.png}
\end{center}

\subsection{Light Weapons}\index{Light Weapons} Short Sword, Light Mace, Rapier, Scimitar, One-handed Axe, Dagger\label{lightweaponlist}

\begin{itemize}[leftmargin=*]

\item 4 points: You can use Dexterity instead of Strength on attack rolls.
\item 5 points: You can draw the weapon as part of the Move Action.
\item 7 points: You can draw the weapon as an Immediate Action.
\item 9 points: Increase the weapon's damage die by one step. If the damage die becomes 8 or more the weapon gains EDX on the maximum value of the die.
\item 11 points: Increase the weapon's damage die by one step. EDX is reduced by 1.
\item 16 points: Using a Reaction avoids the first melee attack of the round and can make a response attack.


\end{itemize}

\subsection{Dual Weapons} \index{Dual Weapons} Double Great Axe, Double Flail, Double-Bladed Sword, Urgrosh\label{double weapon list}

\begin{itemize}[leftmargin=*]
\item 4 points: your proficiency in the use of these weapons makes you extremely versatile, giving you the possibility at the start of your round to choose whether to be defensive or offensive, increasing your attack roll or defense by 1. It doesn't cost Actions.
\item 5 points: taking -4 to attack rolls on the first attack you make in the round gives +4 to Defense.
\item 7 points: using a non-light double weapon does not add an additional -3 to the attack roll.
\item 9 points: your technique leaves no points uncovered, for each successful attack roll in the round you get +2 to Defense.
\item 11 points: Swipe with your weapon. The first hit is equivalent to two hits.
\item 16 points: Whenever you hit with a critical roll you can deliver a blow with the other end of the weapon. This attack roll cannot itself cause critical attacks and is -1d6 to the roll.
	
\end{itemize}

\subsection{Graceful Weapons}\index{Graceful Weapons} Rapier, Scimitar, Glaive\\\label{gracious weapon list}

\begin{center}
\includegraphics[width=0.7\linewidth]{immagini/sciabole.png}
\end{center}

\begin{itemize}[leftmargin=*]
\item 4 points: your style is very similar to a dance. You can use your Charisma or Dexterity value on the attack roll.
\item 5 points: You can use your Perform score in place of Weapon Proficiency on attack rolls.
\item 7 points: you know how to hit where it really hurts. The first Critical Hit adds 2 weapon dice.
\item 9 points: The weapon die increases by one category.
\item 11 points: using a Reaction you can try to intercept your opponent's attacks. Using a Reaction adds +2 to Defense.
\item 16 points: Your dance blocks your opponent from facing you. Force the opponent in melee with you to attack you and no one else in the next round. 1 Reaction.

\end{itemize}

\subsection{Weapons of Death}\index{Weapons of Death} Light Pike, Heavy Pike, Scythe, Sickle\label{deathweaponslist}


\begin{center}
\includegraphics[width=0.7\linewidth]{immagini/scythe-types.png}

\textit{Eric Sloane. A Museum of Early American Tools.}

\end{center}

\begin{itemize}[leftmargin=*]
\item 4 points: you can perform a Coup de Grace with the cost of 1 Action.
\item 5 points: The first critical hit you land on your opponent adds an additional critical hit.
\item 7 points: Increase the weapon's damage die by one step.
\item 9 points: The first critical hit causes 2 weapon dice of damage.
\item 11 points: Increase the weapon's damage die by one step.
\item 16 points: Increase the weapon's damage die by one step.

\end{itemize}

\subsection{Stun weapons}\index{Stun weapons} Empty Fist, Truncheon, Spiked Gauntlet\label{stun weapon list}

\begin{itemize}[leftmargin=*]
\item 4 points: An unaware opponent if hit with these weapons (during the surprise round) must make a DC 15 Fortitude saving throw or be Slowed 1/1r.
\item 5 points: If you cause critical damage the opponent must make a Fortitude save at DC 13 or be weakened 1/1r.
\item 7 points: You double your Strength damage bonus. The 4-point skill's saving throw becomes 19.
\item 9 points: If you cause critical damage the opponent must make a Fortitude save at DC 17 or be weakened 1/1r.
\item 11 points: Your stun weapon does 1d6 more nonlethal damage. The skill's saving throw at 4 and 9 points becomes 23
\item 16 points: Whenever you hit an opponent with critical damage, a teammate in melee with that opponent can use a Reaction to make an attack against him.

\end{itemize}

\subsection{Thrownable Weapons} One-handed Axe, Javelin, Trident. Slingshot, Dagger\index{Thrownable weapons}\label{listarmarrow}

You gain the \textbf{Devastating Shot} ability: you can throw one of your weapons with such force that it does two critical damage but your accuracy suffers -1d6 on the attack roll. It costs 2 Actions.



\begin{itemize}[leftmargin=*]
\item 4 points: you have become extremely precise in throwing your weapon, you have a +1 to hit and a +1 to damage.
\item 5 points: The first critical hit you land on your opponent adds an additional critical hit.
\item 7 points: your skill allows you to have no downtime after throwing a weapon you can instantly draw another one without consuming actions.
\item 9 points: the first attack roll throws 2 weapons. You start with a -5 to attack roll.
\item 11 points: You have a -1d6 range penalty beyond the standard.

\item 16 points: you have become extremely precise in throwing your weapon, you have +4 to hit and +4 to damage.


\end{itemize}

\subsection{Lethal weapons} Katana, Machete\index{Lethal weapons}\label{lethallist}

\begin{center}
\includegraphics[width=0.6\linewidth]{immagini/katana3.png}

\textit{Katana}
\end{center}

\begin{itemize}[leftmargin=*]

\item 4 points: against surprised opponents add your Weapon Expertise to the damage.
\item 5 points: The first critical hit you land on your opponent adds an additional critical hit.
\item 7 points: Increase the weapon's damage die by one step. If this causes the weapon to have the d8 as damage die, it also acquires EDX equal to 8.
\item 9 points: The first critical hit you land on your opponent adds two additional critical hits.
\item 11 points: EDX earnings. It is applied only by doing the maximum damage with the die, if the weapon already has an EDX (for example because with the previous bonus it reached 1d8 damage) this decreases by 1.

\item 16 points: Increase the weapon's damage die by one step.
\end{itemize}



\subsection{Polearms} \index{Polearms}Javelin, Estoc, Trident, Halberd\label{armiaste list}

\begin{center}
	\includegraphics[width=0.8\linewidth]{immagini/alabarda2.png}
	
	\textit{Halberds}
\end{center}

\begin{itemize}[leftmargin=*]

\item 4 points: if you make at least one critical roll with the attack roll you can leave the weapon in the opponent's body, penalizing him with -1 Dexterity. The weapon when removed deals critical damage.
\item 5 points: You can make an attack of opportunity against opponents who cross your melee zone.
\item 7 points: You can use your long weapon in melee within one meter without penalty. The skill's damage at 4 points becomes 2 critical damage.
\item 9 points: The 4-point skill's damage becomes 4 critical damage.
\item 11 points: the range if absent becomes 3 meters, if present double it.
\item 16 points: Using a Reaction you can follow your opponent while maintaining your current melee distance.

\end{itemize}


\subsection{Crossbows}\index{Crossbows}Light crossbow, Heavy crossbow, One-handed crossbow\label{listarmibalestr}


\begin{center}
\includegraphics[width=0.9\linewidth]{immagini/arma-balestra.png}
\end{center}


\begin{itemize}[leftmargin=*]

\item 4 points: Gain the Rapid Shot Skill.
\item 5 points: The first critical hit you land on your opponent adds an additional critical hit.
\item 7 points: each Action you dedicate to aiming, up to a maximum of 2, grants you +2 to hit.
\item 9 points: the first critical hit you land on your opponent adds two additional critical hits, it does not stack with the advantage at point 4
\item 11 points: Your first hit against an opponent adds an additional critical hit.
\item 16 points: Reduce the penalty for shooting beyond standard range by 1d6.

\end{itemize}


\subsection{Lance} \index{Lance}Halberd, Urgrosh, Infantry Lance, Naginata, Spear Glaive, Spear, Brandistocco\label{listarmilance}


\begin{center}
	\includegraphics[width=0.85\linewidth]{immagini/arma-asta.png}
	
	\textit{1 Landsknecht skewer; 2 Pike; 3 Spear; 4 Hunting skewer; 5 Buttfire; 6 Glaive; 7 Partisan; 8 Halberd; 9 Halberd; 10 Roncone; 11 Mazzapicchio; 12 Berdica}
\end{center}

\begin{itemize}[leftmargin=*]
\item 4 points: Used against a charge or while charging, as long as it has the Counter Charge ability, you deal additional critical damage.
\item 5 points: you can also use it against opponents within 1 meter without penalty.
\item 7 points: used against a charge or while charging, as long as it has the Counter Charge ability, you deal two additional critical damage
\item 9 points: You use 3 Actions. Make an attack roll at -5 and compare the result to the Defense of all creatures in melee to see if you hit them.
\item 11 points: Your spear's range becomes 10 feet.
\item 16 points: You use 3 Actions and make a single attack roll. If it hits, you deal 3 additional critical hits.
\end{itemize}

\subsection{Spinballs} Flail, Heavy Flail, Double Flail, Spiked Chain, Whip\label{list of Spinning Balls}


\begin{center}
	\includegraphics[width=0.6\linewidth]{immagini/mazzafrusto.png}
\end{center}

\begin{itemize}[leftmargin=*]
\item 4 points: if the attack roll is successful you can make a further CT (without consuming Actions) at -5 against an opponent in melee with you who is not the opponent already hit.
\item 5 points: If you hit your opponent twice, the second attack roll generates additional critical damage.
\item 7 points: the impact of your shots is enough to stun enemies. If you hit your opponent with a critical hit they will suffer -4 Defense until the end of the next round.
\item 9 points: You can use a reaction and use your weapon to try to deflect an attack roll at another teammate within melee range of you. The companion gets +2 to Defense
\item 11 points: the precision and skill in swinging your weapon is such that it confuses the enemy's defense, you ignore the protection (Defense) given by the shield.
\item 16 points: You can use a reaction and use your weapon to try to deflect an attack roll at another teammate within melee range of you. The companion gets +4 to Defense.
\end{itemize}


\subsection{Empty Fist} Punches and Kicks\index{Empty Fist}\hypertarget{empty fist}{}\label{listarmibarefist}

You have trained your body to become the ultimate weapon. You are trained to use kicks and punches effectively and lethally.

The Empty Fist List does not benefit from the Critical Hit, except for the advantage taken at 9 points.

\textbf{Empty Fist}: Each time you take this skill the damage increases following this progression: 1d6 (list taken 2 times), 1d8 (3), 2d6 (5), 2d8 (7), 2d10 (9), 3d6 ( 11), 3d8 (13), 3d10 (15), 4d6 (17).

The player can also decide to do non-lethal damage without incurring any penalties; he can apply the Strength or Dexterity value to the damage as he wishes.

\begin{itemize}[leftmargin=*]
	\item 1 point: your fists deal lethal damage (1d4). You can use your Strength or Dexterity value on attack rolls and damage rolls.
	\item 4 points: Wisdom of the Empty Hand. You can use Wisdom on Hit and Damage in place of Strength or Dexterity. Multiple attack penalties become -4 instead of -5.
	\item 5 points: Base Defense score goes from 10 to 11.
	\item 9 points: solitary strike. You use three Actions to deliver a single devastating blow, if the blow hits, it adds 2 additional critical hits.
	\item 11 points: You gain a bonus to hit and damage equal to double the characteristic used to determine this bonus.
\end{itemize} 

See \hyperlink{magical equivalences}{Vulnerability, Resistance and Immunity} to find out how magical your strike is.

\subsection{Skull Breaker} \index{Skull Breaker} Flail, Big Club, War Maul, War Hammer, Light Mace, Heavy Mace, Spiked Mace, Club
\label{list of skull-breaking weapons}

\begin{itemize}[leftmargin=*]
\item 4 points: you have become so skilled that you can control the force of your blows, you can deal non-lethal damage without penalty to hit (otherwise -1d6 on attack roll).

You can choose to reduce the attack roll by 4 to increase the damage by 8 (does not stack with Power Strikes).
\item 5 points: The first critical hit you land on your opponent adds an additional critical hit.
\item 7 points: your shots daze the enemy. Each successful critical hit lowers Defense by 1 point. Duration 1 minute starting from the first successful critical hit.
\item 9 points: Increase the weapon's damage die by one step.
\item 11 points: The first critical hit you land on your opponent adds two additional critical hits. It does not cumulate with the advantage in point 5.
\item 16 points: using a Reaction every time you hit with a critical roll you can make another attack roll with the same score against a different opponent as long as it is within melee range.

\begin{center}
	\includegraphics[width=0.9\linewidth]{immagini/arma-mazza3.png}
\end{center}

\end{itemize}

\subsection{Shields}\index{Shields} Light, Medium, Heavy Shields\label{listarmiscudi}

You are a master in the use of shields, even as a weapon.

You can use the shield as a weapon, a small shield does 1d4 damage (B/T), a medium shield does 1d6 damage (B/T), a heavy shield does 1d8 damage (B/T).
You have no penalty for hitting with the shield, for you the shield is not an improvised weapon. This Weapon List does not have the 6 point bonus and the 18 point bonus common to other Weapon Lists.

Your technique effectively mixes defense and attack. You can throw your shield with a range of 20 feet.


\begin{itemize}[leftmargin=*]
\item 1 point: you are proficient with all shield types. You are not constrained by the Strength 1 limit on Heavy Shields.
\item 2 points: the Defense bonus when you raise your shield increases by 1 and every 4 times you take the proficiency. Using the shield as a weapon does not cause you to lose the Defense bonus given by the shield.
\item 3 points: the Magical Proficiency penalty given by the shield decreases by one die and every 4 times you take the proficiency.
\item 4 points: the Weapon Proficiency penalty decreases by 1



\begin{center}
\includegraphics[width=0.9\linewidth]{immagini/scudotorre.png}
\textit{Henry Justice Ford. Heavy Shield}
\end{center}

\item 5 points: increases the damage category of the shield by 1 (1d4 - 1d6 - 1d8 - 1d10 - 2d6 - 2d8 - 2d10) and every 4 additional points in the list (9,13,17..). 
\item 8 points: Each ally adjacent (within 1 meter) to you has +1 Defense. You can throw your shield to defend a teammate by granting them +2 Defense, to use as a reaction. The shield falls to the ground where you defended your partner. You can throw your shield with a range of 30 feet. The Magical Proficiency penalty given by the shield decreases by one die.
\item 12 points: you can throw your shield as if it were a weapon with a range of 12 meters. If you get a critical roll on the shield throw once it hits it returns to your hands at the end of the round. Each ally adjacent (within 1 meter) to you has +2 Defense
\item 16 points: If an opponent makes at least your attack rolls and misses, you gain a shield attack against them in the round. Costs 1 Reaction.
\item 18 points: The thrown shield has a range of 18 meters and returns to your hands unless prevented. This allows you to make multiple attacks even from throwing with the same shield. You can throw your shield to defend a teammate by granting them +4 Defense, to use as a reaction. The shield falls to the ground where you defended your partner.

These bonuses cannot be applied if you use more than one shield.


\end{itemize}

\subsection{Armors}\index{Armor List} \label{Armor List}

This List only grants the cumulative bonuses listed here when armor is donned.

\begin{itemize}[leftmargin=*]
	\item 1 point: Halve the time it takes to don and undon armor
	\item 2 points: the Defense granted by the armor increases by 1 point, sleeping in medium armor does not cause fatigue
	\item 3 points: Proficiency Penalty decreases by 1 point, sleeping in heavy armor does not cause fatigue
	\item 4 points: Movement penalty decreases by 1 meter, armor Defense increases by 1 point
	\item 5 points: Magic Test penalty dice decrease by 1 die, Expertise Penalty decreases by 1 point, Movement penalty decreases by 1 meter
	\item 6 points: You negate the additional damage of the first Critical Roll on every attack, melee or ranged. Wearing armor no longer forces you to take the Magic Test
	\item 7 points: you have no Competence Penalty and Movement penalty
	\end{itemize}
	
\subsection{Axes and Hatchets}\index{Axes and Hatchets} One-handed Axe, Battle Axe, Hammer Axe, Great Double Axe\label{axe list}

Call \textit{Axes} in Table: Weapon List

\begin{center}
\includegraphics[width=0.9\linewidth]{immagini/scurieaccette.png}
\end{center}

\begin{itemize}[leftmargin=*]

\item 4 points: The fury of your attacks is such that you gain +2 to on-hit damage.
\item 5 points: The first critical hit you land on your opponent adds an additional critical hit.
\item 7 points: the wounds you cause are so deep that they cause Bleeding. Each successful attack increases your bleed by 1, up to a maximum of bleed 5.
\item 9 points: each critical hit you cause increases the Bleeding by 2, up to a maximum of 10.
\item 11 points: the wounds you cause are so deep that you cause a lot of bleeding. Not the maximum Bleeding value increases to 15.
\item 16 points: you consume 3 Actions, you make a single attack roll that you compare against all creatures in a cone equal to your movement to see if you hit them.

\end{itemize}

\subsection{Swords}\index{Swords} Short Sword, Long Sword, Two-Handed Greatsword, Bastard Sword, Two-Bladed Sword, Broad Sword, Katana, Two-Bladed Sword


\begin{center}
	\includegraphics[width=0.95\linewidth]{immagini/arma-tipi-di-spade.png}
	
	\textit{A Saber, B Scimitar, C One-handed sword, D Broad sword, D Rapier, E Long sword, F One-and-a-half-handed or bastard sword, G Two-handed broadsword}
\end{center}

\begin{itemize}[leftmargin=*]

\item 4 points: Your mastery of the sword technique gives you +1 to damage and attack rolls.
\item 5 points: The first critical hit you land on your opponent adds an additional critical hit.
\item 7 points: Your mastery of the sword technique gives you +2 to damage and attack rolls.
\item 9 points: The first successful hit in the round adds a critical hit.
\item 12 points: you have reached the pinnacle of swordsmanship your strikes are precise and difficult to predict you gain +1 to damage, attack roll and defense. The EDX of the sword if present is lowered by 1.
\item 16 points: Your mastery of the sword technique gives you +3 to damage and attack rolls.

The hand not holding the sword must be free.

\end{itemize}

\subsection{Swords and Shields}\index{Swords and Shields} Short Sword, Long Sword, Large Sword, Small Shield, Medium Shield\label{listarmispadescudi}

\begin{itemize}[leftmargin=*]

\item 4 points: Your mastery of the sword and shield technique gives you +1 to Defense and Attack Rolls.
\item 5 points: if you hit two consecutive hits with the sword you can make an attack roll, without accumulating further multiattack penalties, with the shield by consuming a reaction.
\item 7 points: Your mastery of the sword and shield technique gives you +2 to Defense and Attack Rolls.
\item 9 points: By using a Reaction you can use your shield to protect a creature in melee with you. His Defense increases by 2 points.
\item 11 points: you have reached the pinnacle of mastery with the sword and shield, your ability gives you +2 to Defense and Attack Roll.
\item 16 points: Add the Shield Defense value to the Reflex saving throws.

\end{itemize}

The character must hold the sword in one hand and the shield in the other.


\subsection{Simple Weapons} Dagger, Light Mace, Club, Spiked Mace, Staff, Crossbow (Light), Javelin.\index{Simple Weapons}\hypertarget{simple.weapons}{}\label{simpleweaponslist}

\medskip

This subdivision can also be chosen by those who have not assigned points to Weapons Proficiency. This Weapon List does not grant specific bonuses.

\subsection{More weapons Weapons List}\index{More weapons Weapons List}\label{listweaponsinmultilists}

When a character uses a weapon present in multiple known Weapon Lists, he can apply only one combat technique (one Weapon List) per opponent, he does not accumulate the advantages of any other lists.

By using 2 Actions he can concentrate and move on to using the bonuses resulting from the application of a different Weapon List. 

\end{multicols}


\vfill


\begin{center}
\includegraphics[width=0.45\linewidth]{immagini/brancastle.png}

\textit{Bran Castle, Transylvania}
\end{center}

\pagebreak

\subsection{Optional - Weapon Maneuver List}\hypertarget{Weapon Talent List}{}\label{Weapon Talent List}\index{Weapon Maneuver List}

\begin{changemargin}{0.3cm}{0.3cm}\begin{emphasis}{
Honesty and Justice, Heroic Courage, Compassion, Kind Courtesy, Complete Sincerity, Honor, Duty and Loyalty (The Seven Principles of Bushido)
}\end{emphasis}\end{changemargin}

\begin{multicols}{2}

The more competent the character becomes with weapons, the more he is able to exploit attack opportunities and carry out weapon maneuvers. Whenever the character makes at least two weapon attacks in the round and \textbf{neither of them hits}, it is possible to consult the Weapon Maneuvers list to understand which maneuver can be used.

Each Maneuver has indicated which situation activates it (Activator) and which is the Effect. A Critical Effect can also be indicated, i.e. the Effect that occurs when a critical failure is obtained in at least one attack roll. As long as the Trigger is always respected, the player can choose between the Effect and the Critical Effect.

The Trigger can specify an odd or even value that is compared to the attack roll.

The Weapon Maneuvers are grouped by level, i.e. the minimum Weapon Proficiency score to be able to use those manoeuvres, the player can choose from all the Weapon Maneuvers accessible to him and which can be activated.


\textbf{Weapon Maneuvers level 6}

\medskip

Name: \textbf{Missed Opportunity}\\
Trigger: You missed\\
\textit{Effect}: You can drink a potion held on your belt\\
\textit{Critical Effect}: the potion can be administered to a companion within melee range.

\smallskip

Name: \textbf{Real fake}\\
Trigger: You rolled a draw\\
\textit{Effect}: you didn't want to miss it but at least it was a feint. Until the end of the next round, add your Intelligence or Wisdom score to your Defense against the same opponent\\
\textit{Critical Effect}: on your next attack add your Intelligence and Wisdom scores to your attack roll against the same opponent

\smallskip

Name: \textbf{Smart Shot}\\
Trigger: You rolled a draw\\
\textit{Effect}: the companion at your side who fights against your opponent gains a bonus to the attack roll equal to your Intelligence by the end of the next round\\
\textit{Critical Effect}: as above but also applies to an additional companion.

\smallskip

Name: \textbf{Continue minor}\\
Trigger: you rolled an odd\\
\textit{Effect}: Compare the attack roll of your last attack with a creature within melee range of you, if it is enough to hit it you deal damage equal to your Strength\\
\textit{Critical Effect}: compare the attack roll of your last attack with a creature within melee range of you, if it is enough to hit it you deal damage equal to twice your Strength

\smallskip

Name: \textbf{Distracted by noise}\\
Trigger: you rolled an odd\\
\textit{Effect}: the clangor of battle distracts an opponent, choose a grabbed comrade, he frees himself if he is Grabbed or Blocked\\
\ textit {Critical Effect}: as above, he can also take a step (1 meter) in the direction he prefers \\


\textbf{Level 8 Weapon Maneuvers}

\medskip

Name: \textbf{Deep Breath}\\
Trigger: You missed\\
\textit{Effect}: You focus too much and miss the opportunity to hit, but you gain +2 to hit on all melee attacks by the end of the next round\\
\textit{Critical Effect}: as above and if you hit you get additional critical damage

\smallskip

Name: \textbf{Unbalanced}\\
Trigger: You missed with an odd\\
\textit{Effect}: Your trip caused you to miss but you gain +2 to Defense until the end of the next round\\
\textit{Critical Effect}: next round you can only use 2 actions, but the first melee attack you make automatically misses

\smallskip

Name: \textbf{Perplexed}\\
Trigger: You missed with an odd\\
\textit{Effect}: you can't decide how and where to hit him. You make a knowledge check to better understand your opponent\\
\textit{Critical Effect}: even one of your companions who fights against the same creature gets the chance to make the same check

\smallskip

Name: \textbf{Misstep}\\
Trigger: You missed with a tie\\
\textit{Effect}: you made a mistake. By the end of the next round the terrain is considered difficult, you have +4 to hit\\
\textit{Critical Effect}: You cannot move by the end of the next round. If you hit you get two critical damage

\smallskip

Name: \textbf{I missed!!!}\\
Trigger: You missed with a tie\\
\textit{Effect}: It was all a ploy, you missed on purpose. By the end of the next round you can make one additional attack without stacking multiple attack penalties.\\
\textit{Critical Effect}: check your attack roll with that of another melee opponent, if you hit you also cause critical damage

\textbf{Weapon Maneuvers level 10}

\medskip

Name: \textbf{On the side}\\
Trigger: You missed\\
\textit{Effect}: You moved to your opponent's flank. Move one meter around the opponent\\
\textit{Critical Effect}: move up to 3 meters around the opponent, by the end of the next round perform only 2 Actions

\smallskip

Name: \textbf{Opening}\\
Trigger: You missed with an odd\\
\textit{Effect}: You missed to allow a teammate to hit better. A companion who attacks your opponent gains +4 to attack rolls by the end of the next round. \\
\textit{Critical Effect}: two companions gain the opening described in Effect and until the end of the next round you have -4 to attack rolls

\smallskip

Name: \textbf{Swagger}\\
Trigger: You missed with an odd\\
\textit{Effect}: The opponent is intimidated by your combat mastery. By the end of his next round the first attack he makes gets a -1d6 penalty\\
\textit{Critical Effect}: as above but -2d6, by the end of the next round you have -4 to attack rolls

\smallskip

Name: \textbf{Smooth}\\
Trigger: You missed with a tie\\
\textit{Effect}: You missed just enough, but it was enough to hurt your opponent. Opponent increases Bleeding rating by 1\\
\textit{Critical Effect}: as above but Bleeding is 2, you hit yourself with the weapon and cause damage equal to your Strength

\smallskip

Name: \textbf{Testing the forces}\\
Trigger: You missed with a tie\\
\textit{Effect}: You preferred to evaluate your opponent's capabilities. The first attack roll that hits by the end of the next round automatically deals 1 critical damage\\
\textit{Critical Effect}: as above but 2 critical damage, until the end of the next round you can only perform 2 Actions\\

\smallskip

\textbf{Weapon Maneuvers level 12}

\medskip

Name: \textbf{Tenacious}\\
Trigger: You missed\\
\textit{Effect}: you don't give up and you insist. Each attack you make against this opponent by the end of the next round gains a cumulative +2 to hit until you miss\\
\textit{Critical Effect}: Each attack that hits by the end of the next round you make against this opponent gains cumulative critical damage, you have -2 to attack rolls

\smallskip

Name: \textbf{Persevere}\\
Trigger: You missed with an odd\\
\textit{Effect}: The first attack within the next round will automatically miss, the next attack will cause 2 additional critical damage if it hits\\
\textit{Critical Effect}: As above, but 3 critical damage

\smallskip

Name: \textbf{Battlecry}\\
Trigger: You missed with an odd\\
\textit{Effect}: you missed, it's true, but you took the opportunity to encourage your teammates. By the end of the round all your companions gain +1d6 to attack rolls on their first attack.\\
\textit{Critical Effect}: as above but +2d6, but take one less Action in the next round

\smallskip

Name: \textbf{Eureka}\\
Trigger: You missed with a tie\\
\textit{Effect}: the blow was used to understand how to hit him. The first successful attack by the end of the next round deals critical damage with the weapon die maximized\\
\textit{Critical Effect}: you can only perform 2 actions, until the end of the next round, you and a companion you draw on the first attack, if successful, add 1 critical damage

\smallskip

Name: \textbf{Savage Attack}\\
Trigger: You missed with a tie\\
\textit{Effect}: Your fury is such that you hit something anyway. Compare your attack roll with a creature within melee range of you\\
\textit{Critical Effect}: until the end of the next round you have -4 to Defense, but you gain +1d6 on attack rolls and critical damage if you hit

\end{multicols}


\begin{changemargin}{0.3cm}{0.3cm}\begin{narrator}
Players in agreement with the Storyteller can create their own personalized Weapon Maneuvers based on the character's story and style.
\end{narrator}\end{changemargin}

\vfill

\begin{center}
%\includegraphics[width=0.12\linewidth,angle=90]{images/Bushido_Calligraphy.png}
\includegraphics[width=0.085\linewidth]{immagini/Bushido_Calligraphy.png}

\textit{Kanji transcription of} bushido
\end{center}



\pagebreak

\section{Feats}\index{Feats}\hypertarget{abilita}{}\label{abilita}

\begin{changemargin}{0.3cm}{0.3cm}\begin{emphasis}{Martyrdom is the only way for a man to become famous if he has no ability (George Bernard Shaw, The Devil's Disciple)} \end{emphasis}\end{changemargin}\medskip

\begin{multicols}{2}

\lettrine[lines=2, lhang=0.33, loversize=0.25, findent=1.5em]{F}{eats} are peculiar abilities, the result of training or particular talents. Feat always have a practical effect.

Feat make up a good part of what the character can do, they must be chosen with care and attention. It is by choosing the Feats that the style and capabilities of the character are established, whether you want him to be more of a warrior or a wizard or a healer… or any combination and peculiarity.

\textbf{At first level you get two Feats}. Subsequently, a Feat is taken at all levels except 5,10,15,20. this can be the same one already taken or a new Feat learned during adventures.

It is possible that Requirements are indicated under the name of the Feat, in this case they must be met to take the Feat in question.
Any subsequent requirements are indicated from time to time.

Don't take Feats based on power, strength, combination with others but because they are in line with the story of the character.
Picking a jumble of Feats just because they're strong doesn't make a character powerful but unbalanced, don't be a power-player at any cost.

\medskip

\textbf{Feats must be taken based on the evolutionary path of the character, based on what he experienced and learned during the adventures.}

\medskip

It is possible to change a chosen Feat, still respecting the requirements, by retraining for at least 4 months for 4 hours a day.

The abilities provided by Feats unless otherwise described are cumulative or if it is the same bonus, the greater one applies. Unless explicitly stated, a Feat cannot be taken multiple times.

\subsection{Saving Throws and Abilities}\label{tirisalvezzaedabilita}

Each Feat, except those that directly modify Saving Throws, grants bonuses to Saving Throws that stack between Feats, even when you take the same Feat multiple times.

When choosing an Feat, also note which Saving Throws it raises!

\subsection{Add New Feats}\label{aggiungereabilita}

This list can never be exhaustive given the imagination of the players! But try to figure out if what the player wants is a Feat or Competence.
Carefully evaluate the prerequisites and the advantages it grants, always try to be balanced, rather grant advantages to scale, or by taking the Feat several times.

Also remember to mark the bonuses related to the Saving Throws. Usually a concrete and practical Feat grants a bonus of +3 divided between 2 Saving Throws, a more generic Feat grants 2 points to be divided between a single Saving Throw or two.

\subsection*{Adept of Magic}\index{Adept of Magic}\hypertarget{scuoladimagia}{}\label{adeptodellamagia}

\textbf{Requirement}: Magic Proficiency 1

\textbf{Saving Throws}: +1 on two Saving Throws of your choice.

It is only through this Feat that you can access a Magic List.

By taking this Feat several times and always selecting the same School, it is possible to access the higher levels of the spell.

A spellcaster can take the Magic Adept Feat several times and apply it to a new Magic List or to an already known one.

\subsection*{Animalia}\index{Animalia}\label{amimalia}

\textbf{Requirement}: Follower or Devotee of Ephrem or Shayalia, Magical Proficiency 2.

\textbf{Saving Throws}: +2 Will, +1 Fortitude

You gain the ability to transform into a known creature. Cost 2 Actions.

Your healing spells also work on normal and magical Animals or Plants.\\

The \textbf{first time} you take this feat you can transform into a creature with these characteristics:\\

\textbf{Type of Creatures}: Beasts

\textbf{Characteristics}: the physical ones, Defense, Saving Throws and attack forms are of the animal.

\textbf{Spells}: You cannot cast spells in the new form.

\textbf{Equipment}: Equipment is absorbed into the new form but none have any effect.\\

The \textbf{second time} you take this feat you can also transform into a creature with these characteristics:\\

\textbf{Requirement}: Magical Proficiency 4

\textbf{Type of Creatures}: Plants and Slimes

\textbf{Characteristics}: the Character chooses whether the Physical Characteristics, Defense, Saving Throws are his own or those of the animal.

\textbf{Spells}: You cannot cast spells in the new form

\textbf{Equipment}: Equipment is absorbed into the new form. The magic one has no effect. Armor and Shields apply the magical bonus to the creature's Defense. Items' spell-like abilities cannot be activated.\\

The \textbf{third time} you take this feat you can also transform into a creature with these characteristics:\\

\textbf{Requirement}: Magical Proficiency 10

\textbf{Creature Type}: Elementals

\textbf{Characteristics}: the Character chooses whether the Physical Characteristics, Defense, Saving Throws are their own or those of the animal. 

\textbf{Spells}: You can cast spells in the new form as long as they have only Verbal components

\textbf{Equipment}: Equipment is absorbed into the new form. The magical one continues to have effect if possible. Armor and Shields apply the magical bonus to the creature's Defense, and item spell-like abilities cannot be activated.\\

The \textbf{fourth time} you take this feat you can also transform into a creature with these characteristics:\\

\textbf{Requirement}: Magical Proficiency 16

\textbf{Type of Creatures}: Monstrosities

\textbf{Characteristics}: the Character chooses whether the Physical Characteristics, Defense, Saving Throws are their own or those of the animal. The hit points remain those of the character.

\textbf{Spells}: you can cast spells in the new form as long as they have only Verbal and Somatic components

\textbf{Equipment}: Equipment is absorbed into the new form. The magical one continues to have effect if possible. Armor and Shields apply the magical bonus to the creature's Defense and any magical abilities can be activated.\\

\textbf{Basic rules for transformation}\\

The creature you transform into must have a \textbf{Challenge Rating} within one-third of your Magical Expertise score + the number of times you took the Animalia feat.

The \textbf{Hit Points} remain those of the character.

You can \textbf{remain transformed} for 1 minute per sum of Trait in common with the Patron, with minimum use of 1 minute.

It costs 2 Actions to change shape and before switching from one form to another it is necessary to return to normal form.

The character retains his own Traits, personality, Skills (but the new form does not necessarily allow him to use them) and mental characteristics.

If the creature has a proficiency that the character also has and the creature's bonus is higher than the character's, then use the creature's bonus instead of your own. If the creature has additional or lair actions, the character cannot use them.

Any actions requiring his hands are limited to the capabilities of his new form. The transformation does not interrupt the character's concentration on a spell he has already cast and does not prevent him from performing actions that are part of a spell already cast, such as Call Lightning.

The attack forms are always those of the creatures.

He acquires the characteristics and abilities of the new form, such as senses, movement, languages ​​(but it is not certain that he can speak other languages ​​besides that of the animal).

When transformed you can channel your Magic Points to improve the transformation, for each Magic Point consumed in the round you get a +1 to Attack Rolls, damage with attacks, Defense and Saving Throws. The ability must be declared at the start of the round as an Immediate Action that lasts until the start of your next round.



\begin{center}
	\includegraphics[width=0.9\linewidth]{immagini/animalia3.png}
	\textit{Henry Justice Ford}
\end{center}

\subsection*{Armour of the Devout}\index{Armour of the Devout}\label{armaturadeldoveto}

\textbf{Requirement}: Traits in common 2 (sum of Traits in common with Patron), being a Follower or Devote

\textbf{Saving Throws}: +2 Will, +1 Reflex

Constant training with your Armour allows you to don light Armour without having to make a Magic Test.

The \textbf{second time} you take this Feat, requirement total Traits in common 6, perform the Magic Test with no additional dice given by medium Armour.

The \textbf{third time} you take this Feat, requirement sum Traits in common 8, perform the Magic Test with only 1 additional die given by heavy Armour.

The \textbf{fourth time} you take the Feat, requirement sum Traits in common with the Patron 12, perform the Magic Test with no additional dice given by heavy Armour.

\subsection*{Blade Dance}\index{Blade Dance}\label{danzadellalama}

\textbf{Requirement}: Weapon List: Graceful Weapons at 2, Dexterity or Charisma 1, Perform 1

\textbf{Saving Throws}: +2 Reflexes, +1 Fortitude

With Graceful Weapons you can replace only the damage dealt by Strength in melee attacks with half the value of Charisma or Dexterity.

The \textbf{second time}, requirement Weapons Grace 4, Perform 3, that you take the Skill you can use Charisma as a weapon damage modifier, ignoring Strength damage.

The \textbf{third time}, requirement Weapons Grace 7, Perform 5, that you take the Skill you can use Dexterity or Charisma as a weapon damage modifier, ignoring Strength damage.


The second and third benefits are not cumulative.

\subsection*{Black siphon}\index{Black siphon}\label{Black siphon}

\textbf{Requirement}: Magic Proficiency 6, Adept of Tazher,sum of Traits in common 6

\textbf{Saving Throws}: +1 Fortitude, +2 Will

By increasing by half, rounded up, the Magic Points used in the spell, which must be instantaneous as an effect and cause damage in Hit Points, you recover an amount of Hit Points equal to half of the creature that lost the most.

The casting of the spell if of 2 or less Actions becomes 1 round.

\subsection*{Blindfight}\index{Blindfight}\label{combattereallacieca}

It is the ability to attack opponents who are not clearly perceptible.

\textbf{Requirement}: Awareness 2

\textbf{Saving Throws}: +2 Reflexes, +1 Will

An opponent with light cover gets no bonus to Defense, with medium cover he has a +2 to Defense, with full cover he has a +6 to Defense.

An invisible melee attacker gains no advantage when hitting the character in melee.

The \textbf{second time} you take the Skill, Awareness requirement at 3, reduces the Defense bonus from creatures with full cover by an additional two.

You do not need to make Acrobatics checks to move at full speed while blinded.

The penalty to attack rolls against invisible creatures is -2.

\textit{Zatoichi level}, the \textbf{third time} you take the Skill, Awareness requirement at 5, in melee an invisible creature has no advantage against you nor do you have a penalty against it.

\subsection*{Brewing potions}\index{Brewing potions}\label{distillarepozioni}

\textbf{Requirement}: Magic Proficiency 1

\textbf{Saving Throws}: +1 Fortitude, +1 Will

Proficiency in brewing potions.

You gain a +1d6 bonus on Knowledge Herbalism and in brewing and creating natural potions and poisons.


\subsection*{Called Arrow, Delivered Arrow}\index{Called Arrow, Delivered Arrow}\label{frecciachiamata}

\textbf{Requirement}: Weapon Proficiency 2

\textbf{Saving Throws}: +2 Reflexes

You can shoot 1 arrow, once per day, as a reaction, with no hit penalty from multiattack. The bow must be already draw.

\begin{center}
	\includegraphics[width=0.8\linewidth]{immagini/kameame.png}
	\textit{Kamehameha!}
\end{center}


\subsection*{Cleave}\index{Cleave}\label{proseguire}

\textbf{Requirement}: Weapon Proficiency 1

\textbf{Saving Throws}: +1 Fortitude, +1 Will

If you kill the opponent with your last Attack Action, in melee, you can perform a bonus attack action with the same modifiers as the last Attack Action performed and attack another enemy as long as it is in melee range, if you kill this creature with on one hit, you can't make other attacks on other creatures.

The \textbf{second time} you take this Feat, Weapon Proficiency 6, if you kill the opponent with your last Attack Action, you can perform a bonus attack action with the same modifiers as the last Attack Action made with the weapon and attack another enemy within 1 meter. If you kill it you can continue with a further bonus attack (and move within 1 meter) with the next creature and so on.

Each bonus attack beyond the first has a -2 to hit and a -1 to cumulative damage.


\subsection*{Clinical Eye}\index{Clinical Eye}\label{occhioclinico}

\textbf{Requirement}: Weapon Proficiency 3

\textbf{Saving Throws}: +2 Reflexes

You are able to deal critical damage to creatures normally immune to crit (roll 6 multiple times, burst damage).

\subsection*{Combat Enchanter}\index{War Enchanter}\label{incantatoredacombattimento}

\textbf{Requirement}: Magic Proficiency 1

\textbf{Saving Throws}: +1 Fortitude, +1 Will

When you are Distracted you can roll one fewer die on the Magic Test.

The \textbf{second time}, Magical Proficiency requirement 6, that you take this feat when you are Distracted you can roll one fewer die on the Magic Test.

The \textbf{third time}, Magical Proficiency requirement 8, that you take this feat you can ignore a rolled Magic Test die.



\subsection*{Concentrate}\index{Concentrate}\label{concentratp}

\textbf{Requirement}: Magic Proficiency 2

\textbf{Saving Throws}: +1 Fortitude, +1 Will

Choose a Magic List, the DC of Saving Throws for your spells in that list increase by 1.

The Feat can be taken multiple times on the same Magic List or on other lists as long as the time you take this Feat is less than MP/4.


\subsection*{Coordinated Damage}\index{Coordinated Damage}\label{dannocoordinato}

\textbf{Requirement}: Weapon Proficiency 8, Wisdom 2

\textbf{Saving Throws}: +2 Will

Your experience in managing troops allows you to maximize the effectiveness of attacks.

You can coordinate the attacks of two of your companions, who are within melee range of each other, so that the damage caused by one hits the other's enemy and vice versa. It costs 2 Actions to perform this coordination.

The \textbf{second time} you take this Skill, requirement Weapon Proficiency 4, Intelligence 2, you can coordinate and exchange the damage of three companions as long as they are within melee distance of each other. Cost 2 Actions.

The attack roll must hit an opponent in order to cause damage to another.

\subsection*{Create Magic Item}\index{Create Magic Item}\label{creaoggettimagici}

\textbf{Requirement}: Magic Proficiency 6

\textbf{Saving Throws}: +1 Fortitude, +1 Will

With this Feat, the spellcaster is able to infuse a spell up to level 3 into a magic item.


\begin{center}
	\includegraphics[width=0.8\linewidth]{immagini/oggettimagiciuomo.png}
	
	\textit{Henry Purcell - King Arthur}
\end{center}


\subsection*{Create Greater Magic Item}\index{Create Greater Magic Item}\label{creaoggettimagicisuperiori}

\textbf{Requirement}: Craft Magic Item, Magic Proficiency 12

\textbf{Saving Throws}: +1 Fortitude, +1 Will

Through this Feat the spellcaster is able to infuse a spell up to level 5 into a magic item.

\subsection*{Create Wonderful Magic Items}\index{Create Wonderful Magic Items}\label{creaoggettimagicimeravigliosi}

\textbf{Requirement}: Craft Greater Magic Items, Magic Proficiency 16

\textbf{Saving Throws}: +1 Fortitude, +1 Will

Through this Feat the spellcaster is able to infuse a spell up to level 8 in a magic item.


\subsection*{Create Mythic Magic Item}\index{Create Mythic Magic Item}\label{creaoggettimagicimitici}

\textbf{Requirement}: Craft Wondrous Magic Items, Magic Proficiency 18

\textbf{Saving Throws}: +1 Fortitude, +1 Will

Through this Feat the spellcaster is able to infuse a spell up to level 9 into a magic item.

\subsection*{Crippling Blow}\index{Crippling Blow}\label{colpoparalizzante}

\textbf{Requirement}: Weakening Strike, Sneaking Strike 4, Weapon Proficiency 18

\textbf{Saving Throws}: +2 Reflexes, +1 Fortitude

You dedicate 2 Actions per Round, for 5 rounds, to studying an opponent. In the sixth round using 2 Actions you make a melee or ranged attack. The opponent must make a Fortitude saving throw with a DC equal to the roll or be paralyzed for 3d6 rounds.

\subsection*{Dancing Scourge}\index{Dancing Scourge}\label{flagellodanzante}

\textbf{Requirement}: Weapon Proficiency 1, use a weapon from the Bullets List

\textbf{Saving Throws}: +1 Fortitude, +1 Will

When you use your Scourge you have a +1 bonus to attack and +1 to Defence


\subsection*{Daughter of Shayalia}\index{Daughter of Shayalia}\label{figliadishayalia}

Your connection with nature is strong and concrete

\textbf{Requirement}: Devoted or Follower of Shayalia

\textbf{Saving Throws}: +1 Fortitude, +2 Will

The \textbf{first time} you take this Feat you gain a +2 on Nature checks and a +2 on Saving Throws against natural poisons.

The \textbf{second time} that you take this Feat, requirement sum Traits in common 6, you get a +4 to Nature checks and a +4 to Saving Throws against effects, even magical, of Animals or Plants.

The \textbf{third time} you take this Feat, requirement total Traits in Common 12, you are always under the effect of the Shrine spell to any non-magical animal.

The \textbf{forth time} you take this Feat, requirement Animalia taken 4 times, you can polymorh in any creature but not not fiend or dragon.

\subsection*{Death blow}\index{Death blow}\label{colpomortale}

\textbf{Requirement}: Weapon Proficiency 5

\textbf{Saving Throws}: +2 Reflex, +1 Will

Make the attack roll with a -1d6 penalty, if you hit you cause 3 critical damage. Subsequent attack rolls start at -10 to hit.


\subsection*{Decipher magical writings}\index{Decipher magical writings}\label{decifrarescrittimagici}

\textbf{Requirement}: Magic Proficiency 1

\textbf{Saving Throws}: +1 Fortitude, +1 Will

Know how to read the magical writings. It has a +1d6 bonus to understanding the contents of a scroll and casting the spell it contains. The bonus also applies to a check to copy a spell on your tome of magic.




\subsection*{Defend Mount}\index{Defend Mount}\label{difenderecavalcatura}

\textbf{Requirement}: Ride 1

\textbf{Saving Throws}: +1 Fortitude, +1 Reflex

Whenever the mount is hit, you can make a Ride check to negate the hit. Your Ride check must be greater than your opponent's attack roll

The Feat can only be used once per round, for a single attack, it costs the Reaction.



\subsection*{Detect Magic}\index{Detect Magic}\label{rilevareilmagico}\index{Eye for magic}

\textbf{Requirement}: Magic Proficiency 1

\textbf{Saving Throws}: +1 Will, +1 Fortitude

If you can see it, you know if it's magical. It costs an Action to activate magical vision and lasts one round.

The \textbf{second time} you take the Ability to activate magic sight, with a Magic proficiency requirement of 1, it costs a Reaction.


\subsection*{Dodging traps}\index{Dodging traps}\label{schivaretrappole}

\textbf{Requirement}: Dexterity 2

\textbf{Saving Throws}: +2 Reflexes, +1 Fortitude

The \textbf{first time} you take the Feat, you gain a +1d6 bonus on your Saving Throw to avoid the effects of traps.

The \textbf{second time} you take the Feat, Weapon Proficiency requirement 5, even if the trap doesn't allow a Saving Throw, your natural propensity to avoid damage grants you a Reflex save for half damage.

It is also possible to use this Feat, use a Reaction, to avoid Sneak Attack (Reflex save higher than opponent's attack roll).

%La \textbf{terza volta} che prendi l'Abilità requisiti Competenza Armi 9, il Tiro Salvezza se riuscito ti permette di evitare qualsiasi effetto della trappola, se fisicamente possibile.


\subsection*{Double portion}\index{Double portion}\label{doppiaporzione}

\textbf{Requirement}: Two-weapon combat, Weapon Proficiency 4

\textbf{Saving Throws}: +2 Fortitude, +1 Reflex

Constant dual-weapon training allows you to apply the Strength damage bonus fully to your off-hand weapon as well.


\subsection*{Enchanted Mountain Armour}\index{Enchanted Mountain Armour}\label{armaturamontagnaincantata}

\textbf{Requirement}: Empty Fist Weapon List 1, Weapon Proficiency 1, Constitution 2, Wisdom 1

\textbf{Saving Throws}: +2 Fortitude, +1 Will

Constant training in spirit and body allows you to harden your skin and make it more difficult to hurt. To take advantage of these bonuses you do not have to bring armor or shields or objects that improve Defense. The listed abilities do not stack with the Silver Crane Skill.

The \textbf{first time} you take this Skill your Defense is 10 + Constitution + 1/3 of the points in Empty Fist + any and all.

The \textbf{second time} you take this Skill, Empty Fist requirement 5, you gain damage resistance (DR) of 1/-

The \textbf{third time} you take this Skill, Empty Fist requirement 7, automatically reduce Bleeding by 1 at the end of the round.

The \textbf{fourth time} you take this Skill, Empty Fist requirement 8, you gain damage resistance (DR) of 3/-

The \textbf{fifth time} you take this Skill, Empty Fist requirement 13, you gain damage resistance (DR) of 5/-


\subsection*{Elemental Form}\index{Elemental Form}\label{formaelementale}

\textbf{Requirement}: Follower or Devotee of Erondil, Gaya, Efrem or Shayalia. Must have Lists of Magic on an Element. Magic Proficiency 6

\textbf{Saving Throws}: +1 Fortitude, +1 Will

The \textbf{first time} you take this Feat when you transform into an animal, your animal attack damage can do damage of the elemental type of your choice.

The \textbf{second time} you take this Feat, Magic Proficiency 11, while in animal form you are resistant to the same elemental as you damage. Resistance is to be declared when transforming into an animal and if not declared, protection is from fire.

The \textbf{third time} you take this Feat, Magical Proficiency 14 requirement, your animal attacks do 2d6 more damage of the chosen elemental type.

The Elemental damage type must be from an Elementale Magic List knowed.

If you are a Devotee or Follower of Gaya or Erondil it is not necessary to transform into an animal, the abilities apply to your melee attacks.


\subsection*{Elementalist}\index{Elementalist}\label{Elementalist}

\textbf{Requirement}: At least 2 Elemental Spell Lists

\textbf{Saving Throws}: +1 Will, +1 Fortitude

You are able to swap elements in your spells. You can replace one type of elemental energy damage with damage from a Elementale Magic List known to you. The casting time of the spell increases by 1 Action. If total casting time become more then 3 Actions you cannot use this Feat on this spell.


\subsection*{Enrage}\index{Enrage}\label{fareinfuriare}

Your dialectic Feats are amazing.

\textbf{Requirement}: Weapon Proficiency 2 and Charisma/Strength 2

\textbf{Saving Throws}: +2 Will, +1 Fortitude

You spend 2 Actions defaming and railing against an opponent. The target must make a Will saving throw against your Perform or Intimidate proficiency check or lose the Dexterity bonus (Saving Throws, Attack Rolls, and Defense) until the end of your next round.

The opponent may not understand your language but must have Intelligence of -3 or more.


\subsection*{Expert}\index{Expert}\label{esperto}

\textbf{Requirement}: Characteristic linked to at least -1 value.

\textbf{Saving Throws}: +1 to two Saving Throws of your choice.

You are an expert in a topic. Whenever you take this Feat you gain a +1 on checks on a specific Skill of your choice.

The \textbf{second time} you take this Feat add +2 to the check. You can take 10 on check in 5 rounds instead of 10 rounds. (page \pageref{prendere10}).

The \textbf{third time} you take this Feat add 1d6 to the check. You can take 14 on check in 5 minutes instead of 10 minutes.

The \textbf{fourth time} you take this Feat you treat the total of rolled dice from 4-9 as rolled 10.

Bonus are cumulative.

It is not usable on Awareness (see Perceptive).


\begin{center}
	\includegraphics[width=0.75\linewidth]{immagini/distillare.png}
	
	\textit{The Alchemist Discovering Phosphorus. Joseph Wright of Derby (1771-1795)}
\end{center}


\subsection*{Extended Battery}\index{Extended Battery}\label{batteriaestesa}

\textbf{Requirement}: Magic Proficiency 1

\textbf{Saving Throws}: +1 Fortitude, +1 Will

You can handle the mental stress of casting spells better.

The \textbf{first time} you take this Feat the effects of \hyperlink{quandosihannopochipuntimagia}{When low on Spell Points} (page \pageref{magiequandosihannopochipuntimagia}) are activated at 60\% of the use of Spell Points.

The \textbf{second time} you take this Feat the effects of "When you are low on Spell Points" are activated at 70\% of the use of Spell Points.

The \textbf{third time} you take this Feat the effects of "When you are low on Spell Points" are activated at 80\% of the use of Spell Points.

The \textbf{fourth time} you take this Feat the effects of "When low on Spell Points" no longer apply.


\subsection*{Faithful}\index{Faithful}\label{fedele}

\textbf{Requirement}: Magic Proficiency 1, Sum Value in common Traits 2

\textbf{Saving Throws}: +2 Will, +1 Fortitude

Your connection with the Patron is strong and energetic. You increase your Magic Points by half the amount of Traits you have in common with your Patron.

This Feat does not stack with the Magic Battery Feat.


\subsection*{Fast Step}\index{Fast Step}\label{passoveloce}

\textbf{Requirement}: Dexterity 1

\textbf{Saving Throws}: +2 Reflexes, +1 Fortitude

Your pace is naturally quick.
If you have movement 6m you move to movement 7m, if you have movement 9m you move to movement 10m.

Every two additional times you take the Feat your movement increases by 1 meter per Move Action, up to a maximum of +3 meters per round.


\subsection*{Feign Death}\index{Feign Death}\label{fintamorte}

You are able to simulate death by slowing down the heart.

\textbf{Requirement}: Constitution 0

\textbf{Saving Throws}: +2 Fortitude, +1 Will

As a Reaction Action you are able to fall to the ground (knock down!) dead. Only a DC 20 First Aid check can reveal that you are alive.

The effect lasts a maximum of 2 minutes. Feign death is not repeatable within 10 minutes of each other.

\subsection*{Ferocity}\index{Ferocity}\label{ferocia}

\textbf{Requirement}: Weapon Proficiency 1

\textbf{Saving Throws}: +2 Fortitude, +1 Will

Your rage is such that it temporarily defeats death.

When you drop below 0 Hit Points, you do not pass out and you begin to lose 1 hit point per round.

A creature with Ferocity is knocked unconscious when it has a negative hit point score equal to twice its Constitution score, and dies when its Hit Points drop to a negative score equal to its triple Constitution score+5 (CON*3+5).

The \textbf{second time} you take this Feat, Weapon Proficiency requirement 4, you can make your Strength increase by 2 in combat and gain 6 temporary Hit Points for 10 minutes. At the end of the fight, your fatigue level increases by 1 for 10 minutes.

The \textbf{third time} you take this Feat, Weapon Proficiency requirement 7, you can cause your Strength to increase by 3 in combat and you gain 12 temporary Hit Points for 10 minutes. At the end of the fight your fatigue level increases by 2 for 20 minutes.

The \textbf{fourth time} you take this Feat, Weapon Proficiency requirement 11, you can cause your Strength to increase by 4 in combat and you gain 24 temporary Hit Points for 10 minutes. At the end of the fight your fatigue level increases by 3 for 30 minutes.

The player can choose only one degree of Ferocity to use in the fight (2, 3, 4).

\subsection*{First Blood}\index{First Blood}\label{First Blood}

\textbf{Requirement}: Weapon Proficiency 1

\textbf{Saving Throws}: +1 Fortitude, +1 Will

The first attack roll against a new opponent get a +1d6 bonus.

\subsection*{Forged in fury}\index{Forged in fury}\label{forgiatonellafuria}

\textbf{Requirement}: Weapon Proficiency 5

\textbf{Saving Throws}: +1 Fortitude, +1 Reflex

When you make a critical roll, i.e. you rolled at least 2 6s, you are considered to have rolled an extra 6 for the total count of the number of criticals


\subsection*{Fury}\index{Fury}\label{furia}

\textbf{Requirement}: Weapon Proficiency 1

\textbf{Saving Throws}: +2 Fortitude, +1 Will

Your fighting style is represented by blind killing spree. Add +1d6 to damage on each successful melee attack and your opponents gain +1d6 on hitting towards you. You can decide to activate this Feat round by round. Cost 1 Immediate Action per round.

\begin{center}
	\includegraphics[width=0.9\linewidth]{immagini/Early_Egyptian_juggling_art.png}
	
	\textit{This ancient wall painting appears to depict jugglers.}
\end{center}


\subsection*{Get down!}\index{Get down!}\label{staigiu}

\textbf{Saving Throws}: +2 Fortitude, +1 Will

When you perform two critical hits on an opponent, the force of your blow is enough to knock him prone. The opponent must make a Fortitude save (DC equal to the last attack roll) or fall prone. The Feat works on creatures of a size equal to or smaller than the character.

The \textbf{second time} you take the Feat you can also affect creatures one size larger.

The \textbf{third time} you take the Feat you can also affect creatures two sizes larger.


\subsection*{Hawkeye}\index{Hawkeye}\label{occhiodifalco}

\textbf{Requirement}: Weapon Proficiency 3

\textbf{Saving Throws}: +2 Reflex, +1 Will

The penalty for rolls between the first and second increments has no penalty

The \textbf{second time} you take this Feat, the penalty for rolls up to the third range increment is 1d6.

The \textbf{third time} you take this Feat you are able to extend your roll even further and take it to a fifth increment with a -2d6 penalty to hit. You have no penalty within the first 3 increments while you have -1d6 to hit between the third and fourth increments. \\

\textbf{Example. Tups uses a Shortbow, range 15 meters.}

Normally if he has to shoot an arrow at an orc within 15 m he has no penalty, if the orc is between 15 and 30 meters he has 1d6 to hit, if he is between 30 and 45 meters he has -2d6 to hit. He can't shoot any further.

- \textbf{Tups gets Hawkeye Feat}.

If he has to shoot an arrow at an orc within 15 m he has no penalty, if the orc is between 15 and 30 meters he has no penalty, if he is between 30 and 45 meters he has -2d6 to hit. He can't shoot any further.

- \textbf{Tups gets Hawkeye Feat a second time}.

If he has to shoot an arrow at an orc within 15m he has no penalty, if the orc is between 15 and 30m he has no penalty, if he is between 30 and 45m he has -1d6 to hit, if he is between 45 and 60m he has a -2d6 to hit. He can't shoot any further.

- \textbf{Tups gets Hawkeye Feat a third time}.

If he has to shoot an arrow at an orc within 15 m he has no penalty, if the orc is between 15 and 30 meters he has no penalty, if he is between 30 and 45 meters he has no penalty to hit, if he wants he can shoot the arrow between 45 and 60 meters with -2d6 to hit.


\subsection*{Hit and Run}\index{Hit and Run}\label{toccataefuga}

\textbf{Saving Throws}: +2 Reflexes

Your attacks have a base penalty of -5 and you can take an Action of 1 more move. You cannot take more than one bonus move action in this manner. It costs one Immediate Action.


\subsection*{Hollow head}\index{Hollow head}\label{Hollowhead}

\textbf{Requirement}: Weapon List of crossbows at 4

\textbf{Saving Throws}: +2 Fortitude

You manage to give a deadly effect to your projectiles.

Your crossbow projectiles increases by one size of damage (1d6>1d8>1d10>2d6)

\subsection*{Hound}\index{Hound}\label{segugio}

\textbf{Requirement}: Intelligence 1, Wisdom 1, Weapon Proficiency 1

\textbf{Saving Throws}: +1 Reflex, +1 Will

You have a natural talent for following people

With two Actions you focus on a target that you can see and as long as you see it you stay focused. All your Actions involving that target have a +1 bonus. Staying focused costs 1 Action per round.

The \textbf{second} time you take this Feat, Weapon Proficiency requirement 6, the bonus increases to +2.

The \textbf{third} time you take this Feat, Weapon Proficiency requirement 12, the bonus increases to +3.

The bonus can be used on attack rolls, Saving Throws caused by the opponent, proficiency checks... but not on damage.

Once per day you can transform a 6 rolled by the Arbiter (attack rolls, Feat checks, Saving Throws) into a 1.


\subsection*{Human Mountain}\index{Human Mountain}\label{montagnaumana}

\textbf{Requirement}: Constitution 1

\textbf{Saving Throws}: +3 Fortitude

Maybe you were once puny and weak, now you're a mountain of muscle.

When you take this Feat, you increase the Hit Points taken per level by 1.

The \textbf{second time} you take this Feat increases your Hit Points taken per level by 1.

The \textbf{third time} you take this Feat increases the die for rolling Hit Points (d4 to d6).

Bonuses are cumulative and retroactive to previous levels, except for the increase in hit dice.

The \textbf{fourth time} you take this Feat increases by one size (S > M > H > G).


\begin{center}
	\includegraphics[width=0.65\linewidth]{immagini/elcolosso.png}
	
	\textit{The Colossus (also known as The Giant), is known in Spanish as El Coloso. It is not by Goya but by a pupil.}
\end{center}

\subsection*{Horse Archer}\index{Horse Archer}\label{arcereacavallo}

\textbf{Requirement}: Weapon Proficiency 1

\textbf{Saving Throws}: +1 Reflex, +1 Fortitude

The penalty for shooting horse arrows decreases by 2 each time you take this Feat.

The standard penalties are -4 and -6 depending on whether you trot (move x2) or canter (move x3)

\subsection*{I said FALL!}\index{I said FALL!}\label{hodettocadi}

\textbf{Requirement}: Weapon Proficiency 4

\textbf{Saving Throws}: +2 Fortitude, +1 Will

If you strike 3 times within 2 rounds, an opponent must make a Fortitude save (DC equal to the attack roll of the last second strike) or fall prone.

\subsection*{Iaijutsu}\index{Iaijutsu}\label{iaijutsu}

\textbf{Requirement}: Weapon Proficiency 2

\textbf{Saving Throws}: +2 Reflex, +1 Will

For every -1d6 to Attack Roll you gain +4 to Initiative and vice versa.
The bonus must be used by the end of the next round. The declaration must be made each round that you intend to use at the time of the initiative check.

\subsection*{Immunity to poison}\index{Immunity to poison}\label{immunitaaiveleni}\index{Mithridatism}

\textbf{Requirement}: Constitution 1

\textbf{Saving Throws}: +2 Fortitude, +1 Will

The body becomes accustomed to poisons, the character gains a +2 Saving Throw against poisons.

The \textbf{second time} you take the Feat you become immune to natural poisons. You can no longer get drunk normally.

The \textbf{third time} you have a +1d6 on Saving Throws from magical poisons and are affected by toxic fumes (but you can always suffocate).


\begin{center}
	\includegraphics[width=0.9\linewidth]{immagini/horsearcher.png}
	\textit{Assyrian Archer}
\end{center}



\subsection*{Improved Initiative}\index{Improved Initiative}\label{iniziativamigliorata}

\textbf{Requirement}: Intelligence or Dexterity 1

\textbf{Saving Throws}: +2 Reflexes

Increase initiative by +1. The Feat can be taken up to 2 times and the bonus stacks.

\subsection*{Improvise}\index{Improvise}\label{improvvisare}

\textbf{Requirement}: Weapon Proficiency 1

\textbf{Saving Throws}: +1 Fortitude, +1 Reflexes

Any item that would call itself an improvised weapon to you is not improvised.
You suffer no penalty to hit when using an improvised weapon. Gain +1 damage when using an improvised weapon.




\subsection*{Imbue Magical Energy}\index{Imbue Magical Energy}\label{infuse magicalenergy}

\textbf{Requirement}: Weapon Proficiency 1, Magical Proficiency 2

\textbf{Saving Throws}: +1 Reflex, +1 Fortitude

You know how to manipulate magical energies instinctively and infuse them into weapons. It costs 1 Action to imbue the weapon with magic.

The \textbf{first time} you take this Feat you can use two Magic Points and channel it into your weapon. For the duration of 6 rounds your weapon becomes a +1 magical weapon, if it already has magical abilities the effect does not work.

The \textbf{second time} you take this Feat, Magical Proficiency requirement 4, you can use four Magic Points and a weapon you come into contact with becomes a +2 weapon for 6 rounds, if it is already enchanted with a bonus of +1 makes +2.

The \textbf{third time} you take this Feat, Magical Proficiency requirement 8, you can use six Magic Points and a weapon you come into contact with becomes a +3 weapon for 6 rounds, if it is already enchanted with a bonus of +2 or lower makes it +3.

\subsection*{Infuse Greater Magical Energy}\index{Infuse Greater Magical Energy}\label{infuse greater magical energy}

\textbf{Requirement}: Weapon Proficiency 4, Magical Proficiency 6

\textbf{Saving Throws}: +1 Reflex, +1 Fortitude

You know how to infuse the weapon with magical energy to give it fantastic abilities. It costs 1 Action to activate the infusion of magic into the weapon. The weapon must be magical.

The \textbf{first time} you take this Feat using one Magic Point per round you can make your weapon flaming/electrified or change its form. Each successful hit causes 1d6 additional fire/electricity damage, or you can change the weapon's form (if you stop paying the Magic Point returns to its previous form).

The \textbf{second time} you take this Feat by using two Magic Points per round you can make a weapon you come into contact with extremely dangerous. Each successful hit causes 1 additional critical damage. Magical Expertise Requirement 7.

The \textbf{third time} you take this Feat using three Magic Points per round you can grant a weapon you come into contact with both of the previous abilities.

\subsection*{Instill Courage}\index{Instill Courage}\label{infonderecoraggio}

\textbf{Requirement}: Charisma 2, Perform 1

\textbf{Saving Throws}: +2 Will, +1 Fortitude

Through your performance, singing, ballet, oratory… you are able to instill courage in companions able to hear or see you, within 6 meters.

The first time you take this Feat, your companions have a +1 bonus to attack rolls and damage in combat.

The \textbf{second time} you take this Feat, requirement Perform 4, you can choose to instill up to 2 of these bonuses. +2 AR, +2 Defence, +2 Damage, +2 Will save. Your companions must be within 12 meters of the radius.

The \textbf{third time} you take this Feat, requirement Perform 12, you can choose to imbue up to 2 of these bonuses. +1d6 Attack Roll, +4 Defence, +4 Damage, +1d6 ST. Your companions must be within 24 meters of the radius.

Activating, maintaining or changing the effect of the Feat takes 2 Actions. You can maintain the Feat for a number of rounds, even if not consecutive, equal to your Perform score x 3 per day. Creatures must continue to see/hear the performance to remain affected.



\subsection*{Instill Fear}\index{Instill Fear}\label{infonderepaura}

\textbf{Requirement}: Charisma 2

\textbf{Saving Throws}: +2 Will, +1 Fortitude

Through your performance, singing, ballet, oratory… you are able to instill fear in opponents able to hear you, within 6 meters.

The first time you take this Feat your enemies suffer a -1 penalty to attack rolls and damage in combat. 

The \textbf{second time} you take this Feat, requirement Perform 4, the force of your art attacks enemies and you can select two effects among: -2 Attack Roll, -2 Combat Damage, -2 Defence, -2 at Will saves. Your enemies must be within 12 meters of radius.

The \textbf{third time} you take this Feat, requirement Perform 12, the force of your art attacks enemies and you can select two effects among: -1d6 Attack Roll, -4 Defence, -4 Damage, -1d6 ST. Your enemies must be within 24 meters of radius.

The opponent is allowed a DC Will save equal to 10+CH+Perform score. A creature that succeeds at its Saving Throw is immune to new manifestations of your power that day.

Activating, maintaining or changing the effect of the Feat takes 1 Action. You can maintain the Feat for a number of rounds, even if not consecutive, equal to your Perform score x 3 per day. Creatures must continue to see/hear the performance to remain affected.

\subsection*{Instinctive Knowledge}\index{Instinctive Knowledge }\label{conoscenzaistintiva}

\textbf{Requirement}: Knowledge 1

\textbf{Saving Throws}: +2 Will, +1 Fortitude

You never forget an enemy.

You have an instinctive ability to remember and evaluate an enemy. When you take this Feat, you can make a \hyperlink{riconoscereimostri}{Recognize a Monster} check (page \pageref{riconoscereimostri}) using a Reaction.

\subsection*{Iron Fist}\index{Iron Fist}\label{pugnodiferro}

\textbf{Requirement}: Empty Punch List 3

\textbf{Saving Throws}: +2 Fortitude, +1 Will

Your unarmed combat technique is extremely precise and powerful.

The \textbf{first time} you take this Skill the damage caused by your hits and the attack roll increase by 1.

The \textbf{second time} you take this Skill, Empty Fist requirement 6. Damage increases by +2, attack roll +1. 

The \textbf{third time} you take this Skill, Empty Fist requirement 9. Damage +1, Attack Roll +2.

The \textbf{fourth time} you take this Skill, Empty Fist requirement 12. Damage +2, Attack Roll +1. 

The \textbf{fifth time} you take this Skill, Empty Fist requirement 15. Damage +1, Attack Roll +2. .

The \textbf{sixth time} you take this Skill, Empty Fist Requirement 18. Damage +2, Attack Roll +1.

The bonuses indicated are cumulative.

\subsection*{Iron Will}\index{Iron Will}\label{volontaferrea}

\textbf{Requirement}: Wisdom 0

Over time you have trained your will to resist any weakness and fear.

The first time you take this Feat, you gain a +2 bonus on Will saves. The bonus is cumulative, +2 the first time, +1 the second time, +1 the third time.

The \textbf{fourth time} you take this skill you can decide to automatically succeed on a Will save once per day before rolling the dice.

\subsection*{Juggler}\index{Juggler}\label{giocoliere}

\textbf{Requirement}: Dexterity 2

\textbf{Saving Throws}: +2 Reflexes

You have a natural talent for handling objects.

Any Acrobatics check involving handling objects or balance has a +2 bonus.

You can throw a second dagger as an immediate action to the throw dagger attack action with a -3 to attack roll. Any third thrown dagger has the normal penalty of -5 (and -10.. and so on).

\subsection*{Laying on hands} \index{Laying on hands}\label{laying on hands}

\textbf{Requirement}: Magical Proficiency 1, Common Traits 3, being Devote or Follower

\textbf{Saving Throws}: +2 Will, +1 Fortitude

If your Traits are in common with a positive Patron you can channel healing energy (healing/harmful effect on undead), if they are in common with a neutral or evil Patron you can channel negative energy (harmful/healing effect on undead). Usable a number of times per day equal to (sum of Traits in common with the Patron)/2. 

Through the laying on of hands you can heal/injure 5 hit points to a creature. You can apply multiple uses with single tap.

The feat taken multiple times allows you to remove specific conditions that afflict the creature by consuming multiple uses.

The \textbf{second time} you take this feat, add Traits in common 4, you can remove the Fatigued condition (3 uses).

The \textbf{third time} you take this feat, add Traits in common 6, you can remove the conditions: Frightened, Nauseated, Fainted, Fatigued 2 (3 uses).

The \textbf{fourth time} you take this feat, add Traits in common 8, you can remove the conditions: Blinded, Deafened, Poisoned, Sick* (3 uses).

The \textbf{fifth time} you take this feat, add Traits in common 10, you can remove the conditions: Confused, Paralyzed, Fascinated (4 uses).

The \textbf{sixth time} you take this feat, add Traits in common 12, you can remove the conditions: Dominated, Petrified, Fatigued 3 (5 uses).

Each use of the Lay on Hands feat to heal reduces the Bleeding value by 2 and allows you to recover 3 maximum hit points.

In the case of magical illnesses* a possible contrast check is made with 3d6 + sum of Common Trait points + Wisdom.

The energy comes from the hands (it doesn't matter if there are gloves) and is applied with a Touch Attack. Use 2 Actions. Fortitude save DC 10 + add Traits in common with the Patron + Wisdom to avoid the effect. 2 Actions.

\subsection*{Channel energy} \index{Channel energy}\label{channel energy}

\textbf{Requirement}: Laying on Hands, Magical Proficiency 1, Common Traits 4

\textbf{Saving Throws}: +2 Will, +1 Fortitude

You are able to use the energy of Laying on Hands to create an energetic aura around you.

Through the Laying on of Hands you create an instant aura within a 10 foot radius around you that heals/hurts all creatures present for 5 Hit Points every 2 uses expended.

Each time you take this Ability beyond the first, you increase the radius by 3 feet and can exclude one creature from the aura's effect.

Energy comes from your body and affects yourself and the creatures around you. Reflex save DC 10 + add Traits in common with the Patron + Wisdom to avoid the effect. \index{Channel energy into undead} 2 Actions.


\begin{center}
	\includegraphics[width=0.65\linewidth]{immagini/Portrait_of_V_Greatrakesv2.png}
	
	\textit{Portrait of V. Greatrakes laying on his hands, window, in right-hand corner showing several successful cures, possibly. By W. Faithorne }
\end{center}


\subsection*{Leatherskin}\index{Leatherskin}\label{pellecoriacea}

\textbf{Requirement}: Constitution 2

\textbf{Saving Throws}: +2 Fortitude

Your skin is extremely resilient. You take 1 less damage when hit by slashing, piercing, or bludgeoning weapons.

The \textbf{second time} you take this Feat, Weapon Proficiency requirement 4, you take 1 less damage when hit by slashing, piercing, or bludgeoning weapons. Reduce Bleeding condition by 1 when acquired.

The \textbf{third time} you take this Feat, Weapon Proficiency requirement 8 and Constitution 3, you take 1 less damage when hit by slashing, piercing, or bludgeoning weapons. You take 1 less damage when hit by magic. Reduce Bleeding condition by 1 when acquired.

The \textbf{fourth time} you take this Feat, Weapon Proficiency requirement 16, you ignore 1 critical roll when hit by slashing, piercing, or bludgeoning weapons, and you take 1 damage in hand when hit by magic. Reduce Bleeding condition by 1 when acquired.

Bonuses are cumulative.


\subsection*{Lightning Reflexes}\index{Lightning Reflexes}\label{riflessifulminei}

\textbf{Requirement}: Dexterity 0

Over time you have trained your reflexes to dodge and anticipate any obstacle. The first time you take this Feat, you gain a +2 bonus on Reflex saves. The bonus is cumulative, +2 the first time, +1 the second time, +1 the third time.

The fourth time you take this Feat, you can decide to automatically succeed on a Reflex save once per day as a reaction. It must be declared and does not make a Saving Throw.

\subsection*{Lunge}\index{Lunge}\label{Lunge}

\textbf{Requirement}: Weapon Proficiency 2

\textbf{Saving Throws}: +1 Will, +1 Fortitude

You use a Reaction in conjunction with your Attack Action to deliver the blow as if you had a Long Weapon, i.e. +2 to attack rolls against opponents who do not have long weapons.

\subsection*{Loaded dice}\index{Loaded dice}\label{daditruccati}

\textbf{Requirement}: Magic Proficiency 6

\textbf{Saving Throws}: +1 Fortitude, +1 Reflex

You can increase by 1, within the value of 6, a single die in the Magic Test.

The \textbf{second time} you take this Skill, Magical Proficiency requirement 12, you can increase an additional die on the Magic Test by 1, within the value of 6

\subsection*{Lucky}\index{Lucky}\label{fortunato}

\textbf{Saving Throws}: +1 Fortitude, +1 Reflex

Once per day you can make the Arbiter reroll 1d6 of a check (attack rolls, proficiency checks, Saving Throws) and take the lower of the two rolls.


\begin{center}
	\includegraphics[width=0.55\linewidth]{immagini/streghegoya.png}
	
	\textit{Sabbath of the Witches (Goya, 1798)}
\end{center}


\subsection*{Magic Battery}\index{Magic Battery}\label{batteriamagica}

\textbf{Requirement}: Magic Proficiency 3

\textbf{Saving Throws}: +2 Will, +1 Fortitude

You have a special connection to the magic that Yeru endures.

The \textbf{first time} you take this Skill, you increase your Spell points by 3.

The \textbf{next time} you increase the Spell Points by a value equal to the previous increase +1.


\subsection*{Magic Roots}\index{Magic Roots}\label{radicimagiche}

\textbf{Requirement}: Magic Proficiency 1

\textbf{Saving Throws}: +2 Will, +1 Fortitude

As long as you are affected by your spell, using an Action your weapon gains +1 to hit and damage and is considered a magic weapon till end of the round.

\subsection*{Mental wall}\index{Mental wall}\label{mentalwall}

\textbf{Requirement}: Wisdom +1

\textbf{Saving Throws}: +2 Will, +1 Fortitude

Your mind is trained against those who want to influence it. Whenever you take this skill, you gain +1 on saving throws against spells on the enchantment spell list.

\subsection*{My skin}\index{My skin}\label{La mia pelle}

\textbf{Requirement}: Weapon Proficiency 1

\textbf{Saving Throws}: +3 Fortitude

You have an almost symbiotic relationship with your Armour.

The \textbf{first time} you take this Feat the Defence that grants you the Armour you wear increases by 1.

The \textbf{second time} you take this Feat, Weapon Proficiency requirement 4, choose an Armour type (light, medium, or heavy), all Armour that falls within the chosen type grants +1 Defence.

\textbf{Each subsequent time} you take this Feat, Weapon Proficiency requirement +4 from the previous time, choose a new or already taken Armour type, your Defence with that Armour type increases by 1.

\subsection*{My death your death}\index{My death your death}\label{lamiamortelatuamorte}

\textbf{Requirement}: Weapon Proficiency 1, Strength 1

\textbf{Saving Throws}: +2 Fortitude, +1 Will

For every single combat opponent you can make the first hit of the fight cause an additional damage equal to double your Weapon Proficiency. The opponent gains a bonus to attack rolls and damage equal to your Weapon Proficiency value. It must be declared before the Attack Roll.

\subsection*{My Head is Harder}\index{My Head is Harder}\label{lamiatestaepiudura}

\textbf{Requirement}: Weapon Proficiency 1

\textbf{Saving Throws}: +1 Fortitude, +1 Will

Your Skull Crusher Weapon does +2 damage


\subsection*{One arm, one weapon}\index{One arm, one weapon}\label{unbracciounarma}

\textbf{Requirement}: Weapon Proficiency, 2

\textbf{Saving Throw}: +1 Fortitude, +1 Will

Choose a Weapon List. Strength damage applied by weapons on that list increases by 2.

The Feat can be taken multiple times and the Weapon List score must be 4 times the number of times this Feat is taken if on the same Weapon List.

If you take \textbf{4 times} this Feat on the same Weapon List the damage bonuses are reduced to +4 but you roll damage twice and choose the better result.


\subsection*{One body, one mind, one spirit}\index{One body, one mind, one spirit}\label{USCMS}\index{USCMS}

\textbf{Saving Throws}: +1 of your choice

Assign one point to Weapon Proficiency or Magic Proficiency. This Feat can be taken a maximum of 2 times.

In the manual you will also find this Feat under the name of \textbf{USCMS}.



\subsection*{One with magic}\index{One with magic}\label{tuttunoconlamagia}

\textbf{Requirement}: Adept of Magic

\textbf{Saving Throws}: +1 on two Saving Throws of your choice

Whenever you take the One with Magic Feat, you must determine which Attribute it connects to.
Your Characteristic has a +1 value for determining the maximum level of spell you can cast.

\subsection*{One shot one kill}\index{One shot one kill}\label{ungestounmorto}

\textbf{Requirement}: Magic TestProficiency 1, Adept of Magic 1

\textbf{Saving Throws}: +1 Reflex

The \textbf{first time} you take this skill you gain a +1 on attack rolls to spells that require an attack roll.

The \textbf{second time} the attack roll bonus for Spells becomes +1 for each time you took the Adept of Magic feat. It cannot be combined with the bonus obtained the previous time.


\subsection*{Opportunist}\index{Opportunist}\label{opportunista}\hypertarget{opportunista}{}

\textbf{Requirement}: Weapon Proficiency 2

\textbf{Saving Throws}: +2 Reflex, +1 Will

You may attempt to melee an opponent who \textbf{exits} or \textbf{crosses} a melee area that you threaten or use a thrown weapon against you in your melee area. The ability is usable once per round as a reaction. This attack is also called an attack of opportunity in the rulebook, and there are several creatures that don't trigger its reaction.


\subsection*{Parry}\index{Parry}\label{Parry}

\textbf{Requirement}: Weapon Proficiency 3 or Empty Fist 2

\textbf{Saving Throws}: +1 Reflex, +1 Will

The \textbf{first time} you take this Feat You use a Reaction to increase your Defense by 1.

The \textbf{second time} you take this Feat, Weapon Proficiency requirement 6 or Empty Fist 4, using a Reaction increases your Defense by 2.

The \textbf{third time} you take this Feat, Weapon Proficiency requirement 9 or Empty Fist 6, you gain a Reaction that you can only use to use the Parry Feat.

Using the Parry Feat can be declared even after you know how much you have been hit.


\subsection*{Power of Patron}\index{Power of Patron}\label{poteredelpatrono}

\textbf{Requirement}: Sum Traits common to Patron 1, being Devout

\textbf{Saving Throws}: +1 Fortitude, +2 Willpower

Your faith in the Patron knows no bounds or collapses of trust.

You can declare to use the skill once per day. The next three significant rolls (Saving Throws, Attack Rolls, skill checks) have the sum of the common traits with the Patron as their unique positive modifier.
If all three tests are successful, it is probable that there is a Manifestation of the Patron.


\subsection*{Pet}\index{Familiar}\label{famiglio-abilita}

\textbf{Requirement}: Magic Proficiency 1

\textbf{Saving Throws}: +1 Will, +1 Fortitude

Earn a natural animal. This pet has a Challenge Rating equal to half your Wisdom. You can teach your pet basic actions and make him do simple tasks.

The \textbf{second time} you take this Feat you gain a Familiar (see specific chapter).


\subsection*{Perceptive}\index{Perceptive}\label{percettivo}

\textbf{Requirement}: Wisdom 0

\textbf{Saving Throws}: +1 Reflex, +1 Will

Your Awareness and attention to detail is above average.
You get a +1 bonus on Awareness checks. The Feat can be taken up to 3 times.


\subsection*{Phoenix Wings}\index{Phoenix Wings}\label{alidellafenice}

\textbf{Requirement}: Empty Fist List 3, Silver Crane 1

\textbf{Saving Throws}: +2 Reflexes, +1 Fortitude

Your fighting style emphasizes ranged strikes such as kicks and flying punches.

The \textbf{first time} you take this Feat your melee distance becomes 2 meters.

The \textbf{second time} you take this Feat, Empty Fist List requirement 6, Silver Crane 3, Iron Fist 1, your melee distance becomes 3 meters.

The \textbf{third time} you take this Feat, Empty Fist List requirement 12, Silver Crane 4, Iron Fist 2, your melee range becomes 4 meters.

Until the opponent reaches within melee range of the character, the latter will have a +2 bonus to hit, as if he were using a long weapon.

\subsection*{Polyglot}\index{Polyglot}\label{polyglot}

\textbf{Requirement}: at least Intelligence -1, at character creation

\textbf{Saving Throws}: +2 Will

You have an amazing ability to learn languages. Know two more languages.


\subsection*{Power Punch}\index{Power Punch}\label{pugnopotente}

\textbf{Requirement}: Empty Punch List 3

\textbf{Saving Throws}: +1 Fortitude, +2 Will

Consume 2 Actions. You make a single attack roll with a -5 penalty. 
If you hit, in addition to damage and critical damage, the opponent who must be a maximum of two sizes larger than you must make a Fortitude saving throw with a DC equal to your attack roll or be pushed 10 feet in your direction choice.

If he fails his saving throw critically he takes additional critical damage.

\subsection*{Powerful Blows}\index{Powerful Blows}\label{colpipoderosi}

\textbf{Requirement}: Weapon Proficiency 1

\textbf{Saving Throws}: +2 Fortitude

Your style emphasizes powerful strokes.

Gain +1 to damage with all weapons on a Weapon List.


\subsection*{Powerful Magic}\index{Powerful Magic}\label{magiepotenti}

\textbf{Requirement}: Magic Proficiency 5

\textbf{Saving Throws}: +2 Will

Your spells are extraordinarily effective.

Choose one Magic List, gain a +1d6 Magic Test when casting spells from this school. The Feat can be taken multiple times but the total must be less than MP/4 and the bonus is added or applied to other Magic Lists.


\subsection*{Precise Shot}\index{Precise Shot}\label{tiropreciso}

\textbf{Requirement}: Dexterity 3, Weapon Proficiency 1

\textbf{Saving Throws}: +2 Reflexes

You gain +1 to hit and +1 to damage and to hit, with thrown weapons, bows, or crossbows, with targets within 10 meters.


\subsection*{Prudent Spellcaster}\index{Prudent Spellcaster}\label{prudentspellcaster}

\textbf{Requirement}: Magic Proficiency 8

\textbf{Saving Throws}: +2 Refles, +1 Fortitude

When a hostile creature enters a space within 1 meter of you for the first time, you can use a Reaction to cast a cantrip without any enhancements.

This ability does not affect being Distracted while casting.


\begin{center}
	\includegraphics[width=0.7\linewidth]{immagini/kenilguerriero.png}
\end{center}


\subsection*{Psychic Energy}\index{Psychic Energy}\label{psychic energy}

\textbf{Requirement}: Strength 1, Wisdom 2, Weapon Proficiency 1, Magical Proficiency 1

\textbf{Saving Throws}: +2 Will, +1 Fortitude

After years of training, meditation and internship at Panda Barbat you are able to harvest your Chi Energy.

Every day after at least 6 hours of rest and 2 hours of meditation/training, fill your body with Chi energy equal to Weapon Expertise + Magical Expertise + Wisdom / 2

The \textbf{second time} you take this feat, requirement Strength 1, Wisdom 2, Weapon Proficiency 4, Magical Proficiency 4

You recover 1 Chi point every 10 minutes that the character does not engage in demanding activities.

\subsection*{Psychic Strike}\index{Psychic Strike}\label{psychic strike}

\textbf{Requirement}: Psychic Energy, Dexterity 1

\textbf{Saving Throws}: +2 Will, +1 Fortitude

You concentrate your Chi in your hands.
You can concentrate a number of Chi points equal to your Wisdom.
With a successful Touch Attack, you discharge energy that round that causes 1d6 force damage per point of Chi used, up to a maximum of Chi points equal to your Wisdom score.

The blow is considered to be delivered by a magical weapon with a bonus equal to the Chi points used.

The \textbf{second time} you take this feat, requirement Psychic Strike, Wisdom 3, Weapon Proficiency 2, you consume one less Chi point if the attack roll is successful. 

The \textbf{third time} you take this feat, Weapon Proficiency 3, if the attack roll is successful you consume two fewer Chi points. You can use a number of Chi points at one time equal to one and a half times your Wisdom value.

The \textbf{fourth time} you take this feat, Weapon Proficiency 7, Wisdom 4, if the attack roll is successful you consume three fewer Chi points. You can use a number of Chi points equal to double your Wisdom value at one time.

\subsection*{Psychic Ray}\index{Psychic Ray}\label{psychic Ray}

\textbf{Requirement}: Psychic Strike, Wisdom 3, Weapon Proficiency 5

\textbf{Saving Throws}: +2 Reflexes, +1 Will

You can make a ranged attack within 30 feet using psychic energy.
The attack, on Touch, deals 1d6 force damage per point of Psychic spent focused on the damage.

It is possible to focus one or more Psychic points to increase the distance by 9 meters each time.
You cannot use more total Chi points (for distance and damage) than your Wisdom.

The \textbf{second time} you take this feat requirement Wisdom 3, Weapon Proficiency 9, you can use up to double your Wisdom score to enhance your Psychic ray.


\subsection*{Pure Blood}\index{Pure Blood}\label{pure blood}

\textbf{Requirement}: Animalia

\textbf{Saving Throws}: +1 Will, +2 Fortitude

With this Skill, each attack you make when you transform with Animalia causes 1 additional damage and is considered a +1 magical attack. By concentrating on your step you can leave the footprints of an animal that you can transform into and the terrain is considered doubly difficult.

The \textbf{second time} you take this feat, Magical Proficiency 8, when you use Animalia's feat you can perform a partial transformation or take the type of Movement or Senses of the creature you transform into. When you use the Animalia ability you can select a creature with a Challenge Rating increased by 1. Leaving different footprints is considered difficult terrain.

The \textbf{third time} you take this feat, Magical Proficiency 12, when you transform into an animal, you can use your Magical Proficiency in place of Weapon Proficiency on natural attacks. When you use the Animalia ability you can select a creature with a Challenge Rating increased by 1. Leaving different footprints is not considered difficult terrain.

Skills two and three are cumulative.

\subsection*{Quick draw}\index{Quick draw}\label{quickdraw}

You are extremely quick to draw your weapon.

\textbf{Requirement}: Weapon Proficiency 1

\textbf{Saving Throws}: +1 Reflexes, +1 Will

You can extract a weapon that is not too big for you with the cost of a Reaction.

The \textbf{second time} you take this Skill you can put away your current weapon and draw another one as a Move Action.

The \textbf{third time} you take this Skill you can put away your current weapon and draw another one as a free Action.

\subsection*{Rapid Shot}\index{Rapid Shot}\label{tirorapido}

\textbf{Requirement}: Dexterity 3, Accurate Shot, Weapon Proficiency 2

\textbf{Saving Throws}: +2 Reflexes

When using a bow, crossbow, or throwing a weapon, the penalties for multiple attack are lower.

Each projectile fired beyond the first takes a -4 to the cumulative attack roll (and not the -5). The first shot has a normal attack roll, the second has a -4, the third a -8 ...

\subsection*{Ready Defense}\index{Ready Defense}\label{readydefense}

\textbf{Requirement}: Weapon Proficiency 2

\textbf{Saving Throws}: +2 Reflexes

You are always alert and attentive when you risk your life. You have +4 Defense against attacks of opportunity.

\subsection*{Really evil person}\index{Really evil person}\label{personaveramentemalvagia}

\textbf{Requirement}: Weapon Proficiency 1

\textbf{Saving Throws}: +1 Reflex, +1 Will

Twice per day, add your Weapon Proficiency value to damage, to an opponent you wish to melee hit. The Feat must be declared before the attack roll. It costs one Action.



\subsection*{Reprisal}\index{Reprisal}\label{rappresaglia}

\textbf{Requirement}: Weapon Proficiency 1, at least follower of Gradh or Nedraf or Orlaith or Sumkjr

\textbf{Saving Throws}: +2 Will, +1 Fortitude

Seeing your friends hurt fills you with anger.
When a companion (or yourself) drops below half your Hit Points, you gain a +1 on attack rolls and Saving Throws. The maximum duration of the effect is 1 minute (6 rounds) per day and must be consecutive. The player chooses whether or not to activate the Feat. The companion must be within 9 meters.
You can take this Feat up to 3 times, each time the attack and Saving Throw bonuses increase by 1.

\subsection*{Sage}\index{Sage}\label{sapiente}

\textbf{Requirement}: Magic Proficiency 4

\textbf{Saving Throws}: +2 Will

Your interest and connection with magic is unmatched. You can know one more spell (while meeting the maximum level constraints you can choose).

\subsection*{Shoot and Run}\index{Shoot and Run}\label{shootandrun}

\textbf{Requirement}: Crossbow List 3

\textbf{Saving Throws}: +1 Fortitude, +1 Reflexes

While taking a move action, you can reduce the loading time of your crossbow by 1 action. In the case of Light or one-handed crossbows you can then reload it while moving, in the case of heavy crossbows you reduce the reload time by 1 Action.


\subsection*{Second Skin}\index{Second Skin}\label{secondapelle}\hypertarget{secondapelle}{}

\textbf{Requirement}: Weapon Proficiency 1

\textbf{Saving Throws}: +2 Fortitude

Constant use of Armour allows you to wear them without major penalties.

The penalty on proficiency checks granted by the Armour decreases by 1.

The \textbf{second time} you take this Feat, Weapon Proficiency requirement 6, the penalty to proficiency checks decreases by an additional 1. The penalty to movement penalties decreases by 1 meter.
You can sleep in medium Armour without being fatigued.

The \textbf{third time} you take this Feat, Weapon Proficiency requirement 11, the penalty on proficiency checks decreases by an additional 1. The penalty on movement penalties decreases by an additional 1 meter.
You can sleep in heavy Armour without being fatigued.


\subsection*{Silver Crane}\index{Silver Crane}\label{grudargento}

\textbf{Requirement}: Empty Fist List 2, Dexterity 1

\textbf{Saving Throws}: +2 Reflexes, +1 Will

To take advantage of these bonuses you do not have to bring armor or shields or objects that improve Defense. The listed abilities do not stack with the Enchanted Mountain Armor feat.

The \textbf{first time} you take this feat your base Defense is 11 + Dexterity + 1/3 of the points in Empty Fist + any and all.

The \textbf{second time} you take this feat, Empty Fist List requirement 4, your Initiative increases by 2 (only with unarmed attacks).

The \textbf{third time} you take this feat, Empty Fist List requirement 9 and Dexterity 2, you have a bonus on Will saving throws 2 (cumulative).

The \textbf{fourth time} you take this feat, Empty Fist List requirement 11, your base Defense and Initiative increase by 2 (cumulative).

The \textbf{fifth time} you take this feat, Empty Fist List requirement 13 and Dexterity 3, you have a bonus on Reflex saving throws of 2 (cumulative).

The bonuses are active even if you are not fighting.


\subsection*{Sneak strike}\index{Sneak strike}\index{Attack from behind
}\label{attaccoallespalle}\label{colpofurtivo}

\textbf{Requirement}: Weapon Proficiency 3

\textbf{Saving Throws}: +2 Reflex, +1 Will

When the opponent is melee attacked from behind, the first successful attack of the fight with a melee weapon causes two additional critical damage.

The \textbf{second time} you take this Feat, Weapon Proficiency requirement 6, you do 3 critical damage.

The \textbf{third time} you take this Feat, Weapon Proficiency requirement 10, you do 4 critical damage.

The \textbf{fourth} you take this Feat, Weapon Proficiency requirement 12, causes 5 critical damage.


\subsection*{Specialist}\index{Specialist}\label{tattico}

\textbf{Requirement}: Weapon Proficiency 1, Intelligence 1

\textbf{Saving Throws}: +1 Fortitude, +1 Will

You have an almost instinctive Feat to manage and predict the outcome of battles.

The \textbf{first time} you take this Feat you can exchange, round by round, the outcome of the Initiative between you and a companion respectively in melee range. Cost 1 Action.

The \textbf{second time} you take this Feat, Intelligence requirement 2, Weapon Proficiency 6, you can swap the initiative, round by round, between two of your companions that are within melee range of each other. Cost 1 Action.




\subsection*{Spring}\index{Spring}\label{Molla}

\textbf{Requirement}: Strenght 0

\textbf{Saving Throws}: +1 Reflex, +1 Fortitude

You can ignore the 3m run-up requirement before a jump.

The \textbf{second time} you take this Feat when you make a check to jump long or high, roll an extra 1d6.



\subsection*{Stepper}\index{Stepper}\label{passofelpato}

\textbf{Requirement}: Stealth 1

\textbf{Saving Throws}: +1 Reflex, +1 Fortitude

Your step is naturally silent.

The \textbf{first} time you take this Feat the penalty for moving at full speed with the Stealth Feat becomes -1d6.

The \textbf{second time} you take this Feat, Dexterity requirement 3, Stealth 8, you have no penalty to moving at full speed.


\begin{center}
	\includegraphics[width=0.8\linewidth]{immagini/teseo.png}
	
	\textit{Henry Justice Ford - Backstab!}
\end{center}

\subsection*{Stone Strength}\index{Stone Strength}\label{resistenzadellapietra}

\textbf{Requirement}: Constitution 0

Over time you have trained your Constitution to withstand shocks, transformations, poisons and anything else that wants to change your body. The first time you take this Feat, you gain a +2 bonus on your Fortitude save. The bonus is cumulative, +2 the first time, +1 the second time, +1 the third time.

The fourth time you take this Feat, you can decide to automatically succeed on a Fortitude save once per day as a reaction. It must be declared and does not make a Saving Throw.

\subsection*{Storm of Fury}\index{Storm of Fury}\label{tempestadifuria}

\textbf{Requirement}: Empty Fist list 7, Dexterity 1, Strength 1

\textbf{Saving Throws}: +2 Reflex, +1 Will

When you use this Feat you can declare to use the Storm of Fury as your only action (3 Actions).

Make a single attack roll -1d6 and on a hit deal critical damage equal to Empty Fist list/4 times.\\


\subsection*{Supreme}\index{Supreme}\label{supremo}\hypertarget{supremo}{}

\textbf{Requirement}: the sum of WP+MP is at least 4 points higher than the previous time it was taken.

\textbf{Saving Throws}: +1 to any two Saving Throws

Through the Supreme Feat it is possible to increase a Characteristic by 1 point respecting the rules of \hyperlink{aumentarelecaratteristiche}{Increase Characteristics} (page \pageref{aumentarelecaratteristiche}).

The Feat can be taken multiple times and the sum of WP+MP must fall within the range 1-4.5-8.9-12.13-16 only once.



\subsection*{Sure-footedness}\index{Sure-footedness}\label{passosicuro}

\textbf{Requirement}: Wisdom 1

\textbf{Saving Throws}: +2 Fortitude, +1 Reflex

It is the Feat not to be slowed down in a hostile environment. It is necessary to declare on which environment the Feat is taken. In these environments the natural terrain is not difficult for you.

\bigskip

\begin{tabular}{l|l}
	\textbf{Environment} & \textbf{Environment}\\
	\toprule
	Jungle & Aquatic \& Coastal\\
	Swamp & Hill \& Forest \\
	Plains & Desert \\
	Mountains & Glaciers \& Tundra \\
	Urban& Underground \\
\end{tabular}\\

Each time you take this Skill again you choose a different environment and add to the previous one or specialize in the same one.

The \textbf{second time} you take this Skill on the same terrain, you acquire a specific ability depending on the terrain.

\textit{Jungle / Forest / Hill / Plain}: your movement increases by 1 meter in this terrain\\
\textit{Costal / Aquatic}: swim speed equal to your movement\\
\textit{Swamp}: +2 on saving throws against poison\\
\textit{Desert}: Fire damage reduction equal to level\\
\textit{Mountain / Glaciers / Tundra}: Cold damage reduction equal to the level\\
\textit{Underground}: Twilight vision 9 meters\\
\textit{Urban}: +1 Language, +1 of your choice in two Knowledge


\subsection*{The bigger they are, the more noise they make when they fall}\index{The bigger they are, the more noise they make when they fall}\index{Giant Killer}\label{piusonogrossipiufannorumore}

\textbf{Requirement}: Weapon Proficiency 1

\textbf{Saving Throws}: +2 Fortitude, +1 Will

When you attack a creature at least 2 sizes larger than you, you do +1 additional damage for every 2 Weapon Proficiency points. If it's only one size larger, add 1 more damage for every 3 Weapon Proficiency points.




\subsection*{The Patron is with me}\index{The Patron is with me}\label{ilpatronoeconme}

\textbf{Requirement}: Devote, Sum common Traits with Patron 2

\textbf{Saving Throws}: +1 Will, +1 Reflex

The \textbf{first time} you take this Feat 1 time per day you can reroll a die obtained in the Magic Test for casting a spell.

The \textbf{second time} you take this Feat, requirement sum Traits common to the Patron 6, 2 times a day you can reroll 2 dice obtained in the Magic Test for casting the spell. 

The \textbf{third time} you take this Feat, requirement sum Traits common to the Patron 12, 3 times a day you can rerol 3 dice obtained in the Magic Test for casting the spell. 

The Feat can also be declared after the roll of the dice. Any new value obtained with the new roll must be kept or this Feat is used again.

\subsection*{The Patron is my Weapon}\index{The Patron is my Weapon}\label{ilpatronoelamiaarma}

\textbf{Requirement}: Sum Traits common to Patron 1

\textbf{Saving Throws}: +1 Will, +1 Reflex

The \textbf{first time} you take this skill you have a +1 to attack roll and damage roll when using your patron's favored weapon.

The \textbf{second time} you take this Skill, common Trait sum requirement 5, Weapon Proficiency 1, the penalty for multiple attacks with the Patron's favored weapon becomes -4.

The \textbf{third time} you take this Skill, requirement adds Traits common with the Patron 10, Weapon Proficiency 2, add +1d6 to the attack roll when you make the third attack with the Patron's weapon.

The \textbf{fourth time} you take this Skill, requirement sums Patron Traits 15, Weapon Proficiency 3, you increase the damage of your Patron weapon by one rank.

The \textbf{fifth time} you take this Skill, requirement sums Common Traits with the Patron 19, Weapon Proficiency 4, you gain an additional +1 to Attack Roll and +1 to Damage. The first successful attack in the round with the Patron's weapon always causes critical damage.


\subsection*{The shield is my friend}\index{The shield is my friend}\label{loscudoemioamico}

\textbf{Requirement}: Weapons Proficiency 1

\textbf{Saving Throws}: +1 Fortitude, +1 Reflex

You can wear a buckler without having to make a magic check when you cast a spell.

The \textbf{second time} you take this skill, Weapon Proficiency requirement 3, you can use medium shields without having to make a Magic Check.

The \textbf{third time} you take this skill, Weapon Proficiency requirement 5, the penalty on the attack roll decreases by 1.




\subsection*{This is my dagger}\index{This is my dagger}\label{questoeilmiopugnale}

\textbf{Requirement}: Weapon Proficiency 1

\textbf{Saving Throws}: +2 Fortitude, +1 Reflex

When you deal critical damage with your dagger you add your Weapon Proficiency to the damage. The Feat is usable once per opponent and is automatically applied to the first critical damage done.

\subsection*{This is my weapon!}\index{This is my weapon!}\label{questaelamiaarma}

\textbf{Requirement}: Weapon Proficiency 1

\textbf{Saving Throws}: +2 Fortitude, +1 Will

Every time you hit the same opponent, starting from the second round, you do an additional damage (Max +1 per round of combat, even if you hit him multiple times in the round) up to a maximum of +5. The first time you don't hit the opponent in the round the bonus returns to +0. The bonus can only be applied to one opponent at a time.

The \textbf{second time} you take this Feat you can miss the opponent with one hit and not lose the benefits.


\begin{center}
	\includegraphics[width=0.6\linewidth]{immagini/turning-undead-six.png}
\end{center}

\subsection*{Turning undead}\index{Turning undead}\label{scacciarenonmorti}

\textbf{Requirement}: Sum Traits common 2

\textbf{Saving Throws}: +2 Will, +1 Fortitude

By focusing on the power of your Patron you channel positive energy and drive away or destroy the undead.

Make a check by rolling 1d6 + the trait score in common with the highest ranked Patron, this total is your God Power.

Starting with the weakest undead around you, within a 6 meters radius, subtract the Challenge Rating from the God Power until it is no longer enough to turn anyone.

If the God Power is at least double the Challenge Rating, the undead is destroyed and double the Challenge Rating is subtracted from the God Power value.

The undead make a DC Will save equal to 10 + Divine Power to resist the effect. The value of the Challenge Rating is still subtracted even if the Saving Throw is successful.

The Feat is usable a number of times per day equal to Wisdom, but an undead can only be affected once per day by your effect.

An undead that is turned is below \hyperlink{condizionepaura}{Fear} for 1d4 rounds, a destroyed undead is reduced to dust and divine energy.

A Devotee of Sixiser, instead of turning and destroying, can dominate the undead for 2d4 rounds or 1 real hour respectively.

A Devotee of Thaft get +1d6 at God Power.


\subsection*{Two-weapon combat}\index{Two-weapon combat}\index{Two Weapons}\label{duearmi}

\textbf{Requirement}: Dexterity 2, Strength 1, Weapon Proficiency 2

\textbf{Saving Throws}: +2 Reflexes, +1 Fortitude

The \textbf{first time} you take this Skill the constant and continuous training allows you to reduce the multiattack penalty given by the attack with the secondary weapon. When you attack with your secondary weapon you gain a -4 hit penalty instead of -5 if the weapon is light.

\textbf{Requirement} Dexterity 3, Weapon Proficiency 12

The \textbf{second time} if the secondary weapon is not light you do not accumulate the additional -3 to hit.

\textbf{Requirement} Weapon Proficiency 18

The \textbf{third time} the first attack made with the secondary weapon does not accumulate the multiple attack penalty.

\subsection*{Typist}\index{Typist}\label{typist}

\textbf{Requirement}: Magical Competence 1

\textbf{Saving Throws}: +1 Fortitude, +1 Will

You are extremely quick at copying new spells to your Tome of Magic. The time to copy a spell decreases from 1 hour to 30 minutes per page (a spell occupies a number of pages equal to its level). The cost in inks goes from 10 gp per page to 5 gp per page.


\subsection*{Uncanny Dodge}\index{Uncanny Dodge}\label{schivataprodigioso}

\textbf{Requirement}: Dexterity 3

\textbf{Saving Throws}: +2 Reflexes

As a Reaction to an attack Action you can add +2 to your Defence. You can apply the bonus after your opponent's attack roll even after you know he has hit you.

You can use the Feat up to 3 times per day.

The \textbf{second time} you take the Feat, Weapon Proficiency requirement 4, an opponent does not take the flanking strike bonus against you.

The \textbf{third time} you take the Feat, Weapon Proficiency requirement 8, an opponent does not get the bonus to hit for backstabbing you.


\subsection*{Unlucky}\index{Unlucky}\label{sfortunato}

\textbf{Requirement}: Lucky, at least 6 points in the sum of Traits

\textbf{Saving Throws}: +1 Fortitude, +1 Will


\subsection*{Vampire}\index{Vampire}\label{abilitavampiro}

\textbf{Requirement}: Blood Smell (Perks)

\textbf{Saving Throws}: +2 Fortitude, +1 Will

Your thirst for blood becomes a cure. The bloodthirst bonus can increase up to +5.

If the bonus increases from +3 to +4 or +5 you can, by swallowing the opponent's blood, you can heal yourself of 1d6 using a 2 Actions



\subsection*{Versatile Litany}\index{Versatile Litany}\label{itaniaversatile}

\textbf{Requirement}: Proficiency Perform 6

\textbf{Saving Throws}: +1 Will, +1 Reflex

Through your performance, you can choose to instill courage or fear in creatures within 10 meters of you. Each round you can decide to apply up to 2 modifiers among: +1d6 bonus to attack roll or +4 to Defence or -1d6 to attack roll or -4 to Defence. Changing bonuses costs 1 Action.

The opponent is allowed a DC Will save equal to 10+CH+Perform score. A creature that succeeds at its Saving Throw is immune to new manifestations of your power for 24 hours that day.

Activating and maintaining the Feat takes 2 Actions. You can maintain the Feat for a number of rounds, even if not consecutive, equal to your Perform score x 3 per day. Creatures must continue to see/hear the performance to remain affected.


\subsection*{Warrior of Magic}\index{Warrior of Magic}\label{guerrierodellamagia}

\textbf{Requirement}: Weapon Proficiency 2, Magic Proficiency 2

\textbf{Saving Throws}: +1 Will, +1 Reflex

You don't just go the way of magic or even that of the sword, your style embraces both in a slash of pure magic.

The \textbf{first time} you take this Feat you are able to discharge a melee ranged spell with your weapon. You make the normal Attack Roll (3d6+WP+Strength+...) and if you hit, in addition to the damage from the attack, you also discharge the spell. You are not distracted, but you must make a Magic Test. This way you can only make one attack with the weapon and the spell can't take more than two Actions to cast. It costs 3 Actions.

The \textbf{second time} you take this Feat, requirement Weapon Proficiency 3, Magic Proficiency 3, by consuming 3 Actions you are able to discharge a spell with a ranged weapon. You do not make a Magic Test while performing these Actions. You are distracted by them. First time rules apply.

The \textbf{third time} you take this Feat, requirement Weapon Proficiency 6, Magic Proficiency 3, by consuming 3 Actions you are able to combine up to two melee attacks with a spell's discharge. First time rules apply.

You cannot download spells higher than 3 with this Feat.



\subsection*{Weakening Blow}\index{Weakening Blow}\label{colpoindebolente}

\textbf{Requirement}: Sneak Strike 3, Weapon Proficiency 12

\textbf{Saving Throws}: +2 Reflex, +1 Will

Suppressing Strike is an advanced form of sneak strike. Each Weakening Strike lowers Strength or Dexterity (player's choice) by the number of times you took Sneak Strike.

The opponent is allowed a Fortitude saving throw with a DC equal to the attack roll. It causes the additional damage of the Sneak Strike or the loss of ability points.


\subsection*{Weapon Artist}\index{Weapon Artist}\label{artistadellafuca}

\textbf{Requirement}: Weapon Proficiency 2

\textbf{Saving Throws}: +1 Will, +1 Fortitude

Choose a Weapon List, on these weapons you get +1 to hit.

The Feat can be taken multiple times and the Weapon List score must be 4 times higher than the times this Feat was taken.

If you take \textbf{4 times} this Feat on the same Weapon List the bonuses on hitting are reduced to +1, instead of +4, but you make two attack rolls for the first two attacks of the round and choose which roll hold.

\begin{center}
	\includegraphics[width=0.9\linewidth]{immagini/Historia_Mundi_Naturalis.png}
	\textit{Woodcut illustration from an edition of Pliny the Elder's Naturalis Historia (1582)}
\end{center}


\subsection*{Weapon Focus}\index{Weapon Focus}\label{armafocalizzata}

\textbf{Requirement}: Weapon Proficiency 1

\textbf{Saving Throws}: +1 Reflex, +1 Fortitude

Choose a weapon. Gain +1 to Initiative and Attack Roll when using this weapon you are proficient with.


\subsection*{Whirlwind Attack}\index{Whirlwind Attack}\label{attaccoturbinante}

\textbf{Requirement}: Weapon Proficiency 12

\textbf{Saving Throws}: +2 Reflexes, +1 Fortitude

Using 3 Actions you can perform a single attack (with a 5 penalty on the attack roll) against all melee opponents around you.


\subsection*{Without Trace}\index{Without Trace}\label{senzatraccia}

\textbf{Requirement}: Surefootedness

\textbf{Saving Throws}: +2 Will, +1 Reflex

The Feat not to leave fingerprints in the chosen environment. Each time you take this Feat you can choose a different environment (see Sure Step Feat) whose Feat you took. The difficulty of the Tracking check to pursue you increased by 10.




\end{multicols}

%\vfill

%\begin{center}

%\includegraphics[width=0.8\linewidth]{immagini/Granblue.Fantasy.full.2108782.png}

%\filltopageendgraphics[width=0.7\linewidth]{immagini/Granblue.Fantasy.full.2108782.png}

%\textit{around a fire, victorious another day!}
%\end{center}

\pagebreak

\begin{multicols}{2}


\subsection{Grouping Feats by Style}

To ease the transition from those coming from other role-playing games with classes, the Feats for the more canonical classes are divided here.

They are clearly only indications, in OBSS the character can be built as you prefer and as the story that lives is educating him.

These are suggestions to facilitate the construction of a character for those who are new to the Old Bell School System, of course it is possible to "draw" from all groupings presented!

\end{multicols}

\bigskip

\begin{multicols}{3}


\begin{flushleft}
\textbf{Warrior}

Blade Dance\\
Continue\\
Dancing Scourge\\
Fighting Blindly\\
I said FALL!\\
Iaijutsu\\
Kill shot\\
Leathery skin\\
Lightning reflexes\\
Lunge\\
My head is harder\\
One arm, one weapon\\
Powerful blows\\
Ready Defense\\
Second skin\\
Stay down!\\
Stone resistance\\
This is my weapon!\\
Truly evil person\\
Weapon Focus\\
Whirlwind attack\\


\textbf{Barbarian}

Ferocity\\
Forged in fury\\
Fury\\
Human mountain\\
Instinctive Knowledge\\
Iron will\\
My death your death\\
One arm, one weapon\\

\textbf{Thief}

Called Arrow, delivered Arrow\\
Crippling Blow\\
Dodge traps\\
Improvise\\
Infuriate\\
Juggler\\
Masterful Dodge\\
First Blood\\
Quick Draw\\
Opportunist\\
Shot and Run\\
Sneak strike\\
This is my dagger\\
Touch and run\\
Weakening Blow\\

\textbf{Paladin}

Channel energy\\
Channel selective energy\\
Lay on of hands\\
Mental Wall\\
Pitiful Touch\\
Reprisal\\

\textbf{Bard}

Adept of Magic\\
Coordinated damage\\
Instill Courage\\
Instill Fear\\
Loaded dice\\
Lucky\\
Polyglot\\
Tactical\\
Versatile litany\\

\textbf{Ranger}

Accurate shot\\
Clinical eye\\
Defend Mount\\
Double portion\\
Hawk eye\\
Hollow head\\
Horse archer\\
Hound\\
Quick shot\\
Shot and Run\\
Sure-footedness\\
Two-weapon combat\\
Without Trace\\

\textbf{Druid}

Adept of Magic\\
Animalia\\
Brew potions\\
Daughter of Shayalia\\
Elemental form\\
Pureblood\\
The Patron is with me\\

\textbf{Cleric}

Adept of Magic\\
Devout Armour\\
Loyal\\
One with the magic\\
Power of Patron\\
Specialist\\
The Patron is with me\\
Turning undead\\
Typist\\

\textbf{Wizard/Sorcerer}

Adept of Magic\\
Create Greater Magic Items\\
Create Magic Items\\
Create Mythic Magic Items\\
Create Wonderful Magic Items\\
Decipher magical writings\\
Detect the Magic\\
Elementalist\\
Extended battery\\
Magic Battery\\
One shot one kill\\
One with the magic\\
Pet / Familiar\\
Powerfull Magic\\
Specialist\\
The Patron is my Weapon\\
The Patron is with me\\
Wise\\

\textbf{Monk}

Fast step\\
Human mountain\\
Iron fist\\
Magic Mountain Armour\\
Mental Wall\\
Phoenix Wings\\ 
Psychic energy\\
Psychic Ray\\
Psychic Strike\\
Silver Crane\\
Spring\\
Storm of Fury\\

\textbf{Gish (Warrior/Mage)}

Adept of Magic\\
War Enchanter\\
Devout Armour\\
Infuse Magical Energy\\
Infuse Greater Magical Energy\\
Magic roots\\
One with the magic\\
Prudent Enchanter\\
The Patron is my Weapon\\
The shield is my friend\\
Warrior of Magic\\

\end{flushleft}

\end{multicols}

\pagebreak

\section{Familiar}\index{Familiar}\label{famiglio}
\medskip

\begin{changemargin}{0cm}{0.5cm}\begin{emphasis}{
Mr. Wing's nephew: Look mister, there are three rules to follow, though.

Rand: Oh yeah? And what would they be?

Mr. Wing's nephew: Keep him out of the light, he hates bright light, especially the sun. He would die. And he keeps him away from the water, don't let him get wet. But most importantly, the rule he must never forget is that even if he cries, even if he makes a show of it and begs you, you should never, ever feed him after midnight. Do you understood? (Gremlins, Film, 1984)} \end{emphasis}\end{changemargin}\medskip


\begin{multicols}{2}

\lettrine[lines=2, lhang=0.33, loversize=0.25, findent=1.5em]{P}{ets} are animals chosen by the character, through the Pet Feat, to help him in adventures and for company . A familiar has a special bond with its master.

A familiar is a normal animal but is treated as a magical creature for the purpose of determining any effect depending on its type.

Only a normal, unmodified animal can become a familiar.

A familiar grants its master special abilities, these special abilities apply only when master and familiar are within 100 m of each other.

A particular 4-hour ritual in the animal's native environment is required for it to become a familiar.

If a familiar is dismissed, lost, or dies, it can be replaced a week later with a special ritual that costs the character 2 points of temporary Constitution. It takes 8 hours to complete the ritual.

\medskip


\textbf{Table: Types of Familiar}\index{Table Types of Familiar}

\medskip

\begin{tabularx}{0.45\textwidth}{lX}
\textbf{Family} & \textbf{Feat gained from master}\\
\toprule
Owl & +2 to Arcana Checks\\
Crow & +2 on Intimidate checks\\
\textit{Dobi} & +2 on Will save vs Enchantment\\
Weasel & +1 on Intelligence checks\\
Hawk & +2 on Sight Awareness\\
Cat & +2 on Stealth checks\\
Owl & +2 on Hearing Awareness\\
Otter & +2 to Swim checks\\
Lizard & +2 on Survival checks\\
Bat & +2 on Acrobatics checks\\
Rat & +1 on Save vs Disease\\
Hedgehog & +1 on Will save\\
Toad & +2 on Saving Throw vs poison\\
Monkey & +2 to Fairy Hands, Escape Artist checks\\
\textit{La Topi} & you become Topi's pet!!!\\
Fox & +1 on Reflex saves\\
\end{tabularx}

\bigskip

Use the base stats of a creature of the familiar's species, making the following changes.

\medskip{}

\textbf{Attacks}: Use master's Weapon Proficiency if higher. Use the familiar's Dexterity or Strength modifier, whichever is higher, to calculate the familiar's attack bonus with natural attacks. The damage is equal to that of a normal creature of the familiar's species. The familiar acts on the master's round.

\medskip

\textbf{Defence}: The familiar has a Defence equal to that of the standard animal plus a bonus due to the master's Magic Proficiency. See the Familiar Abilities table.

\medskip

\textbf{Saving Throw}: For each Saving Throw, use the familiar's save or the master's save, whichever is better. The familiar applies its characteristics values as a bonus on Saving Throws and takes none of the bonuses its master may have.

\textbf{Actions of Familiar}\index{Actions of Familiar}: commanding a familiar takes 1 Action. The familiar performs 2 Actions per round. Without commands, the Familiar does nothing but defend itself and attack those who attack it.

\medskip

\begin{center}
\includegraphics[width=0.65\linewidth]{immagini/donnadrago2.png}

\textit{Henry Justice Ford}
\end{center}

\medskip

\textbf{Description of Familiar Abilities}

All familiars possess special abilities (or attribute them to their masters) according to the master's Magic Proficiency score. The special abilities listed in the table are cumulative.


\end{multicols}


\textbf{Table: Familiar Feats and bonuses}\index{Table Familiar Feats and bonuses}

\medskip

\begin{tabularx}{0.95\textwidth}{cccX}
\textbf{Master's MP} & \textbf{Defence Bonus} & \textbf{Intelligence Bonus} & \textbf{Special}\\
\toprule
1-2 & +1 & 0 & Alert, Share Spells, \\
& & & Empathic Link\\
3-4 & +1 & +1 & Deliver Spell Touch\\
5-6 & +1 & +1 & Talk to Animals of Its Kind\\
7-8 & +2 & +1 & Talk to Master\\
9-10 & +2 & +2 & -\\
11-12 & +2 & +2 & See Through Familiar\\
13-14 & +3 & +2 & -\\
15-16 & +3 & +3 & -\\
17-18 & +3 & +3 & -\\
19-20 & +4 & +3 & -\\
\end{tabularx}

\begin{multicols}{2}

\textbf{Master's Magic Proficiency}: The number listed here is the familiar's master's Magic Proficiency value.

\textbf{Defence Bonus}: The number listed here is in addition to the familiar's Defence.

\textbf{Intelligence Bonus}: This value is added to the familiar's Intelligence score.

\textbf{Special}: The special abilities gained by the familiar (and/or master).

\textit{\textbf{Alert}}: When the familiar is within arm's reach of the master, the master gains +1 on Awareness checks

\textit{\textbf{Sharing Spells}}: At his discretion, the master may cast any Spells that affect "himself" on his familiar (such as a Touch Spell), in place of himself.

The master can cast spells on his familiar even if they normally do not affect creatures of the familiar's type (magical creatures).

\textit{\textbf{Empathic Link}}: The master has an empathic link with his familiar up to a distance of 1 km. The master cannot see through the familiar's eyes, but can communicate with it telepathically. Due to the limited nature of the bond, only generic emotions (fear, tranquility, joy...) can be communicated.

\textit{\textbf{Deliver Touch Spells}}: The familiar can transmit Touch Spells for him. If the master and familiar are within 10 meters when the master casts a touch spell, he can designate his familiar as the "spell deliverer" (if touch).

The familiar can deliver the Spell just like the master. The familiar use one of his action to deliver the spell attack. 

\textit{\textbf{Talk to Master}}: The familiar and the master can communicate verbally, as if using a common language. Other creatures or animals are unable to understand their conversation, except by using magical aids. The ability works within 50m and they must be able to hear each other's words.

\textit{\textbf{Talking with Animals of Its Species}}: The familiar is able to communicate with animals of its generic species: bats with bats, mice with rodents, cats with felines, hawks and owls, and crows with birds, snakes and lizards with reptiles, toads with amphibians, monkeys with other primates, weasels with stoats and mustelids... Communication is limited by the Intelligence of the creatures the familiar communicates with.

\textit{\textbf{See Through Familiar}}: The master can see through the familiar. Activating this ability costs 1 Immediate Action. The familiar must be within 50 meters.

\medskip

However special, intelligent and unique a familiar remains an animal and as such cannot use magic items or scrolls, it can even use a potion if it has the ability to drink it. An very intelligent familiar could perform simple and immediate tasks.

\end{multicols}

\vfill

\begin{center}
\includegraphics[width=0.5\linewidth]{immagini/familiar.png}

\textit{Henry Justice Ford, a very good familiar...}
\end{center}



\pagebreak

\section{Other Special Abilities}


\begin{multicols}{2}

\lettrine[lines=2, lhang=0.33, loversize=0.25, findent=1.5em]{T}{hese} Abilities are not selectable by the player, but can be innate in creatures.

\subsection{Ethereal}\index{Ethereal}\label{etero}

A creature that has become Ethereal is located on the Ethereal Plane that overlaps the Material Plane.

An ethereal creature is invisible, insubstantial, and capable of moving in any direction, even up and down. An ethereal creature can move through solid objects, including other living creatures. An ethereal creature can see and hear what is happening on the Material Plane, but everything appears gray and ephemeral. An ethereal creature's sight and hearing on the Material Plane are limited to a distance of 10 meters.

Spells if not properly formulated and modified do not act on ethereal creatures. An ethereal creature has resistance to damage against light or void, and ignores all other forms of energy.

An ethereal creature can't attack a material creature, and spells cast while in the ethereal state can affect only ethereal elements. Some creatures or material objects have special attacks or effects that also work on the Ethereal Plane. An ethereal creature treats all other ethereal creatures as if they were all material.

\subsection{Damage Resistance}\index{Damage Resistance}\label{resistenzaaldanno}

Certain creatures or protections grant the ability to resist a type of damage.

Being resistant to damage automatically means halving the damage received before applying any other protection or Saving Throw.

Damage Resistance can also take values. When Damage Resistance: Lightning is written, the subject automatically halves the electricity damage, if written Damage Resistance: Lightning 10, it means that it reduces the electricity damage by 10 points before applying the Saving Throw or other bonuses.

A creature with resistance to fire halves (reduces) all damage it takes from flames, magical or otherwise.

There may be ability or spells that ignore this resistance. Multiple equal resistances don't add up, due to the fact that two objects give me fire resistance I don't reduce the damage by a quarter, if only one applies.
If an ability ignores damage resistance it will pass resistance even if I have two sources of resistance.

\subsection{Damage Reduction - DR}\index{Damage Reduction - DR}\label{resistenzaaldannodr}\hypertarget{riduzionedeldanno}{}

Certain creatures or Abilities grant the supernatural ability to resist damage from certain types of weapons or up to a certain amount (per attack).

Usually assumes the value of XX/ZZ or how much damage (XX) is ignored if you are not attacked with (ZZ). Ignoring damage also means that effects connected to the attack don't work, such as poison on the weapon.

A single DR is applicable in case there are more than one at the same time, the choice must be made at the start of the battle and remains the same until the battle is over.

\begin{center}
\includegraphics[width=0.8\linewidth]{immagini/morteachille.png}

\textit{Paris shot Achilles with an arrow - Pieter Paul Rubens - Date 1630-1632}
\end{center}

Certain weapons, particularly magical, can ignore DR \index{Ignoring DR}

\medskip

\textbf{Table: Magic Weapon Equivalence}\index{Table Magic Weapon Equivalence}


\begin{tabular}{lll}
\textbf{DR to overcome} & \textbf{Enchantment} & \textbf{Attack}\\
& \textbf{weapon}& \textbf{Natural}\\
\toprule
Enchantment +1 & +1 & Level 3\\
Enchantment +2 & +2 & Level 6\\
Cold Iron & +2 & Level 9\\
Silver & +2 & Level 9\\
Adamantium & +3 & Level 12\\
\end{tabular}

\medskip

\textbf{Projectiles (arrows, darts, rocks) fired from magical weapons are NOT considered magical.}\index{Magical Projectiles}

\subsection{Magic Resistance}\index{Magic Resistance}\label{resistenzaallamagia}

Magic resistance can be indicated in two different ways.

It can be indicated with a dice, e.g. \textit{Resistance to Magic. The deva has +1d6 on saving throws against spells and other magical effects}. In this case, the bonus applies as indicated.

Or followed by a number and level, e.g. \textit{Magic resistance: 3lv}. In this case the creature is unaffected by spells of that level or lower. A spell is considered one level higher for each Magic Critical gained in the Magic Test.

While unaffected by direct effects, it is still affected by indirect effects, for example it can fall into the pit created by a Disintegrate spell.

Magic resistance can't be lowered even by the creature that has it.

\subsection{Damage Immunity}\index{Damage Immunity}\label{immunitaaldanno}

It is extremely rare but there are creatures or magical effects that make one immune to a form of damage, whether it be physical (weapon damage..) or magical (the various forms of energy).

A creature immune to a form of damage takes no damage from that attack. However, a creature that has the ability to have its own damage irresistible, meaning that it can't be reduced by resistance, will only partially penetrate the creature's immunity, making it only resistant to that damage.

A creature that reads "Immunity to Void Damage, Poison; Weapons +2" means that it is immune to Void damage, Poison damage, and that to wound it requires a weapon with a +3 magic bonus or higher, or a character attacking with weapons natural and is level 12 or higher or has taken the Empty Fist Weapon List at least 6 times (page \pageref{equivalenzaarmimagiche}).

\subsection{Vulnerability to Damage}\index{Vulnerability to Damage}\label{vulnerabilitadanno}

Certain creatures or spells make some effects more effective by causing greater damage to the vulnerable subject.

Being vulnerable to a specific type of damage automatically means double the damage received before applying any other protection or Saving Throw.

A creature with a vulnerability to fire doubles all damage taken then makes the Saving Throw indicated by the spell or effect if possible.

\subsection{Fear}\index{Fear}\label{paura}

Spells, Magic Items, and certain creatures can affect characters with fear. A creature with fear can't suppress its aura if it's innate unless otherwise noted. The difficulty with which to make the Will save is always noted. A creature immune to fear can't be frightened whether the source is natural or magical.

\textbf{Frightened}\label{Frightened}\index{Frightened}

A frightened creature has -1d6 on attack rolls, Saving Throws, and proficiency checks as long as the source of its fear is visible. A frightened creature cannot voluntarily approach the source of its fear.

\subsection{Paralyzed}\index{Paralyzed}\label{paralyzed}

There are several methods to Paralyze a creature, both magical and natural. While natural ones often have systems for freeing oneself later, magical systems can provide for freeing oneself from paralysis or not, perhaps only after a certain period of time.

A paralyzed character cannot perform Actions or Reactions or speak, melee attacks against her have a +2d6 bonus. The creature is aware of its surroundings and does not drop objects. The creature automatically fails Reflex saving throws. The creature loses its Dexterity bonus to Defense.


\end{multicols}


\vfill

\begin{center}
\includegraphics[width=0.4\linewidth]{immagini/the-scream.png}

\textit{The Scream (original title: Skrik)\\Edvard Munch - Date 1893–1910}
\end{center}

\pagebreak

\section{The Magic}\index{Magic}\label{lamagia}

\begin{changemargin}{0.3cm}{0.3cm}\begin{emphasis}{
The magic is not in the pendulum, but in whoever uses it. (NCIS - Crime Control Unit)\\


Thou shalt not suffer a witch to live (Exodus, 22 18)\\

A sorcerer is never late, Frodo Baggins. Nor in advance. He arrives exactly when he intends to. (Gandalf, The Lord of the Rings - The Fellowship of the Ring. J.R.R. Tolkien)} \end{emphasis}\end{changemargin} \medskip


\begin{multicols}{2}

\lettrine[lines=2, lhang=0.33, loversize=0.25, findent=1.5em]{M}{agic} permeates the game worlds and its most common form is that of a spell. This chapter provides the rules for casting spells.

\medskip

\subsection{What is a Spell?}\index{What is a Spell?}\index{Definition of Spell}

A spell is a manifestation of power. Each spell is the result of power and knowledge, the spellcaster is a superior medium who channels the power of the Patrons. In casting a spell, a character composes gestures, words and uses objects that connect him to the source.

Spells can manifest weapons or protective barriers, can inflict damage or heal vital energies. Countless spells have been created throughout the history of the multiverse, many of which have been forgotten. Some may still be hidden within the pages of dusty tomes of spells within ancient ruins or locked away in the minds of dead deities. Or they could one day be reinvented by a character with enough power and ability to do so.

\subsection{How to do magic! (In summary)}\index{How to do magic! (In summary)}

Your character must have taken at least one Magic List by investing the first point in Magic Proficiency or having taken the Adept of Magic Feat.

Access to the Magic List allows you to access and therefore be able to learn the spells that belong to them. By assigning a point to Magic Proficiency you can access the Universal List.

Magic Proficiency allows you to have more Spell Points, more spells and also to make your spells harder to resist and together with the Magic Adept Feat to access higher level spells.

Don't worry though, this chapter contains everything you need to know!

\subsection{The characteristics of spells}\index{The characteristics of spells}\label{caratteristicheincantesimi}

The description of each spell begins with a block of information that includes the spell's level, Spell Lists, casting time, range, components, and duration. The rest of the description informs us of the effect of the spell.

When a character casts any spell, the following basic rules are used regardless of the spell's effect.

\medskip

\begin{center}

\includegraphics[width=0.6\linewidth]{immagini/Hex32.png}

\textit{The Witchcraft Art of Jacques de Gheyn II}

\end{center}


\subsubsection{Casting Time}\index{Casting Time}\label{magietempodilancio}\index{Spell, Actions for casting}

Most spells can be cast with two Actions. Some spells take an immediate action, a reaction action, or much longer to cast.

\textbf{Immediate Action}

A spell cast as an immediate action is especially quick. You can use an immediate action during your round to cast the spell that is immediate, as long as you haven't already taken an immediate action during your round. You cannot cast another spell during the same round, unless it is a 0-level spell (called Cantrips).

\textbf{Reactions}

Some spells can be cast as reactions. These spells take a fraction of a second to create and can be cast in response to an event. If a spell can be cast as a reaction, the spell description tells you exactly when you can. You must have a Reaction Action available and not have used it yet.

\textbf{Longer Casting Time}

Certain spells take longer to cast: minutes or even hours. When you cast a spell with a casting time longer than two Actions, each round after the first is considered used in casting the spell. For those rounds, it's as if you had to maintain concentration even if you don't pay the cost of maintaining it.

In the final round, when the casting time is up, you roll the new initiative and use an Action to cast the spell.


\subsubsection{Magic Lists}\hypertarget{lescuoledimagia}{} \index{Magic List}\label{magielistadimagia}\index{Spell, Magic List}

Magical traditions throughout Yeru have formalized spells over the millennia into homogenous lists of type and effect. There are therefore lists that concern Fire or other elements, illusions, healing energies...

The Lists presented here are only those codified and taught in schools of magic. Ancient legends tell of further lists created, curated and disseminated in narrow circles or sects. One such secret list is that of the Devout Gnomes of Shayalia, a purely natural list that mixes the traditional List of Animals and Plants with some spells from the Lists of Elements.
Other more obscure lists are those of demons or Aboleths, some others are related to membership in groups of Devotees. Other more nefarious lists come to corrupt the soul of the characters by also imposing the Traits. These lists will normally be closed to the character but it is not certain that with the increase of the Magic Proficiency he himself will not create new lists of spells.

Magic Lists help describe spells; they have no rules of their own, although some rules may refer to these lists. The related characteristic is indicated next to the name of each List.

\begin{itemize}[leftmargin=*]
\item
\textit{Abjuration} (INT) deals with spells of a protective nature, although it also contains some with aggressive use. These spells create magical barriers, negate harmful effects, harm violators, or banish creatures to other planes of existence.

\item
\textit{Water} (DEX) are the spells that act on the cold and water element and minimally heal

\item
\textit{Air} (CAR) pertains to spells that manipulate and use air as well as electricity.

\item
\textit{Enchantment} (CAR) affects spells that act on the mind of others, influencing or controlling their behavior. These spells can make enemies consider you a friend, force creatures to take certain actions, or even control another creature like a puppet.

\item
\textit{Animals and Plants} (WIS) These are spells that affect animals and plants, natural or magical.

\item
\textit{Heal} (WIS) concerns spells that allow you to recover physical and mental energies and undo weaknesses and poisons.

\item
\textit{Divination} (WIS) pertains to spells that reveal time-lost, forgotten information, visions of the future, the location of hidden objects, the truth behind illusions, or images of distant people and places.

\item
\textit{Conjuration} (INT) pertains to spells that transport objects and creatures from one place to another. Some spells call creatures or objects to the caster's side, while others allow the caster to teleport from one place to another. Some summons create items or effects out of thin air.

\textit{Fire} (STR) The most dangerous spells are in here, with everything needed to burn and incinerate.

\begin{center}
\includegraphics[width=0.65\linewidth]{immagini/Leonids-1833.png}

\textit{The most famous depiction of the famous 1833 Leonids \hyperlink{sciamedimeteore}{Meteor Storm}}
\end{center}


\textit{Illusion} (WIS) pertains to spells that trick the senses and minds of others. They make people see things that don't exist, they don't notice things that exist, they make them hear false noises or remember things that never happened. Some illusions create ghostly images that anyone can see.

\item
\textit{Summoning} (COS) pertains to spells that manipulate magical energy to produce a desired effect.

\item
\textit{Necromancy} (COS) pertains to spells that manipulate the energies of life and death. These spells can bestow an additional reserve of life force, drain the life energy from another creature, create undead, or even bring the dead back to life (if granted).

In OBSS only a Patron has enough power to bring a dead man back to life.

\item
\textit{Earth} (COS) Spells that act and move the earth

\item
\textit{Transmutation} (DEX) pertains to spells that change the properties of a creature, object, or environment.


\item
\textit{Universal} some spells are cornerstones of magic itself and as such accessible to all spellcasters. To access the spells contained in this Magic List you must have at least one point in Magical Expertise. The maximum level of spells you can cast is equal to the number of times you have taken Adept of Magic.

\end{itemize}

\begin{changemargin}{0.3cm}{0.3cm}\begin{narrator}{In OBSS players have potentially access to all the spells in the known lists with therefore the possibility of having an extremely varied and powerful set of spells. To ensure that the choices are different between the various players and between the adventures, insist on the characterizing aspect of one list compared to another. The Magic Lists taken give a thickness and a different role to each other. Also make sure that new spells are not as easy to find as treasure.

}\end{narrator}
\end{changemargin}


\subsubsection{Range}\index{Range, Spell}\label{magiegittata}\index{Spell, Range}

The target of a spell must be within range of the spell. For a spell like Arcane Dart, the target is a creature. For a spell like fireball, the target is the point in space from which the fireball explodes. Most spells have a range in meters. Some spells can only target a creature (including you) that you are in physical contact with. Other spells, such as the shield spell, affect only you—these spells have a personal range. A spell that has "an ally" as its area of effect can also be cast on itself.

Spells that create cones or lines of effect that originate from you also have personal range, indicating that you are the point of origin of the spell's effect (see "Areas of Effect" later in this chapter).

\subsubsection{Casting Spell in Armour}\index{Casting Spell in Armour}\label{magielanciareincantesimiinarmatura}\index{Spell, in Armor}

Given the mental concentration and precise gestures required, the Armour distracts and unbalances the flows. The Magic Test in casting the spell is modified as indicated in the section of \hyperlink{armatureemagie}{armor} (page \pageref{armatureemagie}).

\subsubsection{Optional - Enchantments in Armour}\index{Optional - Enchantments in Armour}

The Armour blocks magical flows and does not allow correct channelling.
This option causes all spells cast by the caster to become Touch Ranged, i.e. downloadable only through the caster's hand. No Magic Tests are required.

\subsubsection{Duration}\index{Spell Duration}\label{magiedurata}\index{Spell, Duration}

A spell's duration is the length of time it persists. Duration can be expressed in rounds, minutes, hours, or even years. Some spells specify that their effects last until the spell is dispelled or destroyed. A spell can be interrupted by its caster as an immediate action. \index{Interrupt own spell}

If a magic critical success doubles the duration, it is always intended to refer to the initial duration. E.g. if the duration is 2 hours after the first doubling it becomes 4 hours, with the second it becomes 6 hours and then 8 hours.\index{Magical Critical success on duration}


\begin{itemize}[leftmargin=*]

\item
\textit{Snapshot}

Many spells are instantaneous. The spell wounds, heals, creates, or alters a creature or object so that it cannot be dispelled, since its magic exists only for an instant.

\item

\textit{Concentration}\index{Concentration}\index{Spell, Concentration}

Some spells require you to maintain concentration to keep the magic going. If you lose concentration, the spell will end. If a spell must be maintained through concentration, this is indicated under the Duration entry, the spell specifies how long you can maintain concentration on it. You can end the concentration at any time using a reaction.

Normal activities, such as moving and attacking, do not interfere with concentration. Maintaining concentration costs 1 Action per round.
\end{itemize}

\subsubsection{Components}\index{Components}\label{magiecomponenti}\index{Spell, Components}

The components of a spell are the physical requirements you must meet to cast it. Each spell's description indicates whether it requires verbal (V), somatic (S), or material (M) components. If you are unable to supply one or more of the spell's components, you will not be able to cast it.

Most spells require you to chant mystical words. The words, the rhythm, the cadence and resonance set in motion the harmony with the Patron.

\begin{changemargin}{0.3cm}{0.3cm}\begin{narrator}
For more immediate management of the components you can replace the indicated components by consuming an equivalent in gold coins equal to the cube of the level ($LV^3$) in gem dust.
\end{narrator}\end{changemargin}


\textbf{Somatic (S)}

The gestures of casting a spell can include forced gesturing or intricate series of gestures. If a spell requires a somatic component, the caster must be free to use at least one hand to perform these gestures.

\medskip

\textbf{Material (M)}\index{Spell, Components}\index{Spell Components}

Casting some spells requires particular items, specified in parentheses under components. The character must acquire that specific component before he can cast the spell.

If a spell indicates that the material component is consumed by the spell, the caster must supply this component for each casting of the spell.
A spellcaster must have one hand free to access these components, but it can be the same hand used to cast somatic components.

\subsubsection{Be incapacitated or killed}\index{Be incapacitated or killed}\label{magieessereucciso}\index{Spell, incapacitated}

If you drop below zero Hit Points, you lose half of your remaining Spell Points, with a minimum of 10 points lost. Any spells you are concentrating on are interrupted.

\subsubsection{Targets}\index{Targets}\label{magiebersagli}\index{Spell, Targets}

A normal spell requires you to choose one or more targets to be affected by its magic. The spell description tells you whether the spell targets creatures, objects, or a point of origin to generate an area of effect (described below). Unless the spell has a discernible effect, a creature may never realize it has been the target of a spell. An effect such as crackling lightning is obvious, but a subtler effect, such as attempting to read a creature's thoughts, usually goes unnoticed unless the spell says otherwise.

Casting a spell is an action that does not go unnoticed. A Hide check (difficulty 13) or casting the spell as if distracted allows you to conceal the casting, if it doesn't happen directly in front of the observer.

\textbf{Clear Path To Target}\index{Spell, Clear path to target}

\textbf{To target a creature or object, you must see it and have a clear path to it}, and so \textbf{it can't be behind full cover}. If you place an area of effect in a spot you can't see and an obstruction, such as a wall, is between you and that spot, the point of origin is created on your closest side of the obstruction (a Fireball behind a closed door explodes on contact with the door on your side and does not manifest beyond the door).\index{Magic see the target}

\textbf{Targeting Yourself}\index{Target yourself}\index{Spell, Target yourself}

If a spell targets a creature of your choice or an ally, you can also choose yourself, unless the creature must be hostile or specified that it can't be you. If you are within the area of effect of a spell you cast, you will be affected as well.

%\begin{center}
%\includegraphics[width=0.6\linewidth]{immagini/tarothanged.png}
%
%\textit{Tarocchi - L'Impiccato}
%\end{center}


\begin{center}
\includegraphics[width=0.6\linewidth]{immagini/3dforme.png}

\textit{Cone, Sphere, Cylinder, Cube}
\end{center}

\subsubsection{Areas of Effect}\index{Area of Effect}\label{magieareedieffetto}\index{Spell, Area of Effect}

Spells such as Burning Wave and cone of cold cover an area, allowing him to strike multiple creatures at once.

A spell description specifies its area of effect, which usually falls in one of five shapes: cylinder, cone, cube, line, or sphere. Each area of effect has a point of origin, a place from which the spell's energy manifests. The rules for each shape specify how to place its point of origin. Usually the point of origin is a point in space, but some spells have an area whose origin is a creature or object. The origin point must always be valid.

\begin{itemize}[leftmargin=*]
\item \textit{\textbf{Cylinder}}: A cylinder's point of origin is the center of a circle of specified radius, as indicated in the spell description. The circle must be on the floor or level with the spell's effect. The energy in a cylinder expands in straight lines from the point of origin to the perimeter of the circle, forming the base of the cylinder. The effect of the spell then starts from bottom to top or from top to bottom, up to a distance equal to the height of the cylinder. The point of origin of the cylinder is included in its area of effect.

Eg. A 9 meter Cone of Cold is 9 meters wide at the end and stretches 9 meters from the point of origin, 3 meters away from the point of origin it is 3 meters wide.


\item

\textit{\textbf{Cone}}: A cone extends in a direction of your choice from its point of origin. The diameter of a cone at a given point along its length is equal to that point's distance from the point of origin. A cone's area of effect specifies its maximum length. The point of origin of the cone is not included in its area of effect unless you decide otherwise.

\item
\textit{\textbf{Cube}}: Select the point of origin of a corner of the cube. The dimensions of the cube are expressed as the length of each of its edges. The cube's point of origin is not included in its area of effect unless you decide otherwise.

\item
\textit{\textbf{Line}}: A line extends from its point of origin in a straight path along its entire length and covers an area defined by its width. The point of origin of the line is not included in its area of effect, unless you decide otherwise.

\item
\textit{\textbf{Sphere}}: select the point of origin of a sphere, which must be valid (see Range and Targets) and the sphere will extend from that point until it encounters an insurmountable obstacle or its size expressed in the radius . The size of the sphere is indicated as the radius in meters that extends from that point. The sphere's point of origin is included in its area of ​​effect.

A fireball that is spawned in a 9x9m room will take a good chunk of it, and in a 6x6m room it will fill it all. In a 3x3m room, if it has the opportunity to exit through a door or window, it will continue its explosion up to a 6m radius. A fireball in a 3x3m corridor will saturate it 6m back and forth from its point of origin.

\end{itemize}

\subsubsection{Spell Rarity}\index{Spell Rarity}\label{magieraritaincantesimi}\index{Spell, Rarity of Spell}

On some Enchantments the Rarity is indicated or how likely it is to find this enchantment or how much it can be known. The rarity depends not only on the level of the spell itself, obviously the most powerful spells are also the rarest, but also on how widespread and known they are normally in the list.
The Arbiter on 3d6 roll will use this scale: Common (1-14) - Uncommon (15) - Rare (16) - Very Rare (17) - Legendary (18).

\subsubsection{Combining Magical Effects}\index{Combining Magical Effect}\label{magiecombinareeffettimagici}\index{Spell, Combining Magical Effects}

The effects of different spells stack until their durations overlap. However, effects from the same spell or that give the same bonus cast multiple times on the same target do not combine. Instead, the most powerful spell among those cast, the one of the highest level and, all things being equal, the one that has obtained the most Magic Criticals will be applied as long as the durations overlap.

In the case of instantaneous spells, the effects act individually if they act in the same initiative segment. E.g. If I am hit by a bolt of lightning with initiative segment 4 and then by another bolt of lightning with initiative segment 8 I will make two separate saving throws with related damage management, if they were in the same initiative segment I would only suffer the more powerful one (see above ).

\subsection{Magic rules}\index{Magic Rules}\label{magieregoledibase}\index{Spell, Magic Rules}

\begin{itemize}[leftmargin=*]

\item
When casting his first spell, the spellcaster chooses whether to use the Characteristic linked to the first Magic List taken as a modifier to the Magical Expertise check or, if he is a Devotee, he can choose the Characteristic indicated by the Patron.\index{Spells, Modifier to the Magical Proficiency check. CM}

Once the choice has been made it is no longer possible to change it. This modifier is called \textbf{spell ability modifier}.\index{Spell ability modifier}
\item
When the character assigns the first point of Magical Expertise, he learns a List of Magic.
\item
Each time the character learns a new Magic List, either through the Adept of Magic skill or by Magical Expertise score, he learns 2 spells from his Tome of Magic + a number of Cantrips equal to his ability modifier from that Magic List or from the Universal List.
\item
Each time a spellcaster gains a point in Magical Expertise she learns two new spells that she has available in her Tome and that are within the maximum castable level.
\item
Each time you gain a point in Magical Expertise, you may forgo learning a spell of level 1 or higher to learn two cantrips (level 0) that you know.
\item
Every time the caster acquires a point in Magical Expertise, it is possible to forget a number of spells equal to the CM score and replace them with others available in the Tome, as long as they are within the maximum castable level.
\item
The number of spells that can be cast per day depends on the caster's ability. See \textbf{Magic Points and Magical Proficiency Table}.
\item
A Follower gains +1d6 on Magic Checks in the schools favored by the Patron. She can use the Patron's favored energy in your spells.
\item
A Devotee adds +1d6 to Magic Tests in the Patron's favored schools and may ignore a die rolled on the Magic Test. He must use the Patron's preferred energy in his spells.
\item
The term \textbf{learned}\index{Spells, Learned} means a spell present in the Tome of Magic that has been memorized and can be cast whenever desired.

The term \textbf{known}\index{Spells, Known} means a spell present in the Tome of Magic which however has not been learned, that is, it has not been memorized and cannot be cast when desired.

\end{itemize}

\subsubsection{Access to Spell Lists}\index{Access to Spell Lists}

The spellcaster can cast spells only if they belong to a known Magic List or the Universal List and if learned or memorized among those present in the Tome of Magic.

A spellcaster learns a new Spell List when he assigns his first point of Magical Proficiency and every 6 points of Magical Proficiency total thereafter.

Further access to the Magic Lists occurs through the choice of the Adept of Magic Feat.

\subsubsection{Maximum level of castable spell}\hypertarget{schools and levels}{}\index{Level of spells by skill}\label{spellsaccess to magic lists}\index{Spells, Maximum level of castable spell}\index{Maximum level of castable spells}

While Magical Competence indicates the study and dedication to Magic in the most abstract form, it is only the Adept of Magic Skill that allows you to understand how good you are at formulating spells. 

To establish the specific maximum castable level for each Magic List it is necessary to know the Magical Expertise value and how many times the Magic Adept Skill has been taken in that Magic List.

Add to CM the times you took Adept of Magic and divide the result by 2. E.g. CM=8, Adept of Magic taken 4 times, (8+4)/2=6 Spell Level, or CM=13, Adept of Magic Magic 1 time, (13+1)/2=7 Spell Level.

If Adept of Magic has not been taken then the maximum level of spells that can be cast is 1.


\subsubsection{Optional - Ultimate Magic}\index{It's over 9000!}\index{Optional - Ultimate Magic}\label{}\label{opzionalemagiasuprema}

- \textbf{Supreme Magic}\index{Supreme Magic}\hypertarget{supreme magic}{}: if you want high-level spellcasters to dominate magic, make sure that for every 6 points of Magic Proficiency the player can add 1d6 in the Magic Test and ignore a rolled die.

- \textbf{Magic for all}\index{Magic for all}\label{magicforall}: instead for every 4 points assigned in Magic Proficiency the spellcaster learns a List of Magic very 2.

\subsubsection{Magic Test}\index{Magic Test}\index{Magical Critical Success}\index{Magical Critical Failure}\label{magictest}\index{Spells, Magic Test}

Casting a spell is not always enough, many times it is necessary for it to work well and indeed to act beyond its normal expectations. The caster can decide to call upon more energy when casting the spell, or make a \emph{\textbf{Magic Test}} and trust in his abilities.

The caster rolls 3d6+1d6 for every two known Magic Lists (rounded up) plus any bonuses or skills.

%The spellcaster can ignore a 1 rolled on the Magic Test for each time he has taken Adept of Magic in the Magic List of the spell he is casting.

If in the set of rolled dice there are \textbf{at least two 1s} or \textbf{one 1 and two 2s} bad things have happened, this case is called \textbf{Magic Critical Failure}\index{Magic Critical Failure}, the spell does not manifest and \textbf{Magic Points are spent}.

To check how many magical critical failures have been made, first check how many pairs of 1s are present, then check if there is another 1 left to associate with a 1 or two 2s.

\begin{changemargin}{0.3cm}{0.3cm}\begin{narrator}
Grant a +1d6 on the Magic Test, or ignore a 1 rolled on the Magic Test, when the character expertly and passionately recites the casting of the spell. If he says \ emph {I launch a fireball} he will gain no advantages but if with transport he recites \ emph {By the Flame of Genesis may Nedraf destroy you with his sacred flames. Burn unworthy. Fireball!} then yes!.
\end{narrator}\end{changemargin}

Once you have checked the absence of critical failure, if there are at least two 6s in the dice roll you will have obtained a \textbf{Magic Critical Success}\index{Magic Critical Success}, as for the Golden Rules you will continue to roll a die for every 6 made or what are you going to do. Count the 6 you get, every two is a Magical Critical Success! Any 1's rolled following the critical success do not count towards the critical failure. \textbf{For each critical success the spell's saving throw DC increases by 1}.

Any result other than a magical critical success or a magical critical failure will cause the spell to manifest without any particular effect unless explicitly requested.

When required to pass or make a Magic Test it is sufficient not to make a Magical Critical Failure.

By paying twice the cost of the spell you can roll 2d6 more, by paying three times you can roll 3 more, by paying 4 times the cost of the spell you roll 4 more dice and so on.To declare before rolling the Magic Test.\index{Magic Test, more dice}

You can, after rolling the Magic Test, additionally pay double the spell's cost to ignore one rolled dice, and by paying four times you can ignore two rolled dice.\index(Magic Test, ignore dice)

A spellcaster can also willfully fail a Magic Test.


%\medskip
%
%\begin{center}
%\includegraphics[width=0.65\linewidth]{immagini/Arthur-Pyle_The_Enchanter_Merlin.png}
%
%\textit{Merlin. Howard Pyle, The Story of King Arthur and His Knights (1903)}
%\end{center}


\subsubsection{Critical Failure Magic Test}\index{Critical Failure Magic Test}\label{magiefallimentocriticonellaprovadimagia}\index{Spell, Failure Magic Test}

If the Magic Test had a magical critical failure, roll 3d6 and consult the following table. For each additional magical critical failure subtract 1d6, until you roll a single 1d6.

%\end{multicols}

\textbf{Table: Spell Critical Failure Effects}\index{Table Spell Critical Failure Effects}

\medskip
{\small
\begin{tabularx}{0.45\textwidth}{lX}
\hline
1 & Increase fatigued by 2 ranks\\
2 & For 1 day you are no longer able to channel magical energies. You cannot cast spells except by making a critical magical success in the Magic Test \\
3 & You exhibit a minor body modification\\
4 & You are hit by a thundering column of Light and Void. In a 3m radius around you, everyone must make a DC 15 Reflex save for half or take 1d6 damage per spell level \\
5 & For 3 rounds you are under the influence of the Confusion spell \\
6 & You are paralyzed for 3 rounds\\
7 & Teleport within 3d10 meters in a random direction\\
8 & Become Invisible \& unable to speak for 6 rounds\\
9 & Only you are enveloped in a curtain of impenetrable magical darkness for 6 rounds\\
10 & You can't speak well, you stutter. Each spell cast forces you to pass a Magic Test. Duration 3 rounds\\
11 & Next spell you cast has effects minimized if possible\\
12 & Your heartbeat is like the beating of a drum, it can be heard within 50 meters\\
13 & All your body hair falls out, luckily it can grow back\\
14 & You emit a loud and pestilential flatulence. A 1m x 50cm luminous sign above your head indicates and mocks you\\
15 & Every item you hold falls to the floor\\
16 & Gain 2d6 Spell Points\\
17 & Anvil falls, 3d6 damage Reflex save DC 15 to halve, on a random creature, excluding you, within 6 meters \\
18 & All creatures, except you, within 6 meters of you take 3d10 unresistable damage
\end{tabularx}}

%\begin{multicols}{2}

\subsubsection{Spell Points}\index{Spell Points}\label{magiepuntimagia}\index{Spell, Spell Points}

Depending on the score in Magic Proficiency, the spellcaster has a certain amount of Spell Points available.

\textbf{Spells have a cost in Spell Points equal to their level}

%\textbf{Gli incantesimi hanno un costo in Punti Magia pari al livello dell'incantesimo +1.}\index{Incantesimi, Costo in Punti Magia}

Each time a spell is cast, the cost is subtracted from the Spell Points available for the day.
In case of Cantrips these do not consume Spell Points but it is necessary to have at least 1 Spell Point remaining.

The caster has a \textbf{bonus} to Spell Points equal to his characteristic modifier for spellcasting.

Spell Points are all recovered with 8 hours of rest.\index{Spell, Recovering Spell Points}

\medskip

\textbf{Table: of Magic Proficiency Spell Points}\index{Table of Magic Proficiency Spell Points}

\medskip

\begin{tabularx}{0.45\textwidth}{XX|XX}
\textbf{Comp. Magic} & \textbf{Spell Points}&\textbf{Magic Comp.} & \textbf{Spell Points}\\
\hline
1&	2 	&11&43\\
2&	4	&12&47\\
3&	8	&13&50\\
4&	10	&14&54\\
5&	16	&15&58\\
6&	19	&16&62\\
7&	23	&17&71\\
8&	27	&18&76\\
9&	36	&19&82\\
10&	41	&20&89\\
20+&prec.+ 4&&\\
\end{tabularx}


%\begin{tabularx}{0.45\textwidth}{XX|XX}
%\textbf{Livello Inc.} & \textbf{Punti Magia}&\textbf{Livello Inc.} & \textbf{Punti Magia}\\

%1& 	2	& 6& 	9\\
%2& 	3	& 7&	10\\
%3&		5	& 8& 	11\\
%4& 	6	& 9& 	13\\
%5&	7&&\\
%\end{tabularx}


%\begin{tabularx}{0.45\textwidth}{XX|XX}
%\textbf{Comp. Magia} & \textbf{Punti Magia}&\textbf{Comp. Magia} & \textbf{Punti Magia}\\
%1 &	4	&2 	&6\\
%3 &	14	&4 	&17\\
%5 &	27	&6 	&32\\
%7 &	38	&8	&44\\
%9 &	57	&10 &64\\
%11&	73	&12 &73\\
%13 &83	&14 &83\\
%15 &94	&16 &94\\
%17 &107	&18 &114\\
%19 &123	&20 &133\\
%\end{tabularx}


\begin{changemargin}{0.3cm}{0.3cm}\begin{narrator}
If you want a more difficult approach, for each degree of Fatigue you recover 20\% less than the maximum Spell Points per night of rest.
\end{narrator}\end{changemargin}

\subsubsection{When low on Spell Points}\index{When low on Spell Points}\hypertarget{quandosihannopochipuntimagia}{}\label{magiequandosihannopochipuntimagia}\index{Spell, Low Spell Points}

When the spellcaster drops below 50\% of the Spell Points available, any further spell casting must be done with a Magic Test.

\subsubsection{Optional - Spells as Rituals}\index{Optional - Spells as Rituals}

Especially at the first levels it can be very annoying not to have learned a spell despite having it available in the Tome of Magic.

With this optional rule the spellcaster can cast a spell, within the 3rd level, which is present in his Tome of Magic or even which he has learned, lengthening its casting time to 1 hour per Magic Point cost. In case of a spell casted in this way, no Magic Points are used but a Magic Test must be passed.


\subsubsection*{Optional - The Vice of Magic}\index{Optional - The Vice of Magic}

If you want an approach that reduces the amount of spells cast by spellcasters, establish that the cost in Magic Points of each spell is equal to the spell's Level + the cost itself x the times it has already been cast that day, but avoid the Magic Test for When low on Spell Points.

\subsubsection{Automagic Critical Success}\index{Automagic Critical Success}\index{Nova}\label{magienova}\index{Spell, Automagic Critical Success}

The spellcaster can decide to additionally spend the \textbf{double the normal Spell Points} of the spell to automatically have a \textbf{magical critical success}.
The choice can be made multiple times and each time the cost of the spell doubles compared to the previous one. The declaration of wanting to use the Automagic Critical Success must be declared before carrying out, and passing, the Magic Test.

The casting time of a spell enhanced in this way increases by 1 Action.

Ex. Fireball, I want it to hit 2 magic critical success, I pay 3 Spell Points to cast it, plus 6 for the first magic critical success plus 12 for the second magic critical success, and possibly 24 for a third magic critical success. 

No more than half of the current Spell Points can be spent to enhance a spell.

\subsubsection{The Tome of Magic}\index{Tome of Magic}\label{magietomodellamagia}\index{Spell, Tome of Magic}

If the Patrons are the source of magic, it is only the application of ancient rites and formulas that allows this raw energy to be manifested in a form and expression that we call spell.

Every user of magic has one or more \textbf{Tome} of spells, don't just think of a large ancient tome bound in leather, different cultures have developed over time the ability to inscribe the runes of spells on cards, sticks, slabs of stone, tattoos... take your pick when creating the character.
This choice will not prevent you from copying spells from \textbf{Tomes} made differently (tobacco leaves, liquids of knowledge...) for you it will always be easy (Arcana DC 12 test) to understand if you are dealing with a Tome of some guy.

When the character learns a List of Magic for the first time, he writes in the Tome of Magic a number of spells, including cantrips and otherwise, equal to the ability modifier for spells +2. These spells will either be cantrips of first level or from the Universal List. Any other spells he wants to learn he will have to find and write down in her Tome.

Each spell occupies a number of pages in the Tome equal to its level, with a minimum of one; copying a spell page takes 1 hour of work and 10 gp of precious ink.\index{Copying Spells on the Tome}

A Tome (book) of spells costs 10 gp per page.


\medskip
\begin{center}
\includegraphics[width=0.7\linewidth]{immagini/spellbook.png}
\end{center}

\medskip

A spellcaster can copy into his Tome spells that belong to a Magic List known to him (or Universal) and the maximum level that can be copied is one level higher than his maximum castable level (see \hyperlink{schoolsandlevels}{Magic Lists}).

If the spell is more than two levels higher or from an unknown Magic List the caster must make a Magic Check and gain a Magical Critical Success. If the character is a Devotee and the spell belongs to a Magic List known to him and preferred by the Patron, then the Magic Test is performed only if the spell is three or more levels higher than the maximum castable level.\index{Spell unknown}

If he does not achieve at least one Magical Critical Success he cannot attempt to copy that spell until the next point of Magical Proficiency gained. If he rolls a Magical Critical Failure, bad things will happen to the Tome and 1d4 random spells will be erased from the Tome itself.

The source of new spells can be another tome, staff, scroll... in short, anything that the previous spellcaster used to store spells. A magical object (magic staff, ring, rod... wand...) is not suitable as a source from which to copy the spell it contains, it must be copied from the equivalent tome or scroll of another spellcaster. A spell when copied to the new Tome vanishes from the original Tome.

During the adventures your enchanter will be able to copy many and numerous spells on his Tome but he will not be able to learn them immediately. When the character acquires a new point of Magical Expertise, he will be able to forget a number of spells equal to his Magical Expertise score to learn other spells present in his Tome that are from a known Magic List and within the maximum possible spell level.

\begin{changemargin}{0.3cm}{0.3cm}\begin{tcolorbox}[title = Choose Spells]
		Spells are not learned alone, they are not chosen from a ready-made list. Each spell is a precious treasure that must be found and learned.
		
		You will have to undertake perilous adventures, pay mercenaries, search for ancient tomes and reveal the darkest and most forgotten secrets in order to learn new spells.
		
		Each spell is like a magical object, a true treasure to seek and obtain!
\end{tcolorbox}\end{changemargin}


\begin{changemargin}{0.3cm}{0.3cm}\begin{narrator}
		Spells become full-fledged magical objects and rewards. Harness your characters' thirst for knowledge and power to build interesting adventures that revolve around ancient tomes and legendary lost spells.
\end{narrator}\end{changemargin}


\subsubsection{Changing Spells}\index{Changing Spells}\label{Changing Spells}

Through a long and difficult magical rite the caster can replace a learned spell with a known spell present in the Tome. After 8 hours of ritual the caster performs a Magic Test and only if he succeeds can he change up to 1d4 spells, if the Magic Test achieves a Critical Success then he can change up to 1d4+4 spells. If the check fails critically, the caster forgets 1d4 spells.


\begin{changemargin}{0.3cm}{0.3cm}\begin{narrator}
Spells become magical items and rewards in all respects. Harness the characters' thirst for knowledge and power to build interesting adventures that revolve around ancient tomes and legendary lost spells.
\end{narrator}\end{changemargin}

\subsubsection{Studying spells}\index{Studying spells}\label{magiestudiareincantesimi}\index{Spell, Studying}

The character who wants to cast spells must review the ancient formulas on his Tome every day. This operation is quite quick, taking only 3 minutes per Magic Proficiency.

If the spellcaster hasn't reviewed the spells upon waking or in any case before casting them, he must make a Magic Test for each spell until he has reviewed.

\subsubsection{Attack Roll with Spell}\index{Attack Roll with Spell}\label{magietiropercolpireconlemagie}\index{Spell, Attack roll}

Several spells must be cast and hit an opponent to work.

When the spell tells you to make a \textit{Spell attack roll} (ranged or melee) you must make an attack roll against your opponent's Defense.

This attack roll is made with 3d6+ \textbf{Weapon Proficiency} + \textbf{Spell characteristic modifier} + \textbf{Skills} and \textbf{miscellaneous modifiers}.

It is also possible that a Touch spell Attack Roll is required, in this case as Touck Attack you get a +1d6 on hit rolls.

\medskip

When the spell is area effect it is not necessary to make an attack roll except for difficult and specified areas, i.e. you aim in a well circumscribed area and you want to avoid hitting someone with an area spell.

Weapon Attack rolls or Spell Attack rools stack the multiple attack penalties.\index{Multiple attack spell penalty}

\subsubsection{Optional - Attack Roll with Spells}\index{Optional - Attack Roll with Spells}

If you want to make it easier for spellcasters to hit, you can decide that the attack roll is modified not by weapon proficiency but by magical proficiency.

\subsubsection{The explosion of 6 in Magic}\index{The explosion of 6 in Magic}\label{magieesplosionedelsei}\index{Magic Critical}

Also in the Magic Test the 6s explode, the 6s rolled in the Magic Test are rerolled, and rerolled again just in case.

Keep track of how many criticals (two 6s rolled) you roll, they could be used to obtain "special" effects in the spell! Remember that for each magical critical the DC of the saving throw increases by 1.

\subsubsection{Saving Throw - Resist the spell}\index{Saving Throw - Resist the spell}\index{Spell, Saving Throws}\label{magietirosalvezza}\hypertarget{magietirosalvezza}{} 

Saving Throw as required by the spell has difficulty (DC) equal to 10 + Magic Proficiency + Characteristic modifier per spell + 2 x Adept of Magic taken in that Magic List +1 per critical success rolled in the Magic Test.

When you cast a spell, for example Lighting Bolt, you impose a Reflex saving throw to try to avoid it and if in the Magic Test you had obtained at least one magical critical success you would have done 9d6 damage and the DC of the saving throw would have increased by 1.

The spell description states that a Saving Throw must be made.

If it's you who has to resist a spell, the Arbiter won't tell you to make a Saving Throw at difficulty 18, he'll compare your roll with the difficulty, he can tell you that the check is complex, difficult or easy...

\begin{itemize}[leftmargin=*]

\item
If you roll 3 times 6 on your Saving Throw, you pass, regardless of the total, and you get a \textbf{Critical Success Save}.

\item
If the Saving Throw is successful and you roll at least two 6s on the dice roll you gain a \textbf{save critical success}\index{Spell Critical Success}.

\item
If you roll 3 times 1 on your Saving Throw, you fail the roll, regardless of the total, and you get a \textbf{Critical Failure Saving}.\index{Three 1 on Saving Throws}

\item
If the Saving Throw fails and you roll at least two 1s or a 1 and two 2s on dice rolls you get a \textbf{critical fail save}\index{Saving Throw Critical Failure}.\index{Two 1s on Saving Throw}

\end{itemize}

It is also possible that the spell description states what happens on a successful or critical Saving Throw.

For \textbf{monsters} or in any case for a spell cast given by innate magical abilities, if not specified, the \textbf{DC of the Saving Throw is equal to 10 + 2 x spell level + Intelligence}.\index{Saving Throw DC for Monster's Spells}\index{DC for Monster spells}

\begin{changemargin}{0.3cm}{0.3cm}\begin{tcolorbox}[title = Tups throw a Lighting Bolt!]
Tups who has Intelligence 4, Magical Proficiency 6, and has taken Adept of Magic 1 time in the Air list, casts the Lightning Bolt spell. The difficulty (DC) of the Reflex saving throw will be equal to 10 + 6 (MP) + 4 (characteristic modifier for spell, Intelligence) + 2 x 1 (Adept of Magic taken 1 time in the Air List) or 10+6+4 +2x1 = 22 to halve damage. If he had made a Magic Test and it had two magical critical success the DC would have become 23.

\end{tcolorbox}\end{changemargin}

\subsubsection{Optional - Saving Throw Value}\index{Optional - Saving Throw Value}

The saving throw calculation system aims to reward the character who invests in his own knowledge and specialization, while making the calculation more complex for the player.

An alternative is to set the save DC to 10+2*Adept of Magic (in any list)+Ability Modifier per spell+Spell Level+1 per Spell Critical.


\subsubsection{Distracted - Problems in spellcasting}\index{Distracted - Problems in spellcasting}\index{Distracted}\label{magiedistratto}

If the caster is severely \textbf{Distracted}, trying to hide the casting of magic, is impeded, disturbed, is bleeding, threatened, is under attack while trying to cast a spell, not a cantrip, he must make a \textbf{Magic Check}.

For each critical damage suffered in the round the Magic Test is made with an additional 1d6.\index{Critical damage if spell is cast}

\subsubsection{Concentration}\index{Hit while concentrating}\index{Concentration}\label{magieconcentrazione}

You lose concentration on a spell if you cast another spell that requires concentration. You can't concentrate on two spells at once. Breaking concentration costs a Reaction. Keep Concentration cost 1 Action for round.

If you are hit\index{Hit while maintaining concentration} while concentrating on a spell you must make a Magic Check and gain at least 1 magical critical success or lose concentration. Also in this case you can pay the additional Magic Point cost to ignore 1 or 2 (\hyperlink{magieefficaci}{Effective Spells} page \pageref{magieefficaci}).

While concentrating, you can only cast Cantrips.

\subsubsection{Optional - Multiple Concentrations}\index{Optional - Multiple Concentrations}

Every 6 points of CM you can maintain concentration on an additional spell, without being limited to cantrips alone. If concentration is interrupted, all spells held in concentration are lost.
For each spell you maintain concentration on, you pay 1 Action.


\subsubsection{Conserve Magic}\index{Conserve Magic}\label{magieconservare}

The spellcaster can cast the spell (usually 2 Actions) and hold it in his fist, without manifesting it. To do so he must cast the spell, then can hold it for up to 1 round per point of spell Characteristic modifier +2 rounds per time he has taken Adept of Magic on the spell's Magic List.

To withhold the spell, the caster must remain Concentrated (cost 1 Action per round) and pay 1 Spell Point as cost per round.
To cast the stored spell it is sufficient to roll the initiative and use 1 Action. You cannot cast further spells that are not Cantrips as long as you retain a spell and in the round in which you cast it.

\subsubsection{Affected by multiple spells}\index{Affected by multiple spells}\label{magieinfluenzatodapiumagie}

When a character is affected by \textbf{two or more magical effects} that grant the same type of bonus, penalty or damage in the same initiative segment (protection against fire, Defence bonus or ST... , multiple acid balls ), only the one with the highest Saving Throw or bonus is taken into account

A character who takes 2 Fireballs in the same Initiative segment will save only for the one with the higher Save, regardless of whether it is the one with the higher damage. If he catches a Fireball in two different times of the same round he will make two separate Saving Throws taking the relative damage.

\subsubsection{Attempting multiple spells in the same round}\index{Attempting multiple spells in the same round}\index{Multiple spells in round}\hypertarget{piumagieround}{}\label{piumagieround}

It is not possible to cast multiple spells per round even if the sum of Actions allows it. Some obscure rites and esoteric practices allow, with great risk, to try to cast even more spells, as long as they are always in the 3 Actions per round. It is necessary to have at least 3 Magic Proficiency.

The spellcaster cast the first spell then make a Magic Test. If he succeeds in a magical critical success (at least 2 times 6) then he succeeds in casting the second spell, if the Magic Test instead does not obtain a magical critical then it is considered a magical critical failure, with the appropriate effects.

\subsubsection{Alter Spells}\index{Alter Spells}\label{magiealteraremagie}

The caster can modify spells in several ways. These possibilities add versatility to the caster and it is advisable for the player to always have them present in the most critical situations.

\begin{itemize}[leftmargin=*]
\item
\textbf{Effective Magic}\index{Effective Magic}\label{magieefficaci}: By paying an additional cost of once the spell's Points, you can ignore one rolled dice; by paying double, you can ignore two, and by paying four times, you can ignore three. Effective Magic can also be used by a companion of the caster by sacrificing the indicated Magic Points cost and using the same number of Actions. Immediate/Reaction Action.
\item
\textbf{Ethereal Magic}\index{Ethereal Magic}: increasing the Magic Points spent in the spell by 3, your spells have full effect on ethereal or incorporeal creatures, increasing the difficulty of the Saving Throw by 2.
\item
\textbf{Magic Sacrifice}\index{Magic Sacrifice}: the spellcaster by reducing his Maximum Hit Points by 4 acquires 1 Spell Points. You can't sacrifice more than half of your current HP at a time.
\item
\textbf{Merciful Magic}\index{Merciful Magic}: Spells deal temporary damage by increasing Spell Points spent by 3.
Spells that deal damage of a particular type (such as fire) deal temporary damage of the same type.
\item
\textbf{Targeted Magic}\index{Targeted Magic}: For each time you took Adept of Magic in that list beyond the first you can make a creature of your choice immune to the effect of the spell you cast. Cost 1 Magic Point per excluded creature, 1 Action.
\item
\textbf{Far Magic}\index{Far Magic}: by increasing the Magic Points used by 1, you increase the casting distance of the spell by up to 9 meters per time you have taken Adept of Magic in that list. 1 Action.
\item
\textbf{Increase time}\index{Increase casting time} of casting from 2 Actions to 3 Actions decreases by 1 in Spell Points spent on casting a spell, with a minimun cost of 1 Spell Point.
\item
\textbf{Collaborative Magic}\index{Collaborative Magic}: only another another spellcaster, by sacrificing half the Spell Points of the companion who casts the spell, using the same number of Actions, can grant +1d6 to the companion's Magic Test. Collaborative Magic stacks with Effective Magic. Magic Proficiency Requirement 3
\item
- \textbf{Circle of Power}\index{Circle of Power}: multiple spellcasters who are all Devotees or Followers of the same Patron can collaborate so that one of them is better at casting a spell.
Each spellcaster sacrifices half the Magic Points of the spell cast by his companion and passes a Magic Test. For every two companions who pass the Magic Test, a magical critical success is generated, up to a maximum of 7 magical critical successes. The casting time of a spell via Circle of Power becomes at least 1 Turn. Magical Proficiency Requirement 5.
\end{itemize}

The chances granted by Alter Spells stack with each other.

\textbf{Minor modifications} \index{Minor spell modifications} to the manifestation of the spell can be agreed with the Arbiter for a cost of additional Spell Points or with a successful Magic Test.

\subsubsection{Attempt Spells with Impairments}\index{Attempt Spells with Impairments} \index{Impedimenti}\label{magieconimpedimenti}

Casting a spell is tied to particular and unique gestures and words. When the character is in a situation where he can't gesture or speak then he can attempt to cast the spell anyway even if it becomes much more difficult.

The Spell Points required to cast spells are tripled if she cannot gesticulate and tripled further if she cannot speak, it is also necessary in each case to make a Magic Test.

If the spell also has material components, these must still be provided (placed within 30 cm of the caster) or the spell cannot be cast.

\subsubsection{Spell target definitions}\index{Spell target definitions}\label{magiedefinizioniobiettivi}

In the spells listed below you will often find references to the types of subjects and targets that can be influenced as well as to different types of energy and elements.

- The \textbf{Creatures} \textbf{Natural} are Insects, Reptiles, Beasts, Humanoids, Plants, Aquatic Creatures, Monstrosities, Oozes.

- The \textbf{Creatures} \textbf{Magical} are: Fiends (Devils and Demons), Fey, Spirits, Undead, Giants, Celestials, Elementals, Constructs, Aberrations (anything alien or unnatural) and Dragons.
If a Natural Creature has magical powers then it is also considered a Magical Creature. A more complete description of these "categories" can be found in the Monstruario Chapter.

- \textbf{Energy} includes: Force, Fire, Light, Sound, Electricity, Positive Energy, Negative Energy, Cold, Void.

\subsubsection{Energy, Light and Void Damage}

The damage caused by \textbf{Light}\index{Light} is half fire and half positive energy, ie a resistance to fire or positive energy only applies to half of the damage caused by the attack.

The damage caused by \textbf{Void}\index{Void} is half cold and half negative energy, any wards apply to the respective halves of the damage.

\textbf{negative energy} alone damages\index{Negative Energy} the living and heals the undead, \textbf{positive energy}\index{Positive Energy} alone damages the undead but does not heal the living (at the Arbiter's discretion the exposition might be equivalent to a lesser restoration spell for one round), see also descriptions of the Planes. A target takes full damage from Light or Void if it has no inherent resistances.

A special case is the \textbf{Healing positive energy}\index{Healing positive energy} which heals the living and harms the undead. This energy is that of lay on hands, channel energy, and healing spells.\index{Positive Energy on undead}\\


\subsection{Magic List Ability}


\begin{changemargin}{0.3cm}{0.3cm}\begin{emphasis}{
I wanted, and I always wanted, and very strongly I wanted (Vittorio Alfieri, 06/09/1783, Letter to Ranieri de' Calzabigi)
}\end{emphasis}\end{changemargin}


The study of magic and the in-depth knowledge of the Magic Lists leads the spellcaster to deepen and learn aspects of it that are not always known. The higher the ability in a Spell List the more the spellcaster will be able to exploit it better than any general user.

Here are abilities that the spellcaster acquires naturally without having to spend any Skills or Feats. The Spell List is indicated, the number of times the list has been taken, by Adept of Magic, the name of the ability and the effect.\\

\textbf{Abjuration List}

\textbf{2: Minor Shield.} Using a Reaction you are able to channel the magical energies that pervade you, manifesting a protection. Until the end of the round, you have +1 Defense.

\textbf{3: Greater Protection.} Using a Reaction you are able to channel the magical energies that pervade you, manifesting a protection. Choose up to 2 creatures within 6 meters, until the end of the round they get +2 Defense or +1 on Saving Throws.

\textbf{List of Water}

\textbf{2: Deep water.} Using a Reaction, he gains resistance 5 to cold and fire until the end of the round.

\textbf{3: Clear waters.} Using a Reaction you can touch a creature and help to free it from poisons and toxins. A new saving throw is allowed (if possible) to lose the poisoned condition.

\textbf{List Air}

\textbf{2: In the clouds.} Using a Reaction you are able to cast the Feather Fall spell on yourself without using any spell points.

\textbf{3: Shock.} Your hand crackles with electricity, the next spell you cast in the round that has an attack roll causes 1d8 more points of electricity damage. It costs a Reaction.

\textbf{Enchantment List}

\textbf{2: Distraction.} When a creature you can observe within 9m of you
makes an attack, you can use a Reaction to distract it. The creature has -2 on attack rolls.

\textbf{3: Major Distraction.} When a creature you can observe within 30 feet of you makes a weapon or spell attack, you can use a Reaction to distract it. Roll 1d6, if the result is 3-4-5 the creature has -2 on the attack roll, if the result is 6 the target of the attack is random.

\textbf{List of Animals and Plants}

\textbf{2: Bark.} Using a Reaction makes your skin tougher and more resistant. You have 2 damage reduction until end of round.

\textbf{3: Claws.} Using an Action makes your natural attacks even sharper for that round. Each natural attack caused by Bleed 1, stacks up to Bleed 5.

\textbf{Heal List}

\textbf{2: Hot hand.} Using a Reaction, the first healing spell you cast in the round on a single subject heals a number of additional Hit Points equal to the level of the spell itself.

\textbf{3: Benevolent Spirit.} By using an Action you channel the residual energy of one of your spells to heal you. In that round, each healing spell you cast causes you to recover 1 hit point.

\textbf{Divination List}

\textbf{2: Premonition.} Using a Reaction he has a fleeting prediction of future events. Until the end of the round, you have a +1 on Reflex saves.

\textbf{3: Blind Spot.} Using a Reaction you can touch a creature, until the end of its round it has a +2 to attack roll.

\textbf{Summon List}

\textbf{2: Hollow hand.} With a Reaction, you can make an object of volume L disappear and reappear whenever you want. You cannot hold more than three objects in this way.

\textbf{3: Long step.} With a Reaction you cause the next Move Action to cause no attacks of opportunity.

\textbf{Fire List}

\textbf{2: Sore Throat.} With a Reaction, you spit a jet of fire in a square adjacent to you. The ground is considered difficult terrain and crossing it or staying in it causes 1d6 Fire damage. It lasts until the end of the round.

\textbf{3: Napalm.} With a Reaction, you touch a weapon. The weapon is engulfed in flames and causes an additional 1d6 Fire damage until the end of the round.

\textbf{Illusion List}

\textbf{2: Prestidigitation.} With a Reaction, you can use the Prestidigitation spell.

\textbf{3: Abundance.} With a Reaction, you can create an inorganic object of volume 1 or less worth 1 gp or less. The object persists until this ability is used again.

\textbf{Invocation List}

\textbf{2: Hope.} With a Reaction, you can illuminate your hand for one round. The hand illuminates only your square and is dim light in the adjacent squares after.

\textbf{3: Augury.} With a Reaction, you touch a creature and turn its luck around. The creature gains a +1 bonus to Hit rolls, Defense, or Saving Throws of its choice until the end of its round.

\textbf{Necromancy List}

\textbf{2: Black Blood.} Using a Reaction for one round, you ignore the fatigued condition.

\textbf{3: Dead Blood.} Using a Immediate Action, you touch a creature. The creature gains a +2 bonus to Fortitude Saving Throws and a -1 penalty to Reflex Saving Throws untile end of round.

\textbf{Earth List}

\textbf{2: Glue.} You are able to cast the Mending spell as a Reaction with no spell point cost.

\textbf{3: Titan.} Using an Action, every time you cast a spell from the Earth List, as long as you are in contact with solid ground, you regain a number of Hit Points equal to the level of the spell cast.

\textbf{Transmutation List}

\textbf{2: Sharing.} Using a Reaction, you touch a creature, and the creature gains an additional Reaction.

\textbf{3: Transition.} With a Reaction, you alter your presence in space. Roll a d6, if you roll 6, you become ethereal until the end of the round.

\textbf{Universal List}

\textbf{2: Hearing.} You have a +4 bonus to checks to recognize spells being cast.

\textbf{3: Sight.} With a Reaction, you can cast the Detect Magic spell without expending Spell Points.

\textbf{4: Knowledge.} You can cast the Identify spell with a Reaction without expending Spell Points.

\end{multicols}

\

\vfill

\begin{center}
	
\includegraphics[width=0.6\linewidth]{immagini/Voynich_Manuscript.png}

\small skip

\textit{A page from Voynich's manuscript, still undeciphered.}
\end{center}


\pagebreak


\section{The Spells}

\begin{multicols}{2}

\medskip

\begin{changemargin}{0.3cm}{0.3cm}\begin{tcolorbox}[title = More special effects!]
The spells listed are those of the 5th edition plus some of my suggestions and other reinterpretations. If you have any suggestions for the Arbiter to handle unexpected crits, talk to him! The spirit of collaboration must always be constructive.
\end{tcolorbox}\end{changemargin}

~

\begin{changemargin}{0.3cm}{0.3cm}\begin{narrator}\index{Optional - Alternatives to Magical Critial Success}
An alternative to the effects of magical Critical Success may be that with spells that cause direct damage or cures, instead of the additional dice, just add half the value of the dice rounded up for each dice. Eg. A 1d10 + 1d8 damage per critical effect becomes 1d10+5.
\end{narrator}\end{changemargin}

\medskip\textbf{Acid Arrow}\index[Spells]{Acid Arrow}\\
\textbf{School}: Water, Earth\\
\textbf{Level}: 2, Common\\
\textbf{Casting Time}: 2 Actions\\
\textbf{Range}: 27 meters\\
\textbf{Components}: V, S, M (a powdered rhubarb leaf and a python stomach)\\
\textbf{Duration}: Instant\\
A glowing green arrow shoots at a target within range and explodes with a spray of acid. Make a ranged spell attack against the target. On a hit, the target takes 4d4 points of acid damage immediately and 2d4 points of acid damage at the end of its next round. On a miss, the arrow sprays the target with acid, dealing half initial damage and doing no damage at the end of the target's next round.\\
\textbf{For each magical critical success rolled} in the Magic Test the damage increases by 2d4.

\medskip\textbf{Acid Surge}\index{Cantrip - Acid Surge}\\
\textbf{School}: Summon\\
\textbf{Level}: 0, Common\\
\textbf{Casting Time}: 1 Action\\
\textbf{Range}: 18 meters\\
\textbf{Components}: V, S\\
\textbf{Duration}: Instant\\
Shoot a bubble of acid. Choose one creature within range or two creatures within 1 meter of each other within range. The target must succeed at a Reflex save or take 1d6 points of acid damage.\\
The spell's damage increases by 1d8 when you reach MP 5, MP 11 and MP 17, but it costs 2 Actions to cast it empowered and 2 Spell Points, you must also have taken Adept of Magic in this Spell List a number of times equal to the empowerments that you want to apply.\\
\textbf{For every two magical critical successes you roll} in Magic Test, shoot one more bubble of acid within range.


\medskip\textbf{Accurate Shot}\index{Cantrip - Accurate Shot}\\
\textbf{School}: Divination\\
\textbf{Level}: 0, Common\\
\textbf{Cast Time}: 2 Actions\\
\textbf{Range}: 9 meters\\
\textbf{Components}: S\\
\textbf{Duration}: 1 round\\
You reach out and point your finger at a target within range. Your magic grants you a brief understanding of the target's Defences. By the end of the next round, you gain +1d6 on your first attack roll against that target. \\
\textbf{For each magical critical success rolled} the bonus lasts for one additional round.

\medskip\textbf{Alarm}\index[Spells]{Alarm}\\
\textbf{School}: Abjuration\\
\textbf{Level}: 1, Common\\
\textbf{Cast Time}: 1 minute\\
\textbf{Range}: 9 meters\\
\textbf{Components}: V, S, M (a bell and a piece of fine silver thread)\\
\textbf{Duration}: 8 hours\\
Set up an alarm against unwanted intrusions. Choose a door, window, or area within range that is no larger than a 6 meters cube. Until the spell ends, you will be warned by an alarm whenever a creature of Tiny size or larger comes into contact with or enters the protected area. When you cast the spell, you can designate creatures that won't trigger the alarm. You also choose whether the alarm is audible or just mental. A mental alarm, if you are within 1.5 kilometers of the protected area, warns you with a noise in your mind. Noise can wake you up if you are sleeping. An audible alarm produces a bell sound for 10 seconds, audible within 20 meters.\\
\textbf{For each magical critical success rolled} in the Magic Test, the duration doubles.

\medskip\textbf{Alter Self}\index[Spells]{Alter Self}\\
\textbf{School}: Transmutation\\
\textbf{Level}: 2, Common\\
\textbf{Casting Time}: 2 Actions\\
\textbf{Range}: Personal\\
\textbf{Components}: V, S\\
\textbf{Duration}: 1 minute per Magic Proficiency\\
Take on a different form. When you cast this spell, you choose one of the following options, the effect of which lasts for the spell's duration. For the spell's duration, you can terminate one option to gain the benefit of another.\\
Aquatic adaptation. You adapt your body to an aquatic environment by developing gills and webbed toes. You can breathe underwater and gain a swim speed equal to your walking speed.\\
\textit{Natural Weapons}. You develop claws, fangs, spikes, horns, or a different natural weapon of your choice. Your unarmed strikes deal 1d6 bludgeoning, piercing, or slashing damage, as appropriate for the chosen natural weapon with which you are proficient. Finally, the natural weapon is magical and you receive a +1 bonus on attack and damage rolls made when using it.\\
\textit{Appearance Change}. Transform your appearance. Decide on your outward appearance, including height, weight, facial features, the sound of your voice, hair length, complexion, and any quirks you desire. You can appear as a member of another race, though none of your stats change. You also can't appear as a creature of a different size than your own, and your base form remains the same; if you are bipedal, you cannot use this spell to become quadrupedal, for example.\\
At any point during the spell's duration, you can use two Actions to change in appearance again in this way.\\
\textbf{For each magical critical success you roll} in the Magic Test you can alter another subject or double the duration.

\medskip\textbf{Anathema}\index[Spells]{Anathema}\\
\textbf{School}: Enchantment\\
\textbf{Level}: 1, Common\\
\textbf{Cast Time}: 1 minute\\
\textbf{Range}: 9 meters\\
\textbf{Components}: V, S, M (a drop of blood)\\
\textbf{Duration}: 1 minute\\
Up to three creatures of your choice that you can see, and that are within range, must make a Will save. Any target that fails this Saving Throw and makes an attack or Saving Throw before the spell ends must roll a d4 and subtract the resulting number from the attack or Saving Throw. \\
\textbf{For each magical critical success you roll} in Magic Test, you may target an additional creature.

\medskip\textbf{Animal Forms}\index[Spells]{Animal Forms}\\
\textbf{School}: Animals and Plants\\
\textbf{Level}: 8, Rare\\
\textbf{Casting Time}: 2 Actions\\
\textbf{Range}: 9 meters\\
\textbf{Components}: V, S\\
\textbf{Duration}: 24 hours\\
You magically transform other creatures into beasts. Choose any number of willing creatures that are within range and can see. You transform each target into the form of a Large or smaller beast with a challenge rating of 4 or lower. On subsequent rounds, you may use 2 Actions to transform affected creatures into new forms.\\
The transformation lasts for each target for the duration of the spell, or until that target drops to 0 Hit Points or dies. You can choose a different shape for each target. The target's game statistics are replaced by the chosen beast's stats, except for its Intelligence, Wisdom, and Charisma scores and Traits, which remain those of the target.
target. The target assumes the Hit Points of its new form, and when it reverts to its normal form, it reverts to the number of Hit Points it had before transforming. If it reverts because it dropped to 0 Hit Points, the excess damage is applied to its original form. As long as the excess damage doesn't reduce the creature's normal form to 0 Hit Points, it isn't knocked unconscious. The creature is limited in the actions it can perform by the nature of its new form, and it cannot speak or cast spells.\\
The target's equipment melds into the new form. The target cannot activate, wield, or otherwise benefit from its equipment.

\medskip\textbf{Animate Dead}\index[Spells]{Animate Dead}\\
\textbf{School}: Necromancy\\
\textbf{Level}: 3, Common\\
\textbf{Cast Time}: 1 minute\\
\textbf{Range}: 3 meters\\
\textbf{Components}: V, S, M (a drop of blood, a piece of meat and a pinch of bone dust)\\
\textbf{Duration}: Instant\\
This spell creates an undead minion. Choose a pile of bones or a corpse of a Medium or Small humanoid within range. Your spell imbues the target with a nefarious semblance of life, reanimating them as an undead creature. The target becomes a skeleton if you choose bones or a zombie if you choose corpse. During each of your rounds, you can use an Action to mentally command any creature you create with this spell that is within 20 meters of you (if you control multiple creatures, you can command all or some of them at the same time, by sending the same command to all). You decide what action the creature will take and where it will move during its next round, or issue it a general command, such as to guard a particular room or corridor. If you send no commands, the creature simply defends itself against hostile creatures. Once given an order, the creature will continue to do so until it is fulfilled. The creature is under your control for 24 hours, after which it will stop following any commands you give it. To maintain control over the creature for another 24 hours, you must cast this spell on it again before the current 24-hour period ends. This use of the spell reasserts your control over up to four creatures you have animated with this spell, rather than animating a new one.\\.
\textbf{For each magical critical success you roll} in the Magic Test you animate or reassert control over two undead creatures. Each of these creatures must come from a different corpse or pile of bones.

\medskip\textbf{Animate Objects}\index[Spells]{Animate Objects}\\
\textbf{School}: Transmutation\\
\textbf{Level}: 5, Common\\
\textbf{Cast Time}: 1 minute\\
\textbf{Range}: 36 meters\\
\textbf{Components}: V, S\\
\textbf{Duration}: Concentration, max 1 minute\\
Objects come to life at your command. Choose up to ten nonmagical items that are within range and are not being worn or carried. Medium targets count as two items, Large targets count as four items, Huge targets count as eight items. You cannot animate objects larger than Huge. Each target animates and becomes a creature under your control until the spell ends or until it is reduced to 0 Hit Points.\\
As an Action, you can mentally command any creature you spawned with this spell that is within 150 meters of you (if you control multiple creatures, you can command only some or all of them at the same time, giving the same command to each). You decide what action the creature will take and where it will move during its next round, or you can issue a generic command, such as to guard a particular room or corridor. If you issue no commands, the creature will simply defend itself against hostile creatures. Once given an order, the creature will continue to follow it until it has completed its task.\\
\textbf{For each magical critical success rolled} in the Magic Test, the maximum duration doubles.
\bigskip

\medskip\textbf{Grease}\index[Spells]{Grease}\\
\textbf{School}: Animals and Plants\\
\textbf{Level}: 1, Common\\
\textbf{Casting Time}: 2 Actions\\
\textbf{Range}: 18 meters\\
\textbf{Components}: V, S, M (a piece of pork rind or butter or "unto topetto") \\
\textbf{Duration}: 1 minute\\
Grease grease covers the ground in a 3m square, centered on a point within range, and rounds it into hindering terrain for the duration\\
When the blubber appears, each target standing in the area must succeed at a Reflex save or be knocked prone. A creature that enters the area or ends its round there must succeed at a Reflex save or be knocked prone.

\medskip\textbf{Anti-Detection}\index[Spells]{Anti-Detection}\\
\textbf{School}: Abjuration\\
\textbf{Level}: 3, Uncommon\\
\textbf{Casting Time}: 2 Actions\\
\textbf{Range}: Contact\\
\textbf{Components}: V, S, M (a pinch of diamond dust worth 25 gp sprinkled on the target, which the spell consumes)\\
\textbf{Duration}: 8 hours\\
For the duration, hide the target you were in contact with from divination magic. The target can be a willing creature or a place or object that occupies a space equivalent to a cube no larger than 3 meter in edge. The target cannot become the target of any divination magic or be perceived by magical scrying senses.

\medskip\textbf{Anti-Life Shell}\index[Spells]{Anti-Life Shell}\\
\textbf{School}: Animals and Plants\\
\textbf{Level}: 5, Uncommon\\
\textbf{Casting Time}: 2 Actions\\
\textbf{Range}: Self (10' radius)\\
\textbf{Components}: V, S\\
\textbf{Duration}: maximum 1 hour\\
A barrier of light extends out to a 3m radius around you, moving with you and staying centered on you, keeping out creatures other than undead or constructs. The barrier remains for the duration. \\
The barrier prevents a subject creature from passing through it in any way. An affected creature can cast spells or make attacks with ranged or reach weapons across the barrier. If you move so that an affected creature is forced through the barrier, the spell ends.

\medskip\textbf{Anti-Magic Field}\index[Spells]{Anti-Magic Field}\\
\textbf{School}: Abjuration\\
\textbf{Level}: 8, Rare\\
\textbf{Casting Time}: 2 Actions\\
\textbf{Range}: Self (3m-radius sphere)\\
\textbf{Components}: V, S, M (a pinch of powdered iron or iron files)\\
\textbf{Duration}: Concentration, max 1 hour\\
You are surrounded by an invisible 3m-radius sphere of anti-magic. This area is separated from the magical energy that permeates the multiverse. Spells cannot be cast within the sphere, summoned creatures disappear, and magic items also become normal. Until the spell ends, the sphere moves with you, centered on you. Spells and other magical effects, except those created by an artifact or Patron, are suppressed within the sphere and cannot penetrate it. A slot expended to cast a suppressed spell is expended. While an effect is suppressed, it doesn't work, but the time it spends suppressed counts towards its duration.\\

\medskip\textbf{Arcane Eye}\index[Spells]{Arcane Eye}\\
\textbf{School}: Divination\\
\textbf{Level}: 4, Common\\
\textbf{Cast Time}: 2 Actions\\
\textbf{Range}: 9 meters\\
\textbf{Components}: V, S, M (a piece of bat cloak)\\
\textbf{Duration}: Concentration, max 1 hour\\
You create an invisible magical eye within range, which floats in the air for the duration.\\
You mentally receive visual information from the eye, which has normal vision and darkvision out to 10 meters. The eye can look in all directions. As a move action, you can move the eye 10 meters in any direction. There is no limit to how far the eye can move, but it cannot enter another plane of existence. A solid barrier blocks movement of the eye, but it can pass through an opening as small as 2.5 centimeters in diameter.

\medskip\textbf{Arcane Hand}\index[Spells]{Arcane Hand}\\
\textbf{School}: Invocation\\
\textbf{Level}: 5, Uncommon\\
\textbf{Casting Time}: 2 Actions\\
\textbf{Range}: 36 meters\\
\textbf{Components}: V, S, M (an eggshell and a snakeskin glove)\\
\textbf{Duration}: Concentration, 1 minute\\
You create a Large hand composed of transparent, luminous energy in an unoccupied space that you can see within range and see. The hand lasts for the duration, and moves at your command, mimicking the motions of your hand.\\
The hand is an object that has Defence 25 and Hit Points equal to your maximum Hit Points. It has Strength 4 and Dexterity 0. The hand does not fill its space.
When you cast the spell and as 2 Actions during your subsequent rounds, you can move your hand up to 20 meters and then generate one of the following effects.\\

- \textit{Grabbing Hand}. The hand attempts to grab a Huge or smaller creature within 1 meter of it. To resolve the grapple action you use Hand Strength. If the target is Medium or smaller, you get +1d6 on the check. While the hand is grappling the target, you can use an Action to make the hand constrict the target. When you do, the target takes bludgeoning damage equal to 2d6 + your Intelligence or Wisdom \\
- \textit{Hand of Strength}. The hand tries to push a creature 1 meter in a direction you choose. Make a hand Strength check contested by the target's Strength check. If the target is Medium or smaller, you get +1d6 on the check. If you win the contest, the hand pushes the target 1 meter plus 1 meter multiplied by the Intelligence or Wisdom value (minimum 1 meter). The hand moves with the target to stay within 1 meter of them.\\
- \textit{In between Hand}. The hand comes between you and a creature of your choice until you give it a different command. The hand moves so that it stays between you and the target, giving you half cover against the target. The target can't move through the hand's space if its Strength score is equal to or lower than the hand's Strength score. If its Strength score is higher than the hand's Strength score, the target can move through the hand's space, but it treats that space as hindering terrain. \\
- \textit{Clenched Fist}. The hand strikes a creature or object within 1 meter of it. Make a melee spell attack using your hand. On a hit, the target takes 4d8 force damage.\\

\medskip
\textbf{For each magical critical success rolled} in the Magic Test the clenched fist option's damage increases by 1d8 and the grabbing hand option's damage increases by 1d6.


\medskip\textbf{Magic Lock}\index[Spells]{Magic Lock}\\
\textbf{School}: Abjuration\\
\textbf{Level}: 2, Common\\
\textbf{Cast Time}: 2 Actions\\
\textbf{Range}: Contact\\
\textbf{Components}: V, S, M (gold dust worth at least 25 gp, which is consumed by the spell) \\
\textbf{Duration}: Until dissolved\\
You cast the spell against a locked door, window, portal, chest, or other entrance, and it becomes locked for the duration. You and the creatures you indicated when you cast this spell can open the item normally. You can also readied a password that, when spoken within 1 meter of the item, suppresses the spell for 1 minute. Otherwise the opening is impassable until it is destroyed or the spell is dispelled or suppressed. Casting lock pick on the item suppresses Magic Lock for 10 minutes.\\
While affected by this spell, the item is more difficult to destroy or force open; the DC for breaking it or picking a lock on it increases by 10.\\
\textbf{For each magical critical success rolled} in the Magic Test you can affect another lock.

\medskip\textbf{Arcane Sword}\index[Spells]{Arcane Sword}\\
\textbf{School}: Invocation\\
\textbf{Level}: 7, Rare\\
\textbf{Casting Time}: 2 Actions\\
\textbf{Range}: 18 meters\\
\textbf{Components}: V, S, M (a miniature platinum sword with a copper-zinc hilt and pommel, valued at 250 gp)\\
\textbf{Duration}: Concentration, max 1 minute \\
For the duration, you create a plane of force in the shape of a floating sword within range. When the sword appears, you make a melee attack with MP modifier + spell modifier against a target of your choice within 1 meter of the sword. On a hit, the target takes 3d10 force damage. Until the spell ends, you can use an action each of your rounds to move the sword 6 meters to a spot you can see and repeat this attack against the same or a different target.

\medskip\textbf{Arcanist's Magic Aura}\index[Spells]{Arcanist's Magic Aura}\\
\textbf{School}: Illusion\\
\textbf{Level}: 2, Uncommon\\
\textbf{Casting Time}: 2 Actions\\
\textbf{Range}: Contact\\
\textbf{Components}: V, S, M (a small square of silk)\\
\textbf{Duration}: 24 hours\\
You place an illusion on a creature or object you touch, so that divination spells reveal false information about it. The target can be a willing creature or an object that isn't being carried or worn by another creature. When you cast this spell, choose one or both of the following effects. The effect lasts for the duration. If you cast this spell on the same creature or object every day for 30 days, placing the same effect each time, the illusion will last until it is dispelled.\\
\textit{False Aura}. You change how the target appears to magical spells and effects, such as detect magic, that detect magical auras. You can make a normal item appear magical, a magic item appear nonmagical, or change the item's magical aura so that it appears to belong to a Magic List of your choice. When you employ this effect on an object, you can make the false magic appear to any creature that manipulates it.\\
\textit{Mask}. You change how the target appears to spells and magical effects that detect creature type or Traits, such as activating the symbol spell. Choose a creature type or Trait, and other spells and magical effects will treat the target as being a creature of that type or Trait, and no longer than the original.

\medskip\textbf{Astral Projection}\index[Spells]{Astral Projection}\\
\textbf{School}: Necromancy\\
\textbf{Level}: 9, Very Rare\\
\textbf{Casting Time}: 2 Actions\\
\textbf{Range}: 3 meters\\
\textbf{Component}: V, S, M (for each creature affected by this spell, you must supply a hyacinth worth at least 1000 gp and an elegantly carved silver ingot worth at least 100 gp, all of which are consumed by the spell)\\
\textbf{Duration}: Special\\
You and up to eight other willing creatures within range project your astral bodies into the Astral Plane (the spell fails and the casting is wasted if you were already in that plane). The material body you leave behind is unconscious and in a state of suspended animation; it does not need food or water and does not age.\\
Your astral body closely resembles your mortal form, replicating your in-game stats and items. The main difference is the addition of a silver cord that extends from the shoulder blades to a 30cm behind you, then becoming invisible. The cord is your connection to your material body. As long as this connection remains intact, you can go home. If the cord is cut (an event that occurs only when a specific effect indicates so) your soul and body are separated, killing you instantly.\\
Your astral form can freely travel the Astral Plane and pass through portals that lead to other planes from there. If you enter a new plane or return to the plane you were on when you cast the spell, your body and items are transported along the silver cord, allowing you to reenter your body upon entering the new plane. Your astral form is a separate incarnation. Any damage or other effects that apply to it do not affect your physical body, nor do they appear there upon your return.\\
The spell ends for you and your companions when you use an action to end it. When the spell ends, the creature it affects returns to its physical body, and awakens. The spell may also have an early end for you or one of your companions. A successful dispel magic spell used on the astral or physical body ends the spell for that creature. If the creature's original body or astral form drops to 0 Hit Points, the spell ends for that creature. If the spell ends and the silver cord is intact, the cord pulls the creature's astral form back to its body, ending its state of suspended animation.\\
If you are returned to your body prematurely, your companions must remain in their astral forms and find their way back to their bodies on their own, usually dropping to 0 Hit Points.

\medskip\textbf{Awakening}\index[Spells]{Awakening}\\
\textbf{School}: Animals and Plants\\
\textbf{Level}: 5, Rare\\
\textbf{Casting Time}: 8 hours\\
\textbf{Range}: Contact\\
\textbf{Components}: V, S, M (an agate worth at least 1000 gp, which the spell consumes)\\
\textbf{Duration}: Instant\\
After spending the casting time drawing magical patterns with a precious gem, you touch a Huge or smaller beast or plant. The target must either have no Intelligence score or have Intelligence -3 or less. The target gains 0 Intelligence. The target also gains the ability to speak a language you know. If the target is a plant, it gains the ability to move its limbs, roots, vines, vines, and so on, and gains human-like senses. The Arbiter will choose the statistics appropriate to the awakened plant type, such as the statistics for awakened bush or awakened tree. \\
The awakened beast or plant is fascinated by you for 30 days or until you or your companions do it damage. When the charmed condition ends, the awakened creature chooses whether to remain friendly to you, based on how you treated it while it was charmed.\\
\textbf{For each magical critical success you roll} in the Magic Test you double the duration of the fascination up to a maximum of 1 year.

\medskip\textbf{Banishment}\index[Spells]{Banishment}\\
\textbf{School}: Abjuration\\
\textbf{Level}: 4, Common\\
\textbf{Cast Time}: 2 Actions\\
\textbf{Range}: 18 meters\\
\textbf{Component}: V, S, M (an object despised by the target)\\
\textbf{Duration}: 1 minute\\
You try to send a creature within range and that you can see to another plane of existence. The target must succeed at a Will save or be banished. If the target is native to the plane of existence you are on, you exile the target to a harmless demiplane. While there, the target is incapacitated. The target remains there until the spell ends, when it will reappear in the space it left or in the nearest unoccupied space if its original space is now occupied. If the target is native to a different plane of existence than the one you are on, the target vanishes with a soft bang, returning to its home plane. If the spell ends before 1 minute has elapsed, the target reappears in the space it left or in the nearest unoccupied space if its original space is occupied. \\
\textbf{For each magical critical success you roll} in the Magic Test you can affect another creature or banish the creature for a week.

\medskip\textbf{Barkskin}\index[Spells]{Barkskin}\\
\textbf{School}: Animals and Plants\\
\textbf{Level}: 2, Common\\
\textbf{Cast Time}: 2 Actions\\
\textbf{Range}: Contact\\
\textbf{Components}: V, S, M (a handful of oak bark)\\
\textbf{Duration}: 1 hour\\
The skin of the target you are in contact with when you cast the spell becomes rough and bark-like in appearance until the spell ends, and the target's Defence cannot be lower than 16, whatever Armour it is wearing. is wearing.

\medskip\textbf{Blessing of Cattalm}\index[Spells]{Blessing of Cattalm}\\
\textbf{School}: Enchantment, Fire\\
\textbf{Level}: 3, Very Rare\\
\textbf{Casting Time}: 2 Actions\\
\textbf{Range}: 18 meters\\
\textbf{Components}: V, S, M (a splash of vinegar)\\
\textbf{Duration}: Instant\\
You call down the wrath of Cattalm on your opponent. The target creature takes 4d6 fire damage, must make a Will saving throw or suffer a -1d6 penalty on the next proficiency check, attack roll, or saving throw, and the caster increases his pool of Fate Points by one. .\\
\textbf{For every two magical critical successes you roll} in the Magic Test you can affect another one creature.

\medskip\textbf{Blessing of Life}\index[Spells]{Blessing of Life}\\
\textbf{School}: Abjuration\\
\textbf{Level}: 3, Rare\\
\textbf{Casting Time}: 2 Actions\\
\textbf{Range}: 9 meters\\
\textbf{Components}: V, S\\
\textbf{Duration}: 1 minute, Concentration\\
This spell bestows hope and vitality. Choose up to 6 creatures within range. For the duration, each target has +2 on Will saves and heal +1 hit per round.\\
\textbf{If you roll 2 magical Critical Successes and also have the Healing List} each round the chosen creatures recover 1 more Hit Point.

\medskip\textbf{Bestow curse} \index[Spells]{Bestow curse}\\
\textbf{School}: Necromancy\\
\textbf{Level}: 3, Uncommon\\
\textbf{Casting Time}: 2 Actions\\
\textbf{Range}: Contact\\
\textbf{Components}: V, S\\
\textbf{Duration}: 1 minute\\
A creature you touch must succeed at a Will save or be cursed for the duration of the spell. When you cast this spell, choose the nature of the curse from the following options:\\


- Choose an ability score. While cursed, the target has -1d6 on ability checks and Saving Throws based on that ability score, if any.\\
- While cursed, the target has -1d6 on attack rolls against you. \\
While cursed, the target must make a Will save at the start of each of its rounds. On a failed save, he wastes his round's actions doing nothing.\\
- While the target is cursed, your attacks and spells deal an additional 1d8 Void damage against them.\\

The remove curse spell (see description) ends this effect. At the Arbiter's discretion, you may choose a curse with a different effect, but it should still be no more powerful than the ones described above. The Arbiter holds the final judgment on the effect of a curse.\\
\textbf{If you score a crit} the duration of the curse is one day. If you get 3 crits the duration is permanent.


\medskip\textbf{Bestow Lesser Curse}\index[Spells]{Bestow Lesser Curse}\\
\textbf{School}: Universal\\
\textbf{Level}: 1, Common\\
\textbf{Casting Time}: 2 Actions\\
\textbf{Range}: Contact\\
\textbf{Components}: V, S\\
\textbf{Duration}: 1 minute\\
A creature you touch must succeed on a Will save or be cursed for the spell's duration. When you cast this spell, choose the nature of the curse from the following options:\\

- Choose an ability score. While cursed, the target has -1 on basic proficiency checks and saving throws based on that ability score if applicable.\\
- While cursed, the target has -2 on attack rolls against you.\\
- While cursed, the target must make a Will saving throw at the start of each of its rounds. If he fails, he wastes 1 Action of that round without doing anything.\\

The remove curse spell (see description) ends this effect. At the Storyteller's discretion, you may choose a curse with a different effect, but it should not be more powerful than those described above. The Storyteller holds the final judgment on a curse's effect.\\
\textbf{For each Magical Critical Success} obtained in the Magic Test, choose another creature within 6 meters of the first.


\medskip\textbf{Black Tentacles}\index[Spells]{Black Tentacles}\\
\textbf{School}: Summon\\
\textbf{Level}: 4, Uncommon\\
\textbf{Casting Time}: 2 Actions\\
\textbf{Range}: 27 meters\\
\textbf{Components}: V, S, M (a piece of tentacle from a giant octopus or giant squid)\\
\textbf{Duration}: 1 minute\\
Slimy ebony tendrils fill a 6 meters square on the ground that you can see and are within range. For the duration of the spell, these tentacles turn the area into hindering terrain.\\
When a creature enters the affected area for the first time in a round or begins its round there, it must succeed at a Reflex save or take 3d6 bludgeoning damage and remain \hyperlink{intralciato}{entangled} by the tentacles until the spell ends. A creature that begins its round in the area and is already entangled by the tentacles takes 3d6 bludgeoning damage. A creature ensnared by the tentacles can use 2 Actions to make a new Saving Throw to be free that round.

\medskip\textbf{Blade Barrier}\index[Spells]{Blade Barrier}\\
\textbf{School}: Invocation\\
\textbf{Level}: 6, Common\\
\textbf{Casting Time}: 2 Actions\\
\textbf{Range}: 18 meters\\
\textbf{Components}: V, S\\
\textbf{Duration}: 10 minutes \\
You create a vertical wall of spinning blades made of magical energy, razor sharp. The wall appears within range and lasts for the duration. You can create a straight wall up to 30 meters long, 6 meters high, and 1 meter thick, or a circular wall up to 20 meters in diameter, 6 meters high, and 1 meter thick. The wall provides three-quarters of cover to creatures behind it, and its space is hindering terrain. \\
When a creature enters the wall's area for the first time in a round or begins its round there, the creature must make a Reflex save. It takes 6d10 slashing damage on a failed save, or half as much damage on a successful one. \\
A spellcaster who is at a distance of one meter from the barrier of blades is considered distracted.

\medskip\textbf{Bless Water}\index[Spells]{Bless Water}\\
\textbf{School}: Universal\\
\textbf{Level}: 2, Common\\
\textbf{Casting Time}: 10 Minutes\\
\textbf{Range}: Contact\\
\textbf{Components}: V, S, M (25 gold coins offered to the church)\\
\textbf{Duration}: Instant\\
Bless up to a liter of liquid, enough to create 5 bottles of Holy Water.\\
You must be a Follower or Devotee to cast this spell.\\
\textbf{For each magical critical success you roll} in the Magic Test, bless an additional quart of liquid. 

\medskip\textbf{Blessing}\index[Spells]{Blessing}\\
\textbf{School}: Universal\\
\textbf{Level}: 1, Common\\
\textbf{Cast Time}: 2 Actions\\
\textbf{Range}: 9 meters\\
\textbf{Components}: V, S, M (a splash of holy water)\\
\textbf{Duration}: 1 minute\\
Bless up to three creatures of your choice within range. Targets gain +1 on Saving Throws and attack rolls.\\
More blessings, even from different Patrons, do not add up. You must be a Follower or Devotee to cast this spell.\\
\textbf{For each magical critical success you roll} in Magic Test, you may add one creature as a target.

\medskip\textbf{Blindness/Deafness}\index[Spells]{Blindness/Deafness}\\
\textbf{School}: Necromancy\\
\textbf{Level}: 2, Common\\
\textbf{Casting Time}: 2 Actions\\
\textbf{Range}: 9 meters\\
\textbf{Components}: V\\
\textbf{Duration}: 1 minute, Concentration\\
You can blind or deafen an enemy. Choose a creature that is within range and that you can see. The target must make a Fortitude Saving Throw. On a failed save, the target is blinded or deafened (your choice) for the duration.\\
\textbf{For every two magical critical successes you roll} in the Magic Test, you can add another target within range. If you roll 3 magical critical successes the target is affected by the spell for the whole day.

\medskip\textbf{Advanced Blindness/Deafness}\index[Spells]{Advanced Blindness/Deafness}\\
\textbf{School}: Necromancy\\
\textbf{Level}: 3, Uncommon\\
\textbf{Cast Time}: 2 Actions\\
\textbf{Range}: 36 meters\\
\textbf{Components}: V,S,M (cerumen or a piece of black cloth)\\
\textbf{Duration}: 10 minutes\\
You can blind or deafen an enemy. Choose a creature within range and that you can see. The target must make a Fortitude saving throw. On a failed save, the target is blinded or deafened (your choice) for the duration.\\
\textbf{For each Magical Critical Success} obtained on the Magic Test you can target one additional creature.\\
\textbf{Critical Failure Saving Throw}: On a critical failure the effect is permanent.


\medskip\textbf{Blazing Smite}\index[Spells]{Blazing Smite}\\
\textbf{School}: Invocation\\
\textbf{Level}: 1, Rare\\
\textbf{Casting Time}: 1 Immediate Action\\
\textbf{Range}: personal\\
\textbf{Components}: V\\
\textbf{Duration}: 1 minute\\
The target struck by the blow takes an extra 1d6 points of fire damage. Each round he must make a Fortitude save or suffer 1d6 fire damage, this effect ends after a minute or when the save is successful.\\
You must pass a Magic Test for casting this spell while fighting.\\
\textbf{For each magical critical success you roll} in the Magic Test you deal +1d6 Fire damage.

\medskip\textbf{Blinding Smite}\index[Spells]{Blinding Smite}\\
\textbf{School}: Invocation\\
\textbf{Level}: 3, Raro\\
\textbf{Casting Time}: 1 Immediate Action\\
\textbf{Range}: personale\\
\textbf{Components}: V\\
\textbf{Duration}: 1 minute\\
The target hit by the strike takes an extra 3d8 Light damage, and the target must succeed on Fortitude Saving Throw or become Blinded until the spell ends. At the end of each of its round, the Blinded target repeats the saving throw, ending the spell on itself on a success. \\
You must pass a Magic Test for casting this spell while fighting.\\
\textbf{For each magical critical success you roll} in Magic Test, you deal +1d8 light damage.

\medskip\textbf{Block Monster}\index[Spells]{Block Monster}\\
\textbf{School}: Enchantment\\
\textbf{Level}: 5, Common\\
\textbf{Cast Time}: 2 Actions\\
\textbf{Range}: 27 meters\\
\textbf{Components}: V, S, M (a small straight piece of iron)\\
\textbf{Duration}: 1 minute\\
Choose a creature that is within range and that you can see. The target must succeed at a Will save or be paralyzed for the duration. This spell has no effect on undead or constructs. At the end of each of its rounds, the target can make another Will save. If it succeeds, the spell ends for that target. \\
\textbf{For each magical critical success you roll} in the Magic Test you can add one creature as a target as long as they are within 10 meters of each other.

\medskip\textbf{Hold Person}\index[Spells]{Hold Person}\\
\textbf{School}: Enchantment\\
\textbf{Level}: 2, Common\\
\textbf{Casting Time}: 2 Actions\\
\textbf{Range}: 18 meters\\
\textbf{Components}: V, S, M (a small straight piece of iron)\\
\textbf{Duration}: 1 minute\\
Choose one humanoid that is within range and that you can see. The spell has no effect on creatures of CR 4 or higher. The target must succeed at a Will save or be paralyzed for the duration.\\
\textbf{For each magical critical success you roll} in the Magic Test you can add one creature as a target as long as they are within 10 meters of each other.

\medskip\textbf{Hold Person Advanced}\index[Spells]{Block Person Advanced}\\
\textbf{School}: Enchantment\\
\textbf{Level}: 4, Uncommon\\
\textbf{Cast Time}: 2 Actions\\
\textbf{Range}: 18 meters, radius 6 meters\\
\textbf{Components}: V, S, M (a small straight piece of silver)\\
\textbf{Duration}: 1 minute\\
Blocks up to 2d4 CR (or levels) of creatures within 20 meters of you in a 6 meters radius. You start by blocking the lowest CR creatures and subtracting the CR from the 2d4 you roll, work this until you have no more points to block creatures. Targets must succeed at a Will save or be paralyzed for the duration. The spell has no effect on creatures of CR 6 or higher\\
\textbf{For each magical critical success rolled} in the Magic Test you can add 2 points to the 2d4 rolled.

\medskip\textbf{Blur}\index[Spells]{Blur}\\
\textbf{School}: Illusion\\
\textbf{Level}: 2, Common\\
\textbf{Casting Time}: 2 Actions\\
\textbf{Range}: Personal\\
\textbf{Components}: V\\
\textbf{Duration}: 1 minute \\
Your body becomes blurry, indistinct and shaky to anyone who sees you. For the duration of the spell, all creatures have a -1d6 on attack rolls against you. Attackers who don't rely on sight are immune to this effect, such as if they have blindsight or are able to distinguish illusions, such as true seeing.

\medskip\textbf{Bond of Warding}\index[Spells]{Bond of Warding}\\
\textbf{School}: Abjuration\\
\textbf{Level}: 2, Common\\
\textbf{Casting Time}: 2 Actions\\
\textbf{Range}: Contact\\
\textbf{Components}: V, S, M (a pair of platinum rings worth 50 gp each, which you and the target must wear for the duration)\\
\textbf{Duration}: 1 hour\\
You cast the spell in touch with a creature you want to protect. You create a mystical connection between you and the target until the spell ends. While the target is within 20 meters of you, it gains a +1 bonus to Defence and Saving Throws, and has resistance to all damage. Also, whenever the target takes damage, you take the same amount. The spell ends if you drop to 0 Hit Points or you and the target move more than 20 meters away. It also ends if you cast it again on the same creature it's already affecting. You can end the spell as an action.

\medskip\textbf{Bubble of Life}\index[Spells]{Bubble of Life}\\
\textbf{School}: Air, Abjuration\\
\textbf{Level}: 4, Uncommon\\
\textbf{Cast Time}: 1 minute\\
\textbf{Range}: 9 meters\\
\textbf{Component}: V, S, M (silver and diamond dust per 100 gp consumed)\\
\textbf{Duration}: 1 hour per Magic Proficiency\\
You can create up to 6 bubbles that surround creatures you designate.
The total duration is 1 hour per point of Magic Proficiency divided as desired among the creatures in the bubbles.
This bubble allows subjects to breathe freely, even underwater or in a vacuum, and renders them immune to noxious gases and vapors, including inhaled diseases and poisons, and spells such as foul-smelling cloud and mist of death. The bubble protects subjects from extreme temperatures (but not those that cause damage each round) and extreme pressure.

\medskip\textbf{Burning Blade}\index[Spells]{Burning Blade}\\
\textbf{School}: Fire\\
\textbf{Level}: 2, Common\\
\textbf{Casting Time}: 1 Immediate Action\\
\textbf{Range}: Personal\\
\textbf{Components}: V, S, M (a sumac leaf)\\
\textbf{Duration}: Concentration, max 10 minutes \\
You create a flaming blade in your hand. The blade is similar in size and shape to a scimitar, and remains for the duration. If you let go of the blade, it disappears, but you can create another one as an Action. You can use 2 Actions to make a melee attack with the flaming blade. On a hit, the target takes 3d6 fire damage. The flaming blade sheds bright light in a 3m radius and dim light for an additional 3 meter.\\
\textbf{For every two criticals gained} in the Magic Test, the damage increases by 1d6.

\medskip\textbf{Burning Wave}\index[Spells]{Burning Wave}\\
\textbf{School}: Fire\\
\textbf{Level}: 1, Common\\
\textbf{Cast Time}: 2 Actions\\
\textbf{Range}: Self (4-meter cone)\\
\textbf{Components}: V, S\\
\textbf{Duration}: Instant\\
Keep your hands closed in front of you, a powerful hot wave is generated from your every punch. Each creature in a 5m cone must make a Reflex save. A creature takes 1d4 damage per Magic Proficiency, up to a maximum of 5d4, fire damage on a failed save, or half as much on a successful one. Heat ignites flammable objects in the area that are not worn or carried.\\
\textbf{For each magical critical success rolled} in the Magic Test the damage increases by 1d4.\\
\textbf{Save Critical Success/Failure}: On a critical failure the damage is doubled, on a critical success the damage is further halved.


\medskip\textbf{CTRLC+CTRLV (Copy Paste)}\index[Spells]{CTRLC+CTRLV (Copy Paste)}\\
\textbf{School}: Universal\\
\textbf{Level}: 1, Very Rare\\
\textbf{Cast Time}: 2 Actions\\
\textbf{Range}: Personal\\
\textbf{Components}: V, S, M (three small ceramic cubes bearing the letter C, the letter V and the CTRL glyph)\\
\textbf{Duration}: 1 minute per Magic Proficiency\\
This spell allows you to copy text from one source to another. In the case of a non-magical source this can be a book, a parchment, runes on a slab or a staff. The destination that is placed on the source will copy the symbols in shape and size up to its capacity, for a maximum of 1 (destination) page per minute.

\medskip\textbf{Call Lightning}\index[Spells]{Call Lightning}\\
\textbf{School}: Air\\
\textbf{Level}: 3, Common\\
\textbf{Casting Time}: 1 round\\
\textbf{Range}: 36 meters\\
\textbf{Components}: V, S\\
\textbf{Duration}: Concentration, max 10 minutes\\
A storm cloud appears as a 3m-tall cylinder with a radius of 20 meters, centered on a point you can see 30 meters above you. The spell automatically fails if you cannot see the spot in the air where the storm cloud will appear (for example, if you are in a room that cannot accommodate the cloud). When you cast the spell, you choose a spot you can see within range. Lightning will strike that point from the cloud. Each creature within 1 meter of that point must make a Reflex save. A creature takes 3d10 Electricity damage on a failed save, or half as much damage on a successful one. During each of your rounds until the spell ends, you can use two Actions to call down another bolt of lightning in this way, targeting the same or a different spot.\\
If you are outside in stormy conditions when you cast this spell, the spell gives you control of the existing storm rather than creating a new one. Under these conditions, the spell's damage increases by 1d10. \\
\textbf{For each magical critical success rolled} in the Magic Test the damage increases by 1d8.

\medskip\textbf{Calm Emotions}\index[Spells]{Calm Emotions}\\
\textbf{School}: Enchantment\\
\textbf{Level}: 2, Common\\
\textbf{Casting Time}: 2 Actions\\
\textbf{Range}: 18 meters\\
\textbf{Components}: V, S\\
\textbf{Duration}: Concentration, max 1 minute\\
You try to suppress strong emotions in a group of people. Each humanoid in a 6 meters-radius sphere centered on a point you choose within range must make a Will save; if she wishes, a creature can choose to fail this Saving Throw. If a creature fails its Saving Throw, choose one of these two effects. \\
\textit{Appease}. You can suppress any effect that makes the target charmed or frightened. When this spell ends, the suppressed effects resume, provided their duration has not expired in the meantime. \\
\textit{Indifference}. You can make a target indifferent to a creature of your choice toward which it is hostile. This indifference ends if the target is attacked or damaged by a spell or if it sees one of its friends being damaged. When the spell ends, the creature becomes hostile again, unless the Arbiter determines otherwise.

\medskip\textbf{Cattalm's Slap}\index[Spells]{Cattalm's Slap}\\
\textbf{School}: Summon\\
\textbf{Level}: 1, Uncommon\\
\textbf{Casting Time}: 1 Reaction, which you can take in response to damage done to you by a creature within 20 meters of you that you can see\\
\textbf{Range}: 18 meters\\
\textbf{Components}: V, S\\
\textbf{Duration}: Instant\\
You point your finger, and the creature that harmed you is momentarily engulfed in fiendish flames. The creature must make a Reflex save. It takes 2d10 fire damage on a failed save, or half as much damage on a successful one. \\
\textbf{For each magical critical success rolled} in the Magic Test the damage increases by 1d6.

\medskip\textbf{Chained Lightning}\index[Spells]{Chained Lightning}\\
\textbf{School}: Air\\
\textbf{Level}: 6, Rare\\
\textbf{Cast Time}: 2 Actions\\
\textbf{Range}: 45 meters\\
\textbf{Components}: V, S, M (some fur; a piece of amber, glass, or a crystal rod; and three silver pins)\\
\textbf{Duration}: Instant\\
You create a bolt of electricity that strikes a target of your choice you can see within range. This generates a further bolt that hits the closest target within 6 meters. The process continues until 7 targets have been hit or there are no more new ranged opponents. A target can be a creature or object of at least medium size and can be the target of only one bolt. A target must make a Reflex save. The target takes 8d6 lightning damage on a failed save, or half as much damage on a successful one. \\
\textbf{For each magical critical success rolled} in the Magic Test the bolt shoots towards an additional target.\\
\textbf{Save Critical Success/Failure}: On a critical failure the damage is doubled, on a critical success the damage is further halved

\medskip\textbf{Modify Memory}\index[Spells]{Modify Memory}\\
\textbf{School}: Enchantment\\
\textbf{Level}: 5, Very Rare\\
\textbf{Cast Time}: 2 Actions\\
\textbf{Range}: 9 meters\\
\textbf{Components}: V, S\\
\textbf{Duration}: Concentration, max 1 minute\\
You attempt to reshape another creature's memories. A creature you can see must make a Will save. If you're fighting it, the creature has +1d6 on its Saving Throw. On a failed save, the target becomes fascinated by you for the spell's duration. The charmed target is incapacitated and unaware of its surroundings, though it can still hear you. If he takes damage or becomes the target of another spell, this spell ends, and none of the target's memories are affected. \\
While the target is fascinated by this spell, you can affect the target's memories of an event it experienced in the last 24 hours that lasted no more than 10 minutes. You can permanently erase all memories of the event, allow the target to remember the event with perfect clarity and fine detail, modify the memory of the event's details, or create the memory of another event. You must be able to talk to the target to describe how their memories will be affected, and they must be able to understand your language in order for the modified memories to become embedded in their memory. If the spell ends before you finish describing the altered memories, the creature's memory is unaffected. Otherwise, the modified memories take effect after the spell ends.\\
A modified memory doesn't necessarily affect the creature's behavior, particularly if its memories contradict the creature's natural inclinations, Traits, or faith. An illogically altered memory, such as implanting a memory of how much the creature loves dousing itself in acid, is removed, like a bad dream. The Arbiter may judge a modified memory too nonsensical to have any effect on a creature. A remove curse or greater restoration spell cast on the target restores its true memories.\\
\textbf{For each magical critical success you roll} in Magic Test you can alter a target's memories of an event that occurred up to 7 days ago, 30 days ago, 1 year ago, or any point in the creature's past.

\medskip\textbf{Charm Person}\index[Spells]{Charm Person}\\
\textbf{School}: Enchantment\\
\textbf{Level}: 1, Common\\
\textbf{Cast Time}: 2 Actions\\
\textbf{Range}: 9 meters\\
\textbf{Components}: V, S\\
You attempt to charm a humanoid that is within range and that you can see. He must make a Will save, and will gain +1d6 if he is fighting you or your allies. On a failed save, he is fascinated by you until the spell ends or until you or your allies do something harmful to him. The charmed creature treats you as a friendly acquaintance. When the spell ends, the creature is aware that it has been charmed by you. Whenever the creature is threatened by you or a friend of yours, it can reroll its Saving Throw with a +2 bonus.\\
\textbf{For each magical critical success rolled} in Magic Test you may add one creature as a target. When you cast the spell, the target creatures must be within 10 meters of each other.


\medskip\textbf{Chill Touch}\index{Cantrip - Chill Touch}\\
\textbf{School}: Necromancy\\
\textbf{Level}: 0, Common\\
\textbf{Casting Time}: 1 Action\\
\textbf{Range}: 36 meters\\
\textbf{Components}: V, S\\
\textbf{Duration}: 1 round\\
You create a skeletal spectral hand in a creature's space within range. Make a ranged spell attack against the creature, pelting it with the chill of death. On a hit, the target takes 1d8 void damage, and can't regain Hit Points until the start of your next round. Until then, the hand will remain locked on the target. If you hit an undead target, it will also have -1d6 on attack rolls against you until the end of its next round.\\
The spell's damage increases by 1d8 when you reach MP 5, MP 11 and MP 17, but it costs 2 Actions to cast it empowered and 2 Spell Points, you must also have taken Adept of Magic in this Spell List a number of times equal to the empowerments that you want to apply.\\
\textbf{For every two magical critical successes you roll} in the Magic Test you create an additional skeletal hand that must attack a different creature within range.


\medskip\textbf{Circle of Death}\index[Spells]{Circle of Death}\\
\textbf{School}: Invocation\\
\textbf{Level}: 6, Very Rare\\
\textbf{Cast Time}: 2 Actions\\
\textbf{Range}: 45 meters\\
\textbf{Components}: V, S, M (a powdered black pearl worth at least 500 gp)\\
\textbf{Duration}: Instant\\
A 20m radius sphere of negative energy erupts at a point within range. Each creature in that area must make a Fortitude save. A target takes 8d6 void damage on a failed save, or half as much damage on a successful one. \\
\textbf{For each magical critical success rolled} in the Magic Test the damage increases by 4d6.\\
\textbf{Save Critical Success/Failure}: On a critical failure the damage is doubled, on a critical success the damage is further halved

\medskip\textbf{Clairvoyance}\index[Spells]{Clairvoyance}\\
\textbf{School}: Divination\\
\textbf{Level}: 3, Common\\
\textbf{Cast Time}: 10 minutes\\
\textbf{Range}: 1.5 kilometers\\
\textbf{Component}: V, S, M (a focus worth at least 100 gp, whether a jeweled horn to hear or a glass eye to see)\\
\textbf{Duration}: Concentration, max 10 minutes\\
You create an invisible sensor in a place that is familiar to you and is within range (a place you have visited or seen before) or in a place that is obvious but unfamiliar to you (such as behind a door or corner, or in the middle of a grove of trees). The sensor remains in place for the duration, and cannot be attacked or otherwise interacted with. When you cast this spell, you choose to see or hear. You can use the sense chosen by the sensor, as if you were in its space. With two actions, you can switch between hearing and hearing. A creature that can see the sensor (a creature with see invisibility or true seeing) perceives it as a luminous, intangible orb the size of your fist.\\
\textbf{For each magical critical success rolled} in the Magic Test the duration increases by 10 minutes or the range increases by 500m.

\medskip\textbf{Clone}\index[Spells]{Clone}\\
\textbf{School}: Necromancy\\
\textbf{Level}: 8, Uncommon\\
\textbf{Range}: Contact
\textbf{Components}: V, S, M (a diamond worth at least 1000 gp and at least 16 cubic centimeters of flesh from the creature to be cloned, which the spell consumes, and a vessel worth at least 2000 gp that has a sealable lid and is large enough to hold a Medium creature, such as a large urn, coffin, mud-filled pit in the ground, or crystal container filled with salt water) \\
\textbf{Duration}: Instant\\
This spell produces an inert duplicate of a living creature as a safeguard against death. This clone is formed inside a sealed container and reaches its maximum size and maturity after 120 days; you can also decide that the clone is a younger version of the same creature. It remains inert and survives indefinitely, as long as the container remains undisturbed.\\
At any time after the clone matures, if the original creature dies, its soul transfers into the clone, provided the soul is free and willing to return. The clone is physically identical to the original and has the same personality, memories and characteristics, but none of the equipment of the original. The physical remains of the original creature, if they still exist, become inert and cannot be brought back to life, since the creature's soul is elsewhere. \\
\textbf{This spell is not selectable if Patrons are active}

\medskip\textbf{Cloud of Mist}\index[Spells]{Cloud of Mist}\\
\textbf{School}: Water, Air\\
\textbf{Level}: 1, Common\\
\textbf{Cast Time}: 2 Actions\\
\textbf{Range}: 36 meters\\
\textbf{Components}: V, S\\
\textbf{Duration}: 1 hour\\
You create a 6 meters-radius sphere of mist centered on a point within range. The sphere propagates around corners, and its area is in dim light. It remains for the duration of the spell or until a wind of moderate speed or higher (at least 15 kilometers per hour) dispels it.\\
\textbf{For each magical critical success rolled} in the Magic Test the radius of the mist increases by 6 meters.

\medskip\textbf{Color Spray}\index[Spells]{Color Spray}\\
\textbf{School}: Illusion\\
\textbf{Level}: 1, Common\\
\textbf{Cast Time}: 2 Actions\\
\textbf{Range}: Self (4-meter cone)\\
\textbf{Components}: V, S, M (a pinch of dust or sand that is colored red, yellow and blue)\\
\textbf{Duration}: 1 round\\
A burst of dazzling, colorful light shoots from your hand. Roll 6d10; the total is the amount of Hit Points of creatures this spell affects. Creatures, in a 5m cone originating from you, are affected in ascending order of their current Hit Points (ignoring unconscious creatures and creatures that can't see).\\
Starting with the creature that has the fewest current Hit Points, each creature affected by this spell is blinded until the spell ends. Subtract each creature's Hit Points from the total before moving on to the creature with the next lowest hit point total. A creature's Hit Points must be equal to or less than the remaining total for the spell to affect it. \\
\textbf{For each magical critical success rolled} in the Magic Test, roll an additional 2d10 Hit Points.

\medskip\textbf{Command}\index[Spells]{Command}\\
\textbf{School}: Enchantment\\
\textbf{Level}: 1, Common\\
\textbf{Casting Time}: 2 Actions\\
\textbf{Range}: 18 meters\\
\textbf{Components}: V,s\\
\textbf{Duration}: 1 round\\
You speak a one-word command to a creature that you can see within range and a gesture. The target must succeed at a Will save or execute the command within its next round. The spell has no effect if the target is undead, if he doesn't understand your language, or if your command would harm him. Here are some typical commands and their effects. You can issue commands other than those described here, in which case the Arbiter will determine the target's behavior. If the target cannot carry out your command, the spell ends.\\

- \textit{Get closer}. The target moves towards you by the shortest and most direct route, ending its round if it comes within 1 meter of you.\\

- \textit{Stop}. The target does not move and then ends its round. A flying creature stays in place as long as possible. If it has to move to stay in the air, it flies the minimum distance necessary to do so.\\

- \textit{Throw away}. The target throws whatever it is holding and then ends its round.\\

- \textit{Run away}. The target spends its round moving away from you by the fastest means at its disposal.\\

- \textit{Stripe}. The target falls prone and then ends its round.\\

\textbf{For each magical Critical Success you roll} in the Magic Test you can act on an additional creature. At the time you cast the spell, the target creatures must be within 10 m of each other and execute the same command.


\medskip\textbf{Communion}\index[Spells]{Communion}\\
\textbf{School}: Divination\\
\textbf{Level}: 5, Rare\\
\textbf{Cast Time}: 1 minute\\
\textbf{Range}: Personal\\
\textbf{Components}: V, S, M (incense and a vial of holy or blasphemous water)\\
\textbf{Duration}: 1 minute\\
You communicate with your Patron and ask him up to three questions that can be answered with yes or no. You must ask the questions before the spell ends. You will get the correct answer to each question. Divine creatures aren't necessarily omniscient, so you may get "it's not clear" in response to a question involving information not pertinent to the Patron's knowledge. In case a one word answer could be misleading or contrary to the interests of the Patron, the Arbiter could instead give a short sentence as an answer.\\
If you cast the spell two or more times before the new dawn has risen there is a cumulative 25\% chance that for each cast after the first you get no answer. The Arbiter makes this roll secretly.\\
\textbf{NOTE:} you must be at least a Follower to cast this spell.


\medskip\textbf{Communion with Nature}\index[Spells]{Communion with Nature}\\
\textbf{School}: Divination\\
\textbf{Level}: 5, Very Rare\\
\textbf{Cast Time}: 1 minute\\
\textbf{Range}: Personal\\
\textbf{Components}: V, S\\
\textbf{Duration}: Instant\\
For an instant you become one with nature and gain insight into the surrounding area. In outdoor environments, the spell gives you information about the area within 5 kilometers of you. In caves and other underground natural environments, the range is limited to 100 meters. The spell does not work in places where nature has been replaced by buildings, such as dungeons and villages.\\.
Instantly learn about up to three topics of your choice on one of the following subjects, related to the area:

- land and bodies of water\\
- plants, minerals, animals and prevalent populations\\
- powerful celestials, elementals, fey, demons or undead\\
- influences from other planes of existence\\
- buildings\\
\textbf{For each magical Critical Success achieved} in the Magic Test you learn an additional topic.\\
\textbf{NOTE}: If you are a Devotee of Ephrem you are considered to have made one additional Magical Critical Success.



\medskip\textbf{Compulsion}\index[Spells]{Compulsion}\\
\textbf{School}: Enchantment\\
\textbf{Level}: 4, Uncommon\\
\textbf{Casting Time}: 2 Actions\\
\textbf{Range}: 9 meters\\
\textbf{Components}: V, S\\
\textbf{Duration}: Concentration, max 1 minute\\
Creatures of your choice within range that you can see and can hear you must make a Will save. A target automatically succeeds at its Saving Throw if it cannot be charmed. Until the spell ends, you can use an Action during each of your rounds to indicate a direction horizontal to you. Each affected target must use as much of its movement as possible during its next round to move in that direction. The target cannot take any actions before moving. After moving in this manner, the target can make another Will save to attempt to end the effect. \\
A target can't be forced to move into an obviously lethal hazard, such as a flame or pit.

\medskip\textbf{Conceal}\index[Spells]{Conceal}\\
\textbf{School}: Transmutation\\
\textbf{Level}: 7, Rare\\
\textbf{Cast Time}: 2 Actions\\
\textbf{Range}: Contact\\
\textbf{Components}: V, S, M (a powder composed of diamond, emerald, ruby, and sapphire dust worth at least 50,000 gp, which the spell consumes)\\
\textbf{Duration}: Until dispelled \\
Through this spell, a willing creature or object can be hidden, undetectable for the duration. By casting this spell and coming into contact with a target, they become invisible and cannot be targeted by divination spells, nor sensed by scrying sensors created by divination spells.\\
If the target is a creature, it enters a state of suspended animation. For him time ceases to flow, and does not age. \\
You can set a condition for the spell to end prematurely. The condition can be anything you like, but it must occur or be visible within 1.5 kilometers of the target. Examples include "at the next judgment of the Patrons" or "when the tarrasque awakens". This spell also ends if the target takes damage.

\medskip\textbf{Cone of Cold}\index[Spells]{Cone of Cold}\\
\textbf{School}: Water\\
\textbf{Level}: 5, Common\\
\textbf{Cast Time}: 2 Actions\\
\textbf{Range}: Self (20m cone)\\
\textbf{Components}: V, S, M (a small crystal or glass cone)\\
\textbf{Duration}: Instant\\
A blast of cold air erupts from your hands. Each creature in a 20m cone must make a Fortitude save. A creature takes 8d8 cold damage on a failed save, or half as much damage on a successful one. A creature slain by this spell becomes an ice statue until it thaws.\\
\textbf{For each magical critical success rolled} in the Magic Test the damage increases by 4d8\\
\textbf{Save Critical Success/Failure}: On a critical failure the damage is doubled, on a critical success the damage is further halved

\medskip\textbf{Confusion}\index[Spells]{Confusion}\\\hypertarget{incconfusione}{}\label{incconfusione}
\textbf{School}: Enchantment\\
\textbf{Level}: 4, Common\\
\textbf{Casting Time}: 2 Actions\\
\textbf{Range}: 27 meters\\
\textbf{Components}: V, S, M (three nutshells)\\
\textbf{Duration}: 1 minute\\
This spell assaults and bends the mind of creatures, generating illusions and causing uncontrolled actions. When you cast this spell, each creature in a 3m-radius sphere centered on a point you choose within range must succeed at or be affected by a Will save. An affected target cannot take reactions and must roll a d10 at the start of each of its rounds to determine its behavior for that round.

\medskip

\begin{tabularx}{0.45\textwidth}{lX}
\hline
d10 & Behavior\\
1 & The creature uses all its Actions to move in a random direction. To determine direction, roll a d8 to determine direction\\
2-5 & The creature does nothing for the entire round\\
6 & The creature makes an attack against itself and ends the round\\
7-8 & The creature makes an attack against a randomly determined creature within its reach. If she was hit the previous round she will attack the opponent who hit her. If there is no creature within reach, the creature will do nothing this round. Once the attack is made, the round ends.\\
9-10 & The creature can act and move normally.\\
\end{tabularx}

\medskip

At the end of each of its rounds, an affected target can make a Will save. If he succeeds, the effect ends for him. \\
\textbf{For each magical critical success obtained} in the Magic Test, the sphere's radius increases by 1 meter.


\medskip\textbf{Summon Lesser Elementals}\index[Spells]{Summon Lesser Elementals}\\
\textbf{School}: Air, Water, Earth, Fire\\
\textbf{Level}: 4, Uncommon\\
\textbf{Cast Time}: 1 minute\\
\textbf{Range}: 27 meters\\
\textbf{Components}: V, S\\
\textbf{Duration}: 1 hour\\
You summon elementals that will appear in unoccupied spaces within range that you can see. Choose one of the following options to decide what appears:\\

- An elemental of challenge rating 2 or lower\\
- Two elementals of challenge rating 1 or lower\\
- Four elementals of challenge rating 1/2 or lower\\
- Eight elementals of challenge rating 1/4 or lower\\

\medskip
A summoned elemental disappears when it drops to 0 Hit Points or the spell ends. A summoned elemental is friendly to you and your companions. Roll initiative for the summoned elementals as a group, which acts on its own round. They obey any verbal command given to them (if the command is complex it consumes actions). If you don't give commands to the elementals, they will defend themselves from hostile creatures, but take no other actions.\\
Each Spell List can only summon its specific Elemental \\
\textbf{For each magical Critical Success rolled} two more Elementals will appear in the Magic Test.

\medskip\textbf{Contagion}\index[Spells]{Contagion}\\
\textbf{School}: Necromancy\\
\textbf{Level}: 5, Uncommon\\
\textbf{Cast Time}: 2 Actions\\
\textbf{Range}: Contact\\
\textbf{Components}: V, S\\
\textbf{Duration}: 7 days\\
Through touch you can inflict diseases. Make a melee attack against a creature within reach. On a hit, you infect the creature with a disease of your choice from those described below. At the end of each round of the target, it must make a Fortitude save. After failing three of these Saving Throws, the effects of the disease last for the duration, and the creature no longer makes Saving Throws. After successful three such Saving Throws, the creature recovers from the disease, and the spell ends. \\
Since this spell induces a natural disease in its target, any effects that remove diseases or improve the effects of diseases apply to it.\\
- \textit{Rotten Meat}. The creature's skin rots. The creature has -1d6 on Charisma checks, and any damage is doubled.\\
- \textit{Blinding Weakness}. Pain grips the creature's mind as its eyes turn milky white. The creature has -1d6 on Wisdom checks and Will saves, and is blinded.\\
- \textit{Filthy Fever}. A devastating fever ravages the creature's body. The creature has -1d6 on Strength checks and Fortitude saves, and on attack rolls that use the Strenght. \\
- \textit{Stinging}. The creature is overcome with tremors. The creature has -1d6 on Dexterity checks, Reflex saves, and attack rolls that use Dexterity.\\
- \textit{Mindfire}. The creature's mind is in a fever. The creature has -1d6 on Intelligence checks and Will saves, and behaves as if under the effect of the confusion spell in combat.\\
- \textit{Slimy Death}. The creature begins to bleed incessantly. The creature has -1d6 on Constitution checks and Fortitude saves. Additionally, whenever the creature takes damage, it is stunned until the end of its next round.

\medskip\textbf{Contagious Confusion}\index[Spells]{Contagious Confusion}\\
\textbf{School}: Enchantment\\
\textbf{Level}: 8, Very rare\\
\textbf{Casting Time}: 10 minutes\\
\textbf{Range}: Contact\\
\textbf{Components}: V, S, M (tooth powder)\\
\textbf{Duration}: 1 minute\\
This spell assaults and bends the mind of creatures, generating illusions and causing uncontrolled actions. Once this spell is cast, you then have one minute to touch the first creature. This creature can make a Will saving throw to negate the effects.

Any creature touched by the first creature transmit the confusion spell, with a saving throw as the first creature, the confusion effect on new creatures last one minute.

If the caster does not touch a creature within one minute then he himself will be subject to the confusion spell, with no saving throw possible.

\medskip\textbf{Contingency}\index[Spells]{Contingency}\\
\textbf{School}: Invocation\\
\textbf{Level}: 6, Uncommon\\
\textbf{Cast Time}: 10 minutes\\
\textbf{Range}: Personal\\
\textbf{Components}: V, S, M (a statuette of yourself carved from ivory and decorated with gems worth at least 1,500 gp)\\
\textbf{Duration}: 10 days\\
Choose a Level 4 or lower spell that you can cast, that has a casting time of 2 Actions, and that can target you. You cast that spell (called a contingent spell) as part of contingency casting, expending the spell slots of both, but without the contingent spell taking effect. Instead, it will take effect when a certain circumstance occurs. Describe this circumstance as you cast the two spells. For example, contingency cast in conjunction with water breathing might stipulate that water breathing kicks in when you are submerged in water or similar liquid.\\
The contingent spell takes effect immediately after the circumstance first occurs, whether you mean it or not, and then the contingency ends. The contingent spell affects only you, though it can normally target others as well. You can use only one contingency spell at a time. If you cast this spell again, the effect of another contingency spell on you will end. Furthermore, contingency for you ends if the material component is no longer on your person.\\
\textbf{For each magical critical success rolled} in the Magic Test the contingency lasts 10 days longer.

\medskip\textbf{Control Water}\index[Spells]{Control Water}\\
\textbf{School}: Water\\
\textbf{Level}: 4, Common\\
\textbf{Casting Time}: 2 Actions\\
\textbf{Range}: 90 meters\\
\textbf{Components}: V, S, M (a drop of water and a pinch of powder)\\
\textbf{Duration}: Concentration, max 10 minutes\\
Until the spell ends, you control any free water within the area you chose up to a 30m cube. When you cast this spell, you can choose any of the following effects. As an action, during your round, you can repeat the same effect or choose a different one.\\

\medskip\textbf{Control Weather}\index[Spells]{Control Weather}\\
\textbf{School}: Water, Air\\
\textbf{Level}: 8, Very Rare\\
\textbf{Cast Time}: 10 minutes\\
\textbf{Range}: Personal (1.5 kilometer radius)\\
\textbf{Components}: V, S, M (burned incense and some earth and wood mixed in water)\\
\textbf{Duration}: Concentration, max 8 hours \\
For the duration, take control of the weather within 7.5 kilometers of you. You must be outside to cast this spell. Moving to a place where you don't have an open view of the sky ends the spell early. When you cast this spell, it changes the current climatic conditions, determined by the Arbiter based on the season and latitude. You can change the precipitation, temperature and wind. It takes 1d4 x 10 minutes for the new condition to take effect. Once the condition takes effect, you can change it again. When the spell ends, the weather will gradually return to normal.\\
When changing weather conditions, find the current condition on the following table and change it one step up or down. When you change the wind, you can change its direction as well.


\medskip

\textit{Precipitation}

- 1 Limpid

- 2 A few clouds

- 3 Overcast or mist on the ground

- 4 Rain, hail or snow

- 5 Torrential Rain, Heavy Hailstorm, Blizzard\\

\textit{Temperature}

- 1 unbearable heat

- 2 Hot

-3 luke warm

- 4 fresh

- 5 cold

- 6 Polar Cold\\

\textit{Wind}

- 1 calm

- 2 Moderate wind

- 3 Moderate wind

- 4 Fortunate

- 5 Storm\\

\textbf{For each magical critical success rolled} in the Magic Test the duration increases by 8 hours.

\medskip\textbf{Counterspell}\index[Spells]{Counterspell}\\
\textbf{School}: Abjuration\\
\textbf{Level}: 3, Common\\
\textbf{Casting Time}: 1 Reaction, which you take when you see a creature/object within 20 meters manifest an enchantment\\
\textbf{Range}: 18 meters\\
\textbf{Components}: S \\
\textbf{Duration}: Instant\\
You use a Reaction Action to make a DC 13 Arcana check. If the check succeeds, you understand if you can counter the spell's effect with Counterspell. The undone spell must be Level 2 or lower, regardless of whether it is manifested by a spellcaster or item. Each Magic Critical success or buff rolled from the original spell raises the spell's level by 1.\\
\textbf{For every two magical critical successes you roll} in Magic Test you can undo a spell of a higher level.

\medskip\textbf{Create Ale}\index[Spells]{Create Ale}\\
\textbf{School}: Summon\\
\textbf{Level}: 0, Rare\\
\textbf{Casting Time}: 2 Actions or more\\
\textbf{Range}: 9 meters\\
\textbf{Components}: V, S, M (brewer's yeast, malt, water)\\
\textbf{Duration}: 1 hour\\
You create 1 liter of beer. The quality and type of beer depends on the yeast, malt and water used.
The greater the casting time of the spell, the higher the alcohol content, with a casting time of two actions the alcohol content is 4.3, if 1 action is non-alcoholic, each spent action increases the alcohol content by 0.3 vol up to a maximum of 12.5 vol.
After an hour the beer vanishes, when consumed after an hour any alcoholic effects of the same on the people who drank it also end.\\
\textbf{For each magical critical success rolled} in the Magic Test, the duration increases by one hour or produce one more liter.

\medskip\textbf{Create Food and Water}\index[Spells]{Create Food and Water}\\
\textbf{School}: Summon\\
\textbf{Level}: 3, Common\\
\textbf{Casting Time}: 2 Actions\\
\textbf{Range}: 9 meters\\
\textbf{Components}: V, S\\
\textbf{Duration}: Instant\\
You create food and water in containers within range, enough to sustain up to five humanoids or 2 mounts for 24 hours. The food is bland but nutritious, and it rots after 24 hours if not consumed, as does the water.\\
\textbf{For each magical critical success you roll} in the Magic Test you create food for 3 other people or 1 mount.

\medskip\textbf{Create Undead}\index[Spells]{Create Undead}\\
\textbf{School}: Necromancy\\
\textbf{Level}: 6, Uncommon\\
\textbf{Casting Time}: 2 Actions\\
\textbf{Range}: 3 meters\\
\textbf{Components}: V, S, M (an earthenware pot filled with graveyard soil, an earthenware pot filled with brackish water, and a black onyx worth 50 gp for each corpse)\\
\textbf{Duration}: Instant\\
You can only cast this spell at night. Choose up to three Medium or Small humanoid corpses within range. Each corpse becomes a ghoul under your control (the Arbiter owns the game statistics of these creatures). During your round, with two Actions, you can mentally command any creature you animate with this spell, if the creature is within 36 meters of you (if you control multiple creatures, you can command all or just one of them at the same time by issuing the same command). You decide what action the creature will take and where it will move during its next round, or you can issue a general command, such as to guard a specific room or corridor. If you issue no commands, the creatures will simply defend themselves against hostile creatures. Once given a command, the creature will continue to execute it until the task is complete. The creature is under your control for 24 hours, after which it will stop responding to commands you give it. To maintain control of the creature for another 24 hours, you must cast this spell on the creature before the current 24-hour period ends. This use of the spell reasserts your control over up to three creatures you have animated with this spell, rather than animating new ones.\\.
\textbf{If you roll a Critical} in Magic Test you can revive or reassert control over four ghouls. With two Crits you can animate or reassert control over five
ghouls or two ghasts or wights. With three Crits you can animate or reassert control over six ghouls, three ghasts or wights, or two mummies.

\medskip\textbf{Create or Destroy Water}\index[Spells]{Create or Destroy Water}\\
\textbf{School}: Water\\
\textbf{Level}: 1, Common\\
\textbf{Casting Time}: 2 Actions\\
\textbf{Range}: 9 meters\\
\textbf{Components}: V, S, M (a drop of water to create water or a few grains of salt to destroy it)\\
\textbf{Duration}: Instant\\
You create or destroy water.\\
\textit{Create Water}. You create up to 40 liters of clear water from your hands that spray up to 9 metres. Alternatively, the water falls as rain into a 10m cube that is within range, extinguishing any flames exposed in the area.\\
The spell cannot be used on magical flames.\\
\textit{Destroy Water}. Destroy up to 40 liters of water in an open container within range. Alternatively, you can destroy the fog into a 10m cube within range.\\
\textbf{For each magical critical success rolled} in the Magic Test you create or destroy an additional 40 liters of water, or the cube increases in size by 1 meter of edge in fog.\\
The water is drinkable and quenches thirst if drunk within one round of creation.

\medskip\textbf{Creation}\index[Spells]{Creation}\\
\textbf{School}: Illusion\\
\textbf{Level}: 5, Rare\\
\textbf{Cast Time}: 1 minute\\
\textbf{Range}: 9 meters\\
\textbf{Components}: V, S, M (a tiny piece of material of the same type of item you intend to create) \\
\textbf{Duration}: Special\\
You grab pieces of shadow matter from the plane of Shadows to create nonliving plant matter objects within range: soft goods, rope, wood, or the like. You can also use this spell to create mineral items such as stone, crystal, or metal. The crafted object cannot be larger than a 1m cube, and the object must be of a shape and material you've seen before.\\
The duration depends on the material of the object. If the item is made from multiple materials, use the shortest duration.
\medskip
Material Table - Duration
\medskip


\begin{tabularx}{0.45\textwidth}{lX}
\hline
Plant matter &1 day\\
Stone or crystal &12 hours\\
Precious Metals &1 hour\\
Gems &10 Minutes\\
Adamantium or mithral &1 minute\\
\end{tabularx}
\medskip

Using any material created by this spell as a material component of another spell will cause the new spell to fail. \\
\textbf{For each magical critical success rolled} in the Magic Test, the cube increases by 1 meter of edge.


\medskip\textbf{Cruel Hoax}\index{Cantrip - Cruel Hoax}\\
\textbf{School}: Enchantment\\
\textbf{Level}: 0, Common\\
\textbf{Casting Time}: 1 Action\\
\textbf{Range}: 18 meters\\
\textbf{Components}: V\\
\textbf{Duration}: Instant\\
You unleash a series of taunts wrapped in a subtle charm against a creature that you can see within range. If the target can hear you (although it need not understand you), it must succeed at a Will save or take 1d4 points of damage and be -1d6 on the next attack roll it makes before the end of its next round. \\
The spell's damage increases by 1d4 when you reach MP 5, MP 11 and MP 17, but it costs 2 Actions to cast it empowered and 2 Spell Points, you must also have taken Adept of Magic in this Spell List a number of times equal to the empowerments that you want to apply.\\
\textbf{Every 2 magical critical successes rolled} in Magic Test affects another creature.


\medskip\textbf{Cure Critical Wounds}\index[Spells]{Cure Critical Wounds}\\
\textbf{School}: Heal\\
\textbf{Level}: 5, Uncommon\\
\textbf{Casting Time}: 2 Actions\\
\textbf{Range}: Contact\\
\textbf{Components}: V, S\\
\textbf{Duration}: Instant\\
Your hand fills with positive healing energy, a creature you touch regains a number of Hit Points equal to 5d8 + 3* Characteristic modifier for spells. This spell when used on an undead, attack roll with Touch Attack, damages it by the same amount. \\
Unless otherwise stated, this spell cannot be used on animals or plants. \\
\textbf{For each magical critical success rolled} in the Magic Test you heal an additional 1d6 Hit Points.\\
If the caster and the creature healed are both Followers of the same Patron, the spell heals an additional 1d8. \\
If the spellcaster and the cured creature are both Devotees of the same Patron, each value on the dice equal to 1,2,3 will be considered 4.

\medskip\textbf{Cure Light Wounds}\index[Spells]{Cure Light Wounds}\\
\textbf{School}: Water, Heal\\
\textbf{Level}: 1, Common\\
\textbf{Casting Time}: 2 Actions\\
\textbf{Range}: Contact\\
\textbf{Components}: V, S\\
\textbf{Duration}: Instant\\
Your hand fills with positive healing energy, a creature you touch regains a number of Hit Points equal to 1d8 + spell Characteristic modifier. This spell when used on an undead, attack roll with touch Attack, damages it by the same amount. \\
Unless otherwise stated, this spell cannot be used on animals or plants. \\
\textbf{For each magical critical success rolled} in the Magic Test you heal an additional 1d6 Hit Points.\\
If the caster and the creature healed are both Followers of the same Patron, the spell heals an additional 1d8. \\
If the spellcaster and the cured creature are both Devotees of the same Patron, each value on the dice equal to 1,2,3 will be considered 4.

\medskip\textbf{Cure Serious Wounds}\index[Spells]{Cure Serious Wounds}\\
\textbf{School}: Heal\\
\textbf{Level}: 3, Uncommon\\
\textbf{Cast Time}: 2 Actions\\
\textbf{Range}: Contact\\
\textbf{Components}: V, S\\
\textbf{Duration}: Instant\\
Your hand fills with positive healing energy, a creature you touch regains a number of Hit Points equal to 3d8 + 2* Characteristic modifier for spells. This spell when used on an undead, attack roll with touch Attack, damages it by the same amount. \\
Unless otherwise stated, this spell cannot be used on animals or plants. \\
\textbf{For each magical critical success rolled} in the Magic Test you heal an additional 1d6 Hit Points.\\
If the caster and the creature healed are both Followers of the same Patron, the spell heals an additional 1d8. \\
If the spellcaster and the cured creature are both Devotees of the same Patron, each value on the dice equal to 1,2,3 will be considered 4.

\medskip\textbf{Dancing Lights}\index{Cantrip - Dancing Lights}\\
\textbf{School}: Invocation\\
\textbf{Level}: 1, Uncommon\\
\textbf{Cast Time}: 2 Actions\\
\textbf{Range}: 36 meters\\
\textbf{Components}: V, S, M (a piece of phosphorus or enchanted wood, or an earthworm)\\
\textbf{Duration}: 10 minute of real game time\\
You create up to four torch-sized lights within range, causing them to appear as torches, lanterns, or orbs of light that float in the air for the spell's duration. You can also combine the four lights into one Medium-sized, vaguely humanoid glowing form. Whichever form you choose, each light emits a dim light in a 3m radius. As a 1 move action during your round, you can move the lights up to 20 meters to a new point within range.\\
A light must be within 6 meters of another light created with this spell, and the lights vanish if they exceed the spell's range.\\
\textbf{For each Critical gained} in the Magic Test the duration increases by 1 hour.

\medskip\textbf{Darkness}\index[Spells]{Darkness}\\
\textbf{School}: Invocation\\
\textbf{Level}: 1, Common\\
\textbf{Cast Time}: 2 Actions\\
\textbf{Range}: 18 meters\\
\textbf{Components}: V, M (bat hair and a pinch of bitumen or a piece of coal)\\
\textbf{Duration}: 10 minutes\\
The magical darkness spreads from a point of your choice within range to fill a 3m-radius sphere for the spell's duration. Darkness spreads around corners. A creature with darkvision can't see in this darkness, and nonmagical light can't illuminate it.\\
If the spot you chose is on an item you are carrying or one that is not being worn or carried, the darkness emanates from the item and moves with it. Completely covering the source of the darkness with an opaque object, such as a pot or helmet, blocks the darkness.\\
If any part of this spell's area overlaps with the area of light created by a spell of level 2 or lower, the spell that created the light is dispelled.

\medskip\textbf{Darkvision}\index[Spells]{Darkvision}\\
\textbf{School}: Transmutation\\
\textbf{Level}: 2, Common\\
\textbf{Casting Time}: 2 Actions\\
\textbf{Range}: Contact\\
\textbf{Components}: V, S, M (or a pinch of carrot or blueberry dry)\\
\textbf{Duration}: 1 hour of real game time\\
A willing creature you touch gains the ability to see in the dark. For the duration, that creature has darkvision out to a range of 10 meters.\\
\textbf{For each Magical Critical Success rolled} in Magic Test you double the duration.

\medskip\textbf{Daylight}\index[Spells]{Daylight}\\
\textbf{School}: Invocation\\
\textbf{Level}: 3, Common\\
\textbf{Casting Time}: 2 Actions\\
\textbf{Range}: 18 meters\\
\textbf{Components}: V, S\\
\textbf{Duration}: 1 hour of real game time\\
A sphere of light with a 6 meters radius expands from a point of your choice within range. The sphere radiates bright light and dim light for an additional 12 meters. If you pick a spot on an object that you are holding or that is not being worn or carried, light radiates from the object and moves with it. Covering an object completely with something opaque, such as a vase or helmet, blocks light. If any part of this spell's area overlaps with the area of darkness created by a 3-level or lower spell, the spell that created the darkness is dispelled. The light created is considered sunlight.\\
\textbf{NOTE}: Devoted to Ljust or Sumkjr got +1 to Saving Throw while illuminated by this light.

\medskip\textbf{Deadly Mist}\index[Spells]{Deadly Mist}\\
\textbf{School}: Water, Air\\
\textbf{Level}: 5, Rare\\
\textbf{Cast Time}: 2 Actions\\
\textbf{Range}: 36 meters\\
\textbf{Components}: V, S\\
\textbf{Duration}: 10 minutes \\
You create a 6 meters-radius sphere of poisonous yellow-green mist centered on a point of your choice within range. Fog rolls around corners. It remains for the duration of the spell or until a strong wind clears the mist, ending the spell. Its area is in dim light. When a creature enters the spell's area for the first time in a round or begins its round there, that creature must make a Fortitude save. The creature takes 5d8 poison damage on a failed save, or half as much damage on a successful one. Creatures are affected even if they hold their breath or have no need to breathe. The mist moves 3 meter away from you at the start of each of your rounds, moving along the surface of the ground. The vapours, being heavier than air, tend to descend downwards, even reaching the point of creeping into the openings.\\
\textbf{For each magical critical success rolled} in the Magic Test the damage increases by 2d8.

\medskip\textbf{Death Ward}\index[Spells]{Death Ward}\\
\textbf{School}: Abjuration\\
\textbf{Level}: 4, Uncommon\\
\textbf{Cast Time}: 2 Actions\\
\textbf{Range}: Contact\\
\textbf{Components}: V, S\\
\textbf{Duration}: 8 hours\\
You cast the spell while touching a creature. Grant the target protection from death. The first time the target drops to 0 Hit Points as a result of damage taken, the target drops to 1 hit point instead and the spell ends. If the spell is still active when the target is the victim of an effect that would kill them instantly without inflicting damage, that effect is negated on the target instead and the spell ends. \\
\textbf{For every two magical Critical Successes rolled} in the Magic Test the spell protects one more time.

\medskip\textbf{Delayed Fireball}\index[Spells]{Delayed Fireball}\\
\textbf{School}: Fire\\
\textbf{Level}: 7, Rare\\
\textbf{Casting Time}: 2 Actions\\
\textbf{Range}: 45 meters\\
\textbf{Components}: V, S, M (a large ball of bat guano and sulfur)\\
\textbf{Duration}: Concentration, 1 minute\\
A beam of yellow light shoots from your pointing finger, condensing for the duration of the spell into the form of a luminous ball at a point you choose within range. When the spell ends, either because your concentration is broken or because you decide to end it, the ball explodes with a soft roar and turns into a jet of flame that spreads around corners. Each creature in a 6 meters-radius sphere centered on that point must make a Reflex save. A creature takes fire damage equal to the total accumulated damage on a failed save, or half as much damage on a successful one. The spell's base damage is 12d6. If the ball has not yet detonated at the end of your round, the damage increases by 1d6. \\
If the glowing ball is touched before the spell ends, the creature touching it must make a Reflex save. On a failed save, the spell immediately ends, causing the ball to erupt flames. On a successful save, the creature can throw the ball up to 13 meters away. When it hits a creature or solid object, the spell ends and the ball explodes. \\
Fire damages objects in the area and ignites flammable objects that are not worn or carried.\\
\textbf{For each magical critical success rolled} in the Magic Test the damage increases by 1d6.\\
\textbf{Save Critical Success/Failure}: On a critical failure the damage is doubled, on a critical success the damage is further halved.

\medskip\textbf{Demiplane}\index[Spells]{Demiplane}\\
\textbf{School}: Summon\\
\textbf{Level}: 8, Rare\\
\textbf{Casting Time}: 2 Actions\\
\textbf{Range}: 18 meters\\
\textbf{Components}: S\\
\textbf{Duration}: 1 hour\\
You create a shadow door on a flat surface that you can see within range and that you can see. The door is large enough for a Medium creature to pass through without difficulty. When opened, the door leads to a demiplane that appears as an empty room 10 meters in each dimension made of wood and stone. When the spell ends, the door disappears, and any creatures or objects inside the demiplane are trapped there, while the door also disappears on the other side. \\
Each time you cast this spell, you create a new demiplane, or allow the shadow door to connect to a demiplane created by a previous casting of the spell, or raise a previously known demiplane you created another 10 meters in each dimension. \\
Additionally, if you know the nature and contents of a demiplane created by another creature's casting of this spell, you can cause the shadow door to connect to that demiplane instead.

\medskip\textbf{Destroy undead}\index[Spells]{Destroy undead}\\
\textbf{School}: Care\\
\textbf{Level}: 3, Uncommon\\
\textbf{Casting Time}: 2 Actions\\
\textbf{Range}: 36 meters\\
\textbf{Components}: V, S, M (a relic of a Thaft or Sumkjt Devotee)\\
\textbf{Duration}: Instant\\
Pick one undead within 36 meters. A beam of light shoots from your hand and envelops the creature. The undead makes a Fortitude saving throw to halve 4d12 points of positive energy damage. \\
\textbf{For each magical critical success rolled} in the Magic Test the damage increases by 1d12.

\medskip\textbf{Detect Diseases and Poisons}\index[Spells]{Detect Diseases and Poisons}\\
\textbf{School}: Divination\\
\textbf{Level}: 1, Uncommon\\
\textbf{Cast Time}: 2 Actions\\
\textbf{Range}: Personal\\
\textbf{Components}: V, S, M (a yew leaf)\\
\textbf{Duration}: 1 round per Magic Proficiency\\
For the duration, you sense the presence and location of poisons, poisonous creatures, and diseases within 10 meters of you. Also you can identify the type of poison, poisonous creature or disease. The spell can penetrate most barriers, but is blocked by 30cm of stone, 1 cm of base metal, a thin sheet of lead, or 1 meter of wood or earth.\\
\textbf{For each magical critical success rolled} in the Magic Test duration doubles.

\medskip\textbf{Detect Good and Evil}\index[Spells]{Detect Good and Evil}\\
\textbf{School}: Divination\\
\textbf{Level}: 1, Common\\
\textbf{Cast Time}: 2 Actions\\
\textbf{Range}: Personal\\
\textbf{Components}: V, S\\
\textbf{Duration}: 1 round per Magic Proficiency\\
For the duration, you learn whether an aberration, celestial, elemental, fey, demon, or undead is within 10 meters of you, and its location. Likewise, you learn if there is a place or object within 10 meters of you that has been magically consecrated or desecrated.\\
the spell can penetrate most barriers, but is blocked by 30cm of stone, 1 cm of base metal, a thin sheet of lead, or 1 meter of wood or earth.\\
\textbf{For each magical critical success rolled} in the Magic Test the duration doubles.\\
\textbf{Note}: This spell has no effect on creatures that follow Traits. At the Arbiter's discretion it can be used to identify the Patron of a Follower or Devotee.

\medskip\textbf{Detect Magic}\index[Spells]{Detect Magic}\\
\textbf{School}: Universal\\
\textbf{Level}: 1, Common\\
\textbf{Cast Time}: 2 Actions\\
\textbf{Range}: Personal\\
\textbf{Components}: V, S\\
\textbf{Duration}: 1d4 +1 round per Magic Proficiency\\
For the duration, you sense the magic's presence within 10 meters of you. You can use 1 Action to see a faint aura that extends around any visible creature or object in the area that bears magic. With two Actions you also learn its Magic List, if it has it.\\
The spell can penetrate most barriers, but is blocked by 30cm of stone, 1 cm of base metal, a thin sheet of lead, or 1 meter of wood or earth.\\
\textbf{For each magical critical success rolled} in the Magic Test, the duration add 2 rounds.

\medskip\textbf{Detect Thoughts}\index[Spells]{Detect Thoughts}\\
\textbf{School}: Divination\\
\textbf{Level}: 2, Rare\\
\textbf{Cast Time}: 2 Actions\\
\textbf{Range}: Personal\\
\textbf{Components}: V, S, M (a piece of copper)\\
\textbf{Duration}: 1 minute\\
For the duration, you can read the thoughts of certain creatures. When you cast this spell and with two more Actions each round thereafter until the spell ends, you can focus your mind on any creature you can see within 10 meters of you. If the creature you chose has an Intelligence score of -3 or less or speaks no language, the creature ignores the effect.\\
Initially, you only learn the creature's surface thoughts—the most recurring ones. As an action, you can either shift your attention to another creature's thoughts or attempt to probe deeper into the same creature's mind. If you probe deeper, the target must make a Will save. If he fails, you gain insight into his reasoning (if any), his emotional state, and anything that predominates in his thoughts (such as a worry, love, or hate). If it succeeds at its Saving Throw, the spell ends. Either way, the target knows you're probing its mind, and unless you shift your attention to another creature's mind, on its round, the creature can use the 2 Action to make an Intelligence check contested by the creature. your Intelligence check; if he wins, the spell ends.\\
Questions posed verbally to the target creature obviously shape its train of thought, so this spell is particularly effective in interrogations.
You can also use this spell to detect the presence of thinking creatures you cannot see. When you cast this spell or with 2 Actions in its duration, you can search for thoughts within 10 meters of you. The spell can penetrate barriers, but is blocked by 0.5 meter of stone, 5 centimeters of non-lead metal, or a thin sheet of lead. You can't detect a creature with Intelligence -3 or less, or a creature that doesn't speak any languages. Once you detect a creature's presence in this way, you can read its thoughts for the duration of the spell as long as it stays within range, as described above, even if you can't see it.
You will be Distracted while you have this spell active for casting other spells.

\medskip\textbf{Dimension Door}\index[Spells]{Dimension Door}\\
\textbf{School}: Summon\\
\textbf{Level}: 4, Common\\
\textbf{Casting Time}: 2 Actions\\
\textbf{Range}: 150 meters\\
\textbf{Components}: V\\
\textbf{Duration}: Instant\\
You teleport from your current location to anywhere else within range. You arrive exactly at the desired place. It can be a place you can see, one you can visualize, or one you can describe by indicating distance and direction, such as "30 meters down" or "90 meters up northwest at a 45 degree angle." \\
You may carry items whose weight does not exceed your Encumbrance capacity. You can also bring along a willing creature of your size or smaller with gear up to the limit of its carrying capacity. The creature must be within 1 meter of you when you cast this spell. \\
Should you arrive at a place already occupied by an object or creature, you and the creature traveling with you each take 4d6 points of force damage, and the spell fails to teleport.\\
\textbf{For every two magical critical successes you roll} in the Magic Test you can bring one more creature.

\medskip\textbf{Discover Traps}\index[Spells]{Discover Traps}\\
\textbf{School}: Divination\\
\textbf{Level}: 2, Common\\
\textbf{Casting Time}: 2 Actions\\
\textbf{Range}: 36 meters\\
\textbf{Components}: V, S\\
\textbf{Duration}: 10 minutes\\
For the duration of the spell, you sense the presence of any traps within range that are in your line of sight. A trap, for the purposes of this spell, includes anything that is capable of inflicting a sudden or unexpected effect that you may consider harmful or undesirable, and that was expressly intended as such by its creator. Consequently, the spell would sense an area under the alarm spell, a glyph of ward, or a mechanical trap door, but would not reveal a natural weakness in the floor, an unstable ceiling, or a hidden hole.\\
The trap is highlighted in your view with a purple beacon.

\medskip\textbf{Discover the Path}\index[Spells]{Discover the Path}\\
\textbf{School}: Divination\\
\textbf{Level}: 6, Uncommon\\
\textbf{Cast Time}: 1 minute\\
\textbf{Range}: Personal\\
\textbf{Components}: V, S, M (divination tools - ivory sticks, bones, cards, teeth, or engraved runes - worth at least 100 gp and an item from the location you wish to find)\\
\textbf{Duration}: 1 day\\
This spell allows you to find the shortest and most direct physical route to a specific fixed location that you are familiar with and is on the same plane of existence. If you indicate a destination on another plane of existence, a moving destination (such as a mobile fortress), or a non-specific destination (such as "a green dragon's lair"), the spell fails. \\
For the duration of the spell, as long as you are on the same plane of existence as the destination, you will know how far away it is and which direction it is facing. While you are traveling towards it, whenever you are presented with the possibility of choosing between different routes, you will automatically determine which is the shortest way and the most direct (but not necessarily the safest) route to reach the destination.\\
\textbf{For each crit} scored in the Magic Test the spell lasts 8 hours longer.

\medskip\textbf{Disguise Self}\index[Spells]{Disguise Self}\\
\textbf{School}: Illusion\\
\textbf{Level}: 1, Common\\
\textbf{Casting Time}: 2 Actions\\
\textbf{Range}: Personal\\
\textbf{Components}: V, S\\
\textbf{Duration}: 1 hour\\
You change your appearance, along with that of your clothing, Armour, weapons, and other items you wear, until the spell ends or until you take an action to end the spell. You can appear a 30cm shorter or taller, thin, fat, or somewhere in between. You cannot change your physical conformation, so you must adopt a form that has the same distribution of limbs. For everything else, the illusion is limited only by your imagination.\\
Changes made by this spell are unable to withstand physical inspection. For example, if you use this spell to add a hat to your outfit, objects pass through the hat, and anyone touching it would feel nothing and would end up touching your head and hair. If you use this spell to appear thinner than you are, a person's hand attempting to touch you will bounce off you, while appearing to stop in mid-air to the eye. To distinguish your camouflage, a creature can use 2 Actions to inspect your appearance and must succeed on an Awareness +4 check against the spell's save DC.

\medskip\textbf{Disintegrate}\index[Spells]{Disintegrate}\\
\textbf{School}: Transmutation\\
\textbf{Level}: 6, Uncommon\\
\textbf{Cast Time}: 2 Actions\\
\textbf{Range}: 18 meters\\
\textbf{Components}: V, S, M (a magnet and a pinch of dust)\\
\textbf{Duration}: Instant\\
A thin green beam shoots from your pointing finger at a target that you can see within range. The target can be a creature, an object, or a creation of magical force, such as a wall created by wall of force. A creature targeted by this spell must make a Fortitude save. The target takes 10d6 + 40 force damage on a failed save and half damase if Saving Throw is successfull. If this damage reduces the target to 0 Hit Points, it is disintegrated. A disintegrated creature and everything it wears and carries, except magic items, is reduced to a pile of fine gray dust. The creature can only be brought back to life through the intervention of a Patron\\
This spell automatically disintegrates nonmagical items or a Large or smaller creation of magical force. If the target is a Huge or larger nonmagical object or creation of force, this spell disintegrates a portion of it equal to a 3m cube. Magic items ignore this spell.\\
\textbf{For each magical critical success rolled} in the Magic Test damage increases by 4d6.\\
\textbf{Saving Throw Success/Critical Failure}: On a critical failure the damage is doubled, on a critical success the damage is further halved

\medskip\textbf{Dislike/Like}\index[Spells]{Dislike/Like}\\
\textbf{School}: Enchantment\\
\textbf{Level}: 8, Rare\\
\textbf{Cast Time}: 1 hour\\
\textbf{Range}: 18 meters\\
\textbf{Components}: V, S, M (or a piece of alum dipped in vinegar for the dislike effect or a dash of honey for the sympathy effect)\\
\textbf{Duration}: 10 days\\
This spell attracts or repels creatures of your choice. Pick a target within range, whether it's a Huge or smaller object or a creature or area no larger than a 60mt cube. Then specify a kind of intelligent creature, such as red dragons, goblins, or vampires. Imbues the target with an aura that attracts or repels specified creatures for the duration. Choose dislike or sympathy as the effect of the aura.\\
Dislike. The enchantment causes creatures of the type you designate to feel a strong urge to leave the area and avoid the target. When such a creature can see the target or approaches within 20 meters of it, the creature must succeed at a Will save or become frightened. The creature is frightened as long as it can see the target or is within 20 meters of it. While frightened by the target, the creature must use its movement to move to the nearest safe place from which it can no longer see the target. If the creature moves more than 20 meters away from the target and cannot see it, the creature is no longer frightened, but becomes frightened again if it sees the target again or moves within 20 meters of it. \\
Sympathy. The enchantment causes the specified creatures to feel a strong urge to approach the target if they are within 20 meters of it or can see it. When such a creature can see the target or approaches within 20 meters of it, the creature must succeed at a Will save or use its movement each round to enter the area, or move within reach of the target. When the creature does, it can no longer voluntarily move away from the target. If the target harms or otherwise harms the subject creature, it can make Will saves to end the effect, as described below.\\
Finish the effect. If an affected creature ends its round while it is farther than 20 meters from the target or cannot see it, the creature makes a Will save. On a successful save, the creature is no longer subject to the target and recognizes the feeling of repugnance or attraction as magical. Furthermore, a creature subject to the spell is entitled to another Will save every 24 hours of the spell's duration. A creature that successfully saves against this effect is immune to it for 1 minute, after which it can be affected again. 

\medskip\textbf{Dispel Good and Evil}\index[Spells]{Dispel Good and Evil}\\
\textbf{School}: Abjuration\\
\textbf{Level}: 5, Rare\\
\textbf{Casting Time}: 2 Actions\\
\textbf{Range}: Personal\\
\textbf{Components}: V, S, M (Holy Water or silver and iron powder)\\
\textbf{Duration}: Concentration, 1 minute \\
A luminous energy surrounds you and protects you from fey, undead, and creatures native to places beyond the Material Plane. For the duration, celestials, elementals, fey, demons, and undead have -1d6 on attack rolls against you. You can end the spell early by using one of the following special functions.\\
\textit{Break Enchantment}. As an action, you can touch a creature charmed, frightened, or possessed by a celestial, elemental, fey, demon, or undead. The creature you touch is no longer charmed, frightened, or possessed by these creatures.\\
\textit{Leave}. As an action, make a melee attack against a celestial, elemental, fey, demon, or undead within your reach. If you hit it, you can attempt to send the creature back to its home plane. The creature must succeed at a Will save or be sent back to its home plane (if it isn't already there). If not on their home plane, undead are sent back to the Shadow World and fey to the First World.

\medskip\textbf{Dispel Magic}\index[Spells]{Dispel Magic}\hypertarget{dissolvimagie}{}\\
\textbf{School}: Abjuration\\
\textbf{Level}: 3, Common\\
\textbf{Cast Time}: 2 Actions\\
\textbf{Range}: 36 meters\\
\textbf{Components}: V, S\\
\textbf{Duration}: Instant\\
Choose a creature, object, or magical effect within range. Any 3-level or lower spells on the target end. Every magical critical success on spell raises the spell level by one. If cast on an item that manifests an enchantment, it is disabled for 10 minutes.\\
\textbf{For each magical Critical Success rolled} in the Magic Test, the dispel level increases by 1. In case of 3 critical successes, an effect can be permanently dispelled on a non-artifact object.

\medskip\textbf{Dispel Magic Advanced}\index[Spells]{Dispel Magic Advanced}\hypertarget{dissolvimagieavanzato}{}\\
\textbf{School}: Abjuration\\
\textbf{Level}: 5, Rare\\
\textbf{Cast Time}: 3 Actions\\
\textbf{Range}: 36 meters\\
\textbf{Component}: V, S, M (200 gp worth of diamond dust)\\
\textbf{Duration}: Instant\\
Choose a creature, object, or magical effect within range. Any 5-level or lower spells on the target end. Every magical critical success on spell raises the spell level by one. If cast on an item that manifests an enchantment, it is disabled for 10 minutes.\\
\textbf{For each magical critical success rolled} in the Magic Test, the dispelable level increases by 1.

\medskip\textbf{Divination}\index[Spells]{Divination}\\
\textbf{School}: Divination\\
\textbf{Level}: 6, Rare\\
\textbf{Cast Time}: 2 Actions\\
\textbf{Range}: Personal\\
\textbf{Components}: V, S, M (incense and a sacrificial offering appropriate to your religion, whose total value is 25 gp, which will be consumed by the spell)\\
\textbf{Duration}: Instant\\
Your magic and a votive offering put you in communication with a Patron or a Patron's servant. You can ask him a single question about a specific goal, event or activity that needs to happen within 7 days. The Arbiter gives a truthful answer. The reply could be a short sentence, a cryptic rhyme, or an omen. \\
The spell does not take into account any possible circumstances that could modify the result, such as the casting of further spells or the loss or arrival of an ally. \\
If you cast the spell two or more times before finishing the long day, there is a cumulative 25\% chance that for each cast after the first you get an erroneous reading. The Arbiter makes this roll secretly.

\medskip\textbf{Divine Favor}\index[Spells]{Divine Favor}\\
\textbf{School}: Invocation\\
\textbf{Level}: 1, Uncommon\\
\textbf{Casting Time}: 1 Immediate Action\\
\textbf{Range}: Personal\\
\textbf{Components}: V, S\\
\textbf{Duration}: 1 minute\\
Your prayers empower you and your weapon. Until the spell ends, when it hits, your weapon deals an additional 1d4 points of Light damage.\\
\textbf{For each magical critical success rolled} in the Magic Test your weapon deals +1 additional Light damage.

\medskip\textbf{Divine Word}\index[Spells]{Divine Word}\\
\textbf{School}: Invocation\\
\textbf{Level}: 7, Very Rare\\
\textbf{Casting Time}: 1 Immediate Action\\
\textbf{Range}: 9 meters\\
\textbf{Components}: V\\
\textbf{Duration}: Instant\\
You speak a divine word, infused with the power of your Patron. Choose any number of creatures that are within range and can see. Any creature that can hear you must make a Will save. On a failed save, the creature suffers an effect based on its current Hit Points:\\


- 100 Hit Points or less: deafened for 1 minute\\
- 40 Hit Points or less: deafened and blinded for 10 minutes\\
- 30 Hit Points or less: blinded, deafened, and stunned for 1 hour\\
- 20 Hit Points or less: killed instantly\\

Regardless of its current Hit Points, a celestial, elemental, fey, or demon who fails its Saving Throw is forced to return to its plane of origin (if it isn't already there) and cannot return to your current plane before they are 24 hours have passed, unless the wish spell is used.


\medskip\textbf{Dominate Beasts}\index[Spells]{Dominate Beasts}\\
\textbf{School}: Enchantment, Animals and Plants\\
\textbf{Level}: 4, Very Rare - Common\\
\textbf{Cast Time}: 2 Actions\\
\textbf{Range}: 18 meters\\
\textbf{Components}: V, S\\
\textbf{Duration}: Concentration, max 1 minute\\
You try to charm a beast that you can see within range. It must succeed at a Will save or be charmed for the duration, gaining +1d6 on the save if you or your allies are fighting it.\\
While the beast is charmed, you maintain a telepathic link with it as long as the two of you are on the same plane of existence. You can use this telepathic link to issue commands to the creature while you are conscious (requires 1 action), which it will obey as best it can. You can specify a simple, generic course of action, such as "Attack that creature", "Run over there", or "Take that item". If the creature completes the order and receives no further guidance from you, it will defend and preserve to the best of its ability.\\
You can spend 2 of your actions to assume total and precise control of the target. Until the end of your next round, the target will only take actions you decide, and it won't do anything you don't let it do. During this time, you can also make the target use a Reaction Action, but this requires the use of your reaction.\\
Each time the target takes damage, it makes a new Will save against the spell. On a successful save, the spell ends.\\
\textbf{For each magical critical success rolled} in the Magic Test the duration doubles up to a maximum of 8 hours.

\medskip\textbf{Dominate Monster}\index[Spells]{Dominate Monster}\\
\textbf{School}: Enchantment\\
\textbf{Level}: 8, Uncommon\\
\textbf{Cast Time}: 2 Actions\\
\textbf{Range}: 18 meters\\
\textbf{Components}: V, S\\
\textbf{Duration}: Concentration, max 1 hour\\
You try to charm a creature that you can see within range. It must succeed at a Will save or be charmed for the duration, gaining +1d6 on the save if you or your allies are fighting it.\\
While the creature is charmed, you maintain a telepathic link with it as long as the two of you are on the same plane of existence. You can use this telepathic link to issue commands to the creature while you are conscious (requires 1 action), which it will obey as best it can. You can specify a simple, generic course of action, such as "Attack that creature", "Run over there", or "Take that item". If the creature completes the order and receives no further guidance from you, it will defend and preserve to the best of its ability.\\
You can spend two of your Actions to assume total and precise control of the target. Until the end of your next round, the creature will only take actions you decide, and it won't do anything you don't allow it to do. During this time, you can also have the creature use a Reaction Action, but this requires the use of your reaction. Each time the target takes damage, it makes a new Will save against the spell. On a successful save, the spell ends.\\
\textbf{For each magical critical success rolled} in the Magic Test the duration doubles up to a maximum of 8 hours.

\medskip\textbf{Dominate Person}\index[Spells]{Dominate Person}\\
\textbf{School}: Enchantment\\
\textbf{Level}: 5, Uncommon\\
\textbf{Cast Time}: 2 Actions\\
\textbf{Range}: 18 meters\\
\textbf{Components}: V, S\\
\textbf{Duration}: Concentration, max 1 minute\\
You attempt to charm a humanoid within range that you can see. It must succeed at a Will save or be charmed for the duration, gaining +1d6 on the save if you or your allies are fighting it.\\
While the target is charmed, you maintain a telepathic link with the target as long as the two of you are on the same plane of existence. You can use this telepathic link to issue commands to the target while you are conscious (requires 1 action), which it will obey as best it can. You can specify a simple, generic course of action, such as "Attack that creature", "Run over there", or "Take that item". If the target completes the order and receives no further guidance from you, he will defend himself to the best of his abilities.\\
You can spend 2 Actions to assume total and precise control of the target. Until the end of your next round, the target will only take actions you decide, and it won't do anything you don't let it do. During this time, you can also have the target use a Reaction Action, but this requires the use of your reaction. Each time the target takes damage, it makes a new Will save against the spell. On a successful save, the spell ends.\\
\textbf{For each magical critical success rolled} in the Magic Test the duration doubles up to a maximum of 8 hours.

\medskip\textbf{Dream}\index[Spells]{Dream}\\
\textbf{School}: Illusion\\
\textbf{Level}: 5, Uncommon\\
\textbf{Casting Time}: 2 Actions\\
\textbf{Range}: Special\\
\textbf{Components}: V, S, M (a handful of sand, a point of ink, and a writing pen taken from a sleeping bird)\\
\textbf{Duration}: 8 hours\\
This spell shapes a creature's dreams. Choose a creature known to you as the target of the spell. The target must be on the same plane of existence as you. Creatures that don't sleep can't be affected by this spell. You or a willing creature you touch enter a trance-like state, acting as a messenger. While in a trance, the messenger is aware of its surroundings, but cannot take actions or move.\\
For the duration of the spell, if the target is asleep, the messenger appears in the target's dreams and can converse with him as long as he remains asleep. The messenger can also shape the dream environment, creating terrain, objects, and other images. The messenger can emerge from the trance at any time, prematurely ending the spell's effect. Upon awakening, the target remembers his dream perfectly. If the target is awake when you cast the spell, the messenger knows this and can end the trance (and the spell) or wait for the target to fall asleep. At that point the messenger can appear in the target's dreams.\\
You can make the messenger appear monstrous and terrifying to the target. If you do, the messenger can deliver a message of up to ten words, and then the target must make a Will save. On a failed save, the echoes of the hideous monstrosity spawn a nightmare for the duration of the target's sleep that prevents it from gaining any benefit from that rest. Also, when the target awakens, it takes 3d6 points of damage.\\
If you have a lock of hair, clipped fingernails, or similar portion of the target's body, it will make its Saving Throw at -1d6.

\medskip\textbf{Druidic Artifice}\index{Cantrip - Druidic Artifice}\\
\textbf{School}: Universal\\
\textbf{Level}: 0, Uncommon\\
\textbf{Casting Time}: 2 Actions\\
\textbf{Range}: 9 meters\\
\textbf{Components}: V, S\\
\textbf{Duration}: Instant\\
Whispering to nature spirits, you create one of the following effects within range:

- You create a tiny, harmless sensory effect that predicts what the weather will be like where you are for the next 24 hours. The effect could manifest itself as a golden sphere for clear skies, a cloud for rain, snowflakes for snow, and so on. The effect lasts for 1 round.\\

- Immediately make a flower, seed or similar plant bloom.\\

- Create an instant harmless sensory effect, such as falling leaves, a puff of wind, the sound of a small animal, or the faint smell of a skunk. The effect must fit in a 1 meter cube.\\

- Instantly light or extinguish a candle, torch or small bonfire.\\

This spell can only be cast by Followers or Devotees of Ephrem, Erondil, Gaya, Shayalia.

\medskip\textbf{Eithne's Mudball}\index{Eithne's Mudball}\\
\textbf{School}: Earth\\
\textbf{Level}: 1, Uncommon\\
\textbf{Cast Time}: 2 Action\\
\textbf{Range}: 24 meters\\
\textbf{Components}: S\\
\textbf{Duration}: Instant\\
The caster mimes the gesture of throwing a stone with a slingshot towards the target and makes an attack roll with ranged spells.
If the attack roll is successful, the target takes 2d6 bludgeon damage and must make a Reflex saving throw. If the saving throw fails, the target's movement decreases by 2 meters per Action for 1 minute.\\
\textbf{For each Magical Critical Success obtained} in the Magic Test, you throw one more stone.

\medskip\textbf{Earthquake}\index[Spells]{Earthquake}\\
\textbf{School}: Earth\\
\textbf{Level}: 8, Very Rare\\
\textbf{Casting Time}: 2 Actions\\
\textbf{Range}: 150 meters\\
\textbf{Components}: V, S, M (a pinch of dirt, a piece of stone and a lump of clay)\\
\textbf{Duration}: Concentration, max 1 minute\\
You cause a seismic disturbance at a point on the ground that you can see within range and. For the duration, an intense tremor shakes the ground in a 30m radius circle centered on that point, and shakes creatures and structures in that area that are in contact with the ground. The terrain in the area becomes hindering terrain. Each creature on the ground that is concentrating must make a Fortitude save. If he fails, his concentration is broken.\\
When you cast this spell, and at the end of each round you spend concentrating on it, each creature in the area that is on the ground must make a Reflex save. On a failed save, the creature falls prone.\\
This spell has additional effects depending on the type of terrain in the area, at the Arbiter's discretion. Fissures. At the beginning of the round following the one in which you cast the spell, fissures open throughout the spell's area. A total of 1d6 fissures open at locations chosen by the Arbiter. Each is 1d10 x 3 meter deep, 3 meter wide, and extends from one side of the spell's area to the other. A creature standing on the spot where a fissure opens must succeed at a Reflex save or fall into it. A creature that succeeds at its Saving Throw moves to the edge of the fissure as it opens.\\
A fissure that opens under a structure immediately causes it to collapse (see below). Structures. The tremor deals 50 bludgeoning damage to any structure in contact with the ground in the area when you cast the spell and at the end of each of your rounds until the spell ends. If a structure drops to 0 Hit Points, it collapses and may damage nearby creatures. A creature half the height of the structure or less away from the structure must make a Reflex save. On a failed save, the creature takes 5d6 bludgeoning damage, falls prone, and is engulfed in rubble. He must then use 2 actions with a successful DC 20 Dexterity (Athletics) check to free himself. The Arbiter can adjust the DC up or down, depending on the nature of the rubble. On a successful save, the creature takes only half damage and does not fall or become buried.

\medskip\textbf{Elastic Sphere}\index[Spells]{Elastic Sphere}\\
\textbf{School}: Invocation\\
\textbf{Level}: 4, Rare\\
\textbf{Casting Time}: 2 Actions\\
\textbf{Range}: 90 meters\\
\textbf{Components}: V, S, M (a hemispherical piece of clear crystal and a corresponding hemispherical piece of gum arabic)\\
\textbf{Duration}: Concentration, max 1 minute\\
A ball of luminous energy envelops a Large or smaller creature or object within range. An unwilling creature must make a Reflex save. On a failed save, the creature is engulfed in the spell for its duration.\\
Nothing (neither physical objects, nor energy, nor other spell effects) can pass through this barrier, in or out, although a creature inside the sphere can breathe without problem. The sphere is immune to all damage, and a creature inside it can't be damaged by attacks or effects originating from outside, nor can a creature inside the sphere harm anything outside it. The sphere is weightless and just large enough to hold the creature or object within it. An engulfed creature can use 1 Action to push against the walls of the sphere and then roll it up to half the creature's speed. Likewise, the orb can be picked up and moved by other creatures.\\
A disintegrate spell targeting the orb destroys it without harming anything inside.

\medskip\textbf{Endless Flame}\index[Spells]{Endless Flame}\\
\textbf{School}: Universal\\
\textbf{Level}: 2, Legendary\\
\textbf{Casting Time}: 2 Actions\\
\textbf{Range}: Contact\\
\textbf{Component}: V, S, M (ruby dust worth 75 gp, which the spell consumes)\\
\textbf{Duration}: 1 day\\
A light similar to that produced by a torch is given off by an object with which you are in contact. The effect resembles that of a normal flame, but does not produce heat or require oxygen. An endless flame can be concealed or hidden but it cannot be dimmed or extinguished.

\medskip\textbf{Energy Weapon}\index[Spells]{Energy Weapon}\index{Spada laser}\\
\textbf{School}: Air, Water, Earth, Fire\\
\textbf{Level}: 1, Very Rare\\
\textbf{Casting Time}: 1 Action\\
\textbf{Range}: Contact\\
\textbf{Component}: V, S, M (of Fairy's hair)\\
\textbf{Duration}: 6 rounds, Concentration\\
You cast the spell in touch with a weapon that acquires powers according to the Magic List used and is considered magical, as if it had a +1 bonus.
If Energy Weapon is cast using the Air List the weapon becomes shot through with electricity, in case of Water the weapon becomes extremely cold, in case of Earth the weapon gushes acid, in case of Fire it becomes flaming. Whichever List is used, the effect is such that the weapon causes an additional 1d6 damage of the indicated type per successful hit.
A weapon can only have one Energy Weapon effect active at a time.
For each Magic List you possess, you can add one elemental effect and add 1d6 of damage of the chosen type. Each round using 1 action it is possible to change the type of damage. \\
\textbf{For every two magical critical successes rolled} in the Magic Test the damage increases by +1d6.

\medskip\textbf{Enhance}\index[Spells]{Enhance}\\
\textbf{School}: Transmutation\\
\textbf{Level}: 2, Common\\
\textbf{Cast Time}: 2 Actions\\
\textbf{Range}: Contact\\
\textbf{Component}: V, S, M (fur or feather of a beast)\\
\textbf{Duration}: maximum 10 minutes\\
You bestow a magical enhancement on a creature you touch. Choose one of the following effects; the target gains that effect until the spell ends.\\
\textit{Fox's Cunning}. Target has +1d6 on Intelligence and Strength checks \\
\textit{Strength of the Bull}. The target has +1d6 on Strength checks, and its Encumbrance ability is doubled.\\
\textit{Bright Energy Grace}. The target has +1d6 on Dexterity checks. Also, if not incapacitated, he takes no damage from falling 6 meters or less.\\
\textit{Bear's Endurance}. The target has +1d6 on Constitution checks. He also gains 2d6 temporary Hit Points, which are lost at the end of the spell. \\
\textit{Owl's Wisdom}. The target has +1d6 on Wisdom checks. \\
\textit{Eagle's Splendour}. The target has +1d6 on Charisma checks.\\
\textbf{For each magical critical success you roll} in Magic Test, you may target an additional creature

\medskip\textbf{Enlarge/Reduce}\index[Spells]{Enlarge/Reduce}\\
\textbf{School}: Transmutation\\
\textbf{Level}: 2, Common\\
\textbf{Cast Time}: 2 Actions\\
\textbf{Range}: 9 meters\\
\textbf{Components}: V, S, M (a pinch of powdered iron)\\
\textbf{Duration}: 1 minute\\
Causes a creature or object that you can see to grow or shrink for the duration of the spell. Choose a creature or object that is neither worn nor carried. If the target is unwilling, it can make a Fortitude save; if it succeeds, the spell has no effect. If the target is a creature, everything it is wearing and carrying changes size with it. Any item dropped by an affected creature immediately returns to its normal size. \\

- \textit{Enlarge}. The target's size doubles in all dimensions, and its weight is multiplied by eight. This growth increases its size by one category: from Medium to Large, for example. If there isn't enough room for the target to double its size, the creature or object assumes the largest possible size the available space allows. Until the spell ends, the target has +1d6 on Actions based on Strength and Fortitude saves. The target's weapons scale to reach the new size. While these weapons are enlarged, the target's attacks with them will do an additional damage category.\\
- \textit{Reduce}. The target's size is halved in all dimensions, and its weight is reduced to one-eighth. This growth decreases its size by one category: from Medium to Small, for example. Until the spell ends, the target has -1d6 on Actions based on Strength and Fortitude saves. The target's weapons shrink to reach the new size. While these weapons are scaled down, the target's attacks with them will do a lower damage category (but without reducing the weapon's damage below 1).\\
\textbf{For every two Critical success} in the Magic Test the creature increases by another size, or affects another creature within 6 meters of the first.


\medskip\textbf{Enrapture}\index[Spells]{Enrapture}\\
\textbf{School}: Enchantment\\
\textbf{Level}: 2, Common\\
\textbf{Casting Time}: 2 Actions\\
\textbf{Range}: Personal\\
\textbf{Components}: V, S\\
\textbf{Duration}: 1 minute\\
You weave a series of misleading words, causing creatures of your choice within range, who you can see and can hear you to make a Will save. Any creature that can't be charmed makes its Saving Throw automatically, and if you or your companions are fighting a creature, it has +1d6 on its Saving Throw. On a failed save, the target has -1d6 on Awareness checks made to perceive any creature other than you until the spell ends or until the target can no longer hear you.
The spell ends if you are incapacitated or can no longer speak.

\medskip\textbf{Entangle}\index[Spells]{Entangle}\\
\textbf{School}: Animals and Plants\\
\textbf{Level}: 1, Common\\
\textbf{Cast Time}: 2 Actions\\
\textbf{Range}: 27 meters\\
\textbf{Components}: V, S\\
\textbf{Duration}: 1 minute\\
Vines and crushing branches sprout from the ground in a 6 meters square from a point within range. For the duration, these plants turn the terrain in the area into hindering terrain.\\
A creature in the area when you cast this spell must succeed at a Fortitude save or be restrained by these plants until the spell ends. A creature entangled by plants can use two Actions to make a new Saving Throw. If it passes, it breaks free. When the spell ends, the summoned plants vanish.

\medskip\textbf{Ethereal Form}\index[Spells]{Ethereal Form}\\
\textbf{School}: Transmutation\\
\textbf{Level}: 7, Rare\\
\textbf{Casting Time}: 2 Actions\\
\textbf{Range}: Personal\\
\textbf{Components}: V, S\\
\textbf{Duration}: Maximum 8 hours\\
You enter the border regions of the Ethereal Plane, the area that overlaps your current plane. You remain on the Ethereal Verge for the duration or until you use an action to end the spell. If you move up or down, the move cost is doubled, if you move horizontally, the move cost is doubled per move action. You can see and hear the plane you came from, but everything there appears gray to you, and you can't see more than 20 meters away.\\
While on the Ethereal Plane, it can only interact with other creatures on that plane. Creatures not on the Ethereal Plane cannot perceive or interact with you unless a special ability or magic allows them to do so. \\
You ignore all objects and effects that are not on the Ethereal Plane, allowing you to pass through objects you perceive on the plane you came from. When the spell ends, you immediately return to the plane you came from at the point you currently occupy. If you occupy the same space as a solid object or creature when this happens, you are immediately moved to the closest unoccupied space you can occupy and you take 6 force damage for each 0.5m you are moved (or a fraction thereof). This spell has no effect if you cast it while already on the Ethereal Plane or on a plane that doesn't border it, such as one of the Outer Planes. \\
\textbf{For each magical critical success you roll} in the Magic Test, you may take another creature with you.

\medskip\textbf{Fabricate}\index[Spells]{Fabricate}\\
\textbf{School}: Transmutation\\
\textbf{Level}: 4, Common\\
\textbf{Casting Time}: 10 minutes\\
\textbf{Range}: 36 meters\\
\textbf{Components}: V, S\\
\textbf{Duration}: Instant\\
Convert raw materials into finished products of the same material. For example, you can make a small wooden bridge from a pile of trees, a rope from a pile of hemp, and clothing from flax or wool. Choose raw materials that you can see within range. You can craft one Large or smaller item (contained in one 3m cube, or eight connected 1m cubes) given enough raw materials. If you're working with metal, stone, or other mineral substances, the crafted item can't be larger than Medium size (contained in a single 1m cube). The quality of the items created by this spell is commensurate with the quality of the raw materials.\\
You cannot create or transmute magical creatures or items with this spell. You also cannot use it to craft items that normally require a high level of crafting, such as jewelry, weapons, glass, or Armour, unless you are proficient with the type of crafting tools used to craft these items. In case of critical in the Magic Test you can process more volumes or produce with higher quality.

\medskip\textbf{Fairy Dust}\index[Spells]{Fairy Dust}\\
\textbf{School}: Fire, Air\\
\textbf{Level}: 2, Uncommon\\
\textbf{Casting Time}: 2 Actions\\
\textbf{Range}: 36 meters\\
\textbf{Components}: V, S, M (Silver Dust)\\
\textbf{Duration}: 1 round per Magic Proficiency\\
In a 3m-diameter sphere, everyone in it is covered in glittering, luminous dust. The cloud outlines the creatures present, even the invisible ones, and anyone who remains in the area must make a Reflex save at the start of the round or be blinded for the round. The dust will naturally disappear after the duration or if blown away by even a light wind.

\medskip\textbf{False Life}\index[Spells]{False Life}\\
\textbf{School}: Necromancy\\
\textbf{Level}: 1, Common\\
\textbf{Casting Time}: 2 Actions\\
\textbf{Range}: Personal\\
\textbf{Components}: V, S, M (a small amount of distilled alcohol or spirit)\\
\textbf{Duration}: 1 hour\\
Empowering yourself with a necromantic semblance of vitality, you gain 1d4 + 4 temporary Hit Points for the duration.\\
\textbf{For each magical critical success rolled} in the Magic Test you gain 5 temporary Hit Points.

\medskip\textbf{Fast Step}\index[Spells]{Fast Step}\\
\textbf{School}: Transmutation\\
\textbf{Level}: 1, Very Rare\\
\textbf{Cast Time}: 2 Actions\\
\textbf{Range}: Contact\\
\textbf{Components}: V, S, M (a hare's foot)\\
\textbf{Duration}: 1 hour\\
A creature's movement increases by 1 meter until the spell ends. \\
\textbf{For each magical critical success you roll} in Magic Test, you may target one additional creature.

\medskip\textbf{Fatal}\index[Spells]{Fatal}\\
\textbf{School}: Illusion\\
\textbf{Level}: 9, Rare\\
\textbf{Casting Time}: 2 Actions\\
\textbf{Range}: 36 meters\\
\textbf{Components}: V, S\\
\textbf{Duration}: Concentration, max 1 minute\\
By tapping into the innermost fears of a group of creatures, you create illusory creatures in their minds, visible only to them. Each creature in a 10m-radius sphere centered on a point of your choice in range must make a Will save. On a failed save, the creature becomes frightened for the duration. The illusion sinks into the creature's innermost fears, manifesting its worst nightmares as an implacable threat. At the end of each round, the frightened creature must succeed at a Will save or take 4d10 points of damage. On a successful save, the spell ends for that creature.

\medskip\textbf{Fear}\index[Spells]{Fear}\\
\textbf{School}: Illusion\\
\textbf{Level}: 3, Uncommon\\
\textbf{Casting Time}: 2 Actions\\
\textbf{Range}: Self (10m cone)\\
\textbf{Components}: V, S, M (a white feather or a chicken's heart)\\
\textbf{Duration}: 1 minute\\
You project an illusory image of a creature's worst fears. Each creature in a 10m cone must succeed at a Will save or drop whatever it is holding and become frightened for the duration.\\
While frightened by this spell, a creature must, on each of its rounds, take the dash action and move away from you by the safest route, unless it has no room to move. If the creature ends its round in a place where it has no line of sight to you, it can make a Will save, and on a successful one, the spell ends for that creature.

\medskip\textbf{Feather fall}\index[Spells]{Feather fall}\label{featherfall}\\
\textbf{School}: Air\\
\textbf{Level}: 1, Common\\
\textbf{Casting Time}: 1 Reaction, which you take when you or a creature within 20 meters of you falls\\
\textbf{Range}: 18 meters\\
\textbf{Components}: V, M (a small feather or piece of feather)\\
\textbf{Duration}: 1 minute\\
Choose up to 2 creatures within range. A falling creature's rate of descent decreases to 60 feet per round until the spell ends. If the creature lands before the spell ends, it takes no falling damage and can land on its feet; for that creature the spell ends.\\
\textbf{For each Magical Critical Success} obtained in the Magic Test you can move sideways 1 meter or affect another creature.

\medskip\textbf{Finger}\index[Spells]{Finger}\\
\textbf{School}: Enchantment\\
\textbf{Level}: 0, Rare\\
\textbf{Casting Time}: 1 Immediate Action\\
\textbf{Range}: 18 meters\\
\textbf{Components}: S\\
\textbf{Duration}: 3 rounds\\
Give the finger (or raspberry or umbrella gesture) to the opponent who must be able to see (or hear) \\
This must make a Will save, nothing happens on a successful one.
If he fails the Saving Throw by 5 or more he is humiliated, for the next 2 rounds he has a penalty of 2 on attack rolls, Saving Throw and Proficiency checks. \\
If he fails the Saving Throw by 3 or 4, he is mortified, until the end of the next round he has a penalty of 2 on attack and proficiency rolls. \\
If he fails the Saving Throw by 2 or 1, he is punished, until the end of the next round he has a penalty of 2 on Attack or Defence Rolls (choice of target).\\
\textbf{For each magical critical success you roll} in the Magic Test you can affect one other creature that can see the gesture.

\medskip\textbf{Finger of Death}\index[Spells]{Finger of Death}\\
\textbf{School}: Necromancy\\
\textbf{Level}: 6, Common\\
\textbf{Cast Time}: 2 Actions\\
\textbf{Range}: 18 meters\\
\textbf{Components}: V, S\\
\textbf{Duration}: Instant\\
You send a blast of negative energy to a creature that you can see within range, causing it great pain. The target must make a Fortitude save. The target takes 7d8 + 30 void damage on a failed save, or half as much damage on a successful one. \\
A humanoid slain by this spell reanimates as a zombie under your permanent command at the start of your next round, and will do your verbal bidding to the best of its ability.\\
\textbf{Save Critical Success/Failure}: On a critical failure the damage is doubled, on a critical success the damage is further halved

\medskip

\begin{changemargin}{0.3cm}{0.3cm}\begin{emphasis}{
I scatter around to avoid area spells (called by a player to avoid a Fireball)
}\end{emphasis}\end{changemargin}

\medskip\textbf{Fireball}\index[Spells]{Fireball}\\
\textbf{School}: Fire\\
\textbf{Level}: 3, Common\\
\textbf{Casting Time}: 2 Actions\\
\textbf{Range}: 45 meters\\
\textbf{Components}: V, S, M (a tiny ball of bat guano and sulfur)\\
\textbf{Duration}: Istant\\
A beam of yellow light shoots from your pointing finger at a point you choose within range and then explodes with a thunderous roar and transforms into a tongue of flame.\\
Each creature in a 20-foot-radius sphere centered on that point must make a Reflex save. A creature takes 8d6 fire damage on a failed save, or half as much damage on a successful one.\\
The fire spreads and occupies all the available volume within 6 meters of the point of explosion. Fire ignites flammable items in the area that are not being worn or carried.\\
\textbf{For each magical Critical Success rolled} in the Magic Test the base damage increases by 3d6.\\
\textbf{Saving Throw Success/Critical Failure}: On a critical failure the damage is doubled, on a critical success the damage is further halved

\medskip\textbf{Fire Bolt}\index[Spells]{Fire Bolt}\\
\textbf{School}: Fire\\
\textbf{Level}: 1, Common\\
\textbf{Casting Time}: 2 Actions\\
\textbf{Range}: 36 meters\\
\textbf{Components}: V, S\\
\textbf{Duration}: Instant\\
You hurl a fiery spark at a creature or object within range. Make a ranged spell attack against the target. On a hit, the target takes 1d10 fire damage. A flammable object affected by this spell catches fire, if it is not being worn or carried. \\
The spell's damage increases by 1d8 when you reach MP 5, MP 11, and MP 17, but it costs 2 Actions to cast it empowered and 2 Magic Points, it is also necessary to have taken Adept of Magic in this Magic List a number of times equal to the empowerments you want to apply\\
\textbf{For every two magical critical successes you roll} in the Magic Test, you cast an additional spark.

\medskip\textbf{Fire Shield}\index[Spells]{Fire Shield}\\
\textbf{School}: Fire, Water\\
\textbf{Level}: 4, Uncommon\\
\textbf{Casting Time}: 2 Actions\\
\textbf{Range}: Personal\\
\textbf{Components}: V, S, M (some phosphor or a firefly) \\
\textbf{Duration}: 10 minutes\\
Thin, wispy flames engulf your body for the duration, emitting bright light in a 3m radius and dim light for an additional 3 meter. You can end the spell early, using an action to end it.\\
The flames provide you with a hot shield or a cold shield, your choice. Heat shield gives you resistance to cold damage, while cold shield gives you resistance to heat damage.\\
In addition, whenever a creature within 1 meter of you hits you with a melee attack, the shield erupts in flames. The attacker takes 2d8 points of fire damage from a hot shield, or 2d8 points of cold damage from a cold shield.

\medskip\textbf{Flaming Sphere}\index[Spells]{Flaming Sphere}\\
\textbf{School}: Fire\\
\textbf{Level}: 2, Common\\
\textbf{Casting Time}: 2 Actions\\
\textbf{Range}: 18 meters\\
\textbf{Components}: V, S, M (a little tallow, a pinch of sulfur and a handful of iron powder)\\
\textbf{Duration}: 1 minute\\
For the duration, a 1m-diameter sphere appears in a space you choose within range. Any creature that ends its round within 1 meter of the sphere must make a Reflex save. The creature takes 2d6 fire damage on a failed save, or half as much damage on a successful one. \\
As an action, you can move the sphere 10 meters. If you crash the sphere into a creature, the creature must make a Saving Throw against the sphere's damage, and the sphere stops moving for that round.
When you move the sphere, you can move it over barriers up to 1m high, and make it jump over spaces up to 3m wide. The sphere sets flammable objects not being worn or carried on fire, and sheds bright light in a 3 meters radius and dim light for an additional 3 meters.\\
While you have this spell active you are distracted from casting other spells.\\
\textbf{For each magical critical success rolled} in the Magic Test the damage increases by 1d6.

\medskip\textbf{Firecloud}\index[Spells]{Firecloud}\\
\textbf{School}: Fire\\
\textbf{Level}: 8, Rare\\
\textbf{Cast Time}: 2 Actions\\
\textbf{Range}: 45 meters\\
\textbf{Components}: V, S\\
\textbf{Duration}: 1 minute\\
A cloud of swirling smoke shot through with incandescent lava forms into a 6 meters-radius sphere centered on a point within range. The cloud spreads around corners and is in dim light. It remains for the duration of the spell or until a wind of moderate speed or higher (at least 15 kilometers per hour) dispels it.\\
When the cloud appears, each creature within it must make a Reflex save. A creature takes 10d8 fire damage on a failed save, and half as much damage on a successful one. A creature must also make a Saving Throw when it first enters the area or ends its round there. \\
At the start of each of your rounds, the cloud moves 3 meter away from you in a direction of your choice.

\medskip\textbf{Firestorm}\index[Spells]{Firestorm}\\
\textbf{School}: Fire\\
\textbf{Level}: 7, Rare\\
\textbf{Casting Time}: 2 Actions\\
\textbf{Range}: 45 meters\\
\textbf{Components}: V, S\\
\textbf{Duration}: Instant\\
A storm composed of roaring flames appears at a point you choose within range. The storm area consists of up to ten cubes with 3 meter edges, which you can arrange however you like. Each cube must have at least one face adjacent to that of another cube. Each creature in the area must make a Reflex save. It takes 7d10 fire damage on a failed save, or half as much damage on a successful one. Fire damages objects in the area and ignites flammable objects that are not worn or carried. If you wish, the plant life in the area is unaffected by the effects of this spell. \\
\textbf{For each magical critical success you roll} in the Magic Test you increase the area of a 3m cube. \\
\textbf{Save Critical Success/Failure}: On a critical failure the damage is doubled, on a critical success the damage is further halved

\medskip\textbf{Flame Strike}\index[Spells]{Flame Strike}\\
\textbf{School}: Fire\\
\textbf{Level}: 5, Common\\
\textbf{Casting Time}: 2 Actions\\
\textbf{Range}: 18 meters\\
\textbf{Components}: V, S, M (pinch of sulfur)\\
\textbf{Duration}: Instant\\
A vertical column of divine fire descends from the sky and crashes into the location you specify. Each creature in a 3m-radius, 12m high cylinder centered on a point within range must make a Reflex save. A creature takes 8d6 Light damage on a failed save, or half as much damage on a successful one. \\
\textbf{For each magical critical success rolled} in the Magic Test the Light damage increases by 4d6.\\
\textbf{Save Critical Success/Failure}: On a critical failure the damage is doubled, on a critical success the damage is further halved

\medskip\textbf{Flamethrower}\index[Spells]{Flamethrower}\\
\textbf{School}: Fire\\
\textbf{Level}: 2, Rare\\
\textbf{Casting Time}: 2 Actions\\
\textbf{Range}: Personal\\
\textbf{Components}: V, S, M (a 30cm iron pipe, some beans)\\
\textbf{Duration}: 1 minute, Concentration\\
A flame appears at the end of the metal tube you hold in your hand. The flame remains there for the duration of the spell during which you must concentrate and does no harm to you or your equipment. The flame produces bright light in a 1m radius and dim light in a 1m radius. The spell ends if you interrupt it with an action or cast it again.\\
With a Ranged Spell Attack Roll and spending 1 Action you can stretch the flame up to 10 meters to hit a target. On a hit, the target takes 2d6 fire damage, if you hold the target you have a +2 to hit the next round.\\
\textbf{For each Critical scored} in the Magic Test the damage increases by 1d6.

\medskip\textbf{Flesh in Stone - Stone in Flesh}\index[Spells]{Flesh in Stone}\index[Spells]{Stone in Flesh}\\
\textbf{School}: Earth\\
\textbf{Level}: 6, Uncommon - Rare\\
\textbf{Cast Time}: 2 Actions\\
\textbf{Range}: 18 meters\\
\textbf{Components}: V, S, M (a pinch of lime, water and earth)\\
\textbf{Duration}: Permanent\\
You try to turn a creature within range that you can see to stone. If the target's body is made of flesh, the creature becomes slowed 1/6r and must make a Fortitude saving throw. If she fails her save she becomes slowed for 1/10 minutes and her flesh begins to harden. 
The creature that fails the initial saving throw the next round must make a new Fortitude saving throw. If the save is successful there are no further effects. If she fails this new saving throw, she is turned to stone and becomes petrified for the duration.

If the first saving throw succeeds, the creature suffers no further effects. 

If the creature is physically harmed while petrified, it suffers deformities similar to the damage done to the stone if it returns to its original state.\\
The \emph{Stone to Flesh} spell returns a creature to flesh as long as it hasn't been transformed for more than a year. The dispel magic spell cannot negate its effects.


\medskip\textbf{Floating Disc}\index[Spells]{Floating Disc}\\
\textbf{School}: Summon\\
\textbf{Level}: 1, Common\\
\textbf{Casting Time}: 2 Actions\\
\textbf{Range}: 9 meters\\
\textbf{Components}: V, S, M (a drop of mercury)\\
\textbf{Duration}: 1 hour\\
This spell creates a perfectly circular, slightly concave horizontal plane of force 1 meter in diameter and 1 cm thick that floats 1 meter off the ground in an unoccupied space of your choice that you can see within range. The disc remains active for the duration, and can hold 120kg. If a greater weight is placed on it, the spell ends and everything on it falls to the ground. As long as you are within 6 meters of it, the disc is motionless. If you move more than 6 meters away from it, the puck follows you so that it always stays 6 meters away from you. It can move across uneven terrain, up and down stairs, slopes, and the like, but it can't overcome altitude changes of 3 meter or more. For example, the disc can't cross a 3m-deep ditch, nor could it leave the ditch if it were created at the bottom of it. The disc can be grabbed by the caster and moved manually. If you move more than 30 meters away from the puck (usually because it fails to go around an obstacle to follow you) the spell ends.\\
\textbf{For each magical critical success rolled} in the Magic Test, the duration doubles.

\medskip\textbf{Fly}\index[Spells]{Fly}\\
\textbf{School}: Air\\
\textbf{Level}: 3, Common\\
\textbf{Casting Time}: 2 Actions\\
\textbf{Range}: Contact\\
\textbf{Components}: V, S, M (a feather from any bird's wing)\\
\textbf{Duration}: 10 minutes \\
You cast the spell touching a willing creature. For the duration, the target gains a flying speed of 20 meters. When the spell ends, if still in the air, the target falls, unless he is able to stop the descent. \\
Casting a spell while flying is more difficult, being Distracted if failing a DC 11 Fly check.\\
\textbf{For each magical critical success you roll} in the Magic Test you can target an additional creature or double the duration.

\medskip\textbf{Vigor}\index[Spells]{Vigor}\\
\textbf{School}: Heal\\
\textbf{Level}: 4, Rare\\
\textbf{Casting Time}: 2 Actions\\
\textbf{Range}: Meter Contact\\
\textbf{Components}: V, S, M (water, salt, sugar)\\
\textbf{Duration}: 1 round per Magic Proficiency\\
The creature affected by this spell recovers one level of fatigue, gains 3d6 temporary Hit Points. He can focus his energies to take an Attack Action with no multiattack penalty or take an extra Move Action.

\medskip\textbf{Forcecage}\index[Spells]{Forcecage}\\
\textbf{School}: Invocation\\
\textbf{Level}: 6, Rare\\
\textbf{Casting Time}: 2 Actions\\
\textbf{Range}: 30 meters\\
\textbf{Component}: V, S, M (ruby dust worth 1,500 gp)\\
\textbf{Duration}: 1 hour\\
An immobile, invisible cubic prison composed of magical force appears around an area you choose within range. The prison can be a cage or a solid box, your choice. A prison in the form of a cage can be 6 meters on a side and made of 10cm bars spaced 1cm apart, providing complete cover for the creatures inside. A box-shaped prison can be 3 meter on a side, creating a solid barrier that prevents any matter from passing through it and blocking any spells cast from inside or outside the area. When you cast this spell, any creature that is completely inside the cage is trapped. Creatures only partially in the cage area, or those too large to fit, are pushed away from the center of the area until they are completely out of it. \\
A creature inside the cage can't leave it by nonmagical means. If the creature tries to use teleportation or interplanar travel to leave the cage, it must first make a Will save. If it succeeds, the creature can use that spell to escape the cage. On a failed save, the creature can't leave the cage and wastes the use of the spell or effect. The cage also extends onto the Ethereal Plane, thus blocking ethereal travel.\\
This spell cannot be dispelled by \hyperlink{dissolvimagie}{Dispel Magic} but only with \hyperlink{dissolvimagieavanzato}{Advanced Dispel Magic}.

\medskip\textbf{Forecast}\index[Spells]{Forecast}\\
\textbf{School}: Divination\\
\textbf{Level}: 9, Uncommon\\
\textbf{Cast Time}: 1 minute\\
\textbf{Range}: Contact\\
\textbf{Components}: V, S, M (a hummingbird feather)\\
\textbf{Duration}: 8 hours\\
You cast the spell while touching a willing creature to grant it a limited ability to see into the immediate future. For the duration, the target cannot be surprised and has +1d6 on attack rolls, ability checks, and Saving Throws. Also, again for the duration, other creatures have -1d6 on attack rolls against the target. The spell immediately ends if you cast it again before its duration ends.

\medskip\textbf{Freedom of Movement}\index[Spells]{Freedom of Movement}\\
\textbf{School}: Abjuration\\
\textbf{Level}: 4, Common\\
\textbf{Casting Time}: 2 Actions\\
\textbf{Range}: Contact\\
\textbf{Components}: V, S, M (a strip of leather, wrapped around an arm or similar appendage)\\
\textbf{Duration}: 1 hour\\
You cast the spell touching a willing creature. For its duration, the target's movement ignores hindering terrain, and spells or other magical effects cannot reduce its speed or cause the target to be paralyzed or entangled. \\
The target can use two Actions to automatically break free of any nonmagical restraints, such as handcuffs or a creature it's grabbed by. Finally, being underwater does not incur penalties to the target's movement or attacks.\\
\textbf{For two magical critical successes rolled} in the Magic Test you can affect another creature.

\medskip\textbf{Freezing Sphere}\index[Spells]{Freezing Sphere}\\
\textbf{School}: Water\\
\textbf{Level}: 6, Rare\\
\textbf{Cast Time}: 2 Actions\\
\textbf{Range}: 90 meters\\
\textbf{Components}: V, S, M (a small crystal ball)\\
\textbf{Duration}: Instant\\
A freezing orb of cold energy shoots from your fingertips to a point of your choice within range, where it explodes in a 20m-radius sphere. Each creature in the area must make a Fortitude save. On a failed save, a creature takes 10d6 cold damage. On a successful save, he takes half as much damage.\\
If the orb strikes a body of water or a liquid composed primarily of water (but excluding water-based creatures), it freezes the liquid to a depth of 10cm in an area 10 meters square. Ice lasts 1 minute. Creatures that were swimming on the surface of frozen water become trapped in the ice. A trapped creature can use two actions to make a new Saving Throw to free itself.\\
You may refrain from firing the orb after completing the spell if you wish. A small orb, about the size of a slingstone, cold to the touch, appears in your hand. At any time, you, or a creature you gave the orb to, can throw the orb (to a range of 13 meters). This will shatter on impact, with the same effect as normal casting of the spell. You can also place the globe on the ground without it shattering. After 1 minute, if the globe hasn't already been shattered, it will explode.\\
\textbf{For each magical critical success rolled} in the Magic Test the damage increases by 1d6.\\
\textbf{Save Critical Success/Failure}: On a critical failure the damage is doubled, on a critical success the damage is further halved

\medskip\textbf{Friendship with Animals}\index[Spells]{Friendship with Animals}\\
\textbf{School}: Animals and Plants\\
\textbf{Level}: 1, Uncommon\\
\textbf{Casting Time}: 2 Actions\\
\textbf{Range}: 9 meters\\
\textbf{Components}: V, S, M (some food)\\
\textbf{Duration}: 24 hours\\
This spell allows you to convince a natural beast that you don't want to harm it. Choose a beast that you can see within range. This must see and hear you. If the beast's Intelligence is 4 or higher, the spell fails. Otherwise, the beast must succeed at a Will save or be charmed by you for the duration of the spell. If you or one of your companions damages the target, the spell ends.\\
\textbf{For each magical critical success you roll} in the Magic Test, you may act on one additional beast.

\medskip\textbf{Gaseous Form}\index[Spells]{Gaseous Form}\\
\textbf{School}: Transmutation\\
\textbf{Level}: 3, Uncommon\\
\textbf{Casting Time}: 2 Actions\\
\textbf{Range}: Contact\\
\textbf{Components}: V, S, M (a piece of gauze and a wisp of smoke)\\
\textbf{Duration}: Concentration, max 1 hour\\
You turn a willing creature, along with whatever it is wearing and carrying, into a vaporous cloud for the duration. The spell ends if the creature drops to 0 Hit Points. Incorporeal creatures ignore this effect. While in this form, the target's only method of movement is a fly speed of 3 meter. The target can enter and occupy another creature's space. The target has resistance to nonmagical damage, and has +1d6 on Fortitude and Reflex saves. The target can pass through small holes, bottlenecks, and even simple holes, although it treats liquids as solid surfaces. The target cannot fall and remains hovering in the air even if stunned or otherwise incapacitated. \\
While in the form of a vaporous cloud, the target cannot speak or manipulate objects, and any objects it is wearing or carrying cannot be thrown, used, or otherwise employed. The target cannot attack or cast spells.\\
\textbf{For every two magical critical successes you roll} in the Magic Test you can affect one other creature.

\medskip\textbf{Geas}\index[Spells]{Geas}\\
\textbf{School}: Enchantment\\
\textbf{Level}: 5, Rare\\
\textbf{Cast Time}: 1 minute\\
\textbf{Range}: 18 meters\\
\textbf{Components}: V\\
\textbf{Duration}: 30 days\\
You impose a magical command on a creature that you can see within range, either forcing it to perform a certain task or forbidding it from performing an action or course of activity you decide. If the creature can understand you, it must succeed at a Will save or be charmed by you for the duration. While the creature is charmed by you, it takes 3d10 points of damage whenever it acts directly contrary to your instructions, but no more than once per day. A creature that can't understand you is unaffected by this spell. You can issue any command of your choosing, except an activity that would result in certain death. Should you utter a suicidal command, the spell will end.\\
You can end the spell as an action. Remove curse, greater restoration, or wish also end it.\\
\textbf{If you get at least two Criticals} in the Magic Test the duration is 1 year. If you get 3 Critics the spell lasts until it is ended by one of the spells mentioned above.

\medskip\textbf{Giant Insect}\index[Spells]{Giant Insect}\\
\textbf{School}: Animals and Plants\\
\textbf{Level}: 4, Uncommon\\
\textbf{Cast Time}: 2 Actions\\
\textbf{Range}: 9 meters\\
\textbf{Components}: V, S\\
\textbf{Duration}: 10 minutes\\
For the duration, you transform up to ten centipedes, three spiders, five wasps, or one scorpion within range into giant versions of their natural forms. A centipede becomes a giant centipede, a spider becomes a giant spider, a wasp becomes a giant wasp, and a scorpion becomes a giant scorpion. Each creature obeys your spoken commands and, in combat, acts each round during your round. The Arbiter owns the statistics of these creatures, and the Arbiter will always resolve their actions and movements. A creature remains in its giant form for the duration, until it drops to 0 Hit Points, or until you use an action to break the effect on it.\\
The Arbiter may allow you to choose different targets. For example, if you transform a bee, its giant version might have the same stats as the giant wasp.

\medskip\textbf{Glyph of Warding}\index[Spells]{Glyph of Warding}\\
\textbf{School}: Abjuration\\
\textbf{Level}: 3, Common\\
\textbf{Casting Time}: 2 Actions\\
\textbf{Range}: Contact\\
\textbf{Components}: V. S, M (incense and diamond powder worth at least 200 gp, which the spell consumes)\\
\textbf{Duration}: Until dispelled or activated \\
When you cast this spell, you inscribe a glyph that harms other creatures on a surface (such as a table or section of floor or wall) or within an object that can be closed (such as a book, scroll, or chest) to hide the glyph. If you choose a surface area, the glyph can cover a surface area no larger than 3 meter in diameter. If you choose an item, that item must stay in place; if the object is moved more than 3 meter from where the spell was cast, the glyph is broken, and the spell ends without being triggered. \\
The glyph is nearly invisible and can be found with an Intelligence check against the DC of your spell's save. You decide what activates the glyph when casting the spell.\\
For glyphs inscribed on a surface, typical activation involves touching or standing over the glyph, removing another object covering the glyph, approaching a certain distance from the glyph, or manipulating the object on which the glyph is inscribed. glyph. For glyphs inscribed on an object, typical activation involves opening the object, approaching a certain distance from the object, or seeing or reading the glyph. Once the glyph has been activated, the spell ends.\\
You can fine-tune the trigger so that the spell only activates under certain circumstances or according to certain physical characteristics (like height or weight), creature species (for example, the ward might work against aberrations or elves obscure), or specific Traits. You can also set conditions to prevent the glyph from being triggered, such as saying a password.\\
When inscribing the glyph, choose blast runes or enchantment glyph.


\medskip

- \textit{Enchantment Glyph}. You can place a prepared spell of level 2 or lower into the glyph by casting it as part of creating the glyph. The spell must target a single creature or an area. The spell being cast has no immediate effect when cast this way. When the glyph is activated, the spell cast is cast. If the spell has a target, it targets the creature that activated the glyph. If the spell affects an area, the area is centered on that creature. If the spell summons hostile creatures or creates harmful objects or traps, they appear as close as possible to the intruder and attack him. If the spell requires concentration, this is maintained until the end of its normal duration. \\

- \textit{Exploding Runes}. When activated, the glyph erupts magical energy in a 6 meters-radius sphere centered on the glyph. The sphere propagates around corners. Each creature in the area must make a Reflex save. A creature takes 5d8 acid, cold, fire, Electricity, fire, or sound damage on a failed save (your choice when you create the glyph), or half as much damage on a successful save. \\
\textbf{For each magical critical success rolled} in the Magic Test, the damage of the blast rune glyph increases by 1d8.


\medskip\textbf{Good Berry}\index[Spells]{Good Berry}\\
\textbf{School}: Animals and Plants\\
\textbf{Level}: 2, Common\\
\textbf{Casting Time}: 2 Actions\\
\textbf{Range}: Contact\\
\textbf{Components}: V, S, M (a sprig of mistletoe and 8 berry)\\
\textbf{Duration}: Instant\\
You enchant up to 2d4 berries in your hand which are infused with magic for the duration. A creature can use 1 immediate action to eat a berry. Eating a berry restores 1 hit point, and the berry also provides enough nourishment to feed a creature for one day. Only the first berry is effective for the day.\\
The berries lose their effectiveness if not consumed within 8 hours of casting the spell. \\
\textbf{For each magical critical success rolled} in the Magic Test the berries last one day longer or you enchant one more berry (up to a maximum total of 8).

\medskip\textbf{Greater Blessing}\index[Spells]{Greater Blessing}\\
\textbf{School}: Invocation\\
\textbf{Level}: 2, Uncommon\\
\textbf{Casting Time}: 1 Minute\\
\textbf{Range}: 18 meters\\
\textbf{Components}: V, S, M (a splash of holy water, 10 gp)\\
\textbf{Duration}: 1 hour\\
Bless a creature of your choice. The creature within the duration can add 1d6 to a roll before knowing whether the check (AR/ST/Check) was successful or not. This bonus can be used 2 times per hour. You must be a Follower or Devotee to cast this spell.\\
\textbf{For each magical critical success you roll} in Magic Test you may add a creature as a target or add one hour to the duration.

\medskip\textbf{Greater Invisibility}\index[Spells]{Greater Invisibility}\\
\textbf{School}: Illusion\\
\textbf{Level}: 4, Uncommon\\
\textbf{Casting Time}: 2 Actions\\
\textbf{Range}: Contact\\
\textbf{Components}: V, S\\
\textbf{Duration}: 1 minute\\
You cast the spell touching a creature. The target becomes invisible until the spell ends. Anything worn or carried by the target becomes invisible while it remains on the target.\\
Casting spells or attacking actions does not cause you to become visible.

\medskip\textbf{Greater Restoration}\index[Spells]{Greater Restoration}\\
\textbf{School}: Heal\\
\textbf{Level}: 5, Uncommon\\
\textbf{Casting Time}: 2 Actions\\
\textbf{Range}: Contact\\
\textbf{Component}: V, S, M (diamond dust worth at least 100 gp, which the spell consumes)\\
\textbf{Duration}: Instant\\
Imbue a creature in contact with positive healing energy to nullify a debilitating effect.

- An effect that has Charmed the target.\\
- Restore 2 points to a stat on the target. You recover 1 point if the loss was permanent.\\
- Maximum Hit Points return to normal, but do not increase current Hit Points.\\
- You are able to relieve Fatigue conditions by two steps\\

At the Arbiter's discretion, if the condition was caused by magic, the Magical Competence value of the caster must exceed the Magical Expertise of the person causing the effect. \\
\textbf{For each Magical Critical Success} obtained in the Magic Test you have a +4 to your Magical Expertise to overcome the opponent's Magical Competence.

\medskip\textbf{Guidance}\index[Spells]{Guidance}\\
\textbf{School}: Divination\\
\textbf{Level}: 0, Common\\
\textbf{Cast Time}: 1 Reaction\\
\textbf{Range}: 3 meters\\
\textbf{Components}: V, S\\
\textbf{Duration}: 1 Round\\
You cast the spell touching a willing creature. Once, before the spell ends, the target can roll a d4 and add the result rolled to one ability check of its choice. He can roll the die before or after making the competence check. The spell then ends. It is not possible to cast Guidance on the same creature every less than 1 hour.

\medskip\textbf{Gust of Wind}\index[Spells]{Gust of Wind}\\
\textbf{School}: Air\\
\textbf{Level}: 2, Common\\
\textbf{Casting Time}: 2 Actions\\
\textbf{Range}: Self (20m line)\\
\textbf{Components}: V, S, M (a legume seed)\\
\textbf{Duration}: Concentration, max 1 minute\\
A line of strong wind 20 meters long and 3 meter wide explodes out from you in a direction of your choice, for the duration. Any creature that starts its round inside the line must succeed at a Fortitude save or be pushed 5 meters away from you, following the direction of the line.\\
Any creature in the line must expend double movement to get close to you.\\
Gust scatters gases or vapors, extinguishes candles, torches, and similar unprotected flames in area. Warded flames, such as those in lanterns, flare, and have a 50\% chance of extinguishing. As 1 action during each of your rounds, before the spell ends, you can change the direction the line projects from you.\\
A thrown weapon that passes through a gust of wind has a 50\% miss.

\medskip\textbf{Hallow}\index[Spells]{Hallow}\\
\textbf{School}: Invocation\\
\textbf{Level}: 5, Rare\\
\textbf{Cast Time}: 24 hours\\
\textbf{Range}: Contact\\
\textbf{Components}: V, S, M (herbs, oils, and incenses worth at least 1000 gp, which the spell consumes)\\
\textbf{Duration}: Until dissolved\\
Infuse the surrounding area with a point you touch with the power of your Patron. The area can have a maximum radius of 20 meters, and the spell fails if it includes an area already under the effect of a sanctify spell. The affected area generates the following effects.
\textit{First things first}, celestials, elementals, fey, demons, and undead cannot enter the area, nor can such a creature charm, frighten, or possess others within it. Any creature charmed, frightened, or possessed by such a creature is no longer charmed, frightened, or possessed the moment it enters this area. You may exclude one or more types of these creatures from this effect.\\
\textit{Second thing}, you can bind an additional effect to the area. Choose the effect from the list below, or choose one presented to you by the Arbiter. Some of these effects apply to creatures in the area; you can decide whether the effects apply to all creatures, creatures Devoted or Followers of a specific Patron, or creatures of a specific type, such as orcs or trolls. When an affected creature enters this area for the first time in a round or begins its round here, it must make a Will save. On a successful save, the creature ignores the additional effect until it leaves the area.\\

- \textit{Courage}. Affected creatures cannot be frightened while in this area. Extradimensional interference. Affected creatures cannot move or travel using teleportation or other extradimensional or interplanar means.\\
\textit{Languages}. Affected creatures can communicate with any other creature in the area, even if they don't share a common language.\\
- \textit{Daylight}. Bright light fills the area. Magical darkness created by spells of lower level than that used to cast this spell cannot extinguish light. In this case the spell last one week.\\
- \textit{Darkness}. Darkness fills the area. Normal light, and even magical light created by spells of a lower level than that used to cast this spell, cannot illuminate the area.\\.
- \textit{Fear}. Affected creatures are frightened while remaining in this area.\\
- \textit{Energy Protection}. Affected creatures receive resistance to a damage type of your choice (except bludgeoning, piercing, or slashing damage), as long as they remain in the area.\\
- \textit{Inviolate Rest}. Dead bodies buried in the area cannot be turned undead.\\
- \textit{Silence}. No sound can emanate from within the area, and no sound can enter it.\\
- \textit{Vulnerability to Energy}. Affected creatures receive vulnerability to a damage type of your choice (except bludgeoning, piercing, or slashing damage) as long as they remain in the area.

\medskip\textbf{Halucination of Death}\index[Spells]{Halucination of Death}\\
\textbf{School}: Illusion\\
\textbf{Level}: 4, Uncommon\\
\textbf{Cast Time}: 2 Actions\\
\textbf{Range}: 36 meters\\
\textbf{Components}: V, S\\
\textbf{Duration}: Instant\\
You draw upon the nightmares of a creature within range and that you can see, and create an illusory manifestation of its deepest fears, visible only to that creature. The target must make a Will save. \\
On a failed save, the target is frightened for 1 minute and takes 4d10 damage. \\
\textbf{Save Critical Success/Failure}: On a critical failure the damage is doubled, on a critical success the damage is further halved\\
\textbf{For each magical critical success rolled} in the Magic Test the damage increases by 2d10

\medskip\textbf{Haste}\index[Spells]{Haste}\\
\textbf{School}: Transmutation\\
\textbf{Level}: 3, Uncommon\\
\textbf{Casting Time}: 2 Actions\\
\textbf{Range}: 9 meters\\
\textbf{Components}: V, S, M (a grating of licorice root)\\
\textbf{Duration}: 1 minute\\
You change the flow of time by speeding it around up to 1d4 creatures in a 6 meters-edge cube within range. Until the spell ends, the targets can perform an additional Attack or Move Action.
This spell counters and is countered by \hyperlink{lentezza}{Slow}.\\
When the spell ends, the targets cannot move or take Actions until their next round, while falling into sudden drowsiness.\\
\textbf{For each magical critical success you roll} in Magic Test, you can affect one more creature.\\
\textbf{For every three magical critical successes you roll} in the Magic Test you can increase your Actions by an additional 1 per round.

\medskip\textbf{Heal}\index[Spells]{Heal}\\
\textbf{School}: Heal\\
\textbf{Level}: 6, Rare\\
\textbf{Casting Time}: 2 Actions\\
\textbf{Range}: 18 meters\\
\textbf{Components}: V, S\\
\textbf{Duration}: Instant\\
Choose a creature that is within range and that you can see. a wave of positive healing energy overwhelms the creature, causing it to recover 70 Hit Points. The spell also ends any blindness, deafness, and disease (including magical) afflicting the target. This spell deals 50 Hit Points of damage to an undead on a touch attack roll. \\
\textbf{For each magical critical success rolled} in Magic Test the amount healed increases by 20.

If the caster and the creature healed are both \textbf{Followers} of the same Patron the spell heals 90 Hit Points.

If the caster and the cured creature are both \textbf{Devotees} of the same Patron the spell regains full Hit Points.

\medskip\textbf{Healing Word}\index[Spells]{Healing Word}\\
\textbf{School}: Heal\\
\textbf{Level}: 1, Uncommon\\
\textbf{Casting Time}: 1 Reaction Action\\
\textbf{Range}: 18 meters\\
\textbf{Components}: V\\
\textbf{Duration}: Instant\\
A creature of your choice that you can see within range regains Hit Points equal to 1d4 + your Characteristic modifier for spellcasting spells. This spell deals the same amount of damage to an undead.\\
\textbf{For each magical critical success rolled} in the Magic Test the healing increases by 1d4.\\
If the caster and the creature healed are both Followers of the same Patron, the spell heals an additional 1d4. \\
If the spellcaster and the creature cured are both Devotees of the same Patron, the spell heals an additional 2d4.

\medskip\textbf{Heat Metal}\index[Spells]{Heat Metal}\\
\textbf{School}: Fire\\
\textbf{Level}: 2, Uncommon\\
\textbf{Cast Time}: 2 Actions\\
\textbf{Range}: 18 meters\\
\textbf{Components}: V, S, M (a piece of iron and a flame)\\
\textbf{Duration}: 1 minute\\
Choose a metal artifact, such as a medium or heavy metal weapon or metal Armour, that is within range and that you can see. Causes the object to glow red from the heat. Any creature in physical contact with the item takes 1d8 fire damage when you cast this spell. Until the spell ends, you can use 2 Actions to deal this damage again in your next rounds.\\
If a creature is holding or wearing the item and takes damage from it, the creature must succeed at a Fortitude save or discard the item if able. If he doesn't throw the object, he has -1d6 on attack rolls and ability checks until the start of his next round. As long as the creature is beyond 18 meters from the caster, the spell does not end but the object ceases to be hot. \\
\textbf{For each magical critical success rolled} in the Magic Test the damage increases by 1d8.

\medskip\textbf{Aid}\index[Spells]{Aid}\\
\textbf{School}: Heal, Necromancy\\
\textbf{Level}: 2, Uncommon\\
\textbf{Cast Time}: 2 Actions\\
\textbf{Range}: 9 meters\\
\textbf{Components}: V, S, M (a thin strip of white fabric)\\
\textbf{Duration}: 8 hours\\
Your spell increases the toughness and resolve of your allies. Choose up to three creatures within range. For the duration, each target's maximum Hit Points and current Hit Points increase by 5.\\
\textbf{For each magical critical success rolled} in the Magic Test the target's Hit Points are increased by an additional 5 points

\medskip\textbf{Heroes' Feast}\index[Spells]{Heroes' Feast}\\
\textbf{School}: Summon\\
\textbf{Level}: 6, Uncommon\\
\textbf{Cast Time}: 10 minutes\\
\textbf{Range}: 9 meters\\
\textbf{Components}: V, S, M (a gem-encrusted bowl worth at least 500 gp, which the spell consumes)\\
\textbf{Duration}: Instant\\
You create a magnificent banquet, including delicious food and drink. The feast is consumed in 1 hour and disappears at the end of this period, but the beneficial effects are not felt until the end of the hour. Up to twelve other creatures can
attend the banquet. A creature that participates in the feast gains several benefits. The creature is cured of all diseases and poisons, becomes immune to poison and being frightened, and has +2d6 on all Will saves. His maximum Hit Points increase by 2d10, and he heals the same amount as his current Hit Points. These benefits last for 24 hours.\\
\textbf{On two magical critical successes rolled} in the Magic Test the bowl is not consumed.

\medskip\textbf{Heroism}\index[Spells]{Heroism}\\
\textbf{School}: Enchantment\\
\textbf{Level}: 1, Uncommon\\
\textbf{Casting Time}: 2 Actions\\
\textbf{Range}: Contact\\
\textbf{Components}: V, S\\
\textbf{Duration}: 1 minute\\
A willing creature you touch is imbued with courage. Until the spell ends, the creature is immune to being frightened and, at the start of each of its rounds, gains temporary Hit Points equal to your spell modifier. When the spell ends, the target loses any remaining temporary Hit Points derived from this spell. \\
\textbf{For each magical critical success you roll} in the Magic Test you can affect another creature.

\medskip\textbf{Holy Aura}\index[Spells]{Holy Aura}\\
\textbf{School}: Abjuration\\
\textbf{Level}: 8, Common\\
\textbf{Casting Time}: 2 Actions\\
\textbf{Range}: Personal\\
\textbf{Components}: V, S, M (a tiny reliquary worth at least 1000 gp containing a sacred relic, such as a piece of cloth from a Devotee's robe or a piece of parchment from a religious text) \\
\textbf{Duration}: Concentration, 1 minute\\
You radiate divine light from you that gathers in a faint 10m-radius glow around you. When you cast the spell, creatures you choose within this ray shed dim light in a 1m radius and have{+2d6} on all Saving Throws, while other creatures have{-2d6} on attack rolls against you. them until the spell ends. Additionally, when a demon or undead strikes a target creature with a melee attack, the aura glows with a bright light and it must succeed at a Fortitude save or be blinded until the spell ends.

\medskip\textbf{Holy Flame}\index{Cantrip - Holy Flame}\\
\textbf{School}: Universal\\
\textbf{Level}: 0, Common\\
\textbf{Casting Time}: 1 Action\\
\textbf{Range}: 18 meters\\
\textbf{Components}: V, S\\
\textbf{Duration}: Instant\\
A torch-like luminosity descends upon a creature you can see within range. The target must succeed at a Reflex save or take 1d8 light damage. The target does not receive the benefit of cover for this Saving Throw. \\
The spell's damage increases by 1d8 when the sum of Traits in common with the Patron reaches 5, 11 and 17, but it costs 2 Actions to cast it empowered and 2 Spell Points.\\
\textbf{For every two magical critical successes rolled} in the Magic Test an additional flame descends which must strike a different target within range.

\medskip\textbf{Hut}\index[Spells]{Hut}\\
\textbf{School}: Invocation\\
\textbf{Level}: 3, Uncommon\\
\textbf{Launch Time}: 1 minute\\
\textbf{Range}: Personal (3 meter radius hemisphere)\\
\textbf{Components}: V, S, M (a diamond sliver worth 50 gp that the spell consumes)\\
\textbf{Duration}: 8 hours\\
A 10-foot radius half-sphere of motionless force forms around and above you, remaining stationary for the duration. The spell ends if you leave the area. Eight creatures of Medium or smaller size can enter the dome with you. The spell fails if the area includes a larger creature or more than nine creatures. Creatures and objects inside the dome can pass through it freely when you cast this spell. All other creatures and objects must make a Fortitude saving throw or be unable to pass through it that round. Spells and other magical effects can extend beyond the dome or pass through it if they are not cantrips. The atmosphere inside the space is comfortable and dry whatever the climate outside.\\
Until the spell ends you can command the interior lighting to be full, dim, or dark. The dome is opaque from the outside, any color you choose, but is transparent from the inside.\\
\textbf{For each Magical Critical Success} obtained in the Magic Test, the spell lasts 2 hours longer.

\medskip\textbf{Hypnotic Texture}\index[Spells]{Hypnotic Texture}\\
\textbf{School}: Illusion\\
\textbf{Level}: 3, Common\\
\textbf{Casting Time}: 2 Actions\\
\textbf{Range}: 36 meters\\
\textbf{Components}: S, M (a glowing incense stick or crystal vial filled with glow-in-the-dark material)\\
\textbf{Duration}: 1 minute\\
You create a twisted web of color that moves through the air within a 10m cube within range. The texture appears for a moment and then fades away. Each creature in the area that sees the pattern must make a Will save. On a failed save, a creature is charmed for the duration. While charmed by this spell, the creature is incapacitated and has 0 speed. The spell ends for the subject creature if it takes damage or if someone uses an action to jolt it out of its dazed state.

\medskip\textbf{Ice Storm}\index[Spells]{Ice Storm}\\
\textbf{School}: Water, Air\\
\textbf{Level}: 4, Uncommon\\
\textbf{Casting Time}: 2 Actions\\
\textbf{Range}: 90 meters\\
\textbf{Components}: V, S, M (a pinch of powder and a few drops of water)\\
\textbf{Duration}: Instant\\
A hail of ice slams the ground in a 6 meters-radius, 12m high cylinder centered on a point within range. Each creature in the cylinder must make a Reflex save. The creature takes 2d8 bludgeoning damage and 4d6 cold damage on a failed save, or half as much on a successful one. The hail turns the storm's area of effect into hindering terrain until the end of your next round.\\
\textbf{For each magical critical success rolled} in the Magic Test the damage increases by 2d8.\\
\textbf{Save Critical Success/Failure}: On a critical failure the damage is doubled, on a critical success the damage is further halved

\medskip\textbf{Identify}\index[Spells]{Identify}\hypertarget{incantesimoidentificare}{}\\
\textbf{School}: Universal\\
\textbf{Level}: 1, Common\\
\textbf{Cast Time}: 1 minute\\
\textbf{Range}: Contact\\
\textbf{Components}: V, S, M (a gem worth at least 10 gp and an owl feather that the spell consumes)\\
\textbf{Duration}: Instant\\
Choose an object that you must remain in contact with throughout the entire casting of the spell. If it is a magic object or other object imbued with magic, make a check of Arcana at DC 30 with a +2d6 bonus, if you succeed you learn its properties and how to use them and how many charges it has, if any. \\
You learn if any spells are affecting the item and what they are. If the item was created by a spell, you learn which spell created it. If, on the other hand, you remain in contact with a creature during the execution, you learn if any spells are acting on it and what they are.\\
\textbf{Only if you roll a magical Critical Success} learn if the item is \hyperlink{oggettimaledettiid}{cursed}.

\medskip\textbf{Illusionary Terrain}\index[Spells]{Illusionary Terrain}\\
\textbf{School}: Illusion\\
\textbf{Level}: 4, Uncommon\\
\textbf{Casting Time}: 10 minutes\\
\textbf{Range}: 90 meters\\
\textbf{Components}: V, S, M (a stone, a twig and a piece of green plant)\\
\textbf{Duration}: 24 hours \\
You make a piece of natural terrain within range, in a 150 meters cube, look, sound, and smell like some other type of natural terrain. As a result, open fields or a road can be turned into a swamp, hills, a crevasse, or some other type of difficult or impassable terrain. A pond can be transformed into a grassy clearing, a precipice into a gentle slope, a rock-strewn ravine into a wide, smooth road. Built structures, equipment, and creatures within the area do not change in appearance.\\
The tactile characteristics of the terrain are unchanged, so creatures entering the area are likely to reveal the illusion. If the difference isn't obvious to touch, a creature warily examining the illusion can attempt an Intelligence (Investigation) check against your spell's save DC to disbelieve it. A creature that recognizes the illusion for what it is perceives it as a vague image superimposed on the ground.

\medskip\textbf{Illusory Written}\index[Spells]{Illusory Written}\\
\textbf{School}: Illusion\\
\textbf{Level}: 1, Common\\
\textbf{Cast Time}: 1 minute\\
\textbf{Range}: Contact\\
\textbf{Components}: S, M (a lead-based ink worth at least 10 gp, which the spell consumes)\\
\textbf{Duration}: 10 days\\
You write on a scroll, piece of paper, or some other writing material and imbue it with a powerful illusion that lasts for the spell's duration.\\
To you and any creatures you designate when casting the spell, the writing appears normal, in your handwriting, and conveys whatever meaning you intended to convey when you wrote the text. To all others, the writing appears as if written in an unknown or magical script, which is incomprehensible. Alternatively, you can make the writing look like an entirely different message, in a different handwriting and language, although it must be a language you are familiar with.\\
Should the spell be dispelled, both the original writing and the illusion vanish. A creature with true seeing can read the hidden message.

\medskip\textbf{Imprison}\index[Spells]{Imprison}\\
\textbf{School}: Abjuration\\
\textbf{Level}: 9, Rare\\
\textbf{Cast Time}: 2 Actions\\
\textbf{Range}: 9 meters\\
\textbf{Component}: V, S, M (a fleece depiction or figurine engraved with the target's features, and a special component that varies depending on which version of the spell you choose, worth at least 500 gp per Target's Hit Die)\\
\textbf{Duration}: Until dispelled\\
You create magical bonds to hold a creature that is within range and that you can see. The target must succeed at a Will save or be bound by the spell; on a successful save, he is immune to the spell if he casts it again. While affected by this spell, the creature doesn't need to breathe, eat, or drink, and it doesn't age. Divination spells cannot locate or sense the target.\\
When you cast this spell, choose one of the following forms of imprisonment.\\


- \textit{Chain}. Heavy chains, well welded to the ground, keep the target anchored. The target is restrained until the spell ends, and cannot move or be moved in any way until then. The special component for this version of the spell is a precious metal chain.\\
- \textit{Minimum Insulation}. The target shrinks to 1cm in height and is encased in a gem or similar object. Light can pass through the gem normally (allowing the target to see out and other creatures to see in), but nothing else can pass through it, not even by teleportation or planar travel. The gem cannot be cut or broken as long as the spell is in effect. The special component for this version of the spell is a large clear gem, such as corundum, diamond, or ruby.\\
- \textit{Confined Prison}. The spell transports the target to a tiny demiplane that is barred from teleportation and planar travel. The demiplane can be a labyrinth, cage, tower, or any other enclosed structure you choose. The special component for this version of the spell is a miniature representation of the prison made of jade.\\
- \textit{Burial}. The target is buried deep within the earth in a sphere of magical force large enough to contain the target. Nothing can pass through the sphere, nor can any creature teleport or use planar travel to enter or exit it. The special component for this version of the spell is a small sphere of mithral.\\
- \textit{Drowsiness}. The target falls asleep and cannot be awakened. The special component for this version of the spell consists of rare soporific herbs.\\

\medskip
\textit{\textbf{End the spell}}. When casting the spell, in any of its versions, you can specify a condition that would end the spell and free the target. The condition can be as specific or elaborate as you like, but the Arbiter must agree that the condition is reasonable and can come true. Conditions can be based on a creature's name, identity, or Patron, but still based on perceivable actions or qualities, not intangible things like level, Feats, or Hit Points.\\
A dispel spell can end the spell only when cast by a character with at least 18 Spell-Like Proficiency, targeting the dungeon or the material component used to create it.\\
You can use a particular special component to create only one dungeon at a time. If you cast the spell again using the same component, the target of the spell's first casting is immediately freed from its bond.

\medskip\textbf{Inflict Wounds}\index[Spells]{Inflict Wounds}\\
\textbf{School}: Necromancy\\
\textbf{Level}: 1, Common \\
\textbf{Cast Time}: 2 Actions\\
\textbf{Range}: Contact\\
\textbf{Components}: V, S\\
\textbf{Duration}: Instant\\
Make a melee spell attack against a creature within reach. On a hit, the target takes 3d10 void damage, Fortitude save for half.\\
\textbf{For each magical critical success rolled} in the Magic Test the damage increases by 1d8.

\medskip\textbf{Insect Plague}\index[Spells]{Insect Plague}\\
\textbf{School}: Animals and Plants\\
\textbf{Level}: 5, Rare\\
\textbf{Cast Time}: 2 Actions\\
\textbf{Range}: 90 meters\\
\textbf{Components}: V, S, M (a few grains of sugar, a few grains of wheat, a little lard)\\
\textbf{Duration}: 10 minutes\\
A swarm of hungry locusts fills a 6 meters-radius sphere centered on a point you choose within range. The sphere propagates around corners. The sphere remains for the duration, and its area is in dim light. The area of the sphere is difficult terrain.\\
When the area appears, each creature in it must make a Fortitude save. A creature takes 4d10 damage on a failed save, or half as much damage on a successful one. A creature must also make this Saving Throw when it enters the spell's area for the first time in a round, or if it ends its round therein. \\
\textbf{For each magical critical success rolled} in the Magic Test the damage increases by 2d8.

\medskip\textbf{Instant Summons}\index[Spells]{Instant Summons}\\
\textbf{School}: Summon\\
\textbf{Level}: 6, Rare\\
\textbf{Cast Time}: 1 minute\\
\textbf{Range}: Contact\\
\textbf{Components}: V, S, M (a sapphire worth 1000 gp)\\
\textbf{Duration}: Until dispelled \\
You come into contact with an object weighing 5 kilos or less and whose largest dimension does not exceed 180 centimeters. The spell leaves a mark on the object's surface and invisibly inscribes its name on the sapphire used as a material component. Each time you cast this spell, you must use a different sapphire.\\
At any time thereafter, you can use 2 Actions to speak the item's name and shatter the sapphire. The object appears instantly in your hand regardless of the physical or planar distance separating you, and the spell ends.\\
If another creature is holding or carrying the item, shattering the sapphire will not carry the item to you, but instead you will learn who the creature holding it is and roughly where it is at the moment.\\
Dispel magic, or a similar effect successfully applied to the sapphire, ends the spell's effect.

\medskip\textbf{Intermittent}\index[Spells]{Intermittent}\\
\textbf{School}: Transmutation\\
\textbf{Level}: 3, Uncommon\\
\textbf{Cast Time}: 2 Actions\\
\textbf{Range}: Personal\\
\textbf{Components}: V, S\\
\textbf{Duration}: 1 round per Magic Proficiency\\
Roll a 1d6 at the end of each of your rounds for the duration of this spell. If you roll an odd number, you vanish from your current plane of existence and reappear on the Ethereal Plane (the spell fails and the casting is wasted if you were already on that plane). At the start of your next round, and when the spell ends, if you were on the Ethereal Plane, you return to an unoccupied space of your choice that you can see, within 3 meter of the space you vanished from. If no unoccupied spaces are available within this range, you appear in the nearest unoccupied space (determined randomly if more than one space is available). You can break the spell with an action.\\
While on the Ethereal Plane, you can see and hear the plane you came from, which you perceive in shades of gray, but you still cannot perceive anything more than 20 meters away. You can only interact with creatures that are on the Ethereal Plane. Creatures not there can neither sense you nor interact with you unless they have the ability to do so.

\medskip\textbf{Inviolate Repose}\index[Spells]{Inviolate Repose}\\
\textbf{School}: Necromancy\\
\textbf{Level}: 2, Uncommon\\
\textbf{Casting Time}: 2 Actions\\
\textbf{Range}: Contact\\
\textbf{Components}: V, S, M (a pinch of salt and a piece of copper placed on each eye of the corpse, which must remain there for the duration)\\
\textbf{Duration}: 10 days\\
You come into contact with a corpse or other remains. For the duration, the target is protected from the rot and cannot become undead. \\
\textbf{For each magical critical success rolled} in the Magic Test you double the duration up to a maximum of one year.

\medskip\textbf{Invisibility}\index[Spells]{Invisibility}\\
\textbf{School}: Illusion\\
\textbf{Level}: 2, Common\\
\textbf{Cast Time}: 2 Actions\\
\textbf{Range}: Contact\\
\textbf{Components}: V, S, M (an eyelash wrapped in gum arabic)\\
\textbf{Duration}: 1 minute per Magic Proficiency\\
You cast the spell touching a creature. The target becomes invisible until the spell ends. Whatever the target is wearing or carrying becomes invisible while it remains on the target. The spell ends for the target who attacks or casts a spell. \\
\textbf{For each magical critical success you roll} in Magic Test, you may choose an additional target creature or double the duration.

\medskip\textbf{Invisible Cook}\index[Spells]{Invisible Cook}\\
\textbf{School}: Summon\\
\textbf{Level}: 1, Common\\
\textbf{Casting Time}: 2 Actions\\
\textbf{Range}: 18 meters\\
\textbf{Components}: V, S, M (a wooden spoon and a few drops of olive oil, the food you want to cook)\\
\textbf{Duration}: 2 hours\\
This spell creates an almost invisible force only bordered by a slight aura (color of your choice) capable and proficient in cooking. Together with the cook there is also a set of pots and pans as well as crockery and a small camp stove.\\
Based on the ingredients available or herbs and vegetables within a radius of 100 meters (the cook does not go hunting) the cook will cook the best of the ingredients by preparing excellent food for up to 4 people. The spell does not create food or water, this must be available when the spell is cast. \\
Once the ingredients are available within two hours, the invisible cook will prepare the food. It is also possible to hasten the execution but at the expense of quality.\\
None of the pots, pans or fires can be used except by the invisible cook.\\
\textbf{If he rolls two magical critical success} the Cook is summoned with food needed to feed 2 people

\medskip\textbf{Invisible Servant}\index[Spells]{Invisible Servant}\\
\textbf{School}: Summon\\
\textbf{Level}: 1, Common\\
\textbf{Cast Time}: 2 Actions\\
\textbf{Range}: 18 meters\\
\textbf{Components}: V, S, M (a piece of string and a piece of wood)\\
\textbf{Duration}: 1 hour\\
This spell creates a nearly invisible force bordered only by a faint aura (the color of your choice) that performs simple tasks at your command, until the spell ends. The minion forms in an unoccupied field space, within range. He has Defence 10, 1 hit point, Strength 0 and can't attack. If he drops to 0 Hit Points, the spell ends.\\
As an immediate action, during each of your rounds, you can mentally command the minion to move up to 5 meters and interact with an object. The servant can perform simple tasks like a human servant, such as gathering things, cleaning, repairing, folding clothes, lighting fires, serving food, and pouring wine. Once the command is given, the minion will do the task to the best of his ability until he completes it, and then wait for your next command. \\
If you command the minion to perform a task that would cause it to move more than 20 meters away from you, the spell ends.

\medskip\textbf{Irresistible Dance}\index[Spells]{Irresistible Dance}\\
\textbf{School}: Enchantment\\
\textbf{Level}: 8, Legendary\\
\textbf{Casting Time}: 2 Actions\\
\textbf{Range}: 9 meters\\
\textbf{Components}: V\\
\textbf{Duration}: 1 minute\\
Choose a creature that is within range and that you can see. The target begins a comical dance on the spot: swinging his legs, stamping his feet, and hopping for the duration. Creatures that can't be charmed are immune to this spell.\\
A dancing creature must use 2 move actions to dance without leaving its space, and has -1d6 on Reflex saves and attack rolls. While the target is affected by this spell, other creatures have +1d6 on attack rolls against it. By spending 2 Actions, the dancing creature can make a new Will save to regain control of itself. If she succeeds, the spell ends. While dancing he considers himself Distracted\\
\textbf{If you roll 2 magical critical successes} the duration increases by 1 hour

\medskip\textbf{Jump}\index[Spells]{Jump}\\
\textbf{School}: Air\\
\textbf{Level}: 1, Common\\
\textbf{Cast Time}: 2 Actions\\
\textbf{Range}: Contact\\
\textbf{Components}: V, S, M (the hind leg of a grasshopper)\\
\textbf{Duration}: 1 minute\\
The leap distance of the creature you are in contact with when cast is tripled until the spell ends.

\medskip\textbf{Knowledge of Legends}\index[Spells]{Knowledge of Legends}\\
\textbf{School}: Divination\\
\textbf{Level}: 5, Common\\
\textbf{Cast Time}: 10 minutes\\
\textbf{Range}: Personal\\
\textbf{Components}: V, S, M (incense worth at least 250 gp, which the spell consumes, and four ivory strips worth at least 50 gp)\\
\textbf{Duration}: Instant\\
Name or describe a person, place or object. The spell brings to your mind a brief summary of the most important knowledge on the subject you named. If the thing you name has no legendary relevance, you get no information. The more information you have on the subject, the more precise and detailed the information you will receive. The information you receive will be accurate, but maybe hidden in metaphorical language.

\medskip\textbf{Kyrin Lemon Hail}\index[Spells]{Kyrin Lemon Hail}\\
\textbf{School}: Animals and Plants, Earth\\
\textbf{Level}: 3, Rare\\
\textbf{Casting Time}: 2 Actions\\
\textbf{Range}: 30 meters\\
\textbf{Components}: V, S, M (at least 9 drops of lemon, one bottle)\\
\textbf{Duration}: 1 round per Magic Proficiency, Concentration\\
Enchant a bottle with at least 9 drops of lemon inside.
Each round, by spending 1 Action, you can spray up to 2 drops of lemon, of the total 9, against one or more targets within 30 meters. Make a single Attack Roll with ranged spells, with a bonus equal to the times you took Animals and Plants or Earth Lists, per target regardless of the number of drops you roll at it. Each drop on a hit does 1d6+1 acid damage.\\
\textbf{For each magical critical success you roll} in the Magic Test you may create two extra lemon drop.

\medskip\textbf{Kyrin Acorn Hail}\index[Spells]{Kyrin Acorn Hail}\\
\textbf{School}: Animals and Plants\\
\textbf{Level}: 2, Uncommon\\
\textbf{Casting Time}: 1 Action\\
\textbf{Range}: 50 meters\\
\textbf{Components}: V, S, M (9 acorns that are consumed, a piece of rubber)\\
\textbf{Duration}: 1 minute per Magic Proficiency, Concentration\\
You enchant 9 acorns of magical energy and they begin to swirl 15 centimeters above your shoulder.
Each round, by spending 1 Action, you can throw up to 5 acorns at one or more targets. Make only one Attack Roll with ranged spells, with a bonus equal to the number of times you have taken Lists of Animals and Plants, per target regardless of the number of acorns you throw at it. Each acorn does 1d4 bludgeoning damage on a hit.\\
\textbf{For each magical critical success you roll} in the Magic Test, you may enchant two more acorns.

\medskip\textbf{Kyrin Chestnut Hail}\index[Spells]{Kyrin Chestnut Hail}\\
\textbf{School}: Animals and Plants\\
\textbf{Level}: 5, Very Rare\\
\textbf{Casting Time}: 1 Action\\
\textbf{Range}: 60 meters\\
\textbf{Components}: V, S, M (9 browns that are consumed, a piece of rubber)\\
\textbf{Duration}: 1 minute per Magic Proficiency, Concentration\\
You cast 9 browns of magical energy and they begin swirling 60cm above your shoulder.
Each round, by spending 1 Action, you can throw up to 5 browns at one or more targets. Make only one Attack Roll with ranged spells, with a bonus equal to the number of times you have taken Lists of Animals and Plants, per target regardless of the number of acorns you throw at it. Each acorn does 2d8+4 bludgeoning damage if it hits\\
\textbf{For each magical critical success you roll} in the Magic Test you can enchant two more browns.

\medskip\textbf{Kyrin Currant Juice Concentrate}\index[Spells]{Kyrin Currant Juice Concentrate}\\
\textbf{School}: Animals and Plants, Earth\\
\textbf{Level}: 2, Uncommon\\
\textbf{Casting Time}: 2 Actions\\
\textbf{Range}: 9 meters\\
\textbf{Components}: V, M (12 currants that the spell consumes)\\
\textbf{Duration}: 1 minute \\
Extract the acidic sap from currants and project a line of spray of acid 30 feet long and 3 feet wide in a direction of your choice. Each creature in the line must succeed on a Reflex save or be covered in acid for the duration of the spell or until a creature uses two Actions to wash the acid off itself or another creature. A creature covered in acid takes 2d4 points of acid damage at the start of each of its rounds.\\
\textbf{For each magical critical success gained} in the Magic Test, the damage increases by 2d4


\medskip\textbf{Kyrin Fire Acorn Hail}\index[Spells]{Kyrin Fire Acorn Hail}\\
\textbf{School}: Animals and Plants, Fire\\
\textbf{Level}: 3, Rare\\
\textbf{Casting Time}: 2 Action\\
\textbf{Range}: 50 meters\\
\textbf{Components}: V, S, M (9 acorns that are consumed, a piece of rubber)\\
\textbf{Duration}: 1 minute per Magic Proficiency, Concentration\\
You enchant 9 acorns of magical energy and they begin to swirl 30cm above your shoulder.
Each round, by spending 1 Action, you can throw up to 5 acorns at one or more targets. Make a single attack roll with ranged spells, with a bonus equal to the number of times you have taken Lists of Animals and Plants or Fire, per target regardless of the number of acorns you throw at it. Each acorn does 1d4 bludgeoning damage + 1d4 fire damage if it hits.\\
\textbf{For each magical critical success you roll} in the Magic Test, you can enchant two more acorns.

\medskip\textbf{Kyrin reading the land}\index[Spells]{Kyrin reading the land}\index{Ecolocation}\\
\textbf{School}: Earth\\
\textbf{Level}: 2, Uncommon\\
\textbf{Cast Time}: 1 Round\\
\textbf{Range}: Self (30m radius)\\
\textbf{Components}: V, S\\
\textbf{Duration}: Instant\\
You place your hands on the earth and once the spell has been formulated you have a fleeting vision of the environment around you within a spherical radius of 30 meters.
You can sense the position and relative shape of creatures and structures that rest on the ground.\\
\textbf{For each magical critical success rolled} in the Magic Test the radius increases by 10 meters.

\medskip\textbf{Lesser Restoration}\index[Spells]{Lesser Restoration}\\
\textbf{School}: Heal\\
\textbf{Level}: 2, Common\\
\textbf{Cast Time}: 2 Actions\\
\textbf{Range}: Contact\\
\textbf{Components}: V, S\\
\textbf{Duration}: Instant\\
You can end a nonmagical disease or condition that afflicts a creature you are in contact with. The condition can be blinded, deafened or paralyzed. Can reduce your Fatigue level by one degree. You recover 2d6 maximum hit points lost, but do not increase your current hit points. You can recover 1 lost Characteristic point non-permanently. It is not possible to use more than one Lower Restaurant per day.
At the Arbiter's discretion, if the condition was caused by magic, the Magical Competence value of the caster must exceed the Magical Competence of the person causing the effect. \\
\textbf{For each Magical Critical Success} obtained in the Magic Test you have a +4 to your Magical Competence to overcome the opponent's Magical Competence.

\medskip\textbf{Levitation}\index[Spells]{Levitation}\\
\textbf{School}: Air\\
\textbf{Level}: 2, Common\\
\textbf{Casting Time}: 2 Actions\\
\textbf{Range}: 18 meters\\
\textbf{Components}: V, S, M (either a small leather thong or a piece of gold wire bent into a cup shape with a long stem at the end)\\
\textbf{Duration}: 10 minutes \\
A creature or object of your choice that you can see within range rises vertically up to 6 meters and remains suspended for the duration. The spell can levitate a target weighing up to 120kg. An unwilling creature that makes a successful Fortitude save ignores the effect.\\
The target can only move by pushing or pulling towards a fixed object or surface within reach (such as a wall or ceiling). During your round, you can change the target's altitude up to 6 meters in either direction. If you are the target, you can move up or down as part of your move. Otherwise you can use 1 Action to move the target, which must remain within range of the spell. When the spell ends, if still in the air, the target floats gently to the ground.\\
While under the influence of this spell you are considered distracted when casting spells.\\
\textbf{For each magical critical success you roll} in the Magic Test you can move 1 meter to the side or affect another creature.

\medskip\textbf{Light}\index[Spells]{Light}\\
\textbf{School}: Universal\\
\textbf{Level}: 1, Common\\
\textbf{Casting Time}: 2 Actions\\
\textbf{Range}: Contact\\
\textbf{Components}: V, M (a firefly or glow in the dark moss)\\
\textbf{Duration}: 30 minutes of real game time\\
You cast the spell touching an object no larger than 3 meter in any direction. Until the spell ends, the item sheds bright light in a 3m radius and dim light for an additional 6 meter. The light can be any color you like. Covering the object completely with something opaque blocks the light. If a target item is held or worn by a hostile creature, that creature must succeed at a Reflex save to avoid the spell. A creature affected by the spell must make a Fortitude save or be blinded until the end of the next round. You cannot have more than one Light spell active at a time, a subsequent casting extinguishes the previous Light. \\
\textbf{For each Critical gained} in the Magic Test, the duration doubles.

\medskip\textbf{Lightning Bolt}\index[Spells]{Lightning Bolt}\\
\textbf{School}: Air\\
\textbf{Level}: 3, Common\\
\textbf{Cast Time}: 2 Actions\\
\textbf{Range}: Self (30m line)\\
\textbf{Components}: V, S, M (a piece of fur and a rod of amber, crystal or glass)\\
\textbf{Duration}: Instant\\
You explode lightning that forms a line 30 meters long and 1 meter wide starting from where you are in a direction of your choosing. Each creature in the line must make a Reflex save. The creature takes 8d6 Electricity damage on a failed save, or half as much damage on a successful one. \\
Lightning ignites flammable objects in the area that are not being worn or carried.\\
Lightning when thrown at worked hardstone bounces at an angle of 180 degrees - the angle of entry. Lightning thrown into water creates a 3m-radius sphere of electricity where it enters.\\
\textbf{For each magical critical success rolled} in the Magic Test the damage increases by 3d6.\\
\textbf{Save Critical Success/Failure}: On a critical failure the damage is doubled, on a critical success the damage is further halved

\medskip\textbf{Locate Animals and Plants}\index[Spells]{Locate Animals and Plants}\\
\textbf{School}: Animals and Plants\\
\textbf{Level}: 2, Uncommon\\
\textbf{Casting Time}: 2 Actions\\
\textbf{Range}: Personal\\
\textbf{Components}: V, S, M (a piece of hound fur) \\
\textbf{Duration}: Instant\\
Describe or name a specific type of beast or plant. By focusing on the voice of nature in your surroundings, you learn the direction and distance to the nearest creature or plant of that species, if any within 7.5 kilometers.\\
\textbf{For each Magical Critical Success rolled} in the Magic Test you increase the controlled area by 1km

\medskip\textbf{Locate Creature}\index[Spells]{Locate Creature}\\
\textbf{School}: Divination\\
\textbf{Level}: 4, Common\\
\textbf{Casting Time}: 2 Actions\\
\textbf{Range}: Personal\\
\textbf{Components}: V, S, M (a piece of hound fur)\\
\textbf{Duration}: Concentration, max 1 hour\\
Describe or name a creature that is familiar to you. You sense the direction of the creature's location, as long as that creature is within 300 meters of you. If the creature moves, you also know the direction of its movement.\\
The spell can locate a specific creature known to you, or the closest creature of a species (such as a human or unicorn), provided you have seen such a creature up close (within 10 meters) at least once. If the creature you describe or name has a different form, for example it is under the effects of the polymorph spell, this spell will not be able to locate the creature.\\
This spell cannot locate a creature if a stream of running water at least 3 meter wide blocks a direct path between you and the creature. \\
\textbf{For each magical critical success rolled} in the Magic Test increases the distance by another 300m.

\medskip\textbf{Locate Object}\index[Spells]{Locate Object}\\
\textbf{School}: Divination\\
\textbf{Level}: 2, Common\\
\textbf{Casting Time}: 2 Actions\\
\textbf{Range}: Personal\\
\textbf{Components}: V, S, M (a forked twig)\\
\textbf{Duration}: Concentration, max 10 minutes \\
Describe or name an object that is familiar to you. You sense the direction of the object's location, as long as that object is within 300 meters of you. If the object moves, you also know the direction of its movement.\\
The spell can locate a specific object known to you, provided you have seen it up close (within 10 meters) at least once. Alternatively, the spell can locate the nearest object of a particular type, such as certain types of clothing, jewelry, furniture, tools, or weapons.\\
This spell cannot locate an object if any thickness of lead, even a thin sheet, blocks a direct path between you and the object.\\
\textbf{For each Magical Critical Success rolled} in Magic Test you double the duration.

\medskip\textbf{Seem}\index[Spells]{Seem}\\
\textbf{School}: Illusion\\
\textbf{Level}: 5, Uncommon\\
\textbf{Casting Time}: 2 Actions\\
\textbf{Range}: 9 meters\\
\textbf{Components}: V, S\\
\textbf{Duration}: 8 hours\\
This spell allows you to change the appearance of any number of creatures within range and that you can see. Give each target a new illusory appearance. An unwilling creature can make a Will save and ignore the spell on a successful one.\\
The spell disguises physical appearance as well as clothing, Armour, weapons, and equipment. You can make each creature appear a 0.5m shorter or taller, appear thin, fat, or somewhere in between. You can't change the shape of the target's body, and so you must choose a shape that has the same basic distribution of limbs. \\
For everything else, the illusion is limited only by your imagination. The spell lasts for its duration, unless you use an action to end it sooner. Changes made by this spell are unable to withstand physical inspection. For example, if you use this spell to add a hat to a creature's clothing, objects pass through the hat, and anyone touching it would feel nothing and would end up touching the creature's head and hair.
If you use this spell to appear thinner than you are, a person's hand attempting to touch you will bounce off you, while appearing to stop in mid-air to the eye. A creature can use 2 Actions to inspect a target and make an Awareness check against the spell's save DC, if it takes 3 Actions it has a +1d6 bonus. If successful, she realizes the target is disguised.

\medskip\textbf{Loquaciousness}\index[Spells]{Loquaciousness}\\
\textbf{School}: Transmutation\\
\textbf{Level}: 8, Rare\\
\textbf{Cast Time}: 2 Actions\\
\textbf{Range}: Personal\\
\textbf{Components}: V\\
\textbf{Duration}: 1 hour\\
Until the spell ends, when you make a Charisma-based check, you can replace the number rolled with 15. Also, no matter what you say, the magic or analysis that determines whether you are telling the truth will always indicate that you are being honest.\\
\textbf{For each Magical Critical Success rolled} in Magic Test you double the duration.

\medskip\textbf{Loyal Hound}\index[Spells]{Loyal Hound}\\
\textbf{School}: Summon\\
\textbf{Level}: 4, Rare\\
\textbf{Casting Time}: 2 Actions\\
\textbf{Range}: 9 meters\\
\textbf{Components}: V, S, M (a tiny silver whistle, and a piece of bone, and a string)\\
\textbf{Duration}: 8 hours\\
You can summon a phantom watchdog to an unoccupied space that is within range and that you can see, where it will remain for the duration of the spell, until dismissed as an action, or until it moves more than 30 meters away from you.\\
The hound is invisible to all creatures except you and can't be harmed. When a Small or larger creature approaches within 10 meters of it without first speaking the password you specified when you cast the spell, the hound begins to bark at a high volume. The hound sees invisible creatures and can see into the Ethereal Plane. It ignores illusions. At the start of each of your rounds, the hound attempts to bite a creature within 1 meter of it that is hostile to you. The hound's attack bonus is equal to your Characteristic modifier for spellcasting + MP. On a hit, it deals 2d8 points of piercing damage.

\medskip\textbf{Luminescence}\index[Spells]{Luminescence}\\
\textbf{School}: Invocation\\
\textbf{Level}: 1, Uncommon\\
\textbf{Cast Time}: 2 Actions\\
\textbf{Range}: 18 meters\\
\textbf{Components}: V\\
\textbf{Duration}: 1 minute of real game time\\
All objects in a 6 meters-cube within range are surrounded by blue, green, or purple light (your choice). Any creature in the area when the spell is cast is also surrounded by light if it fails a Reflex save. For the duration, the affected objects and creatures emit a dim light with a 3m radius. Any attack roll against a subject creature or object is +1d6 if the attacker can see it, and the creature or object cannot benefit from invisibility.

\medskip\textbf{Magic Circle}\index[Spells]{Magic Circle}\\
\textbf{School}: Abjuration\\
\textbf{Level}: 3, Common\\
\textbf{Cast Time}: 1 minute\\
\textbf{Range}: 3 meters\\
\textbf{Components}: V, S, M (Holy Water or powdered silver and iron worth at least 100 gp, which the spell consumes)\\
\textbf{Duration}: 1 hour\\
You create a 3m-radius, 6 meters-tall cylinder of magical energy centered on a point on the ground that you can see within range and. Glowing runes appear wherever the cylinder intersects with the floor or other surface.\\
Choose one or more of the following creature types: celestials, elementals, fey, demons, or undead. The circle affects a creature of the chosen type in the following ways:\\

- The creature cannot knowingly enter the cylinder by any non-magical means. If the creature tries to use teleportation or planeswalking to do so, it must first succeed at a Will save.\\

- The creature has -1d6 on attack rolls against targets inside the cylinder.\\

- Targets inside the cylinder cannot be charmed, frightened, or possessed by the creature. When you cast this spell, you can have the spell work in the opposite direction, preventing a creature of the specified type from leaving the cylinder and protecting targets outside it.\\

\textbf{For each magical critical success you roll} in Magic Test you can increase the duration by 1 hour.

\medskip\textbf{Magic Hand}\index{Cantrip - Magic Hand}\\
\textbf{School}: Summon\\
\textbf{Level}: 0, Common\\
\textbf{Casting Time}: 2 Actions\\
\textbf{Range}: 9 meters\\
\textbf{Components}: V, S\\
\textbf{Duration}: 1d4 rounds +1 per point of Magic Proficiency\\
A hovering ghostly hand appears at a point you choose within range. The hand remains for the duration of the spell or until it is ended with an action. The hand vanishes if it is more than 10 meters from you or if you cast the spell again.\\
The Actions needed to move and use the magic hand are the same as you would use to use your hand. You can use your hand to manipulate an object, open an unlocked door or container, insert or retrieve an object from an open container, or pour out the contents of a vial. You can move your hand 10 meters each time you use it. The hand cannot attack, activate magic items, or carry items with Encumbrance greater than 2.\\
\textbf{For each magical critical success rolled} in the Magic Test the lifted encumbrance increases by 1 or doubles the duration.


\medskip\textbf{Magic Jar}\index[Spells]{Magic Jar}\\
\textbf{School}: Necromancy\\
\textbf{Level}: 6, Very Rare\\
\textbf{Cast Time}: 1 minute\\
\textbf{Range}: Personal\\
\textbf{Component}: V, S, M (a gem, crystal, reliquary, or some other ornamental container worth at least 500 gp)\\
\textbf{Duration}: Until dispelled\\
Your body enters a catatonic state as your soul leaves it and enters the container you used as a material component. As your soul occupies the container, you are aware of your surroundings as if you were in the space of the container. You cannot move or use reactions. The only action you can take is to project your soul up to 30 meters away, out of the container, returning to your living body (and ending the spell) or trying to possess a humanoid body.\\
You can attempt to possess any humanoid within 30 meters of you that you can see (creatures protected by the protection from good and evil or magic circle spells cannot be possessed). The target must make a Will save, and on a failed save, your soul enters the target's body, while the target's soul remains trapped in the container. If it succeeds, the target resists your attempts to possess it, and you cannot attempt to possess it again until 24 hours have passed.\\
Once you own a creature's body, you can control it. Your game statistics are replaced by the creature's statistics, with the exception of your Traits and your Intelligence, Wisdom, and Charisma scores. Keep the benefits provided by Feats. If the target has any Feats, you may not use any of them.\\
Meanwhile, the possessed creature's soul can sense the container's surroundings using its senses, but cannot move or take any actions.\\
While in possession of a body, you can use 2 Actions to return from the host body to the container, if you are within 30 meters of it, returning the host creature's soul to its body. If the host body dies while you're in it, the creature dies, and you must make a Will save against your DC of spell saves. On a successful run, you return to the container if it is within 30 meters of you. Otherwise, you will die.\\
If the container is destroyed or the spell ends, your soul immediately returns to your body. If your body is more than 30 meters away or if it died while trying to return to it, your soul will die as well. If another creature's soul is in the container when it is destroyed, the creature's soul returns to its body, if the body is alive and within 30 meters, otherwise, the creature dies. When the spell ends, the container is destroyed.

\medskip\textbf{Magic Mark}\index{Cantrip - Magic Mark}\\
\textbf{School}: Universal\\
\textbf{Level}: 0, Common\\
\textbf{Cast Time}: 2 Actions\\
\textbf{Range}: Contact\\
\textbf{Components}: V, S\\
\textbf{Duration}: Permanent\\
This spell allows you to inscribe a personal rune or mark on an object. The writing cannot be longer than 15 cm. The writing can be visible or invisible depending on how you decide when casting the spell.
A detect magic or read magic spell displays the inscription if invisible.
If the writing is placed on a creature, it disappears within a month.\\
\textbf{For each magical Critical Success rolled} in Magic Test write one more logo.

\medskip\textbf{Magic Mouth}\index[Spells]{Magic Mouth}\\
\textbf{School}: Illusion\\
\textbf{Level}: 2, Common\\
\textbf{Cast Time}: 1 minute\\
\textbf{Range}: 9 meters\\
\textbf{Component}: V, S, M (a small piece of honeycomb and jade dust worth at least 10 gp, which the spell consumes)\\
\textbf{Duration}: Until dispelled\\
Implants a message into an object within range, a message that is spoken when the trigger condition is met. Choose an item that you can see and that isn't being worn or carried by another creature. Then speak your message, which must be 25 words or less, but can be spread out over a period of up to 10 minutes. Finally, determine the circumstance that will trigger the spell to deliver your message.\\
When the circumstance manifests itself, a magical mouth appears on the object and recites the message with your voice and at the same volume with which you pronounced it. If the item you choose has a mouth or something that looks like a mouth (for example, the mouth of a statue), the magical mouth appears so that words appear to come from the item's mouth. When you cast this spell, you can cause the spell to end after delivering its message, or to linger and repeat the message whenever the condition triggers.\\
The triggering circumstance can be as general or detailed as you like, but it must be based on visible or audible conditions that occur within 10 meters of the object. For example, you could instruct the mouth to speak when any creature approaches within 10 meters of the object or when a silver bell rings within 10 meters of it.

\medskip\textbf{Magic Armour}\index[Spells]{Magic Armour}\\
\textbf{School}: Abjuration\\
\textbf{Level}: 1, Uncommon\\
\textbf{Casting Time}: 2 Actions\\
\textbf{Range}: Contact\\
\textbf{Components}: V, S, M (a piece of tooled leather)\\
\textbf{Duration}: 8 hours\\
You cast the spell upon touching a willing creature not wearing Armour. A protective magical force surrounds the target until the spell ends. The target's Defence becomes 13 + Dexterity +1/6 Magic Proficiency. The spell ends if the target dons Armour or dismisses the spell as an action.\\
\textbf{For each magical critical success rolled} in the Magic Test, Defence increases by 1.

\medskip\textbf{Magic Club}\index{Cantrip - Magic Club}\\
\textbf{School}: Animals and Plants\\
\textbf{Level}: 0, Common\\
\textbf{Casting Time}: 1 Immediate Action\\
\textbf{Range}: Contact\\
\textbf{Components}: V, S, M (mistletoe, a four-leaf clover, and a club or fighting staff)\\
\textbf{Duration}: 1 minute\\
The wood of a club or quarterstaff you are wielding is infused with the power of nature. For the duration of the spell, using that weapon you can use your spellcasting ability in place of Strength for attack rolls and melee damage, and the weapon's damage die becomes a d8. The weapon also becomes magical if it isn't already. The spell ends if you cast it again or if you let go of the weapon.\\
\textbf{For each magical critical success rolled} in the Magic Test the duration doubles or you take +1 to the damage.


\medskip\textbf{Arcane Dart}\index[Spells]{Arcane Dart}\\
\textbf{School}: Universal\\
\textbf{Level}: 1, Common\\
\textbf{Casting Time}: 2 Actions\\
\textbf{Range}: 36 meters\\
\textbf{Components}: V, S\\
\textbf{Duration}: 1 Turn, Concentration\\
You create a luminous bolt of magical force. Launching one or more darts already summoned costs 1 Action. The bolt strikes a creature of your choice that you can see within range. A bolt deals 1d4 + 1 force damage to its target, and you can direct them to hit one or more creatures.\\
You create an additional bolt when you reach MP 3, MP 5, MP 7, and MP 9. Damage increases by 2 for each time you took Adept of Magic on the Universal List, up to a maximum of 5 increases.\\
\textbf{For each magical critical success rolled} in the Magic Test the spell creates an additional bolt.

\medskip\textbf{Magic Weapon}\index[Spells]{Magic Weapon}\\
\textbf{School}: Transmutation\\
\textbf{Level}: 2, Common\\
\textbf{Casting Time}: 1 Immediate Action\\
\textbf{Range}: Contact\\
\textbf{Components}: V, S\\
\textbf{Duration}: 10 minutes\\
You cast the spell touching a non-magical weapon. Until the spell ends, the weapon becomes a magical weapon with a +1 bonus on attack and damage rolls.\\
\textbf{For each magical critical success obtained} in the Magic Test, the bonus increases to +1.


\medskip\textbf{Major Image}\index[Spells]{Major Image}\\
\textbf{School}: Illusion\\
\textbf{Level}: 3, Common\\
\textbf{Cast Time}: 2 Actions\\
\textbf{Range}: 36 meters\\
\textbf{Components}: V, S, M (a piece of fleece)\\
\textbf{Duration}: Concentration, up to 1 minute per Magic Proficiency\\
You create an image of an object, creature, or some other visible phenomenon no larger than a 6 meters cube. The image appears at a point you can see within range and remains there for the duration. The image looks completely real, and includes sounds, smells, and the temperature appropriate to the thing depicted. You can't generate enough heat or cold to do damage, nor a sound loud enough to deal sound damage or deafen a creature, or an odor that would make a creature sick (such as a troglodyte's stench). While you remain within range of the illusion, you can use an action to cause the image to move to anywhere else within range.\\
As the image shifts position, you can alter its appearance so that its movements appear natural. For example, if you create an image of a creature and move it around, you can alter the image so that it appears to be walking. Similarly, you can use the illusion to produce different sounds at different times, eventually leading to a conversation.\\
Physical interaction with the image reveals it as an illusion, as things pass through it. A creature that uses 3 actions to examine the image can determine that it is an illusion with a successful Intelligence (Investigation) check against your spell's save DC. If a creature recognizes the illusion for what it is, the creature can see through it, and for that creature all other sensory qualities vanish.\\
\textbf{On a critical success} the spell lasts until dispelled, requiring no concentration from you.

\medskip\textbf{Marking Smite}\index[Spells]{Marking Smite}\\
\textbf{School}: Invocation\\
\textbf{Level}: 2, Common\\
\textbf{Casting Time}: 1 Immediate Action\\
\textbf{Range}: Personal\\
\textbf{Components}: V\\
\textbf{Duration}: 1 minute\\
The next time you hit a creature with a melee weapon attack during the spell's duration, the weapon glows with a magical glow as you strike. The attack deals an additional 1d6 points of Light damage to the target, which becomes visible if it is invisible and emits dim light in a 1m radius. Also, the target cannot become invisible until the spell ends. \\
You must pass a Magic Test for casting this spell while fighting.\\
\textbf{For each magical critical success rolled} in the Magic Test the additional damage increases by 1d6.

\medskip\textbf{Mass Cure Wounds}\index[Spells]{Mass Cure Wounds}\\
\textbf{School}: Heal\\
\textbf{Rarity}: Uncommon\\
Same as Cure Wounds but heal up to 4 creatures, in a 10m radius.\\
You use three times more Spell Points than the selected Cure Wounds.\\
\textbf{For each magical critical success rolled} on the check, you heal one more creature.\\
If the caster and the creature healed are both Followers of the same Patron, the spell heals an additional 1d8. \\
Unless otherwise stated, this spell cannot be used on animals or plants. \\
If the spellcaster and the cured creature are both Devotees of the same Patron, each value on the dice equal to 1,2,3 will be considered 4.

\medskip\textbf{Mass Heal}\index[Spells]{Mass Heal}\\
\textbf{School}: Heal\\
\textbf{Level}: 9, Legendary\\
\textbf{Casting Time}: 2 Actions\\
\textbf{Range}: 18 meters\\
\textbf{Components}: V, S\\
\textbf{Duration}: Instant\\
A wave of healing energy flows from you to the wounded creatures around you. You restore up to 700 Hit Points, divided as you like among any creatures within range and that you can see (with a maximum of 70 Hit Points per creature). Creatures healed by this spell are also cured of all disease and any effects that cause them to be blinded or deafened. This spell can deal up to 120 HP of damage to an undead. Save on Fortitude to nullify the effect.

If the spellcaster and creature healed are both \textbf{Followers} of the same Patron, the healing assigned increases by 20\%

If the spellcaster and creature healed are both \textbf{Devoted} of the same Patron, the healing assigned increases by 50\%

\medskip\textbf{Mass Healing Word}\index[Spells]{Mass Healing Word}\\
\textbf{School}: Heal\\
\textbf{Level}: 3, Rare\\
\textbf{Casting Time}: 1 Immediate Action\\
\textbf{Range}: 18 meters\\
\textbf{Components}: V\\
\textbf{Duration}: Instant\\
As you speak words of healing, up to six creatures of your choice that you can see within range regain Hit Points equal to 1d4 + your Characteristic modifier for spellcasting. This spell deals the same amount of damage to undead.\\
\textbf{For each magical critical success rolled} in the Magic Test the healing increases by 1d4.\\
If the caster and the creature healed are both Followers of the same Patron, the spell heals an additional 1d4. \\
If the spellcaster and the creature cured are both Devotees of the same Patron, the spell heals an additional 2d4.


\medskip\textbf{Mass Suggestion}\index[Spells]{Mass Suggestion}\\
\textbf{School}: Enchantment\\
\textbf{Level}: 6, Uncommon\\
\textbf{Cast Time}: 2 Actions\\
\textbf{Range}: 18 meters\\
\textbf{Components}: V, M (a snake's tongue and a piece of honeycomb or a drop of sweet oil)\\
\textbf{Duration}: 24 hours\\
You suggest a course of activity (limited to one or two sentences) and magically affect up to twelve creatures within range that you can see and hear and understand you, chosen by you. Creatures that can't be charmed are immune to this effect. The suggestion must be made so that the course of action sounds reasonable. Asking a creature to stab itself, throw itself on a spear, set itself on fire, or do some other obviously harmful act automatically negates the spell's effects.\\
Each target must make a Will save. On a failed save, it follows the course of action you describe to the best of its ability. The suggested course of action can continue for the entire duration of the spell. If the suggested activity can be completed in a shorter time, the spell ends when the subject finishes doing what is asked.\\
You can also specify conditions that will trigger a special activity for the duration of the spell. For example, you might suggest that a group of soldiers give up all their money to the first beggar they meet. If the condition is not met before the spell ends, the activity will fail. If you or any of your companions harm a creature affected by this spell, the spell ends for that creature. \\
\textbf{For each magical critical success rolled} in Magic Test add one day to the duration.

\medskip\textbf{Maze}\index[Spells]{Maze}\\
\textbf{School}: Summon\\
\textbf{Level}: 8, Rare\\
\textbf{Casting Time}: 2 Actions\\
\textbf{Range}: 18 meters\\
\textbf{Components}: V, S\\
\textbf{Duration}: maximum 10 minutes\\
Banish a creature that you can see within range on a labyrinthine demiplane. The target remains there for the duration of the spell or until it escapes the labyrinth. The target can take 3 Actions to attempt to escape. When he does, he makes a DC 25 Intelligence check.
When the spell ends, the target reappears in the space it left or, if that space is occupied, in the nearest unoccupied space.\\
\textbf{For each magical critical success rolled} in the Magic Test, the duration increases by 10 minutes. With two magical critical successes you can affect another creature.

\medskip\textbf{One with stone}\index[Spells]{One with stone}\\
\textbf{School}: Earth\\
\textbf{Level}: 3, Common\\
\textbf{Casting Time}: 2 Actions\\
\textbf{Range}: Contact\\
\textbf{Components}: V, S\\
\textbf{Duration}: 8 hours\\
You step into a stone object or surface large enough to hold your entire body, fusing with the stone along with any equipment you carry for the duration. Using your movement, you step into the stone at a point you are in contact with. Nothing remains of your presence that remains visible or otherwise detectable by nonmagical senses. While fused with the stone, you cannot see what goes on outside it, and any Wisdom checks you make to hear sounds made outside it are made with -1d6. You remain aware of the passage of time and can cast spells upon yourself while fused with the stone. You can use your movement to leave the stone and reappear where you entered it, thus ending the spell. Otherwise you can't move.\\
Minor damage to the stone does no harm to you, but its partial destruction or change of shape (so that you no longer enter it) expels you from it and deals 6d6 bludgeoning damage to you. The stone's complete destruction (or its transmutation into another substance) causes you to be expelled and deals 50 bludgeoning damage to you. If you are ejected, you fall prone in an unoccupied space, closest to where you entered the stone.\\
\textbf{For each magical critical success rolled} in the Magic Test, the maximum duration increases by 1 hour.

\medskip\textbf{Mental Fell}\index[Spells]{Mental Fell}\\
\textbf{School}: Enchantment\\
\textbf{Level}: 8, Rare\\
\textbf{Casting Time}: 2 Actions\\
\textbf{Range}: 45 meters\\
\textbf{Components}: V, S, M (a handful of clay, crystal, glass, or mineral orbs)\\
\textbf{Duration}: Instant\\
You assault the mind of a creature within range and that you can see, seeking to fragment its intellect and personality. The target takes 4d6 points of damage and must make a Will save. On a failed save, the creature's Intelligence and Charisma scores drop to -4. The creature can't cast spells, activate magic items, understand languages, or communicate in any intelligible way. The creature can, however, identify its friends, track them, and even protect them. After 30 days, the creature can repeat the Saving Throw against the spell. If successful, the spell ends. If it fails, the effect is permanent. \\
The spell can be ended within 30 days with Greater restoration, Heal or Wish.

\medskip\textbf{Message}\index{Cantrip - Message}\\
\textbf{School}: Transmutation\\
\textbf{Level}: 0, Common\\
\textbf{Cast Time}: 2 Actions\\
\textbf{Range}: 36 meters\\
\textbf{Components}: V, S, M (a small piece of copper wire)\\
\textbf{Duration}: 1 round\\
You point your finger at a creature within range and whisper a short message. The target (and only the target) hears the message and can reply in a whisper that only you can hear.\\
You can also cast this spell through solid objects if you are familiar with the target and know that it is behind the barrier. Magic silence, 30cm of stone, 1 cm of plain metal, a thin sheet of lead, or 1 meter of wood blocks the spell. The spell need not follow a straight line, and can freely go around corners or through cracks.\\
\textbf{For each magical critical success rolled} in the Magic Test the spell lasts 1 round longer.


\medskip\textbf{Messenger Animal}\index[Spells]{Messenger Animal}\\
\textbf{School}: Animals and Plants\\
\textbf{Level}: 2, Common\\
\textbf{Casting Time}: 2 Actions\\
\textbf{Range}: 9 meters\\
\textbf{Components}: V, S, M (some food)\\
\textbf{Duration}: 24 hours\\
Through this spell, you use an animal to deliver a message. Choose a Tiny beast that you can see and is within range, such as a squirrel, jay, or bat. You specify a place, which you must have visited in the past, and a recipient that matches a generic description, such as "a man or woman wearing a city guard uniform" or "a red-headed dwarf wearing a fedora". Also speak a message of up to twenty-five words. The target beast travels for the spell's duration to the specified location, covering approximately 45 miles in 24 hours for a flying messenger, or 25 miles for other animals. When the messenger arrives at its destination, it delivers the message to the creature you describe, replicating the sound of your voice. The messenger speaks only to a creature matching the description you provide. If the messenger fails to reach its destination before the spell ends, the message is lost, and the beast returns to where you cast the spell.\\
\textbf{For each magical critical success rolled} in Magic Test the duration of the spell increases by 8 hours

\medskip\textbf{Metamorphosis}\index[Spells]{Metamorphosis}\\
\textbf{School}: Animals and Plants\\
\textbf{Level}: 4, Common\\
\textbf{Casting Time}: 2 Actions\\
\textbf{Range}: 18 meters\\
\textbf{Components}: V, S, M (a caterpillar cocoon)\\
\textbf{Duration}: 1 hour \\
This spell transforms a creature that you can see within range into a new form. An unwilling creature must succeed at a Will save to avoid the effect. Shapeshifters automatically succeed at their saves. The spell has no effect on a target with 0 Hit Points. \\
The transformation lasts for the duration of the spell or until the target drops to 0 Hit Points or dies. The new form can be that of any beast whose challenge rating is half the caster's Magic Proficiency score (or sum of Traits if devoted to Shayalia). The target's game statistics, including mental ability scores, are replaced by the stats of the chosen beast. However, he retains his Traits and personality.\\
The target retains the same Hit Points and regains 1d12 Hit Points in its new form. When it reverts to its normal form, the creature retains the Hit Points it currently has. If he reaches 0 or less Hit Points in the new form then he returns to normal and any effects are also reflected in the current form. \\
The creature is limited in the actions it can perform by the nature of its new form, and cannot converse, cast spells, or take any other action that requires hands or speech. The target's equipment melds into the new form. The creature can't activate, use, wield, or benefit in any way from its equipment.

\medskip\textbf{Meteor Storm}\index[Spells]{Meteor Storm}\hypertarget{sciamedimeteore}{}\\
\textbf{School}: Fire, Earth\\
\textbf{Level}: 9, Legendary\\
\textbf{Cast Time}: 3 Actions\\
\textbf{Range}: 1.5 kilometers\\
\textbf{Components}: V, S\\
\textbf{Duration}: Instant\\
4 fire meteors crash to the ground at four different points within range and that you can see. Each meteorite strikes in a 10-foot radius. Each affected creature must make a Reflex saving throw. A creature takes 20d6 fire damage and 20d6 bludgeoning damage on a failed save, or half as much.
these damages if he exceeds it. If a creature is in the area of ​​more than one meteorite, it is only affected by one.\\
\textbf{Saving Throw Success/Critical Failure}: On a critical failure the damage is doubled, on a critical success the damage is further halved\\
\textbf{Every 3 critical rolls} in the Magic Test you hurl another meteorite.


\medskip\textbf{Mind Shield}\index[Spells]{Mind Shield}\\
\textbf{School}: Abjuration\\
\textbf{Level}: 8, Uncommon\\
\textbf{Casting Time}: 2 Actions\\
\textbf{Range}: Contact\\
\textbf{Components}: V, S\\
\textbf{Duration}: 24 hours\\
Until the spell ends, a willing creature you touch during the casting is immune to any effects that would sense its emotions or read its thoughts, divination spells, and the charmed condition. the spell also negates wish spells and other spells or effects of similar strength used to
affect the mind of the target or to gain information about it.\\
\textbf{For each magical critical success rolled} in the Magic Test, the duration doubles. If you get three crits the duration is permanent.

\medskip\textbf{Minor Illusion}\index[Spells]{Minor Illusion}\\
\textbf{School}: Universal\\
\textbf{Level}: 0, Common\\
\textbf{Cast Time}: 2 Actions\\
\textbf{Range}: 9 meters\\
\textbf{Components}: S, M (a piece of fleece)\\
\textbf{Duration}: 1 minute\\
You create an image of an object or sound within range for the spell's duration. The illusion ends if you dismiss it as an action or cast this spell again.\\
If you create a sound, its volume can range from a whisper to a scream. It can be your voice, someone else's voice, a lion's roar, a drumbeat, or any other sound you choose. The sound continues unceasingly for the duration, or you can make different sounds at different times before the spell ends.\\
If you create an image of an object (such as a chair, a muddy footprint, or a small chest) it cannot be larger than a cube with a 1-metre edge. The image cannot produce sound, light, smell or any other sensory effect. Physical interaction with the object reveals it as an illusion, because things can pass through it.\\
A creature that uses 3 actions to examine the sound or image can determine that it is an illusion with a successful Intelligence (Investigation) check against your spell's save DC. If a creature recognizes the illusion for what it is, the illusion fades for it.

\medskip\textbf{Mirage Arcane}\index[Spells]{Mirage Arcane}\\
\textbf{School}: Illusion\\
\textbf{Level}: 7, Rare\\
\textbf{Cast Time}: 10 minutes\\
\textbf{Range}: View\\
\textbf{Components}: V, S\\
\textbf{Duration}: 10 days\\
You make a piece of terrain within range, in an area up to 1 mile square, look, sound, and smell like some other terrain type. However, the general lay of the land remains the same. Open fields or a road can be turned into a swamp, hills, a crevasse, or some other type of difficult or impassable terrain. A pond can be transformed into a grassy clearing, a precipice into a gentle slope, a rock-strewn ravine into a wide, smooth road.\\
Similarly, you can change the appearance of structures, or add ones where there are none. The spell does not disguise, conceal, or add creatures.\\
The illusion includes audible, visual, tactile, and olfactory elements so that it can turn clear terrain into difficult terrain (or vice versa) or otherwise impede movement in the area. Any illusory piece of ground (such as a stone or staff) that is removed from the spell's area immediately vanishes. Creatures with true seeing can see beyond the illusion and distinguish the true shape of terrain; however, the other elements of the illusion remain, so while the creature is aware of the illusion's presence, it can still physically interact with it.\\
\textbf{With three magical critical successes gained} in Magic Test the duration is permanent.

\medskip\textbf{Mirror Image}\index[Spells]{Mirror Image}\\
\textbf{School}: Illusion\\
\textbf{Level}: 2, Common\\
\textbf{Cast Time}: 2 Actions\\
\textbf{Range}: Personal\\
\textbf{Components}: V, S\\
\textbf{Duration}: 1 minute\\
2d4 illusory duplicates of yourself appear in your space. Until the spell ends, the duplicates move with you and mimic your actions, swapping places so it's impossible to determine which image is real. You can use 1 Action to dismiss illusory duplicates.\\
Every time a creature hits you it actually hit an illusory image.
If a creature makes multiple attacks in turn, it can scatter an image for each successful attack. If you are hit by an area spell, all images vanish.\\
A creature that can't see, or relies on senses other than sight (such as blindsight), or that can distinguish illusions as false (such as true seeing), is unaffected by this spell. \\
\textbf{For each magical critical success you roll} in the Magic Test you create one more duplicate image up to a maximum total of 8 images.

\medskip\textbf{Mislead}\index[Spells]{Mislead}\\
\textbf{School}: Illusion\\
\textbf{Level}: 5, Uncommon\\
\textbf{Casting Time}: 2 Actions\\
\textbf{Range}: Personal\\
\textbf{Components}: S\\
\textbf{Duration}: 1 hour\\
You become invisible the moment an illusory double of you appears in your current location. The double lasts for the spell's duration, but the invisibility ends if you attack or cast a spell. You can use 2 Actions to make the illusory double move up to twice your speed and make it gesture, speak, and behave in any way you like.\\
You can see through his eyes and hear through his ears as if you were in his space. During each of your rounds, as an Action, you can switch from using his senses to using yours, or vice versa. While you are using his senses, you are blinded and deafened to your surroundings.

\medskip\textbf{Move earth}\index[Spells]{Move earth}\\
\textbf{School}: Earth\\
\textbf{Level}: 6, Uncommon\\
\textbf{Casting Time}: 2 Actions\\
\textbf{Range}: 36 meters\\
\textbf{Components}: V, S, M (an iron shovel and a small bag containing a mix of soil types - clay, manure, and sand)\\
\textbf{Duration}: Concentration, max 2 hours\\
Choose an area on the ground within range no larger than 13 meters on a side. For durability, you can reshape dirt, sand, or clay in the area in any way you like. You can raise or lower the altitude of the area, create or fill a moat, erect or lower a wall, or form a pillar. The extent of these changes cannot exceed half the area's largest size. Thus, if you operate on a square 12 meters on each side, you can create a pillar 6 meters high, raise or lower the altitude of the land by 6 meters, dig a ditch 6 meters deep, and so on. It takes 10 minutes to complete these changes. After every 10 minutes you spend concentrating on the spell, you can choose a new area of land to work on.\\
Because the terrain changes slowly, creatures in the area usually can't be entangled or injured by the movement of the terrain. The spell can't manipulate natural stone or stone buildings. Rocks and structures move to adjust to the new terrain. If the way you shape the terrain would make a structure unstable, it could collapse. Likewise, this spell does not directly affect plant growth. The loose earth carries with it any vegetable present.

\medskip\textbf{Occult Missile}\index{Cantrip - Occult Missile}\\
\textbf{School}: Universal\\
\textbf{Level}: 1, Common\\
\textbf{Casting Time}: 1 Action\\
\textbf{Range}: 36 meters\\
\textbf{Components}: V, S\\
\textbf{Duration}: Instant\\
A crackling beam of energy is aimed at a creature within range. Make a ranged spell attack against the target. On a hit, the target takes 1d8 force damage.\\
The spell's damage increases by 1d8 when you reach MP 5, MP 11 and MP 17 but it costs 2 Actions to cast it empowered and 2 Spell Points, you must also have taken Adept of Magic in this Spell List a number of times equal to the empowerments you you want to apply.\\
\textbf{Every 2 magical critical successes rolled} in the Magic Test you create another beam of energy.

\medskip\textbf{Omen}\index[Spells]{Omen}\\
\textbf{School}: Divination\\
\textbf{Level}: 2, Common\\
\textbf{Cast Time}: 1 minute\\
\textbf{Range}: Personal\\
\textbf{Components}: V, S, M (some sticks, bones, or similar items specially marked and worth at least 25 gp) \\
\textbf{Duration}: Instant\\
By throwing gem-inlaid sticks, rolling dragon bones, stacking elaborate cards, or employing some other divining tool, you receive a portent from an otherworldly entity regarding the outcome of a specific course of action you intend to take in the next 30 minutes. The Arbiter chooses from the following omens:\\

- Prosperity, for positive results\\
- Calamity, for negative results\\
- Prosperity and calamity, for both positive and negative outcomes\\
- Nothing, for results that are neither particularly positive nor negative\\

The spell does not take into account any possible circumstances that could modify the result, such as the casting of further spells or the loss or arrival of an ally. If you cast the spell two or more times before the new sun has risen, there is a cumulative 25\% chance that for each cast after the first you get an erroneous reading. The Arbiter makes this roll secretly.

\medskip\textbf{Orb of Invulnerability}\index[Spells]{Orb of Invulnerability}\\
\textbf{School}: Abjuration\\
\textbf{Level}: 6, Common\\
\textbf{Casting Time}: 2 Actions\\
\textbf{Range}: Self (10' radius)\\
\textbf{Component}: V. S, M (a glass or crystal ball that shatters when the spell ends) \\
\textbf{Duration}: Concentration, max 1 minute\\
An immobile, faintly shimmering barrier stands in a 3m radius around you and remains there for the duration.\\
Any spell of Level 4 (excluding higher results from critical magic) or lower cast from outside the barrier cannot affect creatures or objects within it. These spells are suppressed if they target creatures and objects within the barrier or affect the area the barrier is over. \\
\textbf{For every two magical critical successes you roll} in Magic Test you can block a higher level of spell.

\medskip\textbf{Pass without trace}\index[Spells]{Pass without trace}\\
\textbf{School}: Earth, Animals and Plants\\
\textbf{Level}: 2, Common\\
\textbf{Casting Time}: 2 Actions\\
\textbf{Range}: Personal\\
\textbf{Components}: V, S, M (ashes from a burnt mistletoe leaf and a spruce twig)\\
\textbf{Duration}: Concentration, 1 hour
For the duration of the spell, your tracks cannot be followed except by magical means. The creature receiving this bonus leaves no tracks or other signs of its passage.\\
\textbf{For each magical critical success you roll} in the Magic Test you can include another creature in the benefits of the spell.

\medskip\textbf{Pass door}\index[Spells]{Pass door}\\
\textbf{School}: Earth\\
\textbf{Level}: 5, Uncommon\\
\textbf{Casting Time}: 2 Actions\\
\textbf{Range}: 9 meters\\
\textbf{Components}: V, S, M (a pinch of sesame seeds)\\
\textbf{Duration}: 1 hour\\
For the duration, a passageway appears at a point within range that you can see, on a wooden, wall, or stone surface (such as a wall, ceiling, or floor) of your choice. Choose the size of the opening: maximum 1 meter wide, 2.4 meters high and 6 meters deep. The passage does not create instability in the surrounding structure.\\
When the opening disappears, any creatures or objects still in the passage created by the spell are ejected safely into the unoccupied space closest to the surface you cast the spell upon.

\medskip\textbf{Phantom Steed}\index[Spells]{Phantom Steed}\\
\textbf{School}: Illusion\\
\textbf{Level}: 3, Common\\
\textbf{Cast Time}: 1 minute\\
\textbf{Range}: 9 meters\\
\textbf{Components}: V, S\\
\textbf{Duration}: 1 hour\\
A quasi-real horse-like creature of size Large, it appears on the ground in an unoccupied space of your choice and within range. You decide the appearance of the creature, and it appears equipped with saddle, bit and bridle. Any equipment created by the spell vanishes in a cloud of smoke if it is brought more than 3 meter away from the steed. For the duration, you or a creature of your choice may ride the steed. The creature uses the stats of the racehorse, except that it has a speed of 30 meters and can travel 9 miles in an hour, or 12 miles at a fast pace. When the spell ends, the steed gradually begins to fade, giving the rider 1 minute to dismount. The spell ends if you use an action to end it or if the steed takes damage.\\
\textbf{For each magical critical success rolled} in the Magic Test, the duration augment by one hour or create one more additional steed.

\medskip\textbf{Pick Lock}\index[Spells]{Pick Lock}\\
\textbf{School}: Transmutation\\
\textbf{Level}: 2, Common\\
\textbf{Casting Time}: 2 Actions\\
\textbf{Range}: 18 meters\\
\textbf{Components}: V\\
\textbf{Duration}: Instant\\
Choose an object that is within range and that you can see. The object can be a door, box, handcuffs, lock, or other object that has a common or magical method of preventing access.\\
A target that is locked by a common lock or that is locked or barred is opened, unlocked, or freed. If the object has multiple locks, only one of them is opened.\\
If you choose a target that is held Magic Lock that spell is suppressed for 10 minutes, during which time the target can be opened as normal. When you cast this spell, a loud knock, audible up to 100 meters away, emanates from the target object.\\
\textbf{For each magical critical success you roll} in the Magic Test, you may open another padlock/lock within range.

\medskip\textbf{Piercing Gaze}\index[Spells]{Piercing Gaze}\\
\textbf{School}: Necromancy\\
\textbf{Level}: 6, Very Rare\\
\textbf{Cast Time}: 2 Actions\\
\textbf{Range}: Personal\\
\textbf{Components}: V, S\\
\textbf{Duration}: Concentration, max 1 minute\\
For the duration, your eyes turn into a black void infused with terrible power. A creature of your choice within 20 meters of you that you can see must succeed at a Will save or, for the duration, be affected by one of the following effects of your choice. During each of your rounds, until the spell ends, you can use two Actions to target another creature, but you cannot target a creature again that has successfully saved against this casting of piercing gaze. \\


- \textit{Asleep}. The target falls unconscious. It awakens if it takes any amount of damage or if another creature uses 2 Actions to rouse it from sleep.\\
- \textit{Sick}. The target has -1d6 on attack rolls and ability checks. At the end of each of his rounds, he can make another Will save. If successful, the effect ends.\\
- \textit{Panicballed}. The target is scared of you. During each of its rounds, the frightened creature must take two move actions and move away from you by the shortest and safest route possible, unless it has no room to move. If the target moves to a place at least 20 meters away from you, where it can't see you, this effect ends.


\medskip\textbf{Plant Growth}\index[Spells]{Plant Growth}\\
\textbf{School}: Animals and Plants\\
\textbf{Level}: 3, Uncommon\\
\textbf{Cast Time}: 2 Actions or 8 hours\\
\textbf{Range}: 45 meters\\
\textbf{Components}: V, S\\
\textbf{Duration}: Instant\\
This spell channels vitality into plants within a specific area. There are two possible uses for this spell, conferring immediate or long-term benefits. If you cast this spell taking 1 action, choose a point within range. All normal plants in a 30m radius centered on that point become dense and bushy. A creature that crosses the area quadruples the cost of its move. \\
You can exclude one or more areas of any size within the spell's area from its effects. \\
If you cast this spell over the course of 8 hours, you feed the earth. All plants in a 715 meters radius centered on a point within range become superproductive for 1 year. Plants produce twice the normal amount of food at harvest.\\
\textbf{If he gets two magical Critical Successes} you suffer the effects of the 8 hours of casting even if the spell was cast with 2 Actions.


\medskip\textbf{Poison Spray}\index{Cantrip - Poison Spray}\\
\textbf{School}: Animals and Plants\\
\textbf{Level}: 0, Uncommon\\
\textbf{Casting Time}: 1 Action\\
\textbf{Range}: 3 meters\\
\textbf{Components}: V, S\\
\textbf{Duration}: Instant\\
You extend your hand towards a creature you can see within range, and project a cloud of poisonous gas from your palm. The creature must succeed at a Fortitude save or take 1d12 poison damage. \\
The spell's damage increases by 1d8 when you reach MP 5, MP 11 and MP 17, but it costs 2 Actions to cast it empowered and 2 Spell Points, you must also have taken Adept of Magic in this Spell List a number of times equal to the empowerments that you want to apply.\\
\textbf{For every two magical critical successes you roll} in the Magic Test you affect another creature within range.


\medskip\textbf{My Word: Stun}\index[Spells]{My Word: Stun}\\
\textbf{School}: Enchantment\\
\textbf{Level}: 8, Uncommon\\
\textbf{Casting Time}: 1 Immediate Action\\
\textbf{Range}: 18 meters\\
\textbf{Components}: V\\
\textbf{Duration}: 1 minutes\\
You speak a word of power that can overwhelm the mind of a creature within range and that you can see, leaving it confused. If the target has 150 Hit Points or fewer, it is stunned. Otherwise, the spell has no effect.

\medskip\textbf{Prayer of Healing}\index[Spells]{Prayer of Healing}\\
\textbf{School}: Heal\\
\textbf{Level}: 2, Common\\
\textbf{Cast Time}: 10 minutes\\
\textbf{Range}: 9 meters\\
\textbf{Components}: V\\
\textbf{Duration}: Instant\\
Up to six creatures of your choice that you can see within range each regain Hit Points equal to 2d6 + your Characteristic modifier for spellcasting. This spell deals the same amount of damage to undead. \\
\textbf{For each magical critical success rolled} in the Magic Test the healing increases by 1d8.

\medskip\textbf{Prestidigitation}\index{Cantrip - Prestidigitation}\\
\textbf{School}: Universal\\
\textbf{Level}: 0, Common\\
\textbf{Casting Time}: 2 Actions\\
\textbf{Range}: 3 meters\\
\textbf{Components}: V, S\\
\textbf{Duration}: Maximum 1 hour\\
This spell is a minor magic trick that novice spellcasters use for practice. You create one of the following magical effects within range:\\

- You create an instant harmless sensory effect such as a shower of sparks, a gust of wind, a faint musical note or a strange smell.\\
- Instantly light or extinguish a candle, torch or small campfire.\\
- Instantly clear or defile an object no larger than 0.03 cubic meters.\\
- Cool, heat or flavor up to 0.03 cubic meters of non-living material for 1 hour.\\
- Make a color, a small sign or a symbol appear on an object or surface for 1 hour.\\
- You create a non-magical trinket or illusory image that enters your hand and remains until the end of your next round.\\

If you cast this spell multiple times, you can have up to three non-instantaneous effects active at a time, and you can interrupt one of these effects as an action.\\
\textbf{For each magical Critical Success rolled} in Magic Test you can activate one additional magical effect.

\medskip\textbf{Prismatic Spray}\index[Spells]{Prismatic Spray}\\
\textbf{School}: Invocation\\
\textbf{Level}: 7, Rare\\
\textbf{Cast Time}: 2 Actions\\
\textbf{Range}: Self (20m cone)\\
\textbf{Components}: V, S\\
\textbf{Duration}: Instant\\
Eight multi-colored beams of light shoot out of your hand. Each ray is a different color and has a different power and purpose. Each creature in a 20m cone must make a Reflex save. For each target, roll a d8 to determine what color ray hit it. \\


- \textit{1. Red}. The target takes 10d6 fire damage on a failed save, or half as much damage on a successful one. \\
- \textit{2. Orange}. The target takes 10d6 acid damage on a failed save, or half as much damage on a successful one. \\
- \textit{3. Yellow}. The target takes 10d6 electricity damage on a failed save, or half as much damage on a successful one. \\
- \textit{4. Green}. The target takes 10d6 poison damage on a failed save, or half as much damage on a successful one. \\
- \textit{5. Blue}. The target takes 10d6 cold damage on a failed save, or half as much damage on a successful one. \\
- \textit{6. Indigo}. On a failed save, the target is entangled. It must then make a Fortitude save at the start of each of its rounds. If it succeeds at the Saving Throw three times, the spell ends. If he fails his save three times, he is permanently turned to stone and becomes subject to the petrified condition. Successes and failures do not have to be consecutive; keep track of both until the target has obtained three of the same type.\\
- \textit{7. Violet}. On a failed save, the target is blinded. He must then make a Will save at the start of your next round. On a successful save, the blindness ends. On a failed save, the creature is transported to another plane of existence of the Arbiter's choice and is no longer blinded (typically, a creature that is not on its home plane is exiled to it, while other creatures are usually taken to the Astral or Ethereal planes).\\
- \textit{8. Special}. The target is hit by two beams. Roll two more times, rerolling the 8s.


\medskip\textbf{Prismatic Wall}\index[Spells]{Prismatic Wall}\\
\textbf{School}: Abjuration\\
\textbf{Level}: 9, Rare\\
\textbf{Casting Time}: 2 Actions\\
\textbf{Range}: 18 meters\\
\textbf{Components}: V, S\\
\textbf{Duration}: 10 minutes\\
A plane of brilliant, multicolored light forms an opaque vertical wall, up to 30 meters wide, 10 meters high, and 1 cm thick, centered on a point you can see within range. Alternatively, you can shape the wall into a sphere, up to 10 meters in diameter, centered on a point of your choice within range. The wall remains fixed in place for the duration of the spell. If you place the wall so that it passes through a creature's space, the spell fails and the spell slot is wasted. The wall radiates bright light out to a range of 18 meters and dim light for an additional 18 meters. You and the creatures you designate when casting the spell can pass through and stay close to the wall without danger. If another creature that can see the wall moves within 6 meters of it or begins its round there, she must succeed at a Fortitude save or be blinded for 1 minute. The wall consists of seven layers, each of a different color. When a creature tries to dive into or through the wall, it does so one layer at a time, through all layers of the wall. As it dives into or passes through each layer, the creature must succeed at a Reflex save or be affected by the properties of each layer, one at a time, as described below.\\
The wall can be destroyed, one layer at a time, in order from red to violet, in a specific way for each layer. Once a layer is destroyed, it will be destroyed for the duration of the spell. A rod of cancellation destroys a Prismatic Wall, but an anti-magic field has no effect on it.\\

- \textit{1. Red}. The target takes 10d6 fire damage on a failed save, or half as much damage on a successful one. As long as this layer exists, nonmagical ranged attacks can't pass through the wall. The layer can be destroyed by dealing 25 cold damage to it.\\
- \textit{2. Orange}. The target takes 10d6 acid damage on a failed save, or half as much damage on a successful one. As long as this layer exists, magical ranged attacks cannot pass through the wall. The layer can be destroyed by a strong wind. 3. Yellow. The target takes 10d6 Electricity damage on a failed save, or half as much damage on a successful one. This layer can be destroyed by dealing 60 force damage to it.\\
- \textit{4. Green}. The target takes 10d6 poison damage on a failed save, or half as much damage on a successful one. A Pass door spell, or another spell of equal or higher level that can open a portal to a solid surface, destroys this layer.\\.
- \textit{5. Blue}. The target takes 10d6 cold damage on a failed save, or half as much damage on a successful one. The layer can be destroyed by dealing at least 25 fire damage to it.\\
- \textit{6. Indigo}. On a failed save, the target is entangled. It must then make a Fortitude save at the start of each of its rounds. If it succeeds at the Saving Throw three times, the spell ends. If he fails his save three times, he is permanently turned to stone and becomes subject to the petrified condition. Successes and failures do not have to be consecutive; keep track of both until the target has obtained three of the same type. As long as this layer exists, spells cannot be cast through the wall. The layer is destroyed by bright light shed by a daylight spell or a similar higher-level spell.\\
- \textit{7. Violet}. On a failed save, the target is blinded. He must then make a Will save at the start of your next round. On a successful save, the blindness ends. On a failed save, the creature is transported to another plane of existence of the Arbiter's choice and is no longer blinded (typically, a creature that is not on its home plane is exiled to it, while other creatures are usually cast in the Astral or Ethereal planes). This layer is destroyed by a dispel magic spell or a similar spell of equal or higher level that can end spells and magical effects.


\medskip\textbf{Private Shrine}\index[Spells]{Private Shrine}\\
\textbf{School}: Abjuration\\
\textbf{Level}: 4, Very Rare\\
\textbf{Casting Time}: 10 minutes\\
\textbf{Range}: 36 meters\\
\textbf{Components}: V, S, M (a thin sheet of lead, a piece of opaque glass, a cotton ball or tissue, and powdered chrysolite)\\
\textbf{Duration}: 24 hours \\
Magically protect an area. The area is a cube that can be as small as 1 meter of edge or as large as 30 meters of edge. The spell lasts until the duration ends or until you use an action to end it.\\
When you cast the spell, you decide what kind of protection it provides, by choosing one or more of the following properties:\\


- Sound cannot go through the perimeter of the protected area.\\
- The perimeter of the protected area appears dark and foggy, preventing you from seeing through (even darkvision).\\
- Sensors created by divination spells cannot appear within the warded area or pass through its perimeter barrier.\\
- Creatures in the area cannot be the target of divination spells.\\
- Nothing can teleport into or out of the protected area.\\
- Within the protected area, planar travel is prohibited.\\

Casting this spell on the same spot every day for a year makes the effect permanent.\\
\textbf{For each magical critical success you roll} in the Magic Test you can increase the cube's size by 10 meters of edge or increase the duration by 12 hours.


\medskip\textbf{Produce Flame}\index{Cantrip - Produce Flame}\\
\textbf{School}: Fire\\
\textbf{Level}: 0, Common\\
\textbf{Casting Time}: 1 Action\\
\textbf{Range}: Personal\\
\textbf{Components}: V, S\\
\textbf{Duration}: 10 minutes\\
A flame appears in your hand. The flame remains there for the duration of the spell and does no harm to you or your equipment. The flame produces dim light in a 1m radius. The spell ends if you interrupt it with an action or cast it again.\\
You can also use the flame to attack, though doing so ends the spell. When you cast this spell, or as an action in a later round, you can hurl the flame at one creature within 10 meters of you. Perform a ranged spell attack. On a hit, the target takes 1d8 fire damage.\\
The spell's damage increases by 1d8 when you reach MP 5, MP 11 and MP 17, but it costs 2 Actions to cast it empowered and 2 Spell Points, you must also have taken Adept of Magic in this Spell List a number of times equal to the empowerments that you want to apply.\\
\textbf{For each magical critical success you roll} in Magic Test, you may attack one more creature without ending the spell.

\medskip\textbf{Programmed Illusion}\index[Spells]{Programmed Illusion}\\
\textbf{School}: Illusion\\
\textbf{Level}: 6, Uncommon\\
\textbf{Cast Time}: 2 Actions\\
\textbf{Range}: 36 meters\\
\textbf{Components}: V, S, M (a piece of fleece and jade dust worth at least 25 gp)\\
\textbf{Duration}: Until dissolved\\
You create, within range, an illusion of an object, creature, or some other visible phenomenon that activates when a specific condition is met. Until then the illusion is imperceptible. It can't be larger than a 10m cube, and you decide when you cast the spell, how the illusion behaves and what sounds it produces. The programmed performance can last up to 5 minutes. When the conditions you specify are met, the illusion manifests and behaves in the way you describe. Once the illusion has finished its performance, it disappears and lies dormant for 10 minutes. After this period, the illusion can be activated again.\\
The triggering condition can be as general or detailed as you like, although it must be based on visible or audible conditions occurring within 10 meters of the area. For example, you could create an illusion of yourself that appears and warns anyone who tries to open a trapped door, or you could set up the illusion to activate only when a creature says the right word or phrase.\\
Physical interaction with the image reveals it as an illusion, as things pass through it. A creature that uses 3 actions to examine the image can determine that it is an illusion with a successful Intelligence (Investigation) check against the spell's save DC. If a creature recognizes the illusion for what it is, it can see through the image, and any sound made by the image sounds artificial to it.

\medskip\textbf{Prohibition}\index[Spells]{Prohibition}\\
\textbf{School}: Abjuration\\
\textbf{Level}: 6, Uncommon\\
\textbf{Cast Time}: 10 minutes\\
\textbf{Range}: Contact\\
\textbf{Components}: V, S, M (a spray of Holy Water, rare incense, and a ruby powder worth 1000 gp)\\
\textbf{Duration}: 1 day\\
Create a magical travel ward that protects up to 4,000 square meters of floor space, up to a height of 10 meters above the ground. For the duration, creatures cannot teleport into the area or use passageways, such as the one created by the portal spell, to enter the area. The spell protects the area from planar travel, and thus prevents creatures from entering the area via the Astral Plane, Ethereal Plane, or Plane of Shadow, or the planeshift spell. \\
Additionally, the spell damages creature types of your choice when casting. Choose one or more of the following: celestials, elementals, fey, demons, and undead. When a selected creature enters the spell's area for the first time in a round or begins its round here, the creature takes 5d10 Light or Void damage (your choice, when you cast the spell). \\
When you cast this spell, you can establish a password. A creature that speaks the password while entering the spell's area takes no damage from it.\\
The spell's area cannot overlap with the area of another forbidding spell. If you cast forbidding every day for 30 days in the same place, the spell will last until dispelled, and the material components are consumed on the last casting.

\medskip\textbf{Project Image}\index[Spells]{Project Image}\\
\textbf{School}: Illusion\\
\textbf{Level}: 7, Uncommon\\
\textbf{Cast Time}: 2 Actions\\
\textbf{Range}: 750 kilometers\\
\textbf{Components}: V, S, M (a small reproduction of you made of materials worth at least 5 gp)\\
\textbf{Duration}: 1 day\\
You create an illusory copy of yourself that lasts for the duration. The copy can appear anywhere within range that you've already seen, ignoring any obstacles in the way. The illusion reproduces your appearance and sounds but is intangible. If the illusion takes any damage, it disappears, and the spell ends.\\
You can use 2 Actions to make this illusion move up to twice your speed and make it gesture, speak, and behave in any way you like. Imitate your behavior perfectly.\\
You can see through its eyes and hear through its ears as if you were in the space it is in. During each of your rounds, as an Action, you can switch from using her senses to using yours, or vice versa. While you are using her senses, you are blinded and deafened to your surroundings.\\
Physical interaction with the image reveals it as an illusion, as things pass through it. A creature that uses 3 actions to examine the image can determine that it is an illusion with a successful Awareness check against the DC of the spell's save. If a creature recognizes the illusion for what it is, it can see through the image, and any sound made by the image sounds artificial to it.

\medskip\textbf{Protection from Energy}\index[Spells]{Protection from Energy}\\
\textbf{School}: Abjuration\\
\textbf{Level}: 3, Common\\
\textbf{Cast Time}: 2 Actions\\
\textbf{Range}: Contact\\
\textbf{Components}: V, S\\
\textbf{Duration}: 10 minutes\\
You cast the spell touching a willing creature. For the duration, the target has resistance to one type of damage you choose: acid, cold, fire, electricity, or sound. You can sacrifice the entire duration of the spell by ending it to completely negate damage taken from an energy source.\\
\textbf{For each magical critical success you roll} in the Magic Test you can affect another person or double the duration.

\medskip\textbf{Protection from Good and Evil}\index[Spells]{Protection from Good and Evil}\\
\textbf{School}: Abjuration\\
\textbf{Level}: 1, Common\\
\textbf{Casting Time}: 2 Actions\\
\textbf{Range}: Contact\\
\textbf{Components}: V, S, M (Holy Water or powdered silver and iron, which the spell consumes worth 5 gp)\\
\textbf{Duration}: 10 minutes\\
Until the spell ends, a willing creature in contact with you at the time of casting it is protected from certain types of creatures: aberrations, celestials, elementals, fey, demons, and undead.\\
Protection confers several benefits. Creatures of those types have -1d6 on attack rolls against the target. The target cannot be charmed, frightened, or possessed by them. If the target is already charmed, frightened, or possessed by such a creature, the target has +1d6 on any new Saving Throws against that effect. \\
\textbf{This spell is not usable if using Traits. The Arbiter can grant the same effects to Followers and Patrons as other Patrons}

\medskip\textbf{Protection from Poisons}\index[Spells]{Protection from Poisons}\\
\textbf{School}: Abjuration\\
\textbf{Level}: 2, Uncommon\\
\textbf{Casting Time}: 2 Actions\\
\textbf{Range}: Contact\\
\textbf{Components}: V, S\\
\textbf{Duration}: 1 hour\\
For the duration of the spell, the target has +1d6 on Saving Throws against poisoned beings, and has resistance to poison damage.\\
\textbf{On two magical critical successes rolled} in the Magic Test you can nullify a poison circulating on the target.

\medskip\textbf{Protection from energy minor}\index[Spells]{Protection form Energy  minor}\\
\textbf{School}: Abjuration\\
\textbf{Level}: 1, Rare\\
\textbf{Casting Time}: 1 Reaction\\
\textbf{Range}: Contact\\
\textbf{Components}: V, S\\
\textbf{Duration}: 1 minute\\
You cast the spell touching a willing creature. For the duration, the target has damage reduction from the chosen energy equal to 5. You can sacrifice the entire duration of the spell, ending it, to reduce the damage taken from an energy source by 20 (as if you had Damage Resistance 20 from that energy source).\\
\textbf{For each magical critical success you roll} in the Magic Test you can affect another person or double the duration.

\medskip\textbf{Pure Polymorph}\index[Spells]{Pure Polymorph}\\
\textbf{School}: Animals and Plants\\
\textbf{Level}: 9, Rare\\
\textbf{Casting Time}: 2 Actions\\
\textbf{Range}: 9 meters\\
\textbf{Components}: V, S, M (a drop of mercury, a dollop of gum arabic, and a puff of smoke) \\
\textbf{Duration}: 1 hour \\
Choose a nonmagical creature or object that you can see within range. The spell has no effect on a target with 0 Hit Points. You transform the creature into a different creature, the creature into an object, or the object into a creature (the object must not be worn or carried by another creature). The transformation lasts for the duration of the spell or until the target drops to 0 Hit Points or dies. If you concentrate on this spell for its entire duration, the transformation becomes permanent.\\
Shapeshifters ignore this spell. An unwilling creature can make a Will save, and ignores the effect of this spell on a successful one. \\

- \textit{Creature to Creature}. If you transform a creature into another species of creature, the new form can be that of any species you choose, whose challenge rating is equal to or lower than your Magic Proficiency score (or total Traits in Common if Shayalia's Devotee) . The target's game statistics, including mental ability scores, are replaced by the new form's stats. He however retains the traits and personality of him. \\
The target retains the same Hit Points and regains 1d12 Hit Points in its new form. When it reverts to its normal form, the creature retains its current Hit Points. If it reaches 0 or less Hit Points in the new form then it returns to normal and any effects are also reflected in the current form. The creature is limited in the actions it can take by the nature of its new form, and it cannot speak, cast spells, or take any other action that requires hands or speaking, unless the new form is capable of taking these actions. The target's equipment melds into the new form. The creature can't activate, use, wield, or benefit in any way from its equipment. \\

- \textit{Object to Creature.} You can transform an object into any type of creature, as long as the creature's size is no greater than the object's size and the creature's challenge rating is 9 or lower. The creature is friendly towards you and your companions. It works on your rounds. You decide what actions it will perform and how it moves. The Arbiter owns the creature's statistics and will resolve all of its actions and movements.
If the enchantment becomes permanent, you lose control of the creature. Depending on how you treated her, she may remain friendly towards you.\\

- \textit{Creature in Object}. If you transform a creature into an item, it transforms along with whatever it is wearing or carrying. The creature's statistics become those of the item, and after the spell ends and the creature reverts to its normal form, it has no memory of its time in item form.

\medskip\textbf{Purify Food and Drink}\index[Spells]{Purify Food and Drink}\\
\textbf{School}: Animals and Plants\\
\textbf{Level}: 1, Common\\
\textbf{Casting Time}: 2 Actions\\
\textbf{Range}: 3 meters\\
\textbf{Components}: V, S\\
\textbf{Duration}: Instant\\
All nonmagical food and drink in a 1m-radius sphere centered at a point of your choice within range is purified and freed from poisons and diseases. A decaying food is cleaned up and made edible.

\medskip\textbf{Pyroexpert}\index[Spells]{Pyroexpert}\\
\textbf{School}: Fire\\
\textbf{Level}: 2, Uncommon\\
\textbf{Casting Time}: 2 Actions\\
\textbf{Range}: 18 meters\\
\textbf{Components}: V, S, M (a match that is consumed)\\
\textbf{Duration}: Instant\\
You choose an area with fire, at least 1 meter in edge, within range that is directly visible to you. By extinguishing the flames he can create fireworks or smoke.

- \textit{Fireworks}. The target fire explodes in a dazzling display of flame and color. Each creature within 3 meter of the target must make a Fortitude save or be blinded until the end of the next round.

- \textit{Smoking}. Thick black smoke billows from target fire and spreads in a 6 meters radius, moving around corners. The smoke area is heavily obscured and provides medium coverage. The smoke persists for 1 minute or until a strong wind blows it away.

\medskip\textbf{Quick Retreat}\index[Spells]{Quick Retreat}\\
\textbf{School}: Transmutation\\
\textbf{Level}: 1, Uncommon\\
\textbf{Casting Time}: 1 Immediate Action\\
\textbf{Range}: Personal\\
\textbf{Components}: V, S\\
\textbf{Duration}: Concentration, 1 minute\\
This spell allows you to move at an incredible pace. When you cast this spell, you gain a bonus move action.\\
\textbf{For each magical critical success rolled} in the Magic Test, the duration increases by 1 round.

\medskip\textbf{Raise dead}\index[Spells]{Raise dead}\\
\textbf{School}: Necromancy\\
\textbf{Level}: 5, Legendary\\
\textbf{Cast Time}: 1 hour\\
\textbf{Range}: Contact\\
\textbf{Components}: V, S, M (a diamond worth at least 500 gp, which the spell consumes)\\
\textbf{Duration}: Instant\\
You bring a dead creature back to life, provided it hasn't been dead for more than 10 days. If the creature's soul is both willing and free to rejoin the body, the creature returns to life with 1 hit point.\\
This spell also neutralizes any poisons and cures non-magical diseases that afflicted the creature at the time of death. This spell, however, does not remove magical diseases, curses, or similar effects; if these are not removed before casting the spell, they will resume manifesting when the creature returns to life. The spell cannot bring an undead creature back to life.\\
This spell closes all mortal wounds, but does not restore missing body parts. If the creature lacks body parts or organs essential for survival (the head, for example) the spell automatically fails. \\
Returning from the dead is an ordeal. The target takes a -4 penalty on all attack rolls, Saving Throws, and ability checks. Each time the target finishes a night's rest, the penalty is reduced by 1 until it disappears.\\
\textbf{This spell shouldn't be available. Only a Patron can revive.}

\medskip\textbf{Ray of Fatigue}\index[Spells]{Ray of Fatigue}\\
\textbf{School}: Necromancy\\
\textbf{Level}: 2, Common\\
\textbf{Casting Time}: 2 Actions\\
\textbf{Range}: 18 meters\\
\textbf{Components}: V, S\\
\textbf{Duration}: 1 minute\\
A black beam of debilitating energy shoots from your finger directed at a creature within range. Make a ranged spell attack against the target. On a hit, the target will deal half damage with weapon attacks that use the Strenght until the spell ends.\\
\textbf{For every two magical critical successes rolled} in the Magic Test, you increase the target's Fatigue by 1.


\medskip\textbf{Ray of Frost}\index{Cantrip - Ray of Frost}\\
\textbf{School}: Water\\
\textbf{Level}: 0, Common\\
\textbf{Casting Time}: 1 Action\\
\textbf{Range}: 18 meters\\
\textbf{Components}: V, S\\
\textbf{Duration}: Instant\\
A frozen beam of blue light strikes a creature within range. Make a ranged spell attack against the target. On a hit, he takes 1d8 points of cold damage, and his speed is reduced by 3 meter until the start of your next round. \\
The spell's damage increases by 1d8 when you reach MP 5, MP 11 and MP 17, but it costs 2 Actions to cast it empowered and 2 Spell Points, you must also have taken Adept of Magic in this Spell List a number of times equal to the empowerments that you want to apply.\\
\textbf{For every two magical critical successes you roll} in Magic Test you create an additional ice cream bundle.


\medskip\textbf{Read Magic}\index[Spells]{Read Magic}\\
\textbf{School}: Universal\\
\textbf{Level}: 1, Common\\
\textbf{Casting Time}: 1 Action\\
\textbf{Range}: Contact\\
\textbf{Component}: V, S, M (a fragment of an enchanted item)\\
\textbf{Duration}: 1 minute, while used\\
You grant the target the ability to read a scroll or magical writing. For a duration of 1 minute or until used once, whichever comes first, the creature automatically succeeds in understanding a magical scroll or casting the contents of the scroll while meeting the criteria and rules for casting spells from scrolls.
\textbf{For each magical Critical Success rolled} in the Magic Test you may read or understand one more scroll.

\medskip\textbf{Rebreathe}\index[Spells]{Rebreathe}\\
\textbf{School}: Heal, Necromancy\\
\textbf{Level}: 3, Very Rare\\
\textbf{Casting Time}: 10 Minutes\\
\textbf{Range}: Contact\\
\textbf{Components}: V, S, M (diamond worth 300 gp, which the spell consumes)\\
\textbf{Duration}: Instant\\
A creature that died in the last minute and you touch returns to life with 1 hit point. This spell cannot bring people who have died of old age back to life, nor can it restore missing body parts.\\
The creature brought back to life must make a DC 15 Fortitude save or, due to the trauma suffered, it does not come back to life, if it comes back to life it is Fatigued 3.\\
\textbf{Note}: At the Arbiter's discretion this may be the only spell granted to bring a creature back to life, otherwise the rule applies that only a Patron can bring back to life.

\medskip\textbf{Regeneration}\index[Spells]{Regeneration}\\
\textbf{School}: Transmutation\\
\textbf{Level}: 7, Legendary\\
\textbf{Cast Time}: 1 minute\\
\textbf{Range}: Contact\\
\textbf{Components}: V, S, M (a rosary and Holy Water)\\
\textbf{Duration}: 1 hour\\
You cast the spell upon touching a creature to stimulate its natural healing ability. The target regains 4d8 + 15 Hit Points. For the duration of the spell, the target regains 1 hit point at the start of each of its rounds (6 Hit Points per minute). Severed limbs from the target's body (fingers, legs, tails, and so on), if any, are restored in 2 minutes. If you have the severed part and hold it against the stump, the spell causes the limb to sew itself back together in 3 rounds with the stump.\\
\textbf{For each magical critical success rolled} in the Magic Test double the Hit Points recovered per round.

\medskip\textbf{Reincarnation}\index[Spells]{Reincarnation}\\
\textbf{School}: Animals and Plants\\
\textbf{Level}: 5, Rare\\
\textbf{Cast Time}: 1 hour\\
\textbf{Range}: Contact\\
\textbf{Components}: V, S, M (rare oils and ointments worth at least 1000 gp, which the spell consumes)\\
\textbf{Duration}: Instant\\
You come into contact with a dead humanoid or a fragment of a dead humanoid. Provided the creature hasn't been dead for more than 10 days, the spell forms it into a new adult body and then summons its soul to enter the body. If the target's soul is not free or willing to do so, the spell fails.\\
The magic shapes a new body, which will likely cause the creature to change its race. The Arbiter rolls a d10 and consults the following table to determine what form the creature takes upon being brought back to life, or the Arbiter chooses the form.\\


\medskip
\begin{tabular}{ll}
\textbf{d10} &\textbf{Race}\\
\toprule
0 & Wolf/Eagle/Fox/Lynx (roll 1d4)\\
1&Nano\\
2&Elf\\
3&Half-elf\\
4&Half-orc\\
5&Boar/Badger/Dog/Rat (roll 1d4)\\
6&Nibali\\
7&Bear/Owl/Raccoon/Cat (roll 1d4)\\
8&Human\\
9&Same Previous Race\\
\end{tabular}

The reincarnated creature remembers its past life and experiences (same WP and MP, skill and feats). It retains the abilities it had in its original form if it is able to apply them.\\
\textbf{This spell is not available except to Devotees and Followers of Shayalia or Ephrem}\\
\textit{Note}: A Devotee or Follower of Shayalia or Ephrem will always reincarnate the creature into an animal, but being able to choose the type.\\
It is not possible to reincarnate as a gnome if you weren't a gnome before.


\medskip\textbf{Remove Curse}\index[Spells]{Remove Curse}\\
\textbf{School}: Abjuration\\
\textbf{Level}: 3, Common\\
\textbf{Cast Time}: 2 Actions\\
\textbf{Range}: Contact\\
\textbf{Components}: V, S\\
\textbf{Duration}: Instant\\
If the object or person has been cursed with a bestow curse spell, or otherwise the Storyteller decides that the object has a particular curse, then the caster's DC remove curse must be greater than that of the curse.

\textbf{For each magical critical success gained} in the Magic Test you can heal one more person or count +4 in DC to overcome the DC curse's.

Whether it was enough to cast the spell or it was cast with a Magic Test, the curse remains, but the spell allows the object to be removed and thrown away.

\medskip\textbf{Remove Disease}\index[Incantesimi]{Remove Disease}\label{rimuovimalattie}\hypertarget{rimuovimalattie}{} \\
\textbf{School}: Heal\\
\textbf{Level}: 4, Common\\
\textbf{Cast Time}: 1 turn\\
\textbf{Range}: Contact\\
\textbf{Components}: V, S\\
\textbf{Duration}: Instant\\
You can end even a natural disease. In the case of magical diseases your DC must overcome the DC of the disease.

\textbf{For each magical critical success gained} in the Magic Test you can heal one more person or count +4 in DC to overcome the DC disease's.

\medskip\textbf{Remove Poison}\index[Spells]{Remove Poison}\label{incrimuoviveleno}\hypertarget{incrimuoviveleno}{}\\
\textbf{School}: Water, Heal\\
\textbf{Level}: 3, Common\\
\textbf{Casting Time}: 1 round\\
\textbf{Range}: Contact\\
\textbf{Components}: V, S\\
\textbf{Duration}: Instant\\
The target subject to the spell is no longer poisoned.

\textbf{For each magical critical success gained} in the Magic Test you can heal one more person or count +4 in DC to overcome the DC poison's.

\medskip\textbf{Repair}\index{Cantrip - Repair}\\
\textbf{School}: Earth\\
\textbf{Level}: 0, Common\\
\textbf{Cast Time}: 1 minute\\
\textbf{Range}: Contact\\
\textbf{Components}: V, S, M (two magnets)\\
\textbf{Duration}: Instant\\
This spell repairs a single break or fissure in an object you touch, such as a broken chain, two halves of a broken key, a torn cloak, or a leaking wineskin. Provided the break or rift is no larger than 0.5m in any dimension, you are able to repair it, leaving no trace of the damage sustained. This spell can physically repair a magic item or construct, but cannot restore the magical functions of these items.

\medskip\textbf{Resistance}\index{Cantrip - Resistance}\\
\textbf{School}: Abjuration\\
\textbf{Level}: 0, Common\\
\textbf{Casting Time}: 1 Reaction\\
\textbf{Range}: Contact\\
\textbf{Components}: V, S, M (a miniature cloak)\\
\textbf{Duration}: Instant\\
You cast the spell while touching a willing creature. Once before the spell ends, the target can roll a d4 and add the result to a Saving Throw of its choice. He can roll the die before or after making the Saving Throw. Then the spell ends.\\
\textbf{For each magical Critical Success rolled} in the Magic Test another creature get the bonus.

\medskip\textbf{Restser's Furious Transformation}\index[Spells]{Restser's Furious Transformation}\\
\textbf{School}: Transmutation\\
\textbf{Level}: 6, Very Rare\\
\textbf{Casting Time}: 2 Actions\\
\textbf{Range}: Personal\\
\textbf{Component}: V, S, M (20cc of alcoholic drink which is consumed by casting the spell, a magic weapon)\\
\textbf{Duration}: 1 round per Magic Proficiency\\
This spell allows a spellcaster to channel his magical energies to transform himself into a powerful fighter.

Until the end of the spell's duration, your weapon proficiency becomes equal to your Magic Proficiency.

Based on the magic weapon held in hand at the time of the spell, one becomes proficient in the List of Weapons in which that weapon belongs, if the weapon is present in more than one list, the spellcaster will choose the list. The caster gains the abilities of that weapon list as if he had chosen it a number of times equal to half his points in Magic Proficiency.

The caster gains 4 temporary Hit Points per point of Magic Proficiency possessed.
The unmodified score of the physical characteristics (Strength, Dexterity and Constitution) if lower than 2 become 2.

For the duration of the spell, the caster is no longer able to cast spells.

\medskip\textbf{Reverse Gravity}\index[Spells]{Reverse Gravity}\\
\textbf{School}: Transmutation\\
\textbf{Level}: 7, Rare\\
\textbf{Casting Time}: 2 Actions\\
\textbf{Range}: 30 meters\\
\textbf{Components}: V, S, M (a magnet and a wire)\\
\textbf{Duration}: Concentration, up to 1 minute
This spell reverses gravity in a 15 meters-radius, 30m tall cylinder centered at a point within range. When you cast this spell, all creatures and objects that aren't anchored to the ground in any way fall straight up and reach the top of the area. A creature can attempt a Reflex save to grab a stationary object within reach, to avoid falling this way, on a successful one. \\
If a solid object (the ceiling) is encountered along this fall, the falling objects and creatures impact it as they would during a normal fall. If an object or creature reaches the top of the area without hitting anything, it stays there, swaying slightly, for the duration.\\
At the end of the duration, affected objects and creatures fall back to the bottom.

\medskip\textbf{Rope Trick}\index[Spells]{Rope Trick}\\
\textbf{School}: Transmutation\\
\textbf{Level}: 2, Common\\
\textbf{Casting Time}: 1 minute\\
\textbf{Range}: Contact\\
\textbf{Components}: V, S, M (wheat extract powder and parchment string)\\
\textbf{Duration}: 1 hour + 10 minutes per Magic Competence\\
You come into contact with a piece of rope up to 18 meters long. One end of the string rises into the air until the string hangs perpendicular to the ground. At the opposite end of the rope, an invisible entrance opens into an extradimensional space that remains until the spell ends \\
Extradimensional space can be reached by climbing to the top of the rope (Climb check DC 15). The space can contain up to 2 creatures of Medium size or smaller +1 for each time you took Adept of Magic in the Transmutation List. The rope can be dragged through space, causing it to disappear from the sight of those outside it.\\
Attacks and spells cannot pass through the gateway into or out of extradimensional space, but those inside can see out as if they were seeing through a 3-by-3-foot window centered on the string. The Detect Magic spell allows you to see the opening. Anything in extradimensional space falls out when the spell ends.\\
\textbf{For each Magical Critical Success} obtained in the Magic Test the duration doubles or may contain another medium or smaller creature.


\medskip\textbf{Sanctuary}\index[Spells]{Sanctuary}\\
\textbf{School}: Abjuration\\
\textbf{Level}: 1, Common\\
\textbf{Casting Time}: 1 Immediate Action\\
\textbf{Range}: 9 meters\\
\textbf{Components}: V, S, M (a small silver mirror)\\
\textbf{Duration}: 1 minute\\
Protect a creature within range from attacks. Until the spell ends, any creature that targets the protected creature with a harmful attack or spell must first make a Will save. On a failed save, the attacker must choose a new target or lose the attack or spell. This spell does not protect the warded creature from area effects, such as the explosion of a fireball. If the warded creature makes an attack or casts a spell that affects enemy creatures, the spell ends.

\medskip\textbf{Scream of pain}\index[Spells]{Scream of pain}\\
\textbf{School}: Necromancy\\
\textbf{Level}: 1, Rare\\
\textbf{Casting Time}: 1 Reaction\\
\textbf{Range}: personal\\
\textbf{Components}: V\\
\textbf{Duration}: snapshot\\
As a reaction action, you emit a cry of pain when struck in melee. The creature that struck you must make a Fortitude save or take 2d4 Void damage.\\
\textbf{For each magical critical success you roll} in the Magic Test you cause an extra 1d4 of damage.

\medskip\textbf{Scry}\index[Spells]{Scry}\\
\textbf{School}: Divination\\
\textbf{Level}: 5, Rare\\
\textbf{Cast Time}: 10 minutes\\
\textbf{Range}: Personal\\
\textbf{Component}: V, S, M (a focus worth at least 1000 gp, such as a crystal ball, silver mirror, or fountain filled with Holy Water)\\
\textbf{Duration}: Concentration, max 10 minutes\\
You can see and hear a particular creature of your choice that is on the same plane of existence as you. The target must make a Will save, modified by how well you know the target and your physical connection to it. If the target knows you're casting the spell, it can voluntarily fail its Saving Throw if it wishes to be observed by
you.

\medskip

\begin{tabular}{ll}
\toprule
\textbf{Knowledge} & \textbf{Mod. to ST}\\
Have you heard of it &+5\\
You met the target &+0\\
You know the target well &-5\\
\end{tabular}

\begin{tabular}{ll}
\toprule
\textbf{Connection} & \textbf{Mod. ST}\\
Description or image &-2\\
Property or garment & -4\\
Body part (hair...)&-10\\
\end{tabular}

\medskip

If the save is successful, the target is unaffected by the spell, and you cannot use this spell against it again until 24 hours have passed.\\
On a failed save, the spell creates an invisible sensor within 3 meter of the target. Through the sensor you can hear and see as if you were on the spot. The sensor moves with the target, remaining within 3 meter of it for the duration. A creature that can see invisible objects sees the sensor as a glowing ball about the size of a fist.\\
Instead of targeting a creature, you can target a place you've seen before. When you choose this option, the sensor appears in that place but doesn't move.


\medskip\textbf{Searing Ray}\index[Spells]{Searing Ray}\\
\textbf{School}: Fire\\
\textbf{Level}: 2, Common\\
\textbf{Casting Time}: 2 Actions\\
\textbf{Range}: 36 meters\\
\textbf{Components}: V, S\\
\textbf{Duration}: Instant\\
You create three beams of fire and project them at three targets within range. You can throw them at the same target or at different targets.\\
Make one ranged spell attack for each ray. On a hit, the target takes 2d6 fire damage.\\
\textbf{For each magical critical success you roll} in the Magic Test you create an additional ray.

\medskip\textbf{Secret Chest}\index[Spells]{Secret Chest}\\
\textbf{School}: Summon\\
\textbf{Level}: 4, Rare\\
\textbf{Casting Time}: 2 Actions\\
\textbf{Range}: Contact\\
\textbf{Components}: V, S, M (a crafted chest, 1 meter x 50cm x 50cm, made of rare materials worth at least 5000 gp, and a Tiny replica of it made of the same materials and worth at least 50gp) \\
\textbf{Duration}: Instant\\
Hide a chest and all its contents on the Ethereal Plane. When you cast this spell, you must be in contact with the chest and the miniature replica that serves as the material component. The chest can hold up to 0.25 cubic meters of non-living material (1 x meter x 50cm x 50cm). While the chest remains on the Ethereal Plane, you can use an action to contact the replica and summon the chest. It will reappear in an unoccupied space on the ground within 1 meter of you. You can send the chest back to the Ethereal Plane, using an action and touching both the chest and the replica.\\
After 60 days, there is a cumulative 5\% per day that the spell's effect ends. \\
The effect ends if the spell is cast again, if the replica of the chest is destroyed, or if you decide to end the spell as an action. If the spell ends and the chest is on the Ethereal Plane, it is irretrievably lost.

\medskip\textbf{See Invisibility}\index[Spells]{See Invisibility}\\
\textbf{School}: Divination\\
\textbf{Level}: 2, Common\\
\textbf{Casting Time}: 2 Actions\\
\textbf{Range}: Personal\\
\textbf{Components}: V, S, M (a pinch of talcum powder and a handful of silver powder)\\
\textbf{Duration}: 1 hour\\
For the duration, you see invisible creatures and objects as if they were visible, and you can also see in the Ethereal Plane. Ethereal creatures and objects appear ghostly and transparent to you.

\medskip\textbf{Send}\index[Spells]{Send}\\
\textbf{School}: Invocation\\
\textbf{Level}: 3, Common\\
\textbf{Cast Time}: 2 Actions\\
\textbf{Range}: Unlimited\\
\textbf{Components}: V, S, M (a small piece of copper wire)\\
\textbf{Duration}: 1 round\\
You send a short message of 25 words or less to a creature you are familiar with. The creature hears the message in its mind, recognizes you as the sender, and can reply in a similar way. The spell allows creatures with an Intelligence score of at least -2 to understand the meaning of your message even if they do not understand your language.\\
You can send the message across any distance and even to other planes of existence, but if the target is on a plane other than yours, there is a 5\% chance that the message will not arrive.\\
\textbf{For each magical critical success rolled} in the Magic Test you increase the message by 25 words or the duration by one round.

\medskip\textbf{Shatter}\index[Spells]{Shatter}\\
\textbf{School}: Invocation\\
\textbf{Level}: 2, Common\\
\textbf{Casting Time}: 2 Actions\\
\textbf{Range}: 18 meters\\
\textbf{Components}: V, S, M (a metal shard)\\
\textbf{Duration}: Instant\\
A loud, very intense rumble erupts from a point of your choice within range. Each creature in a 3m-radius sphere centered on that point must make a Fortitude save. A creature takes 3d8 sound damage on a failed save, or half as much damage on a successful one. A creature composed of inorganic material, such as stone, crystal, or metal, has a -1d6 Saving Throw. A nonmagical item that is neither worn nor carried also takes damage if it is in the spell's area. \\
\textbf{For each magical critical success rolled} in the Magic Test the damage increases by 1d8.\\
\textbf{Save Critical Success/Failure}: On a critical failure the damage is doubled, on a critical success the damage is further halved

\medskip\textbf{Shining Smite}\index[Spells]{Shining Smite}\\
\textbf{School}: Invocation\\
\textbf{Level}: 2, Uncommon\\
\textbf{Casting Time}: 1 Immediate Action\\
\textbf{Range}: personal\\
\textbf{Components}: V\\
\textbf{Duration}: 1 minute\\
The target struck by the blow takes an extra 2d6 points of Light damage and becomes visible for the duration of the spell. In addition, the creature sheds light in a 1 meter radius.\\
You must pass a Magic Test for casting this spell while fighting.\\
\textbf{For each magical critical success you roll} in the Magic Test, deal +1d6 Light damage.

\medskip\textbf{Shield}\index[Spells]{Shield}\\
\textbf{School}: Abjuration\\
\textbf{Level}: 1, Common\\
\textbf{Casting Time}: 1 Reaction, which you take when hit by an attack or target of the spell Arcane Dart\\
\textbf{Range}: Personal\\
\textbf{Components}: V, S\\
\textbf{Duration}: 1 round\\
A barrier of invisible magical force appears to protect you. Until the start of your next round, you have a +2 bonus to Defence including the trigger attack, and you take no damage from Arcane Dart and Hidden Blast.\\
\textbf{For each magical critical success you roll} in the Magic Test you increase the duration by 1 round.

\medskip\textbf{Shield of Faith}\index[Spells]{Shield of Faith}\\
\textbf{School}: Abjuration\\
\textbf{Level}: 1, Common\\
\textbf{Casting Time}: 1 Immediate Action\\
\textbf{Range}: 18 meters\\
\textbf{Components}: V, S, M (a small parchment with a fragment of sacred text written on it)\\
\textbf{Duration}: 10 minutes\\
A shimmering field appears surrounding a creature of your choice within range, granting it a +2 bonus to Defence for the duration.
\textbf{For each magical critical success you roll} in the Magic Test you affect another creature.

\medskip\textbf{Shocking Grasp}\index{Cantrip - Shocking Grasp}\\
\textbf{School}: Air\\
\textbf{Level}: 0, Common\\
\textbf{Casting Time}: 1 Action\\
\textbf{Range}: Contact\\
\textbf{Components}: V, S\\
\textbf{Duration}: Instant\\
Lightning shoots from your hands, shocking any creature you try to touch. Make a melee spell attack against the target. You have +1d6 on the attack roll if the target is wearing Armour made of metal. On a hit, the target takes 1d8 electricity damage, and can't take reactions until the start of its next round.\\
The spell's damage increases by 1d8 when you reach MP 5, MP 11 and MP 17, but it costs 2 Actions to cast it empowered and 2 Spell Points, you must also have taken Adept of Magic in this Spell List a number of times equal to the empowerments that you want to apply.\\
\textbf{For each magical critical success rolled for two} in the Magic Test, the damage increases by 1d8

\medskip\textbf{Silence}\index[Spells]{Silence}\\
\textbf{School}: Illusion\\
\textbf{Level}: 2, Common\\
\textbf{Cast Time}: 2 Actions\\
\textbf{Range}: 36 meters\\
\textbf{Components}: V, S\\
\textbf{Duration}: 10 minutes\\
For the duration, no sound can be created in or through a 6 meters-radius sphere centered on a point of your choice within range. Any creature or object wholly within the sphere is immune to sound damage, and creatures wholly within it are deafened. It is impossible to cast a spell that includes a verbal component while inside it.\\
\textbf{For each magical critical success rolled} in the Magic Test, the duration doubles.

\medskip\textbf{Silent Image}\index[Spells]{Silent Image}\\
\textbf{School}: Illusion\\
\textbf{Level}: 1, Common\\
\textbf{Cast Time}: 2 Actions\\
\textbf{Range}: 36 meters\\
\textbf{Components}: V, S, M (a piece of fleece)\\
\textbf{Duration}: Concentration, max 3 minutes per Magic Proficiency\\
You create an image of an object, creature, or some other visible phenomenon no larger than a 3m cube. The image appears in a spot you can see within range and remains for the duration. The image is purely visual; it is not accompanied by sounds, smells or other sensory effects. You can use an action to cause the image to move to any other point within range. As the image shifts position, you can alter its appearance so that its movements appear natural. For example, if you create an image of a creature and move it around, you can alter the image so that it appears to be walking.\\
Physical interaction with the image reveals it as an illusion, as things pass through it. A creature that uses 3 actions to examine the image can determine that it is an illusion with an Awareness check against your spell's save DC. If a creature recognizes the illusion for what it is, the creature can see through it.

\medskip\textbf{Simulacrum}\index[Spells]{Simulacrum}\\
\textbf{School}: Illusion\\
\textbf{Level}: 7, Rare\\
\textbf{Casting Time}: 12 hours\\
\textbf{Range}: Contact\\
\textbf{Components}: V, S, M (a lot of snow or ice to make a full-size copy of the duplicated creature; some hair, nails, or other body parts of that creature to place in the snow or to ice; and a powdered ruby worth 1,500 gp, sprinkled over the duplicate and consumed by the spell)\\
\textbf{Duration}: Until dissolved\\
You shape an illusory duplicate of a beast or humanoid that remains within range for the spell's entire casting time. The duplicate is a creature, partly real and made of ice or snow, that can take actions and interact like a normal creature. It appears to be identical to the original, but has half that creature's maximum, hapf of Magic Proficiency and Weapon Proficiency, Hit Points and comes unequipped. Otherwise, the illusion uses all the stats of the creature it duplicates.\\
The simulacrum is friendly towards you and the creatures you designate. It obeys your spoken commands, moving and acting according to yourwishes and acting during your round of combat. The simulacrum lacks the ability to learn or become more powerful, and thus never increases in level or ability, nor can it recover Spell Points.\\
If the simulacrum is damaged, you can repair it in an alchemical laboratory, using rare herbs and minerals worth 100 gp per hit point recovered. The simulacrum remains until it drops to 0 Hit Points, at which point it reverts to snow and melts instantly. If you cast this spell again, any duplicates you created with this currently active spell are immediately destroyed.

\medskip\textbf{Sleep}\index[Spells]{Sleep}\\
\textbf{School}: Enchantment\\
\textbf{Level}: 1, Common\\
\textbf{Casting Time}: 2 Actions\\
\textbf{Range}: 27 meters\\
\textbf{Components}: V, S, M (a pinch of sand, rose petals or a cricket)\\
\textbf{Duration}: 1 minute\\
This spell puts creatures into a magical torpor. Roll 5d8; the total is the number of creature Hit Points the spell can affect. Creatures within 6 meters of the point you choose within range are affected in ascending order of Hit Points (ignoring unconscious creatures).\\
Starting with the creature with the lowest number of current Hit Points, each affected creature is knocked unconscious until the spell ends, the sleeper takes damage, or someone uses an action to shake or slap the sleeper. . Subtract each creature's Hit Points from the total before considering the creature with the next lowest hit point value. A creature's Hit Points must be equal to or less than its remaining total for the effect to affect it. Undead and creatures that can't be charmed are unaffected by this spell.\\
\textbf{For each magical critical success rolled} in the Magic Test affects an additional 2d8 Hit Points.

\medskip\textbf{Sleet Storm}\index[Spells]{Sleet Storm}\\
\textbf{School}: Water\\
\textbf{Level}: 3, Very Rare\\
\textbf{Casting Time}: 2 Actions\\
\textbf{Range}: 45 meters\\
\textbf{Components}: V, S, M (a pinch of powder and a few drops of water)\\
\textbf{Duration}: 1 minute\\
Until the spell ends, freezing rain and sleet rain down in a 6 meters-tall, 12m radius cylinder centered on a point you choose within range. The area is in dim light, while the exposed flames are extinguished. The ground in the area is covered in  ice, making it difficult terrain. When a creature enters the spell's area for the first time in a round or begins its round there, it must make a Reflex save. On a failed save, it falls prone. If a creature in the spell's area is concentrating, it must succeed at a Fortitude save against the DC of the spell's save or lose concentration.

\medskip\hypertarget{lentezza}{\textbf{Slow}}\index[Spells]{Slow}\\
\textbf{School}: Transmutation\\
\textbf{Level}: 3, Uncommon\\
\textbf{Cast Time}: 2 Actions\\
\textbf{Range}: 36 meters\\
\textbf{Components}: V, S, M (a drop of molasses) \\
\textbf{Duration}: 1 minute, Concentration\\
You modify the flow of time around up to 1d4 creatures of your choice in a 6 meters cube within range. Each target must succeed at a Will save or take one less action per round.\\
\textbf{For each magical critical success you roll} in Magic Test, you can affect one more creature.\\
\textbf{Critical Failure Save}: On a critical failure, you are slowed by one more action.

\medskip\textbf{Snooze}\index[Spells]{Snooze}\\
\textbf{School}: Alteration\\
\textbf{Level}: 2, Legendary\\
\textbf{Casting Time}: 1 round\\
\textbf{Range}: 6 meters\\
\textbf{Components}: V, S, M (a feather, a piece of white cotton)\\
\textbf{Duration}: 1 minute\\
This spell allows the caster to put up to 1 creature per Magic Proficiency/4 to rest for 1 hour. The creature must be willing.


This hour of rest is equivalent to 8 hours of rest as regards the recovery of Spell Points and Hit Points. The spell's benefits cannot be used more than once in 36 hours.\\
\textbf{For each magical critical success you roll} in Magic Test you affect 1 more creature.

\medskip\textbf{Speak with Animals}\index[Spells]{Speak with Animals}\\
\textbf{School}: Animals and Plants\\
\textbf{Level}: 1, Common\\
\textbf{Casting Time}: 2 Actions\\
\textbf{Range}: Personal\\
\textbf{Components}: V, S\\
\textbf{Duration}: 10 minutes\\
For the duration of the spell, you gain the ability to understand and communicate verbally with beasts. Many beasts' lore and awareness are limited by their intellect, but at a minimum, beasts can provide you with information about nearby places and monsters, including those they can or have sensed in days past. At the Arbiter's discretion, you may be able to get a beast to do you a small favor.\\
\textbf{For each magical critical success rolled} in the Magic Test, the duration doubles.

\medskip\textbf{Speak with Dead}\index[Spells]{Speak with Dead}\\
\textbf{School}: Necromancy\\
\textbf{Level}: 3, Rare\\
\textbf{Casting Time}: 2 Actions\\
\textbf{Range}: 3 meters\\
\textbf{Components}: V, S, M (lit incense)\\
\textbf{Duration}: 10 minutes\\
You bestow an appearance of life and Intelligence on a corpse of your choice within range, allowing it to answer questions you ask. The corpse must still have a mouth and cannot be undead. The spell fails if the corpse has already been the target of this spell in the last 10 days. Until the spell ends, you can ask the corpse up to five questions. The corpse knows only what it already knew in life, including spoken languages. The answers are usually short, cryptic or repetitive, and the corpse is under no obligation to give you truthful answers if you are hostile to it or recognize you as its enemy. This spell does not return the creature's soul to the body, only the spirit that moves it. As a result, the corpse cannot learn new information, understand nothing of what has happened since it died, and cannot make judgments about future events.

\medskip\textbf{Speak with Plants}\index[Spells]{Speak with Plants}\\
\textbf{School}: Animals and Plants\\
\textbf{Level}: 3, Rare\\
\textbf{Cast Time}: 2 Actions\\
\textbf{Range}: Self (10m radius)\\
\textbf{Components}: V, S\\
\textbf{Duration}: 10 minutes\\
Imbue plants within 10 meters of you with sentience and limited mobility, giving them the ability to communicate with you and execute simple commands. You can question plants about events that have occurred in the spell's area over the past day, gaining information about passing creatures, the weather, and more. You can also turn difficult terrain produced by plant growth (such as bushes and thick undergrowth) into ordinary terrain for the spell's duration.\\
Or you can transform normal terrain containing plants into difficult terrain, which remains for the duration of the spell, causing vines and branches to slow down pursuers, for example. \\
At the Arbiter's discretion, the plants may also perform other tasks on your behalf. The spell does not allow plants to uproot themselves and move, but they can move branches, stalks and stems freely. If a plant creature is in the area, you can communicate with it as if you speak the same language, but you gain no spell-like abilities to affect it. This spell can cause the plants created by the entangle spell to release an entangled creature.

\medskip\textbf{Spider Movement}\index[Spells]{Spider Movement}\\
\textbf{School}: Transmutation\\
\textbf{Level}: 2, Uncommon\\
\textbf{Casting Time}: 2 Actions\\
\textbf{Range}: Contact\\
\textbf{Components}: V, S, M (a drop of bitumen and a spider)\\
\textbf{Duration}: 10 minutes \\
You cast the spell touching a willing creature. Until the spell ends, the creature gains the ability to move up, down, and along vertical surfaces or stand upside down on a ceiling, hands free. The target also gains a climb speed equal to its movement speed. The affected creature is considered distracted when casting other spells.

\medskip\textbf{Spike Growth}\index[Spells]{Spike Growth}\\
\textbf{School}: Animals and Plants\\
\textbf{Level}: 2, Common\\
\textbf{Casting Time}: 2 Actions\\
\textbf{Range}: 45 meters\\
\textbf{Components}: V, S, M (seven sharp thorns or seven twigs, each pointed at one end)\\
\textbf{Duration}: 10 minutes\\
The ground in a 6 meters radius centered on a point within range writhes and generates very sharp spikes and thorns. For the duration, the area becomes difficult terrain. When a creature enters or moves within the area, it takes 2d4 points of damage for every 1 meter it travels.
The transformation of the terrain is so well camouflaged that it seems natural. Any creature that hasn't seen the area when the spell is cast must make an Awareness check against the spell's save DC to recognize the danger posed by the terrain before entering it.

\medskip\textbf{Spiritual Weapon}\index[Spells]{Spiritual Weapon}\\
\textbf{School}: Invocation\\
\textbf{Level}: 2, Common\\
\textbf{Casting Time}: 2 Actions\\
\textbf{Range}: 18 meters\\
\textbf{Components}: V, S\\
\textbf{Duration}: 3 minutes, Concentration\\
At a point in range, you create a floating spectral weapon, which remains for the duration or until you cast this spell again. When you cast the spell, you can make a melee spell attack against a creature within 1 meter of the weapon with a bonus to hit equal to Magic Proficiency/4. On a hit, the target takes force damage equal to 1d4 + your spellcasting Characteristic modifier + Magic Proficiency/4. During your round, as an Action, you can move the weapon 6 meters and make the attack against a creature within 1 meter of the weapon. The weapon can take any form you like, perhaps akin to the Patron. It is considered to have a magical bonus equal to Magic Proficiency/4. \\
The bonuses granted by Magic Proficiency/4 can be replaced by the sum of the Traits in common with the Patron/4.\\
\textbf{For each magical critical success rolled} in the Magic Test, the damage increases by 2.

\medskip\textbf{Nauseating Fog}\index[Spells]{Nauseating Fog}\\
\textbf{School}: Water, Air\\
\textbf{Level}: 3, Uncommon\\
\textbf{Cast Time}: 2 Actions\\
\textbf{Range}: 27 meters\\
\textbf{Components}: V, S, M (a rotten egg or stinky cabbage leaves)\\
\textbf{Duration}: 10 minutes\\
You create, at a point within range, a 6 meters-radius sphere composed of a foul-smelling yellow gas. The fog spreads around corners and its area is in dim light. The fog remains in the air for the duration. Any creature that is completely within the fog at the start of its round must make a Fortitude save against the poison. On a failed save, the creature spends 2 Actions that round vomiting and staggering. Creatures that don't need to breathe or are immune to poison automatically make their Saving Throws. A moderate wind (at least 15 kilometers per hour) disperses the fog after 4 rounds. A strong wind (at least 30 km/h) scatters it after 1 round.

\medskip\textbf{Stone Shape}\index[Spells]{Stone Shape}\\
\textbf{School}: Earth\\
\textbf{Level}: 4, Common\\
\textbf{Casting Time}: 2 Actions\\
\textbf{Range}: Contact\\
\textbf{Components}: V, S, M (malleable clay, which must be worked into a rough shape of the stone object)\\
\textbf{Duration}: Instant\\
Carve into any shape that lends itself to your purposes a Medium or smaller stone object, or a section of stone no larger than 1 meter in any direction, that you touch.\\
So, for example, you could carve a large stone into a weapon, idol, or coffin, or create a small passage through the wall, as long as the wall is less than 1 meter thick. You could also fashion a stone door or its frame to seal the door. The item you create can have up to two hinges and a latch, but it is impossible to create more complex mechanisms.

\medskip\textbf{Stone to Mud - Mud to Stone}\index[Spells]{Stone to Mud}\index[Spells]{Mud to Stone}\\
\textbf{School}: Earth\\
\textbf{Level}: 5, Uncommon - Very Rare\\
\textbf{Casting Time}: 2 Actions\\
\textbf{Range}: 45 meters\\
\textbf{Components}: V, S, M (water and clay)\\
\textbf{Duration}: Instant\\
This spell turns any type of natural rock into an equal volume of mud. The magic stone is unaffected by the spell. The spell has effect up to 2 cubes of 3x3x3 meters. The depth of the created mud cannot exceed 3 meters. Creatures unable to fly, levitate, or otherwise move away from the mud sink to the waist or chest; the terrain becomes doubly difficult and they are Entangled. Creatures large enough to walk on the bottom of the mud pool can wade through the area as difficult terrain.

If stone to mud is thrown at the ceiling of a cave or tunnel, the mud spills onto the floor and expands to form a pool 1 meter deep. The falling mud and ensuing landslide deals 8d6 bludgeoning damage to anyone directly under the area unless you halve the damage with a Reflex save.

Castles and large stone buildings are generally immune to the spell's effects, as turn stone to mud doesn't reach deep enough to undermine the foundations. However, other, smaller buildings often sit on foundations shallow enough to be damaged or even destroyed by the spell's effects.

The mud remains until a successful dispel magic or mud to stone spell is used, which restores its substance, but not necessarily its form. Natural evaporation transforms the mud into normal soil within several days depending on exposure to the sun, wind and natural drying.
If a creature is in the mud at the time of the Mud to Stone spell, it can make a Reflex save to free itself otherwise a DC 22 Strength check or 30 damage is required to break the stone.\\
\textbf{For each magical critical success you roll} in the Magic Test you affect an extra 10x10x3m cube.


\medskip\textbf{Stoneskin}\index[Spells]{Stoneskin}\\
\textbf{School}: Earth\\
\textbf{Level}: 4, Uncommon\\
\textbf{Casting Time}: 2 Actions\\
\textbf{Range}: Contact\\
\textbf{Components}: V, S, M (100 gp worth of diamond dust, which the spell consumes)\\
\textbf{Duration}: 1 hour\\
You cast the spell upon touching a willing creature, whose skin turns to a substance as hard as stone. Roll 1d4+half the MP value, the resulting sum being the number of times an attack with a melee or ranged weapon is nullified (whether it hits or not). \\
\textbf{For each Magical Critical Success rolled} in the Magic Test you increase the attacks negated by 1.

\medskip\textbf{Suggestion}\index[Spells]{Suggestion}\\
\textbf{School}: Enchantment\\
\textbf{Level}: 2, Common\\
\textbf{Cast Time}: 2 Actions\\
\textbf{Range}: 9 meters\\
\textbf{Components}: V, M (a snake's tongue and a piece of honeycomb or a drop of sweet oil)\\
\textbf{Duration}: 8 hours \\
You suggest a course of activity (limited to a sentence or two) and magically affect a creature of your choice within range and that you can see and hear and understand you. Creatures that can't be charmed are immune to this effect. The suggestion must be made so that the course of action sounds reasonable. Asking a creature to stab itself, throw itself on a spear, set itself on fire, or do some other obviously harmful act automatically negates the spell's effects.\\
The target must make a Will save. On a failed save, it follows the course of action you describe to the best of its ability. The suggested course of action can continue for the entire duration of the spell. If the suggested activity can be completed in a shorter time, the spell ends when the subject finishes doing what is asked. \\
You can also specify conditions that will trigger a special activity for the duration of the spell. For example, you might suggest that a knight give up his warhorse to the first beggar you meet. If the condition is not met before the spell ends, the activity will fail. If you or any of your companions damage the target, the spell ends.

\medskip\textbf{Summon Animals}\index[Spells]{Summon Animals}\\
\textbf{School}: Animals and Plants\\
\textbf{Level}: 3, Uncommon\\
\textbf{Casting Time}: 2 Actions\\
\textbf{Range}: 18 meters\\
\textbf{Components}: V, S\\
\textbf{Duration}: 1 hour\\
You summon fey spirits that take the form of beasts and appear in unoccupied spaces that you can see and are within range. Choose one of the following options to determine what appears:\\


- A beast of challenge rating 2 or lower\\
- Two beasts of challenge rating 1 or lower\\
- Four beasts of challenge rating 1/2 or lower\\
- Eight beasts of challenge rating 1/4 or lower\\
\medskip

Each beast is also considered a fey, and disappears when it drops to 0 Hit Points or when the spell ends. \\
Summoned creatures are friendly to you and your companions. Roll initiative for summoned creatures as a group, which acts on its own round. They obey any verbal command given to them (without needing you to take actions). If you don't give commands to the beasts, they will defend themselves from hostile creatures, but will take no other action.\\
\textbf{For each magical Critical Success rolled} two more beasts will appear in the Magic Test.


\medskip\textbf{Summon Mount}\index[Spells]{Summon Mount}\\
\textbf{School}: Animals and Plants\\
\textbf{Level}: 2, Common\\
\textbf{Casting Time}: 10 minutes\\
\textbf{Range}: 9 meters\\
\textbf{Components}: V, S\\
\textbf{Duration}: 1 hour\\
You summon a spirit that takes the form of an unusually intelligent, strong, and loyal mount, forming a lasting bond with it. Appearing in an unoccupied, ranged space, the steed takes the form of your choice, such as that of a warhorse, pony, camel, moose, or hound (the Arbiter may give you the ability to summon steeds as well other types of animals). The steed has the stats of the chosen form, though it is celestial, fey, or demon type (your choice) instead of its normal type. Additionally, if your steed has Intelligence -3 or less, its Intelligence becomes -2, and it gains the ability to understand one language of your choice among those you can speak. Your steed serves as a mount, both in and out of combat, and you possess an instinctive bond with it, allowing you to fight as a whole.\\
When the steed drops to 0 Hit Points, it disappears, leaving no physical form behind. you can dismiss the steed at any time as an action, causing it to disappear. In either case, casting this spell again summons the same steed, restored to full Hit Points.\\
You can't have more than one steed bonded by this spell at a time. As an action, you can free the steed from this bond at any time, making it disappear.\\
\textbf{For each magical critical success rolled} in the Magic Test the spell lasts an hour longer.

\medskip\textbf{Summon elemental}\index[Spells]{Summon elemental}\\
\textbf{School}: Air, Water, Earth, Fire\\
\textbf{Level}: 5, Rare\\
\textbf{Cast Time}: 1 minute\\
\textbf{Range}: 27 meters\\
\textbf{Components}: V, S, M (burning incense for air, malleable clay for earth, sulfur and phosphorus for fire, or water and sand for water) \\
\textbf{Duration}: 1 hour\\
You summon an elemental minion. You choose an area made up of water, air, fire, or earth within range that fills a 3m cube. An elemental of challenge rating 5 or lower appropriate to your chosen area appears in an unoccupied space within 3 meter of it. The elemental disappears when it drops to 0 Hit Points or the spell ends.\\
The elemental is friendly to you and your companions for the duration. He rolls initiative for the elemental, which acts during his own round. He obeys any verbal command given to him (if the command is complex he consumes actions). If you don't issue commands to the elemental, it will defend itself against hostile creatures, but take no other actions.\\
Each Spell List can only summon its specific Elemental \\
\textbf{For two magical critical successes rolled} in the Magic Test the challenge rating of the summoned elemental increases by 1

\medskip\textbf{Sun Flare}\index[Spells]{Sun Flare}\index{Yamato Wave Cannon}\\
\textbf{School}: Invocation\\
\textbf{Level}: 6, Uncommon\\
\textbf{Casting Time}: 2 Actions\\
\textbf{Range}: Self (20m line)\\
\textbf{Components}: V, S, M (a magnifying glass)\\
\textbf{Duration}: Concentration, max 1 minute\\
A beam of brilliant light explodes from your hand in a line 1 meter wide and 20 meters long. Each creature in the line must make a Fortitude save. On a failed save, the creature takes 6d8 light damage and is blinded until your next round. On a successful save, she takes half damage and is not blinded. Undead and oozes have -1d6 on this save. You can create a new line of luminosity as an action during any of your rounds until the spell ends. \\
For the duration, a particle of bright light shines in your hand. It produces light in a 10m radius and dim light for an additional 10 meters. This light is considered sunlight.\\
\textbf{In case of two magical critical successes obtained} the spell ends after the first ray but the line is 6 meters wide, 108 meters long, the Light damage becomes 12d8.

\medskip\textbf{Solar Flare}\index[Spells]{Solar Flare}\\
\textbf{School}: Invocation\\
\textbf{Level}: 8, Rare\\
\textbf{Casting Time}: 2 Actions\\
\textbf{Range}: 45 meters\\
\textbf{Components}: V, S, M (fire and a piece of sunstone)\\
\textbf{Duration}: Instant\\
Intense sunlight illuminates in a 20m radius centered on a point you choose within range. All creatures within the light must make a Fortitude save. On a failed save, a creature takes 12d6 light damage and is blinded for 1 minute. On a successful save, she takes half damage and is not blinded by the spell. Undead and oozes have -2d6 on this save. A creature blinded by this spell makes another Fortitude save at the end of each of its rounds. If she succeeds at her Saving Throw, she is no longer blinded.\\
In its area, this spell dispels any darkness generated by a spell.\\
\textbf{For each magical critical success rolled} in the Magic Test the damage increases by 6d6.

\medskip\textbf{Supervision and Interdiction}\index[Spells]{Supervision and Interdiction}\\
\textbf{School}: Abjuration\\
\textbf{Level}: 6, Uncommon\\
\textbf{Casting Time}: 10 minutes\\
\textbf{Range}: Contact\\
\textbf{Components}: V, S, M (burned incense, a small measure of sulfur and oil, a tied noose, a small amount of earthen colossus blood, and a small rod of silver worth at least 10 gp)\\
\textbf{Duration}: 24 hours\\
Create a ward that protects up to 225 square meters of floor space (an area 15 meters square, or one hundred 1 meter squares, or twenty-five 3 meter squares). The restricted area can be up to 6 meters high, and shaped however you like. You can interdict several floors of a stronghold by dividing the area between them, provided you can walk continuously in each adjacent area while casting the spell\\.
When you cast this spell, you can specify individuals who ignore any or all of this spell's effects. You can also specify a password that, when spoken aloud, renders the user immune to these effects.\\
Guard and interdiction creates the following effects within the interdicted area.\\
\textit{Corridors}. Fog fills all forbidden corridors, making them heavily obscured. Additionally, at each intersection or fork in the passage that offers a choice of direction, there is a 50\% chance that a creature, excluding you, will believe it is going in the opposite direction to the one it chose.\\
\textit{Doors}. All doors in the warded area are magically locked, as if sealed by an Magic Lock spell. Additionally, you can cover up to ten doors with an illusion (equivalent to the illusory object function of the minor illusion spell) to make them look like simple sections of wall.\\
\textit{Stairs}. Webs cover all stairs in the warded area from top to bottom, as the web spell. These threads grow back in 10 minutes if they are burned or ripped out while vigilance and ward remain active.\\
Other Spells in Effect. You may place one of the following magical effects of your choice within the forbidden area of the building\\


- Place dancing lights in four corridors. You can specify a simple program that the lights will repeat for the duration of vigilance and interdiction.\\
- Place magic mouth in two places.\\
- Place foul smelling cloud in two places. The vapors appear in the place you indicate; they return within 10 minutes if dispersed by wind while vigilance and interdiction is still active.\\
- Place a constant gust of wind in a corridor or room.\\
- Place a suggestion in a place. Select an area 1 meter square on a side, and any creature that enters or passes through that area mentally receives the suggestion.\\

The entire restricted area radiates magic. A dispel magic spell cast against a specific effect, if successful, removes only that effect. You can create a perpetually guarded and forbidden structure by casting this spell in it every day for a year.\\
\textbf{If you roll three magical critical success rolled} the duration is permanent.

\medskip\textbf{Supreme Blessing}\index[Spells]{Supreme Blessing}\\
\textbf{School}: Invocation\\
\textbf{Level}: 3, Rare\\
\textbf{Casting Time}: 1 Reaction\\
\textbf{Range}: 27 meters\\
\textbf{Components}: V, S, M (a splash of holy water, 25 gold)\\
\textbf{Duration}: Instant\\
Bless a creature of your choice. The creature can reroll two dice on a single check before knowing whether the check succeeded or failed. The creature chooses whether to take the new rolls or keep the old ones. You must be a Follower or Devotee to cast this spell.\\
\textbf{For each magical critical success rolled} on the Magic Test, the creature gets a +1 bonus on the check.

\medskip\textbf{Symbol}\index[Spells]{Symbol}\\
\textbf{School}: Abjuration\\
\textbf{Level}: 7, Uncommon\\
\textbf{Casting Time}: 2 Actions\\
\textbf{Range}: Contact\\
\textbf{Components}: V, S, M (mercury, phosphorus, and powdered diamond and opal with a total value of at least 1000 gp, which the spell consumes)\\
\textbf{Duration}: Until dispelled or activated\\
When you cast this spell, you inscribe a harmful glyph on a surface (such as a section of floor, wall, or table) or within an object that can be closed to hide the glyph (such as a book, scroll, or chest). ). If you choose a surface area, the glyph can cover a surface area no larger than 3 meter in diameter. If you choose an object, that object must remain in place; if the object is moved more than 3 meter from where the spell was cast, the glyph is broken, and the spell ends without being triggered.\\
The glyph is nearly invisible and can be found with a Survival check against the DC of your spell's save.\\
You decide what activates the glyph when casting the spell.\\
For glyphs inscribed on a surface, typical activation involves touching or standing over the glyph, removing another object covering the glyph, approaching a certain distance from the glyph, or manipulating the object on which the glyph is inscribed. glyph.\\
For glyphs inscribed on an object, typical activation involves opening the object, approaching a certain distance from the object, or seeing or reading the glyph.\\
You can refine the trigger so that the spell triggers only under certain circumstances or according to certain physical characteristics (such as height or weight) or creature species (for example, the protection might work against hags or shapeshifters). You can also set conditions to prevent the glyph from being triggered, such as saying a password.\\
When you inscribe the glyph, choose one of the following options as its effect. Once activated, the glyph glows, filling a 20m radius sphere of dim light for 10 minutes, after which the spell ends. Any creature in the sphere when the glyph activates becomes the target of its effect, as does a creature that enters the sphere for the first time during a round or ends its round there.\\


- \textit{Dementia}. Each target must make a Will save. On a failed save, the target becomes insane for 1 minute. An insane creature can't take actions, doesn't understand what others say to it, can't read, and only speaks in a stutter. The Arbiter controls its movements, which are erratic.\\
- \textit{Discord}. Each target must make a Fortitude save. On a failed save, the target begins bickering and arguing with another creature for 1 minute. During this time, he is unable to make any meaningful communication and has -1d6 on attack rolls and ability checks. Ache. Each target must make a Fortitude save. On a failed save, the target is incapacitated by searing pain.\\
- \textit{Death}. Each target must make a Fortitude save, taking 10d10 void damage on a failed save, or half as much damage on a successful one. \\
- \textit{Fear}. Each target must make a Will save and become frightened for 1 minute on a failed save. While frightened, the target throws whatever it was holding and must move at least 10 meters away from the glyph during each of its rounds, if able.\\
- \textit{Mistrust}. Each target must make a Will save. On a failed save, the target is overcome with despair for 1 minute. During this time, it can't attack or target any creatures with harmful abilities, spells, or other magical effects. \\
- \textit{Sleep}. Each target must make a Will save, and fall unconscious for 10 minutes on a failed save. A creature awakens if it takes damage or if someone uses an action to awaken it.\\
- \textit{Stun}. Each target must make a Will save, and be stunned for 1 minute on a failed save.


\medskip\textbf{Thaumaturgy}\index{Cantrip - Thaumaturgy}\\
\textbf{School}: Universal\\
\textbf{Level}: 0, Uncommon\\
\textbf{Cast Time}: 2 Actions\\
\textbf{Range}: 9 meters\\
\textbf{Components}: V\\
\textbf{Duration}: Maximum 1 minute\\
You manifest a minor trick within range, a sign of supernatural power. You create one of the following magical effects within range:\\

- Your voice sounds three times louder than normal for 1 minute.\\
- Cause the flames to flicker, grow, dim, or change color for 1 minute.\\
- Cause harmless tremors on the ground for 1 minute.\\
- You create an instantaneous noise, such as a rumble of thunder, the cry of a crow, or an eerie whisper, originating from a point within range of your choice.\\
- Cause an unlocked door or window to swing open or slam shut.\\
- Change the look of your eyes for 1 minute.\\

If you cast this spell multiple times, you can keep up to three one-minute effects active at a time, and you can stop these effects as an action.\\
\textbf{For each magical Critical Success rolled} in the Magic Test you can manifest an additional magical effect.

\medskip\textbf{Laydel's Tear}\index[Spells]{Laydel's Tear}\\
\textbf{School}: Invocation\\
\textbf{Level}: 2, Very Rare/Common\\
\textbf{Casting Time}: 2 Action/1 Action\\
\textbf{Range}: 36 meters\\
\textbf{Components}: V, S, M (a teardrop of caster)\\
\textbf{Duration}: Instant\\
The caster imbues a tear with magic which he throws at the opponent, requiring a ranged spell attack roll.
The creature suffers 1d6+2d6 of damage, to establish the type of damage consult the table with the values of the first d6 rolled.

\medskip

\begin{tabular}{l|l}
	\textbf{1d6}&\textbf{Energy}\\
	\hline
	1 &Fire\\
	2 &Electricity\\
	3 &Cold\\
	4 &Sound\\
	5 &Void\\
	6 &Force\\
\end{tabular}

\medskip

The damage the target takes is equal to the Energy type that results from the first d6. If the first die is a 6 and one of the other dice is a 6 as well, then roll 1d6 again and add to the damage.

For a Laydel Devotee this spell is Common and has a casting time of 1 Action and can continue to roll further d6s of damage as long as he continues to roll 6s with that die.

\medskip\textbf{Telekinesis}\index[Spells]{Telekinesis}\\
\textbf{School}: Transmutation\\
\textbf{Level}: 5, Uncommon\\
\textbf{Casting Time}: 2 Actions\\
\textbf{Range}: 18 meters\\
\textbf{Components}: V, S\\
\textbf{Duration}: Concentration, max 10 minutes \\
You gain the ability to move or manipulate creatures or objects through thought. When you cast this spell, and as 2 Actions during each round, you can exert your will on a creature or object that you can see within range, causing the appropriate effect below. You can target the same target round after round, or choose a new one each time. If you switch targets, the previous target is no longer affected by the spell.
\textit{Creature}. You may attempt to move a Huge or smaller creature. Make a Will Saving Throw with bonus given by your spellcasting ability contested by Fortitude Sanving Throw of target creature. If you win the contest, you move the creature 10 meters in any direction, including up, but without exceeding the spell's range. Until the end of your next round, the creature is restrained in your telekinetic grip. A creature lifted high is suspended in mid-air.\\
On subsequent rounds, you can use 2 Actions to attempt to maintain your telekinetic hold on the creature by repeating the contest.
\textit{Object}. You can attempt to move an object weighing up to 250kg. If the item is not being worn or carried, you automatically move it 10 meters in any direction, but without exceeding the spell's range.\\
If the item is worn or carried by a creature, you must make a Will Saving Throw with bonus given by your spellcasting ability contested by Fortitude Sanving Throw of target creature carrying object. If you win the contest, you drag the object away from that creature and move it 10 meters in either direction, without exceeding the spell's range.\\
You can exert precise control over objects with your telekinetic grip, allowing you to manipulate a simple tool, open a door or container, insert or retrieve an object from an open container, or pour material into a vial.

\medskip\textbf{Telepathic Bond}\index[Spells]{Telepathic Bond}\\
\textbf{School}: Divination\\
\textbf{Level}: 5, Rare\\
\textbf{Casting Time}: 2 Actions\\
\textbf{Range}: 9 meters\\
\textbf{Components}: V, S, M (pieces of eggshells from two different species of creatures)\\
\textbf{Duration}: 1 hour\\
You establish a telepathic link between up to eight willing creatures of your choice within range, psychically linking each creature to the others for the duration of the spell. Creatures with an Intelligence score of -3 or lower ignore this spell. Until the spell ends, the targets can communicate telepathically via this link, whether or not they share a common language. Communication is possible at any distance, but cannot extend across different planes of existence.\\
\textbf{For each magical critical success rolled} in the Magic Test the duration increases by 1 hour.

\medskip\textbf{Teleport}\index[Spells]{Teleport}\\
\textbf{School}: Summon\\
\textbf{Level}: 7, Common\\
\textbf{Casting Time}: 2 Actions\\
\textbf{Range}: 3 meters\\
\textbf{Components}: V\\
\textbf{Duration}: Instant\\
This spell instantly teleports you and eight other willing creatures (or a single object) within range and that you can see, of your choice, to a destination of your choosing. If the target is an object, it must be able to fit into a 3m cube, and it can't be held or carried by an unwilling creature.\\
The destination you choose must be known to you, and it must be on the same plane of existence as you are. Your familiarity with the destination determines whether you can get there.\\
The DM rolls a d100 and consults the table.
\end{multicols}
\medskip
\begin{tabular}{lllll}
\toprule
d100 &Error&Similar Area&Off Target&On Target\\
Permanent circle&-&-&-&01-100\\
Associated Object&-&-&-&01-100\\
Very Familiar&01-05&06-13&14-24&25-100\\
Visa by accident&01-33&34-43&44-53&54-100\\
Seen once&01-43&44-53&54-73&74-100\\
Description&01-43&44-53&54-73&74-100\\
False Destination&01-50&51-100&-&-\\
\end{tabular}
\medskip
\begin{multicols}{2}
\textbf{NOTE}: Teleporting from Curyan to Tiya and vice versa has only 5\% of success.	
	

\medskip\textbf{Teleportation Circle}\index[Spells]{Teleportation Circle}\\
\textbf{School}: Summon\\
\textbf{Level}: 5, Uncommon\\
\textbf{Cast Time}: 1 minute\\
\textbf{Range}: 3 meters\\
\textbf{Component}: V, M (rare chalks and inks infused with precious gems worth at least 50 gp, which the spell consumes)\\
\textbf{Duration}: 1 round\\
As you cast the spell, you draw a 3m-diameter circle on the floor, inscribed with sigils that connect your location to a permanent teleportation circle of your choice, whose sigil sequence you know, and which is on the same plane of existence as you are. A glowing portal opens within the circle you draw and remains open until the end of your next round. Any creature that enters the portal instantly reappears within 1 meter of the target circle or in non-space
nearest occupied, if cannot appear within 1 meter of it.\\
Many large temples, guilds, and other important locations have permanent teleportation circles etched somewhere in their vicinity. Each of these circles has a unique sigil sequence: a series of magical runes arranged in a specific pattern.\\ When you gain the ability to cast this spell, you learn the sigil sequences of
two destinations on the Material Plane, determined by the Arbiter. During your adventures you can learn new sigil sequences. You can memorize a sigil sequence after studying it for at least 1 minute.\\
You can create a permanent teleportation circle by casting this spell in the same location every day for one year. You don't have to use the teleportation circle when casting the spell this way.\\
\textbf{NOTE}: Teleporting from Curyan to Tiya and vice versa has only 5\% of success.


\medskip\textbf{Thunder Wave}\index[Spells]{Thunder Wave}\\
\textbf{School}: Air\\
\textbf{Level}: 1, Common\\
\textbf{Casting Time}: 2 Actions\\
\textbf{Range}: Self (3m cube)\\
\textbf{Components}: V, S\\
\textbf{Duration}: Instant\\
A wave of thunderous force projects from you. Each creature in a 2m sphere that originates from you must make a Fortitude save. On a failed save, a creature takes 2d8 sound damage and is moved 3 meter away from you. On a successful save, the creature takes half damage and is not driven away. Additionally, unanchored objects that are wholly within the area are pushed 3 meter away from you by the spell's effect, and the spell produces a thunderous boom audible up to 100 meters.\\
\textbf{For each magical critical success rolled} in the Magic Test the damage increases by 1d8.

\medskip\textbf{Time Stop}\index[Spells]{Time Stop}\\
\textbf{School}: Transmutation\\
\textbf{Level}: 9, Very Rare\\
\textbf{Casting Time}: 2 Actions\\
\textbf{Range}: Personal\\
\textbf{Components}: V\\
\textbf{Duration}: Instant\\
You briefly stop the flow of time for everyone but you. Time does not pass for other creatures, while you take 1d4+1 rounds in a row, during which you can take actions and move as usual. This spell ends if any actions you use during this time, or any effects you create during this time, affect a creature other than yourself or an item worn or carried by someone other than you. Also, the spell ends if you move to a place more than 300 meters away from where you cast it.\\
\textbf{For each magical critical success rolled} in the Magic Test, the duration increases by 1 round. On two Critical Magical Successes you can exclude another creature from stopping time.

\medskip\textbf{Tongues}\index[Spells]{Tongues}\\
\textbf{School}: Divination\\
\textbf{Level}: 3, Common\\
\textbf{Casting Time}: 2 Actions\\
\textbf{Range}: Contact\\
\textbf{Components}: V, M (a small clay model of a ziggurat)\\
\textbf{Duration}: 1 hour\\
This spell gives the creature you were in contact with when you cast the spell the ability to understand any spoken language it hears. Additionally, when the target speaks, any creature that knows at least one language and can hear the target understands what it says. \\
\textbf{For each magical critical success rolled} in the Magic Test the duration doubles or affects another creature.

\medskip\textbf{Tracer Bolt}\index[Spells]{Tracer Bolt}\\
\textbf{School}: Invocation\\
\textbf{Level}: 1, Uncommon\\
\textbf{Casting Time}: 2 Actions\\
\textbf{Range}: 36 meters\\
\textbf{Components}: V, S\\
\textbf{Duration}: 1 round\\
A flash of light travels to a creature of your choice within range. Make a ranged spell attack against the target. On a hit, the target takes 4d6 Light damage, and the next attack roll made against it before the end of your
next round he has +1d6 to AR, thanks to the mystical dim light that will continue to glow around the target until then. \\
\textbf{For each magical critical success rolled} in the Magic Test the damage increases by 1d6.

\medskip\textbf{Transformation}\index[Spells]{Transformation}\\
\textbf{School}: Transmutation\\
\textbf{Level}: 9, Rare\\
\textbf{Cast Time}: 2 Actions\\
\textbf{Range}: Personal\\
\textbf{Components}: V, S, M (a circlet of jade worth at least 1,500 gp, which you must place on your head before casting the spell)\\
\textbf{Duration}: 1 hour\\
For the duration, you take the form of a different creature. The new form can be that of any creature whose challenge rating is equal to or lower than your MP. The creature can't be a construct or undead, and you must have seen it at least once. You turn into an average specimen of that creature, one with no specific Abilities. You can remain in the assumed form until the spell ends. You automatically revert if you fall unconscious, drop to 0 Hit Points, or die. Your game statistics are replaced by the stats of the chosen creature, except for your Traits, and your Intelligence, Wisdom, and Charisma scores. You keep all your proficiency in skills and Saving Throws, in addition to gaining those of the creature. If the creature has the same Feats as you and the bonus listed in its statistics is higher than yours, use the creature's bonus instead of yours. You cannot use any additional actions or lair actions of the new form.\\
When you transform, you assume the creature's Hit Points and Hit Dice. When you revert to your normal form, you return to the number of Hit Points you had before you transformed. However, if you revert because you were reduced to 0 Hit Points, all excess damage is reverted to your original form. Unless the excess damage reduces your normal form to 0 Hit Points, you will not fall unconscious. \\
You retain all the benefits of any Feats you possess, race, or other source, and can use them if the new form is physically capable of using them. However, you can't use any of your special senses, such as darkvision, unless the new form also has the same sense. You can only speak if the creature is normally able to speak.\\
When you transform, you choose whether your equipment falls to the ground in your space, merges with your new form, or is worn by it. The worn equipment functions as normal, but it is up to the Arbiter to decide whether it is comfortable for the new form to wear such a piece of equipment, based on the size and dimensions of the creature. Your equipment does not change size or adapt to the new form, and any equipment that the new form cannot wear must be dropped or merged with the new form. Equipment that fuses is ineffective.\\
During the spell's duration, you can use two actions to assume a different form following the same restrictions and rules as your original form, with one exception: if your new form has more Hit Points than your current form, your Hit Points remain at their current level. \\
\textbf{NOTE}: you must be a Devotee of Efreem or Shayalia to lear this spell

\medskip\textbf{Transport via Plant}\index[Spells]{Transport via Plant}\\
\textbf{School}: Animals and Plants\\
\textbf{Level}: 6, Very Rare\\
\textbf{Casting Time}: 2 Actions\\
\textbf{Range}: 3 meters\\
\textbf{Components}: V, S\\
\textbf{Duration}: 1 round\\
This spell creates a magical bond between an inanimate plant of Large or larger size within range and another plant, at any distance, on the same plane of existence. You must have seen or come into contact with the target vegetable at least once. For the duration of the spell, any creature can enter the target plant and exit the target plant using 1 move action.

\medskip\textbf{Tree Translation}\index[Spells]{Tree Translation}\\
\textbf{School}: Animals and Plants\\
\textbf{Level}: 5, Rare\\
\textbf{Casting Time}: 2 Actions\\
\textbf{Range}: Personal\\
\textbf{Components}: V, S\\
\textbf{Duration}: maximum 1 minute\\
You gain the ability to enter a tree and move from within it into another tree of the same species within 150 meters. Both trees must be alive and at least the same size as you. You must use 1 meter of movement to enter the tree. You instantly learn the location of all other trees of the same species within 150 meters, and as part of the movement it takes to enter the tree, you can pass into one of the other trees or exit the tree you entered. You reappear at a point of your choice within 1 meter of the target tree, using 1 more Move Action. If you have no movement left to use, you reappear within 1 meter of the tree you entered.\\
You can use this carrying ability once per round for the duration of the spell. You must end each round outside a tree.

\medskip\textbf{True Seeing}\index[Spells]{True Seeing}\\
\textbf{School}: Divination\\
\textbf{Level}: 6, Rare\\
\textbf{Casting Time}: 2 Actions\\
\textbf{Range}: Contact\\
\textbf{Components}: V, S, M (an eye ointment costing 25 gp; made of mushroom powder, saffron, and fat; is consumed by the spell)\\
\textbf{Duration}: 1 hour\\
You cast the spell touching a willing creature. The target gains the ability to see things as they really are. For the duration, the creature has true sight, notices secret doors hidden by magic, and can see into the Ethereal Plane, out to a range of 16 meters.

\medskip\textbf{Twigs to Serpents}\index[Spells]{Twigs to Serpents}\\
\textbf{School}: Animals and Plants\\
\textbf{Level}: 3, Uncommon\\
\textbf{Casting Time}: 2 Actions\\
\textbf{Range}: 18 meters\\
\textbf{Components}: V, S, M (several twigs and a drop of snake venom)\\
\textbf{Duration}: Concentration up to 1 minute\\
You transform 1d4 twigs, +1 for each time you took the Animals and Plants Magic List, into venomous snakes. Snakes always act, in your turn, in unison and perform the same Action against the same opponent.

These snakes, considered tiny objects, have Defense 13 and 10 Hit Points. If they drop below 0 Hit Points, they return to broken twigs.

As an Action you can command the snakes to attack. Make an attack roll as per the melee spell attack for each snake against a creature within 1 meter of them. Each snake that strikes deals 1 penetration damage and forces a DC 14 Fortitude save, on a failed save the creature takes 2d4 poison damage, or half on a successful one.

With one Action you can command the snakes to move up to 6 meters.\\
\textbf{Every magical critical success you roll} in the Magic Test creates a new snake.

\medskip\textbf{Uncontrollable Laughter}\index[Spells]{Uncontrollable Laughter}\\
\textbf{School}: Enchantment\\
\textbf{Level}: 1, Uncommon\\
\textbf{Casting Time}: 2 Actions\\
\textbf{Range}: 9 meters\\
\textbf{Components}: V, S, M (small cakes and a feather being waved in the air)\\
\textbf{Duration}: 1 minute
One creature that you choose within range and that you can see perceives everything as tremendously hilarious and amusing, breaking into uproarious laughter while affected by this spell. The target must succeed at a Will save or fall prone, incapacitated and unable to stand for the duration. Creatures with an Intelligence score of -2 or less ignore the effect. \\
At the end of each of its rounds and each time it takes damage, the target can make another Will save. The target has +1d6 on its Saving Throw if it took damage in the round. If it succeeds, the spell ends.

\medskip\textbf{Understanding Languages}\index[Spells]{Understanding Languages}\\
\textbf{School}: Divination\\
\textbf{Level}: 1, Common\\
\textbf{Casting Time}: 2 Actions\\
\textbf{Range}: Personal\\
\textbf{Components}: V, S, M (a pinch of salt and soot)\\
\textbf{Duration}: 1 hour\\
For the duration, you understand the literal meaning of any spoken language you hear.\\
\textbf{For each magical critical success rolled} in the Magic Test, the duration doubles. With three magical critical success you also understand written languages.

\medskip\textbf{Understanding the Writings}\index[Spells]{Understanding the Writings}\\
\textbf{School}: Divination\\
\textbf{Level}: 2, Uncommon\\
\textbf{Casting Time}: 2 Actions\\
\textbf{Range}: Personal\\
\textbf{Components}: V, S, M (a dash of silver and dry ink)\\
\textbf{Duration}: 1 hour\\
For the duration, you understand any non-magical written language you see. You must be in contact with the surface on which the words are written. It takes 1 minute to read a page of text. This spell does not decode secret messages in text or glyph, such as an arcane sigil, that is not part of a written language.\\
\textbf{For each magical critical success rolled} in the Magic Test, the duration doubles.\\
\textbf{NOTE}: If you are a Nethergal Devotee the spell is common and lasts 2 hours.

\medskip\textbf{Vampirich Touch}\index[Spells]{Vampirich Touch}\\
\textbf{School}: Necromancy\\
\textbf{Level}: 3, Common\\
\textbf{Casting Time}: 2 Actions\\
\textbf{Range}: Personal\\
\textbf{Components}: V, S\\
\textbf{Duration}: 1 minute \\
Contact with your shadow-shrouded hand can drain the life force of others to heal your wounds. Each round you can make one melee attack, 2 Actions, with spell against a creature within reach. If you hit, the target takes 3d6 Void damage and you regain Hit Points equal to half the Void damage you dealt.\\
While you have this spell active you are considered distracted from casting other spells.\\
\textbf{For each magical critical success rolled} in the Magic Test the damage increases by 1d12.

\medskip\textbf{Veiled Step}\index[Spells]{Veiled Step}\\
\textbf{School}: Summon\\
\textbf{Level}: 2, Uncommon\\
\textbf{Casting Time}: 1 Immediate Action\\
\textbf{Range}: Personal\\
\textbf{Components}: V\\
\textbf{Duration}: Instant\\
Quickly enveloped in a silver haze, you teleport up to 10 meters to an unoccupied space that you can see.\\
\textbf{If you roll two magical critical successes} in Magic Test, you can switch with a willing creature.

\medskip\textbf{Wall of Fire}\index[Spells]{Wall of Fire}\\
\textbf{School}: Fire\\
\textbf{Level}: 4, Common\\
\textbf{Cast Time}: 2 Actions\\
\textbf{Range}: 36 meters\\
\textbf{Components}: V, S, M (a small piece of phosphorus)\\
\textbf{Duration}: 1 minute\\
You create a wall of fire on a solid surface within range. You can create a wall up to 20 meters long, up to 6 meters high, and 15 centimeters thick, or a circular wall up to 6 meters in diameter, 6 meters high, and 15 centimeters thick. The wall is opaque and remains for the duration. \\
When the wall appears, each creature in its area must make a Reflex save. A creature takes 5d8 fire damage on a failed save, or half as much on a successful one. One side of the wall, selected by you when you cast this spell, deals 5d8 fire damage to each creature that ends its round within 3 meter of that side or inside the wall. A creature takes the same damage when it enters the wall for the first time during a round. The other side of the wall deals no damage.\\
\textbf{For each magical critical success rolled} in the Magic Test the damage increases by 2d8.

\medskip\textbf{Wall of Force}\index[Spells]{Wall of Force}\\
\textbf{School}: Invocation\\
\textbf{Level}: 5, Common\\
\textbf{Cast Time}: 2 Actions\\
\textbf{Range}: 36 meters\\
\textbf{Components}: V, S, M (a pinch of dust made by crushing a clear gem)\\
\textbf{Duration}: Concentration\\
An invisible wall of force forms at a point you choose within range. The wall appears in any orientation you want, as a horizontal or vertical barrier, or at an angle. It can float in the air or rest on a solid surface. You can give it the shape of a hemispherical dome or a sphere with a maximum radius of 3 meters, or you can give it the appearance of a flat surface made up of up to ten panels measuring 3 meters by 3 meters. Each panel must be contiguous to another panel. In any form, the wall is 50cm thick and remains for the duration of the spell. If the wall cuts through a creature's space, when it appears, the creature is pushed to one side of the wall (your choice). Nothing can physically go through the wall, everyone has complete cover behind the wall. It is immune to all damage and cannot be dispelled by dispel magic. However, the wall is instantly destroyed by the disintegrate spell. The wall also extends across the Ethereal Plane, preventing ethereal travelers from passing through.

\medskip\textbf{Wall of Ice}\index[Spells]{Wall of Ice}\\
\textbf{School}: Water\\
\textbf{Level}: 6, Common\\
\textbf{Casting Time}: 2 Actions\\
\textbf{Range}: 36 meters\\
\textbf{Components}: V, S, M (a small piece of quartz)\\
\textbf{Duration}: 10 minutes\\
You create a wall of ice on a solid surface within range. You can create a hemispherical dome or sphere with a maximum radius of 3 meters, or you can create a flat surface composed of up to ten square panels of 3 meters on each side. Each panel must be contiguous to at least one other panel. In any form, the wall is 30cm thick and remains for the duration. \\
If the wall passes through a creature's space when it appears, the creature is pushed to one side of the wall (your choice) and must make a Reflex save. On a failed save, the creature takes 10d6 cold damage, or half as much damage on a successful one. \\
The wall is an object that can be damaged and broken through. Each 3m section has Defence 12 and 30 Hit Points, and is vulnerable to fire damage. Reducing a 3m section to 0 Hit Points destroys it and leaves a breeze of icy wind in the space that was occupied by the wall. A creature that moves through this wind-chill breeze for the first time in a round must make a Fortitude save. On a failed save, the creature takes 5d6 cold damage, or half as much damage on a successful one. \\
\textbf{For each magical critical success rolled} in the Magic Test the damage increases by 2d8.

\medskip\textbf{Wall of Stone}\index[Spells]{Wall of Stone}\\
\textbf{School}: Invocation\\
\textbf{Level}: 5, Common\\
\textbf{Cast Time}: 2 Actions\\
\textbf{Range}: 36 meters\\
\textbf{Components}: V, S, M (a small block of granite)\\
\textbf{Duration}: 10 minutes\\
A nonmagical solid stone wall forms at a point of your choice within range. The wall is 15 centimeters thick and is made up of 10 panels measuring 3 by 3 meters. Each panel must be contiguous to at least one other panel. Alternatively, you can create 3 x 6 meter panels that are just 7.5 centimeters thick.\\
If, when it appears, the wall crosses a creature's space, the creature is pushed to one side of the wall (your choice). If the creature is surrounded on all sides by the wall (or the wall and another solid surface), the creature can make a Reflex save. If she succeeds, she can use her Reaction Action to move up to her speed so she is no longer trapped in the wall.\\
The wall can be any shape you like, though it can't occupy the same space as a creature or object. The wall may not even be vertical or rest on a plane. It must, however, merge with and be supported by already existing stone. Thus, you can use this spell to bridge a chasm or create a ramp.\\
If you create such a non-vertical wall longer than 6 meters, you'll need to halve the size of each panel to create supports. You can roughly shape the stone to create battlements, glacis, and so on. The wall is an object made of stone that can be damaged and broken through. Each panel has Defence 15, Hardness 15, and 15 Hit Points per cm of thickness. Reducing a panel to 0 Hit Points destroys it and may cause connected panels to collapse, at the Arbiter's discretion. If you maintain your concentration on this spell for its entire duration, the wall becomes permanent and can't be dispelled. Otherwise, the wall disappears when the spell ends.

\medskip\textbf{Wall of Thorns}\index[Spells]{Wall of Thorns}\\
\textbf{School}: Animals and Plants\\
\textbf{Level}: 6, Uncommon\\
\textbf{Cast Time}: 2 Actions\\
\textbf{Range}: 36 meters\\
\textbf{Components}: V, S, M (a handful of thorns)\\
\textbf{Duration}: maximum 10 minutes\\
You create a wall of sturdy, malleable, entangled bushes filled with sharp thorns. The wall appears within range on a solid surface and remains for the duration. The wall you can create can be up to 20 meters long, up to 3 meter high, and up to 1 meter thick, or a circle that has a diameter of 6 meters and up to 6 meters high and 1 meter thick. The wall blocks line of sight.\\
When the wall appears, each creature in its area must make a Reflex save. On a failed save, a creature takes 7d8 piercing damage, or half as much damage on a successful one. A creature can move through the wall, albeit slowly and painfully. For every 1 meter the creature moves through the wall, it must expend 6 meters of movement. In addition, the first time a creature enters the wall during a round or ends its round within it, the creature must make a Reflex save. It takes 7d8 slashing damage on a failed save, or half as much damage on a successful one. \\
\textbf{For each magical critical success rolled} in the Magic Test the damage increases by 2d8.

\medskip\textbf{Wall of Wind}\index[Spells]{Wall of Wind}\\
\textbf{School}: Air\\
\textbf{Level}: 3, Uncommon\\
\textbf{Cast Time}: 2 Actions\\
\textbf{Range}: 36 meters\\
\textbf{Components}: V, S, M (a tiny fan and a feather of exotic origin)\\
\textbf{Duration}: 1 minute\\
A wall of strong wind rises from the ground at a point you choose within range. You can create a wall up to 15 meters long, 4 meters high and 30 centimeters thick. You can shape the wall in any way you like as long as it forms a continuous path on the ground. The wall remains for the spell's duration. When the wall appears, each creature within its area must make a Fortitude save. A creature takes 3d8 bludgeoning damage on a failed save, or half as much damage on a successful one. Strong wind keeps haze, smoke and other gases away. Small or smaller flying creatures can't pass through the wall. Light materials dragged into the wall fly up. Arrows, bolts, and other normal ammunition are deflected and automatically miss (boulders hurled by giants and siege engines, and similar ammunition, are unaffected). Creatures in gaseous form cannot pass through it.\\
\textbf{For each magical critical success rolled} in the Magic Test, the duration increases by 1 minute.

\medskip\textbf{Water Breathing}\index[Spells]{Water Breathing}\\
\textbf{School}: Water, Air\\
\textbf{Level}: 3, Common\\
\textbf{Cast Time}: 2 Actions\\
\textbf{Range}: 9 meters\\
\textbf{Components}: V, S, M (a straw or straw)\\
\textbf{Duration}: 24 hours\\
This spell allows up to ten willing creatures that you can see within range to breathe underwater until the spell ends. Affected creatures also retain their normal breathing pattern.\\
\textbf{For each magical critical success you roll} in Magic Test, you may choose one additional creature.

\medskip\textbf{Water Walk}\index[Spells]{Water Walk}\\
\textbf{School}: Water\\
\textbf{Level}: 3, Common\\
\textbf{Casting Time}: 2 Actions\\
\textbf{Range}: 9 meters\\
\textbf{Components}: V, S, M (a piece of cork)\\
\textbf{Duration}: 1 hour\\
This spell grants the ability to move through liquid surfaces (such as water, acid, mud, snow, quicksand, or lava) as if they were harmless solid ground (creatures walking through molten lava can still take heat damage or melt into the acid ). Up to ten willing creatures within range and that you can see receive this ability for the duration. If your target is immersed in liquid, the spell returns the target to the surface of the liquid at a rate of 10 meters per round.\\
\textbf{For each magical critical success rolled} in the Magic Test the spell lasts 1 hour longer or affects another creature.

\medskip\textbf{Web}\index[Spells]{Web}\\
\textbf{School}: Animals and Plants\\
\textbf{Level}: 2, Common\\
\textbf{Casting Time}: 2 Actions\\
\textbf{Range}: 18 meters\\
\textbf{Components}: V, S, M (a piece of cobweb)\\
\textbf{Duration}: 1 hour\\
You summon a thick mass of dense, sticky web at a point of your choice within range. For the duration, the web fills a 6 meters cube from that point. The web is hindering terrain and makes that area slightly darkened.\\
If the web isn't anchored between two solid masses (such as walls or trees) or stretched across a floor, wall, or ceiling, the summoned web collapses on itself, and the spell ends at the start of your next round. Canvases spread out on a flat surface have a depth of 1 meter.\\
Each creature that begins its round in the web or that enters it during its round must make a Reflex save. On a failed save, the creature is restrained for as long as it remains in the web or until it breaks free.\\
A creature entangled in the webs can use 2 Actions to make a new Saving Throw. If she passes it, she is no longer hindered.\\
The web is flammable and if exposed to flames, it ignites immediately and for 2 rounds, dealing 2d4 points of fire damage to each creature within its area.

\medskip\textbf{Windwalk}\index[Spells]{Windwalk}\\
\textbf{School}: Air\\
\textbf{Level}: 6, Uncommon\\
\textbf{Cast Time}: 1 minute\\
\textbf{Range}: 9 meters\\
\textbf{Components}: V, S, M (fire and Holy Water)\\
\textbf{Duration}: 8 hours\\
For the duration, you and up to ten other willing creatures within range that you can see assume gaseous form, becoming clouds. While in cloud form, a creature has a flying speed of 100 meters and has resistance to damage from nonmagical weapons. Reverting to normal form takes 1 minute, during which time the creature is incapacitated and cannot move. Until the spell ends, a creature can revert to cloud form, which requires a one-minute transformation. If a creature is in cloud form and flying when the effect ends, the creature descends 20 meters per round per minute until it lands safely. If it fails to land after 1 minute, the creature will fall the remaining distance.

\medskip\textbf{Wish}\index[Spells]{Wish}\\
\textbf{School}: Summon\\
\textbf{Level}: 9, Legendary\\
\textbf{Cast Time}: 2 Actions\\
\textbf{Range}: Personal\\
\textbf{Components}: V,S,M (gems for 20000 gp)\\
\textbf{Duration}: Instant\\
Wish is the most powerful spell a mortal creature can cast. By simply speaking aloud and consuming the gems hold in hand, you can change the very foundation of reality according to your needs. \\
The basic use of this spell is to reproduce the effect of any other spell of level 8 or lower. You don't have to meet any of the spell's requirements, including any expensive material components. The spell simply takes effect.\\
Alternatively, you can create one of the following effects of your choice:\\
- You create an item worth up to 25,000 gp that is not a magic item. The object cannot be larger than 100 meters in any dimension, and appears in an unoccupied space on the terrain.\\
- Allow up to twenty creatures you can see to regain all Hit Points, and end all effects on them described by the greater restoration spell.\\
- Grant up to ten creatures you can see resistance to a damage type of your choice.\\
- Grant up to ten creatures you can see immunity to a single spell or other magical effect for 8 hours. For example, you could make you and all your companions immune to the lich's life drain attack.\\
- You cancel any recent event forcing you to reroll any rolls made in the last round (including your last round). Reality remodels itself to accommodate the new result. You can make the new roll +2d6 or -2d6, you can choose whether to use the original roll or the new roll. You may even be able to achieve more than the goals in the examples above.\\


\medskip
State your wishes as much as possible to the Arbiter. The Arbiter has great leeway in deciding what happens in these cases; the greater the desire, the greater the chances that something will go wrong. The spell may simply fail, the desired effect may only partially manifest, or you may suffer unforeseen consequences, depending on how you uttered the wish. The stress of casting this spell to create any effect other than reproducing another spell weakens you.\\
After you have withstood its stress, whenever you cast a spell until you finish a night's rest, you take 1d10 void damage per level/2 of the spell. This damage cannot be reduced or decreased in any way. In addition, your Constitution drops to -3, if not already -3 or lower, for 2d4 days. \\
For each day you spend resting and doing nothing but light activity, your remaining recovery time decreases by 2 days.
Roll 1d100, if you roll 1 to 33\% you will never be able to cast wish again due to the stress suffered, 34\%-66\% aged 5 years, 67\% -99\% no particular effect happens, 100\% you immediately recover the stress of casting.\\
\textbf{On 2 magical Critical Successes rolled} you suffer no side effects from casting Wish.

\medskip\textbf{Wither}\index[Spells]{Wither}\\
\textbf{School}: Necromancy\\
\textbf{Level}: 4, Uncommon\\
\textbf{Cast Time}: 2 Actions\\
\textbf{Range}: 9 meters\\
\textbf{Components}: V, S\\
\textbf{Duration}: Instant\\
Necromantic energy engulfs one creature of your choice that you can see within range, draining it of sap and vitality. The target must make a Fortitude save. On a failed save, the target takes 8d8 void damage, or half as much damage on a successful save. The spell has no effect on undead or constructs.\\
If the target is a non-magical plant that isn't also a creature, such as a tree or bush, it makes no Saving Throw, shrivels, and dies instantly. \\
\textbf{For each magical critical success rolled} in the Magic Test the damage increases by 1d8.\\
\textbf{Save Critical Success/Failure}: On a critical failure the damage is doubled, on a critical success the damage is further halved


\medskip\textbf{Walking on air}\index[Spells]{Walking on air}\\
\textbf{School}: Air\\
\textbf{Level}: 4, Uncommon\\
\textbf{Cast Time}: 2 Actions\\
\textbf{Range}: 18 meters\\
\textbf{Components}: V, S, M (a handful of beans and holy water)\\
\textbf{Duration}: 1 Turn\\
For the duration, a willing creature you choose within range that you can see can walk through the air as if walking on solid ground. If a creature is in the air when the effect ends, the creature drops 60 feet per round for one minute, then falls the remaining distance.\\
\textbf{For each Magical Critical Success} obtained in the Magic Test you can add a creature as a target. When you cast the spell, the target creatures must be within 30 feet of each other.


\medskip\textbf{Wonderful Palace}\index[Spells]{Wonderful Palace}\\
\textbf{School}: Summon\\
\textbf{Level}: 7, Legendary\\
\textbf{Cast Time}: 1 minute\\
\textbf{Range}: 90 meters\\
\textbf{Components}: V, S, M (a miniature portal carved from ivory, a small piece of polished marble, and a tiny silver spoon, each of these items must be worth at least 5 gp)\\
\textbf{Duration}: 24 hours\\
Within range, you summon an extradimensional dwelling that lasts for the spell's duration. Choose where your front door is located. The entrance door emits a slight brightness and is 1 meter wide by 3 meters high. You and all the creatures you indicated when you cast the spell can enter the extradimensional dwelling, as long as the door remains open. You can open or close the door if you are within 10 meters of it. While closed, the gate is invisible.\\
Beyond the door is a magnificent entrance, beyond which numerous rooms unfold. The atmosphere is clean, fresh and welcoming. You can create as many floors as you like, but the space cannot exceed 50 cubes each with 3m edges. The place is furnished and decorated as you like. Contains enough food to satisfy a 9-course banquet for 100 people. A staff of 100 nearly transparent servants are at the service of anyone who enters it. It's up to you to decide the visual appearance of these minions and their clothing. They absolutely obey your orders. Each minion can perform any task a normal human minion can perform, but they cannot attack or take any action that could directly harm another creature. The servants can then gather items, clean, mend, fold clothes, light fires, serve food, pour wine, and so on. Minions can go anywhere in the mansion, but cannot leave. The furniture and other objects created by this spell turn to smoke when taken out of the dwelling. When the spell ends, any creatures inside the extradimensional space are ejected into the open space closest to the exit.\\
\textbf{Note}: the spell cast in the same place every day for a year becomes permanent.\\
\textbf{For each magical critical success rolled} in Magic Test double the duration or deduct a month from the count to make it permanent.

\medskip\textbf{My Word: Kill}\index[Spells]{My Word: Kill}\\
\textbf{School}: Enchantment\\
\textbf{Level}: 9, Rare\\
\textbf{Casting Time}: 1 Immediate Action\\
\textbf{Range}: 18 meters\\
\textbf{Components}: V\\
\textbf{Duration}: Instant\\
You speak a word of power that causes one creature you can see within range to instantly die. If the creature you choose has 100 Hit Points or fewer, it dies. Otherwise, the spell has no effect.

\medskip\textbf{Word of Withdrawal}\index[Spells]{Word of Withdrawal}\\
\textbf{School}: Summon\\
\textbf{Level}: 6, Rare\\
\textbf{Casting Time}: 2 Actions\\
\textbf{Range}: 1 meter\\
\textbf{Components}: V\\
\textbf{Duration}: Instant\\
You and up to five willing creatures within 1 meter of you instantly teleport to a previously designated safe location, called a sanctuary. You, and any teleported creatures with you, reappear in the unoccupied space closest to the spot you indicated when you prepared this shrine (see below). If you cast this spell without first preparing a shrine, the spell has no effect.\\
You must indicate a sanctuary, which is dedicated or strongly connected to your Patron. If you attempt to cast the spell to take you to an area that isn't dedicated by your Patron, the spell has no effect.

\medskip\textbf{Wound}\index[Spells]{Wound}\\
\textbf{School}: Necromancy\\
\textbf{Level}: 6, Uncommon\\
\textbf{Casting Time}: 2 Actions\\
\textbf{Range}: 18 meters\\
\textbf{Components}: V, S\\
\textbf{Duration}: Instant\\
You unleash a virulent disease upon a creature you can see within range. The target must make a Fortitude save. The target takes 14d6 void damage on a failed save, or half as much damage on a successful one.  If the target fails its Saving Throw, its maximum Hit Points are reduced for 1 hour by an amount equal to the void damage it took. Any effect that removes a disease allows the character's maximum Hit Points to return to their normal value before that time expires.

\medskip\textbf{Zone of Truth}\index[Spells]{Zone of Truth}\\
\textbf{School}: Enchantment\\
\textbf{Level}: 2, Uncommon\\
\textbf{Casting Time}: 2 Actions\\
\textbf{Range}: 18 meters\\
\textbf{Components}: V, S\\
\textbf{Duration}: 10 minutes\\
You create a magical zone that protects against trickery in a 3m-radius sphere centered on a point of your choice within range. Until the spell ends, a creature that enters the spell's area for the first time in a round, or starts its round therein, must make a Will save. On a failed save, the creature can't deliberately speak lies while in range of the spell. You know whether a creature has succeeded or failed its Saving Throw. A subject creature is aware of this and can therefore avoid answering questions it would normally answer with a lie. This creature can give elusive answers as long as it stays within the bounds of truth.


\end{multicols}

%\vspace{2cm}
%\begin{center}
%\includegraphics[width=0.4\linewidth]{immagini/Bocca_della_Verita.png}
%\medskip
%\textit{La Bocca della Verita', Chiesa di Santa Maria in Cosmedin, Roma}
%\end{center}


\pagebreak

\subsection{Ancient and Lost Spells}

The spells present here have been lost to history and only legends refer to their existence. \\
These spells have not only the Legendary Rarity but only the most learned have heard of them. Very often these are spells that were against the will of some Patron who proceeded to eliminate them from history and knowledge.

\begin{multicols}{2}


\medskip\textbf{Contact Other Planes}\index[Spells]{Contact Other Planes}\\
\textbf{School}: Divination\\
\textbf{Level}: 5, Legendary\\
\textbf{Casting Time}: 1 minute\\
\textbf{Range}: Personal\\
\textbf{Components}: V\\
\textbf{Duration}: 1 minute\\
You mentally contact a demigod, the spirit of a long-dead sage, or some other mysterious entity from another plane. Contacting extraplanar intelligence can strain or even break your mind. When you cast this spell, you make a DC 15 Will save. On a failed save, you take 6d6 points of damage and are insane until dawn the next day. While insane, you can't take actions, you can't understand what other creatures are saying, you can't read, and you only ramble. The greater restoration spell can end this effect. On a successful save, you can ask the entity up to five questions. You must ask the questions before the spell ends. The Arbiter will answer each question with one word: "yes", "no", "maybe", "never", "irrelevant" or "confused" (if the entity does not know the answer to the question). If a one-word answer might be misleading, the Arbiter might instead give a short sentence answer.	


\medskip\textbf{Find Familiar}\index[Spells]{Find Familiar}\\
\textbf{School}: Animals and Plants\\
\textbf{Level}: 1, Legendary\\
\textbf{Casting Time}: 1 hour\\
\textbf{Range}: 3 meters\\
\textbf{Components}: V, S, M (10 gp of charcoal, incense, and herbs which must be consumed by the fire in a brass brazier)\\
\textbf{Duration}: Instant\\
You gain the service of a familiar, a spirit that takes on an animal form of your choice: seahorse, crow, weasel, hawk, cat, crab, owl, lizard, fish (Froet), octopus, bat, spider, frog (toad) , rat or venomous snake. Appearing in an unoccupied space within range, the familiar has the stats of the chosen form, though it is celestial, fey, or demon (your choice) instead of a beast. Your familiar acts independently of you, but always obeys your commands. In combat, he rolls his initiative and acts during his round. A familiar cannot attack, but can perform other actions as normal.
You can have no more than one familiar at a time. \\
\textbf{Check Familiar Feat} For familiar abilities, you must have the Familiar feat.


\medskip\textbf{Guardian of Faith}\index[Spells]{Guardian of Faith}\\
\textbf{School}: Summon\\
\textbf{Level}: 4, Legendary\\
\textbf{Casting Time}: 2 Actions\\
\textbf{Range}: 9 meters\\
\textbf{Components}: V\\
\textbf{Duration}: 8 hours\\
A Large ghostly guardian appears for the duration and floats in an unoccupied space that you choose within range and that you can see. The guardian occupies that space and is indistinguishable except by a glowing sword and shield with your Patron's symbol.\\
Any creature hostile to you that enters a space within 3 meter of the guardian for the first time during a round must make a Reflex save. The creature takes 20 Light/Void damage on a failed save, or half as much damage on a successful one. The guardian vanishes after dealing a total of 60 damage.


\medskip\textbf{Hunter's Mark}\index[Spells]{Hunter's Mark}\\
\textbf{School}: Divination\\
\textbf{Level}: 1, Legendary\\
\textbf{Casting Time}: 2 Actions\\
\textbf{Range}: 27 meters\\
\textbf{Components}: V \\
\textbf{Duration}: Concentration, max 1 hour\\
Choose a creature you can see within range. The creature is mystically marked as your prey. Until the spell ends, you deal an additional 1d6 points of damage to the target each time you hit them with a weapon attack, and you have +1d6 on Mindfulness or Survival checks to find it.\\
If the target drops to 0 Hit Points before the spell ends, you can use an immediate action during your next round to mark a new creature.\\
\textbf{For each magical critical success you roll} in the Magic Test you can maintain your concentration on the spell for another hour.

\medskip\textbf{Moonglow}\index[Spells]{Moonglow}\\
\textbf{School}: Invocation\\
\textbf{Level}: 2, Legendary\\
\textbf{Casting Time}: 2 Actions\\
\textbf{Range}: 36 meters\\
\textbf{Components}: V, S, M (several night beauty seeds and a piece of opalescent plush)\\
\textbf{Duration}: Concentration, max 1 minute\\
A silvery beam of pale light shines in a 1m-radius, 12m tall cylinder centered at a point within range. Until the spell ends, a dim light fills the cylinder. \\
When a creature enters the spell's area for the first time in a round or begins its round there, it is engulfed in spectral flames that cause terrible pain, and it must make a Fortitude save. It takes 2d10 light damage on a failed save, or half as much damage on a successful one. A shapeshifter makes a Saving Throw of -1d6. It immediately reverts to its original form on a failure and cannot assume a different form until it exits the spell's light.\\
During each of your rounds after casting the spell, you can use an action to move the
beam of 18 meters in any direction. \\
\textbf{For each magical critical success rolled} in the Magic Test the damage increases by 1d10.
	
	
\medskip\textbf{Planar Ally}\index[Spells]{Planar Ally}\\
\textbf{School}: Summon\\
\textbf{Level}: 6, Legendary\\
\textbf{Casting Time}: 10 minutes\\
\textbf{Range}: 18 meters\\
\textbf{Components}: V, S\\
\textbf{Duration}: Instant\\
You plead with an otherworldly entity for help. The being must be known to you: a god, a primordial, a prince of demons, or some other creature of great power. That entity sends a celestial, elemental, or demon loyal to it to aid you, causing the creature to appear in an unoccupied space within range. If you know the name of a specific creature, you may speak its name when casting this spell to request that creature's aid, though you may still receive another (Arbiter's discretion).\\
When the creature appears, it is under no compulsion to act in any particular way. You can ask the creature to perform a service in exchange for a reward, but it doesn't have to satisfy you. The required task could be easy ("fly us over the edge" or "help us fight this battle") or complex ("spy on our enemies" or "protect us as we explore the dungeon"). You must be able to communicate with the creature to hire its services. The reward can take many forms. A celestial might request a sizable donation of gold or magical items from an allied temple, while a demon might request a human sacrifice or the gift of treasure. Some creatures may trade their services for a quest you will undertake on their own. As a general rule, a task that can be measured in minutes requires a reward of 100 gp per minute. A task measured in hours requires 1,000 gp per hour. A task measured in days (maximum 10 days) requires 10,000 gp per day. The Arbiter can adjust these rewards based on the circumstances under which the spell was cast. If the task is aligned with the creature's morals, the payment request could be halved or even cancelled. Non-hazardous tasks usually ask for only half of the suggested payment, while very dangerous tasks may require higher donations. It is rare for these creatures to accept tasks that seem suicidal.\\
After the creature completes the task, or when the agreed period of service is over, the creature will return to its home plane after reporting to you, if appropriate to the task performed and if possible. If you are unable to agree on a price for the creature's services, the creature will immediately return to its home plane. A creature enlisted to join your party is treated as a member of it, and receives a full share of the rewards in experience points.

\medskip\textbf{Planar Binding}\index[Spells]{Planar Binding}\\
\textbf{School}: Abjuration\\
\textbf{Level}: 5, Legendary\\
\textbf{Casting Time}: 1 hour\\
\textbf{Range}: 18 meters\\
\textbf{Components}: V, S, M (a jewel worth at least 1000 gp, which the spell consumes)\\
\textbf{Duration}: 24 hours\\
With this spell, you seek to bind a celestial, elemental, fey, or demon into your service. The creature must remain within range for the entire casting of the spell. (Usually, the creature is first summoned to the center of an inverted magic circle to keep it trapped while this spell is being cast.) Upon completion of the casting, the target must make a Will save. On a failed save, he is bound to your service for the duration. If the creature was summoned or created by another spell, that spell's duration is extended to match the duration of this spell. A bonded creature must carry out your instructions to the best of its ability. You might command the creature to accompany you on an adventure, to protect a place, or to deliver a message. The creature obeys your instructions to the letter, but if it is hostile to you, it will try to twist your words to its own ends. If the creature completely complies with your instructions before the spell ends, if you are on the same plane of existence it will return to you to inform you of the success. If you are on different planes of existence, she will return to the place where you bound her and remain there until the spell ends.
\textbf{For each magical critical success you roll} in Magic Test you double the creature's permanence.

\medskip\textbf{Portal}\index[Spells]{Portal}\\
\textbf{School}: Summon\\
\textbf{Level}: 9, Legendary\\
\textbf{Casting Time}: 2 Actions\\
\textbf{Range}: 18 meters\\
\textbf{Components}: V, S, M (a diamond worth at least 5000 gp)\\
\textbf{Duration}: Concentration, max 1 minute\\
You summon in an unoccupied space within range that you can see a portal connected to a specific place on a different plane of existence. The portal is a circular opening you create, from 1 to 6 meters in diameter. You can point the portal in any direction you want. The portal remains for the duration.\\
The portal has a front and back on both planes where it appears. Travel through the portal is only possible by moving from the front. Anything that does is instantly transported to the other plane, appearing in the unoccupied space closest to the portal.\\
Gods and other planar rulers can prevent portals created by spells from opening in their presence or anywhere in their domains. When you cast this spell, you can speak the name of a specific creature (pseudonym, title, or nickname don't work). If that creature is on a different plane than yours, the portal opens in proximity to the named creature and draws the creature through it, to the nearest unoccupied space on your side of the portal. You hold no special power over the creature, and it is free to act as the Arbiter sees fit. He may leave, attack you, or help you.


\medskip\textbf{Planar Shift}\index[Spells]{Planar Shift}\\
\textbf{School}: Summon\\
\textbf{Level}: 7, Legendary\\
\textbf{Casting Time}: 2 Actions\\
\textbf{Range}: Contact\\
\textbf{Components}: V, S, M (a forked metal rod worth at least 250 gp, tuned to a specific plane of existence)\\
\textbf{Duration}: Instant\\
You and up to eight other willing creatures, who join hands in a circle, are transported to a different plane of existence. You can specify a target destination in generic terms, and you will reappear in or near that destination, at the Arbiter's discretion.\\
Alternatively, if you know the sigil sequence of a teleportation circle to another plane of existence, the spell can take you to that circle. If the teleport circle is too small to fit all the creatures you carry, they will reappear in the closest unoccupied space to the circle.\\
You can use this spell to banish an unwilling creature to another plane. Choose a creature within reach and make a melee spell attack against it. On a hit, the creature must make a Will save. On a failed save, the creature is transported to a random location on the plane of existence you specify. A creature thus transported will have to find its own way back to your current plane of existence.


\medskip\textbf{Pure Resurrection}\index[Spells]{Pure Resurrection}\\
\textbf{School}: Necromancy\\
\textbf{Level}: 9, Legendary\\
\textbf{Casting Time}: 1 hour\\
\textbf{Range}: Contact\\
\textbf{Components}: V, S, M (a little Holy Water and diamonds worth 25,000 gp, which the spell consumes)\\
\textbf{Duration}: Instant\\
You cast the spell upon touching a creature, not an Elf, that has been dead no more than 200 years and that died of any reason but not old age. If its soul is free and willing, the creature will return to life with all its Hit Points. \\
This spell closes all wounds, neutralizes any poisons, cures all diseases, and removes any curses that afflicted the creature when it died. The spell replaces damaged organs and limbs.\\
The spell can also provide a new body if the original no longer exists, in which case you must speak the creature's name. The creature will then reappear in an unoccupied space of your choice within 3 meter of you. \\
\textbf{This spell shouldn't be available. Only a Patron can revive.}

\medskip\textbf{Resurrection}\index[Spells]{Resurrection}\\
\textbf{School}: Necromancy\\
\textbf{Level}: 7, Legendary\\
\textbf{Casting Time}: 1 hour\\
\textbf{Range}: Contact\\
\textbf{Components}: V, S, M (a diamond worth at least 1000 gp, which the spell consumes)\\
\textbf{Duration}: Instant\\
You cast the spell upon touching a creature, not an Elf, that has been dead for no more than a century, that hasn't died of old age, and that isn't undead. If her soul is free and willing, the target will return to life with all of her Hit Points.
This spell neutralizes all poisons and cures the normal diseases that afflicted the creature when she died. It does not, however, remove magical diseases, curses, and the like; if these effects are not removed before casting the spell, they will afflict the target upon its return to life.\\
This spell closes all mortal wounds and restores any missing body parts. Returning from the dead is an ordeal. The target takes a -4 penalty on all attack rolls, Saving Throws, and ability checks. Each time the target finishes a night's rest, the penalty is reduced by 1 until it disappears.\\
Casting this spell to bring back a creature that has been dead for a year or more wears you down. Until the end of a night's rest, you can no longer cast spells and have -1d6 on all attack rolls, ability checks, and Saving Throws.\\
The creature brought back to life must make a DC 13 Fortitude save or not come back to life due to the trauma suffered.\\
\textbf{This spell shouldn't be available. Only a Patron can revive.}


\medskip\textbf{Saving the Dying}\index{Cantrip - Saving the Dying}\\
\textbf{School}: Animals and Plants\\
\textbf{Level}: 0, Legendary\\
\textbf{Casting Time}: 1 round\\
\textbf{Range}: Contact\\
\textbf{Component}: V, S, M (an offering to your Patron of at least 5 gp, which the spell consumes)\\
\textbf{Duration}: Instant\\
A creature at 0 Hit Points, with which you are in contact, returns to 1 HP. The spell has no effect on undead or constructs.\\
\textbf{For each magical critical success you roll} in the Magic Test you heal the creature 1d4 Hit Points.

\medskip\textbf{Spirit Guardians}\index[Spells]{Spirit Guardians}\\
\textbf{School}: Summon\\
\textbf{Level}: 3, Legendary\\
\textbf{Casting Time}: 2 Actions\\
\textbf{Range}: Self (4 meter radius)\\
\textbf{Components}: V, S, M (a holy symbol)\\
\textbf{Duration}: Concentration, max 10 minutes\\
You summon spirits to protect you. For the duration, they will float around you at a distance of 4 meters. You designed the aspect of your Spirit Guardians. When you cast this spell, you can designate any number of creatures to be immune to it. An affected creature's speed is halved within the area, and when a creature enters the area for the first time during a round or begins its round there, it must make a Will save. On a failed save, it takes 3d8 Light or Void damage, or half as much damage on a successful one. \\
\textbf{For each magical critical success rolled} in the Magic Test the damage increases by 1d8


\medskip\textbf{Storm of Vengeance}\index[Spells]{Storm of Vengeance}\\
\textbf{School}: Air, Water\\
\textbf{Level}: 9, Legendary\\
\textbf{Casting Time}: 2 Actions\\
\textbf{Range}: View\\
\textbf{Components}: V, S\\ 
\textbf{Duration}: Concentration, max 1 minute\\
A seething storm cloud forms, centered in a point you can see and spreading out in a radius of 110 meters. The area is lit by lightning, thunder echoes in it, and strong winds sweep through it. When the cloud appears, each creature beneath it (that is, no more than 1500 meters below the cloud) must make a Fortitude save. On a failed save, the creature takes 2d6 sound damage and is deafened for 5 minutes.\\
Each round you maintain your concentration on this spell, the storm produces additional effects during your round.\\
\textit{Round 2}. Acid rain falls from the cloud. Each creature and object under the cloud takes 1d6 acid damage.\\
\textit{Round 3}. You call down six bolts of lightning from the cloud to strike six creatures or objects of your choice under the cloud. A specific creature or object cannot be struck by more than one bolt of lightning. An affected creature must make a Reflex save. The creature takes 10d6 electricity damage on a failed save, or half as much damage on a successful one. \\
\textit{Round 4}. The cloud produces a thick hailstorm. Each creature under the cloud takes 2d6 bludgeoning damage.\\
\textit{Rounds 5-10}. Gusts of wind and freezing rain pound the area under the cloud. The area becomes difficult terrain and is in dim light. Each creature in the area takes 1d6 cold damage. Ranged weapon attacks become impossible in the area. Wind and rain are considered a major distraction for the purposes of maintaining concentration on spells.\\ Finally, gusts of strong winds (ranging from 30 to 75 kilometers per hour) automatically disperse fog, mist, and similar phenomena into the area, whether natural or magical.

\medskip\textbf{Summon Celestials}\index[Spells]{Summon Celestials}\\
\textbf{School}: Summon\\
\textbf{Level}: 7, Legendary\\
\textbf{Casting Time}: 1 minute\\
\textbf{Range}: 27 meters\\
\textbf{Components}: V, S\\
\textbf{Duration}: 10 minutes\\
You summon a celestial of challenge rating 4 or lower, which appears in an unoccupied space within range and which you can see. The celestial disappears when it drops to 0 Hit Points or the spell ends. The celestial is friendly to you and your companions for the duration. He rolls initiative for the celestial, which acts during his own round. He obeys any verbal command given to him (without you needing to take actions), as long as he doesn't violate his Traits. If you don't issue commands to the celestial, it will defend itself against hostile creatures, but take no other actions.\\
\textbf{For each magical critical success rolled} in the Magic Test you increase the summoned creature's CR by one.

\medskip\textbf{Summon Imp}\index[Spells]{Summon Imp}\\
\textbf{School}: Summon\\
\textbf{Level}: 6, Legendary\\
\textbf{Casting Time}: 1 minute\\
\textbf{Range}: 27 meters\\
\textbf{Components}: V, S\\
\textbf{Duration}: 1 hour \\
You summon a fey spirit of challenge rating 6 or lower, or a fey spirit that takes the form of a beast of challenge rating 6 or lower. It appears in an unoccupied space that you can see within range. The fey creature disappears when it drops to 0 Hit Points or when the spell ends.\\
The fey creature is friendly towards you and your companions. Roll initiative for the fey creature, which acts on its own rounds. She obeys any verbal command given to her (without requiring you to take actions), as long as it doesn't violate her Traits. If you don't issue commands, it will defend itself against hostile creatures, but take no other actions.\\
\textbf{For each magical critical success rolled} in the Magic Test you increase the summoned creature's CR by 1.

\medskip\textbf{Summon Woodland Creatures}\index[Spells]{Summon Woodland Creatures}\\
\textbf{School}: Summon\\
\textbf{Level}: 4, Legendary\\
\textbf{Casting Time}: 2 Actions\\
\textbf{Range}: 18 meters\\
\textbf{Component}: V, S, M (a holly berry per summoned creature)\\
\textbf{Duration}: 1 hour \\
You summon fey spirits that appear in unoccupied spaces within range that you can see. Choose one of the following options to determine what appears:\\

- A fey of challenge rating 2 or lower

- Two faeries of challenge rating 1 or lower

- Four faeries of challenge rating 1/2 or lower

- Eight faeries of challenge rating 1/4 or lower

\medskip
A summoned creature disappears when it drops to 0 Hit Points or when the spell ends. Summoned creatures are friendly to you and your companions. Roll initiative for summoned creatures as a group, which acts on its own round. They obey any verbal command given to them (without needing you to take actions). If you don't give commands to the fey, they will defend themselves from hostile creatures, but will take no other action.\\
\textbf{For each magical Critical Success rolled} two more creatures will appear in the Magic Test.



\end{multicols}

\pagebreak

\

\vfill

\

\begin{center}
\includegraphics[keepaspectratio,width=0.70\textwidth]{immagini/Goetic_circle_from_The_Lesser_Key_of_Solomon.png}

\textit{The Circle of Solomon and Triangle of Solomon from The Goetia: The Lesser Key of Solomon the King, The Book of Evil Spirits by L. W. De Laurence}
\end{center}

\pagebreak



\subsection{List of Spells by List, Rarity, Level}\hypertarget{elencoscuole}{}

Next to the title of each Magic List is indicated the Characteristic connected to establish the maximum level that can be cast, next to each spell is indicated the Rarity and the level of the magic. 

\begin{multicols}{3}

\flushleft{\textbf{List of Water - Dexterity}}

Ray of Frost, Common, 0\\
Cloud of Mist, Common, 1\\
Create or Destroy Water, Common, 1\\
Cure Light Wounds, Common, 1\\
Energy Weapon, Very Rare, 1\\
Acid Arrow, Common, 2\\
Nauseating Fog, Uncommon, 3\\
Remove Poison, Common, 3\\
Sleet Storm, Very Rare, 3\\
Water Breathing, Common, 3\\
Water Walk, Common, 3\\
Control Water, Common, 4\\
Fire Shield, Uncommon, 4\\
Ice Storm, Uncommon, 4\\
Summon Lesser Elementals, Uncommon, 4\\
Cone of Cold, Common, 5\\
Deadly Mist, Rare, 5\\
Summon Elemental, Rare, 5\\
Freezing Sphere, Rare, 6\\
Wall of Ice, Common, 6\\
Control Weather, Rare, 8\\


\flushleft{\medskip\textbf{List of Air - Charisma}}

Shocking Grasp, Common, 0\\
Cloud of Mist, Common, 1\\
Energy Weapon, Very Rare, 1\\
Feather fall, Common, 1\\
Jump, Common, 1\\
Thunder Wave, Common, 1\\
Fairy Dust, Uncommon, 2\\
Gust of Wind, Common, 2\\
Levitation, Common, 2\\
Call Lightning, Common, 3\\
Fly, Common, 3\\
Lightning Bolt, Common, 3\\
Nauseating Fog, Uncommon, 3\\
Wall of Wind, Uncommon, 3\\
Water Breathing, Common, 3\\
Bubble of Life, Uncommon, 4\\
Ice Storm, Uncommon, 4\\
Summon Lesser Elementals, Uncommon, 4\\
Walking on air, Uncommon, 4\\
Deadly Mist, Rare, 5\\
Summon Elemental, Rare, 5\\
Chained Lightning, Rare, 6\\
Windwalk, Uncommon, 6\\
Control Weather, Very Rare, 8\\
Prismatic Wall, Rare, 9\\

\flushleft{\medskip\textbf{List of Fire - Strength}}

Produce Flame, Common, 0\\
Burning Wave, Common, 1\\
Energy Weapon, Very Rare, 1\\
Fire Bolt, Common, 1\\
Burning Blade, Common, 2\\
Fairy Dust, Uncommon, 2\\
Flamethrower, Rare, 2\\
Flaming Sphere, Common, 2\\
Heat Metal, Uncommon, 2\\
Pyroexpert, Uncommon, 2\\
Searing Ray, Common, 2\\
Blessing of Cattalm, Very Rare, 3\\
Fireball, Common, 3\\
Kyrin Fire Acorn Hail, Rare, 3\\
Fire Shield, Uncommon, 4\\
Summon Lesser Elementals, Uncommon, 4\\
Wall of Fire, Uncommon, 4\\
Flame Strike, Common, 5\\
Summon Elemental, Rare, 5\\
Delayed Fireball, Rare, 7\\
Firestorm, Rare, 7\\
Control Weather, Very Rare, 8\\
Incendiary Cloud, Rare, 8\\
Meteor Storm, Legendary, 9\\
Prismatic Wall, Rare, 9\\

\flushleft{\medskip\textbf{List of Earth - Constitution}}

Repair, Common, 0\\
Energy Weapon, Very Rare, 1\\
Eithne's Mudball, Uncommon, 1\\
Acid Arrow, Common, 2\\
Kyrin Currant Juice Concentrate, Uncommon, 2\\
Kyrin Lemon Hail, Very Rare, 2\\
Kyrin Reading the land, Uncommon, 2\\
Pass Without Trace, Common, 2\\
One with stone, Common, 3\\
Stone Shape, Common, 4\\
Stoneskin, Uncommon, 4\\
Summon Lesser Elementals, Uncommon, 4\\
Pass door, Uncommon, 5\\
Stone to Mud - Mud to Stone, Uncommon - Very Rare, 5\\
Summon Elemental, Rare, 5\\
Wall of Stone, Common, 5\\
Flesh to Stone - Stone to Flesh, Uncommon - Rare, 6\\
Move earth, Uncommon, 6\\
Earthquake, Very Rare, 8\\
Meteor Storm, Legendary, 9\\


\flushleft{\medskip\textbf{Abjuration - Intelligence}}

Resistance, Common, 0\\
Alarm, Common, 1\\
Magic Armour, Uncommon, 1\\
Protection from Energy Minor, Rare, 1\\
Protection from Good and Evil, Common, 1\\
Sanctuary, Common, 1\\
Shield, Common, 1\\
Shield of Faith, Common, 1\\
Bond of Warding, Common, 2\\
Magic Lock, Common, 2\\
Protection from Poisons, Uncommon, 2\\
Anti-Detection, Uncommon, 3\\
Blessing of Life, Rare, 3\\
Counterspell, Common, 3\\
Dispel Magic, Common, 3\\
Glyph of Warding, Common, 3\\
Magic Circle, Common, 3\\
Protection from Energy, Common, 3\\
Remove Curse, Common, 3\\
Banishment, Common, 4\\
Bubble of Life, Uncommon, 4\\
Death Ward, Uncommon, 4\\
Freedom of Movement, Common, 4\\
Private Shrine, Very Rare, 4\\
Dispel Good and Evil, Rare, 5\\
Dispel Magic Advanced, Rare, 5\\
Orb of Invulnerability, Common, 6\\
Prohibition, Uncommon, 6\\
Supervision and Interdiction, Uncommon, 6\\
Symbol, Uncommon, 7\\
Anti-Magic Field, Rare, 8\\
Holy Aura, Common, 8\\
Mind Shield, Uncommon, 8\\
Imprison, Rare, 9\\


\flushleft{\medskip\textbf{Animals and Plants - Wisdom}}

Magic Club, Common, 0\\
Poison Spray, Uncommon, 0\\
Entangle, Common, 1\\
Find Familiar, Legendary, 1\\
Friendship with Animals, Uncommon, 1\\
Grease, Common, 1\\
Purify Food and Drink, Common, 1\\
Speak with Animals, Common, 1\\
Barkskin, Common, 2\\
Good Berry, Common, 2\\
Kyrin Acorn Hail, Uncommon, 2\\
Kyrin Currant Juice Concentrate, Uncommon, 2\\
Locate Animals and Plants, Uncommon, 2\\
Messenger Animal, Common, 2\\
Pass Without Trace, Common, 2\\
Spider Movement, Uncommon, 2\\
Spike Growth, Common, 2\\
Summon Mount, Common, 2\\
Web, Common, 2\\
Kyrin Fire Acorn Hail, Rare, 3\\
Kyrin Lemon Hail, Very Rare, 3\\
Plant Growth, Uncommon, 3\\
Speak with Plants, Rare, 3\\
Summon Animals, Uncommon, 3\\
Twigs to Serpents, Uncommon, 3\\
Dominate Beasts, Common, 4\\
Giant Insect, Uncommon, 4\\
Locate Creature, Common, 4\\
Metamorphosis, Common, 4\\
Anti-Life Shell, Uncommon, 5\\
Awakening, Rare, 5\\
Insect Plague, Rare, 5\\
Kyrin Chestnut Hail, Very Rare, 5\\
Reincarnation, Rare, 5\\
Tree Translation, Rare, 5\\
Transport via Plant, Very Rare, 6\\
Wall of Thorns, Uncommon, 6\\
Animal Forms, Rare, 8\\
Pure Polymorph, Rare, 9\\

\flushleft{\medskip\textbf{Enchantment - Charisma}}

Cruel Hoax, Common, 0\\
Finger, Rare, 0\\
Anathema, Common, 1\\
Charm Person, Common, 1\\
Command, Common, 1\\
Heroism, Uncommon, 1\\
Sleep, Common, 1\\
Uncontrollable Laughter, Uncommon, 1\\
Hold Person, Common, 2\\
Calm Emotions, Common, 2\\
Enrapture, Common, 2\\
Snooze, Legendary, 2\\
Suggestion, Common, 2\\
Zone of Truth, Uncommon, 2\\
Blessing of Cattalm, Very Rare, 3\\
Hold Person Advanced, Uncommon, 4\\
Compulsion, Uncommon, 4\\
Confusion, Common, 4\\
Dominate Beasts, Very Rare, 4\\
Dominate Person, Uncommon, 5\\
Geas, Rare, 5\\
Modify Memory, Very Rare, 5\\
Mass Suggestion, Uncommon, 6\\
Contagious Confusion, very Rare, 8\\
Dislike/Like, Rare, 8\\
Dominate Monster, Uncommon, 8\\
Irresistible Dance, Legendary, 8\\
Mental Fell, Rare, 8\\
My Word: Stun, Uncommon, 8\\
My Word: Kill, Rare, 9\\

\flushleft{\medskip\textbf{Heal - Wisdom}}

Cure Light Wounds, Common, 1\\
Healing Word, Uncommon, 1\\
Aid, Uncommon, 2\\
Lesser Restoration, Common, 2\\
Prayer of Healing, Common, 2\\
Cure Serious Wounds, Uncommon, 3\\
Destroy undead, Uncommon, 3\\
Mass Healing Word, Rare, 3\\
Rebreathe, Very Rare, 3\\
Remove Poison, Common, 3\\
Remove Disease, Common, 4\\
Vigor, Rare, 4\\
Cure Critical Wounds, Uncommon, 5\\
Greater Restoration, Uncommon, 5\\
Heal, Rare, 6\\
Regeneration, Legendary, 7\\
Mass Heal, Legendary, 9\\
Mass Cure Wounds, Uncommon, (variable)\\

\flushleft{\medskip\textbf{Divination - Wisdom}}

Accurate Shot, Common, 0\\
Detect Good and Evil, Common, 1\\
Guidance, Common, 1\\
Understanding Languages, Common, 1\\
Detect Diseases and Poisons, Uncommon, 2\\
Detect Thoughts, Rare, 2\\
Discover Traps, Common, 2\\
Locate Object, Common, 2\\
Omen, Common, 2\\
See Invisibility, Common, 2\\
Understanding the Writings, Uncommon, 2\\
Clairvoyance, Common, 3\\
Tongues, Common, 3\\
Arcane Eye, Common, 4\\
Communion, Rare, 5\\
Communion with Nature, Very Rare, 5\\
Knowledge of Legends, Common, 5\\
Scry, Rare, 5\\
Telepathic Bond, Rare, 5\\
Discover the Path, Uncommon, 6\\
Divination, Common, 6\\
True Seeing, Rare, 6\\
Forecast, Uncommon, 9\\


\flushleft{\medskip\textbf{Summon - Intelligence}}

Acid Surge, Common, 0\\
Create Ale, Rare, 0\\
Magic Hand, Common, 0\\
Cattalm's Slap, Uncommon, 1\\
Floating Disc, Common, 1\\
Invisible Cook, Common, 1\\
Invisible Servant, Common, 1\\
Veiled Step, Uncommon, 2\\
Create Food and Water, Common, 3\\
Black Tentacles, Uncommon, 4\\
Dimension Door, Common, 4\\
Loyal Hound, Rare, 4\\
Secret Chest, Rare, 4\\
Summon Woodland Creatures, Legendary, 4\\
Teleportation Circle, Uncommon, 5\\
Instant Summons, Rare, 6\\
Planar Ally, Legendary, 6\\
Word of Withdrawal, Rare, 6\\
Planar shift, legendary, 7\\
Summon Celestials, Legendary, 7\\
Teleport, Common, 7\\
Wonderful Palace, Legendary, 7\\
Demiplane, Rare, 8\\
Maze, Rare, 8\\
Portal, Legendary, 9\\
Wish, Legendary, 9\\


\flushleft{\medskip\textbf{Illusion - Intelligence}}

Color Spray, Common, 1\\
Disguise Self, Common, 1\\
Illusory Written, Common, 1\\
Silent Image, Common, 1\\
Arcanist's Magic Aura, Uncommon, 2\\
Blur, Common, 2\\
Invisibility, Common, 2\\
Magic Mouth, Common, 2\\
Mirror Image, Common, 2\\
Silence, Common, 2\\
Fear, Uncommon, 3\\
Hypnotic Texture, Common, 3\\
Major Image, Common, 3\\
Phantom Steed, Common, 3\\
Greater Invisibility, Uncommon, 4\\
Halucination of Death, Uncommon, 4\\
Illusionary Terrain, Uncommon, 4\\
Creation, Rare, 5\\
Dream, Uncommon, 5\\
Mislead, Uncommon, 5\\
Seem, Uncommon, 5\\
Programmed Illusion, Uncommon, 6\\
Mirage Arcane, Rare, 7\\
Project Image, Uncommon, 7\\
Simulacrum, Rare, 7\\
Fatal, Rare, 9\\


\flushleft{\medskip\textbf{Invocation - Wisdom}}

Dancing Lights, Uncommon, 0\\
Blazing Smite, Rare, 1\\
Darkness, Invocation, 1\\
Divine Favor, Uncommon, 1\\
Luminescence, Uncommon, 1\\
Tracer Bolt, Uncommon, 1\\
Greater Blessing, Uncommon, 2\\
Laydel's Tear, Very Rare, 2\\
Marking Smite, Common, 2\\
Shatter, Common, 2\\
Shining Smite, Uncommon, 2\\
Spiritual Weapon, Common, 2\\
Blinding Smite, Rare, 3\\
Daylight, Common, 3\\
Hut, Uncommon, 3\\
Send, Common, 3\\
Supreme Blessing, Rare, 3\\
Elastic Sphere, Rare, 4\\
Arcane Hand, Uncommon, 5\\
Hallow, Rare, 5\\
Wall of Force, Common, 5\\
Blade Barrier, Common, 6\\
Circle of Death, Very Rare, 6\\
Contingency, Common, 6\\
Heroes' Feast, Uncommon, 6\\
Solar Flare, Uncommon, 6\\
Arcane Sword, Rare, 7\\
Divine Word, Very Rare, 7\\
Prismatic Spray, Rare, 7\\
Forcecage, Rare, 8\\
Solar Flare, Rare, 8\\


\flushleft{\medskip\textbf{Necromancy - Constitution}}

Chill Touch, Common, 0\\
False Life, Common, 1\\
Scream of pain, Rare, 1\\
Aid, Uncommon, 2\\
Blindness/Deafness, Common, 2\\
Inflict Wounds, Common, 2\\
Inviolate Repose, Uncommon, 2\\
Ray of Fatigue, Common, 2\\
Advanced Blindness/Deafness, Uncommon, 3\\
Animate Dead, Common, 3\\
Bestow curse, uncommon, 3\\
Rebreathe, Very Rare, 3\\
Speak with Dead, Rare, 3\\
Vampirich Touch, Common, 3\\
Wither, Uncommon, 4\\
Contagion, Uncommon, 5\\
Raise Dead, Legendary, 5\\
Create Undead, Uncommon, 6\\
Finger of Death, Common, 6\\
Magic Jar, Very Rare, 6\\
Piercing Gaze, Very Rare, 6\\
Wound, Uncommon, 6\\
Resurrection, Legendary, 7\\
Clone, Uncommon, 8\\
Astral Projection, Very Rare, 9\\
Pure Resurrection, Legendary, 9\\


\flushleft{\medskip\textbf{Transmutation - Dexterity}}

Message, Common, 0\\
Alter Self, Common, 1\\
Fast Step, Very Rare, 1\\
Quick Retreat, Uncommon, 1\\
Darkvision, Common, 2\\
Enhance, Common, 2\\
Enlarge/Reduce, Common, 2\\
Magic Weapon, Common, 2\\
Pick Lock, Common, 2\\
Rope Trick, Common, 2\\
Gaseous Form, Uncommon, 3\\
Haste, Uncommon, 3\\
Intermittent, Uncommon, 3\\
Slow, Uncommon, 3\\
Fabricate, Common, 4\\
Animate Objects, Common, 5\\
Telekinesis, Uncommon, 5\\
Disintegrate, Uncommon, 6\\
Restser's Furious Transformation, Very Rare, 6\\
Conceal, Rare, 7\\
Ethereal Form, Rare, 7\\
Reverse Gravity, Rare, 7\\
Loquaciousness, Rare, 8\\
Time Stop, Very Rare, 9\\
Transformation, Rare, 9\\


\flushleft{\medskip\textbf{Universal - Any}}

Druidic Artifice, Uncommon, 0\\
Holy Flame, Common, 0\\
Magic Mark, Common, 0\\
Prestidigitation, Common, 0\\
Thaumaturgy, Uncommon, 0\\
Bestow Lesser Curse, Common, 1\\
Blessing, Common, 1\\
Detect Magic, Common, 1\\
Identify, Common, 1\\
Light, Common, 1\\
Arcane Dart, Common, 1\\
Minor Illusion, Common, 1\\
Occult Missile, Common, 1\\
Read Magic, Common, 1\\
Bless Water, Common, 2\\
Endless Flame, Legendary, 2\\

\end{multicols}

\vfill

\begin{center}
\includegraphics[width=0.35\linewidth]{immagini/the-discovery-of-witchcraft.png}

\medskip

\textit{"The Discoverie of Witchcraft' by Reginald Scot, 16th century }
\end{center}


\pagebreak

\section{Advantages}\index{Advantages}\hypertarget{vantaggi}{}\label{vantaggiinizio}

\begin{changemargin}{0.3cm}{0.3cm}\begin{emphasis}{I love being a superhero! The working hours are terrible, the pay is non-existent... but at least I don't run the risk of being fired! (PK)
}\end{emphasis}\end{changemargin}\medskip


\begin{multicols}{2}

\lettrine[lines=2, lhang=0.33, loversize=0.25, findent=1.5em]{E}{ach} character can have, and it is not mandatory to have, Advantages. These must be interesting, pleasant, fun and above all playable.

Each Advantage has a cost, payable at each level. As mentioned, it shouldn't be mandatory to take an advantage, nor should you take advantages just because they make you strong. The purpose of an Advantage is to amaze and amuse.

Having an advantage means being different, being a freak, having that detail that makes you special and unique, but not always the strongest, most powerful or invincible. A perk isn't just an ability, it's a role-playing opportunity. The player is invited to be creative in choosing advantages and also in creating new ones, the cost is then decided with the Arbiter. And it is always and in any case the Arbiter who has the last word on the chosen Benefits.

Several advantages do not have a concrete and immediate practical effect but they enrich the background, the history of the character, they introduce opportunities for play and fun. When you choose the advantages, and consequently the disadvantages, it is not like going to stock up on super powers and extraordinary abilities, but on peculiarities, foibles, specialties that the character possesses and which once again make him different, unique, only yours .

Therefore advantages and disadvantages must also and above all be played and interpreted.

The Arbiter could also insert advantages and disadvantages peculiar to the character's characterization such as immunity to diseases, healing touches, extrasensory abilities, abilities that modify the relationship with a familiar... Always be scrupulous in analyzing and evaluating the benefits, remembering that there must also be an adequate score given by Disadvantages.


\begin{itemize}[leftmargin=*]

\item
Advantages with{*} and all those with cost 15 or higher are at the Arbiter's discretion in being admitted to the choice.

\item
Advantages are chosen at the first level, each advantage taken at subsequent levels must be agreed with the Arbiter.

\item
The cost points of an Advantage are paid with the points earned from the Disadvantages.

\item
Bonuses given to skills are specific to the check when indicated in parentheses.

\item Unless otherwise indicated, it costs two Action to activate an Advantage (if the effect is not permanent).

\end{itemize}


\begin{changemargin}{0.3cm}{0.3cm}\begin{emphasis}{
From a great advantage comes a great disadvantage! (cit. "With great power comes great responsibility", Amazing Fantasy 15, Stan Lee)
}\end{emphasis}\end{changemargin}\medskip


\subsection{List of Benefits}\index{List of Benefits}

\textbf{Wings of Providence} \index{Wings of Providence}20 : you have wings, the choice of shape and color is yours, usually they stay on your shoulder blades and make you fly. Unless otherwise agreed, air movement is equal to ground movement.

\textbf{Ambidextrous}\index{Ambidextrous} 10: You can use both hands indifferently. Penalties on checks where two hands are used decrease by 2

\textbf{Friend of animals}\index{Friend of animals} 5: +2 on checks to handle animals (even wild ones)

\textbf{Amphibious}\index{Amphibious} 20: You can breathe both underwater and air

\textbf{Rainbow}\index{Rainbow} 5: You are an artist. Your fingers spontaneously produce color

\textbf{Aura of Courage}\index{Aura of Courage} 15: Around you, in a distance within 3 meters, infuse courage. +2 Saving Throw vs natural or magical effects of fear.

\textbf{Claws}\index{Claws} 5: occasionally remember to trim the hooves. 1d4 damage per attack. Natural attacks with the second hand take the damage bonus given by Strength. For 10 cost point claw will do 1d6 damage.

%\begin{center}
%\includegraphics[width=0.3\linewidth]{immagini/claw.png}
%\end{center}


\textbf{Drinking is good for you}\index{Drinking is good for you} 5: Prerequisite: \textit{Liver doesn't count}. Your body metabolizes alcohol very efficiently. A liter of beer restores 1d4 Hit Points, a bottle of liquor restores 1d8 Hit Points. Not if it's bad quality. You can still get drunk.

10: a liter of beer restores 2d4 Hit Points, a bottle of liquor restores 2d8 Hit Points. Not if it's bad quality. You cannot get drunk on natural liquids.

\textbf{Cat fall}\index{Cat fall} 5: you ignore the first 3 meters of the fall. +2 to Stealth.

\textbf{Chameleon}\index{Chameleon} 10-20: Your skin can change color. Time required 1 minute/1 round.

\textbf{Shapeshifter}\index{Shapeshifter} 40: as the Alter Self spell. Can be used every 10 minutes.

\textbf{Walking on air} \index{Walking on air}30: not too controlled. Anything other than walking requires a Dexterity check or falling prone (but not fall on land).

\textbf{Walking on water} \index{Walking on water} 30: but don't put on airs..

\textbf{Magnetic} \index{Magnetic}5-10: you release light whenever you want. luckily not literally. $\pm1/2$ to Charisma checks.

\textbf{Reduced consumption} \index{Reduced consumption}5: drink and eat half as much as a normal man. You are under weight.

\textbf{Metabolism Control} \index{Metabolism Control} 10: The name alone is great! Each round you reduce the damage from Bleeding by 2.

You regain Hit Points as if you had double your Constitution score.

\textbf{Effective Healing} \index{Effective Healing}10: +1d6 Hit Points healed each time you use a Heal spell on yourself or others.

\textbf{Daredevil} \index{Daredevil}10: you like to get into fights, especially if you are in danger. +2 Attack / Defense Rolls while surrounded by three or more opponents.

\textbf{Teeth} \index{Teeth}5: your bite hurts, 1d6. Brush your teeth every once in a while..

\textbf{Universal digestion} \index{Universal digestion}5: as long as it doesn't hurt you eat, +2 Fortitude save vs poisons. Immune to natural stomach upsets.

\textbf{Direction Absolute} \index{Direction Absolute}5: You always know where magnetic north is. You have a +1d6 on orientation checks.

\textbf{Hard to subdue} \index{Hard to subdue}10: +2 Will save against Charm school spells

\textbf{Hard to kill} \index{Hard to kill}5: You don't faint at 0 Hit Points, but at -LV/2 Hit Points. Die at 15+Constitution x 3 Hit Points.

\textbf{Empathy with plants} \index{Empathy with plants}10: I understand the suffering of crushed grass.

\textbf{Empathy} 5: +2 to Sense Emotion checks.

\textbf{Animal Empathy} \index{Animal Empathy}10: +1d6 to checks to handle animals (even wild ones).

\textbf{Spiritual empathy} \index{Spiritual empathy}5: you don't talk to spirits, but you feel their emotions.

\textbf{Hermaphrodite} \index{Hermaphrodite}0: lgbtE!.

\textbf{Forged of Steel} \index{Forged of Steel}5: Through painful operations your skin has been covered with metal plates. Your base Defense is 13.

\textbf{Shadowform} \index{Shadowform}30: Consider being able to transform into a shadow for 1 hour per level. You can only move to shaded areas.

\textbf{Lucky} \index{Lucky}5: 2 times a day you can reroll a 1 on the die to 6, to be declared before the die roll.

\textbf{Very Lucky} \index{Very Lucky} 10: 1 times a day you can reroll a 1 or 2 on the die to 6, to be declared even before the die roll.

\textbf{Accelerated healing}\index{Accelerated healing}: 5 Every morning you recover double the number of Hit Points you would normally recover. Stacks with Metabolism Control. \index{Metabolism Control}.

\textbf{Healer}\index{Healer} 5: you know where to put your hands. +1d6 on First Aid checks

\textbf{Your liver can't be counted} \index{Your liver can't be counted}10: you can drink a lot and you won't get drunk

\textbf{Illuminated} \index{Illuminated}10-20: shed light.. literally. Emit light in a radius of 3/6 meters, 1 hour for level. You can check (20) the emission or not (10).

\textbf{Immune}\index{Immune} 5-20: to what?

\textbf{Invisible} \index{Invisible}40: Your body is invisible. All time. And it's not magic...

\textbf{Wrath} \index{Wrath}5: you are capable of getting angry. +2 to melee damage and -1 to attack and Defense. Every other 5 points +2 gives -1 Roll to Attack and Defense, max 20 points. Duration 4 (even non-consecutive) rounds every 5 points. It is activated with an Action.

\textbf{My shadow is my friend} \index{My shadow is my friend}10: You can place your shadow anywhere within 3 meters. Your shadow can manipulate objects like an invisible minion. You are considered to be able to cast Touch Spells via your shadow (which must be present) within 3 meter. The conditions must exist for there to be a shadow


\begin{center}
\includegraphics[width=0.35\linewidth]{immagini/shadow.png}
\end{center}

\textbf{Bonds of Fury} 15\index{Bonds of Fury} : You can summon ethereal snares that threaten your enemies. 3 times a day with the cost of 2 Actions, all opponents within a radius of 10 meters around you are affected by the Entangle spell until the next round. DC 10 + 1/2 LV + Charisma.

\textbf{Slow and Firm} 5: \index{Slow and Firm}You are exceptionally stable on your feet. You can't be moved or lifted except by a creature 2 sizes larger.

\textbf{Universal language} \index{Universal language}10. Your language skills are impressive. After two days in contact with a new language you are able to speak it correctly. After 7 days away from the environment you forget the language. You gain a +2 on language-based checks.

\textbf{Explosive Magic} \index{Explosive Magic}10: Your damage-causing Evocation spells have an extra die of damage (when it comes to rolling a die...).

\textbf{Fairy Hands} \index{Fairy Hands}10: +1d6 Fairy Hands and Escape Artist checks involving hands. You can take 14 as you would take a 10 on related tests.

\textbf{Hand/Foot Webbed} \index{Hand Webbed Foot}5: +1d6 to Swim checks.

\textbf{Early riser} \index{Early riser}5-10-15: you only need to sleep 6/5/4 hours a night to be fully rested.

\textbf{Medium} \index{Medium}10-20: sometimes you want it, other times they ask for you.

\textbf{Photographic memory}\index{Photographic memory} 20-50: fortunately it is not permanent (50). +2d6 on checks to remember details (Knowledge and Awareness).

\textbf{Hairy nose} \index{Hairy nose}5: Your nostrils filter the toxins in the air you breathe. +2 to related checks. Your nose is sized… not small.

\textbf{You don't sleep}\index{You don't sleep} 20{*}: and I don't know how you do it..

\textbf{You don't age}\index{You don't age} 20{*}: you don't age (but they can kill you anyway).

\textbf{You don't eat, drink} \index{You don't eat, drink}20: and I don't know how you do it..

\textbf{You're not breathing} \index{You're not breathing}20: and I don't know how you do it..

\textbf{The smell of blood} \index{The smell of blood}10: The smell of blood is a powerful drug
Prerequisites: You cannot have "Liver Doesn't Matter". You gain +1 to attack roll and +1 to damage for each enemy you killed with your weapon in the round. This bonus can't exceed +4/+4. The bonus remains active until the round following the last kill made. Creatures less than 3 LV than you do not count.

\textbf{Oracle} \index{Oracle}20: for some it is a curse. The use must always be agreed with the Arbiter.

\textbf{Excellent eyesight} \index{Excellent eyesight}5: you have excellent eyesight (12/10). +2 on related Awareness checks using sight.

\textbf{Excellent smell and taste} \index{Excellent smell and taste}5: you have excellent taste and smell. +2 on Awareness checks that use smell or taste.

You gain +1d6 on checks to recognize a natural potion or poison.

\textbf{Excellent tact} \index{Excellent tact}5: You have excellent tact. you can read with your fingers. You are able to find a hidden door by touching the wall.


\begin{center}
\includegraphics[width=0.9\linewidth]{immagini/braille2.png}
\end{center}

\textbf{Excellent hearing}\index{Excellent hearing} 5: You have excellent hearing. +2 on Awareness checks involving hearing.

\textbf{Talking with animals}\index{Talking with animals} 20: choose a family (sheep, marsupials, guinea pigs...).

\textbf{Talking to plants} \index{Talking to plants}25: I've always wanted to talk to courgettes..

\textbf{Blindsight}(blindsight): \index{Blindsight}30: You can perceive anything with your senses within 20 meters, from smell to heat. You can "see" through and up to 18 meters, 5cm of stone, 10cm of wood, 0.2cm of metal.

\textbf{Perfect balance} 5:\index{Perfect balance} +2 to related Acrobatics checks.

\textbf{Fast Feet}\index{Fast Feet} 10/20: Your movement increases by 1/2 meters.

\textbf{Green thumb}\index{Green thumb} 5: +1d6 to Profession checks (Herbalist, Gardener...).

\textbf{Iron Lungs}\index{Iron Lungs} 5: you can hold your breath for 2*Constitution minutes (minimum 2 minutes).

\textbf{Precognition} 30{*}\index{Precognition}: As the spell Precognition.

\textbf{Recovery}\index{Recovery} 10: Your body produces caffeine spontaneously. You take half the time to recover from the fatigued condition.

\textbf{Resistance}\index{Resistance} 5-10: +1/+2 Reflex or Fortitude or Will save.

\textbf{Damage resistance}\index{Damage resistance} 10: -1 damage. -1 additional damage for every 5 additional points. It establishes the type of resistance (cut, blow, piercing, fire...).

\textbf{Resist Magic}\index{Resist Magic} 20: You have an RM 2.

\textbf{Rebuilding}\index{Rebuilding} 30: losing a hand has never been a problem..

\textbf{Regeneration}\index{Regeneration} 30: +1 Hit Points per turn (do not regenerate limbs).

\textbf{Fast regeneration} 40: +1 Hit Points per round (does not regenerate limbs). You die if they destroy your body (or all that remains is ash).

\textbf{Downsize} 30: You can go down by up to two sizes. Duration up to 8 hours.

\hypertarget{rhinoceros}{}\textbf{Rhinoceros} 10 : Your charge is destructive. Nothing below the strength of iron bars (hardness 15) is considered to be able to stop your charge. You leave a trail of destruction behind you. +2 on charge attack rolls and +1d6 damage.

\textbf{Mind Shield}\index{Mind Shield} 5: +2 Saving Throw against mental influences and controls.

\textbf{Protected senses}\index{Protected senses} 5: +2 Saving Throw against sounds/lights/vapours or spells that act on and through your senses.

\textbf{Common sense}\index{Common sense} 5: if you are about to make a bad impression, a bell warns you.

\textbf{Fashion sense}\index{Fashion sense} 5: You always know how to dress well, even in just a rag.

\textbf{Sense of vibrations} \index{Sense of vibrations} \index{Sense Telluric} (Sense Telluric) 30: everything makes the earth shake a little, or almost, radius of 18 meters around you.

\textbf{Sense of time} \index{Sense of time}5: you always know what time it is, day or night.

\textbf{Spider Sense}\index{Spider Sense} 15: no a radioactive man didn't bite you, but you are extremely sensitive to danger. +2 initiative, you can't be surprised.

\textbf{Fearless} \index{Fearless}10: You are immune to fear, magical or otherwise.

\textbf{Silent} \index{Silent}5: +1d6 on Stealth.

\textbf{Spine} \index{Spine}5: and you're ugly too. 1d4 of damage.

\textbf{Super Dog Tags} \index{Super Dog Tags}5: Reduce Bleed damage by 1 at the end of each round.

\textbf{Language Talent}\index{Language Talent} 5: You learn two languages ​​by investing 1 point in Linguistic Knowledge.

\textbf{Wild Talent}: \index{Wild Talent}Let's talk about it.

\textbf{Icy Touch} \index{Icy Touch}10: By touching a dead man (within 1 day per level) you can see and hear what happened in his last round of life.

\textbf{Troll} \index{Troll}60: You regenerate 5 Hit Points per round even if the Hit Points are negative. You also regenerate limbs. You can only be "killed" by fire or acid. A condition may still keep you at negative Hit Points (eg, submerged underwater).

\textbf{Subsonic hearing}\index{Subsonic hearing} 10: Hear frequencies inaudible to humans (like a dog)

\textbf{Seeing the invisible} \index{Seeing the invisible}15: X-ray vision is better.. smudge..

\textbf{Understanding the truth}\index{Understanding the truth} 5: The truth has a ring all its own. +1d6 on Sense Emotion checks.

\textbf{Demon Sight} \index{Demon Sight}15: See in total darkness, even magical, up to 18 meters.

\textbf{Perimeter Vision} \index{Perimeter Vision}5: Sole ? +2 on Side Awareness checks.

\textbf{Telescopic Vision}\index{Telescopic Vision} 10: +1d6 on Awareness and vision checks but from afar.

\textbf{Persuasive voice} \index{Persuasive voice}5: +2 on Charisma checks using voice,

\textbf{Subsonic voice}\index{Subsonic voice} 10: Emit sounds inaudible to humans. Dogs hate you.


\end{multicols}

\pagebreak

\section{Disadvantages}\index{Disadvantages}

\begin{changemargin}{0.3cm}{0.3cm}\begin{emphasis}{If you have to be crippled, better be a rich cripple. (Tyrion Lannister)}\end{emphasis}\end{changemargin}\medskip


\begin{multicols}{2}

\lettrine[lines=2, lhang=0.33, loversize=0.25, findent=1.5em]{A}{n} disadvantage characterizes the character, defines its limits and fears. Each character must have at least 1 role disadvantage and this does not give them bonus points.

The points taken with the psycho/physical Disadvantages are used to cover the points spent with the Advantages. Obviously the Evil Arbiter also likes more disadvantages...

\textbf{Each player must play his disadvantages otherwise he will not gain experience points and will be denied the use of Advantages.}

A drawback can be "undone" in the course of the character's story and there must be an adventure to justify it. As always the Arbiter has the final say on each choice of pros and cons.

\bigskip

Tips
\begin{itemize}[leftmargin=*]
\item
Get cons that are fun to play, even if they'll get you in trouble.
\item
Get disadvantages that are interesting to play with other players even if they will get into trouble
\item
Take disadvantages that have to do with the character
\item
Take some cons you won't regret
\end{itemize}

\textbf{Be careful}:

\begin{itemize}[leftmargin=*]
\item
Avoid disadvantages that are difficult to play either because they are completely detached from the system or totally useless or severely harmful to others. If you want to be an extreme pacifist, evaluate the character and the group carefully.
\item
Don't take disadvantages that you might be ashamed to act
\item
Don't take disadvantages that have nothing to do with the character (in perfect contradiction to what has already been said...)
\item
Don't take silly disadvantages (such as fear of turning right, of elevators...)
\item
If you take a severe disadvantage, recite it well, the Arbiter will be able to reward you
\end{itemize}

\subsection{Role Disadvantages and Psycho/Physical Disadvantages}\hypertarget{svantaggi}{}\label{svantaggidiruolo}

Disadvantages are divided into two categories, \textbf{Role Disadvantages} and \textbf{Psycho/physical Disadvantages}.

The \textbf{Role Disadvantages} are small flaws, tics, big and small problems that serve to give a more "human" dimension to the character. They have a deliberately ambiguous and playful description, choose them carefully and discuss with the Arbiter how you intend to interpret this disadvantage.

The player is invited to create new role disadvantages. These drawbacks do not grant a bonus or penalty, nor do they give points for taking advantages. \textit{They're fun though!}

\bigskip

The \textbf{psycho/physical disadvantages} are instead more impactful in the game, in everyday life giving concrete disadvantages. These disadvantages provide the points with which to "pay" the advantages. At the bottom you will find a list of Phobias.

\end{multicols}

\pagebreak

\subsubsection{Disadvantages of Role}\index{Disadvantages of Role}


\begin{multicols}{2}


\textbf{Alcoholism}:\index{Alcoholism} you like to drink, and a lot.. but when do you stop?

\textbf{Fashionable}\index{Fashionable}: probably yours, even with new clothes you never dress well. The combination of colors is always an eyesore.

\textbf{Friend of animals}:\index{Friend of animals} intended as fleas, ticks, lice, bedbugs.. flies. You have a zoo on you.

\textbf{Attract animals}: \index{Attract animals}you don't know why but you are always surrounded by cats, dogs, bunnies, cockatrices..

\textbf{Attracts trouble}\index{Attracts trouble}: It's not my fault the dragon took a detour to come poop here..

\textbf{Banana}: \index{Banana}The one you try to get in your hair, but you can't. Your hair doesn't get along with you.

\textbf{Low pain threshold}: \index{Low pain threshold} she scratched me, help! I'm dying!!!

\textbf{Pimples}: \index{Pimples}full, your face is pockmarked and these disgusting yellow pimples keep forming.

\textbf{Pacifier}: \index{Pacifier}you don't do it often, but when you're more nervous you take out the old wooden pacifier.. (or failing that, your thumb is always good).

\textbf{Coward}:\index{Coward} it's better to run, sorry, let's collect all the information first before attacking.

\textbf{Cogito ergo sum}: \index{Cogito ergo sum}you have a tendency to talk to yourself, but loudly even if there are people around and even if they are not friendly.

\textbf{Advisers}\index{Advisers}: they never understand but you do it for them. They never understand how generous you are with your valuable advice.

\textbf{Gullible}: \index{Gullible}Come on? really ? and how high did the donkey fly?

\textbf{Hamster}: \index{Hamster}understood as memory. You can't associate names with faces.

\textbf{Rotten teeth}: \index{Rotten teeth}the toothbrush you use probably doesn't have real boar bristles...

\textbf{Fingers in the nose}:\index{Fingers in the nose} I hope they are at least good.

\textbf{Diva}: \index{Diva}or at least you think you are. You don't miss an opportunity to show off your non-existent singing, comic, aesthetic skills... with big laughs from everyone.

\textbf{Common face}: \index{Common face}What's your name? I think I've seen you before...

\textbf{Gallant}: \index{Gallant}almost manic, in your every gesture you are formal, appropriate and cordial.

\textbf{Killer}:\index{Killer} no, you are not a killer. But you always have cold hands and feet.

\textbf{Frighten animals}: \index{Frighten animals}may be comfortable too, if it weren't for the horses that run away and the bears that attack...

\textbf{Unable to enjoy himself}: \index{Unable to enjoy himself} and so ? it's your problem, not mine.

\textbf{English}: \index{English}intended as humour. No one ever gets your jokes.

\textbf{Glutton}: \index{Glutton}CIOMP!. Never skimp, it could be your last meal!

\textbf{Meteor}:\index{Meteor} you suffer from compulsive and noisy bloating, not to mention the unpleasant smell.

\textbf{Megalomaniac}:\index{Megalomaniac} let's involve the armies of the seven kingdoms and penetrate the dungeon!

\textbf{Mint}: \index{Mint}if you only ate garlic and onion your breath would be less stinky

\textbf{Musician}: \index{Musician}with the mouth. You whistle all the time, whenever you're absent-minded or very tense... you start whistling.

\textbf{Not empathic}: \index{Not empathic}why is the kid whose teddy bear I just set on fire crying?

\textbf{Obsession}:\index{Obsession} more, more, more. Another tube of skin cream!!

\textbf{Package}: \index{Package}yours. You always have a hand over there. Maybe the pants are tight? and no, I'm not shaking your hand.

\textbf{Bad temper}: \index{Bad temper}it's okay to be gruff.. but do you always have to make it clear?

\textbf{Bronze}: \index{Bronze}no, not the cow or your mare but your armpit. You sweat profusely, whether it's hot or cold... or nervous.

\textbf{Mental Stiffness}:\index{Mental Stiffness} no, I don't understand, the map says to go right. I don't care if there isn't a right.

\textbf{Knowledgeable}\index{Knowledgeable}: the right answer is only yours. There is no doubt.. for you.

\textbf{Nosebleed}: \index{Nosebleed}it happens, and always as soon as you see a woman/man (depending on your taste) that you like.

\textbf{Scarf}: \index{Scarf}you must always have a garment of a certain type on and visible, otherwise you won't come out of the cave.

\textbf{Secret}: \index{Secret}I have a secret, so secret that I don't know if I know it myself...

\textbf{Follow Chaos}: \index{Follow Chaos}is stronger than you, you can never obey any law or authority.

\textbf{Follow the Law}: \index{Follow the Law}is stronger than you, no matter what the law is, you don't break it.

\textbf{Tattooed}: \index{Tattooed}tattoos are the way to live. You have at least 30\% of your body already tattooed and you don't miss opportunities to get new tattoos.

\textbf{Mice}:\index{Mice} you are The Topi!

\textbf{Umarel}\index{Umarel}: Whenever there is a construction job is stronger than you, you have to stop and comment on the bad ability of the workers or engineers.

\textbf{Nails}:\index{Nails} you are a compulsive nail eater, your fingertips bleed sometimes.

\textbf{Last Word}\index{Last Word}: is stronger than you, you must have the last word in every speech.

\textbf{Old at heart}\index{Old at heart}: "Ehhh in my time!". It's not a matter of age. You always have to complain about anything, your vitality is that of an octogenarian.

\end{multicols}

\pagebreak

\subsection{Psycho/Physical disadvantages}\index{Psycho/Physical disadvantages}


\begin{multicols}{2}

\textbf{Albino}\index{Albino}

You are White, almost like milk. You don't tan and you can't stand the light, your skin is delicate.

\textbf{13}: In addition to being extremely recognizable, you have the following disadvantages: Myopia and Photosensitivity and Sensitive Skin.

\textbf{Allergy}\index{Allergy}

You have some form of allergies. I hope not serious. Make sure you always have a potion of poison remover with you.

\textbf{5:} In the presence of a specific allergen, the character sneezes loudly until the allergen is removed, -1 to all checks. (e.g. allergic to beer)

\textbf{10}: The character suffers from coughing attacks, watery eyes, dizziness, -2 on all checks. DC 10 Fortitude save or suffocate. The roll is repeated every 20 rounds until you have cleared the allergen.

\textbf{15:} The character suffers from violent fits of coughing, nausea, cold sweats, palpitation. -1d6 on all checks, requires a DC 15 Fortitude save or falls unconscious. The saves are repeated every 5 rounds until the allergen is removed.

\textbf{20}: The character falls prey to a respiratory crisis, and is unable to take any action other than vomit, gasp in vomit, and try to survive. Failing a DC 25 Fortitude save causes the character to die in inhuman spasms. The saves is repeated every round until the allergen is removed.

Note: Allergens that are too rare do not count.

\textbf{Hallucinations}\index{Hallucinations}

there is something wrong in your head, every now and then a spark is ignited.

\textbf{10}: Character sees and hears things that aren't there. Every day you roll a 1d6.
If 1 or 2 comes up, nothing happens.
With 3,4 or 5 one or two hallucinatory episodes will occur with methods and times at the discretion of the Arbiter.
On a 6 the character will be the victim of horrendous and disgusting visions (or the opposite) with a duration of 1d4 hours.

\textbf{Amnesia}\index{Amnesia}

\textbf{10}: You have forgotten your past and with it the memory of friends, enemies, goals. There is no way to recover lost memories.

\textbf{Ascetic}\index{Ascetic}

10, the rule says so. You will not bring more than 10 items with you.

\textbf{20}: You cannot have more than 10 items with you, magic or normal or coins or weapons. Luckily the clothes don't count.

\textbf{Stutterer}\index{Stutterer}

You can talk, but badly.

\textbf{5:} You have an annoying tendency to stammer just when you have something important to say. In these critical situations only sketchy sounds come out of your lips.

\textbf{Bad Character}\index{Bad Character}

Good manners are never an option.

\textbf{5}: You have never mastered the art of diplomacy, and you hate being contradicted or insulted. This does not mean that you take action, but that in the face of an insult or a frank criticism you tend to silence your interlocutor with very unpleasant expressions. You have a -2 on Charisma-based checks

\textbf{Big Spender}\index{Big Spender}

\textbf{10}: you must spend half of your quest earnings on trivial pleasures (eating expensive food, drinking fine wine and spirits, luxurious clothes, no weapons or magic items)

\textbf{15}: you must spend all your quest earnings on futile pleasures (eating expensive food, drinking fine wine and spirits, luxurious clothes, no weapons or magic items)

\textbf{Charitable}\index{Charitable}

\textbf{10}: You must donate half of your quest earnings to charity

\textbf{15}: can't hold more than 10 gp in cash

\textbf{Blindness}\index{Blindness}

\textbf{10}: You are blind, impaired lateral vision, trouble understanding the distance of things.
Skills such as Awareness and Attack Rolls to hit with thrown weapons have a -4. Defence worsens by 2.

\textbf{20}: you are blind. You do not see. all enemies are Invisible.

\textbf{Kleptomania}\index{Kleptomania}

\textbf{5}: You feel the irresistible urge to steal "interesting" items from time to time. If you haven't stolen at least one item in a day, you won't be able to use Fate Points for that day.

\textbf{Code of Ethics/Vote}\index{Code of Ethics/Vote}\index{Vote}

You made a vow, a promise, an oath that conditions your actions.

5-10 : establish the rules well, in black and white, and be clear with the Arbiter

\textbf{Compulsive}\index{Compulsive}

There are certain behaviours, necessary for you, which you absolutely cannot do without (e.g.: walking avoiding spots on the ground or passing only over them, removing the weapon only in a certain way, etc.)
These behaviors must be declared and made explicit when choosing the disadvantage.

\textbf{5-10}: when in the grip of compulsive behavior you have a -2 on Mindfulness checks / you are always the last to act regardless of initiative rolled or marching order.

\textbf{Daltonism}\index{Daltonism}

You are color blind, a sunset will be something sad seen in grey

\textbf{5}: You lack awareness of colors (achromatopsia). See everything in grayscale.

\textbf{Deformity}\index{Deformity}

Not everyone is born handsome or straight. There are also those who are born crooked and ugly.

\textbf{5}: Minor malformation, affects your choice of Strength or Dexterity or Constitution. Subtract 1 point from this stat.

\textbf{10}: Two characteristics of your choice cannot exceed 2 points except magically. You have half movement.

\textbf{20}: Serious malformation. Three characteristics of your choice cannot exceed 1 point except magically. You have half movement

\textbf{Depression}\index{Depression}

Every day is a bad day and nothing will make it better

\textbf{8}: You love the Blues but unfortunately you have lost the joy of living, the enthusiasm, the hope.

Nothing seems to matter, you just drag yourself wearily from day to day. -2 to each core skill check

\textbf{Dependency}\index{Dependency}

\textbf{10}: You have an addiction, whether it be alcohol, drugs, women...If you don't consume a fair amount of it every day (the Arbiter will be able to tell you how much) you take a -2 on all Saving Throws. After 3 days of abstinence you also become Depressed

\textbf{Dyslexia}\index{Dyslexia}

jk j0j zo mdbbdfd

\textbf{10}: You are unable to read and write. You are unable to correctly associate sounds with letters and shapes with sounds

\textbf{Compulsive Dishonesty}\index{Compulsive Dishonesty}

Lie, it's stronger than you.

\textbf{5}: The character is driven by his own insecurity to always lie. Whenever the character is forced to admit his responsibilities or otherwise speak against his own interest, or in any situation in which he feels "examined", he will invent rather imaginative stories even endangering friends and relatives.

\textbf{Chronic Pain}\index{Chronic Pain}

oh how bad. Devote do you use a cure on me again today?

\textbf{10}: You do not recover Hit Points except magically

\textbf{Hemophilia}\index{Hemophilia}

you tend to bleed all the time, even at inopportune moments

\textbf{8}: BAND AID!!! (each attack you take automatically stacks Bleeding +1)

\textbf{Epilepsy}\index{Epilepsy}

always and only at the least opportune moments

\textbf{15}: Whenever you roll a 3 on a Saving Throw or attack roll, you fall to the ground for 1d6 rounds convulsing, the attack roll or save is considered to have failed. You are considered helpless.

\textbf{Fetish}\index{Fetish}

If you don't smell a woman's foot, you become depressed.

\textbf{5}: The character is irresistibly attracted to an object, body, category ... Every day that he is away from the source of his pleasure, consider himself fallen into Depression.

\textbf{Memories}\index{Memories}

Hey are you there? why are you paralyzed? And when did you learn these things?

\textbf{5}: on each proficiency check, roll a d4. With 1-2 you do the normal check, with 3 you do the check with a -2, with 4 you do the check with a +2

\textbf{Phobias}\index{Phobias}

\textbf{Miscellaneous}: The character is terrified by an object, by a category of people or living beings, by a situation. In the presence of the triggering cause, the character falls prey to a panic attack: his only desire is to flee as far as possible from the source of his terror, by any means; anyone who blocks his path is to be considered an enemy. If the character is unable to escape, he falls into a catatonic state until the trigger is eliminated. See table below for possible phobias

\textbf{Photosensitivity}\index{Photosensitivity}

The light, even if light, bothers you.

\textbf{5}: The character has a -1 on any roll where the brightness is at least daytime

\textbf{10}: The character has a -2 on any roll where the brightness is at least that of a lantern

\textbf{20}: The character has a -3 on any roll where the brightness is at least that of a torch.

The character is so sensitive that it is impossible for him to move freely in directly or darkly lit places, he will prefer to move and travel at night.

\textbf{Dormouse}\index{Ghiro}

you like to sleep and a lot. Ronf

\textbf{5}: +2 for every 2 hours after 8, otherwise you are fatigued.

\textbf{Awkwardness}\index{Awkwardness}

\textbf{10}:Your Dexterity score can't exceed 2. You have a -2 on all checks that require Dexterity (disable devices, pick pockets, climb, initiative....)

\textbf{Hygienist}\index{Hygienist}

I'm out of soap. I'VE RUN OUT OF SOAP! .. I do not touch that sword, even if it shines with holy light and flies in mid-air until it is disinfected!

\textbf{5}: you have the urge to clean yourself constantly and clean everything you touch.

\textbf{Unconsciousness}\index{Unconsciousness}

\textbf{5}: You are not afraid of anything. Literally. If you have to do something, the most direct and immediate plan is the best choice. You can't come up with plans that last more than a minute. You get a +1 to Initiative and a -1 to Attack Roll.

\textbf{Undecided}\index{Undecided}

Let's not do it, let's wait for tomorrow.. maybe it's better!

\textbf{10}: You never act first. -1d6 on initiative checks

\textbf{Recurring Nightmares}\index{Recurring Nightmares}

\textbf{10}: The character cannot sleep well. Every night he rolls a d4. With 1 the character sleeps normally, with 2 or 3 the character sleeps restlessly and wakes up tired, with 4 you wake up screaming in the middle of the night, in the morning you are fatigued.

\textbf{Open Book}\index{Open Book}

yes, I know, I can shut up, you've already understood everything anyway.

\textbf{5}: it's not that you're unable to lie, it's that you have a -1d6 on Deceive checks

\textbf{Migraine}\index{Migraine}

It's never a good day. You suffer from constant and fierce headaches.

\textbf{15}: The character suffers from violent headaches. Every day the character rolls a d4: on a 1 the character does not complain of any effect, on a 2 or 3 he suffers a -1 penalty on alls, on a 4 the penalty becomes -2.

\textbf{Cursed}\index{Cursed}

You are Cursed. A dark fate has stained your soul

\textbf{5-10}: you carry a curse. Talk to the Arbiter

\textbf{Myopia}\index{Myopia}

Hoping to find some glasses

\textbf{5}: You see very little. You have a -2 on all proficiency checks with ranged weapons and Awareness checks beyond 13 meters.

\textbf{15}: You see very little. You have a -1d6 proficiency check with ranged weapons and Wisdom checks beyond 10 meters.

\textbf{Mute}\index{Mute}

You can't talk and worst of all you can't even slander the guy who's stepping on your foot

\textbf{10}: You are unable to make sounds. You don't speak or rather no one hears you. Take a -1d6 on checks based on Charisma and oral skills

\textbf{Dyscalculia}\index{Dyscalculia}

1+1= ?

\textbf{10}: the character has an ailment that prevents him from mastering the concept of numbering. Not only is he unable to perform the simplest operations, he is also unable to understand the concepts of major/minor, or quantitative information of any kind.
Pay attention to the change they give you...

\textbf{Obesity}\index{Obesity}

You are definitely out of shape, by a lot.

\textbf{10}: Dexterity cannot be above 2. You have a -1d6 on Dexterity checks and Reflex saves. You gain a +2 on Fortitude saves

\textbf{Bad Smell/Taste}\index{Bad Smell/Taste}

Burnt nose, palate, tongue, excessive use of chilli pepper or wasabi… there can be many causes

\textbf{5}: -2 two on checks that use taste or smell. You don't feel flavors and smells if not extreme.

\textbf{Compulsive Honesty}\index{Compulsive Honesty}

\textbf{10}: You don't know how to lie, the mere idea of telling a lie makes you nervous.

\textbf{Crystal Bones}\index{Crystal Bones}

It would be called osteogenesis imperfecta but for you it's just continuous pain.

\textbf{5}: The character has brittle bones. Each damage caused by a bludgeoning weapon causes 2 more Hit Points of damage

\textbf{10}: The character has brittle bones. Each damage caused by a bludgeoning weapon causes 5 more Hit Points of damage

\textbf{Maimed}\index{Maimed}

You are one-armed, the choice is yours which hand is.

\textbf{7}: You lack your off hand

\textbf{13}: you lack the leading hand. -2 to all rolls involving the use of the hand.

\textbf{Paranoiose}\index{Paranoiose}

You are paranoid and boring.

\textbf{5}: You always behave stealthily, even if there is no real need to, thus arousing suspicion in the people around you.

Each opposed Awareness check has an additional -5 difficulty, and a critical failure indicates that the target has something vital to hide.

\textbf{Sensitive Skin}\index{Sensitive Skin}

You don't love the sun, or at least your skin doesn't love it.

\textbf{5}: Your character burns easily, prolonged exposure without adequate protection causes painful and unsightly burns and discomfort.

\textbf{10}: You are extremely sensitive to ultraviolet rays. Each fire or light damage causes 2 additional damage.

\textbf{Lazy}\index{Lazy}

you are slow and listless

\textbf{5}: -2 to initiative

\textbf{Loud}\index{Loud}

You don't do it on purpose, but there is always some noise around you. A sword flapping, a yawn, a burp, a clattering shoe…

\textbf{5}: You have a -2 on Stealth checks

\textbf{Leakblood}\index{Leakblood}

\textbf{10}: The character's immune system is definitely worthless. -2 on Fortitude saves

\textbf{Carelessness}\index{Carelessness}

Oops, I didn't notice!

\textbf{10}: You tend not to notice what's going on around you, unless you have very good reason to be alert, or you're actively looking for something take a -1d6 to Awareness

\textbf{Schizophrenia}\index{Schizophrenia}

It was not me, but the other!

\textbf{4}: You have more personality, or perhaps the other is convinced of it.

The character has at least a second personality (max 6).
Each additional Personality to manage, besides the first, grants a +1 to the cost.
So having 3 personalities brings the disadvantage to 6 points

1d6 is rolled each day. On 6, during the day the second (or third) personality comes to light.

\textbf{Unfortunate}\index{Unfortunate}

things don't just happen, you have to know how to look for them too

\textbf{5}: you ignore the first crit you make (AR or ST) in the day

\textbf{7}: You ignore the first three crits you make (AR or Save) of the day

\textbf{Manic Depressive Syndrome}\index{Manic Depressive Syndrome}\index{Depression}

Today is Friday !!! It's Friday!!!

\textbf{7}: The character alternates between states of euphoria and moments of gloomy despair. 1d4 is rolled each day. On a 1 the character is in a "normal" mood. On a 2 or 3 he is considered to be in Depression, on a 4 he is in a state of joyous elation (see Unconsciousness) and bravado.

\textbf{Awe}\index{Awe}

I apologize

\textbf{10}: The character is very insecure and tends to blindly trust others, especially if they are charismatic. Take a -2 on Intimidate and Perform checks
You take a -2 on Charm and Domination Saving Throws

\textbf{Light Sleep}\index{Light Sleep}\index{affaticato}

Every noise disturbs you, you can never sleep well

\textbf{5}: If you sleep in an area with natural/human noises (forest/city) you cannot sleep well. In the morning you are Fatigued. You can avoid the problem by using earplugs, which require you to make a -1d6 on Awareness hearing checks to wake you up.

\textbf{Deafness}\index{Deafness}

Silence has a sound of its own says those who hear us, for you it's just a heartbreaking silent scream.

\textbf{10}: You don't hear us. You cannot make Awareness checks that require the use of hearing. You can't hear people talking. But you can read lips if you know how.

\textbf{Vertigo}\index{Vertigo}

The discomforts appear when the character is aware of his height. Just for walking in an elevated position he has no penalty

\textbf{5}: At heights above 20 meters you tend to freeze. You get a -2 on all basic proficiency checks, attack rolls, and saving throws.

\textbf{7}: At heights above 10 meters you tend to freeze. You get a -3 on all basic proficiency checks, attack rolls, and saving throws.

\textbf{10}: At heights above 6 meters you tend to freeze. You get a -1d6 on all basic proficiency checks, attack rolls, and saving throws.


\textbf{Reduced Night Vision}\index{Reduced Night Vision}

Your eyes don't work well with reduced brightness.

\textbf{5}: When the brightness is equal to or less than that of a torch (or dim) the character has a -2 on attack rolls.

\textbf{Shyness}\index{Shyness}

\textbf{5}: You are shy and reserved.

You have a -2 on Charisma-based checks

\textbf{Lame}\index{Lame}

you are lame

\textbf{5}: Your movement is reduced by 2 meters (9 to 7, 6 to 4)

\textbf{7}: Your movement is halved (from 9 to 4, from 6 to 3)

\textbf{10}: you are significantly crippled. -2 to checks requiring Dexterity, your movement is halved

\bigskip

\end{multicols}

\textbf{Phobia table (5-15 points)}\index{Fobie}\index{Table of Phobia}

\begin{tabular}{ll}
\textbf{Phobia Name} & \textbf{Description}\\
\toprule
Blennophobia & Fear Of Slimy Things\\
Keraunophobia & Fear of Thunder\\
Hypochondriasis & Fear Of Illness\\
Claustrophobia & Fear of Closed Places\\
Coimetrophobia & Fear of the Cemetery\\
Hedonophobia & Fear Of Physical Pleasure\\
Eisoptrophobia & Fear of Mirrors\\
Glossophobia & Fear Of Public Speaking\\
Monophobia & Fear Of Being Alone\\
Necrophobia & Fear of Dead Bodies\\
Nyctophobia & Fear Of The Dark\\
Acrophobia & Fear of Heights\\
Agoraphobia & Fear of Outdoors\\
Rupophobia & Fear Of Dirty And Unhygienic. Feel the need to clean\\
Haphephobia & Fear Of Touch \& Being Touched\\
Asymmetrophobia & Fear of Non-Symmetrical Things\\
Gymnophobia & Fear Of Nudity\\
Hemophobic & Fear of Blood\\
Traumatophobia & Fear of Injury\\
Sciophobia & Fear of Shadows\\
\end{tabular}


\pagebreak

\section{Optional - Iconic Abilities}\index{Optional - Iconic Abilities}\hypertarget{abilitaiconiche}{}\label{abilitaiconiche}


\begin{changemargin}{0.3cm}{0.3cm}\begin{emphasis}{
Life has gotten immeasurably better since I was forced to stop taking it seriously. (Daniel Day Lewis)
}\end{emphasis}\end{changemargin}\medskip



\begin{multicols}{2}

%{\small

These ability represent the apex of a character, not intended as the capstone ability of the 20th level, but as ability related to the way of role, to the type of character that has been created and grown. These ability should only be given to characters who have been raised from level 1 to at least 12th level, it's an acknowledgment to the player.

They are optional ability because they are strong, peculiar, unique and "broken". The Arbiter should give them at the end of a long and single campaign when the characters are \textit{legends}. Each character can have only one iconic ability, a capacity that distinguishes the heroes, capable of actions at the limit and beyond the human. Players are encouraged to create new Iconic ability based on character development.\\

{\Large{\textbf{A Light against the Darkness}}}\index{A Light against the Darkness}

\textbf{Suggested Requirement: Patron Ljust, Sumkjr}

Once per day you emit for 60 minutes around you holy light that has the effects of the spell of Protection from Good and Evil against Devotees and Followers not of your Patron. You can channel light once per day and all creatures who are followers or devotees of other Patrons within a 10m radius of you must make a DC 10 Fortitude save + sum of Traits in common with the Patron + Wisdom or be stunned for 2d6 rounds. \\

{\Large{\textbf{The Blacksmith}}}\index{The Blacksmith}


\textbf{Requirement: Metalworking Skill}

Your skills in working with weapons and Armour are legendary.
Any Armour you make clutters and weighs one category lower, weapons do damage one die category higher. \\

{\Large{\textbf{The Oracle of War}}}\index{The Oracle of War}


\textbf{Requirement: Master Melee Fighter}

Every weapon in your hands is lethal. The weapon die doubles as the damage caused by the Strength is doubled. Eg a longsword does 2d8 of damage and if you have Strength +3 the total damage becomes 2d8+6\\

{\Large{\textbf{The Fearless Hero}}}\index{The Fearless Hero}


\textbf{Requirements: Brave and Steadfast}

Once per combat you can ignore (for 1d4 rounds) the conditions that affect you as a Reaction.\\


{\Large{\textbf{Mindmaster}}}\index{Mindmaster}

\textbf{Requirement: A life of adventure managed with intelligence and coolness}

You can use the score in a mental characteristic (Intelligence, Wisdom or Charisma) instead of a physical one (Strength, Dexterity, Charisma) which affects all checks.\\


{\Large{\textbf{On a pale horse}}}\index{On a pale horse}


\textbf{Requirements: Do not fear death, do not be reckless}

You are the closest thing to death your enemies will ever see.
When you kill an enemy, all opponents (who may have seen the scene) within a 10m radius must make a DC 10 Will save + Weapon Proficiency + Charisma, cost a Reaction, or be affected as by the Fear spell. The ability is usable 3 times per day.\\

{\Large{\textbf{The Magical Fury}}}\index{The Magical Fury}


\textbf{Requirement: A life dedicated to explosive magic}

You are capable of unleashing hell with magic. The difficulty (DC) of each of your spells increases by 2, when you make a Magic Test you roll an extra 3d6 and don't consider 1 dice rolled.\\

{\Large{\textbf{The Shadow}}}\index{The Shadow}

\textbf{Requirement: A life devoted to hiding and surprising enemies}

Three times a day you can switch positions with an opponent within 30 meters as long as you are both in a shaded area. DC Will save equal to the Stealth check +10.\\

{\Large{\textbf{The Mother}}}\index{The Mother}

\textbf{Requirements: Spent more time in animal form than own}

It has the innate ability to leave the tracks of any animal, compatible with your size, even if you are not transformed. You can speak with any animal as if you were still under the effect of the speak with animals spell.\\

{\Large{\textbf{The Dead}}}\index{The Dead}


\textbf{Requirement: A lifetime on the brink of death}

Three times per day when your Hit Points drop below 1, as a Reaction Action you regain 3d12 Hit Points. This ability can also be used when HP is negative or you should be directly dead. \\

{\Large{\textbf{The Hunter}}}\index{The Hunter}


\textbf{Requirements: A lifetime dedicated to hunting and stalking}

Your Survival checks have a +2d6 bonus. The first hit that lands on an opponent automatically scores 2 crits.
%}

\end{multicols}



\pagebreak

\section{Cosmology}\index{Cosmology}\hypertarget{cosmologia}{}\label{cosmologia}

\begin{changemargin}{0.3cm}{0.3cm}\begin{emphasis}{
It's easier to rule over those who don't believe in anything (The Neverending Story, Kmorf)


\medskip

Do you believe there is only one God? You are right; even the demons believe it and tremble! (James the Just 2, 19. NdA Referring to one's Patron...)}\end{emphasis}\end{changemargin}\medskip


\begin{changemargin}{0.3cm}{0.3cm}\begin{narrator}
Deities in OBSS are different from traditional RPG deities.

The divinities, the Patrons, love to get their hands dirty, participate in the affairs of the creatures who adore them, for them it is a continuous challenge to have more believers, followers and people more similar, by Traits, to them.

Patrons were created as \href{https://www.merriam-webster.com/dictionary/paroxysm}{paroxysm} of the human soul, where everything is an excess. As spirits released from Pandora's box they have the sole purpose of bringing their Traits to domination making them the most common and present among creatures, especially among the most powerful.
\end{narrator}\end{changemargin}

\begin{multicols}{2}

\bigskip

\lettrine[lines=2, lhang=0.33, loversize=0.25, findent=1.5em]{I}{n} principle was nothingness which contained everything in itself.

The Energy deriving from the most primordial impulses exploded in all its power without any control.

Love, hate, fear, pain, joy, serenity...everything was tangled in a thick and infinite skein whose nucleus was forming.

These energies, emotions and impulses have begun to create three entities: Atmos, the one who is responsible for controlling the progress of time and space, the one who assists and the scribe; Ljust the positive energy, heat, light, life and syntropy; Calicante, the negative energy, icy hatred, destruction, death and entropy.

While Atmos does not have a definable form Ljust and Calicante have manifested themselves as two tongues of flame of a single progenitor energy.

\textbf{Ljust} \index{Ljust}is the representation of what light and life always bring with them. It represents the purity of the feeling of love, the protection of life, respect for others, curiosity for the new, the desire to always improve, the strength to fight with courage and value for the common good. It is the vital drive for change, chaos that evolves but does not destroy.

\textbf{Calicante}\index{Calicante} is the representation of darkness, hatred, anger and violence. Calicante is vengeance and cold destruction, there is no interest in any form of life rather he uses them, exploits them and only in such cases suffers their presence. He sadistically loves suffering. It is entropy that annihilates and annihilates and finds pleasure in doing so.

\textbf{Atmos} \index{Atmos}is the witness, the one who marks the passage of time and transcribes every event of Yeru and among the Patrons of Genesis. An entity born of creation to prevent absolute destruction. He supervises and transcribes what the Patrons of Genesis do, the divinities who generated creation.

Together the two Patrons of Genesis gave birth to everything we know. Calicante created Tiya\index{Tiya} and Ljust created Curyan\index{Curyan}, the two kingdoms that make up our world, Yeru.

They played with forms and energies creating two realms specular but opposed and distinct. Tiya and Curyan, like Calicante and Ljust, are part of a whole but, exactly like the Patrons of Genesis, they are also profoundly different and magically divided. In fact, there is a physical barrier formed by deadly perennial sea storms and also magical ones, which delimit its borders and keep them clearly divided.

But just as their two creators who totally divided and distant cannot stay, cannot exist, so Tiya and Curyan are yes divided but also in contact with each other through the Portals. Portals that are generated autonomously, without any control and prediction, due to the magical energy that presses, pushes and feeds itself in the "non-place" on the border of the two kingdoms and which is generated by the continuous emotions of the Patrons of Genesis.

It is these magical ways that allow you to move between Tiya and Curyan and travel to the "non-place" or what is outside of Yeru.

Ljust and Calicante decided, strangely by mutual agreement, to generate a Patron who would oversee these rifts, who would be able to sense, open and block these Portals. Thus was created \textbf{Lynx}, the Guardian of Portals.

Many try to move from Tiya to Curyan to seek peace, serenity. Others try to cross the reverse border in search of adventure and power, some try by normal ways, others through the Portals, many are lost forever in the "non-place".

Lynx \index{Lynx}oversees the cosmic void, the access to the Planes, to the portals which with the alternation of chaos and order, of good and evil, of light and darkness are increasingly creating fractures on the border between the two realms. Lynx perceives them, "feels" them, knows where they are being generated or shut down, with the passage of time in fact some of these Portals have become stable and definitive, while others continue to be generated randomly and always in a totally unknown way remain active or run out . Traveling continuously in the non-place Lynx closes the largest portals but for one that closes another one opens. Lynx stripped the Magic Lists of many of the spells that affect the planes, to protect Yeru from outside creatures.

Precisely in carrying out this important role, Lynx collided with a strange creature, reptilian, gigantic, winged, powerful, strong, wise and magical. A red Dragon,\index{Tàhil} named Tàhil.
The latter moved in the "non-place" with maximum freedom, without any difficulty and approached Lynx. The Atmos diaries tell of how Lynx tried to stop him and speak to him, how he was ferociously attacked, the screams of the Patron Guardian that were heard echoing in both realms, the sound almost similar to a guttural roar that pierced the silence in the realms of Tiya and Curyan. Of the intervention of Ljust and Calicante. The first to save Lynx and the second to discover, know this fascinating new "weapon".

Lynx was saved. Ljust infused him with his healing spells and helped him regenerate. However, he left himself scarred in memory of the meeting.

Tàhil led many other Dragons on Yeru and these moved by their thirst for knowledge and power have also spread through the Portals present on both Tiya and Curyan.

Hordes of dragons of all colors have darkened the skies for decades. No nation was saved. Looting, raids and violence were perpetrated indifferently in the two kingdoms. They were highly intelligent, cunning and violent, powerful beyond belief and extremely evil. They had a sturdiness out of the ordinary. But above all, they did not fear the Patrons. They did not submit to them.

Atmos, concerned for the balance of creation, channeled the primordial and divine energies of the Patrons of Genesis going to create deities that could rival the dragons and could defend Yeru.

The first created by Atmos, with the help of Ljust, and the intervention of Calicante, was \textbf{Gradh}\index{Gradh}, Patron of Humanity (and of all sentient races), the one who would defend creation by dragons and other Patrons. Gradh embodies the dualism of the two Patrons of Genesis, the innate instinct for protection, Defence and care of Ljust and Calicante's instinct for revenge, violence and fury.

He courageously throws himself into battles, fearlessly attacks the enemy, protects the weakest, defends life but is not afraid to take the path of the most destructive revenge towards those who exploit and destroy lives for no reason. Gradh loves to "immerse" among people and live with them, like them. He doesn't feel totally at ease in the Pantheon with the other Patrons nor among ordinary people, he is Human among Patrons and Patron among Humans. Passionate and kind, he is the Patron who most cares about the fate of Yeru and its races.

The tongues of divine energies were too intense, chaotic, and pure for Atmos to govern them to shape the further Patrons alone. Using the raw power of the Patrons of Genesis he created other Patrons, each influenced differently by Calicante or Lust These Patrons turned out to be less perfect and divine than his intentions, more imperfect and "human" as they originated from emotions, uncontrollable and pure Patrons of Genesis. These new Patrons shape wills, establish kingdoms, rule in the shadows as pawns the creatures that dare to ask for their favors.

Gradh immediately perceived that the Dragons represented an element of further chaos, further suffering and war. As Patron of Yeru and its creatures he felt the Dragons as alien creatures, not original, not part of the plan of Genesis.

Mistrustful by nature Gradh decided to propose to the Patrons of Genesis to make a pact with the Dragons.

Here it is that a little more than 300 years ago, on 15 Prineva of 65 of the sixth cycle, on the unreachable island of Alantia which divides Tiya and Curyan, the flame of Ljust and Calicante were found on one side, Lynx and Gradh on the other, while Tàhil, the evil and immortal red dragon and Elysan\index{Elysan}, the wise and good silver dragon on the other. Atmos everywhere kept track of events.

Gradh tried to enforce the banishing of Dragons and Lynx the permanent closure of the Portals. Ljust tried to mediate understanding that not all Dragons were evil and that they could bring knowledge and a new evolutionary boost to Yeru.

Calicante pretended to agree with Ljust with the sole purpose of bringing more chaos and destruction through the Dragons.

Realizing that the outcome of the meeting was already decided Gradh and Lynx left the Plain of Solitude leaving the Dragons and the Patrons of Genesis to formalize the partition of Yeru. It had been a sound defeat for both of them, Gradh was even more wary, if not prejudiced, of all dragons ever since.

Tàhil became the First General of Calicante and the Dark Hatred created for him a secret and unreachable kingdom, a land for him and the Dragons. Elysan swore allegiance and trust to Ljust and together they promised to rule Curyan as best they could.

Lynx, now an external spectator, did not remain without doing anything, fearing the worst he created a Portal that led to a new Yeru, a different planet and out of the influence of Dragons and Patrons. His searches take him to an almost idyllic world full of nature, animal life and without Patrons, as he liked. He named it Ker \index{Ker} in memory of an old love. There is very little information on this world, only a few high Devotees of Lynx know it and even fewer have visited it.

The fact remains that the joining portal exists and is stable, exactly where it is is not known but the gift of him to Yeru has already done so, the Gnomes \index{Gnomes}.

In this apparent calm the Patrons perpetuate their interests, to become the strongest the most important, those who have the greatest followers. The goal is only one to have as many people as possible following their Traits.

If a Patron acts personally or indiscriminately he knows that he will trigger the reaction of Gradh or the intervention of Atmos which will prevent him from an uncontrolled and massive use of his powers directly on the world. This rarely stops them and nature itself, creatures and plants, are often influenced by the will of the Patrons.

In Tiya, but sometimes also in Curyan, aberrations are born more and more often, ever new diseases, cursed lands where nothing can grow, not to mention madness that often involves those who should instead protect the citizens. It's a hard life for the common man who continually has to face drought or floods, animal deaths and an irregular if not absurd weather, hordes of creatures that have come from nowhere who just want to exterminate everyone. At every step he has to look around because he can never know who sold his soul to a Patron to live one more day.

In Curyan one sees the development of harmony and almost perfect coexistence between nature and different races. There is pain, there is sickness and death but all as a natural cycle of life as part of it which is protected, guided and helped. A rich and generous land and for this reason increasingly victim of the machinations of the Patrons if they follow Calicante.

Everywhere the strongest enemies are the Dragons who raid to bring destruction and death and sow fear and horror.

All is not always idyllic, vast regions of Curyan are becoming breeding grounds for dark and evil races, legions of the undead led by powerful necromancers amass the borders, Dragons train their corrupt adepts, and dark black coils in the sky promise storms.

\subsection{Patrons}\index{Patrons}\hypertarget{patroni}{}\label{patroni}

\begin{changemargin}{0.3cm}{0.3cm}\begin{emphasis}{
Conan: Which of the prayers?

Subotai: I pray to the rooftops and you?

Conan: I pray to Crom, but only rarely... he doesn't listen. (Conan the Barbarian, 1982 movie)

\medskip

For as the body without the spirit is dead, so also faith without works is dead. (James the Just 2, 26. NdA Referring to the scores of the Traits connected to the Patron...)
}

\end{emphasis}\end{changemargin}\medskip


All creatures, even those who don't use magic, can feel the influence of these powers, these Patrons.

If a character, due to his way of being (playing) and behaving, has at least one Trait in common with a Patron and indeed matures and strengthens these convictions, even if he has not sworn allegiance to a Patron he could still feel the influence of the Patron and receive of gifts from him.

A Patron is very happy if someone follows his dictates, Traits, and gives those who do small powers as recognition for the loyalty reserved for him, intentionally or not. The powers listed under "Common Traits" are cumulative. Unless otherwise indicated, the powers can be used 1 time per day and cost 2 Actions.
When a spell is indicated, it is manifested without Magic Tests or Armour penalties.

Each \textbf{Patron prefers one or more energy forms}, if you are a Follower you can use that energy in your spells, if you are a Devotee instead you must, in the same way the preferred Magic Lists are indicated, i.e. lists in which the Devotee it has usage advantages.

The forms of Energy are distinguished between positive, neutral and negative sources, they also serve you to better understand your Master, pardon the Patron you serve.

Do the sum of the elements, if positive the Patron can be considered good, if with a zero value the Patron is neutral, if with a negative value the Patron is evil.

In the Patron's description you will also find his manifestation, i.e. what happens when a character acts in a particularly and significantly consonant manner with the Traits followed by the Patron. The effect is purely scenic and circumstantial but always leaves anyone who can observe it impressed, and usually guarantees an advancement point in some Trait connected to the Patron.

There is also an indication of the Patron's favorite weapon. There are no benefits to using it, the choice is purely personal and left to the devotion of the character.

Below the indication of the preferred weapon there is an indication of the Rule \index{Rule} or the behavior that the Devotee must try to respect.


A spellcaster who relies on a Patron, at least 3 Traits in common, becomes a Devotee. If he has at least 2 Traits in common and relies on a Patron then he is said to be a Follower. The \textbf{Advantage} shown is for the Devout only.

He may not even follow any Patron even though he has multiple Traits in common.

\begin{changemargin}{0.3cm}{0.3cm}\begin{narrator}
The Arbiter can still grant being a Follower or Devoted even if the Traits don't match perfectly. At the request of the player and at his discretion, he can evaluate the similarity of some Traits of the character to those of the Patron and evaluate them suitable for being a Follower or Devotee. In these situations it is necessary to understand how the player frames the character and understand not only if the Traits but also the feeling of the character is similar to the chosen Patron.
\end{narrator}\end{changemargin}

The skills you gain related to the Common Traits are independent of being a Devotee, Follower, or simply "atheist." \index{Advantages}

\bigskip

\textbf{Table Elements - Energy}\index{Table Elements - Energy}

\medskip

\begin{tabular}{lll}
\textbf{Positive} (+1) & \textbf{Neutral} (0) & \textbf{Negative} (-1)\\
\toprule
en. Positive & Fire & En. Negative\\
Light & Cold & Void\\
& Sound & \\
& Electricity & \\
\end{tabular}

\begin{changemargin}{0.3cm}{0.3cm}\begin{tcolorbox}[title = Devotees and Followers]

Being a Devotee or a Follower is your choice, no one forces it on you. You must feel it as a role-playing opportunity, as a character enrichment and not a constraint. Being Devotees or Followers does not mean being prone to the will of the Patron, on the contrary, it means being even more convinced of one's own Traits, of one's personality. A Patron does not ask for prayers, but asks to be oneself.

\end{tcolorbox}\end{changemargin}

\subsubsection{Miracles, Interventions and Wonders}\index{Miracles, Interventions and Wonders}

In a world where deities are so capricious, fickle but thirsty for devotees, it is their game to be generous with those who can then spread their Traits.

A favor asked of a Patron always has a price that is neither obvious nor obvious. The Storyteller must carefully evaluate the character's plea and judge whether the request is relevant to the Patron's Traits, if so roll 1d100 and score less than half the highest Trait score in common with the Patron. Or decide independently according to the course of the adventure.


\subsubsection{Ljust}\label{ljust}

\index{Ljust}

The Lady of Light radiates warmth and love, and is the generatrix of impulses of love, protection, kindness, joy, and forgiveness. She embodies the protective aspect of a mother, the strength and audacity of a fighter, the passion of a young lover, and the joy and imagination of a child. Ljust embodies the beauty of life, and every creature that contemplates her sees her maximum harmony and falls under her charm.

Ljust can only be chosen by a character with four traits in common with her. Essentially, one is born to be a Devotee of Ljust. Over time, Ljust decided to select, choose, and reward women who displayed an innate and deep love for life, curiosity for the new, unshakable strength, dedication, trust, respect, and care for others. She gave them the powers and possibility to study and grow as Pupils of the Light. These students must follow the 8-step rule..\\

\noindent- \textbf{Symbol}: A star surrounded by sunbeams\\
- \textbf{Feature}(Devoted): Wisdom or Charisma\\
- \textbf{Traits}: Empathy, Courage, Patience, Respect, Sincerity, Perseverance, Honesty. The Devotee of Ljust has 4 Traits in common with the Patron.\\
- \textbf{Manifestation}: Golden light floods the caster.\\
- \textbf{Sum of common Traits to 5 points}: you can cast the Light spell as a Reaction, 3 times per day\\
- \textbf{\textbf{Sum of common Traits to 10 points}}: you gain a +2 on Fortitude saves\\
- \textbf{Sum of common Traits at 15 points}: an Armour of light protects you, you gain +2 on all Saving Throws and Defence, the effect is permanent.\\
- \textbf{Sum of common Traits to 20 points}: You can cast the Solar Flare spell. Once a day.\\
- \textbf{Elements} (Follower/Devotee): Positive Energy, Light\\
- \textbf{Advantage} (Devoted): Effective healing\\
- \textbf{Privileged Spell Lists}(Follower/Devoted): Heal, Abjuration\\
- \textbf{Favorite Weapon}: Bastard Sword\\
- \textbf{Rule}: Accept invitation to a dance


\medskip

\textbf{The 8 Steps of the Students}\index{The 8 Steps of the Students}\index{Students}

The Pupils of the Light are a secret group of Devotees who, out of total affinity with Ljust, have embarked on the hard path of good and love. It is among the oldest groups founded in Yeru. The Lovers, 99 as a maximum number, but unfortunately often less numerous, are Devotees of Ljust and must follow the 8 Steps of the Light\\

1. Love and protect those around you with all of yourself, with total and sincere dedication. \\
2. Do not let your inaction generate suffering.\\
3. Yes, a point of comparison. Let your Light uplift the people around you and they can see in You are hope, serenity, calm, protection and security.\\
4. Use intelligence, cunning and wit. She is forward-looking and resolute in action.\\
5. Your work is for the common good. Let your Light always be high and intense.\\
6. Do not seek any other Light than yours and that of your sisters.\\
7. Be bright but don't blind those around you.\\
8. Be the difference between despair and hope.\\

\medskip

The Lovers built a harmonious dance by transforming the steps of their Rule into dance.

There are also Lovers of uncertain gender, rare but historically proven.

\subsubsection{Calicante}\index{Calicante}\label{calicante}
\begin{changemargin}{0.3cm}{0.3cm}\begin{emphasis}{
Superstition is the religion of weak spirits. (Edmund Burke)
}\end{emphasis}\end{changemargin}


It is dark, cold, and filled with anger. Calicante embodies hate, violence, destruction, revenge, and perpetual dissatisfaction. It combines the capricious and disgruntled personality of a child, the violent and sadistic boredom of a young man, the destructive force of a hurricane, and the rage of a fighter who has nothing left to lose. Calicante's mere presence makes you uncomfortable and feel in danger. It fascinates, but with the weapons of fear and inconstancy.

Calicante can only be chosen by characters who share four traits with him. His devotees are the best assassins, and it is their closest profession. They show the greatest contempt for danger and the lives of others. Calicante favors those who are feared, hated, violent, and cruel but are deadly efficient and decisive in any combat situation.\\

\noindent- \textbf{Symbol}: A black whirlwind\\
- \textbf{Feature}: Strength or Dexterity\\
- \textbf{Traits}: Malice, Resentment, Lust, Greed, Cynicism, Selfishness, Pride. The Devotee of Calicante has 4 Traits in common with the Patron.\\
- \textbf{Manifestation}: sword dripping with black blood\\
- \textbf{Sum of common Traits to 5 points} points: You can cast the Darkness spell. Once a day\\
- \textbf{Sum of common Traits to 10 points}: Your weapon is cloaked in shadow. You gain +2 on attack rolls and +1d4 void damage for 2d6 rounds, once per day.\\
- \textbf{Sum of common Traits to 15 points}: Create 4 void bolts, each bolt deals 2d6 damage, automatically hits within 18 m. Once a day.\\
- \textbf{Sum of common traits to 20 points}: You create a zone of protective energy around you within a 3 meter radius, you halve all damage you take, it is not possible to recover Hit Points in the area. Duration 10 consecutive minutes, once a day.\\
- \textbf{Elements}: Negative Energy, Void\\
- \textbf{Advantage}: Mind Shield\\
- \textbf{Privileged Spell Lists}: Fire, Necromancy\\
- \textbf{Favorite Weapon}: Machete\\
- \textbf{Rule}: Never leave a direct offense unpunished


\subsubsection{Atmos}\index{Atmos}\label{atmos}

\begin{changemargin}{0.3cm}{0.3cm}\begin{narrator}
So what is time? If nobody asks me, I know; if I want to explain it to anyone who asks me, I don't know anymore. (Augustine of Hippo)
\end{narrator}\end{changemargin}


The keeper of Time and of the Clock Tower, who started time and the creation of new Patrons, will also stop their challenge. The surviving Patrons will be judged, their works evaluated, and Ljust or Calicante will benefit from it. With a challenge from a single copper coin, new Patrons and new ideals will be created, and we, small creatures, will witness the birth of new civilizations and flourishing kingdoms. The story is little known, only the few Devotees of Atmos, scribes and scholars of the Library of Time, know the secret and the passage of time and the race. The others, ignorant, will live their time with a master surely guided by a Patron.

Atmos, the Patron of Time, is the keeper of history and time, and he is the one who keeps track of the thousand and one worlds that have been created. He has the task of starting and stopping time. Atmos has the unique power reserved only for him to banish a Patron from creation should they become too strong and threaten Calicante and Ljust. Atmos has used this power before. Atmos has never taken sides, both because of his totally neutral nature and his role.

All Patrons fear Atmos for his power, which is most terrible to them, namely their alienation, oblivion, forgetfulness, and being turned away from time and challenge.

To become a Devotee of Atmos during the ritual, the future Devotee must possess at least four traits in common with him, love history and knowledge.

Dressed in a soft brown habit and leather shoes, he moves among the infinite shelves of the Library of Knowledge, always with a strange time gauge hanging from his waist. \\

\noindent- \textbf{Symbol}: A blank book with a pocket watch placed upon it\\
- \textbf{Feature}: Intelligence or Wisdom\\
- \textbf{Traits}: Indifference, Justice, Perseverance, Responsibility, Resentment, Intransigence. The Devotee of Calicante has 4 Traits in common with the Patron.\\
- \textbf{Manifestation}: the spell develops as in slow motion, it's just an illusory effect\\
- \textbf{Sum of common Traits to 5 points}: You always know the exact date and time.\\
- \textbf{Sum of common traits to 10 points}: You have an innate intuition for knowledge. You have +1d6 on Knowledge checks\\
- \textbf{Sum of common Traits to 15 points}: You can cast the spell Orb of Invulnerability, 1 time per day.\\
- \textbf{Sum of common Traits at 20 points}: Whenever you have to make an Arcana you can take the 18 as if you took 10\\
- \textbf{Elements}: Sound, Cold\\
- \textbf{Advantage}: Sense of time\\
- \textbf{Privileged Spell Lists}: Divination, Abjuration\\
- \textbf{Favorite Weapon}: Light Mace\\
- \textbf{Rule}: Do not accept or give compensation if it is not deserved


\subsubsection{Lynx}\index{Lynx}\label{lynx}

\begin{changemargin}{0.3cm}{0.3cm}\begin{narrator}
	People don't make trips, trips make people. (John Steinbeck)
\end{narrator}\end{changemargin}


Lynx is the Patron of Portals and can only be chosen by characters who share at least 3 Traits with him. He is the first Patron generated by Ljust and Calicante, created to protect Yeru from external attacks.

Serious, with icy eyes of a very light blue, he is the Guardian of Portals and of what lies Beyond. A lethal guardian for those who try to pass without permission, a careful guide for those who ask for his help and permission. He uses his scars as a shield to keep everyone at bay. He is the solitary controller of the world.

His Devotees are the ultimate travelers, those who guard and protect Yeru from what is alien, from what could disturb creation.\\

\noindent- \textbf{Symbol}: A portal to darkness\\
- \textbf{Feature}: Dexterity or Intelligence\\
- \textbf{Traits}: Responsibility, Courage, Honesty, Cynicism, Intransigence, Resentment.\\
- \textbf{Manifestation}: as if the panorama had no more horizon\\
- \textbf{Sum of common Traits to 5 points} points: Once per day you can perform one more move Action\\
- \textbf{Sum of common Traits at 10 points}: You acquire one more move Action per round\\
- \textbf{Sum of common Traits to 15 points}: You can cast the Banishment spell, 1 time per day, DC 30.\\
- \textbf{Sum of common Traits at 20 points}: You can teleport yourself only 500km per day (even more teleports or teleport as long as the total sum does not exceed 500km)\\
- \textbf{Elements}: Fire, Electricity\\
- \textbf{Advantage}: Slow and Still\\
- \textbf{Privileged Spell Lists}: Summon, Water\\
- \textbf{Favorite Weapon}: Short Sword\\
- \textbf{Rule}: Don't leave an environment unexplored


\subsubsection{Gradh}\index{Gradh}\label{gradh}

\begin{changemargin}{0.3cm}{0.3cm}\begin{emphasis}{
The man who has ceased to fear has ceased to worry. (Francis Herbert Bradley)
}\end{emphasis}\end{changemargin}


The first Patron created by Atmos under the guidance of Ljust and the influence of Calicante.

Gradh embodies Ljust's innate instinct for protection, defence and care. Gradh is the most similar and deeply linked to Ljust that has been generated. He is balance, rationality and empathy.

Where there is defence, care and protection there is Gradh.

Gradh does not like to openly challenge Cattalm because he knows that he would play exactly his game, so he cunningly tries to lure him to his playground, where no life will be in danger and there he shows off his strategic and combat superiority.

But Calicante could not allow the creation of a Patron totally devoted to Ljust and so he instilled in Gradh the coldness of vengeance and the fury of anger. So then Gradh in the act of defending humanity, often must first of all protect it from himself.

Passionate and cold, he is perhaps the most human Patron of the current Pantheon. His warm and charismatic gaze that when he loves and protects is a reassuring chocolate color, can become cold and sharp with the shades of the cold frozen earth when he is prey to the fury of battle or revenge. Gradh loves to study the world around him and go unnoticed. He often hides among people and "lives" his human life of him. But he doesn't really let anyone get close to him.
Gradh draws to him as easily as he pushes away.

The Devotee of Gradh is fierce and proud, untamed and protective, and grieved, for no matter how hard he tries to punish evil it always continues to prosper.\\

\noindent- \textbf{Symbol}: A shield engraved with two intertwined spirals.\\
- \textbf{Feature}: Strength\\
- \textbf{Traits}: Courage, Kindness, Responsibility, Respect, Sincerity, Pride, Envy\\
- \textbf{Manifestation}: two coils one black as a shadow and one shiny as a spark surround your weapon intertwining\\
- \textbf{Sum of common Traits to 5 points} points: You can cast the Cure Serious Wounds spell, but it does you 1d6 damage. 1 time per day\\
- \textbf{Sum of common traits to 10 points}: For 10 consecutive minutes you have a +1d6 Saving Throw bonus on Reflexes and Fortitude. Once a day.\\
- \textbf{Sum of common Traits to 15 points}: You emanate an aura that grants all your companions within a 3 meter radius a +2 ST. Once a day, for 30 consecutive minutes\\
- \textbf{Sum of common Traits to 20 points}: Explode your wrath in a Fireball of 60 damage. The damage is from negative energy. DC 25 Reflexes to halve. 2 times a day\\
- \textbf{Elements}: Positive Energy - Negative Energy\\
- \textbf{Advantage}: Protected Senses\\
- \textbf{Privileged Spell Lists}: Abjuration, Invocation\\
- \textbf{Favored Weapon}: Heavy Mace\\
- \textbf{Rule}: Do not allow a fiend to walk on Yeru


\subsubsection{Atherim}\index{Atherim}\label{atherim}

\begin{changemargin}{0.3cm}{0.3cm}\begin{emphasis}{
The one to whom you confide your secret becomes the master of your freedom. (François de La Rochefoucauld)\\
			
There is nothing hidden that will not be revealed, nor secret that will not be known. (Luke, 12, 1-7)
}\end{emphasis}\end{changemargin}


The Guardian Patron. Many see in Atherim's generous bosom a sign of voluptuousness and passion. They are enchanted by her ample beauty and do not see the crystal eyes that instill fear in those who dare to even think of approaching her.

Atherim is the guardian of dreams and hopes, the one to whom desires can be entrusted, like a mother. She is the Patron of Children, Secrets, and Midwives.

With her cheerful smile and kind soul, she will always be ready to help you achieve your dreams. And like a mother, Atherim protects and guards secrets and passions. Atherim is mute. She is the one who forever guards the secrets of Yeru within her soul.

The Devotee of Atherim takes to heart those who have made a promise, punishes those who break them and those who reveal secrets. Many Devotees of Atherim are diplomats, notaries, and midwives.\\

\noindent- \textbf{Symbol}: A gloved woman's hand holding a flask full of fluxes\\
- \textbf{Feature}: Wisdom\\
- \textbf{Traits}: Honesty, Empathy, Gratitude, Responsibility, Sincerity, Intransigence\\
- \textbf{Manifestation}: A serene and soothing silence descends around the caster\\
- \textbf{Sum of common Traits to 5 points}: You can add 1d6 to a Saving Throw after you roll it but before knowing whether it was successful or not. Once per day, as a Reaction.\\
- \textbf{Sum of common Traits to 10 points}: Gain 30 temporary Hit Points. Duration 1 hour, once per day, as an immediate action.\\
- \textbf{Sum of common Traits to 15 points}: You can cast the Zone of Truth spell 3 times per day\\
- \textbf{Sum of common Traits to 20 points}: Every potion you drink has double the duration or effect if immediate.\\
- \textbf{Elements}: Positive Energy, Electricity\\
- \textbf{Advantage}: Metabolism control\\
- \textbf{Privileged Spell Lists}: Enchantment\\
- \textbf{Favorite Weapon}: Dagger\\
- \textbf{Rule}: Do not reveal a confided secret


\subsubsection{Belevon}\index{Belevon}\label{belevon}

\begin{changemargin}{0.3cm}{0.3cm}\begin{emphasis}{
			Where there is a man, there is also a lie. (Robert Louis Stevenson)\\
			
			No one has such a good memory that they can be a perfect liar. (Abraham Lincoln)
}\end{emphasis}\end{changemargin}

He is the Patron who best embodies lies and pretense for the sake of his own gain. He only loves himself. He is a narcissist who only surrounds himself with people who pander and flatter him. He abhors loneliness but at the same time he hates being touched by anyone.

He is always looking for new things, for wonderful objects that he exchanges and reciprocates with other objects. He likes to argue and bargain until he always gets what he wants at the cost of other people's lives.

Belevon is a hideous and deformed Patron worshiped by abject creatures of the deepest caves, envious of what others possess.

The Belevon Devotee is well described by a lizardman surrounded by trinkets and human remains.\\


\noindent- \textbf{Symbol}: A golden cage\\
- \textbf{Feature}: Intelligence\\
- \textbf{Traits}: Envy, Lust, Selfishness, Malice, Empathy, Perseverance, Generosity\\
- \textbf{Manifestation}: As if the golden bars of a cage weave around the caster\\
- \textbf{Sum of common Traits to 5 points} points: You can cast the Prestidigitation spell, 3 times per day.\\
- \textbf{Sum of common traits to 10 points}: You gain the ability to cast the Greater Image spell once per day.\\
- \textbf{Sum of common Traits to 15 points}: You can cast the Killing Hallucination spell. 1 time per day\\
- \textbf{Sum of common Traits at 20 points}: touching an object you get to know in brief the history of who created it. Once a day. It costs 3 Actions.\\
- \textbf{Elements}: Fire, Sound\\
- \textbf{Advantage}: Lucky\\
- \textbf{Privileged Spell Lists}: Illusion\\
- \textbf{Favorite Weapon}: Light pike\\
- \textbf{Rule}: Haggle on the price whether buying or selling


\subsubsection{Cattalm}\index{Cattalm}\label{cattalm}

\begin{changemargin}{0.3cm}{0.3cm}\begin{emphasis}{
It's not being angry that matters, it's being angry about the right things. I said to her: look at it from the Darwinian perspective. Anger serves to make you efficient. This is its function for survival. That's why it was given to you. If it makes you inefficient, drop it like a hot potato. (Philip Roth)
}\end{emphasis}\end{changemargin}


Generated directly by Calicante in response to the creation of Gradh by Ljust, Cattalm is pure destruction, chaos, and entropy. Cattalm's sole purpose is to destroy, bring chaos and disease, earthquakes and floods.

Cattalm is among the few Patrons who dare to openly challenge Gradh, and he does so with joy because he knows that their battle will only bring further destruction. Cattalm accepts and invites any creature capable of hatred, capable of destruction and harm to be his Devotee. Many of his Devotees are monstrous creatures or aberrations.

Cattalm, on the other hand, is one of the most beautiful Patrons, with a shiny white skin, soft feathered wings, and a light silver armor. Although his delicate features make him a beautiful being, he aspires to destruction.

Cattalm loves the chaos that manifests itself in the most violent ways with earthquakes, floods, tsunamis, diseases, and even raining fire. He rarely acts directly but lets chaos and destruction work for him.

Ljust could not help but intervene in the creation of a Patron so explicitly evil and, secretly from Calicante, instilled in Cattalm love and protection for children. Cattalm destroys, poisons, weakens but not children, not even indirectly. Rather, he himself acts to nullify the harm caused by his nature.

Whenever a calamity happens, it is often said that "Cattalm has stomped his foot".\\

\noindent- \textbf{Symbol}: A giant wave rolling over the coast\\
- \textbf{Feature}: Strength\\
- \textbf{Traits}: Cynicism, Pride, Selfishness, Intransigence, Lust, Sincerity, Perseverance\\
- \textbf{Manifestation}: Thunderclap\\
- \textbf{Sum of common Traits to 5 points} points: Through your weapons you weaken the targeted opponent. After a critical roll, you can increase fatigue by one level. Once per day as a Reaction.\\
- \textbf{Sum of common Traits to 10 points}: Your touch rots food (up to 50kg/Encumbrance 10) and water (a cube with a 10m edge). Once a day\\
- \textbf{Sum of common Traits to 15 points}: Your gaze blinds with anger. You cast the Confusion spell, but the only possible result is that the targets attack random subjects. DC 25. 1/day\\
- \textbf{Sum of common Traits to 20 points}: You cast the Cone of Cold spell 60 damage, but the damage is from Void. DC 25. Once a day\\
- \textbf{Elements}: Negative Energy - Void\\
- \textbf{Advantage}: Hard to kill\\
- \textbf{Privileged Spell Lists}: Fire\\
- \textbf{Favored Weapon}: Great Double Axe\\
- \textbf{Rule}: Don't do good deeds without gain


\subsubsection{Efrem}\index{Efrem}\label{efrem}\hypertarget{efrem}{}

\begin{changemargin}{0.3cm}{0.3cm}\begin{emphasis}{
Not deviating from nature and forming ourselves on its laws and examples is wisdom. (Lucius Anneus Seneca)
}\end{emphasis}\end{changemargin}

He is the Patron Saint of those who make nature their home. He embodies within himself the purest aspects of nature itself, aggressive as only the most lethal felines can be; but also wild like the most hidden clearings and rigorous like only nature can be.

Efrem aims to defend Nature from the contamination of man, from this infesting species that destroys everything it encounters.

The Devotees of Ephrem. also called druids\index{Druid}, they are more closely linked to the natural element. They manipulate mainly elemental magic and also defend or attack using animals and natural creatures. It is said that the most powerful force even the Dragons to obey.

Ephrem Devotees have the supreme goal of protecting animals and plants, places and everything that is natural and not artificial. Usually solitary and grumpy, he cannot understand the reason for the hatred that, from his point of view, the man unloads on Yeru.

An Ephrem Devotee respects life as well as death, in the natural process that is evolution and the life cycle. Sometimes he decides to settle in a certain environment and elects it as his territory and protects it as if it were his home. Other times he decides to be a wanderer and intervene throughout Yeru to protect his beloved plants and animals.

In the most desolate lands, in the most natural regions, the Devotees of Ephrem build utopias between humanoids and animals, where the balance is maintained with the blood of anyone who rebels against their will.\\


\noindent- \textbf{Symbol}: A stirrup with a vine coiled around it\\
- \textbf{Feature}: Constitution\\
- \textbf{Traits}: Respect, Indifference, Laziness, Indolence, Honesty, Perseverance\\
- \textbf{Manifestation}: Coils of leaves wrap around the sword\\
- \textbf{Sum of common Traits to 5 points} points: Your touch makes non-magical animals tame. Will save 20 to resist. 3 times a day. Cost 2 Actions.\\
- \textbf{Sum of common Traits to 10 points}: You gain a +1d6 on all Survival checks made in a natural environment.\\
- \textbf{Sum of common Traits to 15 points}: You can cast the Good Berry spell 1 time per day. Each berry heals 1d6 Hit Points and removes nonmagical disease or poison.\\
- \textbf{Sum of common Traits to 20 points}: Your touch is that of the master. You can tame even magical creatures, but not Aberrations or Dragons, that you touch. Will save DC 30. Once per day. Cost 2 Actions\\
- \textbf{Elements}: Electricity, Sound\\
- \textbf{Advantage}: Empathy with plants\\
- \textbf{Privileged Spell Lists}: Animals and Plants and an Elemental Spell List.\\
- \textbf{Favorite Weapon}: Staff\\
- \textbf{Rule}: Nature is always your first choice


\subsubsection{Erondil}\index{Erondil}\label{erondil}\hypertarget{erondil}{} 

\begin{changemargin}{0.3cm}{0.3cm}\begin{emphasis}{
	Good reasoning is stronger than two strong hands. (Sophocles)
}\end{emphasis}\end{changemargin}


Patron of Earth and Air, Erondil is the Lord of the most concrete and rational elements. He who, endowed with infinite power and rationality, gives his Devotees the power of manipulating the earth, the gift of creating gigantic constructions of millenary strength from simple mud. He concludes his works with attention and precision.
Even with difficulty because if the final result doesn't satisfy him he unleashes his lightning to destroy it instantly. Perfectionist and insatiable, something is rarely exactly as he imagined it.

Orderly and exuberant he is the lord of storms, thunder and lightning, earthquakes and destruction. He loves to surround himself with the roar of thunder, the roar of the crumbling earth. He can be destructive towards those who do not respect Yeru.
He has arms and chest covered in almost silvery tattoos that tell the legends of Earth and Air.

Erondil Devotees are the engineers of the impossible.\\

\noindent- \textbf{Symbol}: a sand castle with a lightning bolt above it\\
- \textbf{Feature}: Wisdom\\
- \textbf{Traits}:Pride, Resentment, Selfishness, Empathy, Gratitude, Respect\\
- \textbf{Manifestation}: Storm sound and landslide rumble\\
- \textbf{Sum of common Traits 5 points}: You no longer fear falls. You can cast the Feather Fall spell 3 times per day, only on you.\\
- \textbf{Sum of common Traits to 10 points}: Your touch shapes the stone. You can cast the Pass door spell 1 time per day.\\
- \textbf{Sum of common Traits to 15 points}: You can cast the Lightning spell from your hands. Reflex save DC 30 to halve. Cost 2 Actions.\\
- \textbf{Sum of common Traits to 20 points}: You are able to create a very deep pit (1km) under your opponent (size up to large). Reflex save 30 or fall. Once a day. After 1 minute the pit closes with whoever is inside. Cost 2 Actions.\\
- \textbf{Elements}: Fire, Electricity\\
- \textbf{Advantage}: Universal digestion\\
- \textbf{Privileged Spell Lists}: Air, Earth\\
- \textbf{Favored Weapon}: Warhammer\\
- \textbf{Rule}: You must not allow the destruction of architectural monuments


\subsubsection{Gaya}\index{Gaya}\label{gaya}\hypertarget{gaya}{} 

\begin{changemargin}{0.3cm}{0.3cm}\begin{emphasis}{
SPLENDOR of ended day floating and filling me,

Hour prophetic, hour resuming the past,

Inflating my throat, you divine average,

You earth and life till the last ray gleams I sing. (Song at Sunset, Walt Whitman)
}\end{emphasis}\end{changemargin}

Patron of Water and Fire, in the depths of the earth, where water and lava meet, Gaya enjoys painting. He loves to surround himself with flows of fire and water as if to create a dance between them. He loves the sounds of nature, the crashing of waves on the rocks, the falling of raindrops on the cobblestones, the hum of a crackling fire.

He paints by mixing hot and cold. The crystalline and impetuous water with the intriguing and burning fire. Jealous of beauty and the arts, she keeps all her works safe in an almost maniacal and protected order. As a true artist, she uses the elements to make the wonders of nature shine. Gaya is the painter of sunsets and storms.

Gaya Devotees are fickle and over-the-top artists. They are those who recreate the magic of dawn or sunset or the stormy sea in their works, they are those who put poetry and madness into normality.

But Gaya also has a much more devious and violent side, a streak of evil madness that loves to bring destruction with flames and water. Deep in the caves, creatures akin to water or fire worship Gaya and kill anyone who disagrees with them.\\

\noindent- \textbf{Symbol}: a brush on the sky\\
- \textbf{Feature}: Intelligence\\
- \textbf{Traits}: Generosity, Kindness, Sincerity, Cynicism, Malice, Envy\\
- \textbf{Manifestation}: coils of fire and water envelop the caster\\
- \textbf{Sum of common Traits to 5 points} points: You can create up to 5 liters of water or 1 liter of good quality liquor. Once a day. Cost 2 Actions.\\
- \textbf{Sum of common Traits to 10 points}: Your metabolism can stand the cold. You resist magical cold damage and are immune to natural damage.\\
- \textbf{Sum of common Traits to 15 points}: You can breathe underwater like you breathe air. Resist non-magical fire damage\\
- \textbf{Sum of common Traits to 20 points}: You generate a rain of fire. You cast the spell Flame Strike, DC 25 once per day. Resist magical fire damage.\\
- \textbf{Elements}: Electricity, Fire\\
- \textbf{Advantage}: Rainbow\\
- \textbf{Privileged Spell Lists}: Water, Fire\\
- \textbf{Favorite Weapon}: Trident\\
- \textbf{Rule}: Don't stop yourself from listening to good music


\bigskip

\textbf{Gaia} and \textbf{Erondil} are like the two sides of the same coin and oversee the elements, Gaia water and fire and Erondil Air and Earth; they act as a direct expression of the major Patrons, they are small manifestations of their immense power.

\subsubsection{Krondal}\index{Krondal}\label{krondal}

\begin{changemargin}{0.3cm}{0.3cm}\begin{emphasis}{
Freedom, Sancho, is one of the most precious gifts that the heavens have ever given to men; neither the treasures which the earth encloses nor which the sea covers are to be compared with it; for freedom, as for honour, one can and must risk one's life. (Miguel de Cervantes)
}\end{emphasis}\end{changemargin}

Krondal the madman, Krondal the killer, Krondal the savior.

These and many others are the names of Krondal, the Patron you can never truly fully understand.

Krondal embraces the anarchic and free spirit in the most absolute way. According to Krondal, everyone should only do what they want.

His motto is \emph {Nobody knows Nobody} because you can't know the future and what awaits you.

Krondal has a deep respect for freedom and cannot criticize choices, whether extreme or not, yet by divine dictate he pursues uncompromisingly to bring justice.

A Devotee of Krondal is typically a bodyguard, a protector, the sheriff who is not interested in the reasons for the choice but who knows how to judge the actions performed.\\

\noindent- \textbf{Symbol}: A sword held vertically in front of you\\
- \textbf{Feature}: Charisma\\
- \textbf{Traits}:Intransigence, Responsibility, Sincerity, Pride, Perseverance, Justice\\
- \textbf{Manifestation}: the Devotee's cloak or robe becomes soiled with dirt and blood\\
- \textbf{Sum of common Traits to 5 points} points: Curse your opponent. You cast the bestow curse spell once per day. DC 20 to resist.\\
- \textbf{Sum of common Traits to 10 points}: You cannot be tied up or handcuffed. Twice per day, you can only cast Freedom of Movement on yourself.\\
- \textbf{Sum of common Traits to 15 points}: Your presence blinds opponents. Designate up to 6 creatures within 10 meters, they must make a DC 30 Fortitude save or be blind to you only for 1d4 rounds.\\
- \textbf{Sum of common traits to 20 points}: Your weapon is more effective against enemies. Each creature struck must make a DC 20 Will save or be paralyzed for 3 rounds. Once the creature succeeds at its Saving Throw, it cannot be affected again for the next 24 hours. Once per day, activating the skill costs 1 Action and lasts 1 minute.\\
- \textbf{Elements}: Positive Energy, Fire\\
- \textbf{Advantage}: Magnetic\\
- \textbf{Privileged Spell Lists}: Abjuration\\
- \textbf{Favorite Weapon}: Longsword\\
- \textbf{Rule}: Do not allow abuse


\subsubsection{Ledyal}\index{Ledyal}\label{ledyal}\label{laydel}\hypertarget{ledyal}{} \hypertarget{laydel}{} 

\begin{changemargin}{0.3cm}{0.3cm}\begin{emphasis}{
A man's soul is immortal and incorruptible. (Plato)\\
			
It's not body that defines me (free creature)
}\end{emphasis}\end{changemargin}


He is the Patron without a precise face, without a voice but a song. Mutable in body and without a clear definition of his being. He/She manifests himself with a long fiery red cape made of a thousand butterflies. Her touch is life and peace, she protects those who need his favors whether he asks for them or not. He desires a world without suffering, with only happiness and harmony. Suspicious and deeply introverted, he does not believe those who agree with him. Her heart is full of life and goodness but he doesn't have a body to love with.

Ledyal also has a twin sister/brother, or maybe another personality. Or maybe they are the same Patron, no one has ever seen them together. The "twin" \textbf{Laydel}\index{Laydel} does not tolerate suffering, despises those who cause pain, fearlessly kills any creature who has sinned against an innocent, anyone who has caused suffering.\\

\noindent- \textbf{Symbol}: A butterfly dripping blood while flying\\
- \textbf{Feature}: Wisdom (Ledyal) - Strength (Laydel)\\
- \textbf{Ledyal Traits}:Gratitude, Empathy, Kindness, Honesty, Patience, Humility \\
- \textbf{Laydel Traits}: Resentment, Perseverance, Empathy, Intransigence, Envy, Cynicism\\
- \textbf{Manifestation}: As if a cloak of butterflies enveloped the Devotee\\
- \textbf{Sum of common Traits to 5 points} points: Your touch is life/attack. 3 times per day you can touch a living creature and heal/cause it 1d6 Hit Points. Cost 2 Actions (also includes Touch Action)\\
- \textbf{Sum of common Traits to 10 points}: Your touch is peace. You can cast the Sabctuary spell two times per day.\\
- \textbf{Sum of common Traits to 15 points}: Your aura protects your companions. Within a 6m radius your companions have +4 to Defence and +2 to Saving Throws. Duration 10 consecutive minutes, once a day. Cost 2 Actions.\\
- \textbf{Sum of common Traits to 20 points}: Radiate a healing sphere around you. Each creature within a 6 meters radius is healed for 60 Hit Points. Once a day. In Laydel's case the effect is the opposite. Cost 2 Actions\\
- \textbf{Elements}: Positive Energy, Electricity\\
- \textbf{Advantage}: Healer (Ledyal) or Fearless (Laydel)\\
- \textbf{Privileged Spell Lists}: Heal or Invocation\\
- \textbf{Favored Weapon}: Truncheon/Spiked Chain\\
- \textbf{Rule}: Don't allow violence against a creature's gender


\subsubsection{Nethergal}\index{Nethergal}\label{nethergal}

\begin{changemargin}{0.3cm}{0.3cm}\begin{emphasis}{
Dreams are answers to questions that we are not yet able to formulate. (X-Files)\newline
			
An uninterpreted dream is like an unread letter. (Talmud)
}\end{emphasis}\end{changemargin}


The Patron Messenger. On the feather of a goose flies the Nethergal letter. Quick, impetuous, direct, Nethergal is the messenger, the one to whom to entrust thoughts and writings. Sarcastic and talkative, she will inquire about your goals, she will ask you for information on the writings entrusted to her with explicit frankness and she will always have something to say about the message to bring but she will also be just as direct and precise in delivering it.

Nethergal is not just talk and gossip, whatever text is written she knows it, there is no written code or secret that she does not know.

The Devotee of Nethergal is a fine linguist, an expert in riddles and rebuses, a Devotee who, unlike Atmos, does not limit himself to keeping the writings but disseminates their knowledge.

A Nethergal Devotee is a teacher, a college language teacher, a learned expert on a thousand subjects.

Nethergal also has another role, is the Patron of dreams and visions, she shares this task with Sixiser who dominates the nightmares instead.\\

\noindent- \textbf{Symbol}: a white iridescent feather\\
- \textbf{Feature}: Dexterity\\
- \textbf{Traits}: Sincerity, Indolence, Resentment, Responsibility, Honesty, Perseverance\\
- \textbf{Manifestation}: cascade of feathers, a flying goose\\
- \textbf{Sum of common Traits to 5 points} points: You can send a message of up to 144 characters to a subject that you can see within 50 meters without being heard/seen. Once an hour. Cost 1 Action. The subject must understand the language used.\\
- \textbf{Sum of common Traits to 10 points}: By placing your hand on a book you learn its contents as if you had read it. One book a week. You lose the knowledge thus gained after a week. Time 1 Turn. The written language of the tome must be known.\\
- \textbf{Sum of common Traits to 15 points}: You can fly, as a spell of the same name, 1 hour per day. Cost 1 Reaction.\\
- \textbf{Sum of common Traits to 20 points}: You can cast the Zone of Truth spell, 3 times per day. DC 30 to resist. Cost 2 Actions.\\
- \textbf{Elements}: Electricity, Sound\\
- \textbf{Advantage}: Absolute Leadership\\
- \textbf{Privileged Spell Lists}: Transmutation, Air\\
- \textbf{Favorite Weapon}: Light crossbow\\
- \textbf{Rule}: Do not destroy a book or letter


\subsubsection{Nedraf}\index{Nedraf}\label{nedraf}

\begin{changemargin}{0.3cm}{0.3cm}\begin{emphasis}{
Does he really think he's fighting for anything other than his survival? (Matrix Revolutions, movie)
}\end{emphasis}\end{changemargin}


The Surviving Patron, the never tired old wolf who has gone through and fought a thousand battles. His flesh is wounded, his body covered in war scars and bruises but nothing will bring him down. Tenacity, passion, experience and a lot of anger make Nedraf not only an excellent fighter on any occasion but a connoisseur of the environment around him. Thanks to his impeccable training he knows how to make the most of the resources available. He can passionately goad the men at his behest.
Nedraf represents the one you would always want by your side in every battle.

Many mercenary captains and commanding officers are Nedraf's Devotees. The Devotee of Nedraf does not give up, does not give up, does not abandon his companions but this does not mean that he is rash or irrational in his choices.\\

\noindent- \textbf{Symbol}: a strong hand, wrapped in a bloodstained bandage brandishing a sword\\
- \textbf{Feature}: Constitution\\
- \textbf{Traits}: Perseverance, Responsibility, Gratitude, Courage, Intransigence, Pride\\
- \textbf{Manifestation}: the smell of blood and metal spreads in the air\\
- \textbf{Sum of common traits to 5 points} points: You can wear light Armour without penalty to the Magic Test\\
- \textbf{Sum of common Traits to 10 points}: Acquire a bonus point on a Weapon List. May or may not be known\\
- \textbf{Sum of common Traits to 15 points}: You can wear medium Armour without penalty to the Magic and Dexterity Test\\
- \textbf{Sum of common Traits to 20 points}: Acquire a bonus point on a Weapon List. May or may not be known\\
- \textbf{Elements}: Positive Energy, Sound\\
- \textbf{Advantage}: Accelerated healing\\
- \textbf{Privileged Spell Lists}: Enchantment, Earth\\
- \textbf{Favored Weapon}: Greatsword\\
- \textbf{Rule}: Don't abandon your teammates


\subsubsection{Nihar}\index{Nihar}\label{nihar}

\begin{changemargin}{0.3cm}{0.3cm}\begin{emphasis}{
What is a hero? He is an individual with great talent and extraordinary courage, who knows how to choose good over evil, who sacrifices himself to save others, but above all ... who acts when he has everything to lose and nothing to gain. (Lo chiamavano Jeeg Robot, movie)
}\end{emphasis}\end{changemargin}

He is the Patron Saint of accidental heroes. Thoughtful and calm, he loves good wine and carousing. He is the one you would never choose as a comrade in arms because of his \emph {common} appearance and his jovial attitude. But then when it's time to be there, to fight, to make the difference, with a lucky shot he solves the challenge.

He has the appearance of a small man, with sumptuous and refined clothes and a guarded and cheerful expression. He always protects himself no matter what, showing the world exactly what the world wants to see. He carefully monitors the reality around him and even if it is always easier to see him with a glass in his hand, if you do not let yourself be fooled by appearances you will notice how his eyes never lose sight of the danger, the problem. He is careful, he doesn't trust anything or anyone. He made his weaknesses into his strengths.

\noindent- \textbf{Symbol}: A dagger leaning next to a goblet of wine\\
- \textbf{Feature}: Intelligence\\
- \textbf{Traits}: Humility, Courage, Empathy, Responsibility, Envy, Greed\\
- \textbf{Protest}: the sound of a toast or the uncorking of a bottle\\
- \textbf{Sum of common Traits to 5 points} points: You can turn water into wine. One liter a day. Cost 2 Actions. 2 times a day.\\
- \textbf{Sum of common Traits to 10 points}: One Immediate Action, gains +2d6 bonus to one Action that round. Once a day.\\
- \textbf{Sum of common Traits to 15 points}: Your light weapon always deals critical damage on a hit. The bonus is always active.\\
- \textbf{Sum of common Traits to 20 points}: The delicacies you prepare are very good. Anyone satisfied with a dish prepared by you recovers 2d6 Hit Points and is cured of poisons, including magical ones. Max 6 people per day. 0.5 hours of preparation per person.\\
- \textbf{Elements}: Positive Energy, Fire\\
- \textbf{Advantage}: Universal language\\
- \textbf{Privileged Spell Lists}: Enchantment, Divination\\
- \textbf{Favorite Weapon}: Short sword\\
- \textbf{Rule}: Don't refuse a good glass of wine


\subsubsection{Orudjs}\index{Orudjs}\label{orudjs}

\begin{changemargin}{0.3cm}{0.3cm}\begin{emphasis}{
Nothing is easier than deluding yourself. Because man believes what he desires is true. (Demostesthenes)
}\end{emphasis}\end{changemargin}


Orudjs crawls in the depths of the caves, surrounded by gems, treasures, zealous servants.

Described as a shapeless slime by those who have perceived a semblance of the form, Orudjs is the Patron Saint of illusion and pretence.

With just his thought he convinces anyone of anything he wants. He loves the theater for what it is for him, the representation of falsehood, of being many people and in reality no one.

Where he dominates, chaos reigns where everyone is convinced that they are right and wars between clans feed their endless hunger for him.

He pretends to listen to those around him but in reality he is not interested in other people's stories because his are always the best. He is a coward without limits and a liar with always an advantage.

His Devotees are weak creatures, who need a master, a voice that constantly tells them what they need and what they want.

But also skilled actors and entertainers, he undercover spies, diplomats or politicians.\\

\noindent- \textbf{Symbol}: A theatrical mask with only an open mouth and eyes\\
- \textbf{Feature}: Charisma\\
- \textbf{Traits}:Laziness, Lust, Selfishness, Indifference, Hypocrisy, Empathy\\
- \textbf{Manifestation}: The sound of deep, contagious laughter\\
- \textbf{Sum of common Traits to 5 points} points: Your speech is already legendary. +2 to Perform checks.\\
- \textbf{Sum of common Traits to 10 points}: You can cast Silent Image 3 times per day.\\
- \textbf{Sum of common Traits to 15 points}: Your speech is already legendary. +1d6 additional on Perform checks. You can cast Greater Image 1 time per day.\\
- \textbf{Sum of common Traits to 20 points}: Your voice is soft. A creature you detect that listens to you for at least one minute must make a DC 30 Will save or be under the influence of dominate person. Once a day\\
- \textbf{Elements}: Electricity, Fire\\
- \textbf{Advantage}: Soft voice\\
- \textbf{Privileged Spell Lists}: Enchantment, Illusion\\
- \textbf{Favorite Weapon}: Rapier\\
- \textbf{Rule}: You must always have the last word


\subsubsection{Orlaith}\index{Orlaith}\label{orlaith}

\begin{changemargin}{0.3cm}{0.3cm}\begin{emphasis}{
I'm made to fight crime, not rule it. The time has not yet come when honest men can serve their country with impunity. The defenders of liberty will always be outcasts as long as the gang of scoundrels rules. (Maximilian de Robespierre)
}\end{emphasis}\end{changemargin}


Or the Patron of Justice and Vengeance. He slavishly follows the laws and demands that his subordinates carry out the orders given without any discussion. He is moved by a kind and good spirit which however he keeps well hidden behind his direct and incisive, shameless and deadly actions. Orlaith is vengeance made law. He acts out of a sense of justice with his methods of his. His bearing and proud gaze attract him.

Devotees of Orlaith are often judges and executioners, people who have decided to bring justice everywhere, because Orlaith cannot stand still, there is always someone to judge and punish.\\

\noindent- \textbf{Symbol}: A hand stretched over a closed book\\
- \textbf{Feature}: Strength\\
- \textbf{Traits}: Justice, Courage, Responsibility, Lust, Indifference, Intransigence, Resentment\\
- \textbf{Manifestation}: the image of a steelyard, unbalanced.\\
- \textbf{Sum of common Traits to 5 points} points: You call back to you 1 (normal) mastiff that obeys your commands. Duration 1 minute. Once a day. Cost 2 Actions.\\
- \textbf{Sum of common Traits to 10 points}: A pair of handcuffs manifests around the creature's wrists (maximum size large) within 27 meters. DC 25 Reflex save to cancel. Cost 2 Actions. Once a day. Strength/Escape Artist DC 20 to break free.\\
- \textbf{Sum of common Traits to 15 points}: Your hearing is only for the truth. Around you for 3 meters, including yourself, Zone of Truth is always active.\\
- \textbf{Sum of common Traits to 20 points}: You create a ray of Light 27 meters long and a few centimeters wide. Each creature crossed takes 5d6 damage, DC 25 Reflex to halve. Once a day. Cost 2 Actions.\\
- \textbf{Elements}: Light, Sound\\
- \textbf{Advantage}: Common Sense\\
- \textbf{Privileged Spell Lists}: Illusion, Fire\\
- \textbf{Favorite Weapon}: Infantry spear\\
- \textbf{Rule}: Do not refuse an order from a legitimate authority


\subsubsection{Rezh}\index{Rezh}\label{rezh}


\begin{changemargin}{0.3cm}{0.3cm}\begin{emphasis}{
Greed, I can't find a better word, is good, greed is right, greed works, greed clarifies, penetrates and captures the essence of the evolutionary spirit. Greed in all its forms: greed for life, for love, for knowledge, for money, has marked the forward momentum of all humanity. And greed, listen to me, will not only save Teldar Charter, but also the other dysfunctional corporation that goes by the name of America. (Gordon Gekko from the movie Wall Street, 1987)
}\end{emphasis}\end{changemargin}

The Patron who despises everything. Rezh loves, wants, touches, admires only his shiny and shiny coins. They are never enough, no wealth is ever enough for her. Rezh, the miser keeps everything to herself, knows no compassion, knows no charity, knows no sharing. Her hunger for money and riches makes her prone to any meanness. She despises everything and everyone and judges everything and everyone following only her personal yardstick. In every coin there is a little Rezh. Rezh's imprint can be seen in the oxidation of each coin.

In the depths of the caves the devotees of Rezh dig for treasure, desecrate catacombs and hunt, hunt and kill anyone who has something precious with them, even their best friend.

Among humans, the Devotees of Rezh become explorers, grave robbers, people always looking for treasure and an extra coin.\\

\noindent- \textbf{Symbol}: a stack of coins with a rat nearby\\
- \textbf{Feature}: Intelligence\\
- \textbf{Traits}: Greed, Hypocrisy, Cynicism, Selfishness, Indifference, Envy, Perseverance\\
- \textbf{Manifestation}: A sound of falling coins envelops the caster\\
- \textbf{Sum of common Traits to 5 points} points: You are an expert in coins and gems, no counterfeiter can deceive you. +1d6 on related Awareness and Knowledge checks.\\
- \textbf{Sum of common Traits to 10 points}: You use gems as receptacles. You can download a 3rd-level or lower spell into a gem, which must have a minimum value of 10gp. The gem holds the spell for 6 hours. To activate the gem, you use 2 actions and the spell it contains is performed.\\
- \textbf{Sum of common Traits to 15 points}: You can take 1 gold out of your pockets whenever you want. Max 10 gp per day. Cost 1 Action.\\
- \textbf{Sum of common Traits to 20 points}: Your Armour is covered in golden glitter and gems. You gain +4 to Defence and +1d6 Fortitude saves for 1 hour. Cost 1 Reaction, once per day.\\
- \textbf{Elements}: Void, Electricity\\
- \textbf{Advantage}: Fairy Hands\\
- \textbf{Privileged Spell Lists}: Abjuration\\
- \textbf{Favorite Weapon}: Sickle\\
- \textbf{Rule}: Don't leave treasure unattended


\subsubsection{Sumkjr}\index{Sumkjr}\label{sumkjr}

\begin{changemargin}{0.3cm}{0.3cm}\begin{emphasis}{
Anything that is not given is lost. (Dominique Lapierre)
}\end{emphasis}\end{changemargin}


Patron of the Arcanum of Light. Sumkjr is goodness, fairness, loyalty, justice, protection.

Sumkjr is the knight who protects the innocent, he is "Ljust's" sword in the final battle. He defends the weak and heals wounds.

Sumkjr brings the Light of Ljust everywhere, no danger can ever stop Sumkjr from his continuous, infinite, search for good.

A Sumkjr Devotee acts loyally and honorably, always pursuing the ultimate good, his being cannot be bent to evil, injustice, dishonor.

With courage and determination, the Devotee faces every challenge, not only out of a sense of duty, but because he is deeply devoted to his destiny. Sumkjr knows that few people hold up to this standard because unlike the Devotees of the Patroness of Genesis, his Devotees are not born to be such, but become so thanks to their deep and determined willpower.

For this reason Ljust intervenes in their favor with the elaborate Rite of Renewal, thanks to which every year the deserving and repentant Devotee of having lost, even if only for a little while, the right direction, the Light, is made to recover every Trait point lost because acted outside from the 7 Luminous Rules.

Sumkjr is a brave soldier, the best friend of the righteous.

Calicante, horrified at the sight of such a Patron, deprived him of the ability to love and experience true feelings of affection. Doing good for a Sumkjr Devotee is something normal just as it is normal not to be able to empathize with those who suffer. The Devotee knows what he has to do and why, but he is unable to be moved or to love in the face of the sufferings or caresses of a woman/man.\\

\noindent- \textbf{Symbol}: three drops of blood falling one after the other\\
- \textbf{Feature}: Charisma\\
- \textbf{Traits}: Kindness, Courage, Generosity, Justice, Honesty, Sincerity, Pride\\
- \textbf{Manifestation}: the Devout is wrapped in a cloak of golden brocade\\
- \textbf{Sum of common Traits to 5 points} points: The touch of your sword is life. A creature touched with your weapon regains 3d6 Hit Points. Once a day. Cost 2 Actions.\\
- \textbf{Sum of common Traits to 10 points}: Your Will is stronger than metal. You gain a +2 on Will saves\\
- \textbf{Sum of common Traits to 15 points}: You can cast the Cone of Cold spell, but the damage is from Electricity. DC 25 to halve. Once a day. Cost 2 Actions.\\
- \textbf{Sum of common Traits to 20 points}: You sacrifice your life to bring back to life a creature that has been dead for up to 1 week. Once. Cost 3 Actions.\\
- \textbf{Elements}: Positive Energy, Electricity\\
- \textbf{Advantage}: Aura of Courage\\
- \textbf{Privileged Spell Lists}: Heal\\
- \textbf{Favorite Weapon}: Bastard Sword\\
- \textbf{Rule}: Do not perform sexual acts\\


\textbf{The 7 Luminous Rules}\index{The 7 Luminous Rules}\\

The Seven Luminous Rules are a set of rules and behaviors held, for various reasons, by the Devotees who want to follow the path of the Light of Ljust.

Sumkjr's Devotees must follow all 7 of them under penalty of loss of power (Trait points), other Devotees of other Patrons, always positive or at least neutral, follow only some of these dictates, as a rule not to fall into Calicante's arms\\

1. Protect the weak and those who cannot defend themselves from abuse\\
2. Love life and protect it.\\
3. Fight against injustices and those who bring suffering and pain\\
4. Soothe wounds and pains. Calm souls and promote peace and harmony\\
5. Honesty and Loyalty are your foundation\\
6. You are a teacher of virtue. Let others be inspired by your deeds\\
7. Don't let your inaction breed suffering.\\


\subsubsection{Shayalia}\index{Shayalia}\label{shayalia}


\begin{changemargin}{0.3cm}{0.3cm}\begin{emphasis}{
He who plants a garden sows happiness (Chinese proverb)\\

Nothing great in the world has been accomplished without passion. (Georg Wilhelm Friedrich Hegel)
}\end{emphasis}\end{changemargin}


Patron of the Arcanum of Darkness. Shayalia is the dark soul of perdition, of betrayal, of the most sordid and sinful lust. She loves brothels. She likes the acrid smell of sweat, the shiny skin of oils and perfumes. The passions, the revenges that are consumed there, the physical and moral destruction that is perpetrated in those places is her life.

Shayalia is a woman who enjoys herself, who lives on more physical pleasures. She lives on long and carefully planned revenges. Vengeful and amoral, she does not judge by human standard, enjoy for her is not even remotely understandable. Shayalia is the closest to Calicante that has been created. It's the passions, the drives, the humoral liquids that make her inebriate.

Shayalia is the concubine who bewitches and destroys you, drop by drop. Her poisons are her weapons, human weaknesses her field.

Shayalia Devotees are spies, bastard children, lovers of powerful lords who operate in the shadows.

Ljust disgusted by the vision of such a Patron instilled in Shayalia a love of nature, plants and animals. And so many of the most famous botanists, herbalists and zoologists are Shayalia Devotees, perhaps the only things Shayalia can truly love.\\

\begin{changemargin}{0.3cm}{0.3cm}\begin{narrator}
While Efrem is the patron of pristine nature, Shayalia embodies devotion and love for nature. The first supervises nature from above, the second descends and becomes one with it, building magnificent gardens.
\end{narrator}\end{changemargin}

\noindent- \textbf{Symbol}: a crumpled pillow covered in blood\\
- \textbf{Feature}: Charisma\\
- \textbf{Traits}: Lust, Cynicism, Empathy, Responsibility, Hypocrisy, Resentment\\
- \textbf{Manifestation}: the Devotee is wrapped in a cloak of black velvet\\
- \textbf{Sum of common Traits to 5 points} points: The time to prepare a potion is halved. Healing spells also affect animals and plants.\\
- \textbf{Sum of common Traits to 10 points}: Your touch is life to nature. Your healing spells work on natural animals and plants to a maximum extent.\\
- \textbf{Sum of common Traits to 15 points}: From your palm you secrete poison. Your touch, or melee weapon delivers the poison. Fortitude save DC 25 or -2 to Wisdom and Dexterity for 10 minutes, a poisoned subject cannot be poisoned again for 24 hours. Cost 1 Action.\\
- \textbf{Sum of common Traits to 20 points}: Your touch is life to nature. You can heal magical animals and plants. You are immune to natural poisons. +1d6 Knowledge Nature.
- \textbf{Elements}: Void, Electricity\\
- \textbf{Advantage}: Animal Empathy\\
- \textbf{Privileged Spell Lists}: Illusion or Animals and Plants and an Elemental Spell List\\
- \textbf{Favored Weapon}: Whip\\
- \textbf{Rule}: Don't give up on humiliating


\bigskip

\textbf{Sumkjr} and \textbf{Shayalia} complement each other in holding the elusive rows of creatures. They act as a direct expression of the Patrons of Genesis.

\subsubsection{Sixiser}\index{Sixiser}\label{sixiser}

\begin{changemargin}{0.3cm}{0.3cm}\begin{emphasis}{
The strength that opposes fate is actually a weakness. (Franz Kafka)\\

Hush now baby, baby, don’t you cry.


Mother’s gonna make all your nightmares come true. (Mother, 1979 The Wall,  Pink Floyd)
}\end{emphasis}\end{changemargin}



The Patron who is indifferent to the present as totally, compulsively obsessed with the future and its destiny. In the most remote corners of the known worlds it is said that Sixiser accumulates everything, indifferent to everything and everyone.

Terrified of the future he sees, of a hypothetical end of himself and altogether lives a life of retreat, spiritual and physical. He voluntarily deprives himself of everything necessary. But at the same time he hoards any object that crosses his path in the hope of a return.

He's paranoid and doesn't trust anyone. He uses his powers of divination to know and scrutinize everyone.

Sixiser is the master of nightmares, dreams scarier than visions of death. He often uses nightmares as a means of communication with his followers.

Sixiser's Devotees are often necromancers surrounded by undead and other silent, obedient creatures. Those who take refuge in search of solitude and study, those who instead aim to expand and govern entire cities and nations in order to feel safer, are Devoted to Sixiser.\\

\noindent- \textbf{Symbol}: A chest overflowing with everything that cannot be closed\\
- \textbf{Feature}: Wisdom\\
- \textbf{Traits}: Gluttony, Indifference, Intransigence, Malice, Laziness, Sincerity \\
- \textbf{Manifestation}: two hands that surround, as if to hide, the caster's head\\
- \textbf{Sum of common Traits to 5 points} points: acquire crepuscular vision up to 18 meters, or 36 meters if already present.\\
- \textbf{Sum of common Traits to 10 points}: see in darkness even magic within 18 meters. Automatically detect non-magical traps within 3 meters of you.\\
- \textbf{Sum of common Traits to 15 points}: By touching an object you are able to understand all its magical and non-magical properties, even if it is cursed. 3 times a day.\\
- \textbf{Sum of common Traits to 20 points}: You are able to animate a creature that has been dead for no more than a day as an undead with 1 Challenge rank (zombie/skeleton type depending on status). Once a day. Cost 2 Actions.\\
- \textbf{Elements}: Electricity, Negative Energy\\
- \textbf{Advantage}: Reduced consumption\\
- \textbf{Privileged Spell Lists}: Necromancy\\
- \textbf{Favored Weapon}: Glaive\\
- \textbf{Rule}: Don't trust


\subsubsection{Tazher}\index{Tazher}\label{tazher}

\begin{changemargin}{0.3cm}{0.3cm}\begin{emphasis}{
A person often ends up looking like his shadow. (Rudyard Kipling)
}\end{emphasis}\end{changemargin}

The Patron of Shadows; he who is silent, kills you. You'll never know why. You will never know what he looks like, but if you suddenly get a cold feeling, Tazher is behind you ready to take your life.

Double agent with a bad soul, ask for his help only if you are willing to pay the price that he and he alone will decide.

He lives in darkness and blood. Shadows are his friends and darkness is his cloak. 

He surrounds himself with assassins, mercenaries, anyone who kills without feeling. In the depths of the underground he feeds his followers with pain, blood and death.

Ljust, horrified by so much hatred and nihilism, instilled respect for the dead in the Patron. A Deve si Tazher will not attack a deceased person or violate his corpse. Many undead hunters are devotees of Tazher.

The human Devotee of Tazher is the thief, the murderer, the bandit, anyone who lives for darkness and his own gain. A Devotee of Tazher is extremely dangerous in combat.\\

\noindent- \textbf{Symbol}: The gleam of the blade in the dark\\
- \textbf{Feature}: Dexterity\\
- \textbf{Traits}:Selfishness, Malice, Cynicism, Indifference, Pride, Hypocrisy, Perseverance\\
- \textbf{Manifest}: The Devout's shadow comes to life by moving the weapon\\
- \textbf{Sum of common Traits to 5 points} points: You gain +2 on Hide and Stealth checks.\\
- \textbf{Sum of common Traits to 10 points}: Once a day you make one more attack (without penalty). An Immediate Action.\\
- \textbf{Sum of common Traits to 15 points}: As long as you walk on shadows or in darkness (dim light or darkness) you are invisible. You can still be detected with light or divination spells.\\
- \textbf{Sum of common Traits to 20 points}: Every successful melee attack in this round generates a crit. Cost 1 Reaction Action to be declared also after the Attack Roll but before knowing if the roll were successful. Usable 3 times per day.\\
- \textbf{Elements}: Void, Ice\\
- \textbf{Advantage}: My shadow is my friend\\
- \textbf{Privileged Spell Lists}: Transmutation\\
- \textbf{Favorite Weapon}: Pole Glaive\\
- \textbf{Rule}: 5 Seconds. Time to steal from a dead man, no more.


\subsubsection{Thaft}\index{Thaft}\label{thaft}


\begin{changemargin}{0.3cm}{0.3cm}\begin{emphasis}{
When we exist, death is not; and when death exists, we are not. All sensation and consciousness ends with death and therefore in death there is neither pleasure nor pain. The fear of death arises from the belief that in death, there is awareness. (Epicurus)
}\end{emphasis}\end{changemargin}

The Patron who accompanies in birth and in death. Silent, he stands aside and watches the flow of men's lives. Almost humble in his simplicity, Thaft is everywhere. Silent witness of human life; the moment a life slips away, Thaft attends, the moment a life is born, Thaft is present.

Thaft also knows that one can't always be just an observer. Through his sacred and magical notebook he can decide and judge the lives of men, because if a sword wounds, it is only Thaft who decides his death.

The Devotees of Thaft are the priests of the last journey, those who protect and watch over the souls and bodies of the dead. Deeply opposed to the use of the undead, they seek their destruction.

A Devotee of Thaft respects life as well as death and is unafraid to bring destruction for greater balance.

Thaft was shaped by Atmos.\\

\noindent- \textbf{Symbol}: An open book with a skull on it\\
- \textbf{Feature}: Wisdom\\
- \textbf{Traits}: Patience, Kindness, Justice, Perseverance, Resentment, Indifference\\
- \textbf{Manifestation}: The cry of a newborn baby or the sigh of death can be heard\\
- \textbf{Sum of common Traits to 5 points} points: Your touch is lethal to the undead. Your touch deals 2d6 damage to an undead. Cost 2 Actions including touch. Up to 3 times a day.\\
- \textbf{Sum of common Traits to 10 points}: Your touch soothes. Once per day you can remove Blindness or Deafness. Cost 2 Actions.\\
- \textbf{Sum of common Traits to 15 points}: An undead must make a DC 30 Fortitude save or be destroyed if touched by your hand. Cost 2 Actions.\\
- \textbf{Sum of common Traits to 20 points}: Kill the touched creature. Will save DC 30 or death. Once a week. Cost 2 Actions.\\
- \textbf{Elements}: Sound, Electricity\\
- \textbf{Advantage}: Frosty Touch\\
- \textbf{Privileged Spell Lists}: Necromancy, Animals and Plants\\
- \textbf{Favorite Weapon}: Bow\\
- \textbf{Rule}: Don't create an undead


\subsubsection{Torbiorn}\index{Torbiorn}\label{torbion}

\begin{changemargin}{0.3cm}{0.3cm}\begin{emphasis}{
About arrogance, violent ones will suffer for her, but wise men laugh at her. (Tito Livio, attributed to Astimede)
}\end{emphasis}\end{changemargin}


The Patron who best embodies the concept \emph{is never enough}. Tall, beautiful like a classical statue but, just like the latter, without warmth and life, Torbiorn borders on maniacal perfection in his dressing and attitude.

Nothing is ever enough for him. No one is ever up to him. And here he goes, with arrogance and irony, to modify everything that can be modified in order to appease this profound dissatisfaction. If the final result achieved does not satisfy him, and it happens very often, his cynicism takes over and destroys everything without caring about the suffering he is causing to those around him.

In the Forsaken Lands the Devotee of Torbiorn is the iron-fisted Tyrant who acts only out of his own desire and pleasure without caring about anyone else.

The Devotee of Torbiorn is the typical rich and lazy aristocrat, the one who always seeks the easiest and least risky path.

Regardless of others, he enjoys exploiting other people's work and benefiting from it.

\noindent- \textbf{Symbol}: An opaque mirror\\
- \textbf{Feature}: Charisma\\
- \textbf{Traits}: Hypocrisy, Laziness, Pride, Gluttony, Empathy, Cynicism, Indifference\\
- \textbf{Manifestation}: Broken mirror shards all around the Devout like a whirlwind\\
- \textbf{Sum of common Traits to 5 points} points: With a gesture you can refresh your clothes and yourself, making them clean and perfumed. Cost 1 Action. 3 times a day.\\
- \textbf{Sum of common Traits to 10 points}: Your spit is poisonous. If the touch attack roll is successful -2 Strength. Duration 1 minute. Three times a day. Cost 1 Action.\\
- \textbf{Sum of common Traits to 15 points}: Fixing the lens in the eye forces it to stop. The target can no longer move its legs (or paws for movement). Will save DC 30. Once per day. Cost 2 Actions.\\
- \textbf{Sum of common Traits to 20 points}: Tendrils shoot from your fingers that sting up to 10 opponents. Each tendril, up to 20 meters long, causes 2d6 damage, DC 25 Reflex save for half. Cost 2 Actions.\\
- \textbf{Elements}: Fire, Sound\\
- \textbf{Advantage}: Hard to subjugate\\
- \textbf{Privileged Spell Lists}: Transmutation\\
- \textbf{Favored Weapon}: Hand Axe\\
- \textbf{Rule}: Don't be sloppy, poorly dressed or untidy.


\medskip

\begin{changemargin}{0.3cm}{0.3cm}\begin{narrator}

In agreement with the Arbiter, and adequately motivated, it is possible to change Advantage and Privileged Magic Lists.

\end{narrator}\end{changemargin}

\subsubsection{Patron List - Trait}\index{Patron List - Trait}\hypertarget{tabellacollegamentopatronotratto}{}\label{tabellacollegamentopatronotratto}

\medskip


\noindent \textbf{Rezh}: Avarice, Hypocrisy, Cynicism, Selfishness, Indifference, Envy, Perseverance\\

\textbf{Gradh}: Courage, Kindness, Responsibility, Respect, Sincerity, Pride, Envy\\

\textbf{Cattalm}: Cynicism, Pride, Selfishness, Intransigence, Lust, Sincerity, Perseverance\\

\textbf{Ljust}: Empathy, Courage, Patience, Respect, Sincerity, Perseverance, Honesty\\

\textbf{Belevon}: Envy, Lust, Selfishness, Malice, Empathy, Perseverance, Generosity\\

\textbf{Gaya}: Generosity, Kindness, Sincerity, Cynicism, Malice, Envy\\

\textbf{Ledyal}: Gratitude, Empathy, Kindness, Honesty, Patience, Humility\\	

\textbf{Sixiser}: Greed, Indifference, Intransigence, Malice, Laziness, Sincerity\\

\textbf{Atherim}: Honesty, Empathy, Gratitude, Responsibility, Sincerity, Intransigence\\	

\textbf{Nihar}: Humility, Courage, Empathy, Responsibility, Envy, Avarice.\\

\textbf{Torbiorn}: Hypocrisy, Laziness, Pride, Greed, Empathy, Cynicism, Indifference\\

\textbf{Atmos}: Indifference, Justice, Perseverance, Responsibility, Resentment,  Intransigence\\	

\textbf{Krondal}: Intransigence, Responsibility, Sincerity, Pride, Perseverance, Justice\\

\textbf{Orlaith}: Justice, Courage, Responsibility, Lust, Indifference, Intransigence, Resentment\\

\textbf{Sumkir}: Kindness, Courage, Generosity, Justice, Honesty, Sincerity, Pride\\

\textbf{Orudjs}: Laziness, Lust, Selfishness, Indifference, Hypocrisy, Empathy\\

\textbf{Shayalia}: Lust, Cynicism, Empathy, Responsibility, Hypocrisy, Resentment\\

\textbf{Calicante}: Malice, Resentment, Lust, Greed, Cynicism, Selfishness, Pride\\

\textbf{Thaft}: Patience, Kindness, Justice, Perseverance, Resentment, Indifference\\

\textbf{Nedraf}: Perseverance, Responsibility, Gratitude, Courage, Intransigence, Pride	\\

\textbf{Erondil}: Pride, Resentment, Selfishness, Empathy, Gratitude, Respect\\

\textbf{Laydel}: Resentment, Perseverance, Empathy, Intransigence, Envy, Cynicism\\

\textbf{Efrem}: Respect, Indifference, Laziness, Indolence, Honesty, Perseverance\\

\textbf{Lynx}: Responsibility, Courage, Honesty, Cynicism, Intransigence, Resentment\\

\textbf{Tazher}: Selfishness, Malice, Cynicism, Indifference, Pride, Hypocrisy, Perseverance\\

\textbf{Nethergal}: Sincerity, Indolence, Resentment, Responsibility, Honesty, Perseverance\\

\end{multicols}

%\pagebreak

\

\vfill

\begin{center}
\includegraphics[keepaspectratio,width=0.7\textwidth]{immagini/jung-archetipi-eng.png}

\textit{Jung's 12 Archetypes}
\end{center}

\pagebreak

\section{Equipment}\hypertarget{equipaggiamento}{}\label{equipaggiamento}

\subsection{Wealth and Money}\index{Wealth and Money}
\begin{changemargin}{0.3cm}{0.3cm}\begin{emphasis}{I'm ready to go. I have a backpack! (Morgan Grimes, Chuck, TV Series)
}\end{emphasis}\end{changemargin}


\begin{multicols}{2}

\label{ricchezza-e-denaro}

\lettrine[lines=2, lhang=0.33, loversize=0.25, findent=1.5em]{T}{he} common coins come in different denominations based on the relative value of the metal they are made of. The three most common types of coins are the gold coin (gp), the silver coin (sp), and the copper coin (cp).

A skilled (but not exceptional) craftsman can earn one gold piece per day. Gold coin is the standard measure of wealth, although the coin itself is not widely used. When merchants discuss deals involving goods or services worth hundreds or thousands of gold pieces, the transactions usually do not include any exchange of cash.

Gold coin is instead a measure of value, and the exchange is made in gold bars, letters of credit or valuable goods. A gold coin is worth ten silver coins, the most used type of coin among the population. A silver coin can cover the daily wages of a laborer and buy a cruet of lamp oil or a bed for a night in a cheap inn.

One silver coin is worth ten copper coins, which are normally used by workers and beggars. A single copper coin can buy a candle, a torch or a piece of chalk.

Sometimes, however, unusual coins made of other precious metals appear among the treasures. The electrum coin (ep) and platinum coin (pp) come from forgotten empires and lost kingdoms, and when used in transactions they sometimes arouse suspicion and skepticism. An electrum coin is worth five silver coins while a platinum coin is worth ten gold coins.

A common coin weighs about ten grams, so that fifty coins weigh half a kilo.

A character who starts playing generally has enough gold to buy the basic elements: some weapons, second-hand Armour (the least expensive one) and some miscellaneous equipment. As the character embarks on adventures and accumulates loot, he can afford better equipment and magical items. At first level, characters have coins and equipment totaling about 125 gp.

\subsubsection{Coins}\index{Coins}

The most common currency is the gold coin (gp). A gold coin is worth 10 silver coins (sp). Each silver coin is worth 10 coppers (cp). In addition to copper, silver and gold coins there are also platinum (mp) coins worth 10 gold and electrum (ep) coins worth 5 silver each.

\medskip

\textbf{Table: Coin Equivalence}\index{Table Coin Equivalence}

\medskip

\begin{tabular}{llllll}

\textbf{Coin} & \textbf{CP}&\textbf{SP}&\textbf{EP}&\textbf{GP}&\textbf{PP}\\
\toprule
Copper & 1 & 1/10 & 1/50 & 1/100 & 1/1000\\
Silver & 10 & 1 & 1/5 & 1/10 & 1/100\\
Electrum & 50 & 5 & 1 & 1/2 & 1/10\\
Gold & 100 & 10 & 2 & 1 & 1/10\\
Platinum & 1000 & 100 & 20 & 10 & 1\\
\end{tabular}

\medskip

Usually payments over 100 gold coins are made in bars of 1,2,5 kilograms of gold, equivalent to 100, 200 and 500 gold coins or better still in gems. In the case of even larger sums, a letter of credit from some bank may be issued (but valid in very few important cities).

\subsubsection{Level 1 Wealth}\index{Level 1 Wealth}\index{Coins at first level}

Usually a character with 1 part Weapon Proficiency or Magic Proficiency with 125 gp that he can spend on basic equipment.

\subsubsection{Other Wealth - Trade Goods}\index{Other Wealth}

Merchants usually trade goods even without the use of coins.
To get an idea of commercial transactions, some trade goods are described in the table.

\medskip

\textbf{Table: Other wealth examples}\index{Table Other wealth examples}

\medskip


\begin{tabular}{ll}
\textbf{Cost} & \textbf{Item}\\
\toprule
1 cp & Wheat (0.5 kg)\\
2 cp & flour (0.5 kg) or chicken (1)\\
1 sp& Iron (0.5 kg)\\
5 sp& Tobacco or copper (0.5 kg)\\
1 gp & Cinnamon (0.5 kg) or goat \\
2 gp & Ginger or pepper (0.5 kg) or mutton (1)\\
3 gp & Pork (1) \\
4gp & Linen (1m\textsuperscript{2}\\
5 gp & Salt or Silver (0.5 kg) \\
10 gp& Silk (1 m) or cow (1)\\
15 gp& Saffron(0.5 kg)/ox (1)\\
30 gp&cloves (1kg)\\
\end{tabular}

\medskip

See also the chapter on Encumbrance in Movement and Transport.

\end{multicols}

\pagebreak

\section{Equipment - Weapons}\index{Equipment - Weapons}\index{Armi}\label{equipaggiamentoarmi}
\hypertarget{equipaggiamento.armi}{}

\label{equipaggiamento---armi}
\begin{changemargin}{0.3cm}{0.3cm}\begin{emphasis}{
This is my rifle. There are many like him, but this is mine. My rifle is my best friend, it's my life. I must master it as I master my life. Without me my rifle is nothing; without my rifle I am nothing. I have to know how to hit the target, I have to shoot better than my enemy who is trying to kill me, I have to shoot before he shoots me and I will. In the presence of God I swear by this creed: my rifle and myself are the defenders of the country, we are the rulers of our enemies, we are the saviors of our lives and so be it, until there is no more enemy but only peace, amen. (Full Metal Jacket, Movie 1987)

\medskip

The really good sword is the one that stays in its sheath. (Sanjuro)}\end{emphasis}\end{changemargin}


\medskip

Remember that using a Weapon without the proper proficiency imposes a -1d6 to hit

The table presents the name of the weapon, its cost in gold coins, the damage and the type of damage (if from Cut, Hit or Point), the range, the Weapon List it belongs to and the special characteristics it can to have. \hyperref[sec:Special Actions in combat]{See section Special Actions in Combat}, see also \hyperref[sec:load-capacity-and-transport-encumbrance]{Cargo and Transport Capacity.}

\medskip

\textbf{Table: Weapon List}\index{Table Weapon List}

\begin{xltabular}{0.99\textwidth}{lllX}
\textbf{Weapon}&\textbf{Cost}&\textbf{Size/Damage} & \textbf{Range, List, Special}\\
\toprule
Axe Hammer& 16 & M/1d6 T/B& \textbf{Axes}\\
Bastard sword& 35 & M/1d8 S&\textbf{Swords}, 1d8 one-handed, 2d6 two-handed\\
Battle Axe& 10 & M/1d10 S&\textbf{Axes}\\
Brandistocco& 10 & M/2d4 P/S& \textbf{Spear}, Counter-Charge, Long Weapon\\
Broadsword& 12 & M/2d4 S&\textbf{Swords}, 2d4 one-handed, 3d4 two-handed\\
Composite Longbow& note*& G/Arrows& 36 meters, \textbf{Bows}\\
Composite Shortbow& note*& M/Arrows& 20 metres, \textbf{Bows}\\
Cudgel& 1& P/1d6 B& \textbf{Simple Weapons}, \textbf{Skull Breaker}\\
Dagger& 2& P/1d4 P& 6 meters, \textbf{Simple Weapons}, \textbf{Light Weapons}, \textbf{Thrown Weapons}\\
Double Flail& 90 & M/1d10 B& \textbf{Whirling Balls}, \textbf{Double Weapons}\\
Estoc& 25& G/1d8 P& \textbf{Spears}, Long Weapon\\
Falchion& 75 & M/2d4 S& \textbf{Graceful Weapons}, \textbf{Spears}, ED7\\
Flail& 8& M/1d8 B& \textbf{Whirling Balls}, \textbf{Skull Breaker}\\
Great Cudgel& 2& M/1d8 B&\textbf{Skull Breaker}\\
Great Double Axe& 25 & G/1d12 S& \textbf{Axes}, \textbf{Double Weapons}, Long Weapon\\
Halberd& 10 & G/1d10 P/S& \textbf{Lance}, \textbf{Spear}, Counter-Charge, Long Weapon, ED9 \\
Hand Axe& 6 & M/1d6 S& 6 m, \textbf{Axes}, \textbf{Thrown Weapons}, Versatile\\
Heavy crossbow& 50 & G/Bolts& 30 meters \textbf{Crossbows}\\
Heavy Flail& 15 & M/1d10 B& \textbf{Whirling Balls}\\
Heavy Mace& 5& M/1d8 B/S& \textbf{Skull Breaker}\\
Heavy Pike& 8& G/1d6 P&\textbf{Weapons of Death}, Long Weapon\\
Infantry spear& 2& M/1d8 P&3 m, \textbf{Lance}, Long Weapon, Countercharge\\
Javelin& 1& P/1d6P& 12 meters, \textbf{Spears}, \textbf{Thrown Weapons} \textbf{Simple Weapons}\\
Katana& 300& M/1d10 S& \textbf{Swords}, \textbf{Lethal Weapons}, ED9\\
Light crossbow& 35 & P/Bolts& 15 meters, \textbf{Simple Weapons}, \textbf{Crossbows}\\
Light Mace& 3& P/1d6 B/S& \textbf{Simple Weapons}, \textbf{Light Weapons}, \textbf{Skull Breaker}\\
Light Pike& 4& M/1d4 HP&\textbf{Weapons of Death}\\
Longbow& 75 & G/Arrows& 20 meters, \textbf{Bows}\\
Longsword& 15 & M/1d8 S&\textbf{Swords}\\
Machete& 10 & M/1d6 S&\textbf{Lethal Weapons}\\
Naginata& 8& G/1d10 S&\textbf{Lance}, Long Weapon, ED9\\
One-Handed Crossbow& 100& M/Bolts& 6m, \textbf{Crossbows}\\
Pole Glaive& 12 & G/1d10 P/S& \textbf{Lance}, Countercharge, Long Weapon, ED9\\
Punch/Barefoot& note*& P/1d4 B&Versatile\\
Quaterstaff& 3& M/1d6 B& \textbf{Simple Weapons}, Long Weapon, Versatile\\
Rapier& 20 & P/1d6 P& \textbf{Simple Weapons}, \textbf{Graceful Weapons}, Versatile\\
Scimitar& 15 & M/1d6 S&\textbf{Simple Weapons}, \textbf{Graceful Weapons}, Versatile\\
Scythe& 18 & G/2d4 P/S& \textbf{Weapons of Death}, Long Weapon\\
Shortbow& 30 & M/1d6 P& 15 m, \textbf{Bows}\\
Shortsword& 10 & P/1d6 P&\textbf{Simple Weapons}, \textbf{Swords}, Versatile\\
Sickle& 6& P/1d6 S& \textbf{Weapons of Death}\\
Sling& -& P/1d4 B& 10 m, \textbf{Thrown Weapons}\\
Spear& 10 & G/1d8 P&\textbf{Spear}, Long Weapon, Countercharge\\
Spiked Chain& 25 & G/2d4 P& 3 meters, \textbf{Whirling Balls}, Long Weapon\\
Spiked gauntlet& 5& P/1d4 P&\textbf{Stun Weapons}\\
Spiked Mace& 6& M 1d8 B/P& \textbf{Simple Weapons}, \textbf{Skull Breaker}\\
Trident& 15 & M/1d6 P/S& 3 meters, \textbf{Spears}, \textbf{Thrown Weapons}, Long Weapon, Countercharge\\
Truncheon& 1& P/1d6 B& \textbf{Stun Weapons}, non-lethal\\
Two-bladed sword& 100& G/1d8 S& \textbf{Dual weapons}, \textbf{Swords}\\
Two-handed sword& 50 & G/2d6 S&\textbf{Swords}\\
Urgrosh& 18 & M/1d6 T/P& \textbf{Lance}, \textbf{Dual Weapons}\\
Warhammer& 5& M/1d8 W/P& 6 m, \textbf{Skull Breaker}\\
Warmaul& 7& G/1d10 B& \textbf{Skull Breaker}\\
Whip& 1& M/1d3 S& \textbf{Whirling Balls}, Long Weapon\\

\end{xltabular}

\medskip

A \textbf{Weapon}Small has \textbf{Encumbrance} 1, a Medium Weapon has Encumbrance 2, a Large Weapon has Encumbrance 4, a Huge Weapon has Encumbrance 8.\index{Weapon Encumbrance Armi}\index{Encumbrance Weapon}

\medskip

\textbf{Table: List of bullets - Bows - Crossbows - Slingshots}\index{Table List of bullets - Bows - Crossbows - Slingshots}

\begin{tabular}{lcc}
\textbf{Projectile Name}& \textbf{Number of shots / Cost (gp)} & \textbf{Damage / Type}\\
\toprule
Crossbow bolts (one-handed, light) & 10/1 gp & 1d6 P\\
Crossbow bolts (heavy) & 3/1 gp & 1d10 P\\
Hunting Arrows (Shortbow, Longbow) & 20/1 gp & 1d6 P\\
War Arrows (Longbow) & 10/1 gp & 1d8 P\\
Marble ball (slingshots) & 15/1 gp & 1d4 W\\
Rock (slingshot)& -& 1d2 W\\
\end{tabular}

\medskip

A \textbf{Quiver} (full or empty) of Bullets has \textbf{Encumbrance} 2.\index{Encumbrance bullets}.\\
An \textbf{heavy bolt} for crossbow penetrates metal armor more easily, dealing an additional +2 damage.\index{Heavy crossbow bolt}\index{Heavy bolt}\index{Bolt, Heavy}


\begin{multicols}{2}

A +1 Weapon costs 1500 gp, +2 5000 gp. It is practically not possible to buy weapons with enchantments higher than +2, they have to be "found".

A magic arrow/dart/rock with a +1 bonus costs 25 gp, if +2 it costs 100 gp. Bullets with magic bonus greater than +2 are practically impossible to find.

\textbf{A projectile does not acquire magical properties because its launcher is magical.}

\medskip

\textbf{Empty Fist}: \hyperlink{pugnovuoto}{see List of Weapons}

\bigskip

\textbf{Composite Bow}\index{Composite Bow}
A composite bow is a particularly strong and stiff bow that requires a certain amount of strength to use.
A composite longbow has a fixed modifier, from +1 to +5. If the wearer has Strength greater than this modifier, he can apply a bonus equal to the composite bow's modifier to the arrow's damage.
A +3 composite bow used by a character with Strength 2 can draw the bow incompletely and thus the arrow that is fired will have a +2 damage modifier.
A +1 composite bow used by a character with Strength 4 will be fully fired but the damage modifier can be at most +1.

The cost of a composite bow depends on its modifier.
A composite bow with a +1 modifier costs 75 gp, +2 150 gp, +3 300 gp, +4 600 gp, +5 1500 gp. It is not possible to buy composite bows with bonuses higher than +3, they must be "found".

A composite shortbow has a maximum Strength modifier of +3.

\textbf{Crossbow}\index{Crossbow}\index{Recharge Crossbow}
A heavy crossbow requires two Actions to reload, thus allowing you to fire a bolt per round. A light or one-handed crossbow takes 1 action to reload.

\textbf{Range}\index{Range}
The indicated distance is the one at full attack roll. Each ranged weapon can strike within three times the indicated range.

If the target is within the indicated distance there is no penalty on hitting, if the target is between the first and second increments the penalty on hitting is -1d6. If the target is between the second and third increments the penalty to hit is -2d6.

A javelin thrown within 12 meters has no penalty, but thrown within 24 meters has a -1d6 to hit, at a distance between 24 and 36 meters a -2d6 to hit, beyond that it cannot be thrown.

\medskip

\begin{center}
\includegraphics[width=0.7\linewidth]{immagini/bow2.png}
\end{center}

\medskip

A \textbf{Arrow or Dart that hits is considered destroyed}, if it misses it is considered to have a 50\% (4-5-6 on a d6) probability that it is still intact.

A Magic Arrow/Dart/Rock adds its bonuses to those of the caster to determine the Attack Roll and Damage.

Remember that a normal projectile fired from a magic launcher does not become magical.

\medskip

The \textbf{Weapon Size} is indicated as P (small), M (medium), G (large). See section \hyperref[armatroppogrande]{Weapon too big}.

A \textbf{larger weapon} \index{Bigger weapons} such as a Longsword forged for an Ogre increases its damage die by one category (1d4-1d6-1d8-1d10-2d6-2d8-2d10..)


Weapons have a \textbf{Damage Type}\index{Damage Type}, i.e. S/B/P.
These letters indicate whether the damage is of the type Slashing, Bludgeoning, Piercing. This feature can be important because certain creatures may be immune or suffer less damage from a particular type of wound (eg a skeleton against a piercing weapon or a gelatinous cube against a slashing weapon...).

A weapon can be used to deal a different type of damage (from slashing to piercing or bludgeoning) by reducing the damage die by one category (e.g. Longsword to deal bludgeoning damage causes 1d6).

\medskip


\textbf{Masterwork Weapons}\index{Masterwork Weapons}

A masterwork weapon is a weapon created by a very skilled gunsmith which, although not magical, thanks to its perfect balance and sharpness, has a +1 attack roll.

To create a masterwork weapon, a gunsmith must exceed the DC set for the creation of the weapon by 10.

A masterwork weapon costs twice as much as a normal weapon.

\textbf{Improvised Weapons}\index{Improvised Weapons}

Sometimes items that weren't meant to be weapons can have some combat effectiveness. Since these are not objects designed for this use, the creature that attacks with one of them takes a -1d6 penalty on the attack roll. Improvised small weapon (bottle) does 1d3 of damage, medium-sized (chair leg) 1d6, large (table leg) does 1d8 of damage.

An improvised thrown weapon has a range of 3 meter.

\medskip

\textbf{Throwing Weapons}\index{Throwing Weapons}

A sword or other weapon not meant to be thrown can still be thrown at the opponent. The attack roll takes a -1d6 and the weapon does a lower damage category (long sword does 1d6, short sword 1d4..). The range is 3 meters.

\medskip

\textbf{Using a weapon without proper proficiency if it is not a simple weapon} imposes a -1d6 on the attack roll.

\textbf{Example}: A small creature using a halberd in close combat has -1d6 because the weapon is large, -1d6 because it is not proficient, -1d6 because it uses the weapon in melee.

In this case, as the penalties are higher than 3d6, the character does not roll dice but only uses his Weapons Proficiency and Strenght as a value to hit.

\end{multicols}

\vfill

\begin{center}
\includegraphics[width=0.6\linewidth]{immagini/armiriempitivo3.png}
\end{center}


\pagebreak

\section{Equipment - Armour and Shields} \index{Armour}\index{Shields}\hypertarget{equipaggiamento.armature.scudi}{}\label{equipaggiamentoarmature}

\label{equipaggiamento---armature-e-scudi}

\begin{changemargin}{0.3cm}{0.3cm}\begin{emphasis}{
Armour (s.f.). Dress that is worn if one's tailor is a blacksmith. (Ambrose Bierce)


\medskip

Fantozzi armour: weather vane 4 winds acting as a plume, scary Viking helmet with zero visibility, bronze jockstrap taken from the statue of Pepin the Short and, at the feet, irons with molten lead charcoal. Overall weight of Fantozzi armor: 4 quintals, 32 kilos and 7 and a half hectograms. (Superfantozzi, Film)} \end{emphasis}\end{changemargin}\medskip

\lettrine[lines=2, lhang=0.33, loversize=0.25, findent=1.5em]{T}{he} Armour helps to be unaffected (raises Defence) and penalizes Magic Test and proficiencys.

The Skills penalty is the penalty that applies to skill checks influenced by the weight and Encumbrance of the armour. Different specific or magical Armours have different ratings, this table serves as a guideline for the Arbiter.\index{Penalty due to armor}

\subsubsection{Armour Table}\index{Table Armour}

\label{tabella-armature}
\begin{tabular}{llllllll}
%\begin{xltabular}{0.95\textwidth}{lXXXXXXX}
\textbf{Armour} & \textbf{Cost (gp)} & \textbf{Defence} & \textbf{Penalty Skill} & \textbf{Type} & \textbf{Move} & \textbf{Magic Test}&\textbf{Encumbrance}\\
\toprule
Padded & 5 & 1 & 0 & L & 0 & NO&2\\
Leather & 10 & 2 & 0 & L & 0 & SI&2\\
Studded Leather & 25 &3 & 0 & L & 0 &SI&2\\
Chain shirt & 15 & 4 & -1 & M & 0 &+1d6&4\\
Scales & 50 & 5 & -1 & M & 0 &+1d6&4\\
Chain Rings & 150 & 6 & -1 & M & 0 &+1d6&4\\
Breastplate & 200 & 6 & -2 & M & 0 &+1d6&4\\
Bands & 250 & 7 & -2 & P & 0 &+2d6&8\\
Half Armour & 1200 & 8 & -2 & P & 1 &+2d6&8\\
from Field& 1400 & 9 & -3 & P & 2 &+2d6&8\\
Complete & 1500 & 10 & -4 & P & 3 &+2d6&8\\
\end{tabular}

\begin{multicols}{2}

\textbf{Cost}: It's for one medium size Armour.

\textbf{Defence}: is the bonus given to Defence

\textbf{Penalty Skill.}: it is the penalty given to Skill checks given by the weight and Encumbrance of the armour.

\textbf{Type}: Indicates whether the Armour is \textbf{L}light, \textbf{M}medium, or \textbf{P}heavy. Light \textbf{Armour} has \textbf{Encumbrance} 2, Medium Armour has Encumbrance 4, Heavy Armour has Encumbrance 8.\index{Armour Encumbrance}

\textbf{Mov. (movement)}: is the reduction in meters of movement to be applied per Movement Action.

\textbf{Magic Test}: indicates whether the check is to be done (YES) or not (NO). If dice are indicated (+1d6,+2d6..) it means that the Magic Test is to be done with the added dice marked. See Armour and Shields and Magic.

\textbf{Costs}: A +1 Armour or shield costs 2250gp, +2 10000gp. It is practically not possible to buy Armour or shields or weapons with enchantments higher than +2, they have to be \textit{found}.

\subsubsection{Armour, Shields and Magic}\index{Armour and Magic}\index{Shield and Magic}\hypertarget{armatureemagie}{}\label{armatureemagie}

All Armour, with the exception of Padded Armour. force the caster to pass a Magic Test disregarding any critical successes.

Light Armour and Shields make the Magic Test with no dice added, Medium ones with +1d6, Heavy ones with +2d6. In practice, the heavier the Armour or shield, the more dice you roll, the more chances there are of a failure.

Even if it is the player who requests a Magic Test, only one test will be performed with the sum of the dice due to armor and/or shields. Wearing the armor will negate any critical successes rolled.

\medskip

\begin{changemargin}{0.3cm}{0.3cm}\begin{narrator} When calculating the Encumbrance given by the Armour and shield \textbf{worn} you must divide it by two.

Encumbrance of Armour and shields is to be understood when it is "loaded in the backpack", i.e. carried but not worn.\end{narrator}\end{changemargin}

\subsubsection{Description of Armour}

\textbf{Light Armour}

Made of lightweight and flexible materials, light Armour favors agile adventurers as it offers protection without sacrificing mobility.

\textit{Padded}. Padded Armour consists of layers of fabric and padding stitched together.

\textit{Leather}. The bodice and shoulder pads of this Armour are made of leather hardened after being boiled in oil. The rest of the Armour is made up of
softer and more flexible materials.

\textit{Studded Leather}. Made of tough but supple leather, the stiffened leather Armour is studded with rivets or spikes.

\medskip

\textbf{Medium Armour}

Medium Armour offers more protection than light Armour, but limits movement.

\textit{Chain Shirt}. Composed of intertwined metal rings, a coat of mail is worn over layers of clothing or leather. This type of Armour offers modest protection to the upper body, while the noise of the rings rubbing together is muffled by the other layers.

\textit{Scales}. This Armour consists of a coat and greaves (sometimes also a separate skirt) of leather covered by overlapping pieces of metal, similar to the scales of a fish. The Armour comes complete with gauntlets.

\textit{Chain Rings}. This Armour is leather Armour with heavy rings sewn onto it. The rings serve to reinforce the Armour against sword and ax blows. The Armour comes complete with gauntlets.

\textit{Breastplate}. This Armour consists of a metal bodice worn over a layer of leather. Although it leaves the arms and legs relatively uncovered, the Armour provides good protection to the character's vital organs, without taking too much space.

\begin{center}
\includegraphics[width=0.7\linewidth]{immagini/donnacavalierecavallo.png}
\end{center}

\textbf{Heavy Armour}

\textit{Bands}. This Armour is made of strips of metal sewn to a sturdy back of leather and mail. The size of the metal plates, interconnected to the metal bands, and the layers of Armour underneath make it one of the most protective of Armours.

\textit{Half Armour}. Half plate Armour consists of shaped metal plates that cover much of the character's body. It does not include leg guards other than simple greaves tied with leather laces.

\textit{Field Armor}. Very similar to full plate Armour but lighter in construction sacrificing some protection for greater flexibility and mobility.

\textit{Complete}. This Armour consists of interlocking shaped metal plates that cover the entire body. Plate Armour includes gauntlets, heavy leather boots, a visor helmet, and a thick layer of padding under the Armour. Buckles and laces distribute the weight of the Armour throughout the body.


\subsubsection{Basic rules for using Armour}

\textbf{Using Armour without the proper proficiency} prevents the use of the Dexterity bonus and decreases the Defence bonus provided by 1.

\textbf{Using a shield without proper proficiency} worsens the attack roll by 1 and decreases the Defence bonus granted by 1.

\textbf{Sleeping in Armour}: If you sleep in medium or heavy Armour, you are automatically \hyperlink{affaticato}{Fatigued} the following day.

Sleeping in light Armour does not cause fatigue.

The \textbf{movement range} of the character will remain the same up to the banded Armour then it will decrease progressively. The value indicated in the column Mov. are the fewer meters the character travels per Move Action.

For example a human in full Armour has movement 6 meters, a dwarf 3 meters.

\textbf{Weight}: the indicated weight refers to the version for Medium-sized characters. Armour adapted for Small-sized characters weighs half as much, while for Large-sized characters it weighs twice as much.

\textbf{Masterwork Armour}\index{Masterwork Armour}

A perfect Armour is an Armour created by a very skilled blacksmith which, although not magical, thanks to its perfect balance and sharpness, has a +1 to Defence.

To create a masterwork weapon, a blacksmith must exceed the DC set for the creation of Armour by 10.

Masterwork Armour costs twice as much as normal Armour.\\

\subsubsection{Magic Armour}\index{Magic Armour}\index{Magic Shield}

A magical armor or magical shield not only protects better but is also lighter and similar to magic.

+2 armor lowers the Proficiency penalty by 1 and the Movement penalty by 1 meter.
+3 armor or shield also removes 1 die from the Magic Test if added. +4 armor further removes 1 from the Proficiency penalty, 1m from Movement and removes 1 die from the Magic Test.


\subsubsection{The Shields}

The \textbf{Shields} \index{Shields}allow you to increase your Defence, the more the shield is imposing and heavier the more it protects, the more the penalties to Magic Proficiencys increase and the less it makes it easier to fight (Attack Roll penalty to hit).

Shields can be of Light, Medium, Heavy type.

\end{multicols}

\subsubsection{Shields table}\index{Shields table}

\label{tabella-scudi}

\begin{tabular}{lccccc}
\textbf{Shields} & \textbf{Cost} & \textbf{Defence Bonus} & \textbf{AR Penalty} & \textbf{Magic Test} & \textbf{Type}\\
\toprule
Buckler & 5 gp & 1 & 0& SI & L\\
Light wooden shield & 3 gp & 1 & 0& SI & L\\
Light Metal Shield & 9 gp & 1 & 0& SI & L\\
Medium wood shield & 5 gp & 2 & 0& +1d6& M\\
Medium metal shield & 12 gp & 2 & 0& +1d6 & M\\
Heavy wooden shield & 9 gp & 3 & 1& +2d6 & P\\
Heavy metal shield & 20 gp & 3 & 1& +2d6 & P\\
\end{tabular}

\begin{multicols}{2}

\textbf{The bonus is granted to Defense}.

\textbf{Attack roll Penalty}: it is the penalty to the attack roll that occurs when the shield is worn.

\textbf{Type}: indicates the size of the shield. \textbf{L}light, \textbf{M}edium, \textbf{P}heavy.

A Light \textbf{Shield} has \textbf{Encumbrance} 1, a Medium Shield has Encumbrance 2, a Heavy Shield has Encumbrance 4.\index{Encumbrance for Shields}

The penalty to the \textbf{Magic Check} is added to any penalty due to the armor and is applied when the shield is worn.\index{Shield and Armor Magic Penalty}

A shield can be used as an \textbf{improvised weapon}. The attack roll is penalized by -1d6 and a small shield does 1d4 damage (B/T), a medium shield does 1d6 damage (B/T), a heavy shield does 1d8 damage (B/T).

Using the shield as an improvised weapon does not apply its bonus to Defense unless you take an Reaction to reset it to Defense after attacking.

Holding a shield takes up one hand/arm.


\subsubsection{Donning and Remove Armour}\index{Donning and Remove Armour}

Putting on and taking off Armour is an operation that requires time and attention, doing it quickly doesn't help and indeed tends to worsen the protection given by the Armour.

\end{multicols}

\textbf{Table: Times for putting on and taking off Armour}\index{Table Times for putting on and taking off Armour}

\begin{tabular}{llll}
\textbf{Armour Type}& \textbf{Put On} & \textbf{Put On Quickly} & \textbf{Take Off}\\
\toprule
Shield& 1 action & - & 1 action\\
Padded, Leather, Studded Leather & 1 minute& 3 rounds & - \\
Chain shirt& 1 minute& 5 rounds & 5 rounds\\
Scales, Rings, Breastplate, Bands & 4 minutes & 1 minute{*} & 1 minute\\
Half Armour, Field, Complete & 4 minutes{*}{*}& 4 minutes{*}& 1d4+1 minutes\\
\end{tabular}

\bigskip

\begin{multicols}{2}

{*} If someone helps, the time is halved. A single character doing nothing else can help one or two characters adjacent to him. Two characters cannot help each other put on Armour at the same time.

{*}{*} You need help to don this Armour. Without help it can only be put on quickly.

\textbf{Donning Armour quickly} implies a -1 penalty to Defence and an additional +1 penalty to proficiency checks.\\


\end{multicols}


\vfill

\begin{center}
%\includegraphics[width=0.4\linewidth]{immagini/armaturacorpetto.png}
\includegraphics[width=0.4\linewidth]{immagini/buckler.png}

\textit{Buckler, front and back}

\end{center}

\pagebreak


\section{Goods and Services}\index{Goods}\index{Services}


\subsection{Wealth and Money}\index{Wealth and Money}


\begin{changemargin}{0.3cm}{0.3cm}\begin{emphasis}{
- Doc... you just need a little plutonium.

\medskip

- Ah, I'm sure that in '85 plutonium can be bought in the local grocery store, but in '55 the matter is much more complicated! (Back to the Future, Movie 1985)}
\end{emphasis}\end{changemargin}\medskip

\begin{multicols}{2}

\subsubsection{Selling Treasures}

\lettrine[lines=2, lhang=0.33, loversize=0.25, findent=1.5em]{I}{n} dungeons you explore will have ample opportunity to find treasure, equipment, weapons, Armour and more. Usually, you will be able to sell treasures and trinkets when you reach a town or other settlement, as long as you can find buyers and merchants interested in your loot.

\medskip

\textbf{Weapons, Armour and Other Equipment }

As a general rule, undamaged weapons, Armour, and other equipment cost half the price when sold. Weapons and Armour used by monsters are unlikely to be in prime condition for sale.

\medskip

\textbf{Magic Items}

Selling magical items is a problem. Finding someone willing to buy a potion or scroll isn't too difficult, but most items are beyond the reach of anyone but the wealthiest nobles. Also, aside from a few common magic items, it's hard to find magic items or spells for sale. The value of magic winks at base money and should always be treated with respect.

\medskip

\textbf{Gems, Jewels and Works of Art}

These items retain their full market value, and you can choose to trade them for cash or use them as currency in transactions. In the case of exceptionally valuable treasures, the Arbiter may require that you first find a buyer in a large country or even a larger community.

\medskip

\textbf{Goods}

On the borderlands, most transactions are by barter. Like gems and objets d'art, commodities—iron ingots, sacks of salt, livestock, and so on—can be traded as standard money at their full value.

\medskip

\textbf{Perfect Equipment}

A perfect tool as well as costing much more than the normal version, grants a +1 to the check in which it is used.


\end{multicols}

\vfill

\begin{center}
\includegraphics[width=0.7\linewidth]{immagini/jewelry-box-2931784_1280.png}
\end{center}

\pagebreak

\begin{multicols}{2}

\subsubsection{Adventure Gear}\index{Adventure Gear}\index{Things to buy}

This is a short, non-exhaustive list of equipment your characters may be interested in purchasing. The list is by no means exhaustive or complete but may provide you with pricing guidelines.

As a Arbiter, always use common sense in requests, carefully evaluate the type of request, the need for the object, the place where you buy it and how you buy it.

Depending on the type of companion, additional items such as firearms or alchemicals may be available.

{\small
\begin{tabularx}{0.42\textwidth}{lll}
\textbf{Item}&\textbf{Cost}&\textbf{Encumbrance}\\
\toprule

Abaco&2 gp&L\\
Acid (ampoule) & 10 gp & L \\
Alchemist's fire (flask)& 20 gp& L\\
Alchemist's laboratory & 200 gp & 5\\
Ampoule (empty)& 3 cp& L \\
Antitoxin (flask) & 50 gp & L\\
Backpack & 2 gp & 1 \\
Bag for Components & 25 gp& L\\
Bag of Impediment & 50 gp & L \\
Bag& 5 ma&L\\
Bandolier & 3 gp & L\\
Banquet (per person) & 10 gp & -\\
Barge& 3000 gp & -\\
Barrel (empty)& 2 gp& 4\\
Basket (empty) & 4 but& 1 \\
Bedroll& 3 gp& 2 \\
Beer Mug& 5 cp& L\\
Bell & 1 gp& - \\
Belt Pouch (empty) & 1 gp& L\\
Bit and bridle & 2 gp&1\\
Bottle of ink or potion & 1 gp& L \\
Bread (per loaf) & 2 cp& -\\
Bucket (empty)& 5 sp& L\\
Burglary tools & 30 gp & 1\\
Camouflage Tricks & 50 gp& L\\
Candle & 1 cp& -\\
Canvas (per sq m)& 1 sp& L\\
Carafe beer& 2 sp& L\\
Carriage & 100 gp & -\\
Cart & 15 gp & 10\\
Case for Darts or Arrows & 1 gp & 1 \\
Case for maps or scrolls & 1 gp& 1 \\
Ceramic jug & 2 cp& L\\
Ceramic jug (5lt) & 2 cp& L\\
Ceramic mug & 2 cp& L\\
Chain (3m) & 30 gp & 1\\
Chalk, (1 piece) & 1 cp& - \\
Cheese (1 piece)& 1 sp& \\
Chest & 5 gp&4\\
Chest (small) & 2 gp & 3 \\
Climbing gear & 80 gp& 1\\
Coarse hemp rope (15 m)& 2 gp& 2 \\
Common Wine (carafe) & 2 sp& 1\\
Common Lantern& 1 gp& 2 \\
Common musical instrument& 5 gp& 2\\
Courtier's dress & 30 gp & 1\\
Craftsman Dress& 1 gp& 1\\
Craftsman's tools& 5 gp& 2\\
Crappy Inn & 2 sp& -\\
Crayfish& 5 gp&3\\
Crowbar& 2 gp& 1 \\

\end{tabularx}

\begin{tabularx}{0.42\textwidth}{lll}
\textbf{Oggetto}    & \textbf{Costo} & \textbf{Ingombro}\\

Devout Robe & 5 gp & 1\\
Entertainer Suit & 3 gp& 1\\
Exotic saddle& 40 gp& 3\\
Explorer outfit & 10 gp & 1\\
Firewood (per day)& 1 cp& 4 \\
Fish hook & 1 sp& - \\
Fishing net (2.25 m)& 4 gp& 1 \\
Fishing rod & 1 gp&1\\
Galley & 30k gp & -\\
Galloping saddle & 30 gp & 2\\
Glass bottle & 2 gp& L \\
Good Inn & 2 gp& -\\
Grapple & 1 gp& 1 \\
Hammer& 5 ma& 1 \\
Handcuffs & 15 gp & L \\
Healer's Bag & 50 gp & 1\\
Hemp Rope (15m)& 1 gp& 1\\
Holly and Mistletoe & - & -\\
Holy water (ampulla) & 25 gp& L\\
Hourglass& 25 gp & -\\
Ink (30 g bottle)& 8 gp& - \\
Iron pot & 8 ma& 1 \\
Lock/padlock Good & 80 gp & -\\
Lock/padlock Medium & 40 gp& \\
Longship & 10k gp & -\\
Magnifying glass & 100 gp & -\\
Mallet& 1 gp& 2 \\
Meals (per day) Normal& 3 Tue&-\\
Meals (per day) Poor & 1 Tu&-\\
Meals (per day) Voucher & 5 Tue&-\\
Meat (1 piece) & 3 ma& L\\
Merchant's balance & 2 gp& 1\\
Metal Spheres (100) & 3 gp & 1\\
Metal hook & 1 gp& L\\
Military saddle & 50 gp & 3\\
Miner's pick & 3 gp& 2 \\
Monk dress & 5 gp& 1\\
Nobleman's dress & 75 gp & 2\\
Oar & 2 gp& 2\\
Oil for lantern& 1 sp& 1 \\
Pack saddle & 15 gp & 2\\
Paper (sheet)& 4 sp& -\\
Parchment (Sheet) & 2 sp& - \\
Peasant dress& 1 ma& 1\\
Perfume & 5 gp & L\\
Potion of Healing & 80 gp & L\\
Potion of Empowered Healing & 125 gp & L\\
Portable ram & 10 gp& 3 \\
Projecting Lens Lantern & 12 gp & 1 \\
Pulley and tackle & 20 gp& 2 \\
Regular Inn& 5 sp& -\\
Rock peg& 1 ma&L\\
Rod (3 m) & 5 cp& 2\\
Rowboat & 50 gp & 12\\
Royal dress & 200 gp & 3\\
Sack (empty) & 1 sp& L \\
Saddlebags & 4 gp& 2\\
Sailing ship & 10k gp & -\\
Scholar's Robe & 5 gp& 1\\
Sealing wax& 1 gp& -\\
Sewing needle & 5m &- \\
Shieldable lantern& 7 gp& 1 \\
Shovel or shovel & 2 gp& 1 \\
Signet Ring & 5 gp& - \\
Silk Rope (15 m)& 10 gp & L\\
Silver Holy Symbol & 25 gp& L\\
Simple lock/padlock & 20 gp & -\\
\end{tabularx}

\begin{tabularx}{0.42\textwidth}{lll}
\textbf{Oggetto}    & \textbf{Costo} & \textbf{Ingombro}\\

Sled& 20 gp & 3 \\
Small metal mirror & 10 gp & L\\
Soap (per 0.5 kg) & 5 sp& - \\
Spade or Shovel & 1 gp&1\\
Stabling (per day) & 5 sp& -\\
Staff & 2 gp& 1\\
Steel and flint & 1 gp&\\
Stepladder (3 m) & 2 ma& 3 \\
Superior lock/padlock& 150 gp & - \\
Telescope & 1,000 gp & 1 \\
Tent & 10 gp & 3 \\
Torch& 1 ma& 1\\
Travel rations (per day) & 5 sp& L \\
Traveller's Dress & 1 gp& 2\\
Caltrops (20) & 1 sp& L \\
Wine, good (bottle) & 10 gp & 1\\
Wagon & 35 gp& 20\\
Warship & 25k gp & -\\
Water clock & 1,000 gp & -\\
Waxed&5 ma&1\\
Whetstone & 2 cp& L \\
Whistle & 8 ma& - \\
Wineskin & 1 gp& 1 \\
Winter blanket & 5 ma& 1 \\
Winter dress & 8 gp& 2\\
Wooden holy symbol & 1 gp& L\\
\end{tabularx}}

\medskip


\begin{changemargin}{0.3cm}{0.3cm}\begin{emphasis}{
Any sufficiently advanced technology is indistinguishable from magic. (Arthur C. Clarke, from Profiles of the Future)
}\end{emphasis}\end{changemargin}


\textbf{Intense Acid}. As an action, you can scatter the contents of this vial on a creature within 3 feet of you or throw the vial up to 20 feet, shattering it on impact. In either case, make a ranged attack roll against the creature or object, treating the acid as an improvised weapon (-1d6 attack roll). On a hit, the target takes 2d6 acid damage.

\textbf{Holy Water}. As an action, you can scatter the contents of this flask on a creature within 3 feet of you or throw the flask up to 20 feet, shattering it on impact. In either case, make a ranged attack roll against the creature or object, treating the holy water as an improvised weapon. On a hit, and the target is a fiend or undead, it takes 2d4 points of positive energy damage.

\textbf{Ampolla (empty)}: small glass or ceramic amphora with a thin neck.

\textbf{Signet Ring:} metal circle, generally valuable, with an engraving suitable for imprinting seals on sealing wax.

\textbf{Poison Ring:} +20 gp, compared to ring cost, this ring has a small compartment under the gem, usually used to hold poison. Opening and closing it requires an action; doing so without being noticed requires a DC 20 Fey Hands check.

\textbf{Antitoxin}. A creature that drinks from this vial of liquid gets +1d6 on saving throws against poison for 1 hour. Grants no bonus to undead and constructs.

\textbf{Portable Ram}. You can use a portable battering ram to break down doors. When you do so, you gain a +1d6 bonus on Strength checks. Another character can help you with the use of the ram, giving you +2 on the check.

\textbf{Fishing Equipment}. This kit includes a wooden rod, silk thread, wooden cutter, steel hooks, lead weight, velvet lures and a landing net.

\textbf{Bandolier}. This specialized belt for holding small items such as potions or scrolls is worn around the neck. Extracting and drinking an item from it costs 1 Action.\index{Bandolier}

\textbf{Metal Marbles}. As an action, you can scatter a single bag of these tiny metal marbles to cover a flat area 10 feet square on a side. A creature that passes through the covered area must succeed at a DC 12 Reflex save or be knocked prone. A creature that passes through the area at half speed need not make a saving throw.

\textbf{Merchant Scales}. A merchant scale includes a small balance wheel, a pan, and an assortment of weights up to 1 pound. With it, you can measure the exact weight of small items, such as precious metals or commodities, to help you determine their value.

\begin{center}
	\includegraphics[height=0.6\linewidth]{immagini/stadera.png}
\end{center}

\textbf{Components Exchange}. A component pouch is a small, waterproof leather belt pouch with compartments containing all the material components and other special items you need to cast your spells, except for those components that have a specific cost or are uncommon materials ( as indicated in the spell description).

\textbf{Bag}. A cloth or leather pouch can hold, among other things, up to 20 slingshot projectiles or 50 blowgun needles. A pouch divided into compartments for holding spell components is called a component pouch.

\textbf{Candle}. For 1 hour of real game time, a candle sheds dim light in a  meter radius.

\textbf{Raincoat}. It is a cape treated to be water repellent, it allows you to stay dry even in the rain.

\textbf{Spyglass}. Objects viewed through a telescope are magnified to twice their size.

\textbf{Pulley and Tackle}. A series of levers connected by a cable and a hook to attach to objects, pulley and tackle allow you to pull up to four times the
weight you can normally lift.

\textbf{Chain}. A chain has 15 Hit Points and hardness 6. It can be broken with a successful DC 24 Strength check.

\textbf{Colony of Scavenger Roaches}: 3 gp, this glass jar contains carnivorous scavenger beetles. Cockroaches must be fed at least 125 grams of meat per day or they die. When released on a dead organism, they devour its flesh in 1d4 days, leaving only the bones. Scavenger beetles eat only dead flesh and cannot harm living creatures. Once released, cockroaches cannot be returned to the jar.

\textbf{Rope}. A rope, whether made of hemp or silk, has 2 Hit Points and can be broken with a successful DC 19 Strength check. The large version has 4 Hit Points, DC 22.

\textbf{Silk Rope} (15 m): 10 gp, this rope has 4 Hit Points and can be broken with a DC 24 Strength check

\textbf{Quiver}. A quiver can hold up to 12 arrows.\index{Quiver}

%\begin{center}
%\includegraphics[width=0.5\linewidth]{immagini/chest.png}
%\end{center}


\textbf{Alchemical Fire}. This sticky fluid ignites when it comes into contact with air. As a two action, you can throw this flask up to 20 feet, smashing it on impact. Make a ranged attack roll against the creature or object, treating alchemical fire as an improvised weapon. On a hit, the target takes 1d6 points of fire damage at the start of each of its rounds. A creature can end this damage by spending two Actions and making a successful DC 12 Dexterity check. If the check succeeds, the flames are extinguished.

\textbf{Healer's Bag}. This kit is a leather bag containing bandages, ointments and splints. The kit can be used ten times. Grants a +2 to first aid checks.

\textbf{Lunch Kit}. 4 gp. This small tin box contains a bowl and simple cutlery. The two sides of the box can be detached, and one side used as a cooking pot and the other as a plate or container

\textbf{Climbing Kit}. 8 gp A climber's kit includes special pitons, boot spikes, gloves, and a harness. You can anchor yourself using the climber's kit as an action; when you do, you cannot fall more than 7 meters from the anchored point, and you cannot climb more than 7 meters away from the anchored point without first undoing the anchor.

\textbf{Lantern}. A Lantern casts bright light in a 10-foot radius and dim light for an additional 20 feet. Once ignited, it burns for 3 hours of real-time play with a flask (0.5 litre) of oil.

%\begin{center}
%\includegraphics[width=0.6\linewidth]{immagini/lantern.png}
%\end{center}

\textbf{Potion of Healing}. This generic healing potion allows you to recover 1d8+1 Hit Points.

\textbf{Potion of Empowered Healing}. This generic healing potion allows you to recover 3d8+3 Hit Points.

\begin{changemargin}{0.3cm}{0.3cm}\begin{narrator} As much as I am personally against the purchase of magical items by characters, Potions of Healing must be available.
\end{narrator}\end{changemargin}


\textbf{Protruding Lens Lantern}. A projecting lens lantern casts light in a 10-foot cone and dim light an additional 30 feet. Once ignited, it burns for 3 hours of in-game real-time with a flask (0.5 litre) of oil.

\textbf{Shieldable Lantern}. A shieldable lantern casts light in a 20-foot radius and dim light for an additional 20 feet. Once ignited, it burns for 1 hour of real game time with a flask (0.5 litre) of oil. As an action, you can lower the shield, reducing the light to dim with a 3-foot radius.

\textbf{Magnifier Lens}. This magnifier allows you to take a closer look at small objects. It is also a useful substitute for flint and steel when starting a fire. Starting a fire with the magnifying glass requires at least as much light as the sun, wood to light, and about 5 minutes for the wood to catch fire. A magnifying glass provides aid (+1d6) on any check made to evaluate or analyze a small or highly detailed object.

\textbf{Hunter's Lens:} 100 gp, this intricate lens is placed over one eye and occupies the eye slot when in use. When used with a ranged attack, you reduce your ranged attack penalties by 1d6. Objects within 30 feet become difficult to see, and you take a -1d6 penalty on sight-based Awareness checks and attack rolls.

\textbf{Handcuffs}. These metal tools can bind a Small or Medium creature. To free yourself from the handcuffs you must pass a DC 24 Dexterity check. To break them you must pass a DC 24 Strength check. Each set of handcuffs comes with a key. Without the key, a creature can use Escape Artist or Disable Device to open the lock with a successful DC 18 check. The manacles have 15 Hit Points and hardness 2

\textbf{Oil}. It is usually bought in a clay ampoule that holds 0.5 litres. As an action, you can sprinkle the oil in this flask on a creature within 3 feet of you or throw it up to 20 feet, shattering it on impact. In either case, make a ranged attack roll against the creature or object, treating the oil as an improvised weapon. On a hit, the target is covered in oil. If the target takes any amount of fire damage before the oil dries (after 1 minute), the target takes an additional 1d6 points of fire damage from the burning oil per round. If ignited, the oil burns for 2 rounds and deals 1d6 points of fire damage to any creature that enters the area or ends its round in it. A creature can only take this damage once per round. You can also pour a cruet of oil on the floor to cover a square of 1 meter on each side, as long as the surface is flat.

\textbf{Tome of Magic}. A 10-page Tome of Magic, meaning it can contain 10 spell levels, costs 100gp. Usually if the enchanter comes from a Magic Academy he can buy it at half the price.\index{Tome of Magic, buy}

\textbf{Crowbar}. Using a crowbar gives +1d6 on Strength checks whenever the crowbar's leverage can be applied.

\textbf{Rations}. Rations consist of dry food suitable for long journeys, and include dried meat, dried fruit, biscuits and nuts.

\textbf{Box with Bait}. This small container contains stone, flint and tinder (usually a dry rag soaked in oil) used to start a fire. Using it to light a torch (or any other easily ignitable object) requires two actions. Lighting any other fire takes 1 minute.

\textbf{Box for Maps or Scrolls}. This cylindrical leather box can hold, rolled up, up to ten pieces of paper or five sheets of parchment.

\textbf{Quiver for crossbow bolts}. This wooden box holds up to 12 crossbow bolts.

\textbf{Lock}. A key is supplied with the lock. Without the key, a creature can pick this lock with a successful DC 17 Disable Device check. The Storyteller may decide that better quality locks are available for more money.

%\begin{center}
%\includegraphics[width=0.6\linewidth]{immagini/lock.png}
%\end{center}

\textbf{Sacred Symbol}. A sacred symbol is the depiction of a Patron. It could be an amulet depicting the symbol of a Patron, the same symbol carefully engraved or intertwined on an emblem or shield, or a tiny box containing a sacred relic.

\textbf{Earplugs} 3 cp, made of cotton or waxed cork, earplugs grant a +2 bonus on saving throws against effects that require hearing but inflict a -4 penalty on hearing-based Awareness checks.

\textbf{Tent}. A simple portable canvas shelter, a tent holds two people. It takes about 20 minutes to pitch a tent.

\textbf{Torch}. A torch burns for \textbf{1 hour of actual game time}, providing light in a 10-foot radius and dim light for an additional 20 feet. If you roll to hit with a lit torch, improvised weapon, and hit, you deal 14d of damage plus 1 additional fire damage. \index{Torch}

\textbf{Hunting Trap}. 12 gp, 2. You use two actions to set this trap, formed of a serrated steel ring, which springs when a creature steps on the metal plate in the center of it. The trap is attached by a heavy chain to an immovable object, such as a tree or spike driven into the ground. A creature that steps on the plate must succeed at a DC 15 Reflex save or take 1d4 points of piercing damage and stop moving. Thereafter, until the creature frees itself from the trap, its movement is limited by the length of the chain (usually 3 feet long). A creature can use 2 actions to succeed on a DC 15 Strength check, and on a successful roll breaks free or another creature within reach. Each failed attempt deals 1 piercing damage to the trapped creature.

\textbf{Caltrops}. As an action, you can scatter a single bag of these tiny caltrops to cover an area 3 feet square on a side. A creature that passes through the covered area must succeed on a DC 15 Reflex save or take 1 piercing damage. Until the creature regains at least 1 hit point, its walking speed is decreased by 10 feet. A creature that passes through the area at half speed need not make a saving throw.

\begin{center}
	\includegraphics[width=0.4\linewidth]{immagini/tribolo.png}
\end{center}

\textbf{Basic Poison}. You can use the poison in this vial to cover one slashing or piercing weapon, or up to three pieces of ammo. Applying the poison requires an action. A creature struck by a poisoned weapon or ammunition must succeed on a DC 12 Fortitude save or take 1d4 points of poison damage.
Once applied, the poison remains effective for 1 minute before drying.


\subsubsection{Equipment}
The starting equipment a character receives at first level depends on the profession and also includes a set of adventuring items. The contents of each bundle are listed below. If the character chooses to purchase her own starting equipment, she can purchase one equipment at the listed price, which is generally cheaper than purchasing the individual items separately.

\textbf{Adventurer's Gear (12 gp)}. Includes a rucksack, crowbar, hammer, 10 rock pitons, 10 torches, flint and steel, 10 daily rations, and a skin bag. The equipment also includes 15 meters of hemp rope tied to the backpack.

\textbf{Hunter's Gear, 24 gp}: contains flint and flint, a belt pouch, an 18m rope, a couch, an oilskin, a wineskin, an iron pot, traveling rations (5 days), torches (10) and a backpack.

\textbf{Diplomat's Set (57 gp)}. Includes a treasure chest, 2 cases for maps and scrolls, a fine dress, a bottle of ink, a quill, a lamp, 2 bottles of oil, 5 sheets of paper, a vial of perfume, sealing wax and soap.

\textbf{Devotee's Gear (30 gp)}: contains flint and steel, a belt pouch, a pouch for spell components, candles (10), 18m rope, a bedstead, an iron pot, a waterskin, travel rations (for 5 days), soap, a wooden holy symbol, a cheap holy text, torches (10), and a backpack.

\textbf{Scout's Gear (15 gp)}. Includes a backpack, a bed, a mess tin, a flint and steel, 10 torches, 10 daily rations and a skin. The equipment also includes 15 meters of hemp rope tied to the backpack.

\textbf{Cave Explorer Gear, 27 gp}: Contains a basic set of tools for exploring ruins and abandoned cities Includes 2 candles, chalk, hammer, and 4 rocker's nails, 20 meters of rope, shieldable lantern with 5 oil flasks, 2 sacks, 2 torches, travel rations (for 3 days)

\textbf{Entertainer's Endowment (60 gp}). Includes a backpack, a bed, 2 costumes, 5 candles, 5 daily rations, a wineskin and camouflage tricks.

\textbf{Burglar's Gear (26 gp)}. Includes a backpack, a bag with 1000 metal balls, 3 meters of string, a bell, 5 candles, a crowbar, a hammer, 10 rock-climbing pitons, a shieldable lantern, 2 oil flasks, 5 daily rations, a flint and flint and a wineskin. The equipment also includes 15 meters of hemp rope tied to the backpack.

\textbf{Scholar's Endowment (40 gp)}. Includes a backpack, study book, ink bottle, stylus, 10 sheets of parchment, sandbag and pocketknife.

\end{multicols}

\subsubsection{Container Capacity}

\begin{tabularx}{0.95\textwidth}{lll}
\textbf{Container}&\textbf{Capacity}&\textbf{Encumber carried}\\
\toprule
Ampoule or Mug&0.5 liters&L\\
Barrel&160 liters of liquid, 4 cubes with edge of 30cm&35\\
Bag&1 cube with 10 cm edge/3 kg of equipment&1\\
Bottle&1 liter of liquid&L\\
Jug or carafe&4 liters of liquid&2\\
Basket&2 cubes with 30 cm edge/20 kg of equipment&5\\
Vial&120ml liquid&L\\
Chest&12 cubes with 30 cm edge/150 kg of equipment&35\\
Wineskin&2 liters of liquid&1\\
Bag&1 cube with 30 cm edge/15 kg of equipment&3\\
Bucket&12 liters of liquid, 1 cube with edge of 25 cm&3\\
Iron Vase&4 liters of liquid&2\\
Backpack*&2 cube with 30 cm edge/30 kg of equipment&6\\
\end{tabularx}

\medskip

\begin{multicols}{2}

\subsubsection{Tools}

The list of tools presented helps characters perform checks related to their professions.

Checks related to professions are usually related to Wisdom.

For example, a "Calligraphy" check is resolved with a Wisdom check, if the character has the appropriate tools at his disposal ("\textit{Calligrapher's Supplies}") he gets a +2 bonus to the check.

If the character has to carry out a check on her profession, this will be done with a bonus equal to half the character's level, if he also has the tools available, he checks a further bonus of +2.

\end{multicols}

\medskip

\begin{tabularx}{0.95\textwidth}{Xll|Xll}
\textbf{Item}&\textbf{Cost}&\textbf{Encumb.}&\textbf{Item}&\textbf{Cost}&\textbf{Encumb.}\\
\toprule
Burglary/Forger's Tools&25 gp&1&Herbalist's Bag&5 gp&1\\
Dice&1 sp&-&Deck of Cards&5 sp&-\\
Dragon Chess&1 gp&1&Three Dragons in the Dark&1 gp&-\\
Poisoner's Substances&50 gp&1&Alchemist's Supplies&50 gp&2\\
Calligrapher's Supplies&10 gp&1&Mealer's Supplies&20 gp&2\\
Cobbler's Tools&5 gp&2&Cartographer's Tools&15 gp&2\\
Leathercrafter's Tools&5 gp&2&Builder's Tools&10 gp&2\\
Blacksmith Tools&20 gp&3&Carpenter Tools&8 gp&2\\
Jeweler's Tools&25 gp&1&Carver's Tools&1 gp&2\\
Inventor's Tools&50 gp&2&Painter's Tools&10 gp&1\\
Blower's Tools&30 gp&2&Weaver's Tools&1 gp&2\\
Potter's Tools&10 gp&2&Cook's Tools&1 gp&2\\
Navigator's Tools&25 gp&2&Shack&2 gp&1\\
Bagpipes&30 gp&1&Horn&3 gp&L\\
Dulcimer&25 gp&2&Flute&2 gp&0L\\
Pan flute&12 gp&L&Lyre&30 gp&L\\
Lute&35f&1&Drum&6f&1\\
Purple&30 gp&1&Camouflage Tricks&25 gp&1\\
\end{tabularx}

\begin{multicols}{2}

\medskip

\subsubsection{Mounts and Vehicles}

A good mount can enable a character to traverse wilderness quickly, but its primary purpose is to carry equipment that would otherwise slow its master.

The table \textit{Mounts and Other Animals} indicates the costs of transport animals. For information on the movement and transport capacity of each animal, see instructions in the chapter \hyperlink{tabella-cavalcature-e-veicoli}{Table Mounts and Vehicles} (page \pageref{tabella-cavalcature-e-veicoli}).


There are other mounts in fantasy worlds than those listed in this section, but these are rare mounts that are not normally available for purchase, such as certain flying mounts (pegasi, griffins, hippogriffs, and other similar animals), or even some aquatic (such as giant seahorses).

Obtaining such a mount often requires stealing an egg and raising the creature yourself, making a pact with a powerful entity, or negotiating with the mount itself.

\textbf{Barding}. A barding is Armour designed to protect an animal's head, neck, chest, and body. Each type of Armour listed in the Armour table in this chapter can be purchased as barding. The cost is four times the equivalent Armour made for humanoids, while the weight is double.

\begin{center}
\includegraphics[width=0.7\linewidth]{immagini/bardatura.png}

\textit{Full harness}
\end{center}


\textbf{Saddle}. A rider can strap into a military saddle to remain at his seat on an active mount during battle. A military saddle grants +1d6 on checks the character makes to remain in the saddle. An exotic saddle is required to ride an aquatic or flying creature.


\begin{center}
\includegraphics[height=0.7\linewidth]{immagini/sella2.png}
\end{center}


\textbf{Rowing Boats}. Barges and rowboats are usually used on lakes and rivers. If a vessel goes with the current, the speed of the current (typically 4.5km per hour) is added to its speed. Generally it is not possible to row against the current if the current has a significant intensity, but it is possible to make these boats go up a waterway by bringing them to the shore and having them towed by one or more beasts of burden. A rowboat weighs 50 kg (Encumbrance 10) should the adventurers have to carry it overland.

\subsubsection{Mounts and Other Animals}


\textbf{Table: Mounts and Other Animals}\index{Table Mounts and Other Animals}

\begin{tabularx}{0.42\textwidth}{ll}
	\toprule
	\textbf{Mount}&\textbf{Cost}\\
	Donkey or Mule&8 gp\\
	Camel&50 gp\\
	Elephant&200 gp\\
	Mastino&25 gp\\
	Galloping Horse&75 gp\\
	War Horse&400 gp\\
	Draft Horse&50 gp\\
	Pony&30 gp\\
\end{tabularx}


\bigskip

\textbf{Harness and Range Vehicles}\\
\begin{tabularx}{0.45\textwidth}{llX}
\toprule
\textbf{Item}&\textbf{Cost}&\textbf{Weight}\\
Harness&x4&x2\\
Biga&250 gp&50 kg\\
Saddlebags&4 gp&4 kg\\
Cart&15 gp&100 kg\\
Wagon&35 gp&200 kg\\
Carriage&100 gp&300 kg\\
Bit and Bridle&2 gp&0,5 kg\\
Nutrition (per day)&5 cp&15 kg\\
\end{tabularx}

\bigskip

\textbf{Saddle}\\
\begin{tabularx}{0.45\textwidth}{llX}
\toprule
\textbf{Item}&\textbf{Cost}&\textbf{Weight}\\
From Load&5 gp&7,5 kg\\
From Gallop&10 mp&12,5 kg\\
Exotic & 60 gp & 20 kg\\
Military&20mo&15kg\\
Sledge&20 gp&150 kg\\
Stabling (per day)&5 but&\\
\end{tabularx}

\bigskip

\textbf{Boats}\\
\begin{tabularx}{0.45\textwidth}{llX}
\toprule
\textbf{Item}&\textbf{Cost}&\textbf{Speed}\\
Rowboat&50 gp&2.25 kph\\
Barge&3000 gp&1.5 km/h\\
Galley&30000 gp&6 kph\\
Sailing ship&10000 gp&3 km/h\\
Warship&25,000 gp&3.75 kph\\
Longship & 10,000 gp & 4.5 km per hour\\
\end{tabularx}


\subsubsection{Expenses}
When not descending into the bowels of the earth, exploring ruins in search of lost treasures, or waging war against the forces of impending darkness, even adventurers must think of the most common needs. Even in a fantastic way, people have to meet basic needs such as food, shelter and clothing. All of this has a cost, even if certain lifestyles cost more than others.

\medskip

\textbf{Lifestyle Expenses}


Lifestyle expenses are an easy way to account for the costs of living in a fantasy world. They cover housing, food, drink, and all other essential needs of a character. These expenses also cover the cost of maintaining your character's equipment, so you'll be ready when your next call to adventure comes. At the start of each week or month (player's choice), each character chooses a lifestyle from the "Lifestyle Expenses" table and pays the price required to maintain that lifestyle. The prices listed are per day, so those wishing to calculate their cost of living for a thirty-day period should multiply the listed price by 30. A character can change his lifestyle from one period to another, based on funds at his disposition, or he can maintain the same lifestyle throughout his career.

Your lifestyle choice can have consequences. A character who maintains a wealthy lifestyle can more easily make connections with the rich and powerful, but runs the risk of attracting some thieves. Similarly, a poor lifestyle may help him avoid criminals, but it is unlikely that he will make important contacts.

\bigskip

\textbf{Lifestyle Expenses}

\medskip

\begin{tabular}{ll}
Lifestyle&Price per Day\\
\toprule
Miserable&-\\
Dingy&1 but\\
Poor&2 but\\
Modest&1 gp\\
Wealthy&2 gp\\
Rich&4 gp\\
Aristocrat&Minimum of 10 gp\\
\end{tabular}

\bigskip

\textbf{Miserable}. The character lives in inhuman conditions. He doesn't have a place he can call home and takes shelter where he can, sneaking into a barn, curling up in an old chest or relying on the good heart of those luckier than him. A miserable lifestyle presents dangers aplenty. Violence, disease, and hunger follow the character wherever he goes. Other wretches might set their sights on his Armour, weapons, and adventuring gear, which represent a fortune by their standards. Most people don't pay much attention to the character.


\begin{center}
	\includegraphics[width=0.8\linewidth]{immagini/mercante.png}
\end{center}


\textbf{Shabby}. The character lives in a drafty barn, a mud-floored hut just outside the village, or a flea-filled hostel in the worst part of town. He enjoys minimal shelter from the elements, but he lives in a desperate and often violent environment, in places plagued by disease, hunger and misfortune. Most people ignore it at all and the law protects it little or nothing. Most people who lead this lifestyle are marked by some terrible misfortune: branded as exiles, suffering from a mental disorder or an illness of some kind.

\textbf{Poor}. A poor lifestyle means having to get by without the comforts available in a stable community. Basic food and lodging, poor quality clothing, and unpredictable living conditions result in a lifestyle that may be sufficient to survive, but certainly not very pleasant. The character sleeps in a hostel or in a common room on the first floor of a tavern. He enjoys a modicum of legal protection, but he still has to contend with acts of violence, crime and disease. Unskilled laborers, junkers, beggars, thieves, mercenaries, and other disreputable figures tend to adopt this lifestyle.

\begin{center}
\includegraphics[width=0.7\linewidth]{immagini/mendicante.png}

\textit{Beggar - Francesco Londonio}
\end{center}

\textbf{Modest}. A modest lifestyle keeps a character out of the slums and allows him to take care of his equipment. The character lives in an old part of town, has a rented room in a boarding house, inn, or temple. He doesn't go hungry or thirsty and lives in a clean, even if spartan environment. Ordinary individuals who lead modest lifestyles include soldiers with families, laborers, students, priests, amateur charmers, and so on.

\textbf{Affluent}. A character capable of adopting a comfortable lifestyle can afford quality clothes and take care of his equipment without difficulty. He lives in a house on a reputable block or has a private room at a quality inn. He associates with merchants, skilled craftsmen and military officers.

\textbf{Rich}. A character who adopts a wealthy lifestyle lives in luxury, even if he may not have achieved the social prestige associated with the old values of nobility and royal blood. He leads a lifestyle comparable to that of a highly successful merchant, a valued servant of a royal house, or an owner of some small business. He lodges in a respectable abode, usually a spacious house in a respectable part of town or a comfortable apartment at a well-known inn. He is probably assisted by a small group of servants.


\begin{center}
\includegraphics[width=0.6\linewidth]{immagini/lucullo.png}

\textit{Lucius Licinius Lucullus. Rome, 117 BC, Naples 56 BC). Roman military and politician}
\end{center}


\textbf{Aristocrat}. The character lives comfortably and in abundance and frequents environments populated by the most powerful figures in the community. He has an excellent abode, perhaps a house in the most elegant part of the city or perhaps a series of rooms in the most renowned inn. He dines at the best restaurants, helps himself to the most skilled and fashionable tailors and can count on various servants who take care of his every need. He receives invitations to the society events of the rich and powerful and spends his evenings in the company of politicians, guild leaders, high priests and nobles. He also has to deal with the deceptions and betrayals perpetrated at the highest levels. The greater his wealth, the greater the chances that he will be drawn into some political intrigue, sometimes as a pawn, sometimes as an active participant.



\subsubsection{Services}


Adventurers can pay nonplayer characters to aid or act on their behalf under any number of circumstances. Most of these cohorts have almost ordinary skills, while others have mastered an art or trade, and some have specialized in some adventuring skill.

Other common cohorts include the many inhabitants of a typical town or city whom adventurers can hire to perform a specific task. For example, a spellcaster might pay a carpenter to build a fine chest (and its miniature replica) to use for a spell.
A warrior might commission a blacksmith to forge a special sword.

\medskip

\textbf{Services}

\bigskip

\begin{tabular}{ll}
Service&Cost\\
\toprule
Carriage within a city&5 cp per 1km\\
Carriage between two towns&1 sp per 1 km\\
Skilled Follower&2 gp per day\\
Inexperienced Wingman&5 sp per day\\
Messenger&5 cp per 1.5km\\
Passage by ship&1 sp for 1.5 km\\
Road toll or entrance fee&5cp/5sp\\
\end{tabular}


\subsubsection{Magical Services}

\textbf{Spell Level x Spell Level ×100 gp}

This is the cost to have a magic-manipulating spellcaster. This cost assumes that you can go to the caster and ask him to manipulate a certain magic to his liking (usually it takes him at least 8 hours to prepare). If you want to take the spellcaster somewhere to use magic you need to negotiate with him, and the basic answer is "no".

\begin{center}
\includegraphics[width=0.8\linewidth]{immagini/riempitivocavalieriapranzo.png}
\end{center}

If the spell a has dangerous consequences, the spellcaster must receive certain evidence that the character has the ability to pay and that he will not fail to do so if these consequences occur (provided he agrees to cast the required spell, which is not sure at all). When it comes to spells that transport the character and caster a distance, the spell must be paid twice even if the character does not wish to travel back with the caster.

Not all villages and towns have a spellcaster capable enough to manipulate magic. As a general rule, you need to travel to at least a small country to be sure enough of finding a spell caster. In a small town one might find a spellcaster capable of casting level 2 spells, in a large town one at level 3, a small town for one at level 5, in a large city for one at level 6, in a metropolis for those of level 8. Not even in a metropolis are you sure to find a spellcaster capable of casting spells with level 9 or higher.

\subsubsection{Special Items and Substances}\index{Special Substances}

\textbf{Antiemetic} 25 gp, this sweet and savory green liquid creates a sense of warmth and comfort. The syrup protects the stomach and makes it more resistant. For 1 hour after drinking it, you gain a +4 bonus on saving throws to resist effects that make you nauseated or against ingested poisons.

\textbf{Antibiotic} (vial) 50 gp. Drinking a vial of this foul-tasting milky-white liquid grants a +4 bonus on saving throws against disease made in the next hour. If already infected, you can make two saving throws to resist the disease that day (without the +4 bonus) and keep the better result. Single dose.

\textbf{Antitoxin} (flask) 50 gp, if you drink the antitoxin, you gain a +4 bonus on all Fortitude saves against poison for 1 hour. Single dose.

\textbf{Staff of Smoke} 20, this alchemically treated wooden staff instantly creates a thick, opaque smoke when ignited. The smoke fills a 10-foot cube (melee range), except that the smoke is dispelled in 1 round by a moderate or stronger wind. The staff is consumed in 1 round, and the smoke then dissipates naturally. All creatures in the affected area have total cover.

\textbf{Alchemist's Caffettone} 1 gp, much loved by young people, it is a brown crystalline powder. Mixed with water, it creates a bitter drink that cures the hangover. Single dose. DC 15 job

\textbf{Bag of Impedance} 50 gp, this round leather bag is filled with molasses, resin, or other goo. When you hurl the pouch at a creature (as a ranged touch attack with a range of 10 feet), the pouch opens, and the substance inside entangles and entangles the victim, becoming tough and springy upon exposure to air.

The substance does not act on Huge or larger creatures. A flying creature isn't stuck to the ground, but it must make a DC 15 Reflex save or lose the ability to fly (if it uses its wings to do so), falling to the ground. The bag of impediment does not function underwater.

\textbf{Bloodstop} 25 gp, this pink, sticky substance helps heal wounds. Using a dose grants a +4 bonus on First Aid checks. 6 uses.

\textbf{Alkaline Flask} 15 gp, this flask of caustic liquids reacts with the natural acids in oozes. You can throw an alkaline flask as a splash weapon with a range of 10 feet. Against non-oil creatures, an alkaline flask works like a flask of acid. Against oozes and other acidic creatures, the alkaline flask deals double the damage indicated by Acid Flask.

\textbf{Smoke Bomb} 25 gp, this small clay sphere contains two alchemical substances separated by a thin barrier. When the sphere is broken, the substances combine and fill a melee area with a cloud of harmless blackish smoke. The smoke bomb functions like a smokestick, but the smoke lingers for 1 round before dissipating. You can throw a smoke bomb as a touch attack with a range of 10 feet.

\textbf{Alchemist's Fire} 20 gp, you can throw a flask of alchemist's fire as a splash weapon. Treat the attack as a ranged touch attack, with a range of 10 feet.

The direct hit deals 1d6 points of fire damage. All creatures within melee range of where the flask fell take 1 point of fire damage as an effect of the splash. In the round following the direct hit, the victim takes an additional 1d6 points of fire damage. The victim can spend 1 Action to attempt to put out the flames before taking this additional damage. A successful DC 15 Reflex save is required to put out the flames. Using 2 Actions gives the character a +2 bonus on saving throws. Diving into water or quenching the flames by magical means automatically extinguishes the flames.

\textbf{Plaster Casting:} 5 but, this dry white powder, mixed with water, thickens within an hour to create a solid material. It can be used to create a cast of a footprint or relief, fill holes or cracks in walls or (if applied to a cloth cover) to secure a broken bone. Hardened plaster has hardness 1 and 5 Hit Points per inch of thickness. A 5-pound pot of plaster can cover a melee range for a depth of 1 inch, create five casts for the forearm or calf of a Medium creature, or two full casts for an arm or leg. Single dose.

\textbf{Liquid Ice} (vial) 40 gp, also called "alchemist's ice", this crystalline blue fluid begins to evaporate as soon as it is removed from the container. For the next 1d6 rounds, you can use it to freeze a liquid or cover an object with a thin layer of ice. You can also throw liquid ice as a splash weapon. A direct hit deals 1d6 points of cold damage, while creatures within melee range take 1 point of cold damage from the splash. The package contains 3 doses.

\textbf{Alchemical Fat} 5 gp, each pot of this blackish substance can cover one Medium or two Small creatures. Covering yourself in alchemical grease grants you a +4 bonus on grapple checks and to escape grapples. The effect lasts for 4 hours or until the grease is washed off.

\textbf{Detect Light} 1 gp, this hand-sized metal plate is covered in a clear, light-sensitive cream. When exposed to light, the cream darkens and becomes opaque depending on how much light is present. Bright light causes it to darken in 1 round, normal light in 3 rounds, dim light in 10 rounds.
The plate is sold wrapped in a heavy cloth to prevent accidental exposure.

\textbf{Advanced Light Detector} 50 gp, this metal plate similar to the Light Detector plate is approximately 50cm*50cm in size. If exposed to light, it imprints on it the image of the surrounding environment within 3 meters.

\textbf{Thunderstone} 30 gp, you can hurl this stone with a ranged attack with a range of 6 meters. When it strikes a hard surface (or is struck hard), it creates a deafening noise that is equivalent to a sonic attack. Creatures within 10 feet must make a DC 15 Fortitude save or be deafened for 1 hour. Single use.

\textbf{Lightning Powder} 50 gp, this silvery powder burns and explodes almost instantly when exposed to fire, rubbing it or throwing it forcefully against a surface (1 action). Creatures within a 10-foot radius are blinded for 1 round (Fortitude DC 13 negates). The package contains 3 doses.

\textbf{Bladeguard} 40 gp, this transparent resin protects a weapon from attacks by Ooze, Rust Mourners, and effects that corrode or melt weapons, making it immune to such attacks for 24 hours. One jar can cover one two-handed weapon, two hand or light weapons, or 50 rounds of ammunition. Applying it requires 2 Actions. The package contains 3 doses.

\textbf{Universal Solvent} (vial) 20 gp, this bubbling purple jelly eats stickers. Each vial can cover one melee range. It destroys ordinary adhesives (such as tar, resin, or glue) in 1 round, but takes 1d4+1 rounds to dissolve more powerful adhesives (bags of impediment, cobwebs, etc.). Has no effect on Magic Stickers.

\textbf{Burning Ember} 1 gp, the alchemical substance at the tip of this small wooden stick ignites when rubbed against a rough surface. Creating a flame with a burning ember is much faster than creating one with flint, flint (or magnifying glass), and tinder. Lighting a torch with a burning ember costs 1 Action (instead of 2 Actions), and lighting any other fire takes at least 3 Actions.

\subsubsection{Alchemical Equipment}

\textbf{Reagent Paper} 1 gp, this piece of paper can help identify liquids. Its color changes depending on traits such as acidity, salinity and magic. Consuming a sheet grants a +2 bonus on Work (alchemy) or Arcane checks to identify potions or other liquids.

\textbf{Exploding Ink} (vial) 40 gp, this alchemically infused ink helps ensure that a secret message is destroyed after it is read. If light hits the ink after it has dried, the chemicals cause it to spontaneously combust within 1 minute
This burning is small: it is not significant enough to ignite anything but paper. Ink used on other materials such as stone or wood simply vanishes, leaving no trace of the writing
One vial of this ink contains enough to write 10 short messages of no more than 50 words each.

\textbf{Luthiers' Oil} 50 gp, this golden oil smells of ancient wood. When applied to the case of a wooden musical instrument, it improves the sound quality. For 1 hour, anyone playing the instrument gains a +2 bonus on the appropriate Perform check.

\textbf{Nightingale Lozenge} 50 gp, this honey-coated candy is made of calming reagents. If eaten, it takes 1 round to begin taking effect, after which it grants a +2 bonus on Perform (sing) checks for 1 hour.

\textbf{Waystones} 50 gp, these small white pebbles are alchemically treated so that they give off a soft glow when activated by rubbing them against each other. The luminescence is dim, just enough to illuminate the stone. The duration is 8 hours.

\textbf{Tracer Powder} 30 gp, when scattered on the ground, this very fine light blue powder reveals the traces of any creature or individual that has passed through the area in the last 48 hours.
The dust also provides a +8 bonus on Survival checks to spot tracks. A single application can cover an area of ​​3 meters. The tracer powder is sold in small leather pouches that hold 10 applications each.

\subsubsection{Alchemical Remedies}\index{Alchemical Remedies}

\label{alchemical-remedies}

\textbf{Carbonated Help} 25 gp, this packet is filled with prickly-edged leaves and has a pungent odor almost strong enough to make your eyes water. While chewing the leaves, one ignores the effects of being fatigued. The leaves last for 10 rounds, after which only a lump of goo is left.
When the effect of the Carbonated Aid wears off, you increase your fatigue level by 1 rank. One package is enough for 1 time only.

\textbf{Anti-Venom Balm} 15 gp, this herbal balm can be applied directly to the skin to prevent the effects of poisons on contact. If a creature touches a poison by touch, but applies the balm to itself within 1 round of the contact, it makes the saving throw twice, taking the higher result. Single use.

\textbf{Coagulant Balm} 5 but, applying this herbal balm on a wound heals 1 damage, it is not possible to apply more than two doses per day on the same patient. The pack is for 3 uses.

\textbf{Bitter Fortifying} 20 gp, this alcoholic liquid generates a pleasant sensation of warmth when ingested. For the next hour, you gain a +2 bonus on saving throws against fear. Using multiple doses in the same 24 hours makes Nauseati for 1 hour. The pack is for 3 uses.


\begin{center}
	\includegraphics[width=0.7\linewidth]{immagini/zaino.png}
\end{center}


\subsubsection{The Standard Backpack\texorpdfstring{\huge{\textregistered}}{\textregistered}} \index{The Standard Backpack}

The Standard Backpack \textregistered \space is a list of objects that I have marked over time by adding everything that I had needed during my adventures.
Take it as a cue to figure out which objects to have behind, don't write them all down otherwise the Arbiter will start seriously looking at the rules of Encumbrance!

These are the contents of the adventurer's backpack: belt, 3 candles, 6 torches, tinder and lighter, 7 dry rations, flask of water, rolled up mattress, tarpaulin, tent, 18-metre rope, net, metal mirror, crowbar , compass, 3 lantern oil, ink, chalk, charcoal, hook, spade, fish hook, rags, wire rope, whistle, 6 potion vials, marble marbles, brass bell, 1kg bag of flour, 3 wedges, 12-metre metal chain, handcuffs, pitons, hammer, pulley, grappling hook.

\end{multicols}

%\vfill

%\begin{center}
%\includegraphics[width=0.6\linewidth]{immagini/carrozza.png}
%\end{center}


\pagebreak

\subsection{Special Materials}\index{Special Materials}

\begin{changemargin}{0.3cm}{0.3cm}\begin{emphasis}{
For this purpose, Captain De Medici had all the Armour browned, to surprise the enemy even in the dark. (The craft of arms, Ermanno Olmi, film 2001)}\end{emphasis}\end{changemargin}\medskip


\begin{multicols}{2}

Armour and weapons can be constructed from materials that possess special innate qualities. If you craft Armour or weapon with more than one special material, you receive the benefits of only the prevalent material. However, you can build a double weapon with each head made of a different special material.

\subsubsection{Living Steel}\index{Living Steel}\index{Table Living Steel}

\label{acciaio-vivente}

\begin{tabularx}{0.45\textwidth}{Xl}
\textbf{Living Steel Item Type} & \textbf{Cost Modifier}\\
\toprule
Ammo & +40 gp per ammo\\
Weapon & +1000 gp\\
Light Armour & +3000 gp\\
Medium Armour & +8000 gp\\
Heavy Armour & +12000 gp\\
Shield & +600 gp\\
Other items & 3000 gp/kg\\
\end{tabularx}

\medskip
A living steel tree is characterized by a particularly hard wood like steel. The origin of these trees remains a mystery to almost everyone. A living steel tree is an ordinary tree planted by a Devotee of Ephrem or Shayalia and given a blessing.

Living steel Armour and shields are formally wooden but have the same characteristics as adamantium. This particular wood is the favorite of those who fight and live for nature. It is not easy to identify a living steel tree for a non-expert and also for this reason it is extremely rare to find it rough, at most it is possible to find weapons or Armour already made.

Living steel has 35 Hit Points per cm of thickness and hardness 15.

\subsubsection{Adamantium}\index{Adamantio}\index{Table Adamantium}

\label{adamantio}

\begin{tabularx}{0.45\textwidth}{Xl}
\textbf{Adamantine Item Type} & \textbf{Cost Modifier}\\
\toprule
Ammunition & +60 gp per ammunition\\
Weapon & +1500 gp\\
Light Armour & +5000 gp\\
Medium Armour & +10000 gp\\
Heavy Armour & +15000 gp\\
Shield & +1000 gp\\
Other items & 5000 gp/kg\\
\end{tabularx}

\medskip
This extremely hard metal is found only in meteorites and contributes to the quality of a weapon or Armour.

Thus adamantium weapons and ammunition have a +1 bonus on attack rolls, and the penalty given by the Armour (Skills and Magic Tests penalty) is decreased by 1 (or one die) compared to a normal Armour of its same kind. Items without metal parts cannot be crafted with adamantium. An arrow may be adamantium, but a quarterstaff is not.

Weapons and Armour normally made of steel and constructed with adamantium have one-third more Hit Points than normal. Adamantium has 40 Hit Points per cm of thickness and hardness 20.

\subsubsection{Alchemical Silver}\index{Alchemical Silver}\index{Table Alchemical Silver}

\label{argento-alchemico}

\begin{tabularx}{0.45\textwidth}{Xl}
\textbf{Alchemical Silver Item Type} & \textbf{Cost Modifier}\\
\toprule
Ammo & +2 gp per ammo\\
Light Weapon & +20 gp\\
Medium Weapon & +90 gp\\
Heavy Weapon & +180 gp\\
Shield & +100 gp\\
\end{tabularx}

The alchemical silvering process can only be applied to metallic weapons and does not work on special metals such as adamantium, cold iron, and mithral.

A complex process involving metallurgy and alchemy can bond silver to a weapon made of steel so that it bypasses the damage reduction of creatures such as lycanthropes.

An alchemical silver weapon retains the hardness and hit points of the original weapon.

\subsubsection{Cold Iron}\index{Cold Iron}

\label{ferro-freddo}

This iron is mined deep underground and is known for its effectiveness against demons and fey. It is forged at a lower temperature to retain its delicate properties. Crafting weapons made of cold iron costs twice as much as their regular counterparts. In addition, any magical enhancements cost an additional 2,000 gp. This increase is applied the first time the item is upgraded, not once per added quality.

Items without metal parts cannot be made of cold iron. An arrow might be made of cold iron but a club is not (unless all metal). A double weapon that is only half cold iron increases its cost by 50\%.

Cold iron has 30 Hit Points per cm of thickness and hardness 10.


\subsubsection{Mithral}\index{Mithral}\index{Table Mithral}

\label{mithral}

\begin{tabularx}{0.45\textwidth}{Xl}
\textbf{Item Type in Mithral} & \textbf{Cost Modifier}\\
\toprule
Light Armour & +1000 gp\\
Medium Armour & +4000 gp\\
Heavy Armour & +9000 gp\\
Shield & +1000 gp\\
Other items & +1000 gp/kg\\
\end{tabularx}

\bigskip

\begin{center}
\includegraphics[width=0.9\linewidth]{immagini/mithral.png}
\end{center}


Mithral is a very rare, shimmering, silver-like metal that is lighter than iron but just as hard. When worked like steel, it makes a wonderful material with which to make Armour, and is occasionally used for other items as well. Most mithral Armour is one tier lighter than normal, and is more accommodating due to movement and other limitations. Heavy Armour is treated as medium Armour, and medium Armour is treated as light, but light Armour remains light.

This decrease does not apply to the proficiency required to don the Armour in question (to wear heavy mithral Armour, you must have Weapon Proficiency 3, although this counts as average for other factors). You must be proficient in the appropriate type of Armour, otherwise you incur the associated penalties as normal.

The chances of failing a spell for mithral Armour and shields decrease by 2 dice (the check remaining common) and the penalty to proficiency checks decrease by 2 (to a minimum of 0), movement penalties decrease by 1 meter .

Mithral has 30 Hit Points per cm of thickness and hardness 15.

\subsubsection{Dragon Skin}\index{Dragon Skin}

\label{pelle-di-drago}

Armour makers can process dragon hides into Armour or shields.
A dragon provides enough scales for a single full set of armor, equivalent to heavy armor, for a creature one size smaller than the dragon, or two medium pieces of armor for a creature two sizes smaller, or 4 light pieces of armor for creatures 3 sizes smaller ' small.

\begin{center}
\includegraphics[width=0.9\linewidth]{immagini/dragonhide.png}
\end{center}


Dragonskin Armour or shield cannot be bought, it is always necessary to bring the raw material to the craftsman who will build the Armour.

In any case, there is always enough hide to make a light or heavy shield in addition to Armour, provided the dragon is Large or greater.
If the dragonskin comes from a dragon that has immunity to an energy type, the Armour is also immune to that energy type, though it confers no protection on the wearer. If the shield or Armour is later given the ability to protect the wearer from a specific energy type, the cost of this enhancement is reduced by 25\%.

A Dragonskin Armor forces you to make a Magic Test without additional dice when casting spells, the Skill penalty decreases by 1 (to a minimum of 0), movement penalties decrease by 1 meter.

Dragonhide armor costs 10 times as much armor of that type, but does not take longer to craft. Magical or medium or heavy type armor must be found.

Dragon skin has 10 Hit Points per cm of thickness and hardness 10. Dragon skin is typically 1 to 1 cm thick.

\end{multicols}

\pagebreak

\section{Break and Enter}\index{Break}\index{Enter}

\begin{changemargin}{0.3cm}{0.3cm}\begin{emphasis}{
In a man's life, sooner or later there comes a day when, to get where he has to go, if there are no doors or windows, he has to break through the wall. (Bernard Malamud)\\

The crime of theft will be punished with the branding of the thieves in the chest. In case of repetition of the crime, first the ears and then two fingers will be cut off. (Twoslad, Citizens' Rights and Duties)}\end{emphasis}\end{changemargin}\medskip

\begin{multicols}{2}

\label{sfondare-ed-entrare}

\lettrine[lines=2, lhang=0.33, loversize=0.25, findent=1.5em]{W}{hen} you try to break an object, there are two choices: hit it with an object (weapon?) or break it with brute force.

\smallskip

\textbf{Size matters...}

Depending on the size of the object this can be more or less easy to hit.\\

\textbf{Table: Size and Defense of Objects - Hitting an Object}\index{Table Size and Defense of Objects - Hitting an Object}

\medskip

\begin{tabularx}{0.43\textwidth}{lll}
\textbf{Cut} & \textbf{Mod. Defense} & \textbf{Dimensions}\\
\toprule
Colossal & -8 &18m+\\
Mammoth & -6 &9-18m\\
Huge & -4 &4-9m\\
Large & -2 &2.4-4m\\
Average & +0 &1.2-2.4m\\
Small & +2 &60-120cm\\
Lowercase & +4 &30-60cm\\
Petite & +6 &15-30cm\\
Very small & +8 &5-20cm\\
\end{tabularx}

\medskip
\textbf{Defense Modifier}

Objects are easier to hit than creatures since they usually don't move, but many are tough enough to ignore damage with each hit. An object's Defense is equal to 10 + its Size modifier (see Table: Hitting an Object) + its Dexterity modifier (if it has one).

If you use 3 Actions to aim, you automatically hit with a melee weapon and gain a +2d6 bonus to hit with a ranged weapon.

\subsection{Hardness}

The following table indicates elements and objects with relative hardness, hit points and DC to break or break through\index{Breaking}\index{Breaking}

When you try to break or break through something with brute force rather than by inflicting damage, you must make a Strength check to see if you succeed.
Since Hardness does not affect the DC to break the item, this value depends more on how the item is constructed than on the material. The DC indicated is for common objects, a 20cm thick glass will not have DC 6 to break.

\end{multicols}


\textbf{Table: Hardness and Hit Points of objects}\index{Table of Hardness and Hit Points of objects}

\begin{tabularx}{0.95\textwidth}{lllll}
\toprule{}
\textbf{Material} & \textbf{Hardness}& \textbf{PF} & \textbf{DC} & \textbf{Example Objects}\\
Paper, Glass, Cloth & 0 & 1 & 3 & Sheets of paper, window glass, light fabric\\
Heavy Cloth & 1 & 4 & 12 & Cloth Armor, Heavy Jacket, Sack, Tent\\
Glass & 1 & 4 & 6 & Glass block, glass table, heavy vase\\
Rope, Leather & 2 & 4 & 19 & Hemp Rope\\
Thin Wood & 3 & 12 & 14 & Simple Door, Chair, Wooden Shield, Club\\
Leather Armor & 4 & 16 & 22 & Leather Armor, saddle, thick hemp rope\\
Thin stone & 4 & 16 & 20 & Blackboard, slate tiles, stone cladding\\
Steel or thin iron &5 & 20 & 23 & Silk rope, steel shield, sword\\
Wood & 5 & 20 & 18 & Chest, sturdy door, table\\
Stone & 7 & 28 & 35 & Paving stone, statue\\
Steel or Iron & 9 & 36 & 26 & Chain, Steel or Iron Armor\\
Wooden structure & 10 & 40 & 20 & Wooden wall, reinforced door\\
Stone structure & 14 & 56 & 35 & Stone wall, reinforced iron door\\
Steel or iron structure & 18 & 90 & 45 & Iron plate wall\\
\end{tabularx}

\begin{multicols}{2}

\subsection{Damaging objects}

\textbf{Energy Attacks}: almost all objects have Damage Resistance towards energy attacks (fire, electricity...), divide the damage by 2 before applying Hardness. Certain types of energy may be particularly effective against certain objects, at the Storyteller's discretion.

For example, fire might deal double damage to scrolls, cloth, and other items that burn easily. Crystal or ceramic objects and creatures may take double damage (vulnerability) against a sonic attack.

Negative or Positive Energy does not damage objects, only living or non-living creatures.

\textbf{Ineffective Weapons}: Certain weapons simply cannot deal damage to certain objects. For example, a bludgeoning weapon cannot cut a rope.
Likewise, it is decidedly difficult to break down a door or stone wall with most melee weapons unless they are specifically designed to do so, such as pickaxes and hammers.

\textbf{Immunity}: Inanimate objects are immune to nonlethal damage and critical hits (but not burst damage). Animated objects, if not treated as creatures, also have these immunities.\index{Immunity to critical objects}

\textbf{Damaged Objects}: A damaged object remains fully functional until its Hit Points reach 0, at which point it is considered destroyed. Damaged objects (but not destroyed ones) can be repaired by a Craftsman Profession and some Spells.

\textbf{Saving Throw}: Unattended nonmagical objects never save. They are considered to have failed their saving throws, and are therefore always subject to the spell and other attacks that allow a saving throw to resist or negate the effect.

An object held by a character (whether held, touched, or worn) succeeds on the saving throw if the character succeeds.

\textbf{Magic Items always have Saving Throws}. The bonus on Fortitude, Reflex, or Will saving throws of a Magic Item is equal to 2 + level x2 of the most powerful spell it contains. If the item does not have an enchantment, it is considered a bonus of +4 for every +1 bonus possessed. Kept (worn) Magic Items make a saving throw only if their owner fails his own. If an effect specifically affects the magical item and not the wearer then only the magical item makes the saving throw.

An \textbf{enchanted object}\index{Damaging enchanted objects} such as a weapon or armor has double the Hardness and Hit Points of a normal object and the DC to break it increases by half.


\textbf{Animated Objects}: Animated objects count as creatures when determining their Defense (they are not considered inanimate objects).


%\begin{center}
%\includegraphics[width=0.45\linewidth]{immagini/portarinforzata2.png}
%
%\textit{Reinforced door}
%\end{center}

\subsubsection{Breaking Objects}\index{Breaking Objects}

\label{rompere-oggetti}

When attempting to break something with brute force rather than dealing damage, you must make a Strength check to see if you succeed.

Since Hardness does not affect the DC to break the item, this value depends more on how the item is constructed than on the material. See the table below for a list of the most common DCs related to breaking objects.

\textbf{Table: Hardness, Hit Points, and Strength DC for Breaking Objects}\index{Table Hardness, Hit Points, and Strength DC for Breaking Objects}

\medskip

\begin{tabular}{llll}
\textbf{Object}& \textbf{Dur.} & \textbf{HP} & \textbf{DC}\\
\toprule
Rope (1" diameter) & 0 & 2 & 23\\
Forcing tied strings & 0 & 2 & 15\\
Handcuffs & 10 & 10 & 22\\
Perfect Handcuffs & 10 & 10 & 24\\
Chain & 10 & 5 & 26\\
Small Box & 5 & 1 & 17\\
Treasure Chest & 5 & 15 & 23\\
Simple wooden door & 5 & 10 & 13\\
Good wooden door & 5 & 15 & 15\\
Sturdy Wooden Door & 5 & 20 & 18\\
Iron Door (5cm thick) & 10 & 60 & 28\\
Reinforced door & 10 &10 & 25\\
Bending Iron Bars & 10 &10 & 24\\
Stone wall (30cm thick) & 8 & 90 & 35\\
Cut stone (90 cm thick) & 8 & 540 & 50\\
\end{tabular}

\medskip

Creatures larger or smaller than Medium have bounty bonuses or penalties on the Strength check to break down a door:

\medskip

\textbf{Table: Strength check modifiers based on your size}\index{Table Strength check modifiers based on your size}

\medskip

\begin{tabular}{ll|ll}
\textbf{Size} & \textbf{Mod.}&\textbf{Size} & \textbf{Mod.}\\
\toprule
Fine & -16 & Large & +4\\
Diminutive & -12 & Huge & +8\\
Tiny & -8 & Large &+12\\
Small & -4 & Gargantuan & +16\\
Medium & +0&&\\
\end{tabular}

\medskip

A \textbf{crowbar}\index{Crowbar} or a \textbf{portable battering ram}\index{Battering ram} increases the character's chance of breaching a door by +1d6.

\end{multicols}

\vfill


\begin{center}
\includegraphics[width=0.47\linewidth]{immagini/kitladro.png}

\textit{Rogue's Kit}
\end{center}

\pagebreak

\section{Environment}\index{Environment}

\label{ambiente}
\begin{changemargin}{0.3cm}{0.3cm}\begin{emphasis}{
Nature isn't cruel, it's just ruthlessly indifferent. This is one of the hardest lessons a human being has to learn. (Richard Dawkins)

\medskip

The main antidote to a bad environment is, of course, to replace it with a good one. (Robert Baden-Powell)}\end{emphasis}\end{changemargin}\medskip


\begin{multicols}{2}

\lettrine[lines=2, lhang=0.33, loversize=0.25, findent=1.5em]{F}{rom} lifeless deserts to trap-filled dungeons, the environment helps define the world, make it alive, dynamic and rich. Allows you to create an exciting and immersive gaming experience.

\subsection{Environmental Rules}

\label{regole-ambientali}

\subsubsection{Vision and Light}\index{Vision}\index{Light}

\label{sec:visione-e-luce}

In a natural environment lighting can take on different gradations and these gradations help to understand how far a creature can see.

The gradations of light can be:
\begin{itemize}[leftmargin=*]
\item
\textbf{Darkness}': Pitch dark, can be natural or magical
\item
\textbf{Dim Light/Slightly Dimmed/Twilight}: dim light, allows you to recognize silhouettes
\item
\textbf{Light}: intense light, a bright, opaque, sunny light
\end{itemize}

It will be the light sources, or their absence, that determine how much light there is and how far away. The Light Sources Table indicates for the most common light sources the fully illuminated beam, the less illuminated one (Dim Light) and the duration.

Many spells and objects use the \textit{real game time} as duration, that is, the rounds or turns are not counted to establish the duration but rather the time the torch, lantern or spell is turned on is marked on the card. Another method can be to set a timer on your smartphone. In this way, management will be easier and greater attention will be paid to consumable resources.


\begin{changemargin}{0.3cm}{0.3cm}\begin{tcolorbox}[title = Note on light sources]
You will have noticed or you will soon, that magical light sources work differently, very often they last much less or generate little light. This is due to the will of a Patron and as such only a Patron can undo the effects (or the Arbiter!).
\end{tcolorbox}\end{changemargin}

\begin{changemargin}{0.3cm}{0.3cm}\begin{narrator}The different functioning of the light sources aims to make exploration darker, darker and more difficult, especially in caves and areas without light sources. No more groups casting Luce every minute. Darkness helps the imagination and raises the level of tension.
\end{narrator}\end{changemargin}

\medskip

\begin{center}
\includegraphics[width=0.8\linewidth]{immagini/oscurita.png}

\textit{Henry Justice Ford}
\end{center}

\bigskip

\textbf{Table: Light sources}\index{Table Light sources}

\medskip

\index{Dim Light}

\begin{tabular}{l|cc|c}
\textbf{Source of} &\multicolumn{2}{c}{\textbf{Radius in meters}}& \textbf{Duration} \\
\textbf{Light}& \textbf{Light} & \textbf{Dim Light} &\\
\toprule
Candela & 1 meter & - & 1 hour\\
Flashlight & 3 meters & 6 meters & 1 hour\\
Lantern & 6 meters & 12 meters & 3 hours \\
\multicolumn{4}{c}{\textbf{spells}}\\
Light & 3 meters & 6 meters &3T \\
Daylight & 6 meters & 12 meters & 1 hour \\
\end{tabular}

Duration is expressed as real time game duration

\medskip

\textbf{Dim Light}\index{Dim Light} is light beyond a light source. It is passing through a 3-metre corridor if it is lit only by dim candles. It's a full moon night.
Generally speaking, a light source creates dim light in a beam twice the normal light beam. A creature has -2 on Awarness check and -1 To Hit Rolls.


\medskip

\textbf{Darkness}\index{Darkness}: It is complete darkness with no light source. For creatures with normal vision, darkness is what lies beyond the dim Light.
The \textbf{character who is blind}\index{Blindess} or who fights in the dark (and cannot see in the dark) has -1d6 to Awareness and all opponents are invisible.

\medskip

The \textbf{Light}\index{Light} is the light outdoors in the sun, but also if you are holding a torch in your hand or in a lantern-lit corridor. In its own small way, even a candle provides light, but only enough to envelop ourselves.

\subsubsection{Types of Vision and Light}

\begin{itemize}[leftmargin=*]
\item
A creature with \textbf{Normal vision} \index{Normal vision}sees up to the distance, as a circular beam around the light source, indicated in Light. Beyond is Dim Light and beyond still Darkness.

\item
A creature with \textbf{Low-light vision} \index{Low-light vision}sees well up to a distance, as a circular beam around the light source, denoted in Dim Light, or denoted by race if minor, beyond it is darkness.

\item
A creature with \textbf{darkvision} \index{darkvision} sees in darkness as if there is Dim Light up to the distance indicated by its darkvision ability.
Darkvision is black and white vision.
\end{itemize}


\subsubsection{Darkness}\index{Darkness}

\label{buio}

Torches and lanterns can be suddenly extinguished by a gust of wind, magical light sources can be dispelled or thwarted, and some magical traps can create areas of impenetrable darkness.

In certain cases, some characters or monsters may be able to see while others are blinded. For the purposes of the rules that follow, a blinded creature is simply a creature that can't see its surroundings.

\subsubsection{Blinded}\index{Blinded}\index{Invisible}

\label{accecato}

Blinded creatures lose their Ability to deal extra damage caused by such as Sneak Strike feat (but not by burst damage or critical on hit).

Blinded creatures move at half speed\index{Moving in darkness}. They must make a DC 12 Acrobatics check per move Action to move at normal speed. If the check fails, they fall prone. Blinded creatures cannot charge.

A blinded creature, or fighting an invisible creature,\index{Invisible} can make an Awareness check at difficulty 20 (or 10+Stealth of the opponent if he doesn't want to be found) to detect the creature as long as it is within twice melee distance from the character.

A blinded creature \index{Blinded} takes a -1d6 penalty on Awareness checks and a -2 penalty on Strength- and Dexterity-based checks, and automatically fails any sight-dependent Awareness check.

In addition, a blinded creature cannot use spells that involve the use of gaze and is immune to spells that involve the use of gaze.

See attack modifier details in \hyperlink{invisibilita}{Invisibility} (page \pageref{invisibility}).

\subsubsection{Falls}\index{Fall}\index{Falling}\hypertarget{cadute}{}

\label{cadute}

Creatures that fall get hurt. Divide the height of the fall (in meters) by 3, round down, the resulting number is the d6 of damage suffered. Eg 16 meters of fall is 16/3=5d6 damage. For falls from great heights it is suggested to apply 3 damage for every 3 meters of fall (1 damage per meter).

Creatures that take damage from a fall land prone.

A successful DC 15 Acrobatics check allows the character to reduce damage by 3 when falling from less than 20 feet.

Falls onto soft surfaces (soft ground, mud, etc.) reduce damage by 3. 

A character can end his move Action with a fall, but only if he has not done himself any damage can he continue with the same Action, otherwise he must first get up from prone.


\begin{center}
\includegraphics[width=0.8\linewidth]{immagini/oggetticadenti.png}

\textit{Henry Justice Ford}
\end{center}


In a round of free fall you fall 150 meters (50d6 or 150 damage), in the first segment you fall 20 meters, then 80m then 150m. A character cannot cast spells while falling unless the fall is 100 meters or more. You are distracted while trying to cast a spell while falling.\index{Casting spells while falling}


\medskip

\textbf{Fall into the Water}\index{Fall into the Water}

Falls into water are handled a little differently. As long as the water has a depth of at least 3 meters and the dive is from a height of within 12 metres, no damage is suffered.

To determine the damage from falling into water, subtract 15 meters from the height of the fall, add 1d6 to damage and for every 3 meters remaining there is an addidional 1d6 damage.

Characters who willingly dive into the water take no damage if they succeed on a DC 15 Swim check and if the water is at least 20 feet deep. The DC of the test increases by 5 every 5 meters above 15 meters in height.

\subsubsection{Effects of Acid}\index{Acid}

\label{effetti-dellacido}

Corrosive acids deal 1d6 points of damage per round of exposure, except in the case of total immersion (such as in a vat of acid), which deals 10d6 points of damage per round. An acid attack, such as from a thrown flask or a monster's spit/breath, counts as one round of exposure.

The vapors produced by most acids are equivalent to inhaled poisons. Those who get very close to a large lump of acid must make a DC 13 Fortitude save or take 1 point of Constitution damage per round. This poison has no frequency, so a creature is safe if it moves away from the acid.

Creatures immune to the caustic properties of acid may still drown if fully immersed in it (see Drowning).

\subsubsection{Effects of Smoke}\index{Smoke}

\label{effetti-del-fumo}

A character forced to breathe thick smoke must make a Fortitude save each round (DC 15, +1 for each previous check) or spend the round coughing and choking. A character who continues to choke for 2 consecutive rounds takes 1d6 points of non-lethal damage per additional round of exposure. The smoke obscures vision, providing Light Cover (+2 Defence) to characters within it.

\subsubsection{Hunger and Thirst}\index{Hungry}\index{Thrist}

\label{fame-e-sete}

Characters may find themselves without water or food and without the means to obtain any. In normal climates, Medium characters need at least 2 liters of fluids and 0.5 kg of decent food per day to stave off hunger, Small characters need half that. In very hot climates, characters may need two or three times that amount of water to avoid dehydration.

Every day without food it is necessary to make a Fortitude save at difficulty 11 +1 per day without food, if you don't have anything to drink the difficulty increases to +3.

On a failed save, you take 1d4 points of damage and become increasingly fatigued. Fatigue penalties remain until you eat and drink enough.

\subsubsection{Falling Objects}\index{Falling Objects}

\label{oggetti-cadenti}

Just as characters take damage from falling more than 3 meter, they also take damage from falling objects.

Objects that fall on characters deal damage based on their weight and the distance they fell.

\textbf{Table: Damage from Falling Objects} determines the amount of damage dealt by an object based on its size. The object is assumed to be made of a dense, heavy material, such as stone.
Objects made of lighter materials might inflict half or less of the listed damage, at the Arbiter's discretion. For example, a Huge boulder hitting a character deals 6d6 points of damage, while a wooden cart might only deal 3d6 points.

Also, if the object falls from less than 3 meter away, it deals half the listed damage. If an object falls more than 20 meters away, it deals double damage. The falling object takes the same amount of damage it deals.

\bigskip

\textbf{Table: Damage from Falling Objects}\index{Table Damage from Falling Objects}

\medskip

\begin{tabular}{ll}
\textbf{Item Size} & \textbf{Damage}\\
\toprule
Tiny or Smaller & 1d6\\
Small & 2d6\\
Medium & 3d6\\
Large & 4d6\\
Huge & 6d6\\
Gargantuan & 8d6\\
Colossal & 10d6\\
\end{tabular}

\bigskip

Dropping an object on a creature requires a ranged touch attack (Touch Attack with Dexterity bonus). These attacks usually have a range of 3 meter. If an object falls on a creature (instead of being thrown), that creature must make a DC 15 Reflex save for half damage if it is aware of the falling object. Falling objects that are part of a trap use the rules for traps instead of those described here.

\subsubsection{Water Hazards}\index{Water Hazards}\index{Water}

\label{pericoli-dellacqua}

Any character can cross relatively calm water that is no deeper than his height without needing to check. Similarly, Swim in Calm Water requires a DC 10 check. Trained swimmers (at least 1 point in Swim) can take 10. Remember that heavy Armour or equipment makes any attempt to swim more difficult. Moving through water is considered difficult terrain.

In the event of faster or more violent water with a successful Swim check (DC 13) or a DC 15 Strength check, the characters are in no danger of going underwater. On a failed save, they take 1d3 points of non-lethal damage per round (1d6 points of lethal damage if the waters flow over rocks and hollows).

Very deep water is not only pitch black, making navigation very dangerous, but it inflicts even worse damage due to water pressure on the order of 1d6 points of damage per minute for every 30 meters that separate you from the surface. A successful Fortitude save (DC 15, +1 for each previous check) indicates that the submerged character takes no damage in that minute. Very cold water deals 1d6 points of non-lethal damage per minute of exposure from hypothermia.

\textbf{Drowning}\index{Drowning}\index{Suffocate}\hypertarget{trattenereilfiato}{}

Any character can hold its breath for a number of rounds equal to 6 rounds by its Constitution score, with a minimum of 3 rounds. For each Action taken, the remaining duration decreases by 1 round, casting a spell with Verbal componente consumes 2 more rounds of air. After this period of time, the character must make a DC 12 Fortitude save each round to continue holding his breath. Each round, DC increases by 2.

Wizards casting spell under water is considered Distracted.

\medskip
\begin{center}
\includegraphics[width=0.8\linewidth]{immagini/affogare.png}

\textit{Henry Justice Ford}\end{center}
\medskip

If the Saving Throw fails, the character immediately drops to 0 Hit Points and passes out. From the next round he begins to lose 1 hit point per round until death (or reanimation!)

You can drown in substances other than water, such as sand, quicksand, very fine dust or a silo full of spelled or simply by holding your breath.

\subsubsection{Perils of the Heat}\index{Heat}

\label{pericoli-del-caldo}

A creature subjected to very high temperatures (above 100 F) must succeed at a Fortitude save every hour (DC 15, +1 for each previous check) or take 1d4 non-lethal damage. If she wears heavy clothing or any type of Armour, she takes a -1d6 penalty on these Saving Throws. A character stacks his Survival proficiency value and can give a bonus to companions equal to half that value on the same Saving Throw. Unconscious characters begin taking lethal damage (1d4 damage per hour).

A character who takes non-lethal damage from exposure to heat suffers from heatstroke and is fatigued. These penalties end when the character regains non-lethal damage taken from the heat.

Infernal heat (air temperature above 60C, fire, boiling water, lava) inflicts lethal damage. Breathing air at this temperature deals 1d6 points of fire damage per minute (no save).

Boiling water deals 1d6 points of scalding damage, unless the character is immersed in it, in which case he takes 10d6 points of damage per round of exposure.


\begin{center}
\includegraphics[height=0.7\linewidth]{immagini/desert.png}
\end{center}

\subsubsection{Catching Fire}\index{Catching Fire}\index{Fire}

\label{prendere-fuoco}

Characters exposed to boiling oil, campfires, or non-instantaneous magical fires may have their clothing, hair, or equipment catch fire. Spells specify whether they are able to start fire.

Characters in danger of catching fire can make a DC 15 Reflex save to avoid this fate. If a character's clothing or hair catches fire, he immediately takes 1d6 points of damage. Each round thereafter, the burning character must make another Reflex save. Failure means he takes another 1d6 points of damage that round. Success indicates that the fire is extinguished (that is, once it succeeds at its Saving Throw, it is no longer on fire).

A character on fire can automatically extinguish the flames by jumping into enough water to put it out. If large quantities of water are not available, rolling on the ground or dampening the flame with cloaks or the like may grant the character another Saving Throw with a +1d6 bonus.

Those unfortunate enough to see their gear or clothing catch fire must succeed at a Reflex save (DC 15) for each item. Flammable objects that fail take the same amount of damage as the character.

\begin{center}
\includegraphics[width=0.8\linewidth]{immagini/fuocopericolo.png}
\end{center}

\medskip

\textbf{Lava Effects}\index{Lava}

Lava or magma deals 2d6 points of damage per round of exposure, except in cases of total immersion (such as when a character falls into the crater of an active volcano), which deals 20d6 points of damage per round (plus any falling damage).

The damage caused by the magma continues for 1d3 rounds after the exposure ends, but this additional damage is only half of that inflicted during the last round of effective contact. An immunity or resistance to fire also serves as resistance or resistance to lava or magma. However, creatures that are immune or resistant to fire may drown if immersed in lava (see Drowning).


\subsubsection{Perils of the Cold}\index{Cold}

\label{pericoli-del-freddo}

Characters ill-dressed in cold climates (below 5 C) must make a Fortitude save every hour (DC 15, +1 for each previous check) or take 1d6 non-lethal damage.
In conditions of extreme cold or exposure below -17 C, a face-up character must make a Fortitude save every 10 minutes (DC 15, +1 for each previous check), taking 1d6 lethal damage for each failed save. Characters wearing winter clothing need to check for cold and exposure only once per hour.

A character adds his Survival proficiency value to Saving Throws and can give companions a bonus equal to half that value on the same Saving Throw.

A character who takes non-lethal damage from cold or exposure is subject to frostbite or hypothermia (treat him as fatigued). These penalties end when the character recovers from non-lethal damage taken from cold and exposure.

Intolerably cold or exposed conditions (below -28C) inflict 1d6 lethal damage per minute on characters (with no Saving Throw) unless specifically protected.


\subsubsection{Effects of Ice}\index{Ice}

Characters walking on ice appear to be walking on difficult terrain. Movement is halved, any Acrobatics checks have a +5 difficulty increase. Characters who are in contact with ice for a long time may take extreme cold damage.

\begin{center}
\includegraphics[height=0.6\linewidth]{immagini/snowfall.png}
\end{center}

\subsubsection{Slow Choke}\index{Suffocating}

A Medium-sized character can breathe easily for approximately 6 hours in a sealed chamber measuring 3 meter on a side. After this time, she takes 1d6 points of non-lethal damage every 15 minutes. Each additional Medium-size character or each significant fire (a torch, for example) proportionally reduces the duration of breathing air. Once knocked unconscious from the accumulation of non-lethal damage, players begin taking lethal damage at the same rate. Small-sized characters consume half as much air as Medium-sized characters.

\subsection{Weather - Weather}\index{Weather}

\label{tempo-atmosferico---meteo}

Sometimes the weather can play a big part in an adventure. The Table: Random Weather is a generic table that can be used to determine local weather conditions. The terms in the table are defined below:

\end{multicols}

\medskip

\textbf{Table: Random Weather}\index{Table Random Weather}

\medskip

\begin{tabularx}{0.95\textwidth}{llXXX}
\textbf{d\%} & \textbf{Weather} & \textbf{Cold Weather}& \textbf{Temperate Weather{*}} & \textbf{Desert}\\
\toprule
01-70 & Normal& Cold, calm & Normal for the season{*}{*} & Hot, calm\\
71-80 & Abnormal & Heatwave (01-30) - Coldwave (31-100)&Heatwave (01-50) - Coldwave (51-100)& Hot,breezy \\
81-90 & Inclement & Precipitation (snow)& Precipitation (normal for the season)& Hot, breezy \\
91-99 & Storm & Snowstorm & Lightning Storm - Snowstorm & Dust Storm \\
100& Severe Storm& Blizzard & Blizzard, Blizzard, Hurricane, Tornado & Downpour\\
\end{tabularx}

\medskip

* Temperate includes warm forests, hills, swamps, mountains, plains, and marine areas.

** Winter is cold, summer is hot, autumn and spring are moderate. Swamps are always slightly warmer in winter.

\begin{multicols}{2}

\textbf{Downstorm}: Treat it as rain (see Precipitation below), but it offers cover like fog. It can cause flooding and usually lasts 2d4 hours.

\textbf{Hot}: The temperature is between 15 and 30 C during the day, and between 6 and 11 degrees less at night.

\textbf{Calm}: Light wind (between 0 and 15 km/h).

\textbf{Cold}: Temperature between -17 and 5 C during the day, and between 6 and 11 degrees less at night.

\textbf{Moderate}: Temperature between 5 and 15 C during the day, and between 6 and 11 degrees less at night.

\textbf{Heat Wave}: Raises the temperature by 6C.

\textbf{Cold Snap}: Lowers the temperature by 6C.

\textbf{Precipitation}: Roll a d100 to determine whether the precipitation is fog (01-30), rain/snow (31-90), or sleet/hail (91-00). Snow and sleet only occur when the temperature is 0C or lower. Most precipitation lasts 2d4 hours. Hail, however, lasts only 3d6 minutes but is usually accompanied by 1d4 hours of rain.

\textbf{Storm} (of Lightning/Snow/Dust): The wind is very strong (45 to 75 km/h) and visibility is reduced by three quarters. The storms last 2d4-1 hours. See Storms, below, for more details.

\textbf{Storm} (Storm/Blizzard/Hurricane/Tornado): Wind speed is greater than 75 km/h (see Table: Wind Effects). In addition, blizzards are accompanied by heavy snow (1d3 \texttimes{} 30 cm), and hurricanes are accompanied by showers. Blizzards last 1d6 hours, blizzards 1d3 days. Hurricanes can last up to a week, but the greatest impact to characters will occur between 24 and 48 hours as the center of the storm moves into their area. Tornadoes are very short-lived (1d6 \texttimes{} 10 minutes), usually forming as part of a lightning storm.

\textbf{Hot}: Temperature between 30 and 43 C during the day and between 6 and 11 degrees less at night.

\textbf{Breezy}: Wind speed is moderate to strong (15 to 45 km/h); see Table: Wind Effects.

\textbf{Rain, Snow, Sleet and Hail}

The bad season frequently slows down or blocks land transport and makes navigation practically impossible. Torrential downpours and blizzards obscure vision as much as thick fog would.

Most precipitation occurs as rain, but in cold climates it can manifest as snow, sleet, or hail. Precipitation of any type, followed by a drop in temperature from above 0C to below 0C can produce ice.

\begin{center}
\includegraphics[width=0.9\linewidth]{immagini/Paesaggio-pioggia-Auvers.png}

\textit{Vincent van Gogh, Landscape in the rain in Auvers, 1890, oil on canvas, 50 x 100 cm}
\end{center}


\textbf{Heavy rain}\index{Heavy rain}

Rain halves visibility, and imposes a -1d6 penalty on Wisdom checks. It has the same effect as a very strong wind on flames, ranged weapon attacks, and Wisdom checks as a very strong wind.

\textbf{Snow}\index{Snow}

As it falls, the snow has the same effects as rain on visibility, ranged weapon attacks, and Wisdom checks, and the terrain is considered difficult. A snowfall lasting one day leaves 3d6*2.5 cm of snow on the ground.

\textbf{Thick Snow}

A heavy snowfall has the same effects as a normal snowfall, but obscures visibility like fog (see Fog). A day of heavy snow leaves 2d4 x 30cm of snow on the ground, and the terrain is considered doubly difficult (move/4). A heavy snowfall accompanied by high or very high winds can cause snowdrifts 1d4 x 1 meter deep, especially on and around objects large enough to deflect the wind (a cabin or large tent, for example).
There is a 10\% chance that a heavy snowfall will be accompanied by lightning (see Lightning Storm). Snow has the same effect as moderate wind on flames.

\textbf{Sleet}

This is basically frozen rain, which has the same effects as rain when it falls (except that the chance to extinguish protected flames is 75\%) and those of snow once it settles.

\textbf{Hail}

Hail does not reduce visibility, but the sound of falling hail makes hearing-based Awareness checks more difficult (–1d6 penalty). Sometimes (5\% chance) the hail can be so large that it deals 1 lethal damage (per storm) to anything in the open. Once deposited, hail has the same effect as snow on movement.

\subsubsection{Storms}\index{Storms}

\label{tempeste}

The combined effects of precipitation (or dust) and wind, which accompany all storms, reduce visibility by three-quarters, imposing a –8 penalty on all Wisdom checks. The storms make ranged weapon attacks impossible, except with siege weapons, which take a –1d6 penalty on attack rolls.
Automatically extinguish candles, torches or similar unprotected flames. Shielded flames, such as those from lanterns, are flailed violently and have a 50\% chance of extinguishing. See Table: Effects of Wind for possible effects on creatures caught outside without cover.

Storms are of three types.

\textbf{Dust Storm (Challenge Rank 3)}

these desert storms differ from other storms in that they have no precipitation. Conversely, dust storms carry grains of sand that obscure vision, smother unprotected flames, and can even extinguish protected ones (50\% chance). Most dust storms are accompanied by very high winds and leave behind a deposit of 1d6 \texttimes{} one cm of sand.
There is also a 10\% chance of encountering large dust storms with wind blasts (see Table: Wind Effects). These violent dust storms inflict 1d3 non-lethal damage per round to anyone caught outdoors without cover and also pose a risk of suffocation (see Drowning, except that a character with a scarf or similar protection over his mouth and nose does not initiate choke until after a number of rounds equal to 10 \texttimes{} his Constitution score). Large dust storms settle behind (2d3-1) x 30cm of sand.

\textbf{Snow Storm}

in addition to the winds and precipitation common to other storms, snowstorms deposit 1d6 \texttimes{} one cm of snow on the ground.

\textbf{Lightning Storm}

in addition to winds and precipitation (usually rain, but sometimes hail), lightning storms are accompanied by electrical discharges that pose a danger to characters outdoors without shelter (especially if they are wearing metal Armour). As a general rule, consider one lightning strike per minute over a one-hour period in the heart of the storm. Each bolt deals between 4d8 and 10d8 electricity damage. One in ten lightning storms is accompanied by a tornado.

\textbf{Violent Storms}

Very high winds and torrential rainfall reduce visibility to zero, and make it impossible to make Awareness checks and make ranged weapon attacks. Unprotected flames are automatically extinguished, and there is a 75\% chance of doing so for protected flames as well. Creatures caught in these areas must make a Fortitude save or face effects based on their size (see Table: Wind Effects). Severe storms are divided into the following four types.

\begin{center}
\includegraphics[width=0.95\linewidth]{immagini/Vincent_van_Gogh_tempesta.png}

\textit{Vincent van Gogh, Wheatfield under a stormy sky (Auvers-sur-Oise, July 1890)}

\end{center}


\textbf{Blizzard}: Although they have little or no precipitation, blizzards can cause massive damage due to the force of the wind.

\textbf{Blizzard}: The combination of strong winds, thick snow (usually 1d3 \texttimes{} 30cm), and severe cold make blizzards deadly to anyone unprepared for them.

\textbf{Hurricane}: In addition to very high winds and heavy rain, hurricanes are followed by floods. Many activities in an adventure are impossible under these conditions.

\textbf{Tornado}: In addition to very high winds, tornadoes can seriously injure and kill those caught within them.

\subsubsection{Fog}\index{Fog}

\label{nebbia}

Whether in the form of a low-lying cloud or a mist that rises from the ground, the fog obstructs your vision beyond 3 meter. Creatures farther than 3 meter have Light Cover (+2 Defence).

Fog makes the terrain difficult.

The fog could also be very thick in which case creatures further than 3 meters have medium Cover (+4 to Defence) and those within 1 meter still have light cover.

\subsubsection{Wind}\index{Wind}

\label{venti}

Winds can whirl sand or dust, fan large fires, capsize small boats, and dispel gases or vapors. If they are strong enough, they can even knock characters to the ground (see Table: Wind Effects), interfere with ranged attacks, or impose penalties on some Proficiency Checks.

\medskip

\textbf{Table: Wind Effects Wind Strength}\index{Table Wind Effects Wind Strength}

\medskip

\begin{tabular}{lll}
\textbf{Intensity} & \textbf{Speed} & \textbf{Range Attacks} \\
\toprule
Light & 0-15km &\\
Moderate & 16.5-30 km/h & \\
Strong & 31.5-45 & -2 \\
Very strong & 45.5-75km/h & -4 \\
Blizzard & 76.5-111km/h & Impossible \\
Hurricane & 12-261km/h & Impossible \\
Tornado & 262-450km/h & Impossible\\
\end{tabular}

\bigskip

\textbf{Light Wind}

A gentle breeze, which has no practical effect on gameplay.

\textbf{Moderate Wind}

A sustained wind, which has a 50\% chance of extinguishing any small unprotected flame, such as that of a candle.

\textbf{Strong wind:} Gusts that automatically extinguish unprotected flames (candles, torches, and the like). These flurries impose a –2 penalty on ranged attack rolls and Awareness checks.

\textbf{Very Strong Wind}

In addition to automatically extinguishing unprotected flames, winds of this intensity violently agitate protected flames (such as those of a lantern) and have a 50\% chance of extinguishing them. Ranged weapon attacks and Wisdom checks take a –1d6 penalty.

\textbf{Storm}\index{Storm}

Strong enough to knock down branches or even entire trees, blizzards automatically extinguish unprotected flames and have a 75\% chance to extinguish protected flames, such as lantern flames. Attacks with ranged weapons are impossible, and siege weapons also take a -1d6 penalty on attack rolls. Awareness checks based on hearing take a –8 penalty for howling wind.

\textbf{Hurricane}\index{Hurricane}

Extinguish all flames. Ranged attacks are impossible (except with siege weapons which take a -8 penalty on attack rolls). Awareness checks based on hearing are also impossible, and all characters can hear is the howl of the wind. Hurricanes are often able to bring down trees.

\textbf{Tornado (Challenge Rank 10)}\index{Tornado}

Extinguish all flames. All ranged attacks are impossible (including those with siege weapons), as are hearing-based Awareness checks. Instead of being blown away (see Table: Wind Effects), characters in close proximity to a tornado and who fail a Fortitude save are sucked into the tornado.

Those who come in contact with the conical cloud are lifted off the ground and tossed about for 1d10 rounds, taking 6d6 points of damage per round, before being violently ejected (with falling damage applied).

Although a tornado's rotational speed can reach 450 km/h, the cone itself moves forward at an average of 45 km/h (about 75 meters for each round). A tornado is capable of uprooting trees, destroying buildings, and causing other forms of similar devastation.

\end{multicols}

\vfill

\begin{center}
\includegraphics[keepaspectratio,width=0.7\textwidth]{immagini/blizzard.png}

\end{center}

\pagebreak

\section{Water Adventures}\index{Water Adventures}

\label{avventure-in-acqua}
\begin{changemargin}{0.3cm}{0.3cm}\begin{emphasis}{
He looked at the sea and realized how alone he was now. (The Old Man and the Sea, Ernest Hemingway)}\end{emphasis}\end{changemargin}\medskip


\begin{multicols}{2}

\lettrine[lines=2, lhang=0.33, loversize=0.25, findent=1.5em]{W}{ater} allows societies to exist, but it can also destroy them. Life could not exist without it. Trade and travel are facilitated by its presence. Yet, water can also kill, both by drowning people and by generating large-scale floods and tsunamis. Life on earth is dependent on water but at the same time fears it.

\textbf{Water Adventures}

A water adventure can take place anywhere that water is the main feature of the land: such as swamps, rivers, lakes, ponds, oceans, the Plane of Water, and the like. Aquatic adventures, however, do not require characters to have the ability to breathe underwater; the introduction of Aquatic challenges for low-level adventurers bring a lot of tension and danger to an adventure.

\textbf{Adapting to Aquatic Environments}

The rules for underwater combat apply to creatures that aren't native to this dangerous environment, like most PCs. For extended Aquatic adventures and particularly deep explorations, characters will require the use of magic to continue their adventures. Transformation or Abjuration spells are of obvious use.

Pressure damage can be totally avoided by spells that offer resistance.

\subsection{Underwater combat}\label{underwatercombat}\index{Underwater combat}
Creatures that live on land have considerable difficulty fighting underwater. Water affects a creature's defense, attack rolls, damage, and movement.

\begin{itemize}[leftmargin=*]
	\item
	A creature underwater loses its Dexterity bonus to Defense.
	\item
	A creature underwater that is not under the \emph{Freedom of Movement} spell makes attack rolls with a -1d6 and the opponent is considered to have damage resistance against slashing and bludgeoning weapons.
	
	Weapons such as Trident, Short Spear, Short Sword, Javelin have no penalty to hitting underwater in melee.
	\item
	Moving or swimming in water is considered \textbf{\emph{terrain} difficult}.
\end{itemize}

These penalties are only valid if you do not have a Swim type move.

\subsubsection{Underwater ranged attacks}\index{Underwater attacks}
Thrown weapons are ineffective underwater, even when thrown from land. Ranged weapon attacks take -1 damage for every 2 meters of water they pass through.

\subsubsection{Attacks from the mainland}
Those characters who swim, float, or wade through surface water, or wade through water that is at least chest deep, enjoy medium coverage.

A fully submerged creature has complete cover against opponents on land.

\subsubsection{Magical effects in water}
Magical effects are unaffected, except those that require an attack roll (see above) and fire effects.

Nonmagical \textbf{fire} (including alchemist's fire) does not burn underwater. Fire spells or magical effects are ineffective underwater. A partially submerged creature has fire resistance.

\textbf{Casting spell underwater}\index{Casting spell underwater}

Casting spells while underwater can be difficult for those without the ability to breathe underwater.

A creature that is unable to breathe underwater expends three rounds of holding its breath to cast a spell underwater.

Some spells may work differently underwater, at the Arbiter's discretion

Remember that a character can hold his breath for CON*6, minimum 3 rounds, rounds before starting to drown and each Action consumes an additional round.

\subsection{Drowning}\index{Drowning}\index{Suffocate}\label{drowning}

A character drowns when he is no longer able to hold his breath underwater, or when the Swim check, including penalties, has a value of less than 5.
The Swim check has difficulties based on the state of the water and the type of liquid which at least are considered "difficult terrain", in which you move. In case of calm waters the DC is 10, troubled waters DC 15, stormy waters DC 20. In case of failure of the check you do not move and you have a -1 to the next check, in case of critical failure the next check takes a -1d6.
A drowning character must make a DC 12 Fortitude save each round to hold his breath and each subsequent round the check increases by 1, each Action taken increases the next difficulty by 1. When the check fails the character goes to 0 Points Wounded, she faints, and from the next round she loses 1 hit point per round.


\subsection{Nautical Adventures}

Water can provide the setting for a different and unique gaming experience: nautical adventure. In such a scenario, the effects and perils of underwater adventures are replaced by surface challenges, as characters and their opponents use ships and boats to navigate that environment. Usually, nautical adventures are resolved normally, with a combat aboard a ship similar to a terrestrial one. If combat occurs during a storm or in rough seas, treat the ship's deck as hindering terrain. Remember to consider the effects of weather or roll on Concentration checks.

\begin{center}

\includegraphics[width=0.8\linewidth]{immagini/avventure_acqua_grey.png}

\textit{The Mermaid and the Boy" 1904 by H.J.Ford}
\end{center}


\subsubsection{Fast Sea Combat}

When ships do battle, things change a bit. The following rules are not intended to accurately simulate all aspects of a naval combat, but only to provide you with quick and simple rules for unraveling such situations when they escalate into a nautical adventure, whether it be a battle between two ships or between a ship and a ship. a sea monster.

{Preparation:} Determine which types of ships are involved in the combat (see Table: Ship Statistics). Use a large, empty battle grid to represent the waters where the battle takes place. A single square corresponds to 10 meters of distance. Depict each ship by placing tokens that occupy the appropriate number of squares (toy ships make great tokens and can be found at model stores).

{Starting Combat:} When combat begins, let the characters (and key NPC allies) roll Initiative normally; the ship moves and attacks based on the captain's initiative result. If one of the ships in the battle uses sail to navigate, randomly determine which way the wind is blowing by rolling 1d8 and following the guidelines for Missing Splash Weapons.

{Movement:} Based on the captain's Initiative score, the ship can move at its base speed in a single round as if the Action matched that of the captain himself (or at double its speed as the only action of the round), as long as it has its own minimum full crew. The ship can increase or decrease its speed by 10 meters per round, up to maximum speed. Alternatively, the captain can change direction (maximum one side of a square at a time) (2 Actions). A ship can only change direction at the start of the round.

{Attacks:} Those in excess of a ship's minimum manning requirement may be placed to man Siege Engines. Siege Engines attack based on the Captain's Initiative score.


\begin{center}
\includegraphics[width=0.9\linewidth]{immagini/acquapericoli.png}
\end{center}

A ship may also attempt to ram a target if it is carrying the minimum crew. To ram a target, the ship must move at least 10 meters and end up with its bow in a square adjacent to it.
Then, the ship's captain makes a Profession (sailor) check: if the result equals or exceeds the target's Defence, the ship strikes its target, inflicting damage as indicated on Table: Ship Statistics and at the same time suffering the minimal damage. A ship equipped with a rake deals an additional 3d6 points of damage to the target (the attacking vessel takes no additional damage).

\textbf{Sinking}\index{Sinking}

A ship gains the sinking condition when its Hit Points drop to 0 or less. A sinking ship cannot move or attack and is considered sunk after 10 rounds. For every 25 points of damage taken from a sinking ship, the sinking is reduced by 1 round. The Fabricate spell allows you to repair a sinking ship if its Hit Points are restored above 0, in which case the ship loses its sinking condition. Generally, nonmagical repairs take too long to save a ship from sinking once it begins to sink.

\textbf{Ship Stats}

In the real world, there is a huge variety of boats and ships, from small rafts to massive galleons. Representing this, Table: Ship Statistics classifies seven standard ship sizes and their respective statistics. Just as real-world cultures have created and adapted different types of vessels, so races in fantasy worlds might create their own bizarre ships.
Arbiters may use or modify these statistics to suit the needs of their creations, and still describe such means of transportation as they see fit. All ships have the following traits.

\textbf{Type}: This is a general category listing the basic type of ship.

\textbf{Defence}: The Defence of the ship. To calculate a ship's effective Defence, add the captain's Profession (sailor) score to the base Defence of the ship. Touch attacks against a ship ignore the captain's modifier. A ship is never flat-footed.

\textbf{Basic Save}: A ship's basic Saving Throw modifiers (Fortitude, Reflexes, and Wisdom) have the same value. To determine a ship's effective save modifiers, add the captain's Profession (sailor) modifier to this value.

\textbf{Max Speed}: The maximum speed of a ship in combat. An asterisk indicates that the ship has sails and can move twice as fast if it moves in the same direction as the wind. A ship with only sails can move only in the presence of wind.

\textbf{Armaments}: The number of Siege Engines that can be equipped on the ship. A rake uses one of these slots, and a ship can only be equipped with a rake.

\begin{center}
\includegraphics[width=0.8\linewidth]{immagini/navenotte.png}
\end{center}


\textbf{Ramming}: The amount of damage a ship deals with a successful ramming attack (without a ramming).

\textbf{Squares}: The number of squares the ship occupies on the combat grid.

\textbf{Crew}: The first number indicates the minimum crew the ship needs to operate normally, excluding gun handlers. The second indicates the maximum number of crew plus additional soldiers or passengers. A ship without its minimum crew can only move, change speed, change direction, or ram if its captain succeeds at a DC 20 Profession (sailor) check.
A crew exceeding the minimum number does not affect movement, but crew members can replace fallen members or man additional weapons.

\bigskip

\end{multicols}

\textbf{Table: Ship Statistics}\index{Table Ship Statistics}

\medskip

\begin{tabular}{lllllllll}
\textbf{Type} & \textbf{Defence} & \textbf{HP} & \textbf{Base ST} & \textbf{Sp. (m/s)} & \textbf{Weapon} & \textbf{Spur} & \textbf{Quad}. & \textbf{Crew}\\
\toprule
Raft & 9 & 10& +0& 4.5 & 0 & 1d6 & 2 & 1/4\\
Rowboat & 9& 20& +2& 9 & 0 & 2d6+6 & 6 & 1/3\\
Boat & 8& 60& +4& 9 & 1 & 2d6+6 & 12 & 4/15+100\\
Longship& 6& 75& +5& 18 & 1 & 4d6+18 & 40 & 50/75+100\\
Sailboat & 6& 125 & +6& 18 & 2 & 3d6+12 & 20 & 20/50+120\\
Warship & 2& 175 & +7& 18 & 3 & 3d6+12 & 35 & 60/80+160\\
Galley& 2& 200 & +8& 27 & +4 & 6d6+24 & 60 & 200/250+200\\
\end{tabular}

\pagebreak

\section{City Adventures}\index{City}

\label{avventure-in-citta}
\begin{changemargin}{0.3cm}{0.3cm}\begin{emphasis}{
God created the countryside, and man created the city. (William Cowper)}\end{emphasis}\end{changemargin}\medskip


\begin{multicols}{2}

\lettrine[lines=2, lhang=0.33, loversize=0.25, findent=1.5em]{A}{t} first view, a city is very similar to a dungeon, as it is made up of walls, doors, rooms and corridors . Adventures set in cities differ from those set in dungeons in two main ways. Characters have access to more resources and must be mindful of the presence of law enforcement.

\textbf{Access to Resources}: Unlike dungeons and wilds, characters can buy and sell Equipment very quickly in the city. A large city or metropolis likely has NPCs and high-level experts specializing in the more obscure fields of knowledge who can offer help and interpret clues. And when characters are battered and bruised, they can always return to the comfort of their rooms at the inn.

The freedom to retreat and access market goods means players have more control over the pace of a city adventure.

\textbf{Enforcement}: The other element of distinction between adventuring in a city and exploring a dungeon is that the dungeon is, almost by definition, a place without rules where the only law is that of the jungle: kill or be killed.

A city, on the other hand, is governed by a code of laws, many of which are explicitly designed to prevent the kind of behavior adventurers indulge in more often than not: killing and looting. However, city laws recognize the serious threat monsters pose to city stability, and the prohibition of killing rarely applies to monsters such as aberrations or Fiends.

Most evil humanoids, however, usually enjoy the same protection as all other citizens. Having a set of evil Traits is not a crime (except perhaps in cities where there is a strict theocracy, with magical power necessary to enforce the law); only evil deeds are considered a breach of the law.

\begin{center}
\includegraphics[width=0.8\linewidth]{immagini/cavalieri.png}
\end{center}


Even when the adventurers meet an evildoer committed to committing the most horrific crimes against the city population, the law still frowned upon those who take justice into their own hands by killing the evildoer or otherwise preventing him from being brought before a court to be processed.

\textbf{Weapon and Spell Limitations}

Each city has its own laws regarding weapons that can be carried in public and restrictions on spells.

City laws may not affect all characters equally. A man of faith moving with a weapon in tow is not hindered in any way by the law of lassoing weapons, but a spellcaster suffers a considerable reduction in his power if his Tome is confiscated at his hands. city gates.

\textbf{Urban Elements}

Walls, doors, dim lighting, and uneven ground—in many ways, a city is similar to a dungeon. New elements suitable for a city setting are described below.

\textbf{Walls and Gates}

Many cities are defended by a circle of walls. Normal city walls are made of reinforced stone, 1 meter thick and 6 meters high. Such a wall is fairly smooth and requires a DC 30 Climb check to climb it. The walls have small crenellations on one side to provide a parapet for the guards at the top, and the walking space on the walls is barely enough for a guard.

\textbf{The Walls}

Unlike smaller cities, metropolises often also have internal walls, sometimes the old walls erected when the city was smaller, or walls that separate the various districts from each other. Sometimes these walls are as high and wide as the outer ones, but much more often they are the size of a city large or small.

\textbf{Watchtowers}: Some city walls have watchtowers that pop up at regular intervals. Few cities have enough guards to place on each watchtower, unless the city expects an attack from outside. The towers offer an elevated view of the surrounding countryside as well as a bastion of Defence against enemy invaders.

\medskip

\begin{center}
\includegraphics[width=0.85\linewidth]{immagini/muraparigi.png}

\textit{Chronicles, Jean Froissart. Queen Isabella of France arrives in Paris, 15th century}
\end{center}

\medskip


Watchtowers are usually 3 meter taller than the wall they are part of, and their diameter is 5 times the thickness of the walls. Loopholes for archers open on the upper floors of the tower, and the top is crenellated in the same way as the surrounding walls. In the smaller towers (about 7.5 meters in diameter, along a 1.5 meter thick wall) a simple ladder connects the inside of the tower to the roof. In the larger towers there are real stairs.

Access to the tower is protected by heavy wooden doors, with iron reinforcements and good locks (Disable Devices DC 25). Normally the captain of the guard keeps the access key to the tower, and a second copy is kept in the inner fortress or in the city barracks.

\textbf{Gates}: A typical city gate is composed of a gatehouse with two portcullis and loopholes in the space between them. In towns and small towns, the main entrance is protected by double iron doors set into the city walls.

The gates usually remain open during the day and locked or barred at night. Generally, only one gate lets travelers in after dark, and it is guarded by guards who will only open the gates for someone who looks honest, presents the proper papers, or bribes them sufficiently (depending on the type of city and guards).

\textbf{Guards and Soldiers}

A city typically has full-time service military personnel equal to 1\% of its adult population, in addition to duty or conscript soldiers equal to 5\% of the population. Full-time soldiers are city guards responsible for maintaining order in the city, in a role similar to that of modern police, and (to a much lesser extent) defending the city from outside assaults. Conscripted soldiers are called up in the event of an attack on the city.

A typical array of city guards deploys into three eight-hour duty shifts, with 30\% of its force on daytime duty (8am to 4pm), 35\% on evening duty (from 16 to 24) and 35\% of service in the night shift (from 24 to 8). At any one time, 80\% of the guards on duty are patrolling the streets, while the remaining 20\% are assigned to various posts around the city, ready to react to any alarms. A similar guard post is present at least in every city neighborhood (a neighborhood is made up of several neighborhoods).

The majority of city guards are combatants, mostly 1st level. Officers are higher-level fighters, and maybe even some spellcasters.

\textbf{Siege Engines}\index{Siege Engines}

Siege engines are large weapons, temporary structures, or mechanisms traditionally used to besiege a castle or fortress.

\begin{center}
\includegraphics[width=0.7\linewidth]{immagini/armidaassedio.png}
\end{center}


\textbf{Heavy Catapult}: \index{Catapult}A heavy catapult is a gigantic siege engine capable of hurling boulders or other heavy objects with great force. Since the launch arc of the catapult is very high, the contraption is able to hit even areas outside its line of sight. To fire a heavy catapult, the chief machine operator makes a DC 15 special check using only his Attack Proficiency value with his Intelligence modifier, range penalty, and modifier from the lower section of the Table : Siege Engines.

If the check is successful, the catapult's boulder strikes the melee zone the catapult aimed at, dealing the indicated damage to any objects or characters in the area. Characters who make a successful DC 15 Reflex save take half damage. Once the boulder has hit the zone, subsequent roll will hit the same zone, unless the catapult is redirected or the wind changes direction or speed.

If a catapult's boulder misses, use the weapons table on \hyperlink{spargimento}{splash} (page \pageref{attacchiarmidaspargimento}). The distance covered is 1d4x3 meter.

Loading a catapult requires a series of actions that take up the whole round. It takes a DC 15 Strength check to lower the catapult arm; most catapults have wheels that allow up to two operators to use the Assist Another action to assist the main pulley operator.

A Profession (siege engineer) check with DC 15 will snap the arm into place, and then another Profession (siege engineer) check with DC 15 will load the projectile onto the catapult. It takes four rounds to reload a heavy catapult (several catapult operators can perform these actions in the same round, so four people can reload a catapult in just 1 round). A heavy catapult takes up a space of 5 meters.


\textbf{Light Catapult}: This is a smaller and lighter version of the heavy catapult. It functions essentially as a heavy catapult, except that a DC 10 Strength check is required to snap the arm into place, and only 2 rounds to redirect the catapult. A light catapult occupies a 3m space.

\textbf{Ballista}: \index{Balista}A ballista is basically a huge stationary heavy crossbow. Its size makes it difficult for most creatures to use. Thus, a medium creature takes a -1d6 penalty on attack rolls when using a ballista, and a small creature takes a -6 penalty. A smaller than large creature takes 2 rounds to reload the ballista after firing.

A ballista takes up a space of 2 meters.

\textbf{Ram}:\index{Ram} This massive trunk is sometimes tethered and suspended from a movable trellis that allows its wielder to swing it with ever-increasing force at a target. As the only action of the round, the character closest to the ram's point makes an Attack Roll against the building's Defence, applying a –1d6 penalty for lack of proficiency (it is not possible to have proficiency in using this machine). In addition to the damage listed on Table: Siege Engines, up to nine other characters can push the ram and add their Strength modifiers to the ram's damage, if they reserve an attack action to do so. It takes at least one Huge or larger creature, 2 Large creatures, 4 Medium creatures, or 8 Small creatures to man a ram (Tiny or smaller creatures can't use a ram).

A ram is usually 10 meters long. In a battle, creatures flying a ram must line up in two adjacent lines of equal length with the ram supported between the two lines.

\textbf{Siege Tower}\index{Siege Tower}: This machine is a huge wooden tower mounted on wheels or cylinders which can be pushed against a wall to allow the besiegers to scale the tower and thus get to the top of the walls benefiting from Coverage. The wooden walls of the tower are usually about 30 cm thick.

A typical siege tower occupies a space of 5 meters. Creatures within it push it with a speed of 3 meter (a siege tower can't run). The eight creatures pushing the tower on the ground floor have full cover, those on the upper floors have Improved Cover and can shoot through loopholes for archers.

\end{multicols}

\medskip

\textbf{Table: Catapult Attack Modifiers}\index{Table Catapult Attack Modifiers}

\medskip

\begin{tabular}{p{0.57\textwidth}p{0.35\textwidth}}
\textbf{Circumstance} & \textbf{Modifier}\\
\toprule
Line of sight does not reach target area & -6\\
Consecutive Shot (operators can see where most recent misses landed) & Cumulative +2 per previous miss (max +10)\\
Consecutive Shot (operators can't see where most recent misses landed but a spotter provides guidance) & Cumulative +1 per previous miss (max +5))\\
\end{tabular}

\bigskip

\textbf{Table: Siege Engines}\index{Table Siege Engines}

\medskip

\begin{tabular}{lllll}
\textbf{Machine} & \textbf{Cost (gp)} & \textbf{Damage} & \textbf{Range} & \textbf{Soldiers}\\
\toprule
Heavy Catapult & 800 & 6d6 & 60m & 4\\
Light Catapult & 550 & 4d6 & 45m & 2\\
Ballista & 500 & 3d8 & 36m & 1\\
Ram & 1000 & 3d6 & - & 10\\
Siege Towers & 2000 & - & - & 20\\
\end{tabular}

\begin{multicols}{2}

\bigskip

\textbf{City Streets}\index{City Streets}

Typical city streets are narrow and winding. Most city streets are 3 to 6 meters wide, while alleyways range from 3 meters to only 1 meters wide. If the paved floor is in good condition, it is possible to move normally, while badly damaged and badly damaged roads are considered equivalent to scattered debris, and increase the DC of Acrobatics checks by 2.

Some cities don't have large driveways, especially those that have grown gradually from small settlements. Cities that were either pre-planned, or perhaps consumed by a major fire that allowed authorities to build new roads on what were once inhabited areas, may have some larger roads running through them. These main roads are 8 meters wide, allowing wagons to pass side by side, with 1 meter sidewalks on either side.

\textbf{Crowd}: The city streets are filled with people who come and go, busy with various daily chores. In most cases it is not necessary to include every 1st level commoner on the map when it comes to a fight on the main avenue of the city. Instead, it is sufficient to indicate which areas on the map are occupied by the crowd. If the crowd sees anything dangerous, it will move away at a speed of 10 meters per round at an initiative count of 10. You must have melee distance to make contact with the crowd. Crowd provides Full Coverage.

\textbf{Lead the Crowd}: A DC 15 Diplomacy or DC 20 Intimidate check is required to get a crowd to move in a certain direction, and the crowd must be able to hear or see the character doing so the attempt. It takes a whole round to make the Diplomacy check, while it takes only one Action to make the Intimidate check.

If two or more characters attempt to push the crowd in two different directions, they make opposed Diplomacy or Intimidate checks to determine who the crowd will listen to. The crowd will ignore both if both check results fall below the above DCs.

\textbf{Rooftops}: Climbing a roof usually requires scaling a wall, unless a character can reach a roof by jumping off a higher window, balcony, or bridge. Flat roofs are only common in warm climates (accumulating snow can cause a flat roof to collapse) and are easy to run across. Moving to the top of a roof requires a DC 20 Acrobatics check. Moving horizontally across a sloped roof (moving parallel to its top, basically) requires a DC 15 Acrobatics check. Moving up and down a sloped roof requires an Acrobatics check with DC 10.

Sooner or later a character will reach the end of the roof, and will have to make a long jump to move to the next roof or to get down to the ground. The distance from one roof to the next is usually 3 meters, but the roof on the other side may be 1 meters higher or lower, or the same height. The guidelines given for Acrobatics (the peak height in a long jump is equal to one-quarter the horizontal distance) are used to determine if the character is capable of making a jump.

\textbf{Sewers}: To enter the sewers, characters usually have to open a grate (1 round) and jump down 3 meter. Sewers are built exactly like dungeons, except that the floor is slippery or covered in water. Sewers are also similar to dungeons in terms of the creatures that can be encountered within them. Some cities were built on the ruins of older civilizations, so the sewers could also lead to treasures and dangers from a bygone era.

\textbf{City Buildings}
Most city buildings are divided into three categories. Many buildings in a city are two to five stories high and are built side by side to form long rows, interrupted only by main or side streets. These terraced buildings usually house a shop on the ground floor, with offices or apartments on the upper floors. Inns, wealthier trading establishments, and larger warehouses (as well as any mills, tanneries, and other space-intensive businesses) are usually large free-standing buildings up to five stories tall.
Finally, smaller houses, shops, warehouses and warehouses are simple one-story wooden buildings, especially in poorer neighborhoods.

\textbf{City Lighting}
If a city has large driveways, these will be lit by lanterns hung about 2 meters high on the sides of the buildings. These lanterns are placed at a distance of 9 meters from each other, so the lighting in these streets is practically continuous. Secondary streets and alleyways are not lit; it is customary for citizens to pay a lantern man to accompany them, if they have to go out at night. The alleyways can be dark places even during the day, thanks to the shadows of the surrounding taller buildings. A dark alley in daytime is not dark enough to impart full but light cover.

\subsubsection{Working in the city}\index{Working in the city}\index{Downtime}

During breaks between one adventure and another or because a certain amount of time must pass for a certain thing to happen, the characters can try to make use of their Skills to earn some income.

The characters make one check per day in Crafts or Herbalism or Entertain (depending on the activity they carry out), depending on their success they will earn or not.

Subtract 15 from the check made, if the check is 16 or more, by the character and square the difference, the result is the silver coins earned ($(15-Test)^2$).


\end{multicols}

\vfill

\begin{center}
\includegraphics[width=0.77\linewidth]{immagini/fognelondra.png}

\textit{London Sewer Map, 1880}
\end{center}


\pagebreak

\section{Adventures and Disasters}\index{Adventures}\index{Disasters}

\label{avventure-e-disastri}
\begin{changemargin}{0.3cm}{0.3cm}\begin{emphasis}{
First, no one is left behind. (anonymous)}\end{emphasis}\end{changemargin}\medskip


\begin{multicols}{2}


\lettrine[lines=2, lhang=0.33, loversize=0.25, findent=1.5em]{N}{atural} disasters are terrifying environmental hazards that bring death and devastation. Supernatural ones can be even more destructive, as they can forever disfigure a world. A disaster is more like an adventure than an encounter, and doesn't have a specific Challenge Rating. Rather, each part of the disaster should be treated as a separate encounter designed with a Challenge rating appropriate to the PCs.

Below are the rules for dealing with the effects of three different types of disasters, both natural and supernatural. Some disasters occur rapidly, such as earthquakes and tsunamis, while others proceed through many stages, such as forest fires, volcanoes, and undead risings. Adjust the adventure outline to fit the disaster, to allow events to unfold over the course of a few minutes or several days depending on what you need.

\textbf{Volcanoes}\index{Volcanoes}

When the earth's crust breaks and expels its molten heart, one of the most dramatic natural disasters takes place: the eruption of a volcano. Volcanic eruptions offer a variety of options to the Arbiter, including lava, lava bombs, poisonous gases, and pyroclastic flows. Arbiters might also consider foretelling a dramatic volcanic eruption (or volcanic dragons) with pre-existing dangers, such as minor avalanches and earthquakes.

\textbf{Lava}\index{Lava}

Lava flows are generally associated with non-explosive eruptions and can be a permanent feature of active volcanoes. Lava flows are mostly slow, moving at 5 meters per round, but hotter lava flows are rapid, reaching 13 meters per round. Channeled lava, such as in a lava tube, is very dangerous, moving at a rate of 16 meters per round (4 move actions per round) (a challenge rating of 6 hazard). Creatures reached by a lava flow must succeed at a DC 20 Reflex save or be engulfed in lava. Success indicates they are in contact with the Lava but not Immersed.

\textbf{Lava Bombs} (Challenge Rank 2 or 8)\index{Lava Bombs}

Clusters of molten stone can be hurled many miles from an erupting volcano, cooling into solid stone before reaching the ground. A typical lava bomb strikes a point designated by the Arbiter and explodes in a 6 meters radius. All creatures in the area must succeed on a DC 15 Reflex save or take 4d6 points of damage. Creatures that have cover or are able to cover themselves (such as with a shield) gain a +2 bonus on this roll. Very large lava bombs sometimes form, dealing 12d6 points of damage. Normal lava bombs have a challenge rating of 2, large ones have a challenge rating of 5.

\textbf{Poisonous Gases} (Challenge Rank 5)\index{Poisonous Gases}

One of the most insidious threats from a volcano is toxic gas, often unnoticed amidst fire and destruction. Different types of poisonous vapors arise from a volcanic eruption, some visible, some not. Poisonous gases deal 1d3 points of Constitution damage per round when inhaled (Fortitude DC 15 negates, DC increases by 1 with each previous save), and visible ones also function as thick smoke. The clouds of poisonous gases flow downwards, and generally reach a height of 6 meters. Strong winds can deflect gas clouds, as can tall barriers, provided the gas has somewhere else to go.

\textbf{Pyroplastic Flows} (Challenge rank 10)\index{Pyroplastic Flows}

Some volcanic eruptions create a devastating wave of burning ash, hot gases and volcanic debris called a pyroclastic flow that can travel for miles. A pyroclastic flow is treated as an avalanche that travels 150 meters per round, combined with the poison gas effects noted above. Contact with the burning debris of the flow deals 2d6 points of fire damage per round, while any creature buried by the flow takes 10d6 points of damage per round.

\textbf{Tsunami}\index{Tsunami}

Tsunamis, sometimes attributed to tidal waves, are tremendous surges of water caused by underwater earthquakes, volcanic explosions, landslides or asteroid impacts. Tsunamis cannot be detected until they reach shallow water, when the body of water forms a large wave. Depending on the size of the tsunami and the slope of the coast, the wave can cover any distance from a hundred meters to over a kilometer on land, leaving a trail of destruction in its wake. The water then recedes, carrying all sorts of debris and creatures out to sea.

The exact devastation caused is subject to the Arbiter's discretion, but a typical tsunami topples or uproots all temporary or poorly constructed structures in its path, destroys approximately 25\% of well-constructed buildings (causing significant damage to those remaining ) and leaves solid fortifications slightly damaged. At least 1/4 of the population living in the area (including animals and monsters) perish in the disaster, washed out to sea, drowned on the beach or buried under rubble.

A creature can avoid being swept out to sea with a DC 25 Swim check; otherwise it is washed 6d6 x 3 meter from shore. The waters after a tsunami are always considered rough or stormy, barring magical influences. A creature caught in the collapse of a building takes 6d6 points of damage (DC 15 Reflex save halves), or half that much if the structure is particularly small. there is a 50\% chance that the creature will be buried (as with a Collapse), or that the tsunami can destroy the building, freeing the creature from the rubble.

\textbf{Undead Rising}\index{Undead Rising}

The result of an ancient curse or necromantic acts, one of the most terrifying supernatural disasters is the undead rising: the dead who emerge from the grave to claim the living. This disaster can affect any area where the dead have been buried, not just towns and cities. More than one battlefield has seen the rise of a legion of withered undead combatants. Undead uprisings occur in waves, with the timing varying according to the major forces at play. Events can unfold over the course of a few days, with a city devastated, or stretch over weeks with a terrified population cowering behind locked doors and struggling to survive. During the day, life often returns to some semblance of normality, as daylight temporarily suppresses the power of undeath.

\textbf{The Restless Dead}

In the first few nights of an undead uprising, the recently dead reanimate as zombies. Those buried in consecrated ground do not revive, but bodies left unburied or in mass graves stagger out into the streets, wreaking havoc. Initially, only a few corpses are able to free themselves from their coffins and graves, but every evening, the number of living corpses increases. When dawn comes, the dead seek safety in their graves or other hidden places. Anyone caught in the daylight flails Confused until they are destroyed or reach shelter. At the Arbiter's discretion, nonhumanoid corpses may be resurrected as undead on subsequent nights.

\textbf{The Skeletons Awakened}

As the uprising progresses, older and older corpses join the ranks of the undead. Skeletons bearing traces of long-rotten burial garments claws their way out of cemeteries and crypts, and act with a malevolence and organization rarely found among their kin. The undead remain devoid of Intelligence, but the magical power behind the raid gives them the efficiency and tactical acumen of a living army. Skeletons seek out weapons and Armour to equip themselves for battle. Elite Skeleton champions lead the troops, using Magic Items looted from abandoned tombs. Finally, Ghouls and Wights also prowl the streets after dark, along with other lesser Undead with free will

\textbf{Lost Souls}

As the uprising gathers strength, so do the restless souls of corpses long since reduced to dust. Ghosts, Shadows, Wraiths, and even Specters rise to hunt the living. Some Ghosts may break free from the malevolent influence of the uprising, and enterprising characters may glean valuable information from these restless spirits.

The infusion of negative energy fortifies undead within the raid area, granting the benefits of a Blessing. Areas once consecrated are now treated as normal terrain, and can serve as new sources of corpses for undead armies; the sanctified ground remains inviolate.

As the undead grow stronger, the rising tide of negative energy approaches the Plane of Shadow, dulling or graying colors except during the brightest hours of the day. Even the undead most vulnerable to light can move with impunity from late afternoon to midmorning.

\textbf{Necropolis}

The flow of negative energy is irreversible, the darkness eventually claiming the area, covering it in perpetual shadow. The hallowed ground remains a rare sanctuary, but only until it is destroyed by malevolent outside forces.

Heroes who die in battles return as fearsome undead generals. The few living survivors are subjected as slaves. The area becomes a city of death or construction begins if no city existed or survived. Undead with free will gather in this new sanctuary, and only the greatest heroes are able to return from this now withered area to the world of the living.

\end{multicols}

\vfill

\begin{center}
\includegraphics[width=0.45\linewidth]{immagini/anubis.png}

\textit{Representation \href{https://it.wikipedia.org/wiki/Anubi}{Anubi}}
\end{center}



\pagebreak

\section{Dungeon Adventures}\index{Dungeon}

\begin{changemargin}{0.3cm}{0.3cm}\begin{emphasis}{

\st{Linux} Dungeon is user friendly. It's just very picky about who its friends are. (anonymous)

\medskip

The dungeon is tilted. The creatures are angry because they can't play marbles (Dungeon Keeper 2, Video Game, 1999)

}\end{emphasis}\end{changemargin}\medskip

\label{avventure-nei-dungeon}
\begin{multicols}{2}

\lettrine[lines=2, lhang=0.33, loversize=0.25, findent=1.5em]{O}{f} of all the strange places an adventurer can explore, none more deadly than a dungeon. These labyrinths, filled with deadly traps, hungry monsters, and wondrous treasures, check every skill and ability of the characters. These rules can be applied to any type of dungeon, from a shipwreck to a vast underground cave complex.

\begin{changemargin}{0.3cm}{0.3cm}\begin{narrator}
The dungeon, cave, tunnel, call it what you prefer is a cornerstone of the adventure!

A dungeon is a recipe made of humidity, stench, stale air, dirt, mud, remains of creatures, traps, treasures, traps (plenty...), monsters, enemies, monsters (plenty!), darkness, sinister noises, mushrooms , creaks, yelps, screams, moans.. but also of fear, tension, shiver terror \& horror, emphasis, ecstasy, pain, disappointment, joy and treasures!

Your dungeon is never just a cave. NEVER!
\end{narrator}
\end{changemargin}

Whether they are caverns, caverns, quarries, caves, lairs, caves, "Dungeons" often represent the focal point of adventure and exploration.

The characters will spend a lot of time in these environments and the Arbiter must be prepared and ready for the environment they will encounter.

When preparing a cave it is necessary to think intelligently about the type of cave and the creatures it will encounter.
Placing a group of lizardfolk without thinking about what they eat, where they sleep, what kind of organization they have is dangerous, not to mention placing a chimera.
Will it have atrophied wings because the cave is 3 meters high and 3 wide and it is hard to move? What did she feed on during this period? Rather better to use a gorgon that feeds on minerals...

If designed with attention and care, a cave can become an excellent experience of encounters, situations and adventure.

\subsection{The underground}\index{The underground}

The natural conditions of the subsoil depend on various factors but there are certainly points in common for all.

- No lights to illuminate the spaces. There may be sporadic fluorescent mushrooms but nothing that can light up the whole environment

- Wet environment

- Usually cool ambient temperature, there are rarely caves with extreme temperatures in both heat and cold.

\subsubsection{Lighting}\index{Lighting}

In a cave there are no sources of artificial or natural light other than those introduced by sentient creatures. There may be groups of mushrooms, lichens, which dimly illuminate the ground where they grow but nothing else around.
Furthermore, if torn from the ground, they lose their bioluminescence after 2d4 rounds.

Cave-dwelling creatures must have become accustomed to the darkness probably by developing some form of alternate vision, such as darkvision, telluric sense, or blindsight.

All who are unseen have full cover, and hiding in shadows is done with a +2d6 bonus.

Even the torch itself can provide limited relief since its beam is 3 meter plus 3 meter of dim light and lasts for an hour before going out.

\subsubsection{Movement}\index{Movement}

If you don't have the means to see the terrain, it is considered difficult and holes, precipices and various obstacles can be very dangerous.

In total darkness and in a natural environment, a DC 12 Dexterity check must be made every 30 meters or stumble and suffer 1 temporary damage.

\subsection{Types of caves}

Different types of caves can be identified:

\begin{itemize}[leftmargin=*]


\item created by flowing water. In this case the tunnel can be quite chaotic in its unraveling due to the type of rocks that the water has encountered. There may still be underground rivers and lakes.

\item created by erosion. In this case the water is probably gone if not in a minimal part, the resulting caves can also be very large with rooms tens if not hundreds of meters wide.

\item may have been created by a volcano with flowing lava. In this case the tunnel dug out of the rock is often linear and somewhat smooth, the lava once congealed then crumbled over the millennia.

\item may be arctic caverns, dug out of the ice by water. In this case, carefully evaluate the surrounding environment and the freezing temperature.

\item can be artificial caves, built by creatures of different types.

\end{itemize}

\subsubsection{The four types of dungeons}\index{The four types of dungeons}

The four basic types of dungeons are defined by their current state. Many dungeons are variants of these basic types or combinations of several types. Occasionally, old dungeons are used repeatedly by new dwellers for different purposes.

\textbf{Ruined Structure}: Once inhabited, this place is now abandoned (in whole or in part) by its original creators and is occupied by other creatures. Many subterranean creatures seek out abandoned, subterranean buildings in which to make their lairs. Any traps that may have existed have probably already been removed or activated, but wandering beasts can be found.

\textbf{Occupied Structure}: This dungeon is still in use. Creatures (usually intelligent) still inhabit it, though they may not be the creators of the dungeon. An occupied structure could be a house, a fortress, a temple, an active mine, a prison, a headquarters.

This type of dungeon is less likely to have traps or wandering beasts, and more likely to have organized guards, both stationary and on patrol. The traps and wandering beasts that may be encountered are often under the control of the occupants. Occupied structures have furniture suitable for the inhabitants, as well as decorations, food reserves, and the possibility for the inhabitants to move around.

\medskip
\begin{center}
\includegraphics[width=0.8\linewidth]{immagini/avventure_dungeon.png}

\textit{The Red Romance Book, Illustration by H.J. Ford}
\end{center}
\medskip

The inhabitants may also have a communication system, and almost always control at least one access to the outside.

Some dungeons are partially occupied and partially empty or in ruins. In these cases, the occupants are usually not the original builders of the place, but rather a group of intelligent creatures who have established their base, lair, or fortification within the abandoned dungeon.

\textbf{Safe Cover}: When someone wants to protect something, he often buries it underground. Whether the object he wants to protect is a fabulous treasure, a forbidden artifact or the corpse of an important man, these valuable objects are placed inside a dungeon and surrounded by barriers, traps and guardians.

The safe haven type dungeon is the one that will have more traps and fewer wandering beasts. It is normally built for function rather than appearance, although it is sometimes decorated with statues and painted walls, especially for the tombs of important people.

\begin{center}
\includegraphics[width=0.8\linewidth]{immagini/dungeon.png}
\end{center}

Sometimes, however, a treasure hall or crypt is built to house living guardians. The problem with this strategy is that you need to keep the creatures alive between one intrusion attempt and another. Magic is usually the best way to supply these creatures with food and water. Builders of tombs and sepulchres typically place undead and constructs, which need no sustenance or rest, to protect their dungeons. Magic traps can attack intruders by summoning monsters into the dungeon that disappear when they finish their task.

\textbf{Natural Cave Complex}: Subterranean caverns provide shelter for any type of creature of the deep. Naturally created and connected by a system of labyrinthine passageways, these caverns lack any semblance of order, logic or decoration. With no intelligent power building it, this type of dungeon is the least likely to feature traps or doors.

Many varieties of fungi live in caves, sometimes growing into huge forests of mushrooms and puffballs, where subterranean predators prowl for those who feed on these vegetables. Some varieties of mushrooms produce a phosphorescent glow that can provide the natural cave complex with its own limited source of illumination. In other areas, the use of Daylight spells can ensure sufficient light for green plants to grow.

Often, a natural cave complex is connected to other dungeon types, having been discovered when the man-made dungeon was built. A cave complex can connect two independent dungeons, sometimes resulting in a strange mixed environment. A natural cave complex joined to another dungeon often provides a path that subterranean creatures can use to reach and populate a man-made dungeon.

\subsection{Exploration}\index{Exploration}\index{Moving with attention}

Moving through a dungeon requires attention and a cool head. Uneven floors, sinister noises, trapdoors and trapdoors, lights that appear and disappear make it not easy to safely venture into these dangerous environments.

The characters will have to be careful, actively look for traps, observe in the distance and keep a cautious attitude. All of this means that movement is halved if characters \textit{take precautions} to avoid trouble.

Describing what the character does to look for traps, passages.. \textit{problems} or requesting a check (Survival or Awareness) at DC 12 can give general indications on the \textit{feeling} that something is wrong.

\subsection{Dungeon Terrain}\index{Dungeon Terrain}

The following rules pertain to the basic terrains that can be found in a dungeon.

\subsubsection{Walls}\index{Walls}

Sometimes, brick walls (stones stacked on top of each other usually, but not always, held together with lime) divide dungeons into corridors and rooms. Dungeon walls can also be carved out of bare rock, giving a chiseled appearance, or they can be composed of smooth, plain stone as found in natural caves. Dungeon walls are difficult to damage or break through, but are usually easily climbable.

\end{multicols}
\textbf{Table: Walls}\index{Table Walls}
\medskip

\begin{tabularx}{0.95\textwidth}{XllllX}
\textbf{Type of Wall} & \textbf{Thickness} & \textbf{Breakthrough} & \textbf{Hardness} & \textbf{Hit Points} & \textbf{DC to Scale}\\
\toprule
Bricks & 30cm & 35 & 8 & 90 & 20\\
Top bricks & 30cm & 35 & 8 & 120 & 25\\
Reinforced bricks & 30 & 45 & 8 & 180 & 20\\
Etched Stone & 90 & 50 & 8 & 540 & 25\\
Rough stone & 150cm & 65 & 8 & 900 & 25\\
Iron & 7.5cm & 30 & 10& 90& 25\\
Paper & variable & 1 & --& 1 & 30\\
Wood & 15cm& 20 & 5 & 60& 21\\
\end{tabularx}

\medskip

\textbf{Table: Dig a tunnel}\index{Table Dig a tunnel}

\medskip

\begin{tabular}{llll}
\textbf{Miner}&\multicolumn{3}{c}{\textbf{Material to Excavate - 1 minute}}\\
&\textbf{Soil}&\textbf{Stone} \textbf{soft}&\textbf{Hard stone}\\
\toprule
Human&30cm&15cm&7cm\\
Gnome &45cm&30cm&15cm\\
Dwarf/Orc & 55cm&45cm&20cm\\
Stone Giant& 3m& 1.5m& 75cm\\
Xorn &6m&6m& 6m\\
Earth Elemental & 9m&9m&9m\\
\end{tabular}

\medskip

The excavated distances indicated are assumed to be obtained with spades or picks, otherwise reduce to a third.

\begin{multicols}{2}

\textbf{Brick Walls}: The most common type of wall for a dungeon, brick walls are usually at least a 0.5m thick. Often these ancient walls have holes and crevices, inside which sludge and small creatures can lurk, waiting there for their prey. Brick walls block out all but the loudest noises. A DC 20 Climb check is required to move along a brick wall.

\textbf{Higher Grade Brick Walls}: Sometimes brick walls are better built (smoother, with better fitting, less damaged stones) and occasionally these higher grade walls are covered in mortar or stucco. These walls are often embellished with paintings, bas-reliefs or other decorations. Higher quality brick walls are no harder to damage than normal brick walls, but are harder to Climb (DC 25).

\textbf{Reinforced Walls} These are brick walls with iron bars on one or both sides, or inserted into the wall itself to reinforce it. The hardness of the reinforced wall remains the same, but the Hit Points are doubled and the DC for the Strength check to break through it is increased by 10.

\textbf{Carved Stone Walls}: These walls are usually found in rooms or passageways carved into bare rock. The rough surface of a sculpted wall has tiny protrusions where fungi can grow and crevices where vermin, bats, or subterranean snakes can live.

When such a wall has an "other side" (the wall separates two rooms in a dungeon), the wall is at least three feet thick; if it were thinner it would risk collapsing everything because it would not be able to support the weight of the stone vault. A DC 25 Climb check is required to climb a carved stone wall.

\textbf{Coarse Stone Walls}: These surfaces are uneven and rarely flat. They are smooth to the touch but full of tiny holes, hidden alcoves and protrusions at various heights. They are usually wet or at least damp, as natural caves are usually the product of water infiltration. When such a wall from the "other side", the wall is usually at least 150 centimeters thick.

a DC 15 Climb check is required to move along a rough stone wall.

\textbf{Iron Walls}: These walls are placed within dungeons around important locations such as treasure halls.

\textbf{Paper Walls}: Paper walls are the opposite of iron ones, used as screens to block vision but nothing more.

\textbf{Wooden Walls}: Wooden walls are often found as recent additions to older dungeons, used to create animal pens, storerooms, or even just to divide a larger series into a series of smaller rooms.

\textbf{Magically Treated Walls}: These walls are stronger than average, with a higher Hardness, with more Hit Points, and breaking through them requires greater DC. Magic can usually double the wall's hardness and Hit Points and add up to +20 to its DC to break through it. A magically treated wall also gains a Saving Throw against spells that might affect it, with the save bonus equal to 2 + half the caster level of the magic reinforcing the wall. Creating a magical wall requires the Craft Wondrous Item feat and the expenditure of 1,500 gp for each 10-by-3m section.

\textbf{Slitwalls}: Slitwalls can be constructed of any strong material, but are usually made of brick, carved stone, or wood. They allow defenders to fire arrows or crossbow bolts at intruders while remaining behind the relative protection of a wall. Archers behind the loopholes have superior cover that gives them a +8 bonus to Defence, a +1d6 bonus on Reflex saves.

\subsubsection{Floors}\index{Floors}

As with walls, there are many types of dungeon floors.

\textbf{Paverstone}: Like brick walls, floors can be made of interlocking stones. They are usually full of fissures and usually just level. Slimes and molds grow inside these cracks. In some cases the water flows in small drains through the stones or forms stagnant pools. Flagstone is the most common type of floor in dungeons.

\textbf{Rumpy Paving}: Over time, some floors can become so uneven that they require a DC 10 Acrobatics check to run or Charge across their surface. Those who fail the check cannot move during that round. Such dangerous floors should really be the exception and not the rule.

\textbf{Carved Stone Floor}: Rough and uneven, carved stone floors are usually covered with loose stones, gravel, dust, and other debris. A DC 10 Acrobatics check is required to run or Charge on such a floor. Failure means that the character can still act, but cannot run or charge that round.

\textbf{Low Cobblestone}: Small and sparse debris is present on the ground. A floor with little cobblestone on it adds 2 to the DC of Acrobatics checks.

\textbf{Dense Cobblestone}: The ground is covered in debris of all sizes. Cobblestone is considered difficult terrain. A floor strewn with dense cobblestone adds 5 to the DC of Acrobatics checks, and adds 2 to the DC of Awareness (Stealth) checks

\textbf{Smooth Stone Floor}: Smooth, perfect, and sometimes even polished floors are found only in dungeons created by skilled and careful builders.

\textbf{Natural Stone Floor}: The floor of a natural cave is as uneven as the walls. It is difficult for these caverns to have large flat surfaces; their floors are more likely to be multi-level.

Some adjacent surfaces could vary in elevation by as little as a foot, so that moving from one point to another is no more difficult than climbing a step on a ladder, but in some places the floor could go up or down for over 1.5 meter, forcing the character makes a Climb check (pag. \pageref{Climb}) to move from one surface to another.

Unless there is a path dug by time or well beaten the ground is considered difficult and therefore the movement is halved, for convenience steps under 50cm consider them difficult terrain and those within 1.5m terrain doubly difficult.\index{Steps} Charging and running in these environments are impossible, except on the routes in question.


\begin{center}
\includegraphics[width=1\linewidth]{immagini/pavimento_grey.png}
\end{center}

\textbf{Slippery}: Water, ice, slime, or blood can make any floor described in this section more treacherous. Slippery floors increase the DC of Acrobatics checks by 5.

\textbf{Grate}: A grate often covers a pit or area below the main floor. Grates are usually made of iron, but larger ones may also be made of reinforced tree trunks. Many grates have hinges that allow access to the area below (these grates can be locked like a door), while others are fixed and designed not to be moved. A typical 1cm thick iron grate has 25 Hit Points, hardness 10, and DC 27 to break through or dislodge.

\textbf{Protrusions}: Protrusions allow creatures to walk above an area below. They are often arranged around trenches along the course of subterranean rivers, like balconies surrounding a large room, or they provide a position from which archers can position themselves to attack enemies from above.

Narrow ledges (less than 15 centimeters wide) require Acrobatics checks for those walking over them. Failure causes the moving character to fall off the ledge (DC 15).

Sometimes the ledges have a railing. In these cases, the characters gain a +1d6 bonus on Acrobatics checks to move along the ledge. A character near the railing has a +2 bonus on his opposed Strength check to avoid being pushed off the ledge.

The ledges can sometimes even be delimited by balustrades 60-90 centimeters high. Such walls provide cover from attackers within 3 meter of the other side of the wall, provided the target is closer to the attacker's railing.

Transparent Floors: Transparent floors, made of hardened glass or magical materials, allow you to observe a dangerous environment from above. Transparent floors are usually placed above lava pools, arenas, monster lairs, and torture chambers. Can be used by defenders
to monitor an area.

\textbf{Sliding Floors}: A sliding floor is a type of trap door, designed to be moved and reveal something underneath. A sliding floor usually moves so slowly that anyone standing on it can avoid falling into the opening, provided they have room to move. If such a floor slides so fast that there's a chance a character will fall into what's underneath it (sharp spears, a vat of boiling oil, or a shark-infested pool, acid...) then it's a trap.

\textbf{Trap Floors}: These floors are designed to become suddenly dangerous. With the right amount of weight applied or a nearby lever operated, spikes shoot out of the floor, flames or puffs of steam shoot out of hidden holes, or the entire floor moves. These strange floors are usually found inside arenas, designed to make the fights more exciting and deadly. This type of floor is constructed in the same way as a trap.

\subsection{The doors}\index{Doors}

\textbf{Doors} \index{Porte}Doors within dungeons are more than just entrances or exits. Often they can be real encounters. Dungeon doors come in three basic types: wooden, stone, and iron.

\end{multicols}

\textbf{Table: Doors}\index{Table Doors}

\medskip

\begin{tabular}{llllll}
\textbf{Door type} & \textbf{Typical thickness (cm)} & \textbf{Hardness} & \textbf{Hit Points} & \multicolumn{2}{c}{\textbf{DC to Open}} \\

&& & & Stuck & Locked\\
\toprule
Plain Wood & 2.5 & 5 & 10& 13 & 15\\
Good Wood & 3.75 & 5 & 15& 16 & 18\\
Strong Wood& 5 & 5 & 20& 23 & 25\\
Stone & 10 & 8 & 60 & 28 & 28\\
Iron & 5 & 10& 60& 28 & 28\\
Wooden portcullis & 7.5 & 5 & 30& 25& 25\\
Iron Gate & 5 & 10& 60& 25& 25\\
Lock & - & 15& 30& - & -\\
Hinges & - & 10& 30& - & -\\
\end{tabular}

\medskip

\begin{flushleft}
	Stuck: Strenght check for force open (+1d6 with crowbar)\\
Locked: Disable Device check to open\\

Breaking through a door with your shoulder/kicks costs 1 Action. Forcing it with a crowbar costs 2 Actions.\index{Breaking Down Doors Action}\index{Forcing Doors Action}


\textbf{Critically failing} a Strength check means having hurt yourself in the breakthrough maneuver. Until at least 10 minutes pass, it is no longer possible to break down a door.\index{Fail strenght check open door}\index{Break door}

\end{flushleft}

\begin{multicols}{2}

\bigskip

\textbf{Wooden Doors}\index{Wooden Doors}: Constructed of thick nailed planks, sometimes reinforced with iron bars (also placed to prevent dungeon moisture warping), wooden doors are the most common type of door. Wooden doors vary in hardness: they can be plain, good or sturdy. Simple doors (DC 15 to break through) are not designed to keep motivated attackers out.

Well-crafted doors (DC 18 to break through them), while strong and durable, are still not designed to take a great deal of damage. The strong doors (DC 25 to break through) are lined with iron and are fairly strong barriers against those who try to pass through them. Iron hinges hold the door up, and usually a circular ring in the center serves to open it. Sometimes, instead of a ring, a door has an iron bar on one or both sides that functions as a handle.

In inhabited dungeons these doors are usually well maintained (unlocked) and unlocked, although important areas will likely be locked.

\textbf{Stone Doors}\index{Stone Doors}: Constructed from blocks of solid stone, these heavy and unwieldy doors are often designed to pivot when opened, though dwarves and other skilled craftsmen are able to construct hinges strong enough to support the weight of a stone door.

Secret doors hidden along a stone wall are usually made of stone. Otherwise, doors of this type are designed to become strong barriers that protect anything beyond them. As a result they are often found locked or barred.

\textbf{Iron Doors}\index{Iron Doors}: Rusted but strong, iron doors in a dungeon are hinged like wooden doors. These doors are the strongest doors of the non-magical type. They are usually locked or barred.

\textbf{Breakthrough}\index{Door Breakthrough}: Dungeon doors can be locked, trapped, strengthened, barred, magically sealed, or sometimes simply locked.

All but the weakest characters will be able to knock down a door with a heavy tool such as a mallet, numerous spells and magical items can offer characters an easy way through a locked door.

\textbf{DC 10 or lower}: A door that anyone can break through.

\textbf{DC 11--15}: A door that a strong person would have to break down in one go, and that a person of average strength might have some hope of knocking down in one go.

\textbf{DC 16--20}: A door that pretty much anyone could break through given the time it takes.

\textbf{DC 21--25}: A door that only a strong or very strong person has a hope of breaking through, and probably not on the first try.


\begin{center}
	\includegraphics[width=0.85\linewidth]{immagini/porta_grey.png}
\end{center}


\textbf{DC 26 or higher}: A door that only a person of exceptional strength has any hope of breaking through.

\textbf{Locks}: Dungeon doors are often locked and so the Disable Device skill comes in handy. The locks are recessed on the edge opposite the hinges or straight into the center of the door. Locks usually control an iron or wooden bar that extends from the door into the wall that supports it.

The padlocks secure between two rings, one on the door and one on the wall. More complex locks, such as combination or enigma locks, are usually built into the door itself.

The DC to pick a lock with a Disable Device check often falls between 15 and 30, although locks with higher or lower DC exist. A door can have more than one lock, each of which must be opened separately.\index{Pick a lock of door}. Open a closed door or lock without picking tools carries a -1d6 penalty on the check.\index{Open door without picking tools}

A critical failure on open door or locks cause the break of picking tools.\index{Breaking picking tools}\index{Failing open lock}

Locks are often fitted with traps, usually poison needles that snap out to prick a burglar's fingers.

\subsubsection{Smashing a lock}\index{Smashing a lock}

A special door might have a keyless lock, but one that requires guessing the right combination of nearby levers or pressing symbols on a panel in the correct order in order to open it.

\textbf{Locked Doors}: Dungeons are often damp places, and in some cases doors get stuck, particularly if they are made of wood. It is usually assumed that approximately 10\% of wooden doors and 5\% of other doors are blocked. These values can be doubled (to 20\% and 10\% respectively) in the case of long-abandoned or neglected dungeons.

\textbf{Barred Doors}: When a character tries to break through a barred door, it is the quality of the bar that makes the difference, not the material of the door itself. Breaking through a door closed by a wooden bar requires a Strength check with a DC of 25, and the DC rises to 30 in the case of a metal bar.

Characters can attack the door and destroy it, leaving the bar hanging in the passage clear. Using a crowbar to force open a jammed/stuck door grants a +1d6 to the check.\index{Crowbar on door}

\textbf{Magic Seals}: Spells placed on a door can make it difficult to pass through a door.

A door on which a magic lock has been cast is considered closed even if it does not physically have a lock. It takes a knocking spell or Destroy Magic or a successful Strength check to pass through a door locked in this way.

\textbf{Hinges}: Most doors have hinges. Obviously sliding doors aren't (these rather feature grooves in the floor, allowing them to slide aside with ease).

\textbf{Standard Hinges}: These hinges are metal and hold the door to its post or wall. Remember that the door opens towards the side where the hinges are (so if the hinges are on the PCs' side, the door will open towards them; otherwise it will open towards the other direction).

Adventurers can remove the hinges one at a time by making various Disable Device checks (only if, of course, they are facing the side of the door on which the hinges are located). Such an action has a DC of 20, as many of the hinges are rusted or jammed.

Breaking a hinge is difficult. Most have hardness 10 and 30 Hit Points. The DC to break a hinge is the same as to knock down the door

\textbf{Put hinges}: These hinges are much more complex and only found in areas of excellent construction. These hinges are built into the wall and allow the door to open in both directions. Characters cannot reach the hinges to remove them unless they break through the door post or wall. Insert hinges are usually found on stone doors, but are sometimes seen on wooden or iron doors as well.

\begin{center}
\includegraphics[width=0.9\linewidth]{immagini/cardini.png}
\end{center}

\textbf{Pins}: The pins are not true hinges, but simple pegs that protrude from the top and bottom of the door and thread into the holes in its holder, allowing it to turn. The advantages of pins are that they cannot be removed like hinges and that they are easy to make. The downside is that since the door spins on its center of gravity (usually in the middle), nothing larger than half the door's width can fit through.

Doors fitted with pivots are usually of stone and often also wide enough to overcome the disadvantage. Another solution is to place the pin toward one end and make the door thicker at that end and thinner at the other, so it opens more or less like a normal door.

Secret doors within walls often rotate, as the lack of hinges makes it easier to conceal the door's presence. The pins also allow items such as a bookcase to be used as secret doors.

\textbf{Secret Doors}: Disguised as an ordinary portion of wall (or floor or ceiling), bookcase, hearth, fountain, a secret door leads to a secret passage or room.

Someone scanning the area can find a secret door (if one exists) with a successful Wisdom check (with DC 20 for a common secret door and DC 30 for a very well hidden door).

Many secret doors require a special method to open, such as a hidden button or pressure plate. Secret doors can open like common doors, pivot, slide, sink, rise, or even drop like a drawbridge.

A builder might place a secret door very low near the floor or very high up on a wall, making it more difficult to find and use the door.

\begin{center}
\includegraphics[width=0.6\linewidth]{immagini/arcoserpenti.png}

\textit{Henry Justice Ford}
\end{center}


\textbf{Magical Doors} Enchanted by the original builder, a door can call out to explorers not to proceed. It may be protected from harm, with a higher hardness or a higher number of Hit Points, as well as an improved Saving Throw bonus. A magical door may not lead to the space behind it, but may actually be a portal to a very distant place or even to another plane of existence. Other magical doors may require a password or special keys to open.

\textbf{Sluice gates}: These special doors are made of rods of iron or thick buttressed wood that swing down from a recess at the top of an arch. Sometimes a penstock has horizontal bars forming a grid, other times it doesn't. Usually raised by a winch or similar machine, the portcullis can be brought down quickly, and the bars end in spikes to discourage anyone from passing under them (or attempting to run through them as they are lowering). Once dropped, a portcullis closes, unless it is so large that no normal person would be able to lift it. In any case, lifting a typical portcullis requires a DC 25 Strength check.

\textbf{Walls, Doors and Detection actions}

Stone walls, iron walls, and iron doors are generally thick enough to block most divinations. Wood walls, wood and stone doors are generally not thick enough to do the same. However, a secret stone door built into a wall and as thick as the wall itself (at least a 0.5m) will block most of these Actions.

\textbf{Stairs} The most traditional method of connecting different levels of a dungeon is via stairs. A character can climb or descend a ladder as part of his movement without penalty, but he cannot run. Increase the DC of any Acrobatics check made on a ladder by 4. Some particularly steep stairs are treated as difficult terrain.

\subsection{Dangers in Dungeons}\index{Dangers in Dungeons}

In dungeons and caves, in addition to monsters, there are also other dangers such as collapses, molds, fungi and more.

\subsubsection{Crumbling and Sagging (Challenge Rank 8)}\index{Crumbling and Sagging}

Collapses and subsidence in tunnels are extremely dangerous. Not only do dungeon explorers run the risk of being crushed by tons of stone, but also, should they survive, get stuck under a pile of debris or be unable to reach an exit.

A cave-in buries anyone in the middle of the buried area, and so the rolling debris will inflict damage on anyone in the outlying areas of the buried area. A typical collapse-prone corridor might have a 3m-radius buried zone and a melee-radius creep zone at the far end of the buried one.

A failing ceiling can be identified with a DC 20 Engineering check or a DC 20 Mason Profession check. A Dwarf can make this check simply by walking within 3 meter of a failing ceiling.

A crumbling ceiling can collapse under the impact of a large force. A character can cause a collapse by destroying half of the pillars holding up the ceiling.

Characters in the buried area take 8d6 points of damage, or half damage if they succeed on a DC 15 Reflex save. They are then buried. Characters in the swipe zone take 3d6 points of damage, or no damage if they succeed on a DC 15 Reflex save. Characters in the swipe zone are also buried if they fail their save.

Buried characters take 1d6 points of non-lethal damage for every minute they remain under rubble. If a character in this condition is knocked unconscious, he must make a DC 15 Constitution check. If the character fails the check, he begins to take 1d6 lethal damage per minute until he is freed or dies.

Characters who have not been buried can pull their companions out from under the rubble. In 1 minute, using only their hands, a character can move an amount of rock and debris equal to five times their heavy load limit. The amount of loose rock that fills a melee area weighs approximately 1 ton (1000 kg). Equipped with the right tools, such as a pickaxe, crowbar, or shovel, a digger can take half as long as doing it by hand. A buried character could also be allowed to free himself by passing a DC 25 Strength check.

\subsubsection{Sludge, Mold and Fungi}\index{Sludge, Mold and Fungi}

In the dank dark recesses of the dungeons, mold and mildew thrive, fear the mold columns! As far as spells and other special effects are concerned, all slimes, molds and fungi are considered vegetables. Like traps, dangerous slimes and molds have a Challenge rating, and characters gain Experience Points for encountering them.

A glossy organic ooze coats anything that remains too long immersed in the darkness and dampness of the dungeons. This type of mud, while it may be repulsive, is not dangerous. Mold and mildew abound in dark, cold, damp places. While some are as harmless as normal dungeon slimes, others are quite dangerous. Edible mushrooms, blisters, yeasts, molds and other types of fibrous, bulbous mushrooms or entire beds of fungal spores can be found in most dungeons. They are usually harmless and are often edible (although most are unappealing or have a strange taste).

\textbf{Screeching Boleto}\index{Screeching Boleto}: These human-sized purple mushrooms emit a piercing sound that lasts for 1d3 rounds whenever there is movement or a light source within 3 meter. This shout makes it impossible to hear other sounds or noises within melee range. The sound attracts nearby creatures who are willing to investigate. Some creatures that live near shriekers have learned that noise most often means food.

\begin{center}
\includegraphics[width=0.9\linewidth]{immagini/funghi.png}

\textit{They glow in the dark, trust me! and fried are even better!}
\end{center}

\textbf{Green Slime}\index{Green Slime} (Challenge Rank 4): This dungeon hazard is an insidious variety of regular slime. The green slime eats away at flesh and organic matter that comes into contact with it, and is even capable of dissolving metals. Bright green, wet and sticky, it spreads in patches on walls, floors and ceilings and reproduces by consuming organic material. Drops from walls and ceilings when it detects movement (and possible nourishment) below it.

Green slime deals 1 point of Constitution damage each round it devours flesh. On the first round of contact, the slime can be scrubbed off a creature (probably destroying the object used to scrub it off), but after the first round it must be frozen, burned, or cut (also dealing damage to its victim) to remove. . Anything that deals fire or cold damage, sunlight, or a remove disease spell destroys a patch of green slime. In the case of wood or metal, the green slime deals 2d6 points of damage per round, ignoring the hardness of the metal but not that of the wood. It does not damage the stone.  Defense 10, PHit Point 30, Saving Throws F 3, R 0, W 1.

\textbf{Glow Mushroom}\index{Glow Mushroom}: This strange subterranean mushroom gives off a faint purplish glow that lights up caverns and subterranean passageways like a candle. Rare spots of this mushroom light up like a torch.

\textbf{Yellow Mold} \index{Yellow Mold}(Challenge Rank 6): Releases a cloud of poisonous spores within a 3m radius when disturbed. All within 3 meter of the mold must succeed at a DC 15 Fortitude save or take 1d3 points of Constitution damage. Another DC 15 Fortitude save is required once per round for the next 5 rounds or to avoid taking another 1d3 points of Constitution damage. A successful Saving Throw blocks this effect. Fire destroys yellow mold, while sunlight renders it inert. Defense 10, Hit Point 25, Saving Throws F 3, R 0, W 1, Vulnerability to Fire.


\textbf{Brown Mold} \index{Brown Mold}(Challenge rank 2): Brown mold feeds on heat, extracting it from everything around it. It usually appears in patches with a diameter of scrum size and the temperature around the mold is always cold within a 3m radius. Living creatures within melee distance of it take 3d6 non-lethal cold damage. If a fire source is brought into melee by the mold, it immediately doubles in size. Cold damage, such as that dealt by a cone of cold, instantly destroys it.
Defense 10, Hit Point 12, Saving Throws F 3, R 0, W 1, Vulnerability to Cold, absorb to hit point damage dealt by fire.

\subsubsection{Example of Dungeon Traps}\index{Example of Dungeon Traps}

The name of the trap is indicated, the DC for the Survival check to find the trap and the indications for using it.\\

\textbf{Flooded room, DC 17}: If the characters do not notice the pressure plate on the floor it will cause the entrance door to seal and the room will begin to fill with water.
The room fills with water in 10 rounds. A DC 15 Survival check, combined with a DC 12 Swim check, detects the plate that triggers the escaping water.

\textbf{Crushing Room, DC 15}: If the PCs fail to notice the pressure plate on the floor it will seal the entrance door and very loud grinding and gearing noises will fill the room. The walls will begin to move closer together like the ceiling to the floor. If the characters do not find the hidden tile (DC 17) they will suffer 10d6 constricting damage. The trap is easier to detect than others because the walls are thicker making the room smaller.

\textbf{Crushing ceiling, DC 18}: if the characters don't notice the activation system (pressure plate, cable, interrupted beam of light...) a 3m x 3m section of ceiling will fall on the characters with damage of 3d6.

\textbf{Tunnel of Cobwebs, DC 12}: This tunnel is evidently filled with thick, dense, sturdy cobwebs. If the characters enter, they are considered Entangled. After 1d4 rounds of permanence, an activator will generate a spark setting fire to the webs. You take 2d4 fire damage each round inside the tunnel.

\textbf{Pit, DC 15}: Careless character will cause a 3m x 3m section of floor to collapse into a pit. This can be a simple pit (1d6 falling damage), with spikes (1d6+2d4), with acid (1d6 per round), with undead...

\textbf{Garrotte, DC 14}: This trap can be very insidious. A magically sharpened wire is 1 meter above the ground, between one wall and the opposite one and flows towards the players.
A DC 13 Athletics check is required or take 2d6 of slashing damage.

\textbf{Crushing Mail, DC 16}: As soon as you touch this door, it rotates on central hinges and as it rotates it hits the character (or characters if a large door). Inflicts 1d6 bludgeoning damage and continues to spin for 1d6 rounds.

\textbf{Finger Shredder, DC 14}: This trap is very sneaky. It has a hole of about 1 cm in diameter and 7 cm deep. Anything that touches the bottom will trigger the trap, dealing 2d4 damage to the inserted finger/object. The blade could also be poisoned.

\end{multicols}

\pagebreak

\section{Dangerous Adventure}\index{Dangerous Adventure}


\begin{changemargin}{0.3cm}{0.3cm}\begin{emphasis}{
An adventure is a reasonable outcome. Two is better, three deserves to be passed down, and four...no one can ever dispute four adventures. (John Steinbeck)


\medskip


Runs less danger who, even if he is safe, is on his guard. (Publilius Syrus)}
 \end{emphasis}\end{changemargin}\medskip

\label{pericoli-in-avventura}

\begin{multicols}{2}

\lettrine[lines=2, lhang=0.33, loversize=0.25, findent=1.5em]{T}{he} world is full of dangers as well as dragons and hungry fiends. Hazards are area-based threats that have a lot in common with traps, but are usually part of the place instead of being built. Hazards fall into three main categories: environmental, living, and magical.

Environmental hazards include landslides, fires and the like. Living hazards include creatures that aren't considered monsters but pose a threat to unwary adventurers, such as slimes, fungi, and mosses. Magical dangers are the most unpredictable and can be leftovers from arcane experiments, strange subterranean radiation, or failed ancient spells.

\medskip

\begin{center}
\includegraphics[width=0.8\linewidth]{immagini/boscopericoli.png}
\end{center}

\textbf{Antidweomer (Challenge Rank 6)}\index{Antidweomer}

Zone of magic-destroying entropy, antidweomers form at the sites of great wizarding duels, through the destruction of powerful artifacts, or from vortexes of mystical energy on the fringes of antimagic zones. Sizes vary from small bubbles of just a few meters to large areas the size of a city.

A successful Arcana check with DC 20 reveals the proximity of an antidweomer with a tingle in the air. Active magic brought into an antidweomer may be dispelled, and any spell cast into it is subject to an immediate counterspell. If you get a critical success in the Magic Test, it passes the counter spell but generates no further effects.

If the spell fails, the release of magical energy deals 2d6 force damage in a 3m-radius explosion centered on the caster; a DC 15 Reflex save allows you to halve this damage.

A spell manifested by an object always fails.

If multiple overlapping blasts hit the same target, only the most damaging one applies. A spell that has resisted an attempt to dispel isn't affected again unless it exits and reenters the antidweomer.

Stronger antidweomers are even more destructive. Each +1 increase in Challenge rating increases the DC damage of the save by 1d6 by 1.

\medskip
\textbf{Stale Air (Challenge Rank 1 or 4)}\index{Stale Air}

An unseen danger, gas pockets pose a risk to miners, spelunkers and adventurers investigating caves. Nonflammable gases such as carbon dioxide or nitrogen are Challenge 1 and require a DC 25 Survival check to notice.

Creatures that breathe that air must make a Fortitude save (DC 15 +1 for each preceding roll) every hour or become fatigued. Once fatigued, they begin to choke slowly. Creatures holding their breath can avoid these effects.

Flammable vapors such as coal gas are much more dangerous (Challenge grade 4). This gas replaces the breathable air in the lungs, causing fatigue—in addition, any open flame or spark causes an explosion that deals 6d6 points of damage (DC 15 Reflex saves half) to anyone in the cave or within 3 meter of an entrance. Fire burns the oxygen in the air, making it unbreathable for 2d4 minutes. After an explosion, flammable gas typically takes several days to return to dangerous levels.

\medskip
\textbf{Parasites}\index{Parasites}

Parasites such as ear-eaters or scavenging larvae cause parasites, a type of affliction similar to disease. Parasitosis can only be cured through specific treatments; no matter how many Saving Throws are made, the parasitosis continues to afflict the target. Even if a remove disease (or similar effect) immediately kills a parasite, immunity to disease offers no protection, since it is caused by parasites.

\medskip
\textbf{Ear Finder (Challenge Rank 5)}\index{Ear Finder}

Earfinches are tiny white worms that live in rotten wood or other organic debris. They can be noticed with an Awareness check (DC 15). Otherwise, a living creature searching their lair inadvertently transfers itself to one or more earplugs, which then seek out a warm spot on the creature's body, favoring the ear canal, and lay 2d8 eggs before dying.

The eggs hatch 4d6 hours later and the larvae devour the surrounding flesh. Upon the death of their host, the worms crawl out and look for a new one.

Remove Disease kills all earfinches or unhatched eggs on a host. Some earfinders prefer to live in tainted wood, often hiding in dungeon doors. The small holes left by this variant are very difficult to notice (DC 20 Awareness).

\medskip
\textbf{Earfinder}

Type: Parasitosis

Save: DC Fortitude 15

Onset: 4d6 hours

Saving Throw Frequency: 1 every hour

Effects: 1d3 to Constitution on failed save

\medskip
\textbf{Mnemonic Crystals (Challenge Rank 3)}\index{Mnemonic Crystals}

Memory crystals are large (10-13 meters tall) clusters of purple quartz crystals that radiate an aura of strong disruption. To identify them requires an Arcana check with DC 25.

Memory crystals build magical energy to grow and defend themselves, draining the prepared spells of spellcasters who must make a DC 22 Will save each round while within 3 meter of the crystals.

If the roll fails, they lose 1 spell slot (one less castable spell) available. By damaging or breaking the crystals, the absorbed spells are expelled in a blast of mental energy that deals 1d4 points of Wisdom damage to all within a 6 meters radius.

Memory crystals are very fragile (hardness 0, 1 hit point).
In areas rich in crystals, creatures that pass through them must succeed at a DC 10 Acrobatics check to avoid walking on them or brushing against them by breaking them.

\medskip
\textbf{Carnivorous Grubs (Challenge Rank 4)}\index{Carnivorous Grubs}

Once they have occupied a living body, the larvae burrow towards the heart, brain and other key internal organs of the host, eventually causing its death.

In the first round of parasitism, applying fire to the entrance hole can kill the larvae and save the host, but the host takes 1d6 points of fire damage.

Extracting them also works, but the longer the larvae remain in the host, the more damage this method causes. Extracting the larvae requires a slashing weapon and a DC 20 First Aid check, inflicting 1d6 points of damage for each round that the host was afflicted with parasites. If the First Aid check succeeds, one larva is removed. Remove Disease kills all scavenger larvae on a host.

\medskip
\textbf{Necrophagous Larvae}

Type: Parasitosis

ST: DC Fortitude 17

Onset: Immediate

Frequency: 1/round

Effects: 1d2 points of Constitution damage per grub

\medskip
\textbf{Magnetized Ore (Challenge Rank 2)}\index{Magnetized Ore}

The strange energies of the underworld can charge stones and veins of minerals with powerful magnetic fields, creating a danger to those who carry or wear ferrous metals. All iron or steel things brought within a 3m radius of the ore are pulled toward it.

\begin{center}
\includegraphics[width=0.8\linewidth]{immagini/neodimio.png}

\textit{Neodymium}
\end{center}


Small creatures are dragged even with 7.5 kg of metal, Large ones only with 30 kg. For creatures of other sizes, the weight changes according to the Carrying Capacity rules. Creatures wearing metal Armour take a penalty, those struck are dragged up to 10 meters, take 2d6 points of damage from the impact with the rock, and are considered grappled. Freeing a struck object requires a DC 20-25 Strength check

\medskip
\textbf{Cursed Pool (Challenge Rank 3)}\index{Cursed Pool}

The lingering effects of ancient curses or the noxious energy emanating from a submerged cursed magical item can turn a mere pool of water into a hazardous magical hazard. A cursed pool draws passers-by into its depths through the illusion (DC 16 Will save to doubt) of a glittering treasure on the 3m-deep bottom. Any creature that gets to the treasure activates the curse.

A creature within the pool must succeed at a DC 16 Will save or be affected by the curse, distorting its perception of the pool. The water seems to thicken in a viscous sapropelite (Editor's note: also sapropel or fetid slime, used in geology to indicate a blackish, pasty and more or less constipated sludge, originating from the deposit in stagnant or slightly moved water of the remains of organisms mixed with calcareous or siliceous shells of microorganisms and clayey substances), while the pool seems to reach a depth of 12 metres.

Your Swim in the Pool checks take a –10 penalty, your speed is reduced to half normal due to these effects, you are considered distracted when casting spells.

A cursed pool radiates strong magic, and can be destroyed by Destroy Magic or Remove Curse.

\medskip
\textbf{Poisonous Oak (Challenge Rank 1 or 3)}\index{Poisonous Oak}

Contact with a poison oak (Challenge rank 1) causes a painful, itchy rash that renders the victim sickened until the damage heals. Full body contact or inhalation of smoke from a burning poison oak could be fatal (Challenge rank 3). A check of Nature (or Herbalism) with DC 15 reveals the dangers inherent in the apparently innocuous plant. This hazard can also be used for similar harmful plants (poison ivy, poison sumac or stinging nettles, but the latter are harmless when burned).

\textbf{Poison Oak}

Type: Poison, contact

ST: DC Fortitude 13

Onset: 1 hour

Effects: 1d4 Dex damage, creature is sickened until damage heals

Heals: 1 ST


\subsection{Preparing for rest}\index{Preparing for rest}\index{Sleeping}\index{Guard shifts}

Every adventurer must rest from time to time, he must do it carefully and being careful not to run into nasty and dangerous surprises.

Whenever a character ends a 24-hour period without sleeping for at least 8 hours, he must succeed at a DC 12 Fortitude save or become fatigued.

Each further missed rest will make him even more fatigued by cumulating the relative penalties. If the character stays awake for more days, fighting sleep becomes more difficult. After the first 24 hours, DC increases by 4 for each consecutive 24-hour period without 8 hours of sleep. DC resets to 12 when character completes a rest of at least 8 hours.

Sleeping in medium or heavy Armour causes you to be fatigued, except if you have the Feat\hyperlink{secondapelle}{Second Skin}.

You are unable to sleep the 8 hours at intervals of less than 16 hours. \index{Sleeping more time for day}

A demanding activity such as fighting, casting spells, riding, if continued for more than 10 minutes invalidates the benefits of the rest taken, forcing a new rest.


\subsubsection{Arranging Watch Shifts}

If the group is large, the guard shifts to watch over and control the environment become shorter.

\medskip{}

\textbf{Table: Duration of guard shifts}\index{Table Duration of guard shifts}

This table indicates the duration of the guard shifts and the total rest time of the group, in the hypothesis of resting at least 8 hours.

\medskip{}

\begin{tabularx}{0.45\textwidth}{XXX}
\textbf{Members} &\textbf{Duration}&\textbf{Duration}\\
\textbf{group}&\textbf{turn}&\textbf{Total}\\
\textbf{2}	& 8 h	& 16 h\\
\textbf{3}	& 4 h & 12 h\\
\textbf{4}	& 2 h e 30 min. & 10 h e 30 min.\\
\textbf{5}	& 2 h	& 10 h\\
\textbf{6}	& 1 h e 30 min. & 9 h e 30 min.\\
\end{tabularx}

\medskip{}

A sharp noise grants a DC 15 Awareness check, or equal to the opponent's Stealth check +8, to wake up.\index{Waking up to noise
}\index{Rumor on night}

\end{multicols}

\vfill

\begin{center}
\includegraphics[width=0.75\linewidth]{immagini/mappaparigi.png}

\textit{Ancient Map of Paris}
\end{center}

\pagebreak

\subsection{Adventures and Traps}\index{Traps}\label{trappole}

\begin{changemargin}{0.3cm}{0.3cm}\begin{emphasis}{
Anyone who always sets the trap in the same place will not catch any iguanas. (African proverb)}\end{emphasis}\end{changemargin}\medskip



\begin{multicols}{2}

Almost everywhere a trap can be encountered. Traps can be magical or mechanical in nature. Mechanical traps include pits, arrows, falling boulders, water-filled rooms, whirling blades, and anything else that depends on a mechanism to operate. Magic traps are magical trap devices or trap spells. Magic trap devices when activated generate the effects of a spell. Trap spells are spells such as glyph of ward and symbol that function like traps.

\textbf{The Traps in the Game}
When the adventurers encounter a trap, you should know how the trap activates and what it does, as well as have an idea of how the characters can spot the trap and be able to disarm or avoid it.

\subsubsection{Trigger a Trap}
Most traps are triggered when a creature travels to or touches something that the creator of the trap intended to protect. Normal activation systems are pressure plates or false floor sections, pulling a cord, turning a handle and using the wrong key in the lock. Magical traps often activate when a creature enters an area or touches an object. Some magical traps (such as the glyph of warding spell) have more complex triggering conditions, including using passwords to prevent the trap from activating.

\subsubsection{Locate and Disable a Trap}
Usually, some elements of a trap are clearly visible upon close inspection.

The description of the trap specifies the checks and DCs needed to detect it, disable it, or both. A character actively seeking a trap can attempt a \textbf{Survival} check against the trap's DC.

The Arbiter can also compare the DC to detect the trap against the Survival score (at roll 8) of the characters in order to determine whether a party member notices the trap. If the adventurers notice the trap before activating it, they might attempt to disarm it, either permanently or long enough for them to pass.

The Arbiter might request a Disable Device check. If you don't have burglary tools \index{Burglary tools} or adequate tools, you make the check with a -1d6 penalty. \index{Disable devices without tools}The Survival skill can also be used albeit with a -2d6 to deactivate a trap, lock..., in this case the duration of the operation is equal to 1 Action per DC of the trap.

If you want to temporarily disable \index{Temporary disabling traps} a trap, add 6 to the difficulty. This will disable the trap for 2d4 minutes.

A magical trap can be disabled with a Disable Device check as long as the Arcana value is at least 1/4 of the trap's DC, in addition to any other checks listed in the trap's description. Dispel magic spell has a chance to nullify most magical traps.\index{Disable magic traps}

If the check to disable or disable the trap fails with a critical roll \index{Failure on disabling traps} (rolls two 1s or two 2s and a 1 on the check) the trap springs.

In most cases, the description of the trap is clear enough for the Arbiter to judge whether a character's actions locate or foil the trap.

Use common sense, drawing on the description of the trap to determine what happens. No trap design could ever be able to anticipate every possible action the characters might attempt to take.

The Arbiter should allow a character to discover a trap without making proficiency checks if his actions or description of what he does would clearly reveal the presence of the trap.

Foiling traps can be a little more complicated. Let's take the case of a chest defended by a trap. If the chest is opened without pulling on the two handles placed on the sides, a mechanism placed inside shoots a barrage of poisoned needles towards anyone who is in front of it.

After inspecting the chest and making a few checks, the characters are still not certain that he is trapped. Rather than opening the chest, they aim a shield in front of it and open it from a distance using an iron rod. In this case, the trap is activated, but the barrage of needles is fired at the shield without harming anyone.

Traps are often designed with mechanisms that allow them to be disarmed or bypassed.

\subsubsection{Trap Effects}
The effects of traps can range from mere inconvenience to lethal. A trap's description specifies what happens when it is triggered.
A trap's attack bonus, save DC to resist its effects, and the damage it deals can vary based on how dangerous the trap is.

Use the DC table of Trap Saving Throws and Attack Bonuses and the Severity of Damage by Level table as suggestions about the three levels of trap severity.

\medskip

\begin{center}
\includegraphics[width=0.9\linewidth]{immagini/medusa.png}
\end{center}


\textbf{Table: DC of Saving Throws and Attack Bonuses of Traps}\index{Table DC of Saving Throws and Attack Bonuses of Traps}

\medskip

\begin{tabularx}{0.45\textwidth}{XXX}
Trap Danger&DC Saving Throw& Attack Bonus\\
\toprule
Setback&13-14&+4 to +6\\
Dangerous&16-20&+8 to +10\\
Deadly&21-26&+12 to +15\\
\end{tabularx}

\medskip

\textbf{Table: Severity of Damage by Level}\index{Table Severity of Damage by Level}

\medskip

\begin{tabularx}{0.45\textwidth}{XlXX}
Character Level&Mishap&Dangerous&Deadly\\
\toprule
1st-4th&1d10&2d10&4d10\\
5th-10th&2d10&4d10&10d10\\
11th-16th&4d10&10d10&18d10\\
17th-20th&10d10&18d10&24d10\\
\end{tabularx}

\medskip

\subsubsection{Complex Traps}
Complex traps work like normal traps, except that once triggered, they perform a series of actions each round.

A complex trap turns the process of dealing with a trap into something more like a combat encounter. When activating a complex trap, roll its initiative.

The trap's description includes an initiative bonus. During its round, the trap activates again, often performing an action, whether it's an attack, an effect that changes over time, a dynamic challenge. Otherwise, the complex trap can be located and disabled in the usual ways.

\subsubsection{Example Traps}
\textbf{Poisoned Needle}

Mechanical trap

A poisoned needle is hidden inside the lock of a chest, or other object that can be opened. Opening the chest without the proper key would trip the needle, which dispenses a dose of poison.

When the trap is triggered, the needle extends 10 cm from the lock. A creature within range takes 1 piercing damage and 11 (2d10) poison damage, and must succeed at a DC 20 Fortitude save or take -1d6 on attack rolls and -1d6 on Proficiency checks for 1 hour.

The character who passes a Survival check with DC 22 can deduce the presence of the trap from the modifications made to the lock to house the needle. A successful Disable Device check using picking tools disarms the trap by removing the needle from the lock. A failed check to pick the lock sets off the trap. Declaring to stick a stick in the lock is just as effective in disabling the trap.

\medskip

\textbf{Poison Darts}

Mechanical trap

When a creature steps on a hidden pressure plate, poison darts are fired from a spring mechanism or pressure tubes cunningly hidden within the surrounding walls. An area might feature multiple pressure plates, each connected to its own set of darts.

The tiny holes in the walls are hidden by dust and cobwebs, or cleverly hidden among the bas-reliefs, murals or frescoes that adorn the room. The DC of the check to notice them (Survival) is 18.

The character who passes a DC 18 Survival check can deduce the presence of the hidden pressure plate from the differences in the flooring it is made of compared to the rest of the floor.

Wedging an iron spike or other object under the pressure plate prevents the trap from activating. Filling the holes with tissue or wax prevents the darts contained inside from escaping.

The trap is activated when more than 10 kilos of weight are placed on the pressure plate, thus firing four darts. Each dart makes a ranged attack with a +10 attack bonus against a random target within 3 meter of the pressure plate (camera has no impact on this attack roll).

If there are no targets in the area, the dart hits nothing. A hit target takes 2 (1d4) piercing damage and must make a DC 18 Fortitude save, taking 11 (2d10) poison damage on a failed save, or half as much damage on a successful one.


\medskip

\textbf{Dit}

Mechanical trap

We present below four basic types of pits.

\medskip

\textit{Simple Pit}

The simple grave is a hole dug in the ground. The hole is covered with a thick fabric anchored to the edge of the pit and camouflaged with dirt and debris.
The DC to notice the pit is 12. Anyone who steps on the fabric falls inside the hole and pulls the fabric after them, taking damage based on the depth of the pit (usually 3 meter, but some pits are deeper).

\medskip

\textit{Hidden Pit}

This pit has a cover made of material identical to that of the surrounding floor.
Passing an Awareness check with DC 18 shows the absence of traces in the section of floor that forms the cover of the pit.

A successful DC 18 Survival check is required to confirm that that section of floor actually covers a pit.

When a creature steps on the cover, it opens like a trap door, plunging the intruder into the pit below. The pit is usually between 3 and 6 meters deep, but it can be even more.

Once the pit has been located, an iron spike or similar object can be driven between the pit cover and the surrounding ground to prevent the cover from opening, making passage safe. The cover can also be magically held shut by the Magic Lock spell or similar magic.

\medskip
\textit{Snap Pit}

This pit is identical to the hidden pit trap, with one key exception: the trapdoor covering the pit hides a spring mechanism. After a creature falls into the pit, the cover snaps shut to trap the victim inside.

A successful DC 20 Strength check is required to force open the cover. The cover can also be destroyed. A character inside the pit can also attempt to disable the spring mechanism from within by making a successful Disable Device check on DC 18 and using burglary tools, provided he can reach and see the mechanism in question. In some cases, another mechanism causes the pit to reopen.

\medskip

\textit{Spiked Pit}

The pit is a simple, hidden or snap pit, on the bottom of which there are wooden spikes or iron spikes. A creature that falls into the pit takes 11 (2d10) piercing damage from the spikes, in addition to the falling damage.

More cruel versions of this trap have poison sprinkled on the spikes placed at the bottom of the pit. In that case, anyone taking piercing damage from the spikes must also make a DC 16 Fortitude save, taking 22 (4d10) poison damage on a failed save, or half as much damage on a successful one.


\medskip

\textbf{Falling Net}

Mechanical trap

This trap uses a wire to release a net hanging from the ceiling.

The wire is placed 7 centimeters above the ground and extends between two columns or trees. The web is hidden by cobwebs or foliage. The DC (Survival) for noticing the wire and the net is 15. A successful Disable Devices check with DC 20 using burglary tools disables the wire.

A character without burglary tools can still attempt the check with -1d6 using a sharp weapon or tool. If the check fails, the trap activates.

When the trap is triggered, the net is released covering an area 3 meters square on each side. All creatures in the area are ensnared by the web and are restrained, while those that fail a Fortitude save, with a Strength modifier, at DC 13 are also knocked prone.

A creature can use 2 Actions to make a DC 13 Strength check, freeing itself or another creature within reach on a successful one.

The net has Defence 10 and 20 Hit Points. Dealing 5 slashing damage to the net destroys a 1m square section of it, freeing any creatures trapped in that section.

\medskip

\textbf{Rolling Orb}

Mechanical trap

When 5kg or more are placed on the trap's pressure plate, a hidden trap door in the ceiling opens, releasing a 3m-diameter sphere made entirely of stone.

With a successful DC 20 Survival check, a character can see the trapdoor and pressure plate. If an examination of the floor is accompanied by a successful DC 20 Survival check, it will reveal the presence of the pressure plate by the difference in floor texture that accommodates it. The same check performed while checking the ceiling will reveal the presence of a trap door. Wedging an iron spike or other object under the pressure plate will prevent the trap from activating.

Activating the sphere causes all creatures present to roll for initiative. The sphere rolls initiative with a +8 bonus.

During its round, the sphere moves 18 meters in a straight line. The sphere can move through a creature's space, and creatures can move through the space it occupies, treating it as difficult terrain.

Whenever the sphere enters a creature's space or a creature enters its space while the sphere is rolling, the creature must succeed at a DC 15 Reflex save or take 55 (10d10) bludgeoning damage and fall prone.

The sphere stops when it hits a wall or similar barrier. It can't turn corners, but skilled dungeon builders incorporate slight curving twists and turns into nearby passageways that allow the sphere to keep moving.

As a 2 action, a creature within 1 meter of the sphere can attempt to slow it with a successful DC 20 Strength check. If successful, the sphere's speed is reduced by 5 meters. If the ball's speed drops to 0, it stops moving and is no longer a threat.

\medskip

\textbf{Collapsed Ceiling}

Mechanical trap

This trap uses a wire to collapse the supports that hold up an unstable section of ceiling.

The wire is placed 7 centimeters above the ground and extends between the two supports. The DC (Survival) to notice the wire is 13. A successful Disable Devices check with DC 20 using burglary tools disables the wire.

A character without burglary tools can still attempt the check with -1d6 using a sharp weapon or tool. If the check fails, the trap activates.

Anyone who inspects the props can easily deduce that they are just props. As an action, you can drop a prop and activate the trap.

The ceiling above the cavetto is in bad shape, and anyone who can see it can tell that it is in danger of collapsing. When the trap is triggered, the unstable ceiling collapses. All creatures in the area below the unstable section must make a DC 20 Reflex save, taking 22 (4d10) bludgeoning damage on a failed save, or half as much damage on a successful one. Once the trap is triggered, the floor of the area is filled with rubble and becomes difficult terrain.


\medskip

\textbf{Fire Breathing Statue}

Magic trap

This trap is activated when an intruder steps on a hidden pressure plate, unleashing a burst of magical flame from a nearby statue.

The DC (Survival) to notice the pressure plate or burn marks on the floor and walls is 20. A spell or other effect that can sense the presence of magic, such as detect magic, reveals a magical aura of invocation around the statue.

The trap is activated when more than 10 kilos of weight are placed on the pressure plate, causing a 10m cone of fire to shoot from the statue. All creatures in the cone must make a DC 17 Reflex save, taking 22 (4d10) fire damage on a failed save, or half as much damage on a successful one.

Sticking an iron spike or other object under the pressure plate prevents the trap from activating. A DC 20 Disable Device check (and requires 3 Arcana) disables the trap. A dispel magic (DC 17) cast on the statue destroys the trap.

\medskip

\textbf{Enchantment Traps and Dispel Magic}\index{Enchantment Traps and Dispel Magic}

The above traps may come with a spell that activates with the trap.
The saving throws to resist the spell are the same as for the spell cast by object or as indicated in the trap description.

A Dispel Magic cancels the enchantment on the trap if it has a Challenge Rating of 2 or less and disables its magical effect for 10 minutes if it has a Challenge Rating of 3.
An Advanced Dispel Magic cancels the enchantment on the trap if it is CR 4 or lower and disables its magical effect for 10 minutes if it is CR 5.


\bigskip

\subsubsection{More examples of traps}

Additional traps are presented here for your delight.


\medskip

\textbf{Little legend}:

Challenge Rating: Indicates the trap's challenge rating

Type: whether the trap is mechanical or magical

DC Survival: what is the check and difficulty to reveal the trap

DC Deactivate Devices: what is the check and difficulty to deactivate the trap.

Activator: whether it is activated by contact or distance

Reset: Whether the trap can be reset once it has been sprung

Effect: What is the effect of the trap


\medskip


\textbf{Poison Dart}

Challenge rank: 1

Type: mechanical

DC Survival: 20

DC Disable Devices: 20

Activator: contact

Recovery: none

Effect: Ranged Attack 12 meters +10 (1d3 damage plus Fermented Slime of Lucos)


\textbf{Arrow}

Challenge rank: 1

Type: mechanical

DC Survival: 20

DC Disable Devices: 20

Activator: contact

Recovery: none

Effect Ranged attack 13 meters +15 (1d8+1/×3)


\textbf{Pit}

Challenge rank: 1

Type: mechanical

DC Survival: 20

DC Disable Devices: 20

Trigger: location

Reset: manual

Effect 3m-deep pit (2d6 fall damage)

ST: Reflex DC 20 avoid

Target: Multiple targets (all targets 3m radius)


\textbf{Cutting Blade}

Challenge rank: 1

Type: mechanical

DC Survival: 20

DC Disable Devices: 20

Trigger: location

Reset: manual

Effect melee attack +10 (1d8+1/×3)

Target: Multiple targets (all targets in a line within 3 meters)


\textbf{Pit with Spikes}

Challenge rank: 2

Type: mechanical

DC Survival: 20

DC Disable Devices: 20

Trigger: location

Reset: manual

Effect 10' deep pit (1d6 falling damage) + spikes (Melee Attack +10, 1d4 spikes per target for 1d4+2 damage each)

ST: Reflex DC 20 avoid

Target: Multiple targets (all targets in a 3m square)


\textbf{Burning Wave}

Challenge rank: 2

Type: magical

DC Survival: 26

DC Disable Device/Arcana: 4/26

Trigger: Proximity (Alarm)

Recovery: none

Effect: 2d4 fire damage

ST: DC 11 Reflexes halve

Target: Multiple targets (all targets in a cone of 6 meters long and 3 meters final)


\textbf{Javelin}

Challenge rank: 2

Type: mechanical

DC Survival: 20

DC Disable Devices: 20

Trigger: location

Recovery: none

Effect: Ranged attack 13 meters +15 (1d6+6), within 6 meters radius


\textbf{Acid Arrow}

Challenge rank: 3

Type: magical

DC Survival: 27

DC Disable Device/Arcana: 4/27

Trigger: Proximity (Alarm)

Recovery: none

Effect: Attack ranged 15 meters (2d4 acid damage for 4 rounds)


\textbf{Hidden Pit}

Challenge rank: 3

Type: mechanical

DC Survival: 25

DC Disable Devices: 20
Trigger: location

Reset: manual

Effect medium deep pit (3d6 falling damage)

ST: Reflex DC 20 avoid

Target: Multiple targets (all targets in a 3m square)


\textbf{Electric Arc}

Challenge rating: 4

Type: magical

DC Survival: 25

DC Deactivate Device/Arcana: 20/3

Activator: contact

Recovery: none

Effect: Arc flash, 4d6 electricity damage

Save: DC Reflexes 20 halve

Target: multiple targets (all targets in a line at a distance of 6 meters)


\textbf{Wall Scythe}

Challenge rating: 4

Type: mechanical

DC Survival: 20

DC Disable Devices: 20

Trigger: location

Recovery: automatic

Effect melee attack +20 (2d4+6)


\textbf{Falling Block}

Challenge rating: 5

Type: mechanical

DC Survival: 20

DC Disable Devices: 20

Trigger: location

Reset: manual

Effect: Melee Attack +15 (6d6)

Target: Multiple targets (all targets in a 3m square)

\textbf{Fiery Strike}

Challenge rank: 6

Type: magical

DC Survival: 30

DC Disable Device/Arcana: 30/5

Trigger: Proximity (Alarm)

Recovery: none

Effect: 8d6 fire damage, range 3 meter

ST: DC 17 Reflexes halve

Target: Multiple targets (all targets in a 3m radius cylinder)

\textbf{Poisoned Arrow}

Challenge rank: 6

Type: mechanical

DC Survival: 20

DC Disable Devices: 20

Trigger: location

Recovery: none

Effect Ranged attack 20 meters +15 (1d6 plus poison ×3)


\textbf{Cold Fangs}

Challenge rating: 7

Type: mechanical

DC Survival: 25

DC Disable Devices: 20

Activator: position

Duration: 3 rounds

Recovery: none

Effect 10' range (spray of frozen water, 3d6 cold damage)

Save: DC Reflexes 20 halve

Target: multiple targets (all targets in a 3x3x3 meter room)


\textbf{Gas Trap}

Challenge rating: 8

Type: mechanical

DC Survival: 25

DC Disable Devices: 20

Trigger: location

Recovery: Repairable

Effect: Poisonous gas

Target: multiple targets (all targets in a 3x3x3 meter room)


\textbf{Arrow Volley}

Challenge rank: 9

Type: mechanical

DC Survival: 25

DC Disable Devices: 25

Trigger: Visual ( Arcane Eye)

Recovery: Repairable

Effect Ranged Attack +20 (6d6)

Target: Multiple targets (all targets in a 6m line)


\textbf{Concealed Pit with Spikes}

Challenge rating: 8

Type: mechanical

DC Survival: 25

DC Disable Devices: 20

Trigger: location

Reset: manual

Effect 15m deep pit (5d6 falling damage) + spikes (+15 melee attack, 1d4 spikes per target for 1d6+5 damage each)

ST: Reflex DC 20 avoid

Target: multiple targets (all targets in a cube with side 3x3x3 meters)


\textbf{Dazzling Floor}

Challenge rank: 9

Type: magical

DC Survival: 26

DC Disable Device/Arcana: 4/26

Trigger: Proximity (Alarm)

Duration: 1d6 rounds

Recovery: none

Effect: Melee touch attack +9, 4d6 shock damage

Target: multiple targets (all targets in a 6x6x3 meter room)

\textbf{Energy Drain}

Challenge rank: 10

Type: magical

DC Survival: 34

DC Disable Device/Arcana: 34/5

Trigger: Visual (True Seeing)

Recovery: none

Effect: Ranged touch attack 18 meters +10, max Hit Points drop by 10d4 + fatigued.

ST: Fortitude DC 23 negates after 24 hours


\textbf{Room of Blades}

Challenge rank: 10

Type: mechanical

DC Survival: 25

DC Disable Devices: 20

Trigger: location

Duration: 1d4 rounds

Recovery: Repairable

Effect melee attack +20 (3d8+3)

Target: Multiple targets (all targets in a 3x3x3 meter room)


\textbf{Cone of Ice Shards}

Challenge rating: 11

Type: magical

DC Survival: 30

DC Disable Device/Arcana: 30/5

Trigger: Proximity (Alarm)

Recovery: none

Effect: Cone of ice spears, 15d6 cold damage

ST: DC 17 Reflexes halve

Target: Multiple targets (all targets in a cone of 18m long and 6m trailing)


\textbf{Deadly Spear}

Challenge rank: 18

Type: mechanical

DC Survival: 30

DC Disable Devices: 30

Activator: visual

Reset: manual

Effect Ranged Attack 16 meters +20 (1d8+6 plus poison)


\textbf{Hellfire}

Challenge rank: 13

Type: magical

DC Survival: 31

DC Disable Device/Arcana: 5/31

Trigger: Proximity (Alarm)

Recovery: none

Effect: 60 fire damage

ST: Reflexes DC 14 halve

Target: Multiple targets (all targets in a 6m radius burst)


\textbf{Crushing Boulder}

Challenge rank: 15

Type: mechanical

DC Survival: 30

DC Disable Devices: 20

Trigger: location

Reset: manual

Effect: Melee Attack +15 (16d6)

Target: Multiple targets (all targets in a 3m square)


\textbf{Enhanced Attack}

Challenge rank: 16

Type: magical

DC Survival: 33

DC Disable Devices: 33

Trigger: Visual (True Seeing)

Recovery: none

Effect: +9 touch ranged 20 meters, 30d6 damage, Save: Fortitude DC 19 reduces to 5d6 damage


\textbf{Gallery of Lightning}

Challenge rank: 17

Type: magical

DC Survival: 29

DC Deactivate Devices: 29

Trigger: Proximity (Alarm)

Duration: 1d6 rounds

Recovery: none

Effect: 8d6 electricity damage)

Save: Reflexes DC 16 halve

Target: All targets in a 12x3x3m corridor


\textbf{Poisoned Pit}

Challenge rank: 12

Type: mechanical

DC Survival: 25

DC Disable Devices: 20

Trigger: location

Reset: manual

Effect Pit 15m deep (5d6 falling damage) + spikes (+15 melee attack, 1d4 spikes per target for 1d6+5 damage each plus poison)

ST: Reflexes DC 25 avoid

Target: Multiple targets (all targets in a 3x3 meter square)


\textbf{Meteor Storm}

Challenge rank: 19

Type: magical

DC Survival: 34

DC Disable Devices: 34

Activator: visual

Recovery: none

Effect 4 meteors to separate targets, +9 touch 30 meters range, 2d6 impact plus 6d6 fire damage

ST: Reflex DC 23 halves fire damage

Target: Multiple targets (four targets, two of which cannot be more than 12m apart)


\textbf{Destruction}

Challenge rank: 20

Magical type

DC Survival: 34

DC Disable Devices: 34

Activator: Proximity (Alarm)

Recovery: none

Effect: Death Saving Throw

ST: Fortitude DC 23 reduces damage to 5d12 otherwise 10d12

\end{multicols}

\medskip

\begin{changemargin}{0.3cm}{0.3cm}\begin{tcolorbox}[title = Tups and the trap]{\small
In this example I bring you the old school approach when it was assumed that there were traps. Nothing prevents the Arbiter from allowing Survival checks or Disable Devices. I can only say that this approach is more engaging though.

\medskip

\textit{Arbiter}: A 3m-wide corridor leads north into darkness.

\textit{Tups}: We prowl the floor with our 3m pole.

\textit{Arbiter}: The pole was left stuck in the collision with the stone idol.
[\textit{If he had used the pole the trap would have been discovered easily}.]
Do you continue in the corridor?

\textit{Tups}: No, I'm suspicious. Can I see some cracks in the floor, perhaps square in shape?

\textit{Arbiter}: No, there are millions of cracks, you can't see a pit that clearly[\textit{Arbiter assesses the pit is well camouflaged and Tups has poor lighting to see well}]

\textit{Tups}: Ok, I'll take my flask of water from my backpack. I'm going to pour some water on the floor. Does it seem to dig into the floor somewhere or reveal some form of texture?

\textit{Arbiter}: Yes, the water seems to flow around a square shape, slightly raised off the floor.

\textit{Tups}: Looks like a covered grave?

\textit{Arbiter}: Could be

\textit{Tups}: can I disable it?

\textit{Arbiter}: How?[\textit{The Arbiter deliberately does not make a check, but involves the player}]

\textit{Tups}: I jam the crowbar into it so the mechanism doesn't open the trapdoor[\textit{Tups doesn't ask you to roll a die to figure out how to disarm or disarm it directly, he just explains to the Arbiter how he does it }]

\textit{Arbiter}: You walk through the area now safely and see that it opens into a small room with two reinforced wooden doors... }

\medskip

Freely inspired by \href{https://friendorfoe.com/d/Old%20School%20Primer.pdf}{\textbf{Quick Primer for Old School Gaming}}

\end{tcolorbox}\end{changemargin}

\begin{changemargin}{0.3cm}{0.3cm}\begin{narrator}
A "visible/obvious" trap forces players to interact with it, make an effort to understand how it works, and strive to avoid or deactivate it. When you can avoid resolutions based only on the die roll (Looking for traps/Disabling traps), rather reward the player's ingenuity, even simple but creative, to avoid the danger... and maybe sooner or later they will remember to retrieve the crowbar. ..!
\end{narrator}\end{changemargin}

\pagebreak

\subsection{Optional - Reputation and Fame}\index{Reputation}\index{Fame}\index{Optional - Reputation and Fame}


\begin{changemargin}{0.3cm}{0.3cm}\begin{emphasis}{
Fame and honor sometimes come more easily to those who don't seek them. (Livy)}\end{emphasis}\end{changemargin}\medskip

\begin{multicols}{2}

While some heroes settle for the rewards of their exploits or hide behind a veneer of humility, others seek to live forever in the sagas and songs of their epic exploits. History measures a hero's success with tales of triumph and daring, repeated for generations.

A hero who cannot tell anyone his story is soon forgotten, along with his untold efforts. The story of the prodigious deeds becomes the yardstick by which a hero is measured, and sculpts both his identity and his reputation.

Reputation represents how the general public positively or negatively perceives the character. This perception precedes him, speaks for him in his absence and determines how he will be treated by those who have heard of him. Reputation does different things for different character types, based on the social and cultural values of different regions. A character who embodies the qualities of a hero in one region might be considered depraved or lewd in another. An icon widely revered and respected in her homeland could slip from fame to oblivion if she travels to a neighboring kingdom.

When using these reputation rules, the Arbiter must establish what reputation means to the players and NPCs in the campaign. For example, a Viking-themed campaign might base its reputation on plunder.

If one manages to acquire a strong or notable reputation, one might be commended for one's actions and rewarded with resources superior to those obtainable from lesser-known individuals. Similarly, reputation can be used to influence people socially, politically, or economically.

Renown rises and falls based on your actions. Current Renown determines overall reputation, Sphere of Fame defines places where reputation benefits can be applied.

\end{multicols}

\textbf{Table: how to acquire Fame points}\index{Table how to acquire Fame points}

\medskip

\begin{tabularx}{0.95\textwidth}{lX}
\textbf{Events}&\textbf{Mod. Fame}\\
\toprule
\textbf{Positive Events}&\\
Acquiring a remarkable treasure from a worthy opponent&+1\\
Consecrate a temple to your Patron&+1\\
Create a Powerful Magic Item&+12\\
Level Up&+1\\
Detect and disarm three or more traps with appropriate CR in a row&+1\\
Make a noteworthy historical, scientific, or magical discovery&+1\\
Own a legendary item or artifact&+14\\
Receiving a medal or similar honor from a public figure&+1\\
Returning a significant Magic Item or relic to its owner&+1\\
Looting a powerful noble's stronghold(enemy)&+1\\
Defeat in single combat an enemy with a CR higher than your level&+15\\
As a group win a fighting match with an APL +3 plus&+1\\
Defeat a public slanderer in combat&+2\\
Pass a Profession check with DC 30 or more to create a work or object &+2\\
Pass a public Intimidate check with DC 30 or more (must be witnesses)&+2\\
Pass public Perform check with DC 30 or more (must be witnessed)&+2\\
Complete an adventure with a difficulty appropriate to your level&+3\\
Obtain a formal title (lady, lord, knight, etc.)&+3\\
Defeat a key (campaign) rival in combat&+5\\
\textbf{Negative Events}&\\
Being convicted of a minor crime&-1\\
Accompanying an unseemly person&-18\\
Being convicted of a serious non-violent crime&-2\\
Publicly fleeing an encounter with a weaker opponent&-3\\
Attacking innocent people&-5\\
Being convicted of a serious violent crime&-5\\
Publicly losing a match to a weaker opponent&-5\\
Being convicted of murder&-8\\
To be convicted of treason&-10\\
\end{tabularx}

\bigskip

\begin{multicols}{2}

\subsubsection{Fame}

You begin the game with a Fame equal to your character level + your Charisma modifier. Fame ranges from -100 to 100, with 0 representing lack of notoriety.

Throughout the campaign, words and deeds help build a reputation. While an adventurer accomplishes many feats, not all of them are significant enough to warrant a change of Fame. If possible, the Arbiter should stick to those deeds that directly affect the story or campaign, and not award points for secondary victories.

The significance of a specific deed should be at the Arbiter's discretion, but Table: Events of Fame provides some examples. If Fame drops below 0, see Discredit and Infamy below.

\subsubsection{Sphere of Notoriety}

A character's reputation travels hand in hand with the tale of her exploits. Though he is a great hero in his own land, when he travels elsewhere he will soon find that his reputation dwindles and that sooner or later he will arrive in regions where he is completely unknown. The higher the reputation, the wider the affected area is.

Renown determines the maximum radius of the Sphere of Reputation. The Sphere of Fame has a radius of 150 kilometers, and usually increases by another 150 kilometers when Renown reaches 10, 20, 30, 40 and 55.

Increasing the Sphere of Notoriety is not always automatic, one can express an opinion on where one's reputation is concentrated. For example, you might request that your sphere extend further south to a large city and ignore barbarian tribes to the east, or that it extend inland to another country instead of out over the ocean.

While reputation may spread by accident, it usually does so on purpose, as wandering Arbiters embellish stories of a character's deeds to make them more entertaining, his allies amplify the most common deeds, his enemies repeat gossip about him to hire others and fight it, or the character himself tells his story to pleased listeners.

Where these tales are told determines where you will be known and creates a Sphere of Renown: a heroic sorceress might hire Bards to boast of her magic in a neighboring kingdom she plans to visit, while an antagonistic Barbarian might push south the wounded survivors of his raids, to spread fear among his next victims.

The following actions and conditions affect your Charisma, Diplomacy, or Intimidate check modifier for the purpose of expanding your Sphere of Reputation.

\medskip

\textbf{Table: Sphere of Fame Modifiers}\index{Table Sphere of Fame Modifiers}

\end{multicols}

\medskip

\begin{tabularx}{0.95\textwidth}{Xl}
\textbf{Action}&\textbf{Modifier to Trial}\\
\toprule
Allies or minions spread stories of your exploits before your arrival & + 5 \\
A Bard spreads stories or songs of the PC's deeds before his arrival & + 1/2 Bard's Entertainment Score \\
You have contact with the NPCs of the settlement&+1\\
You have enemies in the settlement&+1\\
Distance from own Sphere of Fame&-1 per 15 kilometers\\
The main language of the settlement is different from your own&-5\\
\end{tabularx}

\begin{multicols}{2}

\subsubsection{The Level of Fame}

\begin{itemize}[leftmargin=*]


\item The Fame score is what makes the character popular.

\item A fame score within 10 points will make him a local, small town hero.

\item A fame score between 10 and 20 points will make him a public figure, known to everyone in a small town or a neighborhood hero in a big city.

\item A score between 20 and 30 points makes the character known to everyone even in a big city, his deeds are also known in the region, perhaps not with all the details.

\item A score between 30 and 40 is a real celebrity in his city, known by name even in neighboring cities and respected throughout the region.

\item A fame between 40 and 50 points makes the character a true eminence respected in the state.

\item A score over 55 points make the character a legend whose deeds are handed down and magnified in the centuries to come.

\end{itemize}

\subsubsection{Discredit and Infamy}


If your Fame drops below 0, your reputation is based on infamy rather than fame. Treat Renown as a positive number instead of a negative number for all rules related to Renown, Sphere of Fame, and Prestige Points (for example, a villain's Fame of -20 equals a hero's Fame of 20 for his admirers.


In the event that an event would increase your Renown, you may choose to increase your Renown (bringing it closer to 0) or decrease it (making it a larger negative number). For example, if a character's Fame is 20 and you publicly roll a 30 on a Profession check to create a sword (which is usually worth +2), you can either increase your Fame to 18 or decrease it to 22.

Negative events that decrease Renown always count as negative (an adversary attacking innocent people does not inspire sympathy from the public).

If you have negative Fame, non-evil NPCs will often have ill-disposed or hostile reactions (see Table: Negative Fame Reactions). Note that if you have a reputation for being powerful and dangerous, NPCs may avoid you rather than confront you.


\end{multicols}

\textbf{Table: Negative Fame Reactions}\index{Table Negative Fame Reactions}

\medskip

\begin{tabularx}{0.95\textwidth}{lX}
\textbf{Fame}&\textbf{Reaction}\\
\toprule
-5&Merchants, mercenaries, and innkeepers charge an additional 10\% to the PC to discourage them from doing business in their community.\\
-8&Merchants, mercenaries and innkeepers refuse to do business. The PC entering a shop is immediately asked to leave. If he refuses, the owner calls the authorities or fellow citizens to throw him out.\\
-10&When the PC approaches, the shops close their windows and bar their doors. Most citizens refuse to talk to him. Others urge him to leave immediately. If he stays longer than 24 hours or acts blatantly against the citizens, his Renown decreases by 5 and the citizens rally to drive the PC away.\\
-15&Inflamed by the PC's shameless audacity to present himself in the community, an angry mob gathers. If the PC does not leave within a few minutes, the mob begins to bombard him with rotten fruit, branches and stones.\\
-20&An angry mob forms immediately after the PC enters the city. Not wanting to wait for a potentially corrupt trial, they try to capture and execute him for his crimes.\\
-25&An authority figure has issued an edict of arrest against the PC, including a reward for anyone who catches him. This is well known and many want to collect it.\\
-30&An authority figure has put a bounty on the PC's head. This is well known and many want to collect it.\\

\end{tabularx}

\vfill

\begin{center}
\includegraphics[keepaspectratio,width=0.55\textwidth]{immagini/Eastern_Story_Teller_1878.png}

\textit{Legends are told. Travelers in the Middle East Archive, Wilhelm Gentz}
\end{center}

\pagebreak

\section{Poisons, Potions and Disease}\index{Poisons}\index{Potions}\index{Disease}

\label{veleni-e-pozioni}


\begin{changemargin}{0.3cm}{0.3cm}\begin{emphasis}{
One day, a man was shot with a poisoned arrow. Anxious friends and relatives called a doctor. When they approached him to take the arrow, the man said to them: "Before I do so, I would like to know who pierced me with this arrow... Was he a slave, a king, or a brahmin? Was he big? Small? Of what color was his skin? Where did he live? And how was the arrow made? What poison was used? ..."

While he was asking himself all these questions... the poison took its effect and the wounded man ended up dying. (Buddha)
}\end{emphasis}\end{changemargin}\medskip


\begin{multicols}{2}

\subsection{Type of poison and potion}\label{tipidiveleno}

\lettrine[lines=2, lhang=0.33, loversize=0.25, findent=1.5em]{P}{oisons} and potions can be distinguished based on how you come into contact with them.
Not all poisons are toxic when ingested or inhaled.

To identify a natural potion, a check of Herbalism is required at DC 12 + the rarity of the plant or, in the case of poisons, the difficulty is equal to the Saving Throw of the same. It costs 1 Action every 10 of DC or with Herbalism 6 or more it costs 1 Action every 15 DC and with 12 points it costs 1 Action every 20 DC. Potions unless otherwise described must be drunk (ingested).

\textbf{Touch (C)}: Contracted the moment someone touches the poison with bare skin. Contact poisons usually have an onset time of 1 round. A contact poison can be an ointment, balm, liquid of any density, or even powder if specified for contact and not inhalation.

\textbf{Ingested (I)}: Trigger when a creature eats or drinks them. Ingested poisons usually have an onset time of 10 minutes.

\textbf{Wounding (F)}: Transferred mainly with the attacks of some creatures and with weapons doused with poison. Wounding poisons usually have an instantaneous onset time.

\textbf{Inhalation (R)}: Activate the moment a creature enters an area containing such poisons. Many inhaled poisons fill a volume equal to a cube with an edge of 3x3x3 meters per dose. Creatures may attempt to hold their breath while within the area to avoid inhaling the toxin.
A creature can hold its breath for 6 rounds per its Constitution score, with a minimum of 3 rounds, and each Action decreases the remaining time by 1 round.
After the time has elapsed they must make a Fortitude save at difficulty 12 each round to avoid inhaling the gas. Each round you hold your breath, the difficulty check increases by 1.
See also the rules for holding your breath and choking in \hyperlink{trattenereilfiato}{Environment}

\subsection{Onset and Effect}\index{Onset and Effect of Poison}\index{Activating time for poison}\label{insorgenzaveleno}

By onset we mean how long it takes for the poison or potion to take effect. If the onset time is 1 Turn it means that for the effects of the poison/potion and the Saving Throw it is done after 10 minutes. If it is not specified in the onset poison/potion table, it means that the effect is immediate after coming into contact with the poison.

The effect of a poison/potion is immediate after onset. Check the description of the poison to understand its effect. If the Fortitude save succeeds, the poison has had no effect and can be considered neutralized.

There are some cases in which the Frequency item is present, in these rare occasions the Saving Throw must be repeated every time the indicated Frequency passes, in case of failure of the Saving Throw the indicated damages are re-applied.

\begin{center}
\includegraphics[height=0.4\linewidth]{immagini/potion.png}
\end{center}

\begin{changemargin}{0.3cm}{0.3cm}\begin{narrator}
The poisons proposed here are some of the many present and possible. Use them as guidelines. If for your ethics and style you don't like poisons, especially the nastier ones, I suggest you use the Generic Potions that you find at the end of the chapter. They are milder and less personal poisons, probably more easily usable by players as well.
\end{narrator}\end{changemargin}

\subsubsection{Poisoned}\index{Poisoned}\label{avvelenato}

\textbf{First Dose}: When exposed to a poison for the first time (either during your action or someone else's), you must make a Saving Throw within the onset to avoid being poisoned.

\textbf{Success}: Poison is resisted. You take no ill effects, and no further Saving Throws are required.

\textbf{Failure}: You have been poisoned and immediately suffer the listed effect.

\textbf{Multiple doses}: If exposed to multiple doses of the same poison in the same round, the difficulty of the Saving Throw increases by 1 per additional dose.\index{Poison more doses}

\textbf{At different times}: If you are exposed to the poison at different times, each time there will be a new Saving Throw and you will suffer any effects on schedule.

If you are exposed to different poisons, you must make a Saving Throw for each type of poison taken.

\begin{changemargin}{0.3cm}{0.3cm}\begin{tcolorbox}[title = Poison ?]
{Poison is a double-edged sword. As long as you use it it's fine but if they use it against you, maybe the same, it becomes a problem. There are also ethical aspects to using poisons, consider whether your Traits allow you to use poisons and what types.
}\end{tcolorbox}\end{changemargin}

\subsection{Apply Poison}\index{Apply Poison}\label{applicareveleno}

Applying poison to a weapon or ammunition takes 3 Actions.

Each time a character applies or prepares a poison for use he must roll 3d6+Intelligence and if roll a critical faiulure he has come into contact with the poison and must make a Saving Throw against the poison as normal. This does not consume the dose of poison.

Whenever a character attacks with a poisoned weapon, if he rolls a critical failure  on his attack roll, he is exposed to the effects of the poison. This consumes the poison on the weapon.
A potion of poison is enough to cover a medium weapon or 3 arrows with poison. The poison is thus consumed and remains active on the weapon until it hits.

A creature under the effects of a poison, whether already unleashed or not, has the poisoned condition.

\subsection{Remove Poison}

The \textbf{Remove Poison} spell removes poisons, and therefore the poisoned condition, which have not yet taken effect as long as the DC of the poison is lower than the DC of the Remove Poison spell. If the DC of Poison isn't specified the casto of Remove Poison spell's is enough to dispel the effect of poison.

Each Magic Critical obtained with the Magic Test in casting the spell is equivalent to +4 in the calculation of the DC see (\hyperlink{spell save}{Saving Throws - Resist the spell}, page \pageref{magietirosalvezza}) to overcome the DC of the poison.

A First Aid check\index{First Aid and Poisons}\index{Poisons and First Aid}, which is at least half the DC of the poison within the onset time, allows you to make a new saving throw. Once the check has been done, it is no longer possible to do it again until after the onset.
A continuous treatment of First Aid for 8 hours allows you to make a new saving throw after the activation of the poison.


\subsection{Create Natural Poisons}\index{Create Natural Poisons}\label{crearevelenonaturale}

Natural poisons can be crafted using Herbalism. The DC to brew a poison is equal to the DC of a Fortitude save which requires -5. If you buy the ingredients, the cost to prepare the potion is half of the indicated selling cost, if you look for them in nature, the production cost drops to a quarter. The time to prepare these potions/drugs is equal to DC/2 in hours.

A critical failure on Herbalism check exposes you to the poison during its preparation. If the DC Herbalism check is successful, 1d2+1 doses are prepared.

The following examples represent just a few of the possible poisons. All costs are expressed in Gold Coins.

Poisons are presented, especially in the Monstruarium, with this wording: Poison Name, Use (I/R/F/C), Onset time, DC of Saving Throw, Effect.

\begin{center}
\includegraphics[height=0.35\linewidth]{immagini/poison.png}
\end{center}

\begin{changemargin}{0.3cm}{0.3cm}\begin{narrator}
Poisons are part of the long tradition of trouble and adversity in RPGs. When you want to use a poison think first of all why it is there, for whom it was to be used, for what purpose. Not all poisons have to kill, a skilled thief could also use poisons that stun or weaken the will of his target just enough to get the safe opened.
\end{narrator}\end{changemargin}

\end{multicols}

\vfill

\begin{center}
\includegraphics[width=0.4\linewidth]{immagini/funeralebarca.png}
\end{center}



\textbf{Table: Poisons}\index{Table Poisons}\label{tabellaveleni}

\medskip

\begin{tabularx}{1\textwidth}{m{4.5cm}lllm{6.5cm}l} %{XlllXl}
\toprule
\textbf{Poison Name} & \textbf{Use} & \textbf{ST} & \textbf{Ins.} & \textbf{Effect (damage)} & \textbf{MO}\\
\toprule
Purple Berry of Barsar\index{Purple Berry of Barsar} & I & 18 & 1 Turn & Incapable of violence for 3d8 hours & 40 \\
\toprule
Ditch Blue Berries \index{Ditch blue berries} & I & 21 & 1 Turn & -1d3 Intelligence and Wisdom for 6 hours & 55\\
\toprule
Fermented Slime of Lucos \index{Fermented Slime of Lucos}& F & 15 & - & 1d8 Hit Points & 25\\
\toprule
Yellow Bark Ash \index{Yellow Bark Ash} & F & 15 & 6 rounds & Unconscious for 1d3 hours & 25\\
\toprule
Purple Concentrate \index{Purple Concentrate} & F & 15 & & 2d6 Hit Points & 15\\
\toprule
Fingers of Daraka\index{Fingers of Daraka} & F & 17 & - & -1d6 Strength, for 1 hour & 35\\
\toprule
Pink Spike Grass \index{Pink Spike Grass} & I & 22 & 1 Turn & -1d6 Dexterity, for 1 hour & 60\\
\toprule
Purple Shrew Liver \index{Purple Shrew Liver} & I & 25 & 1 hour & 2d6 damage to Wisdom and Intelligence. Permanent & 75 \\
\toprule
White Mucot Flake \index{White Mucot Flake} & C & 20 & - & Sleeps for 2d12 hours & 20\\
\toprule
Fumes of Curna\index{Fumes of Curna} & R & 18 & - & -1d3 Wisdom & 40\\
\toprule
Blue Frost \index{Blue Frost} & F & 18 & & 3d6 Cold Hits & 25\\
\toprule
Purple Shrew Fat \index{Purple Shrew Fat} & C & 13 & 1 round & 2d12 Hit Points & 15\\
\toprule
Tongue of Kreex \index{Tongue of Kreex} & F & 20 & - & The wound bleeds. +1 bleed damage. 1 use in 24 hours. & 50 \\
\toprule
Red Mixture \index{Red Mixture} & F & 13 & - & -1d6 AR/ST for 10 minutes & 10\\
\toprule
Yellow Moss \index{Yellow Moss}& I & 20 & 1 round & creature gains a bounty. -2 Int and Sag. Duration 10 minutes & 50\\
\toprule
Hazel of Dennar \index{Hazel of Dennar} & I & 13 & 1 Turn & -1d2 Strength, for 3d & 15\\
\toprule
Oil of Nabar \index{Oil of Nabar} & R-F& 20 & - & Confused for 2d6 rounds & 50\\
\toprule
Azure Toad Skin \index{Azure Toad Skin} & C & 22 & 1 minute & Paralyzed for 1d6 rounds & 60\\
\toprule
Omro's Rose Pollen\index{Omro's Rose Pollen} & I & 15 & - & -1d3 Constitution and Dexterity, for 1 hour & 25\\
\toprule
Scent of Ragmor \index{Scent of Ragmor} & R & 16 & - & -1d3 Charisma, for 1 day & 30\\
\toprule
Blood of Thrun \index{Blood of Thrun} & C & 26 & - & -1d3 Constitution & 80\\
\toprule
Juice of Ythis\index{Juice of Ythis} & I & 14 & 1 Turn & -1d2 Intelligence, per 1g & 20\\
\toprule
Poison of Octalm\index{Poison of Octalm} & F & 20 & - & Death or -1d2 Permanent Constitution & 50\\
\toprule
Blood Serpent Venom \index{Blood Serpent Venom} & F & 25 & - & Paralysis for 1d6 hours -1d4 Strength for 7 days & 75 \\
\end{tabularx}

\medskip

\textbf{Application}: \textbf{I}(gestion), \textbf{F}(ertion), \textbf{C}(ontouch), \textbf{R}(exhalation).

The Saving Throw is always Fortitude unless specified otherwise

Lost ability points recover at a rate of 1 per day unless non-permanent or otherwise noted.


\begin{center}
\includegraphics[width=0.20\linewidth]{immagini/mandragola2.png}

\textit{Mandrake plant}
\end{center}

\subsection{Natural Potions}\index{Potions}\label{pozioninaturali}

\begin{changemargin}{0.3cm}{0.3cm}\begin{emphasis}{
I believe that a leaf of grass is no less than a day's work done by the stars. (Walt Whitman)
}\end{emphasis}\end{changemargin}


\begin{multicols}{2}

The time to prepare these potions/drugs is equal to the DC/2 in hours, while the difficulty of the Herbalism check is equal to the DC -5. If you buy the ingredients, the cost to prepare the potion is half of the indicated selling cost, if you look for them in nature, the production cost drops to a quarter.

If the DC Herbalism check is successful, 1d2+1 potions are prepared (from 1 dose).

You cannot benefit from more than one dose of natural potions (of any type) per day, unlike magical ones.

\end{multicols}

\medskip
{\small
\begin{xltabular}{0.95\textwidth}{llllXlc}
\textbf{Name} & \textbf{Usage} & \textbf{Ins.} & \textbf{DC} & \textbf{Effect}& \textbf{Loc.} & \textbf{Cost} \\
\toprule
Arduuar\index{Arduuar} & I & 1 round & 25& Remove Poison & SZ7 & 75 \\
\toprule
Arkasun\index{Arkasun} & C & 1 Turn & 25& Heals 1d6 Hit Points per Turn for 3 rounds& TM7 & 75 \\
\toprule
Arlan\index{Arlan} & C & 5 rounds & 15& Heals 1d6+3 Hits & TT5 & 50 \\
\toprule
Arlandas\index{Arlandas} & R & 1 hour& 24& Heal the fractures & CF5 & 200 \\
\toprule
Attarna\index{Attarna} & I & 1 Turn & 20& Grants a new Disease save with a +1d6 & TF7 & 50 \\
\toprule
Berries of Ljust \index{Berries of Ljust} & I & 1 round & 16& Taken in the evening you recover double the minimum Hit Points 4) & AZ6 & 10 \\
\toprule
Ljust's Kiss\index{Ljust's Kiss} & C & 1 round & 35& Heal 100 Hit Points& HO8 & 500 \\
\toprule
Barannie\index{Barannie} & I & 1 minute & 15& Removes nausea & MD6 & 3 \\
\toprule
Burthelas \index{Burthelas} & I & 1 Turn & 32& Regenerate hands& HD7 & 410 \\
\toprule
Dagmathir Bark\index{Dagmathir Bark} & R & 1 round & 25& Removes one level of Fatigue & SS5 & 15 \\
\toprule
Bark of Aklent\index{Bark of Aklent} & I & 1 Turn & 10& The bark chewed for at least 10 rounds grants a +1 Saving Throw vs poison for the next 24 hours & TM6 & 1 \\
\toprule
Culcoa\index{Culcoa}& C & 1 round & 16& You recover 2d6 from fire damage & ST7 & 15 \\
Darsirion\index{Darsirion} & C & 1 round & 25& Heals 1d4 Hit Points& MP4 & 5 \\
\toprule
Delrean Plus\index{Delrean Plus} & I & 1 round & 18& Drives away bugs for 3 days & CC6 & 5 \\
\toprule
Delrean\index{Delrean} & C & 1 round & 15& Drives away bugs for 1 day & CC6 & 2 \\
\toprule
Draaf \index{Draaf} & C & 1 round & 20& Heals 1d8 Hit Points& SO6 & 50 \\
\toprule
Eldrin'tail\index{Eldrin'tail}& I & 1 round & 15& Grants a new poison save & FH7 & 18 \\
\toprule
Illa Berry Extract\index{Illa burned Berry Extract}& I & 1 round & 15& +2 Initiative, +2 Dexterity, -1d6 Will save, for 10 minutes & MS6 & 5 \\
\toprule
Gisenosa root extract\index{Gisenosa root extract} & I & 3 rounds & 15& Cure cold and cough & TM6 & 3 \\
\toprule
Febfendi \index{Febfendi}& C & 1 Turn & 25& Regenerate Ears & CF7 & 75 \\
\toprule
Garioe\index{Garioe}& I & 1 round & 25& Heals 2d6 Hits& AZ7 & 95 \\
\toprule
Geffnull \index{Geffnull}& I & 5 rounds & 28& Heals 3d8+3 Hit Points & EV8 & 150 \\
\toprule
Gusterbloon \index{Gusterbloon} & C & 1 round & 20& Skin becomes darker granting +1d6 to Hide check & CM5 & 8 \\
\toprule
Gylvert\index{Gylvert} & I & 1 minute & 25& Grants breathing underwater for 4 hours & MO7 & 3 \\
\toprule
Harfy \index{Harfy} & C & - I & 12& -1 to bleed & SS6 & 3 \\
\toprule
Harfindar\index{Harfindar} & I & 1 Turn & 15& Abort& SS7 & 3 \\
\toprule
Jojopo\index{Jojopo}& C & 1 round & 15& You recover 2d6 cold damage & FM6 & 18 \\
\toprule
Kelventare\index{Kelventare} & I & 1d4 rounds & 28& Heal 2d6 Hit Points & TT7 & 100 \\
\toprule
Klagul\index{Klagul}& C & 1 Turn & 20& Cleans teeth & SS4 & 2 \\
\toprule
Klandor\index{Klandor} & I & I & 15& Removes paralysis. Increases fatigue level by 1& HB6 & 18 \\
\toprule
Klynkyx\index{Klynkyx} & C & 6 Turn & 15& Loses all hair for 1d6+4 days & MO6 & 4 \\
\toprule
White Musk Yeast \index{White Musk Yeast} & I & 1 minute & 12& Baked products using this yeast cause uncontrollable bloating and incredibly smelly for 12 hours & CA3 & 1 \\
\toprule
Red Tongue of Xabax\index{Red Tongue of Xabax}& C & 1 Turn & 20& Heals 2d6 Hit Points but if there is disease or poison removes it causing 2d6 HP of damage & Save7 & 13 \\
\toprule
Melandrir\index{Melandrir} & I & 1 round & 15& Grants a new disease save with +5 & CF7 & 100 \\
\toprule
Mirenna\index{Mirenna} & I & 1 round & 20& Heal 5 Hit Points & CM6 & 30 \\
\toprule
Blend 31\index{Blend 31}& I & 1 Turn & 20&The mount is extremely durable. +6 hours of galloping per day & SM6 & 15 \\
\toprule
Silvermoss\index{Silvermoss}& I & I & 25& Remove magical diseases & MU8 & 250 \\
\toprule
Musekiss\index{Musekiss} & C & 1 hour& 30& Regenerate lower limbs & TH9 & 550 \\
\toprule
Nazamuse \index{Nazamuse}& I & I & 30& Removes Poisons and Natural Diseases & EW9 & 175 \\
\toprule
Nelthalion \index{Nelthalion} & I & I & 15& Makes vomit& SR3 & 1 \\
\toprule
Lisbeth's Petals \index{Lisbeth's Petals} & I & 1 Turn & 15&+2 Intelligence, -2 Dex for 10 minutes & MC6 & 20 \\
\toprule
Green Rose Pollen\index{Green Rose Pollen}& R & 3 rounds & 25& Heal 2d4 damage Intelligence and Wisdom & FA8 & 35 \\
\toprule
Dry Root of Kathaus\index{Dry Root of Kathaus} & R & 1 round & 20& +2 Strength and Dexterity for 1 hour & FW6 & 50 \\
\toprule
Rewky\index{Rewky} & I & 1 Turn & 25& Heals 2d8 Hit Points& TD6 & 20\\
\toprule
Siranmuse\index{Siranmuse} & I & 1 day & 30& Regenerate internal organs & SS8 & 850 \\
\toprule
Ucsaboo \index{Ucsaboo} & C & 1 Turn & 30& Regenerate Eyes & MO8 & 400 \\
\toprule
Urk Egg\index{Urk Egg}& I & 1 Turn & 12& 1 day's food& FH7 & 1 \\
\toprule
Uscaboo \index{Uscaboo} & R & 1 Turn & 25& Remove blindness & MO7 & 125 \\
\toprule
Wickalim\index{Wickalim} & I & 1 hour & 15& Heals 2 Hits & TD3 & 5 \\
\toprule
Yaveth\index{Yaveth}& I & 1 Turn & 20& Heals 2d8 Hit Points& MO5 & 100 \\
\end{xltabular}}


\subsubsection{Notes on Poisons and Potions}

\textbf{Purple Shrew Liver}: poisoning recognizable by the typical bloodshot eyes


\textbf{Lucos fermented slime}: Lucos is a herbivorous and peaceful lizard. The collected slime must be fermented in the oxen for 1 week before being usable.

\textbf{Purple Shrew}: according to many, the Shrew is Cattalm's favorite pet. Aggressive, violent, dangerous in every fiber.

\textbf{Daraka Fingers}: Daraka Fingers are the fruit of the Daraka tree. The elongated black pod is reminiscent of the fingers of the ancient goddess of darkness

\textbf{Nabar oil}: the small Nabar berries are exclusively eaten by Shrews, immune to their evil effects. Boiled for a long time it becomes an excellent ointment for the skin.

\textbf{Urk egg}: Urk is a large beetle, the egg is little bigger than a hazelnut. It is usually first smoked with beech wood, eaten raw the taste is moldy and earthy.

\textbf{Mixture 31}: A studied mix of drugs for horses. Once the effect ends, the creature must make a DC 23 saving throw or fall unconscious for 12 hours.

\textbf{Barsar's Purple Berry}: curiosity the purple Shrew is disgusted by these berries.

\textbf{Poison of Ottalm}: the Ottalm is a variant of the purple Shrew endowed with a venomous sting

\textbf{Yellow Bark Ash}: the bark is first macerated and beaten in water and salt. The resulting mush is dried and then heated without burning it directly

\textbf{Red Tongue of Xabax}: it is the long petal of the Xabax. Of the 7 petals, only the long one has the substances necessary to prepare the ointment.

\textbf{Kathaus dry root}: small, extremely hard and woody black tuber. It is usually left to dry in the sun before grinding it

\textbf{Lisbeth's petals}: extremely fragrant, they resemble those of roses

\textbf{Flumes of Curna}: the Curna is the inflorescence of the milk thistle

\textbf{Aklent Bark}: also called \textit{Skunk Bush} due to its pungent and characteristic odor.

\textbf{Gisenosa root extract}: thistle-like plant, extremely thorny. It tends to grow surrounded by the \textit{Tribulus terrestris} or "footkisser".

\textbf{Silvermoss}: very similar, for a non-expert, to White Musk. The berries are picked.


\subsection{Where to find plants}

Ex: Gusterbloon FT5. The first letter indicates the CLIMATE, the second indicates the ENVIRONMENT, the third indicates the RARITY. The rarity indicates the possibility, on a d10, of finding the herb/plant sought. Roll 1d10 and do more than the number indicated, clearly if there is a match of climate and environment.

\textbf{Table: Correspondence Potions - Places}\index{Table Correspondence Potions - Places}

\medskip

\begin{tabular}{ll|ll|ll}
\textbf{1st letter} & \textbf{Climate} & \textbf{2nd letter} & \textbf{Environment} & \textbf{2nd letter} & \textbf{Environment} \\
\toprule
A & Arid & A & Alpine & B & Gorges\\
C & Cold & C & Coniferous Forest & D & Deciduous Forest\\
E & Perennial ice & F & River, stream embankments & G & Frozen fields\\
F & Severe Cold & H & Dry Fields &J & Jungle, Rain Forest\\
H & Wet \& Hot & M & Mountain & N & Ocean, Salt Flats\\
M & Temperate & S & Shortgrass & T & Tallgrass\\
S & Semi Arid & U & Caves \& Dungeons & V & Volcanic\\
T & Cool Temperate & W & Landfill, Waste & Z & Desert\\
X & Unknown & X & Unknown&&\\
\end{tabular}

\subsection{Generic Potions}\index{Generic Potions}\index{Potions}\label{pozionigeneriche}

The Arbiter is free to use all of the above potions and poisons, or use ready-to-use generic potions that can be purchased at most herbal or potion shops.

The table shows the costs and effects of these potions. The onset is always immediate, the duration for cures is immediate, for the others it is 1 hour (therefore the Remove Poison potion "immunises" you for 1 hour against a poison). For potions that cause damage, the Saving Throw is to negate their effects.

\textbf{Table: generic potions}\index{Table generic potions}

\medskip

\begin{tabularx}{0.95\textwidth}{lXcc}
\textbf{Potion Name}& \textbf{Effect}& \textbf{Cost (gp)}& \textbf{Application}\\
\toprule
Heal & recovers 1d8+1 Hit Points & 50 & Ingested\\
Empowered Heal & recover 3d8+3 Hit Points & 125 & Ingested\\
Weakening & -1d6 AR. Save DC 15 Fortitude & 34 & Ingested\\
Empowered Weakening & -1d6 AR. Save DC 18 Fortitude& 50 & Wounding \\
Poison & you take 2d6+2 damage. Save DC 15 Fortitude & 30 & Ingested \\
Empowered poison & you take 2d8+2 damage. Save DC 18 Fortitude & 25 & Wounding \\
Remove Poison & grants a new Saving Throw with +1d6 & 75 & Ingested\\
Generic Potions & Level Spell*Level Spell*50mo &&Ingested\\
\end{tabularx}

These generic potions like natural potions only take effect the first time they are taken within 24 hours. If the character dedicates 1 minute to drinking a Healing Potion it will have maximized effect.\index{Potions maximized effect}

The \emph{Generic Potions} serve as a purchase example for potions not listed. See also the \hyperlink{creatingpotions}{Creating Potions} section.


\subsection{Optional - Drugs}\index{Drugs}\index{Optional - Drugs}\hypertarget{droghe}{}\label{droghe}

\textbf{Table: Drug List}\index{Table Drug List}

\medskip

\begin{tabularx}{0.99\textwidth}{llllXrr}
\textbf{Name} & \textbf{Usage} & \textbf{Ins.} & \textbf{DC} & \textbf{Effect}& \textbf{Loc.} & \textbf{Cost} \\
\toprule
Fermented leaves of Luside\index{Fermented leaves of Luside} & I & 1 Turn & 17& Sensory hallucinations for 2d4 hours. +2 Charisma, Intelligence & SF7 & 5 \\
\toprule
Ferpillon \index{Ferpillon}& I & 1 round & 20& Sleep for 24 hours& SC5 & 50 \\
\toprule
Gray Grease \index{Gray Grease} & I & 1 round & 24& Removes mental conditioning caused by spells of level 5 or lower& AH9 & 80 \\
\toprule
Ash of Arpasur \index{Ash of Arpasur} & R & 1 round & 20& Removes 2 levels of fatigued & FT6 & 10 \\
\toprule
Purple Shrew Dried Meat \index{Purple Spider Dried Meat} & I & 1 round & 24& +4 Strength -4 Intelligence (minimum -3) for 1 Turn& SH7 & 30 \\
\toprule
Alcoholic Extract of Melzaa\index{Alcoholic Extract of Melzaa} & I & 1 round & 20& +1d4 Strength, +1d4 Dexterity. -1d6 Will save. For 3 hours & AF6 & 25 \\
\toprule
Perfumed Essence of Inut\index{Perfumed Essence of Inut} & R & I & 15& +2 Intelligence, for 1d8 hours& HB6 & 15 \\
\toprule
Julnnaus Pollen\index{Julnnaus Pollen} & R & I & 20& +3 Constitution for 2 hours & ST6 & 25 \\
\toprule
Erain's Flower Pollen \index{Erain's Flower Pollen} & R & 1 round & 20& +2 Strength and Intelligence and Dexterity. +3d6 Temporary Hit Points, for 1 hour & FT7 & 75 \\
\end{tabularx}

\begin{multicols}{2}

\medskip

\textbf{The use of drugs is completely optional, it is the Arbiter who decides their presence and availability also based on the sensitivity of the players}.

Drugs are addictive. Once the effect ends within 24 hours, make a Will save at difficulty 15 or take another dose, the next Saving Throw will have difficulty +1 and so on.

Each time you take a new dose within 2 weeks of the first, the Saving Throw against becoming addicted increases by 1. Not taking a dose increases your Fatigue level by one.

It takes 7 successful Saving Throws in a row to end the addictive effect.

\subsection{Optional - Drinking too much}\index{Drinking too much}\index{Optional - Drinking too much}\hypertarget{alcoholism}{}\label{drugs}\index{Drunks}

A creature can drink a number of mugs of ale equal to its Constitution score. Any new mug forces the creature to make a Fortitude saving throw at DC 11, each subsequent mug increases the saving throw by +2. When the saving throw fails, the creature is drunk and is considered to be under the effects of the confusion spell.

Higher strength beers or spirits require a more difficult saving throw.

Narratore could handle heavy driker as only roleplaying aspect.

\subsection{Diseases}\index{Diseases}\hypertarget{diseases}{}\label{diseases}

In principle, diseases are managed like poisons, a Saving Throw is performed to check if the disease has taken and therefore will take effect.
Usually the triggering time of a disease is not as immediate as a poison and yet the magical ones can be disruptive and act in a few minutes.

A disease requires a saving throw to completely avoid taking the disease and then multiple successful saving throws in succession to heal once the first fails.

Each disease must have indicated the time of onset, the initial saving throw, how often the saving throw must be redone, how many successes on the Saving Throw are needed to heal, the effects suffered.

Ex. Lesser Demonic Fever: 1 minute, save Fortitude DC 18, 6 hours, 3 successes, -1 Constitution and Wisdom.

Lesser Demonic fever forces a DC 18 Fortitude save after only one minute of taking it. Subsequently every 6 hours the saving throw must be redone and the disease persists until at least 3 consecutive successes have been made on the TS. Every 6 hours the sufferer loses 2 points to Constitution and Wisdom.

To recover from a non-natural disease, such as those afflicted by monsters, it is necessary to pass the required Saving Throws or have a \hyperlink{remove disease}{Remove Disease} spell available (page \pageref{rimuovimalattie}).

A \textbf{First Aid}\index{First Aid and Disease}\index{Disease and First Aid} test, with DC equal to at least half the DC of the disease (or 15 if not indicated), carried out between one Saving Throw and the next, allows you to have a +1 to the Saving Throw to resist the effects of the disease.

The remove disease spell grants healing from the affliction as long as the spell's DC is higher than the disease's DC.

Each Magic Critical obtained with the Magic Testin casting Remove Disease equals +4 in the calculation of the DC (see \hyperlink{magietirosalvezza}{Saving Throw Resist the spell}, page \pageref{magietirosalvezza}) to overcome that of the disease.

Being affected several times by the same disease does not increase the difficulty of healing or change the times and effects of the disease.

Examples of Diseases:\\

\textbf{Daemonic Influence}: 1 minute, save Fortitude DC 16, 12 hours, 2 successes, -1 Constitution/12 hours\index{Demon Influence}

\textbf{Rezh's Corruption}: 1 day, Will save DC 18, 1 hour, 2 successes, -1d6 Max Hit Points\index{Rezh's Corruption}

\textbf{Funginean Plague}\index{Funginean Plague}: 8 hours, TS Fortitude DC 24, 12 hours, 2 successes, -1 point to Dexterity and Intelligence

\textbf{Violent Torpor}\index{Violent Torpor}: 24 hours, Will save DC 12, 12 hours, 1 success, +1 Damage with Melee Weapons and -1 Wisdom

\textbf{Lesser Demonic Fever}\index{Lesser Demonic Fever}: 1 minute, save Fortitude DC 18, 6 hours, 3 successes, -1 Constitution and Wisdom

\textbf{Black Blood}\index{Black Blood}: 10 minutes, save Fortitude 28, 12 hours, 1 success, loss of half Hit Points remaining

\textbf{Pox T}\index{Pox T}: 1 minute, save Fortitude 30, 2 hours, 3 success, if not save three time in a row trasform in a zombi

\end{multicols}

\vfill

\begin{center}
	\includegraphics[width=0.20\linewidth]{immagini/plaguedoctor.png}
	
	\textit{Engraving of the Plague Doctor, Paul Furst, 1656}
\end{center}


\pagebreak

\section{Movement and Transport}\index{Movement}\index{Transport}

\label{movimento-e-trasporto}

\begin{changemargin}{0.3cm}{0.3cm}\begin{emphasis}{
My left foot works great, but I still wouldn't be able to walk without my right foot! (Madagascar 3 - Europe's Most Wanted, Film )

\medskip

When you can't run anymore, walk fast; when you can no longer walk fast, walk; when you can no longer walk, use the cane; but never hold back. (Mother Teresa of Calcutta)}

\end{emphasis}\end{changemargin}\medskip


\begin{multicols}{2}

\lettrine[lines=2, lhang=0.33, loversize=0.25, findent=1.5em]{M}{ovement} can be distinguished based on which situation it applies.

\medskip

\begin{itemize}[leftmargin=*]
\item Tactical, when fighting, use precise distances, grid and 1m squares
\item Local, for exploring an area, measured in meters per minute.
\item Overland, to move from one place to another, measured in km per hour or per day.
\end{itemize}

\subsection{Types of Movement}\label{tipodimovimento}

When moving in different movement situations (Tactical, Local Overland), creatures generally walk or run.

\textbf{Walking}:\index{Walking} Walking represents an unhurried but purposeful movement of approximately 4 km per hour for a human with no Encumbrance.

\textbf{Running}\index{Running}: Means moving about 12 km per hour for a human.

The running character suffers a 1d6 penalty on attack rolls and 4 on Defence in the round in which he runs.
Running as a move action doubles your movement speed, not triples it. Only in non-combat situations does running triple movement (local, land movement)

\subsection{Table: Movement \& Distance \& Speed: Walk}\index{Movement by walking}\index{Table Movement \& Distance \& Speed: Walking}

This table shows the basic ground movement values in non-combat situations.

\bigskip

\begin{tabularx}{0.43\textwidth}{lccc}
\multirow{2}*{Type of movement} &
\multicolumn{3}{c}{Movement} \\
\cmidrule(lr){2-4} & 6m & 9m & 12m \\
\midrule
\multicolumn{4}{c}{\textbf{Tactical Movement)}}\\
Walking & 6m & 9m & 12m \\
Run (x2) & 12m & 18m & 24m \\
\multicolumn{4}{c}{\textbf{One minute (local)}} \\
Walking & 36m & 54m & 72m \\
Run (x3) & 108m & 162m & 216m \\
\multicolumn{4}{c}{\textbf{One hour (by land)}} \\
Walking & 3km & 4km & 6km \\
Running (x3) & 9km & 12km & 18km \\
\multicolumn{4}{c}{\textbf{One day (by land)}} \\
Walking & 24km & 32km & 54km \\
\end{tabularx}


\subsection{Tactical Movement}\index{Tactical Movement}\label{movimentotattico}

During combat, you use Tactical Movement.
Distances are measured in squares of one meter, movement is managed through Move Actions.

A character can use 1 (Move) Action to move up to his full movement. He can perform the Move Action several times in the round, up to 3 times, thus moving three times as much as he does.

He can also take a Sprint Action \index{Sprint Action} or travel twice as far as he does in a single Action. However, he runs into penalties for running (-1d6 to hit, -4 Defence).

A character can perform up to 3 Sprint Actions, i.e. runs for the whole round thus covering his movement * 6.

\subsubsection{Difficult terrain}\index{Difficult terrain}\label{terrenodifficile}

Difficult terrain, obstacles or poor visibility can impede movement. When movement is hindered, you move at half speed, 2 Actions are needed to cover your distance of 9 meters (if you're human without encumbrance..). Or with a move Action you only cover 4 meters.

If there is more than one particular condition, add all applicable additional costs to each other, i.e. if terrain is difficult and crawling means moving a quarter of your movement.

In some situations movement is so hindered \index{Impossibile movement} that the distance that can be covered per Action is minimal, in which case all 3 Actions can be used to move only 1 meter in any direction.

Do not apply this rule to cross impassable terrain or to move when there is no way to do so.

You cannot \textbf{Sprint} (Run) or \textbf{Charge} \index{Charge on difficult terrain} \index{Sprint on difficult terrain}easily through a \textbf{path that hinders movement}, i.e. difficult terrain. The player can attempt an Acrobatics check at DC 20 to successfully charge or run, however only traveling half the distance. The Acrobatics check is not necessary, even if you complete half the movement, if you have the Advantage \hyperlink{rinoceronte}{Ryno}.

\textbf{Moving Prone}\index{Moving Prone}\index{Move on all fours
}, Swimming or Crawling\index{Crawling} is considered difficult terrain.

Terrain where creature bodies are present is considered difficult\index{Muoversi su corpi}.

\subsubsection{Through enemies}\index{Through enemies}\index{Through occupied zone}\label{attraversonemici}

\index{Passing through enemies}\index{Movement through}A character can \textbf{cross} but not stop in \textbf{an area occupied} by a companion without being \textbf{restricted}{\textbf{restricted}}\index {Restricted}. If the movement leads him to \textbf{share}\index{Share the square} the same map square (assuming both medium-sized creatures) will have a -1d6 on the attack roll and a -4 on the Defense.

To cross terrain where there is a hostile creature it is necessary to perform an Opposed Dexterity or Strength Test (your choice) with the opposing creature whose terrain you want to cross. \textbf{Crossing the space occupied by an enemy costs 1 Action}.
\textbf{Crossing the space occupied by an enemy costs 1 Action}. Crossing the space occupied by an enemy costs 1 Action and it is considered difficult terrain.

If you fail the check, you return to the nearest free square and both the movement and the overpass actions are considered finished.

If the enemy has the \hyperlink{opportunist}{Opportunist} Skill, in addition to obstructing the passage, he can perform an attack (1 Reaction).

A medium-sized or smaller creature can share the same square with a small-sized creature.


\subsubsection{Exchange places}\index{Exchange places}\label{scambiarsidiposto}

A character in contact with another creature can use \textbf{an Action} to \textbf{swap places}, if the creature is hostile an opposed Fortitude save is required to be able to swap places. For each \textbf{size difference}, whoever has the larger one gets a +1d6 bonus on the check.\index{Exchange places}. It costs 1 Action.

\begin{changemargin}{0.3cm}{0.3cm}\begin{tcolorbox}[title = Tups in the tunnel] %box giocatore
Tups is with his companions in a narrow tunnel in single file. It is in fourth position.

Suddenly an enemy steps in front of him and Tups is the fastest to react, using a Move Action \textit{\textbf{cross}} the 3 companions in front of him remaining \textbf{restricted} with the first in line.

He could decide to (among the various options):

- don't move, delay the actions until it is possible to move normally

- stay restricted and attack

- push the partner into the previous square, making him squeeze with another partner

- push your partner to the next square! causing him to cross the enemy's square

- go back to its starting square

-try to cross the opponent, but if he fails he would have only 1 more Action left and would be restricted to his partner, damaging both

\end{tcolorbox}\end{changemargin}

%\begin{changemargin}{0.3cm}{0.3cm}\begin{narrator} %box narratore
%Se volete un crudo realismo allora è terreno difficile attraversare anche zone dove ci sono creature amichevoli. \end{narrator}\end{changemargin}

\subsubsection{Pass through bottlenecks}\index{Pass through bottlenecks}

Passing through a gap one size smaller is equivalent to moving through difficult terrain. For example, a medium creature, occupying 1 square, that must pass through a half-square (half-meter) narrow passage treats that path as difficult terrain.

You can't pass through restrictions narrower than half the creature's size.


\subsection{Local Movement}\index{Local Movement}\label{movimentolocale}

Characters exploring an area use local movement, measured in meters per minute.

In these situations it is not essential to measure the distance precisely but as soon as the situation becomes "problematic" or requires attention, the map is converted into tactical movement, squared and measured.

\medskip

\begin{itemize}[leftmargin=*]
\item
Walking: A character can walk without difficulty at a room scale for 8 hours per day.
\item
Run: A character can run for a number of minutes equal to three times his Constitution score on a local scale without needing to rest (minimum of one round).
\end{itemize}


\subsection{Ground Movement}\index{Ground Movement}\label{movimentoviaterra}

Characters who travel long distances use land movement. Overland movement is measured in hours or days. One day represents 8 hours of real travel time. For rowboats, a day means rowing for 10 hours. For sailing vessels it represents 24 hours of movement.

\textbf{Walking}\index{Walking}\label{camminare}

You can walk 8 hours in a day's journey without any problems.

Longer walking can wear you down (see Forced March, below).

\textbf{Go Fast}\index{Go Fast}\label{andareveloci}

You can go fast (movement*2) for 1 hour without any problems. Hustling for a second hour between two sleep cycles deals 1 point of non-lethal damage, and each additional hour causes double the damage taken in the previous hour. A character who takes non-lethal damage from a fast stride is considered fatigued.

A fatigued character cannot run or charge.

\textbf{Run}\index{Run}\label{correre}

You can't Run for a long time. Attempts to run and rest in cycles work like going fast.

\textbf{Ground}\index{Ground}\label{terreno}

The terrain you travel over affects how much distance you travel in an hour or day (see Table: Terrain and Overland Movement). A highway is a main road, straight and paved. A common road is usually an arduous path. A trail is like a common road except that it allows travel only in single file and does not benefit a group traveling in vehicles. Clear terrain is a wilderness area with no marked trails.

\bigskip

\textbf{Optional - Table: Terrain and Overland Movement}\index{Optional - Table Terrain and Overland Movement}

The table shows the multipliers for the distance travelled.

\medskip

\begin{tabularx}{0.45\textwidth}{XXXX}
\textbf{Terrain} & \textbf{Highway} & \textbf{Common road} & \textbf{Untrodden path}\\
\toprule
Moor & x1 & x1 & x3/4\\
Hill & x1 & x3/4 & x1/2\\
Sandy Desert & x1 & x1/2 & x1/2\\
Forest & x1 & x3/4 & x1/2\\
Jungle & x1 & x3/4 & x1/4\\
Mountain & x3/4 & x3/4 & x1/2\\
Swamp & x1 & \texttimes 3/4 & \texttimes 1/2\\
Plain & x1 & \texttimes 3/4 & \texttimes 1/2\\
Frozen Tundra & x1 & \texttimes 3/4 & \texttimes 3/4\\
\end{tabularx}

\bigskip

\textbf{Forced march}\index{Forced march}\label{marciaforzata}

On a normal walking day, one can walk for 8 hours. The rest of the day is used to make and unpack the camp, rest and eat.

If you walk for more than 8 hours, you must make a Fortitude save (difficulty 11+1) for each consecutive day of forced walking or you become fatigued. The Saving Throw is made every 2 hours after 8 hours of walking.

The forced march can be held for a number of days equal to the Constitution value +1 before incurring fatigue regardless of the result of the Saving Throw.


\textbf{Movement on the saddle}\index{Movement on the saddle}\label{movimentoacavallo}

A mount carrying a rider can move at a fast pace. However, the damage it takes is normal damage instead of non-lethal. She can also be forced into a forced march, but her Constitution checks automatically fail and again the damage she takes is normal damage. Mounts are also considered fatigued when they take damage from fast stride or forced march.

\end{multicols}

%\medskip
%\begin{center}
%\includegraphics[height=0.3\linewidth]{immagini/carretto.png}
%\end{center}

\subsection{Table: Mounts and Vehicles}\index{Mounts}\index{Vehicles}\index{Table  Mounts and Vehicles}\index{Horse movement}\index{Day movement on horse}

\medskip

\label{tabella-cavalcature-e-veicoli}\index{Dog}\index{Pony}\index{Cart}\index{Raft}\index{Ship}\index{Boat}

\begin{tabularx}{0.95\textwidth}{llXX}
	\textbf{Mount or Vehicle} & \textbf{Carried Encumbrance} & \textbf{Movement} & \textbf{Movement}\\
	&(CdC)&\textbf{Per hour} & \textbf{Per day}\\
	\toprule
	Gallop Dog* & 30 & 6km & 36km \\
	Light Horse* & 60 & 8km & 48km \\
	Heavy Horse* & 75 & 7km & 42km \\
	Pony* & 25 & 5km & 30km \\
	Donkey or Mule & 55 & 6km & 48km \\
	Camel & 40 & 8km & 48km \\
	Elephant & 160 & 6km & 36km \\
	& & & \\
	\textbf{Boat} & & & \\
	\toprule
	Raft or Barge (pole or trailer) & 1000kg & 0.75km & 7.5km \\
	Rowing Boat** & 2000kg & 1.5km & 15km \\
	Rowing Boat** & 1000kg & 2.25km & 22.5km \\
	Sailing Vessel & 4000kg & 3km & 72km \\
	Warship (sails and oars) & 8000kg & 3.75km & 90km \\
	Long Ship (sails and oars) & 1200kg & 4.5km & 108km \\
	Galley (oars and sails) & 15000kg & 6km & 144km \\
\end{tabularx}


\begin{multicols}{2}

\bigskip

\textbf{*Quadrupeds}, like horses, can carry higher loads than characters (x4). See Carrying Capacity for more information.

A mount can carry a creature only if it is smaller than itself. The movement per day is intended for 6 hours of riding, beyond these hours the mount gets tired requiring a full day of rest.\index{Hours of riding per day}

**Rafts, barges and barges are used on lakes and rivers. If they go with the current, add the speed of the current (usually 4.5 km/h) to the speed of the boat. In addition to being rowed for 10 hours, the craft can also float for another 14 hours, if someone can steer it, adding another 63 km to the daily distance traveled. These boats cannot be rowed against a very strong current, but they can be pulled against the current by pack animals on the shore.

\textbf{Mount Harnesses}\index{Mount Harnesses}\index{Horse armor}

A mount can be barded with Armour. Generally, light Armour will grant a +2 Defence bonus, Medium Armour will grant a +4 Defence bonus, reducing movement by 25\%, Heavy Armour will give +6 Defence, reducing movement by 33\%.

\subsection{Escape and Chase}\index{Escape}\index{Chase} \label{fugainseguimento}

In round-to-round movement it is impossible for a slow character to escape a fast character without some kind of help. Likewise, it's no problem for a fast character to escape a slower one.

When the speed of the two characters involved is equal, there is a fairly simple method for resolving a chase: if one creature is chasing another and both move at the same speed, and the chase continues for at least a few rounds, the pursuer and pursued make 3 consecutive Opposing Reflex saves.

Whoever wins the challenge manages to make them lose track or grab the fugitive.

In the case of a long chase where there is no possibility of hiding or losing track, make 3 opposed Fortitude saves to determine which side can maintain the pace longer. Whoever gets the most successes manages to escape or it is the pursuer who manages to reach it.

\subsection{Carrying and Loading Capacity: Overall Dimensions}\index{Carrying and Loading Capacity: Overall Dimensions}\index{Encumbrance}

\label{sec:capacita-di-carico-e-trasporto-ingombro}

\subsubsection{Weight and Dimensions}\index{Encumbrance}\index{Weight}

Carrying treasure, dragon pieces, full Armour not to mention oversized weapons or battering rams, pulleys and tackle, make movement difficult.

When you evaluate the weight carried, also think about the size!
Carrying a roll of 12 meters x 6 meters of silk is not a demanding physical activity, it will be a few kilos, but the size is such that it cannot allow further load.

There may be light but extremely bulky objects (hollow trunks, silk carpets, in fact...) or small but very heavy objects (mercury spheres, gold-woven clothes), for all these objects the weight value must also be reasoned according to the clutter.

Each object has its own Encumbrance value, in principle \textbf{every 3 kg there is 1 as Encumbrance factor}. This value can also become 5Kg if the object is easily transportable. The Encumbrance values of the objects add together to give the total load carried to compare with the Cargo Capacity of the creature.\index{Weight and Encumbrance}

Objects with low weight and volume have encumbrance \textbf{Light} (L). These items count as 1 Encumbrance for every 10. Every 500 coins has 1 Encumbrance.\index{Encumbrance of money}

\subsubsection{Carrying Capacity}\label{capacitadicarico}\index{Carrying Capacity}

A creature's Carrying Capacity is the sum of factors such as Size, Strength, and Constitution.

A creature's size grants a bonus to \textbf{CoC} (Carrying on Capacity) equal to 9 if Small, 16 if Medium, 25 if Large. A creature's Encumbrance when dragged is equal to half its Carrying Capacity plus encumbrance.\index{Encumbrance transported creature}

When the total CoC is exceeded then moving around and making Dexterity-based proficiency checks becomes problematic. You become weighed down and your movement range drops by half, and Dexterity-based Proficiency checks have a -3 penalty.

If the CoC is doubled then it is no longer possible to move due to the encumbrance of the weights carried.

\textit{Remember that worn Armour and shield have a Encumbrance half of their marked.}

Ex. Tups is wearing a Chain Ring (encumbrance 2 being worn), a long sword (medium weapon, encumbrance 2), a spiked mace (eng. 2), 18 light objects (eng. 1), a backpack (eng. 1), a tent (ing. 2), a lantern (eng. 1). Total Encumbrance = 11.

Tups is a Medium creature with Strength -1 and Constitution -1 (he's a bit puny and weak..) this gives his a Carrying on Capacity of 16-1-1 = 14.

The CoC of Tups is higher than he encumbrance but he has to be careful, maybe it's better if he leaves the tent on the horse ...


If the load is placed on a cart you can push it at full movement if within your CoC, at half movement if within double your CoC and at a quarter of the movement if within quadruple your CoC.

If multiple creatures push or pull a cart, consider the highest one as the CoC and add half of the other creatures. A chariot can be pushed by 1 creature +1 per chariot size above medium.


\subsubsection{Larger and Smaller Creatures}

The \textbf{Table: CdC transported based on size}\index{Encumbrance transported based on size} shows the Load Capacity based on size. The Strength and Constitution values ​​must be added to the value given by the size.
\medskip

\begin{tabularx}{0.45\textwidth}{ll|ll}
	\textbf{Cut} & \textbf{Eng.}&\textbf{Cut} & \textbf{Eng.}\\
	\toprule
	Very small &1/4& Large & 25\\
	Petite & 1 & Huge & 36\\
	Tiny & 4& Mammoth&49\\
	Small & 9 & Colossal&64\\
	Average & 16&&\\
\end{tabularx}


Creatures with 4 legs or more can carry larger loads.


%\begin{center}
%\includegraphics[height=0.5\linewidth]{images/horse.png}
%\end{center}

\textbf{Table: Transportation modifiers for multi-legged creatures}\index{Table of transportation modifiers for multi-legged creatures}

\medskip

\begin{tabularx}{0.45\textwidth}{ll}
	\textbf{Creature Paws} & \textbf{CdC}\\
	\toprule
	4 legs & x2\\
	6 legs & x2.5\\
	8 legs & x3\\
	12 legs & x4\\
	every other 2 legs & +0.5\\
\end{tabularx}

\medskip

These tables are to be used for unusual animals not listed or similar to those in the Mounts and Vehicles Table.


\subsection{Other Movement Types}

\label{altri-tipi-di-movimento}

\subsubsection{Swim}\index{Swim}\label{nuotare}

A creature with a swim speed can move through water at its listed speed without making Swim checks. You gain a +2d6 bonus on any Swim check to perform a particular action or avoid a hazard. The creature can always choose to take 10 on a Swim check, even if distracted or endangered when swimming. He can't take the 10 only in case of stormy waters. Such a creature can use the run action while swimming, provided it swims in a straight line.

Caster is considered Distracted while swimming.

If you don't have the Swim \textbf{move through water} move type, it's considered \textbf{difficult terrain} so you move at half the speed indicated by move. The Swim check is necessary every time you have to move, in case of failure you don't move and you have a -1 to the next check, in case of critical failure the next check takes a -1d6. When the check is less than 5 you begin to sink and drown.

See the Waer Adventures chapter for swimming and underwater combat rules (page \pageref{underwatercombat}).

\subsubsection{Climb}\index{Climb}\label{scalare}

A creature with a Climb speed has a +2d6 bonus on all Climb checks. If the creature must make a Climb check to climb any wall or slope, it can always choose to take 10, even if in a hurry or threatened during the climb.

If a creature with a Climb speed attempts a rapid climb (see above), it is as if it were taking a dash action and makes a single Climb check at a -5 penalty.

A creature has no penalty to Defence while climbing, and has no penalty to attack rolls while attacking.

If you don't have the type of \textbf{Scale movement} it is treated as \textbf{hindering terrain}, and therefore moves at half the speed indicated by movement.

\subsubsection{Digging}\index{Digging}\label{scavare}

A creature with a burrow speed can tunnel through earth, but not through rock unless the descriptive text says otherwise. Creatures can't charge or run while digging.

Most burrowing creatures leave no tunnels for other creatures to use (either because the material they burrow through fills the tunnel behind them or because they don't actually displace material when they burrow), see the individual creature's description for details.

\subsubsection{Walking - Ground Speed}

Land speed is the normal speed for characters who don't climb, swim, or fly.

\subsubsection{Fly}\index{Fly}\label{volare}

Flying for a creature with this ability is like walking for a "ground" creature. A creature with flight uses its actions to move but is unlikely to be affected by hindering terrain.


\end{multicols}

\

\vfill

\begin{center}
\includegraphics[width=0.5\linewidth]{immagini/grifonicastello.png}
\end{center}




\pagebreak

\section{Mastering}\index{Mastering}\index{Arbiter} 

\label{masterizzare}




\begin{changemargin}{0.3cm}{0.3cm}\begin{emphasis}{
Who commands the story is not the voice: it's the ear. (Italo Calvino)


\medskip

Dungeons \& Dragons RPG (\textit{and so OBSS}) is about storytelling in worlds of swords and sorcery. It shares elements with childhood pretend play. Like those games, D\&D is driven by the imagination. It's about imagining the imposing castle under the stormy night sky and imagining how a fantasy adventurer might react to the challenges the scene presents. (DnD 5e Basic Rules)\\

It's not the DM's (\textit{Arbiter}) job to entertain the players and make sure they have fun. Every person playing the game is responsible for the fun of the game. Everyone speeds the game along, heightens the drama, helps set how much roleplaying the group is comfortable with, and brings the game world to life with their imaginations. Everyone should treat each other with respect and consideration, too—personal squabbles and fights among the characters get in the way of the fun.

Different people have different ideas of what's fun about D\&D (\textit{OBSS}). Remember that the “right way” to play D\&D is the way that you and your players agree on and enjoy. If everyone comes to the table prepared to contribute to the game, everyone has fun. (Dungeon Master Guide 4e)

}\end{emphasis}\end{changemargin}\medskip

\begin{multicols}{2}

\label{il-narratore}

\subsection{The Arbiter}

\lettrine[lines=2, lhang=0.33, loversize=0.25, findent=1.5em]{W}{hile} the player plays a character in an adventure, the Arbiter is the one who manages it. He certainly has a lot more work, but creating an entire world for your friends to explore can be very satisfying.

The role of the Arbiter is not easy but grants enormous privileges. Seeing your friends playing, having fun, "going mad" behind doubts, riddles and situations you create gives a lot of fun and moments of true conviviality.

Your role is that of the great orchestrator, planner or even landscaper if you prefer, with a few simple brushstrokes you outline the structure and it will then be the players who add details and situations.

Your \textit{work and fun} is fundamental and very important, the quality of the game session depends also on you. Your aim is to entertain, involve but also terrify and ingenious.


\begin{changemargin}{0.3cm}{0.3cm}\begin{narrator}
OBSS wants to help you and other players have fun. Always use common sense when applying a rule. Your goal is not to kill characters but to create worlds and campaigns that evolve around the characters and worlds, their actions and decisions. Incorporate the things players care about, keep them involved, make them understand that the world is alive and they are a part of it. If you are good your adventures situations will echo in other sessions and off the table.
\end{narrator}\end{changemargin}

You are not the protagonist in the adventure, but the players, your friends, don't steal the show but like a great dance, be the conductor of the orchestra where the instruments are the possibilities offered by OBSS, the music is the adventure and the dancers the players.

In OBSS, Experience Points (XP) that characters earn are used to determine their level, and therefore their abilities and skills at their disposal.

Characters earn XP by defeating monsters, but also by achieving objectives, ideas, performing special actions, and even by finding treasure!

The main suggestion is to reward characters who have contributed the most to the success of the adventure and the session. XP doesn't just measure success, but also participation in the game. Therefore, it's possible to have characters with different XP and potentially even different levels.

The XP Table by Level shows the XP required to advance from one level to the next.

\medskip

\textbf{Table: Experience Points per Level}\index{Table of Experience Points per Level}\label{table of experience points}

\begin{tabularx}{0.45\textwidth}{lX|lX}
\textbf{Level} & \textbf{Experience Points}&\textbf{Level} & \textbf{Experience Points}\\
\toprule
1&0 &11&300000\\
2&2000 &12&390000\\
3&8000 &13&490000\\
4&15000 &14&600000\\
5&35000 &15&740000\\
6&60000 &16&890000\\
7&90000 &17&1050000\\
8&120000 &18&1250000\\
9&170000 &19&1470000\\
10&220000 &20&1730000\\
			&+prev*0.2&\\
\end{tabularx}

\medskip

XP earned from defeating a monster is listed in the Monster Manual, for example, Challenge Rating 13 (10,000 XP). These XP should be divided among all the characters who participated in the encounter in any way.

Never over-assign XP or you risk unbalancing the game and having to make significant changes to the adventure. Be clear with players at the start of the campaign, in Session Zero, about how XP will be calculated, distributed, and what they can do to earn more.

\medskip

However, you should not only consider the Experience Points granted by adversaries but also evaluate the characters and the group during the session.

Whenever a character or group:
\begin{itemize}[leftmargin=*]
\item \textbf{Achieves the set objectives} (reward for the group or the character);
\item \textbf{Fully utilizes and even alternates the use of their Feats and skills} (reward for the character);
\item \textbf{Solves problems in a creative, imaginative, and functional way} (reward for the character);
\item \textbf{Proposes plans and actions that are functional and alternative to what is expected} (reward for the character);
\item \textbf{Discovers or initiates adventure clues and creates new plots} (reward for the character);
\item \textbf{Uses a skill or item in an intelligent and cunning manner} (reward for the character);
\item \textbf{Uses a spell in a brilliant (and alternative) way} (reward for the character);
\item \textbf{Performs an action that jeopardizes their own life for the group} (reward for the character);
\item \textbf{Performs actions following the creed of their Patron (for Devouts). These should give Trait points} (reward for the character);
\item \textbf{Convert an NPC, of ​​equivalent level, to his Patron (for Devotees only)} (character reward);
\item \textbf{Collect at least 500*level in gold coins (or equivalent treasure)} (reward to the group, once per session maximum);
\end{itemize}

I also suggest that you evaluate these actions to reward the player's effort
\begin{itemize}[leftmargin=*]
\item \textbf{Be collaborative with other players} (group or character reward);
\item \textbf{Help a player in difficulty} (character or group reward);
\end{itemize}

Grant 200 Experience Points * Character Level.

A further approach, but to be considered only amongst the most cohesive and mature groups, is to ask the players at the end of each session to choose who among them played best, in a combination of role, inspiration, incisiveness and collaboration. Reward his character with the 200 PX*LV bonus. 

These Experience Points can be assigned to the group and therefore to all characters or to a single character.
There is no need to give Experience Points at the end of the game session, keep track of them and inform the players when there is a moment of pause, reflection on what happened and done.
In this system, about 6/10 sessions are needed to level up, potentially even much less if the players prove to be good and face situations brilliantly.

Build the session so that all the characters can participate and no one feels left out.

When I say "encounter" don't think just about fighting monsters, an encounter is any role-playing event that challenges and tests the characters. This challenge can be a witty discussion with the noble who does not want to pay them at the end of a mission, the challenge of a riddle, rebus, well-placed traps. Give experience points based on challenge difficulty.
A monster doesn't necessarily have to be killed to get its Experience Points, it's enough to defeat it, capture it, win in a different way. In case of retreat by the characters or the enemy, grant half the Experience Points foreseen for the fight if there has been at least an attempt to challenge.

As far as possible, each session should include a role part, an exploration part, three combat parts (even many more than three), a rest part.

\begin{changemargin}{0.3cm}{0.3cm}\begin{narrator}
It may seem anachronistic when the sixth edition of the most famous role-playing game is already in development to return to rewarding the characters based on the gold taken from the monsters.

However, I can guarantee you that if your group is particularly "poor" in role-playing or simply prefers a more combative style, knowing that the gold collected is equivalent to experience can make going on an adventure much more dynamic and exciting.

OBSS refers to the principles of the OSR and as such the exploration and combat phase has its own important and vital weight.
\end{narrator}\end{changemargin}

Let it be clear that nothing prevents you from giving level passage based on fixed points (milestones) during the adventure. Your the table, your rules!.


\subsection{Encounters}\index{Encounters}

\label{incontri}

\begin{changemargin}{0.3cm}{0.3cm}\begin{changemargin}{0.3cm}{0.3cm}\begin{emphasis}{What is life without hope? A throw of the dice in the darkness, in the delirium. Ambrogio Bazzero}

\end{emphasis}\end{changemargin}
\end{changemargin}

An encounter is a moment of tension and hope, fear and challenge. It is an opportunity to show and demonstrate one's skills and to work as a group.

An encounter is not an opportunity to show off one's absolute power, either as a Arbiter or as a Player. The Arbiter will \st{punish} be able to educate the player who wants to be beyond the group and not part of it.

You will find on the following pages the instructions for creating easy (0 experience points), medium and high (1 experience points), extraordinary (2 experience points) , Deadly (3 experience points) and epic (4 experience points) challenges.

In any case, it will always be you, the Arbiter, who establishes and knows if a challenge is trying or not, if it is challenging and critical for the players and therefore evaluate both its impact as experience points and as difficulty.

An encounter is an event that confronts characters with a specific problem they must solve. Many are fights with monsters or hostile NPCs, but there are other types: a corridor full of traps, a political interaction with a suspicious king, a dangerous passage over a rickety rope bridge, an awkward argument with a friendly NPC feels that a character has betrayed him, or anything that adds a bit of drama to the game.

Puzzles, role-playing challenges, and skill checks are the classic methods for solving encounters, but the most complex encounters to construct are the more common combat encounters.

A clash can also be born clearly unbalanced, it will be the foresight of the players to understand when to escape!

When planning a combat encounter, first decide what level of challenge you want the PCs to face, then follow the steps outlined below.


\begin{center}
\includegraphics[width=0.7\linewidth]{immagini/deathbeowulf.png}

\textit{Henry Justice Ford}
\end{center}

\textbf{Determine APL}: \index{APL}Determine the average level of the characters: this is the Average Party Level (APL for short, Average Party Level). You should round this value to the nearest whole number (this is one of the few exceptions to the round down rule).

Note that this reference to creating an encounter assumes a party of four or five PCs. If your group has six or more players, add one to their average level. If your party contains three or fewer players, subtract one from their average level. For example, if your party consists of six players, two 4th level and four 5th level, the APL is 6th (28 total levels, divided by six players, rounding up and adding one to the final result). .

\textbf{Determine Challenge Rating}: Challenge Rating (or Challenge Rating, CR) is a convenience number used to indicate the relative risks presented by a monster, trap, hazard, or other encounter: plus the higher the Challenge rating, the more dangerous the encounter. Refer to Table: Determining Encounters to determine the Degree of Challenge your party should face, based on the difficulty of the challenge you want and the APL.

\medskip

\textbf{Table: Determining Encounters}\index{Table Determining Encounters}

\medskip

\begin{tabular}{ll}
\textbf{difficulty} & \textbf{Challenge Rating (CR)}\\
\toprule
Easy & APL\\
Medium & APL +2\\
High & APL +3\\
Extraordinary & APL +4\\
Deadly & APL +6\\
Epic & APL +8\\
\end{tabular}

\begin{changemargin}{0.3cm}{0.3cm}\begin{changemargin}{0.3cm}{0.3cm}\begin{emphasis}{
The essence of the world is play… we play the serious, we play the authentic, we play reality, work and struggle, we play love and death, and we even play the game. (Eugen Fink)				
}\end{emphasis}\end{changemargin} \end{changemargin}


\subsubsection{How many fights to face}\index{How many fights to face}\label{quantiincontri}

There is no single answer. It's your choice, the system finds its balance between 3 and 5 fights per day. Of course they don't all have to be on High difficulty!. Clashes are ultimately a management of resources to be used against an enemy. These resources are your Hit Points, spells, potions, and scrolls or consumable items you possess. If you place an Extraordinary encounter as the first encounter, it is probable that the players then decide to rest to recover their energy, otherwise you could opt to tire them slowly with medium encounters and then try them with a higher difficulty. Finally, remember that a \textit{clash} doesn't necessarily have to be physical, but also traps, puzzles/riddles, alternative challenges.. anything that consumes resources and makes you think.

Always evaluate where they move and what's around, it will be natural to find the right number and types of clashes and enemies.

\begin{center}
\includegraphics[width=0.75\linewidth]{immagini/impegnativa.png}

\textit{Henry Justice Ford}
\end{center}

\subsubsection{Building the Encounter}\index{Building the Encounter}\label{costruireincontro}

To build a match, first calculate the value of the APL.

To develop your encounter, add creatures, traps, and hazards until you get to your scheduled APL.

Start by calculating the challenges with the highest Challenge rating of the match, completing the rest with lower challenges.

For example, you want your group of six level 7 characters to have a Medium Challenge and face some Gargoyles (Challenge 2 rank each), Xorn (Challenge 5 rank) and their leader, a Stone Giant (Challenge Rank 7). Characters have APL 8 and the Table: Determine Encounters states that a Medium challenge for an APL 8 is a Challenge 10 rank match.

Starting from an established Challenge degree (10) follow this table to establish how many "monsters to insert in the fight".

\medskip

\textbf{Table: CR weight for APL calculation}\index{Table Challenge rating weight for APL calculation}

\medskip

\begin{tabularx}{0.45\textwidth}{XXX}
\textbf{Challenge target} & \textbf{Challenge creature versus APL target} & \textbf{"Weight" per single creature}\\
\toprule
Challenge Rating & -7/8 & 5\\
& -6 & 10\\
& -5 & 15\\
& -4 & 20\\
& -3 & 30\\
& -2 & 50\\
& -1 & 65\\
& 0 & 80\\
& +1 & 90\\
& +2 & 100\\
\end{tabularx}

\bigskip

\textbf{To achieve the goal we have to add "the weights" until we reach 100, or 100\% of the challenge.}

In our example, a Stone Giant has a Challenge rating of 7, i.e. a Challenge rating -3 compared to our goal difficulty a Challenge rating of 10, the Xorn has a Challenge rating of 5, or -5 compared to a Challenge rating of 10, the Gargoyles have Challenge rating 2 or -8 compared to Challenge rating 8.

An enemy with a -3 Challenge rating has weight 30, a -5 Challenge rating has a weight of 15, a -8 Challenge rating has a weight of 5.

To achieve the goal of a Challenge rank of 10 I will put 1 Challenge rank -3 (aka a stone giant), 3 Challenge rank -5 (aka three Xorns) and 5 Challenge rank -8 (aka five gargoyles). The total will be 30 (one Stone Giant) + 3*15 (three Xorns) + 5*5 (gargoyle) = 30+45+25 = 100. Goal achieved!

The total of Experience Points earned will be: 2900+3*1800+5*450 = 10550 Experience Points / 6 Characters = 1750 Experience Points per character!


Opponents with a Challenge rating lower than 8 compared to the APL are counted, "weighed", only if they are higher than 20 as a unit.

\subsubsection{Fight is too fast}

One problem that you may encounter is the fight ending too quickly. There could be various reasons for this, and correspondingly, several solutions.

If players anticipate fewer encounters, they may exhaust their best resources and options at the start of the fight, resulting in a quick defeat of the enemy. In this case, surprising them with successive waves of enemies can add a challenge.

It's also possible that there are too few enemies, and even if they are powerful, they become easy prey for the characters by focusing all their attacks on a single target. In this case, introducing followers or preventing the enemy from resting can make the encounter more challenging.

It's evident that the fight may not be well-balanced, making it too easy. However, it is the easiest problem to solve as experience will teach you to balance encounters by adding or replacing enemies.


\subsubsection{The Boss fight}\index{The Boss fight}

When preparing for a boss fight, which involves a significant enemy that holds considerable importance in the campaign's development, it is crucial to focus on creating an engaging and compelling battle.

If the clash must be memorable, it is not enough to put the villain, organize everything so that all the events are engaging and exciting.

Arrange enemies to:

- come in multiple waves so there is a sense of false success\\
- that the enemies come from many sides so as not to concentrate forces only on one side\\
- that more or less difficult enemies are interspersed so that there is a sense of false security\\
- that the environment is significant and plays an important role in combat\\
- divide the characters on several fronts\\
- make the attack not look like an attack\\
- play with wit and don't get demoralized.

\subsubsection{Add NPCs}\index{Add NPCs}

A creature that has levels, Skills, proficiencies, that could be a character is considered an NPC. These creatures can play a very important role and should not be used as mere "monsters". Give it a thickness and you will create unforgettable characters.

\subsubsection{Challenge Rank Ad Hoc Changes}\index{Challenge Rank Ad Hoc Changes}

While you can adjust the specific Challenge rating of the monster by advancing it, applying modifications or levels, you can also adjust the difficulty of the encounter by applying ad hoc modifications to the encounter or the creature itself.

Described here are three additional ways you can alter the difficulty of the encounter.

\subsubsection{Favourable terrain for the PCs}\index{Favourable terrain for the PCs}

An encounter against a monster that is not in its preferred element (such as a Yeti encountered in a lava-filled cave, or a huge Dragon encountered in a very small room) gives the characters an advantage. Treat the encounter as having a lower Challenge rating than its actual Challenge rating.

\subsubsection{Terrain unfavorable to the PCs}\index{Terrain unfavorable to the PCs}

Monsters are designed with the assumption that they are encountered in their preferred terrain: encountering an Aboleth underwater does not increase the encounter's Challenge rating, even if neither character is able to breathe underwater.

If, on the other hand, the terrain has a more significant impact on the encounter (such as an encounter against a creature with blindsight in an area that suppresses all light sources), you may, if necessary, increase the challenge rating of the encounter. meeting was of a higher degree.

\begin{center}
	\includegraphics[width=0.7\linewidth]{immagini/tesoro2.png}
\end{center}

\subsubsection{Changes to NPC Equipment}\index{Changes to NPC Equipment}

You can increase or decrease the difficulty given by NPCs by modifying their Equipment. An NPC encountered without Equipment should have a Challenge rating reduced by 1 (provided that the loss of Equipment is really counterproductive to the NPC), while an NPC who has Equipment equivalent to that of a character (as indicated on Table: Wealth of Characters per Level) has a Challenge rating 1 higher than its Royal Challenge rating.

Care should be taken to equip NPCs with this extra equipment, especially at higher levels, where you can consume the entire treasure of your adventure in one fell swoop!

\begin{changemargin}{0.3cm}{0.3cm}\begin{narrator}
But how many fights to handle per session?

There isn't an exact answer (2/4 per session/day ???), a lot depends on the type of players you have and what game they play.
Stay focused on the adventure, don't go thinking that it's better not to tire the characters otherwise they won't hold up the next fight. You don't have to worry too much about the characters, that's up to the players. You have to think based on the environment where you are and who is around.

In any case, common sense always helps. Tire the characters but make sure it's not a "Total party kill" every session.
\end{narrator}\end{changemargin}

\subsubsection{Assign PX}\index{Assign PX}\label{assegnarepuntiesperienza}

Characters level up by defeating monsters, overcoming challenges, having fun, completing adventures and grabbing treasures: in doing so they earn Experience Points (XP for short). You can award Experience Points as soon as a challenge is cleared, but this may interrupt the flow of the game. It is easier to award experience points at the end of a game session (or sessions) which allows the characters to reflect on what happened. The player can use the time available between game sessions to refresh the card.

\subsubsection{Arranging Treasures}\index{Arranging Treasures}\label{disporretesori}

As characters level up, the amount of treasure they carry and use also increases. In OBSS it is assumed that all characters of the same level have roughly the same amount of treasure and magic items. Since the primary income for a character comes from treasure and loot gained from adventures, it's important to moderate wealth and treasure in your adventures.

To help you lay out treasure, the amount of magic items and loot characters receive for their adventures is tied to the Challenge rating of the encounters they face: the higher the Challenge rating of the encounter, the more treasure will be awarded.

\begin{changemargin}{0.3cm}{0.3cm}\begin{narrator}
How to distribute treasure is an important matter. Treasures shouldn't be slammed in your face, much less hidden that you can't find them.
		
A tip is to make sure that the treasures (and coins) found in the dungeons are distributed according to this criterion:
		
\smallskip
- a third will have monsters on them

\smallskip
- a third will be hidden behind secret passages or traps
		
\smallskip
- a third will be scattered around
		
\smallskip
		
This will motivate players to continue exploring, face monsters, and actively search the dungeon.
\end{narrator}\end{changemargin}

\textbf{Table: Treasure Values per Encounter} lists the amount of treasure each encounter should award based on the average character level and XP progression rate of the campaign. Easy encounters should award treasure one-half the level of the average PC level. The most dangerous, difficult, and heroic encounters should respectively award a treasure one, two, or three levels higher than the PCs' average level. If the spell is rare in the game, halve these values. If the game is more epic, double these values.

\medskip

\textbf{Table: Treasure Values per Encounter}\index{Treasure}\index{Table Treasure Values per Encounter}\label{valoretesoroincontro}

\begin{tabularx}{0.45\textwidth}{XX}
\textbf{Challenge Rating} & \textbf{per Encounter (GP)} \\
\toprule
\textbf{0 (Easy)} & LV*LV*3\\
\textbf{+2 (Medium)} & LV*LV*5\\
\textbf{+3 (High)} & LV*LV*7\\
\textbf{+4 (Extraordinary)} & LV*L*10\\
\textbf{+6 (Deadly)} & LV*LV*15\\
\textbf{+8 (Epic)} & LV*LV*20\\
\end{tabularx}
\bigskip{}

Use this table to get an indication of the Treasure value of the encounter.\index{Monster Treasure}

A Double Treasure\index{Double Treasure} means that a party facing a match with a Challenge Rating of +2 compared to APL will find the equivalent of Level * Level * 10 * 2 in equivalent gold pieces.
Ex. APL 6lv faces an encounter on High difficulty (APL+3), they will get the equivalent of 6*6*15 gold from the encounter, 540 gp.

An accidental treasure indicates that there may be treasures, objects, coins only "accidentally" obtained, perhaps by killing other creatures. Stated differently, the creature does not collect treasures, but it could still have accumulated something.

I suggest that accidental treasure be valued at one-quarter the value of normal treasure. An accidental treasure lends itself very well to providing particular objects that can give indications or provide new adventure plots (what was this adamantium forged ampoule for?)...

Encounters against NPCs usually reward three times more treasure than with a monster, thanks to the NPC's Equipment. To compensate, make sure the characters go through a couple of extra encounters that award little in the way of treasure.

Animals, Plants, Constructs, unintelligent Undead, Oozes, and traps make excellent "little treasure encounters." Alternatively, if the characters are facing a number of creatures with little or no treasure, they should have the opportunity to obtain a number of items of more significant value in the near future to compensate for the imbalance. As a general rule, characters shouldn't possess any magical items worth more than half of the character's total wealth, so check carefully before rewarding characters with very expensive items. 

\subsection{Character Wealth by Level}

The \textbf{Table: Character Wealth by Level} by Level indicates how much treasure, in equivalent gold piece, each character should have at a specific level. Note that this table is based on a standard game model.

Adventures with rare magic might only award half this value, while more epic games might double it. It is assumed that some of the treasure is consumed in the course of an adventure (such as potions and scrolls) and that some of the less used items are sold for half their value to buy more useful equipment.

Table: Character Wealth by Level can also be used to allocate Equipment for characters starting after 1st level, such as a new character created to replace a dead one. Characters should spend no more than a third their total wealth on a single item.

For a balanced method, characters that are created after 1st level should spend 25\% of their wealth on weapons, 25\% on Armour and protective items, 25\% on other magic items, 15\% for expendable items such as wands, scrolls, and potions, and 10\% for normal equipment and coins. Different character types may spend their wealth differently than suggested; for example, arcane spellcasters might spend more on consumable and magical items than on weapons.\\

\textbf{Table: Character Wealth by Level in gold piece}\index{Table Character Wealth by Level}

\smallskip

\begin{tabular}{ll|ll}
	\textbf{Level} & \textbf{Wealth} & \textbf{Level} & \textbf{Wealth}\\
	\toprule
	1 & 100 & 11 & 13900\\
	2 & 160 & 12 & 19900\\
	3 & 220 & 13 & 25900\\
	4 & 340 & 14 & 37900\\
	5 & 530 & 15 & 49800\\
	6 & 2030 & 16 & 67700\\
	7 & 3660 & 17 & 85700\\
	8 & 5780 & 18 & 142000\\
	9 & 8100 & 19 & 253000\\
	10 & 11000 & 20 & 365000\\
\end{tabular}


\subsection{Magical Treasures}\index{Magical Treasures}\label{tesorimagici}

The magical treasures must be entered sparingly and with reasoning, try to resist the temptation to be generous with the players because they will easily get used to it and you will hardly be able to recover the situation.

In principle, each magic item found by the players is placed by the Arbiter with a reason, purpose and reasoning.

Nothing prevents you from cheering up \hyperlink{generazionetesorimagici}{Random Generation of Magical Objects} (page \pageref{generazionetesorimagici}) I just ask you to be careful.

You can grant that a monster with Standard Treasure has 1\% of having a magic item, a Double Treasure has 3\% and a Triple Treasure 5\%. Obviously evaluate case by case depending on the type of monster and if, for example, it has a lair, it goes hunting for adventurers...\index{Probability of Magical Object in treasure}

If you prefer an established and balanced distribution, follow these guidelines. \index{Magical object per level}

As for \textit{permanent} magic items (weapons, armor, other items with no charges or with daily charges) you can distribute the items according to this scheme:

\begin{itemize}[leftmargin=*]
	\item level 1st to 4th: one uncommon item
	\item level 5th to 7th: second uncommon item
	\item level 8 to 10: one rare item
	\item level 11 to 13: secondd rare item
	\item level 14 to 16: one very rare item
	\item beyond 17 level: second very rare or legendary item.
\end{itemize}

Concerning \textit{consumables} such as potions, scrolls or items with a decreasing use of charges

\begin{itemize}[leftmargin=*]
	\item 1 common consumable each level from 1 to 5
	\item 1 uncommon consumable every level from 6 to 10
	\item 1 rare consumable every level from 11 to 15
	\item 1 very rare consumable for each level from 16 to 19
	\item 1 legendary consumable at level 20
\end{itemize}

All of this clearly depends on the level of magic you want to give the adventure.

\subsubsection{Building the Loot}\index{Building the Loot}\label{costruireunbottino}

Often it is sufficient to tell your players that they have found 5,000 gp in gems and 10,000 gp in jewels. But sometimes it's more interesting to give details. Giving a treasure a personality can not only help the likelihood of the game, but can sometimes spark new adventures.

The information on the following pages can help you determine random treasure types: many of the items have been given values, but you can assign them as you see fit. It's easier to place the most expensive items first: if you like, you can also determine the magic items randomly using the tables in Magic Items, to determine which items are in the treasury.

Once you have used up a considerable portion of the treasure's value, the rest can simply be scattered coins and non-magical items with values defined according to your needs.

\textbf{Coins}: Coins in a hoard can be copper, silver, gold, and platinum: silver and gold are the most common, but you can decide otherwise. For coins and their exchange value go to Equipment.

\textbf{Gems}: While you can assign any value to a gem, some may be worth more than others. Use the value categories below (and associated gemstones) as a guideline when assigning gemstone values.

\textbf{Low Quality Gems} (10 gp): agate; azurite; blue quartz; hematite; lapis lazuli; malachite; obsidian; rhodochrosite; eye of the Tiger; turquoise; river pearl (irregular).

\textbf{Semi-Precious Gems} (50 gp): heliotrope, carnelian; chalcedony; chrysoprase; citrine; jasper; lunar; onyx; chrysolite; rock crystal (clear quartz); sardonyx; sardonyx; rosy, smoky or star rose quartz; zircon.

\textbf{Medium Quality Gems} (100 gp): amber; amethyst; chrysoberyl; coral; red or green-brown garnet; jade; jet; white, golden, pink or silver pearl; red, red-brown, or dark green spinel; tourmaline.

\textbf{High Quality Gems} (500 gp): alexandrite; aquamarine; purple garnet; black Pearl; dark blue spinel; golden yellow topaz.

\textbf{Jewels} (1000 gp): emerald; white, black, or fire opal; blue sapphire; fiery yellow or vermilion corundum; blue or black star sapphire.

\textbf{Exceptional Jewels} (5000 gp or more): crystalline brilliant green emerald, diamond, hyacinth, ruby, crystalline topetto honey.

\textbf{Nonmagical Treasures} This category includes trinkets, fine clothing, wares, alchemical items, masterwork items, and others.

Unlike gems, many of these items have set values, but you can always increase the value of the item by decorating it with precious stones or particularly artistic workmanship.

This cost increase grants no additional capabilities: a 40,000 gp gem-embellished cold iron scimitar functions as a normal 330 gp cold iron scimitar. Below you will find numerous examples of non-magical treasures, with typical values.

\textbf{Fine Works of Art} (100 gp or more): While some works of art are composed of precious materials, the value of most paintings, sculptures, works of literature, fine clothing, and the like consists in workmanship with which they are made and in the skill of those who made them. Works of art are often bulky or difficult to move, and fragile, making their recovery and transportation an adventure in itself.

\textbf{Minor Jewels} (50 gp): This category includes jewels made with materials such as brass, bronze, copper, ivory, or exotic woods, sometimes embellished with very small or flawed low-quality gems. Minor jewelery includes rings, bracelets and earrings.

\textbf{Jewellery Normal} (100-500 gp): Most jewelery is made from silver, gold, jade, or coral, and often decorated with semi-precious gems or medium-grade gemstones. Normal jewelery includes all types of minor jewelery plus bracelets, necklaces and brooches.

\textbf{Precious Jewels} (500 gp or more): Precious jewels are made of gold, mithral, platinum, or similar rare metals. Such objects include types of plain jewelry plus sceptres, pendants, and other large objects.

\textbf{Finely Crafted Tools} (100-300 gp): This category includes tools for Professions or Skills: see Equipment for details and costs of these items.



\begin{changemargin}{0.3cm}{0.3cm}\begin{narrator}
Never go overboard with treasures, especially magical ones. A treasure shouldn't become a habit even more if it is something special and particular. One thing can be coins, gems and "consumables" one thing are the real treasures, the magical, particular, unique ones.

Respecting the Law of the Prize does not mean lining the pockets of the characters, otherwise they will get bored risking their lives for new treasures and objects. When you find a magical object, always think in perspective. It's true that it can be nice to see players happy with what they've found but then you'll be forced the next adventure to give something even more powerful.

\end{narrator}\end{changemargin}

\medskip

\textbf{Common Items} (up to 1000 gp): There are many valuable items of an alchemical or common nature that can be used as treasure. Most alchemical items are portable and estimable items, but others such as locks, sacred symbols, spyglasses, fine wines, or fine clothing can also form interesting parts of a hoard. Trade goods can also serve as treasure: 5 kg of saffron, for example, is worth 150 gp.

\textbf{Treasure Maps and Information Items} (variables): Items like treasure maps, ship and house deeds, watch or informer lists, passwords, and the like can be fun items to find in a treasure: you can determine the value of these items as you like and they can be of double use as they can generate ideas for new adventures.

\textbf{Magic Items}

Of course, the discovery of a Magic Item is the true prize for any adventurer. Be careful about placing Magic Items in a hoard: it is much more satisfying for many players to find a magic item than to buy it.

While you should generally place items with careful thought about their likely effects on your campaign, it can be fun to spawn magic items into a random hoard. Be careful though! it's easy, with a little luck (or bad luck) of the dice, to inflate your game with too much treasure or to deprive it of the same. The placement of random magic items should always be tempered by the Arbiter's common sense.

Even spells are real treasures and prizes like magic items. Consider carefully what can be found. Remember that a magical ability is not a copyable spell, only those present in tomes, scrolls and anything else specially created to be a repository of spells are eligible for copying.

\begin{center}
	\includegraphics[width=0.9\linewidth]{immagini/Hoxne_Hoard_1.png}
	
	\textit{Hoxne Treasure Reproduction}
\end{center}


\subsection{Role Playing}\index{Role Playing}\label{ruolare}

\label{recitare}

An RPG is not a simple roll of dice, it is a meeting of thoughts, opinions, challenges, struggles. It is a cathartic, liberating, evolutionary and instructive game.

There should be fighting, fighting, blood fear and action, likewise there should be the possibility to play your own characters with their drawbacks, advantages, powers and stories and personal dramas too.

The player must always impersonate the character, empathize and actively participate.

There may be side situations, managed quickly, which are done in the third person, yet every time it is necessary to play then it must be true, done by the player fully immersing himself in the character.

\textbf{When a player interprets well and describes the action he is going to carry out in a participatory, engaging, inspired way, give him a prize, grant a +1 bonus to the action he is carrying out}

Make it clear to the player that thanks to his interpretation he has that bonus.

At the same time there may be situations that prove unpleasant for some players to manage and play. Be very careful in this case, going against the sensitivity of a player, a friend, is not like going against the ethics or morals of a character. If you feel a sense of unease and embarrassment, stop the game immediately and clarify the situation with the players and resume only when you have agreed on how to change the situation to prevent it from happening again.

\begin{changemargin}{0.3cm}{0.3cm}\begin{emphasis}
{Focus on the people, not the rules. Push for a group play style; the interpretation is fun but must not hinder the plan; support your teammates. (Frank Mentzer)}\end{emphasis}\end{changemargin}

\subsection{About OBSS and dice rolls}\index{About OBSS and dice rolls}\label{obssedadi}

OBSS uses a unique dice rolling system mixing a 3d6 distribution with the potential of exploding 6s. This system manages to guarantee a good variance and even if by concentrating the results around the central values of the distribution it leaves the upper limit open to particularly lucky rolls.

If you want to have fun studying the corresponding curve, I recommend the site \href{https://anydice.com/}{Anydice}. This is the pseudo code to insert:

{\small function: explode ROLLEDVALUE:n \{

if ROLLEDVALUE = 6 \{ result: 6 +[explode d6] \}

if ROLLEDVALUE = 1 \{ result: 0 \}

result: ROLLEDVALUE\}

output 3d[explode d6]}

or click \href{https://anydice.com/program/2610e}{here} for the code already entered.

\subsubsection{Optional - Critical Shots Variant}\index{Optional - Critical Shots Variant}\label{variantetiricritici}

In OBSS the rule applies that if a success is achieved by obtaining at least two 6s on the dice it is defined as a critical success, in the same way if I fail the test by obtaining two 1s in the dice roll then it is defined as a critical failure.

These critical rolls, which are very dependent on the randomness of the dice, may not appeal to all players and as per the \hyperlink{variantetiricritici}{Optional - Variant critical roll} rule (page \pageref{variantetiricritici}) this optional rule establishes that:

- treat a failed check by 6 or more as a critical failure.

- treat a successful check of 6 or more as a critical success.

- for each multiple of 6 for which the test is successful or failed, an additional critical success or failure is accumulated respectively.

The same principle applies to the Magic Check, which is performed by rolling 3d6 + Magical Proficiency + 2*Adept of Magic, and the result must be equal or exceed a difficulty of 10 + Level of Spell*3.



\subsection{The Adventures in OBSS} \hypertarget{OSR}{} \index{OSR}\index{Adventures in OBSS}\label{avventureinobss}

I suggest reading the full article: \href{https://lithyscaphe.blogspot.com/p/principia-apocrypha.html} {Principia Apocrypha}

https://lithyscaphe.blogspot.com/p/principia-apocrypha.html The following is my adapted and modified summary of the guidelines I follow when mastering OBSS.

OBSS follows the principles of \href{https://it.wikipedia.org/wiki/Old_School_Renaissance}{OSR} (wikipedia). Adventures in OBSS aim to be lethal, have a freely explorable world, a sketchy plot, push problem-solving, and have a reward system focused on exploration, treasure, and party participation. OBSS doesn't care too much about balancing encounters and appreciates the resourcefulness of players and captures their ideas by putting them into the adventure.

To me, OSR isn't about random encounter tables and chaotic randomization or a specific rulebook, it's rather the spirit of adventure, wonder, fear, glory, amazement and challenge that develops in adventures. Don't be too linear, too predictable, add the right mix in the adventures that always make them unique.

If you don't like the method, use the one that suits you best, personally over the decades I have learned to appreciate and see the spontaneity and naturalness that the cornerstones of the OSR bring to the game.

\bigskip

\textbf{These are basic rules for the Arbiter that I suggest for conducting adventures.}\index{Guidelines for Arbiters}\index{OSR}

\medskip

\begin{itemize}[leftmargin=*]

\item
You are the Arbiter, your Rules, your World.

Don't be limited by the adventure, the system, the monster list, always feel free to modify and adapt according to the needs of the adventure and the party

\item
Remember to be fair and correct. Improvise, adapt as much as you like but be consistent. If you establish a rule (or a change to a rule) follow it through.

At the same time, if you need a rule and can't find it, use common sense, it's definitely the right choice at that moment.

Respect the dice and the results obtained, as will happen to the players, particular results will happen to you too. Rightly so.

\item
You don't have to save the characters' asses. You are not their friend or their enemy. Your role is to tell stories that arise from the stories of the characters, their actions and inactions.

\item
Sketch out the story, write the middle parts or read to the players but don't let yourself be dominated or bound by what you expect. Often the players will amaze you, it's better to know where they move and what they have around them in order to always be able to react promptly.

The players set the direction of the adventure and you unravel it.

\item
Appreciate chance and create different situations where players can choose different paths or weave new ones. You are lucky to have creative players who know how to surprise you.

\item
Don't force anyone to do something, let the players make mistakes, let them pay for their choices. You don't have to hinder them nor you have to cue them for a direction. It requires considerable imagination and adaptability on your part, but the adventure will certainly benefit from it.

\item
The characters are explorers, by definition. Focus on exploration, the more you explore the more situations you create, the more hooks you create in the adventure, the more you know other NPCs the more there are areas to explore.

It makes it clear that treasures are experience, in a literal and practical sense. You will never have to push them into a dungeon but their lust for experience and treasure.

\item
Have the players solve the problems, not the characters. Let the scenes roll, they're always better than a die roll. Encourage the player to interact and only ask for proof as a last resort. Propose problems that don't necessarily have to be solved with a die roll but rather through multiple actions, even complex ones.

It rewards creative actions and courageous choices, above all intelligence and the desire to find alternative and creative situations.

\item
Make the players ask you for information, compare themselves with the environment and with each other. Encourage interaction with the outside world and only as a last resort allow a dice roll.

\item
It is natural that the characters know what the players know, try to limit this exchange, for the benefit of everyone's skills and abilities.

\item
Great challenges and risks always yield great rewards. Don't disappoint players (except for the purpose of adventure) by denying them the right treasure or experience, the deeper they go the more lethal the dangers the greater the reward (Law of the Prize).

\item
There must be no habits, customs. Don't create a standard.
Always try to surprise players with monsters that are out of place (but make sense), freak traps, alternate environments. Different situations will stimulate players to solve each problem differently.

Prepare different solutions and accept different solutions. Put problems and situations into the adventure that together allow for a solution, each room shouldn't be an aseptic environment but contain clues and solutions for other problems even without a direct solution.

\item
Accept death. A fight if such is always lethal, don't be afraid to injure or kill the characters. Make them think, study the enemy, understand which is the best approach; and finally amaze them. Characters must first outwit and plan their enemies if they are to survive.
If you protect the characters, the game will lack tension and players will solve all problems with brute force.
Dungeons don't have to be environments that need to be cleared of monsters. The purpose of monsters is to limit and direct actions, to consume options.

If players are always looking for head-on collision then give it to them, as they request.

\item
Keep attention high. It makes the passage of time have consequences, if players fear the passage of time they will make more daring or perhaps wrong choices. Maintain the tension between the desire to explore and loot and the dread of sitting still for too long.

\item
You are the source of information, the players process it, the characters use it.

Don't hide information that the characters need to know or already know, you won't have to be a professor but in the same way make sure they are aware of their surroundings.
At the same time you don't have to reveal everything right away, make them investigate, snoop. Like an onion, the information they obtain will be hidden under layers of other information that may be of lesser importance.

\item
Clues create situations. Let your clues, specific and curious, attract the attention of the players. Like a bait on a hook, it lures players into dubious situations, where they investigate and figure out what's going on.

Don't stuff the adventure with useless details, leave room for the creativity and imagination of the players, but the details you provide must not only make sense but be necessary for the adventure.

\item
If players tend to forget useful information given try using an NPC who has memory or have them take notes, there's no harm in being prepared.

\item
The adventure is never static, much less the world where the characters move.
The world matters as much if not more than the adventure itself. Player actions can trigger global events. Always think about the consequences of gestures.

\item
If you use NPCs (non-player characters) don't make them mere specks, make sure that the characters can become attached and consider the NPC one of the group on a par with everyone else.

\item
Monsters don't have to be stupid. Make them talk, reason, run away.. they want to live too!

\item
Remember the Law of the Prize. Reward the bold, reward those who go deeper into the caves. Reward those who survive.

\end{itemize}

\subsection{Session Zero}\index{Session Zero}\label{sessionezero}\hypertarget{sessionezero}{}

The Session Zero, the first game session, holds a special significance and importance. It can be the session where everyone meets for the first time, often the session where character creation takes place, and always the session where common rules and expectations are established.

The Session Zero serves to determine what and how the game will be played, what the main features of the campaign and the group being formed will be.

To start off as a good group of players, it's important to personally get to know each other and have trust and respect for one another. You don't have to reveal everything about yourself, but at least share your passions, interests, curiosities – things that, at least initially, help build a friendship.

I suggest that Arbiter establish clear rules for good gameplay. Unfortunately, experience teaches us that we are all different people with different styles, perspectives, and expectations, often opposing our own way of playing and being. Getting to know each other also serves this purpose – to understand if your character can fit well within the group of characters and if your own person and personality are somehow compatible or not with the others.

\textbf{Before starting, the Arbiter should clarify the essential rules at their table}. An example of rules can be:

Know the boundaries of others. Each person has a different sensitivity to certain topics (rape, slavery, racism, violence...), and it is crucial to discuss together what the limits are never to be crossed.

Respect every person you play with. This includes being punctual and not canceling without an important reason. Also, respect the limits that others have communicated.

Players must create a cohesive group of individuals who collaborate.

Players must know or at least try to learn the rules of the game.

As a Arbiter, you must understand the players' level of system knowledge and, if necessary, take the time to explain the rules.

\textbf{Basic adventure information should be shared and established.}

Provide a general overview of the campaign or adventures that will take place. Indicate the genre (heroic, dark, gothic, horror, political, endless caves, exploration, survival...) and difficulty level.

Provide necessary information about the setting or provide handouts and manuals on the subject. Indicate if there are any skills that may be less useful than others.

If you are using optional rules, explain them well and ensure they apply to both players and adversaries.

Indicate the list or type of accepted Traits and if there are any limits on choosing Patrons.

\textbf{Other useful guidelines include}:

What is allowed to bring and use at the table and what is not (drinks, food, cell phones, alcoholic beverages, smoking...). Find out if there are any animals in the house.

Set a minimum number of players to conduct the session. Establish the game day and schedule.\

Ultimately, the Session Zero is fundamental to establish a solid foundation for a good role-playing experience. It helps create a collaborative environment where everyone feels involved and contributes to avoiding potential problems and disagreements throughout the campaign.

Even in the best, well-established group, it's always a good idea to remember and share these suggestions at the beginning of each campaign.


\end{multicols}

%\vfill

%\begin{center}
%\includegraphics[keepaspectratio,width=0.4\textwidth]{immagini/dungeonsample.png}

%\textit{Dungeon detail}
%\end{center}

\pagebreak

\section{Create Magic Item}\index{Create Magic Item}

\begin{changemargin}{0.3cm}{0.3cm}\begin{emphasis}{
To create is to live twice. (Albert Camus)}\end{emphasis}\end{changemargin}\medskip


\begin{multicols}{2}

\label{creare-oggetti-magici}

\lettrine[lines=2, lhang=0.33, loversize=0.25, findent=1.5em]{T}{he} Crafting Magic Items requires having the Crafting Magic Item Skills.

The costs listed here are those of production, the revenue can be at least around 20\% of the production price.

Knowing the spell (or having it available via Scroll) that applies to the item is a requirement of any magic item you create.

\subsubsection{Magic Item Cost Modifiers}\index{Magic Item Cost Modifiers}\label{magic item cost modifier}

Magical items have as their basic component the application of a spell to the item itself.

It is important to evaluate the rarity of the enchantment as it is used to determine the cost of the item.

The costs reported for the creation of the various types of objects refer to the use of a spell with a common rarity. If the rarity is Uncommon multiply the price x1.5, if it is Rare x2, Very Rare x5, Legendary x10.


\begin{changemargin}{0.3cm}{0.3cm}\begin{narrator}
Crafting magical items can break the balance of the game. A character with abundant resources and time can create items that upset the balance of the adventure. I suggest that NPCs, the non-player characters managed by the Arbiter, create the most wondrous items. At the same time the sale of valuables above 2000gp should be as limited as possible.
\end{narrator}\end{changemargin}

\subsubsection{Create Magic Rings}\index{Magic Rings}\index{Create Magic Rings}\label{creareanellimagici}

To create a magic ring, a character needs a source of heat. She also needs a supply of materials, the most obvious of which is a ring or ring pieces to assemble. The cost of the materials is included in the cost of creating the ring.

The production cost of the ring is equal to level*level*2000, an Invisibility Ring costs 2*2*2000=8000 gp

\begin{center}
\includegraphics[width=0.5\linewidth]{immagini/onering2.png}

\textit{Needless to say Ring is...}
\end{center}

A ring allows you to fix a spell to make the effect always active.
The ring must have an intrinsic value equal to at least 500gp*level of the spell it is to house.

A ring can hold a 9th level spell or if multiple spells the maximum level is 7.

It is also possible to insert an activation spell, in this case consult the costs of the Rods.

Forging a ring takes 1 day for every 500 gp of the base price. In the case of multiple spells, the costs and times add up.

\medskip

\textbf{Item Creation Feat Required}: Craft Greater Magic Items.

\subsection{Crafting Magic Armour and Shields}\index{Create Magic Armour and Shields}\label{crearearmaturemagiche}

To create magical Armour or shields, a character needs a source of heat and some tools for working iron, wood, or leather. He also needs a supply of materials, the most obvious being the Armour/shield itself or the Armour pieces to assemble. An armour/shield to be enchanted must be of quality.

\begin{center}
\includegraphics[width=0.5\linewidth]{immagini/Rustning_Gustav_Vasa.png}

\textit{Armour for Gustav I of Sweden by Kunz Lochner, c. 1540 (Livrustkammaren)}
\end{center}

If the prerequisites for crafting the Armour include spells, the caster must know those spells.

The production cost of +1 mage Armour costs 2050 gp, +2 7500 gp, +3 12000 gp, +4 25000 gp, +5 45000 gp plus the price of the Armour itself.

Infusing a spell into Armour costs the same as creating a ring with that spell.

Crafting mage Armour/shields takes one day for every 1,000 gp of base price value.

\medskip

\textbf{Item Creation Feat Required}: Craft Magic Items.

\subsection{Create Magic Weapons}\index{Create Magic Weapons}\label{crearearmimagiche}

To create a magical weapon, a character needs a source of heat and some tools for working iron or the material from which the weapon is made. You also need a supply of materials, the most obvious being the weapon itself or weapon pieces to assemble. Only a quality weapon can be enchanted to become a magic weapon, and its cost is added to the total enchantment cost to determine the final market value.

A magic weapon must have at least a +1 bonus to have any special ability or spell.

\medskip

\begin{center}
\includegraphics[width=0.6\linewidth]{immagini/exacaliburfuori.png}

\textit{The drawing of the sword from the stone, Henrietta Elizabeth Marshall's Our Island Story (1906)}
\end{center}

\medskip

If the prerequisites for crafting the weapon include spells, the caster must know those spells.

At the time of creation, the spellcaster must decide whether or not the weapon sheds light, as a secondary effect of the magic imbued in the weapon. This decision does not affect the price or crafting time, but once the item is completed, the decision is final.

Crafting dual weapons is treated as analogous to crafting two weapons in terms of cost, time, and special abilities.

The production cost of a +1 Weapon is 1200 gp, +2 4000 gp, +3 11000 gp, +4 25000 gp, +5 45000 gp plus the price of the weapon (only relevant if it is made of some rare or precious material ).

The production cost of a +1 Arrow is 20 gp, +2 75 gp, +3 325 gp. More powerful enchantments are extremely rare.

Instilling a spell in a weapon has a cost as if you were going to create a ring with that spell, if continuous, otherwise if single use as a potion.

Crafting a magic weapon takes one day for every 1,000 gp of the base price's value.

\medskip

\textbf{Item Creation Feat Required}: Craft Magic Items.

\subsection{Create Wands}\index{Create Wands}\label{crearebacchette}

The production cost of the wand is equal to level*level*400, a wand with invisibility costs 2*2*400=1600 gp

A wand is a magical item that retains a previously charged spell.

To recharge a wand, a spellcaster must imbue the same spell and have the Craft Magic Item skill. The wand recovers a charge but the caster, in addition to having used Spell Points, spends the equivalent of 100*level gold coins in components.

A wand can hold a maximum spell level of 5.

To create a wand, a character needs a supply of materials, the most obvious being a wand or wand pieces to assemble. Wands are always fully charged (20 charges) upon creation.

The caster must know the spell he puts into the wand.

\begin{center}
\includegraphics[width=0.5\linewidth]{immagini/wand.png}
\end{center}

Crafting a wand takes 1 day for every 500 gp of the base price's value.

\medskip

\textbf{Item Creation Feat Required}: Craft Magic Items.

\subsection{Create Staff}\index{Create Staff}\label{crearebastoni}

\textbf{Base Costs of Staffs}

\bigskip

The production cost of the Staff is equal to level*level*600, a Staff with Invisibility costs 2*2*600=2400 gp

\bigskip

A Staff is a magical item into which one or more spells are charged.

When a staff is activated, one spell can be used at a time.

To create a staff, a character needs a supply of materials, the most obvious being a staff or pieces of a staff to assemble.

Staffs are always fully charged, 10 charges, upon creation.

\begin{center}
\includegraphics[width=0.3\linewidth]{immagini/staff2.png}
\end{center}

A staff can contain a maximum spell level of 8, or in the case of several spells the maximum level is 6.

Crafting a staff takes 1 day for every 500 gp of the base price.

\medskip

\textbf{Item Creation Feat Required}: Craft Greater Magic Items.

\subsection{Create Scrolls}\index{Create Scrolls}\index{Scrolls}\index{Isy Scroll}\label{crearepergamene}
\index{Buy spells}
\medskip

There are two types of magical Scrolls, those that can be performed by everyone (called ISY SCROLL, or Easy) and those that require the magical ability to cast spells, i.e. Magic Proficiency greater than or equal to 1.

Easy scrolls cost level*level*80 gp to produce.

Normal, not easy scrolls have a production cost of level*level*40 gp.

The cost of the parchment must be evaluated according to the rarity and level of the spell. A very rare or high level spell can easily cost level*level*level*80.

\begin{center}
\includegraphics[width=0.4\linewidth]{immagini/scroll3.png}
\end{center}

If a parchment includes several spells, the cost is equal to the sum of the various spells. On an ISY SCROLL scroll there cannot be normal scroll spells and vice versa.

The spellcaster must know the spells he places on the scroll. Preparing a scroll requires 30 minutes of work per spell level present.

An ISY parchment can contain spells of level 3 as a maximum, while a normal parchment can contain a maximum spell level of 9, in case of multiple spells the maximum level is 8.

\medskip

To read a parchment you need:\\

\textbf{in case of ISY SCROLL scrolls}:

- to understand the content, a check of Intelligence (or Arcana if known) with difficulty DC 10 is sufficient

- to be able to read and cast the parchment spell requires an Intelligence check (or Arcana if known) at difficulty 12.\\

\medskip

\textbf{in case of normal scrolls}:\\

- to understand its contents, a check of Arcana at difficulty 15 is necessary

- to be able to read and cast the parchment spell it is necessary to check Arcana at difficulty 11+Level of the spell and to have access to the Magic List of the spell contained and that this is of a level equal to the maximum that can be cast +1.

\medskip

The \textbf{casting time} of a spell from a scroll is equal to the casting time of the present spell.

\textbf{Item Creation Feat Required}: Craft Magic Item

Proficiency used in creation: Arcana or Profession (scribe).

A scroll is destroyed when used or copied.

\textbf{Note}: A Tome of Magic is equivalent to a set of normal scrolls. A character in desperate situations can read the spell from the Tome of Magic and manifest magic as if from a scroll. The pages containing the spell will pulverize and the spellcaster will have to find a source from which to copy the spell back to Tome.\index{Tome of Magic as scroll}

\begin{center}
\includegraphics[width=0.6\linewidth]{immagini/potion2.png}

\textit{A witch, raising her arm above a flaming cauldron, recites a spell; a young woman kneels in front of the cauldron. Mezzotint by J. Dixon after J.H. Mortimer, 1773}
\end{center}

\subsection{Create Potions}\index{Create Potions}\index{Potions}\label{crearepozioni}

A potion contains the brew of only one spell, so each potion is disposable.

\medskip

The production cost of the Potion is equal to level*level*40, a Potion with Invisibility costs 2*2*40=160 gp

\bigskip

To create a potion, a character needs a horizontal work surface and some containers for mixing liquids, as well as a heat source for boiling the brew.

A Potion can normally contain a maximum spell level of 3. Spells of a higher level increase the price to level*level*level*20 gp, if the Storyteller allows it.

All ingredients and materials for brewing a potion must be fresh and unused.

The caster must know the spell he puts into the potion. The preparation time of a potion is equal to twice the level of the spell contained in hours.

\medskip

\textbf{Item Crafting Feat Required}: Brew Potions.

\subsection{Create Rods}\index{Create Rods}\index{Rods}\label{creareverghe}

A rod is a special wand that is capable of regenerating its charges. They are precious and very expensive objects.

To create a rod, a character needs a supply of materials, the most obvious being a rod or pieces of a rod to assemble.

\medskip

The production cost of the Rod is equal to level*level*1600, a Rod with Invisibility costs 2*2*1600=6400 gp

\bigskip

A rod is able to cast its spell once per day.

Multiply the cost by 4 if you can cast it 2 times, multiply by 8 if you can cast it 3 times per day.

You can also cast the spell contained in the rod once more per day, after which the rod is destroyed.

A rod can contain a maximum spell level of 3.

The caster must know the spell he places on the rod.

Crafting a rod takes 1 day for every 500 gp of the base price.

\textbf{Item Creation Feat Required}: Craft Greater Magic Items.

\subsection{Add New Capabilities}\index{Add New Capabilities}\label{aggiungerecapacitamagiche}

Sometimes, lack of funds or time makes it impossible to craft the desired magic item, but fortunately, it is possible to enhance or modify a created magic item. Only time, gold, and the various prerequisites required by the new ability one wishes to add to the magic item place restrictions on the type of additional powers one can instill.

The cost to add additional abilities to an item is the same as if the item were nonmagical, minus the value of the original item. Thus, a +1 longsword can become a +2 vorpal longsword, and the cost to create is equal to a +2 vorpal longsword minus the cost of a +1 longsword.

When determining the price of a concocted magic item, many factors must be considered. The easiest way to decide the price is to compare the new item to an item that already has a price, and use that price as a guide.

\end{multicols}

\vfill

\begin{center}
\includegraphics[width=0.2\linewidth]{immagini/Rod_of_asclepius.png}

\textit{Asclepius' rod is an ancient Greek symbol associated with medicine. It consists of a snake coiled around a rod.}
\end{center}


\pagebreak

\section{Magical Item Rules}\index{Magical Item Rules}\hypertarget{identificareom}{}\label{regoleoggettimagici}

\begin{multicols}{2}

\lettrine[lines=2, lhang=0.33, loversize=0.25, findent=1.5em]{T}{hese} are the indications on the use of magic items.

\label{oggetti-magici}
\begin{itemize}[leftmargin=*]
\item
A character can \textbf{carry numerous (up to 10) magic items} on him but to determine the \textbf{Defence} bonus no more than 2 items can be added (eg 1 magic ring and a bracelet). Armour and Shield are not considered in this count.
\item
The same principle applies to the bonus to \textbf{Saving Throws}, you can only add bonuses from two objects.
\item
If the bonus is at \textbf{Characteristics} only the one with the highest bonus is counted.
\item
A character \textbf{may not wear more than two magic rings} otherwise they resonate causing 1d6 damage (not magically reducible or curable) per round for each ring beyond the second.
\item
To \textbf{recognize a magic item} and its abilities, a DC 30 Arcana check is required. \textbf{10 minutes}. With Arcana score 6 it costs 5 minutes, with 12 it costs 1 minute, with Arcana 18 it costs 1 Round.
\item
A \textbf{magic item that manifests spells} does not perform any Magic Tests. The \textbf{Saving Throw} it imposes, if not specified, is equal to 10 + level*2 of the spell it manifests.\index{Saving Throw from magical objects}
\item
\textbf{Activating Spell-Like Abilities}: Unless otherwise noted, activating an item's magical ability costs 2 Actions.
\item
A magical item that provides a \textbf{static bonus (or penalty)} applies its value even if the item has not been identified, the Arbiter will silently apply this bonus to Defence, Attack Rolls, Saving Throws.. . informing the player who perceives how the object interacts with the situation.

\end{itemize}

\subsubsection{Weapons}

\textbf{Weapons}: A weapon with a special ability must have at least a +1 bonus. Weapons cannot have the same special ability more than once.

The magical bonus of a \textbf{weapon can be identified} following two critical strikes in an attack roll or dedicating 1 hour of training, any talents or magical abilities remain hidden.

\subsubsection{Armour and Shields}

\textbf{Special Abilities}: Armor or shield with a special ability must have at least a +1 bonus. Armor and Shields cannot have the same special ability more than once.

+1 armor lowers the Competence penalty by 1 and the Movement penalty by 1 meter

An armor or \emph{shield} +2 removes 1 die from the Magic Test if added. 

+3 armor further removes 1 from the Proficiency penalty, reduces the Movement penalty by 1m and removes 1 die from the Magic Test.

\textbf{The cost of Weapons and Armour:} larger than Medium is at least double (or quadruple depending on size). Small Armour or Small Weapons cost the same amount as medium weapons and Armour while requiring less material.

\subsection{Size and Magic Items}\label{tagliaoggettimagici}

\label{taglia-e-oggetti-magici}

When a piece of magical clothing or jewelry is discovered, most often size isn't a concern—many magical clothing is easy for everyone to wear or magically adjusts to the wearer. As a rule, bounty should not prevent characters of various physical types from using a magic item.

There may be some rare exceptions, especially with items made for a specific race.

Randomly dropped weapons and Armour have a 30\% chance of being Small (01--30), 60\% of being Medium (31--90), and 10\% of be another size.

\subsection{Magical Items on Body}\index{Magical Items on Body}

\label{oggetti-magici-sul-corpo}

Many magic items must be worn by a character who wants to use them or benefit from their abilities. A creature of humanoid form can wear up to 10 magic items at a time. Each of these items must be worn over a specific part of the body called a "slot".

A humanoid-shaped body can wear magical equipment consisting of one item from each of the following groups, related to the part of the body on which the item is worn.

\textbf{Ring} (two max): Rings.

\textbf{Clothing}: cuirasses, Armour, tunics, and robes

\textbf{Belt}: belts.

\textbf{Neck}: amulets, necklaces, medallions, scarabs, brooches, talismans and scarves

\textbf{Hands}: gloves and gauntlets.

\textbf{Eyes}: Eyes, glasses and lenses.

\textbf{Feet}: shoes, boots and slippers.

\textbf{Wrist}: Bracelets and bangles.

\textbf{Shield}: shields.

\textbf{Shoulders}: capes and cloaks.

\textbf{Head}: hats, diadems, helmets, masks, crowns, bands and phylacteries

\textbf{Chest}: shirts, jackets, sweaters and cloaks.

\medskip

Of course, a character can own as many items of the same type as she likes. But additional magic items of the same type, beyond those provided in the slots, will not work.

\subsection{Saving Throws Against Magic Item Powers}\index{Saving Throw}

\label{tiri-salvezza-contro-i-poteri-degli-oggetti-magici}

Magic items normally reproduce spells or other magical effects. For a Saving Throw against magic or a magical effect generated by a magical item, the DC is 10 + manifested spell level x2 unless otherwise specified.

\subsection{Damaging Magic Items}\index{Damaging Magic Items}

\label{danneggiare-gli-oggetti-magici}

A magic item does not need to make a Saving Throw unless it is unattended, is specifically targeted by the effect, or its wielder rolls a natural 0 (three times 1) on its Saving Throw.

Magic items are always entitled to a Saving Throw against something that could harm them, even when a normal item of the same type would have no chance of making a Saving Throw. Magic items always use the same bonus on Saving Throws, regardless of type (Fortitude, Reflex, or Will). A magic item's Saving Throw bonus is equal to 2 + 2xlevel of the most powerful spell it hosts (or a +4 for each +1 it has). The only exceptions to this rule are intelligent magic items, which make Will saves based on their Wisdom scores.

\subsection{Repair Magic Items}\index{Repair Magic Items}
\label{riparare-gli-oggetti-magici}

Repairing a magic item takes materials and time, equal to half the time and cost to create it.

\subsection{Fillers, Doses and Multiple Uses}\index{Fillers}\index{Doses}\index{Multiple Uses}

\label{cariche-dosi-e-usi-multipli}

Many objects, especially wands and staves, have power limited to the number of charges they contain. Normally, items with charges never exceed the maximum of 20 charges (10 for staves). If similar items are found as a random part of a treasure, roll a 1d10+10 to determine the number of charges remaining. If an item has a maximum number of charges other than 20, roll randomly to see how many charges are left.

The prices shown are for items with full charges (when an item is crafted, it always has full charges). The value of an object depends on the number of residual charges, in the case of objects that can have a use even with few or no charges, the value remains higher.

\end{multicols}

\subsection{Acquire Magic Items}\index{Acquire Magic Items}\index{Table Acquire Magic Items}

\label{acquisire-oggetti-magici}

\bigskip

\begin{tabular}{lllll}
\textbf{Community Size} & \textbf{Base Value} & \textbf{Common} & \textbf{Uncommon} & \textbf{Rare}\\
\toprule
Settlement & 50gp & 1d2 items && \\
Hamlet & 200gp& 1d4 items && \\
Village & 500gp & 1d6 items & 1d2 items & \\
Small town & 1000gp & 1d4 items & 1d2 items & \\
Great country & 2000gp & 1d6 items & 1d4 items & 1d2 items\\
Small town & 4000gp & 2d4 items & 1d6 items & 1d4 items\\
Great city & 8000gp & 3d4 items & 2d4 items & 1d6 items\\
Metropolis & 16000gp &{*} & 3d4 objects & 2d4 objects\\
\end{tabular}

{*} Almost all minor magic items are found in a metropolis.

\begin{multicols}{2}

\bigskip

Magic items are valuable, and most large cities have at least one or two purveyors of magic items, from a simple potion seller to a blacksmith who specializes in forging magic swords. Of course, not every item in this manual is available in every city.

The following guidelines help Arbiters determine what items are available in a specific community. Assuming a campaign with an average level of magic. Some cities may deviate greatly from this baseline at the Arbiter's discretion. The Arbiter should keep a list of items available from each merchant and should occasionally replenish the stock with new acquisitions.

The number and types of magic items available in a community depend on its size. Each community has a base value tied to it (see Table: Available Magic Items).

there is a 75\% chance that any item of that value or less will be easily found for sale in that community. Additionally, the community has a number of other items for sale. These objects are determined at random and are divided into categories (minor, medium or major).

After determining the number of items available in each category, consult the Random Magic Item Generation chapter to determine the type of each item (potion, scroll, ring, weapon, etc.) before moving on to specific tables to determine the item exact. Reroll whenever items do not match the community's base value.

If the use of magic in the campaign in which you play is rare, you should halve the base value and the number of items in each community. In campaigns with extremely rare magic or no magic, there may be no magic items for sale at all. Arbiters running these types of campaigns should plan to make adjustments to the challenges characters face due to the lack of magic items.

Campaigns with abundant magic items may have communities with double the established base value and random magic items available. Alternatively, all communities could be set to be one size category larger for the purpose of determining the available magic items. In a campaign with very common magic, all magic items can be bought in a metropolis.

Nonmagical items and tools are typically available in a community of any size unless the item is very expensive, such as full plate Armour, or made of an unusual material, such as an adamantium longsword. These items should follow the base value guideline for determining their availability, at the Arbiter's discretion.

\end{multicols}

\vfill

\begin{center}
\includegraphics[keepaspectratio,width=0.90\textwidth]{immagini/Alchemical_laboratory_Wellcome_M0005193.png}

\textit{Alchemical laboratory}
\end{center}

\pagebreak


\section{Random Generation of Magic Items}\index{Random Generation of Magic Items}\label{generazionetesorimagici}\hypertarget{generazionetesorimagici}{}

\begin{changemargin}{0.3cm}{0.3cm}\begin{emphasis}
{Like any unrequited love, even that pays off for things in the long run. (Adolfo Bioy Casares)}

\end{emphasis}\end{changemargin}\medskip

\begin{multicols}{2}

In preparing the adventure, the Arbiter can place the magic objects he prefers, if there is a need, or in pure OSR style he can rely on random generation.

This approach is not always recommended, the results could upset the adventure if not the whole campaign! Usually if an "enemy" has a magic item there is a reason and this item will have a purpose. The fact remains that every now and then, rolling dice on the tables of random generation of magical treasures is very satisfying and fun!

\begin{changemargin}{0.3cm}{0.3cm}\begin{narrator} %box narratore
Yeru is a world with a low magical profile, magical objects exist but are rare and even more rare the powerful ones. While natural potions and small trinkets can be found everywhere what has prevented the creation of so many items is the cost to create them. Maybe they were once more accessible but now the creation of the most powerful items, and I mean swords +3 not wondrous or almost unique items, requires resources that almost nobody owns or has an interest in spending.

This means that if you want to have any hope of obtaining some magic item you need to explore the oldest, least known areas... and the deeper you go, the more likely you are to find something.
\end{narrator}\end{changemargin}

First of all, it is necessary to establish what type of object will be generated.

\medskip

\textbf{Table: Types of Magic Item}\index{Table Types of Magic Item}

\medskip

\begin{tabular}{lc}
\textbf{Type of magic item}&\textbf{3d6}\\
Amulets, Necklaces, Jewels&3-4\\
Belts, Helmets, Boots and Gloves&5-6\\
Armour and Shields&7-8\\
Magic Weapons&9-10\\
Potions, Filters and Oils&11-13\\
Wands, Staves and Rods&14\\
Rings&15\\
Hats, Cloaks, Glasses, Tunics&16\\
Manuals and Tomes&17\\
Miscellaneous Magic Items&18\\
\end{tabular}

\subsubsection{Weapons}

\textbf{Table: Weapon Generation}\index{Table Weapon Generation}

\medskip

\begin{tabularx}{0.45\textwidth}{lX}
\textbf{1d100} & \textbf{Magic Bonus}\\
1-50 & +1\\
51-65 & -1 Cursed\\
66-72 & +2\\
73-76 & +3\\
77-79 & +4\\
80 & +5\\
81-87 & reroll + Weapon Special Ability Type 3\\
88-91 & reroll + Type 2 Weapon Special Ability\\
92-94 & reroll + Type 1 Weapon Special Ability\\
95-100 &-2 Cursed\\
\end{tabularx}

\medskip

When in \textit{Magic Bonus} there is written \textit{roll back + Weapon Type Special Ability...} it means that you must roll back the 1d100, ignoring other results above 80 and keep the magic bonus obtained, then you can roll on the \textit{Weapon Type Special Abilities Table...} resulting.

\medskip

\textbf{Table: Weapon Special Ability Type 1}\index{Table Weapon Special Ability Type 1}

\medskip

\begin{tabular}{ll}
\textbf{1d100} & \textbf{Weapons Special Ability Type 1}\\
1-8 &Accumulate Spells\\
9-16 &Anathema\\
17-21& Dancing\\
22-27& Defensive\\
28-34& Destroyer of Giants\\
35-41& Destruction\\
42-47& Bright Energy\\
48-54& Gloriosa\\
55-60& Guardian\\
61-63& Lucky\\
65-70& Thief of the Nine Lives\\
71-73& Sacred\\
74-80& Ghost Touch\\
81-86& Vampire\\
87-92& Speed\\
93-99& Cursed Weapon\\
100 &Vorpal\\
\end{tabular}

%\medskip
%
%\begin{center}
%\includegraphics[width=0.55\linewidth]{immagini/armatura-med.png}
%\end{center}
%
%\medskip

\textbf{Table: Weapon Special Ability Type 2}\index{Table Weapon Special Ability Type 2}

\medskip

\begin{tabular}{ll}
\textbf{1d100} & \textbf{Weapons Special Ability Type 2}\\
1-8& Conductive\\
9-16&Courageous\\
17-23&Cruel\\
24-30&Duel\\
31-36&Fury Within\\
37-43&Vital Pulse\\
44-58&Immoral\\
59-60&Lethal\\
61-65&Perfidious\\
66-69&Piecious\\
70-74&Punitive\\
75-79&Cursed\\
80-85&Disdainful\\
87-95&Terror\\
95-100 & Titanic\\
\end{tabular}

\bigskip

\begin{center}
\includegraphics[width=0.7\linewidth]{immagini/shield1.png}
\end{center}

\textbf{Table: Weapon Special Ability Type 3}\index{Table Weapon Special Ability Type 3}

\medskip

\begin{tabular}{ll}
\textbf{1d100} & \textbf{Weapons Special Ability Type 3}\\
1-4& Adaptive\\
5-8   & Sharp\\
9-12  & Dragon Slayer\\
13-16 & Giant Slayer\\
17-20 & Hunter\\
21-24 & Corrosive\\
25-28 & Designator\\
29-32 & Distance\\
33-36 & Extinguish Fire\\
37-40 & Fanatizer\\
41-44 & Injury\\
45-48 & Dazzling\\
49-52 & Gelida\\
53-56 & Fiery\\
57-60 & Marine\\
61-65 & Masking\\
65-69 & Ghost Ammo\\
70-72 & Infinite Ammo\\
73-76 & Planar\\
77-80 & Prehensile\\
81-82 & Researcher\\
83-84 & Returning\\
85-88 & Thundering\\
89-91 & Transforming\\
92-95 & Thing Finder\\
96-100 & Cursed Weapon\\
\end{tabular}

\medskip

\subsubsection{Armour and Shields}

\textbf{Table: Armour/Shield generation}\index{Table Armour/Shield generation}

\medskip

\begin{tabularx}{0.45\textwidth}{lX}
\textbf{1d100} & \textbf{Magic Bonus}\\
1-50 & +1\\
51-65 & -1 Cursed\\
66-72 & +2\\
73-76 & +3\\
77-79 & +4\\
80 & +5\\
81-85 & reroll + Armour / Shields Special Ability Type 2\\
86-90 & reroll + Armour / Shields Special Ability Type 1\\
91-100 &-2 Cursed\\
\end{tabularx}

\medskip

When in \textit{Magic Bonus} there is written \textit{reroll + Armour/Shield Special Ability Type...} it means that you must reroll the 1d100, ignoring other results above 80 and keep the magic bonus obtained, then you can roll on the resulting \textit{Table Armour/Shield Special Ability Table...}.

\textbf{Table: Armour/Shields Special Ability Type 1}\index{Table Armour/Shields Special Ability Type 1}

\begin{tabularx}{0.45\textwidth}{lX}
\textbf{1d100} & \textbf{Armour/Shield Special Ability Type 1}\\
1-5 & Aries\\
6-10 &Balanced\\
11-15& Archer's Bracers\\
16-20& Bracers of Defence\\
21-25& Greater Defence Bracers\\
26-30& Brilliant\\
31-35& Determination\\
36-40& Spell Defence\\
41-45& Elegant\\
46-50& Hospital\\
51-55& Resistance to Poison\\
56-60& Energy Resistance\\
61-65& Greater Energy Resistance\\
66-70& Wild\\
71-75& Dragonscale\\
76-80& Animated Shield\\
81-85& Bullet Attraction Shield\\
86-90& Breath of the Dragon\\
91-95& Ghost Touch\\
95-100 & Armour/Cursed Shield\\
\end{tabularx}

\medskip

\textbf{Table: Armour/Shields Special Ability Type 2}\index{Table Armour/Shields Special Ability Type 2}

\begin{tabularx}{0.45\textwidth}{lX}
\textbf{1d100} & \textbf{Armour/Shield Special Ability Type 2}\\
1-5 &Blinding\\
6-10 &Adamantium\\
11-15& Amorphous\\
16-20& Antihemorrhagic\\
21-25& Wrangler\\
26-30& Load\\
31-35& Demon Armour\\
36-40& Denial\\
41-45& Sweatshirt\\
46-50& Ethereal Form\\
51-55& Invulnerability\\
56-60& Untraceable\\
61-65 &Masking\\
66-70 &Mithral\\
71-75 &Shadow\\
76-80 &Perceptive\\
81-85 & Titanic\\
86-90 &Vulnerability\\
91-100& Armour/Cursed Shield\\
\end{tabularx}

\medskip

When the Cursed special ability is indicated, you must reroll and reverse the weapon's magic bonuses, so a +2 Armour or shield becomes a -2 Armour or shield.

\subsubsection{Amulets, Necklaces and Jewels}\index{Table Amulets, Necklaces and Jewels}


\begin{tabular}{ll}
\textbf{Item Type}&\textbf{1d8}\\
Amulets, Necklaces and Jewels Type 1&1-6\\
Amulets, Necklaces and Jewels Type 2&7-8\\
\end{tabular}

\medskip

\begin{tabularx}{0.45\textwidth}{lX}
\textbf{1d100} & \textbf{Amulets, Necklaces and Jewels Type 1}\\
1-8 & Anti-Poison Charm\\
8-12 & Gangrene Charm\\
12-18 & Healing Charm\\
19-26 & Amulet Against Possession\\
27-34 & Inescapable Tracking Charm\\
35 & Amulet of the Planes\\
36-42 & Amulet of Protection from Detection and Tracking\\
42-46 & Charm of Physical Endurance\\
47-53 & Circlet of Blast\\
53-60 & Necklace of Adjustment\\
61-70 & Necklace of Strangulation\\
71-77 & Fireball Necklace\\
78-83 & Rosary Necklace\\
84-90 & Death Beetle\\
91-100 & Protection Beetle\\
\end{tabularx}

\medskip

\begin{tabularx}{0.45\textwidth}{lX}
\textbf{1d100} & \textbf{Amulets, Necklaces and Jewels Type 1}\\
1-7 & Elemental Gem\\
8-13& Gem of Luminosity\\
9-16& Sight Gem\\
17-26& Monster Lure Jewel\\
27-33& Medallion of Thoughts\\
34-41& Feather Fall Medallion\\
42-49& Pearl of Wisdom\\
50-57& Pin of Defence\\
58-60& Talisman of Pure Good\\
61-62& Talisman of Extreme Evil\\
63-70& Protection from Poison Talisman\\
71-78& Talisman of Health\\
79-85& Sphere Talisman\\
86-100& Worthless jewel
\end{tabularx}


\begin{center}
\includegraphics[width=0.8\linewidth]{immagini/gauntlet.png}\\
\end{center}


\subsubsection{Belts, Helmets, Boots and Gloves}\index{Table Belts, Helmets, Boots and Gloves}

\begin{tabularx}{0.45\textwidth}{lX}
\textbf{1d100} & \textbf{Belts, Helmets, Boots \& Gloves}\\
1-3 &Girdle of Giants\\
3-6 &Girdle of the Dwarves\\
6-11 &Helmet of Understanding Tongues\\
12 &Helmet of Luster\\
13-17 &Helmet of the Underwater Movement\\
18-22 &Helmet of Telepathy\\
23-26 &Helm of Teleportation\\
27-31 & Bullet Grabber Gloves\\
31-35 &Gloves of Orc Power\\
36-41 & Swimming and Climbing Gloves\\
41-46 &Gloves of Dexterity\\
47-52 &Clumsy Gloves\\
53-58 &Spider's Slippers\\
59-63 & Winged Boots\\
64-66 & Running and Jumping Boots\\
67-77 &Elf Boots\\
78-83 &Winter Boots\\
84-90 &Boots of Levitation\\
91-95 & Boots of Speed\\
96-100 &Dancing Boots\\
\end{tabularx}

\subsubsection{Wands, Staves and Rods}\index{Table Wands, Staves and Rods}

Roll 1d8 to determine if a Wand or Staff or Rod is found.

\medskip

\begin{tabular}{ll}\\
\textbf{Item Type}&\textbf{1d8}\\
Wands&1-4\\
Wands&5-7\\
Rods&8\\
\end{tabular}

\medskip

\textbf{Table: Wand Generation}\index{Table Wand Generation}

\medskip

\begin{tabularx}{0.45\textwidth}{lX}
\textbf{1d100} & \textbf{Wand}\\
1-5& Metal Search Wand\\
6-10 &Wand of Spellbolts\\
11-15 &Wand of Convenience\\
16-20 &Wand of Lightning\\
21-25& Wand of Fire\\
26-30& Wand of Ice\\
31-35& Individual's Wand. of Magic\\
36-38& Individual's Wand. of Enemies\\
39-44& Wand of Illusions\\
45-48& Wand of Finding Secret Doors\\
46-50& Wand of Light\\
51 &Warmage Wand\\
52 &Wand of Metamorphosis\\
53 &Wand of Wonders\\
54 &Wand of Denial\\
55-60& Fireball Wand\\
61-65 &Wand of Paralysis\\
66-70& Wand of Fear\\
71-75 &Trap Detecting Wand\\
76-80& Wand of Secrets\\
81-85& Web Wand\\
86-90& Wand of Binding\\
91-95& Wand of Assisted Escape\\
96-100&Cursed Wand\\
\end{tabularx}

\medskip

\textbf{Table: Staff Generation}\index{Table Staff Generation}

\medskip

\begin{tabularx}{0.45\textwidth}{lX}
\textbf{1d100} & \textbf{Staff}\\
62 &Archmage's Staff\\
63-65& Withering Stick\\
66-67& Staff of the Woods\\
68-70& Staff of Charm\\
71-72& Staff of Striking\\
73-74& Staff of Fire\\
75-76& Frost Staff\\
77-78& Staff of Healing\\
79-80& Staff of Swarming Insects\\
81-82& Python's Staff\\
83 &Staff of Power\\
84-86& Staff of Thunder and Lightning\\
87 &Staff of Sorcery\\

\end{tabularx}

\medskip

\textbf{Table: Rod Generation}\index{Table Rod Generation}

\medskip

\begin{tabularx}{0.45\textwidth}{lX}
\textbf{1d100} & \textbf{Rod}\\
1-10&Rod of Enchantment\\
11-20&Rod of Absorption\\
21-30&Immovable Rod\\
31-41&Rod of Mighty Stroke\\
42-50&Rod of Sovereign Strength\\
51-60&Rod of Readiness\\
61-70&Rod of Security\\
71-80&Rod of Sovereignty\\
81-90& Tentacle Rod\\
91-100& Cursed Rod\\
\end{tabularx}

\begin{center}
\includegraphics[width=0.8\linewidth]{immagini/cupdrinking.png}\\

\textit{Drinking cup depicting scenes from the Odyssey, Athens 550–525 B.C.}
\end{center}

\subsubsection{Potions, Filters and Oils}\index{Table Potions, Filters and Oils}

\begin{tabular}{ll}
\textbf{Potion}&\textbf{1d8}\\
Potion Type 1 &1-4\\
Potion Type 2 &5-7\\
Potion Type 3 &8\\
\end{tabular}

\medskip

\begin{tabular}{ll}
\textbf{1d100} & \textbf{Potion Type 1}\\
1-8 &Potion of Climbing\\
9-15 &Potion of Growth\\
16-23 &Potion of Heroism\\
24-29 &Potion of Gas Shape\\
30-35 &Potion of Giant Strength\\
36-46 &Potion of Healing\\
47-53 &Potion of Deception\\
54-64 &Potion of Invisibility\\
65-74 &Potion of Levitation\\
77-78 &Potion of Endurance\\
79-84 &Water Breathing Potion\\
84-90 &Potion of Shrinking\\
91-95 &Potion of Speed\\
96-100 &Potion of Flight\\
\end{tabular}

\medskip

\begin{tabular}{ll}
\textbf{1d100} & \textbf{Potion Type 2}\\
1-10 & Animal Clairaudience Potion\\
11-20 & Animal Clairvoyance Potion\\
21-28 & Animal Control Potion\\
29-33 &Dragon Control Potion\\
34-38 &Potion of Undead Control\\
39-49 & People Control Potion\\
50-55 &Plant Control Potion\\
56-66 &Potion of Invulnerability\\
67-77 & Potion of Mind Reading\\
78-85 &Potion of Poison\\
86-95 &Major Healing Potion\\
96-100 &Potion of Major Poison\\
\end{tabular}

\medskip

\begin{tabular}{ll}
\textbf{1d100} & \textbf{Potion Type 3}\\
1-13 & Love Potion\\
14-27 & Treasure Discoverer Filter\\
28-40 & Oil of Sharpness\\
41-53 & Ethereal Form Oil\\
54-66 & Slipperiness Oil\\
67-79 & Animal Friendship Potion\\
80-85 & Potion of Longevity\\
86-95 & Potion of Metamorphosis\\
96-100& Major Poison Potion\\
\end{tabular}

\subsubsection{Rings}\index{Table Rings}

\begin{tabular}{ll}
\textbf{Ring}&\textbf{3d6}\\
Ring Type 1 &3-16\\
Ring Type 2 &17-18\\
\end{tabular}

\medskip

\begin{tabular}{ll}
\textbf{1d100} & \textbf{Rings Type 1}\\
1-5 & Spell Storage Ring\\
6-13 & Ring of Aries\\
14-21 & Feather Drop Ring\\
22-28 & Water Walk Loop\\
29-35 & Ring of Heat\\
36-41 & Ring of Weakness\\
42-47 & Ring of Evasion\\
48-50 & Animal Influence Ring\\
51-55 & Ring of Deception\\
56-61 & Ring of Freedom of Action\\
61-67 & Swimming Ring\\
68-77 & Protection Ring\\
76-84 & Ring of Resistance\\
85-93 & Jump Ring\\
93-100 & Telekinesis Ring\\
\end{tabular}

\medskip

\begin{center}
\includegraphics[width=0.8\linewidth]{immagini/romanring.png}
\end{center}

\begin{tabular}{ll}
\textbf{1d100} & \textbf{Rings Type 2}\\
1-8 & Ring of Control people\\
9-17& Plant Control Ring\\
18-23& Water Elemental Ring\\
24-29& Ring of Air Elementals\\
31-36& Fire Elemental Ring\\
37-42& Earth Elemental Ring\\
43-48& Djinni Summoning Ring\\
49-56& Spell Repelling Ring\\
57-65& Ring of Invisibility\\
66-75& Regeneration Ring\\
76-83& Mindshield Ring\\
84-90& Shooting Star Ring\\
91-92& Ring of Three Wishes\\
92-96& Ring of Three Wishes sold out\\
97-100 & Ring of X-Ray Sight\\

\end{tabular}


\subsubsection{Hats, Cloaks, Glasses, Tunics}\index{Table Hats, Cloaks, Glasses, Tunics}

\begin{tabularx}{0.45\textwidth}{lX}
\textbf{1d100} & \textbf{Hats, Cloaks, Glasses, Tunics}\\

1-3 &Bandana of Intelligence\\
4-10 &Camouflage Hat\\
11-17& Arachnid Cloak\\
18-23& Cloak of the Charlatan\\
24-29& Cloak of Distortion\\
30-40& Elven Cloak\\
41-45& Manta Ray\\
46-50& Bat Cloak\\
51-57& Cloak of Protection\\
58-62& Spell Resistance Cloak\\
63-68& Cloak of Venom\\
69-72& Petrification Eyes\\
73-75& Charming Eyes\\
76-77& Eagle Eyes\\
78-80& Eyes of Detailed Sight\\
80-82& Night Glasses\\
83-86 &Tunic of Camouflage\\
87 &Tunic of the Archmage\\
88 &Tunic of Shimmering Colors\\
89-91& Tunic of Weakening\\
92-94 &Tunic of the Eyes\\
95-99 &Tunic of Useful Items\\
100 &Tunic of the Stars\\
\end{tabularx}


\subsubsection{Manuals and Tomes}\index{Table Manuals and Tomes}

\begin{tabularx}{0.45\textwidth}{lX}
\textbf{1d100} & \textbf{Manuals and Tomes}\\
1-9 & Golem Handbook\\
10-24 & Handbook of Good Health\\
25-40 & Action Speed Manual\\
40-54 & Exercise Manual\\
55-69 &Tome of Authority and Influence\\
70-84 &Tome of Understanding\\
85-100& Tome of Clear Thought\\
\end{tabularx}

\subsubsection{Miscellaneous Magic Items}\index{Table Miscellaneous Magic Items}

Roll 1d10 to determine if a rare or legendary magic item or one from the miscellaneous magic item lists is found

\medskip

\begin{tabular}{ll}
\textbf{Item Type} & \textbf{1d12}\\
Miscellaneous Magic Items 1&1-3\\
Miscellaneous Magic Items 2&4-5\\
Miscellaneous Magic Items 3&6-7\\
Miscellaneous Magic Items 4&8-9\\
Miscellaneous Magic Items 5&10-12\\
Rare and Legendary&10\\
\end{tabular}

\medskip


\subsubsection{Rare and Legendary}\index{Table Rare and Legendary}

\medskip

\begin{tabularx}{0.45\textwidth}{lX}
\textbf{1d100} & \textbf{Magic Item}\\
1-3 &Wings of Flight\\
4-6 &Iron Vial\\
7-10 &Water Elemental Amphora\\
11-12& Crab Apparatus\\
13-15& Folding Boat\\
17-20& Type III Preservative Bag\\
21-24& Type IV Preservative Bag\\
25-28& Bean Bag\\
29-30& Efreeti Bottle\\
31 &Potions Pitcher\\
32-33& Invocation Candle\\
34-35 & Horn of Valhalla\\
36-39 &Philactery of youth\\
40-42 &Fortress Instant\\
43-45 &Deck of Wonders\\
46-49 &Miniature of Wonderful Power\\
50-53 &Ammo of Kill\\
54-58 &Crystal Ball\\
59-62 &Scroll Against Elementals\\
63-65 &Scroll Against the Undead\\
66-70 & Pipe of the Sewers\\
71-75 & Pigments of Wonders\\
76-83 &Cubic Portal\\
84-85 &Well of Many Worlds\\
86-87 &Mirror of Mental Ability\\
88-89 &Mirror Traps Life\\
90-91 &Sphere of Annihilation\\
92-94 &Thurible Air Elemental\\
95-96 & Portable Compartment\\
97-98 &Hooves of Speed\\
99-100 &Hooves of the Zephyr\\
\end{tabularx}

\medskip

\subsubsection{Miscellaneous Magic Items 1}\index{Table Miscellaneous Magic Items 1}

\begin{tabularx}{0.45\textwidth}{lX}
\textbf{1d100} & \textbf{Miscellaneous Magic Items 1}\\
1-8& Purifying Water\\
9-17&Battle of the Opening\\
18-27&Bag Preservative Type I\\
28-34& Climbing Rope\\
35-43&Quiver Efficient\\
44-48&Locate Arrow\\
49-52&Crossbow of Arcane Dart\\
53-60&Lantern of Revelation\\
61-70&Pearl of Power\\
71-80&Good Luck Stone\\
81-83&Universal Solvent\\
84-94&Restorative Ointment\\
95-100&Practical Backpack\\
\end{tabularx}

\subsubsection{Miscellaneous Magic Items 2}\index{Table Miscellaneous Magic Items 2}


\begin{tabularx}{0.45\textwidth}{lX}
\textbf{1d100} & \textbf{Miscellaneous Magic Items 2}\\
1-8 &Brazier of Fire Elementals\\
9-17 &Brazier of Cursed Sleep\\
18-27& Cold Protection Cube\\
28-34& Air Elemental Censer\\
35-43& Entangling Net\\
44-52& Entrapping Net\\
53-60& Broom of Animated Attack\\
61-70& Broom Steering wheel\\
71-80& Broom of Cursed Flight\\
81-88& Mirror of Duplication\\
89-90& Flying Carpet\\
99-100& Hoe of the Titans\\
\end{tabularx}

\medskip
\subsubsection{Miscellaneous Magic Items 3}\index{Table Miscellaneous Magic Items 3}

\begin{tabularx}{0.45\textwidth}{lX}
\textbf{1d100} & \textbf{Miscellaneous Magic Items 3}\\
1-8 &Cancellation Scholarship\\
9-18 & Pitcher of Infinite Water\\
19-26& Dimension Logs\\
27-35& Supreme Glue\\
36-40& Incense of meditation\\
41-51& Protection scroll against magic\\
52-60& Developer Powder\\
61-70& Dust of Vanishing\\
71-82& Sneeze Powder\\
83-90& Arcane Stone\\
91-96& Stone of Weight\\
97-100& Arcane Fan\\
\end{tabularx}

\begin{center}
\includegraphics[width=0.8\linewidth]{immagini/ancientdrum.png}
\end{center}


\subsubsection{Miscellaneous Magic Items 4}\index{Table Miscellaneous Magic Items 4}

\begin{tabularx}{0.45\textwidth}{lX}
\textbf{1d100} & \textbf{Miscellaneous Magic Items 4}\\
1-5 & Vial of Curses\\
6-10 &Battle of Cannibalism\\
11-16& Type II Preservative Bag\\
17-20& Devouring Bag\\
21-25& Steaming Bottle\\
26-31& Portable Hole\\
32-37& Healthy Air Necklace\\
38-43& Rope of Entanglement\\
44-48& Choke Rope\\
49-50& Horn of Destruction\\
51-52& Cube of Force\\
53-58& Binding Iron Bands\\
59-64& Phylactery against motionlessness\\
65-69& Incense of Obsession\\
70-71& Illusion Deck\\
72-76& Hypnotic Crystal Ball\\
77-82& Scroll against werewolves\\
83-84& Earth Elemental Stone\\
85-89& Fife of Fright\\
90-92& Arcane Feather\\
93-94& Dryness Dust\\
95-96& Drums of Panic\\
97-98& Drums of Stunning\\
99-100& Thurible of Cursed Evocation\\
\end{tabularx}


\begin{center}
\includegraphics[width=0.6\linewidth]{immagini/ancientbraziers2.png}

\textit{Teotihuacano Old God vessels: Top - stone brazier in Natural History Museum of Los Angeles County}
\end{center}


\pagebreak

\section{Description of Magic Items}\index{Description of Magic Items}


Magic items are presented alphabetically by grouping categories. A magic item's description provides the item's name, its category, rarity, and magical properties.

Although the costs are reported, it is always a good idea to grant magical objects such as prizes, treasure, following a mission.

In principle, a Common item, the only one that could easily be found in a large city, can cost from 50 to 100 gp, an Uncommon one between 150 and 500 gp, a Rare one between 500 and 5000 gp, a Very Rare up to 30,000 gp and over will be only legends...

Items with a bonus over +2, or Legendary, are never bought, it must be an epic adventure to find them.



\medskip

Spells are also magical objects and as such, if the Arbiter allows, they can be purchased (horror! there is nothing more beautiful than finding a new spell among the treasures of an adventure).

A spell costs level*level*level*80 gold coins\index{Buy spells}

\bigskip

\subsection{Magic Weapons Special Abilities}

\lettrine[lines=2, lhang=0.33, loversize=0.25, findent=1.5em]{W}{eapon} with a special ability must have at least +1 magic bonus.

Here are listed the magical abilities that an armour, shield or weapon can have in addition to the generic magical bonus (+1,+2....). Use this list as guidelines and examples, same for prices, use them as an indication of rarity.

\index[MagicItem]{Magic Weapons!Magic Weapons}\smallskip* \textbf{Magic Weapon}

\textit{Weapon (any)} +1 1800 gp, +2 6000 gp, +3 17000 gp, +4 45000 gp, +5 80000 gp

You have a bonus on attack rolls and damage rolls made with this weapon. The bonus is determined by the rarity of the weapon. Some magical weapons have additional properties, such as emitting light.

\smallskip* \textbf{Accumulate Spells}\index[MagicItem]{Magic Weapons!Accumulate Spells}

A spellstorage weapon allows a spellcaster to store one targeted spell up to level 3 in the weapon. The spell must have a standard casting time of 2 Actions. Whenever the weapon strikes a creature and the latter takes damage, the wielder of the weapon can unleash the spell as an immediate action.

Once the spell is cast, a spellcaster can store any other targeted spell within it, always up to level 3.

The weapon magically reveals to the wielder the name of the spell currently held. A randomly created Spell Storage weapon has a 50\% chance of already having an enchantment contained within it. This special ability can only be added to melee weapons.

A spell-storing weapon emits a strong aura of the evocation school, plus the aura of the spell it contains.

\textbf{Details}: Strong and variable Invocation Aura; Greater Magic Item Crafting Requirements, Cost +3000 gp.

\smallskip* \textbf{Adaptive}\index[MagicItem]{Magic Weapons!Adaptive}

This ability can only be added to composite bows. An adaptive bow reacts to the wielder's strength, acting like a bow with a Strength bonus equal to that of the wielder. The wielder can fire with a lower Strength bonus (and cause less damage) if she wishes.

\textbf{Details}: Weak Transmutation Aura; Craft Magic Item Requirements, Animals and Plants List; Cost +1500 gp.

\smallskip* \textbf{Sharpened}\index[MagicItem]{Magic Weapons!Sharpened}

This ability on a critical strike counts the number of 6s rolled by 1. Only slashing or piercing melee weapons can be sharpened.

\textbf{Details}: Moderate Transmutation Aura; Greater Magic Item Crafting Requirements, Land List; Cost +5000 gp.

\index[MagicItem]{Magic Weapons!Dragon Slayer}\smallskip* \textbf{Dragon Slayer}

When you strike a dragon with this weapon, the dragon takes an additional 3d6 points of damage of the weapon's type. For the purposes of this weapon, "dragon" is any creature of the type dragon.

\textbf{Details}: Moderate Invocation Aura; Craft Greater Magic Item requirements; Cost +8000 gp.

\index[MagicItem]{Magic Weapons!Giant Slayer}\smallskip* \textbf{Giant Slayer}

When you strike a giant with this weapon, the giant takes an additional 2d6 points of damage of the weapon's type and must succeed at a DC 18 Fortitude save or be knocked prone. For the purposes of this weapon, "giant" means any creature of the type giant.

\textbf{Details}: Moderate Invocation Aura; Craft Greater Magic Item requirements; Cost +8000 gp.

\smallskip* \textbf{Destroyer of Giants}\index[MagicItem]{Magic Weapons!Destroyer of Giants}

You must wear a \textit{belt of giants} (any variety) and the \textit{gloves of orc power} in order to use this weapon.

As you use the hammer, your Strength score increases by 2 (to a maximum of 7).

When you roll a crit on an attack roll made with this weapon against a giant, the giant must succeed on a DC 21 Fortitude save or die.

You can expend 1 charge and make a ranged weapon attack by throwing it as if it had a range of 6 meters. If the attack hits, the hammer produces thunder audible up to 100 meters away. The target and all creatures within 10 meters of it must succeed on a DC 21 Fortitude save or be stunned until the end of your next round.

The hammer has 5 charges, and regains 1 expended charge each day at dawn.

\smallskip* \textbf{Anathema}\index[MagicItem]{Magic Weapons!Anathema}

A bane weapon excels at attacking certain creatures. Against the favored enemy, its effective bonus becomes +2. The weapon also deals an additional +2d6 points of damage against that foe. The following table is used to randomly determine the weapon's favored enemy:

\medskip

\begin{tabular}{ll}
d\% &Enemy Favored\\
01-05 &Aberrations\\
06-09 &Beasts\\
10-16 &Constructs\\
17-22 &Dragons\\
23-27 &Fatati\\
28-60 &Humanoids (choose subtype)\\
61-70 &Magical Creatures\\
71-72 &Slimes\\
73-88 &Fiends\\
89-90 &Plants\\
91-98 &Undead\\
99-100 &Insects\\
\end{tabular}

\medskip

\textbf{Details}: Moderate Summoning Aura; Craft Greater Magic Item Requirements, Abjuration List; Cost +3000 gp.

\smallskip* \textbf{Hunter}\index[MagicItem]{Magic Weapons!Hunter}

A hunter's weapon helps the wielder locate and capture prey. When the weapon is held in hand, the wielder gains the weapon's bonus on Survival checks made to track any creatures the weapon has damaged during the previous day. Deals +1d6 damage to creatures tracked with Survival by the wielder during the previous day.

\textbf{Details}: Moderate Divination Aura; Requirements to Craft Greater Magic Items, Locate Animals and Plants; Cost +3000 gp.

\smallskip* \textbf{Conductive}\index[MagicItem]{Magic Weapons!Conductive}

A conductive weapon is able to channel the energy of a magical ability that requires a melee or ranged touch attack to strike its target.

When the wielder makes a successful attack of the appropriate type, she can choose to expend two uses of her spell-like ability to channel it through the weapon, to strike the opponent, who is affected by the weapon's attack and those special ability (such as channel energy, lay on hands...).

This weapon special ability can only be used once per round (even if you have multiple conductive weapons).

\textbf{Details}: Moderate Necromancy Aura; Craft Greater Magic Item Requirements, Mage Hand; Cost +3000 gp.

\smallskip* \textbf{Brave}\index[MagicItem]{Magic Weapons!Brave}

This special ability can only be added to a melee weapon. A Brave weapon fortifies the wearer's courage and morale in battle. The wielder gains a bonus on Saving Throws against fear equal to the weapon's bonus.

\textbf{Details}: Weak Enchantment Aura; Craft Magic Item requirements, Heroism, Fear; Cost +3000 gp.

\smallskip* \textbf{Corrosive}\index[MagicItem]{Magic Weapons!Corrosive}

On command, a Corrosive weapon coats itself in a layer of acid that deals an additional 1d6 points of acid damage when it hits the target. The acid does not harm the wielder. The effect lasts until a new command is given.

\textbf{Details}: Moderate Invocation Aura; Craft Magic Item Requirements, Acid Arrow; Cost +3000 gp.

\smallskip* \textbf{Cruel}\index[MagicItem]{Magic Weapons!Cruel}

A cruel weapon feeds on fear and pain. When the wielder strikes a frightened creature with a dire weapon, the creature is sickened for 1 round. When the wielder uses the weapon to knock unconscious or kill a creature, she gains 5 temporary Hit Points that last for 10 minutes.

\textbf{Details}: Weak Necromancy Aura; Craft Magic Item Requirements, Fear, Cost +3000 gp.

\smallskip* \textbf{Dancing}\index[MagicItem]{Magic Weapons!Dancing}

As a standard action, a dancer weapon can be released to fight on its own. The weapon fights for 4 rounds using the user's Defence and then falls to the ground.

It always stays with the person who released it, even if they move by physical or magical means. If the one who released it has a free hand he can pick up the weapon it is attacking on its own, as an immediate action, but once it is picked up, the sword can no longer dance (attack on its own) for 4 rounds.

This ability can only be added to melee weapons.

\textbf{Details}: Strong Transmutation Aura; Requirements to Craft Greater Magic Items, Animate Objects; Cost +25,000 gp.

\smallskip* \textbf{Designator}\index[MagicItem]{Magic Weapons!Designator}

This special ability can only be added to ranged weapons or ammunition. Whenever a ranged weapon or ammunition with this ability strikes a creature, it magically designates the target. All allies gain a +2 bonus on attack rolls for 1 round. Multiple successful hits on the same target do not increase bonuses or their duration.

\textbf{Details}: Moderate Enchantment Aura; Craft Requirements Greater Magic Items, Light; Cost +6000 gp.

\smallskip* \textbf{Defensive}\index[MagicItem]{Magic Weapons!Defensive}

A defensive weapon allows the wielder to transfer some or all of the weapon's bonus to his Defence as a bonus that stacks with any other bonuses. As an immediate action, the wielder can choose how to dispose of the weapon's bonus at the start of the round, before using it, and the Defence bonus lasts until the next round.

\textbf{Details}: Aura Moderate Abjuration; Craft Requirements Greater Magic Items, shield; Cost +3000 gp.

\smallskip* \textbf{Distance}\index[MagicItem]{Magic Weapons!Distance}

This special ability can only be added to projectiles. A Ranged projectile has twice the range given by the weapon it fires.

\textbf{Details}: Moderate Divination Aura; Craft Magic Item requirements, clairvoyance; Cost +3000 gp.

\smallskip* \textbf{Destruction}\index[MagicItem]{Magic Weapons!Destruction}

A weapon of Destruction is the bane of all undead. Any undead creature struck in combat must succeed at a DC 14 Will save or be destroyed or take an additional 2d8 points of light damage. A Destruction weapon must be a bludgeoning melee weapon.

\textbf{Details}: Strong Summoning Aura; Craft Greater Magic Item Requirements, Healing; Cost +6000 gp.

\smallskip* \textbf{Duel}\index[MagicItem]{Magic Weapons!Duel}

This ability can only be given to a melee weapon. A dueling weapon (which must be one that can be used with the Weapon Finesse feat) grants the wielder a +1d6 bonus on initiative checks, provided the weapon was drawn and wielded when the duel check is made. Initiative.

\textbf{Details}: Weak Aura Transmutation; Craft Magic Item Requirements, Animals and Plants List; Cost +7000 gp.

\smallskip* \textbf{Bright Energy}\index[MagicItem]{Magic Weapons!Bright Energy}

This object looks like the handle of a longsword, but without the blade. When you grasp its handle, you can use two actions to cause a blade of pure luminescence to form, or cause the blade inserted into the handle to disappear.

As long as the sword exists, this magical longsword has the Versatile property. If you are proficient with short swords or longswords, you are also proficient with the sun blade.

You gain a +2 bonus on attack and damage rolls made with this weapon, which deals Light damage instead of slashing damage. When you strike an undead creature with it, the target takes an additional 1d8 points of Light damage.

The sword's luminous blade emits bright light in a 5m radius and dim light for an additional 5 meters. The light is sunlight. While the blade is active, you can use two actions to expand or shrink the beam of the bright and dim light by 1 meter each, to a maximum of 10 meters or a minimum of 3 meter each.

\textbf{Details}: Strong Transmutation Aura; Craft Wondrous Magic Item Requirements, Everlasting Flame, Sunburst; Cost +45,000 gp.

\smallskip* \textbf{Extinguish Fire}\index[MagicItem]{Magic Weapons!Extinguish Fire}

This special ability can only be added to melee weapons. An extinguish fire weapon is capable of extinguishing Medium or smaller non-magical fire. When used against a creature of Fire, it deals an additional 1d6 points of damage. The wielder of an extinguish fire weapon gains a +2 competence bonus on Saving Throws against fire-based effects, and the weapon itself is immune to fire damage.

\textbf{Details}: Weak Aura Transmutation; Craft Magic Item Requirements, Water List; Cost +3000 gp.

\smallskip* \textbf{Fanaticizer}\index[MagicItem]{Magic Weapons!Fanaticizer}

Cursed weapon. This ability grants a +2 bonus on attacks, however, at the start of battle, causes the wearer to go into a rage that cannot be contained. The character will attack the closest creature, enemy or friend, until none are left alive within 18m.

\smallskip* \textbf{Wounding}\index[MagicItem]{Magic Weapons!Wounding}

This ability can only be added to melee weapons. A wounding weapon deals 1 point of bleed damage when it strikes a creature. Multiple damage from this weapon increases the bleed damage to a maximum of 10.
Bleeding creatures take bleed damage at the start of their round.

Creatures immune to critical rolls are immune to bleed damage dealt by this weapon.

\textbf{Details}: Moderate Necromantic Aura; Craft Greater Magic Item Requirements, Contagion; Cost +6000 gp.

\smallskip* \textbf{Shooting}\index[MagicItem]{Magic Weapons!Shooting}

On command, a shock weapon is engulfed in crackling electricity that deals an additional 1d6 points of electricity damage on each successful hit. This electricity does not harm the wielder. The effect always remains active as long as the weapon is drawn.

\textbf{Details}: Moderate Invocation Aura; Greater Magic Item Crafting Requirements, lightning bolt; Cost +3000 gp.

\smallskip* \textbf{Innate Fury}\index[MagicItem]{Magic Weapons!Innate Fury}

This special ability can only be added to melee weapons. A weapon of fury within draws power from the rage and frustration the wielder feels when fighting enemies who refuse to die. Whenever the wielder deals damage to an opponent with the weapon, her bonus increases by +1 when making attacks against that foe (to a maximum total bonus of +5). This added bonus wears off if the opponent dies, or if the wielder uses the weapon to attack a different creature, misses on the attack roll, or passes 1 hour.

\textbf{Details}: Moderate Enchantment Aura; Greater Magic Item Crafting Requirements, Heroism; Cost +4000 gp.

\smallskip* \textbf{Lucky}\index[MagicItem]{Magic Weapons!Lucky}

While you have the sword on you also receive a +1 bonus on Saving Throws.

- \textit{Luck}. If you are carrying the sword, you can rely on its luck (requiring no action) to re-roll an attack roll, ability check, or Saving Throw whose result you are not satisfied with. You are forced to use the second result of the die. This property cannot be used again until the next dawn.

- \textit{Desire}. While holding it, you can use two actions to expend 1 charge and cast the wish spell from it. This property cannot be used again until the next dawn. The sword has 1d4-1 charges, and loses this property if it runs out of charges.

\textbf{Details}: Very strong Invocation Aura; Mythic Craft Magic Item Requirements, Wish; Cost +30,000 gp.

\smallskip* \textbf{Freezing}\index[MagicItem]{Magic Weapons!Freezing}

On command, a frosty weapon becomes engulfed in a terrible chill that deals 1d6 points of cold damage on successful hits. This cold does not harm the wielder. The effect always remains active as long as the weapon is drawn.

\textbf{Details}: Moderate Invocation Aura; Greater Magic Item Crafting Requirements, Water List; Cost +3000 gp.

\smallskip* \textbf{Gloriosa}\index[MagicItem]{Magic Weapons!Gloriosa}

A glorious weapon glows with a dazzling light equal to that of a daylight spell when drawn. The wielder cannot suppress this light, though it can be temporarily suppressed by any effect that can suppress daylight.

When a glorious weapon scores a critical roll, the target is blinded until the start of the wielder's next round (Fortitude DC 14 negates). Only a melee weapon can have the glorious ability.

\textbf{Details}: Moderate Invocation Aura; Requirements Craft Magic Item, blindness/deafness, daylight; Cost +6000 gp.

\smallskip* \textbf{Guardian}\index[MagicItem]{Magic Weapons!Guardian}

This ability can only be added to melee weapons. A guardian weapon allows the wielder to transfer some or all of the weapon's bonus on her Saving Throws as a bonus that stacks with all others. As an immediate action, the wielder chooses how to distribute the weapon's bonus at the start of her round before using the weapon. The bonus on all Saving Throws lasts until its next round. Only the weapon's own bonus can be sacrificed, no other bonus from other effects can be used.

If a weapon has both defensive and guardian abilities, sacrificing a single point of the bonus improves either Defence or Saving Throws, but not both.

\textbf{Details}: Aura Moderate Abjuration; Craft Greater Magic Item Requirements, Resistance; Cost +3000 gp.

\smallskip* \textbf{Immoral}\index[MagicItem]{Magic Weapons!Immoral}

This ability can only be added to melee weapons. When an Immoral weapon strikes an opponent, it produces a void flash that reverberates between the wielder and its target. The energy deals an additional 2d6 points of damage to the opponent and 1d6 points of damage to the wielder.

\textbf{Details}: Moderate Invocation Aura; Craft Greater Magic Item Requirements, Enervation; Cost +3000 gp.

\smallskip* \textbf{Life Pulse}\index[MagicItem]{Magic Weapons!Life Pulse}

This special ability can only be added to melee weapons. A life-surge weapon augments and sustains the wielder's life energy while in the midst of combat. The wielder gains a bonus on Saving Throws against necromancy effects (including ability damage, ability drain, and maximum hit point reductions from undead powers) equal to the weapon's bonus. In addition, whenever the wielder gains temporary Hit Points from any source, she adds the weapon's bonus to them.

\textbf{Details}: Moderate Summoning Aura; Greater Craft Magic Item requirements, cure serious wounds, greater restoration; Cost +6000 gp.

\smallskip* \textbf{Fiery}\index[MagicItem]{Magic Weapons!Fiery}

On command, a flaming weapon is engulfed in flames, dealing 1d6 points of fire damage on successful hits. This fire does no damage to the wielder. The effect remains active until it is deactivated with another command.

\textbf{Details}: Moderate Invocation Aura; Craft Requirements Greater Magic Items, fireball; Cost +3000 gp.

\index[MagicItem]{Magic Weapons!Thief of Nine Lives}\smallskip* \textbf{Thief of Nine Lives}

You gain a +2 bonus on attack and damage rolls made with this magical weapon. If you score a critical roll against a creature that has fewer than 100 Hit Points, it must succeed on a DC 17 Fortitude save or be immediately slain, as the sword drains its life force from its body (constructs and undead are immune to this property).

The sword has 1d8 + 1 charges, and loses 1 charge when a creature is slain. When the sword has no more charges, it loses this property.

\textbf{Details}: Strong Necromantic Aura; Craft Requirements Greater Magic Items, fireball; Cost +25,000 gp.

\smallskip* \textbf{Lethal}\index[MagicItem]{Magic Weapons!Lethal}

This special ability can only be added to melee weapons that normally deal non-lethal, stun damage. All damage from a lethal weapon is normal (lethal). On command, immediate action, the weapon suppresses this ability until the wielder commands him to reactivate it.

\textbf{Details}: Weak Necromancy Aura; Craft Greater Magic Item Requirements, cure light wounds (reverse); Cost +3000 gp.



\smallskip* \textbf{Marine}\index[MagicItem]{Magic Weapons!Marine}

This special ability can only be added to melee weapons. A Marine weapon works well in aquatic environments. With the weapon in hand, the wielder gains a bonus on Swim checks equal to double the weapon's bonus.

In addition, the wielder does not take the normal penalties on attack and damage rolls for being underwater, as if subject to a freedom of movement spell.

\textbf{Details}: Moderate Necromancy Aura; Craft Greater Magic Item Requirements, Freedom of Movement, Cost +3000 gp.

\smallskip* \textbf{Masquerade}\index[MagicItem]{Magic Weapons!Masquerade}

A weapon of the phantom can be commanded to change its shape and appear as another object of similar size. The weapon retains all of its properties (including weight) even when disguised, but radiates no magic. Only true seeing or other similar magic reveals the true nature of the transformed weapon. After a weapon of the phantom is used to attack, this special ability is suppressed for 1 minute.

\textbf{Details}: Moderate Illusion Aura; Requirements Craft Greater Magic Item, Magic Weapon, Disguise Self; Cost +2000 gp.

\smallskip* \textbf{Ghost Ammo}\index[MagicItem]{Magic Weapons!Ghost Ammo}

This ability can only be given to ammunition. Ammo with this weapon special ability dissolves 1 round after it is fired. In addition, if the Bullet hits a target, the wound caused closes as soon as the ammunition disintegrates. The Bullet deals damage normally, but leaves no visible trace of violence.

The price refers to 50 Ghost Munitions.

\textbf{Details}: Moderate Transmutation Aura; Requirements Craft Greater Magic Items, disintegrate, mend; Cost +1000.

\smallskip* \textbf{Infinite Ammo}\index[MagicItem]{Magic Weapons!Infinite Ammo}

Only bows and crossbows can be weaponized by Infinite Ammo. Whenever an infinite ammo weapon is nocked, a single nonmagical arrow or bolt is spontaneously created by its magic, so the wielder never needs to load the weapon with ammunition.

If the wielder attempts to load the weapon with more ammunition, the arrow or bolt created immediately vanishes and you can load the weapon as normal. This ability does not reduce the amount of time required to load or fire the weapon. The created arrow or bolt vanishes if removed from the weapon; persists only if thrown. Unlike normal bow or crossbow ammunition, these arrows and bolts are always destroyed when fired.

\textbf{Details}: Moderate Summoning Aura; Craft Greater Magic Item Requirements, Crafting; Cost +6000 gp.

\index[MagicItem]{Magic Weapons!Wicked}\smallskip* \textbf{Wicked}

When you roll a 17 or 18 on an attack roll with this magical weapon, the target takes an additional 7 points of damage of the weapon's type.

\textbf{Details}: Weak Summoning Aura; Craft Magic Item Requirements, Cause Light Wounds; Cost +3000 gp.

\smallskip* \textbf{Merciful}\index[MagicItem]{Magic Weapons!Merciful}

All damage dealt by the weapon is temporary.

On command, the weapon suppresses this ability until commanded to reactivate it (allowing it to deal lethal damage).

\textbf{Details}: Weak Summoning Aura; Requirements Craft Magic Item, cure light wounds; Cost +3000 gp.

\smallskip* \textbf{Planar}\index[MagicItem]{Magic Weapons!Planar}

A planar weapon is effective against all types of extradimensional beings, being able to overcome their resistance to physical damage. When used to attack outsiders, a planar weapon ignores 5 points of their damage reduction or resistances.

\textbf{Details}: Moderate Summoning Aura; Craft Requirements Greater Magic Items, plane shift; Cost +3000 gp.

\smallskip* \textbf{Prehensile}\index[MagicItem]{Magic Weapons!Prehensile}

This ability can only be granted to whips. A grapple whip can, as a move action, grab onto an object as if it were a grappling hook. The whip can then be used to climb surfaces or swing across a room or any outdoor area.

\textbf{Details}: Moderate Enchantment Aura; Craft Greater Magic Item requirements, Rope Trick; Cost +2,500.

\index[MagicItem]{Magic Weapons!Mace of Punishment}\smallskip* \textbf{Mace of Punishment}

You gain an additional +3 to hit and damage when using this weapon to attack a construct.

When you roll a critical attack roll with this weapon, the target takes an additional 7 bludgeoning damage, or an additional 14 bludgeoning damage if it is a construct. If a construct has 25 or fewer Hit Points left after taking this damage, it is destroyed.

\textbf{Details}: Strong Invocation Aura; Craft Greater Magic Item requirements; Cost +7000 gp.

\smallskip* \textbf{Researcher}\index[MagicItem]{Magic Weapons!Researcher}

This ability can only be added to ranged weapons. A seeker weapon veers toward its target, negating any miss chances that might apply, such as those due to concealment. The wielder must still aim the weapon at the correct square. Arrows fired by mistake into an empty space, for example, don't turn to hit Invisible opponents, if there are any nearby.

\textbf{Details}: Strong Divination Aura; Craft Greater Magic Item Requirements, true seeing; Cost +3000 gp.

\smallskip* \textbf{Returner}\index[MagicItem]{Magic Weapons!Returner}

A returning weapon can teleport to its wielder's hands as an immediate action, even if it is in the possession of another creature. This ability has a maximum range of 30 meters, and effects that block teleportation prevent a returning weapon from returning. A returning weapon must be in a creature's possession for at least 24 hours for this ability to work.

\textbf{Details}: Moderate Summoning Aura; Craft Greater Magic Item Requirements, Teleport; Cost +3000 gp.

\smallskip* \textbf{Sacred}\index[MagicItem]{Magic Weapons!Sacred}

You gain a +3 bonus on attack and damage rolls made with this magical weapon. When you strike a fiend or undead with it, that creature takes an additional 2d10 points of Light damage.

While holding the drawn sword, it creates a 3m-radius aura around you. You and all friendly creatures within the aura gain +1d6 on Saving Throws against spells and other magical effects generated by followers or devotees of other patrons. If you have Traits in common with the Patron 13 or more, the aura's radius increases to 10 meters.

\textbf{Details}: Moderate Invocation Aura; Common traits 12; Craft Greater Magic Item requirements; Cost +6000 gp.



\smallskip* \textbf{Contemptuous}\index[MagicItem]{Magic Weapons!Contempt}

This special ability can only be added to melee weapons. A Reckless weapon helps its wielder survive in desperate conditions. It remains in the wielder's hands even if the latter is frightened, dazed, or unconscious. The wielder adds her bonus as a bonus on First Aid checks when unconscious or dying, and also adds the same to saves against spells that cause instant death.

\textbf{Details}: Strong Abjuration Aura; Requirements to Craft Greater Magic Items, Stabilize; Cost +6000 gp.

\index[MagicItem]{Magic Weapons!Terror}\smallskip* \textbf{Terror}

While wielding it, you can use two actions and expend 1 charge to unleash a wave of terror.
Each creature of your choice within a 10m radius of you must succeed at a DC 17 Will save or be frightened of you for 1 minute. While so frightened, a creature must spend its rounds trying to move as far away from you as possible, and it can't knowingly move into a space that is within 10 meters of you. It also cannot take reactions. As its action, it can only use the Move action to Disengage. If it can't move anywhere, the creature can use the Total Defence action.

At the end of each of its rounds, the creature can repeat the Saving Throw, ending the effect for itself on a successful one. This magical weapon has 3 charges, and regains 1d3 charges each day at dawn.

\textbf{Details}: Moderate Enchantment Aura; Greater Magic Item Crafting Requirements, Fear; Cost +8000 gp.

\smallskip* \textbf{Titanic}\index[MagicItem]{Magic Weapons!Titanic}

This weapon is 3 meter long and weighs nearly 50kg (8 Bulk), can only be used by a giant (or enlarged character). If used as a weapon, it has a +2 bonus to hit and inflicts 1d4x10 wounds. It can also be used to quickly plant piles the size of tree trunks and to tear down doors and gates with just a few blows.

\textbf{Details}: Moderate Transmutation Aura; Greater Magic Item Crafting Requirements, Enlarge/Reduce; Cost +3000 gp.

\smallskip* \textbf{Ghost Touch}\index[MagicItem]{Magic Weapons!Phantom Touch}

A ghost touch weapon deals critical damage when striking creatures with incorporeal movement and ignores resistances to magical weapons. As long as you wield a Ghost Touch weapon you can see ethereal creatures.

\textbf{Details}: Moderate Summoning Aura; Craft Requirements Greater Magic Items, plane shift; Cost +3000 gp.

\smallskip* \textbf{Thundering}\index[MagicItem]{Magic Weapons!Thundering}

A thundering weapon creates a tremendous din similar to that of thunder when it scores a critical roll. The sonic energy does no harm to the wielder and deals an additional 1d8 points of sonic damage for each successful critical roll. Those subjected to a critical roll from a thundering weapon must make a DC 14 or Fortitude save. remains permanently Deaf.

\textbf{Details}: Weak Necromancy Aura; Craft Magic Item requirements, blindness/deafness; Cost +3000 gp.

\smallskip* \textbf{Transformer}\index[MagicItem]{Magic Weapons!Transformer}

This ability can only be added to melee weapons. A transforming weapon alters its form at the wielder's command, becoming any other melee weapon of similar size. For example, a transforming longsword can assume the form of any other Medium one-handed melee weapon, such as a scimitar, a light flail, or a trident, but not a light or Medium two-handed melee weapon (such as a Medium short sword or two-handed greatsword).

The weapon retains all of its abilities, including bonuses and weapon special abilities, except those prohibited by its current new form. If left unattended, the weapon reverts to its original form.

\textbf{Details}: Moderate Transmutation Aura; Greater Craft Magic Item Requirements, Major Crafting; Cost +5000 gp.

\smallskip* \textbf{Thing Finder}\index[MagicItem]{Magic Weapons!Thing Finder}
This ability allows the wielder of this weapon to cast the locate object spell once per day

\textbf{Details}: Light Divination Aura; Craft Magic Item requirements, Locate item; Cost +1000 gp.

\index[MagicItem]{Magic Weapons!Vampire}\smallskip* \textbf{Vampire}

When you attack a creature with this magical weapon and roll a critical attack roll, the target, other than constructs and undead, takes an additional 10 void damage and you gain 10 temporary Hit Points.

\textbf{Details}: Moderate Necromantic Aura; Craft Greater Magic Item Requirements, Vampire Touch; Cost +8000 gp.

\smallskip* \textbf{Speed}\index[MagicItem]{Magic Weapons!Speed}

When making multiple attacks (2 Actions), the wielder of a haste weapon can make one additional attack with the weapon. The additional attack does not have the penalties of multiple attacks. This ability cannot be combined with similar spells or effects.

\textbf{Details}: Moderate Transmutation Aura; Greater Magic Item Crafting Requirements, Haste; Cost +15,000 gp.

\index[MagicItem]{Magic Weapons!Vorpal}\smallskip* \textbf{Vorpal}

Although it is a +1 magic weapon, it is considered a +5 magic weapon to evaluate immunity and bonus to attack and damage. Additionally, the weapon ignores resistance to slashing damage. When you attack a creature that has at least one head with this weapon and you roll 3 time 6 on attack roll, you sever one of the creature's heads. The creature dies if it cannot survive without losing its head.

A creature is immune to this effect if it is immune to slashing damage, doesn't have or need a head, or the Arbiter decides the creature is too large for its head to be severed by this weapon.

Such a creature takes an additional 6d8 slashing damage from the hit instead.

\textbf{Details}: Very strong Invocation Aura; Mythic Craft Magic Item Requirements; Cost +150,000 gp, legendary.

\subsection{Special Abilities of Magic Armour and Shields}

Most magical Armour and shields have only bonuses, but some have some of the special abilities described below. Armour or shield with special abilities must have at least a +1 bonus.

\index{Armour / Magic Shield}\smallskip* \textbf{Armour / Magic Shield}

\textit{Armour (any)} +1 2500 gp, +2 10000 gp, +3 18000 gp, +4 35000 gp, +5 80000 gp

\textit{Shields (small, medium, heavy)}: +1 1500 gp, +2 4000 gp, +3 9000 gp, +4 20000 gp, +5 35000 gp

While wielding this shield/Armour, you gain a Defence bonus determined by the shield/Armour's magical bonus. This bonus is in addition to the normal Defence bonus provided by the shield/Armour.

\smallskip* \textbf{Blinding}\index[MagicItem]{Armour and Shields!Blinding}

A shield endowed with this enchantment sheds a blinding light up to twice per day at the wielder's command. All but the wielder within 6 meters of the shield must succeed at a DC 14 Reflex save or be blinded for 1d4 rounds.

\textbf{Details}: Moderate Invocation Aura; Building Requirements Craft Greater Magic Item, daylight; Cost +3000 gp.

\index[MagicItem]{Armour and Shields!Adamantium}\smallskip* \textbf{Adamantium}

Armour (medium or heavy, but not leather), uncommon +700 gp above base Armour price. While wearing it, any critical roll you take become normal hits (but do not protect against burst damage).


\smallskip* \textbf{Amorpha}\index[MagicItem]{Armour and Shields!Amorpha}

Once per day on command, the wearer of the Armour (along with any equipment he wears) can assume the form of a viscous liquid that is able to pass through any gap through which thick mud could reasonably flow. While using this ability, your speed is reduced to 3 meter and you can only take move actions. You can assume this form for 1 minute or until you take a move action to revert to your natural form. Amorphous Armour must be made primarily of leather, cloth, or other flexible, organic material.

\textbf{Details}: Moderate Transmutation Aura; Building Requirements Craft Greater Magic Item, Polymorph, Cost +2,250 gp.


\smallskip* \textbf{Antihaemorrhagic}\index[MagicItem]{Armour and Shields!Antihaemorrhagic}

A haemorrhagic Armour helps stop blood loss from the wearer's wounds, automatically tightening like a tourniquet at the appropriate points while also magically reducing the extent of the injury.

Anti-hemorrhagic Armour reduces your hit point damage by 1 per hit taken, and you cannot take bleed damage.

\textbf{Details}: Moderate Transmutation Aura; Building Requirements Craft Greater Magic Item, Cure Critical Wounds, Lesser Restoration, or Stabilize; Cost +3000 gp.

\smallskip* \textbf{Ram}\index[MagicItem]{Armour and Shields!Ram}

These shields are very solid and often bear the emblem of a ram or bull. When the wearer of a ram shield makes a shield attack as part of a charge, the shield's Defence bonus applies to attack and damage rolls. This does not stack with any other buffs that have the shield. This ability doesn't apply to light-type shields.

\textbf{Details}: Weak Invocation Aura; Building Requirements Craft Greater Magic Items, Cost +3000 gp.


\smallskip* \textbf{Brawler}\index[MagicItem]{Armour and Shields!Brawler}

The wearer of brawler Armour gains a +2 bonus on attack and damage rolls on unarmed attacks. His unarmed strikes count as magic weapons for the purpose of overcoming damage reduction. The brawler ability can only be applied to light Armour.

\textbf{Details}: Weak Aura Transmutation; Building Requirements Craft Magic Item, Bull's Strength; Cost +15,000 gp

\smallskip* \textbf{Balanced}\index[MagicItem]{Armour and Shields!Balanced}

This Armour repels anything that threatens to knock the wearer down. The wearer gets a +1d6 bonus against anyone who tries to push or knock him down.

Knocking down while wearing balanced Armour is a move action instead of an immediate action. The balanced ability can be applied to light or medium Armour, but not to heavy Armour or shields.

\textbf{Details}: Weak Aura Transmutation; Building Requirements Craft Magic Item, Cost +3000 gp.

\index[MagicItem]{Armours and Shields!Bracelets of the Archer}\smallskip* \textbf{Bracelets of the Archer}

While wearing these bracers, you are proficient with the longbow and shortbow, and gain a +2 bonus on damage rolls on ranged attacks made with these weapons.

\textbf{Details}: Weak Aura Transmutation; Building Requirements Craft Magic Item, Cost +3000 gp.

\index[MagicItem]{Armour and Shields!Bracelets of Defence}\smallskip* \textbf{Bracelets of Defence}
\textit{Wonderful item, rare}

While wearing these bracers, you have a +1, +2, +3, +4+, +5 bonus to your Defence if you are wearing no Armour and using no shield.

\textbf{Details}: Abjuration Aura; Building Requirements Craft Greater Magic Items, Cost +6,000 gp, 15,000 gp, 30,000 gp, 45,000 gp, 60,000 gp.

\index[MagicItem]{Armours and Shields!Bracelets of Greater Defence}\smallskip* \textbf{Bracelets of Greater Defence}
\textit{Wonderful item, legendary}

These bracers function like Armour but are not Armour. You are wrapped in an invisible magic shield that grants you Defence 15, 17, 19, 21, 23. Defence can be increased with magic items that improve Defence, except Armour and shields.

\textbf{Details}: Abjuration Aura; Building Requirements Craft Greater Magic Item, Cost +12,000 gp, 24,000 gp, 36,000 gp, 50,000 gp, 75,000 gp


\smallskip* \textbf{Brilliant}\index[MagicItem]{Armour and Shields!Brilliant}

Armour and shields with the brilliant special ability radiate light like a torch when worn, which can be suppressed or reactivated on command. The object's appearance is usually characterized by bright colors and a brilliant sheen even when unlit. Once per day, the wearer can command the Armour or shield to glow with the intensity of a daylight spell for 10 minutes or until commanded to dim it.

This Armour needs to be cleaned at least once a week or loses its powers for a week.

\textbf{Details}: Moderate Invocation Aura; Building Requirements Craft Magic Item, Daylight; Cost +3,750 gp.

\smallskip* \textbf{Load}\index[MagicItem]{Armour and Shields!Load}

Cargo Armour distributes the weight carried by the wearer more effectively, allowing him to carry more without suffering the effects of encumbrance. The wearer's encumbrance capacity is increased by 50\%.

\textbf{Details}: Weak Aura Transmutation; Building Requirements Craft Magic Item, Passive Armour; Cost +2000 gp.

\index[MagicItem]{Armour and Shields!Demon Armour}\smallskip* \textbf{Demon Armour}

While wearing the Armour, you can understand and speak Abyssal. In addition, the clawed knobs of the Armour transform unarmed blows made with your hands into magical weapons that deal slashing damage, with a +1 bonus on attack rolls and damage rolls, and a d8 damage die.

\textbf{Details}: Strong Summoning Aura; Building Requirements Craft Greater Magic Items; Cost +5000 gp.

\smallskip* \textbf{Deneger}\index[MagicItem]{Armour and Shields!Deneger}

When the wearer of the Armour is the target of a critical roll or burst of damage done with a melee weapon, he can automatically negate this critical and make it a normal attack. This ability can only be applied to heavy Armour. The ability is usable a number of times per day equal to the weapon's magical bonus.

\textbf{Details}: Strong Abjuration Aura; Building Requirements Craft Greater Magic Items; Cost +25,000 gp.

\smallskip* \textbf{Resolve}\index[MagicItem]{Armour and Shields!Resolve}

A shield or suit of Armour grants the ability to fight in seemingly impossible circumstances. Once per day, when the wielder reaches 0 or fewer Hit Points, the item automatically activates the spell cure serious wounds.

\textbf{Details}: Moderate Summoning Aura; Building Requirements Craft Greater Magic Item, cure serious wounds; Cost +15,000 gp.

\index[MagicItem]{Armour and Shields!Defence from Spells}\smallskip* \textbf{Defence from Spells}

You have +1d6 on Saving Throws against spells and other magical effects.

\textbf{Details}: Strong Abjuration Aura; Building Requirements Craft Greater Magic Items; Cost +5000 gp.

\index[MagicItem]{Armour and Shields!Elegant}\smallskip* \textbf{Elegant}

You can use two actions to speak the command word to cause the Armour to take the form of ordinary clothing or some other type of Armour. You decide the appearance, including color, style and accessories, but the Armour / shield retains its normal Bulk and weight. The illusory aspect lasts until you use this property again or take off your Armour.

\textbf{Details}: Moderate Illusion Aura; Building Requirements Craft Greater Magic Items; Cost +3000 gp.


\smallskip* \textbf{Sweatshirt}\index[MagicItem]{Armour and Shields!Sweatshirt}

Armour with the Fleece ability counts towards the wearing penalties of light Armour. The character can move almost without difficulty with this Armour.

\textbf{Details}: Strong Transmutation Aura; Building Requirements Craft Greater Magic Items; Cost +6000 gp.

\smallskip* \textbf{Ethereal Form}\index[MagicItem]{Armour and Shields!Ethereal Form}

On command, this property allows the wearer to become ethereal (as the spell ethereal form) once per day. The character can remain Ethereal for as long as he wishes, but once he returns to normal, he can no longer become Ethereal for that day.

\textbf{Details}: Strong Transmutation Aura; Building Requirements Craft Wondrous Magic Item, Ethereal Form; Cost +24,500 gp.

\smallskip* \textbf{Invulnerability}\index[MagicItem]{Armour and Shields!Invulnerability}

This Armour grants the wearer damage reduction of 5/magic. Armour with invulnerability emits a strong abjuration aura.

\textbf{Details}: Strong Abjuration Aura; Building Requirements Craft Mythic Magic Item, Wish; Cost +15,000 gp.

\smallskip* \textbf{Untraceable}\index[MagicItem]{Armour and Shields!Untraceable}

Untraceable Armour eases the wearer's steps and disguises their appearance. Survival checks to track the wearer take a –5 penalty, and the wearer of the Armour gains a +5 bonus on Stealth checks. Only leather or hide Armour can be Untraceable.

\textbf{Details}: Weak Aura Transmutation; Building Requirements Craft Magic Item, pass without trace; Cost +3,750 gp.

\smallskip* \textbf{Masquerade}\index[MagicItem]{Armour and Shields!Masquerade}

On command, such Armour changes its shape and appears as a normal set of clothing. The Armour retains all of its properties (including weight) even when masked. Only true seeing or other similar magic reveals the true nature of the transformed Armour.

\textbf{Details}: Moderate Illusion Aura; Building Requirements Craft Greater Magic Item, Disguise Self; Cost +1,350 gp.

\index[MagicItem]{Armour and Shields!Mithral}\smallskip* \textbf{Mithral}

Medium or heavy Armour, but not leather, uncommon +800 gp above base Armour price. Mithral is a light and flexible metal. A mail jacket or mithral breastplate may be worn under normal clothing. Reduces the weight category by 1 to apply penalties to proficiency and Magic tests.

\smallskip* \textbf{Shadow}\index[MagicItem]{Armour and Shields!Shadow}

This Armour makes the wearer blurry whenever he attempts to hide, providing a +5 bonus on his Stealth checks. The Armour check penalty applies normally.

\textbf{Details}: Weak Illusion Aura; Building Requirements Craft Magic Item, Invisibility, Silence; Cost +1,875 gp.

\smallskip* \textbf{Hospitable}\index[MagicItem]{Armour and Shields!Hospitable}

A suit of Armour or shield with this special ability hides live animals within its iconography for safe keeping. The bearer with a command word magically stores an animal to which he is bonded, such as a familiar or a mount. The stored animal appears as a symbol on the Armour or shield, whether it is an imitation of the animal's appearance or a more symbolic, abstract representation.

While stored, the animal sleeps and provides no benefit (such as a familiar's skill bonus) to the wearer. The size of animals that can be stored depends on the type of Armour or shield. Light or medium Armour and light or heavy shields can store an animal up to the wearer's size. Heavy Armour or a tower shield can store an animal up to one size category larger than the wearer. A second command word releases the animal stored in the Armour or hospitable shield. A released animal immediately awakens, appears in a space adjacent to the bearer, and can take actions on the round it appears.

Since the stored animal is sleeping rather than in suspended animation (or even hibernating), it ages and hungers at the normal rate while stored. A hospitable Armour or shield automatically releases a stored animal 24 hours after it is stored within it.

\textbf{Details}: Moderate Summoning Aura; Building Requirements Craft Greater Magic Item, Secret Chest; Cost +3,750 gp.


\smallskip* \textbf{Perceptive}\index[MagicItem]{Armour and Shields!Perceptive}

Perceptive Armour comes to the rescue when the wearer has been blinded, is in total darkness (if the wearer lacks darkvision or the see in the dark ability), or is in magical darkness. When one of these conditions affects the wearer, a perceptive Armour immediately grants him blindsight in a 1m radius. As soon as the wearer can see again, the additional senses cease. The wearer of the Armour cannot gain these abilities by closing his eyes.

\textbf{Details}: Strong Divination Aura; Building Requirements Craft Wondrous Magic Item, true seeing; Cost +15,000 gp.

\smallskip* \textbf{Resistance to Poison}\index[MagicItem]{Armour and Shields!Resistance to Poison}

An Armour or shield with this special ability grants the wearer a +3 bonus on Saving Throws against poison.

\textbf{Details}: Weak Aura Transmutation; Building Requirements Craft Greater Magic Item, remove poison; Cost +1,125 gp.

\smallskip* \textbf{Energy Resistance}\index[MagicItem]{Armour and Shields!Energy Resistance}

This type of Armour or shield protects against one type of energy (Fire, Light, Sound, Electricity, Positive Energy, Negative Energy, Cold, Void) and is decorated with designs depicting the element from which it protects. The Armour or shield absorbs the first 10 points of energy damage per attack that would normally be taken by the wearer.

\textbf{Details}: Weak Abjuration Aura; Building Requirements Craft Magic Item, protection from energy; Cost +9000 gp.

\smallskip* \textbf{Superior Energy Resistance}\index[MagicItem]{Armour and Shields!Superior Energy Resistance}

This type of Armour or shield protects against one type of energy (Fire, Light, Sound, Electricity, Positive Energy, Negative Energy, Cold, Void) and is decorated with designs depicting the element from which it protects. The Armour or shield grants resistance to the listed energy.

\textbf{Details}: Aura Moderate Abjuration; Building Requirements Craft Greater Magic Item, energy protection; Cost +21,000 gp.

\smallskip* \textbf{Selvatica}\index[MagicItem]{Armour and Shields!Selvatica}

Armour with this special ability generally appears to be made of magically hardened animal hide. Wearers of Armour or shields with this ability retain Defence even while transformed into an animal (neither for Spell or Skill).

Armour and shields with this ability usually bear leaf motifs. While the wearer is in wild shape, the Armour is not visible.

\textbf{Details}: Moderate Transmutation Aura; Building Requirements Craft Greater Magic Item, Polymorph; Cost +15,000 gp.

\index[MagicItem]{Armour and Shields!Dragon Scales}\smallskip* \textbf{Dragon Scales}

This Armour or shield is made from the scales of some kind of dragon.

While wearing it, you have +1d6 on Saving Throws against frightening presence and dragon breath weapons, and you have resistance to a type of damage determined by the species of dragon that provided the scales.

In addition, with two actions you can focus your senses to magically determine the distance and direction of the closest dragon within 28 miles that is of the same species as the Armour. This special action cannot be used again until the next dawn.

\textbf{Details}: Aura Moderate Abjuration; Building Requirements Craft Greater Magic Items; Cost +8000 gp.

\smallskip* \textbf{Animated Shield}\index[MagicItem]{Armour and Shields!Animated Shield}

While holding this shield, with two actions you can speak a command word and cause it to animate. The shield will float in the air within your space to protect you as if you were wielding it, leaving your hand free.

The shield remains animated for 1 minute, until you use two actions to end its effect, are incapacitated, or die: at which point the shield will fall to the ground or return to your hand if you have a free one.

\textbf{Details}: Strong Transmutation Aura; Building Requirements Craft Greater Magic Items, Animate Items; Cost +6000 gp.

\index[MagicItem]{Armour and Shields!Shield of Attracting Bullets}\smallskip* \textbf{Shield of Attracting Bullets}

While wielding this shield you apparently have resistance to damage from ranged weapon attacks.

\textit{Cursed version}.

Removing the shield does not end the curse. Whenever a ranged weapon attack is made against a target within 3 meter of you, the curse causes you to become the target of the attack.

\textbf{Details}: Strong Transmutation Aura; Building Requirements Craft Greater Magic Items, Animate Items; Cost +2000 gp.


\smallskip* \textbf{Dragon's Breath}\index[MagicItem]{Armour and Shields!Dragon's Breath}

A shield with this special ability is usually made with a dragon's jaws gaping across the front. A shield with the dragon breath special ability is tied to an energy type (poison, electricity, cold, or fire). The shield regains 1d4 charges each dawn and can hold up to 10.

On command, 2 actions, the wearer can expend 1 to 5 shield charges to cause it to emit a breath weapon in a 5m cone that deals 1d4 points of energy damage per expended charge (Reflex DC 11 half). This damage is of the same energy type tied to the shield. A shield cannot have more than one dragon breath ability.

\textbf{Details}: Weak Invocation Aura; Building Requirements Craft Magic Item, Burning Wave; Cost +2,500 gp.

\smallskip* \textbf{Titanic}\index[MagicItem]{Armour and Shields!Titanic}

Armour with the titanic property is almost comically oversized, even if the effect is only on the exterior and the interior accommodates a creature as normal, requiring no modification. A creature wearing titan Armour is considered one size category larger, including for the purpose of using items and weapons or being affected by size-dependent special attacks, such as swallow whole and trample.

\textbf{Details}: Moderate Transmutation Aura; Building Requirements Craft Greater Magic Item, Enlarge; Cost +15,000 gp.

\smallskip* \textbf{Phantom Touch}\index[MagicItem]{Armour and Shields!Phantom Touch}

This Armour or shield looks almost transparent. The Defence value given by the Armour is counted against the attacks of corporeal and incorporeal creatures. The Armour or shield can be picked up, moved, and put on at any time by both corporeal and incorporeal creatures. Incorporeal creatures gain the item's bonus against corporal and incorporeal attacks, and still retain the ability to pass through solid objects.

\textbf{Details}: Strong Transmutation Aura; Building Requirements Craft Wondrous Magic Item, Ethereal Form; Cost +15,000 gp.

\index[MagicItem]{Armour and Shields!Vulnerabilities}\smallskip* \textbf{Vulnerabilities}

While wearing it, you have resistance to one of the following damage types: bludgeoning, piercing, or slashing. The Arbiter chooses the type. The Armour is cursed, while you are cursed, you have vulnerability to two of the three damage types associated with the Armour (other than what you have resistance to).

\textbf{Details}: Moderate Necromantic Aura; Building Requirements Craft Magic Item, Bestow Curse; Cost +3000 gp.


\subsection{Amulets, Necklaces and Jewels}

\index[MagicItem]{Magic Items!Anti-Poison Amulet}\smallskip* \textbf{Anti-Poison Amulet}
3,000 gp, uncommon, this gem hanging on a silver chain is black and shiny. The wearer has a +1d6 Saving Throw against poison.

\index[MagicItem]{Magical Items!Amulet of Gangrene}\smallskip* \textbf{Amulet of Gangrene}
this engraved gem hanging on a chain appears to be of little value. If a character keeps it with him for more than 1 day, he is struck by a terrible gangrene that causes him to permanently lose 1 point of Dexterity, Constitution and Charisma per week. The gem (and gangrene) can only be countered by remove curse and cure disease, followed by healing or wish. Gangrene can also be defeated by grinding an amulet of health and sprinkling its dust on the afflicted character

\index[MagicItem]{Healing Charm}\smallskip* \textbf{Healing Charm}
25,000 gp, very rare, this gem hanging from a gold chain is red and brilliant. The wearer regains Hit Points twice as fast as normal (even maximum Hit Points). The amulet prevents you from taking damage from Bleeding.

\index[MagicItem]{Amulet Against Possession}\smallskip* \textbf{Amulet Against Possession}
32,000 gp, very rare, the bearer of this copper amulet becomes immune to possession and domination spells.

\index[MagicItem]{Amulet of Unavoidable Tracking}\smallskip* \textbf{Amulet of Unavoidable Tracking}
this cursed amulet has the appearance of an untraceable amulet. On the contrary, it makes the wielder vulnerable to this type of magic. The chance of observing the wielder and the duration of spells used for this purpose are doubled.

\index[MagicItem]{Amulet of the Planes}\smallskip* \textbf{Amulet of the Planes}
160,000 gp, legendary, While wearing this amulet, you can use two actions to name a place you are familiar with that is on another plane of existence. Make a DC 18 Intelligence check. If the check succeeds, you cast the plane shift spell via the amulet. On a failed check, you and each creature and object within 5 meters of you are transported to a random destination. Roll a 1d8. From 1 to 4, you reach a random destination on the floor you named. On a 5 to 8, you reach a randomly determined plane of existence.

\index[MagicItem]{Amulet of Protection from Detection and Tracking}\smallskip* \textbf{Amulet of Protection from Detection and Tracking}
20,000 gp, rare, while wearing this amulet you are concealed from divination magic. You cannot be targeted by these spells or sensed by magical scrying sensors.

\index[MagicItem]{Amulet of Physical Resistance}\smallskip* \textbf{Amulet of Physical Resistance}
8,000 gp, rare, not while wearing this amulet you have a +2 on Fortitude saves.

\index[MagicItem]{Circle of Burst}\smallskip* \textbf{Circle of Burst}
1500 gp, uncommon, While wearing this circlet, you can use two actions to cast the searing ray spell from it. The circlet cannot be used in this way again until the next dawn.


\index[MagicItem]{Necklace of Adjustment}\smallskip* \textbf{Necklace of Adjustment}
1500 gp, uncommon, While wearing this necklace, you can breathe normally in any environment that has air, and you have +1d6 on Saving Throws made against noxious gases and vapors.

\index[MagicItem]{Necklace of Strangulation}\smallskip* \textbf{Necklace of Strangulation}
this necklace looks like a very valuable piece of jewelry. Once worn, it tightens quickly around the neck, inflicting 6 damage per round. It cannot be removed in any way except with a wish or remove curse, remaining close to its victim's neck even after death. The necklace will only come loose when the victim has become a skeleton, ready for an unsuspecting treasure hunter to pick up.

\index[MagicItem]{Necklace of Fireballs}\smallskip* \textbf{Necklace of Fireballs}
depending on the spheres present: 500 gp, 1000 gp, 1600 gp, 2300 gp, 3100 gp, 4000 gp, 4500 gp, 5000 gp, 5500 gp, 6000 gp, uncommon/rare/very rare: 1d6 + hang from this necklace 3 spheres. You can use two actions to detach a sphere and throw it up to 20 meters away. When it reaches the end of its trajectory, the sphere detonates as a fireball spell (DC 18).

\index[MagicItem]{Rosary Necklace}\smallskip* \textbf{Rosary Necklace}
3000 gp + variable, rare, this necklace has 1d4 + 2 magic orbs made of aquamarine, black pearl, or topaz. He also has several non-magical orbs. If a magical orb were removed from the necklace, that orb would lose its magic.

There are six types of magic balls. The Arbiter decides the type of each sphere that is part of the necklace. A necklace can have more than one sphere of the same type. To use it, you must wear the necklace. Each sphere contains a spell that you can cast with two actions, with the spell's DC equal to 10+2xLevel on a Saving Throw. Once a magic sphere's spell has been cast, you cannot use that sphere again until the next dawn.

\medskip

\begin{tabularx}{0.45\textwidth}{llX}
\textbf{3d6} &\textbf{Sphere of...} &\textbf{Enchantment}\\
\hline
3-5 &Blessing &Blessing\\
6-11& Cure &cure serious wounds or lesser restoration\\
12-14 &Please& restaurant superior\\
15-16& Smite &branding punishment\\
17 &Wind& walking in the wind\\
18 &Summon &planar ally\\
\end{tabularx}


\index[MagicItem]{Elemental Gem}\smallskip* \textbf{Elemental Gem}
1200 gp, uncommon, this gem contains a spark of elemental energy. When you use two actions to break the gem, it summons an elemental as if you had cast a summon elementals spell, and the gem's magic vanishes. The type of gem determines the elemental summoned by the spell.

\medskip

\begin{tabular}{ll}
\textbf{Gem} &\textbf{Summoned Elemental}\\
\hline
Red Corundum & Fire Elemental\\
Yellow Diamond & Earth Elemental\\
Emerald & Water Elemental\\
Blue Sapphire&Elemental of Air\\
\end{tabular}

\medskip

\index[MagicItem]{Gem of Luminosity}\smallskip* \textbf{Gem of Luminosity}
5,000 gp, rare, this prism has 50 charges. While holding it, you can use two actions to speak one of three command words to cause one of the following effects:

\begin{itemize}[leftmargin=*]
\item
The first command word causes the gem to shed bright light in a 10m radius and dim light for an additional 10 meters. The effect does consume 1 charge. Lasts until you use two actions to repeat the command word or until you employ another function of the gem or 6 hours passed.

\item
The second command word expends 1 charge and causes the gem to project a bright beam of light at a visible creature within 20 meters of you. The creature must succeed at a DC 17 Fortitude save or be blinded for 1 minute.

\item
The third command word expends 5 charges and causes the gem to shed blinding light in a 10m cone originating from you. Each creature within the cone must make a Saving Throw as if struck by the beam created by the second command word.

\end{itemize}

\medskip

When all of the gem's charges have been expended, the gem becomes a common jewel worth 50 gp.

\index[MagicItem]{Gem of Sight}\smallskip* \textbf{Gem of Sight}
32,000 gp, very rare, with two actions, you can speak the gem's command word and expend 1 charge. For the next 10 minutes, when you look through the gem, you have True Seeing up to 16 meters away. The gem has 3 charges, and regains 1 expended charge each day at dawn.

\index[MagicItem]{Jewel Attract Monsters}\smallskip* \textbf{Jewel Attract Monsters}
this magical jewel is cursed, the owner attracts roaming monsters twice as likely. Monsters will also chase him twice as likely if he runs away. The jewel cannot be dropped and will immediately reappear on the owner's person whenever he tries to get rid of it. Only Remove Curse will allow the wielder to leave the jewel behind.

\index[MagicItem]{Medallion of Feather Fall}\smallskip* \textbf{Medallion of Feather Fall}
400 gp, uncommon, this medallion automatically triggers the Feather Fall spell when the wearer falls from a height of 2 meters or more.

\index[MagicItem]{Medallion of Thoughts}\smallskip* \textbf{Medallion of Thoughts}
3,000 gp, uncommon, While wearing this medallion, you can use two actions and expend 1 charge to cast the detect thoughts spell (save DC 15) with it. The medallion has 3 charges, and recovers 1 expended charge each day at dawn.

\index[MagicItem]{Pearl of Wisdom}\smallskip* \textbf{Pearl of Wisdom}
20,000 gp, rare, this magical bead gives an extra point of Wisdom that keeps it for 4 weeks. After this time the pearl must always be worn in order not to lose its benefits. There is a 5\% chance that a pearl will be cursed and have the opposite effect. In this case, after 4 weeks, the negative effect is permanent and can only be canceled by desire.

\index[MagicItem]{Death Beetle}\smallskip* \textbf{Death Beetle}
this beetle brooch looks like a simple good luck charm. However, if held for 1 round or carried for 1 turn, it transforms into a hideous carnivorous insect. Equipped with powerful jaws, the ravenous creature pierces through leather and fabric, sinking into the flesh and reaching the heart in 1 round. After killing its victim, the creature reverts to the shape of a brooch. Only the heat that comes from contact with a living being can animate the monstrous insect, so placing the brooch in a box or display case is a sufficient precaution to avoid any danger.

\index[MagicItem]{Protection Beetle}\smallskip* \textbf{Protection Beetle}
36,000 gp, legendary, if you hold this scarab-shaped medallion in your hands for 1 round, an inscription appears on it, revealing its magical nature. While on you, it provides two benefits

- You have +2 on Saving Throws against spells.

- The scarab has 12 charges. If you fail a Saving Throw against a necromancy spell or harmful effect originating from an undead creature, you can use a reaction action to expend 1 charge and make the failed save a success. The scarab crumbles to dust and is destroyed when its last charge is expended.

\index[MagicItem]{Brooch of Defence}\smallskip* \textbf{Brooch of Defence}

7500 gp, uncommon, brooch can absorb 101 points of damage from Force spells, then loses its magical properties.

\index[MagicItem]{Talisman of Pure Good}\smallskip* \textbf{Talisman of Pure Good}
50000 gp, legendary, a Devotee of Gradh or Sumkjr in possession of this item can cause a chasm of flames to appear at the feet of a Devotee of Calicante or Shayalia within 30m. The victim is engulfed in fire and plummets screaming towards the center of the Earth. A talisman of pure good has 6 charges and cannot be recharged. If a Calicante or Shayalia Devotee touches it, it suffers 6d6 wounds. Any other Devotees or Followers are unaffected. The talisman pulses with light within 16 meters of a devotee or follower of Calicante or Shayalia.

\index[MagicItem]{Talisman of Extreme Evil}\smallskip* \textbf{Talisman of Extreme Evil}
50,000 gp, legendary, this talisman functions exactly like the talisman of pure good but with the Patrons reversed.

\index[MagicItem]{Talisman of Protection from Poison}\smallskip* \textbf{Talisman of Protection from Poison}
5,000 gp, rare, poisons have no effect on you while wearing this pendant. You are immune to the poisoned condition and have immunity to poison damage.

\index[MagicItem]{Talisman of Health}\smallskip* \textbf{Talisman of Health}
5,000 gp, rare, while wearing this pendant you are immune to the possibility of contracting any disease. If you are already infected with a disease, its effects are suspended as long as you wear this pendant.

\index[MagicItem]{Sphere Talisman}\smallskip* \textbf{Sphere Talisman}
75,000 gp, legendary, when you make an Arcana check to control an orb of annihilation while wielding this talisman you have a bonus of 5. Also, when you start the round in control of an orb of annihilation, you can use two actions to levitate it 3 meter plus an additional number of meters equal to 3 x your Intelligence value.

\subsection{Belts, Helmets, Boots and Gloves}

\index[MagicItem]{Belt of Giants}\smallskip* \textbf{Belt of Giants}

10000/15000/20000/30000/45000 gp, varies rarity, while wearing this belt, your score equals the score conferred by the belt. If your Strength score is already equal to or higher than the wall's score, the item has no effect on you.

There are four variants of this wall, each corresponding to a species of true giants. The stone giant's belt and the frost giant's belt look different, but have the same effect.

\medskip

\begin{tabular}{lll}
\textbf{Type of Giant}& \textbf{Strength} &\textbf{Rarity}\\
\hline
\textbf{Hill} &5& Rare\\
\textbf{Frost/Stone}& 6 &Very Rare\\
\textbf{Fire} &7& Very Rare\\
\textbf{Clouds} &8& Legendary\\
\textbf{Storms}& 9& Legendary\\
\end{tabular}

\index[MagicItem]{Girdle of the Dwarves}\smallskip* \textbf{Girdle of the Dwarves}
86,000 gp, rare, while wearing this belt, you gain the following benefits:

- Your Constitution score increases by 1, to a maximum of 5.

- you have +2 on Charisma checks made to interact with dwarves.

Also, while wearing the belt you have a 50\% chance each day at dawn to see a thick beard appear, if it can grow, or to see yours even thicker, if you already have one.

If you're not a dwarf, you gain the following additional benefits when wearing this belt:

- you have +2 on Saving Throws against poison and you have resistance to poison damage. You have darkvision with a range of 20 meters. You can speak, read, and write Dwarven.

\index[MagicItem]{Helmet of Understanding Tongues}\smallskip* \textbf{Helmet of Understanding Tongues}
600 gp, common, While wearing this helm, you can use two actions to cast the spell comprehend languages from it at will.

\index[MagicItem]{Helmet of Luster}\smallskip* \textbf{Helmet of Luster}
75,000 gp, legendary, this glowing helm is set with 1d10 diamonds, 2d10 rubies, 3d10 fire opals, and 4d10 opals. Any gems extracted from the helmet are reduced to dust. When all gems are removed or destroyed, the helmet loses its magic. While wearing it you get the following benefits:

\medskip

\begin{itemize}[leftmargin=*]
\item
You can use two actions to cast one of the following spells, using a helmet gem of the specified type as a component: daylight (opal), wall of fire (ruby), fireball (fire opal), or Prismatic Spray (diamond ). When the spell is cast the gem is destroyed and disappears from the helm.

\item
As long as it has at least one diamond, the helm emits light in a 10m radius when at least one undead is within this area. Any undead that starts its round within the area takes 1d6 points of light damage.

\item
As long as the helmet has at least one ruby in it, you have resistance to fire damage.
\end{itemize}

\medskip

As long as the helm has at least one fire opal, you can use two actions and speak a command word to cause a weapon you're holding to be engulfed in flames. The flames emit light in a 3m radius and dim light for an additional 3 meter. The flames are harmless to you and the weapon. When you hit with an attack made with the flaming weapon, the target takes an additional 1d6 points of fire damage. The flames last until you use two actions to speak the command word again or until you drop or sheath the weapon.

If you're wearing the helmet and take fire damage from a critical save vs. a spell, the helmet emits a beam of light from the remaining gems. Each creature within 20 meters of the helmet, other than you, must succeed on a DC 21 Reflex save or be hit by the beam, taking Light damage equal to the number of gems in the helmet x 5. Then, the gems and the helmet are destroyed.

\index[MagicItem]{Helmet of Underwater Movement}\smallskip* \textbf{Helmet of Underwater Movement}
4,000 gp, rare, this helmet, usually fish skin, grants the ability to breathe underwater, move Swim 20 meters, echolocation 20 meters. The power is usable for 6 hours per day and recharges at dawn.

\index[MagicItem]{Helmet of Telepathy}\smallskip* \textbf{Helmet of Telepathy}
12,000 gp, rare, While wearing this helm, you can use two actions to cast the detect thoughts spell (save DC 13) from it. As long as you maintain your concentration on the spell, you can use two actions to send a telepathic message to the creature you focus on. She can reply (using two actions to do so) as long as you continue to focus on her.

While focusing on a creature with detect thoughts, you can use two actions to cast the suggestion spell (save DC 13) on that creature via the helm. Once used, the suggestion property cannot be used again until the next dawn.

\index[MagicItem]{Teleport Helm}\smallskip* \textbf{Teleport Helm}
64,000 gp, rare, While wearing this helm, you can use two actions and expend 1 charge to cast the teleport spell from it. The helmet has 3 charges, and recovers 1 each morning at dawn.

\index[MagicItem]{Gloves Catch Bullets}\smallskip* \textbf{Gloves Catch Bullets}
3000 gp, uncommon, these quanta almost seem to melt into your skin when you wear them. When a ranged weapon attack hits you while wearing them, you can use a reaction action to reduce the damage by 1d10 + Dexterity, provided you have one hand free. If you reduce the damage to 0 and the projectile is small enough to hold, you can grab it.

\index[MagicItem]{Gauntlets of Orcish Power}\smallskip* \textbf{Gauntlets of Orcish Power}
9000 gp, rare, while wearing these gauntlets your Strength is 4. The gauntlets have no effect if your Strength is already 4 or higher.

\index[MagicItem]{Swimming and Climbing Gloves}\smallskip* \textbf{Swimming and Climbing Gloves}
2000 gp, uncommon. While wearing both of these gauntlets, climbing and swimming cost you no additional movement. In addition, you have a +1d6 bonus on Constitution and Wisdom checks made while climbing or swimming.

\index[MagicItem]{Gloves of Dexterity}\smallskip* \textbf{Gloves of Dexterity}
12,000 gp, rare, these gloves give the wielder a minimum Dexterity of +2 and if he already has a score of +2 this increases by 1 (up to a maximum of +4). In addition, the bearer gains +1d6 in Fairy Hands Proficiency

\index[MagicItem]{Clumsy Gloves}\smallskip* \textbf{Clumsy Gloves}
these gauntlets may be of soft leather or heavy protective material suitable for use with armour. In the first case they appear to be gloves of dexterity. In the latter case they appear to be gauntlets of orc power. At each trial the gloves appear to have the above functions until the wearer is under attack or in a life-and-death situation. At that moment the curse is activated. The character becomes clumsy, with a 50 \% chance each round of dropping an item he is holding. The gloves reduce Dexterity by 2 points. Once the curse is active, the gloves can only be removed with a remove curse spell or a wish.

\index[MagicItem]{Spider's Slippers}\smallskip* \textbf{Spider's Slippers}
5000 gp, uncommon, while wearing these lightweight shoes, you can move up, down, and along vertical surfaces and upside down on ceilings, leaving your hands free. You have a climbing speed equal to your walking speed. However, slippers don't allow you to move in this way on difficult terrain, such as walls covered in ice, oil, rubble...

\index[MagicItem]{Winged Boots}\smallskip* \textbf{Winged Boots}
15,000 gp, rare, While wearing these boots, you have a flying speed equal to your walking speed. You can use these boots to fly for up to 4 hours, all together or split into short flights, each taking a minimum of 1 minute in duration. If the duration ends while you are flying, you descend at a rate of 10 meters per round until you land. Boots regain 2 hours of flight ability each sunrise.

\index[MagicItem]{Boots of Running and Leaping}\smallskip* \textbf{Boots of Running and Leaping}
5,000 gp, uncommon, While wearing these boots, your walking speed becomes 10 meters unless faster, and your speed isn't reduced if you're encumbered or wearing heavy Armour. Additionally, you jump three times the normal distance, up to a maximum of 10 meters.

\index[MagicItem]{Elf Boots}\smallskip* \textbf{Elf Boots}
3000 gp, uncommon, While wearing these boots, your footsteps make no sound, no matter what surface you're crossing. You have +1d6 on Stealth checks.

\index[MagicItem]{Winter Boots}\smallskip* \textbf{Winter Boots}
10,000 gp, rare, while wearing these boots you have resistance to cold damage, ignore hindering terrain produced by snow or ice. You can tolerate temperatures down to -45C without the need for additional protection. If you wear warm clothes, you can tolerate temperatures as low as -75C.

\index[MagicItem]{Boots of Levitation}\smallskip* \textbf{Boots of Levitation}
5,000 gp, rare, While wearing these boots, you can use two actions at will to cast the levitation spell on yourself.

\index[MagicItem]{Boots of Speed}\smallskip* \textbf{Boots of Speed}
5000 gp, rare, while wearing these boots, you can use a bonus action to use only to move. You can end the effect whenever you want. The effect lasts until worn off, for a maximum of 10 minutes per day. The capacity recharges at dawn.

\index[MagicItem]{Dancing Boots}\smallskip* \textbf{Dancing Boots}
these cursed boots function like other magical boots. However, when you enter combat or attempt to flee from potential combat, you are affected by an irresistible dance spell, with no Saving Throw possible. You can remove the dancing boots with a remove curse or wish spell.

\subsection{Wands, Rods and Staves}

\index[MagicItem]{Metal Searching Wand}\smallskip* \textbf{Metal Searching Wand}
500 gp, uncommon, when a charge is expended, the wand points in the direction of any metal mass of at least 100 kg within 6 meters. The wielder of the wand has an intuitive perception of the type of metal identified.

\index[MagicItem]{Wand of Enchanted Bolts}\smallskip* \textbf{Wand of Enchanted Bolts}
8,000 gp, rare, While holding this wand, you can use two actions to expend 1 or more of its charges to fire a Arcane Dart from it, such as the spell of the same name. Each charge generates 1 dart. The wand has 7 charges. The wand regains 1d3+1 expended charges at dawn each day. However, if you expend the wand's last charge, roll 1d6; if you roll a 1, the wand crumbles to dust and is destroyed.

\index[MagicItem]{Wand of Comforts}\smallskip* \textbf{Wand of Comforts}
300 gp, common, The wielder can expend 1 charge to cast the invisible servant or invisible cook or floating disc spells. The wand has 7 charges which are recovered at dawn.

\index[MagicItem]{Wand of Lightning}\smallskip* \textbf{Wand of Lightning}
32,000 gp, rare, While holding this wand, you can use two actions to expend 1 charge to cast the lightning spell with it (save DC 18).
This wand has 7 charges. The wand regains 1d3 + 1 expended charges at dawn each day. However, if you expend the wand's last charge, roll 1d6; if you roll a 1, the wand crumbles to dust and is destroyed.

\index[MagicItem]{Wand of Fire}\smallskip* \textbf{Wand of Fire}
18,000 gp, very rare, a wand of fire produces several spells and consumes 1 charge + level of the manifested spell. The manifestable spells are: Burning Wave, pyroexpert, fireball, wall of fire. As long as the wand is held, each 1 on the dice for fire damage it deals counts as 2. The wand has 7 charges and regains 1 at dawn.

\index[MagicItem]{Ice Wand}\smallskip* \textbf{Ice Wand}
15,000 gp, very rare, a wand of fire produces several spells and consumes 1 charge + level of the manifested spell. The manifestable spells are: ray of frost, sleet storm, ice storm, cone of cold. As long as the wand is held in hand, each 1 on the dice for cold damage it inflicts is treated as 2. The wand has 7 charges and recovers 1 at dawn.

\index[MagicItem]{Wand of Detect Magic}\smallskip* \textbf{Wand of Detect Magic}
1500 gp, uncommon, While holding this wand, as a two action you can expend 1 charge to cast the detect magic spell with it. This wand has 7 charges, and regains 1d3 expended charges each morning at dawn.

\index[MagicItem]{Wand of Enemy Detection}\smallskip* \textbf{Wand of Enemy Detection}
4,000 gp, rare, While holding this wand, you can use two actions and expend 1 charge to speak its command word. For the next minute, you know which way the closest hostile creature within 20 meters of you is facing, but not the distance between you. The wand can sense the presence of hostile creatures that are ethereal, invisible, disguised, or hidden, as well as those in plain sight. The effect ends if you stop holding the wand. This wand has 7 charges. The wand regains 2 expended charges at dawn each day. However, if you expend the wand's last charge, roll 1d6; if you roll a 1, the wand crumbles to dust and is destroyed.

\index[MagicItem]{Wand of Illusions}\smallskip* \textbf{Wand of Illusions}
3,000 gp, rare, wielder of this wand can cast greater image (3), silent image (1), mirror image (2). Each spell costs a number of charges equal to level +1. While concentrating on the effect, the character can only move at half speed. If hit, he must succeed at a Magic Test or the illusion immediately vanishes.

\index[MagicItem]{Wand of Finding Secret Doors}\smallskip* \textbf{Wand of Finding Secret Doors}
300 gp, uncommon, this wand points to the closest secret passageway within 6 meters. The effect consumes one charge of the 7 available, every day at dawn all charges are recovered.

\index[MagicItem]{Wand of Light}\smallskip* \textbf{Wand of Light}
3500 gp, rare, a wand of light manifests several spells and consumes 1 charge + level of the manifested spell. The manifestable spells are: dancing lights, light, everlasting flame, daylight. Finally, by expending 5 charges, the wielder can create a beam of intense sunlight. The intense yellow-gold light has a range of 36 m, and forms a sphere of light with a diameter of 12 m. Anyone in the area of effect is blinded and stunned for 1 round if they fail a DC 17 Fortitude save. The golden orb has a devastating effect on undead, inflicting 6d6 light wounds with no Saving Throw possible. This wand has 7 charges. The wand regains 2 expended charges at dawn each day. However, if you expend the wand's last charge, roll 1d6; if you roll a 1, the wand crumbles to dust and is destroyed.

\index[MagicItem]{Warmage Wand}\smallskip* \textbf{Warmage Wand}
1500/5500/25000 gp, uncommon (+1), rare (+2), or very rare (+3), while holding this wand, you gain a bonus on spell attack rolls determined by the wand's rarity. Also, you ignore light cover when you make a spell attack.

\index[MagicItem]{Wand of Metamorphosis}\smallskip* \textbf{Wand of Metamorphosis}
32,000 gp, very rare, While holding this wand, you can use two actions to expend 1 charge to cast the polymorph spell with it (DC 18, Will save). This wand has 3 charges. The wand regains 1 expended charge at dawn each day. However, if you expend the wand's last charge, roll 1d6; if you roll a 1, the wand crumbles to dust and is destroyed.

\index[MagicItem]{Wand of Wonders}\smallskip* \textbf{Wand of Wonders}
25,000 gp, very rare, While holding this wand, you can expend 1 charge as a two action and choose a target within 16 meters of you. The target can be a creature, an object, or a point in space. The Arbiter decides or randomly determines what will happen when you use the wand. Spells cast with the wand have a DC of 18. If the spell normally has a range expressed in meters, the range becomes 36 meters if it isn't already. If an effect covers an area, you must center the spell on the target and include it. If an effect affects as many subjects as possible, the Arbiter randomly determines who is affected.

This wand has 7 charges. The wand recovers 1 charge each day at dawn. If you expend the wand's last charge, roll 1d6; if you roll a 1, the wand crumbles to dust and is destroyed.

Each time you use your wand of wonders, roll a d100 and consult this table.

\end{multicols}

\vfill

\begin{center}
\includegraphics[width=0.55\linewidth]{immagini/bacchette.png}

\end{center}

\medskip

\begin{tabularx}{0.95\textwidth}{lX}
\textbf{d100}& \textbf{Contents}\\
\hline
01-05 &Cast slow.\\
06-10 &You throw fairy fire.\\
11-15 &You are stunned until the start of your next round, and you believe something amazing has happened.\\
16-20 &Launches gust of wind.\\
21-25 &You cast detect thoughts at your chosen target. If your target isn't a creature, you take 1d6 damage instead.\\
26-30 & Casts foul-smelling cloud.\\
31-33 & Heavy rain falls in a 20m radius centered on the target. The area becomes slightly darkened. The rain keeps falling until the start of your next round.\\
34-36 &An animal appears in the unoccupied space closest to the target. The animal is not under your control and is acting as normal. Roll a d100 to determine what kind of animal appears.01-25, a rhinoceros; 26-50, an elephant; 51-100, a rat.\\
37-46 &Throw Lightning.\\
47-49 &A cloud of 600 huge butterflies fills a 10m radius around the target. The area becomes heavily darkened. Butterflies stay for 10 minutes.\\
50-53 &Enlarge the target as if you had cast the enlarge/reduce spell. If the target can't be affected by the spell, or if it isn't a creature, you become the target.\\
54-58 &You cast darkness.\\
59-62 &Thick grass sprouts in a 20m radius around the target.If there is already grass, it grows tenfold and stays that way for 1 minute.\\
63-65 &An object of the Arbiter's choice disappears on the Ethereal Plane.Item must not be worn or carried, must be within 36 meters of the target, and no larger than 3 meters in each dimension.\\
66-69 &You shrink as if you had cast the enlarge/reduce spell on yourself.\\
70-79 & Fireball throws.\\
80-84 &You cast invisibility on yourself.\\
85-87 &Leaves are growing on the target. If you've chosen a point in space as a target, leaves will sprout on the creature closest to that point. Unless torn off, the leaves will turn brown and fall off after 24 hours.\\
88-90& A stream of 1d4 x 10 gems worth 1 gp each shoots from the wand's tip in a line 10 meters long and 1 meter wide. Each gem deals 1 bludgeoning damage, and their total damage is divided evenly among all creatures in the line.\\
91-95 &A flurry of shimmering, colorful light extends from you in a 10m radius. You and all creatures in the area must succeed at a DC 15 Fortitude save or be blinded for 1 minute. A creature can repeat the Saving Throw at the end of each of its rounds, ending the effect on itself on a successful one.\\
96-97 &Target's skin turns deep blue for 1d10 days. If you chose a point in space, the subject will be the creature closest to that point.\\
98-00 &If the target is a creature, it must make a DC 18 Fortitude save. If the target isn't a creature, the target becomes you, and you make the Saving Throw. On a failed save by 5 or more, the target is petrified. On fewer failed saves, the target is restrained and begins to turn to stone. While restrained in this way, the target must repeat the Saving Throw at the end of each of its rounds, becoming petrified on a failed save or ending the effect on a success. The target remains petrified until freed from stone to flesh or similar magic.\\
\end{tabularx}

\medskip

\begin{multicols}{2}

\index[MagicItem]{Wand of Denial}\smallskip* \textbf{Wand of Denial}
35,000 gp, very rare, this wand negates spells or similar effects produced by magical items. The wielder points the wand at an object within 16 meters, and it emits a light gray beam that strikes the target. The ray automatically negates the manifestation of similar spells or effects of level 3 or lower. Each use of the wand costs 1 charge, and it can only be used once per round. This wand has 3 charges. The wand recovers 1 charge each day at dawn. If you expend the wand's last charge, roll 1d6; if you roll a 1, the wand crumbles to dust and is destroyed.

\index[MagicItem]{Wand of Fireballs}\smallskip* \textbf{Wand of Fireballs}
32,000 gp, rare, While holding this wand, you can use two actions to expend 1 charge to cast the fireball spell (save DC 18) with it. This wand has 7 charges. The wand regains 1 expended charge at dawn each day. However, if you expend the wand's last charge, roll 1d6; if you roll a 1, the wand crumbles to dust and is destroyed.

\index[MagicItem]{Wand of Paralysis}\smallskip* \textbf{Wand of Paralysis}
16,000 gp, rare, While holding this wand, you can use two actions to expend 1 charge to cause a thin beam to shoot from its tip at a creature visible within 20 meters of you. The target must succeed on a DC 17 Fortitude save or be paralyzed for 1 minute. At the end of each round of the target, he can make a DC 15 Fortitude Saving Throw, ending the effect on himself on a successful one. This wand has 7 charges. The wand recovers 1 spent charge at dawn each day. However, if you expend the wand's last charge, roll 1d6; if you roll a 1, the wand crumbles to dust and is destroyed.

\index[MagicItem]{Wand of Fear}\smallskip* \textbf{Wand of Fear}
13,000 gp, rare, this wand has 7 charges for the following properties. The wand recovers 1 spent charge at dawn each day. However, if you expend the wand's last charge, roll a 1. If you roll a 1, the wand crumbles to dust and is destroyed.

\textbf{Command} While holding this wand, you can use two actions to expend 1 charge and command another creature to flee or crawl, as per the command spell (save DC 18)

\textbf{Cone of Fear} While holding this wand, you can use two actions to expend 2 charges, causing the wand's tip to emit light in a 20m cone. Each creature in the cone must succeed at a DC 18 Will save or be frightened of you for 1 minute. While frightened in this way, a creature must spend its rounds trying to move as far away from you as possible, and it can't voluntarily move within 10 meters of you.

It also cannot take reactions. As its action, the creature can only use the dash action or try to free itself from an effect that prevents it from moving. If it can't move anywhere, the creature can use the All Defence action. At the end of each of its rounds, the creature can repeat the Saving Throw, ending the effect on itself on a successful one. This wand has 7 charges. The wand recovers 1 spent charge at dawn each day. However, if you expend the wand's last charge, roll 1d6; if you roll a 1, the wand crumbles to dust and is destroyed.

\index[MagicItem]{Trap-detecting wand}\smallskip* \textbf{Trap-detecting wand}
400 gp, uncommon, this wand targets the nearest trap within 6 meters. The effect consumes a charge. This wand has 7 charges. The wand recovers all expended charges at dawn each day.

\index[MagicItem]{Wand of Secrets}\smallskip* \textbf{Wand of Secrets}
500 gp, uncommon, While holding this wand, you can use two actions to expend 1 charge and detect if a secret door or trap is within 10 meters of you, the wand pulses and points to the one closest to you. The wand has 3 charges. The wand recovers all expended charges at dawn each day.


\index[MagicItem]{Wand of the Web}\smallskip* \textbf{Wand of the Web}
8,000 gp, uncommon, While holding it, you can use two actions to expend 1 charge to cast the web spell from it (save DC 18). This wand has 7 charges. The wand regains 1 expended charge at dawn each day. However, if you expend the wand's last charge, roll 1d6; if you roll a 1, the wand crumbles to dust and is destroyed.

\index[MagicItem]{Wand of Binding}\smallskip* \textbf{Wand of Binding}
10,000 gp, rare, this wand has 7 charges for the following properties. The wand recovers 1 spent charge at dawn each day. However, if you expend the wand's last charge, roll 1d6; if you roll a 1, the wand crumbles to dust and is destroyed. While holding this wand, you can use two actions and expend some of its charges to cast one of the following spells (save DC 21):

\textbf{hold monster} (5 charges) or \textbf{hold people} (2 charges).

\index[MagicItem]{Wand of Assisted Escape}\smallskip* \textbf{Wand of Assisted Escape}
2000 gp, rare, While holding this wand, you can use the reaction action and expend 1 charge to gain +1d6 on Saving Throws you make to avoid being paralyzed or restrained, or you can expend 1 charge to gain +1d6 on any checks made to escape an attempt to grab.

\index[MagicItem]{Archmage's Staff}\smallskip* \textbf{Archmage's Staff}
125,000 gp, legendary, the archmage's staff is a very powerful version of the staff of sorcery. It provides the bearer with various spells. The staff can be used to manifest spells: Magic Lock, detect magic, enlarge/reduce, and light. These abilities do not require the consumption of charges. In addition, the staff has the following abilities that cost 1 charge per use: dispel magic, lightning bolt, invisibility, wall of fire, fireball, Pass door, pyroexpert, web, lock lock, and ice storm. The following powerful abilities cost 2 charges per use: summon elemental, planar shift, telekinesis. The wielder of the staff receives a +2 bonus on Saving Throws against spells. The staff can be recharged, but only by absorbing the magical energies thrown at the wielder, who can absorb them in quantities equal to 1 charge per level of the spell. This operation is the only possible action in a round, and the staff cannot be used for other effects in the same round in which it absorbs energy. Each staff has a maximum number of possible charges, and it will only absorb charges up to its limit without incurring any deleterious effects. The wielder has no way of knowing this limit, or how many charges have been used, unless using some magical method. If the staff absorbs excess energy, it explodes as in the case of an ultimate blow, described below. An archmage staff can be used for an ultimate strike, requiring that it be broken by its wielder. The break must not be accidental and must be declared. All charges stored in the staff are instantly released within a 10m radius. All creatures within 3 meter take wounds equal to 10 times the number of charges in the staff; between 3 m and 6 m the wounds are 6 times the number of charges; and between 6m and 9m the wounds are 4 times the number of charges. A DC 25 Fortitude save reduces the damage in half. The character who breaks the staff has a 50\% chance of going to another plane of existence, or else the explosive release of magical energy destroys him. When all charges have been used up, the staff becomes a +2 staff. If the charges are depleted it cannot be used for a final blow.

\index[MagicItem]{Staff of Wither}\smallskip* \textbf{Staff of Wither}
3,000 gp, rare, staff can be wielded as a magical fighting staff. On a hit, it deals damage like a normal quarterstaff, and you can expend 1 charge to deal an additional 2d10 void damage to the target. In addition, the target must succeed on a DC 18 Fortitude save or be -1d6 for 1 hour on any ability check or save that requires Constitution. This staff has 3 charges and regains 1d3 expended charges at midnight.

\index[MagicItem]{Staff of the Woods}\smallskip* \textbf{Staff of the Woods}
44,000 gp, rare, the staff can be wielded as a magical quarterstaff that grants a +2 bonus on attack and damage rolls made with it. When wielded, you also have a +2 bonus on attack rolls with spells.
This staff has 10 charges for the following properties. Recovers 2 expended charges each day at dawn. If you expend the staff's last charge, roll 1d6; if you roll a 1, the staff turns black, turns to ash, and is destroyed.

- \textit{Enchantments}. You can use two actions to expend 1 or more staff charges to cast one of the following spells with it, using your spell save DC: friendship with animals (1 charge), locate animals and plants (1 charge), wall of thorns (6 charges), speak with animals (3 charges), leathery skin (2 charges), or awaken (5 charges). You can also use two actions to cast the pass without trace without spell with the staff
spend charges.

- \textit{Tree Shape}. You can use two actions to plant one end of the staff in fertile ground and spend 1 charge to transform the staff into a vigorous fruit tree. The tree is 18 meters tall, with a trunk 1 meters in diameter; at the top its branches extend 6 meters. The tree looks like a normal tree but radiates a faint aura of transmutation magic if it is targeted by the detect magic spell. While in contact with the tree and using another action to speak its command word, you return the staff to its normal shape. Any creature on the tree falls when it transforms back into a staff.

\index[MagicItem]{Staff of Charming}\smallskip* \textbf{Staff of Charming}
12,000 gp, rare, While holding this staff, you can use two actions to expend 1 charge to cast charm person, command, or understand languages with it, using your DC of spell saves. The staff can be used as a magical fighting staff.

If you are wielding the staff and fail a Saving Throw against an enchantment spell that targets only you and not an area, you can make the failed save a success. You cannot use this staff property again until dawn the following day.

If you succeed on a Saving Throw against an enchantment spell that targets only you, with or without the staff's intervention, you can use a reaction action to expend 3 charges from the staff and turn the spell against the caster. as if the spell was cast by you.

The staff has 7 charges, and regains 1 expended charge each day at dawn. If you expend the last charge, roll 1d6 and if you roll a 1 the staff becomes a normal fighting staff.

\index[MagicItem]{Staff of Striking}\smallskip* \textbf{Staff of Striking}
25,000 gp, very rare, this staff can be wielded as a magical fighting staff that grants a +3 bonus on attack and damage rolls made with it. When you hit with a melee attack using the staff, you can expend up to 3 of its charges. For each charge expended, the target takes an additional 1d6 force damage. The staff has 10 charges, and regains 2 expended charges each day at dawn. If you expend the last charge, roll 1d6 and if you roll a 1 the staff becomes a normal fighting staff.

\index[MagicItem]{Fire Stick}\smallskip* \textbf{Fire Stick}
16,000 gp, very rare, while holding this staff, you have resistance to fire damage.
Additionally, you can use two actions to expend 1 or more of its charges to cast one of the following spells with it: Burning Wave (1 charge, DC 13), wall of fire (4 charges, DC 19), or fireball (3 charges DC 17).

The staff has 10 charges, and regains 2 expended charges each day at dawn. If you expend the staff's last charge, roll 1d6; if you roll a 1, the staff turns black, turns to ash, and is destroyed.

\index[MagicItem]{Staff of Frost}\smallskip* \textbf{Staff of Frost}
26,000 gp, very rare, while holding this staff, you have resistance to cold damage.
In addition, you can use two actions to expend 1 or more of its charges to cast one of the following spells with it.

- \textit{Spells}: cone of cold (5 charges, DC 21), wall of ice (4 charges, DC 19), cloud of fog (1 charge, DC 13) or ice storm (4 charges, DC 19 ).

The staff has 10 charges, and regains 2 expended charges each day at dawn. If you expend the staff's last charge, roll 1d6; if you roll a 1, the staff turns to water and destroys itself.

\index[MagicItem]{Staff of Healing}\smallskip* \textbf{Staff of Healing}
13,000 gp, rare, While holding it, you can use two actions to expend 1 or more of its charges to cast one of the following spells with it: cure light wounds (1 charge), lesser restoration (2 charges), remove disease (3 charges ). This staff has 10 charges, and regains 1 expended charge each day at dawn. If you expend the staff's last charge, roll 1d6; if you roll a 1, the staff vanishes in a flash of light, lost forever.

\index[MagicItem]{Staff of Swarming Insects}\smallskip* \textbf{Staff of Swarming Insects}
160,000 gp, rare, this staff has 10 charges you can expend to use the properties described below, and regains 1 charge each day at dawn. If you expend the staff's last charge, roll 1d6 if you roll 1 a swarm of insects consumes and destroys the staff, and then scatters.

- \textit{Enchantments}. While holding this staff, you can use two actions to expend its charges and cast one of the following spells: giant insect (4 charges, DC 19) or insect plague (5 charges, DC 21).

- \textit{Cloud of Bugs}. While wielding this staff, you can use two actions and expend 1 charge to cause a swarm of harmless insects to spread in a 10m radius around you. The bugs stay for 10 minutes, making the area heavily blacked out for everyone but you. The swarm moves with you, staying centered on you. A wind of at least 15 kilometers per hour disperses the swarm and ends the effect.

\index[MagicItem]{Staff of the Python}\smallskip* \textbf{Staff of the Python}
2000 gp, uncommon, you can use two actions to speak the staff's command word and hurl it to the ground up to 3 meter away. The staff becomes a giant constrictor snake under your control and acts on its own initiative count. Using two actions to say the command word again, you return the staff to its normal shape in the space previously occupied by the snake.

During your round, you can mentally command the snake as long as it is within 20 meters of you and you are not incapacitated. You decide what actions the snake will take and where it will move during its next round, or you can give it a generic command, such as to attack your enemies or defend a location. If the snake is reduced to 0 Hit Points, it dies and reverts to its staff form. Then, the stick shatters and is destroyed. If the snake reverts to staff form before losing all of its Hit Points, it regains all lost Hit Points.

\index[MagicItem]{Staff of Power}\smallskip* \textbf{Staff of Power}
150,000 gp, legendary, this staff can be wielded as a magical quarterstaff that grants a +2 bonus on attack and damage rolls made with it. While wielding it, you receive a +2 bonus to Defence, Saving Throws, and spell attack rolls. This staff has 20 charges for the following properties. Recovers 1d8 + 1 expended charges each day at dawn. If you expend the staff's last charge, roll 1d6. If you roll a 1 or less, the staff retains its +2 bonus on attack and damage rolls but loses all other properties.

- \textit{Power Stroke}. When you hit with a melee attack using this staff, you can expend 1 charge to deal an additional 1d6 force damage to the target.

- \textit{Enchantments}. While holding this staff, you can use two actions to expend 1 or more of its charges to cast one of the following spells with it: hold monster (5 charges, DC 21), cone of cold (5 charges, DC 21), Orb of Invulnerability ( 6 charges, DC 22), levitation (2 charges DC 15), wall of force (5 charges, DC 21), fireball (3 charges DC 17), Arcane Dart (1 charge), ray of fatigue (1 charge DC 11) or lightning (3 DC 17 charges).

- \textit{Shot of Revenge}. You can use two actions to break the staff on your knee or against a solid surface, performing a revenge strike. The staff is destroyed and releases its remaining magic in a burst that expands to fill a 10m-radius sphere centered on it.

You have a 50\% chance to instantly travel to a random plane of existence, thus avoiding the explosion. If you fail to avoid the effect, you take force damage equal to 16 x the number of charges in the staff. Each other creature in the area must make a DC 27 Reflex save. On a failed save, the creature takes an amount of damage based on the distance from the point of origin of the explosion, as shown on the following table.

On a successful save, the creature takes half as much damage.

\medskip

\begin{tabularx}{0.45\textwidth}{Xl}
\hline
\textbf{Distance from origin} &\textbf{Damage}\\
3 meter or less &8 x staff charges\\
Up to 6 meters & 6 x staff charges\\
Up to 10 meters & 4 x staff charges\\
\end{tabularx}

\medskip

Note: the Staff of the Archimage and of Power are similar, this is because they were prepared by two bitter enemies who wanted to create the most powerful Staff.

\index[MagicItem]{Staff of Thunder and Lightning}\smallskip* \textbf{Staff of Thunder and Lightning}
10,000 gp, very rare, the staff can be wielded as a magical quarterstaff that grants a +2 bonus on attack and damage rolls made with it. It also has the following properties. When one of these properties is used, it cannot be used again until the next dawn.

- \textit{Lightning Bolt}. When you hit with a melee attack using the staff, you can cause the target to take an additional 2d6 lightning damage.

- \textit{Thunder}. When you hit with a melee attack using the staff, you can cause the staff to make a thunderous sound, audible up to 100 meters away. The affected target must succeed at a DC 21 Fortitude save or be stunned until the end of your next round.

- \textit{Lightning Strike}. You can use two actions to cause a bolt of lightning to leap from the tip of the staff in a line 1 meter wide and 16 meters long. Each creature in the line must make a DC 21 Reflex save, taking 9d6 lightning damage on a failed save, or half as much damage on a successful one.

- \textit{Thunderclap}. You can use two actions to cause the staff to produce a deafening roll of thunder, audible up to 200 meters away. Each creature within 20 meters of you (excluding you) must make a DC 21 Fortitude save. On a failed save, the creature takes 2d6 sound damage and is deafened for 1 minute. On a successful save, she takes half damage and is not deafened.

- \textit{Thunder and Lightning}. You can use two Actions to use the Lightning Strike and Thunderclap properties together. Doing so does not consume the daily use of those properties, only the use of this one.

\index[MagicItem]{Staff of Sorcery}\smallskip* \textbf{Staff of Sorcery}
85,000 gp, very rare, in combat, this staff functions as a +1 staff. Can be used to cast summon elemental, invisibility, Pass door, and web. The staff can be used as a wand of paralysis. Each of these powers requires a charge. It is possible to break the stick to produce a "final blow", the effect of which depends on the number of residual charges. The staff explodes in a large ball of flame, striking all creatures within 10 meters (including the staff's owner) and inflicting 8 wounds per charge remaining, Fortitude save DC 27 to halve.

\index[MagicItem]{Rod of Enchantment}\smallskip* \textbf{Rod of Enchantment}
28,000 gp, rare, by expending 1 charge, the wielder can cast dominate beasts, with 2 charges dominate people, and with 3 charges dominate monsters.

\index[MagicItem]{Rod of Absorption}\smallskip* \textbf{Rod of Absorption}
50,000 gp, very rare, While holding this rod, you can use an Action to absorb a spell that targets only you and has no area of effect. The absorbed spell's effect is canceled, and the spell's energy (not the spell itself) is absorbed by the rod. During its existence the rod can absorb and contain up to a sum of 31 Levels of spells. Once the rod has absorbed 8 spells (max level 4), it cannot absorb any more. If you are the target of a spell that the rod cannot contain, the rod has no effect on the spell. When you pick up the rod, you know how many spells the rod has absorbed so far. If you are a spellcaster and hold the rod, you can convert all the energy contained to have 10 more Spell Points.

\index[MagicItem]{Immovable Rod}\smallskip* \textbf{Immovable Rod}
5,000 gp, Uncommon, this flat iron rod has a button on one end. You can use two actions to press the button, which causes the rod to magically stay in place. Until you or another creature uses two actions to press the button again, the rod won't move, even if it defies gravity. The rod can support up to 4000 kilos of weight. More weight causes the rod to deactivate and fall. A creature can use two actions to make a DC 30 Strength check, moving the rod 3 meter on a success.

\index[MagicItem]{Rod of Mighty Stroke}\smallskip* \textbf{Rod of Mighty Stroke}
30,000 gp, very rare, rod of mighty blow deals 1d8+3 wounds, and functions as a +3 magic light mace. When the rod is used against golems, it consumes 1 charge per hit, and inflicts 2d8+6 wounds. Note that when the rod is used as a weapon against a golem, a critical attack roll instantly annihilates it. In addition, this rod deals additional wounds to fiends and undead. When attacking these monsters, a Critical Attack Roll consumes 1 charge, and the rod inflicts triple wounds.

\index[MagicItem]{Rod of Sovereign Strength}\smallskip* \textbf{Rod of Sovereign Strength}
50,000 gp, legendary, this rod has a flanged head, and functions as a magical mace that grants a +3 bonus on attack and damage rolls made with it. The rod has properties associated with the six different buttons that are arranged along the handle. It also has three other properties described below.

\textbf{Six Buttons}. You can press one of the rod's six buttons with two actions. The button's effect lasts until you press a different button or press the same button again, returning the rod to its normal form.

- If you press the \textit{button 1}, the rod becomes a tongue-of-fire weapon, and a flaming blade shoots out from the end opposite the flanged head.

- If you press \textit{button 2}, the flanged head of the rod folds back and two crescent-shaped blades protrude, transforming the rod into a magical battle-axe granting a +3 bonus on attack and damage rolls made with it.

- If you press \textit{button 3}, the flanged head of the rod folds back, and a spearhead shoots out of the end of the rod, while the handle extends up to 1.8 meters, turning the rod into a magical spear that grants a +3 bonus on attack and damage rolls made with it.

- If you press \textit{button 4}, the rod transforms into a climbing pole up to 15 meters long, as requested by you. On hard surfaces like granite, one spike at the bottom and three at the top hold the rod in place. Horizontal bars 10 cm long run along the sides of the rod, 15 centimeters apart, to form a ladder. The rod can support 2000 kilos. More weight or lack of solid anchoring causes the rod to return to its normal shape.

- If you press \textit{button 5}, the rod transforms into a battering ram and grants the user a +10 bonus on Strength checks made to breach doors, barricades or other barriers.

- If you press \textit{button 6}, the rod assumes or remains in its normal form and points to magnetic north (nothing happens if this rod function is used in areas without magnetic north). The rod also gives you a rough knowledge of how deep underground you are and how high you are above sea level.

\textit{Drain Life}. When you hit a creature with a melee attack using the rod, you can force the target to make a DC 21 Fortitude save. On a failed save, the target takes an additional 4d6 void damage and is subtracted from its hit point maximum, and you regain a number of Hit Points equal to half the Void damage inflicted. Once used, this property cannot be used again until dawn the following day.

\textbf{Paralyze}. When you hit a creature with a melee attack using the rod, you can force the target to make a DC 21 Fortitude save. On a failed save, the target is paralyzed for 1 minute. The target can repeat the Saving Throw at the end of each of its rounds, ending the effect on itself on a successful one. Once used, this property cannot be used again until dawn the next day.

\textit{Terrify}. While holding this rod, you can force each creature you see within 10 meters of you to make a DC 21 Will save. On a failed save, the target is frightened of you for 1 minute. The frightened target can repeat the Saving Throw at the end of each of its rounds, ending the effect on itself on a successful one. Once used, this property cannot be used again until dawn the next day.

This rod cannot be recharged. When the charges run out, one remains

\index[MagicItem]{Rod of Readiness}\smallskip* \textbf{Rod of Readiness}
25,000 gp, very rare, this flanged-headed rod has the following properties.

\textit{Readiness}. While holding this rod, you have +2 on Wisdom checks and initiative rolls.

\textit{Enchantments}. While holding this rod, you can use two actions to cast one of the following spells with it: detect good and evil, detect magic, detect poison and disease, or see invisibility.

\textit{Protective Aura}. With two actions, you can drive the pointed end of the rod into the ground. At that point, the rod's head will shed bright light in a 20m radius and dim light for an additional 20 meters. Within this bright light, you and any creatures friendly to you gain a +1 bonus to Defence and Saving Throws, and you can sense the location of any hostile invisible creatures that are also within the bright light. The rod's head stops emitting light and ends after 10 minutes, or when a creature uses two actions to pull the rod out of the ground. This property cannot be used again until dawn the following day.

\index[MagicItem]{Rod of Security}\smallskip* \textbf{Rod of Security}
90,000 gp, very rare, while holding this rod, you can use two actions to activate it. As a result, the rod transports you and up to 199 other visible willing creatures to a paradise located in extraplanar space. You will choose the shape of this paradise. It could be a peaceful garden, a pleasant clearing, a cheerful tavern, a huge palace, a tropical island, or a fantastic fair or anything else you can imagine. Whatever its nature, paradise contains enough food and drink to feed its visitors. Anything that can be interacted with in extraplanar space can only exist within it.

For each hour spent in this paradise, a visitor regains Hit Points as if she had had a night's rest. Furthermore, as long as the creatures remain in the paradise they do not age, although time passes normally. Visitors can stay in the paradise for up to 200 days divided by the number of creatures present (round down).

When the time runs out or you use two actions to end it, all visitors reappear in the location they occupied when you activated the rod, or in the closest unoccupied space to that. The rod cannot be used again until ten days have passed.

\index[MagicItem]{Rod of Sovereignty}\smallskip* \textbf{Rod of Sovereignty}
16,000 gp, rare, you can use two actions and present the rod and command obedience from each visible creature within 16 meters of you of your choice. Each target must succeed at a DC 17 Will save or be charmed by you for 8 hours. While charmed in this way, the creature considers you a trusted leader. If it is dealt damage by you or your companions, or ordered to do something contrary to its nature, the target will cease to be charmed in this manner. The rod cannot be used again until the next dawn.

\index[MagicItem]{Twisted Rod}\smallskip* \textbf{Twisted Rod}
5,000 gp, rare, this rod is a magical weapon that terminates in three leather tentacles. While holding the rod, you can use two actions to direct each tentacle to attack a visible creature within 5 meters of you. Each tentacle makes a melee attack roll with a +9 bonus. On a hit, the tentacle deals 1d6 bludgeoning damage. If you hit a target with all three tentacles, it must make a DC 15 Fortitude save. On a failed save, the creature's speed is halved, it has -1d6 on Reflex saves, and for 1 minute it can't use his reactions. Also, during each of his rounds, he can take two actions or two actions but not both. The target can repeat the Saving Throw at the end of each of its rounds, ending the effect on itself on a successful one.


\subsection{Potions - Oils}

\index[MagicItem]{Potion of Friendship with Animals}\smallskip* \textbf{Potion of Friendship with Animals}

uncommon, 200 gp, when you drink this potion, you can cast the spell friendship with animals (save DC 15) at will for 1 hour.

\index[MagicItem]{Potion of Climbing}\smallskip* \textbf{Potion of Climbing}

common, 250 gp, when you drink this potion, you gain a climb speed equal to your walking speed for 1 hour. During this time, you have +1d6 on Endurance checks you make to attempt a climb.

\smallskip* \textbf{Potion of animal clairaudience}\index[MagicItem]{Potion of animal clairaudience}

uncommon, 500 gp, this potion grants the drinker the ability to hear sounds through the ears of an animal within a 20m radius. A lead barrier blocks this effect.

\smallskip* \textbf{Potion of Animal Clairvoyance}\index[MagicItem]{Potion of Animal Clairvoyance}
uncommon, 500 gp, this potion grants the drinker the ability to see through the eyes of any animal within a 20m radius. A lead barrier blocks this effect.

\smallskip* \textbf{Potion of Animal Control}\index[MagicItem]{Potion of Animal Control}
rare 1500 gp, anyone who drinks this position is as if they have cast Dominate Beasts


\smallskip* \textbf{Potion of Dragon Control}\index[MagicItem]{Potion of Dragon Control}
legendary, 5,000 gp, this potion grants power equivalent to the dominate monster spell on a single dragon type. You can control a dragon within 20 meters for 5d4 rounds.

\smallskip* \textbf{Potion of Undead Control}\index[MagicItem]{Potion of Undead Control}
2500 gp, rare, although undead are normally immune to this type of effect, this potion allows the drinker to affect 3d6 HD of undead (intelligent or otherwise) as if using the charm spell. The duration of the effect is 5d4 rounds.

\smallskip* \textbf{Potion of People Control}\index[MagicItem]{Potion of People Control}
500 gp, uncommon, once ingested, this potion grants the drinker a power analogous to the charm spell.

\smallskip* \textbf{Potion of Plant Control}\index[MagicItem]{Potion of Plant Control}
1500 gp, rare, the drinker of this potion is able to control all plants and plant creatures (including mushrooms) in a 6x6m square area and within a distance of 27m. The effect lasts for 5d4 rounds. Plants obey according to their possibilities (for example, lianas can twist and thicken, causing slowness or impairment of vision). You can command sentient plant creatures, but they are entitled to a DC 19 Will save. As with other types of enchantments, you cannot command a controlled creature to hurt itself.

\index[MagicItem]{Potion of Growth}\smallskip* \textbf{Potion of Growth}
300 gp, uncommon, when you drink this potion, you gain the "enlarge" effect of the enlarge/reduce spell for 1d4 hours (does not require concentration).

\index[MagicItem]{Potion of Heroism}\smallskip* \textbf{Potion of Heroism}
200 gp, rare, when you drink this potion, you gain 10 temporary Hit Points that last for 1 hour. For the same duration you are under the effect of the blessing spell (does not require concentration).

\index[MagicItem]{Potion of Gas Shape}\smallskip* \textbf{Potion of Gas Shape}

1500 gp, rare, when you drink this potion, for 1 hour or until you end the effect with two actions, you gain the effect of the gaseous form spell (does not require concentration).

\index[MagicItem]{Potion of Giant's Strength}\smallskip* \textbf{Potion of Giant's Strength}
varies in rarity, varies in cost, when you drink this potion, your Strength score changes for 1 hour. The type of giant determines the score (see table below). The potion has no effect if your Strength score is equal to or higher than your new score. The Frost Giant's Strength Potion and the Stone Giant's Strength Potion have the same effect.

- of the hills, Strength 5, Uncommon 500 gp

- stone or frost, Strength 6, Rare 1000 gp

- fire, Strength 7, Rare 2000 gp

- of the clouds, Strength 8, Very rare 5000 gp

- of the storms, Strength 9, Legendary 10000 gp

\index[MagicItem]{Potion of Healing}\smallskip* \textbf{Potion of Healing}
varies rarity, varies cost, when you drink from this potion, you recover a number of Hit Points that varies depending on the rarity of the healing potion.

- Common, Hit Points 2d4 + 2, 75 gp

- Major, Hit Points 4d4 + 4, 150 gp

\index[MagicItem]{Potion of Healing}\smallskip* \textbf{Potion of Greater Healing}
varies rarity, varies cost, when you drink from this potion, you recover a number of Hit Points that varies depending on the rarity of the healing potion.

- Superior, Hit Points 8d4 + 8, 350 gp

- Ultimate, Hit Points 10d4 + 20, 1500 gp

\smallskip* \textbf{Potion of Deception}\index[MagicItem]{Potion of Deception}

500 gp, rare, this potion is most aptly named, as it convinces the drinker that they have ingested a potion of another kind. For example, a fake "clairaudience potion" could make the drinker hear sounds that don't actually exist. If multiple people taste this potion, there is a 90\% chance that they will agree that it is the same type.

\index[MagicItem]{Potion of Invisibility}\smallskip* \textbf{Potion of Invisibility}
200 gp, very rare, when you drink this potion, you become invisible for 1 hour. While you are invisible, everything you carry or wear is also invisible with you. The effect ends when you attack or cast a spell.

\smallskip* \textbf{Potion of invulnerability}\index[MagicItem]{Potion of invulnerability}
800 gp, rare, a potion of invulnerability grants the drinker a +2 bonus on Saving Throws and a 2-point improvement to Defence.

\index[MagicItem]{Potion of Mind Reading}\smallskip* \textbf{Potion of Mind Reading}
200 gp, rare, when you drink this potion, you gain the effect of the detect thoughts spell (save DC 15).

\smallskip* \textbf{Potion of Levitation}\index[MagicItem]{Potion of Levitation}
200 gp, uncommon, this potion has the same effect as the levitation spell.

\smallskip* \textbf{Potion of Longevity}\index[MagicItem]{Potion of Longevity}
15,000 gp, legendary, this potion rejuvenates 1d12 years. Regained youth not only negates natural aging, but also aging caused by magical effects or creatures. There is a danger in using this potion, as each time a potion of longevity is drunk, there is a cumulative 1\% chance that all previously gained benefits with potions of this type will be reversed. A partial dose of this potion cannot be consumed.

\smallskip* \textbf{Potion of Metamorphosis}\index[MagicItem]{Potion of Metamorphosis}
2500 gp, rare, this potion grants a power analogous to the polymorph spell.

\index[MagicItem]{Potion of Resistance}\smallskip* \textbf{Potion of Resistance}
300 gp, uncommon, when you drink this potion, you gain resistance to one type of damage for 1 hour. The Arbiter chooses the type of damage or determines it randomly (Acid, Cold, Fire, Strength, Lightning, Void, Poison, Light, Sound)

\index[MagicItem]{Potion of Water Breathing}\smallskip* \textbf{Potion of Water Breathing} \textit{Potion, uncommon} 200 gp

After drinking this potion, you can breathe underwater for 1 hour.

\index[MagicItem]{Potion of Shrinking}\smallskip* \textbf{Potion of Shrinking}
300 gp, rare, when you drink this potion, you gain the "reduce" effect of the enlarge/reduce spell for 1d4 hours (does not require concentration).

\index[MagicItem]{Potion of Poison}\smallskip* \textbf{Potion of Poison}
250 gp, uncommon, this distillate looks, smells, and tastes like a healing potion or other beneficial potion. However it is actually poison masquerading as illusion magic. The identify spell reveals its true nature.

If you drink it, you take 3d6 points of poison damage, and you must succeed on a DC 13 Fortitude save or be poisoned one more round and take 1d6 points of damage at the start of the next round.

\index[MagicItem]{Potion of Poison}\smallskip* \textbf{Potion of Major Poison}
450 gp, uncommon, this distillate looks, smells, and tastes like a healing potion or other beneficial potion. However it is actually poison masquerading as illusion magic. If identified, the true nature is understood.

If you drink it, you take 5d6 poison damage, and you must succeed on a DC 18 Fortitude save or be poisoned. At the start of each of your rounds, as long as you are poisoned in this way, you take 2d6 poison damage. You can re-roll the Saving Throw at the end of each of your rounds. On a successful save, poison damage taken on subsequent turns decreases by 1d6. The poison ceases its effects when the damage drops to 0d6.

\index[MagicItem]{Potion of Speed}\smallskip* \textbf{Potion of Speed}
400 gp, very rare, when you drink this potion, you gain the effect of the haste spell for 1 minute (does not require concentration).

\index[MagicItem]{Potion of Flight}\smallskip* \textbf{Potion of Flight}
500 gp, very rare, when you drink this potion, for 1 hour you gain a flying speed equal to your normal walking speed and can hover. If the potion runs out while you are flying, you fall unless you have some other method of staying in the air.

\index[MagicItem]{Love Potion}\smallskip* \textbf{Love Potion}
120 gp, uncommon, you will be fascinated for 1 hour by the first creature you see within 10 minutes of drinking this potion. If the creature is of a species or gender to which you are normally attracted, as long as you are fascinated, you will consider it your one and only love.

\smallskip* \textbf{Treasure Finder Filter}\index[MagicItem]{Treasure Finder Filter}
500 gp, rare, whoever drinks this potion can perceive treasures that contain precious metals or gems within 72 meters, as long as they have a value of at least 50 gold pieces. You can sense the direction of the treasure, but not its exact distance. No nonmagical barrier can prevent you from sensing treasures, except a sheet of lead.

\index[MagicItem]{Oil of Sharpness}\smallskip* \textbf{Oil of Sharpness}
3200 gp, very rare, this oil can coat one slashing or piercing weapon or up to 5 slashing or piercing ammunition. Applying the oil takes 1 minute. For 1 hour, the oil-coated weapon is magical and has a +3 bonus on attack and damage rolls.

\index[MagicItem]{Ethereal Oil}\smallskip* \textbf{Ethereal Oil}
2,000 gp, rare, one dose of oil is sufficient to coat a Medium or smaller creature and the equipment it wears and carries (one additional vial is required for each size category above Medium). Applying the oil takes 10 minutes. After that, the creature gains the effect of the ethereal form spell for 1 hour.

\index[MagicItem]{Oil of Slipperiness}\smallskip* \textbf{Oil of Slipperiness}
500 gp, uncommon, the oil can cover a Medium or smaller creature, along with any equipment it wears or carries (one additional vial is required for each size category above Medium). Applying the oil takes 10 minutes. The creature then gains the benefit of the freedom of movement spell for 8 hours. Alternatively, with two actions you can pour the oil onto the ground, doubling the effect of the anointed spell on that area for 8 hours.


\subsection{Rings}

\index[MagicItem]{Ring Storing Spells}\smallskip* \textbf{Ring Storing Spells}
24,000 gp, rare, this ring stores spells cast upon it, retaining them until the wearer uses them. The ring can accumulate up to 3 Spells for a total of 17 Spell Points with a maximum of 6 single Spell Points.

Any creature can cast an accumulated spell of level 1 to 5 on the ring by touching it. The spell has a DC equal to 10 + 2 x spell level, any attack roll is made by the caster of the spell.

The caster must aim at the ring for it to absorb. If the ring cannot contain the spell, the spell manifests normally. A spell cast through this ring is no longer contained within it, freeing up space for other spells.

\index[MagicItem]{Ring of Ram}\smallskip* \textbf{Ring of Ram}
5,000 gp, rare, While wearing this ring, you can use two actions to expend 1 to 3 charges to attack a visible creature within 20 meters of you.

The ring produces a spectral ram's head and makes its attack roll with a +7 bonus. On a hit, for each charge expended, the target takes 2d10 force damage and is pushed 1 meter away from you.

Alternatively, you can expend 1 to 3 ring charges as a two action attempt to break a visible object within 20 meters of you that is not worn or carried. The ring makes a Strength +5 check for each expended charge.

This ring has 3 charges, and regains 1d3 expended charges each morning at dawn.

\index[MagicItem]{Ring of Feather Fall}\smallskip* \textbf{Ring of Feather Fall}
2000 gp, rare, falling more than 1 meter and wearing this ring triggers the Feather Fall spell

\index[MagicItem]{Ring of Waterwalking}\smallskip* \textbf{Ring of Waterwalking}
1500 gp, uncommon. While wearing this ring, you can stand or move on any liquid surface as if it were solid ground.

\index[MagicItem]{Ring of Heat}\smallskip* \textbf{Ring of Heat}
5000 gp, uncommon, while wearing this ring, you have resistance to cold damage. Plus, you and everything you wear and carry are immune to the effects of temperatures as low as -45C.

\index[MagicItem]{Ring of Water Elementals}\smallskip* \textbf{Ring of Water Elementals}
250,000 gp, legendary. this ring is connected to the Elemental Plane of Water. While wearing it, you have +1d6 on attack rolls against elementals of the Water Elemental Plane, and they have -1d6 on attack rolls made against you.

You can expend 2 ring charges to cast dominate monsters on a water elemental. Also, you can stand and walk on liquid surfaces as if they were solid ground. You can speak and understand Aquan.

If you help kill a water elemental while wearing the ring, you gain access to the following additional properties:

\smallskip- You can breathe underwater and have speed again equal to your walking speed.

\smallskip- You can cast the following spells through the ring, expending the required number of charges: create or destroy water (1 charge), control weather (3 charges), wall of ice (3 charges), or ice storm (2 charges).

\medskip

The ring has 5 charges. Recovers 1d4 + 1 charges each day at dawn. Spells cast via the ring have a save DC of 21.

\index[MagicItem]{Ring of Air Elementals}\smallskip* \textbf{Ring of Air Elementals}
250,000 gp, legendary, this ring is connected to the Elemental Plane of Air. While wearing it, you have +1d6 on attack rolls against elementals of the Air Elemental Plane, and they have -1d6 on attack rolls made against you.

You can expend 2 ring charges to cast dominate monsters on an air elemental. Also, when you fall, you drop 20 meters per round and take no damage from the fall. You can speak and understand Ictum.

If you help kill an air elemental while wearing the ring, you gain access to the following additional properties:

\smallskip- You have resistance to lightning damage.

\smallskip- You have a flying speed equal to your walking speed and can hover.

\smallskip- You can cast the following spells through the ring, expending the required number of charges: Chained Lightning (3 charges), gust of wind (2 charges), or wall of wind (1 charge).

\medskip

The ring has 5 charges. Recovers 1d4 + 1 charges each day at dawn.

Spells cast via the ring have a save DC of 21.


\index[MagicItem]{Ring of Fire Elementals}\smallskip* \textbf{Ring of Fire Elementals}
250,000 gp, legendary, this ring is connected to the Elemental Plane of Fire. While wearing it, you have +1d6 on attack rolls against elementals of the Elemental Plane of Fire, and they have -1d6 on attack rolls made against you.

You can expend 2 ring charges to cast dominate monsters on a fire elemental. Additionally, you have resistance to fire damage. You can speak and understand Ignan.

If you help kill a fire elemental while wearing the ring, you gain access to the following additional properties:

\smallskip- You have immunity to fire damage.

\smallskip- You can cast the following spells through the ring, expending the required number of charges: Burning Wave (1 charge), wall of fire (3 charges), or fireball (2 charges).

\medskip

The ring has 5 charges. Recovers 1d4 + 1 charges each day at dawn.

Spells cast via the ring have a save DC of 21.

\index[MagicItem]{Ring of Earth Elementals}\smallskip* \textbf{Ring of Earth Elementals}
250,000 gp, legendary, this ring is connected to the Elemental Plane of Earth. While wearing it, you have +1d6 on attack rolls against elementals of the Earth Elemental Plane, and they have -1d6 on attack rolls made against you.

You can expend 2 ring charges to cast dominate monsters on an earth elemental. Additionally, you can move across hindering terrain made up of rubble, stone, or dirt as if it were normal terrain. You can speak and understand Tremun.

If you help kill an earth elemental while wearing the ring, you gain access to the following additional properties:

\smallskip- You have resistance to acid damage.

\smallskip- You can move through earth or solid rock as difficult terrain. If you finish your round there, you are thrown out onto the nearest unoccupied space you last occupied.

\smallskip- You can cast the following spells from the ring, expending the required number of charges: stone shape (2 charges), stone wall (3 charges), or stone skin (1 charge).

\medskip

The ring has 5 charges. Recovers 1d4 + 1 charges each day at dawn.

Spells cast via the ring have a save DC of 21.

\smallskip* \textbf{People's Ring of Control}\index[MagicItem]{People's Ring of Control}
2500 gp, rare, this ring grants the wearer the ability to use the charm spell once per day. The effect lasts until the controller ends it, 1 hour passes, or dispel magic is used.

\smallskip* \textbf{Plant Ring of Control}\index[MagicItem]{Plant Ring of Control}
5000 gp, very rare, the wearer of this ring can control plants and plant creatures in a 3x3 m square area within a distance of 18 meters. Even if a plant is immobile, it can move while under the effect of this ring. The control lasts as long as the one exercising it maintains total concentration, which prevents any other action.

\smallskip* \textbf{Ring of Weakness}\index[MagicItem]{Ring of Weakness}
rare, once worn, this ring can only be removed by removing curse. Over the course of 6 rounds, the wearer's strength is reduced to -3.

\index[MagicItem]{Ring of Three Wishes}\smallskip* \textbf{Ring of Three Wishes}
75,000 gp, legendary, While wearing this ring, you can use two actions to expend 1 of its 1d3 charges to cast the wish spell from it. The ring loses its magic when you use the last charge.

\index[MagicItem]{Ring of Evasion}\smallskip* \textbf{Ring of Evasion}
5,000 gp, rare, While wearing this ring and failing a Reflex save, you can use your reaction action to expend 1 charge to succeed on the save you just failed. This ring has 3 charges, and regains 1d3 expended charges each morning at dawn.

\index[MagicItem]{Djinni Summoning Ring}\smallskip* \textbf{Djinni Summoning Ring}
35,000 gp, legendary, While wearing this ring, you can speak its command word with two actions to summon a specific djinni of the Elemental Plane of Air. The djinni appears in an unoccupied space of your choice within 36 meters of you. It remains as long as you concentrate (as if concentrating on a spell), for up to 1 hour, or until it drops to 0 Hit Points. Then it returns to its home plane.

While summoned, the djinni is friendly to you and your companions. It obeys any command you give it, no matter what language you use. If you give no commands, the djinni will defend itself against attacks but take no other action.

After the djinni's departure, it cannot be summoned again for 24 hours, and if the djinni dies, the ring loses its magic.

\index[MagicItem]{Ring of Animal Influence}\smallskip* \textbf{Ring of Animal Influence} \textit{Ring, rare} 4000 gp

While wearing this ring, you can use two actions to expend 1 of its charges to cast one of the following spells with it: friendship with animals (save DC 15), speak with animals, fear (save DC 15, takes only target beasts that have Intelligence -2 or less).

This ring has 3 charges, and regains 1d3 expended charges each day at dawn.

\smallskip* \textbf{Ring of Deception}\index[MagicItem]{Ring of Deception}
rare, the wearer of this cursed ring is convinced that it has a power chosen by the Arbiter or determined randomly.

\index[MagicItem]{Ring of Invisibility}\smallskip* \textbf{Ring of Invisibility}
10,000 gp, very rare, while wearing this ring, you can make yourself invisible with two actions. Everything you wear or carry becomes invisible with you. You remain invisible until the ring is removed, you attack or cast a spell, or until you use two actions to become visible again.

\index[MagicItem]{Ring of Freedom of Action}\smallskip* \textbf{Ring of Freedom of Action}
20,000 gp, rare, While wearing this ring, hindering terrain costs you no additional movement. Furthermore, the magic can neither reduce your speed nor make you paralyzed or entangled.

\index[MagicItem]{Swimming Ring}\smallskip* \textbf{Swimming Ring}
3000 gp, uncommon, while wearing this ring, you have a swim speed of 12 meters.

\index[MagicItem]{Ring of Protection}\smallskip* \textbf{Ring of Protection}
cost varies, rarity varies, while wearing this ring, you have a +1 (5000 gp, rare), +2 (7500 gp, rare), +3 (12,000 gp, very rare) bonus to Defence and Saving Throws.

\index[MagicItem]{Enchantment Keeping Ring}\smallskip* \textbf{Enchantment Keeping Ring}
35,000 gp, legendary, While wearing this ring, you have +1d6 on Saving Throws against any spell that targets only you and not an area of effect. Additionally, if you make a critical saving success and the spell is 6-level or lower, the spell has no effect on you and instead targets the caster who cast the spell.

\index[MagicItem]{Ring of Regeneration}\smallskip* \textbf{Ring of Regeneration}
12,000 gp, very rare, while wearing this ring, you regain 1d6 Hit Points every 10 minutes, provided you have at least 1 hit point left. If you lose a body part, the ring causes the missing part to grow back to full function in 1d6 + 1 days, provided you always have at least 1 hit point left over the entire period.

\index[MagicItem]{Ring of Resistance}\smallskip* \textbf{Ring of Resistance}
6000 gp, rare, while wearing this ring, you have resistance to one type of damage. The gem set in the ring indicates the type of damage, which is chosen or determined randomly by the Arbiter.

\medskip

\begin{tabular}{lll}
\textbf{d10} & \textbf{Damage Type} & \textbf{Gem}\\

\hline
1 &Acid &Pearl\\
2& Strength &Sapphire\\
3& Cold &Tourmaline\\
4& Lightning Bolt &Citrine\\
5& Fire & Garnet\\
6& Void& Jet\\
7& Positive Energy &Jade\\
8& Light &Topaz\\
9& Sound &Spinel\\
10& Negative Energy &Amethyst\\
\end{tabular}

\medskip

\index[MagicItem]{Jump Ring}\smallskip* \textbf{Jump Ring}
2500 gp, uncommon, While wearing this ring, as a two action you can cast the jump at will spell with it, but only you can be the target.

\index[MagicItem]{Mindshield Ring}\smallskip* \textbf{Mindshield Ring}
16,000 gp, uncommon. While wearing this ring, you are immune to magic that allows other creatures to read your thoughts, determine if you are lying, know your Traits, or learn what kind of creature you are. Creatures can communicate with you telepathically only if you let them.

You can use two actions to make the ring invisible until another action makes it visible again, until you remove it or die. If you die while wearing this ring, your soul is captured in it, unless it already harbors another soul. You can decide to stay in the ring or reach the afterlife. As long as your soul remains in the ring, you can communicate telepathically with any creature wearing it. The wearer cannot prevent this form of telepathic communication.

\index[MagicItem]{Ring of Shooting Stars}\smallskip* \textbf{Ring of Shooting Stars}
14,000 gp, very rare, While wearing this ring in dim light or darkness, you can cast dancing lights and light from it at will. Casting either spell via the ring requires two actions. The ring has 6 charges for the following other properties.

The ring regains 1d6 expended charges each day at dawn.

\textit{Luminescence}. Spend 1 charge with two actions to cast the luminescence spell using the ring.

\textit{Ball of Lightning}. You can expend 2 charges as a two action to create one to four 1m-diameter spheres of lightning. The more orbs you create, the less powerful each orb will be individually.
Each sphere appears in a visible unoccupied space within 36 meters of you. The orb lasts as long as you concentrate on it (as if concentrating on a spell), up to 1 minute. Each orb radiates dim light in a 10m radius. With two actions you can move each sphere up to 10 meters, but not more than 16 meters away from you. When a creature other than you is within 1 meter of a sphere, the sphere discharges lightning at that creature and then disappears. That creature must make a DC 18 Reflex save. On a failed save, the creature takes lightning damage based on the number of orbs you created (4 orbs, 2d4 damage; 3 orbs, 2d6 damage; 2 orbs, 2d6 damage; 5d4 damage; 1 sphere, 4d12 damage).

\textit{Shooting Stars}. You can expend 1 to 3 charges with two actions. For each charge expended, you fire a spark of light from the ring to a visible point within 20 meters of you. Each creature in a 5m cube originating from that point is covered in sparks and must make a DC 15 Dexterity Saving Throw, taking 5d4 fire damage on a failed save, or half as much damage on a successful one. .

\index[MagicItem]{Telekinesis Ring}\smallskip* \textbf{Telekinesis Ring}
80,000 gp, very rare, While wearing this ring, you can cast the telekinesis spell at will, but you can only target objects that are not being worn or carried.

\index[MagicItem]{Ring of X-Ray Sight}\smallskip* \textbf{Ring of X-Ray Sight}
6,000 gp, rare, While wearing this ring, you can use two actions to speak its command word. When you do, you can see through solid matter for 1 minute. This sight has a radius of 10 meters. To you, solid objects within the beam appear transparent and don't block light from passing through them.

This sight can penetrate 30cm of stone, 1 cm of base metal, or up to 1 meter of wood or earth. Dense substances block vision, as does a thin sheet of lead. Whenever you use the ring again before finishing a night's rest, you must succeed at a DC 18 Fortitude save or become fatigued.

\subsection{Hats, Cloaks, Glasses, Tunics}


\index[MagicItem]{Bandana of Intelligence}\smallskip* \textbf{Bandana of Intelligence}
8,000 gp, rare, while wearing this bandana your Intelligence is +4. The band has no effect if you already have Intelligence is already +4 or higher.

\index[MagicItem]{Camouflage Hat}\smallskip* \textbf{Camouflage Hat}
5,000 gp, uncommon, While wearing this hat, you can use two actions to cast the disguise self spell at will. The spell ends when the hat is removed.

\index[MagicItem]{Cloak of the Arachnid}\smallskip* \textbf{Cloak of the Arachnid}
8000 gp, very rare, while wearing this elegant black silk dress woven with silver threads, you get the following benefits:

\medskip

\begin{itemize}[leftmargin=*]
\item
You have resistance to poison damage.
\item
You have a climbing speed equal to your walking speed.
\item
You can move up, down and along vertical surfaces and upside down on ceilings, keeping your hands free.
\item
You cannot be caught by any sort of web and you move through the webs as difficult terrain.
\item
You can use two actions to cast the web spell (save DC 15). The web created by the spell fills twice its normal area. Once used, this property of the Cloak cannot be used again until the next dawn.
\end{itemize}

\index[MagicItem]{Charlatan's Cloak}\smallskip* \textbf{Charlatan's Cloak}
8,000 gp, rare, While wearing this cape that smells faintly of sulfur, you can use it to cast the dimension door spell with two actions. This cape's property cannot be used again until dawn. When you disappear, you leave behind a cloud of smoke, and reappear at your destination within a similar cloud of smoke. This smoke slightly obscures the space you left and the one you reappear in, and dissipates at the end of your next round. A light or stronger wind disperses the smoke.

\index[MagicItem]{Cloak of Distortion}\smallskip* \textbf{Cloak of Distortion}
60,000 gp, rare, While wearing this cloak, it projects an illusion that makes you appear as if you are somewhere near your actual location, causing all creatures to have -1d6 on attack rolls against you. If you take damage, the property ceases to function until the start of your next round. This property is suppressed while you are incapacitated, restrained, or otherwise unable to move.

\index[MagicItem]{Cloak of the Elves}\smallskip* \textbf{Cloak of the Elves}
5,000 gp, uncommon, While wearing this cloak by pulling up the hood, Awareness checks made to notice you are -1d6, and you have +1d6 on Dexterity checks made to hide. Pulling the hood up or down requires two actions.

\index[MagicItem]{Cloak of the Manta}\smallskip* \textbf{Cloak of the Manta}
6000 gp, uncommon, while wearing this cloak with the hood pulled up, you can breathe underwater, and have a swim speed of 20 meters. Pulling the hood up or down takes 1 action.

\index[MagicItem]{Cloak of the Bat}\smallskip* \textbf{Cloak of the Bat}
6,000 gp, rare, while wearing this cloak, you have +1d6 on Dexterity checks. In areas of dim light or darkness, you can grasp the edges of the cloak with both hands and use it to move at a flying speed of 12 meters. If you were to stop holding the edges of the Cape while flying this way, you would lose your flying speed. While wearing the Cloak in an area of dim light or darkness, you can use your action to cast polymorph upon yourself, transforming you into a bat. While in bat form, maintain your Intelligence, Wisdom, and Charisma scores. The Cloak cannot be used in this way again until the next dawn.

\index[MagicItem]{Cloak of Protection}\smallskip* \textbf{Cloak of Protection}
varies in rarity, varies in cost, while wearing this cloak, you gain a +1 (uncommon, 3500 gp), +2 (rare, 6000 gp), +3 (very rare, 15000 gp) bonus to Defence and Saving Throws.

\index[MagicItem]{Cloak of Spell Resistance}\smallskip* \textbf{Cloak of Spell Resistance}
uncommon, 3,000 gp. While wearing this cloak, you have +2 on Saving Throws vs. spells.


\smallskip* \textbf{Cloak of Venom}\index[MagicItem]{Cloak of Venom}
rare, 4,000 gp, this cloak is usually made of wool, though it can also be leather. The garment can be manipulated without danger, but it deals 5d6 poison damage as soon as it is worn. Each subsequent round, a DC 21 Fortitude save can be made to reduce the damage by 1d6 to a minimum of 1d6 damage remaining. The cloak can only be removed with a remove curse or wish spell.

\smallskip* \textbf{Eyes of petrification}\index[MagicItem]{Eyes of petrification}
these two magical crystal lenses overlap the pupils of the eyes. When a creature puts on these lenses, it is immediately petrified with no Saving Throw. About a quarter of these items (25\% chance) instead allow the wearer to petrify with a gaze, but in this case the victims are entitled to a Saving Throw. It is not possible to combine two types of magic lenses.

\index[MagicItem]{Fascinating Eyes}\smallskip* \textbf{Fascinating Eyes}
3,000 gp, uncommon, While wearing these crystal goggles before your eyes, you can expend 1 charge as a two action to cast the charm person spell (save DC 15) on one humanoid within 10 meters of you, provided you and the target can see you. Lenses have 3 charges and regain 1 charge of those spent each day at dawn.

\index[MagicItem]{Eagle Eyes}\smallskip* \textbf{Eagle Eyes}
4500 gp, uncommon, while wearing these crystal glasses over your eyes, you have +1d6 on sight-based Awareness checks. In clear visibility conditions, you can see the details of even very distant creatures and objects as small as 50 centimeters.

\index[MagicItem]{Eyes of Detailed Sight}\smallskip* \textbf{Eyes of Detailed Sight}
2500 gp, uncommon, while wearing these crystal lenses in front of your eyes, you can see much better than normal up to a distance of 30 centimeters. You have +1d6 on sight-based Awareness checks while patrolling an area or studying an object at close range.

\index[MagicItem]{Nightglasses}\smallskip* \textbf{Nightglasses}
1500 gp, uncommon, while wearing these dark lenses, you have darkvision, with a range of 20 meters. If you already have darkvision, wearing these glasses increases its range by 20 meters.

\smallskip* \textbf{Tunic of Camouflage}\index[MagicItem]{Tunic of Camouflage}
1500 gp, rare, when wearing this robe, a character immediately understands its power. A cloak of camouflage allows the character to blend into whatever surroundings they may have and to hide. It has +1d6 on Stealth checks. The wielder can assume the appearance of another humanoid at will, as with the alter self (change appearance) spell. In this case, the owner's friends and those who know him very well are instinctively aware of his true identity.

\smallskip* \textbf{Archmage's Robe}\index[MagicItem]{Archmage's Robe}
8,000 gp, legendary, this seemingly ordinary suit can be yellow (01-45 on 1d100), gray (46-75), or black (76-00). May only be worn by a spellcaster with Magic Proficiency 2 or higher. Grants the following bonuses:

- Defence 15

- +2 to Saving Throws against spells and magic items

\index[MagicItem]{Tunic of Shimmering Colors}\smallskip* \textbf{Tunic of Shimmering Colors}
6,000 gp, very rare, this robe has 3 charges, and regains 1d3 expended charges each day at dawn. When wearing it, you can use two actions and expend 1 charge to cause the garment to produce a shifting weave of dazzling colors until the end of your next round. During this time, the robe sheds bright light in a 10m radius and dim light for an additional 10 meters. Creatures that see you have -1d6 on attack rolls against you. In addition, any creature in the bright light that sees you when the robe's power is activated must succeed at a DC 17 Will save or be stunned until the effect ends.

\smallskip* \textbf{Tunic of Weakening}\index[MagicItem]{Tunic of Weakening}
5,000 gp, rare, a robe of weakening looks like a magical robe of another kind. As soon as a character wears it, his Strength and Intelligence drop to -3 and he loses the ability to cast spells. The robe can be removed easily, but restoring the attributes requires remove curse followed by healing.

\index[MagicItem]{Tunic of Eyes}\smallskip* \textbf{Tunic of Eyes}
30,000 gp, rare, this robe is adorned with a design of eyes. While wearing it, you get the following benefits:

- The robe allows you to see in all directions and you have +1d6 on sight-based Awareness checks.

- You have darkvision with a range of 36 meters.

- You can see invisible creatures and objects, as well as in the Ethereal Plane, up to a range of 36 meters.

The eyes of the robe cannot be closed or averted, and while you are wearing this robe it is never considered eyes closed or averted.

A light spell cast on the robe or a daylight spell cast within 1 meter of the robe blinds you for 1 minute. At the end of each of your rounds, you can make a Fortitude save (DC 13 for light or DC 17 for daylight), ending the blinded condition on a successful one.

\index[MagicItem]{Tunic of Useful Items}\smallskip* \textbf{Tunic of Useful Items}
300 gp, uncommon, While wearing this robe covered in patches of various shapes and colors, you can use two actions to detach one of the patches, causing it to become the object or creature it represents. When the last patch is removed, the dressing gown becomes a normal garment. Robe has two of each of the following patches:

3 meter rod, Hemp rope (15 meters, coiled), Lantern with protruding lens (full and lit), Dagger, Sack, Steel mirror.

Additionally, the robe has 4d4 other patches. The Arbiter chooses the patches or determines them at random, choosing between totally different properties from those already present.

He rolls a d100 on the following table to find out the properties of the other 4d4 robe patches of useful items.

\end{multicols}

\medskip
\begin{tabularx}{0.95\textwidth}{lX}
\textbf{d100} & \textbf{Effect}\\
\hline
01-08 & Pouch with 100 gp.\\
09-15& Silver chest (30 cm long, 15 cm wide and deep) worth 500 gp.\\
16-22& Iron gate (maximum 3 meters wide and high, barred on the side of your choice), which you can place on any opening within reach; it adapts to enter the opening, fixing itself and creating hinges.\\
23-30 &10 gems worth 100 gp each.\\
31-44 & A wooden ladder (7.5 metres).\\
45-51 &A racehorse with saddlebags 52-59 Fossa (a 3m cube), which you can place on the ground within 3 meter of you.\\
60-68 &4 healing potions. \\
69-75 & Rowboat (3.5 meters long).\\
76-83& Scroll of Spells containing a 1st to 3rd level spell.\\
84-90& Two mastiffs.\\
91-96 &Window (60 x 120 cm, maximum depth 60 cm), which you can place on any vertical surface within reach.\\
97-100 &Portable ram.\\
\end{tabularx}

\begin{multicols}{2}

\medskip

\index[MagicItem]{Tunic of the Stars}\smallskip* \textbf{Tunic of the Stars}
60,000 gp, rare, While wearing this robe, you gain a +1 bonus on Saving Throws. Six stars, positioned on the top front of the dressing gown, are larger than the others. While wearing this robe, you can use two actions to extract one of the stars and use it to cast Arcane Dart. Every day at sunset, the removed star reappears on the dressing gown. While wearing the robe, you can use two actions to enter the Astral Plane along with anything you're wearing or carrying. You will stay there until you use two actions to return to the floor you were on before. You reappear in the last occupied space you occupied, or if that space is occupied, in the nearest unoccupied space.

\subsection{Manuals, Tomes, Books}


\index[MagicItem]{Manual of Golems}\smallskip* \textbf{Manual of Golems}
10,000 gp, very rare, this tome contains the information and enchantments necessary to build a particular type of golem. The Arbiter chooses the type of golem that can be built or determines it randomly. To decipher and use the handbook, you must have at least Magic Proficiency 10. A creature that can't use the golem handbook and you try to read it takes 6d6 force damage.

\medskip

\begin{tabular}{llll}
3d6 &Golem &Time &Cost\\
\hline
3-4 &Clay &30 days &65000 gp\\
5-16 &Meat &60 days& 50000 gp\\
17 &Iron &120 days &100000 gp\\
18 &Stone& 90 days &80000 gp\\
\end{tabular}


\medskip

To create a golem, you must spend the amount of time indicated above working without interruption with the available manual and resting no more than 8 hours per day. You also have to pay the specified cost to purchase the necessary materials.

Once finished creating the golem, the book is consumed by arcane flames. The golem comes to life when the handbook's ashes are scattered over it. It will be under your control and understands and obeys orders given by you.

\smallskip* \textbf{Handbook of Good Health}\index[MagicItem]{Handbook of Good Health}
15,000 gp, very rare, this tome contains instructions for strengthening the body and health. To read the book you need 24 hours in a minimum of 3 days. His instructions will be followed for 4 weeks, after which the player will permanently earn a point of Constitution. Once read, the book loses its magical power and the reader will never be able to use a similar one again.

\smallskip* \textbf{Manual of Action Speed}\index[MagicItem]{Manual of Action Speed}
15,000 gp, very rare, this tome contains exercises for balance and coordination. Works like a manual of good health, but scores one point of Dexterity.

\smallskip* \textbf{Manual of Physical Exercise}\index[MagicItem]{Manual of Physical Exercise}
15,000 gp, very rare, this tome works exactly like the health manual, but grants the reader one point of Strength.

\index[MagicItem]{Tome of Authority and Influence}\smallskip* \textbf{Tome of Authority and Influence}
15,000 gp, very rare, this book contains instructions on how to influence and charm others, and his words are suffused with magic. If you spend 48 hours in a period of 6 days or less studying the book's contents and practicing its directions, your Charisma score increases by 1. Then the manual loses its magic, only to regain it after a century.

\index[MagicItem]{Tome of Understanding}\smallskip* \textbf{Tome of Understanding}
15,000 gp, very rare, this book contains exercises in intuition and discernment, and its words are suffused with magic. If you spend 48 hours in a period of 6 days or less studying the book's contents and practicing its directions, your Wisdom score increases by 1, and so does your maximum score for that ability. Then the manual loses its magic, only to recover it after a century.

\index[MagicItem]{Tome of Clear Thought}\smallskip* \textbf{Tome of Clear Thought}
15,000 gp, very rare, this book contains exercises in memory and logic, and its words are suffused with magic. If you spend 48 hours in a period of 6 days or less studying the book's contents and practicing its directions, your Intelligence score increases by 1. Then the manual loses its magic, only to recover it after a century.

\subsection{Various Magic Items}

\smallskip* \textbf{Purifying Water}\index[MagicItem]{Purifying Water}
500 gp, rare, this sweet liquid can be used to purify water (even to desalinate sea water) and to transform poisons, acids, and other noxious liquids into a drinkable drink. Additionally, purifying water neutralizes the effectiveness of any other potion. This potion can transform up to 1000 cubic meters of almost any water-based liquid, but only 10 cubic meters of acid. The effects are permanent, and a purified liquid cannot be degraded or contaminated again for a period of 5d4 rounds.

\index[MagicItem]{Wings of Flight}\smallskip* \textbf{Wings of Flight}
54,000 gp, legendary, While wearing this cloak, you can use two actions to speak its command word, turning it into a pair of bat or bird's wings that sprout from your back for 1 hour or until you repeat the command word with an action. The wings give you a flying speed of 18 meters. When they disappear, you won't be able to use them until dawn of day after hours.

\index[MagicItem]{Iron Vial}\smallskip* \textbf{Iron Vial}
35,000 gp, legendary, this iron bottle has a brass cap. You can use two actions to speak the flask's command word, targeting one visible creature within 20 meters of you. If the target is a native of a different plane of existence than yours, it must succeed at a DC 21 Will save or be trapped in the flask. If the target has already been trapped in the flask, it receives +1d6 on its Saving Throw. Once trapped, the creature will remain in the flask until freed. The flask can only hold one creature at a time. A creature trapped in the flask doesn't need to breathe, eat, or sleep, and it doesn't age. You can use two actions to remove the bottle's cap and free the creature it contains. The creature will be friendly to you and your companions for 1 hour and will obey your commands for that duration. If you give her no commands or give her one that would cause her death, she will defend herself but take no other action. At the end of the duration, the creature will act according to its normal behavior

The identify spell reveals that a creature is inside the flask, but the only way to determine what sort of creature it is is to open the flask. A newly discovered iron flask may already contain a creature chosen by the Arbiter or randomly determined.

\medskip

\begin{tabular}{ll}
\hline
d100 &Contains\\
1-50 &Blank\\
51-66 &Demon \\
67 &Angelo Deva\\
68-69 & Devil (upper)\\
70-73 &Devil (lower)\\
74-75 &Djinni Genie\\
76-77 &Genio Efreeti\\
78-83 &Elemental (any)\\
84-86 & Invisible Persecutor\\
87-90 &Night hag\\
91 &Angelo Planetary\\
92-95 &Salamander\\
96 &Angelo Solar\\
97-99 &Succubus/Incubus\\
100 &Xorn\\
\end{tabular}
\medskip


\smallskip* \textbf{Water elemental amphora}\index[MagicItem]{Water elemental amphora}
2500 gp, rare, this amphora can be used to summon and control a water elemental similar to the summon elemental spell. You must prepare the magic item and conduct a ritual for one turn before the actual conjuration, which takes one round. After the elemental has been summoned, you must maintain concentration in order to give it commands. The amphora is usable once per day.

\index[MagicItem]{Crab Apparatus}\smallskip* \textbf{Crab Apparatus}
15,000 gp, legendary, this item appears as a sealed iron barrel of size Large and weighing 250kg. The barrel hides a catch, which can be found with a successful DC 25 Intelligence check. Removing the catch opens a compartment at one end of the apparatus, which allows two Medium or smaller creatures to enter. At the opposite end are ten levers, each in a neutral position, capable of moving up or down. When certain levers are employed, the apparatus transforms and resembles a giant lobster.

The apparatus is a Large item with the following stats.

Defence: 20, Hit Points: 200, Speed: 9m, swim 9m (or 0m both if legs and tail are not extended)

Damage Immunity: Poison

To be used as a vehicle, the apparatus needs a pilot. When the unit door is closed, the compartment is sealed, and does not allow air or water to enter. The compartments hold enough air for 10 hours, divided by the number of creatures inside. The apparatus floats in water and can even go underwater up to a depth of 270 meters. Below this threshold, the device takes 2d6 bludgeoning damage per minute from the pressure. A creature inside the compartment can use two actions to move up to two levers up or down. After each use, the lever returns to its neutral position. Each lever, from left to right, functions as shown in the table below.

1: Extends legs and tail, allowing apparatus to walk and swim. It retracts its legs and tail, reducing the item's speed to 0 and rendering it unable to benefit from speed bonuses.

2: Opens the front porthole. Closes the front porthole.

3: Opens the side portholes (two on each side). Closes the side portholes (two on each side).

4: Extends two claws from the front side of the apparatus. Retracts the claws.

5: Make a melee weapon attack with each claw extended: +8 to attack, reach 5 ft., one target. Hit: 7 (2d6) bludgeoning damage. Make one melee weapon attack with each claw extended: +8 to attack, reach 5 ft., one target. Hits: Target is grabbed (DC 18 to flee).

6: The apparatus walks or swims forward. The apparatus walks or swims backwards.

7: The apparatus turns 90 degrees to the left. The apparatus turns 90 degrees to the right.

8: Frontal slits emit bright light in a 10m radius and dim light for a further few feet. Turn off the lights.

9: The apparatus sinks 6 meters in liquids. The apparatus ascends 6 meters from liquids.

10: Unlocks and opens the tailgate. Closes and seals the tailgate.


\smallskip* \textbf{Vial of Curses}\index[MagicItem]{Vial of Curses}
800 gp, rare, this item has the appearance of a cruet, bottle, decanter, container, flask, or jug. It may contain a liquid or give off smoke. When the flask is first uncorked, all creatures within 10 meters are cursed.

\index[MagicItem]{Foldable Boat}\smallskip* \textbf{Foldable Boat}
12,000 gp, rare, this item looks like a wooden box measuring 12" long, 6" wide, and 6" deep. It weighs 2 kilos and floats. It can be opened to place objects inside. This item has three command words, each of which requires two actions to be spoken. A command word causes the box to unfold into a boat 3 meter long, 1 meter, and 50 centimetres  deep. The boat has a pair of oars, an anchor, a mast and a sail. The boat can hold up to four Medium-sized creatures.

The second command word causes the box to unfold into a vessel 7.2 meters long, 2.5 meters wide and 2 meters deep. The vessel has a bridge, rowing rows, five sets of oars, a rudder, an anchor, a cabin, and a square-rigged mast. The ship can hold fifteen Medium-sized creatures.

The third command word causes the folding boat to fold back into the box, provided no creatures are aboard. Any items on board that cannot fit inside the box stay outside the box as it folds. Any item on board that can fit in the box does.

\smallskip* \textbf{Drowning Basin}: This cursed basin has the appearance of a water elemental amphora. However, instead of summoning an elemental, it unleashes a globe of water that wraps around the player's head. He drowns in 2d4 rounds unless he succeeds on a Saving Throw vs. spells. Water is "sticky" and can only be removed by magic (dispel magic or destroy water).

\index[MagicItem]{Battle of the Opening}\smallskip* \textbf{Battle of the Opening}
1500 gp, rare, this hollow metal tube measures about 15 centimeters long and weighs 100gr. You can defeat it with two actions, aiming it at an object within 36m that can be opened, such as a door or lock. The clapper makes a clear sound, and a lock or lasso on the item opens unless the sound is prevented from reaching the item. If there are no locks or snares left to open, the item opens itself.

The clapper can be used ten times. After the tenth, it cracks and becomes unusable.

\smallskip* \textbf{Battle of Cannibalism}\index[MagicItem]{Battle of Cannibalism}
this item looks like an opening clapper. It functions as such for the first round of use (and has 1d4x10 charges for this purpose). However, at the second jingle all creatures within 20 meters must succeed at a DC 21 Will save or fall prey to ravenous hunger, attacking the nearest humanoid to kill and devour it. A new Saving Throw is allowed every other round. If no humanoids are present, the affected creatures will attack other creatures present.

\index[MagicItem]{Bag of Holding}\smallskip* \textbf{Bag of Holding}

There are different types of Storage Bags and they all have in common the ability to be able to contain much more than what they should given their size.

The Preservative Bags are divided into 4 types (Type I, II, III; IV) according to their conservation capacity.

If the pouch is overloaded, punctured, or torn, the pouch ruptures and is destroyed and its contents scattered across the Astral Plane. If the bag is turned inside out, its contents are expelled, unharmed, but the bag must be turned right around before it can be used again. Breathing creatures placed in the bag can survive there for a number of minutes equal to 10 divided by the number of creatures (minimum 1 minute), after which they will begin to suffocate.

Placing a bag of holding within the extradimensional space generated by a utility backpack, portable hole, or similar item destroys both items and opens a portal to the Astral Plane. The portal originates where one object has been placed inside another. Any creature within 3 meter of the portal is sucked into it and reappears in a random place on the Astral Plane, then the portal closes again. The portal is one-way and cannot be reopened.


Some spellcasters prefer to create holding chests, which function in the same way as holding bags.

\index[MagicItem]{Bag of Holding Type I}\smallskip* \textbf{Bag of Holding Type I}
500 gp, uncommon, this is the smallest model of the holding bags. It is apparently a small bag 20 cm in diameter with a mouth about as wide.
It is not possible to bring in objects that have a width greater than 20cm and a length greater than 50cm.
The maximum capacity is 20 kg/footprint 7.

\index[MagicItem]{Bag of Holding Type II}\smallskip* \textbf{Bag of Holding Type II}
1000 gp, uncommon, this is the average model of holding bags. It is apparently a bag of 40 cm in diameter with a mouth about as wide.
It is not possible to enter objects that have a width greater than 40cm and a length greater than 100cm.
The maximum capacity is 100 kg/Size 25.

\index[MagicItem]{Bag of Holding Type III}\smallskip* \textbf{Bag of Holding Type III}
1500 gp, rare, apparently a bag 80 cm in diameter with a mouth about as wide.
It is not possible to bring in objects that have a width greater than 80cm and a length greater than 150cm. The maximum capacity is 200 kg/Size 50.

\index[MagicItem]{Bag of Holding Type IV}\smallskip* \textbf{Bag of Holding Type IV}
5000 gp, very rare, apparently a big bag 120 cm in diameter with a mouth about as wide.
It is not possible to bring in objects that have a width greater than 120cm and a length greater than 200cm. The maximum capacity is 300 kg/Size 75.

\index[MagicItem]{Crossbow of Arcane Dart}\smallskip* \textbf{Crossbow of Arcane Dart} This small, one-handed crossbow has the ability to manifest a magical bolt.
By spending 2 Actions, it is possible to fire a magical missile as if it were a single Arcane Dart.

\index[MagicItem]{Devouring Bag}\smallskip* \textbf{Devouring Bag}
2000 gp, rare, bag looks like a holding bag. If the bag is turned over its properties stop working. An extradimensional creature attached to the bag can sense anything placed inside it. Animal or vegetable matter placed entirely inside the bag is devoured and is lost forever. When part of a living creature is placed in the bag, there is a 50\% chance that the creature will be pulled into the bag. A creature inside the bag can use two actions to try to escape with a successful DC 18 Strength check.

Another creature can use two actions to grab the creature inside the bag and pull it out, making a DC 20 Strength check (provided it isn't itself pulled into the bag). Any creature that starts its round inside the bag is devoured, its body destroyed.

Inanimate objects up to 27 dm3 of material can be placed inside the bag. However, once per day, the pouch swallows any object placed within it and spits it out into another plane of existence. The Arbiter determines the moment and the plan. If the bag is torn apart or torn apart, it is destroyed, and whatever it contains is transported to a random location in the Astral Plane.

\index[MagicItem]{Efreeti Bottle}\smallskip* \textbf{Efreeti Bottle}
15000 gp, very rare, this painted brass bottle weighs 500 grams. When you use two actions to remove the cap, a cloud of thick smoke billows from the bottle. At the end of your round, the smoke dissipates in a harmless flash of fire, and an efreeti appears in an unoccupied space within 10 meters of you. The first time the bottle is opened, the Arbiter randomly determines what happens.

\medskip

\begin{tabularx}{0.45\textwidth}{lX}
\textbf{3d6} &\textbf{Effect}\\
\hline
3-5 & The efreeti attacks you. After fighting for 5 rounds, the efreeti disappears and the bottle loses its magic.\\
6-16 &The efreeti obeys you for 1 hour, acting on your commands. He then he goes back into the bottle, and a new cork can hold it. The cap cannot be removed until 24 hours have passed. The next two times the bottle is opened, the same effect occurs again. If the bottle is opened a fourth time, the efreeti escapes and disappears, and the bottle loses its magic.\\
17-18 & The efreeti can cast a wish spell in your favor three times. He disappears when he grants the final wish or after 1 hour, when the bottle loses its magic.
\end{tabularx}


\index[MagicItem]{Bag of Beans}\smallskip* \textbf{Bag of Beans}
5,000 gp, rare, contains 3d4 dried beans inside this bag. The bag weighs 250 grams plus 125 grams for each bean it contains.

If you spill the contents of the bag onto the ground, the beans explode within a 3m radius. Each creature in the area, including you, must make a DC 18 Reflex save, taking 5d4 fire damage on a failed save, or half as much damage on a successful one.

Fire ignites flammable items in the area that are not being worn or carried. If you remove the bean from the bag, plant it in soil or sand, and water it, the bean will take effect 1 minute later, starting at the spot in the soil where it was planted. The Arbiter chooses the effect or determines it randomly.

\end{multicols}

\vfill

\begin{center}
\includegraphics[width=0.5\linewidth]{immagini/borsetta.png}

\textit{Bag of Holding, classic model, Type II}
\end{center}


\medskip

\begin{tabularx}{0.95\textwidth}{lX}
\textbf{d100} & \textbf{Effect}\\
\hline
01 &5d4 Mushrooms sprout. If a creature eats a mushroom, roll a dice. If the result is odd, he must succeed on a DC 15 Fortitude save or take 5d6 poison damage and be poisoned for 1 hour. If the result is even, he gains 5d6 temporary Hit Points for 1 hour.\\
02-10 &Erupts a geyser that spews water, beer, juice, tea, vinegar, wine, or oil (Arbiter's choice) 10 meters in the air for 1d12 rounds.\\
11-20 & A tree man sprouts. There is a 50\% chance that the treeman is chaotic evil and will attack you.\\
21-30 &A stone statue animated in your likeness rises from the ground. She will start verbally threatening you. If you were to leave and other people arrived, the statue would describe you as the most dangerous of criminals, and would urge them to seek out and attack you. If you are on the same plane of existence as the statue, it will always know where you are. After 24 hours the statue will become inanimate.\\
31-40 &A campfire that produces blue flames rises from the ground and burns for 24 hours (or until extinguished).\\
41-50 &Spit 1d6 + 6 howler mushrooms.\\
51-60 &1d4 + 8 fuchsia toads appear. Whenever a toad is touched, it transforms into a Large or smaller monster of the Arbiter's choice. The monster stays for 1 minute and then disappears in a puff of fuchsia smoke. 61-70 A bulette comes out of the ground and attacks.\\
71-80 &A fruit tree grows. It has 1d10 + 20 fruits. 1d8 of these function as a randomly determined magic potion, while one of them acts as ingested poison of the type determined by the Arbiter. The tree vanishes after 1 hour. The harvested fruits, on the other hand, remain, and retain their magic for 30 days. \\
81-90 &A nest appears with 1d4 + 3 eggs. Any creature that eats an egg must make a DC 28 Fortitude save.
creature permanently increases its lowest ability score by 1, choosing randomly on a tie. On a failed save, the creature takes 10d6 force damage from a magical blast within it.\\
91-99 &A pyramid with a square base of 18 meters is protruding from the ground. Inside is a sarcophagus containing a sovereign mummy. The pyramid is regarded as the lair of the mummy ruler, and its sarcophagus contains treasure of the Arbiter's choice.\\
100 &A huge beanstalk grows on the spot, to a height chosen by the Arbiter. The summit leads wherever the Arbiter wants, whether it's a cloud giant's castle or another plane of existence.
\end{tabularx}

\begin{multicols}{2}

\medskip

\index[MagicItem]{Smoking Bottle}\smallskip* \textbf{Smoking Bottle}
1200 gp, uncommon, smoke continuously flows from the mouth of this brass bottle, held in by its lead cap. The bottle weighs 500 grams. When you use two actions to remove the cork, a cloud of thick smoke spreads in a 20m radius around the bottle. The cloud area is heavily obscured. For each minute the bottle remains open and within the cloud, the radius increases by 3 meter until it reaches the maximum radius of 16 meters.

The cloud persists as long as the bottle is open. Closing the bottle requires you to say its command word with two actions. Once the bottle is closed, the cloud disperses after 10 minutes. A moderate wind (15 to 30 km/h) can disperse the smoke in 1 minute, and a strong wind (more than 30 km/h) can disperse it in 1 round.

\smallskip* \textbf{Bag of Nullification}\index[MagicItem]{Bag of Nullification}
9,000 gp, rare, this magical bag functions as a bag of holding for 1d6 days. After this period, all material in it or new material added is subject to a transformation depending on its nature. Precious stones become useless stones, and precious metals are transformed into less valuable metals such as lead. Magical items lose their power without a Saving Throw, and transform into mundane items of their own type. Only extremely powerful magic items are possibly immune to this effect.

\index[MagicItem]{Brazier of Fire Elementals}\smallskip* \textbf{Brazier of Fire Elementals Command}
8,000 gp, rare, While fire burns within this brass brazier, you can use two actions to speak the brazier's command word and summon a fire elemental, as if you had cast the summon elementals spell. The brazier cannot be used in this way again, until the next dawn.

The bowl weighs 2.5 kilos.

\smallskip* \textbf{Brazier of Cursed Sleep}\index[MagicItem]{Brazier of Cursed Sleep}
this brazier looks like, and functions like, a fire element command brazier. However, when activated, the smoke gathers in a 3m radius around the brazier, putting anyone in the area into cursed sleep unless they succeed at a DC 21 Will save. A fire elemental appears normally, but is hostile and attacks all creatures present. Creatures subject to cursed sleep sleep indefinitely until slain, unless remove curse is used.

\index[MagicItem]{Jug of Infinite Water}\smallskip* \textbf{Jug of Infinite Water} 12000 gp 12000 gp, uncommon, this capped flask makes a liquid sound when shaken, as if it contains water. The jug weighs 1 kilo. You can use two actions to remove the cap and speak one of three command words, at which point an amount of fresh water or salt water (your choice) will pour out of the flask, until the start of your next round. Choose one of the following options:

\medskip

\begin{itemize}[leftmargin=*]
\item
"Stream" produces 4 liters of water.
\item
"Fountain" produces 20 liters of water.
\item
"Geyser" produces 150 liters of water which are projected from a geyser 9 meters long and 30 centimeters wide. As a two action, while holding the jug, you can target a visible creature within 10 meters of you with the geyser.

The target must succeed at a DC 15 Fortitude save or take 1d4 bludgeoning damage and be knocked prone. Instead of a creature, you can target an object that is not being worn or carried and that weighs no more than 50kg. The object is knocked over or pushed 5 meters away from you.
\end{itemize}

\smallskip* \textbf{Potion Pitcher}\index[MagicItem]{Potion Pitcher}
18,000 gp, legendary, this blue pottery jug has a solid gold stopper. The jug contains 1d4+1 magical potions, each of which can be poured every 2 days. Specific potions are determined at random, stay the same over time, and must always be poured in the same order. Not all of them are necessarily beneficial.

\index[MagicItem]{Portable Hole}\smallskip* \textbf{Portable Hole}
10,000 gp, rare, this elegant, silky-soft black fabric folds to the size of a handkerchief. It unfolds in a circular layer 1 meter in diameter. You can use 1 round to unfold a portable hole and place it on or against a solid surface, upon which the portable hole creates a hole 3 meter deep. Any creature small enough can use the portable hole to pass through the wall or surface it is resting on as long as it is less than 3 meter deep.

You can use 1 round to close a Portable Hole by pinching the edges of the fabric and folding it back. Folding the fabric closes the hole, and any creature or object inside is ejected with a 50\% chance of exiting one way or the other.

Placing a portable hole within the extradimensional space created by a holding bag, portable compartment, utility backpack, or similar item instantly destroys both items and opens a portal to the Astral Plane. The portal originates from where one object has been placed inside another. Any creature within 3 meter of the portal is sucked into it and deposited in a random location on the Astral Plane. Then the portal closes. The portal is one-way and cannot be reopened.

\index[MagicItem]{Candle of Summoning}\smallskip* \textbf{Candle of Summoning}
8000 gp, very rare, this long and thin candle is dedicated to a Patron and shares his Traits. The Traits of the candle can be identified through a 1-hour ritual of flanking the candle.

The Arbiter chooses the Patron and the Traits associated with it or determines it randomly.

Candle magic is activated when the candle is lit with two actions. After burning for 4 hours, the candle is destroyed. You can decide to turn it off in advance to use it again later. Deduct the time the candle remains before extinguishing in 1-minute increments, to determine how long the candle has burned.

When lit, the candle radiates dim light in a 10m radius. Any creature within the candle's Devout or Follower light makes attack rolls, Saving Throws, and skill checks with +1d6.

Alternatively, when you light the candle for the first time, you can cast the gate spell. Doing so destroys the candle.

\index[MagicItem]{Dimensional Logs}\smallskip* \textbf{Dimensional Logs}
4000 gp, rare, you can use 2 Actions to place these manacles on an incapacitated creature. The handcuffs fit any Small to Large creature. In addition to serving as ordinary handcuffs, the shackles prevent a creature bound with them from using any method of extradimensional movement, including teleportation or travel to different planes of existence. They do not, however, prevent a creature from passing through an interdimensional portal.

You and any creatures you designate when using the shackles can use two actions to remove them. Once every 30 days, the bound creature can make a DC 40 Strength check. On a successful one, the creature breaks free and destroys the shackles.

\index[MagicItem]{Supreme Glue}\smallskip* \textbf{Supreme Glue}
400 gp, uncommon, this milky-white, viscous substance can form a permanent adhesive bond between any two objects. It must be contained in a jar or cruet that has been coated inside with oil of slipperiness. When found, its container holds 1d6 + 1 per 500gr. 30 grams of glue can cover a square area of 30 centimeters on each side. The glue takes 1 minute to set. Once the glue sets, the bond created can only be broken by the universal solvent or oil of the ethereal form, or by the wish spell.

\smallskip* \textbf{Necklace of Salubrious Air}\index[MagicItem]{Necklace of Salubrious Air}
2500 gp, uncommon, this necklace is a chain with a platinum medallion. The necklace's magic surrounds the wearer with a bubble of pure air, making them immune to the effects of vapors and gases. The bubble allows you to survive in an airless environment for a week.

\index[MagicItem]{Climbing Rope}\smallskip* \textbf{Climbing Rope}
2000 gp, uncommon, this 18-meter-long silk rope weighs 1.5 kilos and can support up to 1,500 kilos. If you hold one end of the rope and use two actions to say the command word, the rope animates. With two actions you can command the other end to move to a destination of your choice. That end moves 3 meter during your round when it receives your first command, and 3 meter during each subsequent round until it reaches its destination, up to its maximum length, or until you tell it to stop. You can also tell the rope to tighten or unhook from an object, knot or unknot, or coil itself for carrying. If you tell the rope to tie a knot, large knots will appear at 25cm intervals along the rope. While knotted, the rope diminishes to a length of 15 meters and grants +1d6 on checks made to climb it.

The rope has Defence 20, Hardness 3, and 20 Hit Points. He regains 1 hit point every 5 minutes while he has at least 1 hit point. If the rope drops to 0 Hit Points, it is destroyed.

\index[MagicItem]{Rope of Entanglement}\smallskip* \textbf{Rope of Entanglement}
4,000 gp, rare, this rope is 10 meters long and weighs 2kg. If you hold one end of the rope and use two actions to speak its command word, the other end will lunge forward to snag a visible creature within 6 meters of you. The target must succeed at a DC 18 Reflex save or be restrained. You can release the creature by using two actions to speak a second command word. A target entangled in the rope can use two actions to make a DC 18 Strength or Escape Artist check (target's choice). On a successful save, the creature is no longer restrained by the rope.

The rope has Defence 20 and 20 Hit Points. He regains 1 hit point every 5 minutes while he has at least 1 hit point. if the rope drops to 0 Hit Points, it is destroyed.

\smallskip* \textbf{Choke Cord}\index[MagicItem]{Choke Cord}
rare, this magical cord, while normal in appearance, can come alive and attack those who attempt to use it, wrapping around the neck and attempting to strangle its victim. The choke cord is long enough to strangle up to 1d4 creatures in a 3m radius, inflicting 2d6 wounds per round on each. A successful DC 19 Reflex save is required to avoid being caught. The rope has Defence 22 and 25 Hit Points, but only those who are not strangled can attack it. Victims cannot free themselves in any way or cast spells.

\index[MagicItem]{Horn of Destruction}\smallskip* \textbf{Horn of Destruction}
750 gp, rare, you can use two actions to speak the horn's command word and then sound it, emitting a thunderous blast in a 10m cone and audible up to 200 meters away. Each creature within the cone must make a DC 18 Fortitude save. On a failed save, the creature takes 5d6 sound damage and is deafened for 1 minute. On a successful save, the creature takes half damage and is not deafened. Creatures and objects made of glass or crystal have -1d6 on Saving Throws and take 10d6 sound damage instead of 5d6.

Each use of the horn's magic has a 20\% chance to cause it to explode. The blast deals 10d6 fire damage to the player who blows it and destroys the horn.

\index[MagicItem]{Horn of Valhalla}\smallskip* \textbf{Horn of Valhalla}
6,000 gp, rare, you can use two Actions to blow this horn. In response, the warrior spirits of Asgard appear within 18 meters of you. These spirits use berserker stats. They return to Asgard after 1 hour or when they drop to 0 Hit Points. Once used, the horn cannot be used again until 7 days have passed.


\index[MagicItem]{Cube of Force}\smallskip* \textbf{Cube of Force}
16,000 gp, rare, this cube has an 1 cm to the edge. Each face has a unique mark that can be pressed. The cube starts with 36 charges, and regains 3d6 expended charges each day at dawn. You can use two Actions to press either face of the cube, expending a number of charges based on the face of the cube.

Each face has a different effect. If there are no charges left in the cube, nothing happens. Otherwise, an invisible force barrier arises, forming a 5m cube. The barrier is centered on you, moves with you, and lasts for 1 minute, until you use two actions to press the sixth face of the cube, or the cube runs out of charges. You can change the effect of the barrier by pressing a different side of the cube and expending the required number of charges, resetting its duration.

If your movement causes the barrier to touch a solid object that cannot pass through the cube, you cannot approach the object while the barrier remains.

\medskip

The cube loses charges when the barrier is targeted by certain spells or comes into contact with certain spells or magical item effects, as shown in the following table.

\medskip

\begin{tabular}{ll}
\textbf{Spell or Item} &\textbf{Charges Lost}\\
\hline
Arcane Dart (5 hits) &1\\
Disintegrate &1d12\\
Wall of fire &1d4\\
Wallpass& 1d6\\
Prismatic Spray &3d6\\
\end{tabular}

\medskip

\begin{tabularx}{0.45\textwidth}{llX}
\textbf{Face} & \textbf{Charges}& \textbf{Effect}\\
\hline
1& 1& Gases, wind and fog cannot penetrate the barrier\\
2& 2 &Nonliving matter cannot pass through the barrier. Walls, floors and ceilings can pass through it at your discretion.\\
3 &3 &Living matter cannot pass through the barrier.\\
4 &4 &The spell's effects cannot pass through the barrier.\\
5 &5 &Nothing can cross the barrier. Walls, floors and ceilings can pass through it at your discretion.\\
6 &0& The barrier deactivates. \\
\end{tabularx}


\smallskip* \textbf{Cold Protection Cube}\index[MagicItem]{Cold Protection Cube}
2500 gp, rare, this cubic charm is activated and deactivated by pressing one side (immediate action). When activated, it exudes a cubic protective field with a 10-ft. edge (similar to that of a cube of force but with a different effect). The temperature inside the protective field is maintained at 21 C. The field absorbs all cold attacks, completely negating them. If it negates more than 50 cold damage in one round (either from a single attack or from multiple attacks), however, the magical field collapses and cannot be reactivated for one hour. If the field negates more than 100 cold wounds in a round, the cube is destroyed.

\index[MagicItem]{Iron Binding Bands}\smallskip* \textbf{Iron Binding Bands}
5000 gp, rare, this rusty iron sphere measures 7.5 centimeters in diameter and weighs 500 grams. You can use two actions to speak a command word and hurl the sphere at a visible Huge or smaller creature within 20 meters of you. The sphere moves in the air, opening up in a grid of metal bands. Make a Ranged Attack Roll, if you hit, the target is entangled until you take two actions to speak a command word and free them. Doing so, or missing the attack, causes the fascia to contract back into a sphere.

A creature, including the entangled one, can use two actions to make a DC 25 Strength check to break the iron bands. If it succeeds, the item is destroyed, and the entangled creature is free. If the check fails, any further attempts made by the creature automatically fail until 24 hours have passed. Once the wraps have been used they cannot be used again until the next dawn.

\index[MagicItem]{Quiver Efficient}\smallskip* \textbf{Quiver Efficient}
2500 gp, rare, each of the quiver's three compartments is connected to an extradimensional space that allows it to carry numerous objects never weighing more than 1kg.

The smallest compartment can hold up to 60 arrows, bolts or similar items. The middle compartment can hold up to 18 javelins or similar items. The longest compartment can hold up to 6 long items, such as bows, fighting sticks or spears. You can draw any item contained in the quiver as if you were taking it from a normal quiver or scabbard.

\smallskip* \textbf{Phylactery vs Undead}\index[MagicItem]{Phylactery vs Undead}
1000 gp, rare, this sacred item allows you to use the Turning Undead Featwith a +2 bonus to the sum of Traits in common with the Patron.

\smallskip* \textbf{Phylactery of Youth}\index[MagicItem]{Phylactery of Youth}
10,000 gp, legendary, the parchment strip of this phylactery is usually encased in a small metal tube worn around the neck. When a character wears it, their natural rate of aging drops to 75\%, while any magical aging is reduced by half.

\index[MagicItem]{Fortress Instant}\smallskip* \textbf{Fortress Instant}
75,000 gp, very rare, you can use two actions to place this one cm metal cube on the ground and speak its command word. The cube quickly grows into a fortress that will remain until you use two actions to say the command word to dismiss it, which only works when the fortress is empty.

The fortress is a square tower, 6 meters on each side and 9 meters high, with loopholes on all sides and battlements at the top. Its interior is divided into two floors, with a staircase running along one wall to join them. The staircase ends with a trap door that opens onto the roof. When activated, the tower features a small door on the side facing you. The door opens only on your command, which you can pronounce with two actions. It is immune to the knock spell and similar magic, such as that of the clapper of opening.

Each creature in the area where the keep appears must make a DC 17 Reflex save, taking 10d10 bludgeoning damage on a failed save, or half as much damage on a successful one. In both cases, the creature is pushed to a space outside the keep but in close proximity to it. Objects in the area that are not being worn or carried take the same damage and are automatically pushed.

The tower is made of adamantium, and its magic prevents it from being toppled. The roof, door, and walls each have 100 Hit Points, immunity to damage from nonmagical weapons except siege weapons, and resistance to all other damage.

Only the wish spell can repair the fortress. Each wish cast causes the roof, door, or one of the walls to recover 50 Hit Points.


\smallskip* \textbf{Locating Arrow}\index[MagicItem]{Locating Arrow}
400 gp, uncommon, this arrow can be used up to 8 times per day. It is thrown into the air, and when it lands it indicates a desired direction or place. Possible indications include the nearest exit or entrance, stairways, passageways, caves, and similar areas.

\index[MagicItem]{Air Elemental Command Censer}\smallskip* \textbf{Air Elemental Command Censer}
8,000 gp, rare, While incense burns within this censer, you can use two actions to speak the brazier's command word and summon an air elemental, as if you had cast the summon elementals spell. The censer cannot be used in this way again until the next dawn. This 15cm wide and 30cm high censer looks like a goblet with a decorated lid. Weighs 0.5 kilos.

\smallskip* \textbf{Incense of meditation}\index[MagicItem]{Incense of meditation}
5,000 gp, rare, this sweet-smelling incense block is indistinguishable from ordinary incense until lit. When it burns, its distinctive fragrance and pearly smoke are recognizable with an Arcana check at DC 13. After a spellcaster has spent 8 hours reviewing the Tome and meditating near a lit block, he gains the ability to cast his spells with maximum effect and maximum possible duration, spells that require a Saving Throw will impose an additional -1 penalty. Each incense block burns for 8 hours and the effect persists for another 8 hours. Usually 2d4 incense blocks are found in the same case.

\index[MagicItem]{Lantern of Revelation}\smallskip* \textbf{Lantern of Revelation}
5,000 gp, uncommon, While lit, this lantern burns for 6 hours with 1 flask of oil, radiating bright light in a 10m radius and dim light for an additional 10 meters. Invisible creatures and objects are made visible while under the lantern's bright light.

It can use two actions to lower its cover, reducing light to dim with a 1m radius.

\smallskip* \textbf{Incense of Obsession}\index[MagicItem]{Incense of Obsession}
rare, very similar to the incense of meditation, this incense also gives the user the impression of its effect, but will be confused for 24 hours if he fails a DC 23 Will save.

\index[MagicItem]{Deck of Illusions}\smallskip* \textbf{Deck of Illusions}
6500 gp, uncommon, this box contains one set of scroll cards. A full deck contains 34 cards, each depicting a different creature. The creatures represented are left to the discretion of the Arbiter. Decks found lying around usually lack 3d6-3 cards.

Deck magic works only if the cards are drawn at random (you can use a deck of normal playing cards modified to simulate the illusion deck). You can use two actions to draw a card from the deck and throw it to a spot on the ground 10 meters away from you.

The illusion of one or more creatures forms on top of the thrown card and lasts until dispelled. The illusory creature appears real, of the appropriate size, and behaves as a real creature, except it can't deal damage. As long as you are within 16 meters of the illusory creature and can see it, you can use two actions to magically move it to anywhere within 10 meters of the card. Any physical interaction with the illusory creature reveals it as an illusion, as objects pass through it. Someone who uses two actions to visually inspect the creature identifies it as illusory with a successful DC 17 Intelligence check. The creature will then appear transparent to her.
The illusion lasts until the card is moved or the illusion is dispelled. When the illusion ends, the image on the card disappears, and that card can no longer be used.

\end{multicols}

\medskip

\begin{center}
\includegraphics[width=0.55\linewidth]{immagini/Incenso.png}

\end{center}

\begin{tabular}{ll|ll}
\textbf{Playing Card}& \textbf{Illusion}&\textbf{Playing Card}& \textbf{Illusion}\\
\hline
Ace of Hearts & Red Dragon & Ace of Diamonds & Beholder\\
King of Hearts & Knight and Four Guards & King of Diamonds & Archmage and Apprentice Magus \\
Queen of Hearts & Succubus or Incubus & Queen of Diamonds & Night Hag\\
Jack of Hearts & Druid & Jack of Diamonds & Assassin \\
Ten of Hearts & Cloud Giant & Ten of Diamonds & Fire Giant\\
Nine of Hearts &Ettin&Nine of Diamonds &Oni\\
Eight of Hearts& Bugbear&Eight of Diamonds &Gnoll\\
Two of Hearts & Goblin & Two of Diamonds & Kobold \\
Ace of Spades & Lich & Ace of Clubs & Iron Golem\\
King of Spades & Priest and Two Acolytes & King of Clubs & Bandit Captain and Three Bandits\\
Queen of Spades & Medusa & Queen of Clubs & Erinyes\\
Jack of Spades & Veteran & Jack of Clubs & Berserker \\
Ten of Spades &Frost Giant&Ten of Clubs &Hill Giant\\
Nine of Spades &Trolls&Nine of Clubs &Ogre\\
Eight of spades & Hobgoblin & Eight of clubs & Ogre\\
Two of Spades & Goblin & Two of Clubs & Kobold \\
Joker (2) &You (the owner of the deck)&&\\
\end{tabular}

\begin{multicols}{2}

\medskip

\index[MagicItem]{Deck of Wonders}\smallskip* \textbf{Deck of Wonders}
100,000 gp, legendary, usually found in a pouch or box, which contains cards made of ivory or fleece. Most of these decks (75\%) have only thirteen cards, while the remaining decks have twenty-two.

Before drawing a card, you must declare how many cards you intend to draw and then draw them randomly (you can use a modified deck of playing cards to simulate the deck). Any card drawn in excess of this number has no effect. Otherwise, as soon as you draw a card from the deck, its spell takes effect.

You must draw each card within 1 hour of the previous draw. If you don't draw the chosen number of cards, the remaining number of cards will come out of the deck spontaneously and take effect at the same time. Once a card is drawn, it will vanish from existence. Unless the card is the Fool or the Fool, the card reappears in the deck, making it possible to draw the same card twice.

\medskip

\end{multicols}

\begin{tabularx}{0.95\textwidth}{lX|lX}
\textbf{Playing Card}& \textbf{Card}&\textbf{Playing Card}& \textbf{Card}\\
\hline
Ace of diamonds& Vizier*&Ace of hearts &Fate*\\
King of Diamonds & Sun & King of Hearts & Throne\\
Queen of Diamonds & Moon & Queen of Hearts & Key\\
Jack of Diamonds & Star & Jack of Hearts & Knight\\
Two of Diamonds & Comet* & Two of Hearts & Gem*\\
Ace of Clubs & Spurs* & Ace of Spades & Dungeon * \\
King of Clubs & The Void & King of Spades & Ruin\\
Queen of Clubs & Flames & Queen of Spades & Euryale\\
Jack of Clubs & Skull & Jack of Hearts & Villain \\
Two of Clubs &Idiot&Two of Spades &Hanging*\\
Jolly & Fool*&Jolly & Buffoon\\
\end{tabularx}

\begin{multicols}{2}

\medskip

* Only in 22-card deck

\textit{Hanging} (only in decks of 22). Your mind is blown, and you switch 2 Traits

\textit{Fool}. You get 35 XP or you may draw two additional cards in addition to your declared draws.

\textit{Knight}. You gain the services of one WP 4-level fighter who appears in a space of your choice within 10 meters of you. The warrior is of your own kind and he will serve you loyally until he dies, believing that it was fate that brought him into your service. The character is controlled by you.

\textit{Key}. A rare, very rare, or legendary magic weapon with which you are proficient appears in your hands. The Arbiter determines what type of weapon it is.

\textit{Comet} (only in deck of 22). If you single-handedly defeat the next hostile monster or group you encounter, you will gain enough experience points to gain a level. Otherwise, this card will have no effect.

\textit{Euriale}. You are cursed by the card, and you take a -2 penalty on all Saving Throws as long as you are so cursed. Only a Patron or the magic of the Fate card can end this curse.

\textit{Fate} (22 deck only). The fabric of reality dissolves and reforms, allowing you to avoid or erase an event as if it never happened. You can use this card's magic as soon as you draw it, or wait any other time until you die.

\textit{Flames}. A mighty devil becomes your enemy. The devil will try to ruin and infest your existence, savoring your suffering until he tries to kill you. This enmity will last until your death or the devil's.

\textit{Rogue}. One non-player character of the Arbiter's choice becomes hostile towards you. The identity of the new enemy is unknown until the NPC or someone else reveals it. Nothing short of a wish or divine intervention can end the NPC's hostility towards you.

\textit{Gem} (only in decks of 22). Before your feet appear twenty-five jewels worth 2,000 gp each, or fifty gems worth 1,000 gp each.

\textit{Idiot} (deck of 22 only). Permanently reduce your Intelligence score by 2 (to a minimum score of 3). You may draw one more card before your other declared draws.

\textit{Moon}. You gain the ability to cast the wish spell 1d3 times.

\textit{Mate} (only in packs of 22). You lose 10000 XP, discard this card, and draw from the deck again, counting both draws as just one of your draws. If losing that amount of XP would cause you to lose a level, you'll instead be left with just enough XP to maintain your level.

\textit{Ruin}. You lose all wealth you have on you, apart from other magical items. Businesses, buildings, and the lands you own are lost in the least reality-altering way. Any documents proving you own anything you lost to this card disappear.

\textit{Sun}. Gain 35 XP, and a wondrous item (determined by the Arbiter) appears in your hand.

\textit{Dungeon} (only in decks of 22). You disappear and are buried in a state of suspended animation within an extradimensional sphere. Anything you were wearing or carrying remains in the space you occupied when you disappeared. You will remain imprisoned until you are found and removed from the sphere. You cannot be located by any divination magic, but the wish spell can reveal the location of your prison. No further cards are drawn.

\textit{Spurs} (only in packs of 22). Any magical items you wear or carry are disintegrated. Artifacts in your possession are not disintegrated, but vanish.

\textit{Stella}. Increase your ability score by 1. The score can exceed 5, but cannot exceed 7.
\textit{Skull}. You summon an avatara of death (a ghostly humanoid skeleton wrapped in a tattered black robe, holding a ghostly scythe). It appears in a space of the Arbiter's choice within 3 meters of you and attacks you, warning all others that you must win the battle alone. The avatara fights until you die or until it drops to 0 Hit Points, at which point it vanishes. If someone tries to help you, they will summon their death avatara. A creature slain by an avatara of death cannot be brought back to life.

\textit{Throne}. Gain +1d6 in Diplomacy. In addition, you get the right to own a small fortress somewhere in the world. However, the keep is currently occupied by monsters, which you will need to hunt down before you can claim it as your own.

\textit{Vizir} (only in decks of 22). At any time of your choice, within a year of drawing this card, you may meditate and ask for an answer to one of your questions and receive a truthful answer to it. Aside from providing information, the answer can help you solve a complex problem or dilemma. In other words, knowledge is provided along with wisdom on how to use it.

\textit{Blank}. This black card indicates disaster. Your soul is snatched from your body and imprisoned inside an object at a location of the Arbiter's choosing. One or more powerful creatures protect this place. As long as your soul is thus trapped, your body is incapacitated. The wish spell cannot restore your soul, but it can reveal the whereabouts of the object containing it. No more cards are drawn.

\textit{Avatar of Death}

Average undead, neutral evil

\textbf{Strength} +3

\textbf{Dexterity}' +3

\textbf{Intelligence} +3

\textbf{Wisdom} +3

\textbf{Charisma} +3

\textbf{Defence} 20

\textbf{Hit Points} half his summoner's Hit Points

\textbf{Movement}: Speed 18m, fly 18m (floats)

\textbf{Immunity to Damage}: Void, poison

\textbf{Condition Immunity}: Charmed, Poisoned, Paralyzed, Petrified, Frightened, Unconscious

\textbf{Senses}: darkvision 18m, True Seeing 18m

\textbf{Languages}: All languages known to its summoner

\textbf{Challenge} (0 XP)

\textbf{Incorporeal Movement}. The avatar can pass through creatures and objects as if they were hindering terrain. It takes 5 (1d10) force damage if it ends its round inside an object.

\textbf{Immunity to Turning}. The avatara is immune to effects that turn undead.

\textbf{Actions}

\textbf{Reaper's Scythe}. The avatara plunges its spectral scythe into a creature within 1 meter of it, dealing 7 (1d8 + 3) piercing damage plus 4 (1d8) negative energy damage.

\index[MagicItem]{Miniature of Wondrous Power}\smallskip* \textbf{Miniature of Wondrous Power}
varying rarity, varying cost, a miniature of wondrous power is a figurine of a beast, small enough to fit in your pocket. If you use two actions to speak a command word and throw the figure to a point on the ground within 20 meters of you, the figure becomes a living creature. If the space where the creature would appear is occupied by another creature or object, or if there is not enough space for the creature, the figure does not transform.

The creature is friendly towards you and your companions. It understands your languages and obeys your orders. If you don't command it, the creature defends itself but takes no other actions. See the Bestiary for the creature's other stats.

The creature remains for the duration specified for each figure. At the end of the duration, the creature reverts to its miniature form. It transforms early if it drops to 0 Hit Points or if you use two actions to speak the command word again while touching it. After the creature reverts to a miniature, its properties can't be used again until a certain amount of time has passed, as specified in the miniature's description.

\textit{Onyx Dog} (Rare, 500 gp). This onyx figurine depicts a dog. It can become a mastiff for up to 6 hours. The hound has Intelligence -2 and can speak Common. He also has darkvision with a range of 20 meters and can see invisible creatures and objects within that range. Once used, it cannot be used again until 7 days have passed.

\textit{Ivory Goat (Rare. 1000 gp)}. These ivory goat figurines are always created in sets of three. Each billy goat has a unique look and functions differently from the others. Their properties are as follows:

The billy goat of terror can become a giant billy goat for up to 3 hours. The billy goat can't attack, but you can remove its horns and use them as weapons. One horn becomes a +1 lance while the other becomes a +2 longsword.

Removing a horn takes two actions, and the weapons disappear and the horns reappear when the billy goat reverts to its miniature form. In addition, the billy goat radiates an aura of terror with a radius of 10 meters while you ride it. Any creature hostile to you that starts its round within the aura must succeed at a DC 17 Will save or remain
frightened by the billy goat for 1 minute, or until the billy goat reverts to miniature form. The frightened creature can repeat the Saving Throw at the end of each of its rounds, ending the effect on a successful one. Once a creature makes a successful Saving Throw against this effect, it is immune to the billy goat's aura for the next 24 hours. Once used, the miniature cannot be used again until 15 days have passed.

The labor billy goat can grow into a giant billy goat for up to 3 hours. Once used, it cannot be used again until 30 days have passed.
The traveling billy goat can become a large billy goat with the same stats as a racehorse. It has 24 charges, and each hour or portion of it you spend in beast form costs 1 charge. As long as it has charges, you can use it as much as you like. Once the charges are finished, it reverts to being a miniature and cannot be used again until 7 days have passed, when it has recovered all its charges.

\textit{Silver Raven} (Uncommon, 300 gp). This silver figurine depicts a raven. Can become a crow for up to 6 hours. Once used, it cannot be used again until 2 days have passed. While in raven form, the miniature allows you to cast the animal messenger spell upon it at will.

\textit{Obsidian Steed} (Very Rare, 1000 gp). This smooth obsidian figurine becomes a nightmare for up to 24 hours. The nightmare fights only to defend itself. Once used, it cannot be used again until 5 days have passed.

\textit{Marble Elephant} (Rare, 1500 gp). This marble figurine is approximately 10 centimeters wide and 10 centimeters tall. It can become an elephant for up to 24 hours. Once used, it cannot be used again until 7 days have passed.

\textit{Bronze Griffin} (Rare, 1250 gp). This bronze statuette depicts a rampant griffin. Can become a griffin for up to 6 hours. Once used, it cannot be used again until 5 days have passed.

\textit{Serpentine Owl} (Rare, 400 gp). This serpentine owl figurine can grow into a giant owl for up to 8 hours. Once used, it cannot be used again until 2 days have passed. If you are on the same plane of existence, the owl can telepathically communicate with you at any distance.

\textit{Golden Lions} (Rare, 800 gp). These golden lion figurines are always created in pairs. You can use one or both thumbnails at the same time. Each can become a lion for up to 1 hour. Once one of the lions is used, it cannot be used again until 7 days have passed.

cd \index[MagicItem]{Ammo of Killing}\smallskip* \textbf{Ammo of Killing}
700 gp, very rare, if a creature of the type, race, or group with which the arrow of slaying is associated takes damage from the arrow, the creature must make a DC 21 Fortitude save, taking an additional 6d10 piercing damage if so fails, or half as much damage on a successful one.

Once the arrow of slaying has dealt additional damage to the creature, it becomes a nonmagical arrow.

\index[MagicItem]{Crystal Ball}\smallskip* \textbf{Crystal Ball}
50,000 gp, very rare or legendary, a typical crystal ball is about 10 cm in diameter. While touching it, you can cast the scrying spell (save DC 21) with it. The following variant crystal balls are legendary items and have additional properties.

\textit{Crystal Ball of Mind Reading}. This crystal ball is approximately 12cm in diameter. While touching it, you can cast the scrying spell (save DC 21) through it. You can use two actions to cast the detect thoughts spell (save DC 21) while you are scrying through this crystal ball, targeting creatures you can see and are within 10 meters of the spell's sensor. You don't have to focus on this detect thoughts to maintain it for its duration, which ends when scrying ends.

\textit{Crystal Ball of Telepathy}. This crystal ball is approximately 12cm in diameter. While touching it, you can cast the scrying spell (save DC 21) through it. While scrying through this crystal ball, you can communicate telepathically with creatures you can see and are within 10 meters of the spell's sensor. You can also use two actions to cast the suggestion spell (save DC 21) on one of these creatures via the sensor. You don't have to concentrate on this suggestion to hold it for its duration, which ends if scrying ends. Once used, the crystal ball's suggestion power cannot be used again until the next dawn.

\textit{True Seeing Crystal Ball}. This crystal ball is approximately 12cm in diameter. While touching it, you can cast the scrying spell (save DC 21) through it. While scrying with this crystal ball, you have True Seeing with a 16 meters radius centered on the spell's sensor.

\smallskip* \textbf{Hypnotic Crystal Ball}\index[MagicItem]{Hypnotic Crystal Ball}
rare, this cursed item is indistinguishable from a normal Crystal Ball. However, anyone attempting to use the device is charmed for 1d6 turns, and a telepathic suggestion is implanted in their mind if they fail a DC 27 Will save. is under the influence of a powerful spellcaster, or even a power or being from another plane of existence. With each further use the user falls more and more under the influence of the controller, as a servant or as a tool. The user is always unaware of being enslaved.

\index[MagicItem]{Scroll of Spells}\smallskip* \textbf{Scroll of Spells}
variable rarity, see scroll crafting costs, a spell scroll contains the words of a single spell, written in a mystical code.

To read a scroll you need:

\textbf{in case of ISY SCROLL scrolls}:

to understand the content, a check of Arcana on DC 10 difficulty is sufficient

to be able to read and cast the scroll's spell requires an Intelligence check (or Arcana if known) at difficulty 12.

\textbf{in case of normal scrolls}:

to understand its contents requires a check of Arcana at difficulty 15

in order to read and cast the scroll's spell, a difficulty 20 Arcana check is required.

Casting the spell by reading it from a scroll takes the spell's normal casting time. Once the spell has been cast, the words on the scroll vanish, and the scroll is reduced to dust. If the throw is interrupted, the scroll does not dissolve.

\smallskip* \textbf{Scroll of protection against elementals}\index[MagicItem]{Scroll of protection against elementals}
800 gp, rare, protects against all elementals for 20 rounds, granting +4 to Defence and Saving Throws against attacks or effects produced by elementals.

\smallskip* \textbf{Scroll against werewolves}\index[MagicItem]{Scroll of protection against werewolves}
700 gp, uncommon, protects against all lycanthropes for 20 rounds, granting +4 to Defence and Saving Throws against attacks or effects produced by lycanthropes.

\smallskip* \textbf{Scroll against the undead}\index[MagicItem]{Scroll of protection against the undead}
900 gp, uncommon, protects against all undead for 20 rounds, granting +4 to Defence and Saving Throws against attacks or effects produced by the undead.

\smallskip* \textbf{Scroll Against Magic}\index[MagicItem]{Scroll Protecting Against Magic}
1500 gp, rare, scroll casts an Anti-Magic Field spell.

\index[MagicItem]{Pearl of Power}\smallskip* \textbf{Pearl of Power}
6,000 gp, uncommon, while you have the pearl with you, you can use two actions to recover 2d4 Spell Points. Once used, the pearl cannot be used again until the next dawn. There are more powerful variants that recover more points.

\smallskip* \textbf{Stone of Weight}\index[MagicItem]{Stone of Weight}
this object looks like a smooth and shiny black stone. When the wearer is involved in combat or flight, he is suddenly affected by the slow spell. Once taken, the stone cannot be thrown away normally, as after a short time it magically reappears on the owner's person. To get rid of the stone permanently requires the remove curse spell.

\index[MagicItem]{Arcane Stone}\smallskip* \textbf{Arcane Stone}\index{Ioun Stone}
variable cost, variable rarity, there are many types of arcane stone, each type a specific combination of shapes and colors.

When you use two actions to throw one of these stones into the air, the stone begins to orbit your head at a distance of 1d3 x 15 centimeters and grants you a boon.
After that, another creature must use two actions to grab or snag the stone and separate it from you, making a successful attack roll vs. Defence 24 or making a DC 31 Dexterity check. You can use two actions to snag and set aside. the stone, ending its effect.

A stone has Defence 24, 10 Hit Points, and resistance to all damage. As it orbits your head it is considered a worn item.

\textit{Dexterity} (very rare, 3000 gp). As it orbits your head, your Dexterity score increases by 1, to a maximum of 5.

\textit{Absorption} (very rare, 6000 gp). As it orbits your head, you can use your Action to cancel a 4-level or lower spell cast by a visible creature that targets only you. Once the stone has cleared 5 Spells, it depletes and turns dull gray, losing its magic.

\textit{Authority} (very rare, 3000 gp). As it orbits your head, your Charisma score increases by 1, to a maximum of 5.

\textit{Awareness} (rare, 12000 gp). As it orbits your head you cannot be surprised.

\textit{Strength} (very rare, 3000 gp). As it orbits your head, your Strength score increases by 1, to a maximum of 5.

\textit{Intelligence} (very rare, 3000 gp). As it orbits your head, your Intelligence score increases by 1, to a maximum of 5.

\textit{Insight} (very rare, 3000 gp). As it orbits your head, your Wisdom score increases by 2, to a maximum of 5.

\textit{Protection} (rare, 10000 gp). While orbiting your head, gain a +1 bonus to Defence.

\textit{Sustenance} (rare, 3500 gp). As it orbits your head, you don't need to eat or drink.

\index[MagicItem]{Stone of Good Luck}\smallskip* \textbf{Stone of Good Luck}
4500 gp, uncommon, while the stone is with you, you gain a +1 bonus on ability checks and Saving Throws.

\index[MagicItem]{Earth Elemental Control Stone}\smallskip* \textbf{Earth Elemental Control Stone}
8,000 gp, rare, If the stone touches the ground, you can use two actions to speak the command word and summon an earth elemental, as if you had cast the summon elementals spell. The stone cannot be used in this way again, until the next dawn. The stone weighs 2.5 kilos.

\index[MagicItem]{Sewer Pipe}\smallskip* \textbf{Sewer Pipe}
2000 gp, uncommon, must be proficient with wind instruments to use this fife. While using this fife, normal rats and giant rats are indifferent towards you and will not attack you unless you threaten or harm them. If you play the fife with two actions, you can use two actions to expend 1 to 3 charges, summoning a swarm of rats for each expended charge, provided there are enough rats within 750 m of you to summon in this manner (Arbiter's discretion). ). If there aren't enough rats to form a swarm, the charge is wasted. Called swarms move towards the music via the shortest possible route, but are otherwise not under your control. The fife has 3 charges and regains 1d3 expended charges each day at dawn.

Whenever a swarm of rats not under the control of another creature approaches within 10 meters of you while you're playing the fife, you can make a Charisma check contested by the swarm's Wisdom check. If you lose the contest, the swarm behaves as normal and cannot be distracted by fife music again for the next 24 hours. If you win the contest, the swarm is attracted to the fife's music and becomes friendly towards you and your companions as long as you continue to play the fife with two actions each round. A friendly swarm obeys your commands. If you don't issue commands to a friendly swarm, it will defend itself but take no other action.

If a friendly swarm cannot hear the fife's music at the start of the round, your control over that swarm ends, and the swarm behaves as it normally would and cannot be drawn back to the fife's music for the next 24 hours.

\index[MagicItem]{Frightening Piper}\smallskip* \textbf{Frightening Piper}
6,000 gp, uncommon, must be proficient with wind instruments to use this fife. You can use two actions to play it and spend 1 charge to create an enchanting, ghostly sound. Each creature within 10 meters of you that hears you play must succeed at a DC 17 Will save or be frightened of you for 1 minute. If you wish, all creatures in the area that aren't hostile to you can automatically succeed on their Saving Throw. A creature that fails its save can repeat it at the end of its round, ending the effect on itself on a successful one. A creature that successfully saves is immune to this fife's effect for 24 hours. The fife has 3 charges and regains 1d3 expended charges each day at dawn.

\index[MagicItem]{Pigments of Wonders}\smallskip* \textbf{Pigments of Wonders}
400 gp, very rare, usually found in 1d4 jars inside elegant wooden boxes together with a brush (weighing a total of 500 grams), these pigments allow you to create three-dimensional objects, painting them in two dimensions. Paint flows from the brush to form the desired object as you focus on the image

Each pot of paint is sufficient to cover 90 square meters of surface area, allowing you to create inanimate objects and terrain features (doors, pits, flowers, trees, cells, rooms, or weapons) that take up a total of 270 cubic meters. It takes 10 minutes to cover 90 squares.

When you complete the painting, the painted terrain feature or object becomes a real, non-magical object. Thus, painting a door on a wall creates a real door that can be opened to access what lies beyond it. Painting a pit on the floor creates a real pit, the depth of which counts towards the total area of items you can craft.

Nothing created from pigments can be worth more than 25 gp. If you paint a higher value object (a diamond or a pile of gold), the object will appear authentic, but close examination will reveal that it is made of plaster, bone, or some other worthless material.

If you paint a form of energy, such as fire or lightning, the energy appears but dissipates as soon as you complete the painting, doing no damage to anything.

\index[MagicItem]{Arcane Feather}\smallskip* \textbf{Arcane Feather}
variable cost, variable rarity, this tiny item resembles a feather. There are several types of arcane feathers, each with a single use effect. The Arbiter chooses the type of arcane feather.

\textit{Tree}. You must be outdoors to use this Arcane Feather. You can use two actions to place it on an unoccupied space on the ground. The feather vanishes, and a nonmagical oak tree grows in its place. The tree is 18 meters tall and has a trunk 1 meter in diameter. At the top, its branches extend up to 6 meters. 50 gp

\textit{More}. You can use two actions to place the Arcane Feather on a boat or ship. For the next 24 hours, the vessel may not be moved in any way. Touching the vessel with the Arcane Feather again ends this effect. When the effect ends, the feather vanishes. 50 gp

\textit{Whip}. You can use two actions to throw the Arcane Feather to a point within 3 meter of you. The feather vanishes and a floating whip appears in its place. You can then use two actions to make a melee spell attack against a creature within 3 meter of the whip, with a +9 attack bonus. On a hit, the target takes 1d6 + 5 force damage. During your round, as a two action you can direct the whip to fly up to 6 meters and repeat the attack against a creature within 3 meter of it. The whip vanishes after 1 hour, when you use two actions to dismiss it, or when you are incapacitated or die. 250 gp

\textit{Swan Ship}. You can use two actions to place the arcane plume on a body of water at least 20 meters in diameter. The feather vanishes and a 15m long and 6m wide swan-shaped boat appears in its place. The boat moves by itself and moves through the water at a speed of 9 kilometers per hour. You can use two actions while aboard to command it to move or turn 90 degrees. The boat can carry up to thirty-two Medium or smaller creatures. A Large creature counts as four Medium creatures, while a Huge creature counts as nine Medium creatures. The boat vanishes after 24 hours. You can dismiss the boat with two actions. 3000 gp

\textit{Bird}. You can use two actions to launch the Arcane Feather 1 meter in the air. The feather vanishes and a huge multicolored bird takes its place. The bird has the stats of a Roc, but obeys simple commands and cannot attack. It can carry up to 250 kilos while flying at its maximum speed (24 kilometers per hour for a maximum of 216 kilometers per day, with an hour of rest for every 3 hours of flight), or 500 kilos of weight at half speed. The bird vanishes after flying the maximum possible distance in a day or if it drops to 0 Hit Points. You can dismiss the bird with two actions. 3000 gp

\textit{Fan}. If you are on a boat or ship, you can use two actions to throw the Arcane Feather up to 3 meter in the air. The feather vanishes and a giant fan appears in its place. The fan floats and creates a wind strong enough to inflate the ship's sails, increasing its speed by 7.5 kilometers per hour for 8 hours. You can dismiss the fan with two actions. 250 gp

\index[MagicItem]{Dust of Aridity}\smallskip* \textbf{Dust of Aridity}
120 gp, rare, this small package contains 1d6 + 10 cm of powder. You can use two actions to sprinkle a pinch of dust on the water. The dust turns a 5m cube of water into a marble-sized ball of dust that floats or settles where it was throw the dust. The weight of the ball is negligible.

Anyone can use two actions to smash the ball against a hard surface, causing the ball to break and release the absorbed water from the dust. Doing so depletes the ball's magic.

An elemental composed primarily of water and exposed to a pinch of this dust must make a DC 15 Fortitude save, taking 10d6 void damage on a failed save, or half as much damage on a successful one.

\smallskip* \textbf{Discovering Powder}\index[MagicItem]{Discovering Powder}
500 gp, uncommon, this fine powder looks like a very light metallic speck. A handful of this substance sprayed into the air coats all objects in a 3m radius, making everything visible. When sprayed through a blowgun, the powder fills a cone 6 meters long and 1 meter wide at the tip. The dust negates illusion power effects, warp cloak, elven cloak, and special abilities of creatures such as unstable molossers and warp panthers; the effect lasts for 2d10 turns. The telltale powder is usually stored in small silk pouches or hollow tubes made of bone; 5d10 doses of powder are usually found.

\index[MagicItem]{Vanishing Dust}\smallskip* \textbf{Vanishing Dust}
700 gp, rare, found in small bags, this powder looks like very fine sand. There is enough in one bag for one use. When you use two actions to fling the powder into the air, you and each creature and object within 3 meter of you become invisible for 2d4 minutes. The duration is the same for all subjects, and when the spell takes effect, the dust is consumed. If a creature affected by the dust attacks or casts a spell, the invisibility ends only for that creature.

\index[MagicItem]{Sneezing and Choking Powder}\smallskip* \textbf{Sneezing and Choking Powder}
480 gp, uncommon, found in small containers, this powder looks like fine sand. It appears similar to vanishing dust, and the identify spell reveals it as such. There is enough for one use. When you use two actions to fling a handful of dust into the air, you and all creatures that need to breathe within 10 meters of you must succeed on a DC 17 Fortitude save or stop breathing, and begin sneezing in uncontrollable way. A creature afflicted in this way is incapacitated and suffocates. As long as it is conscious, the creature can repeat the Saving Throw at the end of each of its rounds, ending the effect on a successful one. Even the lesser restoration spell can end the effect affecting the creature.

\index[MagicItem]{Cubic Portal}\smallskip* \textbf{Cubic Portal}
40,000 gp, legendary, this 5cm cube radiates palpable magical energy. The six faces of the cube are each connected to a different plane of existence, one of which is the Material Plane. The other faces are connected to planes determined by the Arbiter.

You can use two actions to press a face of the cube to cast the portal spell through it, opening a passage to the plane connected to that face. Alternatively, if you use two actions to press a face twice, you can cast the plane shift spell (save DC 17) via the cube and transport its targets to the plane connected to that face. The cube has 3 charges. Each use of the cube expends 1 charge. The cube recovers 1 expended charge each day at dawn.

\index[MagicItem]{Well of Many Worlds}\smallskip* \textbf{Well of Many Worlds}
75000 gp, legendary, this elegant black fabric, soft as silk, is wrapped up to the size of a handkerchief. It unfolds into a circular sheet 1.8 meters in diameter. You can use two actions to unfold and place the Well of Many Worlds on a solid surface, upon which it creates a two-way portal to another world or plane of existence. Each time the item opens a portal, the Arbiter decides where it leads. You can use two actions to close an open portal by grabbing the edges of the fabric and folding them back. Once a well of many worlds has opened a portal, it cannot do so again until 1d8 hours have passed.

\smallskip* \textbf{Entangling Net}\index[MagicItem]{Entangling Net}
800 gp, rare, this 3m-square net can be thrown at an opponent to entangle them. The net is very resistant and it takes the strength of a giant (Str 5) to tear it apart with bare hands. The net also resists cuts, and must be hit with extreme precision (Defence 25, HP 30) for it to yield. The net can also be hung or placed on the ground as a trap, which will magically activate at the wielder's command.

\smallskip* \textbf{Trapping Net}\index[MagicItem]{Trapping Net}
900 gp, rare, this net can only be used underwater, but functions exactly like an entangling net on the surface, floating if it takes up to 10 meters to entangle an opponent.

\smallskip* \textbf{Broom of Animated Attack}\index[MagicItem]{Broom of Animated Attack}
this item is indistinguishable in appearance from a regular broom. In all checks it is identical to a flying broom, up to 6 meters in height. When this happens, the broom spins and knocks its pilot onto its head from a height of 1d4+5 x 15 centimeters (no falling damage is dealt since the distance is less than 3 meter). The broom then attacks the victim, hitting him in the face with the brush and beating him with the handle. The broom makes two attacks per round with each end (two attacks with the brush and two with the handle for a total of four attacks). The brush blinds the victim for 1 round when it strikes. The handle deals 1d3 wounds. The broom has Defence 13, 18 Hit Points, and has +4 on attack rolls.

\index[MagicItem]{Flying Broom}\smallskip* \textbf{Flying Broom}
8,000 gp, uncommon, this wooden broom, weighing about 1kg, functions like a normal broom until you sit on it and say the command word. It then begins to float below you and can be ridden in the air. It has a flight speed of 15 meters. It can carry up to 200 kilos, but its flight speed becomes 10 meters if it has to carry more than 100 kilos. When you land, the broom stops floating.

By saying the command word, naming the place and if you're familiar with it, you can send the broom by itself to a place up to 1.5 kilometers away from you. The broom will come back to you when you say another command word, as long as it is still within 1.5 kilometers of you.

\smallskip* \textbf{Broom of Cursed Flight}\index[MagicItem]{Broom of Cursed Flight}
this magic broom looks like a flying broom. However, when activated, it flies up to 15m high or to the ceiling (whichever is lower) and then stops functioning, sending the rider falling. Then the broom falls to the ground and loses its magical power.

\index[MagicItem]{Sphere of Annihilation}\smallskip* \textbf{Sphere of Annihilation}
250,000 gp, legendary, this 50cm diameter black sphere is actually a hole in the fabric of the multiverse, floating in space and stabilized by the magical field that surrounds it.

The sphere annihilates all matter that passes through it and all matter that passes through it. The only exception is artifacts. Unless the artifact is susceptible to damage from the sphere of annihilation, it can pass through the sphere without problem. Anything else that touches the sphere and isn't completely engulfed and annihilated by it takes 4d10 force damage per round.

The sphere remains motionless until someone controls it. If you are within 20 meters of an uncontrolled sphere, you can take two actions to make a DC 30 Arcana check. If successful, the sphere levitates in a direction of your choice, a number of meters equal to 5 x the Intelligence (minimum 1 meter). If you fail, the sphere moves 3 meters towards you. A creature whose space the sphere enters must succeed on a DC 15 Reflex save or be touched by it, taking 4d10 force damage.

If you attempt to control a sphere that's under the control of another creature, you make a contested arcana versus other creature's arcana check. The winner of the contest gains control of the sphere and can levitate it as normal.

If the sphere contacts a planar portal, such as one created by the portal spell, or an extradimensional space, such as that within a portable hole, the Arbiter randomly determines what happens, using the table below.

\medskip

\begin{tabularx}{0.45\textwidth}{lX}
\textbf{3d6}& \textbf{Result}\\
\hline
3-10 &The orb is destroyed\\
12-16& The sphere moves through the portal or within the extradimensional space.\\
17-18 &A space rift sends every creature and object within 54 meters of the sphere, including the sphere, into a random plane of existence.\\
\end{tabularx}

\medskip

\index[MagicItem]{Universal Solvent}\smallskip* \textbf{Universal Solvent}
300 gp, legendary, this tube contains a white liquid with a strong smell of alcohol. You can use two actions to pour its contents onto a surface within reach. The liquid instantly dissolves 1000cm x cm of adhesive it contacts, including glue supreme.

\smallskip* \textbf{Mirror of Mental Ability}\index[MagicItem]{Mirror of Mental Ability}
15,000 gp, very rare, this object looks like an ordinary mirror five feet high and 0.5 meter wide. On command, the wielder can use it in the following ways:

- I would read a person's thoughts reflected on its surface with telepathy (without needing to understand an unknown language).

- Seeing other places as with a crystal ball, with the possibility of seeing in other planes, as long as they are sufficiently familiar to the observer.

- Create a portal to visit other places. The owner must first visualize the place, then physically enter the mirror, alone or with the companions he wishes. The mirror will create an invisible portal on the other side, through which the wielder, or anyone who can spot it, can pass through.

- Once a week, the mirror can accurately answer a question concerning a person reflected on its surface (an effect similar to the knowledge of legends spell.

\smallskip* \textbf{Mirror of Duplication}\index[MagicItem]{Mirror of Duplication}
legendary, this mirror is a little over a meter high and a little less wide. When a creature reflects off the mirror's surface, its reflected image (an identical duplicate in every way) comes out to attack the original. The duplicate has all the equipment and powers of the original, including magic. The duplicate immediately disappears, along with all its items, upon its or the original's death.

\index[MagicItem]{Mirror Traps Life}\smallskip* \textbf{Mirror Traps Life}
18,000 gp, rare, when this 1.2m tall mirror is viewed indirectly, its surface shows a vague image of the creature. The mirror weighs 25 kilos, has Defence 11, 10 Hit Points, and vulnerability to bludgeoning damage. It shatters and is destroyed when reduced to 0 Hit Points.

If the mirror is hanging from a vertical surface and you are within 1 meter of it, you can use two actions to speak its command word and activate it. It will remain active until you say the command word again.

Any creature other than you that sees its reflection in the activated mirror while within 10 meters of it must succeed at a DC 17 Will save or be trapped, along with everything it wears or carries, in one of the twelve extradimensional mirror cells. This Saving Throw receives +1d6 if the creature knows the nature of the mirror and the constructs automatically succeed on the Saving Throw.

An extradimensional cell is an infinite space filled with a thick mist that reduces visibility to 3 meter. Creatures trapped in mirror cells do not age, and have no need to eat, drink, or sleep. A creature trapped within a cell can escape it using magic that allows planeswalking. Otherwise, the creature is confined to the cell until released.

If the mirror traps a creature but its twelve extradimensional cells are already occupied, the mirror releases one of the trapped creatures at random to house the new prisoner. The released creature appears in an unoccupied space within sight of the mirror but facing away from it. If the mirror is broken, all creatures it contains are released and reappear in an unoccupied space near it.

While within 1 meter of the mirror, you can use two actions to speak the name of one of the creatures trapped within it or summon a particular cell number. The creature named or contained in the named cell appears as an image on the mirror surface. After that, you and the named creature can communicate normally.

In a similar way, you can use two actions to speak a second command word and free one of the creatures trapped in the mirror. The released creature appears, along with all of its properties, in the unoccupied space closest to the mirror and facing away from it.

\smallskip* \textbf{Drums of Panic}\index[MagicItem]{Drums of Panic}
1500 gp, uncommon, these drums are timpani-like (small, easily portable percussion instruments). They are found in pairs and have an inconspicuous appearance. If both are played, all creatures within 215 meters (except those within a 3m circle centered on the drums) are assailed by Fear and flee for 30 rounds at full speed. A DC 21 Will save is allowed to save itself from the effects.

\smallskip* \textbf{Drums of Stun}\index[MagicItem]{Drums of Stun}
rare, these two paired drums resemble panic drums; when both are sounded, all creatures within 3 meter must succeed on a DC 21 Fortitude save and be stunned for 2d4 rounds. All creatures within 21 m are immediately deafened. Spells greater restoration, healing, regeneration, or similar effects can cure deafness.

\index[MagicItem]{Flying Carpet}\smallskip* \textbf{Flying Carpet}
15000 gp, very rare, you can speak the command word of the carpet with two actions to make the carpet float and fly. It moves in the directions spoken to it, as long as you are within 10 meters of it.

There are four sizes of flying carpet. The Arbiter chooses the size of the rug or determines it randomly.

\medskip

\begin{tabular}{llll}
d100 &Size (cm)&Capacity &Speed. by Volo\\
01-20& 90 x 150 &100 kg/25&24 metres\\
21-55& 120 x 180 &200 kg/50&18 metres\\
56-80& 150 x 210 &300 kg/75&12 metres\\
81-100& 180 x 270 & 400 kg/100& 9 meters\\
\end{tabular}

\medskip
The Capacity value indicates both the weight carried and the Encumbrance. The carpet can carry up to double the load listed on the table, but flies at half speed if it carries more.

\smallskip* \textbf{Elemental Thurible of Air}\index[MagicItem]{Elemental Thurible of Air}
1500 gp, rare, this censer can be used to summon and control an air elemental in a manner analogous to the summon elemental spell. You must prepare the magic item and conduct a ritual for one Turn before the actual conjuration, which takes one round. After the elemental has been summoned, you must maintain concentration in order to give it commands.

\smallskip* \textbf{Thurible of Cursed Summoning}\index[MagicItem]{Thurible of Cursed Summoning}
rare, this censer looks like, and appears to function like, an air elemental censer. However, once ignited it is impossible to extinguish it for 1d4 rounds. Each round, one air elemental emerges and attacks all nearby creatures.

\index[MagicItem]{Restorative Balm}\smallskip* \textbf{Restorative Balm}
5,000 gp, uncommon, this glass jar, 5cm in diameter, holds 1d4+1 doses of a thick brew. The jar and its contents weigh 250 grams. In two actions, a dose of ointment can be swallowed or applied to the skin. The receiving creature recovers 2d8 + 2 Hit Points, stops being poisoned, and is cured of any disease.

\index[MagicItem]{Compartment Compartment}\smallskip* \textbf{Compartment Compartment}
10,000 gp, rare, this elegant, silky-soft black fabric folds to the size of a handkerchief and unfolds into a circle 2 meters in diameter. You can use two actions to deploy a Compartment and place it on or against a solid surface, upon which the Compartment creates an extradimensional hole 3 meter deep. The cylindrical space inside the hole is in a different plane, and therefore cannot be used to open passages. Any creature inside an open Compartment can climb out of it.

You can use two actions to close a Laptop Compartment by taking the edges of the fabric and folding it back. Folding the fabric closes the Compartment, and any creatures or objects inside remain in the extradimensional space. No matter what it contains, the Vano weighs nothing.

If the Compartment is collapsed, a creature within the Compartment's dimensional space can use two actions to make a DC 10 Strength check. If the check succeeds, the creature breaks free and reappears within 1 meter of the Compartment. or the creature carrying it. A breathing creature can survive inside a closed portable hole for up to 10 minutes, after which it begins to suffocate.

Placing a Compartment within the extradimensional space created by a bag of holding, utility backpack, or similar item instantly destroys both items and opens a portal to the Astral Plane. Any creature within 3 meter of the portal is sucked into it and deposited in a random location on the Astral Plane. Then the portal disappears.

\index[MagicItem]{Arcane Fan}\smallskip* \textbf{Arcane Fan}
1500 gp, uncommon, While holding this fan, you can use two actions to cast the gust of wind spell with it (save DC 15). Once used, the fan
it should not be used again until the next dawn. Each time it is used before then, there is a cumulative 20\% chance that it will fail and break into useless magic-free shreds.


\index[MagicItem]{Practical Backpack}\smallskip* \textbf{Practical Backpack}
7,000 gp, rare, this pack has a central and two side pouches, each of which is actually an extradimensional space. Each side bag can contain 10 kilos of material, not exceeding a volume of 60 dm3

The large central bag can contain up to 240 dm3 or 40 kilos of material. The backpack always weighs 2.5 kilos, whatever its contents.

Placing an object inside the backpack follows the normal rules for interacting with objects. Retrieving an item from the backpack requires the use of two actions. When you search for an item in your backpack, it will magically always be on top of the pile of items it contains.

The backpack has some limitations. If overloaded, or a sharp object cuts or tears it, the pack will crack and be destroyed. If the backpack is destroyed, its contents are lost forever, although an artifact will always reappear somewhere in the multiverse. If the backpack is turned inside out, its contents are expelled without harming it, and the backpack must be turned over before it can be used again. If a breathing creature is placed inside the rucksack, it can survive there for up to 10 minutes before it begins to suffocate.

Placing the backpack inside the extradimensional space created by a holding bag, portable hole, or similar item immediately destroys both items and opens a portal to the Astral Plane. The portal originates from where objects have been placed inside each other. Any creature within 3 meter of the portal is sucked through it and dragged to a random location on the Astral Plane. Then the portal closes. The portal is one-way and cannot be reopened.

\smallskip* \textbf{Titan Hoe}\index[MagicItem]{Titan Hoe}
2000 gp, uncommon, this oversized tool is 10' long and weighs 120kg (30 Encumbrance), and can only be used by a giant (or enlarged character) to move large amounts of dirt and build earthworks (a 3-cube m per shift). The hoe can also be used to split stone with great speed. When used as a weapon, it has a +3 bonus on hitting and inflicts 5d6 wounds.

\index[MagicItem]{Hoofs of Speed}\smallskip* \textbf{Hooves of Speed}
5,000 gp, rare, these iron hooves come in sets of four. When all four hooves are attached to a horse or similar creature, they increase that creature's walking speed by 10 meters.

\index[MagicItem]{Zephyr Hooves}\smallskip* \textbf{Zephyr Hooves}
1500 gp, very rare, these iron hooves come in sets of four. When all four hooves are attached to a horse or similar creature, they allow that creature to move normally, while floating about 25 cm above the ground. This effect means that the creature can walk or pass over non-solid or unstable surfaces, such as water or lava. The creature leaves no tracks and ignores hindering terrain. Additionally, the creature can move at its normal speed for up to 12 hours per day without suffering fatigue from forced marching.

\end{multicols}

\pagebreak

\section{Cursed Items}\index{Cursed Items}

\begin{changemargin}{0cm}{0.5cm}\begin{emphasis}{When a wicked curses his opponent, he curses himself. (Sirach)


\medskip

If you curse one person there will be two graves. (Japanese proverb)}
\end{emphasis}\end{changemargin}\medskip

\begin{multicols}{2}

\label{oggetti-maledetti}

\lettrine[lines=2, lhang=0.33, loversize=0.25, findent=1.5em]{C}{ursed} items are magical items that have a potentially negative influence on the character.

Cursed items are almost never made intentionally, but rather are the result of botched work, inexperienced craftsmen, or lack of suitable components, or broken pacts with some Patron.

The Arbiter can ask for an Arcana check with a DC equal to 10+ days used to build the magic item in case of particularly complex objects or there have been problematic situations in the creation and when the check fails by 10 or more or there has been a critical failure (two 1s, two 2s, and one 1) roll on the table to determine the type of curse the item has.

A curse can manifest as a result of extreme negative or emotional influences affecting an object.

\medskip

\textbf{Common Item Curses}

\medskip

\begin{tabular}{ll}
\textbf{\%} & \textbf{Curse}\\
\toprule
01-15 & Deception\\
16-40 & Effect or Opposing Target\\
41-50 & Discontinuous Operation\\
51-65 & Requirement\\
66-90 & Inconvenience\\
91-100 & Completely different effect\\
\end{tabular}

\medskip

Cursed items are \hypertarget{oggettimaledettiid}{identificati} like any other magic item with one exception: unless the Arcana check to identify the item exceeds 35 or the Identify spell is cast with a Magic Test and rolls a magic critical (2 times 6) the curse is not detected. If the check is under 35 or without a magic crit all that is revealed is the original purpose of the magic item.

If the object is known to be cursed, the nature of the curse can be determined by using DC \hyperlink{identificareom}{standard} to identify the object.

\begin{center}
\includegraphics[width=0.75\linewidth]{immagini/vasobasano.png}

\textit{Vase of Basano. This vase was made in the second half of the 15th century and is made of silver. DC 35}
\end{center}


\begin{changemargin}{0.3cm}{0.3cm}\begin{narrator}
A curse is always a particular \textit{inconvenience}, which is not used at random. Think carefully about the cursed objects that you will let the characters find because they will ask you for a lot of information and you will have to be ready.

There is no need for the curse to be excessive and limiting it can very well be ridiculous or particular, make sure it is characterizing. The character should not feel (except if you want it) condemned forever, take advantage of the opportunity to build new adventures and team spirit.
\end{narrator}\end{changemargin}


\subsection{Remove Cursed Objects}\index{Remove Cursed Objects}

If the item has been cursed with the bestow curse spell, or otherwise the Storyteller decides that the item has a particular curse, then the DC of the caster Remove Curse (see \hyperlink{magietirosalvezza}{Saving Throw Resist the spell}, page \pageref{magietirosalvezza}) must be compared with the DC of the item's curse. Only if if the caster of the Remove Curse spell has a DC higher than that of the object then it is possible to remove the object.

A spellcaster can cast remove curse with a Magic Test and for each critical success adds +4 to his DC to compare with that of the item.

If the count is higher the item can be removed in the next round, but the curse remains and strikes again if the item is used/worn again.

Each cursed object has its own method of being destroyed, from being thrown into an active volcano, to being struck by the Thunder god's hammer (or Patron...) or devoured by a colossal sandworm if not struck by the breath of a red dragon and a white dragon at the same time...

For many cursed objects, casting the Remove Curse spell is sufficient, without specifying the DC to beat.

\subsection{Common Effects of Cursed Items}

The most common effects of cursed items are as follows, the Arbiter can invent new special effects for specific cursed items.

\subsubsection{Deception}

The user of the object continues to believe that it is what it seems at first sight, but in reality it has no power other than to deceive. The user is mentally tricked into believing that it works, and cannot be convinced otherwise except with the use of Remove Curse

\begin{center}
\includegraphics[width=0.70\linewidth]{immagini/mirror.png}

\textit{The mirror in The Myrtles Plantation. DC 28}
\end{center}

\subsubsection{Effect or Opposing Target}

These cursed objects tend to have malfunctions which in some cases generate effects diametrically opposite to those desired by their creator, while in other cases they tend to affect the user instead of someone else.

But the most interesting thing is that these objects may not even be a disadvantage for their owner. The category of magical items with opposite effects also includes weapons that inflict penalties on attack and damage rolls, rather than bonuses.

Since a character shouldn't know immediately what a magic item's bonus is, he shouldn't even know the nature of his curse. Once he learns of it, the Remove Curse Spell will be needed to free himself from the object.

Some particularly strong curses, at the Arbiter's discretion, can be removed by Remove Curses cast by a very experienced spellcaster (control Magic Proficiency value).

\subsection{Discontinuous Operation}

Discontinuous objects work exactly as they should, when they do. Determine if the object is Unreliable, Conditional or Uncontrollable.

\medskip
\subsubsection{Unreliable}

Each time the item is activated, there is a 5\% chance that it will not work.

\subsubsection{Conditional}

This item only works in certain situations. To determine what these are, choose a trigger condition or consult the table below.

\subsubsection{Uncontrollable}

An uncontrollable object tends to activate randomly. Roll a d\% each day. On a roll of 01--05 the item spontaneously activates at a certain time of day.

\medskip

\begin{tabularx}{0.45\textwidth}{lX}
\textbf{\%} & \textbf{Situation}\\
\toprule
01-03 & Temperature below zero\\
04-05 & Temperature above zero\\
06-10 & During the day\\
11-15 & During the night\\
16-20 & Exposed to sunlight\\
21-25 & In the absence of sunlight\\
26-34 & Underwater\\
35-37 & Out of Water\\
38-45 & Underground\\
46-55 & Above ground\\
56-60 & Within 3 meter of creature type\\
61-64 & Within 3 meter of a race or creature type\\
65-72 & Within 3 meter of a spellcaster\\
73-80 & Within 3 meters of a Follower or Devotee of a specific Patron\\
81-85 & In the hands of a non-spellcaster character\\
86-90 & In the hands of a spellcaster character\\
91-95 & In the hands of a creature with particular Trait\\
96 & In the hands of a creature of a particular kind\\
97-99 & On non-sacred days or during particular astronomical anniversaries\\
100 & More than 150km from a given location\\
\end{tabularx}

\subsection{Requirement}

Some items have much more difficult requirements to meet for them to work. For this item to work, one of the following conditions may need to be met:

\begin{itemize}[leftmargin=*]
\item The character must eat twice as much as normal.
\item Character must sleep twice as much as normal.
\item The character must complete at least one specific mission.
\item The character must sacrifice (destroy) 100 gp worth of precious objects or materials per day.
\item The character must swear allegiance to a particular noble or his family.
\item Character must drop all other magic items.
\item Character must be a Follower or Devotee of a specific Patron
\item The character must have a minimum number of ranks in a particular skill.
\item The character must sacrifice part of his life energy (1 point of permanent Constitution) the first time he uses the item.
\item The item must be purified with the Holy Water of a specific Patron every day.
\item The object must be bathed in at least half a liter of blood (animal or humanoid) per day.
\item Item must be used to kill one living creature per day.
\item The item must be used at least once per day, or it stops working for its current owner.
\item Item must draw blood when wielded (weapons only). It cannot be set aside or exchanged for another item until it has scored a hit.
\end{itemize}

\medskip

\begin{center}
\includegraphics[width=0.8\linewidth]{immagini/donnalemb.png}

\textit{Woman of Lemb or Goddess of Death Statue, 3500 BC , DC 40}
\end{center}

\medskip

The requirements depend on the convenience of the object and should never be determined at random. An intelligent object with a requirement often imposes its requirement thanks to its personality.

If the requirement isn't met, the object stops working. If met, however, the item typically functions for a full day before having to meet the requirement again (although some requirements need to be met once, some once a month, and still others continuously).

\subsection{Inconvenience}

Items that have drawbacks usually have positive effects on the user, but they also have negative aspects. While sometimes the drawbacks only become apparent when the items are being used (or held, in the case of items such as weapons), they usually remain present until the character gets rid of the item in question.

Unless otherwise noted, the drawbacks remain in effect for as long as the item remains in the character's possession. The DC of the Saving Throw to avoid these effects is equal to 10 + DC of the curse (if you have not established the difficulty, set the Saving Throw, usually on Will, to DC 25)

\end{multicols}

\medskip

\begin{changemargin}{0.3cm}{0.3cm}\begin{narrator}The list is an example to be able to randomly generate effects on the owner of the object. Get inspired and be creative! However, don't let a curse make it impossible to play the character, rather it must be experienced as an opportunity to try, do, something different. Never throw a random cursed object into the pile of treasures, always think about what will happen and what consequences will be generated. A cursed item always requires a high level of attention and planning from the Arbiter\end{narrator}\end{changemargin}

\bigskip

\textbf{Table: Effects of cursed magic items}\index{Table Effects of cursed magic items}

\medskip

{\small
\begin{tabularx}{0.95\textwidth}{lX}
\textbf{\%} & \textbf{Inconvenience}\\
\toprule
01-02& Character's hair grows 2.5 cm per hour.\\
02-04& Character's nails grow 1 cm every 8 hours\\
05-06 & Character's height decreases by 5d10 cm \\
07-09 & Character height increases by 5d10 cm \\
10-11 & The temperature around the object is 5C cooler than normal.\\
12-13 & The temperature around the object is 20C cooler than normal.\\
14-15 & The temperature around the object is 5C hotter than normal.\\
16-17 & The temperature around the object is 20C hotter than normal.\\
18-20 & Character's hair color changes.\\
21-23 & Character's skin color changes.\\
24& Character's hair color changes every hour\\
25& Character's skin color changes every hour\\
26 & Horns like a ram grow on the character's head\\
27 & Antlers like a moose grow on the character's head\\
28-29 & The character now bears a distinctive sign (a tattoo, a strange glow, etc.).\\
30-32 & Character's gender changes every day at dawn.\\
33-34 & Character's race or species changes.\\
35 & The PC is afflicted with a randomly determined Disease, which cannot be cured.\\
36-39 & The object constantly emits unpleasant sounds (moans, curses, insults...).\\
40 & The object has a ridiculous appearance (bright color, shape, glows with a pink halo, etc.).\\
41 & A blue unicorn, visible only with magic, of small size always flies around the Character giving useless advice and making stupid jokes.\\
42& Every day you have a sudden desire and ability to crochet for at least 1 hour.\\
43-45 & The character becomes extremely possessive towards the object.\\
46-49 & The character has an uncontrollable fear of losing or damaging the item.\\
50 & A Stroke is replaced\\
51& The character's metabolism changes and becomes exclusively carnivorous\\
52& The character's metabolism changes and becomes exclusively vegetarian\\
53-54 & The character must attack the creature closest to him (5\% chance each day).\\
55-57 & The character is stunned for 1d4 rounds each time the item has served its purpose \\
58-60 & The character becomes deaf\\
61-64 & Maximum Hit Points drop by 10 permanently (remaining at a minimum of 1).\\
65 & Maximum Hit Points are permanently decreased by 20 (remaining at a minimum of 1).\\
66-68 & The PC acquires a random Phobia.\\
69-71 & Saving Throw Will each day at dawn with mod. Intelligence or takes 1 permanent Intelligence damage.\\
72-74 & Saving Throw Will each day at dawn or takes 1 permanent Wisdom damage.\\
75-77 & Will save every day at dawn with mod. Charisma or takes 1 permanent Charisma damage.\\
78-80 & Saving Throw on Fortitude every day at dawn with mod. Strength or takes 1 permanent Strength damage.\\
81-83 & Saving Throw on Fortitude every day at dawn with mod. Dexterity or takes 1 permanent Dexterity damage.\\
84-85 & Saving Throw Fortitude save each day at dawn or takes 1 permanent Constitution damage.\\
86-89& The PG begins to talk about himself in the third person.\\
90-92& Horses, dogs and cats become hostile.\\
93& A Patron will do anything to kill you.\\
94 & The PC is teleported 10d100 km away every day at dawn.\\
95 & The character is transformed into a random creature of a specific species (5\% chance each day).\\
96 & The character is transformed into a specific creature (5\% chance each day).\\
97 & Character can no longer use magic items or Spells above level 5\\
98 & Character can no longer use magic items or Spells above level 3\\
99 & Character can no longer use Spells\\
100 & Roll Twice\\
\end{tabularx}}


\pagebreak

\section{Yeru}\index{Yeru}\index{Atilantis}

\begin{changemargin}{0cm}{0.5cm}\begin{emphasis}{
So the Earth is really round. But I had no idea it was blue. Why do men who live on such a beautiful planet do nothing but fight each other? (Nadia - The mystery of the blue stone)


\medskip

The planet does not belong to us, we belong to it. We are passing through, he stays. (Pierre Rabhi)}\end{emphasis}\end{changemargin}\medskip

\begin{multicols}{2}

\label{yeru}

\lettrine[lines=2, lhang=0.33,loversize=0.25, findent=1.5em]{Y}{eru} is the reference planet of OBSS. A planet split both physically and magically.

Around Yeru revolve two stars Sparka and Andhakara.\index{Sparka}\index{Andhakara}

Sparka is of a warm golden color and she is the one who brings heat and light, around her Yeru makes a complete tour in 336 days of 24 hours each.

Sparka only ever illuminates the northern hemisphere of Yeru, called Curyan \index{Curyan}.

Andhakara always illuminates only the southern hemisphere of Yeru, Tiya, and is instead a blue and cold star, devoid of life, it is the one that brings energy storms and strange natural events. She brings a cold twilight.

If 14 (06-20) daylight hours see Sparka and Andhakara protagonists in their dance in the sky; the 10 nocturnal hours see the two moons of Yeru named Idam and Kenatu as total protagonists.

The inhabitants of Yeru call them their moons even if in reality they are not really just moons but real inhabited planets.

The two moons are large and imposing on the night sky, Idam of a reddish gray color and Kenata of a warm pearly gray command the tides and influence navigation with their presence.

Yeru has a peculiar and unique distribution of lands, fruit of the whim of the Gods of Genesis (Ljust and Calicante), you can imagine it as a mirror system on the equator.

The lands do not join at the equator, leaving about 200 km of open sea.

The emerged lands that make up the northern hemisphere and the southern hemisphere are almost symmetrical and symbiotic with each other. The shape and subdivision of the lands are very similar to each other. But from a climatic point of view there are profound differences.

The border open sea area is wild and inscrutable. The deepest and most powerful storms continuously discharge their energies and even magic cannot penetrate. In the eye of this perennial and gigantic maelstrom is the civilized and very powerful Alantia, considered by many to be a legendary island and the cradle of civilization.

Many areas of Yeru are still unmapped and unexplored, the primordial chaos rules these areas and everything becomes possible.

There are few cities that exceed 50,000 inhabitants. Every state has a capital which, due to the mocking fate of Yeru, is very often destroyed or disappears. The law is often absent and only that of the strongest is in force.


Wide moors unfold where ancient remains of vanished civilizations are a refuge for new inhabitants. Layer upon layer of historic civilizations beneath your feet with treasures, secrets, caves and protectors.

Curyan is governed by the force of life, this region experiences a sort of perennial hot season with temperature gradations and atmospheric phenomena that vary according to latitude.

Zones with a torrid and humid climate intersect with others with a dry heat and no rainfall; there are phenomena such as dust storms in the desert areas and strong and devastating tropical storms in the lush central bays.
There are pleasantly warm territories and others refreshed by fresh breezes from the northern glaciers.

Tiya, on the other hand, is a semi-withered hemisphere, the light that arrives is just enough to allow agriculture and farmed animals look pale and emaciated.

The richest area is the one closest to the equator where the cold light of Andhakara fades only slightly, making room for some Sparka rays.
In this narrow band agriculture is more flourishing and there are fewer devastating meteorological phenomena.

It is the hemisphere where the law of the strongest is in force, where people fight for a living and there are few states that have an effective protection system.

The sea that embraces the equatorial is strong and tumultuous, very few boats venture from one continent to another, this means that exchanges between Tiya and Curyan are almost nil by sea, only very few captains, and secretly, dare to cross the maelstrom.

\subsection{Adventures in Yeru}\index{Adventures in Yeru}\label{avventureinyeru}

The "problem" for adventurers and explorers is the extreme diversification and mutability not only of the environment but also of the cultures and civilizations that can be encountered.
Since every thousand years something very important changes it is possible that entire islands disappear or appear with once forgotten civilizations, entire cities are suddenly submerged by water or vegetation, or even worse entire empires become warring undead (yes, it happened this too, along with a couple of zombie apocalypses lasting several centuries).

You're never sure what you can find in Yeru!.
Moreover, the worst events are those that affect the rich and vital Curyan while more positive events take Tiya.

Yeru can never be said to be explored, the same area can change from one day to the next because a Patron has decided so. Curious, ineffable, fickle they are capable of building the adventure of life in a clap of hands just to enjoy the show.

Whether you are from Curyan or Tiya your life will not be easy, nothing will be given to you. Young people from both continents are fleeing poverty, abuse and violence to embark on a new life even richer in poverty, abuse and violence but which at least is only theirs, the result of their own choices.

When you decided to embark on this new career, or you were torn from the previous one, you knew it would not be easy, that Yeru himself, through his Patrons, would do everything to defeat and humiliate you, but the Law of Reward is superior even to Patrons and you would have had your prize even if what remained unknown.

\subsection{Companies}\index{Companies in Yeru}\label{societainyeru}

Kings are rarely destined to reign for more than a few generations, fratricidal wars, attacks from outside, the wishes of Patrons, make the most rigid corporate forms struggle to prosper and the \emph{democratic} spirit is not always so developed to allow the creation of advanced societies that can adapt to the situation.

Nations thus have very vague borders, often defined by geography rather than by conquests. Armies cannot always defend them from external attacks and even more often the militias must concentrate on defending the main city from internal attacks, rebellions or sudden hordes of monsters emerging from who knows what footprint of Cattalm.

Throughout Yeru the most widespread form of government and society are the City States, strongholds and lands gathered around a city capable of defending them and protecting them from external assaults. Usually commanded by a strong leader with the support of a Patron.

Small and large villages arise everywhere in the territory, around water sources and natural resources, but they are often at the mercy of gangs of desperate people if not gablins.
It is often here that our heroes have their first training in the attempt to defend first their home, then the village, from the assault of some cunning and bloodthirsty enemy.

The remains of magnificent cities of the past still stand and these often become populated again even if the curse that condemned the ancient kingdom often hovers over the inhabitants.

The underground may be caves, catacombs or infinite tunnels if not true underground cities, they are throughout Yeru, an imperishable, stratified and re-stratified memory of its history. There is never an end to how deep you can go, there is always something else underneath that is even more magnificent and dangerous.

\medskip

\begin{changemargin}{0.3cm}{0.3cm}\begin{narrator}
Use the setting of your choice! Yeru is an example of a chaotic and slightly anarchic world dominated by the ever-changing moods of capricious deities.
I personally prefer a less high fantasy setting, but anything from the Forgotten Realms to Golarion to Mystara will do. You are the Arbiter, you are the world, you are the one who projects light and dark!.

The first suggestion I give you is to know \textbf{well} the setting, the greater your knowledge, the more easily you will know how to adapt to the situation that will happen to you.

\end{narrator}\end{changemargin}



\subsection{Notable places of Yeru}\label{luoghidiyeru}

\subsubsection{Kranguran Desert}

In this immense desert gigantic monsters are hidden. Some hidden under the sand like huge dinosaurs use telluric sense to hunt their prey.

Every creature in this desert is gigantic, monstrous and disproportionate in appearance, as if born from someone's nightmare.

The vegetation itself in the few oases present is enormous and hypertrophic.

\subsubsection{City of Knandir}

This rich, prosperous and populous ancient city was destroyed overnight by a gigantic cataclysm.

It is said that Cattalm's will was so pervasive that all the buildings were destroyed or severely damaged. Not satisfied with the work, I condemn the city to be out of phase with reality, making it disappear in the eyes of all the others.

The few surviving inhabitants perished in atrocious suffering, condemned to not being able to go out, not having anything to eat or drink.

The city was cursed and in the few days of the year in which it is possible to reach it, every person who sets foot there to loot the immense treasures contained seems condemned to never go out again, victim of the curse or of the numerous ghosts, spirits and undead of the previous inhabitants .

Furthermore, the city never appears in the same place but moves following a not well understood pattern. The Lost Scroll of Knandir is said to explain their whereabouts.

\subsubsection{The Silent Sea}

There is a particular area of sea, between three major islands and containing several smaller islands, where all sound is silenced. A sound that is generated in those waters, and not on dry land, is silenced. Two distinct floating cities dedicated to ancient psionic traditions have arisen.

\subsubsection{Blue Gorilla Tower}

The origin of this ancient and magical building is now forgotten, it is said that it was created to challenge a Patron, probably Gradh. The tower, with a square base of 20 meters on each side, is apparently 7 floors high. In each floor, whose map seems to be constantly changing, blue gorillas appear, absolutely brutal and with the intention of killing anyone in the tower. Once the last gorilla on the floor is defeated, the door leading to the stairs to the next floor opens and the characters can go up. With each floor the gorillas get stronger, tougher and more intelligent. It is known that already on the 4th floor they also acquire magical powers. Entered characters can exit whenever they want, if they should die inside the tower they will automatically be teleported out, but alive at 1 Hit Point and extremely fatigued, without the most precious object they were wearing at the moment of death. There are no magic items inside the tower, at least in the known floors, the only thing the characters gain is experience for the fights done. The current record is reaching the 7th floor. Will new heroes make it to the end (???) of the tower, and what rewards will there be for those who survive?


\end{multicols}


\subsection{The Portals}\index{Portals}

\begin{changemargin}{0cm}{0.5cm}\begin{emphasis}{
Never open doors to those who open them even without your permission. (Stanislaw Jerzy Lec)}\end{emphasis}\end{changemargin}\medskip


\begin{multicols}{2}

\label{i-portali}

In a world where sea transfers don't work except between islands of the same hemisphere as well as the Teleport spell, the ability to use portals to transfer goods and people has gained significant traction.

This proliferation of small, large, lasting or instantaneous tunnels has caused a rift in the dimensional fabric of Yeru generating in turn a proliferation of more or less large and lasting spontaneous tunnels.

And these Gates are the cause of so much trouble for both Tiya and Curyan as they not only bind the two hemispheres but connect all of Yeru to other worlds (or so they think since few have returned to report it..).

There are known and stable portals, so far, linking Tiya to Curyan, almost all under the control, not to mention inside the castle, of royalty or powerful.

There are areas where portals open more frequently but the destination is not always certain.

Then there are the dragon portals. Dragons are not native to Yeru but have been drawn to these magical doors, causing havoc and terror to Tiya and Curyan.

The dragons understood Yeru's nature well, and with their keen intelligence and innate ability to shape magic, they built their portals by summoning hundreds of dragons. All wicked.

Yes, there are no "good" dragons on Yeru except with a few exceptions.

People have always tried to destroy dragon portals, with sacrifice and blood. Many have been destroyed, others have been generated. It is an endless war, the only one that can unite the peoples and destinies of the two hemispheres.

\end{multicols}



\vfill

\begin{center}
	\includegraphics[width=0.4\linewidth]{immagini/ancientwell.png}\\
	
	\medskip
	
	\textit{Ancient well and Portal}
\end{center}

\pagebreak

\subsection{The Calendar}\index{Calendar}

\begin{changemargin}{0cm}{0.5cm}\begin{emphasis}{
I have often ended up on a calendar. But never for a specific date. (Marilyn Monroe)\\

It all began at the thirteenth hour of the thirteenth day of the thirteenth month... We were there to discuss the printing errors on the calendars purchased by the school. (The Simpsons)}
 \end{emphasis}\end{changemargin}\medskip


\begin{multicols}{2}

\label{il-calendario}

Based on the Kenatu cycle it has 12 months of 28 days.

These are the names of the months starting from what is defined as the beginning of the year\\

1) Ianas

2) Prineva

3rd) Marc

4th) Epral

5th) Meea

6) Vernam

7th) Ilai

8th) Arkast

9th) Cester

10th) Koper

11th) Narava

0th) Raanant* (special)

12th) Kartan\\

\emph{Raanant} is the month celebrated at the end of the Millennial Cycle, every thousand years. It is a month of freedom from the Patrons, from the Laws, it is the month of catharsis and violence, of freedom and rebirth.\\


The week is itself divided into 7 named days

1) Kalint (or Sparka day, usually a holiday)

2) Iratam

3) Munrat

4th) Arai

5th) Venran

6th) Kittam

7th) Viltar

The day is divided into 24 hours

\subsection{Beyond Death}

Yeruites have a rather pessimistic view of what happens after death. For most after death there is nothing but the dissolution of the body.

Devotees and Followers believe that their spirit will reunite with the Patron, making him stronger.

Still others still believe that each spirit incarnates 4 times before being judged by the Patrons of Genesis and sent to its assigned plane.

What the truth is unknown.

\end{multicols}

\subsection{The Millennial Cycles}\index{The Millennial Cycles}

\begin{changemargin}{0cm}{0.5cm}\begin{emphasis}{
Then I saw an angel descending from heaven with the key of the Abyss and a great chain in his hand.

He seized the dragon, the ancient serpent - that is, the devil, satan - and chained him for a thousand years;

he threw him into the abyss, shut him up and sealed the door over him, so that he would no longer deceive the nations until the thousand years were completed. After these he will have to be dissolved for some time. (Revelation 20,1-3, Apostle John)
}\end{emphasis}\end{changemargin}\medskip

\begin{multicols}{2}

The myth says that every thousand years the Yeru dies to be reborn again, more beautiful than before.

It's not quite like that but it's very close.

It is known to a few scholars of Atmos that every thousand years the recognized Patrons and from which many derive their powers disappear and leave their place, after exactly 1 year, to new Patrons.

Suddenly spells stop working, only magic items that can absorb and retain magic work (such as a Potion, Armour or Weapon if not a Ring or Staff that has charges, but not items that automatically recharge such as the Rods), even the Devotees or Followers no longer have access to any Spell Lists.

With some exceptions. The Patrons of Genesis, Atmos and Lynx and the Conqueror are the only ones to remain constant in not mutating and only their Devotees and Followers can continue to use the known Magic Lists.

From the sixth month old Followers and Devotees begin to hear voices, to dream of new faces and new Patrons.

Each new Patron, based on the Traits he commands, approaches a believer and tries to convince him to accept him as a new Patron.

This Follower/Devotee must have at least two Traits in common with the new Patron to be his Follower and at least 3 to remain Devoted to him (as always).

Enchanters only at the end of the year will be able to use spells, regardless of whether they follow a Patron or not.

It is an extremely turbulent and agitated period where wars and vendettas break out taking advantage of the absence of magic. For many it is a period of pure hatred and violence where the lowest instincts are vented, knowing then that one will not be judged by any deity.

The truth is that every thousand years the Patrons of Genesis judge their creatures, the Patrons, evaluating who did better and who worse. It is a challenge between Calicante and Ljust to who has, through the Patrons, obtained more Followers and Devotees.

The Patron who most of all has shown himself capable of maintaining and conquering people will also remain in the following millennium, this will be the Winner and the believers of him will sing his glory and power for another thousand years.

Inebriated by the victory, the Patron of Genesis will express a wish that the other will have to try to respect as much as possible. Of course he / she herself could manifest it but the satisfaction of making the other do something that he hates is superior to everything.

And that's why something happens every thousand years, in addition to the birth of new Patrons. It can be a continent, sea.. moon, animals... something impressive changes for all yeruiti. It is a time of global upheaval.

Only the supreme Devotees of Atmos know this truth as they know that the Patrons of Genesis after the victory lie together for six months generating the new Patrons.

\bigskip

\begin{changemargin}{0.3cm}{0.3cm}\begin{narrator}
Evaluate well when to start your campaigns, based on the duration and the year you could run into these events.
Exploit the change of Patrons in your favor and benefit from the adventure, let play some of the "rest" from magic, help players with more magical characters to recover.
\end{narrator}\end{changemargin}


\subsection{The Ancient Gods and the 7 Cycles}\index{The Ancient Gods and the 7 Cycles}\index{Old Gods}\index{7 Cycles}

Legend has it, practically unknown even to the most erudite, that 7 cycles ago the Patrons weren't there. They weren't always there to bribe you, drag you to perdition, or unleash hordes of enemies on you.

Once, the first time, the Patrons of Genesis entrusted Atmos with the \textbf{Book of the Gods}\index{Book of the Gods}, a tome of extraordinary power, capable of creating divinities.

Atmos wrote their Names, capabilities, ethics and purpose inside. Thus were born 15 deities, some good, others indifferent, few purely evil, with the aim of educating, instructing, improving the creatures of Yeru.

What exactly happened then is not known, some sages assume that even these gods failed in the noble mission of raising Yeru from barbarism and incivility, others say that their contribution was not direct enough, in any case they were banished and replaced with the Patrons.

Certainly more direct, ruthless and interested in life on Yeru. They weren't meant to improve life perhaps, but they undoubtedly gave much more enjoyment to the Patrons of Genesis.

Legends tell that the Book of the Gods was then destroyed, dispersed and only one page of that magnificent book was saved and whoever finds it and writes her name will become a deity, or perhaps a Patron slave of his impulses?


\end{multicols}


\vfill

\begin{center}
\includegraphics[width=0.4\linewidth]{immagini/Aztec_calendar.png}\\

\medskip

\textit{Ancient Aztec calendar}
\end{center}



\pagebreak

\section{The Planes}\index{The Planes}\label{ipiani}

\begin{multicols}{2}

\lettrine[lines=2, lhang=0.33, loversize=0.25, findent=1.5em]{A}{lthough} if endless adventures await you in Yeru, there are other worlds beyond this one, other continents, other planets, other galaxies . However even beyond the existence of innumerable planets there are other worlds, completely different dimensions from reality, known as planes of existence. Traveling between Planes is complex and each has its own rules.

Although the number of planes is limited only by the imagination, they all fall into five general types: the Material Plane, the Transitional Planes, the Energy Planes, the Outer Planes, and the innumerable demi-planes.

\begin{changemargin}{0.3cm}{0.3cm}\begin{narrator} %box narratore
Consistent with the setting of OBSS the Planes should not be reachable. Lynx doesn't allow it. Yeru was born as a closed and isolated planet although this has not stopped the Dragons and other fiends from arriving. The Arbiter decides Yeru's level of isolation.
\end{narrator}\end{changemargin}

\subsection{What is a Plan?}

You can imagine the Planes as gigantic spheres of indeterminate diameter that float as if they were planets in a cosmic \textit{vacuum} which is the Astral Plane.

For practicality, and only to help the limited human mind, we imagine Yeru (which corresponds to the Material Plane) as in the centre, arranged in a star around there are the Energetic Planes (the Elemental Planes and those of Energy), more distant External Planes. Among the various Plans there are those of Transition.

\textit{Material Plane}\index{Material Plane}: The Material Plane tends to be most similar to Yeru and function using the same rules of nature. Its "size" depends on the campaign: it may conform only to the actual game world, or encompass an entire universe of planets, moons, stars and galaxies. The Material Plane is the basic plane for the game.

\textit{Transition Planes}\index{Transition Planes}: These planes have one important element in common: they all coexist with the other planes and serve to travel between overlapping realities. These planes are strongly interconnected with the Material Plane, and can be accessed using numerous spells. They also have native inhabitants. Some Transition Plans are described below.

\begin{itemize}[leftmargin=*]
\item
\textit{Astral Plane}\index{Astral Plane}: It is the void between the planes, a silver Plane that connects the Material Plane to the Energetic Planes and to the Outer Planes, the Astral Plane is the means through which the souls of the deceased reach the beyond. A traveler in the Astral Plane sees the plane as an infinite void periodically punctuated by tiny slivers of physical reality detached from the innumerable superimposed planes. Powerful spellcasters use the Astral Plane for a brief fraction of a second when teleporting, or can use it to planeswalk with spells such as astral projection.

\item
\textit{Ethereal Plane}\index{Ethereal Plane}: The Ethereal Plane is a nebulous, hidden dimension superimposed on the Material Plane and the Plane of Shadow. Travelers who traverse the Ethereal Plane experience the real world as insubstantial and can move among solid objects without being seen in the real world. Bizarre creatures inhabit the Ethereal Plane, as do ghosts and dreams, many of which can sometimes extend their influence into the real world in mysterious and terrifying ways. Powerful spellcasters use the Ethereal Plane with spells such as Ethereal Form, Blink.

\item
\textit{Plane of Shadow}\index{Plane of Shadow}: The mysterious and deadly Plane of Shadow is a gray and colorless version of the Material Plane. It overlaps the Material Plane and is in many ways a twisted and perverse "reflection" of the Material Plane, infused with negative energy (see Energy Planes) and inhabited by terrible monsters such as shadows or even worse creatures. Powerful spellcasters use the Plane of Shadow to rapidly travel immense distances on the Material Plane.
\end{itemize}

\medskip

\textit{Energy Plans}\index{Energy Plans}: These plans are the embodiments of the basic elements that build the universe. They are composed of a single type of energy or element. The inhabitants of a specific Inner Plane themselves are composed of the element of the plan. Among the Energy Plans there are:

\medskip

\begin{itemize}[leftmargin=*]
\item
Elemental Planes\index{Elemental Planes}: The four classic Inner Planes are the Plane of Water, Plane of Air, Plane of Fire, and Plane of Earth. From these planes come creatures known as elementals, but they are also inhabited by other bizarre creatures, such as genies, xorns, mephits, and unseen stalkers.

\item
Planes of Energy\index{Planes of Energy}: There are two planes of energy, The Positive Energy Plane (where life sparks come from) and the Negative Energy Plane (where undead corruption comes from). The energy of both planes is infused into reality, and the flow of this energy flows through all creatures from birth to death. Devotees use the power of these two planes when Channeling Positive or Negative Energy. As easy as it is to think that Ljust is of the Positive Energy Plane and Calicante of the Negative Energy Plane, so it is not both being a single source of energy that transcends the Planes.

\end{itemize}

%\medskip
%\end{multicols}
%\pagebreak

%\vfill
%\begin{center}
%\includegraphics[width=0.96\linewidth]{immagini/mappaplanare3.png}\\
%\medskip
%\textit{Mappa Planare. Concessa in licenza dall'autore.\\  https://www.reddit.com/r/ImaginaryGolarion/comments/97rog0/pathfinder\_map\_to\_the\_planes}
%\end{center}
%\pagebreak

%\begin{multicols}{2}

\textit{Outer Planes}\index{Outer Planes}: Vast beyond imagination it is to them that the souls of the dead come and it is here that the Patrons dwell. Each of them has its own set of Traits, representing a particular moral or ethical aspect, and their inhabitants tend to behave according to these Traits.

The Outer Planes are also the final resting place of spirits from the Material Plane, whether they are destined for a peaceful continuation or eternal damnation. The denizens of the Outer Planes form civilizational mythologies, including angels and demons, titans and devils, and countless other incarnations of the possible. Each game world should have different Outer Planes that suit specific needs and themes, but the classic Outer Planes include Heaven (lawful and good traits), Abyss (anarchist and evil traits), Hell (lawful and wicked) and Elisha (freedom and goodness). Powerful spellcasters can contact the Outer Planes for guidance and counsel with spells such as Communion and Contact Other Planes, or they can summon allies with the spells of Summons.

\textit{Demiplanes}\index{Demiplanes}: This category collects all other extradimensional spaces that function like planes but have measurable and limited access and dimensions. The other types of planes are in theory infinite in size, but a demiplane could be as long as only a few hundred meters. There are countless demiplanes drifting in the astral plane, and while many are connected to the Astral Plane and the Ethereal Plane, others are cut off entirely from the Transitional Planes and can only be reached through well-hidden portals or dark magic. A demiplane usually combines the aspects and characteristics of multiple planes.

\subsection{Journey to the Planes}
Two floors that are separate from each other do not overlap and do not directly connect to each other. They are like planets in different orbits. The only way to move from one plane to another is to traverse a third plane, such as a Transition Plane.

\textit{Adjacent Planes}: Those planes that connect to each other at specific points are considered adjacent. Where they touch, there is a connection through which travelers can step out of one reality and into the other. Usually demiplanes can act as a connecting portal.

\subsection{Planar Features}\index{Planar Features}
Each plane of existence has its own peculiarities; the natural laws of his universe. Planar features are divided into general areas. All plans have the following features.

\textit{Physical Characteristics}: Determine the physical and natural laws of the plane, including how gravity and time work.

\textit{Elemental and Energy Characteristics}: The influence of elemental and energy forces is determined by these characteristics.

\textit{Traits}: Just as characters may have Traits, so many planes are tied to a particular moral or ethic.

\textit{Magical Features}: Magic works differently from plane to plane; magical characteristics draw the line between what magic can and cannot do on each plane.

\textit{Physical Characteristics}

The two most important natural laws determined by physical traits concern the function of gravity and time. Other physical characteristics concern the size and shape of a plane and the way in which its nature can be altered.\\

\textbf{Gravity}\\

The direction of gravitational attraction may be unusual, and may even change directions within the same plane.\\

\textbf{Time}\\

The rate at which time passes can vary in different planes, though it remains constant within any specific plane. Time is always subjective for the viewer. The same subjectivity applies to the various plans. Travelers may find that they are gaining or wasting time moving between planes, but in their view, time passes naturally.


\textit{Normal Time}: Sets the passage of time on the Material Plane. One hour on a plane with normal time equals 1 hour on the Material Plane. Unless otherwise specified in a plane's description, it is assumed that it is in normal weather.
\textit{Irregular Time}: Some planes are characterized by slowing and quickening time, so an individual may lose or gain time as they move between planes like this one and others. For the inhabitants of such a plane, time passes naturally and the shift goes unnoticed.


\textit{Flowing Time}: On some planes, the flow of time is considerably faster or slower. Someone could travel to another plane, spend a year there, and then return to the Material Plane to find that only 6 seconds have passed. Everything on the plane you returned to lived just a few seconds longer. For the traveler and the items and spells and effects working on him, that year away was completely real. When designing how time works on planes with flowing time, think first of the time flow of the Material Plane, then of the flow in the other plane.


\textit{Timelessness}: Time passes on planes with this trait, but its effects are limited. How timelessness affects certain activities and conditions such as hunger, thirst, aging, poison effects, and healing varies from plane to plane. The danger of a timeless plane is that when an individual leaves that plane for another where time flows normally, conditions such as starvation and aging retroactively occur. If a plane has timelessness in relation to magic, any spells cast with non-instantaneous duration become permanent until dispelled.\\


\textbf{Elemental and Energy Characteristics}\\

Four basic elements and two types of energy combine to shape everything; the elements are water, air, fire and earth; the types of energy are positive and negative. The Material Plane reflects a balance of these elements and energies: it is possible to find them all. Each of the Inner Planes is dominated by an element or type of energy. Many planes have no elemental or energetic characteristics; these characteristics are specified in the description of a floor only if present.

\textit{Dominant Water}: Planes with this trait are mostly liquid. Visitors who cannot breathe underwater or who cannot reach an air pocket are likely to die by drowning. Fire creatures are extremely uncomfortable on water-dominant planes. These creatures, made of fire, take 1d10 points of damage each round.

\textit{Aria Dominante}: Composed essentially of open space, planes with this characteristic house just a few pieces of floating stone or other solid material. They usually have a breathable atmosphere, although such a plane might have clouds of acidic or toxic gas. Creatures of the earth subtype find themselves uncomfortable on air-dominant planes due to little or no natural earth with which to come into contact. However, they suffer no actual damage.

\textit{Land Dominant}: Planes with this trait are mostly solid. Travelers who reach it are at risk of suffocation unless they reach a cave or other crevice. Worse, individuals who lack the Burrow ability become trapped underground and must dig their own way out (1 meter per round).
Creatures of the air subtype are uncomfortable on land-dominant planes, considering them cramped and claustrophobic, but aside from having difficulty moving, they experience no other drawbacks.

\textit{Dominant Fire}: Planes with this trait consist of flames that burn continuously without depleting their power source. Fire-dominant planes are extremely hostile to Material Plane creatures, and those without resistance or immunity to fire are quickly incinerated. Wood, paper, unprotected cloth, and other flammable materials catch fire almost instantly, as do those wearing flammable, unprotected clothing. In addition, individuals take 3d10 points of fire damage each round they remain on a fire-dominant plane. Creatures of the water subtype are extremely uncomfortable on fire-dominant planes. These creatures, made of water, take double damage each round.

\textit{Dominant Negative Energy}: Planes with this trait are vast, empty recesses that suck the life essence of travelers who pass through them. They tend to be barren, haunted plains, stripped of color and filled with winds that carry the faint wails of those who died within them. There are two types of traits based on dominant negative energy: lesser and greater dominant negative energy. In the former, living creatures take 1d6 points of damage per round. At 0 Hit Points or less, these are reduced to ashes.

The latter are even more dangerous. Each round, those within it must make a DC 25 Fortitude save (maximum Hit Points drop by 6 if it dies in this manner becomes a wraith). The death ward spell protects the traveler from the damage and energy drain of a plane with dominant negative energy.

\textit{Dominant Positive Energy}: An abundance of life distinguishes planes with this trait. As with the planes with dominant negative energy, the planes with dominant positive energy can also be minor and higher.
A plane with minor dominant positive energy is a tumultuous explosion of life in all its forms. The colors are brighter, the fires hotter, the noises louder and the sensations more intense thanks to the positive energy diffused in the plane. All individuals in a plane with dominant positive energy regenerate 2 Hit Points per round.\\

\textbf{Traits}\\

Some planes have a bias toward a specific set of Traits. The denizens of these planes mostly share this set of Traits or some of them. The set of Traits of a plane influences its social interactions. Characters who have different Traits from most denizens may have difficulty dealing with the natives and situations of the plane. Traits have multiple components. First of all there are the moral and ethical components. Second, there may be a specific indication of whether this set of Traits manifests itself moderately or more markedly. Many planes have no Traits; the latter are specified in the description of a floor only if present

In principle, the elemental, astral and ethereal planes have no Traits.\\

\textbf{Magic Features}\\

A plane's magical characteristics define magic on that plane relative to the Material Plane. In particular places on a plane (such as those under the direct control of deities) a different magical feature might apply.


\textit{Normal Magic}: This magical feature means that all spells and supernatural abilities act as described. Unless otherwise described, a plane is assumed to have the normal magic trait.

\textit{Dead Magic}: Marks planes where magic does not exist at all. A plane with the dead magic characteristic functions in all respects as an antimagic field. Divination spells cannot detect someone in a plane of dead magic, nor can a spellcaster use the teleport spell to move in and out of it. The only exception to the "no magic" rule is permanent planar gates, which still function normally.

\textit{Enhanced Magic}: On planes with this magical trait, particular spells and spell-like abilities are easier to use or produce more powerful effects than they do on the Material Plane. Natives of a plane are aware of which spells and spell-like abilities are enhanced, but planar travelers may find out on their own. If a spell is empowered, it functions as if it scored a crit in the Magic Test.

\textit{Hampered Magic}: Particular spells and spell-like abilities are more difficult to use on planes with this trait, often because the nature of the plane hinders them. To cast a hindered spell he must score a crit on the Magic Test. If the check fails, the spell has no effect but is still wasted. If the check succeeds, the spell takes effect normally.

\textit{Limited Magic}: Planes with this trait only allow the use of spells and spell-like abilities that meet particular requirements. Magic can be limited in its effects by certain schools or subschools, by effects with certain descriptors, or by effects of a given level (or by any combination of these aspects). Spells and spell-like abilities that don't meet the requirements simply have no effect.

\textit{Wild Magic}: On a plane with the wild magic characteristic, spells and spell-like abilities work in totally different and sometimes dangerous ways. There is a chance that any spell or spell-like ability used on a wild magic plane has no effect. When the spellcaster casts a spell he must make two checks if even one fail causes something unusual to happen; roll a d100 and consult the

\medskip

\end{multicols}

\textbf{Table: Effects of Wild Magic}\index{Table Effects of Wild Magic}

\medskip

\begin{xltabular}{0.95\textwidth}{lX}
d100&Effect\\
\toprule
01-19&The spell bounces back to the caster with normal effect. If the spell cannot affect the caster, it has no effect.\\
20-23&A circular pit 5 meters in diameter opens under your feet; its depth is 3 meter per Magic Proficiency of the caster.\\
24-27&The spell has no effect, but the target or targets of the latter are pelted with a shower of small objects (anything from flowers to rancid fruit), which disappear as soon as they hit. The attack continues for 1 round. During this time the targets are Blinded and must make a Magic Test to cast spells.\\
28-31&The spell affects a random target or area. Randomly choose a different target from among those within the spell's range or center the spell on a random location within that range. To randomly generate direction, roll 1d8 and count clockwise, starting from south. To randomly generate the range, roll 3d6. Multiply the result by 1 meter for short-range spells, 6 meters for medium-range ones, and 24 meters for long-range ones.\\
32-35&The spell functions normally, but any material components are not consumed and Spell Points aren't used, an item does not lose charges, and the effect does not affect the use limit of an item or a spell-like ability.\\
36-39&The spell has no effect. Instead, someone (friend or foe) within 10 meters of the caster receives the effect of a Heal spell.\\
40-43&The spell has no effect. Instead, effects of Deep Darkness and Silence cover a 10m radius around the caster for 2d4 rounds.\\
44-47&The spell has no effect. Instead, a Reverse Gravity effect covers a 10m radius around the caster for 1 round.\\
48-51&The spell takes effect, but shimmering colors swirl around the caster for 1d4 rounds. Treat this area as an effect of Shimmer Dust with a DC 10 + the level of the spell that generated this result. \\
52-59&Nothing happens. The spell has no effect. Any material component is used. The spell or spell slot is used up, an item loses charges, and the effect affects the use limit of an item or spell-like ability.\\
60-71&Nothing happens. The spell has no effect. Any material component is not used. The spell does not disappear from the caster's mind (a spell slot or prepared spell can still be used). An item does not lose charges, and the effect does not affect the use limit of an item or a spell-like ability.\\
72-98&The spell takes effect normally.\\
99-100&The spell has an enhanced effect. Magic Test automatically generates a crit\\
\end{xltabular}

%\addvspace{2cm}

\begin{multicols}{2}


\subsection{The Plans}

\subsubsection{Material Plan}\index{Material Plan}\label{pianomateriale}
The Material Plane is the core of most cosmologies and defines what counts as normal. This is the plan on which most campaigns focus.\\
The Material Plane has the following traits:\\
\textit{Normal Gravity}\\
\textit{Normal Time}\\
\textit{No Elemental or Energy Traits}: However, specific locations may exhibit such traits.\\
\textit{Moderately Neutral}: Though it may have high concentrations of evil or good, law or chaos Traits in places.\\
Normal Magic\\

\subsubsection{Plane of Shadows}\index{Plane of Shadows}\label{pianoombre}
The Plane of Shadows is a dimly lit dimension that simultaneously coincides and coexists with the Material Plane. It overlaps the Material Plane as much as the Ethereal Plane, so the planar traveler can use the Plane of Shadow to cover great distances quickly. The Plane of Shadows is a black and white world: the environment lacks color. Were it not for this, it would resemble the Material Plane. Despite the absence of light sources, some plants, animals and humanoids consider the Plane of Shadow as their home.

The Plane of Shadows has the following characteristics:

\textit{Geography imperfect}: Parts of the Plane of Shadows flow continuously to other planes. Therefore, despite the presence of reference points, creating an accurate map of the floor is almost impossible.

\textit{Traits}: Undisciplined, Free, Superficial, Vengeful, Pessimistic

\textit{Enhanced Magic}: Spells that work with shadow are enhanced on the Plane of Shadow. Despite the dark nature of the Plane of Shadow, spells that generate, use, or manipulate darkness are unaffected by the plane.

\textit{Thwarted Magic}: Light spells or spells that use or generate light or fire are hindered on the Plane of Shadow. Spells that generate light are less effective in general, since all light sources on this plane have half their range.


\subsubsection{Plane of Negative Energy}\index{Plane of Negative Energy}\label{pianoenergianegativa}
There is very little to see on the Negative Energy Plane for an observer. It is a dark and empty place, an endless grave into which the traveler could fall until the plane has erased light and life. The Negative Energy Plane is the most hostile of the Inner Planes, the most indifferent and intolerant of life. Only creatures immune to its drain effects can survive here.

The Negative Energy Plane has the following characteristics:

\textit{Greater Dominant Negative Energy}: Creatures that aren't undead take 10 HP of void damage per round. Upon death you become a Wraith.

In areas of Lesser Dominant Negative Energy, creatures that aren't undead suffer 2 HP of void damage per round.

\textit{Enhanced Magic}: Spells and spell-like abilities that use negative energy are enhanced. Skills that harness negative energy, such as channel negative energy, gain a +4 bonus to the DC of the save to resist the ability.

\textit{Impaired Magic}: Spells and spell-like abilities that use positive energy (including healing spells) are impeded. Characters on this plane must pass the Magic Test with a Magic Critical to cast spells that cure or remove negative effects.


\subsubsection{Positive Energy Plan}\index{Positive Energy Plan}\label{pianoenergiapositiva}
The Positive Energy Plane has no surface and is similar to the Air Plane with its totally open space. However, every corner of this plane is brightly lit with an innate power. Such power is dangerous to mortal forms, not predisposed to experience it. Despite its beneficial effects, it is one of the more hostile Inner Planes. A character without Defences will overflow with power as positive energy is channeled into him. But, since his mortal form is unable to contain such power, he will be incinerated, like a speck of dust caught on the edge of a supernova. Visits to the Positive Energy Plan are of short duration, and even then travelers must be adequately protected.
The Positive Energy Plan has the following characteristics:

\textit{Greater Dominant Positive Energy}: Every 10 rounds you suffer the effect of the Greater Restoration spell. 10 hp are regenerated per round, once on maximum Hit Points 10 temporary ho are gained every round, when the temporary hp reach double the maximum hp the creature explodes in colored energy.

In minor zones of dominant positive energy, you are affected by the lesser restoration spell every 10 rounds. 2 HP are regenerated per round, once at most 2 HP are taken in temporary rounds, when the temporary HP reaches double the maximum HP the creature explodes in colored energy.

\textit{Enhanced Magic}: Spells and spell-like abilities that use positive energy are enhanced. Skills that harness positive energy, such as channel positive energy, gain a +4 bonus to DC to resist the ability.

\textit{Thwarted Magic}: Spells and spell-like abilities that use negative energy (including inflict spells) are hindered.

\subsubsection{Elemental Plane of Water}\index{Elemental Plane of Water}\label{pianoacqua}
The Elemental Plane of Water is a sea without bottom or surface, a liquid environment illuminated by a diffused light. It is one of the more hospitable Inner Planes, once the traveler overcomes the problem of breathing underwater. The infinite oceans of this plane range from freezing cold to searing heat, and between salt water and fresh water. The plane's permanent settlements spawn around chunks of flotsam suspended in this never-ending fluid, drifting with the tides.

The Elemental Plane of Water has the following characteristics:

\textit{Dominant Water}

\textit{Enhanced Magic}: Spells and spell-like abilities that use spells or effects from the Water Elemental List or Water Outsiders are enhanced.

\textit{Hampered Magic}: Spells and spell-like abilities that use spells or effects from the Fire Elemental List or outsiders of the Fire subtype are hampered.


\subsubsection{Elemental Plane of Air}\index{Elemental Plane of Air}\label{pianoaria}
The Plane of Air is an empty plane, made up of sky in every direction. It is the most comfortable and livable of the inner planes and is home to all kinds of creatures of the air. Indeed, flying creatures gain great advantage in this plane. While travelers can survive well here even without the ability to fly, they are still at a disadvantage.
The Elemental Plane of Air has the following characteristics:

\textit{Dominant Air}

\textit{Enhanced Magic}: Spells and spell-like abilities that use spells or effects from the Air Elemental List or Air Outsiders are enhanced.

\textit{Hampered Magic}: Spells and spell-like abilities that use spells or effects from the Air Elemental List or outsiders of the Air subtype are hampered.


\subsubsection{Elemental Plane of Fire}\index{Elemental Plane of Fire}\label{pianofuoco}
On the Plane of Fire everything is illuminated. The ground consists of nothing but vast and shifting layers of compressed fire. The air is stirred by the heat of the continuous rains of fire and the most common liquid is magma. The oceans are composed of liquid flame and the mountains ooze molten lava. Here the fire lasts without anlimentation or air, but the flammable elements introduced on the surface are quickly consumed.

The Plan of Fire has the following characteristics:

\textit{Dominant Fire}

\textit{Greater Dominant Fire}: You take 10 HP of non resistable fire damage each round if you are not immune to fire.

\textit{Lesser Dominant Fire}: You take 2 HP of fire damage each round.

\textit{Enhanced Magic}: Spells and spell-like abilities that use spells or effects from the Fire Elemental List or Fire Outsiders are enhanced.

\textit{Hampered Magic}: Spells and spell-like abilities that use spells or effects from the Water Elemental List or outsiders of the Water subtype are hindered.

\subsubsection{Earth Elemental Plane}\index{Earth Elemental Plane}\label{pianoterra}
The Plane of the Earth is a solid place composed of earth and stone. An imprudent traveler could find himself buried by this vast solid mass: his pulverized remains will remain as a warning to anyone who dares to follow him. Despite its solid and rigid nature, the Plane of the Earth varies in texture, ranging from soft soil to veins of harder, more precious metal.

The Plane of the Earth has the following characteristics:

\textit{Dominant Land}

\textit{Enhanced Magic}: Spells and spell-like abilities that use spells or effects from the Earth Elemental List or Outsiders of the Earth are enhanced.

\textit{Hampered Magic}: Spells and spell-like abilities that use spells or effects from the Air Elemental List or outsiders of the Air subtype are hampered.

\subsubsection{Ethereal Plane}\index{Ethereal Plane}\label{pianoetereo}
The Ethereal Plane coexists with the Material Plane and often with other planes as well. The Material Plane itself is visible from the Ethereal Plane, but appears silent and indistinct; the colors blend together and the borders are blurred. While it is possible to see the Material Plane from the Ethereal Plane, the latter is usually invisible to those on the Material Plane. Normally, creatures from the Ethereal Plane can't attack those from the Material Plane, and vice versa. A traveler on the Ethereal Plane is invisible, incorporeal, and totally silent to someone on the Material Plane.

The Ethereal Plane has the following characteristics:

\textit{No Gravity}

\textit{Normal Magic}: Spells function normally on the Ethereal Plane, even if they do not cross the Material Plane. The only exceptions are spells and spell-like abilities that affect ethereal entities.

No magical attacks pass from the Ethereal Plane to the Material Plane, including power attacks.


\subsubsection{Astral Plane}\index{Astral Plane}\label{pianoastrale}
The Astral Plane is the space between all planes. When a character passes through a portal or projects his spirit into another plane of existence, he travels through the Astral Plane. Spells that allow instantaneous movement across a plane also briefly affect the Astral Plane. The latter is a great endless expanse of clear silver sky, both above and below. The occasional chunk of solid matter can be found here, but most of the Astral Plane is open, boundless space.

The Astral Plane has the following characteristics:

\textit{Timelessness}: Age, hunger, thirst, afflictions (such as Diseases, Curses, and Poisons), and natural healing have no effect in the Astral Plane, though they resume their function when the traveler leaves the Astral Plane. floor.

\textit{Enhanced Magic}: All spells and spell-like abilities used in the Astral Plane have a haste of 1 Action. Spells and spell-like abilities that have already been hastened are unaffected, as are spells from magic items. Spells quickened in this way are still prepared and cast at their original level.

\subsubsection{Vacuus}\index{Plane Vacuus}\label{pianovuoto}
The Plane of the Void. Lying wasteland under a putrid sky, Vacuus is enveloped in a sickening black fog, and the oppressive twilight of an endless solar eclipse. The mortal Styx is born in Vacuus, before slipping as a twisted serpent into the other planes. Vacuus is one of the most hostile Outer Planes: realm of the Caridions, fiends of pure evil indifferent to the conflict between law and chaos, representing oblivion and destruction. The Caridions ruled by four archcaridions with godlike powers, are feared as devourers of souls.

Vacuum has the following features:

\textit{Traits}: Destructive, Relentless, Hard to please, Irrational, Wrathful, Sadistic

\textit{Enhanced Magic}: Evil spells and magical abilities are enhanced.

\textit{Hampered Magic}: Benevolent spells and spell-like abilities are impeded.

\subsubsection{Abyss}\index{Plane of Abyss}\label{pianoabisso}
The Abyss, multi-layered plane is formed by gigantic canyons and gorges that gape in the fabric of the Outer Planes and is bordered by the nefarious waters of the River Styx. The endless layers of the Abyss, bordering all the Outer Planes, are connected to each other by constantly shifting paths. In the Abyss there are no rules, laws, order or hope. The Abyss represents the corruption of freedom, a nightmarish realm of absolute horror where desire and suffering take demonic form, a breeding ground for countless races of Demons, among the oldest beings in all creation. It is said that if ever the being at the deepest layer decides to wake up all planes will cease to exist.

The Abyss has the following characteristics:

\textit{Traits}: Anarchic, Vengeful, Touchy, Arrogant, Double agent

\textit{Highly Chaotic and Strongly Evil}

\textit{Enhanced Magic}: Chaotic or evil spells and magical abilities are enhanced.

\textit{Hampered Magic}: Lawful or good spells and spell-like abilities are hampered.


\subsubsection{Elysium}\index{Plane of Elysium}\label{pianoeliseo}
A vast land of untouched wilderness and wilderness, Elysium is the plane of benevolent chaos, freedom and independence, personified in the Yazata natives. In the Elysium, selfless cooperation and fierce competition collide violently, but such conflicts never overshadow the noble concepts of courage, creativity and good unhindered by rules or laws.

Elisha has the following characteristics:

\textit{Traits}: Good, Charitable, Anarchic, Innovative, Competitive

\textit{Enhanced Magic}: Chaotic or good spells and spell-like abilities are enhanced.

\textit{Hampered Magic}: Lawful or evil spells and spell-like abilities are hampered.


\subsubsection{Hell}\index{Plane of Hell}\label{pianoinferno}
The nine layers of Hell form a structured labyrinth of premeditated evil where torment goes hand in hand with purification. Planned by iron cities, burning wastelands, frozen glaciers and endless volcanic peaks, Hell is divided into nine concentric layers, each under the cruel rule of an archdevil. Torture, anguish, and suffering are inevitable in Hell, but they are delivered methodically, not out of spite or whim, and they support a planned design under the watchful eyes of Hell's disciplined ranks of lesser Devils. The nine layers of Hell are, from first to last, Avernus, Dis, Erebus, Phlegethon, Stygia, Malebolgia, Cocytus, Caina and Nessus.

Hell has the following characteristics:

\textit{Traits}: Evil, Disciplined, Wrathful, Sadistic, Arrogant

\textit{Strongly Lawful and Strongly Evil}

\textit{Enhanced Magic}: Lawful or evil spells and magical abilities are enhanced.

\textit{Hampered Magic}: Chaotic or good spells and spell-like abilities are hampered.

\subsubsection{Nirvana}\index{Plane of Nirvana}\label{pianonirvana}
Nirvana is an impartial paradise existing between the two extremes of Elysium and Paradise. Its wondrous mountains, hills, and dense forests meet the traveller's expectations of a pastoral paradise, but Nirvana also contains mysteries that lead to enlightenment. Nirvana is a sanctuary and resting place for all in search of redemption or enlightenment. The Agathos natives of Nirvana have willingly set aside their transcendence to guard the riddles of the plane, while the celestials battle the forces of evil between the planes.

Nirvana has the following characteristics:

\textit{Traits}: Good, Gentle, Calm, Simple, Confident

\textit{Enhanced Magic}: Good spells and spell-like abilities are enhanced.

\textit{Hampered Magic}: Evil spells and spell-like abilities are impeded.


\subsubsection{Paradise}\index{Plane of Paradise}\label{pianoparadiso}
The Plane of Heaven is an ordered realm of honor and compassion that is divided into seven layers. The slopes of Paradise are full of orderly and well-structured cities and clean and well-kept gardens and orchards. Though they began their lives as mortals, the native Archons of Paradise see law and good as two inseparable halves of the same supreme concept and stand against the cosmic corruptions of chaos and evil.

Paradise has the following characteristics:

\textit{Traits}: Good, Stiff, Combative, Practical. Sincere, Valiant

\textit{Enhanced Magic}: Lawful or good spells and spell-like abilities are enhanced.

\textit{Hampered Magic}: Chaotic or evil spells and spell-like abilities are hampered.


\subsubsection{Purgatory}\index{Plane of Purgatory}\label{pianopurgatorio}
Every soul passes through Purgatory to be judged before being sent to its final destination. Vast graveyards and wastelands fill its gloomy wastes, along with dusty, echoing courts of judgment for the dead. Purgatory is the abode of the Aeons, a race which embodies the dualistic nature of existence and whose members are constantly at war and at peace with each other and with themselves.

Purgatory has the following characteristics:

\textit{Timelessness}: Age, hunger, thirst, afflictions (such as Diseases, Curses, and Poisons), and natural healing have no effect in Purgatory, though they resume their function when the traveler leaves the plane .

\textit{Enhanced Magic}: Spells and spell-like abilities involving death or rest are enhanced.


\subsubsection{Utopia}\index{Plane of Utopia}\label{pianoutopia}
Utopia is a stronghold of order pitted against anarchy and the endless demonic hordes of the Abyss. It is a great city of eternal perfection, whose streets and buildings are models of architecture and aesthetics: everything is in order and nothing happens by chance. While Utopia is ruled by no race, the Ordinax make it their home, constantly seeking to expand their perfect city.

Utopia has the following characteristics:

\textit{Traits}: Stiff, Disciplined, Serious, Direct, Cold

\textit{Enhanced Magic}: Lawful spells and magical abilities are enhanced.

\textit{Hampered Magic}: Chaotic spells and spell-like abilities are hampered.


\subsubsection{Genesis}\index{Plane of Genesis}\label{pianogenesi}

Tradition has it that the Patrons of Genesis are on a plane on the edge of everything and within everything. What this \textit{place} is like is not known to any mortal. Legends, purely fantastic stories, tell of an environment of pure divine energy capable of materializing everything.

No one can get to the Plane of Genesis, if it ever existed, without an explicit invitation from Calicante, Ljust or Atmos. The plan is also forbidden to Patrons except Lynx.


\end{multicols}

\pagebreak

\section{OBSS Monsterarium}\index{Monsterarium}

\textbf{The Monsters Are Coming...}

\begin{changemargin}{0cm}{0.5cm}\begin{emphasis}{Whoever fights monsters must be careful not to become a monster by doing so. And if you gaze long into an abyss, the abyss will also gaze into you. (Friedrich Nietzsche)

\medskip

Monsters can only be defeated by their own kind. (Claymore)

\medskip

The tragedy of monsters is that they are too big and too powerful for mankind to accept. (Ishiro Honda)}\end{emphasis}\end{changemargin}\medskip


\begin{multicols}{2}

\lettrine[lines=2, lhang=0.33, loversize=0.25, findent=1.5em]{W}{elcome} to a universe full of enemies bad violent sneaky smart petty gigantic.. and whatever else you want. Monsters are the cornerstone of any fantasy RPG.

Monsters are explained and presented here, certainly not all nor exhaustive, use them to populate the adventures of your companions with nightmares.

\medskip

\begin{center}

\includegraphics[width=0.8\linewidth]{immagini/sangiorgioedrago.png}

\textit{St. George and the Dragon (c. 1460) by Paolo Uccello. National Gallery of London}
\end{center}

\subsection{Introduction}

An adventure is not just a set of monsters but situations, places, surprises, in short, everything that can fascinate, involve, amaze, engage the players. But monsters are useful too. Hitting has a cathartic, liberating aspect.

Insert difficult and dangerous monsters into the adventure where needed but occasionally, rarely make the players feel powerful, make them face monsters that they can resolve in just a few rounds. Describe combat by emphasizing hits, crits, pain, and monster blood. Show how powerful characters can be.

Other times you make monsters scary because they're big, hungry, magical and evil, players need to be afraid for their characters, never take victory for granted.

Confidence in describing the situation, few jokes, staring the players in the eye. Engage the players and once they have your attention the characters will be more attentive too. Try to put monsters that are coherent with the environment, with the adventure, with the situation. Don't roll randomly on tables, a well-organized fight is much more satisfying than random displays that \textit{spawn}.

Don't boil everything down to an MMORG where the goal is just to kill everything and everyone, there can always be so many choices if you put some effort into it.

\begin{changemargin}{0.3cm}{0.3cm}\begin{tcolorbox}[title = Fighting Monsters]
{
Let this old man give you a couple of tips young adventurer!

- Not all enemies can be defeated with a sword, many times a mace is also needed!

- Sometimes weapons and brute force just aren't enough. If you don't have companions who can cast spells make sure you always have a chance to start a fire.

- Escape. It is always a valid option if you have the opportunity and see that the situation does not bode well.

- Organized! don't enter the dungeon with your head down and never stop except when you're dead! Rest, explore, check your environment, and when you're safe and better, move on! even your enemies organize themselves and rest in the meantime, be careful!

- Sometimes you can even talk to enemies, they also don't want to die always.

- If you have to kill do it with malice and speed. Don't waste time and optimize your shots, save your energy and immediately prepare for another fight.

}\end{tcolorbox}\end{changemargin}

\subsection{Modify Creatures}

Despite the motley collection of monsters in this book, you may still find yourself at a loss when it comes to finding the perfect creature for your adventure. Feel free to modify existing creatures and turn them into something more useful to you, perhaps borrowing a trait or two from a different monster.

Keep in mind that editing a monster may change its challenge rating.

\subsection{Size and Hit Dice}

A monster can be Tiny, Small, Medium, Large, Huge, or Gargantuan and Colossal in size. The Size Categories table shows a creature's average size, how much space it occupies on the grid, and what hit dice it uses to determine its Hit Points.

If not indicated, the range of a creature depends on the size and the weapon used (think of a gigantic greatsword brandished by a titan..)

\end{multicols}

\textbf{Table: Size Categories, Occupied Squares and Capacity}\index{Table Size Categories, Occupied Squares and Capacity}\index{Reach for creatures}\index{Square for creatures}\index{Size and squares}\index{Creatures for squares}

\medskip

\begin{tabularx}{0.95\textwidth}{llllll}
\toprule
\textbf{Cut}& \textbf{Space} & \textbf{Example}&\textbf{Squares}&\textbf{Hit Die}&\textbf{Reach}\\
Tiny & 25 x 25 cm&Cat, Sprite& 1/4&d4&0m\\
Small & 0.5 x 0.5 m & Goblin, Dog, Gnome&1/2&d6&1m\\
Medium & 1 x 1 m & Orc, Human, Elf, Dwarf, Nibali &1&d8&1m\\
Large & 2 x 2 m& Ogre&2x2&d10&2m\\
Huge & 3x3 m & Giant, Ent&3x3&d12&2m\\
Gargantuan & 4x4 m&Kraken, Dragon&4x4&3d6&2m\\
Colossal & 12 x 12 m&Elder Dragon, Tarrasque&6x6&2d12&6m\\
\end{tabularx}

\medskip

The dimensions occupied on the grid are reduced compared to the actual size as the miniatures on the market are prepared with the scale 1 square = 1.5 metres. \textit{When OBSS becomes the world's dominant RPG then it will have its own scale miniatures!}

\begin{multicols}{2}

\subsection{Type}

A monster's type refers to its basic nature. Certain spells, magic items, Abilities, and other effects in the game interact in special ways with creatures of a specific type. For example, a \textit{dragon slaying arrow} deals extra damage not only to dragons but also to all other creatures of the dragon type, such as turtle dragons and wyverns.

The game includes the following types of monsters:

\smallskip\textbf{Aberrations}, totally alien creatures. Many of them possess innate magical abilities that draw on the creature's alien mind rather than the mystical forces of the world. Classic examples of aberrations are abolets, watchers, mind flayers, and the Chaos Batrachians.

\smallskip\textbf{Beasts}, non-humanoid creatures that are a natural component of a fantasy world. Some possess magical powers, but most lack Intelligence and have no form of society or language. Classic examples of beasts are all common animal species, dinosaurs, and giant versions of animals.

\smallskip\textbf{Celestials}, creatures native to the Higher Planes. Many of them are servants of the deities, employed as messengers or agents in the mortal world and for the planes.\\
Celestials are good natured, classic examples of celestials are angels, couatls, and pegasi.

\smallskip\textbf{Constructs}, are created and not birthed. Some are programmed by their creators to follow a simple set of instructions, while others are sentient and able to think for themselves. Golems are the most representative constructs.

\smallskip\textbf{Dragons}, are large reptilian creatures of ancient origin and enormous power. True dragons, including good Ljust dragons and evil Tàhil dragons, are highly intelligent and possess innate magical talents. Creatures distantly related to true dragons, but less powerful, less intelligent, and less magical, such as wyverns and pseudodragons, also fall into this category.

\smallskip\textbf{Elementals}, are creatures native to the elemental planes. Some creatures of this type are little more than animate masses of their respective element, and include creatures simply called elementals. Other creatures possess biological forms infused with elemental energy. The genie races, including djinn and efreet, form the major civilizations of the elemental planes. Other elemental creatures are azers, invisible stalkers, and water freaks.

\smallskip\textbf{Faeries}, are magical creatures closely related to the forces of nature. They live in hidden glades and misty forests. Examples of fairies are dryads, pixies, fairies and satyrs, and La Topi.

\smallskip\textbf{Giants}, they tower over humans and their kind. They are human in shape, although some have multiple heads (ettin) or deformities (fomorians). The six variants of true giants are hill giant, stone giant, frost giant, fire giant, cloud giant, storm giant. Besides these, ogres and trolls are also giants.

\smallskip\textbf{Fiends}, wicked creatures native to the Lower Planes. Some serve deities, but many more serve archdevils and demon princes. Sometimes evil priests and spellcasters summon fiends into the material world to do their wills. While an evil celestial is a rarity, a good fiend is practically inconceivable. Fiends include demons, devils, hellhounds, and rakshasas.

\smallskip\textbf{Slimes}, are gelatinous creatures that hardly have a fixed shape. They live mainly underground, settling in caves and dungeons, feeding on refuse, carcasses or creatures unfortunate enough to stumble across them. Black puddings and gelatinous cubes are among the most recognizable slimes.

\smallskip\textbf{Monstrosities}, are monsters in the strictest sense of the term frightening creatures that are not common, nor truly natural, and almost never benign. Some are the result of magical experiments gone wrong, while others are the product of terrible curses (including the minotaur). They defy categorization, and somehow serve as an all-encompassing category for those creatures that don't match any other monster type.

\smallskip\textbf{Undead}, are once-living creatures brought to a hideous state of undeath through the practice of necromantic magic or some blasphemous curse. The undead include walking corpses, such as vampires and zombies, or incorporeal spirits, such as ghosts and wraiths. Most intelligent undead speak Exspiram a whispering language of foul sound.

\smallskip\textbf{Plants}, in this context these are plant creatures, not ordinary flora. Most of them are mobile, and some are carnivorous. The most classic example of plants are walking mounds and ent. Fungoid creatures such as gas spores and myconids also fall into this category.

\begin{center}
\includegraphics[width=0.60\linewidth]{immagini/sanmichelesatana.png}\\
\textit{St. Michael defeats Satan. Raphael and assistants (1518). Louvre Museum}
\end{center}

\smallskip\textbf{Humanoids}, they are the main population of the game worlds, civilized and wild, they include humans and a wide range of other species. They possess a language and culture, little or no innate magical abilities (although many humanoids can learn spells), and a bipedal form. The most common races of humanoid are those best suited as player characters: humans, dwarves, elves, and nibals, various. Almost as numerous, but more brutal and savage, and nearly all evil, are the goblinoid races (goblins, hobgoblins, and bugbears), orcs, gnolls, lizardfolk, and kobolds.\\

\medskip

These categories can in turn be grouped into types of Creatures:
\smallskip
\begin{itemize}[leftmargin=*]
\item
The \textbf{Natural Creatures}: are Insects, Reptiles, Beasts, Humanoids, Plants, Aquatic Creatures, Monstrusities, Oozes
\item
The \textbf{Magical Creatures} are: Fiends, Fey, Spirits, Undead, Giants, Celestials, Constructs, Aberrations (anything alien or unnatural), and Dragons.

If a Natural Creature has magical powers then it is also considered a Magical Creature.
\end{itemize}


\medskip\textbf{Labels}

A monster can have one or more tags listed in parentheses, following its type. For example, an orc has the type \textit{humanoid (orc)}. The labels in parentheses provide further categorization for certain creatures. Tags do not have their own specific rules, but some game elements, such as magic items, can refer to them. For example, a spear that is particularly effective against demons would work against any monster that has the demon tag.

\subsection{Traits}

The monsters do not have a detailed list of Traits, you will only find the indication on the axes of Chaos, Law, Good and Evil. Remember that they are indications, exceptions can happen especially in the most intelligent species.
Certain creatures are \textbf{unaligned}, that is, they have no moral or ethical conduct.

\subsection{Defence}

A monster wearing Armour or carrying a shield has a Defence that takes into account Armour, shield, and Dexterity. Otherwise, a monster's Defence is based on its Dexterity value and natural Armour if it has any. If a monster has natural Armour, is wearing Armour, or is carrying a shield, it is listed in parentheses after its Defence value.

If the monster is \textbf{taken by surprise} subtract -4 from Defence.

\subsection{Hit Points}

Usually when it drops to 0 Hit Points, a monster dies or is destroyed.

A monster's Hit Points are presented both as a set of dice and as an average value. For example, a monster with 2d8 Hit Points has an average of 9 Hit Points (2 x 4.5).

It will happen that players ask you \textbf{\textit{how is the monster}}, I suggest you never go into details by saying how many HP it has in all or lost, but stay in these grades: Not wounded (HP full), Wounded (30\% HP suffered), Seriously wounded (at least 50\% HP suffered), or give a generic description of the state. \index{How is the moster}\index{Ask health of monster}

A monster's Constitution value also affects the number of Hit Points it has. Its Constitution value is multiplied by the number of Hit Dice it has, and the result is added to its Hit Points. For example, a monster that has Constitution 1 and 2d8 Hit Dice, and will therefore have 2d8+2 Hit Points (average 11).

\subsubsection{Angry}\index{Angry}\index{Bloodied}\label{mostroarrabbiato}

At the Arbiter's discretion, once a creature has lost at least 50\% of its total Hit Points, it accesses a reserve of "rage" which allows it to perform particular actions.
Monsters with a Challenge Rating of 5 or higher can have an \textbf{Angry} note. The Angry ability can be used once for fight.

Particularly ferocious and powerful creatures could have more notes of Angry and both, respecting any marked conditions, can be activated.


\subsubsection{Optional - All Angry}\index{Optional - All Angry}

For greater aggressiveness you can make the monster when it drops below half HP take +1d6 on the attack roll or on the damage or Saving Throws depending on the type of creature.\index{Bloody}\index{Bloodied}\index{Angry}\\
Creature could also negate a negative effect running.

\begin{center}
\includegraphics[width=0.65\linewidth]{immagini/roc.png}\\
\textit{Henry Justice Ford}
\end{center}

\subsection{Movement}

A monster's Move tells you how far it can move during its round per Move Action

All creatures have a walking move, simply called a monster's move. Creatures that have no form of land movement have a walking speed of 0 feet.

Some creatures have one or more of the following additional movement modes.

\smallskip\textbf{Swimming}

A monster that has a swim speed doesn't have to spend extra movement to swim (it's not difficult terrain).

\smallskip\textbf{Climb}

A monster that has a climb speed can use all or part of its movement to move up vertical surfaces. The monster does not have to spend extra movement (x4) to climb.

\smallskip\textbf{Excavation}

A monster that has a burrow speed can use its speed to move through sand, dirt, mud, etc. A monster can't burrow through solid rock unless it has a special trait that allows it to.

\smallskip\textbf{Flight}

A monster that has a flying speed can use all or part of its movement to fly. Some monsters have the ability to \textbf{float}, which makes them difficult to take down. The monster stops floating when it dies.

\subsection{Ability Scores}

Each monster has six ability scores (Strength, Dexterity, Constitution, Intelligence, Wisdom, Charisma)

\subsection{Skills}

The Skills entry is reserved for those monsters that are proficient in one or more skills. For example, a monster that is very alert and stealthy might have bonuses on Wisdom (Awareness) and Dexterity checks.

Other modifiers may also apply, for example, a monster might have a larger-than-expected bonus to account for its great skill.

If not indicated the Weapon Proficiency has value as Challenge Rating.

\subsection{Vulnerabilities, Resistances and Immunities}\index{Magic Weapon equivalence}\index{Fist, Magical}
Some creatures have vulnerabilities, resistances, or immunity to a certain type of damage. Particular creatures are even resistant or immune to nonmagical attacks (a magical attack is an attack delivered via a spell, magic item, or other source of magic).

It is also possible that a specific minimum magical bonus is indicated to be able to damage the creature. Critical immune creature's is for both weapons and spells, burst damage is applied.\index{Critical on monsters}

Additionally, certain creatures are immune to certain conditions. If a monster is immune to a game effect that isn't treated as damage or a condition, it instead has a special trait.

The table below indicates which magical weapon enchantment is necessary to overcome the indicated immunity. The minimum Weapons Proficiency score is also indicated if you hit with kicks and punches.

In case of character with Weapon List \textbf{Empty Fist} check how many times the list has been taken.

\medskip

\textbf{Table: Magic Weapon Equivalence}\index{Table Magic Weapon Equivalence}\hypertarget{equivalenzemagiche}{}\label{equivalenzaarmimagiche}

\medskip

%\begin{tabular}{lp{0.055\textwidth}p{0.06\textwidth}p{0.07\textwidth}}
\begin{tabularx}{0.5\textwidth}{lXXX}
\toprule 
\textbf{Immunity} & \textbf{Magic Weap.} & \textbf{Nat. Attack}& \textbf{Empty Fist}\\
+1 & +1 &  3&  2\\
+2 & +2 &  6&  4\\
Ferro Freddo & +1 &  4&  2\\
Argento  & +1 &  4&  2\\
Adamantio & +2 &  6&  4\\
+3  & +3 &  12& 8\\
+4 & +4 &  16& 12\\
+5 & +5 & - &  16\\
\end{tabularx}

\subsection{Senses}

The Senses entry lists any special senses the monster has. The special senses are described below. If the Senses entry is not present, the creature has standard senses (vision, smell, taste, touch...).

If not specified the Awareness of monster is his Challenge/2 + Wisdom.

\subsubsection{Telluric Perception}

A monster with tremorsense can locate and find the origins of vibrations within a specified radius, as long as the monster and the source of the vibration are in contact with the same ground or substance. Tremorsense cannot be used to detect flying or incorporeal creatures. Many burrowing creatures, such as ankhegs and earth behemoths, possess this special sense.

\subsubsection{Twilight Vision or Darkvision}

A creature with low-light vision can see in the dimest of lights, but not in complete darkness unlike those with darkvision. Many creatures that live underground possess this special sense. See chapter \hyperlink{visioneeluce}{Special Features}.

\subsubsection{True Seeing}

A monster with true seeing can, up to a specified range, see through normal and magical darkness, see invisible creatures and objects, automatically detect illusions and succeed on Saving Throws against them, and sense a shapeshifter's original form. or a creature transformed by magic. In addition, the creature can see into the Ethereal Plane up to the same range.


\begin{center}
	\includegraphics[width=0.60\linewidth]{immagini/ciclope.png}
	
	\textit{Henry Justice Ford}
\end{center}

\subsubsection{Blindsight}

A creature with blindsight can perceive its surroundings, without relying on sight, up to a specified range.

Eyeless creatures such as grimlocks and oozes and creatures with echolocation or enhanced senses, such as bats and dragons, possess this sense.

If a monster is naturally blind, this is noted in parentheses, indicating that the range of its blindsight also defines the maximum range of its perception.


\subsection{Languages}

The languages a monster can speak are listed in alphabetical order. Sometimes a monster can understand a language but not speak it, and this is indicated in this entry. If a monster does not have the Languages note, it means it knows no languages other than its own tongue (if applicable).

\subsection{Telepathy}

Telepathy is an ability that allows a monster to communicate mentally with another creature within a specified range. The contacted creature need not speak the same language as the monster to communicate in this way. A creature without telepathy can receive and respond to telepathic messages but cannot initiate or end a telepathic conversation.

A telepathic monster need not see the contacted creature and can end the telepathic contact at any time. The contact is broken as soon as the two creatures are no longer within range or if the telepathic monster contacts another creature within range. A telepathic monster can initiate or end a telepathic conversation without using an action, but while the monster is incapacitated, it cannot initiate telepathic contact, and any ongoing contact is ended. To initiate telepathic communication the target must have at least been identified.

A creature in the area of a \textit{anti-magic field} or anywhere else where magic doesn't work can send or receive telepathic messages.

\subsection{Challenge}

A monster's \textbf{challenge rating} (CR) tells you how great a threat it poses. A properly equipped and rested party of four adventurers must be able to defeat a monster of a challenge rating equal to their average level without taking casualties. For example, a party of four 3rd-level characters might find a challenge rating 3 monster a worthy challenge, but not lethal.

Monsters that are significantly weaker than 1st-level characters have a challenge rating of less than 1. Monsters with a challenge rating of 0 present no problems except in large numbers; those with no real attacks are worth no experience points.

Some monsters present a greater challenge than even a 20th-level fellowship can handle. These monsters have a challenge rating of 21 or higher and are designed specifically to check the abilities of the characters.

\subsection{Recognizing Monsters}\label{riconoscereimostri}\hypertarget{riconoscereimostri}{} \index{Recognizing Monsters}

Knowing how to recognize a monster can be extremely useful and is something that should never be underestimated.

To \textbf{recognize a monster} a Knowledge check is made. (\textbf{1 Action}) on:\\

\textit{Arcana}: Giants, Constructs, Spirits, Monstrosities, Aberrations, Dragons

\textit{Planes}: Elementals

\textit{Occult}: Fiends (Devils and Demons), Spirits, Undead

\textit{Religion}: Spirits, Undead, Celestials

\textit{Dungeon}: Aberrations, Monstrosities, Oozes, and underground creatures

\textit{Nature}: Beasts, Plants, Fairies\\

The DC of the check is equal to 10 + the creature's Challenge rating + any rarity factor (common (0), uncommon (+1), rare (+2), very rare (4), legendary (8)) .

The information that can be obtained depends on the degree of success achieved. \\

- \textit{within 2}: name, type, main feature\\
- \textit{within 7}: which is the best saving throw, a resistance/immunity to Conditions, a vulnerability to Conditions, typical attack\\
- \textit{within 12}: which is the worst Saving Throw, a resistance/immunity to Conditions, an immunity to Damage, a vulnerability to Conditions, a vulnerability to a type of Damage\\
- \textit{within 15}: two immunities to Conditions, one immunity to Damage, one vulnerability to Conditions, one vulnerability to Damage type\\
- \textit{within 17}: relative degree of challenge or whether it is an easy, medium, high, extraordinary, deadly or epic match \\
- \textit{within 20}: attack and special defenses\\

The information obtained is cumulative, i.e. if the test succeeds by 14 you get the information within 2, 7 and 12.


\subsection{Special Traits}

Special traits (which appear after a monster's challenge rating but before any actions or reactions) are quirks that will likely play a role in a combat encounter and that require explanation.
 
\begin{center}
\includegraphics[width=0.7\linewidth]{immagini/lich2.png}

\textit{Lich - Battle of Wesnoth}

\end{center}

\subsection{Enchantments}

A monster with the Spellcasting feature is able to cast Spells.

A monster can cast a spell from its spell list without making a Magic Test. The DC is \textbf{10 + spell level x2 + Intelligence or Wisdom whichever is better or indicated}. A monster with spells cannot convert the Spell of higher level into lower level spells, except if it has a Magic Proficiency value (eg Lich, Mummy, Naga...).

\subsection{Innate Spells}

A monster with the innate ability to cast spells has the Spellcasting special trait.
A monster's innate spells cannot be exchanged for other spells.

A monster never willingly makes a Magic Test unless it has Intelligence greater than 1.


\subsection{Actions}

Monsters also act according to the 3 Actions scheme available per round. Skills and abilities that allow him to perform a higher number of Actions can be scored.

When a monster performs its actions, it can choose from the options in the Actions section of its stat block or use one of the actions available to all creatures, such as Dash or Hide.

\subsubsection{Melee and Ranged Attacks}

The most common action a monster will take in combat will be a melee or ranged attack. These can be spell attacks or weapon attacks, where the weapon can be an artifact or a natural weapon, such as claws or tail spikes.

\textit{\textbf{Creature vs. Target}.} The target of a melee or ranged attack is usually a creature or target.

\textbf{Reach}: the indicated range is the distance \textbf{within} how many meters the creature can hit the opponent. Even if the reach is greater than that of the opponent, advantages such as Long Weapon (+2 to AR) are not considered with natural attacks. A creature with 0 reach must be on you to strike you, extremely small creatures usually have 0 reach.

\textit{\textbf{Hit.}} Any damage dealt or other effects that occur as a result of an attack hitting the target are described in the annotation ``\textit{Hit}''. You can choose to take the average damage or roll the dice; for this reason both average damage and a dice formula are presented.

I suggest that critical roll is still applicable for enemies while Burst Damage is to be used if you want a more lethal campaign.\index{Critical Damage Monsters}


\textbf{\textit{Miss}.} If an attack has a miss effect, that information is provided by the ``\textit{Miss}'' annotation.

\textit{\textbf{Damage.}} If a monster wields manufactured weapons, it deals damage appropriate to the weapon. Bigger monsters usually wield larger weapons that deal extra damage when they strike. If they use this type of weapon, the damage is already marked, otherwise if they pick up or use an unexpected weapon, double the weapon dice if the creature is Large, triple them if Huge and quadruple them if Gargantuan if they use weapons of their size.

A creature has -1d6 on attack rolls with a weapon made for a size larger than itself. The Arbiter may decide that weapons two or more sizes larger than the attacker's are entirely impossible to use.

\begin{changemargin}{0.3cm}{0.3cm}\begin{narrator}
A creature of at least Challenge Rating 6 at the Arbiter's discretion can make an attack of opportunity (see \hyperlink{opportunista}{Opportunist} page \pageref{opportunista})\\.

To enhance the monsters and make them more effective you can decide that each monster has a \hyperlink{riduzionedeldanno}{Damage Reduction} equal to half its Challenge Rating ( $\frac{GS}{2}$/- )

\end{narrator}\end{changemargin}

\subsubsection{Multiattack}

A creature that can make multiple attacks during its round has the Multiattack ability. The Multiattack Action consumes 2 Actions even if it carries more than 2 attacks. Each Attack follow rules of \hyperlink{attacchimultiplimischia}{multiattack} (page \pageref{attacchimultiplimischia}).

Other Attack Actions listed under Multiattack but not part of those listed in the Multiattack description cost 1 Actions.\index{Multiattack}


\subsubsection{Grabbing Rules for Monsters}

Many monsters have a special attack that allows them to quickly grab their prey. When a monster hits with such an attack, it doesn't have to make another ability check to determine if the grab succeeds, unless the attack says otherwise.

A creature grabbed by the monster can use an action to attempt to escape. To do so, he must succeed on an opposed Strength check (Fortitude save with Strength bonus) against the vanishing DC in the monster's stat block. If no escape DC is provided, assume the DC equals the monster's Fortitude + Strength save bonus.

\subsubsection{Ammunition}

A monster carries enough ammunition to make its ranged attacks. You can assume that a monster has 2d4 projectiles for an attack with thrown weapons (javelins, boulders...), and 2d10 projectiles for a projectile weapon such as a bow or crossbow.

\subsubsection{Reactions}

If a monster can do something special with its reactions, it's listed here. If a creature has no special reactions, this section is absent.


\subsubsection{Limited Use}

Some special abilities have restrictions on the number of times they can be used.

\textbf{\textit{X/Day}.} The notation ``X/Day'' indicates a special ability that can be used X times before the break of dawn to recover expended uses. For example, ``1/Day'' indicates a special ability that can be used once before the monster has to wait for the new dawn.

\textit{\textbf{Recharge X-Y.}} The notation ``Recharge X-Y'' indicates that the monster can use a special ability once and that the ability has a random chance to recharge each round thereafter of fighting. At the start of each monster's round, roll a d6. If the result is one of the numbers in the reload annotation, the monster recovers the use of the special ability. The ability also recharges at the dawn of a new day.

%\begin{center}
%\includegraphics[width=0.6\linewidth]{immagini/cupido.png}
%
%\textit{Eros con il suo arco. Musei Capitolini}
%\end{center}

For example, "Cooldown 5-6" indicates that a monster can use its special ability once. Then, at the start of the monster's round, it regains use of the ability if it rolls a 5 or 6 on a d6.

\begin{center}
\includegraphics[width=0.55\linewidth]{immagini/polpo.png}

\textit{Alphonse de Neuville - Hetzel edition of 20000 Lieues Sous les Mers}
\end{center}


\subsection{Equipment}

The stat block refers to the equipment, other than the weapons or Armour used by the monster. A creature that normally wears clothing, such as a humanoid, is assumed to be dressed appropriately.

You can equip monsters with further equipment however you like, using chapter \hyperlink{equipaggiamento}{Equipment} as a source of inspiration, and you decide how much of the monster's equipment is salvageable after the creature is slain or if any of its equipment is still usable. For example, dented Armour made for one monster is unlikely to be usable by anyone else. If a spellcasting monster requires material components to cast its spells, assume it has the material components to cast spells in its stat block.

\subsection{Additional Actions}

Certain creatures can perform special actions outside their rounds, and some can extend their power into the environment, causing extraordinary magical effects to occur in their vicinity.

A creature with extra actions can take a number of special actions -- called extra actions -- outside of its round. Only one additional action can be used at a time, and only at the end of another creature's round. A creature with additional actions regains the additional actions she used at the start of her round. She is not required to use her additional actions and may not use the additional actions while incapacitated or otherwise unable to perform actions. If surprised, she cannot use them until after her first round of combat.

If a creature takes the form of a creature with additional actions, perhaps through a spell, it does not gain the additional actions, the lair actions.

\subsubsection{A Creature's Lair}

A creature with additional actions may have a section describing its lair and the special effects it can create there while it's there, either by its own will or simply by its presence. This section applies only to legendary creatures that spend a lot of time in their lairs and are highly likely to be encountered there.

\subsubsection{Lairs Actions}

If a creature with additional actions has a lair action, it can use it to harness the ambient magic of its lair. On initiative count 10, losing ties, the creature can use one of its lair action options. She cannot do this while incapacitated or otherwise unable to perform actions. If surprised, she cannot use it until after her first round of combat.

\subsection{Types of Treasury}

Each type of creature can prefer a different type of treasure (intended as objects, coins, gems...). These are just hints on how to build the monster's treasure.

\medskip

\begin{itemize}[leftmargin=*]

\item \textbf{Aberration}
Many aberrations have little regard for treasure, possessing only what they take from the remains of their previous victims. Others are cunning adversaries who use various magical items and treasures to enhance their abilities.

\item \textbf{Animal}
Animals don't care for treasures at all, instead leaving coins and items with the remains of their meals. For those with treasure, it is typically found in their lairs, scattered among bones and other scraps.

\item \textbf{Magical Beast}
Caring little for values, most magical beasts are solely in search of their next meal. The hideouts of these creatures are often littered with precious trinkets and magical items.

\item \textbf{Construct}
The only treasure carried by constructs is usually part of the construct itself, such as a weapon or magical item. Constructs, however, are typically used to guard treasure or more valuable magical items.

\item \textbf{Dragon}
Known for their precious treasures, dragons often brood over piles of coins, gems, magical items, and other expensive items.

\item \textbf{External}
Outsiders are among the most diverse of creature types, and as a result they could really have all sorts of treasure on them or hidden in their lairs. The Arbiter should consider the individual creature when determining what type of treasure best suits that outsider.

\item \textbf{Leprechaun}
Above all else, fey value beautiful and magical objects. They have little regard for the instruments of exchange and commerce used by the more civilized races, such as coins and valuables.

\item \textbf{Slime}
Oozes have no conception of such a thing as treasure and leave behind all they find in their search for the next meal. Any treasure they may carry is completely accidental.

\item \textbf{Undead}
The treasures carried by the undead vary depending on whether or not it is an intelligent creature. Mindless undead typically have only the meager valuables they carried with them in life, rarely truly usable as treasure, while intelligent ones harness a wide variety of magical items to destroy the living.

\item \textbf{Parasite}
Like other mindless creatures, parasites do not covet treasure, although these creatures are sometimes found to infest areas where valuables are kept.

\item \textbf{Humanoid}
Creatures of this type are very diverse, but even the most primitive humanoids use magical equipment and items to some extent. In larger groups, such as communities, humanoids often possess a great deal of treasure that they collectively guard.

\item \textbf{Vegetable}
Like animals, plant creatures disregard treasure, and anything that might be found where they grow is simply the undigested remains of a previous victim.

\end{itemize}

\subsection{Experience Points per CR}

If \textit{defeated} each monster grants a certain amount of Experience Points to be divided among all the participants in the fight.

This table indicates the relative Experience Points for GS.

\medskip


\textbf{Table: Experience Points by Challenge Level}\index{Table of Experience Points by Challenge Level}

\medskip

\begin{tabularx}{0.42\textwidth}{ll|ll|ll}
	
\textbf{CR} & \textbf{PX} &\textbf{CR} & \textbf{PX} &\textbf{CR} & \textbf{PX}\\
0 & 10 & 9 & 5000 & 21&33000\\
1/8 & 25 & 10 & 5900 & 22&41000\\
1/4 & 50 & 11 & 7200 & 23&50000\\
1/2 & 100 & 12 & 8400 & 24 & 62000\\
1 & 200 & 13 & 10000 & 25 & 75000\\
2 & 450 & 14 & 11500 & 26 & 90000\\
3 & 700 & 15 & 13000 & 27 & 105000\\
4&1100&16&15000&28&120000\\
5&1800&17&18000&29&135000\\
6&2300&18&20000&30&155000\\
7 & 2900 & 19 & 22000 & &\\
8 & 3900 & 20 & 25000 & &\\
\end{tabularx}

\subsubsection{Optional - Experience per Challenge}\index{Optional - Experience per Challenge}

An alternative approach to the experience is not to reward the character based on the \textit{monster defeated} but based on the relative challenge incurred.
With this system, Experience Points are given based on relative difficulty, i.e. a Troll (Challenge 5, 1800 XP) does not always give 1800 XP but if faced by a low level group it will give more while faced by a high level group he will give less.

With this system, every 1000 Experience Points you level up. All the considerations in the Mastering chapter to prepare for battles apply.

\medskip

\textbf{Table: Experience Points by Level of Challenge}\index{Table of Experience Points by Level of Challenge}

\begin{tabular}{ll|ll}
\textbf{Degree of Challenge} & \textbf{PX}&\textbf{Degree of Challenge} & \textbf{PX}\\
\toprule
Easy & 20 & Medium & 30\\
High & 50 & Extraordinary & 80\\
Deadly & 120 & Epic & 160\\
\end{tabular}

\medskip

This system is also used to calculate the XP earned for traps or challenges that have been overcome. The Experience Points reward for each personal or group goal achieved are 10.


\end{multicols}

\vfill

\begin{changemargin}{0cm}{0.5cm}\begin{emphasis}{
I am the monster that breathing men would kill. I am Dracula. (Bram Stoker's Dracula)}\end{emphasis}\end{changemargin}\medskip


\pagebreak

\subsection{The Monsters}


\begin{changemargin}{0.3cm}{0.3cm}\begin{narrator}
The creatures presented here are meant to be a full-bodied example of the enemies your friends might encounter. Attention, it is not said that they are all enemies or necessarily that they have negative intentions.

More civilized creatures will have their own individual ethical and moral conduct, even within the same group of "enemies" there are those who could be more "enemy" or simply indifferent.

Take advantage of the peculiarities and uniqueness of the creatures to create unexpected and tactically challenging encounters. Don't be obvious but not even absurd in your choices, there must always be consistency in choosing creatures.
\end{narrator}\end{changemargin}


\bigskip

\begin{changemargin}{0cm}{0.5cm}\begin{emphasis}{

Amon Goth: Control is power. This is power.

Oskar Schindler: Is that why they fear us?

Amon Goth: We have the power to kill. That's why they fear us.

Oskar Schindler: They fear us because we have the power to kill wantonly. A man commits a crime, he had to think about it, we have him killed and we feel at peace. Or we kill it ourselves and feel even better. That's not power though! This is justice, it is different from power. Power is when we have every justification to kill and we don't.

Amon Goth: Is this the power?

Oskar Schindler: Emperors had this. A man steals something, is brought before the emperor and falls on the ground trembling, begging for mercy. He is aware that he is about to leave. And the emperor forgives him instead. That man, undeserving, sets him free.

(Schindler's list - Film, 1993)
}\end{emphasis}\end{changemargin}\medskip



\bigskip

\begin{multicols}{2}
	
\index[Monsters]{Aboleth}\textbf{Aboleth}

\textit{Large aberration, lawful evil}

\textbf{STRENGTH} +5

\textbf{DEXTERITY} -1

\textbf{CONSTITUTION} +2

\textbf{INTELLIGENCE} +4

\textbf{WISDOM} +2

\textbf{CHARISMA} +4

\textbf{Initiative} +4 -- \textbf{Defence} 22

\textbf{Hit Points} 135 (18d10 + 36)

\textbf{Movement} 3m, swim 12m

\textbf{Saving Throws} Fortitude +14, Reflexes +9, Will +12

\textbf{Skills} Awareness +10, History +12

\textbf{Senses} darkvision 40m

\textbf{Languages} Deep language, telepathy 36m

\textbf{Challenge} 10 (5900 XP)

\textit{\textbf{Amphibious.}} The aboleth can breathe air and water.

\textit{\textbf{Cloud of Mucus.}} While underwater, the aboleth is enveloped in mutant mucus. A creature that touches the aboleth, or that hits it with a melee attack while within 1 meter of it, must make a DC 19 Fortitude save. On a failed save, the creature is diseased for 1d4 hours. . The diseased creature can only breathe underwater.

\textit{\textbf{Telepathic Probe.}} If a creature communicates telepathically with the aboleth, and the aboleth can see it, the aboleth learns its greatest wishes.

\textbf{Actions}

\textit{\textbf{Multiattack.}} The aboleth makes three tentacle attacks

\textit{\textbf{Tentacle.} Melee Weapon Attack}: +9 to hit, reach 3m, one target.

\textit{Hit:} 12 (2d6 + 5) bludgeoning damage. If the target is a creature, it must succeed on a DC 19 Fortitude save or become diseased. The disease has no effect for 1 minute and can be removed by any magic that cures disease. After 1 minute, the diseased creature's skin becomes transparent and slimy, the creature can't regain Hit Points unless underwater, and the disease can only be removed by \textit{heal} or another cure disease spell level 3 or higher. When the creature is outside a body of water, it takes 6 (1d12) acid damage every 10 minutes unless its skin is wet before the 10 minutes have passed.

\textit{\textbf{Tail.} Melee Weapon Attack}: +9 to hit, reach 3m, one target.

\textit{Hit:} 15 (3d6 + 5) bludgeoning damage.

\textit{\textbf{Enslave (3/Day).}} The aboleth targets one creature it can see within 10 meters of it. The target must succeed on a DC 19 Will save or be magically charmed by the aboleth until the aboleth dies or the two are on different planes of existence. The charmed target is under the aboleth's control and cannot take reactions. The aboleth and the target can communicate with each other telepathically at any distance.

Whenever the charmed target takes damage, it can repeat the Saving Throw. If it succeeds, the effect ends. No more than once every 24 hours, it can repeat the Saving Throw when it is at least 1 mile away from the aboleth.

\textbf{Additional Actions}

The aboleth can perform 3 additional Actions, chosen from the options below. It can only use one Additional Action at a time, and only at the end of another creature's turn. The aboleth regains expended additional Actions at the start of its round.

\textbf{Locate.} The aboleth makes a Wisdom (Awareness) check.

\textbf{Psychic Drain (Costs 2 Actions).} A creature charmed by the aboleth takes 10 (3d6) damage, and the aboleth regains a number of Hit Points equal to the damage taken by the creature.

\textbf{Tail Sweep.} The aboleth makes a tail attack.

\textbf{Ecology}\\
Environment any aquatic\\
Organization: Solitary, Pair, Brood (3-6) or Pack (7-19)\\
\textbf{Treasure}: Double\\
\textbf{Description}\\
As their primitive appearance suggests, hermaphrodite aboleths are among the oldest life forms in the world. Already ancient when the gods began to take an interest in the Material Plane, the aboleths have always lived apart from other mortals: they are alien, cold and always busy weaving planes. They once ruled the world in a vast empire, and today they view other life forms as food or slaves…sometimes both at the same time. They despise the gods, as they believe they are the true masters of creation, an aboleth is 21 meter long and weighs approximately 8.5 tons. In the darkest depths of the sea, the aboleths still dwell in their grotesque, cyclopean, nauseating cities. They are served by countless slaves taken from every land and sea nation, and those on land are doubly enslaved by their masters and their slime, which allows them to breathe underwater, aboleths encountered alone are usually scouts from these hidden cities, looking for new slaves.

\

\textbf{Dragon, Ancient Yellow}\index[Monsters]{Ancient Yellow Dragon}\\
\textit{Gargantuan Dragon, Neutral Evil}\\
\textbf{Strength}: +10\\
\textbf{Dexterity}: +1\\
\textbf{Constitution}: +8\\
\textbf{Intelligence}: +3\\
\textbf{Wisdom}: +2\\
\textbf{Charisma}: +4\\
\textbf{Defence}: 27 (natural Armour) - \textbf{Initiative}: +4\\
\textbf{Hit Points}: 481 (26x3d6 + 208)\\
\textbf{Movement}: 12m, dig 24m, climb 24m, fly 12m\\
\textbf{Saving Throws}: Fortitude +21, Reflexes +13, Will +19\\
\textbf{Skills}: Crime +7, Awareness +17\\
\textbf{Damage Immunity}: electricity\\
\textbf{Senses}: Darkvision 36m, blindsight 18m\\
\textbf{Languages} Common, Draconic\\
\textbf{Challenge}: 23 (50000 XP)\smallskip\\
\textit{\textbf{Legendary Endurance (3 / Day).}} If the dragon fails a Saving Throw, it may choose to succeed instead. \\
\smallskip\textbf{Actions}\\
\textit{\textbf{Multiattack.}} The dragon can use its Frightening Presence. Then make three attacks: one with the bite and two with the claws.\\
\textit{\textbf{Claw.} Melee weapon attack}: +30 on hit, reach 3m, one target.\\
\textit{Hit:} 16 (2d6 + 9) slashing damage, 3 bleed damage (to a maximum of 20).\\
\textit{\textbf{Tail.} Melee Weapon Attack}: +30 on hit, reach 6m, one target.\\
\textit{Hit:} 18 (2d8 + 9) bludgeoning damage.\\
\textit{\textbf{Bite.} Melee Weapon Attack}: +30 on hit, reach 5 meters, one target.\\
\textit{Hit:} 20 (2d10 + 9) piercing damage plus 11 (2d10) electricity damage.\\
\textit{\textbf{Fearful Presence.}} Any creature of the dragon's choice that is within 16 meters of it and aware of its presence must succeed at a DC 25 Will save or be frightened for 1 minute. A creature can repeat the Saving Throw at the end of each of its rounds, ending the effect on a successful one. If the creature's Saving Throw succeeds or the effect ends for it, the creature is immune to the dragon's Frightening Presence for the next 24 hours.\\
\textit{\textbf{Fire Breath (Cooldown 5-6).}} The dragon exhales scorching air in a line 36 meters long and 3 meters wide. Each creature in that line must make a DC 25 Reflex save and take 88 (16d10) fire damage on a failed save, or half as much damage on a successful one. \\
\textbf{Additional Actions}\\
The dragon can perform 3 additional actions, chosen from the options below. It can only use one Additional Action at a time, and only at the end of another creature's round. The dragon regains expended Additional Actions at the start of its round.\\
\textbf{Wing Attack (Costs 2 Actions).} The dragon flaps its wings. Each creature within 5 meters of the dragon must succeed on a DC 25 Reflex save or take 16 (2d6 + 9) bludgeoning damage and be knocked prone. The dragon can then fly up to half its flying speed.\\
\textbf{Tail Attack.} The dragon performs a tail attack.\\
\textbf{Locate.} The dragon makes a Wisdom (Awareness) check.\\
\textbf{Ecology}\\
Environment: Hot Deserts\\
Organization: Solitary\\
\textbf{Treasure}: Triple\\
\textbf{Description}\\
Yellow dragons have scales of various shades of yellow that as they grow grow more and more like the color of the sands they inhabit, ranging from pale yellow to brick ochre.

They are very intelligent but being solitary by nature, they have no interest in communicating with other races.

They live in deserts where they often ambush their prey by hiding at the bottom of large holes dug in the sand.
As soon as they perceive movement above them, they come out and devour any creature.
They have a fondness for dwarven meat, which they find tasty even when dry.

The Yellow Dragon, although intelligent, is a killing machine and hardly comes to terms, only if it is in serious danger.

A Yellow Dragon has +1d6 on Magic Test and can ignore one die rolled on a Fire List check and is immune to fire.\\

\textbf{Enchantments}\index{Yellow Dragon Spell}\\
This dragon's favorite spells are:\\
- Create Food and Water\\
- Wall of Fire\\
- Fire Shield


\

\index[Monsters]{Angel, Deva}\textbf{Deva, Angel}

\textit{Heavenly Medium, Lawful Good}

\textbf{STRENGTH} +4

\textbf{DEXTERITY} +4

\textbf{CONSTITUTION} +4

\textbf{INTELLIGENCE} +3

\textbf{WISDOM} +5

\textbf{CHARISMA} +5

\textbf{Initiative} +4 -- \textbf{Defence} 22

\textbf{Hit Points} 136 (16d8 + 64)

\textbf{Move} 9m, fly 27m

\textbf{Saving Throws} Fortitude +16, Reflexes +14, Will +15

\textbf{Skills} Sense Emotions +9, Awareness +9

\textbf{Resistance to Damage} from Light; from a non-magical weapon

\textbf{Condition Immunity} fascinated, fatigued, frightened

\textbf{Senses} darkvision 40m

\textbf{Languages} all, telepathy 36m

\textbf{Challenge} 10 (5900 XP)

\textit{\textbf{Angelic Weapons.}} The deva's weapon attacks are magical. When the deva hits with any weapon, the weapon deals an additional 4d8 points of light damage (already included in the attack).

\textit{\textbf{Innate Spells.}} The deva's innate spellcasting ability is Charisma (DC 20 for spell saves). The deva can innately cast the following spells, using only the verbal components:

At will: \textit{detect good and evil}

1/day: \textit{communion, raise dead}

\textit{\textbf{Resistance to Magic.}} The deva has +1d6 on Saving Throws against spells and other magical effects.

\textbf{Actions}

\textit{\textbf{Multiattack.}} The deva makes two melee attacks.

\textit{\textbf{Mace.} Melee Weapon Attack}: +19 to hit, reach 1m, one target.

\textit{Hit:} 7 (1d6 + 4) bludgeoning damage plus 18 (4d8) Light damage.

\textit{\textbf{Healing Touch (3/Day).}} The deva touches another creature. The target magically regains 20 (4d8 + 2) Hit Points and is free from all blindness, disease, curse, deafness, or poison.

\textit{\textbf{Shapeshift.}} The deva can magically transform into a humanoid or beast whose challenge rating is equal to or lower than her own, or revert to her true form. Upon death he reverts to his true form. Any equipment he is wearing or carrying is absorbed or transported into the new form (the deva's choice).

In the new form, the deva retains its game stats and the ability to speak, but its Defence, movement methods, Strength, Dexterity, and special senses are replaced by those of the new form, and it gains any stats or abilities (Additional Actions and lair actions) possessed by his new form and not his original.

\textbf{Ecology}
Environment: Any plane with good Traits\\
Organization: Solitary, pair, or squadron (3-6)\\
\textbf{Treasure}: Double (+1 Flaming Greatsword, other treasure)\\
\textbf{Description}\\
Movanic devas make up the infantry ranks of the celestial armies, though they spend most of their time patrolling the Positive, Negative, and Material Planes. On the Positive Plane they watch over wandering good souls. On the Negative Plane they fight the undead, and other strange beings that hunt in the ravenous void. Their rare sojourns on the Material Plane are usually intended to aid mighty mortals when great peril threatens to plunge an entire realm into the hands of evil.

\

\index[Monsters]{Angel, Planetar}\textbf{Planetar, Angel}

\textit{Large Celestial, Lawful Good}

\textbf{STRENGTH} +7

\textbf{DEXTERITY} +5

\textbf{CONSTITUTION} +7

\textbf{INTELLIGENCE} +4

\textbf{WISDOM} +6

\textbf{CHARISMA} +7

\textbf{Initiative} +5 -- \textbf{Defence} 27

\textbf{Hit Points} 200 (16d10 + 112)

\textbf{Move} 12m, fly 36m

\textbf{Saving Throws} Fortitude +19, Reflexes +11, Will +19

\textbf{Skills} Awareness +11

\textbf{Resistance to Damage} from Light;

\textbf{Condition Immunity} charmed, fatigued, frightened, weapons +1

\textbf{Senses} True Seeing 36 m

\textbf{Languages} all, telepathy 36m

\textbf{Challenge} 16 (15000 XP)

\textit{\textbf{Angelic Weapons.}} The Planetar's weapon attacks are magical. On a hit with any weapon, the weapon deals an additional 5d8 points of Light damage (already included in the attack).

\textit{\textbf{Divine Awareness.}} The Planetar recognizes lies immediately.

\textit{\textbf{Innate Spells.}} The Planetar's innate spellcasting ability is Charisma (DC 24 for spell saves). The Planetar can innately cast the following spells, requiring no material components:

At will: \textit{detect good and evil}, \textit{invisibility} (personal only)

3/day: \textit{blade barrier, Flame Strike, dispel good and bad} \textit{evil, raise dead}

1/day: \textit{communion, control weather, insect plague}

\textit{\textbf{Resistance to Magic.}} The Planetar has +1d6 on Saving Throws against spells and other magical effects.

\textbf{Actions}

\textit{\textbf{Multiattack.}} The Planetar makes two melee attacks.

\textit{\textbf{Greatsword.} Melee Weapon Attack}: +26 to hit, reach 2m, one target.

\textit{Hit:} 21 (4d6 + 7) slashing damage plus 22 (5d8) Light damage.

\textit{\textbf{Healing Touch (4/Day).}} The Planetar touches another creature. The target magically regains 30 (6d8 + 3) Hit Points and is free from all blindness, disease, curse, deafness, or poison.

\textit{\textbf{Enraged}}: 

- the Planetar summons angelic powers to help him. Using 3 Actions he summons 1d4 Deva Angels. 

- the Planetar cause a critical damage (2d6) every time hits with his greatsword till end of fight. Cost 1 Action.


\textbf{Ecology}
Environment: Any plane with good Traits\\
Organization: Solitary or pair\\
\textbf{Treasure}: Double (Holy Greatsword +3)\\
\textbf{Description}\\
Planetars are the generals of celestial armies bent on the destruction of evil. A Planetar is usually 2.7 meters tall and weighs about 250 kg. They are excellent diplomats, but against fiends they would rather wage war than negotiate a peace.


\

\index[Monsters]{Angel, Solar}\textbf{Solar, Angel}

\textit{Large Celestial, Lawful Good}

\textbf{STRENGTH} +8

\textbf{DEXTERITY} +6

\textbf{CONSTITUTION} +8

\textbf{INTELLIGENCE} +7

\textbf{WISDOM} +7

\textbf{CHARISMA} +10

\textbf{Initiative} +7 -- \textbf{Defence} 31

\textbf{Hit Points} 243 (18d10 + 144)

\textbf{Move} 15m, fly 45m

\textbf{Saving Throws} Fortitude +25, Reflexes +14, Will +23

\textbf{Skills} Awareness +14

\textbf{Resistance to Damage} from Light;

\textbf{Immunity to Damage} Void, Poison, Weapons +2

\textbf{Condition Immunity} charmed, poisoned, fatigued, frightened
\textbf{Senses} True Seeing 36 m

\textbf{Languages} all, telepathy 36 m

\textbf{Challenge} 21 (33000 XP)

\textit{\textbf{Angelic Weapons.}} The Solar's weapon attacks are magical. On a hit with any weapon, the weapon deals an additional 6d8 points of Light damage (already included in the attack).

\textit{\textbf{Divine Awareness.}} The Solar recognizes lies immediately.

\textit{\textbf{Innate Spells.}} The Solar's innate spellcasting ability is Charisma (DC 25 for spell saves). The Solar can innately cast the following spells, requiring no material components:

At will: \textit{detect good and evil}, \textit{invisibility} (personal only)

3/day: \textit{blade barrier, Flame Strike, dispel good and evil, resurrection}

1/day: \textit{communion, check weather}

\textit{\textbf{Resistance to Magic.}} The Solar has +1d6 on Saving Throws against spells and other magical effects.

\textbf{Actions}

\textit{\textbf{Multiattack.}} The Solar makes two attacks with its greatsword.

\textit{\textbf{Large Greatsword.} Melee Weapon Attack}: +30 to hit, reach 1m, one target.

\textit{Hit:} 22 (4d6 + 8) slashing damage plus 27 (6d8) Light damage.

\textit{\textbf{Longbow of Slaying.} Ranged weapon attack}: +30 to hit, range 45m, one target.

\textit{Hit:} 15 (2d8 + 6) piercing damage plus 27 (6d8) Light damage. If the target is a creature with 100 Hit Points or fewer, it must succeed on a DC 25 Fortitude save or die.

\textit{\textbf{Flying Sword.}} The Solar releases his greatsword to magically float in an unoccupied space within 1 meter of him. If the solar can see the sword, as a bonus action he can mentally command it to fly up to 15 meters and make an attack against a target or return to the Solar's hand. If the floating sword is the target of an effect, it is considered to be wielded by the solar. If the Solar dies, the floating sword falls to the ground.

\textit{\textbf{Healing Touch (4/Day).}} The Solar touches another creature. The target magically regains 40 (8d8 + 4) Hit Points and is free from all blindness, disease, curse, deafness, or poison.

The Solar can take 3 additional actions, chosen from the options below. He can only use one Additional Action at a time and only at the end of another creature's round. The Solar regains any additional expended actions at the start of its round.

\textbf{Incandescent Blast (Costs 2 Actions).} Solar emits divine magical energy. Each creature of your choice within a 3m radius must make a DC 25 Reflex save, taking 14 (4d6) fire damage plus 14 (4d6) Light damage on a failed save, or half as much on a failed save. succeeds.

\textbf{Blinding Gaze (Costs 3 Actions).} The Solar targets one creature within 10 meters that it can see. If the target can see Solar, the target must succeed at a DC 23 Fortitude save or be blinded until a spell such as \textit{lesser restoration} removes the blindness.

\textbf{Teleport.} The Solar magically teleports itself up to 16 meters away, along with any equipment it is wearing or carrying, to an unoccupied space that it can see.

\textbf{Ecology}\\
Environment: Any plane with good Traits\\
Organization: Solitary or pair\\
\textbf{Treasure}: Double (Full Armour +5, Greatsword Dancing +5, Composite Longbow +5)\\
\textbf{Description}\\
Solars are the most powerful of the angels, usually right-hand men of a deity or champions of causes that benefit an entire world or plane. A Solar is usually almost human in appearance, although some of them resemble other humanoid races and some even have more unusual shapes. A Solar is about 9 feet tall, weighs about 250kg, and has a deep, commanding voice that's impossible to ignore. Most of them have silver or gold skin.

Blessed with a host of more powerful spell-like abilities, Solars are fearsome foes capable of single-handedly killing the most powerful evil creatures. Among celestials they are considered the finest trackers, and the best of them are said to be able to track days-old tracks left by a pit devil across the Astral Plane. Some of them take on the mantle of monster slayers and hunt down powerful fiends and undead such as devourers, night hags, Nightshades, and Pit Fiends, even making raids into evil planes and the Plane of Negative Energy. to destroy these creatures at their source, before they can harm mortals. Some of the older Solars have accomplished their missions, and have a reputation as slayers of now-extinct creatures.

Solars accept the role of guardians, usually of supernatural concepts or objects or creatures of great importance. On one world, a group of Solars protect the sun's energy conduits against attempts by evil races such as the Elves to shut it down and bring eternal darkness. On another, seven Solars watch over seven mystical chains that keep the evil gods imprisoned in a demiplane. On yet another, a Solar with a flaming sword guards the Garden of Eden, preventing all creatures from entering.

In worlds where gods can take physical form, Solars are sent to become prophets and gurus (often in the guise of mortals), thus laying the foundations for cults that would eventually become great religions. In worlds oppressed by evil, Solars are the clandestine priests who bring hope to the oppressed or who allow themselves to be martyred so that their essence can explode in the surrounding regions and grow in the hearts of future heroes.

While not deities, Solars approach that of demigods in power, and they often act as advisors to younger or weaker deities. In some polytheistic faiths, mortals worship one or more Solars as aspects or equal servants of the true deities (though never without the approval of the deity in question) or consider the more famous solars to be children, consorts, or lovers of the true deities (which which, depending on the deity, could be true).

Unlike other angels, most Solars are created as direct servants of the gods, amalgamating good souls and pure divine energy, but increasingly these powerful angels are created through the "promotion" of lesser angels such as devas and Planetars. Rarely does particularly powerful and pure souls ascend directly to solar status. The oldest of them predate the creation of mortals, and are among the earliest creations of the gods. These Solar are champions among their own kind, and have little or no interaction with mortals, focusing instead on abstract concepts such as gravity, entropy, dark matter, and primeval evil.

Solars who spend a lot of time on the Material Plane, especially those who take mortal form, are sometimes the source of aasimar or half-celestial bloodlines in human families, sometimes due to romance, sometimes simply for the closeness of mortals to their celestial emanations. They rarely have direct descendants, and when they do, it is always a mortal mother who carries the child. Although Solars can appear of any gender, the gods have not granted them the ability to bear a child. That's why Solars tend to seek out a mortal lover. Other Solars have little regard for one of their kind who bears a child to a mortal, so solar fathers tend to avoid contact with their offspring, to avoid bringing shame upon themselves. Solars, however, tend to monitor their children from afar and, in times of trouble, help them, albeit in mysterious and unobtrusive ways.

All angels respect the power and wisdom of Solars, and though they tend to work alone, they sometimes command Planetar led armies and act as generals for great raids against the legions of Hell or the hordes of the Abyss.

\

\index[Monsters]{Animated Objects, Animated Armour}\textbf{Animated Armour}

\textit{Average construct, unaligned}

\textbf{STRENGTH} +2

\textbf{DEXTERITY} +0

\textbf{CONSTITUTION} +1

\textbf{INTELLIGENCE} -5

\textbf{WISDOM} -4

\textbf{CHARISMA} -5

\textbf{Initiative} +0 -- \textbf{Defence} 19

\textbf{Hit Points} 33 (6d8 + 6)

\textbf{Move} 7m

\textbf{Saving Throws}: Fortitude +2, Reflexes +0, Will -4

\textbf{Immunity to Damage} Poison

\textbf{Condition Immunity} blinded, charmed, deafened, poisoned, paralyzed, petrified, fatigued, frightened

\textbf{Senses} blindsight 18m (blind beyond this range)

\textbf{Languages} -

\textbf{Challenge} 1 (200 XP)

\textit{\textbf{False Appearance.}} While the Armour remains immobile, it is indistinguishable from normal Armour.

\textit{\textbf{Susceptibility to Anti-Magic.}} The Armour is incapacitated if it is in the area of a \textit{anti-magic field}. If targeted by \textit{dispel} \textit{magic}, the Armour must succeed on a Fortitude save against the spell's save DC or be unconscious for 1 minute.

\textbf{Actions}

\textit{\textbf{Multiattack.}} The Armour makes two melee attacks.

\textit{\textbf{Slam.} Melee Weapon Attack}: +4 to hit, reach 1m, one target.

\textit{Hit:} 5 (1d6 + 2) bludgeoning damage.

\

\index[Monsters]{Animated Objects, Flying Sword}\textbf{Flying Sword}

\textit{Small construct, unaligned}

\textbf{STRENGTH} +1

\textbf{DEXTERITY} +2

\textbf{CONSTITUTION} +0

\textbf{INTELLIGENCE} -5

\textbf{WISDOM} -3

\textbf{CHARISMA} -5

\textbf{Initiative} +2 -- \textbf{Defence} 18

\textbf{Hit Points} 17 (5d6)

\textbf{Move} 0m, fly 15m (float)

\textbf{Saving Throws} Fortitude +1, Reflexes +3, Will -4

\textbf{Immunity to Damage} Poison

\textbf{Immunity to Conditions} blinded, charmed, deafened, poisoned, paralyzed, petrified, frightened

\textbf{Senses} blindsight 18m (blind beyond this range)

\textbf{Languages} -

\textbf{Challenge} 1/4 (50 XP)

\textit{\textbf{False Appearance.}} While the weapon remains motionless and not flying, it is indistinguishable from a normal sword.

\textit{\textbf{Susceptibility to Anti-Magic.}} The sword is incapacitated if it is in the area of a \textit{anti-magic field}. If targeted by \textit{dispel} \textit{magic}, the sword must succeed on a Fortitude save against the spell's save DC or be unconscious for 1 minute.

\textbf{Actions}

\textit{\textbf{Longsword.} Melee Weapon Attack}: +3 to hit, reach 1m, one target.

\textit{Hit:} 5 (1d8 + 1) slashing damage.


\

\index[Monsters]{Animated Objects, Rug of Suffocation}\textbf{Rug of Suffocation}

\textit{Large construct, unaligned}

\textbf{STRENGTH} +3

\textbf{DEXTERITY} +2

\textbf{CONSTITUTION} +0

\textbf{INTELLIGENCE} -5

\textbf{WISDOM} -4

\textbf{CHARISMA} -5

\textbf{Initiative} +2 -- \textbf{Defence} 13

\textbf{Hit Points} 33 (6d10)

\textbf{Move} 3m

\textbf{Saving Throws}: Fortitude +4, Reflexes +2, Will -4

\textbf{Immunity to Damage} Poison

\textbf{Condition Immunity} blinded, charmed, deafened, poisoned, paralyzed, petrified, frightened

\textbf{Senses} blindsight 18m (blind beyond this range)

\textbf{Languages} -

\textbf{Challenge} 2 (450 XP)

\textit{\textbf{False Appearance.}} While the carpet remains motionless, it is indistinguishable from a normal carpet.

\textit{\textbf{Susceptibility to Anti-Magic.}} The carpet is incapacitated while in the area of a \textit{anti-magic field}. If targeted by \textit{dispel} \textit{magic}, the carpet must succeed on a Fortitude save against the DC of the caster's save or be knocked unconscious for 1 minute.

\textit{\textbf{Damage transfer.}} While grappling a creature, the carpet takes only half the damage dealt to it, and the creature grabbed by the carpet takes the other half.

\textbf{Actions}

\textit{\textbf{Choke.} Melee Weapon Attack}: +5 to hit, reach 1m, Medium or smaller creature.

\textit{Hit:} Creature is grabbed (DC 13 to escape). Until the grab ends, the target is restrained, blinded, and in danger of suffocating, but the carpet cannot suffocate another target. In addition, at the start of each target's turn, the target takes 10 (2d6 + 3) bludgeoning damage.

\

\index[Monsters]{Ankheg}\textbf{Ankheg}

\textit{Large monstrosity, unaligned}

\textbf{STRENGTH} +3

\textbf{DEXTERITY} +0

\textbf{CONSTITUTION} +1

\textbf{INTELLIGENCE} -5

\textbf{WISDOM} +1

\textbf{CHARISMA} -2

\textbf{Initiative} +0 -- \textbf{Defence} 15, 12 while prone

\textbf{Hit Points} 39 (6d10 + 6)

\textbf{Move} 9m, dig 3m

\textbf{Saving Throws}: Fortitude +3, Reflexes +2, Will +3

\textbf{Senses} Darkvision 18m, tremorsense 18m,

\textbf{Languages} -

\textbf{Challenge} 2 (450 XP)

\textbf{Actions}

\textit{\textbf{Bite.} Melee Weapon Attack}: +5 to hit, reach 1m, one target.

\textit{Hit:} 10 (2d6 + 3) slashing damage plus 3 (1d6) acid damage. If the target is a Large or smaller creature, it is grabbed (DC 13 to escape). Until the grab ends, the ankheg can bite only the grabbed creature and has +1d6 on attack rolls against it.

\textit{\textbf{Acid Spray (Cooldown 6).}} The ankheg spits acid in a line 10 meters long and 1 meter wide, as long as it's grappling no creatures. Each creature in that line must make a DC 13 Reflex save, taking 10 (3d6) acid damage on a failed save, or half as much damage on a successful one.

\textbf{Ecology}\\
Environment: Temperate or warm lowlands\\
Organization: Solitary, pair or nest (3-6)\\
\textbf{Treasure}: Accidental\\
\textbf{Description}\\
Ankhegs are an all too common blight on rural areas. This horse-sized burrowing monster generally avoids densely populated areas, but its fondness for the flesh of cattle and humans keeps them away from uninhabited areas. Their favorite habitat is represented by the rural countryside, since the loose soil makes it very easy for them to move around by digging. Larger ankhegs are reported to live in remote deserts, feed on scorpions and camels, and rarely come into contact with civilization (a desert ankheg is a Huge advanced ankheg).

In combat, ankhegs prefer to attack with their bite. Against multiple opponents, an ankheg Grabs one of its targets and attempts to retreat underground. A creature dragged underground can breathe, albeit with difficulty (the ankheg also has to, so the tunnels are quite porous), but is often eaten alive before its companions can rescue it.

Ankhegs dig with their legs and jaws, moving at lightning speed through dirt, sand, and gravel (not rock). A digging ankheg often stops to build tunnels, smearing the walls with a thick oral secretion. If an ankheg wants to build a tunnel while it is digging, it must move at half its burrowing speed. A typical ankheg tunnel is 3 meter high and wide, roughly circular in shape, and 60 to 115 meters long ([1d10+5]×10). Ankheg groups share the same territory and create complex networks of tunnels under the countryside, sometimes creating chasms where too many of them dig at once.

While ankhegs resemble huge insects, they are more intelligent and, with a little time and a good trainer, can become pets or pack animals. The fact that even "tamed" ankhegs tend to spit acid when startled or surprised makes them unsafe in more civilized regions, but among savage races, such as Hobgoblins, Troglodytes, and especially Orcs, they are popular as guardians or even pets. living room. An ankheg can reach a length of 3 meters and weigh around 400 kg.

\

\textbf{Ape}\index[Monsters]{Ape}

\textit{Medium beast, unaligned}

\textbf{STRENGTH} +3

\textbf{DEXTERITY} +2

\textbf{CONSTITUTION} +2

\textbf{INTELLIGENCE} -2

\textbf{WISDOM} +1

\textbf{CHARISMA} -2

\textbf{Initiative} +2 -- \textbf{Defence} 13

\textbf{Hit Points} 19 (3d8 + 6)

\textbf{Movement} 9m, climb 9m

\textbf{Saving Throws}: Fortitude +3, Reflexes +3, Will +2

\textbf{Skills} Acrobatics +5, Awareness +3

\textbf{Languages} -

\textbf{Challenge} 1/2 (100 XP)

\textbf{Actions}

\textit{\textbf{Multiattack.}} The ape makes two punch attacks.

\textit{\textbf{Punch.} Melee Weapon Attack}: +5 to hit, reach 1m, one target.

\textit{Hit:} 6 (1d6 + 3) bludgeoning damage.

\textit{\textbf{Rock.} Ranged Weapon Attack}: +5 to hit, range 8m, one target.

\textit{Hit:} 6 (1d6 + 3) bludgeoning damage.

\

\index[Monsters]{Aquatic Man}\textbf{Aquatic Man}

\textit{Medium Humanoid (Aquatic Man), Neutral}

\textbf{STRENGTH} +0

\textbf{DEXTERITY} +1

\textbf{CONSTITUTION} +1

\textbf{INTELLIGENCE} +0

\textbf{WISDOM} +0

\textbf{CHARISMA} +1

\textbf{Initiative} +1 -- \textbf{Defence} 12

\textbf{Hit Points} 11 (2d8 + 2)

\textbf{Movement} 3m, swim 12m

\textbf{Saving Throws}: Fortitude +3, Reflexes +1, Will -1; +2 vs. Enchantment

\textbf{Skills} Awareness +2

\textbf{Languages} Aquan, Common

\textbf{Challenge} 1/8 (25 XP)

\textit{\textbf{Amphibian.}} Aquatic humans can breathe air and water.

\textbf{Actions}

\textit{\textbf{Spear.} Melee or Ranged weapon attack}: +2 to hit, reach 1m or range 6m, one target.

\textit{Hit:} 3 (1d6) piercing damage, or 4 (1d8) piercing damage when used with two hands to make a melee attack.

\textbf{Ecology}\\
Environment: Temperate oceans\\
Organization: Solitary, patrol (2-6), gang (6-10 plus one 3rd level lieutenant, company (11-60 plus 3 3rd level lieutenants, 2 5th level commanders, 1 7th level commodore and 3-12 Squid\\
\textbf{Treasure}: NPC gear (Trident, Light crossbow with 10 bolts, other treasure)\\
\textbf{Description}\\
Physically, the Fishmen resemble their ancestors, with expressive foreheads, pale skin, dark hair, and purple eyes. They have three thin gills on their necks, but can pass as Human for short periods if they wish.

\

\textbf{Hawk}\index[Monsters]{Hawk}

\textit{Little beast, unaligned}

\textbf{STRENGTH} -2

\textbf{DEXTERITY} +2

\textbf{CONSTITUTION} +0

\textbf{INTELLIGENCE} -4

\textbf{WISDOM} +2

\textbf{CHARISMA} -2

\textbf{Initiative} +2 -- \textbf{Defence} 13

\textbf{Hit Points} 3 (1d6)

\textbf{Move} 3m, fly 18m

\textbf{Saving Throws}: Fortitude +3, Reflexes +4, Will +2

\textbf{Skills} Awareness +4

\textbf{Languages} -

\textbf{Challenge} 0 (10 XP)

\textit{\textbf{Honed Vision.}} The eagle has +1d6 on Wisdom (Awareness) checks based on sight.

\textbf{Actions}

\textit{\textbf{Spurs.} Melee Weapon Attack}: +4 to hit, reach 1m, one target.

\textit{Hit:} 4 (1d4 + 2) slashing damage.

\

\textbf{Awakened Tree}\index[Monsters]{Awakened Tree}

The awakened tree is a normal tree granted by ability magic
feeling and mobility.

\textit{Huge plant, unaligned}

\textbf{STRENGTH} +4

\textbf{DEXTERITY} -2

\textbf{CONSTITUTION} +2

\textbf{INTELLIGENCE} +0

\textbf{WISDOM} +0

\textbf{CHARISMA} -2

\textbf{Initiative} +0 -- \textbf{Defence} 14

\textbf{Hit Points} 59 (7d12 + 14)

\textbf{Move} 6m

\textbf{Saving Throws}: Fortitude +6, Reflexes -1, Will +1

\textbf{Damage Vulnerability} fire

\textbf{Damage Resistances} bludgeoning, piercing

\textbf{Languages} a language known to its creator

\textbf{Challenge} 2 (450 XP)

\textit{\textbf{False Appearance.}} While the tree remains motionless, it is indistinguishable from a normal tree.

\textbf{Actions}

\textit{\textbf{Slam.} Melee Weapon Attack}: +6 to hit, reach 3m, one target.

\textit{Hit:} 14 (3d6 + 4) bludgeoning damage.

\

\textbf{Axe Beak}\index[Monsters]{Axe Beak}

The ax-bill is a large, slender wingless bird with powerful legs, a wedge-shaped beak, and a bad temper.

\textit{Large beast, unaligned}

\textbf{STRENGTH} +2

\textbf{DEXTERITY} +1

\textbf{CONSTITUTION} +1

\textbf{INTELLIGENCE} -4

\textbf{WISDOM} +0

\textbf{CHARISMA} -3

\textbf{Initiative} +1 -- \textbf{Defence} 12

\textbf{Hit Points} 19 (3d10 + 3)

\textbf{Move} 15m

\textbf{Saving Throws}: Fortitude +3, Reflexes +1, Will +1

\textbf{Languages} -

\textbf{Challenge} 1/4 (50 XP)

\textbf{Actions}

\textit{\textbf{Beak.} Melee Weapon Attack}: +4 to hit, reach 1m, one target.

\textit{Hit:} 6 (1d8 + 2) slashing damage.

\

\index[Monsters]{Azer}\textbf{Azer}

\textit{Elemental Average, Lawful Neutral}

\textbf{STRENGTH} +3

\textbf{DEXTERITY} +1

\textbf{CONSTITUTION} +2

\textbf{INTELLIGENCE} +1

\textbf{WISDOM} +1

\textbf{CHARISMA} +0

\textbf{Initiative} +1 -- \textbf{Defence} 18 (natural Armour, shield)

\textbf{Hit Points} 39 (6d8 + 12)

\textbf{Move} 9m

\textbf{Saving Throws} Fortitude +2, Reflexes +1, Will +1

\textbf{Immunity to Damage} Fire, Poison

\textbf{Condition Immunity} poisoned

\textbf{Languages} Ignan

\textbf{Challenge} 2 (450 XP)

\textit{\textbf{Heated Weapons.}} When the Azer strikes with a metal melee weapon, he deals an additional 3 (1d6) fire damage (already included in the attack).

\textit{\textbf{Heated Body.}} A creature that touches the Azer or hits it with a melee attack while within 1 meter of it takes 5 (1d10) fire damage.

\textit{\textbf{Living fire.}} An Azer needs no food, drink, or sleep.

\textit{\textbf{Illumination.}} The Azer sheds bright light in a 3m radius and dim light for an additional 3 meter.

\textbf{Actions}

\textit{\textbf{Warhammer.} Melee Weapon Attack}: +6 to hit, reach 1m, one target.

\textit{Hit:} 7 (1d8 + 3) bludgeoning damage, or 8 (1d10 + 3) bludgeoning damage when used two-handed to make a melee attack, plus 3 (1d6) fire damage.

\textbf{Ecology}\\
Environment any terrain (Plane of Fire)\\
Organization: Solitary, pair, group (3-6), squad (11-20 plus 2 sergeants of 3rd level and 1 leader of 3rd-6th level), or clan (30-100 plus 50\% of non combatants plus 1 3rd level sergeant for every 20 adults, 5 5th level lieutenants and 3 7th level captains)\\
\textbf{Treasure}: Standard (masterwork scale mail, masterwork warhammer, lighthammer, other treasure)\\
\textbf{Description}\\
A proud and industrious Race from the Plane of Fire, the Azers toil in their bronze and brass fortresses, always ready to fight their long and simmering war against the Efreet. The Azers live in a society where every member knows his place. Born with specific duties, usually related to the activities of their father or mother, the Azers dedicate themselves to these occupations throughout their lives. A caste system further keeps Azer society in line. Nobles, who rule without accountability, wear ornate brass kilts as a symbol of their caste, while those of merchants and shopkeepers are durable bronze. Copper kilts are worn by the working caste, consisting of servants, artisans and labourers.

Able to channel heat through metal weapons and tools, the Azers almost never use non-metal weapons, preferring close combat over ranged attacks. They used to take prisoners, taking them back to their fortresses and forcing them to work for them for a year and a day.

More than half a million Azers live in the legendary City of Brass. Most of these unfortunate Azers live a life of slavery under the efreet. Azers enslaved by this Slavery continue to perform their duties without question, preferring to wait for their contracts to conclude or hoping their masters die or are defeated. Commitment to order burns intensely in this Race, to the extent that some of the Azer Slaves act as overseers over their own kind. Outside of the City of Brass, the Azers are free to go about their lives, often in other planar metropolises, crafting items, selling goods, and running taverns.

To the untrained eye the Azers are strikingly similar to each other. They are 1.2 meters tall but weigh 100 kg.

\

\index[Monsters]{B.O.C.}\textbf{B.O.C.}

\textit{Large monstrosity, lawful evil}

\textbf{STRENGTH} +4

\textbf{DEXTERITY} +3

\textbf{CONSTITUTION} +2

\textbf{INTELLIGENCE} -2

\textbf{WISDOM} +1

\textbf{CHARISMA} -1

\textbf{Initiative} +2 -- \textbf{Defence} 17

\textbf{Hit Points} 42 (8d8 + 10)

\textbf{Move} 13m

\textbf{Saving Throws} Fortitude +6, Reflexes +7, Will +5

\textbf{Skills} Stealth +8, Awareness +6

\textbf{Resistance} +4 on Saving Throws to spells from the Illusion list

\textbf{Senses} darkvision 20m, low-light vision 18m

\textbf{Languages} common, can only understand it

\textbf{Challenge} 4 (1100 XP)

\textbf{Actions}

\textit{\textbf{Multiattack.}} The B.O.C makes two claw attacks and one bite attack, or makes two tentacle attacks

\textit{\textbf{Claws.} Melee Weapon Attack}: +6 to hit, reach 3m, one target, 1 Bleed damage.

\textit{Hit:} 7 (1d6 + 4) slashing damage.

\textit{\textbf{Bite.} Melee Weapon Attack}: For each claw that hit the B.O.C it gains +2 on bite hit. +8 to hit, reach 10', one target.

\textit{Hit:} 10 (1d8 + 6) slashing damage.

\textit{\textbf{Tentacles.} Melee Weapon Attack}: Each tentacle can strike up to 6 meters away and each can strike a different target, +6 to hit.

\textit{Hit:} 6 (1d4 + 4) bludgeoning damage

\textit{\textbf{Deflect the light.}} The B.O.C. is constantly affected by an effect that alters its position, each attack roll has -1d6. This penalty is eliminated if you can attack the B.O.C. without using your eyesight to locate it.

The B.O.C. it constantly bends the light around itself, appearing almost a meter away from its real position. This ability is not affected by normal type visions, only truesight, blindsight, or telluric sense can correctly perceive the B.O.C.

\textbf{Ecology}\\
Environment: Hills and forests\\
Organization solitary, pair, or herd (2d4)\\
\textbf{Treasure}: Accidental\\
\textbf{Description}\\
The Black Ops Cat better known by B.O.C. it is a large predatory feline, obviously black in colour. Ferocious, insatiable, he kills for the sake of hunting. He usually acts in packs and is extremely loyal to the group.

\

\textbf{Baboon}\index[Monsters]{Baboon}

\textit{Little beast, unaligned}

\textbf{STRENGTH} -1

\textbf{DEXTERITY} +2

\textbf{CONSTITUTION} +0

\textbf{INTELLIGENCE} -3

\textbf{WISDOM} +1

\textbf{CHARISMA} -2

\textbf{Initiative} +2 -- \textbf{Defence} 13

\textbf{Hit Points} 3 (1d6)

\textbf{Move} 9m, climb 9m

\textbf{Saving Throws}: Fortitude +3, Reflexes +4, Will +1

\textbf{Languages} -

\textbf{Challenge} 0 (10 XP)

\textit{\textbf{Pack tactics.}} The baboon has +1d6 on attack rolls against a creature if at least one of the baboon's allies is within 1 meter of the creature and that ally isn't incapacitated.

\textbf{Actions}

\textit{\textbf{Bite.} Melee Weapon Attack}: +1 to hit, reach 1m, one target.

\textit{Hit:} 1 (1d4 - 1) piercing damage.


\

\index[Monsters]{Banshee}\textbf{Banshee}

\textit{Medium Undead, Chaotic Evil}

\textbf{STRENGTH} -5

\textbf{DEXTERITY} +5

\textbf{INSTITUTION} +0

\textbf{INTELLIGENCE} +1

\textbf{WISDOM} +1

\textbf{CHARISMA} +4

\textbf{Initiative} +5 -- \textbf{Defence} 15

\textbf{Hit Points} 58 (13d8)

\textbf{Move} 0m, fly 18m (float)

\textbf{Saving Throws}: Fortitude +4, Reflexes +9, Will +5

\textbf{Damage Resistances} acid, electricity, fire, sound; magic weapon +1

\textbf{Damage Immunity} Void, Poison, Cold, non magical weapon

\textbf{Condition Immunity} charmed, grabbed, poisoned, restrained, paralyzed, petrified, prone, fatigue, bleeding

\textbf{Senses} Darkvision 18m

\textbf{Languages} elven, common, necril

\textbf{Challenge} 4 (1100 XP)

\textit{\textbf{Location of Life}}. The Banshee senses the presence of creatures other than undead and constructs within a 5 km radius. She knows the general direction they are in, but not their precise location.

\textit{\textbf{Incorporeal Movement}}. The Banshee can move through other creatures and objects as if they were hindering terrain. She takes 5 (1d10) force damage if she ends her round inside an object.

\textit{\textbf{Undead Nature.}} The Banshee needs no air, food, drink, or sleep.

\textit{\textbf{Sensitivity to Light}}. While in sunlight, the Banshee has -1d6 on attack rolls, as well as Wisdom (Awareness) checks based on sight.

\textbf{Actions}

\textit{\textbf{Corrupting Touch}}. Melee Touch Attack vs. Attack: +6 to hit, reach 1m, one target.

\textit{Hit}: 12 (3d6 +2) void damage.

\textit{\textbf{Terrific Face}}. Any non-undead creature within 20 meters of the Banshee that can see it must succeed at a DC 17 Charisma Will save or be frightened for 1 minute. A frightened target can repeat the Saving Throw at the end of each of its rounds, taking a -1d6 if the Banshee is within line of sight; if he succeeds, the effect ends for him. If a target succeeds at its Saving Throw or the effect ends for it, that target is immune to the banshee's dread visage for the next 24 hours.

\textit{\textbf{Lament (1/day)}}. The Banshee emits an ominous wail, as long as she is not exposed to sunlight. This wail has no effect on constructs or undead. Each other creature within 10 meters of her and able to hear her must make a DC 17 Fortitude save; on a failed save, he drops to 0 Hit Points, while on a successful one, he takes 35 (10d6) psychic damage.

\textbf{Ecology}\\
Environment: Any\\
Organization: Solitary\\
\textbf{Treasure}: None\\
\textbf{Description}\\
The Banshee is the enraged spirit of an elf who has betrayed loved ones or has been betrayed herself. Mad with her pain, the Banshee wreaks vengeance upon every living creature (innocent or guilty) with her fearsome touch and her deadly cries.

\

\index[Monsters]{Basilisk}\textbf{Basilisk}

\textit{Medium monstrosity, unaligned}

\textbf{STRENGTH} +3

\textbf{DEXTERITY} -1

\textbf{CONSTITUTION} +2

\textbf{INTELLIGENCE} -4

\textbf{WISDOM} -1

\textbf{CHARISMA} -2

\textbf{Initiative} -1 -- \textbf{Defence} 17

\textbf{Hit Points} 52 (8d8 + 16)

\textbf{Move} 6m

\textbf{Saving Throws}: Fortitude +5, Reflexes +2 Will +2

\textbf{Senses} Darkvision 18m

\textbf{Languages} -

\textbf{Challenge} 3 (700 XP)

\textit{\textbf{Petrifying Gaze.}} If a creature begins its round within 10 meters of the basilisk and the two can see each other, if not incapacitated the basilisk can force the creature to make a DC 13 Fortitude save If the creature fails its Saving Throw, it magically begins to round to stone and is restrained. The creature must repeat the Saving Throw at the end of its next round. If it succeeds, the effect ends. On a failed save, the creature is petrified until freed by the \textit{restoration} \textit{greater} spell or other magic.

A creature that isn't surprised can look away to avoid the Saving Throw at the start of its round. In that case, he cannot see the basilisk until the start of his next round, when he can look away again. If he were to look at the basilisk in the meantime, he should immediately make the Saving Throw.

If the basilisk is within 10 meters of its reflection in bright light and sees it, it mistakes it for a rival and becomes the target of its gaze.

\textbf{Actions}

\textit{\textbf{Bite.} Melee Weapon Attack}: +7 to hit, reach 1m, one target.

\textit{Hit:} 10 (2d6 + 3) piercing damage plus Basilisk poison's.

\textit{Poison.} Basilisk poison's, F, istant, 14, slowed 1/3r.

\textbf{Ecology}\\
Environment: Any\\
Organization: Solitary, pair or colony (3-6)\\
\textbf{Treasure}: Accidental\\
\textbf{Description}\\
The basilisk, often called the "King of Serpents" is an eight-legged reptile of an aggressive disposition that has the ability to turn creatures to stone with its gaze. Legend has it that, like the Cockatrice, the first basilisks were born from eggs laid by snakes and brooded by roosters, but very little in basilisk physiology leaves room for this theory.

Basilisks live in almost any dry environment, from forest to desert, and their skin tends to mirror their surroundings: a desert basilisk can be bronze or brown, while one that lives in forests can be green. switched on. They tend to use caves, burrows or other sheltered areas as a refuge. These shelters are often marked by statues depicting people and animals in natural poses, which are nothing more than the petrified remains of the unfortunate who came across a basilisk.

Basilisks have the ability to consume petrified creatures; the acid produced by their stomachs dissolves and extracts nutrients from the stone, although the process is slow and inefficient, making them sluggish and inert. As a result, basilisks rarely attack or hunt prey that avoid their gaze, relying on their stealth and element of surprise in order not to run out of food. When not waiting for the small mammals, birds, or reptiles that form part of their diet, basilisks spend their time sleeping in burrows. Those brave enough to capture basilisks or hide treasure near them find that these beings can serve as guardians or watchdogs.

An adult basilisk is almost 4 meters long, half of which is taken up by the long tail, and weighs 135 kilos. Some breeds have small curved horns on their noses or small crown-like crests of bony stingers above their heads. Although they are generally solitary creatures that gather only to mate and lay eggs, in particularly dangerous areas they can gather in small groups for protection and attack intruders en masse.

For unknown reasons, weasels and ferrets are immune to the basilisk's gaze, and will sometimes sneak into burrows while the adult is hunting to feed on its young. Some legends narrate that the blood of a basilisk can transform ordinary stones into another material, but probably these are witnesses who have misinterpreted the magical restoration of previously petrified creatures or body parts.

\

\index[Monsters]{Behir}\textbf{Behir}

\textit{Massive Monstrosity, Neutral Evil}

\textbf{STRENGTH} +6

\textbf{DEXTERITY} +3

\textbf{INSTITUTION} +4

\textbf{INTELLIGENCE} -2

\textbf{WISDOM} +2

\textbf{CHARISMA} +1

\textbf{Initiative} +3 -- \textbf{Defence} 23

\textbf{Hit Points} 168 (16d12 + 64)

\textbf{Movement} 15m, climb 12m

\textbf{Saving Throws}: Fortitude +15, Reflexes +14, Will +11

\textbf{Skills} Stealth +7, Awareness +6

\textbf{Damage Immunity} Electricity

\textbf{Senses} darkvision 27m

\textbf{Languages} Draconic

\textbf{Challenge} 11 (7200 XP)

\textbf{Actions}

\textit{\textbf{Multiattack.}} The Behir makes two attacks: one to bite and one to constrict.

\textit{\textbf{Bite.} Melee Weapon Attack}: +16 to hit, reach 3m, one target.

\textit{Hit:} 22 (3d10 + 6) piercing damage.

\textit{\textbf{Constrict.} Melee Weapon Attack}: +16 to hit, reach 1m, Large or smaller creature.

\textit{Hit:} 17 (2d10 + 6) bludgeoning damage plus 17 (2d10 + 6) slashing damage. Target is grappled (DC 16 to escape) If the Behir isn't already constricting another creature, the target is grappled and restrained until the grab ends.

\textit{\textbf{Swallow swallow.}} The Behir makes a bite attack against a Medium or smaller target it is grappling. If the attack hits, the target is engulfed, and the grab ends. The swallowed target is blinded and restrained, has full cover against attacks and other effects outside the behir, and takes 21 (6d6) acid damage at the start of each round of the behir. The Behir can swallow only one creature at a time.

If the Behir takes 30 or more damage in a single round from a creature it swallowed, it must succeed on a DC 18 Fortitude save at the end of that round or vomit the creature, which falls prone in a space within 3 meter of the behir. If the Behir dies, a swallowed creature is no longer restrained by it and can exit the corpse using 1 meter of movement, coming prone.

\textit{\textbf{Breath of Lightning (Cooldown 5-6).}} The Behir exhales lightning in a line 6 meters long and 1 meter wide. Each creature in that line must make a DC 18 Reflex save and take 66 (12d10) lightning damage on a failed save, or half as much damage on a successful one.

\textit{\textbf{Enraged}}: the Behir recharge the Breath of Lightning. Cost 2 Actions.

\textbf{Ecology}\\
Environment: Hills and Hot Deserts\\
Organization: Solitary or pair\\
\textbf{Treasure}: Double\\
\textbf{Description}\\
Instinctive and greedy, the Behir spends much of its time creeping over the sandy hills and desert rocks that make up its territory, hunting down any creature that dares enter its territory. Its six pairs of stout, clawed legs remain folded at its sides much of the time, and it only extends itself in combat to grab foes, Run at a gallop, or climb the slopes of sheer cliffs, their lairs. creatures.

On average the Behir is 12 meters long and weighs about 1800 kg. In addition to the two prominent horns on the head, many have decorative quills at regular intervals along the spine.

While territorial and bestial in its fury, the Behir is neither stupid nor necessarily evil, though due to its self-centeredness and tendency to claim everything in existence as its own, it often comes into conflict with other races. As such, a Behir can be bribed or convinced by intrepid negotiators willing to approach him. In these cases, a behir's tendency to attack first and reason later (or not reason at all) means that anyone trying to strike a deal must have good cause and immediately impress the Behir with a tempting offer.

It is often said that behirs are somehow related to blue dragons, but the true nature of this bond remains a mystery. Many dragons disavow any Bonding and frown on behirs for their low intelligence—an affront that infuriates the already impulsive behir. Because of this, many behirs harbor a grudge against dragons and are ready to attack any dragon that enters their territory.

\

\textbf{Black Bear}\index[Monsters]{Black Bear}

\textit{Medium beast, unaligned}

\textbf{STRENGTH} +2

\textbf{DEXTERITY} +0

\textbf{CONSTITUTION} +2

\textbf{INTELLIGENCE} -4

\textbf{WISDOM} +1

\textbf{CHARISMA} -2

\textbf{Initiative} +0 -- \textbf{Defence} 12

\textbf{Hit Points} 19 (3d8 + 6)

\textbf{Movement} 12m, climb 9m

\textbf{Saving Throws}: Fortitude +4, Reflexes +1, Will +1

\textbf{Skills} Awareness +3

\textbf{Languages} -

\textbf{Challenge} 1/2 (100 XP)

\textit{\textbf{Enhanced sense of smell.}} The bear has +1d6 on Wisdom (Awareness) checks based on smell.

\textbf{Actions}

\textit{\textbf{Multiattack.}} The black bear makes two attacks: one with its bite and one with its claws.

\textit{\textbf{Claws.} Melee Weapon Attack}: +3 to hit, reach 1m, one target.

\textit{Hit:} 7 (2d4 + 2) slashing damage, 1 bleed damage.

\textit{\textbf{Bite.} Melee Weapon Attack}: +3 to hit, reach 1m, one target.

\textit{Hit:} 5 (1d6 + 2) piercing damage.

\

\index[Monsters]{Black Knight}\textbf{Black Knight}

\textit{Medium Undead, Chaotic Evil}

\textbf{STRENGTH} +5

\textbf{DEXTERITY} +1

\textbf{CONSTITUTION} +5

\textbf{INTELLIGENCE} +1

\textbf{WISDOM} +2

\textbf{CHARISMA} +3

\textbf{Initiative} +3 -- \textbf{Defence} 28

\textbf{Hit Points} 171 (18d8+90)

\textbf{Movement} 9 meters

\textbf{Saving Throws}: Fortitude +22, Reflexes +18, Will +20

\textbf{Skills} Intimidate +12, Religion +8, Planes Knowledge +8, Arcane Knowledge +5

\textbf{Damage Resistances} Cold, Electricity

\textbf{Immunity to Damage} Void, Poison; weapons +1

\textbf{Condition Immunity} charmed, poisoned, paralyzed, fatigued, frightened, bleeding

\textbf{Senses} Darkvision 40m

\textbf{Languages} Common, Abyssal, Exspiram

\textbf{Challenge} 18 (20000 XP)

\textit{\textbf{Spells.}} The Black Knight has MP 7. His spellcasting ability is Charisma, +3 to hit on spell attacks. The Black Knight knows the following spells:

level 1 (4 slots): \textit{Command, Arcane Dart, Burning Wave, shield}

level 2 (3 slots): \textit{block person, magic weapon}

level 3 (3 slots): \textit{counterspell, dispel magic, fireball}

level 4 (3 slots): \textit{banishment, Blazing Smite (with 1 automatic crit magic, void damage)}

\textit{\textbf{Undead nature.}} The Black Knight has no need for air, food, drink, or sleep.

\textit{\textbf{Legendary Endurance (1 / Day).}} If the Black Knight fails a Saving Throw, he can choose to succeed instead.

\textit{\textbf{Resistance to Turning.}} The Black Knight has +1d6 on Saving Throws against effects that turn undead.

\textbf{Actions}

\textit{\textbf{Multiattack.} 3 longsword attacks +3}: +27 on hit, reach 1m, up to three different creatures, or 1 sword slash with Corruption

\textit{Hit:} 13 (1d10+5+3) slashing damage + Blazing Smitee (Void damage)

\textit{Corruption:} 15 (1d10+10) slashing damage. The target must make a DC 22 Will save or lose 1/10 of a trait point tied to a good patron, if any.

\textbf{Ecology}\\
Environment: Any\\
Organization: Solitary\\
\textbf{Treasure}: Longsword +3 or full plate +3, rest of equipment disappears on Black Knight's death.\\

\textbf{Description}
Damned to the depths of his soul the Black Knight is the antithesis of the knight of Sumkjr, indeed often born from the corruption of a knight of Sumkjr. Formidable, cunning, tactical opponent, he loves to behave and reason, maliciously, like a person who is still alive.His tactic is to throw the Fireball as soon as possible and then consume the victim with a Blazing Smite.


\

\index[Monsters]{Blazing Skull}\textbf{Blazing Skull}

\textit{Little Undead, Evil Traits}

\textbf{STRENGTH} +0

\textbf{DEXTERITY} +1

\textbf{CONSTITUTION} +1

\textbf{INTELLIGENCE} +1

\textbf{WISDOM} +0

\textbf{CHARISMA} +0

\textbf{Initiative} +1 -- \textbf{Defence} 13

\textbf{Hit Points} 7 (1d8 + 3)

\textbf{Movement} fly 10m

\textbf{Saving Throws}: Fortitude +1, Reflexes +2, Will +1

\textbf{Damage Resistances} Void

\textbf{Damage Immunity} Fire, poison, from non-magical weapon

\textbf{Condition Immunity} charmed, poisoned, paralyzed, fatigued, frightened, bleeding

\textbf{Senses} Darkvision 18m

\textbf{Challenge} 2 (200 XP)

\textit{\textbf{Spells.}} A Flaming Skull can innately cast the following spells.

at will: \textit{Produce Flame}

1/day: \textit{Hail of Kyrin's Fire Acorns hail}

\textit{\textbf{Undead nature.}} The Flaming Skull needs no air, food, drink, or sleep.

\textbf{Ecology}\\
Environment: Any\\
Organization solitary, pair, patrol (2d4)\\
\textbf{Treasure}: none\\

\textbf{Description}

Flaming skulls are created from the corpses of spellcasters skilled in the Fire magic List, via a variant of the raise dead spell.

Used as keepers and torches they are often a first line of Defence in dungeons.

\

\textbf{Blood Hawk}\index[Monsters]{Blood Hawk}

Owing its name to its crimson feathers and aggressive nature, the bloodhawk attacks fearlessly using its pointed beak.

\textit{Little beast, unaligned}

\textbf{STRENGTH} -2

\textbf{DEXTERITY} +2

\textbf{CONSTITUTION} +0

\textbf{INTELLIGENCE} -4

\textbf{WISDOM} +2

\textbf{CHARISMA} -3

\textbf{Initiative} +2 -- \textbf{Defence} 13

\textbf{Hit Points} 7 (2d6)

\textbf{Move} 3m, fly 18m

\textbf{Saving Throws}: Fortitude +3, Reflexes +6, Will +3

\textbf{Skills} Awareness +4

\textbf{Languages} -

\textbf{Challenge} 1/8 (25 XP)

\textit{\textbf{Packing tactics.}} The falcon has +1d6 on attack rolls against a creature if at least one of the falcon's allies is within 1 meter of the creature and that ally isn't incapacitated.

\textit{\textbf{Enhanced Sight.}} The falcon has +1d6 on Wisdom (Awareness) checks based on sight.

\textbf{Actions}

\textit{\textbf{Beak.} Melee Weapon Attack}: +4 to hit, reach 1m, one target.

\textit{Hit:} 4 (1d4 + 2) piercing damage.

\

\textbf{Boar}\index[Monsters]{Boar}

\textit{Medium beast, unaligned}

\textbf{STRENGTH} +1

\textbf{DEXTERITY} +0

\textbf{CONSTITUTION} +1

\textbf{INTELLIGENCE} -4

\textbf{WISDOM} -1

\textbf{CHARISMA} -3

\textbf{Initiative} +0 -- \textbf{Defence} 12

\textbf{Hit Points} 11 (2d8 + 2)

\textbf{Move} 12m

\textbf{Saving Throws}: Fortitude +2, Reflexes +1, Will -1

\textbf{Languages} -

\textbf{Challenge} 1/4 (50 XP)

\textit{\textbf{Charge.}} If the boar moves at least 6 meters directly towards the target and hits with a tusk attack during the same turn, the target takes an additional 3 (1d6) slashing damage. If the target is a creature, it must succeed on a Fortitude save DC 11 or fall prone.

\textit{\textbf{Relentless (Recharges after 1 hour).}} If the boar takes 7 damage or less that would reduce it to 0 Hit Points, it drops to 1 hit point instead.

\textbf{Actions}

\textit{\textbf{Fang.} Melee Weapon Attack}: +3 to hit, reach 1m, one target.

\textit{Hit:} 4 (1d6 + 1) slashing damage.

\

\medskip\index[Monsters]{Bone Blossom}\textbf{Bone Blossom}

\textit{Great undead, unaligned}

\textbf{FORCE} +3

\textbf{DEXTERITY} +2

\textbf{CONSTITUTION} +4

\textbf{INTELLIGENCE} -2

\textbf{WISDOM} -2

\textbf{CHARISMA} -3

\textbf{Initiative} +2 -- \textbf{Defense} 18

\textbf{Hit Points} 105 (6d10 + 64)

\textbf{Movement} 12 m

\textbf{Saving Throws}: Fortitude +10, Reflexes +8, Will +4

\textbf{Vulnerability to Damage} from Void

\textbf{Immunity to Damage} Poison

\textbf{Damage Resistances} piercing, slashing, from Light

\textbf{Immunity to Conditions} poisoned, fatigued, bleeding, slow

\textbf{Senses} Blindsight 18 m

\textbf{Languages} understands Common, Druidic, Sylvan but cannot speak

\textbf{Challenge} 6 (2300 PX)

\textit{\textbf{One foot in Nature.}} As long as Bone Blossom is in contact with the earth, it regenerates 6 Hit Points at the beginning of its round.

\textit{\textbf{One in Nature.}} As long as Bone Blossom is in a natural environment and doesn't move, it attacks by surprise if not noticed. An Awareness 21 check is required to notice this.

\textit{\textbf{Undead Nature.}} Bone Blossom requires no air, food, drink, or sleep.

\textbf{Actions}

\textit{\textbf{Multiattack}} Bone Blossom can attack with Great Cudgel 3 times or Breath of Spores and perform an attack with Great Cudgel

\textit{\textbf{Great Cudgel.} Melee Weapon Attack}: +9 to hit, reach 2m, one target.

\textit{Hit:} 17 (2d10 + 6) bludgeoning damage

\textit{\textbf{Breath of Spores}}: radius of 6 meters. Bone Blossom releases spores and pollen all around it. Any creature breathing within 20 feet of the Bone Blossom must make a DC 17 Fortitude saving throw. On a failed save, the creature takes 3d8 poison damage and is affected by the \hyperlink{lentezza}{Slow} spell. On a successful save, it takes half damage and is slowed until the end of the next round.

\textit{\textbf{Enraged}}: the Bone Bloom gathers the energies of nature around it by whithering it. Recover 50 Hit Points. Cost 2 Actions.

\textbf{Ecology}\\
Environment: Any Forest\\
Organization solitary, groups (2d12)\\
\textbf{Treasure}: Accidental\\
\textbf{Description}\\
Bone Blossom are creatures that died in the thick of the forest for the most varied reasons. Nature, not wanting to waste anything, animates the creature to make it its defender. At first glance, a Bone Blossom is no different than a collection of colorful lichens, small mushrooms and turf.


\index[Monsters]{Brains Eater}\textbf{Brains Eater}

\textit{Small aberration, chaotic evil}

\textbf{STRENGTH} +1

\textbf{DEXTERITY} +6

\textbf{CONSTITUTION} +5

\textbf{INTELLIGENCE} +3

\textbf{WISDOM} +0

\textbf{CHARISMA} +3

\textbf{Initiative} +10 -- \textbf{Defence} 22

\textbf{Hit Points} 84 (8d8 + 48)

\textbf{Move} 12m

\textbf{Saving Throws} Fortitude +14, Reflexes +14, Will +9

\textbf{Damage Resistance} non-magical weapons, cold, electricity

\textbf{Damage Immunity} Fire

\textbf{Condition Immunity} spells from the Illusion and Charm spell lists

\textbf{Senses} Blindsight 18 m

\textbf{Languages} telepathy 50m

\textbf{Challenge} 9 (3900 XP)

\textit{\textbf{Eyes of Magic.}} The Brains Eater has detect magic always active.

\textit{\textbf{Innate Spells.}} The Brains Eater's spellcasting ability is Charisma. The Brains Eater can innately cast the following spells, requiring no material components:

At Will: \textit{Confusion (single target), Inflict Serious Wounds, Invisibility}

3/day: \textit{cure moderate wounds, Orb of Invulnerability}

\textbf{Actions}

\textit{\textbf{Multiattack.}} The Brains Eater can make 4 attacks, one per claw

\textit{\textbf{Claw.} Melee weapon attack}: +9 to hit, reach 3 ft., one creature.

\textit{Hit:} 3 slashing damage (1d4+1), 1 bleed damage.

\textbf{Special Abilities}

\textit{\textbf{Theft of the body}}

By spending 3 Actions a Brains Eater can become tiny and crawl into the mouth/nose/ears of a helpless or dead creature and get to the brain to feed on it. This is an action that kills the creature. The brain devourer assumes control of the body and can use it at will, as if controlling the victim with a dominate monster spell. The Brains Eater has full access to all defensive and offensive abilities of the host except for spell-like abilities and spells (although the Brains Eater can still use its own spell-like abilities). A host body must not have been dead for more than 1 day for this ability to work, and even after successful occupation the bodies decompose and become useless in 7 days (unless this period is extended with the inviolate rest spell). While the Brains Eater occupies the body, he knows (and can speak) the languages known to the victim and information about his identity and personality of hers, but cannot possess the specific memories and knowledge. Damage dealt to the host body, which has double its original Hit Points, does no harm to the brain devourer, and if the host body is destroyed the brain devourer exits and is stunned for 1 round.

\textbf{Ecology}\\
Environment: Any dungeon\\
Organization: Solitary, Brood (2-6) or Tribe (7-16)\\
\textbf{Treasure}: Double\\
\textbf{Description}\\
A Brains Eater is nothing more than a brain of about 50 cm with 4 powerful clawed feet.

Thought by some to be invaders from another dimension or planet, the sinister Brain-eaters are certainly one of the cruellest races in the world. Unable to feel emotions or wallow in the sins of their own physical pleasure, Brains Eaters are forced to steal bodies to satisfy their gluttony, lust, and cruelty. There are stories that tell of entire underground cities of these creatures wearing bodies as if they were dressed to consume frightening orgies and macabre feasts. Devour Lonely Brains often live in ruins or caves on the fringes of civilized regions in order to make periodic forays into the city to "purchase" an enticing new body.

Shayalia's garden is said to be full of Brains Eaters.

A Brains Eater is 90 cm long and weighs approximately 30 kg.

\

\textbf{Brown Bear}\index[Monsters]{Brown Bear}

\textit{Large beast, unaligned}

\textbf{STRENGTH} +4

\textbf{DEXTERITY} +0

\textbf{CONSTITUTION} +3

\textbf{INTELLIGENCE} -4

\textbf{WISDOM} +1

\textbf{CHARISMA} -2

\textbf{Initiative} +0 -- \textbf{Defence} 12

\textbf{Hit Points} 34 (4d10 + 12)

\textbf{Movement} 12m, climb 9m

\textbf{Saving Throws}: Fortitude +6, Reflexes +2, Will +3

\textbf{Skills} Awareness +3

\textbf{Languages} -

\textbf{Challenge} 1 (200 XP)

\textit{\textbf{Enhanced sense of smell.}} The bear has +1d6 on Wisdom (Awareness) checks based on smell.

\textbf{Actions}

\textit{\textbf{Multiattack.}} The bear makes two attacks: one with its bite and one with its claws.

\textit{\textbf{Claws.} Melee Weapon Attack}: +5 to hit, reach 1m, one target.

\textit{Hit:} 11 (2d6 + 4) slashing damage.

\textit{\textbf{Bite.} Melee Weapon Attack}: +5 to hit, reach 1m, one target.

\textit{Hit:} 8 (1d8 + 4) piercing damage.

\

\index[Monsters]{Bubbling Maw}\textbf{Bubbling Maw}

\textit{Medium Aberration, Neutral}

\textbf{STRENGTH} +0

\textbf{DEXTERITY} -1

\textbf{CONSTITUTION} +3

\textbf{INTELLIGENCE} -4

\textbf{WISDOM} +0

\textbf{CHARISMA} -2

\textbf{Initiative} -1 -- \textbf{Defence} 10

\textbf{Hit Points} 67 (9d8 + 27)

\textbf{Move} 3m, swim 3m

\textbf{Saving Throws} Fortitude +8, Reflexes +4, Will +5

\textbf{Condition Immunity} prone

\textbf{Senses} Darkvision 18m

\textbf{Languages} -

\textbf{Challenge} 2 (450 XP)

\textit{\textbf{Gurgling.}} As long as the maw can see a creature and isn't incapacitated, it speaks incoherent sentences. Any creature that begins its round within 6 meters of the maw and can hear its gurgling must make a DC 11 Will save. On a failed save, the creature can't take reactions until the start of its next round and rolls a d8 to determine what he will do during his round. On a 1 to 4, the creature does nothing. On a 5 or 6, the creature takes no action or bonus action and uses all of its movement to move in a randomly determined direction. On a 7 or 8, the creature makes a melee attack against a randomly determined creature within its reach, or does nothing if unable to make such an attack.

\textit{\textbf{Aberrant terrain.}} Terrain in a 3m radius around the maw is considered difficult terrain. Any creature that begins its round in that area must succeed on a DC 10 Fortitude save or have its movement reduced to 0 until the start of its next round.

\textbf{Actions}

\textit{\textbf{Multiattack.}} The gibberish makes a bite attack and, if able, a Blinding Spit.

\textit{\textbf{Bite.} Melee Weapon Attack}: +3 to hit, reach 3 ft., one creature.

\textit{Hit:} 17 (5d6) piercing damage. If the target is Medium or smaller, it must succeed on a DC 11 Fortitude save or be knocked prone. If the target is killed by this damage, it is absorbed into the maw.

\textit{\textbf{Blinding Spit (Cooldown 5-6).}} The Maw spits a chemical orb at a visible point within 5 meters of it. The orb explodes on impact in a blinding flash of light. Each creature within 1 meter of the flash must succeed on a DC 13 Reflex save or be blinded until the end of the maw's next round.

\textbf{Ecology}\\
Environment any dungeon\\
Organization: Solitary\\
\textbf{Treasure}: Standard\\
\textbf{Description}\\
Disgusting, nauseating, and hungry - these are the only words that adequately describe the bubbling maw. Loathsome beasts that lurk in caves, sewers, and nightmares, maws have no other social, ecological, or religious significance other than their ability to drive those who hear them mad. Some scholars believe that the gurgling jaws are a smaller variant of the much more dangerous shoggoth, while others theorize that it is a punishment from some powerful entity or deity inflicted on those who have wronged it.

\

\index[Monsters]{Bugbear}\textbf{Bugbear}

\textit{Medium humanoid (goblinoid), chaotic evil}

\textbf{STRENGTH} +2

\textbf{DEXTERITY} +2

\textbf{CONSTITUTION} +1

\textbf{INTELLIGENCE} -1

\textbf{WISDOM} +0

\textbf{CHARISMA} -1

\textbf{Initiative} +2 -- \textbf{Defence} 17

\textbf{Hit Points} 27 (5d8 + 5)

\textbf{Move} 9m

\textbf{Saving Throws} Fortitude +2, Reflexes +3, Will +1

\textbf{Skills} Stealth +6, Survival +2

\textbf{Senses} Darkvision 18m

\textbf{Languages} Common, Goblin

\textbf{Challenge} 1 (200 XP)

\textit{\textbf{Surprise Attack.}} If the Bugbear surprises a creature and hits it with an attack during the first round of combat, the target takes an additional 7 (2d6) damage
from the attack.

\textit{\textbf{Brute.}} A melee weapon deals an additional die of damage when the Bugbear hits with it (already included in the attack).

\textbf{Actions}

\textit{\textbf{Shocked Mace.} Melee Weapon Attack}: +4 to hit, reach 1m, one target.

\textit{Hit:} 11 (2d8 + 2) piercing damage.

\textit{\textbf{Javelin.} Melee or Ranged weapon attack}: +4 to hit, reach 1m or range 12m, one target.

\textit{Hit:} 9 (2d6 + 2) piercing damage in melee or 5 (1d6 + 2) piercing damage within range.

\textbf{Ecology}\\
Environment: Temperate Mountains\\
Organization: Solitary, pair, group (3-6), or warband (7-12 plus 2 1st-level Fighters and 1 3rd-5th level captain)\\
\textbf{Treasure}: NPC gear (Leather Armour, Light Wooden Shield, Spiked Mace, 3 Javelins, other treasure)\\
\textbf{Description}\\
The Bugbear is the largest of the goblinoid race, a heavy-moving brute that outnumbers most humans by at least a head. They are loners who prefer to live and kill alone rather than in tribes, although it is not uncommon to find a small band of Bugbears teaming up or living with a tribe of Goblins or Hobgoblins serving as elite guards or executioners.

Bugbears don't form large settlements like goblins or nations like hobgoblins; they prefer something smaller and more chaotic that leaves them free to do what they like (killing and torturing) on a more personal level. Humans are bugbears' favorite prey, and most bugbears count human flesh as a staple of their diets. Gruesome trophies such as ears and fingers are common decorations among bugbears.

Bugbears, when turning to religion, favor deities of murder and violence, with various demon lords being favorites. A typical Bugbear stands 2.1 meters tall and weighs 200 kg.

\

\index[Monsters]{Bulette}\textbf{Bulette}

\textit{Large beast, unaligned}

\textbf{STRENGTH} +4

\textbf{DEXTERITY} +0

\textbf{CONSTITUTION} +5

\textbf{INTELLIGENCE} -4

\textbf{WISDOM} +0

\textbf{CHARISMA} -3

\textbf{Initiative} +0 -- \textbf{Defence} 20

\textbf{Hit Points} 94 (9d10 + 45)

\textbf{Movement} 12m, digging 12m

\textbf{Saving Throws}: Fortitude +10, Reflexes +5, Will +5

\textbf{Skills} Awareness +6

\textbf{Senses} Darkvision 18m, tremorsense 18m

\textbf{Languages} -

\textbf{Challenge} 5 (1800 XP)

\textit{\textbf{Standing Jump.}} A Bulette can jump up to 10 meters long and up to 12 meters high with or without a running start.

\textbf{Actions}

\textit{\textbf{Bite.} Melee Weapon Attack}: +11 to hit, reach 1m, one target.

\textit{Hit:} 30 (4d12 + 4) piercing damage.

\textit{\textbf{Leap Jump.}} If the Bulette can jump at least 5 meters as part of its move, it can then use this action to land on its feet in a space containing one or more creatures. Each such creature must succeed on a DC 16 Fortitude or Reflex save (target's choice) or be knocked prone and take 14 (3d6 + 4) bludgeoning damage plus 14 (3d6 + 4) slashing damage. On a successful save, the creature takes only half damage, is not knocked prone, and is pushed 1 meter outside the Bulette's space into an unoccupied space of the creature's choosing. If there are no unoccupied spaces within range, the creature falls prone in the Bulette's space.

\textit{\textbf{Enraged}}: Last Energies: The creature regains three times its CR in Hit Points. Cost 1 Action.

\textbf{Ecology}\\
Environment: Temperate Hills\\
Organization: Solitary or pair\\
\textbf{Treasure}: None\\
\textbf{Description}\\
The creation of an unknown wizard of the past, the Bulette has now become a ferocious hill raider. Burrowing rapidly under the ground, it cleaves the surface with its dorsal fin leaving a distinctive trail in its wake. The Bulette leaps out, freeing itself from stones and dirt, to tear apart its prey without remorse, thus giving rise to its nickname of "land shark".

Bulettes are notorious for their bad temper, and they fearlessly attack creatures much larger than themselves. Solitary beasts except for the occasional breeding pair, they spend most of their time patrolling their ranges, which can exceed 4 km2, hunting and punishing intruders with a hillside-shaking fury.

Bulettes are perfect machines for devouring and destroying bone, Armour, and even magical items with their powerful jaws and bubbling acid stomachs. Failing that, a Bulettes might munch on mundane objects, but for some reason it doesn't willingly eat the flesh of elves, perhaps a sign of elven magic involved in their creation, or of dwarves, though it can slaughter members of both. Stingray. Halflings, on the other hand, are among the beasts' favorite foods, and no sane halfling would venture into Bulette territory lightly.

The Bulette is a cunning fighter, surprising enemies with impressive agility. One of his favorite tactics is to charge and pounce on his prey and attack with his razor-sharp claws. The flesh behind the dorsal crest of the beast is said to be particularly tender, and that those willing or able to wait for the fin to be raised in the heat of combat or mating may attempt to deliver a killing blow there, even if the Most of those who have faced a land shark agree that the best way to win a fight with a Bulette is to avoid it altogether.

\

\textbf{Camel}\index[Monsters]{Camel}

\textit{Large beast, unaligned}

\textbf{STRENGTH} +3

\textbf{DEXTERITY} -1

\textbf{CONSTITUTION} +2

\textbf{INTELLIGENCE} -4

\textbf{WISDOM} -1

\textbf{CHARISMA} -3

\textbf{Initiative} -1 -- \textbf{Defence} 10

\textbf{Hit Points} 15 (2d10 + 4)

\textbf{Move} 15m

\textbf{Saving Throws}: Fortitude +5, Reflexes +6, Will +0

\textbf{Languages} -

\textbf{Challenge} 1/8 (25 XP)

\textbf{Actions}

\textit{\textbf{Bite.} Melee Weapon Attack}: +5 to hit, reach 1m, one target.

\textit{Hit:} 2 (1d4) bludgeoning damage.

\

\textbf{Cat}\index[Monsters]{Cat}

\textit{Tiny beast, unaligned}

\textbf{STRENGTH} -4

\textbf{DEXTERITY} +2

\textbf{CONSTITUTION} +0

\textbf{INTELLIGENCE} -4

\textbf{WISDOM} +1

\textbf{CHARISMA} -2

\textbf{Initiative} +2 -- \textbf{Defence} 13

\textbf{Hit Points} 2 (1d4)

\textbf{Movement} 12m, climb 9m

\textbf{Saving Throws}: Fortitude +1, Reflexes +4, Will +1

\textbf{Skills} Stealth +4, Awareness +3

\textbf{Languages} -

\textbf{Challenge} 0 (10 XP)

\textit{\textbf{Enhanced sense of smell.}} The cat has +1d6 on Wisdom (Awareness) checks based on smell.

\textbf{Actions}

\textit{\textbf{Claws.} Melee Weapon Attack}: +0 to hit, reach 1m, one target.

\textit{Hit:} 1 slashing damage.

\

\index[Monsters]{Centaur}\textbf{Centaur}

\textit{Large monstrosity, neutral good}

\textbf{STRENGTH} +4

\textbf{DEXTERITY} +2

\textbf{CONSTITUTION} +2

\textbf{INTELLIGENCE} -1

\textbf{WISDOM} +1

\textbf{CHARISMA} +0

\textbf{Initiative} +2 -- \textbf{Defence} 13

\textbf{Hit Points} 45 (6d10 + 12)

\textbf{Move} 15m

\textbf{Saving Throws}: Fortitude +4, Reflexes +4, Will +3

\textbf{Skills} Acrobatics +6, Awareness +3, Survival +3

\textbf{Languages} Elven, Sylvan

\textbf{Challenge} 2 (450 XP)

\textit{\textbf{Charge.}} If the centaur moves at least 10 meters directly towards the target and hits with a pike attack during the same turn, the target takes an additional 10 (3d6) piercing damage.

\textbf{Actions}

\textit{\textbf{Multiattack.}} The centaur makes two attacks: one with the pike and one with the hooves or two with the longbow.

\textit{\textbf{Pike.} Melee Weapon Attack}: +6 to hit, reach 3m, one target.

\textit{Hit:} 9 (1d10 + 4) piercing damage.

\textit{\textbf{Hooves.} Melee Weapon Attack}: +6 to hit, reach 1m, one target.

\textit{Hit:} 11 (2d6 + 4) bludgeoning damage.

\textit{\textbf{Longbow.} Ranged weapon attack}: +4 to hit, range 45m, one target.

\textit{Hit:} 6 (1d8 + 2) piercing damage.

\textbf{Ecology}\\
Environment: Temperate plains and forests\\
Organization solitary, pair, warband (3-10), tribe (11-30 plus 3 3rd-level hunters and 1 6th-level leader)\\
\textbf{Treasure}: Standard (plate mail, heavy metal shield, longsword, spear, other treasure)\\
\textbf{Description}\\
Legendary hunters and skilled warriors, centaurs are part man and part horse. Generally located on the fringes of civilization, this stoic population varies greatly in appearance: usually the color of the skin is very tan but similar to that of humans from neighboring regions, while the lower body has the hues of the local equines. They have dark colored hair and eyes and rather marked facial features, while their total size depends on the size of the horse whose lower body they have. Thus, while an average centaur stands 2.1 meters tall and weighs more than 1000 kg, there are multiple regional variants, from slender plainsrunners to massive mountain hunters.

Centaurs live on average about 60 years. Distant from other races and in conflict with others of their kind, the centaurs are an ancient race slowly beginning to accept the modern world. While the majority of centaurs still live in tribes roaming vast plains or on the edges of mystical forests, some have abandoned the isolationist ways of their ancestors to settle in cosmopolitan cities. Often these free spirits are considered outcasts and are despised by their tribes, and therefore the decision to abandon them is a heavy one. In some cases, however, entire tribes led by progressive chieftains have begun to trade or form alliances with other communities of humanoids, especially Elves, sometimes Gnomes, and more rarely Humans or Dwarves. Many races remain wary of centaurs, however, mostly due to legends that portray them as territorial and ferocious creatures and the periodic violent confrontations they have with headstrong settlers and expanding countries.

\

\textbf{Giant Centipede}\index[Monsters]{Centopiedi Gigante}

\textit{Little beast, unaligned}

\textbf{STRENGTH} -3

\textbf{DEXTERITY} +2

\textbf{CONSTITUTION} +1

\textbf{INTELLIGENCE} -5

\textbf{WISDOM} -2

\textbf{CHARISMA} -4

\textbf{Initiative} +2 -- \textbf{Defence} 14

\textbf{Hit Points} 4 (1d6 + 1)

\textbf{Move} 9m, climb 9m

\textbf{Saving Throws}: Fortitude -2, Reflexes +3, Will -2

\textbf{Senses} blindsight 9 m

\textbf{Languages} -

\textbf{Challenge} 1/4 (50 XP)

\textbf{Actions}

\textit{\textbf{Bite.} Melee Weapon Attack}: +4 to hit, reach 1 meter, one creature.

\textit{Hit:} 4 (1d4 + 2) piercing damage and the target must succeed on a DC 11 Fortitude save or take 10 (3d6) poison damage. If the poison damage reduces the target to 0 Hit Points, the target is stable but remains poisoned, -1 Strenght and Dexterity, for 1 hour, even after regaining Hit Points, and while poisoned in this way becomes paralysed.

\

\index[Monsters]{Chimera}\textbf{Chimera}

\textit{Large monstrosity, chaotic evil}

\textbf{STRENGTH} +4

\textbf{DEXTERITY} +0

\textbf{CONSTITUTION} +4

\textbf{INTELLIGENCE} -4

\textbf{WISDOM} +2

\textbf{CHARISMA} +0

\textbf{Initiative} +0 -- \textbf{Defence} 17

\textbf{Hit Points} 114 (12d10 + 48)

\textbf{Move} 9m, fly 18m

\textbf{Saving Throws}: Fortitude +10, Reflexes +6, Will +8

\textbf{Skills} Awareness +8

\textbf{Senses} Darkvision 18m

\textbf{Languages} understands Draconic but cannot speak

\textbf{Challenge} 6 (2300 XP)

\textbf{Actions}

\textit{\textbf{Multiattack.}} The chimera makes three attacks: one with its bite, one with its horns, and one with its claws. When fire breath is available, it can use the breath weapon in place of its bite or horns.

\textit{\textbf{Claws.} Melee Weapon Attack}: +10 to hit, reach 1m, one target.

\textit{Hit:} 11 (2d6 + 4) slashing damage, 1 bleed damage.

\textit{\textbf{Horns.} Melee Weapon Attack}: +10 to hit, reach 1m, one target.

\textit{Hit:} 10 (1d12 + 4) bludgeoning damage.

\textit{\textbf{Bite.} Melee Weapon Attack}: +10 to hit, reach 1m, one target.

\textit{Hit:} 11 (2d6 + 4) piercing damage.

\textit{\textbf{Fiery Breath (Cooldown 5-6).}} Dragon's head exhales fire in a 5-meter cone. Each creature in that area must make a DC 15 Reflex save and take 31 (7d8) fire damage on a failed save, or half as much damage on a successful one.

\textit{\textbf{Enraged}}: Chimera charge own energy. The Fiery Breath is recharged. Cost 1 Action.

\textbf{Ecology}\\
Environment: Temperate Hills\\
Organization: Solitary, pair, herd (3-6), or flock (7-12)\\
\textbf{Treasure}: Standard\\
\textbf{Description}\\
Chimeras are monstrous creatures born of primordial evil. Hateful and ravenous, they hunt both on the ground and in the air. A chimera's dragon head can be any type of evil dragon, with the corresponding breath weapon and wings usually having the same scales as the head. Chimeras speak in three overlapping voices, but they rarely do so, typically only to flatter a more powerful creature. A chimera is 1 meter high at the withers, reaching a length of 3 meters and a weight of 350 kg.\\
Chimeras prefer meat, but can subsist on vegetables if necessary (although when forced to do so, their moods become even worse). The fact that they fly means they can choose their prey carefully, and they generally hunt over large areas looking for the easy ones. They are too stupid and belligerent to acquire followers, although a tribe of kobolds can occasionally make them offerings. On the contrary, they are intelligent and headstrong enough to make mediocre pets, and only a creature far more powerful than themselves can subdue them. They can form equal partnerships with respectful humanoids or similar creatures, and they also agree to be used as mounts. A pride of chimeras has a similar hierarchy to that of lions, with a dominant male commanding the group and most of the hunts performed by females. A solitary chimaera can be a solitary young male or a female with pups nearby.


\

\index[Monsters]{Chuul}\textbf{Chuul}

\textit{Large aberration, chaotic evil}

\textbf{STRENGTH} +4

\textbf{DEXTERITY} +0

\textbf{CONSTITUTION} +3

\textbf{INTELLIGENCE} -3

\textbf{WISDOM} +0

\textbf{CHARISMA} -3

\textbf{Initiative} +0 -- \textbf{Defence} 18

\textbf{Hit Points} 93 (11d10 + 33)

\textbf{Movement} 9m, swim 9m

\textbf{Skills} Awareness +4

\textbf{Immunity to Damage} Poison

\textbf{Condition Immunity} poisoned

\textbf{Senses} Darkvision 18m

\textbf{Languages} understands the Deep Language but cannot speak

\textbf{Challenge} 4 (1100 XP)

\textit{\textbf{Amphibious.}} The Chuul can breathe air and water.

\textit{\textbf{Sense of Magic.}} The Chuul senses magic within 36 meters of itself. This trait functions like the spell \textit{detect} \textit{of magic} but is not magical in itself.

\textbf{Actions}

\textit{\textbf{Multiattack.}} The Chuul makes two attacks with its claws. If the Chuul is grabbing a creature, it can also use its tentacles once.

\textit{\textbf{Claw.} Melee Weapon Attack}: +10 to hit, reach 3m, one target.

\textit{Hit:} 11 (2d6 + 4) bludgeoning damage. A target is grappled (DC 14 to flee) if it is Large or smaller and the Chuul isn't already grappling two other creatures.

\textit{\textbf{Tentacles.}} A creature grabbed by the Chuul must succeed on a DC 14 Fortitude save or be poisoned for 1 minute. Until the poison ends, the target is paralyzed. The target can repeat the Saving Throw at the end of each of its rounds, ending the effect on itself on a success.

\textbf{Ecology}\\
Environment: Temperate Marshes\\
Organization: Solitary, pair, or pack (3-6)\\
\textbf{Treasure}: Standard\\
\textbf{Description}\\
Chuul are Armoured crustacean-like predators, always lurking beneath the surface of shallow pools and bogs, coming out of hiding to grab their prey with their pincers and then paralyze them with their mouth tentacles before eating them alive.

Chuul are excellent swimmers, but prefer to attack creatures on land or those accustomed to shallow water. Once they have grabbed their victims, Chuul often drag them into deep water. Lizardfolk are the chuul's favorite prey, though the pale, subterranean species of Chuul prefer deep gnome, unwary elves, and other unfortunates who get too close to their subterranean streams, except for troglodytes whose flavor Chuul find it particularly disgusting.

Chuul are surprisingly intelligent, and many engage in idle speculation about their origins and motivations. They speak a chittering, gurgling dialect of Common, but few of them are inclined to chat to those outside their race, and if there is a Chuul society outside of the frenzied mating season, no one has yet discovered it. Instead, Chuul minds seem devoted only to finding the perfect ambush location to attack other intelligent creatures, and how to decorate their elaborate lairs with trophies of their victims. While Chuul seem uninterested in using tools, they have a compulsive need to collect those of their victims. A typical Chuul stands 8 feet tall and weighs 350kg.

\

\index[Monsters]{Cockatrice}\textbf{Cockatrice}

\textit{Little monstrosity, unaligned}

\textbf{STRENGTH} -2

\textbf{DEXTERITY} +1

\textbf{CONSTITUTION} +1

\textbf{INTELLIGENCE} -4

\textbf{WISDOM} +1

\textbf{CHARISMA} -3

\textbf{Initiative} +1 -- \textbf{Defence} 12

\textbf{Hit Points} 27 (6d6 + 6)

\textbf{Move} 6m, fly 12m

\textbf{Saving Throws}: Fortitude +1, Reflexes +2, Will +1

\textbf{Senses} Darkvision 18m

\textbf{Languages} -

\textbf{Challenge} 1/2 (100 XP)

\textbf{Actions}

\textit{\textbf{Bite.} Melee Weapon Attack}: +3 to hit, reach 3 ft., one creature.

\textit{Hit:} 3 (1d4 + 1) piercing damage, and the target must succeed on a DC 11 Fortitude save or be Slowed 1/1m by progressive petrification. If subsequent bites cause the creature to have no more Actions, the creature is petrified for 24 hours.

\textbf{Ecology}\\
Environment: Temperate lowlands\\
Organization: Solitary, pair, squadron (3-5), or flock (6-12)\\
\textbf{Treasure}: None\\
\textbf{Description}\\
Stupid, malevolent, and repulsive, cockatrices are shunned by other creatures for their ability to turn flesh to stone. Legends state that the first cockatrice emerged from an egg laid by a rooster and incubated by a toad. Whether this story is true or not, modern-day cockatrices breed among themselves in terrifying, filthy burrows dug haphazardly by at least a dozen clucking creatures. Males greatly outnumber females in these flocks, and are distinguished only by barbels and crests. A typical cockatrice is just over 60cm tall and weighs 2.5kg.

Although their diet mainly consists of seeds and petrified insects (which act as both gastroliths and food in the creature's stomach), cockatrices fiercely defend their territory from anything they deem a threat, and the wandering males wander in search of new places to build lairs sometimes lead them into inadvertent contact with humans, with devastating results.

The cockatrice's uncanny ability to turn other creatures to stone is her best Defence, and her lair is invariably filled with the remains of petrified enemies. Ironically, however, weasels and ferrets, the creatures most likely to end up in cockatrice nests to eat their eggs, appear to be completely immune to this effect. For unknown reasons, cockatrices are both terrified and furious with common roosters, and are equally likely to flee or attack when confronted.


\

\textbf{Constrictor Serpent}\index[Monsters]{Constrictor Serpent}

\textit{Large beast, unaligned}

\textbf{STRENGTH} +2

\textbf{DEXTERITY} +2

\textbf{CONSTITUTION} +1

\textbf{INTELLIGENCE} -5

\textbf{WISDOM} +0

\textbf{CHARISMA} -4

\textbf{Initiative} +2 -- \textbf{Defence} 13

\textbf{Hit Points} 13 (2d10 + 2)

\textbf{Movement} 9m, swim 9m

\textbf{Saving Throws}: Fortitude +3, Reflexes +2, Will +0

\textbf{Senses} blindsight 3 m

\textbf{Languages} -

\textbf{Challenge} 1/4 (50 XP)

\textbf{Actions}

\textit{\textbf{Bite.} Melee Weapon Attack}: +4 to hit, reach 3 ft., one creature.

\textit{Hit:} 5 (1d6 + 2) piercing damage.

\textit{\textbf{Constrict.} Melee Weapon Attack}: +4 to hit, reach 3 ft., one creature.

\textit{Hit:} 6 (1d8 + 2) bludgeoning damage, and the target is grappled (DC 14 to flee). Until the grab ends, the creature is restrained, and the snake can't constrict another target.

\

\textbf{Coral Shark}\index[Monsters]{Coral Shark}

Reef sharks are 2 to 3 meters long and live in shallower waters and along coral reefs.

\textit{Medium beast, unaligned}

\textbf{STRENGTH} +2

\textbf{DEXTERITY} +1

\textbf{CONSTITUTION} +1

\textbf{INTELLIGENCE} -5

\textbf{WISDOM} +0

\textbf{CHARISMA} -3

\textbf{Initiative} +1 -- \textbf{Defence} 13

\textbf{Hit Points} 22 (4d8 + 4)

\textbf{Movement} 0m, swim 12m

\textbf{Saving Throws}: Fortitude +2, Reflexes +2, Will +1

\textbf{Skills} Awareness +2

\textbf{Senses} blindsight 9 m

\textbf{Languages} -

\textbf{Challenge} 1/2 (100 XP)

\textit{\textbf{Water Breathing.}} The shark can only breathe underwater.

\textit{\textbf{Polding tactics.}} The shark has +1d6 on attack rolls against a creature if at least one of the shark's allies is within 1 meter of the creature and that ally isn't incapacitated.

\textbf{Actions}

\textit{\textbf{Bite.} Melee Weapon Attack}: +4 to hit, reach 3 ft., one target.

\textit{Hit:} 6 (1d8 + 2) piercing damage.

\

\index[Monsters]{Couatl}\textbf{Couatl}

\textit{Celestial medium, Lawful Good}

\textbf{STRENGTH} +3

\textbf{DEXTERITY} +5

\textbf{CONSTITUTION} +3

\textbf{INTELLIGENCE} +4

\textbf{WISDOM} +5

\textbf{CHARISMA} +4

\textbf{Initiative} +5 -- \textbf{Defence} 21

\textbf{Hit Points} 97 (13d8 + 39)

\textbf{Move} 9m, fly 9m

\textbf{Saving Throws} Fortitude +9, Reflexes +13, Will +14

\textbf{Damage Resistances} from Light

\textbf{Immunity to Damage} from non-magical weapon

\textbf{Senses} True Seeing 36 m

\textbf{Languages} all, telepathy 36m

\textbf{Challenge} 4 (1100 XP)

\textit{\textbf{Magic Weapons.}} The couatl's weapon attacks are magical.

\textit{\textbf{Innate Spells.}} The couatl's innate spellcasting ability is Charisma. The Couatl can cast these spells innately, using only verbal components:

At will: \textit{detect good and evil, detect magic, detect thoughts}

3/day each: \textit{blessing, create food and water, heal wounds,} \textit{protection from poisons, lesser restoration, sanctuary, shield} 1/day each: \textit{greater restoration, scrying, dream }

\textit{\textbf{Guarded Mind.}} The Couatl is immune to scrying and any effects that sense his emotions, read his thoughts, or pinpoint his location.

\textbf{Actions}

\textit{\textbf{Bite.} Melee Weapon Attack}: +8 to hit, reach 1 meter, one creature.

\textit{Hit:} 8 (1d6 + 5) piercing damage, and the target must succeed on a DC 13 Fortitude save or be poisoned for 24 hours. Until the poison ends, the target is unconscious. Another creature can take an action to awaken the target.

\textit{\textbf{Constrict.} Melee Weapon Attack}: +6 to hit, reach 3m, Medium or smaller creature.

\textit{Hit:} 10 (2d6 + 3) bludgeoning damage, and the target is grappled (DC 15 to flee). Until the grab ends, the target is entangled, and the Couatl cannot constrict another target.

\textit{\textbf{Shapeshift.}} The Couatl can magically transform into a humanoid or beast whose challenge rating is equal to or lower than its own, or revert to its true form. Upon death it returns to its true form. Any equipment he is wearing or carrying is absorbed or carried into the new form (the couatl's choice).

In the new form, the Couatl retains its game stats and ability to speak, but its Defence, movement methods, Strength, Dexterity, and other actions are replaced by those of the new form, and it gains any stats or abilities (Additional Actions and lair action) possessed by his new form and not his original. If the new form has a bite attack, the Couatl can use its own bite in the new form.

\textbf{Ecology}\\
Environment: Warm Forests\\
Organization: Solitary, pair, or flock (3-6)\\
\textbf{Treasure}: Standard\\
\textbf{Description}\\
Couatls are servants of lawful good deities, though some operate independently of any higher entities. Respected and admired for their wisdom and beauty, they seek to bring mortals to the right path and use their powers to battle evil, especially those known to planeswalk. Some Couatl are viewed as benevolent deities by isolated societies, and while couatls cringe at the thought of posing as a deity, they allow these misconceptions to continue as they allow them to lead these societies down paths of peace and cooperation with their neighbors. . A Couatl is about 3.6 meters long, with a wingspan of about 5 meters, and weighs 900 kg.

As native outsiders, couatls must eat. They prefer the same foods as true snakes, such as mammals and birds, although they have been known to devour evil humanoids. Because they prefer to spend their time pursuing their pursuits rather than hunting, they appreciate food offerings, especially small boars and fowl. A Couatl sometimes shows its appreciation to an adventurer or party that has done it a service by gifting 1d4 of its brightly colored feathers. These freely obtained feathers, when used as an additional material component, allow a spellcaster casting Planar Ally to summon that specific Couatl without paying the normal cost in gold or other valuables, provided the Couatl approves of the caster's requested service.

\

\index[Monsters]{Crawling Tentacled Worm}\textbf{Crawling Tentacled Worm}

\textit{Large monstrosity, unaligned}

\textbf{STRENGTH} +4

\textbf{DEXTERITY} +1

\textbf{CONSTITUTION} +3

\textbf{INTELLIGENCE} -4

\textbf{WISDOM} 1

\textbf{CHARISMA} -3

\textbf{Initiative} +2 -- \textbf{Defence} 17

\textbf{Hit Points} 55 (7d10 + 31)

\textbf{Move} 9m, climb 9m

\textbf{Saving Throws}: Fortitude +5, Reflexes +4, Will +7

\textbf{Senses} Darkvision 18m

\textbf{Languages} -

\textbf{Challenge} 4 (1000 XP)

\textit{\textbf{Climb as Spider.}} The Crawling Tentacled Worm can climb difficult surfaces, including standing upside down on ceilings, without needing to make an ability check.

\textbf{Actions}

\textit{\textbf{Multiattack.}} The Crawling Tentacled Worm makes 3 attacks, one with its bite and two with its tentacles.

\textit{\textbf{Bite.} Melee Weapon Attack}: +8 to hit, reach 1m, one target.

\textit{Hit:} 10 (2d8 + 6) piercing damage.

\textit{\textbf{Tentacle.} Melee Weapon Attack}: +7 to hit, reach 3m, one creature.

\textit{Hit:} 1 bludgeoning damage. The target must make a DC 18 Fortitude save or be paralyzed until the end of the following round.

\textbf{Ecology}\\
Environment: Any dungeon\\
Organization solitary, pair, tribe (8-12 +3d6 small)\\
\textbf{Treasure}: Accidental\\
\textbf{Description}\\

A typical Crawling Tentacled Worm is an annelid nearly 10.5 meter long and weighing around 500kg. Dark in color (of various shades from blue to green to brown) it is a large worm with a powerful mouth and long, light tentacles along the entire head.

The Crawling Tentacled Worm, even if equipped with short "legs", does not walk but crawls secreting a sticky mucus that allows it to climb even on surfaces in any orientation.

They are ravenous creatures that never miss an opportunity to hunt and devour or conserve corpses where to sow their eggs. They love Nibali's flesh and feed on any living creature (often rats given the typical sewer environment).

The origins of the Crawling Tentacled Worm are rather speculative, some hypothesize that an enchanter tried, critically failing, to transform himself into a Purple Worm, others firmly believe that the gardens of Shayalia needed more fertilization and so the Patroness transformed normal earthworms into these terrifying creatures to devour and digest the buried corpses.

\

\index[Monsters]{Creeping Mound}\textbf{Creeping Mound}

\textit{Large plant, unaligned}

\textbf{STRENGTH} +4

\textbf{DEXTERITY} -1

\textbf{CONSTITUTION} +3

\textbf{INTELLIGENCE} -3

\textbf{WISDOM} +0

\textbf{CHARISMA} -3

\textbf{Initiative} -1 -- \textbf{Defence} 18

\textbf{Hit Points} 136 (16d10 + 48)

\textbf{Movement} 6m, swim 6m

\textbf{Skills} Stealth +2

\textbf{Damage Resistances} cold, fire

\textbf{Damage Immunity} Electricity

\textbf{Condition Immunity} blinded, deafened, fatigued

\textbf{Senses} blindsight 18m (blind beyond this range)

\textbf{Languages} -

\textbf{Challenge} 5 (1800 XP)

\textit{\textbf{Electricity Absorption.}} Whenever the creeping mound takes electricity damage, it takes no damage and regains a number of Hit Points equal to the electricity damage inflicted.

\textbf{Actions}

\textit{\textbf{Multiattack.}} The creeping mound makes two slam attacks. If both attacks hit a Medium or smaller creature, the target is grappled (DC 14 to flee) and the creeping mound uses envelop on it.

\textit{\textbf{Slam.} Melee Weapon Attack}: +11 to hit, reach 1m, one target.

\textit{Hit:} 13 (2d8 + 4) bludgeoning damage.

\textit{\textbf{Envelop.}} The creeping mound envelops a Medium or smaller creature it has grappled. The engulfed target is blinded, restrained, and unable to breathe, and must succeed on a DC 14 Fortitude save at the start of each round of the cairn or take 13 (2d8 + 4) bludgeoning damage. If the mound moves, the engulfed target moves with it. The mound can only envelop one creature at a time.

\textit{\textbf{Enraged}}: The Creeping mound unleashes a wave of electricity. All creatures within 10 feet take 3d6 points of electricity damage. Cost 2 Actions.


\textbf{Ecology}\\
Environment Temperate Forests or Swamps\\
Organization: Solitary\\
\textbf{Treasure}: Standard\\
\textbf{Description}\\
Creeping mounds, also called just creepers, look like decaying plant masses. They are intelligent carnivorous plants, with a penchant for elven flesh. The brain and sensory organs are located in the upper body. Creeping mounds are usually 2.4 meters in circumference and 1.8 to 2.7 meters high. They weigh around 1,900 kg.

Creeping mounds are strange creatures, more like a tangle of parasitic vines than a single rooted plant. They are omnivores, capable of drawing sustenance from just about anything, whether they cling to trees to suck their sap, dig their roots into the ground to absorb simple nutrients, or consume the flesh and bones of prey.

Creeping mounds are incredibly stealthy in their natural environment. They blend in with the surrounding terrain and can wait motionless for days for the arrival of potential prey. They can be practically anywhere and attack at any time without any warning and regardless of whether or not there are survivors, as long as they have food.

Common creeping mounds lead a nomadic, solitary existence in deep forests and fetid swamps, but they can also be found underground, amid thickets of mushrooms. Troubling rumors speak of groups of creeping mounds congregating around large mounds deep in jungles and swamps, often during severe lightning storms. The reason for this behavior is unknown, and many sages wonder if there may be a dark and alien purpose behind it.

\

\textbf{Crocodile}\index[Monsters]{Crocodile}

\textit{Large beast, unaligned}

\textbf{STRENGTH} +2

\textbf{DEXTERITY} +0

\textbf{CONSTITUTION} +1

\textbf{INTELLIGENCE} -4

\textbf{WISDOM} +0

\textbf{CHARISMA} -3

\textbf{Initiative} +0 -- \textbf{Defence} 13

\textbf{Hit Points} 19 (3d10 + 3)

\textbf{Movement} 6m, swim 9m

\textbf{Saving Throws}: Fortitude +6, Reflexes +4, Will +2

\textbf{Skills} Stealth +2

\textbf{Languages} -

\textbf{Challenge} 1/2 (100 XP)

\textit{\textbf{Hold Breath.}} The crocodile can hold its breath for 15 minutes.

\textbf{Actions}

\textit{\textbf{Bite.} Melee Weapon Attack}: +4 to hit, reach 1 meter, one creature.

\textit{Hit:} 7 (1d10 + 2) piercing damage, and the target is grappled (DC 12 to escape). Until the grab ends, the target is restrained, and the crocodile can't use the bite against another target.

\

\index[Monsters]{Cursed Immortal}\textbf{Cursed Immortal}

\textit{Medium aberration (human), leaning towards insane}

\textbf{STRENGTH} +3

\textbf{DEXTERITY} +1

\textbf{CONSTITUTION} +2

\textbf{INTELLIGENCE} -1

\textbf{WISDOM} +1

\textbf{CHARISMA} -2

\textbf{Initiative} +3 -- \textbf{Defence} 15

\textbf{Hit Points} 75 (12d8 + 21)

\textbf{Move} 9m

\textbf{Saving Throws} Fortitude +6, Reflexes +5, Will +5

\textbf{Resistance to Magic} the Accursed Immortal has +1d6 on each spell Saving Throw

\textbf{Skills} Awareness +3, profession he had in life

\textbf{Damage Immunity} cold, fire, void

\textbf{Condition Immunity} charmed, poisoned, petrified, frightened

\textbf{Unconscious} the Accursed has no sense of taste or smell

\textbf{Languages} Common, dwarven, elven

\textbf{Challenge} 4 (1100 XP)

\textit{\textbf{Immortal}} The Cursed Immortal regenerates 1 Hit Point per round, which allows him to regenerate limbs and return to life. The only way to kill it is by dissolving it in magical acid. Remove Curse DC 30 kills it instantly.

\textit{\textbf{Different nature}} The Cursed Immortal does not eat, drink, sleep, grow old. He's not undead

\textbf{Actions}

\textit{\textbf{Multiattack.}} The Cursed Immortal makes three attacks with the longsword.

\textit{\textbf{Sword.} Melee Weapon Attack}: +6 to hit, reach 1m, one target.

\textit{Hit:} 12 (1d10 + 7) slashing damage.

\textbf{Ecology}\\
Environment: Any\\
Organization: Solitary\\
\textbf{Treasure}: NPC gear (Studded Leather Armour, 2 Daggers, Sword, more sword)\\
\textbf{Description}\\
The Cursed Immortal is a person cursed often by a Patron or a powerful spellcaster with the curse of insane immortal life. The curse upsets the balance of the person and he finds himself wandering around without a goal or a goal. Every now and then they remember who they were and then they continue in search of who cursed them.
With the aim of finally being killed, he throws himself into every fight hoping that the opponent will be able to kill him once and for all.


\

\index[Monsters]{Dark hugger}\textbf{Dark hugger}

\textit{Little monstrosity, unaligned}

\textbf{STRENGTH} +3

\textbf{DEXTERITY} +1

\textbf{CONSTITUTION} +1

\textbf{INTELLIGENCE} -4

\textbf{WISDOM} +0

\textbf{CHARISMA} -3

\textbf{Initiative} +1 -- \textbf{Defence} 12

\textbf{Hit Points} 22 (5d6 + 5)

\textbf{Move} 3m, fly 9m

\textbf{Saving Throws}: Fortitude +5, Reflexes +3, Will +0

\textbf{Skills} Stealth +3

\textbf{Senses} blindsight 18 m

\textbf{Languages} -

\textbf{Challenge} 1/2 (100 XP)

\textit{\textbf{Echolocation.}} The Dark hugger cannot use blindsight when deafened.

\textit{\textbf{False Appearance.}} While the Dark hugger remains motionless, it is indistinguishable from a rock formation such as a stalactite or stalagmite.

\textbf{Actions}

\textit{\textbf{Splitting.} Melee Weapon Attack}: +5 to hit, reach 3 ft., one creature.

\textit{Hit:} 6 (1d6 + 3) bludgeoning damage and the Dark hugger sticks to the creature. If the target is Medium or smaller, the Dark hugger has +1d6 on attack rolls, sticks by wrapping around the target's head, who are blinded and unable to breathe as long as the Dark hugger remains stuck in this way.

While stuck to the target, the Dark hugger can't attack any other creature except the target, but it has +1d6 to its attack rolls. The Dark hugger's speed becomes 0, and he cannot benefit from any speed bonuses when moving with the target.

A creature can detach the Dark hugger as an action and with a successful DC 13 Strength check. During its round, the Dark hugger can detach itself from the target by using 1 meter of movement.

\textit{\textbf{Aura of Darkness (1/day).}} A magical darkness with a 5 meter radius extends from the Dark hugger, moving with it, and spreading around corners. The darkness lasts as long as the Dark hugger maintains concentration, up to 10 minutes (as if concentrating on a spell). Darkvision cannot penetrate this darkness, nor can it be illuminated by any natural light. If any part of the darkness overlaps an area of light created by a spell of 2 level or lower, the spell that is creating the light is dispelled.

\textbf{Ecology}
Environment: Any (Dungeon)\\
Organization: Solitary, pair or brood (3-12)\\
\textbf{Treasure}: None\\
\textbf{Description}\\
a Dark hugger's sprawling opening is just under 1 meter wide; when it is hung from the vault of a cave, disguised as a stalactite, its length varies between 60 and 90 cm. A typical Dark hugger specimen weighs 20 kg. The creature's head and body are usually the color of basalt or dark granite, but its membranous tentacles can change color to match its surroundings.

Dark huggers are not particularly skilled climbers, but they are able to hang from a cave vault like bats, hooked by means of hooks at the end of their tentacles, so that their dangling body is almost indistinguishable from a stalactite. From this hidden position, the creature waits for the prey to pass under it and, at this point, it detaches itself by lunging towards it, slamming into the target and trying to wrap its membranous tentacles around it. If the Dark hugger misses its prey, it will ascend and lunge at its prey again, until either the latter is defeated or the Dark hugger is badly injured (in which case it will flutter to the ceiling to hide, hoping its "prey" will leave it lose). This creature's innate ability to cloak its surroundings in magical darkness gives it an added advantage against opponents who require light to see.

Dark huggers prefer to live and hunt in caves and burrows closest to the surface, since these offer a more frequent passage of prey for these monsters to hunt. They are not limited to these dark caverns, however, and can sometimes be encountered in abandoned fortresses or even in the sewers of crowded cities. Any place where food is plentiful and there is a ceiling to hang from is a possible lair for a Dark hugger.

A Dark hugger's life cycle is rapid: hatchlings become adults within a few months, and most die of old age within a few years. As a result, Dark hugger generations follow each other rapidly, and over the years the evolution of these creatures is equally rapid. For this reason, a cavern's ecosystem can have dramatic effects on a Dark hugger's appearance, capabilities, and tactics. Dark huggers capable of swimming may develop in aquatic caves, while creatures inhabiting places prone to volcanism may develop a specific resistance to fire. Other variants of Dark hugger might have stronger skins and instead of falling to crush their prey they might simply dive into it trying to pierce it similar to real stalactites. The deepest, darkest caverns are rumored to harbor Dark huggers of incredible size, capable of suffocating several human-sized targets at once in their enveloping embrace.


\

\textbf{Darktorch}\\\index[Monsters]{Darktorch}
\textit{Medium, Undead, Evil}\\
\textbf{Strength}: +3\\
\textbf{Dexterity}: +1\\
\textbf{Constitution}: +2\\
\textbf{Intelligence}: +0\\
\textbf{Wisdom}: -1\\
\textbf{Charisma}: -2\\
\textbf{Defence}: 17 -- \textbf{Initiative}: +2\\
\textbf{Hit Points}: 75 (12d10 +20)\\
\textbf{Movement}: 6m\\
\textbf{Saving Throws}: Fortitude +9, Reflexes +8, Will +7 \\
\textbf{Senses}: Darkvision, sees in magical darkness\\
\textbf{Damage Resistances} Void; from a non-magical or non-silver weapon\\
\textbf{Immunity to Damage} Poison\\
\textbf{Condition Immunity} poisoned, fatigue, bleeding\\
\textbf{Vulnerability} Light\\
\textbf{Invisible in Dark} A Darktorch is completely invisible while in darkness\\
\textbf{Languages}: Understands the Common, but does not speak\\
\textit{\textbf{Undead nature.}} Darktorch needs no air, food, drink, or sleep.\\
\textbf{Challenge} 4 (1100 XP)\\
\textit{\textbf{Sensitivity to Light}}. While in sunlight, Darktorch has -1d6 to attack rolls\\
\textbf{Multiattack}\\
\textit{\textbf{Attack}} Darktorch attacks twice with his torch or performs Cone of Sadness\\
\textit{\textbf{Torch}} Melee attack, +8 on hit\\
\textit{\textbf{Hit}} 7 (1d6+3) bludgeoning damage, casts the Darkness spell on hit target, lasting until Darklight is destroyed\\
\textit{\textbf{Cone of Sadness}} 6 meters cone. Affected creatures must make a DC 14 Will save or fall into a sad despair that grants -1d6 to attack, -2 to melee damage.\\
\textbf{Ecology}\\
Environment: Dungeon\\
Organization solitary, group 2d4\\
\textbf{Treasure}: Special\\
\textbf{Description}\\
A Darktorch was an adventurer, like you, who died in terror after the last torch went out. A Darktorch is an undead, usually humanoid, vaguely nondescript in appearance, who wields a torch that emanates pure darkness. The purpose of him is to kill new adventurers by enveloping them in eternal darkness.

Usually the Darktorch hides in the darkness waiting to touch the opponent and envelop him in his curse. A creature slain by a Darktorch returns to life as a Darktorch after 1d3 days.

A Darktorch leaves its torch on the ground when destroyed. This torch, of pure darkness, can cast the Darkness spell three times per day, outside of the hands of a Darktorch if exposed to sunlight it is destroyed in 2d4 rounds.


\

\index[Monsters]{Deathpillager}\textbf{Deathpillager}

\textit{Large construct, undead, unaligned}

\textbf{STRENGTH} +5

\textbf{DEXTERITY} +0

\textbf{CONSTITUTION} +4

\textbf{INTELLIGENCE} -4

\textbf{WISDOM} -2

\textbf{CHARISMA} -5

\textbf{Initiative} +2 -- \textbf{Defence} 21

\textbf{Hit Points} 105 (10d10 + 50)

\textbf{Move} 9m

\textbf{Saving Throws} Fortitude +15, Reflexes +9, Will +7

\textbf{Awareness} +4

\textbf{Immunity to Damage} Poison

\textbf{Condition Immunity} poisoned, charmed, fatigued, paralyzed, petrified, bleeding

\textbf{Senses} darkvision 30m

\textbf{Languages} understands all of the creature's languages but cannot speak

\textbf{Challenge} 6 (2300 XP)

\textit{\textbf{Undead Nature.}} The Deathpillager do not need air, food, drink, or sleep.

\textit{\textbf{Immutable Form.}} As a construct, he cannot be affected by spells or effects that change his form.

\textit{\textbf{Container.}} The Deathpillager has an opening compartment with a door on the metal back that can hold up to 100kg of objects, large to small.

\textit{\textbf{Resistance to Air.}} The Deathpillager has innate resistance to spells from the Air Magic List.

\textit{\textbf{Sensitive to Fire.}} The Deathpillager takes one less action the next round if it takes fire damage.

\textbf{Actions}

\textit{\textbf{Multiattack.}} The Deathpillager attacks with two claws or attacks with one claw and uses the Crippling Eye.

\textbf{\textit{Chela.}} +9 on hit, reach 1 meter

\textit{Hit}: 16 (2d10 + 5) bludgeoning damage

\textit{\textbf{Crippling Eye}}: The affected creature, within 20 meters, must make a DC 16 Fortitude save or be paralyzed for 2d4 rounds.

\textbf{Ecology}\\
Environment: Any, caves\\
Organization 1-2 Deathpillager, 1d4+1 guardians\\
\textbf{Treasure}: What you collected\\
\textbf{Description}\\
The Deathpillager are particular undead constructed from pieces of various corpses and pieces of iron to resemble some species of large Armoured crabs.
The completely metallic back acts as a container for the treasures that the Deathpillager finds, the claws, in variable numbers between 6 and 8, are just over a meter long and have the characteristic of each leaving a different imprint being assembled from pieces of metal and verse bodies.

The large central eye, perhaps once belonging to a humanoid, allows the controller and builder of the Deathpillager to see and command it. A Deathpillager's purpose is to explore, usually a system of caverns or pathways, looking for the remains of past raiders and adventurers for their magical items and treasures.

Usually a Deathpillager is always accompanied by several guardians (other creatures at the command of the controller) who help it in "fixing" any "resistance" still active.


\

\index[Monsters]{Deep Gnome}\textbf{Deep Gnome}

\textit{Small humanoid (gnome), neutral good}

\textbf{STRENGTH} +2

\textbf{DEXTERITY} +2

\textbf{CONSTITUTION} +2

\textbf{INTELLIGENCE} +1

\textbf{WISDOM} +0

\textbf{CHARISMA} -1

\textbf{Initiative} +2 -- \textbf{Defence} 16 (mail shirt)

\textbf{Hit Points} 16 (3d6 + 6)

\textbf{Move} 6m

\textbf{Saving Throws}: Fortitude +6, Reflexes +6, Will +2

\textbf{Skills} Stealth +4, Awareness +2

\textbf{Senses} darkvision 40m

\textbf{Languages} Gnomish, Deep Language, Tremun

\textbf{Challenge} 1/2 (100 XP)

\textit{\textbf{Gnome cunning.}} The gnome has +1d6 on Saving Throws against magic.

\textit{\textbf{Stone camouflage.}} The gnome has +1d6 on Dexterity (Hide) checks made to hide in rocky terrain.

\textit{\textbf{Innate Spells.}} The gnome's innate spellcasting ability is Intelligence. The gnome can cast these spells innately, requiring no components:

At will: \textit{anti-detection} (personal)

1/day each: \textit{disguise self, blind/deafness, blur}

\textbf{Actions}

\textit{\textbf{War Pickaxe.} Melee Weapon Attack}: +4 to hit, reach 1m, one target.

\textit{Hit:} 6 (1d8 + 2) piercing damage.

\textit{\textbf{Poison Dart.} Ranged weapon attack}: +4 to hit, range 9m, one target.

\textit{Hit:} 4 (1d4 + 2) piercing damage, and the target must succeed on a DC 12 Fortitude save or be poisoned, -1 Strenght and Dexterity, for 1 minute. The target can repeat the Saving Throw at the end of each of its rounds, ending the effect on itself on a success.

\textbf{Ecology}
Environment: Any dungeon\\
Organization: Solitary, company (2-4), squad (5-20 plus 1 3rd-6th leader and two 3rd level sergeants), or gang (30-50 plus 1 3rd level sergeant for every 20 adults, 5 5th level lieutenants, 3 7th level captains, and 2-5 Medium earth elementals)\\
\textbf{Treasure}: NPC gear (Heavy Pickaxe, Light Crossbow with 10 bolts, other treasure)\\
\textbf{Description}\\
Deep gnomes are a branch of the gnome race. They dwell underground, in hidden cities, safe from dark elves and other subterranean races. Their skin is the color of rock, usually gray or brown. Males are bald and females have sparse gray hair.

\

\textbf{Deer}\index[Monsters]{Deer}

\textit{Medium beast, unaligned}

\textbf{STRENGTH} +0

\textbf{DEXTERITY} +3

\textbf{CONSTITUTION} +0

\textbf{INTELLIGENCE} -4

\textbf{WISDOM} +2

\textbf{CHARISMA} -3

\textbf{Initiative} +3 -- \textbf{Defence} 14

\textbf{Hit Points} 4 (1d8)

\textbf{Move} 15m

\textbf{Saving Throws}: Fortitude +2, Reflexes +3, Will +2

\textbf{Languages} -

\textbf{Challenge} 0 (10 XP)

\textbf{Actions}

\textit{\textbf{Bite.} Melee Weapon Attack}: +2 to hit, reach 1m, one target.

\textit{Hit:} 2 (1d4) piercing damage.

\

\index[Monsters]{Demon, Balor}\textbf{Balor}

\textit{Huge fiend (demon), chaotic evil}

\textbf{STRENGTH} +8

\textbf{DEXTERITY} +2

\textbf{CONSTITUTION} +6

\textbf{INTELLIGENCE} +5

\textbf{WISDOM} +3

\textbf{CHARISMA} +6

\textbf{Initiative} +5 -- \textbf{Defence} 29

\textbf{Hit Points} 262 (21d12 + 126)

\textbf{Move} 12m, fly 24m

\textbf{Saving Throws} Fortitude +29, Reflexes +17, Will +25

\textbf{Damage Resistances} cold, electricity;

\textbf{Damage Immunity} Fire, poison, weapons +1

\textbf{Condition Immunity} poisoned

\textbf{Damage Vulnerability} cold iron

\textbf{Senses} True Seeing 36 m

\textbf{Languages} Abyssal, telepathy 36m

\textbf{Challenge} 19 (22000 XP)

\textit{\textbf{Magic Weapons.}} The demon's weapon attacks are magical.

\textit{\textbf{Aura of Fire.}} At the start of each of the demon's rounds, each creature within 1 meter of it takes 10 (3d6) fire damage, and flammable objects in the aura and which are not worn or carried catch fire. A creature that touches the demon or hits it with a melee attack while within 1 meter of it takes 10 (3d6) fire damage.

\textit{\textbf{Resistance to Magic.}} The demon has +1d6 on Saving Throws against spells and other magical effects.

\textit{\textbf{Death Throat.}} When the demon dies, it explodes; each creature within 10 meters of it must make a DC 25 Reflex save, taking 70 (20d6) fire damage on a failed save, or half as much damage on a successful one. The explosion sets fire to flammable objects that are not worn or carried, and destroys the demon's weapons.

\textbf{Actions}

\textit{\textbf{Multiattack.}} The demon makes two attacks: one with the longsword and one with the whip.

\textit{\textbf{Whip.} Melee Weapon Attack}: +30 to hit, reach 10m, one target.

\textit{Hit:} 15 (2d6 + 8) slashing damage plus 10 (3d6) fire damage, and the target must succeed on a DC 25 Fortitude save or be pulled 21 meter towards the demon.

\textit{\textbf{Longsword.} Melee Weapon Attack}: +30 to hit, reach 3m, one target.

\textit{Hit:} 21 (3d8 + 8) slashing damage plus 13 (3d8) electricity damage. If the demon scores a critical roll, it rolls damage three times instead of twice.

\textit{\textbf{Teleport.}} The demon magically teleports itself, along with all equipment it wears or carries, to an unoccupied space it can see within 36 m.

\textbf{Ecology}\\
Environment: Any (Abyss)\\
Organization: Solitary or warband (1 Balor and 2-5 Glabrezu)\\
\textbf{Treasure}: Standard (+1 Unholy Longsword,+1 Flaming Whip, other treasure)\\
\textbf{Description}\\
When people whisper terrifying tales of demonic creatures, they mostly imagine a towering figure of fire and flesh, a horned nightmare armed with flaming whip and sword, flying through the night in search of its prey. The demon these people fear is the Balor, and this fear is fully justified, since few demons can match the mighty Balor in strength or brutality.

In the Abyss, balor mostly serve demon lords, either as generals or captains (when not extremely powerful balors, known as balor lords). A balor usually commands vast legions of demons, and while he often allows these greedy, slavering minions to fight his battles, he is far from a coward. If the opportunity to join a fight presents itself, few balors choose to hold back.

A Balor is 14 feet tall and weighs 2500kg. Only the cruellest mortal souls can fuel the creation of a balor—unlike other demons, it often takes numerous souls of powerful villains to birth a new balor.

\

\index[Monsters]{Demon, Demogorgon}\textbf{Demogorgon}

\textit{Huge fiend (demon prince), chaotic evil}

\textbf{STRENGTH} +9

\textbf{DEXTERITY} +2

\textbf{CONSTITUTION} +8

\textbf{INTELLIGENCE} +5

\textbf{WISDOM} +3

\textbf{CHARISMA} +7

\textbf{Initiative} +5 -- \textbf{Defence} 35

\textbf{Hit Points} 468 (26d10+208)

\textbf{Movement} 15m, swim 9m

\textbf{Saving Throws}: Fortitude +34, Reflexes +28, Will +29

\textbf{Skills} all +15

\textbf{Damage Resistances} cold, electricity, fire

\textbf{Immunity to Damage} Void, Poison; weapons +2

\textbf{Condition Immunity} charmed, poisoned, paralyzed, fatigued, frightened

\textbf{Senses} True Seeing 40 m

\textbf{Languages} all, telepathy 45m

\textbf{Challenge} 26 (90000 XP)

\textit{\textbf{Spells.}} The Demogorgon has MP 20. His spellcasting ability is Charisma, +7 to hit on spell attacks. The Demogorgon knows the following spells:

At will: detect magic, major image

level 3 (4 slots): \textit{dispel magic, fear, telekinesis}

level 4 (1 slot): \textit{projection image, mental regression}

\textit{\textbf{Demon Nature.}} The Demogorgon does not need air, food, drink, or sleep.

\textit{\textbf{Legendary Endurance (3 / Day).}} If the Demogorgon fails a Saving Throw, he can choose to succeed instead.

\textit{\textbf{Resistance to Turning.}} The Demogorgon has +1d6 on Saving Throws against effects that turn undead.

\textit{\textbf{Two heads.}} Demogorgon has +1d6 on Saving Throws against being blind, deaf, unconscious

\textbf{Actions}

\textit{\textbf{Multiattack.} 2 tentacle attacks}: +30, reach 3 meter, one creature. All of Demogorgon's attacks are considered magical +2.

\textit{Hit:} 35 (4d12 +9) bludgeoning damage. The affected creature must make a DC 29 Fortitude save or its maximum Hit Points decrease by the same amount.

\textit{\textbf{Glance}} Demogorgon stares at a creature he can see within 40 m. The target must make a DC 28 Will save.

\textit{Glare Effect:} Demogorgon chooses one of these effects or it's random:

1. Power Gaze. The target is unconscious until the next round or until the Demogorgon is out of line of sight

2. Hypnotic Gaze. The target is dominated by the Demogorgon who decides every action. This look requires the use of both heads of the Demogorgon.

3. Gaze of Madness. The target is under the influence of the Confusion spell which lasts, with no further Saving Throw, as long as Demogorgon is within sight. The Demogorgon does not have to remain focused for the effect to last.

\textbf{Additional Actions}

The Demogorgon can perform 3 additional actions, chosen from those below and one per round only at the end of another creature's round.

\textbf{Tail.} The Demogorgon attacks with its tail. +30 to hit, reach 5m, one target. On a hit, 31 Hit Points of bludgeoning damage plus +4d6 Void damage

\textbf{Gaze of Madness.} Demogorgon uses either Gaze of Power or Gaze of Madness

\textbf{Ecology}\\
Environment: Abyss\\
Organization: Unique\\
\textbf{Treasure}: Triple\\

\textbf{Description}
Demogorgon is a huge demon, prince of the abyss and madness about 5 meters tall. He appears as a bipedal reptilian with two baboon heads, the necks are long and serpentine like the tentacled arms. Demogorgon's two heads have distinct personalities who loathe each other. They often try to dominate each other and many of the stories involving the Demogorgon deal with just how one or the other heads to dominate the whole. There is a strong rivalry between the Demogorgon and Orcus.


\

\index[Monsters]{Demon, Dretch}\textbf{Dretch}

\textit{Little fiend (demon), chaotic evil}

\textbf{STRENGTH} +0

\textbf{DEXTERITY} +0

\textbf{CONSTITUTION} +1

\textbf{INTELLIGENCE} -3

\textbf{WISDOM} -1

\textbf{CHARISMA} -4

\textbf{Initiative} +0 -- \textbf{Defence} 12

\textbf{Hit Points} 18 (4d6 + 4)

\textbf{Move} 6m

\textbf{Saving Throws}: Fortitude +1, Reflexes +0, Will -1

\textbf{Damage Resistances} cold, electricity, fire

\textbf{Immunity to Damage} Poison

\textbf{Condition Immunity} poisoned

\textbf{Damage Vulnerability} cold iron

\textbf{Senses} Darkvision 18m

\textbf{Languages} Abyssal, telepathy 18m (only works on creatures that understand Abyssal)

\textbf{Challenge} 1/4 (50 XP)

\textbf{Actions}

\textit{\textbf{Multiattack.}} The demon makes two attacks: one with its bite and one with its claws.

\textit{\textbf{Claws.} Melee Weapon Attack}: +2 to hit, reach 1m, one target.

\textit{Hit:} 5 (2d4) slashing damage.

\textit{\textbf{Bite.} Melee Weapon Attack}: +2 to hit, reach 1m, one target.

\textit{Hit:} 3 (1d6) piercing damage.

\textit{\textbf{Fetid Cloud (1/day).}} A disgusting green gas extends within a 3 meter radius of the demon. The gas propagates around corners, and its area is slightly obscured. Remains for 1 minute or until dispersed by a strong wind. Any creature that starts its round in that area must succeed on a DC 11 Fortitude save or be poisoned until the start of its next round. While poisoned in this way, the target can take only one action or a reaction action, but not both, and cannot take reactions during its round.

\textbf{Ecology}\\
Environment: Any (Abyss)\\
Organization: Solo, Pair, Gang (3-5), Group (6-12), or Crowd (13+)\\
\textbf{Treasure}: None\\
\textbf{Description}\\
Even the lowest demon of the Abyss is dangerous and possesses a compelling need to spread doom and dismay. The wretched dretch is as horrifying and fetid as he is cruel, even if he lacks the strength and power to satisfy his urge to brutalize others in his native realm. The purpose of dretch existence is to serve more powerful demons as expendable victims, and only the lucky few manage to survive long enough to evolve.

Dretches are favorite targets for amateurs in Abysmal summoning. Relatively weak and easy to intimidate, dretches can often be forced into long periods of servitude using vague promises of opportunities to vent their frustrations and anger against weaker opponents. Yet the would-be dretch summoner had better remember that these demons are just as cowardly and treacherous as other demons. A dretch facing a more powerful foe will happily trade any information he has in exchange for his miserable life.

Unlike most demons, the dretch's slovenly personality and disdain for sustained physical labor rarely pay off. Advanced dretches are rare, but those who can find the strength in themselves to become more than they were at the time of their creation become the poor rulers of the Abyss, cruel and embittered, ruling over parasitic, broken souls, mindless dead and other dretches. Their empires are confined to abandoned stretches of sewers beneath forgotten cities, unstable swampy wastes shunned by sane minds, and other unwelcome corners of the Abyss that even demons find uncomfortable or repugnant. Yet to the dretch lords these realms are their empires, and they defend them with pitiful tenacity.

A dretch is 4 feet tall and weighs 100kg. Dretches usually form from the souls of evil and slothful mortals—only a small fragment of soul is needed to give rise to such a horrifying birth. A single soul can often cause a small army of dretches to appear, and the sight of a horde of fledgling dretches breaking free from the pulsating protomatter of the Abyss is both nauseating and terrifying.

\

\index[Monsters]{Demon, Glabrezu}\textbf{Glabrezu}

\textit{Large fiend (demon), chaotic evil}

\textbf{STRENGTH} +5

\textbf{DEXTERITY} +2

\textbf{CONSTITUTION} +5

\textbf{INTELLIGENCE} +4

\textbf{WISDOM} +3

\textbf{CHARISMA} +3

\textbf{Initiative} +4 -- \textbf{Defence} 22

\textbf{Hit Points} 157 (15d10 + 75)

\textbf{Move} 12m

\textbf{Saving Throws} Fortitude +18, Reflexes +4, Will +11

\textbf{Damage Resistances} cold, electricity, fire; from a non-magical weapon

\textbf{Immunity to Damage} Poison

\textbf{Condition Immunity} poisoned

\textbf{Damage Vulnerability} cold iron

\textbf{Senses} True Seeing 36 m

\textbf{Languages} Abyssal, telepathy 36m

\textbf{Challenge} 9 (5000 XP)

\textit{\textbf{Innate Spells.}} The demon's spellcasting ability is Intelligence. The demon can cast these spells innately, requiring no material components:

At will: \textit{dispel magic, detect magic, darkness}

1/day each: \textit{confusion, power word stun, Fly}

\textit{\textbf{Resistance to Magic.}} The demon has +1d6 on Saving Throws against spells and other magical effects.

\textbf{Actions}

\textit{\textbf{Multiattack.}} The demon makes four attacks: two with its claws and two with its fists. Alternatively, he can make two claw attacks and cast a spell.

\textit{\textbf{Chela.} Melee Weapon Attack}: +14 to hit, reach 3m, one target.

\textit{Hit:} 16 (2d10 + 5) bludgeoning damage. If the target is a Medium or smaller creature, it is grabbed (DC 15 to flee). The glabrezu has two claws, each of which can grip a target.

\textit{\textbf{Punch.} Melee Weapon Attack}: +14 to hit, reach 1m, one target.

\textit{Hit:} 7 (2d4 + 2) bludgeoning damage.

\textit{\textbf{Enraged}}: the glabrezu creates a duplicate of itself from the plane of shadows. This duplicate has the same characteristics as the glabrezu but does not attack. When attacking the glabrezu you have a 50\% chance to attack the shadow duplicate.

\textbf{Ecology}\\
Environment: Any (Abyss)\\
Organization solitary or squad (1 glabrezu, 1 succubus, and 2-5 vrock)
\textbf{Treasure}: Standard\\
\textbf{Description}\\
While the succubus is a demon who lures her prey by exploiting their carnal desires and needs, the glabrezu is a tempter of another kind. Ferocious and bestial in form, the glabrezu is actually a master of deception and lies. With his ability to hide his true form behind pleasing illusions, he uses his magic to grant the wishes of mortal humanoids, as a form of reward for those who succumb to his deceptions and tricks. A wish granted by a glabrezu satisfies the wisher's need in the most crippling way possible, though these consequences may not prove immediately so. A struggling blacksmith may desire fame and prowess in his chosen profession, only to find that his best patron is a cruel and sadistic murderer who uses weapons to further his own destructive desires. A lonely man who expresses a desire for a mate may see his wish come true with an old flame of his "revived" in vampire form, and other such examples. The glabrezu is highly creative in fulfilling a mortal's wishes.

A glabrezu stands 18 feet tall and weighs just over 3500kg. These evil demons originate from the souls of traitors, deceivers, and subversives—souls of mortals who, while alive, swore falsehood or used treachery and deceit to ruin the lives of others.

\

\index[Monsters]{Demon, Hezrou}\textbf{Hezrou}

\textit{Large fiend (demon), chaotic evil}

\textbf{STRENGTH} +4

\textbf{DEXTERITY} +3

\textbf{CONSTITUTION} +5

\textbf{INTELLIGENCE} 5 (-2)

\textbf{WISDOM} +1

\textbf{CHARISMA} +1

\textbf{Initiative} +3 -- \textbf{Defence} 20

\textbf{Hit Points} 136 (13d10 + 65)

\textbf{Move} 9m

\textbf{Saving Throws} Fortitude +16, Reflexes +3, Will +9

\textbf{Damage Resistances} cold, electricity, fire; from a non-magical weapon

\textbf{Immunity to Damage} Poison

\textbf{Condition Immunity} poisoned

\textbf{Damage Vulnerability} cold iron

\textbf{Senses} darkvision 40m

\textbf{Languages} Abyssal, telepathy 36m

\textbf{Challenge} 8 (3900 XP)

\textit{\textbf{Stenk.}} Any creature that begins its round within 3 meter of the demon must succeed on a DC 16 Fortitude save or be poisoned, -1 Strenght and Dexterity, until the start of its round. On a successful save, the creature is immune to the croaking demon's stench for 24 hours.

\textit{\textbf{Resistance to Magic.}} The demon has +1d6 on Saving Throws against spells and other magical effects.

\textbf{Actions}

\textit{\textbf{Multiattack.}} The demon makes three attacks: one with its bite and two with its claws.

\textit{\textbf{Claw.} Melee Weapon Attack}: +11 to hit, reach 1m, one target.

\textit{Hit:} 11 (2d6 + 4) slashing damage, 2 bleed damage.

\textit{\textbf{Bite.} Melee Weapon Attack}: +11 to hit, reach 1m, one target.

\textit{Hit:} 15 (2d10 + 4) piercing damage and Lesser Demonic Fever disease.

\textit{Lesser Demonic Fever}: 1 minute, save Fortitude DC 18, 6 hours, 3 successes, -1 Constitution and Wisdom.

\textit{\textbf{Enraged}}: Hezrou releases an incendiary cloud of stench. All creatures around him within 10 feet must make a DC 18 Reflex save to halve the 4d10 fire damage. Cost 2 Actions.


\textbf{Ecology}\\
Environment: Any (Abyss)\\
Organization: Solitary or gang (2-4)\\
\textbf{Treasure}: Standard\\
\textbf{Description}\\
The hezrou lives in the vast swamps, marshes, and streams of the Abyss, as comfortable in water as it is on land. The presence of a hezrou has a harmful effect on flora, causing knots and mutations, and surrounding waters, making them foul-smelling and brackish-tasting, peculiarities more easily detectable in the Material Plane than in the Abyss. Prolonged exposure to this corruption causes horrendous transformations and deformities. Often entire isolated communities of misshapen mutants owe their twisted appearance not so much to their depraved customs as to the proximity of a hezrou.

While quite intelligent, a hezrou can honestly be said to waste its intellect. These beings prefer the simpler pleasures: sleep, the taste of torture, the bliss of eating living flesh, or the joy of feeling something beautiful break and crumble in the grip of their fists. They don't often seek to build empires or head cults, though few hezrou would turn down potential followers who volunteer to volunteer.

These monstrous, bestial creatures are born from the souls of evil mortals who have poisoned themselves, their kin, or their environment, for example, drug addicts, murderers, and alchemists who didn't care how their experiments poisoned the natural world.

\

\index[Monsters]{Demon, Marilith}\textbf{Marilith}

\textit{Large fiend (demon), chaotic evil}

\textbf{STRENGTH} +4

\textbf{DEXTERITY} +5

\textbf{CONSTITUTION} +5

\textbf{INTELLIGENCE} +4

\textbf{WISDOM} +3

\textbf{CHARISMA} +5

\textbf{Initiative} +5 -- \textbf{Defence} 26

\textbf{Hit Points} 189 (18d10 + 90)

\textbf{Move} 12m

\textbf{Saving Throws} Fortitude +25, Reflexes +18, Will +13

\textbf{Damage Resistances} cold, electricity, fire

\textbf{Damage Immunity} poison, weapons +1

\textbf{Condition Immunity} poisoned

\textbf{Damage Vulnerability} cold iron

\textbf{Senses} True Seeing 36 m

\textbf{Languages} Abyssal, telepathy 36m

\textbf{Challenge} 16 (15000 XP)

\textit{\textbf{Magic Weapons.}} The demon's weapon attacks are magical.

\textit{\textbf{Reactive.}} The demon can perform a Reaction Action during each round of combat.

\textit{\textbf{Resistance to Magic.}} The demon has +1d6 on Saving Throws against spells and other magical effects.

\textbf{Actions}

\textit{\textbf{Multiattack.}} The demon makes seven attacks: six with longswords and one with its tail.

\textit{\textbf{Tail.} Melee Weapon Attack}: +18 to hit, reach 3m, one creature.

\textit{Hit:} 15 (2d10 + 4) bludgeoning damage. If the target is Medium or smaller, it is grabbed (DC 19 to escape). Until the grab ends, the target is restrained, and the demon can automatically tail the target, but cannot make tail attacks against other targets.

\textit{\textbf{Longsword.} Melee Weapon Attack}: +18 to hit, reach 1m, one target.

\textit{Hit:} 13 (2d8 + 4) slashing damage.

\textbf{Reactions}

\textit{\textbf{Parry.}} The demon adds 5 to its Defence against a melee attack that would hit it. To do so, the demon must be able to see its attacker and wield a melee weapon.

\textit{\textbf{Enraged}}:

- Marilith sharpens her swords together, each longsword attack gains Bleeding 1/20. Costs 2 Actions, lasts until the end of the fight.

- the marilith condemns the opponent to the abyss. Cost 2 Actions. The opponent must make a DC 24 Will save or be transported into the abyss.


\textbf{Ecology}\\
Environment: Any (Abyss)\\
Organization: Solitary, Pair, or Platoon (1 Marilith, 1-3 Glabrezu, and 3-14 Babau)\\
\textbf{Treasure}: Double (6 Longswords, other treasure)\\
\textbf{Description}\\
Rulers of demonic hordes and queens of abysmal nations, fearsome mariliths serve demon lords as rulers, advisors, and even lovers, yet their supremacy as strategists makes them especially in demand as generals and commanders of armies. The most powerful mariliths serve no one, instead commanding ravenous demonic legions.

A marilith stands 6 to 9 feet tall, 6 meters long from head to tip of tail, and weighs 2000kg. Only the most arrogant and proud of evil souls, usually those of cruel rulers, sadistic generals, and particularly violent warlords, can cause the birth of a marilith.

\

\index[Monsters]{Demon, Nalfeshnee}\textbf{Nalfeshnee}

\textit{Large fiend (demon), chaotic evil}

\textbf{STRENGTH} +5

\textbf{DEXTERITY} +0

\textbf{CONSTITUTION} +6

\textbf{INTELLIGENCE} +4

\textbf{WISDOM} +1

\textbf{CHARISMA} +2

\textbf{Initiative} +4 -- \textbf{Defence} 25

\textbf{Hit Points} 184 (16d10 + 96)

\textbf{Move} 6m, fly 9m

\textbf{Saving Throws} Fortitude +22, Reflexes +9, Will +21

\textbf{Damage Resistances} cold, electricity, fire; from a non-magical weapon

\textbf{Immunity to Damage} Poison

\textbf{Condition Immunity} poisoned

\textbf{Damage Vulnerability} cold iron

\textbf{Senses} darkvision 40m

\textbf{Languages} Abyssal, telepathy 36m

\textbf{Challenge} 13 (10000 XP)

\textit{\textbf{Resistance to Magic.}} The demon has +1d6 on Saving Throws against spells and other magical effects.

\textbf{Actions}

\textit{\textbf{Multiattack.}} The demon uses Halo of Horror if able. It then makes three attacks: one with its bite and two with its claws.

\textit{\textbf{Claw.} Melee Weapon Attack}: +18 to hit, reach 3m, one target.

\textit{Hit:} 15 (3d6 + 5) slashing damage, 2 bleed damage.

\textit{\textbf{Bite.} Melee Weapon Attack}: +18 to hit, reach 1m, one target.

\textit{Hit:} 32 (5d10 + 5) piercing damage and Demonic Fever.

\textit{Demonic Fever}: 1 minute, Saving Throw Fortitude DC 23, 4 hours, 3 successes, -1 Constitution and Wisdom/4 hours.


\textit{\textbf{Halo of Horror (Cooldown 5-6).}} The demon emits a shimmering, multicolored magical light. Any creature within 5 meters of the demon that can see the light must succeed on a DC 19 Will save or be frightened for 1 minute. A creature can repeat the Saving Throw at the end of each of its rounds, ending the effect on itself on a successful one. If the creature's Saving Throw succeeds or the effect ends for it, the creature is immune to the Halo of
Horror of the moaning demon for the next 24 hours.

\textit{\textbf{Teleport.}} The demon teleports, along with any equipment it is wearing or carrying, to an unoccupied space that it can see up to 16 meters away. Cost 1 Movement Action.

\textit{\textbf{Enraged}}: The nalfeshnee mimes arcane words and gestures seen within the previous 3 rounds and casts a spell it has witnessed. Cost 3 Actions.


\textbf{Ecology}
Environment: Any (Abyss)\\
Organization: Solitary or Warband (1 nalfeshnee, 1 Hezrou, and 2-5 Vrock)\\
\textbf{Treasure}: Standard\\
\textbf{Description}\\
Few demons understand the internal mechanics that govern the Abyss like the nalfeshnee, and it is not uncommon for these demons to serve the Abyss itself rather than a demon lord. Some oversee the organic realms that spawn new demons, while others guard places of special significance in the hidden recesses of the plane. Often a nalfeshnee's realm in the Abyss is superior in strength and size to the greatest of mortal realms, as these demons have a natural disposition to rule and impose some kind of order on the chaos of the Abyss. Mortal summoners often call upon them for their insane but unrivaled intellect, scrutinizing agreements made with these demons to avoid any hidden consequences and unintended implications, for a nalfeshnee rarely accepts anything that, in some twisted way, doesn't allow it to satisfy the needs and desires of the Abyss.

Nalfeshnee stand 6 meters tall and weigh 4000 kg. They are created from the souls of evil greedy or greedy mortals, especially those who have reigned over empires of slavery, theft, brigandage, and other even more violent vices.

\

\index[Monsters]{Demon, Nightmare Steed}\textbf{Nightmare Steed}

\textit{Large demon, Neutral Evil}

\textbf{STRENGTH} +4

\textbf{DEXTERITY} +2

\textbf{CONSTITUTION} +3

\textbf{INTELLIGENCE} +0

\textbf{WISDOM} +1

\textbf{CHARISMA} +2

\textbf{Initiative} +2 -- \textbf{Defence} 15

\textbf{Hit Points} 68 (8d10 + 24)

\textbf{Move} 18m, fly 24m

\textbf{Saving Throws}: Fortitude +6, Reflexes +5, Will +4

\textbf{Damage Immunity} Fire

\textbf{Languages} understands Abyssal, Common, and Infernal but cannot speak

\textbf{Challenge} 3 (700 XP)

\textit{\textbf{Grant Fire Resistance.}} The nightmarish steed can grant resistance to fire damage to anyone who rides it.

\textit{\textbf{Illumination.}} The nightmarish steed sheds bright light in a 3m radius and dim light for an additional 3 meter.

\textbf{Actions}

\textit{\textbf{Hooves.} Melee Weapon Attack}: +6 to hit, reach 3m, one target.

\textit{Hit:} 13 (2d8 + 4) bludgeoning damage plus 7 (2d6) fire damage.

\textit{\textbf{Ethereal Step.}} The nightmarish steed and up to three willing creatures within 3 meter of it can magically enter the Ethereal Plane from the Material Plane, and vice versa.

\textbf{Ecology}\\
Environment: Any\\
Organization: Solitary\\
\textbf{Treasure}: None\\
\textbf{Description}\\
Nightmares are fiery messengers of death. They allow only the most evil creatures to ride them, and they are never just mounts, but help in the destruction wrought by their riders.

\begin{changemargin}{0.3cm}{0.3cm}\begin{emphasis}{Hell is empty, all the devils are here. (William Shakespeare, The Tempest)}\end{emphasis}\end{changemargin}\

\index[Monsters]{Demon, Orcus}\textbf{Orcus}

\textit{Huge fiend (demon prince), chaotic evil}

\textbf{STRENGTH} +8

\textbf{DEXTERITY} +2

\textbf{CONSTITUTION} +7

\textbf{INTELLIGENCE} +5

\textbf{WISDOM} +5

\textbf{CHARISMA} +7

\textbf{Initiative} +5 -- \textbf{Defence} 30

\textbf{Hit Points} 390 (26d8+182)

\textbf{Movement} 15 meters, fly 15 meters

\textbf{Saving Throws}: Fortitude +33, Reflexes +28, Will +31

\textbf{Skills} all +13

\textbf{Damage Resistances} cold, electricity, fire

\textbf{Immunity to Damage} Void, Poison; weapons +2

\textbf{Condition Immunity} charmed, poisoned, paralyzed, fatigued, frightened

\textbf{Senses} True Seeing 40 m

\textbf{Languages} all, telepathy 45m

\textbf{Challenge} 26 (90000 XP)

\textit{\textbf{Spells.}} Orcus has MP 17. His spellcasting ability is Charisma, +7 to hit on spell attacks. Orcus knows the following spells:

At will: detect magic, icy touch

level 3 (3 slots): \textit{Dispel Magic}

level 6 (3 slots): \textit{Create Undead}

level 9 (1 slot): \textit{Time Stop}

\textit{\textbf{Demon Nature.}} Orcus does not need air, food, drink, or sleep.

\textit{\textbf{Legendary Endurance (3 / Day).}} If the Orcus fails a Saving Throw, he can choose to succeed instead.

\textit{\textbf{Lord of the undead.}} Orcus can always decide the type of undead he creates and this remains under his control indefinitely, moreover he can cast the spell in any condition you find.

\textbf{Actions}

\textit{\textbf{Multiattack.} 2 wand attacks}: +30, reach 3 meter, one creature. All of Orcus's attacks are considered magical +3.

\textit{Hit:} 21 (3d8 + 8) bludgeoning damage + 13 (2d12) void

\textit{\textbf{Tail}} Orcus strikes with his tail. +30, reach 3 meter, one creature

\textit{Hit:} 21 (3d8 + 8) bludgeoning damage + 18 (4d8) poison

\textbf{Additional Actions}

The Orcus can perform 3 additional actions, chosen from those below and one per round only at the end of another creature's round.

\textbf{Tail.} The Orcus attacks with its tail. +30 to hit, reach 5m, one target. On hit 21 (3d8 + 8) bludgeoning damage + 18 (4d8) poison

\textbf{Taste of Death.} Orcus blasphemously casts the Flame Strike spell with Void damage

\textbf{Ecology}\\
Environment: Abyss\\
Organization: Unique\\
\textbf{Treasure}: Triple\\

\textbf{Description}
Orcus is the Daemon Prince of the undead. He favors the company and service of the undead. He wishes to see all life disappear and all life turn into a gigantic necropolis of the undead under his command. Orcus has the head and legs of a goat, ram-like horns, a puffy body, bat-like wings, and a long tail.


\

\index[Monsters]{Demon, Quasit}\textbf{Quasit}

\textit{Tiny fiend (demon, shapeshifter), chaotic evil}

\textbf{STRENGTH} -3

\textbf{DEXTERITY} +3

\textbf{CONSTITUTION} +0

\textbf{INTELLIGENCE} -2

\textbf{WISDOM} +0

\textbf{CHARISMA} +0

\textbf{Initiative} +3 -- \textbf{Defence} 14

\textbf{Hit Points} 7 (3d4)

\textbf{Movement} 12m (3m, fly 12m in bat form; 12m, climb 12m in centipede form; 12m, swim 12m in toad form)

\textbf{Saving Throws} Fortitude +1, Reflexes +4, Will +1

\textbf{Skills} Stealth +5

\textbf{Damage Resistances} cold, electricity, fire; from a non-magical weapon

\textbf{Immunity to Damage} Poison

\textbf{Immunity to Conditions} poisoned

\textbf{Senses} darkvision 40m

\textbf{Languages} Abyssal, Common

\textbf{Challenge} 1 (200 XP)

\textit{\textbf{Shapeshift.}} The demon can use his action to transform into a beastly bat, centipede, or toad form, or back to its true form. His stats are the same in all forms, although the attacks may vary for some forms. Any equipment he is wearing or carrying is not transformed. Upon death it returns to its true form.

\textit{\textbf{Resistance to Magic.}} The demon has +1d6 on Saving Throws against spells and other magical effects.

\textbf{Actions}

\textit{\textbf{Claws (Bite in Beast Form).} Melee Weapon Attack}: +4 to hit, reach 1m, one target. \textit{Hit:} 5 (1d4 + 3) piercing damage. If the target is a creature, it must succeed on a DC 10 Fortitude save or take 5 (2d4) poison damage and be poisoned, -1 Strenght and Dexterity, for 1 minute. The creature can repeat the Saving Throw at the end of each of its rounds, ending the effect on a success.

\textit{\textbf{Invisibility.}} The demon remains invisible until it attacks or ends its concentration. Anything the demon is carrying or wearing is invisible as long as it remains in contact with the demon.

\textit{\textbf{Fright (1/day).}} A creature of the demon's choice that is within 6 meters of him must succeed at a DC 11 Will save or be frightened for 1 minute. The target can repeat the save at the end of each of its rounds, with a -1d6 if the demon is in line of sight, ending the effect prematurely if the save is successful.

\textbf{Ecology}\\
Environment: Any (Abyss)\\
Organization: Solitary or Flock (2-12)\\
\textbf{Treasure}: Standard\\
\textbf{Description}\\
The quasit is perhaps the least powerful demon, but it is not among the least respected: even quasits consider themselves superior to the dretch hordes, and true to their nature, dretches lack the courage or drive to prove them wrong. A quasit's primary role in life is as a familiar in the service of a spellcaster, but those quasits who escape this humiliating servitude gain a will of their own and are far more dangerous. A typical quasit stands 30 cm tall and weighs only 5kg.

Unique among the demonic hordes, quasits are born not from the souls of dead evil mortals, but from living souls: when a spellcaster tries to call a quasit to him as a familiar, its soul brushes against the Abyss and it reacts, creating from its matter a quasit connected to the caster's soul and forging a powerful bond between the two.

Newly created quasits are born directly into the Material Plane, where they become familiars and, as long as they are subject to their master's will, hate and despise him, since they can feel the pulse of his soul and know they could aspire to something more. . A quasit serves, yet watches and watches for mistakes that could cost its master his life, or rather, allow it to turn against its master. When a quasit's master dies, it can attempt to follow its soul to the Great Beyond, succeeding at a DC 15 Will save. This effect functions as plane shift but only affects the quasit and transports it to the Abyss, causing it to become his master's soul, in larva form, rather than using it to create new demonic life forms. In this way, a quasit can use the newly captured soul to bargain with more powerful denizens of the lower planes, and perhaps achieve a vile "promotion" that transforms it into a more powerful life form.

Rarely does a quasit choose to ignore the death of its master and remain on the Material Plane in search of other ways to amuse itself: usually by settling in an urban area where there are many individuals to torment.

\

\index[Monsters]{Demon, Vrock}\textbf{Vrock}

\textit{Large fiend (demon), chaotic evil}

\textbf{STRENGTH} +3

\textbf{DEXTERITY} +2

\textbf{CONSTITUTION} +4

\textbf{INTELLIGENCE} -1

\textbf{WISDOM} +1

\textbf{CHARISMA} -1

\textbf{Initiative} +2 -- \textbf{Defence} 18

\textbf{Hit Points} 104 (11d10 + 44)

\textbf{Move} 12m, fly 18m

\textbf{Saving Throws} Fortitude +13, Reflexes +10, Will +6

\textbf{Damage Resistances} cold, electricity, fire; from a non-magical weapon

\textbf{Immunity to Damage} Poison

\textbf{Condition Immunity} poisoned

\textbf{Senses} darkvision 40m

\textbf{Languages} Abyssal, telepathy 36m

\textbf{Challenge} 6 (2300 XP)

\textit{\textbf{Resistance to Magic.}} The demon has +1d6 on Saving Throws against spells and other magical effects.

\textbf{Actions}

\textit{\textbf{Multiattack.}} The demon makes two attacks: one with its beak and one with its spurs.
or
\textit{\textbf{Beak.} Melee Weapon Attack}: +12 to hit, reach 1m, one target.

\textit{Hit:} 10 (2d6 + 3) piercing damage.

\textit{\textbf{Spurs.} Melee Weapon Attack}: +12 to hit, reach 1m, one target.

\textit{Hit:} 14 (2d10 + 3) slashing damage.

\textit{\textbf{Spore (Cooldown 6).}} A cloud of toxic spores spreads in a 5m radius around the demon. Spores spread around corners. Each creature in that area must succeed on a DC 16 Fortitude save or be poisoned. While poisoned in this way, a target takes 5 (1d10) poison damage at the start of each of its rounds. The target can repeat the Saving Throw at the end of each of its rounds, ending the effect on a successful one. Emptying a vial of holy water on the target also ends the effect.

\textit{\textbf{Stunning Screech (1/Day).}} The demon lets out a horrific screech. Any creature within 6 meters of it that can hear it, and is not a demon, must succeed on a DC 16 Fortitude save or be stunned until the end of the demon's next round.

\textit{\textbf{Enraged}}: The Vrock scrapes its beak with its spurs, making them even sharper. Until the end of the battle, damage caused by Beak and Spurs causes 1 Bleeding damage up to a maximum of 10 damage.


\textbf{Ecology}\\
Environment: Any (Abyss)\\
Organization: Solitary, pair, or gang (3-10)\\
\textbf{Treasure}: Standard\\
\textbf{Description}\\
Unholy champions of the Abyss, vrocks embody all the rage, hatred, and violence of this realm. As voracious and grotesquely opportunistic as the scavenger they resemble, vrocks delight in bloodshed, enjoying the sound and feel of ripping the still-pulsing intestines from a living creature.\\
A typical vrock stands 8 feet tall and weighs 220kg. These creatures usually originate from the souls of evil mortals filled with hatred and rage, particularly those who were professional criminals, mercenaries, or assassins.



\

\index[Monsters]{Devil, Bearded}\textbf{Bearded Devil}

\textit{Medium fiend (devil), lawful evil}

\textbf{STRENGTH} +3

\textbf{DEXTERITY} +2

\textbf{CONSTITUTION} +2

\textbf{INTELLIGENCE} -1

\textbf{WISDOM} +0

\textbf{CHARISMA} +0

\textbf{Initiative} +2 -- \textbf{Defence} 15

\textbf{Hit Points} 52 (8d8 + 16)

\textbf{Move} 9m

\textbf{Saving Throws} Fortitude +9, Reflexes +7, Will +3

\textbf{Damage Resistances} cold; from a non-magical or non-silver weapon

\textbf{Damage Immunity} Fire, poison

\textbf{Condition Immunity} poisoned

\textbf{Senses} darkvision 40m

\textbf{Languages} Infernal, telepathy 36m

\textbf{Challenge} 3 (700 XP)

\textit{\textbf{Resistance to Magic.}} The devil has +1d6 on Saving Throws against spells and other magical effects.

\textit{\textbf{Resolute.}} The devil cannot be frightened as long as he can see an allied creature within 10 meters of him.

\textit{\textbf{Devil's Sight.}} The devil's darkvision is not limited by magical darkness.

\textbf{Actions}

\textit{\textbf{Multiattack.}} The devil makes two attacks: one with his beard and one with his glaive.

\textit{\textbf{Beard.} Melee Weapon Attack}: +7 to hit, reach 1 meter, one creature.

\textit{Hit:} 6 (1d8 + 2) piercing damage, and the target must succeed on a DC 13 Fortitude save or be poisoned for 1 minute. While poisoned in this way, the target cannot regain Hit Points. The target can repeat the Saving Throw at the end of each of its rounds, ending the effect on a successful save.

\textit{\textbf{Glachion.} Melee Weapon Attack}: +7 to hit, reach 3m, one target.

\textit{Hit:} 8 (1d10 + 3) slashing damage. If the target is any creature, excluding constructs and undead, it must succeed at a Fortitude 13 save or lose 5 (1d10) Hit Points at the start of each of its rounds from the infernal wound. Each time the devil hits a wounded target with this attack, the damage dealt by the wound increases by 5 (1d10). Any creature can take an action to seal the wound with a successful DC 12 Wisdom (First Aid) check. The wound also closes if the target receives healing magic.

\textbf{Ecology}\\
Environment: Any (Hell)\\
Organization: Solitary, Pair, Squad (3-10), or Troop (10-40)\\
\textbf{Treasure}: Standard (Glaive, other treasure)\\
\textbf{Description}\\
Elite warriors of the infernal legions, bearded devils, or barbazu, fight savagely in the name of their infernal overlords and command brutal hordes of the damned into battle. They gather and train with their glaives forged in the underworld, among the vaults of Hell's third circle, Erebus, but inevitably return to the first circle, Avernus, to serve alongside the fearsome lord Barbatos.

Barbazu love to make charge attacks with their glaives and try to keep a 3m distance between themselves and their opponents so they can use their signature polearms to maximum effect. Against an opponent who has superior reach (or is able to avoid the devil's favorite tactic), they throw down their glaives and rely on their claws and hideous beards. In an upright stance, bearded devils stand over 2 meters tall (although the crouched stance they hold in battle often makes them appear shorter) and weigh more than 110kg.

\

\index[Monsters]{Devil, Bone}\textbf{Bone Devil}

\textit{Large fiend (devil), lawful evil}

\textbf{STRENGTH} +4

\textbf{DEXTERITY} +3

\textbf{CONSTITUTION} +4

\textbf{INTELLIGENCE} +1

\textbf{WISDOM} +2

\textbf{CHARISMA} +3

\textbf{Initiative} +3 -- \textbf{Defence} 24

\textbf{Hit Points} 142 (15d10 + 60)

\textbf{Move} 12m, fly 12m

\textbf{Saving Throws} Fortitude +12, Reflexes +12, Will +7

\textbf{Skills} Deceive +7, Sense Emotions +6

\textbf{Damage Resistances} cold; from a non-magical or non-silver weapon

\textbf{Damage Immunity} Fire, poison

\textbf{Condition Immunity} poisoned

\textbf{Senses} darkvision 40m

\textbf{Languages} Infernal, telepathy 36m

\textbf{Challenge} 9 (5000 XP)

\textit{\textbf{Resistance to Magic.}} The devil has +1d6 on Saving Throws against spells and other magical effects.

\textit{\textbf{Devil's Sight.}} The devil's darkvision is not limited by magical darkness.

\textbf{Actions}

\textit{\textbf{Multiattack.}} The devil makes three attacks: two with its claws and one with its sting, or one with its barbed pole weapon and one with its sting.

\textit{\textbf{Hooked Pole Weapon.} Melee Weapon Attack}: +12 to hit, reach 3m, one target.

\textit{Hit:} 17 (2d12 + 4) piercing damage. If the target is a Huge or smaller creature, it is grabbed (DC 14 to escape). Until the grab ends, the devil can't use his pole weapon on another target.

\textit{\textbf{Claw.} Melee Weapon Attack}: +12 to hit, reach 3m, one target.

\textit{Hit:} 8 (1d8 + 4) slashing damage, 1 bleed damage.

\textit{\textbf{Sting.} Melee Weapon Attack}: +12 to hit, reach 3m, one target.

\textit{Hit:} 13 (2d8 + 4) piercing damage plus 17 (5d6) poison damage, and the target must succeed on a DC 18 Fortitude save, or be poisoned, -1 Strenght and Dexterity, for 1 minute. The target can repeat the Saving Throw at the end of each of its rounds, ending the effect on a successful one.

\textit{\textbf{Enraged}}: The Bone Devil attacks all creatures around him with his Hooked Pole Weapon. All creatures within a 3-foot radius suffer an attack, without being grabbed. Cost 2 Actions.


\textbf{Ecology}\\
Environment: Any (Hell)\\
Organization: Solitary, squad (2-3), council (4-10), or contingent (1-3 ice devils, 2-6 horned devils, and 1-4 bone devils\\
\textbf{Treasure}: Standard (Icy Spear+1, other treasure)\\
\textbf{Description}\\
Enlightened strategists of Hell's armies, the insectoid ice devils are among the most ingenious and ruthless masterminds in Hell's legions. Known as a gelugon among the ranks of devils, an ice devil hides within its chest a frozen heart stolen from mortals, which allows it to make emotionless decisions. Born in the frozen circle of Cocytus, Hell's seventh circle, most ice devils migrate to Caina, the eighth circle, where they plot to damn the world from courts of icy steel. While they have the most alien and monstrous appearance of all devils, few others are accorded greater respect.

In combat, a gelugon sends his subordinates forward, so that he can evaluate the tactics, strengths and weaknesses of the opponent in the rear, and provide them with support with his magical abilities, avoiding catching them in the area of effect of his Spells: Attitude not due to a sense of camaraderie, but to the cold, logical truth that your allies can survive longer in a fight if they aren't exposed to friendly fire. Gelugons stand 10.5 meter tall and weigh approximately 400kg.


\

\index[Monsters]{Devil, Chain}\textbf{Chain Devil}

\textit{Medium fiend (devil), lawful evil}

\textbf{STRENGTH} +4

\textbf{DEXTERITY} +2

\textbf{CONSTITUTION} +4

\textbf{INTELLIGENCE} +0

\textbf{WISDOM} +1

\textbf{CHARISMA} +2

\textbf{Initiative} +2 -- \textbf{Defence} 20

\textbf{Hit Points} 85 (10d8 + 40)

\textbf{Move} 9m

\textbf{Saving Throws} Fortitude +9, Reflexes +4, Will +3

\textbf{Damage Resistances} cold; from non-magical weapons or not silver

\textbf{Damage Immunity} Fire, poison

\textbf{Condition Immunity} poisoned

\textbf{Senses} darkvision 40m

\textbf{Languages} Infernal, telepathy 36m

\textbf{Challenge} 8 (3900 XP)

\textit{\textbf{Resistance to Magic.}} The devil has +1d6 on Saving Throws against spells and other magical effects.

\textit{\textbf{Devil's Sight.}} The devil's darkvision is not limited by magical darkness.

\textbf{Actions}

\textit{\textbf{Multiattack.}} The devil makes two chain attacks.

\textit{\textbf{Chain.} Melee Weapon Attack}: +16 to hit, reach 3m, one target.

\textit{Hit:} 11 (2d6 + 4) slashing damage. The target is grappled (DC 14 to escape) if the devil isn't already grappling another creature. Until the grab ends, the target is restrained and takes 7 (2d6) piercing damage at the start of each of its rounds.

\textit{\textbf{Animate Chains (Recharges after 1 hour).}} Up to four chains that the devil can see and are within 18 meters of him produce sharp edges and animate under the devil's control, as long as those chains are neither worn nor carried by anyone else.

Each animated chain is an object with Defence 20, 20 Hit Points, resistance to piercing damage, and immunity to sonic damage. When the devil uses Multiattack during his round, he can use each animated chain to make one more chain attack. An animated chain can grab a creature on its own but can't make attacks while grabbing. An animate chain reverts to its inanimate state if it is reduced to 0 Hit Points or if the devil is incapacitated or dies.

\textbf{Reactions}

\textit{\textbf{Unnerving Mask.}} When a creature the devil can see begins its round within 10 meters of the devil, the devil can create an illusion to resemble lost love or a bitter rival of that creature. If the creature can see the devil, it must succeed at a DC 18 Will save or be frightened until the end of its round.

\textit{\textbf{Enraged}}: the Chain Devil waves his chains in front of him. Until the end of the fight the Defense is 23.

\textbf{Ecology}\\
Environment: Any\\
Organization: Solitary, pair, ring (3-6) or chain (7-20)\\
\textbf{Treasure}: Standard\\
\textbf{Description}
Often classified by the layman among the ranks of infernal devils, sadomasochists are not true devils. While some are known to live in Hell, they exist outside the hierarchies established by the underworld's gods and its archdevils and can sometimes be found on other planes, particularly the Plane of Shadow. Many suggest that they are natives of Hell that existed before the advent of evilkind, although others speculate that they were brought to the plane by some sadistic power. Regardless of their origins, they roam the planes indulging their desire to cause and receive suffering, seeking pain through violent abductions and sadistic depravities.


\

\index[Monsters]{Devil, Horned}\textbf{Horned Devil}

\textit{Large fiend (devil), lawful evil}

\textbf{STRENGTH} +6

\textbf{DEXTERITY} +3

\textbf{CONSTITUTION} +5

\textbf{INTELLIGENCE} +1

\textbf{WISDOM} +3

\textbf{CHARISMA} +3

\textbf{Initiative} +3 -- \textbf{Defence} 23

\textbf{Hit Points} 178 (17d10 + 85)

\textbf{Move} 6m, fly 18m

\textbf{Saving Throws} Fortitude +18, Reflexes +17, Will +13

\textbf{Damage Resistances} cold;

\textbf{Damage Immunity} Fire, Poison, weapons +1

\textbf{Condition Immunity} poisoned

\textbf{Senses} darkvision 36m

\textbf{Languages} Infernal, telepathy 36m

\textbf{Challenge} 11 (7200 XP)

\textit{\textbf{Resistance to Magic.}} The devil has +1d6 on Saving Throws against spells and other magical effects.

\textit{\textbf{Devil's Sight.}} The devil's darkvision is not limited by magical darkness.

\textbf{Actions}

\textit{\textbf{Multiattack.}} The devil makes three melee attacks: two with his pitchfork and one with his tail. He can use Throw Flame in place of any melee attack.

\textit{\textbf{Tail.} Melee Weapon Attack}: +18 to hit, reach 3m, one target.

\textit{Hit:} 10 (1d8 + 6) piercing damage. If the target is a creature, excluding constructs and undead, it must succeed on a Fortitude 18 or Bleed 10 (3d6) save. Each time the devil wounds the target with this attack, the damage dealt by the bleed increases by 10 (3d6).

\textit{\textbf{Pitchfork.} Melee Weapon Attack}: +18 to hit, reach 3m, one target.

\textit{Hit:} 15 (2d8 + 6) piercing damage.

\textit{\textbf{Sting.} Melee Weapon Attack}: +18 to hit, reach 3m, one target.

\textit{Hit:} 13 (2d8 + 4) piercing damage plus 17 (5d6) poison damage, and the target must succeed on a DC 18 Fortitude save, or be poisoned, -1 Strenght and Dexterity, for 1 minute. The target can repeat the Saving Throw at the end of each of its rounds, ending the effect if so succeeds.

\textit{\textbf{Throw Flame.} Ranged spell attack}: +7 to hit, range 45m, one target.

\textit{Hit:} 14 (4d6) fire damage. If the target is a flammable object that is not being worn or carried, it catches fire.

\textit{\textbf{Enraged}}: the Horned Devil drains the life that enemies are losing. Until the end of the next round, all Hit Points lost by Bleeding from wounds caused by him.


\textbf{Ecology}\\
Environment: Any (Hell)\\
Organization: Solitary, Pair, or Flock (3-10)\\
\textbf{Treasure}: Standard (Unholy Spiked Chain+1, other treasure)\\
\textbf{Description}\\
Among the deadliest warriors of the archdevils and skilled commanders of the lesser devils, horned devils proclaim the rules of Hell wherever they go. These greater devils are trained, forged, and reforged to be among the most relentless and obedient warriors in the multiverse. The horned devils of the infernal armies' troops are known as cornugon, while the largest among them are called malebranche.

A typical horned devil grows to a remarkable 2.7 meters tall, has wings with a spread of 4.2 meters, and weighs 350 kg.


\

\index[Monsters]{Devil, Ice}\textbf{Ice Devil}

\textit{Large fiend (devil), lawful evil}

\textbf{STRENGTH} +5

\textbf{DEXTERITY} +2

\textbf{CONSTITUTION} +4

\textbf{INTELLIGENCE} +4

\textbf{WISDOM} +2

\textbf{CHARISMA} +4

\textbf{Initiative} +4 -- \textbf{Defence} 25

\textbf{Hit Points} 180 (19d10 + 76)

\textbf{Move} 12m

\textbf{Saving Throws} Fortitude +18, Reflexes +16, Will +16

\textbf{Damage Immunity} cold, fire, poison, weapons +1

\textbf{Condition Immunity} poisoned

\textbf{Senses} blindsight 18m, darkvision 36m

\textbf{Languages} Infernal, telepathy 36m

\textbf{Challenge} 14 (11500 XP)

\textit{\textbf{Resistance to Magic.}} The devil has +1d6 on Saving Throws against spells and other magical effects.

\textit{\textbf{Devil's Sight.}} The devil's darkvision is not limited by magical darkness.

\textbf{Actions}

\textit{\textbf{Multiattack.}} The devil makes three attacks: one with its bite, one with its claws, and one with its tail. Alternatively he makes two attacks: one with his tail and one with a spear.

\textit{\textbf{Claws.} Melee Weapon Attack}: +21 to hit, reach 1m, one target.

\textit{Hit:} 10 (2d4 + 5) slashing damage plus 10 (3d6) cold damage, 1 bleed damage.

\textit{\textbf{Tail.} Melee Weapon Attack}: +21 to hit, reach 3m, one target.

\textit{Hit:} 12 (2d6 + 5) bludgeoning damage plus 10 (3d6) cold damage.

\textit{\textbf{Ice Lance.} Melee Weapon Attack}: +21 to hit, reach 3m, one target.

\textit{Hit:} 14 (2d8 + 5) piercing damage plus 10 (3d6) cold damage. If the target is a creature, it must succeed on a DC 21 Fortitude save, or have its speed reduced by 3 meter for 1 minute; during each of his rounds he can take only one action or a bonus action, but not both; he cannot perform reactions. The target can repeat the Saving Throw at the end of each of its rounds, ending the effect on itself on a success.

\textit{\textbf{Bite.} Melee Weapon Attack}: +10 to hit, reach 1m, one target.

\textit{Hit:} 12 (2d6 + 5) piercing damage plus 10 (3d6) cold damage. Fortidude Saving Throw DC 18 or Slowed 1/1r.

\textit{\textbf{Wall of Ice (Cooldown 6).}} The devil magically forms a wall of opaque ice on a solid surface that he can see within 20 meters of him. The wall is 30 centimeters thick and up to 9 meters wide by a maximum of 3 meters in height, or a hemispherical dome of up to 6 meters in diameter. When the wall appears, any creature in its space is pushed out of it by the shortest route. The creature chooses which side of the wall to end up on, unless the creature is incapacitated. The creature then makes a DC 21 Reflex save, taking 35 (10d6) cold damage on a failed save, or half as much damage on a successful one.

The wall lasts for 1 minute or until the devil is incapacitated or dies. The wall can be damaged and punctured; each 3m section has Defence 5, 30 Hit Points, vulnerability to fire damage, and immunity to acid, cold, void, and poison damage. If a section is destroyed, it leaves a film of freezing air in the space that the wall used to occupy. Whenever a creature ends up moving through this frigid air during a turn, whether willing or not, it must make a DC 21 Fortitude save, taking 17 (5d6) cold damage on a failed save, or half as much damage on a failed save. he succeeds. The chilly air dissipates as the rest of the wall vanishes.

\textit{\textbf{Enraged}}: the Ice Devil aims at the enemy's heart and tries to tear it out. The creature, within 3 feet, must make a DC 18 Fortitude save or have its heart ripped out.


\textbf{Ecology}\\
Environment: Any (Hell)\\
Organization: Solitary, squad (2-3), council (4-10), or contingent (1-3 ice devils, 2-6 horned devils, and 1-4 bone devils\\
\textbf{Treasure}: Standard (Icy Spear+1, other treasure)\\
\textbf{Description}\\

Enlightened strategists of Hell's armies, the insectoid ice devils are among the most ingenious and cruelest minds in Hell. An ice devil hides in its chest a frozen heart stolen from a mortal, which allows it to make decisions free from emotions. Born in the frozen circle of Cocytus, the seventh circle of hell, most ice devils migrate to Caina, the eighth circle, where they plot to damn the world. While they have the most alien and monstrous appearance of all devils, few others are accorded greater respect.

In combat he sends his subordinates forward, so as to be able to evaluate the tactics, strengths and weaknesses of the opponent in the rear, and provide them with support with his magical abilities, avoiding catching them in the area of ​​effect of his spells: attitude not due to a sense of camaraderie, but due to the cold and logical truth that his allies can survive longer in a fight if they are not exposed to friendly fire.

Ice Devils stand 3.6m tall and weigh approximately 400 kg.

\

\index[Monsters]{Devil, Pit}\textbf{Pit Devil}\hypertarget{diavolodellafossa}{}

\textit{Large fiend (devil), lawful evil}

\textbf{STRENGTH} +8

\textbf{DEXTERITY} +2

\textbf{CONSTITUTION} +7

\textbf{INTELLIGENCE} +6

\textbf{WISDOM} +4

\textbf{CHARISMA} +7

\textbf{Initiative} +6 -- \textbf{Defence} 29

\textbf{Hit Points} 300 (24d10 + 168)

\textbf{Move} 9m, fly 18m

\textbf{Saving Throws} Fortitude +27, Reflexes +21, Will +24

\textbf{Damage Resistances} cold; 

\textbf{Damage Immunity} Fire, poison, weapons +2

\textbf{Condition Immunity} poisoned

\textbf{Senses} True Seeing 36 m

\textbf{Languages} Infernal, telepathy 36m

\textbf{Challenge} 20 (25000 XP)

\textit{\textbf{Magic Weapon.}} The pit fiend's weapon attacks are magical.

\textit{\textbf{Aura of Fear.}} Any creature hostile to the devil that starts its round within 6 meters of it must make a DC 26 Will save, unless the devil is incapacitated. On a failed save, the creature is frightened until the start of its next round. If the creature's Saving Throw succeeds, the creature is immune to the devil's fear aura for the next 24 hours.

\textit{\textbf{Innate Spells.}} The pit fiend spellcasting ability is Charisma. The pit fiend can cast these spells innately, requiring no material components:

At will: \textit{detect magic, fireball}

3/day each: \textit{block monster, wall of fire}

\textit{\textbf{Resistance to Magic.}} The devil has +1d6 on Saving Throws against spells and other magical effects.

\textbf{Actions}

\textit{\textbf{Multiattack.}} The devil makes four attacks: one with its bite, one with its claw, one with its club, and one with its tail.

\textit{\textbf{Claw.} Melee Weapon Attack}: +30 to hit, reach 3m, one target.

\textit{Hit:} 17 (2d8 + 8) slashing damage, 3 bleed damage (to a maximum of 20).

\textit{\textbf{Tail.} Melee Weapon Attack}: +30 to hit, reach 3m, one target.

\textit{Hit:} 24 (3d10 + 8) bludgeoning damage.

\textit{\textbf{Mace.} Melee Weapon Attack}: +30 to hit, reach 3m, one target.

\textit{Hit:} 15 (2d6 + 8) bludgeoning damage plus 21 (6d6) fire damage.

\textit{\textbf{Bite.} Melee Weapon Attack}: +30 to hit, reach 1m, one target.

\textit{Hit:} 22 (4d6 + 8) piercing damage. The target must succeed at a DC 26 Fortitude save or be poisoned. While poisoned in this way, the target cannot regain Hit Points, and it takes 21 (6d6) poison damage at the start of each of its rounds. The poisoned target can repeat the Saving Throw at the end of each of its rounds, ending the effect on itself.

\textbf{Ecology}\\
Environment: Any (Hell)\\
Organization: Solitary, pair or council (3-9)\\
\textbf{Treasure}: Double\\
\textbf{Description}
Rulers of hellish realms, generals of the armies of Hell, and advisors to archdevils, pit fiends embody the terrible and frightening pinnacle of the evil race.

Massive, untamed physiques, and gifted with ingenious evil intellects, these diabolical tyrants possess great autonomy whether in the service of the archdevils or in their sovereignty over hellish wastelands of slaves or when busy subjugating the mortal worlds. Solid muscles stretch over their gigantic bodies, Armoured with thick, cutting plates capable of blocking almost any attack. Their dagger-sized jaws and bestial faces hide some of the most insidious masterminds in Hell.

Born deep within Nessus, the ninth and deepest layer of Hell, pit fiends are created from the ranks of cornugon and gelugon only by archdevils and their dukes. Though many travel to the upper layers and beyond Hell, commanding Hell's legions, most remain on Nessus, serving the courts of Hell's mighty or in dark covens of unmentionable purposes.

Pit devils are always over 5 meters tall, with a wingspan of over 6 meters, and weigh over 500kg.

Pit fiends are masters of fire, and they favor territories licked by flames. In Hell, their predisposition makes Avernus, Dis, Malebolge, Nessus, and Phlegethon the circles that most easily host their temple-citadels engulfed in flames. Fanatical obsessed with diabolical superiority and the most uncompromising obedience, pit fiends, if left to run amok, muster huge armies, combing the pits of Hell in search of the most depraved lemures to turn into true devils. Once certain that they have created the perfect legions, they turn their attention to the more vulnerable demiplanes and mortal worlds, relishing their conquest.

Servants of the archdevils or other unique infernal warlords, pit fiends devote themselves to their cause, obeying the will of nobles chosen by some Dark Patron in the hope that, one day, they will win the favor of the Prince of Darkness or of Hell itself. While obedient to the hierarchies of their own race, they are also strict in enforcing their rules, and if a pit fiend were to serve an unworthy master, they would deem it their duty to depose him. Thus, whether lords or servants, pit fiends embody the will of hell's relentless laws and ensure that only the most powerful devils can (or dare) thrive.

Only the most powerful of mortal spellcasters can or dare to summon a pit fiend. The reactions of this type of devil to the summoning are quick and premeditated, usually characterized by an uncontrollable fury at the idea that such an insignificant being could waste their immortal time. Anyone who is unable to face his burning anger is killed and his soul damned to Hell and placed at the service of the summoned devil. Those who can control these greater devils can also intrigue them.

A pit fiend may respectfully serve a mortal lord for centuries, but his purpose remains the same: to corrupt more and more of his soul, ensure his complete damnation, and when he finally dies, claim his soul and begin the process. to make him a totally corrupt lemur servant.

Pit fiends are aware that they are immortal and are intelligent enough to have incredibly disciplined patience. Thus the older pit fiends see in their legions the faces of the countless fools who once claimed to be their masters.


\

\index[Monsters]{Devil, Thorny}\textbf{Thorny Devil}

\textit{Little fiend (devil), lawful evil}

\textbf{STRENGTH} +0

\textbf{DEXTERITY} +2

\textbf{CONSTITUTION} +1

\textbf{INTELLIGENCE} +0

\textbf{WISDOM} +2

\textbf{CHARISMA} -1

\textbf{Initiative} +2 -- \textbf{Defence} 14

\textbf{Hit Points} 22 (5d6 + 5)

\textbf{Move} 6m, fly 12m

\textbf{Saving Throws}: Fortitude +3, Reflexes +4, Will +4

\textbf{Damage Resistances} cold; from a non-magical or non-silver weapon

\textbf{Damage Immunity} Fire, poison

\textbf{Condition Immunity} poisoned

\textbf{Senses} darkvision 40m

\textbf{Languages} Infernal, telepathy 36m

\textbf{Challenge} 2 (450 XP)

\textit{\textbf{Resistance to Magic.}} The devil has +1d6 on Saving Throws against spells and other magical effects.

\textit{\textbf{Flying.}} The devil does not provoke attacks of opportunity when he flies out of an enemy's reach.

\textit{\textbf{Limited Thorns.}} The devil has twelve tail spines. Used thorns grow back at midnight.

\textit{\textbf{Devil's Sight.}} The devil's darkvision is not limited by magical darkness.

\textbf{Actions}

\textit{\textbf{Multiattack.}} The devil makes two attacks: one with its bite and one with its pitchfork or two with its tail spines.

\textit{\textbf{Pitchfork.} Melee Weapon Attack}: +2 to hit, reach 1m, one target.

\textit{Hit:} 3 (1d6) piercing damage.

\textit{\textbf{Bite.} Melee Weapon Attack}: +2 to hit, reach 1m, one target.

\textit{Hit:} 5 (2d4) slashing damage.

\textit{\textbf{Tail Spine.} Ranged Weapon Attack}: +4 to hit, reach 6m, one target.

\textit{Hit:} 4 (1d4 + 2) piercing damage plus 3 (1d6) fire damage.

\textbf{Ecology}\\
Environment: Any (Hell)\\
Organization: Solitary, Pair, Group (3-5), or Platoon (6-11)\\
\textbf{Treasure}: Standard\\
\textbf{Description}\\
Sentinels of the vaults of Hell, jailers of the blackest souls, and living weapons of hellforges, barbed devils, known to diabolists as hamatula, impose their shackles on the damned and guard the nefarious work of the greater devils. A hamatula loves the feel of warm blood on its spines and prefers to leap into battle when given the opportunity to fight.

Hamatulas are collectors and organizers, and are favored allies of eager summoners, often bringing with them tempting treasures from the vaults of Hell or knowing the path to mortal riches. If left to roam free, these devils' hideouts often display the pierced trophies of old victims, hanging like perverse collections of insects on bloodstained walls.

Most hooked devils stand 7 feet tall and up and weigh 150kg, though their lean, muscular bodies appear larger due to the ever-growing spikes that protrude from their bodies, sharp as blades.

\

\index[Monsters]{Dinosaurs, Triceratops}\textbf{Triceratops}

\textit{Huge beast, unaligned}

\textbf{STRENGTH} +6

\textbf{DEXTERITY} -1

\textbf{CONSTITUTION} +3

\textbf{INTELLIGENCE} -4

\textbf{WISDOM} +0

\textbf{CHARISMA} -3

\textbf{Initiative} -1 -- \textbf{Defence} 16

\textbf{Hit Points} 95 (10d12 + 30)

\textbf{Move} 15m

\textbf{Saving Throws} Fortitude +15, Reflexes +8, Will +5

\textbf{Languages} -

\textbf{Challenge} 5 (1800 XP)

\textit{\textbf{Rrumbling Charge.}} If the triceratops moves at least 6 meters directly at a creature and hits it with a gore attack during the same turn, the target must succeed on a DC 15 Fortitude save or fall prone. If the target is prone, the triceratops can make a stomp attack against it as a bonus action.

\textbf{Actions}

\textit{\textbf{Gored.} Melee Weapon Attack}: +13 to hit, reach 1m, one target.

\textit{Hit:} 24 (3d10 + 6) piercing damage.

\textit{\textbf{Stamp.} Melee Weapon Attack}: +13 to hit, reach 1m, one prone creature.

\textit{Hit:} 22 (3d10 + 6) bludgeoning damage.

\textbf{Ecology}\\
Environment: Warm Plains\\
Organization: Solitary, pair, or herd (5-8)\\
\textbf{Treasure}: None\\
\textbf{Description}\\
The Triceratops is a quick-tempered and headstrong herbivore. A typical triceratops is 9 meters long and weighs 10,000 kg.

\

\index[Monsters]{Dinosaurs, Tyrannosaurus}\textbf{Tyrannosaurus}

\textit{Huge beast, unaligned}

\textbf{STRENGTH} +7

\textbf{DEXTERITY} +0

\textbf{CONSTITUTION} +4

\textbf{INTELLIGENCE} -4

\textbf{WISDOM} +1

\textbf{CHARISMA} -1

\textbf{Initiative} +0 -- \textbf{Defence} 17

\textbf{Hit Points} 136 (13d12 + 52)

\textbf{Move} 15m

\textbf{Saving Throws} Fortitude +15, Reflexes +12, Will +10

\textbf{Skills} Awareness +4

\textbf{Languages} -

\textbf{Challenge} 8 (3900 XP)

\textbf{Actions}

\textit{\textbf{Multiattack.}} The tyrannosaurus makes two attacks: one with its bite and one with its tail. It cannot make both attacks against the same target.

\textit{\textbf{Tail.} Melee Weapon Attack}: +14 to hit, reach 3m, one target.

\textit{Hit:} 20 (3d8 + 7) bludgeoning damage.

\textit{\textbf{Bite.} Melee Weapon Attack}: +14 to hit, reach 3m, one target.

\textit{Hit:} 33 (4d12 + 7) piercing damage. If the target is a Medium or smaller creature, it is grabbed (DC 17 to escape). Until the grab ends, the target is restrained, and the tyrannosaurus can't use the bite against another target.

\textit{\textbf{Enraged}}: the Tyrannosaurus is pervaded by a murderous fury. Attack any friendly or enemy creature. The attack roll gains +1d6 and the bite causes Bleeding 3/15.


\textbf{Ecology}\\
Environment Forests and Warm Plains\\
Organization: Solitary, Pair, or Pack (3-6)\\
\textbf{Treasure}: None\\
\textbf{Description}\\
Tyrannosaurus is a primary predator measuring 12 meters in length and weighing 7000 kg.

\

\textbf{Dinwolf (Direwolf)}\index[Monsters]{Dinwolf (Direwolf}

\textit{Large beast, unaligned}

\textbf{STRENGTH} +3

\textbf{DEXTERITY} +2

\textbf{CONSTITUTION} +2

\textbf{INTELLIGENCE} -2

\textbf{WISDOM} +1

\textbf{CHARISMA} -2

\textbf{Initiative} +2 -- \textbf{Defence} 15

\textbf{Hit Points} 37 (5d10 + 10)

\textbf{Move} 15m

\textbf{Saving Throws}: Fortitude +7, Reflexes +6, Will +2

\textbf{Skills} Stealth +4, Awareness +3

\textbf{Languages} -

\textbf{Challenge} 1 (200 XP)

\textit{\textbf{Hearing and Fine Smell.}} The wolf has +1d6 on Wisdom (Awareness) checks based on hearing or smell.

\textit{\textbf{Pack tactics.}} The wolf has +1d6 on attack rolls against a creature if at least one of the wolf's allies is within 1 meter of the creature and that ally isn't incapacitated.

\textbf{Actions}

\textit{\textbf{Bite.} Melee Weapon Attack}: +5 to hit, reach 1m, one target.

\textit{Hit:} 10 (2d6 + 3) piercing damage. If the target is a creature, it must succeed on a DC 13 Fortitude save or fall prone.

\

\textbf{Dobi}\index[Monsters]{Dobi}\\
\textit{Tiny Fairy}\\
\textbf{Strength}: -3\\
\textbf{Dexterity}: -1\\
\textbf{Constitution}: +2\\
\textbf{Intelligence}: -2\\
\textbf{Wisdom}: +1\\
\textbf{Charisma}: +3\\
\textbf{Defence}: 12 -- \textbf{Initiative}: +0\\
\textbf{Hit Points}: 6 (1d8 + 2)\\
\textbf{Movement}: 3m, Swim 9m\\
\textbf{Saving Throws}: Fortitude +2, Reflexes +0, Will +1 \\
\textbf{Senses}: Twilight Vision 18m \\
\textbf{Languages}: - \\
\textbf{Challenge} 0 (10 XP)\\
\textbf{Immunity}: To damage from non-magical non-bludgeoning weapons\\
\textbf{Resistance}: Slashing, Puncture\\
\textit{\textbf{Dobi}} The Dobi sticks, to move it you need to be kind and ask him.\\
\textit{\textbf{Dobi Dobi Dobi}} When the Dobi takes more than 3 Hit Points of damage with a non-bludgeoning weapon it splits into two smaller Dobi each with the same amount of Hit Points left as the previous Dobi. \\
\smallskip\textbf{Actions}\\
\textit{\textbf{Dobi Dobi}} The Dobi projects an aura of calm emotions like the spell of the same name, but no Saving Throw is allowed. The Dobi can only affect one creature at a time with his power.\\
\textbf{Ecology}\\
Environment: Swamps\\
Organization: group\\
\textbf{Treasure}: Accidental\\
\textbf{Description}\\
{\small "...I moved the leaves of the marsh and saw on the ground a strange ball of fur, about ten centimeters in diameter, of a light color. Curious, I picked it up, stroking its soft fur and scrutinized it carefully. It seemed to have no limbs or signs of having a snout with eyes, ears, mouth, but as soon as I stroked it the ball vibrated, making a squeak.
	
	Finally I saw two little black and lively eyes open in all that fur and then two round ears appearing, then two short but robust paws, suitable for jumping, resting on the ground and two more, always short but equipped with five toes each, at half height .
	
	- Dobi! - answered the little animal, expressing a sort of joy and enthusiasm. - Dobi dobi! -.
	
	- So cute! - I exclaimed, caressing him. He was the cutest pet I've ever seen. - But now I'll put you back down -.
	
	"Dobi," answered the furball.
	I put my hand down, but the animal didn't move. I tried to pull it away from my hand, but it stuck to the other. I took it with two fingers, pulling hard and quickly placed it on the ground, but it immediately jumped on my foot and stuck to it. I had to cross the marsh with the dobi attached to my foot, not counting the other four I found clinging to the armour."
	
	From \textit{Journey to the First World} by \textbf{Tristan Cassandiel}}


\

\textbf{Dog of Death}\index[Monsters]{Dog of Death}

\textit{Medium Monstrosity, Neutral Evil}

\textbf{STRENGTH} +2

\textbf{DEXTERITY} +2

\textbf{CONSTITUTION} +2

\textbf{INTELLIGENCE} -4

\textbf{WISDOM} +1

\textbf{CHARISMA} -2

\textbf{Initiative} +2 -- \textbf{Defence} 13

\textbf{Hit Points} 39 (6d8 + 12)

\textbf{Move} 12m

\textbf{Saving Throws}: Fortitude +4, Reflexes +5, Will +2

\textbf{Skills} Stealth +4, Awareness +5

\textbf{Senses} vision in the dark 36 m

\textbf{Languages} -

\textbf{Challenge} 1 (200 XP)

\textit{\textbf{Two-headed.}} The dog has +1d6 on Wisdom (Awareness) checks and on Saving Throws against blinded, charmed, deafened, frightened, stunned, or unconscious conditions.

\textbf{Actions}

\textit{\textbf{Multiattack.}} The dog makes two bite attacks.

\textit{\textbf{Bite.} Melee Weapon Attack}: +4 to hit, reach 1m, one target.

\textit{Hit:} 5 (1d6 + 2) piercing damage. If the target is a creature, it must succeed on a DC 12 Fortitude save against the disease. After every 24 hours, the creature must repeat the Saving Throw, reducing its maximum Hit Points by 5 (1d10) on a failure. This reduction lasts until the disease is cured. The creature dies if the disease reduces its maximum Hit Points to 0.

\textbf{Description}\\
The Dog of Death is a hideous two-headed hound that prowls plains, deserts, and dungeons.

\

\index[Monsters]{Doppelganger}\textbf{Doppelganger}

\textit{Medium monstrosity (shapeshifter), neutral}

\textbf{STRENGTH} +0

\textbf{DEXTERITY} +4

\textbf{CONSTITUTION} +2

\textbf{INTELLIGENCE} +0

\textbf{WISDOM} +1

\textbf{CHARISMA} +2

\textbf{Initiative} +4 -- \textbf{Defence} 16

\textbf{Hit Points} 52 (8d8 + 16)

\textbf{Move} 9m

\textbf{Saving Throws} Fortitude +4, Reflexes +5, Will +6

\textbf{Skills} Deceive +6, Sense Emotions +3

\textbf{Condition Immunity} charmed

\textbf{Senses} Darkvision 18m

\textbf{Languages} Common

\textbf{Challenge} 3 (700 XP)

\textit{\textbf{Shapeshift.}} The doppelganger can use his action to change her form into that of a Small or Medium humanoid he has seen, or back to his true form. His stats, other than size, are the same in all forms. Any equipment he is wearing or carrying is not transformed. Upon death it returns to its true form.

\textit{\textbf{Lurking.}} In the first round of combat, the doppelganger has +1d6 on attack rolls against any creature it has taken by surprise.

\textit{\textbf{Surprise Attack.}} If the doppelganger surprises a creature and hits it with an attack during the first round of combat, the target takes an additional 10 (3d6) damage from the attack.

\textbf{Actions}

\textit{\textbf{Multiattack.}} The doppelganger makes two melee attacks.

\textit{\textbf{Slam.} Melee Weapon Attack}: +6 to hit, reach 1m, one target.

\textit{Hit:} 7 (1d6 + 4) bludgeoning damage.

\textit{\textbf{Read Thoughts.}} The doppelganger magically reads the surface thoughts of a creature within 20 meters of it. The effect can penetrate barriers, but 1 meter of wood or earth, 40 inches of stone, 5 centimeters of metal, or a thin sheet of lead blocks it. While the target is within range, the doppelganger can continue to read its thoughts, as long as the doppelganger's concentration is not broken (such as a spell's concentration). While reading a target's mind, the doppelganger has +1d6 on Wisdom and Charisma checks against the target.

\textbf{Ecology}\\
Environment: Any\\
Organization: Solitary, pair or gang (3-6)\\
\textbf{Treasure}: NPC gear\\
\textbf{Description}\\
Doppelgangers are strange beings who can take the form of those they meet. In its natural form, the creature roughly resembles a humanoid, but slender and frail, with thin limbs and incompletely formed facial features. His complexion is pale, he is hairless, and his eyes are white and vacant.

Doppelgangers prefer to infiltrate societies where they can amass wealth and power, and see little prospect in founding cities with their own kind. Younger doppelgangers try their skills on small tribes of orcs or goblins, then move into more complex societies such as dwarven, elven, and human communities. Rather than becoming targets by occupying positions of leadership, they prefer to hold power from behind the throne, or use multiple identities to manipulate influential citizens or entire guilds.

Doppelgangers make excellent use of their natural mimicry to set ambushes, decoy traps, and infiltrate humanoid societies. While not usually evil, they are interested only in themselves and view all others as toys to be manipulated and deceived. They are very fond of invading human societies to satisfy their desires; some enjoy complex political games while others are constantly trying to change race, gender and love partners. While not the norm, those doppelgangers who use their gifts for cruel and sadistic purposes are very notorious, and these shapeshifters are primarily responsible for their race's sinister reputation. Certainly, a shapeshifting creature has an advantage when trying to avoid capture for its crimes, and some particularly malevolent doppelgangers enjoy severing love affairs by staging betrayals.

Insistent rumors speak of even more powerful doppelgangers capable not only of changing their appearance, but also of making their own abilities, memories and even extraordinary and supernatural abilities of the creatures they choose to impersonate.


\

\textbf{Draft Horse}\index[Monsters]{Draft Horse}

\textit{Large beast, unaligned}

\textbf{STRENGTH} +4

\textbf{DEXTERITY} +0

\textbf{CONSTITUTION} +1

\textbf{INTELLIGENCE} -4

\textbf{WISDOM} +0

\textbf{CHARISMA} -2

\textbf{Initiative} +0 -- \textbf{Defence} 11

\textbf{Hit Points} 19 (3d10 + 3)

\textbf{Move} 12m

\textbf{Saving Throws}: Fortitude +5, Reflexes +1, Will +2

\textbf{Languages} -

\textbf{Challenge} 1/4 (50 XP)

\textbf{Actions}

\textit{\textbf{Hooves.} Melee Weapon Attack}: +6 to hit, reach 1m, one target.

\textit{Hit:} 9 (2d4 + 4) bludgeoning damage.

\medskip

\index{Dragons, Monster}

\begin{changemargin}{0cm}{0.5cm}\begin{emphasis}{
Oh cursed may Lynx close all your portals\\
Oh assassins may Sumkjir exterminate you\\
O ravagers may Nedraf break your bones!\\
(curses against Dragons)}\end{emphasis}\end{changemargin}\medskip

Dragons are fearsome, dangerous, ancient creatures; they represent power itself.

Each Dragon has full access to all spells from a specific spell list depending on its color.

This access is granted by Tàhil or Ljust depending on whether they are dragons loyal to one or the other.

It is from this distinction that dragons are divided into the Dragons of Tàhil and of Ljust. The former represent chaos, destruction, violence and death in various capacities and degrees, while the Dragons of Ljust are the emblem of good, justice, correctness, protection. While Tàhil's dragons are usually definied chromatic dragons, Ljust's dragons are called metallic dragons.

The Dragons of Ljust present on Yeru are transport errors, perhaps because the portal of Tàhil opened while an evil dragon fought a good dragon.

\textbf{Critically failing the Saving Throw} versus breath weapon of dragon double the damage, while critically saving will not halve further.\index{Dragon Breath}

\begin{itemize}[leftmargin=*]
	\item Each Dragon can cast spells up to a level equal to a quarter of his Challenge Rank, with access to the first level at least.
	\item Each Dragon has a number of Spell Points equal to 5 times its Challenge Rating
	\item Each Dragon has a Magic Proficiency score equal to half its Challenge Rating
\end{itemize}

\medskip

\textbf{Table: Access List of Spells for Dragons}\index{Table Access List of Spells for Dragons}

\medskip

\begin{tabular}{ll}
	\hline
	\textbf{Dragon Color}& \textbf{Spell List Name} \\
	White& Water\\
	Blue& Air\\
	Yellow & Fire, Summoning\\
	Black&Water, Necromancy\\
	Purple&Earth\\
	Red&Fire\\
	Green&Animals and Plants\\
	Silver&Transmutation, Illusion\\
	Bronze&Abjuration\\
	Gold&Heal, Summon\\
	Brass&Divination\\
	Copper&Invocation\\
\end{tabular}

\medskip

All Dragons have access to the Universal magic list and favor certain spells that are noted in their descriptions.

\textit{Note}: unfortunately it is not true that the dragons have only been here for 300 years, in reality the last victory of the millennium made us forget that these creatures have been on Yeru for many millennia and have been sowing destruction, chaos and death for just as long.

In the Description of each Ancient Dragon you will find a brief description of the type of dragon.\\

\index[Monsters]{Dragon, Adult Black}\textbf{Adult Black Dragon}

\textit{Huge dragon, chaotic evil}

\textbf{STRENGTH} +6

\textbf{DEXTERITY} +2

\textbf{CONSTITUTION} +5

\textbf{INTELLIGENCE} +2

\textbf{WISDOM} +1

\textbf{CHARISMA} +3

\textbf{Initiative} +2 -- \textbf{Defence} 28

\textbf{Hit Points} 195 (17d12 + 85)

\textbf{Move} 12m, climb 12m, fly 24m

\textbf{Saving Throws} Fortitude +22, Reflexes +19, Will +18

\textbf{Skills} Stealth +7, Awareness +11

\textbf{Damage Immunity} acid

\textbf{Senses} darkvision 40m, blindsight 20m

\textbf{Languages} Common, Draconic

\textbf{Challenge} 17 (18000 XP)

\textit{\textbf{Amphibious.}} The dragon can breathe air and water.

\textit{\textbf{Legendary Endurance (3 / Day).}} If the dragon fails a Saving Throw, it may choose to succeed instead.

\textbf{Actions}

\textit{\textbf{Multiattack.}} The dragon can use its Frightening Presence. Then make three attacks: one with the bite and two with the claws.

\textit{\textbf{Claw.} Melee Weapon Attack}: +26 to hit, reach 1m, one target.

\textit{Hit:} 13 (2d6 + 6) slashing damage, 1 bleed damage.

\textit{\textbf{Tail.} Melee Weapon Attack}: +26 to hit, reach 5 meters, one target.

\textit{Hit:} 15 (2d8 + 6) bludgeoning damage.

\textit{\textbf{Bite.} Melee Weapon Attack}: +26 to hit, reach 3m, one target.

\textit{Hit:} 17 (2d10 + 6) piercing damage plus 4 (1d8) acid damage.

\textit{\textbf{Fearful Presence.}} Any creature of the dragon's choice that is within 16 meters of it and aware of its presence must succeed at a DC 20 Will save or be frightened for 1 minute. A creature can repeat the Saving Throw at the end of each of its rounds, ending the effect on a successful one. If the creature's Saving Throw succeeds or the effect ends for it, the creature is immune to the dragon's Dreadful Presence for the next 24 hours.

\textit{\textbf{Acid Breath (Recharge 5-6).}} The dragon exhales acid in a 20m line that is 1 meter wide. Each creature in that area must make a DC 20 Reflex save and take 54 (12d8) acid damage on a failed save, or half as much damage on a failed save.

\textbf{Additional Actions}

The dragon can perform 3 additional Actions, chosen from the options below. He can only use one Additional Action at a time, and only at the end of another creature's turn. The dragon recovers expended additional Actions at the start of its round.

\textbf{Wing Attack (Costs 2 Actions).} The dragon flaps its wings. Each creature within 3 meter of the dragon must succeed on a DC 19 Reflex save or take 18 (2d6 + 6) bludgeoning damage and be knocked prone. The dragon can then fly for up to half of its flight move.

\textbf{Tail Attack.} The dragon performs a tail attack.

\textbf{Locate.} The dragon makes a Wisdom (Awareness) check.

\textit{\textbf{Enraged}}: The Adult Black Dragon can perform these actions for 2 Actions.

\textit{Focus}: the creature interrupts a mental effect on itself in progress

\textit{Brutality}: the creature attacks with unprecedented ferocity. +1d6 to attack roll, an additional 6 is counted in the critical count until the end of the fight.


\textbf{Ecology}\\
Environment: Warm Swamps\\
Organization: Solitary\\
\textbf{Treasure}: Triple\\
\textbf{Description}\\
Read Ancient Black Dragon Description.\\
\textbf{Enchantments}\index{Black Dragon Spell}\\
This dragon's favorite spells are:\\
- Create Undead\\
- Raise Dead\\
- Bestow Curse


\

\index[Monsters]{Dragon, Adult Blue}\textbf{Adult Blue Dragon}

\textit{Huge Dragon, Lawful Evil}

\textbf{STRENGTH} +7

\textbf{DEXTERITY} +0

\textbf{CONSTITUTION} +6

\textbf{INTELLIGENCE} +3

\textbf{WISDOM} +2

\textbf{CHARISMA} +4

\textbf{Initiative} +3 -- \textbf{Defence} 27

\textbf{Hit Points} 225 (18d12 + 108)

\textbf{Move} 12m, dig 12m, fly 24m

\textbf{Saving Throws} Fortitude +21, Reflexes +16, Will +18

\textbf{Skills} Stealth +5, Awareness +12

\textbf{Damage Immunity} Electricity

\textbf{Senses} darkvision 40m, blindsight 20m

\textbf{Languages} Common, Draconic

\textbf{Challenge} 16 (15000 XP)

\textit{\textbf{Legendary Endurance (3 / Day).}} If the dragon fails a Saving Throw, it may choose to succeed instead.

\textbf{Actions}

\textit{\textbf{Multiattack.}} The dragon can use its Frightening Presence. Then make three attacks: one with the bite and two with the claws.

\textit{\textbf{Claw.} Melee Weapon Attack}: +26 to hit, reach 1m, one target.

\textit{Hit:} 14 (2d6 + 7) slashing damage, 1 bleed damage.

\textit{\textbf{Tail.} Melee Weapon Attack}: +26 to hit, reach 5 meters, one target.

\textit{Hit:} 16 (2d8 + 7) bludgeoning damage.

\textit{\textbf{Bite.} Melee Weapon Attack}: +26 to hit, reach 3m, one target.

\textit{Hit:} 18 (2d10 + 7) piercing damage plus 5 (1d10) electricity damage.

\textit{\textbf{Fearful Presence.}} Any creature of the dragon's choice that is within 16 meters of it and aware of its presence must succeed on a DC 23 Will save or be frightened for 1 minute. A creature can repeat the Saving Throw at the end of each of its rounds, ending the effect on a successful one. If the creature's Saving Throw succeeds or the effect ends for it, the creature is immune to the dragon's Dreadful Presence for the next 24 hours.

\textit{\textbf{Lightning Breath (Cooldown 5-6).}} The dragon exhales lightning in a line 27 meters long and 1 meter wide. Each creature in that line must make a DC 23 Reflex save and take 66 (12d10) electricity damage on a failed save, or half as much damage on a successful one.

\textbf{Additional Actions}

The dragon can perform 3 additional Actions, chosen from the options below. He can only use one Additional Action at a time, and only at the end of another creature's turn. The dragon recovers expended additional Actions at the start of its round.

\textbf{Wing Attack (Costs 2 Actions).} The dragon flaps its wings. Each creature within 3 meter of the dragon must succeed on a DC 23 Reflex save or take 14 (2d6 + 7) bludgeoning damage and be knocked prone. The dragon can then fly for up to half of its flight move.

\textbf{Tail Attack.} The dragon performs a tail attack.

\textbf{Locate.} The dragon makes a Wisdom (Awareness) check.

\textit{\textbf{Enraged}}: The Adult Blu Dragon can perform these actions for 2 Actions.

\textit{Focus}: the creature interrupts a mental effect on itself in progress

\textit{Brutality}: the creature attacks with unprecedented ferocity. +1d6 to attack roll, an additional 6 is counted in the critical count until the end of the fight.

\textbf{Ecology}\\
Environment: Mountain peaks\\
Organization: Solitary\\
\textbf{Treasure}: Triple\\
\textbf{Description}\\
Read Ancient Blu Dragon Description.\\
\textbf{Enchantments}\index{Blu Dragon Spell}\\
This dragon's favorite spells are:\\
- Deadly Mist\\
- Call Lightning\\
- Ice Storm


\

\index[Monsters]{Dragon, Adult Brass}\textbf{Adult Brass Dragon}

\textit{Huge Dragon, Chaotic Good}

\textbf{STRENGTH} +6

\textbf{DEXTERITY} +0

\textbf{CONSTITUTION} +5

\textbf{INTELLIGENCE} +2

\textbf{WISDOM} +1

\textbf{CHARISMA} +3

\textbf{Initiative} +2 -- \textbf{Defence} 25

\textbf{Hit Points} 172 (15d12 + 75)

\textbf{Move} 12m, dig 9m, fly 24m

\textbf{Saving Throws} Fortitude +18, Reflexes +13, Will +14

\textbf{Skills} Stealth +5, Awareness +11, Deceive +8, History +7

\textbf{Damage Immunity} Fire

\textbf{Senses} darkvision 40m, blindsight 20m

\textbf{Languages} Common, Draconic

\textbf{Challenge} 13 (10000 XP)

\textit{\textbf{Legendary Endurance (3 / Day).}} If the dragon fails a Saving Throw, it may choose to succeed instead.

\textbf{Actions}

\textit{\textbf{Multiattack.}} The dragon can use its Frightening Presence. Then make three attacks: one with the bite and two with the claws.

\textit{\textbf{Claw.} Melee Weapon Attack}: +20 to hit, reach 1m, one target.

\textit{Hit:} 13 (2d6 + 6) slashing damage, 1 bleed damage.

\textit{\textbf{Tail.} Melee Weapon Attack}: +20 to hit, reach 5 meters, one target.

\textit{Hit:} 15 (2d8 + 6) bludgeoning damage.

\textit{\textbf{Bite.} Melee Weapon Attack}: +20 to hit, reach 3m, one target.

\textit{Hit:} 17 (2d10 + 6) piercing damage.

\textit{\textbf{Fearful Presence.}} Any creature of the dragon's choice that is within 16 meters of it and aware of its presence must succeed on a DC 20 Will save or be frightened for 1 minute. A creature can repeat the Saving Throw at the end of each of its rounds, ending the effect on a successful one. If the creature's Saving Throw succeeds or the effect ends for it, the creature is immune to the dragon's Dreadful Presence for the next 24 hours.

\textit{\textbf{Breath Weapon (Cooldown 5-6).}} The dragon uses one of the following breath weapons:

\textit{Fiery Breath.} The dragon exhales fire in a line 20 meters long and 1 meter wide. Each creature in the line must make a DC 20 Reflex save, taking 45 (13d6) fire damage on a failed save, or half as much damage on a successful one.

\textit{Sleep Breath.} The dragon exhales sleeping gas in a 20m cone. Each creature in that area must succeed on a Fortitude 20 save or be knocked unconscious for 10 minutes. This effect ends if the unconscious creature takes damage or someone takes an action to awaken it.

\textbf{Additional Actions}

The dragon can perform 3 additional Actions, chosen from the options below. He can only use one Additional Action at a time, and only at the end of another creature's turn. The dragon recovers expended additional Actions at the start of its round.

\textbf{Wing Attack (Costs 2 Actions).} The dragon flaps its wings. Each creature within 3 meter of the dragon must succeed on a DC 20 Reflex save or take 13 (2d6 + 6) bludgeoning damage and be knocked prone. The dragon can then fly up to half its flight move.

\textbf{Tail Attack.} The dragon performs a tail attack.

\textbf{Locate.} The dragon makes a Wisdom (Awareness) check.


\textit{\textbf{Enraged}}: The Adult Brass Dragon can perform these actions at a cost of 2 Actions.

\textit{Focus}: the creature interrupts a mental effect on itself in progress

\textit{Brutality}: the creature attacks with unprecedented ferocity. +1d6 to attack roll, an additional 6 is counted in the critical count until the end of the fight.


\textbf{Ecology}\\
Environment: Hot Deserts\\
Organization: Solitary\\
\textbf{Treasure}: Triple\\
\textbf{Description}\\
Excellent conversationalists, brass dragons prefer to talk rather than fight. Brass dragons lair near humanoid settlements, where they can hear the latest news and gossip.\\
\textbf{Enchantments}\index{Brass Dragon Spell}\\
This dragon's favorite spells are:\\
- True Seeing\\
- Knowledge of Legends\\
- Scrying

\

\index[Monsters]{Dragon, Adult Bronze}\textbf{Adult Bronze Dragon}

\textit{Huge Dragon, Chaotic Good}

\textbf{STRENGTH} +7

\textbf{DEXTERITY} +0

\textbf{CONSTITUTION} +6

\textbf{INTELLIGENCE} +3

\textbf{WISDOM} +2

\textbf{CHARISMA} +4

\textbf{Initiative} +3 -- \textbf{Defence} 27

\textbf{Hit Points} 212 (17d12 + 102)

\textbf{Movement} 12m, swim 12m, fly 24m

\textbf{Saving Throws} Fortitude +21, Reflexes +15, Will +17

\textbf{Skills} Stealth +5, Sense Emotions +7, Awareness +12

\textbf{Damage Immunity} Electricity

\textbf{Senses} darkvision 40m, blindsight 20m

\textbf{Languages} Common, Draconic

\textbf{Challenge} 15 (13000 XP)

\textit{\textbf{Amphibious.}} The dragon can breathe air and water.

\textit{\textbf{Legendary Endurance (3 / Day).}} If the dragon fails a Saving Throw, it may choose to succeed instead.

\textbf{Actions}

\textit{\textbf{Multiattack.}} The dragon can use its Frightful Presence and then make three attacks: one with its bite and two with its claws.

\textit{\textbf{Claw.} Melee Weapon Attack}: +24 to hit, reach 1m, one target.

\textit{Hit:} 14 (2d6 + 7) slashing damage, 1 bleed damage.

\textit{\textbf{Tail.} Melee Weapon Attack}: +24 to hit, reach 5 meters, one target.

\textit{Hit:} 16 (2d8 + 7) bludgeoning damage.

\textit{\textbf{Bite.} Melee Weapon Attack}: +24 to hit, reach 3m, one target.

\textit{Hit:} 18 (2d10 + 7) piercing damage.

\textit{\textbf{Fearful Presence.}} Any creature of the dragon's choice that is within 16 meters of it and aware of its presence must succeed on a DC 20 Will save or be frightened for 1 minute. A creature can repeat the Saving Throw at the end of each of its rounds, ending the effect on a successful one. If the creature's Saving Throw succeeds or the effect ends for it, the creature is immune to the dragon's Dreadful Presence for the next 24 hours.

\textit{\textbf{Breath Weapon (Reload 5-6).}} The dragon uses one of the following breath weapons:

\textit{Lightning Breath.} The dragon exhales lightning in a line 27 meters long and 1 meter wide. Each creature in the line must make a DC 20 Reflex save, taking 66 (12d10) electricity damage on a failed save, or half as much damage on a successful one. \textit{Repulsive Breath.} The dragon exhales repulsive energy in a 10m cone. Each creature in that area must succeed on a DC 20 Fortitude save or be pushed 20 meters away from the dragon.

\textit{\textbf{Shapeshift.}} The dragon can magically transform into a humanoid or beast whose challenge rating is equal to or lower than its own, or revert to its true form. Upon death it returns to its true form. Any equipment it is wearing or carrying is absorbed or carried into the new form (dragon's choice).

In the new form, the dragon retains its Traits, Hit Points, Hit Dice, speech, proficiencies, Legendary Endurance, lair actions, and Intelligence, Wisdom, and Charisma scores, in addition to this action. Its stats and abilities are otherwise replaced by those of the new form, except for the new form's additional Actions.

\textbf{Additional Actions}

The dragon can perform 3 additional Actions, chosen from the options below. He can only use one Additional action at a time, and only at the end of another creature's turn. The dragon recovers expended additional Actions at the start of its round.

\textbf{Wing Attack (Costs 2 Actions).} The dragon flaps its wings. Each creature within 3 meter of the dragon must succeed on a DC 20 Reflex save or take 14 (2d6 + 7) bludgeoning damage and be knocked prone. The dragon can then fly up to half its flight move.

\textbf{Tail Attack.} The dragon performs a tail attack.

\textbf{Locate.} The dragon makes a Wisdom (Awareness) check.

\textbf{Ecology}\\
Environment: Temperate Coastal Zones\\
Organization: Solitary\\
\textbf{Treasure}: Triple\\
\textbf{Description}\\
Bronze dragons are known to ally with travelers and adventurers if cause and reward are just and fitting\\
\textbf{Enchantments}\index{Bronze Dragon Spell}\\
This dragon's favorite spells are:\\
- Orb of Invulnerability\\
- Freedom of Movement


\

\index[Monsters]{Dragon, Adult Copper}\textbf{Adult Copper Dragon}

\textit{Huge Dragon, Chaotic Good}

\textbf{STRENGTH} +6

\textbf{DEXTERITY} +1

\textbf{CONSTITUTION} +5

\textbf{INTELLIGENCE} +4

\textbf{WISDOM} +2

\textbf{CHARISMA} +3

\textbf{Initiative} +4 -- \textbf{Defence} 25

\textbf{Hit Points} 184 (16d12 + 80)

\textbf{Move} 12m, climb 12m, fly 24m

\textbf{Saving Throws} Fortitude +19, Reflexes +15, Will +16

\textbf{Skills} Stealth +6, Deceive +8, Awareness +12

\textbf{Damage Immunity} acid

\textbf{Senses} darkvision 40m, blindsight 20m

\textbf{Languages} Common, Draconic

\textbf{Challenge} 14 (11500 XP)

\textit{\textbf{Legendary Endurance (3 / Day).}} If the dragon fails a Saving Throw, it may choose to succeed instead.

\textbf{Actions}

\textit{\textbf{Multiattack.}} The dragon can use its Frightening Presence. Then make three attacks: one with the bite and two with the claws.

\textit{\textbf{Claw.} Melee Weapon Attack}: +22 to hit, reach 1m, one target.

\textit{Hit:} 13 (2d6 + 6) slashing damage, 1 bleed damage.

\textit{\textbf{Tail.} Melee Weapon Attack}: +22 to hit, reach 5 meters, one target.

\textit{Hit:} 15 (2d8 + 6) bludgeoning damage.

\textit{\textbf{Bite.} Melee Weapon Attack}: +22 to hit, reach 3m, one target.

\textit{Hit:} 17 (2d10 + 6) piercing damage.

\textit{\textbf{Fearful Presence.}} Any creature of the dragon's choice that is within 16 meters of it and aware of its presence must succeed at a DC 20 Will save or be frightened for 1 minute. A creature can repeat the Saving Throw at the end of each of its rounds, ending the effect on a successful one. If the creature's Saving Throw succeeds or the effect ends for it, the creature is immune to the dragon's Dreadful Presence for the next 24 hours.

\textit{\textbf{Breath Weapon (Cooldown 5-6).}} The dragon uses one of the following breath weapons:

\textit{Acid Breath.} The dragon exhales acid in a line 20 meters long and 1 meter wide. Each creature in the line must make a DC 20 Reflex save, taking 54 (12d8) acid damage on a failed save, or half as much damage on a successful one.

\textit{Slowing Breath.} The dragon exhales gas in a 20m cone. Each creature in that area must succeed on a DC 20 Fortitude save. On a failed save, the creature can't use its reaction, has its speed halved, and can't make more than one attack during its round. Also, the creature can use either an action or a bonus action, but not both. These effects last for 1 minute. The creature can repeat the Saving Throw at the end of each of its rounds, ending the effect on itself on a success.

\textbf{Additional Actions}

The dragon can perform 3 additional Actions, chosen from the options below. He can only use one Additional Action at a time, and only at the end of another creature's turn. The dragon recovers expended additional Actions at the start of its round.

\textbf{Wing Attack (Costs 2 Actions).} The dragon flaps its wings. Each creature within 3 meter of the dragon must succeed on a DC 20 Reflex save or take 13 (2d6 + 6) bludgeoning damage and be knocked prone. The dragon can then fly up to half its flight move.

\textbf{Tail Attack.} The dragon performs a tail attack.

\textbf{Spot.} The dragon makes a Wisdom (Awareness) check.

\textit{\textbf{Enraged}}: The Adult Copper Dragon can perform these actions for 2 Actions.

\textit{Focus}: the creature interrupts a mental effect on itself in progress

\textit{Brutality}: the creature attacks with unprecedented ferocity. +1d6 to attack roll, an additional 6 is counted in the critical count until the end of the fight.


\textbf{Ecology}\\
Environment: Warm Hills\\
Organization: Solitary\\
\textbf{Treasure}: Triple\\
\textbf{Description}\\
This capricious dragon tries to hinder and frustrate his enemies during combat.\\
\textbf{Enchantments}\index{Copper Dragon Spell}\\
This dragon's favorite spells are:\\
- Blade Barrier\\
- Wall of Force\\
- Elastic Sphere


\

\index[Monsters]{Dragon, Adult Gold}\textbf{Adult Gold Dragon}

\textit{Huge Dragon, Lawful Good}

\textbf{STRENGTH} +8

\textbf{DEXTERITY} +2

\textbf{CONSTITUTION} +7

\textbf{INTELLIGENCE} +3

\textbf{WISDOM} +2

\textbf{CHARISMA} +7

\textbf{Initiative} +3 -- \textbf{Defence} 28

\textbf{Hit Points} 256 (19d12 + 133)

\textbf{Movement} 12m, swim 12m, fly 24m

\textbf{Saving Throws} Fortitude +24, Reflexes +19, Will +19

\textbf{Skills} Stealth +8, Sense Emotions +8, Awareness +14, Deceive +13

\textbf{Damage Immunity} Fire

\textbf{Senses} darkvision 40m, blindsight 20m

\textbf{Languages} Common, Draconic

\textbf{Challenge} 17 (18000 XP)

\textit{\textbf{Amphibious.}} The dragon can breathe air and water.

\textit{\textbf{Legendary Endurance (3 / Day).}} If the dragon fails a Saving Throw, it can choose to succeed instead.

\textbf{Actions}

\textit{\textbf{Multiattack.}} The dragon can use its Frightening Presence. Then make three attacks: one with the bite and two with the claws.

\textit{\textbf{Claw.} Melee Weapon Attack}: +28 to hit, reach 1m, one target.

\textit{Hit:} 15 (2d6 + 8) slashing damage, 1 bleed damage.

\textit{\textbf{Tail.} Melee Weapon Attack}: +28 to hit, reach 5 meters, one target.

\textit{Hit:} 17 (2d8 + 8) bludgeoning damage.

\textit{\textbf{Bite.} Melee Weapon Attack}: +28 to hit, reach 3m, one target.

\textit{Hit:} 19 (2d10 + 8) piercing damage.

\textit{\textbf{Fearful Presence.}} Any creature of the dragon's choice that is within 16 meters of it and aware of its presence must succeed on a DC 22 Will save or be frightened for 1 minute. A creature can repeat the Saving Throw at the end of each of its rounds, ending the effect on a successful one. If the creature's Saving Throw succeeds or the effect ends for it, the creature is immune to the dragon's Dreadful Presence for the next 24 hours.

\textit{\textbf{Breath Weapon (Cooldown 5-6).}} The dragon uses one of the following breath weapons:

\textit{Fiery Breath.} The dragon breathes fire in a 20m cone. Each creature in the area must make a DC 22 Reflex save, taking 66 (12d10) fire damage on a failed save, or half as much damage on a successful one.

\textit{Weakening Breath.} The dragon exhales gas in a 20m cone. Each creature in that area must succeed on a DC 22 Fortitude save or have -1d6 on Strength-based attack rolls, Strength checks, and Fortitude saves for 1 minute. A creature can repeat the Saving Throw at the end of each of its rounds, ending the effect on itself on a success.

\textit{\textbf{Shapeshift.}} The dragon can magically transform into a humanoid or beast whose challenge rating is equal to or lower than its own, or revert to its true form. Upon death it returns to its true form. Any equipment it is wearing or carrying is absorbed or carried into the new form (dragon's choice).

In the new form, the dragon retains its Traits, Hit Points, Hit Dice, speech, proficiencies, Legendary Endurance, lair actions, and Intelligence, Wisdom, and Charisma scores, in addition to this action. Its stats and abilities are otherwise replaced by those of the new form, except for the new form's additional Actions.

\textbf{Additional Actions}

The dragon can perform 3 additional Actions, chosen from the options below. He can only use one Additional Action at a time, and only at the end of another creature's turn. The dragon recovers expended additional Actions at the start of its round.

\textbf{Wing Attack (Costs 2 Actions).} The dragon flaps its wings. Each creature within 3 meter of the dragon must succeed on a DC 22 Reflex save or take 15 (2d6 + 8) bludgeoning damage and be knocked prone. The dragon can then fly up to half its flight move.

\textbf{Tail Attack.} The dragon performs a tail attack.

\textbf{Locate.} The dragon makes a Wisdom (Awareness) check.

\textit{\textbf{Enraged}}: The Adult Gold Dragon can perform these actions for 2 Actions.

\textit{Focus}: the creature interrupts a mental effect on itself in progress.

\textit{Brutality}: the creature attacks with unprecedented ferocity. +1d6 to attack roll, an additional 6 is counted in the critical count until the end of the fight.

\textbf{Ecology}\\
Environment: Hot Plains\\
Organization: Solitary\\
\textbf{Treasure}: Triple\\
\textbf{Description}\\
Golden dragons are the emblem of virtue. Other Ljust's dragons revere them as agents of divine powers and exemplary members of the draconic race, and often seek them for advice or help.\\
\textbf{Enchantments}\index{Gold Dragon Spell}\\
This dragon's favorite spells are:\\
- Heal\\
- Greater Restoration\\
- Black Tentacles


\

\index[Monsters]{Dragon, Adult Green}\textbf{Adult Green Dragon}

\textit{Huge Dragon, Lawful Evil}

\textbf{STRENGTH} +6

\textbf{DEXTERITY} +1

\textbf{CONSTITUTION} +5

\textbf{INTELLIGENCE} +4

\textbf{WISDOM} +2

\textbf{CHARISMA} +3

\textbf{Initiative} +4 -- \textbf{Defence} 27

\textbf{Hit Points} 207 (18d12 + 90)

\textbf{Movement} 12m, swim 12m, fly 24m

\textbf{Saving Throws} Fortitude +20, Reflexes +16, Will +17

\textbf{Skills} Stealth +6, Deceive +8, Sense Emotions +7, Awareness +12

\textbf{Immunity to Damage} Poison

\textbf{Condition Immunity} poisoned

\textbf{Senses} darkvision 40m, blindsight 20m

\textbf{Languages} Common, Draconic

\textbf{Challenge} 15 (13000 XP)

\textit{\textbf{Amphibious.}} The dragon can breathe air and water.

\textit{\textbf{Legendary Endurance (3 / Day).}} If the dragon fails a Saving Throw, it may choose to succeed instead.

\textbf{Actions}

\textit{\textbf{Multiattack.}} The dragon can use its Frightening Presence. Then make three attacks: one with the bite and two with the claws.

\textit{\textbf{Claw.} Melee Weapon Attack}: +23 to hit, reach 1m, one target.

\textit{Hit:} 13 (2d6 + 6) slashing damage, 1 bleed damage.

\textit{\textbf{Tail.} Melee Weapon Attack}: +23 to hit, reach 5 meters, one target.

\textit{Hit:} 15 (2d8 + 6) bludgeoning damage.

\textit{\textbf{Bite.} Melee Weapon Attack}: +23 to hit, reach 3m, one target.

\textit{Hit:} 17 (2d10 + 6) piercing damage plus 7 (2d6) poison damage.

\textit{\textbf{Fearful Presence.}} Any creature of the dragon's choice that is within 16 meters of it and aware of its presence must succeed at a DC 20 Will save or be frightened for 1 minute. A creature can repeat the Saving Throw at the end of each of its rounds, ending the effect on a successful one. If the creature's Saving Throw succeeds or the effect ends for it, the creature is immune to the dragon's Dreadful Presence for the next 24 hours.

\textit{\textbf{Venom Breath (Cooldown 5-6).}} The dragon exhales poisonous gas in a 20m cone. Each creature in that area must make a DC 20 Fortitude save and take 56 (16d6) poison damage on a failed save, or half as much damage on a successful one.

\textbf{Additional Actions}

The dragon can perform 3 additional Actions, chosen from the options below. He can only use one Additional Action at a time, and only at the end of another creature's turn. The dragon recovers expended additional Actions at the start of its round.

\textbf{Wing Attack (Costs 2 Actions).} The dragon flaps its wings. Each creature within 3 meter of the dragon must succeed on a DC 20 Reflex save or take 13 (2d6 + 6) bludgeoning damage and be knocked prone. The dragon can then fly up to half its flight move.

\textbf{Tail Attack.} The dragon performs a tail attack.

\textbf{Locate.} The dragon makes a Wisdom (Awareness) check.


\textit{\textbf{Enraged}}: The Adult Green Dragon can perform these actions for 2 Actions.

\textit{Focus}: the creature interrupts a mental effect on itself in progress

\textit{Brutality}: the creature attacks with unprecedented ferocity. +1d6 to attack roll, an additional 6 is counted in the critical count until the end of the fight.


\textbf{Ecology}\\
Environment: Temperate Forests\\
Organization: Solitary\\
\textbf{Treasure}: Triple\\
\textbf{Description}\\
Read Ancient Green Dragon Description\\
\textbf{Enchantments}\index{Green Dragon Spell}\\
This dragon's favorite spells are:\\
- Anti-Life Shell\\
- Locate Creature\\
- Remove poison



\

\index[Monsters]{Dragon, Adult Red}\textbf{Adult Red Dragon}

\textit{Huge dragon, chaotic evil}

\textbf{STRENGTH} +8

\textbf{DEXTERITY} +0

\textbf{CONSTITUTION} +7

\textbf{INTELLIGENCE} +3

\textbf{WISDOM} +1

\textbf{CHARISMA} +5

\textbf{Initiative} +3 -- \textbf{Defence} 28

\textbf{Hit Points} 256 (19d12 + 133)

\textbf{Move} 12m, climb 12m, fly 24m

\textbf{Saving Throws} Fortitude +23, Reflexes +17, Will +18

\textbf{Skills} Stealth +6, Awareness +13

\textbf{Damage Immunity} Fire

\textbf{Senses} darkvision 40m, blindsight 20m

\textbf{Languages} Common, Draconic

\textbf{Challenge} 17 (18000 XP)

\textit{\textbf{Legendary Endurance (3 / Day).}} If the dragon fails a Saving Throw, it may choose to succeed instead.

\textbf{Actions}

\textit{\textbf{Multiattack.}} The dragon can use its Frightful Presence and then make three attacks: one with its bite and two with its claws.

\textit{\textbf{Claw.} Melee Weapon Attack}: +28 to hit, reach 1m, one target.

\textit{Hit:} 15 (2d6 + 8) slashing damage, 1 bleed damage.

\textit{\textbf{Tail.} Melee Weapon Attack}: +28 to hit, reach 5 meters, one target.

\textit{Hit:} 17 (2d8 + 8) bludgeoning damage.

\textit{\textbf{Bite.} Melee Weapon Attack}: +28 to hit, reach 3m, one target.

\textit{Hit:} 19 (2d10 + 8) piercing damage plus 7 (2d6) damage from
fire.

\textit{\textbf{Fearful Presence.}} Any creature of the dragon's choice that is within 16 meters of it and aware of its presence must succeed at a DC 20 Will save or be frightened for 1 minute. A creature can repeat the Saving Throw at the end of each of its rounds, ending the effect on a successful one. If the creature's Saving Throw succeeds or the effect ends for it, the creature is immune to the dragon's Dreadful Presence for the next 24 hours.

\textit{\textbf{Fiery Breath (Cooldown 5-6).}} The dragon exhales fire in a 20m cone. Each creature in that area must make a DC 22 Reflex save and take 63 (18d6) fire damage on a failed save, or half as much damage on a successful one.

\textbf{Additional Actions}

The dragon can perform 3 additional Actions, chosen from the options below. He can only use one Additional Action at a time, and only at the end of another creature's turn. The dragon recovers expended additional Actions at the start of its round.

\textbf{Wing Attack (Costs 2 Actions).} The dragon flaps its wings. Each creature within 3 meter of the dragon must succeed on a DC 22 Reflex save or take 15 (2d6 + 8) bludgeoning damage and be knocked prone. The dragon can then fly up to half its flight move.

\textbf{Tail Attack.} The dragon performs a tail attack.

\textbf{Locate.} The dragon makes a Wisdom (Awareness) check.


\textit{\textbf{Enraged}}: The Adult Red Dragon can perform these actions for 2 Actions.

\textit{Focus}: the creature interrupts a mental effect on itself in progress.

\textit{Brutality}: the creature attacks with unprecedented ferocity. +1d6 to attack roll, an additional 6 is counted in the critical count until the end of the fight.


\textbf{Ecology}\\
Environment: Hot Mountains\\
Organization: Solitary\\
\textbf{Treasure}: Triple\\
\textbf{Description}\\
Read Ancient Red Dragon Description\\
\textbf{Enchantments}\index{Red Dragon Spell}\\
This dragon's favorite spells are:\\
- Fireball\\
- Incendiary Cloud\\
- Wall of fire


\

\index[Monsters]{Dragon, Adult Silver}\textbf{Adult Silver Dragon}

\textit{Huge Dragon, Lawful Good}

\textbf{STRENGTH} +8

\textbf{DEXTERITY} +0

\textbf{CONSTITUTION} +7

\textbf{INTELLIGENCE} +3

\textbf{WISDOM} +1

\textbf{CHARISMA} +5

\textbf{Initiative} +3 -- \textbf{Defence} 27

\textbf{Hit Points} 243 (18d12 + 126)

\textbf{Move} 12m, fly 24m

\textbf{Saving Throws} Fortitude +22, Reflexes +16, Will +17

\textbf{Skills} Arcane +8, Hide / Stealth +5, Awareness +11, History +8

\textbf{Damage Immunity} cold

\textbf{Senses} darkvision 40m, blindsight 20m

\textbf{Languages} Common, Draconic

\textbf{Challenge} 16 (1500 XP)

\textit{\textbf{Legendary Endurance (3 / Day).}} If the dragon fails a Saving Throw, it may choose to succeed instead.

\textbf{Actions}

\textit{\textbf{Multiattack.}} The dragon can use its Frightening Presence. Then make three attacks: one with the bite and two with the claws.

\textit{\textbf{Claw.} Melee Weapon Attack}: +27 to hit, reach 1m, one target.

\textit{Hit:} 15 (2d6 + 8) slashing damage, 1 bleed damage.

\textit{\textbf{Tail.} Melee Weapon Attack}: +27 to hit, reach 5 meters, one target.

\textit{Hit:} 17 (2d8 + 8) bludgeoning damage.

\textit{\textbf{Bite.} Melee Weapon Attack}: +27 to hit, reach 3m, one target.

\textit{Hit:} 19 (2d10 + 8) piercing damage.

\textit{\textbf{Fearful Presence.}} Any creature of the dragon's choice that is within 16 meters of it and aware of its presence must succeed at a DC 22 Will save or be frightened for 1 minute. A creature can repeat the Saving Throw at the end of each of its rounds, ending the effect on a successful one. If the creature's Saving Throw succeeds or the effect ends for it, the creature is immune to the dragon's Dreadful Presence for the next 24 hours.

\textit{\textbf{Breath Weapon (Cooldown 5-6).}} The dragon uses one of the following breath weapons:

\textit{Icy Breath.} The dragon exhales an icy blast in a 20m cone. Each creature in the area must make a DC 22 Fortitude save, taking 58 (13d8) cold damage on a failed save, or half as much damage on a successful one.

\textit{Crippling Breath.} The dragon exhales a paralyzing gas in a 20m cone. Each creature in the area must succeed on a Fortitude 22 save or be paralyzed for 1 minute. A creature can repeat the Saving Throw at the end of each of its rounds, ending the effect on itself on a success.

\textit{\textbf{Shapeshift.}} The dragon can magically transform into a humanoid or beast whose challenge rating is equal to or lower than its own, or revert to its true form. Upon death it returns to its true form. Any equipment it is wearing or carrying is absorbed or carried into the new form (dragon's choice).

In the new form, the dragon retains its Traits, Hit Points, Hit Dice, speech, proficiencies, Legendary Endurance, lair actions, and Intelligence, Wisdom, and Charisma scores, in addition to this action. Its stats and abilities are otherwise replaced by those of the new form, except for the new form's additional Actions.

\textbf{Additional Actions}

The dragon can perform 3 additional Actions, chosen from the options below. He can only use one Additional Action at a time, and only at the end of another creature's turn. The dragon recovers expended additional Actions at the start of its round.

\textbf{Wing Attack (Costs 2 Actions).} The dragon flaps its wings. Each creature within 3 meter of the dragon must succeed on a DC 22 Reflex save or take 15 (2d6 + 8) bludgeoning damage and be knocked prone. The dragon can then fly up to half its flight move.

\textbf{Tail Attack.} The dragon performs a tail attack.

\textbf{Locate.} The dragon makes a Wisdom (Awareness) check.


\textit{\textbf{Enraged}}: The Adult Silver Dragon can perform these actions for 2 Actions.

\textit{Focus}: the creature interrupts a mental effect on itself in progress

\textit{Brutality}: the creature attacks with unprecedented ferocity. +1d6 to attack roll, an additional 6 is counted in the critical count until the end of the fight.


\textbf{Ecology}\\
Environment: Temperate Mountains\\
Organization: Solitary\\
\textbf{Treasure}: Triple\\
\textbf{Description}\\
Of all dragons, the silver ones are the bravest, and hold a code of chivalry that requires them to help the weak, defeat evil, and behave honorably.\\
\textbf{Enchantments}\index{Silver Dragon Spell}\\
This dragon's favorite spells are:\\
- Slow\\
- Fabricate\\
- Dream


\

\index[Monsters]{Dragon, Adult White}\textbf{Adult White Dragon}

\textit{Huge dragon, chaotic evil}

\textbf{STRENGTH} +6

\textbf{DEXTERITY} +0

\textbf{CONSTITUTION} +6

\textbf{INTELLIGENCE} -1

\textbf{WISDOM} +1

\textbf{CHARISMA} +1

\textbf{Initiative} +0 -- \textbf{Defence} 25

\textbf{Hit Points} 200 (16d12 + 96)

\textbf{Movement} 12m, swim 12m, dig 9m, fly 24m

\textbf{Saving Throws} Fortitude +19, Reflexes +13, Will +14

\textbf{Skills} Stealth +5, Awareness +11

\textbf{Damage Immunity} cold

\textbf{Senses} darkvision 40m, blindsight 20m

\textbf{Languages} Common, Draconic

\textbf{Challenge} 13 (10000 XP)

\textit{\textbf{Icewalking.}} The dragon can move and climb icy surfaces without needing to make ability checks. Also, hindering terrain composed of ice or snow costs him no additional movement.

\textit{\textbf{Legendary Endurance (3 / Day).}} If the dragon fails a Saving Throw, it may choose to succeed instead.

\textbf{Actions}

\textit{\textbf{Multiattack.}} The dragon can use its Frightful Presence and then make three attacks: one with its bite and two with its claws.

\textit{\textbf{Claw.} Melee Weapon Attack}: +21 to hit, reach 1m, one target, 1 Bleed damage.

\textit{Hit:} 13 (2d6 + 6) slashing damage.

\textit{\textbf{Tail.} Melee Weapon Attack}: +21 to hit, reach 5 meters, one target.

\textit{Hit:} 15 (2d8 + 6) bludgeoning damage.

\textit{\textbf{Bite.} Melee Weapon Attack}: +21 to hit, reach 3m, one target.

\textit{Hit:} 17 (2d10 + 6) piercing damage plus 4 (1d8) cold damage.

\textit{\textbf{Fearful Presence.}} Any creature of the dragon's choice that is within 16 meters of it and aware of its presence must succeed at a DC 19 Will save or be frightened for 1 minute. A creature can repeat the Saving Throw at the end of each of its rounds, ending the effect on a successful one. If the creature's Saving Throw succeeds or the effect ends for it, the creature is immune to the dragon's Dreadful Presence for the next 24 hours.

\textit{\textbf{Icy Breath (Cooldown 4-6).}} The dragon exhales a blast of ice in a 20m cone. Each creature in that area must make a DC 19 Fortitude save and take 54 (12d8) cold damage on a failed save, or half as much damage on a successful one.

\textbf{Additional Actions}

The dragon can perform 3 additional Actions, chosen from the options below. He can only use one Additional Action at a time, and only at the end of another creature's turn. The dragon recovers expended additional Actions at the start of its round.

\textbf{Wing Attack (Costs 2 Actions).} The dragon flaps its wings. Each creature within 3 meter of the dragon must succeed on a DC 19 Reflex save or take 13 (2d6 + 6) bludgeoning damage and be knocked prone. The dragon can then fly up to half its flight move. \textbf{Tail Attack.} The dragon performs a tail attack
.
\textbf{Locate.} The dragon makes a Wisdom (Awareness) check.

\textit{\textbf{Enraged}}: The Adult White Dragon can perform these actions at a cost of 2 Actions.

\textit{Focus}: the creature interrupts a mental effect on itself in progress

\textit{Brutality}: the creature attacks with unprecedented ferocity. +1d6 to attack roll, an additional 6 is counted in the critical count until the end of the fight.


\textbf{Ecology}\\
Environment: Cold Mountains\\
Organization: Solitary\\
\textbf{Treasure}: Triple\\
\textbf{Description}\\
Read Ancient White Dragon Description.\\
\textbf{Enchantments}\index{White Dragon Spell}\\
This dragon's favorite spells are:\\
- Fire Shield\\
- Ice Storm\\
- Sleet Storm


\

\index[Monsters]{Dragon, Ancient Black}\textbf{Ancient Black Dragon}

\textit{Gargantuan dragon, chaotic evil}

\textbf{STRENGTH} +8

\textbf{DEXTERITY} +2

\textbf{CONSTITUTION} +7

\textbf{INTELLIGENCE} +3

\textbf{WISDOM} +2

\textbf{CHARISMA} +4

\textbf{Initiative} +3 -- \textbf{Defence} 33

\textbf{Hit Points} 367 (21x3d6 + 147)

\textbf{Move} 12m, climb 12m, fly 24m

\textbf{Saving Throws} Fortitude +28, Reflexes +23, Will +23

\textbf{Skills} Stealth +9, Awareness +16

\textbf{Damage Immunity} acid, weapons +1

\textbf{Senses} darkvision 40m, blindsight 20m

\textbf{Languages} Common, Draconic

\textbf{Challenge} 21 (33000 XP)

\textit{\textbf{Amphibious.}} The dragon can breathe air and water.

\textit{\textbf{Legendary Endurance (3 / Day).}} If the dragon fails a Saving Throw, it may choose to succeed instead.

\textbf{Actions}

\textit{\textbf{Multiattack.}} The dragon can use its Frightening Presence. Then make three attacks: one with the bite and two with the claws.

\textit{\textbf{Claw.} Melee Weapon Attack}: +30 to hit, reach 3m, one target.

\textit{Hit:} 15 (2d6 + 8) slashing damage, 3 bleed damage (up to a maximum of 20).

\textit{\textbf{Tail.} Melee Weapon Attack}: +30 to hit, reach 6m, one target.

\textit{Hit:} 17 (2d8 + 8) bludgeoning damage.

\textit{\textbf{Bite.} Melee Weapon Attack} : +30 to hit, reach 5 meters, one target.

\textit{Hit:} 19 (2d10 + 8) piercing damage plus 9 (4d6) acid damage.

\textit{\textbf{Fearful Presence.}} Any creature of the dragon's choice that is within 16 meters of it and aware of its presence must succeed on a DC 19 Will save or be frightened for 1 minute. A creature can repeat the Saving Throw at the end of each of its rounds, ending the effect on a successful one. If the creature's Saving Throw succeeds or the effect ends for it, the creature is immune to the dragon's Dreadful Presence for the next 24 hours.

\textit{\textbf{Acid Breath (Recharge 5-6).}} The dragon exhales acid in a 27m line that is 3m wide. Each creature in that area must make a DC 19 Reflex save and take 67 (15d8) acid damage on a failed save, or half as much damage on a successful one.

\textbf{Additional Actions}

The dragon can perform 3 additional Actions, chosen from the options below. He can only use one Additional Action at a time, and only at the end of another creature's turn. The dragon recovers expended additional Actions at the start of its round.

\textbf{Wing Attack (Costs 2 Actions).} The dragon flaps its wings. Each creature within 5 meters of the dragon must succeed on a DC 19 Reflex save or take 15 (2d6 + 8) bludgeoning damage and be knocked prone. The dragon can then fly up to half its flight move.

\textbf{Tail Attack.} The dragon performs a tail attack.

\textbf{Locate.} The dragon makes a Wisdom (Awareness) check.

\textbf{Ecology}\\
Environment: Warm Swamps\\
Organization: Solitary\\
\textbf{Treasure}: Triple\\
\textbf{Description}\\
Black dragons are violent and aggressive, living in swamps and marshes and generally ruling as undisputed masters.

Black dragons are menacing creatures that have large, forward-curving horns.
The head connects to a relatively short neck and a large, muscular lizard-like body.

They have very small wings that are on the sides, but still manage to fly thanks to magic.
They have webbed feet to allow them to swim more easily in the marshy areas where they live.

Black dragons tend to make their lairs in the center of swamp or marsh.
They consider that territory theirs and no one can get wet without feeling their wrath.

A black dragon's lair can be a gigantic pile of logs as well as an underground cavern submerged in water, if not the bottom of a lake.
Being able to breathe underwater, they have no worries about where to build their home.

Their home is always protected by traps and their evil followers who bring them food, possibly live.

The environment where a black dragon lives suffers its effects, acid vapors, destruction, corruption are immediately perceptible.

The Black Dragon represent the Traits of selfishness and violence by hating all life, including the black dragons themselves.

Black dragons have +1d6 on magic checks and can ignore one die rolled on a check with the List of Necromancy and are immune to acid.\\

\textbf{Enchantments}\index{Black Dragon Spell}\\
This dragon's favorite spells are:\\
- Create Undead\\
- Raise Dead\\
- Bestow Curse

The Black Dragon is the only creature on Yeru that can bring a dead man to life despite all the constraints imposed by the Patrons.


\

\index[Monsters]{Dragon, Ancient Blue}\textbf{Ancient Blue Dragon}

\textit{Gargantuan Dragon, Lawful Evil}

\textbf{STRENGTH} +9

\textbf{DEXTERITY} +0

\textbf{CONSTITUTION} +8

\textbf{INTELLIGENCE} +4

\textbf{WISDOM} +3

\textbf{CHARISMA} +5

\textbf{Initiative} +4 -- \textbf{Defence} 34

\textbf{Hit Points} 481 (26x3d6 + 208)

\textbf{Move} 12m, dig 12m, fly 24m

\textbf{Saving Throws} Fortitude +31, Reflexes +23, Will +22

\textbf{Skills} Stealth +7, Awareness +17

\textbf{Damage Immunity} electricity, weapons +1

\textbf{Senses} darkvision 40m, blindsight 20m

\textbf{Languages} Common, Draconic

\textbf{Challenge} 23 (50000 XP)

\textit{\textbf{Legendary Endurance (3 / Day).}} If the dragon fails a Saving Throw, it may choose to succeed instead.

\textbf{Actions}

\textit{\textbf{Multiattack.}} The dragon can use its Frightening Presence. Then make three attacks: one with the bite and two with the claws.

\textit{\textbf{Claw.} Melee Weapon Attack}: +16 to hit, reach 3 m, one target.

\textit{Hit:} 16 (2d6 + 9) slashing damage, 3 bleed damage (up to a maximum of 20).

\textit{\textbf{Tail.} Melee Weapon Attack}: +30 to hit, reach 6m, one target.

\textit{Hit:} 18 (2d8 + 9) bludgeoning damage.

\textit{\textbf{Bite.} Melee Weapon Attack}: +30 to hit, reach 5 meters, one target.

\textit{Hit:} 20 (2d10 + 9) piercing damage plus 11 (2d10) electricity damage.

\textit{\textbf{Fearful Presence.}} Any creature of the dragon's choice that is within 16 meters of it and aware of its presence must succeed at a DC 27 Will save or be frightened for 1 minute. A creature can repeat the Saving Throw at the end of each of its rounds, ending the effect on a successful one. If the creature's Saving Throw succeeds or the effect ends for it, the creature is immune to the dragon's Dreadful Presence for the next 24 hours.

\textit{\textbf{Lightning Breath (Cooldown 5-6).}} The dragon exhales lightning in a line 36 meters long and 3 meters wide. Each creature in that line must make a DC 27 Reflex save and take 88 (16d10) electricity damage on a failed save, or half as much damage on a successful one.

\textbf{Additional Actions}

The dragon can perform 3 additional Actions, chosen from the options below. He can only use one Additional Action at a time, and only at the end of another creature's turn. The dragon recovers expended additional Actions at the start of its round.

\textbf{Wing Attack (Costs 2 Actions).} The dragon flaps its wings. Each creature within 5 meters of the dragon must succeed on a DC 27 Reflex save or take 16 (2d6 + 9) bludgeoning damage and be knocked prone. The dragon can then fly up to half its flight move.

\textbf{Tail Attack.} The dragon performs a tail attack.

\textbf{Locate.} The dragon makes a Wisdom (Awareness) check.

\textbf{Move.} The dragon take a move action.\\
\textbf{Ecology}\\
Environment: Mountain peaks\\
Organization: Solitary\\
\textbf{Treasure}: Triple\\
\textbf{Description}\\
Blue Dragons dwell in the clouds, flying (and levitating) among storms.

Blue Dragons have a serpent-like appearance, elongated and binding, with long backwards horns.

A Blue Dragon's face is less wrinkled and remains smooth.
They are the only dragons that don't have wings even though they fly better than any other dragon.

Their magical but natural ability to fly combined with the fact that they feed on electricity makes them purely flying creatures that almost never descend to the ground (and never touch the ground considering it impure and dirty!), preferring to remain in the clouds, especially among the darker and full of energy to eat

The Blue Dragon's lair is usually among the highest peaks of mountains possibly high enough to reach the clouds. This is never covered and often resembles gigantic nests.

Blue Dragons can assimilate meat but not vegetables, they do not derive nourishment from what they eat having a purely electrical metabolism.

They are social dragons, who enjoy being with their own kind and are very protective of their offspring.
Usually one nest is never found alone, but entire plateaus dominated by dozens of dragons.

They do not get along well with purple dragons, whom they despise for choosing to give up flying to live underground.

Blue dragons have +1d6 on magic checks and can ignore one die rolled on an Air List check and are immune to electricity.\\
\textbf{Enchantments}\index{Blu Dragon Spell}\\
This dragon's favorite spells are:\\
- Deadly Mist\\
- Call Lightning\\
- Ice Storm



\

\index[Monsters]{Dragon, Ancient Brass}\textbf{Ancient Brass Dragon}

\textit{Gargantuan Dragon, Chaotic Good}

\textbf{STRENGTH} +8

\textbf{DEXTERITY} +0

\textbf{CONSTITUTION} +7

\textbf{INTELLIGENCE} +3

\textbf{WISDOM} +2

\textbf{CHARISMA} +4

\textbf{Initiative} +3 -- \textbf{Defence} 30

\textbf{Hit Points} 297 (17x3d6 + 119)

\textbf{Move} 12m, dig 12m, fly 24m

\textbf{Saving Throws} Fortitude +27, Reflexes +20, Will +22

\textbf{Skills} Stealth +6, Awareness +14, Deceive +10, History +9

\textbf{Damage Immunity} Fire, weapons +1

\textbf{Senses} darkvision 40m, blindsight 20m

\textbf{Languages} Common, Draconic

\textbf{Challenge} 20 (25000 XP)

\textit{\textbf{Legendary Endurance (3 / Day).}} If the dragon fails a Saving Throw, it can choose to succeed instead.

\textbf{Actions}

\textit{\textbf{Multiattack.}} The dragon can use its Frightening Presence. Then make three attacks: one with the bite and two with the claws.

\textit{\textbf{Claw.} Melee Weapon Attack}: +30 to hit, reach 3m, one target.

\textit{Hit:} 15 (2d6 + 8) slashing damage, 3 bleed damage (up to a maximum of 20)

\textit{\textbf{Tail.} Melee Weapon Attack}: +30 to hit, reach 6m, one target.

\textit{Hit:} 17 (2d8 + 8) bludgeoning damage.

\textit{\textbf{Bite.} Melee Weapon Attack}: +30 to hit, reach 5 meters, one target.

\textit{Hit:} 19 (2d10 + 8) piercing damage.

\textit{\textbf{Fearful Presence.}} Any creature of the dragon's choice that is within 16 meters of it and aware of its presence must succeed on a DC 27 Will save or be frightened for 1 minute. A creature can repeat the Saving Throw at the end of each of its rounds, ending the effect on a successful one. If the creature's Saving Throw succeeds or the effect ends for it, the creature is immune to the dragon's Dreadful Presence for the next 24 hours.

\textit{\textbf{Breath Weapon (Cooldown 5-6).}} The dragon uses one of the following breath weapons:

\textit{Fiery Breath.} The dragon exhales fire in a line 27 meters long and 3 meters wide. Each creature in the line must make a DC 27 Reflex save, taking 56 (16d6) fire damage on a failed save, or half as much damage on a successful one.

\textit{Sleep Breath.} The dragon exhales sleeping gas in a 27-meter cone. Each creature in that area must succeed on a Fortitude 27 save or be knocked unconscious for 10 minutes. This effect
ends if the unconscious creature takes damage or someone takes an action to awaken it.

\textit{\textbf{Shapeshift.}} The dragon can magically transform into a humanoid or beast whose challenge rating is equal to or lower than its own, or revert to its true form. Upon death it returns to its true form. Any equipment it is wearing or carrying is absorbed or carried into the new form (dragon's choice).

In the new form, the dragon retains its Traits, Hit Points, Hit Dice, speech, proficiencies, Legendary Endurance, lair actions, and Intelligence, Wisdom, and Charisma scores, in addition to this action. Its stats and abilities are otherwise replaced by those of the new form, except for the new form's additional Actions.

\textbf{Additional Actions}

The dragon can perform 3 additional Actions, chosen from the options below. He can only use one Additional Action at a time, and only at the end of another creature's turn. The dragon recovers expended additional Actions at the start of its round.

\textbf{Wing Attack (Costs 2 Actions).} The dragon flaps its wings. Each creature within 5 meters of the dragon must succeed on a DC 27 Reflex save or take 15 (2d6 + 8) bludgeoning damage and be knocked prone. The dragon can then fly up to half its flight move.

\textbf{Tail Attack.} The dragon performs a tail attack.

\textbf{Locate.} The dragon makes a Wisdom (Awareness) check.

\textbf{Ecology}\\
Environment: Hot Deserts\\
Organization: Solitary\\
\textbf{Treasure}: Triple\\
\textbf{Description}\\
Excellent conversationalists, brass dragons prefer to talk rather than fight. Brass dragons lair near humanoid settlements, where they can hear the latest news and gossip.\\
\textbf{Enchantments}\index{Brass Dragon Spell}\\
This dragon's favorite spells are:\\
- True Seeing\\
- Knowledge of Legends\\
- Scrying


\

\index[Monsters]{Dragon, Ancient Bronze}\textbf{Ancient Bronze Dragon}

\textit{Gargantuan Dragon, Chaotic Good}

\textbf{STRENGTH} +9

\textbf{DEXTERITY} +0

\textbf{CONSTITUTION} +8

\textbf{INTELLIGENCE} +4

\textbf{WISDOM} +3

\textbf{CHARISMA} +5

\textbf{Initiative} +4 -- \textbf{Defence} 33

\textbf{Hit Points} 444 (24x3d6 + 192)

\textbf{Movement} 12m, swim 12m, fly 24m

\textbf{Saving Throws} Fortitude +30, Reflexes +22, Will +25

\textbf{Skills} Stealth +7, Sense Emotions +10, Awareness +17

\textbf{Damage Immunity} electricity, weapons +1

\textbf{Senses} darkvision 40m, blindsight 20m

\textbf{Languages} Common, Draconic

\textbf{Challenge} 22 (41000 XP)

\textit{\textbf{Amphibious.}} The dragon can breathe air and water.

\textit{\textbf{Legendary Endurance (3 / Day).}} If the dragon fails a Saving Throw, it may choose to succeed instead.

\textbf{Actions}

\textit{\textbf{Multiattack.}} The dragon can use its Frightening Presence. Then make three attacks: one with the bite and two with the claws.

\textit{\textbf{Claw.} Melee Weapon Attack}: +30 to hit, reach 3m, one target.

\textit{Hit:} 16 (2d6 + 9) slashing damage, 3 bleed damage (up to a maximum of 20).

\textit{\textbf{Tail.} Melee Weapon Attack}: +30 to hit, reach 6m, one target.

\textit{Hit:} 18 (2d8 + 9) bludgeoning damage.

\textit{\textbf{Bite.} Melee Weapon Attack}: +30 to hit, reach 5 meters, one target.

\textit{Hit:} 20 (2d10 + 9) piercing damage.

\textit{\textbf{Fearful Presence.}} Any creature of the dragon's choice that is within 16 meters of it and aware of its presence must succeed at a DC 26 Will save or be frightened for 1 minute. A creature can repeat the Saving Throw at the end of each of its rounds, ending the effect on a successful one. If the creature's Saving Throw succeeds or the effect ends for it, the creature is immune to the dragon's Dreadful Presence for the next 24 hours.

\textit{\textbf{Breath Weapon (Cooldown 5-6).}} The dragon uses one of the following breath weapons:

\textit{Lightning Breath.} The dragon exhales lightning in a line 36 meters long and 3 meters wide. Each creature in the line must make a DC 26 Reflex save, taking 88 (16d10) electricity damage on a failed save, or half as much damage on a successful one. \textit{Repulsive Breath.} The dragon exhales repulsive energy in a 10m cone. Each creature in that area must succeed on a DC 26 Fortitude save or be pushed 20 meters away from the creature.
dragon.

\textit{\textbf{Shapeshift.}} The dragon can magically transform into a humanoid or beast whose challenge rating is equal to or lower than its own, or revert to its true form. Upon death it returns to its true form. Any equipment it is wearing or carrying is absorbed or carried into the new form (dragon's choice).

In the new form, the dragon retains its Traits, Hit Points, Hit Dice, speech, proficiencies, Legendary Endurance, lair actions, and Intelligence, Wisdom, and Charisma scores, in addition to this action. Its stats and abilities are otherwise replaced by those of the new form, except for the new form's additional Actions.

\textbf{Additional Actions}

The dragon can perform 3 additional Actions, chosen from the options below. He can only use one Additional Action at a time, and only at the end of another creature's turn. The dragon recovers expended additional Actions at the start of its round.

\textbf{Wing Attack (Costs 2 Actions).} The dragon flaps its wings. Each creature within 5 meters of the dragon must succeed on a DC 26 Reflex save or take 16 (2d6 + 9) bludgeoning damage and be knocked prone. The dragon can then fly up to half its flight move.

\textbf{Tail Attack.} The dragon performs a tail attack.

\textbf{Locate.} The dragon makes a Wisdom (Awareness) check.

\textit{\textbf{Enraged}}: The Adult Bronze Dragon can perform these actions for 2 Actions.

\textit{Focus}: the creature interrupts a mental effect on itself in progress

\textit{Brutality}: the creature attacks with unprecedented ferocity. +1d6 to attack roll, an additional 6 is counted in the critical count until the end of the fight.


\textbf{Ecology}\\
Environment: Temperate Coastal Zones\\
Organization: Solitary\\
\textbf{Treasure}: Triple\\
\textbf{Description}\\
Bronze dragons are known to ally with travelers and adventurers if cause and reward are just and fitting\\
\textbf{Enchantments}\index{Bronze Dragon Spell}\\
This dragon's favorite spells are:\\
- Orb of Invulnerability\\
- Freedom of Movement


\

\index[Monsters]{Dragon, Ancient Copper}\textbf{Ancient Copper Dragon}

\textit{Gargantuan Dragon, Chaotic Good}

\textbf{STRENGTH} +8

\textbf{DEXTERITY} +1

\textbf{CONSTITUTION} +7

\textbf{INTELLIGENCE} +5

\textbf{WISDOM} +3

\textbf{CHARISMA} +4

\textbf{Initiative} +5 -- \textbf{Defence} 33

\textbf{Hit Points} 350 (20x3d6 + 140)

\textbf{Move} 12m, climb 12m, fly 24m

\textbf{Saving Throws} Fortitude +28, Reflexes +22, Will +23

\textbf{Skills} Stealth +8, Deceive +11, Awareness +17

\textbf{Damage Immunity} acid, weapons +1

\textbf{Senses} darkvision 40m, blindsight 20m

\textbf{Languages} Common, Draconic

\textbf{Challenge} 21 (33000 XP)

\textit{\textbf{Legendary Endurance (3 / Day).}} If the dragon fails a Saving Throw, it may choose to succeed instead.

\textbf{Actions}

\textit{\textbf{Multiattack.}} The dragon can use its Frightening Presence. Then make three attacks: one with the bite and two with the claws.

\textit{\textbf{Claw.} Melee Weapon Attack}: +30 to hit, reach 3m, one target.

\textit{Hit:} 15 (2d6 + 8) slashing damage, 3 bleed damage (up to a maximum of 20).

\textit{\textbf{Tail.} Melee Weapon Attack}: +30 to hit, reach 6m, one target.

\textit{Hit:} 17 (2d8 + 8) bludgeoning damage.

\textit{\textbf{Bite.} Melee Weapon Attack}: +30 to hit, reach 5 meters, one target.

\textit{Hit:} 19 (2d10 + 8) piercing damage.

\textit{\textbf{Fearful Presence.}} Any creature of the dragon's choice that is within 16 meters of it and aware of its presence must succeed at a DC 26 Will save or be frightened for 1 minute. A creature can repeat the Saving Throw at the end of each of its rounds, ending the effect on a successful one. If the creature's Saving Throw succeeds or the effect ends for it, the creature is immune to the dragon's Dreadful Presence for the next 24 hours.

\textit{\textbf{Breath Weapon (Cooldown 5-6).}} The dragon uses one of the following breath weapons:

\textit{Acid Breath.} The dragon exhales acid in a line 27 meters long and 3 meters wide. Each creature in the line must make a DC 26 Reflex save, taking 63 (14d8) acid damage on a failed save, or half as much damage on a successful one.

\textit{Slowing Breath.} The dragon exhales gas in a 27-meter cone. Each creature in that area must succeed on a DC 26 Fortitude save. On a failed save, the creature can't use its reaction, has its speed halved, and can't make more than one attack during its round. Also, the creature can use either an action or a bonus action, but not both. These effects last for 1 minute. The creature can repeat the Saving Throw at the end of each of its rounds, ending the effect on itself on a success.

\textit{\textbf{Shapeshift.}} The dragon can magically transform into a humanoid or beast whose challenge rating is equal to or lower than its own, or revert to its true form. Upon death it returns to its true form. Any equipment it is wearing or carrying is absorbed or carried into the new form (dragon's choice).

In the new form, the dragon retains its Traits, Hit Points, Hit Dice, speech, proficiencies, Legendary Endurance, lair actions, and Intelligence, Wisdom, and Charisma scores, in addition to this action. Its stats and abilities

they are otherwise replaced by those of the new form, except for additional Actions of the new form.

\textbf{Additional Actions}

The dragon can perform 3 additional Actions, chosen from the options below. He can only use one Additional Action at a time, and only at the end of another creature's turn. The dragon recovers expended additional Actions at the start of its round.

\textbf{Wing Attack (Costs 2 Actions).} The dragon flaps its wings. Each creature within 5 meters of the dragon must succeed on a DC 26 Reflex save or take 15 (2d6 + 8) bludgeoning damage and be knocked prone. The dragon can then fly up to half its flight move.

\textbf{Tail Attack.} The dragon performs a tail attack.

\textbf{Locate.} The dragon makes a Wisdom (Awareness) check.

\textbf{Ecology}\\
Environment: Warm Hills\\
Organization: Solitary\\
\textbf{Treasure}: Triple\\
\textbf{Description}\\
This capricious dragon tries to hinder and frustrate his enemies during combat.\\
\textbf{Enchantments}\index{Copper Dragon Spell}\\
This dragon's favorite spells are:\\
- Blade Barrier\\
- Wall of Force\\
- Elastic Sphere


\

\index[Monsters]{Dragon, Ancient Gold}\textbf{Ancient Gold Dragon}

\textit{Gargantuan Dragon, Lawful Good}

\textbf{STRENGTH} +10

\textbf{DEXTERITY} +2

\textbf{CONSTITUTION} +9

\textbf{INTELLIGENCE} +4

\textbf{WISDOM} +3

\textbf{CHARISMA} +9

\textbf{Initiative} +4 -- \textbf{Defence} 34

\textbf{Hit Points} 546 (28x3d6 + 252)

\textbf{Movement} 12m, swim 12m, fly 24m

\textbf{Saving Throws} Fortitude +33, Reflexes +26, Will +27

\textbf{Skills} Stealth +9, Sense Emotions +10, Awareness +17, Deceive +16

\textbf{Damage Immunity} Fire, weapons +1

\textbf{Senses} darkvision 40m, blindsight 20m

\textbf{Languages} Common, Draconic

\textbf{Challenge} 24 (62000 XP)

\textit{\textbf{Amphibious.}} The dragon can breathe air and water.

\textit{\textbf{Legendary Endurance (3 / Day).}} If the dragon fails a Saving Throw, it may choose to succeed instead.

\textbf{Actions}

\textit{\textbf{Multiattack.}} The dragon can use its Frightening Presence. Then make three attacks: one with the bite and two with the claws.

\textit{\textbf{Claw.} Melee Weapon Attack}: +30 to hit, reach 3m, one target.

\textit{Hit:} 17 (2d6 + 10) slashing damage, 3 bleed damage (up to a maximum of 20).

\textit{\textbf{Tail.} Melee Weapon Attack}: +30 to hit, reach 6m, one target.

\textit{Hit:} 19 (2d8 + 10) bludgeoning damage.

\textit{\textbf{Bite.} Melee Weapon Attack}: +30 to hit, reach 5 meters, one target.

\textit{Hit:} 21 (2d10 + 10) piercing damage.

\textit{\textbf{Fearful Presence.}} Any creature of the dragon's choice that is within 16 meters of it and aware of its presence must succeed at a DC 28 Will save or be frightened for 1 minute. A creature can repeat the Saving Throw at the end of each of its rounds, ending the effect on a successful one. If the creature's Saving Throw succeeds or the effect ends for it, the creature is immune to the dragon's Dreadful Presence for the next 24 hours.

\textit{\textbf{Breath Weapon (Reload 5-6).}} The dragon uses one of the following breath weapons:

\textit{Fiery Breath.} The dragon breathes fire in a 27-meter cone. Each creature in the area must make a DC 28 Reflex save, taking 71(13d10) fire damage on a failed save, or half as much damage on a successful one.

\textit{Weakening Breath.} The dragon exhales gas in a 27-meter cone. Each creature in that area must succeed on a DC 28 Fortitude save or have -1d6 on Strength-based attack rolls, Strength checks, and Fortitude saves for 1 minute. A creature can repeat the Saving Throw at the end of each of its rounds, ending the effect on itself on a success.

\textit{\textbf{Shapeshift.}} The dragon can magically transform into a humanoid or beast whose challenge rating is equal to or lower than its own, or revert to its true form. Upon death it returns to its true form. Any equipment it is wearing or carrying is absorbed or carried into the new form (dragon's choice).

In the new form, the dragon retains its Traits, Hit Points, Hit Dice, speech, proficiencies, Legendary Endurance, lair actions, and Intelligence, Wisdom, and Charisma scores, in addition to this action. Its stats and abilities are otherwise replaced by those of the new form, except for the new form's additional Actions.

\textbf{Additional Actions}

The dragon can perform 3 additional Actions, chosen from the options below. He can only use one Additional Action at a time, and only at the end of another creature's turn. The dragon recovers expended additional Actions at the start of its round.

\textbf{Wing Attack (Costs 2 Actions).} The dragon flaps its wings. Each creature within 5 meters of the dragon must succeed on a DC 28 Reflex save or take 17 (2d6 + 10) bludgeoning damage and be knocked prone. The dragon can then fly up to half its flight move.

\textbf{Tail Attack.} The dragon performs a tail attack.

\textbf{Locate.} The dragon makes a Wisdom (Awareness) check.

\textbf{Ecology}\\
Environment: Hot Plains\\
Organization: Solitary\\
\textbf{Treasure}: Triple\\
\textbf{Description}\\
Golden dragons are the emblem of virtue. Other Ljust's dragons revere them as agents of divine powers and exemplary members of the draconic race, and often seek them for advice or help.\\
\textbf{Enchantments}\index{Gold Dragon Spell}\\
This dragon's favorite spells are:\\
- Heal\\
- Greater Restoration\\
- Black Tentacles


\

\index[Monsters]{Dragon, Ancient Green}\textbf{Ancient Green Dragon}

\textit{Gargantuan Dragon, Lawful Evil}

\textbf{STRENGTH} +8

\textbf{DEXTERITY} +1

\textbf{CONSTITUTION} +7

\textbf{INTELLIGENCE} +5

\textbf{WISDOM} +3

\textbf{CHARISMA} +4

\textbf{Initiative} +5 -- \textbf{Defence} 32

\textbf{Hit Points} 385 (22x3d6 + 154)

\textbf{Movement} 12m, swim 12m, fly 24m

\textbf{Saving Throws} Fortitude +29, Reflexes +23, Will +25

\textbf{Skills} Stealth +8, Deceive +11, Sense Emotions +10, Awareness +15

\textbf{Damage Immunity} poison, weapons +1

\textbf{Immunity to Conditions} poisoned

\textbf{Senses} darkvision 40m, blindsight 20m

\textbf{Languages} Common, Draconic

\textbf{Challenge} 22 (41000 XP)

\textit{\textbf{Amphibious.}} The dragon can breathe air and water.

\textit{\textbf{Legendary Endurance (3 / Day).}} If the dragon fails a Saving Throw, it may choose to succeed instead.

\textbf{Actions}

\textit{\textbf{Multiattack.}} The dragon can use its Frightening Presence. Then make three attacks: one with the bite and two with the claws.

\textit{\textbf{Claw.} Melee Weapon Attack}: +30 to hit, reach 3m, one target.

\textit{Hit:} 15 (2d6 + 8) slashing damage, 3 bleed damage (up to a maximum of 20).

\textit{\textbf{Tail.} Melee Weapon Attack}: +30 to hit, reach 6m, one target.

\textit{Hit:} 17 (2d8 + 8) bludgeoning damage.

\textit{\textbf{Bite.} Melee Weapon Attack}: +30 to hit, reach 5 meters, one target.

\textit{Hit:} 19 (2d10 + 8) piercing damage plus 10 (3d6) poison damage.

\textit{\textbf{Fearful Presence.}} Any creature of the dragon's choice that is within 16 meters of it and aware of its presence must succeed on a DC 26 Will save or be frightened for 1 minute. A creature can repeat the Saving Throw at the end of each of its rounds, ending the effect on a successful one. If the creature's Saving Throw succeeds or the effect ends for it, the creature is immune to the dragon's Dreadful Presence for the next 24 hours.

\textit{\textbf{Venom Breath (Cooldown 5-6).}} The dragon exhales poisonous gas in a 27-meter cone. Each creature in that area must make a DC 26 Fortitude save and take 77 (22d6) poison damage on a failed save, or half as much damage on a successful one.

\textbf{Additional Actions}

The dragon can perform 3 additional Actions, chosen from the options below. He can only use one Additional Action at a time, and only at the end of another creature's turn. The dragon recovers expended additional Actions at the start of its round.

\textbf{Wing Attack (Costs 2 Actions).} The dragon flaps its wings. Each creature within 5 meters of the dragon must succeed on a DC 26 Reflex save or take 15 (2d6 + 8) bludgeoning damage and be knocked prone. The dragon can then fly up to half its flight move.

\textbf{Tail Attack.} The dragon performs a tail attack.

\textbf{Locate.} The dragon makes a Wisdom (Awareness) check.

\textbf{Ecology}\\
Environment: Temperate Forests\\
Organization: Solitary\\
\textbf{Treasure}: Triple\\
\textbf{Description}\\
Green Dragons love forests and pristine nature where they consider themselves the undisputed masters and kings.

The mighty green dragons have rounded heads and prominently set ears, the horns are short and blunt.
The claws and jaws are devastating, powerful and capable of slicing through anything.
The nose is wide and the nostrils open as if it were going to blow at any moment.

Green dragon breath is poison, so it can kill living creatures but not plants.

A green dragon's lair is always near a source of water, preferably in the lushest and most pristine part of the forest.

A green dragon does not like to fly and prefers to jump by crushing with its weight and tearing with its claws.

Among the many dragons, the green one is perhaps the one that will make the adventurers talk if they show respect and fear of his royalty.

Green Dragons have +1d6 on Magic Test and can ignore one die rolled on the List of Animals and Plants check and are immune to both magical and natural poisons.\\

\textbf{Enchantments}\index{Green Dragon Spell}\\
This dragon's favorite spells are:\\
- Anti-Life Shell\\
- Locate Creature\\
- Remove poison


\

\index[Monsters]{Dragon, Ancient Silver}\textbf{Ancient Silver Dragon}

\textit{Gargantuan Dragon, Neutral Good}

\textbf{STRENGTH} +10

\textbf{DEXTERITY} +0

\textbf{CONSTITUTION} +9

\textbf{INTELLIGENCE} +4

\textbf{WISDOM} +2

\textbf{CHARISMA} +6

\textbf{Initiative} +4 -- \textbf{Defence} 34

\textbf{Hit Points} 487 (25x3d6 + 225)

\textbf{Move} 12m, fly 24m

\textbf{Saving Throws} Fortitude +32, Reflexes +23, Will +25

\textbf{Skills} Arcane +11, Stealth +7, Awareness +16, Story +11

\textbf{Damage Immunity} cold, weapons +1

\textbf{Senses} darkvision 40m, blindsight 20m

\textbf{Languages} Common, Draconic

\textbf{Challenge} 23 (50000 XP)

\textit{\textbf{Legendary Endurance (3 / Day).}} If the dragon fails a Saving Throw, it may choose to succeed instead.

\textbf{Actions}

\textit{\textbf{Multiattack.}} The dragon can use its Frightening Presence. Then make three attacks: one with the bite and two with the paws
claws.

\textit{\textbf{Claw.} Melee Weapon Attack}: +30 to hit, reach 3m, one target.

\textit{Hit:} 17 (2d6 + 10) slashing damage, 3 bleed damage (up to a maximum of 20).

\textit{\textbf{Tail.} Melee Weapon Attack}: +30 to hit, reach 6m, one target.

\textit{Hit:} 19 (2d8 + 10) bludgeoning damage.

\textit{\textbf{Bite.} Melee Weapon Attack}: +30 to hit, reach 5 meters, one target.

\textit{Hit:} 21 (2d10 + 10) piercing damage.

\textit{\textbf{Fearful Presence.}} Any creature of the dragon's choice that is within 16 meters of it and aware of its presence must succeed at a DC 26 Will save or be frightened for 1 minute. A creature can repeat the Saving Throw at the end of each of its rounds, ending the effect on a successful one. If the creature's Saving Throw succeeds or the effect ends for it, the creature is immune to the dragon's Dreadful Presence for the next 24 hours.

\textit{\textbf{Breath Weapon (Reload 5-6).}} The dragon uses one of the following breath weapons:

\textit{Electricity Breath.} The dragon exhales an lighting blast in a 27-meter cone. Each creature in the area must make a DC 26 Fortitude save, taking 67 (15d8) electricity damage on a failed save, or half as much damage on a successful one.

\textit{Crippling Breath.} The dragon exhales a paralyzing gas in a 24-meter cone. Each creature in the area must succeed on a Fortitude 26 save or be paralyzed for 1 minute. A creature can repeat the Saving Throw at the end of each of its rounds, ending the effect on itself on a success.

\textit{\textbf{Shapeshift.}} The dragon can magically transform into a humanoid or beast whose challenge rating is equal to or lower than its own, or revert to its true form. Upon death it returns to its true form.

Any equipment it is wearing or carrying is absorbed or carried into the new form (dragon's choice).

In the new form, the dragon retains its Traits, Hit Points, Hit Dice, speech, proficiencies, Legendary Endurance, lair actions, and Intelligence, Wisdom, and Charisma scores, in addition to this action. Its stats and abilities are otherwise replaced by those of the new form, except for the new form's additional Actions.

\textbf{Additional Actions}

The dragon can perform 3 additional Actions, chosen from the options below. He can only use one Additional Action at a time, and only at the end of another creature's turn. The dragon recovers expended additional Actions at the start of its round.

\textbf{Wing Attack (Costs 2 Actions).} The dragon flaps its wings. Each creature within 5 meters of the dragon must succeed on a DC 26 Reflex save or take 17 (2d6 + 10) bludgeoning damage and be knocked prone. The dragon can then fly up to half its flying speed.

\textbf{Tail Attack.} The dragon performs a tail attack.

\textbf{Locate.} The dragon makes a Wisdom (Awareness) check.

\textbf{Ecology}\\
Environment: Temperate Mountains\\
Organization: Solitary\\
\textbf{Treasure}: Triple\\
\textbf{Description}\\
Of all dragons, the silver ones are the bravest, and hold a code of chivalry that requires them to help the weak, defeat evil, and behave honorably. The most famous silver dragon on Yeru is Elysan.\\
\textbf{Enchantments}\index{Silver Dragon Spell}\\
This dragon's favorite spells are:\\
- Slow\\
- Fabricate\\
- Dream


\

\index[Monsters]{Dragon, Ancient White}\textbf{Ancient White Dragon}

\textit{Gargantuan dragon, chaotic evil}

\textbf{STRENGTH} +8

\textbf{DEXTERITY} +0

\textbf{CONSTITUTION} +8

\textbf{INTELLIGENCE} +0

\textbf{WISDOM} +1

\textbf{CHARISMA} +2

\textbf{Initiative} +0 -- \textbf{Defence} 30

\textbf{Hit Points} 333 (18x3d6 + 144)

\textbf{Movement} 12m, swim 12m, dig 12m, fly 24m

\textbf{Saving Throws} Fortitude +28, Reflexes +20, Will +21

\textbf{Skills} Stealth +6, Awareness +13

\textbf{Damage Immunity} cold, weapons +1

\textbf{Senses} darkvision 40m, blindsight 20m

\textbf{Languages} Common, Draconic

\textbf{Challenge} 20 (25000 XP)

\textit{\textbf{Icewalking.}} The dragon can move and climb icy surfaces without needing to make ability checks. Also, hindering terrain composed of ice or snow costs him no additional movement.

\textit{\textbf{Legendary Endurance (3 / Day).}} If the dragon fails a Saving Throw, it may choose to succeed instead.

\textbf{Actions}

\textit{\textbf{Multiattack.}} The dragon can use its Frightening Presence. Then make three attacks: one with the bite and two with the claws.

\textit{\textbf{Claw.} Melee Weapon Attack}: +30 to hit, reach 3m, one target.

\textit{Hit:} 15 (2d6 + 8) slashing damage, 3 bleed damage (up to a maximum of 20).

\textit{\textbf{Tail.} Melee Weapon Attack}: +30 to hit, reach 6m, one target.

\textit{Hit:} 17 (2d8 + 8) bludgeoning damage.

\textit{\textbf{Bite.} Melee Weapon Attack}: +30 to hit, reach 5 meters, one target.

\textit{Hit:} 19 (2d10 + 8) piercing damage plus 9 (2d8) cold damage.

\textit{\textbf{Fearful Presence.}} Any creature of the dragon's choice that is within 16 meters of it and aware of its presence must succeed on a DC 25 Will save or be frightened for 1 minute. A creature can repeat the Saving Throw at the end of each of its rounds, ending the effect on a successful one. If the creature's Saving Throw succeeds or the effect ends for it, the creature is immune to the dragon's Dreadful Presence for the next 24 hours.

\textit{\textbf{Icy Breath (Cooldown 4-6).}} The dragon exhales a blast of ice in a 27-meter cone. Each creature in that area must make a DC 25 Fortitude save, taking 72 (16d8) cold damage on a failed save, or half as much damage on a successful one.

\textbf{Additional Actions}

The dragon can perform 3 additional Actions, chosen from the options below. He can only use one Additional Action at a time, and only at the end of another creature's turn. The dragon recovers expended additional Actions at the start of its round.

\textbf{Wing Attack (Costs 2 Actions).} The dragon flaps its wings. Each creature within 5 meters of the dragon must succeed on a DC 25 Reflex save or take 15 (2d6 + 8) bludgeoning damage and be knocked prone. The dragon can then fly up to half its flight move.

\textbf{Tail Attack.} The dragon performs a tail attack.

\textbf{Locate.} The dragon makes a Wisdom (Awareness) check.

\textbf{Ecology}\\
Environment: Cold Mountains\\
Organization: Solitary\\
\textbf{Treasure}: Triple\\
\textbf{Description}\\
White dragons are among the wildest and most "animal" of all dragons.
They love cold, icy places, finding refuge in colder valleys such as icy mountain peaks and frigid steppes.

White Dragons have a wild appearance and almost always show their teeth and claws are drawn to move nimbly on the frozen ground.
They have no movement penalty on these terrains.

They exploit their natural camouflage to attack and capture prey, they are excellent hunters, very intelligent in exploiting the environment.

Little inclined to magic, however, they know how to blow ice shards much more frequently than other dragons. It is immune to cold and ice-based attacks.

Their lairs are frozen caves in the mountains or dug into the most massive glaciers.

White dragons have +1d6 on Magic Test and can ignore one die rolled on the Water List check and are immune to cold.\\

\textbf{Enchantments}\index{White Dragon Spell's}\\
This dragon's favorite spells are:\\
- Fire Shield\\
- Ice Storm\\
- Sleet Storm


\

\index[Monsters]{Dragon, Baby Black}\textbf{Baby Black Dragon}

\textit{Medium Dragon, Chaotic Evil}

\textbf{STRENGTH} +2

\textbf{DEXTERITY} +2

\textbf{CONSTITUTION} +1

\textbf{INTELLIGENCE} +0

\textbf{WISDOM} +0

\textbf{CHARISMA} +1

\textbf{Initiative} +2 -- \textbf{Defence} 18

\textbf{Hit Points} 33 (6d8 + 6)

\textbf{Move} 9m, climb 9m, fly 18m

\textbf{Saving Throws} Fortitude +2, Reflexes +2, Will +0

\textbf{Skills} Stealth +4, Awareness +4

\textbf{Damage Immunity} acid

\textbf{Senses} Darkvision 18m, blindsight 3m

\textbf{Languages} Draconic

\textbf{Challenge} 2 (450 XP)

\textit{\textbf{Amphibious.}} The dragon can breathe air and water.

\textbf{Actions}

\textit{\textbf{Bite.} Melee Weapon Attack}: +4 to hit, reach 1m, one target.

\textit{Hit:} 7 (1d10 + 2) piercing damage plus 2 (1d4) acid damage.

\textit{\textbf{Acid Breath (Recharge 5-6).}} The dragon exhales acid in a 5 meter line that is 1 meter wide. Each creature in that area must make a DC 11 Reflex save and take 22 (5d8) acid damage on a failed save, or half as much damage on a successful one.

\textbf{Ecology}\\
Environment: Warm Swamps\\
Organization: Solitary\\
\textbf{Treasure}: Triple\\
\textbf{Description}\\
Read Ancient Black Dragon Description.

\

\index[Monsters]{Dragon, Baby Blue}\textbf{Baby Blue Dragon}

\textit{Huge Dragon, Lawful Evil}

\textbf{STRENGTH} +3

\textbf{DEXTERITY} +0

\textbf{CONSTITUTION} +2

\textbf{INTELLIGENCE} +1

\textbf{WISDOM} +0

\textbf{CHARISMA} +2

\textbf{Initiative} +1 -- \textbf{Defence} 19

\textbf{Hit Points} 52 (8d8 + 16)

\textbf{Move} 9m, dig 5m, fly 18m

\textbf{Saving Throws} Fortitude +4, Reflexes +1, Will +1

\textbf{Skills} Stealth +2, Awareness +4

\textbf{Damage Immunity} electricity

\textbf{Senses} Darkvision 18m, blindsight 3m

\textbf{Languages} Draconic

\textbf{Challenge} 3 (700 XP)

\textbf{Actions}

\textit{\textbf{Bite.} Melee Weapon Attack}: +5 to hit, reach 1m, one target.

\textit{Hit:} 8 (1d10 + 3) piercing damage plus 3 (1d6) electricity damage.

\textit{\textbf{Lightning Breath (Cooldown 5-6).}} The dragon exhales lightning in a line 10 meters long and 1 meter wide. Each creature in that line must make a DC 13 Reflex save and take 22 (4d10) electricity damage on a failed save, or half as much damage on a successful one.

\textbf{Ecology}\\
Environment: Mountain peaks\\
Organization: Solitary\\
\textbf{Treasure}: Triple\\
\textbf{Description}\\
Read Ancient Blu Dragon Description.


\

\index[Monsters]{Dragon, Baby Brass}\textbf{Baby Brass Dragon}

\textit{Medium Dragon, Chaotic Good}

\textbf{STRENGTH} +2

\textbf{DEXTERITY} +0

\textbf{CONSTITUTION} +1

\textbf{INTELLIGENCE} +0

\textbf{WISDOM} +0

\textbf{CHARISMA} +1

\textbf{Initiative} +0 -- \textbf{Defence} 17

\textbf{Hit Points} 16 (3d8 + 3)

\textbf{Move} 9m, dig 5m, fly 18m

\textbf{Saving Throws} Fortitude +2, Reflexes +0, Will +1

\textbf{Skills} Stealth +2, Awareness +4

\textbf{Damage Immunity} Fire

\textbf{Senses} Darkvision 18m, blindsight 3m

\textbf{Languages} Draconic

\textbf{Challenge} 1 (200 XP)

\textbf{Actions}

\textit{\textbf{Bite.} Melee Weapon Attack}: +4 to hit, reach 1m, one target.

\textit{Hit:} 7 (1d10 + 2) piercing damage.

\textit{\textbf{Breath Weapon (Cooldown 5-6).}} The dragon uses one of the following breath weapons:

\textit{Fiery Breath.} The dragon exhales fire in a line 6 meters long and 1 meter wide. Each creature in the line must make a DC 11 Reflex save, taking 14 (4d6) fire damage on a failed save, or half as much damage on a successful one.

\textit{Sleep Breath.} The dragon exhales sleeping gas in a 5-meter cone. Each creature in that area must succeed on a Fortitude 11 save or be knocked unconscious for 1 minute. This effect ends if the unconscious creature takes damage or someone takes an action to awaken it.

\textbf{Ecology}\\
Environment: Hot Deserts\\
Organization: Solitary\\
\textbf{Treasure}: Triple\\
\textbf{Description}\\
Excellent conversationalists, brass dragons prefer to talk rather than fight. Brass dragons lair near humanoid settlements, where they can hear the latest news and gossip.


\

\index[Monsters]{Dragon, Baby Bronze}\textbf{Baby Bronze Dragon}

\textit{Medium Dragon, Chaotic Good}

\textbf{STRENGTH} +3

\textbf{DEXTERITY} +0

\textbf{CONSTITUTION} +2

\textbf{INTELLIGENCE} +1

\textbf{WISDOM} +0

\textbf{CHARISMA} +2

\textbf{Initiative} +1 -- \textbf{Defence} 18

\textbf{Hit Points} 32 (5d8 + 10)

\textbf{Move} 9m, swim 9m, fly 18m

\textbf{Saving Throws} Fortitude +2, Reflexes +1, Will +1

\textbf{Skills} Stealth +2, Awareness +4

\textbf{Damage Immunity} Electricity

\textbf{Senses} Darkvision 18m, blindsight 3m

\textbf{Languages} Draconic

\textbf{Challenge} 2 (450 XP)

\textit{\textbf{Amphibious.}} The dragon can breathe air and water.

\textbf{Actions}

\textit{\textbf{Bite.} Melee Weapon Attack}: +5 to hit, reach 1 m, one target.

\textit{Hit:} 8 (1d10 + 3) piercing damage.

\textit{\textbf{Breath Weapon (Reload 5-6).}} The dragon uses one of the following breath weapons:

\textit{Lightning Breath.} The dragon exhales lightning in a line 12 meters long and 1 meter wide. Each creature in the line must make a DC 12 Reflex save, taking 16 (3d10) electricity damage on a failed save, or half as much damage on a successful one.

\textit{Repulsive Breath.} The dragon exhales repulsive energy in a 10m cone. Each creature in that area must succeed on a DC 12 Fortitude save or be 10 meters away from the dragon.

\textbf{Ecology}\\
Environment: Temperate Coastal Zones\\
Organization: Solitary\\
\textbf{Treasure}: Triple\\
\textbf{Description}\\
Bronze dragons are known to ally with travelers and adventurers if the cause and reward are just and fitting.

\

\index[Monsters]{Dragon, Baby Gold}\textbf{Baby Gold Dragon}

\textit{Medium Dragon, Lawful Good}

\textbf{STRENGTH} +4

\textbf{DEXTERITY} +2

\textbf{CONSTITUTION} +3

\textbf{INTELLIGENCE} +2

\textbf{WISDOM} +0

\textbf{CHARISMA} +3

\textbf{Initiative} +2 -- \textbf{Defence} 19

\textbf{Hit Points} 60 (8d8 + 24)

\textbf{Move} 9m, swim 9m, fly 18m

\textbf{Saving Throws} Fortitude +3, Reflexes +2, Will +1

\textbf{Skills} Stealth +4, Awareness +4

\textbf{Damage Immunity} Fire

\textbf{Senses} Darkvision 18m, blindsight 3m

\textbf{Languages} Draconic

\textbf{Challenge} 3 (700 XP)

\textit{\textbf{Amphibious.}} The dragon can breathe air and water.

\textbf{Actions}

\textit{\textbf{Bite.} Melee Weapon Attack}: +6 to hit, reach 1m, one target.

\textit{Hit:} 9 (1d10 + 4) piercing damage.

\textit{\textbf{Breath Weapon (Cooldown 5-6).}} The dragon uses one of the following breath weapons:

\textit{Fiery Breath.} The dragon breathes fire in a 5-meter cone. Each creature in the area must make a DC 13 Reflex save, taking 22 (4d10) fire damage on a failed save, or half as much damage on a successful one.

\textit{Weakening Breath.} The dragon exhales gas in a 5-meter cone. Each creature in that area must succeed on a DC 13 Fortitude save or have -1d6 on Strength-based attack rolls, Strength checks, and Fortitude saves for 1 minute. A creature can repeat the Saving Throw at the end of each of its rounds, ending the effect on itself on a success.

\textbf{Ecology}\\
Environment: Hot Plains\\
Organization: Solitary\\
\textbf{Treasure}: Triple\\
\textbf{Description}\\
Golden dragons are the emblem of virtue. Other Ljust's dragons revere them as agents of divine powers and exemplary members of draconic race, and often seek them for advice or aid.

\

\index[Monsters]{Dragon, Baby Green}\textbf{Baby Green Dragon}

\textit{Medium Dragon, Lawful Evil}

\textbf{STRENGTH} +2

\textbf{DEXTERITY} +1

\textbf{CONSTITUTION} +1

\textbf{INTELLIGENCE} +2

\textbf{WISDOM} +0

\textbf{CHARISMA} +1

\textbf{Initiative} +2 -- \textbf{Defence} 18

\textbf{Hit Points} 38 (7d8 + 7)

\textbf{Move} 9m, swim 9m, fly 18m

\textbf{Saving Throws} Fortitude +3, Reflexes +1, Will +0

\textbf{Skills} Stealth +3, Awareness +4

\textbf{Immunity to Damage} Poison

\textbf{Condition Immunity} poisoned

\textbf{Senses} Darkvision 18m, blindsight 3m

\textbf{Languages} Draconic

\textbf{Challenge} 2 (450 XP)

\textit{\textbf{Amphibious.}} The dragon can breathe air and water.

\textbf{Actions}

\textit{\textbf{Bite.} Melee Weapon Attack}: +4 to hit, reach 1m, one target.

\textit{Hit:} 7 (1d10 + 2) piercing damage plus 3 (1d6) poison damage.

\textit{\textbf{Venom Breath (Cooldown 5-6).}} The dragon exhales poisonous gas in a 5-meter cone. Each creature in that area must make a DC 11 Fortitude save, taking 21 (6d6) poison damage on a failed save, or half as much damage on a successful one.

\textbf{Ecology}\\
Environment: Temperate Forests\\
Organization: Solitary\\
\textbf{Treasure}: Triple\\
\textbf{Description}\\
Read Ancient Green Dragon Description

\

\index[Monsters]{Dragon, Baby Red}\textbf{Baby Red Dragon}

\textit{Medium Dragon, Chaotic Evil}

\textbf{STRENGTH} +4

\textbf{DEXTERITY} +0

\textbf{CONSTITUTION} +3

\textbf{INTELLIGENCE} +1

\textbf{WISDOM} +0

\textbf{CHARISMA} +2

\textbf{Initiative} +1 -- \textbf{Defence} 19

\textbf{Hit Points} 75 (10d8 + 30)

\textbf{Move} 9m, climb 9m, fly 18m

\textbf{Saving Throws} Fortitude +4, Reflexes +3, Will +1

\textbf{Skills} Stealth +2, Awareness +4

\textbf{Damage Immunity} Fire

\textbf{Senses} Darkvision 18m, blindsight 3m

\textbf{Languages} Draconic

\textbf{Challenge} 4 (1100 XP)

\textbf{Actions}

\textit{\textbf{Bite.} Melee Weapon Attack}: +8 to hit, reach 1m, one target.

\textit{Hit:} 9 (1d10 + 4) piercing damage plus 3 (1d6) fire damage.

\textit{\textbf{Fiery Breath (Cooldown 5-6).}} The dragon exhales fire in a 5-meter cone. Each creature in that area must make a DC 15 Reflex save and take 24 (7d6) fire damage on a failed save, or half as much damage on a successful one.

\textbf{Ecology}\\
Environment: Hot Mountains\\
Organization: Solitary\\
\textbf{Treasure}: Triple\\
\textbf{Description}\\
Read Ancient Red Dragon Description

\

\index[Monsters]{Dragon, Baby Silver}\textbf{Baby Silver Dragon}

\textit{Medium Dragon, Lawful Good}

\textbf{STRENGTH} +4

\textbf{DEXTERITY} +0

\textbf{CONSTITUTION} +3

\textbf{INTELLIGENCE} +1

\textbf{WISDOM} +0

\textbf{CHARISMA} +2

\textbf{Initiative} +1 -- \textbf{Defence} 18

\textbf{Hit Points} 45 (6d8 + 18)

\textbf{Move} 9m, fly 18m

\textbf{Saving Throws} Fortitude +3, Reflexes +3, Will +2

\textbf{Skills} Stealth +2, Awareness +4

\textbf{Damage Immunity} cold

\textbf{Senses} Darkvision 18m, blindsight 3m

\textbf{Languages} Draconic

\textbf{Challenge} 2 (450 XP)

\textbf{Actions}

\textit{\textbf{Bite.} Melee Weapon Attack}: +6 to hit, reach 1m, one target.

\textit{Hit:} 9 (1d10 + 4) piercing damage.

\textit{\textbf{Breath Weapon (Cooldown 5-6).}} The dragon uses one of the following breath weapons:

\textit{Icy Breath.} The dragon exhales an icy blast in a 5-meter cone. Each creature in the area must make a Fortitude save 13, taking 18 (4d8) cold damage on a failed save, or half as much damage on a successful one.

\textit{Crippling Breath.} The dragon exhales a paralyzing gas in a 5-meter cone. Each creature in the area must succeed at a Fortitude 13 save or be paralyzed for 1 minute. A creature can repeat the Saving Throw at the end of each of its rounds, ending the effect on itself on a success.

\textbf{Ecology}\\
Environment: Temperate Mountains\\
Organization: Solitary\\
\textbf{Treasure}: Triple\\
\textbf{Description}\\
Of all dragons, the silver ones are the bravest, and hold a code of chivalry that requires them to aid the weak, defeat evil, and behave honorably.


\

\index[Monsters]{Dragon, Baby White}\textbf{Baby White Dragon}

\textit{Medium Dragon, Chaotic Evil}

\textbf{STRENGTH} +2

\textbf{DEXTERITY} +0

\textbf{CONSTITUTION} +2

\textbf{INTELLIGENCE} -3

\textbf{WISDOM} +0

\textbf{CHARISMA} +0

\textbf{Initiative} +0 -- \textbf{Defence} 17

\textbf{Hit Points} 32 (5d8 + 10)

\textbf{Movement} 9m, swim 9m, dig 5m, fly 18m

\textbf{Saving Throws} Fortitude +2, Reflexes +1, Will +1

\textbf{Skills} Stealth +2, Awareness +4

\textbf{Damage Immunity} cold

\textbf{Senses} Darkvision 18m, blindsight 3m

\textbf{Languages} Draconic

\textbf{Challenge} 2 (450 XP)

\textbf{Actions}

\textit{\textbf{Bite.} Melee Weapon Attack}: +5 to hit, reach 3m, one target.

\textit{Hit:} 15 (2d10 + 4) piercing damage plus 4 (1d8) cold damage.

\textit{\textbf{Icy Breath (Cooldown 4-6).}} The dragon exhales a blast of ice in a 5-meter cone. Each creature in that area must make a DC 12 Fortitude save, taking 22 (5d8) cold damage on a failed save, or half as much damage on a successful one.

\textbf{Ecology}\\
Environment: Cold Mountains\\
Organization: Solitary\\
\textbf{Treasure}: Triple\\
\textbf{Description}\\
Read Ancient White Dragon Description.

\

\index[Monsters]{Dragon, Red Ancient}\textbf{Red Ancient Dragon}

\textit{Gargantuan dragon, chaotic evil}

\textbf{STRENGTH} +10

\textbf{DEXTERITY} +0

\textbf{CONSTITUTION} +9

\textbf{INTELLIGENCE} +4

\textbf{WISDOM} +2

\textbf{CHARISMA} +6

\textbf{Initiative} +4 -- \textbf{Defence} 34

\textbf{Hit Points} 546 (28x3d6 + 252)

\textbf{Move} 12m, climb 12m, fly 24m

\textbf{Saving Throws} Fortitude +33, Reflexes +24, Will +26

\textbf{Skills} Stealth +7, Awareness +16

\textbf{Damage Immunity} Fire, weapons +1

\textbf{Senses} darkvision 40m, blindsight 20m

\textbf{Languages} Common, Draconic

\textbf{Challenge} 24 (62000 XP)

\textit{\textbf{Legendary Endurance (3 / Day).}} If the dragon fails a Saving Throw, it may choose to succeed instead.

\textbf{Actions}

\textit{\textbf{Multiattack.}} The dragon can use its Frightful Presence and then make three attacks: one with its bite and two with its claws.

\textit{\textbf{Claw.} Melee Weapon Attack}: +30 to hit, reach 3m, one target.

\textit{Hit:} 17 (2d6 + 10) slashing damage, 3 bleed damage (up to a maximum of 20).

\textit{\textbf{Tail.} Melee Weapon Attack}: +30 to hit, reach 6m, one target.

\textit{Hit:} 19 (2d8 + 10) bludgeoning damage.

\textit{\textbf{Bite.} Melee Weapon Attack}: +30 to hit, reach 5 meters, one target.

\textit{Hit:} 21 (2d10 + 10) piercing damage plus 14 (4d6) fire damage.

\textit{\textbf{Fearful Presence.}} Any creature of the dragon's choice that is within 16 meters of it and aware of its presence must succeed on a DC 28 Will save or be frightened for 1 minute. A creature can repeat the Saving Throw at the end of each of its rounds, ending the effect on a successful one. If the creature's Saving Throw succeeds or the effect ends for it, the creature is immune to the dragon's Dreadful Presence for the next 24 hours.

\textit{\textbf{Fiery Breath (Cooldown 5-6).}} The dragon exhales fire in a 27-meter cone. Each creature in that area must make a DC 28 Reflex save and take 91 (26d6) fire damage on a failed save, or half as much damage on a successful one.

\textbf{Additional Actions}

The dragon can perform 3 additional Actions, chosen from the options below. He can only use one Additional Action at a time, and only at the end of another creature's turn. The dragon recovers expended additional Actions at the start of its round.

\textbf{Wing Attack (Costs 2 Actions).} The dragon flaps its wings. Each creature within 5 meters of the dragon must succeed on a DC 28 Reflex save or take 17 (2d6 + 10) bludgeoning damage and be knocked prone. The dragon can then fly up to half its flight move.

\textbf{Tail Attack.} The dragon performs a tail attack.

\textbf{Locate.} The dragon makes a Wisdom (Awareness) check.

\textit{\textbf{Enraged}}: The red dragon shakes and roars. 1x per day, the first time he's Angry, ends all negative conditions on him and all abilities recharge. The breath recharges on 3-6.


\textbf{Ecology}\\
Environment: Volcano Mountains\\
Organization: Solitary\\
\textbf{Treasure}: Triple\\
\textbf{Description}\\
The Red Dragon believe himself to be the King of Dragons due to his physical power and the breath capable of melting the stone.

Red dragons are the largest dragons in both build and wingspan.
Often the scales, almost blood red, have sharp and elongated edges.

Red Dragons prefer warm mountains and if possible directly inside a volcano.

They fight by exploiting their size, wings, bite, claws... in short, everything they are and have at their disposal. A Red Dragon always fights to the death he doesn't retreat or run away or give up a challenge, the pride of which they are bloated does not allow them to show themselves weak.

A Red Dragon has +1d6 on Magic Test and can ignore one die rolled on a Fire List check and is immune to fire.\\

\textbf{Enchantments}\index{Red Dragon Spell}\\
This dragon's favorite spells are:\\
- Fireball\\
- Incendiary Cloud\\
- Wall of fire


\

\index[Monsters]{Dragon, Young Black}\textbf{Young Black Dragon}

\textit{Large dragon, chaotic evil}

\textbf{STRENGTH} +4

\textbf{DEXTERITY} +2

\textbf{CONSTITUTION} +3

\textbf{INTELLIGENCE} +1

\textbf{WISDOM} +0

\textbf{CHARISMA} +2

\textbf{Initiative} +2 -- \textbf{Defence} 22

\textbf{Hit Points} 127 (15d10 + 45)

\textbf{Move} 12m, climb 12m, fly 24m

\textbf{Saving Throws} Fortitude +9, Reflexes +8, Will +7

\textbf{Skills} Stealth +5, Awareness +6

\textbf{Damage Immunity} acid

\textbf{Senses} darkvision 40m, blindsight 20m

\textbf{Languages} Common, Draconic

\textbf{Challenge} 7 (2900 XP)

\textit{\textbf{Amphibious.}} The dragon can breathe air and water.

\textbf{Actions}

\textit{\textbf{Multiattack.}} The dragon can make three attacks: one with its bite and two with its claws.

\textit{\textbf{Claw.} Melee Weapon Attack}: +9 to hit, reach 1m, one target.

\textit{Hit:} 11 (2d6 + 4) slashing damage, 1 bleed damage.

\textit{\textbf{Bite.} Melee Weapon Attack}: +9 to hit, reach 3m, one target.

\textit{Hit:} 11 (2d10 + 4) piercing damage plus 4 (1d8) acid damage.

\textit{\textbf{Acid Breath (Recharge 5-6).}} The dragon exhales acid in a 10m line that is 1 meter wide. Each creature in that area must make a DC 16 Reflex save and take 49 (11d8) acid damage on a failed save, or half as much damage on a successful one.

\textit{\textbf{Enraged}}: the Young Black Dragon rechard his Acid Breath. Cost 1 Action.

\textbf{Ecology}\\
Environment: Warm Swamps\\
Organization: Solitary\\
\textbf{Treasure}: Triple\\
\textbf{Description}\\
Read Ancient Black Dragon Description.\\
\textbf{Enchantments}\index{Black Dragon Spell}\\
This dragon's favorite spells are:\\
- Create Undead\\
- Raise Dead\\
- Bestow Curse


\

\index[Monsters]{Dragon, Young Blue}\textbf{Young Blue Dragon}

\textit{Huge Dragon, Lawful Evil}

\textbf{STRENGTH} +5

\textbf{DEXTERITY} +0

\textbf{CONSTITUTION} +4

\textbf{INTELLIGENCE} +2

\textbf{WISDOM} +1

\textbf{CHARISMA} +3

\textbf{Initiative} +2 -- \textbf{Defence} 23

\textbf{Hit Points} 152 (16d10 + 64)

\textbf{Move} 12m, dig 12m, fly 24m

\textbf{Saving Throws} Fortitude +10, Reflexes +8, Will +8

\textbf{Skills} Stealth +4, Awareness +9

\textbf{Damage Immunity} Electricity

\textbf{Senses} darkvision 40m, blindsight 20m

\textbf{Languages} Common, Draconic

\textbf{Challenge} 9 (5000 XP)

\textbf{Actions}

\textit{\textbf{Multiattack.}} The dragon can make three attacks: one with its bite and two with its claws.

\textit{\textbf{Claw.} Melee Weapon Attack}: +13 to hit, reach 1m, one target.

\textit{Hit:} 12 (2d6 + 5) slashing damage, 1 bleed damage.

\textit{\textbf{Bite.} Melee Weapon Attack}: +13 to hit, reach 3m, one target.

\textit{Hit:} 16 (2d10 + 5) piercing damage plus 5 (1d10) electricity damage.

\textit{\textbf{Lightning Breath (Cooldown 5-6).}} The dragon exhales lightning in a line 20 meters long and 1 meter wide. Each creature in that line must make a DC 16 Reflex save and take 55 (10d10) electricity damage on a failed save, or half as much damage on a successful one.

\textit{\textbf{Enraged}}: the Young Blue Dragon recharg his Lightning breath

\textbf{Ecology}\\
Environment: Mountain peaks\\
Organization: Solitary\\
\textbf{Treasure}: Triple\\
\textbf{Description}\\
Read Ancient Blu Dragon Description.\\
\textbf{Enchantments}\index{Blu Dragon Spell}\\
This dragon's favorite spells are:\\
- Deadly Mist\\
- Call Lightning\\
- Ice Storm


\

\index[Monsters]{Dragon, Young Brass}\textbf{Young Brass Dragon}

\textit{Large Dragon, Chaotic Good}

\textbf{STRENGTH} +4

\textbf{DEXTERITY} +0

\textbf{CONSTITUTION} +3

\textbf{INTELLIGENCE} +1

\textbf{WISDOM} +0

\textbf{CHARISMA} +2

\textbf{Initiative} +1 -- \textbf{Defence} 20

\textbf{Hit Points} 110 (13d10 + 39)

\textbf{Move} 12m, dig 6m, fly 24m

\textbf{Saving Throws} Fortitude +9, Reflexes +8, Will +7

\textbf{Skills} Stealth +3, Awareness +6, Deceive +5

\textbf{Damage Immunity} Fire

\textbf{Senses} darkvision 40m, blindsight 10m

\textbf{Languages} Common, Draconic

\textbf{Challenge} 6 (2300 XP)

\textbf{Actions}

\textit{\textbf{Multiattack.}} The dragon can make three attacks: one with its bite and two with its claws.

\textit{\textbf{Claw.} Melee Weapon Attack}: +7 to hit, reach 1m, one target.

\textit{Hit:} 11 (2d6 + 4) slashing damage, 1 bleed damage.

\textit{\textbf{Bite.} Melee Weapon Attack}: +7 to hit, reach 3m, one target.

\textit{Hit:} 15 (2d10 + 4) piercing damage.

\textit{\textbf{Breath Weapon (Cooldown 5-6).}} The dragon uses one of the following breath weapons:

\textit{Fiery Breath.} The dragon exhales fire in a line 12 meters long and 1 meter wide. Each creature in the line must make a DC 15 Reflex save, taking 42 (12d6) fire damage on a failed save, or half as much damage on a successful one. \textit{Sleep Breath.} The dragon exhales sleeping gas in a 10m cone. Each creature in that area must succeed on a Fortitude 14 save or be knocked unconscious for 5 minutes. This effect ends if the unconscious creature takes damage or someone takes an action to awaken it.

\textbf{Ecology}\\
Environment: Hot Deserts\\
Organization: Solitary\\
\textbf{Treasure}: Triple\\
\textbf{Description}\\
Excellent conversationalists, brass dragons prefer to talk rather than fight. Brass dragons lair near humanoid settlements, where they can hear the latest news and gossip.\\
\textbf{Enchantments}\index{Brass Dragon Spell}\\
This dragon's favorite spells are:\\
- True Seeing\\
- Knowledge of Legends\\
- Scrying

\

\index[Monsters]{Dragon, Young Bronze}\textbf{Young Bronze Dragon}

\textit{Large Dragon, Chaotic Good}

\textbf{STRENGTH} +5

\textbf{DEXTERITY} +0

\textbf{CONSTITUTION} +4

\textbf{INTELLIGENCE} +2

\textbf{WISDOM} +1

\textbf{CHARISMA} +3

\textbf{Initiative} +2 -- \textbf{Defence} 22

\textbf{Hit Points} 142 (15d10 + 60)

\textbf{Movement} 12m, swim 12m, fly 24m

\textbf{Saving Throws} Fortitude +10, Reflexes +8, Will +10

\textbf{Skills} Stealth +3, Sense Emotions +4, Awareness +7

\textbf{Damage Immunity} Electricity

\textbf{Senses} darkvision 40m, blindsight 10m

\textbf{Languages} Common, Draconic

\textbf{Challenge} 8 (3900 XP)

\textit{\textbf{Amphibious.}} The dragon can breathe air and water.

\textbf{Actions}

\textit{\textbf{Multiattack.}} The dragon can use make three attacks: one with its bite and two with its claws.

\textit{\textbf{Claw.} Melee Weapon Attack}: +12 to hit, reach 1m, one target.

\textit{Hit:} 12 (2d6 + 5) slashing damage, 1 bleed damage.

\textit{\textbf{Bite.} Melee Weapon Attack}: +12 to hit, reach 3m, one target.

\textit{Hit:} 16 (2d10 + 5) piercing damage.

\textit{\textbf{Breath Weapon (Cooldown 5-6).}} The dragon uses one of the following breath weapons:

\textit{Lightning Breath.} The dragon exhales lightning in a line 20 meters long and 1 meter wide. Each creature in the line must make a DC 16 Reflex save, taking 55 (10d10) electricity damage on a failed save, or half as much damage on a successful one.

\textit{Repulsive Breath.} The dragon exhales repulsive energy in a 10m cone. Each creature in that area must succeed on a DC 16 Fortitude save or be pulled 13 meters away from the dragon.

\textit{\textbf{Enraged}}: Young Bronze Dragon recharge one of his breath.

\textbf{Ecology}\\
Environment: Temperate Coastal Zones\\
Organization: Solitary\\
\textbf{Treasure}: Triple\\
\textbf{Description}\\
Bronze dragons are known to ally with travelers and adventurers if cause and reward are just and fitting\\
\textbf{Enchantments}\index{Bronze Dragon Spell}\\
This dragon's favorite spells are:\\
- Orb of Invulnerability\\
- Freedom of Movement


\

\index[Monsters]{Dragon, Young Copper}\textbf{Young Copper Dragon}

\textit{Large Dragon, Chaotic Good}

\textbf{STRENGTH} +4

\textbf{DEXTERITY} +1

\textbf{CONSTITUTION} +3

\textbf{INTELLIGENCE} +3

\textbf{WISDOM} +1

\textbf{CHARISMA} +2

\textbf{Initiative} +3 -- \textbf{Defence} 21

\textbf{Hit Points} 119 (14d10 + 42)

\textbf{Move} 12m, climb 12m, fly 24m

\textbf{Saving Throws} Fortitude +9, Reflexes +8, Will +8

\textbf{Skills} Stealth +4, Deceive +5, Awareness +7

\textbf{Damage Immunity} acid

\textbf{Senses} darkvision 40m, blindsight 10m

\textbf{Languages} Common, Draconic

\textbf{Challenge} 7 (2900 XP)

\textbf{Actions}

\textit{\textbf{Multiattack.}} The dragon can make three attacks: one with its bite and two with its claws.

\textit{\textbf{Claw.} Melee Weapon Attack}: +10 to hit, reach 1m, one target.

\textit{Hit:} 11 (2d6 + 4) slashing damage, 1 bleed damage.

\textit{\textbf{Bite.} Melee Weapon Attack}: +10 to hit, reach 3m, one target.

\textit{Hit:} 15 (2d10 + 4) piercing damage.

\textit{\textbf{Breath Weapon (Cooldown 5-6).}} The dragon uses one of the following breath weapons:

\textit{Acid Breath.} The dragon exhales acid in a line 12 meters long and 1 meter wide. Each creature in the line must make a DC 16 Reflex save, taking 40 (9d8) acid damage on a failed save, or half as much damage on a successful one.

\textit{Slow Breath.} The dragon exhales gas in a 10m cone. Each creature in that area must succeed on a DC 16 Fortitude save. On a failed save, the creature can't use its reaction, has its speed halved, and can't make more than one attack during its round. Also, the creature can use either an action or a bonus action, but not both. These effects last for 1 minute. The creature can repeat the Saving Throw at the end of each of its rounds, ending the effect on itself on a success.

\textit{\textbf{Enraged}}: the Young Copper Dragon rechard one of his Breath. Cost 1 Action.

\textbf{Ecology}\\
Environment: Warm Hills\\
Organization: Solitary\\
\textbf{Treasure}: Triple\\
\textbf{Description}\\
This capricious dragon tries to hinder and frustrate his enemies during combat.\\
\textbf{Enchantments}\index{Copper Dragon Spell}\\
This dragon's favorite spells are:\\
- Blade Barrier\\
- Wall of Force\\
- Elastic Sphere

\

\index[Monsters]{Dragon, Young Gold}\textbf{Young Gold Dragon}

\textit{Large Dragon, Lawful Good}

\textbf{STRENGTH} +6

\textbf{DEXTERITY} +2

\textbf{CONSTITUTION} +5

\textbf{INTELLIGENCE} +3

\textbf{WISDOM} +1

\textbf{CHARISMA} +5

\textbf{Initiative} +3 -- \textbf{Defence} 23

\textbf{Hit Points} 178 (17d10 + 85)

\textbf{Movement} 12m, swim 12m, fly 24m

\textbf{Saving Throws} Fortitude +15, Reflexes +12, Will +11

\textbf{Skills} Stealth +6, Sense Emotions +5, Awareness +9, Deceive +9

\textbf{Damage Immunity} Fire

\textbf{Senses} darkvision 40m, blindsight 10m

\textbf{Languages} Common, Draconic

\textbf{Challenge} 10 (5900 XP)

\textit{\textbf{Amphibious.}} The dragon can breathe air and water.

\textbf{Actions}

\textit{\textbf{Multiattack.}} The dragon can make three attacks: one with its bite and two with its claws.

\textit{\textbf{Claw.} Melee Weapon Attack}: +16 to hit, reach 1m, one target.

\textit{Hit:} 13 (2d6 + 6) slashing damage, 1 bleed damage.

\textit{\textbf{Bite.} Melee Weapon Attack}: +16 to hit, reach 3m, one target.

\textit{Hit:} 17 (2d10 + 6) piercing damage.

\textit{\textbf{Breath Weapon (Cooldown 5-6).}} The dragon uses one of the following breath weapons:

\textit{Fiery Breath.} The dragon breathes fire in a 10m cone. Each creature in the area must make a DC 17 Reflex save, taking 55 (10d10) fire damage on a failed save, or half as much damage on a successful one.

\textit{Weakening Breath.} The dragon exhales gas in a 10m cone. Each creature in that area must succeed on a DC 17 Fortitude save or have -1d6 on Strength-based attack rolls, Strength checks, and Fortitude saves for 1 minute. A creature can repeat the Saving Throw at the end of each of its rounds, ending the effect on itself on a success.

\textit{\textbf{Enraged}}: the Young Gold Dragon recharge one of his breaths. Cost 1 Action.

\textbf{Ecology}\\
Environment: Hot Plains\\
Organization: Solitary\\
\textbf{Treasure}: Triple\\
\textbf{Description}\\
Golden dragons are the emblem of virtue. Other Ljust's dragons revere them as agents of divine powers and exemplary members of the draconic race, and often seek them for advice or help.\\
\textbf{Enchantments}\index{Gold Dragon Spell}\\
This dragon's favorite spells are:\\
- Heal\\
- Greater Restoration\\
- Black Tentacles


\

\index[Monsters]{Dragon, Young Green}\textbf{Young Green Dragon}

\textit{Large Dragon, Lawful Evil}

\textbf{STRENGTH} +4

\textbf{DEXTERITY} +1

\textbf{CONSTITUTION} +3

\textbf{INTELLIGENCE} +3

\textbf{WISDOM} +1

\textbf{CHARISMA} +2

\textbf{Initiative} +3 -- \textbf{Defence} 22

\textbf{Hit Points} 136 (16d10 + 48)

\textbf{Movement} 12m, swim 12m, fly 24m

\textbf{Saving Throws} Fortitude +9, Reflexes +7, Will +9

\textbf{Skills} Stealth +4, Deceive +5, Awareness +7

\textbf{Immunity to Damage} Poison

\textbf{Condition Immunity} poisoned

\textbf{Senses} darkvision 40m, blindsight 10m

\textbf{Languages} Common, Draconic

\textbf{Challenge} 8 (3900 XP)

\textit{\textbf{Amphibious.}} The dragon can breathe air and water.

\textbf{Actions}

\textit{\textbf{Multiattack.}} The dragon can make three attacks: one with its bite and two with its claws.

\textit{\textbf{Claw.} Melee Weapon Attack}: +11 to hit, reach 1m, one target.

\textit{Hit:} 11 (2d6 + 4) slashing damage, 1 bleed damage.

\textit{\textbf{Bite.} Melee Weapon Attack}: +11 to hit, reach 3m, one target.

\textit{Hit:} 15 (2d10 + 4) piercing damage plus 7 (2d6) poison damage.

\textit{\textbf{Venom Breath (Cooldown 5-6).}} The dragon exhales poisonous gas in a 10m cone. Each creature in that area must make a DC 16 Fortitude save and take 42 (12d6) poison damage on a failed save, or half as much damage on a successful one.

\textit{\textbf{Enraged}}: Youg Green Dragon recharge his Venom Breath.

\textbf{Ecology}\\
Environment: Temperate Forests\\
Organization: Solitary\\
\textbf{Treasure}: Triple\\
\textbf{Description}\\
Read Ancient Green Dragon Description\\
\textbf{Enchantments}\index{Green Dragon Spell}\\
This dragon's favorite spells are:\\
- Anti-Life Shell\\
- Locate Creature\\
- Remove poison


\

\index[Monsters]{Dragon, Young Red}\textbf{Young Red Dragon}

\textit{Large Dragon, Chaotic Evil}

\textbf{STRENGTH} +6

\textbf{DEXTERITY} +0

\textbf{CONSTITUTION} +5

\textbf{INTELLIGENCE} +2

\textbf{WISDOM} +0

\textbf{CHARISMA} +4

\textbf{Initiative} +2 -- \textbf{Defence} 23

\textbf{Hit Points} 178 (17d10 + 85)

\textbf{Move} 12m, climb 12m, fly 24m

\textbf{Saving Throws} Fortitude +15, Reflexes +10 Will +10

\textbf{Skills} Stealth +4, Awareness +8

\textbf{Damage Immunity} Fire

\textbf{Senses} darkvision 40m, blindsight 30ft.

\textbf{Languages} Common, Draconic

\textbf{Challenge} 10 (5900 XP)

\textbf{Actions}

\textit{\textbf{Multiattack.}} The dragon can make three attacks: one with its bite and two with its claws.

\textit{\textbf{Claw.} Melee Weapon Attack}: +16 to hit, reach 1m, one target.

\textit{Hit:} 13 (2d6 + 6) slashing damage, 1 bleed damage.

\textit{\textbf{Bite.} Melee Weapon Attack}: +16 to hit, reach 3m, one target.

\textit{Hit:} 17 (2d10 + 6) piercing damage plus 3 (1d6) fire damage.

\textit{\textbf{Fiery Breath (Cooldown 5-6).}} The dragon breathes fire in a 10m cone. Each creature in that area must make a DC 17 Reflex save and take 56 (16d6) fire damage on a failed save, or half as much damage on a successful one.

\textbf{\textbf{Angry:}} the Youg Red Dragon recharge the Fiery Breath. Cost 1 Action.

\textbf{Ecology}\\
Environment: Hot Mountains\\
Organization: Solitary\\
\textbf{Treasure}: Triple\\
\textbf{Description}\\
Read Ancient Red Dragon Description\\
\textbf{Enchantments}\index{Red Dragon Spell}\\
This dragon's favorite spells are:\\
- Fireball\\
- Incendiary Cloud\\
- Wall of fire


\

\index[Monsters]{Dragon, Young Silver}\textbf{Young Silver Dragon}

\textit{Large Dragon, Lawful Good}

\textbf{STRENGTH} +6

\textbf{DEXTERITY} +0

\textbf{CONSTITUTION} +5

\textbf{INTELLIGENCE} +2

\textbf{WISDOM} +0

\textbf{CHARISMA} +4

\textbf{Initiative} +2 -- \textbf{Defence} 23

\textbf{Hit Points} 168 (16d10 + 80)

\textbf{Move} 12m, fly 24m

\textbf{Saving Throws} Fortitude +10, Reflexes +8, Will +12

\textbf{Skills} Arcane +6, Hide / Stealth +4, Awareness +8, History +6

\textbf{Damage Immunity} cold

\textbf{Senses} darkvision 40m, blindsight 10m

\textbf{Languages} Common, Draconic

\textbf{Challenge} 9 (5000 XP)

\textbf{Actions}

\textit{\textbf{Multiattack.}} The dragon can make three attacks: one with its bite and two with its claws.

\textit{\textbf{Claw.} Melee Weapon Attack}: +15 to hit, reach 1m, one target.

\textit{Hit:} 13 (2d6 + 6) slashing damage, 1 bleed damage.

\textit{\textbf{Bite.} Melee Weapon Attack}: +15 to hit, reach 3m, one target.

\textit{Hit:} 17 (2d10 + 6) piercing damage.

\textit{\textbf{Breath Weapon (Cooldown 5-6).}} The dragon uses one of the following breath weapons:

\textit{Icy Breath.} The dragon exhales an icy blast in a 10m cone. Each creature in the area must make a DC 17 Fortitude save, taking 54 (12d8) cold damage on a failed save, or half as much damage on a successful one.

\textit{Crippling Breath.} The dragon exhales a paralyzing gas in a 10m cone. Each creature in the area must succeed on a Fortitude 17 save or be paralyzed for 1 minute. A creature can repeat the Saving Throw at the end of each of its rounds, ending the effect on itself on a success.

\textit{\textbf{Enraged}}: the Young Silver Dragon recharge one of his breath.

\textbf{Ecology}\\
Environment: Temperate Mountains\\
Organization: Solitary\\
\textbf{Treasure}: Triple\\
\textbf{Description}\\
Of all dragons, the silver dragons are the bravest, and abide by a code of chivalry that requires them to aid the weak, defeat evil, and behave honorably.\\
\textbf{Enchantments}\index{Silver Dragon Spell}\\
This dragon's favorite spells are:\\
- Slow\\
- Fabricate\\
- Dream

\

\index[Monsters]{Dragon, Young White}\textbf{Young White Dragon}

\textit{Large Dragon, Chaotic Evil}

\textbf{STRENGTH} +4

\textbf{DEXTERITY} +0

\textbf{CONSTITUTION} +4

\textbf{INTELLIGENCE} -2

\textbf{WISDOM} +0

\textbf{CHARISMA} +1

\textbf{Initiative} +0 -- \textbf{Defence} 20

\textbf{Hit Points} 133 (14d10 + 56)

\textbf{Move} 12m, swim 12m, dig 6m, fly 24m

\textbf{Saving Throws} Fortitude +8, Reflexes +7, Will +5

\textbf{Skills} Stealth +3, Awareness +6

\textbf{Damage Immunity} cold

\textbf{Senses} darkvision 40m, blindsight 9m

\textbf{Languages} Common, Draconic

\textbf{Challenge} 6 (2300 XP)

\textit{\textbf{Icewalking.}} The dragon can move and climb icy surfaces without needing to make ability checks. Also, hindering terrain composed of ice or snow costs him no additional movement.

\textbf{Actions}

\textit{\textbf{Multiattack.}} The dragon can use its Frightening Presence. Then make three attacks: one with the bite and two with the claws.

\textit{\textbf{Claw.} Melee Weapon Attack}: +6 to hit, reach 1m, one target.

\textit{Hit:} 11 (2d6 + 4) slashing damage, 1 bleed damage.

\textit{\textbf{Bite.} Melee Weapon Attack}: +6 to hit, reach 3m, one target.

\textit{Hit:} 15 (2d10 + 4) piercing damage plus 4 (1d8) cold damage.

\textit{\textbf{Icy Breath (Cooldown 4-6).}} The dragon exhales a blast of ice in a 10m cone. Each creature in that area must make a DC 15 Fortitude save, taking 45 (10d8) cold damage on a failed save, or half as much damage on a successful one.

\textbf{Ecology}\\
Environment: Cold Mountains\\
Organization: Solitary\\
\textbf{Treasure}: Triple\\
\textbf{Description}\\
Read Ancient White Dragon Description.\\
\textbf{Enchantments}\index{White Dragon Spell}\\
This dragon's favorite spells are:\\
- Fire Shield\\
- Ice Storm\\
- Sleet Storm


\

\index[Monsters]{Driade}\textbf{Driade}

\textit{Medium Fey, Neutral}

\textbf{STRENGTH} +0

\textbf{DEXTERITY} +1

\textbf{CONSTITUTION} +0

\textbf{INTELLIGENCE} +2

\textbf{WISDOM} +2

\textbf{CHARISMA} +4

\textbf{Initiative} +2 -- \textbf{Defence} 12 (17 with \textit{barkskin})

\textbf{Hit Points} 22 (5d8)

\textbf{Damage Vulnerability} cold iron

\textbf{Move} 9m

\textbf{Saving Throws} Fortitude +5, Reflexes +9, Will +7

\textbf{Skills} Stealth +5, Awareness +4

\textbf{Senses} Darkvision 18m

\textbf{Languages} Elven, Sylvan

\textbf{Challenge} 1 (200 XP)

\textit{\textbf{Tree Walk.}} Once during her round, the dryad can use 3 meter of movement to magically enter a living tree within her reach and emerge from another living tree within 20 meters of the first tree, reappearing in an unoccupied space within 1 meter of the second tree. Both trees must be Large or larger.

\textit{\textbf{Innate Spells.}} The dryad's innate spellcasting ability is Charisma (DC 14 for spell saves). The dryad can innately cast the following spells, requiring no material components. At will:

\textit{art of the druid}

3/day each: \textit{Good Berry}, \textit{entangle} 1/day:
\textit{pass without a trace, leathery skin, club} \textit{enchanted}

\textit{\textbf{Speak with animals and plants.}} The dryad can communicate with beasts and plants as if they speak the same language.

\textit{\textbf{Resistance to Magic.}} The dryad has +1d6 on Saving Throws against spells and other magical effects.

\textbf{Actions}

\textit{\textbf{Cudgeon.} Melee weapon attack}: +2 to hit (+6 to hit with \textit{staff}), reach 1m, one target.

\textit{Hit:} 2 (1d4) bludgeoning damage, or 8 (1d8 + 4) bludgeoning damage with \textit{staff}

\textit{\textbf{Fairy Charm.}} The dryad can target one humanoid or beast within 10 meters of her that she can see. If the target can see the dryad, it must succeed at a DC 14 Will save or be charmed by the magic. Charmed creatures consider the dryad a trusted friend to listen to and protect. While the target is not under the dryad's control, he will interpret the dryad's requests or actions as favorably as possible.

Whenever the dryad or her allies deal damage to the target, it can repeat the Saving Throw, ending the effect on a success. Otherwise, the effect lasts for 24 hours or until the dryad dies, she is on a different plane of existence than the target, or ends the effect as a bonus action. If the target's Saving Throw is successful, the target is immune to the dryad's fey charm for the next 24 hours.

The dryad can't charm more than one humanoid or three beasts at a time.

\textbf{Ecology}\\
Environment: Temperate Forests\\
Organization: Solitary, pair, or grove (3-8)\\
\textbf{Treasure}: standard (Masterwork Longbow w/20 Arrows, Dagger, other treasure)\\
\textbf{Description}\\
Dryads are tree-fey who enjoy secluded woods away from humanoids in need of timber. Dryads' primary concern is their own survival and that of their beloved forests, and they have been known to magically compel travelers to help them with tasks they cannot perform. They are friendly with non-evil druids and rangers, as they recognize their empathy or respect for nature.\\
Dryads are benevolent guardians of trees, and while not violent in nature, they can block and thwart threats to their homes or turn enemies into allies. Some keep one or more enthralled humanoids in their territory to defend it or to divert attackers. Enemies incapacitated are usually dragged to the edge of the forest by the dryad's allies and driven away, but evil or hostile ones are slain after combat is over.

\

\index[Monsters]{Drider}\textbf{Drider}

\textit{Large monstrosity, chaotic evil}

\textbf{STRENGTH} +3

\textbf{DEXTERITY} +3

\textbf{CONSTITUTION} +4

\textbf{INTELLIGENCE} +1

\textbf{WISDOM} +2

\textbf{CHARISMA} +1

\textbf{Initiative} +3 -- \textbf{Defence} 22

\textbf{Hit Points} 123 (13d10 + 52)

\textbf{Move} 9m, climb 9m

\textbf{Saving Throws} Fortitude +11, Reflexes +9, Will +9

\textbf{Skills} Stealth +9, Awareness +5

\textbf{Senses} darkvision 36m

\textbf{Languages} Elven, Deep Language

\textbf{Challenge} 6 (2300 XP)

\textit{\textbf{Walking the Web.}} The drider ignores movement restrictions caused by webs.

\textit{\textbf{Fey Ancestry.}} The drider has +1d6 on Saving Throws not to be charmed, and magic cannot put a drider to sleep.

\textit{\textbf{Innate Spells.}} The drider's innate spellcasting ability is Wisdom. The drider can innately cast the following spells, requiring no material components:

At will: \textit{dancing lights}

1/day: \textit{luminescence, darkness}

\textit{\textbf{Climb as Spider.}} The drider can climb difficult surfaces, including standing upside down on ceilings, without needing to make an ability check.

\textbf{Actions}

\textit{\textbf{Multiattack.}} The drider makes three attacks with the longsword or longbow. He can replace one of these attacks with a bite attack.

\textit{\textbf{Bite.} Melee Weapon Attack}: +11 to hit, reach 3 ft., one creature.

\textit{Hit:} 2 (1d4) piercing damage plus 9 (2d8) poison damage.

\textit{\textbf{Longsword.} Melee weapon attack}: +1 to hit, reach 1m, one target.

\textit{Hit:} 7 (1d8 + 3) slashing damage, or 8 (1d8 + 3) slashing damage when used with two hands.

\textit{\textbf{Longbow.} Ranged weapon attack}: +11 to hit, range 45m, one target.

\textit{Hit:} 7 (1d8 + 3) piercing damage plus 4 (1d8) poison damage.

\textit{\textbf{Enraged}}: the Drider collects poisonous saliva and spits it on his weapons. Until the end of the fight, Longsword's attack does 1d8 poison damage. Cost 1 Action.

\textbf{Ecology}\\
Environment any dungeon\\
Organization: Solitary, pair, or group (3-8)\\
\textbf{Treasure}: Double (Masterwork Heavy Mace, Masterwork Composite Longbow[+2 Strength] with 20 Arrows, other treasure)\\
\textbf{Description}\\
Created from the body of an elf, altered and mutated through special poisons and elixirs to take on the characteristics of a giant spider, the drider is a dangerous creature.\\
Driders are sexually dimorphic. The spidery lower portion of a female drider's body is sleek and graceful, often resembling a black widow's body, while the elfin upper torso retains her alluring curves and pretty face (with the exception of her poisonous spiked fangs). . A male drider's lower body is stocky like a tarantula, while its upper body is lean and supports a hideous, more spider-like than elfin face, complete with fanged mandibles.


\

\index[Monsters]{Duergar}\textbf{Duergar}

\textit{Medium humanoid (dwarf), lawful evil}

\textbf{STRENGTH} +2

\textbf{DEXTERITY} +0

\textbf{CONSTITUTION} +2

\textbf{INTELLIGENCE} +0

\textbf{WISDOM} +0

\textbf{CHARISMA} -1

\textbf{Initiative} +2 -- \textbf{Defence} 17 (scale Armour, shield)

\textbf{Hit Points} 26 (4d8 + 8)

\textbf{Move} 8m

\textbf{Saving Throws} Fortitude +4, Reflexes +0, Will +1

\textbf{Damage Resistance} poison

\textbf{Senses} darkvision 40m

\textbf{Languages} Dwarven, Language of the Deeps

\textbf{Challenge} 1 (200 XP)

\textit{\textbf{Duerga Resilience.}} The duergar has +1d6 on Saving Throws against poisons, spells, and illusions, as well as resisting being charmed or paralyzed.

\textit{\textbf{Sensitivity to Light}}. While in sunlight, the duergar has -1d6 on attack rolls, as well as Wisdom (Awareness) checks based on sight.

\textbf{Actions}

\textit{\textbf{Enlarge (Recharges after 1 hour).}} For 1 minute, the duergar magically increases in size, along with anything it is carrying or wearing. While enlarged, the duergar is Large in size, doubles the damage dice from attacks with Strength-based weapons (already included in the attacks), and has +1d6 on Strength checks and Strength Saving Throws. If the duergar doesn't have enough space to grow Large, it gains the maximum size allowed by the available space.

\textit{\textbf{War Pickaxe.} Melee Weapon Attack}: +4 to hit, reach 1m, one target.

\textit{Hit:} 6 (1d8 + 2) piercing damage, or 11 (2d8 + 2) piercing damage when enlarged.

\textit{\textbf{Javelin.} Melee or Ranged weapon attack}: +4 to hit, reach 1m or range 12m, one target. \textit{Hit:} 5 (1d6 + 2) piercing damage or 9 (2d6 + 2) damage piercing when enlarged.

\textit{\textbf{Invisibility (Recharges after 1 hour).}} The duergar becomes magically invisible for up to one hour (as if maintaining concentration for a spell) or until it attacks, casts a spell, use Enlarge or his concentration is broken. All equipment the duergar wears or carries becomes invisible with it.

\textbf{Ecology}\\
Environment any dungeon\\
Organization: Solitary, group (2-5), squad (6-12 plus 3 sergeants at 3rd level and 1 leader at 3rd-8th level), or clan (13-80 plus 25\% non-children combatants plus 1 3rd level sergeant for every 5 adults, 3-6 3rd-6th level lieutenants, and 1-4 9th level captains)\\
\textbf{Treasure}: NPC gear (chainmail, heavy metal shield, warhammer, light crossbow with 20 bolts, 3d6 gp, other treasure)\\
\textbf{Description}\\
Distant relatives of the Dwarves, darker and more misshapen, the Duergar are bad-tempered creatures who hate intruders in their subterranean realms, but never more than the Dwarves. They live in communities deep underground. Their skin is dull gray, as if it were dirty with dust or ash, but this natural shade allows them to blend better in underground areas. They are a race of slavers, but while forcing non-Dwarf captives to backbreaking labor, they mercilessly kill captured Dwarves. In combat, Duergar fire with the crossbow, and then switch to the warhammer a few rounds later. If outnumbered, or if there is adequate danger (and space), a Duergar will use its Enlarge ability and attack.


\textbf{Elementals}

The generic and somewhat mathematical formulas of the elementals are first presented, followed by examples.


\medskip\index[Monstruario]{Generic Water Elemental}\textbf{Generic Water Elemental}\\
\textit{CR/3 (Small, Medium, Large, Huge, Gargantuan, Colossal)}\\
\textbf{STRENGHT} +2+CR/3\\
\textbf{DEXTERITY} 0+CR/4\\
\textbf{CONSTITUTION} +2+CR/3\\
\textbf{INTELLIGENCE} -2+CR/6\\
\textbf{WISDOM} +0+CR/5\\
\textbf{CHARISMA} +0+CR/7\\
\textbf{Initiative} =DEX -- \textbf{Defense} 10+CR+DEX\\
\textbf{HitPoints} CR*CR*4\\
\textbf{Movement} 9 m, swim CR*4 m\\
\textbf{Saving Throws} Fortitude +CR+CON, Reflexes +CR+DEX, Will +CR+Wis\\
\textbf{Damage Resistances} acid; from a non-magical weapon\\
\textbf{Immunity to damage} Poison\\
\textbf{Immunity to Conditions} grabbed, poisoned, restrained, paralyzed, petrified, unconscious, prone, fatigue\\
\textbf{Senses} Darkvision 18m\\
\textbf{Languages} Aquan\\
\textbf{Challenge} CR\\
\textit{\textbf{Freezing.}} If the elemental takes cold damage, it freezes partially; its movement is reduced by 6 meters until the end of its next round.\\
\textit{\textbf{Water form.}} The elemental can enter a hostile creature's space and stop there. He can move through a space as narrow as 3 centimeters without having to squeeze.\\
\textit{\textbf{Elemental nature.}} An elemental does not need air, food, drink, or sleep.\\
\textbf{Actions}\\
\textit{\textbf{Multiattack.}} The elemental makes two slam attacks.\\
\textit{\textbf{Slam.} Melee weapon attack}: +CR+STR to hit, CR/3 meters, one target.\\
\textit{Hit:} CR*1d8 bludgeoning damage.\\
\textit{\textbf{Submerge (Recharge 4-6).}} Each creature in the elemental's space must make a DC CR+CR/2 Fortitude saving throw. On a failed save, the target takes (1d8+1)*CR/2 bludgeoning damage. If it is CR/3 size >=4, the target is also grappled (DC CR*2 to escape). Until the grab ends, the target is restrained and can't breathe unless able to breathe water. On a successful save, the target is pushed out of the elemental's space.\\
The elemental can grab a creature of size CR/3 or 2 of CR/2 or. At the start of each elemental's turn, each grappled target takes (1d6)*CR/2 bludgeoning damage. A creature within 10 feet of the elemental can pull a creature or object out of it, using an action to attempt a successful DC 2+CR*2 Strength check.\\


\medskip\index[Monstruario]{Generic Air Elemental}\textbf{Generic Air Elemental}

\textbf{STRENGHT} +0+CR/4\\
\textbf{DEXTERITY} +3+CR/4\\
\textbf{CONSTITUTION} +0+CR/5\\
\textbf{INTELLIGENCE} -2+CR/6\\
\textbf{WISDOM} -1+CR/5\\
\textbf{CHARISMA} +0+CR/5\\
\textbf{Initiative} =DEX -- \textbf{Defense} 10+CR+DEX\\
\textbf{HitPoints} CR*CR*2\\
\textbf{Movement} 0 m, fly CR*4 m\\
\textbf{Saving Throws} Fortitude +CR+CON, Reflexes +CR+DEX, Will +CR+Wis\\
\textbf{Damage Resistances} lightning, sound; from a non-magical weapon\\
\textbf{Immunity to damage} Poison\\
\textbf{Immunity to Conditions} grabbed, poisoned, restrained, paralyzed, petrified, unconscious, prone, fatigue\\
\textbf{Senses} Darkvision 18m\\
\textbf{Languages} Ictum\\
\textbf{Challenge} CR\\
\textit{\textbf{Form of air.}} The elemental can enter a hostile creature's space and stop there. He can move through a space as narrow as 3 centimeters without having to squeeze.\\
\textit{\textbf{Elemental nature.}} An elemental does not need air, food, drink, or sleep.\\
\textbf{Actions}\\
\textit{\textbf{Multiattack.}} The elemental makes two slam attacks.\\
\textit{\textbf{Slam.} Melee weapon attack}: +CR+STR to hit, CR/3 meters, one target.\\
\textit{Hit:} 1d6*CR/3 bludgeoning damage.\\
\textit{\textbf{Whirlwind (Recharge 4-6).}} Each creature in the elemental's space must make a DC CR*1.5 Fortitude saving throw. On a failed save, the target takes 1d8*CR/3 bludgeoning damage and is knocked CR feet away from the elemental in a random direction and is knocked prone. If a thrown target hits an object, such as a wall or floor, it takes 3 (1d6) bludgeoning damage for every 10 feet it was thrown. If the target is thrown at another creature, that creature must succeed on a DC 13 Reflex save or take the same damage and fall prone.
On a successful save, the target takes half the bludgeoning damage and is not knocked away or knocked prone.


\medskip\index[Monstruario]{Generic Fire Elemental}\textbf{Generic Fire Elemental}

\textbf{STRENGHT} +0+CR/4\\
\textbf{DEXTERITY} +2+CR/4\\
\textbf{CONSTITUTION} +1+CR/4\\
\textbf{INTELLIGENCE} -2+CR/6\\
\textbf{WISDOM} -1+CR/5\\
\textbf{CHARISMA} -2+CR/5\\
\textbf{Initiative} =DEX -- \textbf{Defense} 10+CR+DEX\\
\textbf{HitPoints} CR*CR*3\\
\textbf{Movement} 15 m\\
\textbf{Saving Throws} Fortitude +CR+CON, Reflexes +CR+DEX, Will +CR+Wis\\
\textbf{Damage Resistances} from non-magical weapon\\
\textbf{Immunity to Damage} Fire, Poison\\
\textbf{Immunity to Conditions} grabbed, poisoned, restrained, paralyzed, petrified, unconscious, prone, fatigue\\
\textbf{Senses} Darkvision 18m\\
\textbf{Languages} Ignan\\
\textbf{Challenge} CR\\
\textit{\textbf{Form of fire.}} The elemental can move through a space up to 1 inch wide without constricting. A creature that touches or strikes the elemental with a melee attack while within 3 feet of it takes 5 (1d10) fire damage. In addition, the elemental can enter a hostile creature's space and stop there. The first time it enters a creature's space in a turn, the creature takes CR fire damage and catches fire; until someone takes an action to put out the flames, the creature will take CR fire damage at the start of each of its rounds.\\
\textit{\textbf{Illumination.}} The elemental emits bright light in a radius of CR*2 meters and dim light for a further CR*2 meters.\\
\textit{\textbf{Elemental nature.}} An elemental does not need air, food, drink, or sleep.\\
\textit{\textbf{Susceptibility to water.}} The elemental takes 1 cold damage for every 1 meter it moves in water or for every 4 gallons of water splashed on it.
\textbf{Actions}\\
\textit{\textbf{Multiattack.}} The elemental makes two touch attacks.\\
\textit{\textbf{Slam.} Melee weapon attack}: +CR+STR to hit, CR/3 meters, one target.\\
\textit{Hit:} CR*2 fire damage. If the target is a flammable creature or object, it bursts into flame. Until a creature takes an action to put out the flames, the creature will take CR fire damage at the start of each of its rounds.


\medskip\index[Mostruario]{Generic Earth Elemental}\textbf{Generic Earth Elemental}

\textbf{STRENGHT} +CR\\
\textbf{DEXTERITY} -2+CR/5\\
\textbf{CONSTITUTION} +2+CR/3\\
\textbf{INTELLIGENCE} -3+CR/6\\
\textbf{WISDOM} -1+CR/5\\
\textbf{CHARISMA} -3+CR/6\\
\textbf{Initiative} =DEX -- \textbf{Defense} 10+CR+DEX\\
\textbf{HitPoints} CR*CR*6\\
\textbf{Movement} 9 m, digging 9 m\\
\textbf{Saving Throws} Fortitude +CR+CON, Reflexes +CR+DEX, Will +CR+Wis\\
\textbf{Damage Resistances} from non-magical weapon\\
\textbf{Immunity to damage} Poison\\
\textbf{Immunity to Conditions} grabbed, poisoned, paralyzed, petrified, unconscious, prone, fatigue\\
\textbf{Vulnerability to Damage} sound\\
\textbf{Senses} tremorsense 18m, darkvision 18m\\
\textbf{Languages} Tremun\\
\textbf{Challenge} CR\\
\textit{\textbf{Siege monster.}} The elemental deals double damage to objects and structures.\\
\textit{\textbf{Elemental nature.}} An elemental does not need air, food, drink, or sleep.\\
\textit{\textbf{Earth glide.}} The elemental can burrow through nonmagical, unwrought earth and stone. When he does, the elemental doesn't disturb the material he moves.\\
\textbf{Actions}\\
\textit{\textbf{Multiattack.}} The elemental makes two slam attacks.\\
\textit{\textbf{Slam.} Melee weapon attack}: +CR*2+STR to hit, CR/3 meters, one target.\\
\textit{Hit:} CR*3 bludgeoning damage.



\index[Monsters]{Earth Elemental}\textbf{Earth Elemental}

\textit{Large elemental, neutral}

\textbf{STRENGTH} +5

\textbf{DEXTERITY} -1

\textbf{CONSTITUTION} +5

\textbf{INTELLIGENCE} -3

\textbf{WISDOM} +0

\textbf{CHARISMA} -3

\textbf{Initiative} -1 -- \textbf{Defence} 20

\textbf{Hit Points} 126 (12d10 + 60)

\textbf{Movement} 9m, digging 9m

\textbf{Saving Throws} Fortitude +9, Reflexes +1, Will +6

\textbf{Vulnerability to Damage} sound

\textbf{Damage Resistances} from non-magical weapon

\textbf{Immunity to Damage} Poison

\textbf{Condition Immunity} poisoned, paralyzed, petrified, prone, unconscious, fatigued,

\textbf{Senses} tremorsense 18m, darkvision 18m

\textbf{Languages} Tremun

\textbf{Challenge} 5 (1800 XP)

\textit{\textbf{Siege monster.}} The elemental deals double damage to objects and structures.

\textit{\textbf{Elemental nature.}} An elemental has no need for air, food, drink, or sleep.

\textit{\textbf{Earth Glide.}} The elemental can burrow through nonmagical, unwrought earth and stone. When he does, the elemental doesn't disturb the material he moves.
\textbf{Actions}

\textit{\textbf{Multiattack.}} The elemental makes two slam attacks.

\textit{\textbf{Slam.} Melee Weapon Attack}: +12 to hit, reach 3m, one target.

\textit{Hit:} 14 (2d8 + 5) bludgeoning damage.

\textbf{Ecology}
Environment any (Plane of Earth)\\
Organization: Solitary, pair, or group (3-8)\\
\textbf{Treasure}: None\\
\textbf{Description}\\
Earth elementals are slow, stubborn creatures made of stone or earth. When they stand completely still they are indistinguishable from a pile of stones or a small hill.\\

When an earth elemental gets into heavy motion, his outward appearance may vary, although his statistics remain identical to those of his kin of the same size. Earth elementals are made mostly of rock, earth, or crystal, with glittering gems for eyes. The larger ones look like stone humanoids. Tufts of vegetation often grow on the soil that forms part of an earth elemental's body.\\

A large earth elemental stands 5 meters tall and weighs 3500kg.


\

\index[Monsters]{Elemental of Air}\textbf{Elemental of Air}

\textit{Large elemental, neutral}

\textbf{STRENGTH} +2

\textbf{DEXTERITY} +5

\textbf{CONSTITUTION} +2

\textbf{INTELLIGENCE} -2

\textbf{WISDOM} +0

\textbf{CHARISMA} -2

\textbf{Initiative} +5 -- \textbf{Defence} 18

\textbf{Hit Points} 90 (12d10 + 24)

\textbf{Move} 0m, fly 27m (float)

\textbf{Saving Throws} Fortitude +9, Reflexes +13, Will +2

\textbf{Damage Resistances} electricity, sound; from a non-magical weapon

\textbf{Immunity to Damage} Poison

\textbf{Condition Immunity} grabbed, poisoned, restrained, paralyzed, petrified, unconscious, prone, fatigue

\textbf{Senses} Darkvision 18m

\textbf{Languages} Ictum

\textbf{Challenge} 5 (1800 XP)

\textit{\textbf{Form of air.}} The elemental can enter a hostile creature's space and stop there. It can move through a space as narrow as 3 centimeters without having to squeeze.

\textit{\textbf{Elemental nature.}} An elemental has no need for air, food, drink, or sleep.

\textbf{Actions}

\textit{\textbf{Multiattack.}} The elemental makes two slam attacks.

\textit{\textbf{Slam.} Melee Weapon Attack}: +8 to hit, reach 1m, one target.

\textit{Hit:} 14 (2d8 + 5) bludgeoning damage.

\textit{\textbf{Whirlwind (Cooldown 4-6).}} Each creature in the elemental's space must make a DC 13 Fortitude save. On a failed save, the target takes 15 (3d8 + 2) damage from slam and is thrown 6 meters away from the elemental in a random direction and is knocked prone. If a thrown target hits an object, such as a wall or floor, it takes 3 (1d6) bludgeoning damage for every 3 meter it was thrown. If the target is thrown at another creature, that creature must succeed on a DC 13 Reflex save or take the same damage and fall prone.

On a successful save, the target takes half the bludgeoning damage and is not knocked away or knocked prone.

\textbf{Ecology}\\
Environment: Plane of Air\\
Organization: Solitary, pair, or group (3-8)\\
\textbf{Treasure}: None\\
\textbf{Description}\\
Air elementals are swift flying creatures made of air. Primitive and territorial, they dislike being summoned or controlled by mortals, and prefer to spend their time on the Plane of Air, flying through the endless sky.\\
While all air elementals of the same size have the same stats, the exact appearance of each varies greatly between individuals: one may appear as an animated swirl of wind and smoke, while another as a smoke-like creature. a bird with glittering eyes and windy wings.\\
An air elemental prefers to attack flying creatures, not only because of the advantages it gains from its mastery of air, but also because it loathes touching the ground. An air elemental can move underwater, and while it runs no risk of drowning, it has no ranks in Swim and loses most of its mobility and speed underwater.
An Aria Grande elemental stands 5 meters tall and weighs 2kg.

\

\textbf{Elephant}\index[Monsters]{Elephant}

\textit{Huge beast, unaligned}

\textbf{STRENGTH} +6

\textbf{DEXTERITY} -1

\textbf{CONSTITUTION} +3

\textbf{INTELLIGENCE} -4

\textbf{WISDOM} +0

\textbf{CHARISMA} -2

\textbf{Initiative} -1 -- \textbf{Defence} 14

\textbf{Hit Points} 76 (8d12 + 24)

\textbf{Move} 12m

\textbf{Saving Throws}: Fortitude +13, Reflexes +7, Will +6

\textbf{Languages} -

\textbf{Challenge} 4 (1000 XP)

\textit{\textbf{Rrumbling Charge.}} If the elephant moves at least 6 meters directly towards a creature and hits it with a butting attack during the same turn, the target must succeed at a DC Fortitude save 12 or fall prone. If the target is prone, the elephant can make a stomp attack against it as a bonus action.

\textbf{Actions}

\textit{\textbf{Gore.} Melee Weapon Attack}: +8 to hit, reach 1m, one target.

\textit{Hit:} 19 (3d8 + 6) piercing damage.

\textit{\textbf{Stamp.} Melee Weapon Attack}: +8 to hit, reach 1m, one prone target.

\textit{Hit:} 22 (3d10 + 6) bludgeoning damage.

\

\index[Monsters]{Erinyes}\textbf{Erinyes}

\textit{Medium fiend (devil), lawful evil}

\textbf{STRENGTH} +4

\textbf{DEXTERITY} +3

\textbf{CONSTITUTION} +4

\textbf{INTELLIGENCE} +2

\textbf{WISDOM} +2

\textbf{CHARISMA} +4

\textbf{Initiative} +3 -- \textbf{Defence} 24 (plate Armour)

\textbf{Hit Points} 153 (18d8 + 72)

\textbf{Move} 9m, fly 18m

\textbf{Saving Throws} Fortitude +16, Reflexes +15, Will +13

\textbf{Damage Resistances} cold; from a non-magical or non-silver weapon

\textbf{Damage Immunity} Fire, poison

\textbf{Condition Immunity} poisoned

\textbf{Senses} True Seeing 36 m

\textbf{Languages} Infernal, telepathy 36m

\textbf{Challenge} 12 (8400 XP)

\textit{\textbf{Evil Weapons.}} The erinyes' weapon attacks are magical and deal an additional 13 (3d8) poison damage on a hit (already included in the attacks).

\textit{\textbf{Resistance to Magic.}} The erinyes has +1d6 on Saving Throws against spells and other magical effects.

\textbf{Actions}

\textit{\textbf{Multiattack.}} The erinyes makes three attacks.

\textit{\textbf{Longsword.} Melee Weapon Attack}: +17 to hit, reach 1m, one target.

\textit{Hit:} 8 (1d8 + 4) slashing damage, or 9 (1d10 + 4) slashing damage when used with two hands, plus 13 (3d8) poison damage.

\textit{\textbf{Longbow.} Ranged weapon attack}: +17 to hit, range 45m, one target.

\textit{Hit:} 7 (1d8 + 4) piercing damage plus 13 (3d8) poison damage, and the target must succeed on a DC 14 Fortitude save or be poisoned, -1 Strenght and Dexterity,. The poison remains until removed by a \textit{lesser restoration} or similar spell.

\textbf{Reactions}

\textit{\textbf{Parry.}} The erinyes adds 4 to its Defence against a melee attack that would hit it. To do so, the erinyes must be able to see its attacker and wield a melee weapon.

\textit{\textbf{Enraged}}: The Erinyes channel their magical energy into an attack. The target of the attack is hit by an infernal flame that does 12d6 Void damage. Saving Throw DC 18 Reflexes to halve. Cost 2 Actions.


\textbf{Ecology}\\
Environment: Any (Hell)\\
Organization: Solitary or trio\\
\textbf{Treasure}: Triple (Fiery Composite Longbow+1[Strength +5], Rope, Longsword+1)\\
\textbf{Description}\\
Known by many names, the Fallen, the Ashen Wings and the Furies, the devils known as erinyes insult their angelic form with their lust for vengeance and bloody justice. Executioners, not judges, the erinyes hover above the razor-sharp ledges of Dis, the second cosmopolitan circle of Hell, always alert to seize every opportunity for battle, be it in Defence of hell, at the whim of their diabolical lords or for the impassioned call of capricious mortal summoners. All erinyes weave deadly living ropes from their own hair, which they use in battle to ensnare and lift their enemies into the air, taunting and condemning them for their transgressions before dropping them from great heights.

Erinyes are beautiful and dark angels who deliberately enhance their sensuality with scars and bruises. Yet despite their beauty, the erinyes are not seducers: they lack the subtlety and patience required for this fine emotional art, preferring to solve their problems with acts of swift and excruciating violence. Often an erinyes will hold back her killing blow as she attempts to slay an enemy, only to prolong their suffering. Death is generally the only way to escape the attentions of an erinyes, and the more powerful ones are very good at keeping their enemies alive but helpless, thus prolonging their torment, even going so far as to keep them alive with magic. The most powerful Erinyes torturers are said to have abilities that allow the suffering they inflict to last even after the subject's death. Most erinyes are just under 1.8 meters tall and weigh around 70 kg, and their black feathered wings have a span of more than 3 meters.


\

\index[Monsters]{Ettercap}\textbf{Ettercap}

\textit{Medium Monstrosity, Neutral Evil}

\textbf{STRENGTH} +2

\textbf{DEXTERITY} +2

\textbf{CONSTITUTION} +1

\textbf{INTELLIGENCE} -2

\textbf{WISDOM} +1

\textbf{CHARISMA} 8 (-2)

\textbf{Initiative} +2 -- \textbf{Defence} 14

\textbf{Hit Points} 44 (8d8 + 8)

\textbf{Move} 9m, climb 9m

\textbf{Saving Throws} Fortitude +6, Reflexes +4, Will +6

\textbf{Skills} Stealth +4, Awareness +3, Survival +3

\textbf{Senses} Darkvision 18m

\textbf{Languages} -

\textbf{Challenge} 2 (450 XP)

\textit{\textbf{Web Walk.}} The ettercap ignores movement restrictions caused by webs.

\textit{\textbf{Climb as Spider.}} The ettercap can climb difficult surfaces, including standing upside down on ceilings, without needing to make an ability check.

\textit{\textbf{Web Sense.}} While in contact with a web, the ettercap knows the exact location of any other creature in contact with the same web.

\textbf{Actions}

\textit{\textbf{Multiattack.}} The ettercap makes two attacks: one with its bite and one with its claws

\textit{\textbf{Claws.} Melee Weapon Attack}: +4 to hit, reach 1m, one target.

\textit{Hit:} 7 (2d4 + 2) slashing damage, 1 bleed damage.

\textit{\textbf{Bite.} Melee Weapon Attack}: +4 to hit, reach 1m, one target.

\textit{Hit:} 6 (1d8 + 2) piercing damage plus 4 (1d8) poison damage. The target must succeed on a DC 11 Fortitude save or be poisoned , -1 Strenght and Dexterity, for 1 minute. The creature can repeat the Saving Throw at the end of each of its rounds, ending the effect on a successful save.

\textit{\textbf{Web (Cooldown 5-6).} Ranged Weapon Attack}: +4 to hit, reach 10m, Large or smaller creature.

\textit{Hit:} The creature is entangled in the web. As an action, the entangled creature can make a DC 11 Strength check, breaking free from the web on a successful one. The effect ends if the canvas is destroyed. The web has Defence 10, 5 Hit Points, vulnerability to fire damage, and immunity to bludgeoning and poison damage.

\textbf{Ecology}\\
Environment: Temperate Forests\\
Organization: solitary, pair or nest (3-6 plus 2-8 giant spiders)\\
\textbf{Treasure}: Standard\\
\textbf{Description}\\
Ettercaps are usually 1.8 meters tall and weigh around 100 kg. They are solitary and rarely associate with others of their race, except for mating. When grouped, they tend to attract various species of spiders, forming an odd mix of ettercaps and arachnids.\\
Ettercaps are known for building cunning traps out of cobwebs and other natural materials, which they use to capture prey. They build cobweb shelters in the topmost branches of trees away from other terrestrial predators, and use monstrous spiders as lookouts and guardians.\\
Ettercaps aren't brave, but their traps often prevent the enemy from drawing their weapons. An ettercap attacks with poisonous claws and bites. It generally avoids melee with opponents who can still move and flees if they break free.


\

\index[Monsters]{Ettin}\textbf{Ettin}

\textit{Large giant, chaotic evil}

\textbf{STRENGTH} +5

\textbf{DEXTERITY} -1

\textbf{CONSTITUTION} +3

\textbf{INTELLIGENCE} -2

\textbf{WISDOM} +0

\textbf{CHARISMA} -1

\textbf{Initiative} -1 -- \textbf{Defence} 14

\textbf{Hit Points} 85 (10d10 + 30)

\textbf{Move} 12m

\textbf{Saving Throws} Fortitude +9, Reflexes +2, Will +5

\textbf{Skills} Awareness +4

\textbf{Languages} Giant, Goblinoid

\textbf{Challenge} 4 (1100 XP)

\textit{\textbf{Two Heads.}} The ettin has +1d6 on Wisdom (Awareness) checks and on Saving Throws against conditions blinded, charmed, deafened, unconscious, frightened, and stunned.

\textit{\textbf{Awake.}} When one of the ettin's two heads is asleep, the other is awake.

\textbf{Actions}

\textit{\textbf{Multiattack.}} The ettin makes two attacks: one with the battleaxe and one with the spiked mace.

\textit{\textbf{Battle Axe.} Melee Weapon Attack}: +11 to hit, reach 1m, one target.

\textit{Hit:} 14 (2d8 + 5) slashing damage.

\textit{\textbf{Shocked Mace.} Melee Weapon Attack}: +11 to hit, reach 1m, one target.

\textit{Hit:} 14 (2d8 + 5) piercing damage.

\textbf{Ecology}\\
Environment: Cold Hills\\
Organization: Solitary, pair, gang (3-6), troop (1-2 plus 1-2 Brown Bears, gang (3-6 plus 1-2 Brown Bears), or colony (3-6 plus 1-2 Brown Bears and 7-12 Orcs, or 9-16 Goblins)\\
\textbf{Treasure}: Standard (Leather Armour, 2 Light Flails, 4 Javelins, other treasure)\\
\textbf{Description}\\
Ettins, or two-headed giants, are malevolent and unpredictable nocturnal hunters. The two heads grant him unparalleled powers of perception, making them excellent guardians.\\
Ettins look like hill giants or stone giants, but their fanged faces betray orc ancestry. They have pinkish brown skin and never wash unless they have to, which makes them so dirty and grimy that their skin looks thick and gray.\\
Adults are 3.9 meters tall and weigh 2,600 kg. Ettins live about 75 years.\\
Ettins have no language of their own but speak a mix of Giant, Goblin, and Orc slang. Creatures that speak any of these languages can communicate with an ettin by making a DC 15 Intelligence check. The check is made once for each piece of information; if the other creature speaks two of these languages the DC is 10, while for someone who speaks all three it is 5.\\
While ettins aren't very intelligent, they are cunning warriors. They prefer to ambush their victims rather than engage them in combat, but once battle has begun, an ettin fights furiously until the enemy is killed.\\
Ettins are solitary creatures, making their home in the safety of rock caves and hollows, often surrounded by holes and ditches, and they sometimes keep cave bears as pets or guardians.\\
A particularly powerful ettin can attract a group of followers, especially goblins or orcs. However, these gatherings are more of an exception, and rarely last long, with individualistic ettins going their separate ways as soon as opportunities for looting and robbery diminish or if the leader is killed.\\
Usually they form reproductive pairs to rear their offspring only for short periods before each going their own way. Young ettins mature rapidly, reaching adult size in a year, thus being able to fend for themselves.

\

\index[Monsters]{Explosive Cockroach}\textbf{Explosive Cockroach}

\textit{Little Elemental, neutral}

\textbf{STRENGTH} +1

\textbf{DEXTERITY} +2

\textbf{CONSTITUTION} +1

\textbf{INTELLIGENCE} -5

\textbf{WISDOM} -1

\textbf{CHARISMA} -2

\textbf{Initiative} +2 -- \textbf{Defence} 14

\textbf{Hit Points} 45 (8d8 + 9)

\textbf{Move} 4m, jump 9m, dig 2m

\textbf{Saving Throws} Fortitude +5, Reflexes +6, Will +3

\textbf{Damage Resistances} bludgeoning

\textbf{Damage Immunity} Fire

\textbf{Condition Immunity} fatigued, scared

\textbf{Senses} blindsight 5 m

\textbf{Languages} -

\textbf{Challenge} 2 (450 XP)

\textit{detect fire}: The Explosive Cockroach can sense fires within 100 meters of distance, as long as they are equal to or greater than a torch

\textit{Dug}: The explosive cockroach can burrow into solid ground through its move.

\textbf{Actions}

\textit{\textbf{Multiattack.}} The Explosive Cockroach can make 1 charge attack or emit a fiery goo.

\textit{\textbf{Charge.}} Melee Attack: +6 to hit, reach 1 meter, one target.

\textit{Hit:} 12 (3d6 + 3) bludgeoning damage. The creature must make a DC 11 Fortitude save or fall prone.

\textit{\textbf{Fire Mash}} Ranged Attack: +7 on hit, reach 3 meter. The Blast Roach regurgitates a sticky, flammable liquid into the air. Reload 1/3-6.

\textit{Hit:} 18 (4d6 + 6) fire damage. DC 13 Reflex save to halve.

\textit{\textbf{Death:}} When the Explosive Cockroach dies, the jelly inside in contact with the air explodes all around, within a radius of 1 meter around the cockroach, the flames cause 12 (4d6) of damage, DC 15 Reflex save to halve.

\textbf{Ecology}\\
Environment: Hot Caves\\
Organization: Solitary, Nest (8-64)\\
\textbf{Treasure}: Diamond 1d4x1d50gp\\
\textbf{Description}\\
Explosive Roaches are creatures native to the elemental planes of fire and earth. Usually attracted by environments full of flames, stone or at least heat and earth.
With a shape proportionate to those of a common cockroach if not about 40 cm long and weighing about 4 kg, it is a creature completely devoid of intellect acting only by pure instinct.
They are now common in caves near volcanoes or red dragon lairs having become accustomed to living on Yeru.

In the nest where they live there is at least one queen who commands the cockroaches, much bigger and stronger. Explosive Cockroaches feed on coal, burnt wood, burnt carcasses. They are extremely greedy for diamonds which, once burned, are real delicatessen.

\

\textbf{Falcon}\index[Monsters]{Falcon}

\textit{Tiny beast, unaligned}

\textbf{STRENGTH} -3

\textbf{DEXTERITY} +3

\textbf{CONSTITUTION} -1

\textbf{INTELLIGENCE} -4

\textbf{WISDOM} +2

\textbf{CHARISMA} -2

\textbf{Initiative} +3 -- \textbf{Defence} 14

\textbf{Hit Points} 1 (1d4 - 1)

\textbf{Move} 3m, fly 18m

\textbf{Saving Throws}: Fortitude +2, Reflexes +5, Will +2

\textbf{Skills} Awareness +4

\textbf{Languages} -

\textbf{Challenge} 0 (10 XP)

\textit{\textbf{Enhanced Sight.}} The falcon has +1d6 on Wisdom (Awareness) checks based on sight.

\textbf{Actions}

\textit{\textbf{Spurs.} Melee Weapon Attack}: +5 to hit, reach 1m, one target.

\textit{Hit:} 1 slashing damage.

\

\textbf{Fase Dog}\index[Monsters]{Fase Dog}

The fase dog gets its name from its ability to slip in and out of reality, a talent it uses to attack and avoid being attacked.

\textit{Medium Fey, Lawful Good}

\textbf{STRENGTH} +1

\textbf{DEXTERITY} +3

\textbf{CONSTITUTION} +1

\textbf{INTELLIGENCE} +0

\textbf{WISDOM} +1

\textbf{CHARISMA} +0

\textbf{Initiative} +3 -- \textbf{Defence} 14

\textbf{Hit Points} 22 (4d8 + 4)

\textbf{Damage Vulnerability} cold iron

\textbf{Move} 12m

\textbf{Saving Throws}: Fortitude +5, Reflexes +5, Will +4

\textbf{Skills} Stealth +5, Awareness +3

\textbf{Languages} Fase Dog, understands Sylvan but cannot speak it

\textbf{Challenge} 1/4 (50 XP)

\textit{\textbf{Hearing and Fine Smell.}} The dog has +1d6 on Wisdom (Awareness) checks based on hearing or smell.

\textbf{Actions}

\textit{\textbf{Bite.} Melee Weapon Attack}: +3 to hit, reach 1m, one target.

\textit{Hit:} 4 (1d6 + 1) piercing damage.

\textit{\textbf{Teleport (Cooldown 4-6).}} The dog magically teleports, along with whatever he's wearing or carrying, up to 13 meters in an unoccupied space that he can see. Before or after teleporting, the dog can make a bite attack.

\

\index[Monsters]{Fire Elemental}\textbf{Fire Elemental}

\textit{Large elemental, neutral}

\textbf{STRENGTH} +0

\textbf{DEXTERITY} +3

\textbf{CONSTITUTION} +3

\textbf{INTELLIGENCE} -2

\textbf{WISDOM} +0

\textbf{CHARISMA} -2

\textbf{Initiative} +3 -- \textbf{Defence} 16

\textbf{Hit Points} 102 (12d10 + 36)

\textbf{Move} 15m

\textbf{Saving Throws} Fortitude +8, Reflexes +11, Will +4

\textbf{Damage Resistances} from non-magical weapon

\textbf{Damage Immunity} Fire, poison

\textbf{Condition Immunity} grabbed, poisoned, restrained, paralyzed, petrified, prone, unconscious, fatigue

\textbf{Senses} Darkvision 18m

\textbf{Languages} Ignan

\textbf{Challenge} 5 (1800 XP)

\textit{\textbf{Form of fire.}} The elemental can move through a space up to 1 cm wide without constricting. A creature that touches or strikes the elemental with a melee attack while within 1 meter of it takes 5 (1d10) fire damage. In addition, the elemental can enter a hostile creature's space and stop there. The first time it enters a creature's space in a turn, the creature takes 5 (1d10) fire damage and bursts into flame; until someone takes an action to put out the flames, the creature takes 5 (1d10) fire damage at the start of each of its rounds.

\textit{\textbf{Illumination.}} The elemental sheds bright light in a 10m radius and dim light for an additional 10 meters.

\textit{\textbf{Elemental nature.}} An elemental has no need for air, food, drink, or sleep.

\textit{\textbf{Susceptibility to water.}} The elemental takes 1 cold damage for every 1 meter it moves in water or for every 4 gallons of water splashed on it.

\textbf{Actions}

\textit{\textbf{Multiattack.}} The elemental makes two touch attacks.

\textit{\textbf{Touch.} Melee Weapon Attack}: +7 to hit, reach 1m, one target.

\textit{Hit:} 10 (2d8 + 5) fire damage. If the target is a flammable creature or object, it bursts into flame. Until a creature takes an action to put out the flames, the creature takes 5 (1d10) fire damage at the start of each of its rounds.

\textbf{Ecology}
Environment any (Plane of Fire)\\
Organization: Solitary, pair, or group (3-8)\\
\textbf{Treasure}: None\\
\textbf{Description}\\
Fire elementals are swift and cruel creatures made of living flames. They enjoy frightening those weaker than themselves, and terrorize any creature they set ablaze. A fire elemental cannot enter water or any nonflammable liquid. A body of water is an impenetrable barrier unless the elemental can vault or leap over it, or is covered with flammable material (such as a film of oil).\\
Fire elementals vary in appearance; they typically manifest in the form of serpentine coils made of smoke and flame, but some fire elementals take on more human-like, demon-like, or other monster-like appearances to heighten terror when they suddenly appear. A fire elemental's body appears to be made of flames or puffs of semi-stable sparks, smoke, or ash.\\

A large fire elemental stands 5 meters tall.

\

\index[Monsters]{Fleshworms}\textbf{Fleshworms}

\textit{tiny monstrosity, unaligned}

\textbf{STRENGTH} -4

\textbf{DEXTERITY} +0

\textbf{CONSTITUTION} -2

\textbf{INTELLIGENCE} -4

\textbf{WISDOM} 0

\textbf{CHARISMA} -4

\textbf{Initiative} +0 -- \textbf{Defence} 11

\textbf{Hit Points} 1 (1d6 -2)

\textbf{Move} 1m

\textbf{Saving Throws}: Fortitude -1, Reflexes +0, Will -4

\textbf{Senses} telluric view 3 m

\textbf{Languages} -

\textbf{Challenge} 1 (200 XP)

\textbf{Actions}

\textit{\textbf{Infest flesh.}} These tiny creatures burrow into exposed flesh without making an attack roll as long as the flesh is exposed to contact with them.

\textit{\textbf{Hit.}} Within 2d4 rounds the worms (3d6 creatures) of the flesh burrow into the tissue making their way toward the heart. The worm infestation causes 1 hit point of damage per round as they burrow. Once you get to the heart, each round the character must make a DC 14 Fortitude save, with a cumulative penalty of -1 per round. Once the Saving Throw fails, the character dies.

\textit{\textbf{Kill off the Flesh Worms.}} The only way is to use an open flame (a torch causes 1d6 damage on application or a spell like Burning Wave) on the part where the worms are burrowing. Each application of fire can kill 3d6 worms. A First Aid check ad DC 15 will remove 1d4 worms but will causa 1d4 damage. After 2d4 rounds the worms are too deep and it is useless to apply fire, only a cure disease or heal spell can completely eradicate the infestation.

\textbf{Ecology}\\
Environment: rotten trees, rotting flesh\\
Organization: Groups 3d6\\
\textbf{Treasure}: None\\
\textbf{Description}\\

Fleshworms are among the most feared parasites by adventurers. They are found in damp heaps of leaves or rotten trunks, in rotting corpses, in murky waters. Pale, slimy, equipped with very sharp teeth, just over 4 millimeters long, they penetrate the exposed flesh in a very easy way and perceive the heartbeat where they are directed. While they dig in the flesh they can be perceived and even seen crawling under the skin.


\

\textbf{Flying Serpent}\index[Monsters]{Flying Serpent}

A flying serpent is a richly colored, winged serpent found in remote jungles.

\textit{Tiny beast, unaligned}

\textbf{STRENGTH} -3

\textbf{DEXTERITY} +4

\textbf{CONSTITUTION} +0

\textbf{INTELLIGENCE} -4

\textbf{WISDOM} +1

\textbf{CHARISMA} -3

\textbf{Initiative} +4 -- \textbf{Defence} 15

\textbf{Hit Points} 5 (2d4)

\textbf{Move} 9m, swim 9m, fly 18m

\textbf{Saving Throws}: Fortitude -2, Reflexes +5, Will +1

\textbf{Senses} blindsight 3 m

\textbf{Languages} -

\textbf{Challenge} 1/8 (25 XP)

\textit{\textbf{Flying.}} The snake does not provoke attacks of opportunity when it flies out of an enemy's reach.

\textbf{Actions}

\textit{\textbf{Bite.} Melee Weapon Attack}: +6 to hit, reach 1m, one target.

\textit{Hit:} 1 piercing damage plus 7 (3d4) poison damage.

\

\textbf{Frog}\index[Monsters]{Frog}

\textit{Tiny beast, unaligned}

\textbf{STRENGTH} -5

\textbf{DEXTERITY} +1

\textbf{CONSTITUTION} -1

\textbf{INTELLIGENCE} -5

\textbf{WISDOM} -1

\textbf{CHARISMA} -4

\textbf{Initiative} +1 -- \textbf{Defence} 12

\textbf{Hit Points} 1 (1d4 - 1)

\textbf{Movement} 6m, swim 6m

\textbf{Saving Throws}: Fortitude -4, Reflexes +1, Will -2

\textbf{Skills} Stealth +3, Awareness +1

\textbf{Senses} vision in the dark 9m

\textbf{Languages} -

\textbf{Challenge} 0 (0 XP)

\textit{\textbf{Amphibian.}} The frog can breathe air and water.

\textit{\textbf{Standing Jump.}} A frog can jump up to 3 meters long and up to 1 meter high, with or without a running start.

A \textbf{frog} has no attacks. It feeds on small insects and usually lives near marshes, inside trees or underground.

\

\index[Monsters]{G.E.C.}\textbf{G.E.C.}

\textit{Large aberration, chaotic evil}

\textbf{STRENGTH} +6

\textbf{DEXTERITY} +1

\textbf{CONSTITUTION} +5

\textbf{INTELLIGENCE} +1

\textbf{WISDOM} +1

\textbf{CHARISMA} +1

\textbf{Initiative} +2 -- \textbf{Defence} 20 (chitin)

\textbf{Hit Points} 95 (12d8 + 50)

\textbf{Move} 9m, dig 9m

\textbf{Saving Throws}: Fortitude +15, Reflexes +11, Will +12

\textbf{Resistance} +4 on Saving Throws to spells from the charm and illusion list

\textbf{Skills} Awareness +10

\textbf{Senses} Darkvision 18m, telluric sense 18m

\textbf{Languages} -

\textbf{Challenge} 10 (5900 XP)

\textbf{Actions}

\textit{\textbf{Multiattack.}} The G.E.C. it can attack with two claws or with its bite

\textbf{Claws}: Melee Natural Weapon Attack: +21 to hit, reach 3m, one target.

\textit{Hit:} 15 (3d6 + 5) slashing damage, 1 bleed damage.

\textbf{Bite}: Melee Natural Weapon Attack: +21 on hit, reach 3m, one target

\textit{Hit:} 16 (3d8 + 5) slashing damage, 1 bleed damage, Blurred Vision

\textbf{Blurred vision:} is a poison effect, Will save DC 18 or until the end of the following round the target has -1d6 on the attack roll.

\textit{\textbf{Look.}} It is enough to look at the G.E.C. to be affected by Confusion, as a spell of the same name. Resisting requires a DC 18 Will save. Each round, you can repeat the save to resist the effect.

Fighting without looking at the G.E.C. imposes -1d6 on attack rolls.

\textit{\textbf{Enraged}}: the G.E.C. emits a cacophonous roar. Creatures within 10 feet of him must make a DC 20 Will save or be afflicted with Confusion for 2 rounds. Cost 2 Action

\textbf{Ecology}\\
Environment: Underground\\
Organization: solitary, group (2-4) \\
\textbf{Treasure}: Accidental\\
\textbf{Description}\\
The Great Chitinous Being, or G.E.C, is an insect with a vague humanoid appearance almost 4 meters tall, powerful and equipped with two very strong and resistant claws capable of digging and shearing any material. 4 small, central and multi-faceted eyes give off a faint iridescent luminescence that confuse the creatures that meet their gaze.

Probably the result of some transformation spell gone bad the G.E.C. they are masters of the subsoil. Creatures of real intelligence love elfin flesh and fight tactically and shrewdly.

\

\index[Monsters]{Gablin, Gablin}\textbf{Gablin}

\textit{Little fiend (goblinoid), chaotic evil}

\textbf{STRENGTH} +2

\textbf{DEXTERITY} +1

\textbf{CONSTITUTION} +1

\textbf{INTELLIGENCE} -2

\textbf{WISDOM} -1

\textbf{CHARISMA} -2

\textbf{Initiative} +1 -- \textbf{Defence} 14

\textbf{Hit Points} 6 (2d4 + 2)

\textbf{Move} 9m

\textbf{Saving Throws}: Fortitude +4, Reflexes +2, Will +0

\textbf{Damage Resistance}: Void

\textbf{Senses} Darkvision 18m

\textbf{Languages} understand Common but don't speak it, Abyssal

\textbf{Challenge} 1/2 (100 XP)

\textit{\textbf{Sensitivity to Light}}. While in sunlight, the gablin has -1d6 on attack rolls, as well as Wisdom (Awareness) checks based on sight.

\textbf{Actions}

\textit{\textbf{Short sword.} Melee weapon attack}: +2 to hit, reach 1m, one target.

\textit{Hit:} 5 (1d6 + 2) slashing damage.

\textit{\textbf{Bite.} Melee Weapon Attack}: +3 to hit, touch, one target.

\textit{Hit:} 2 (1d1 + 1) piercing damage.

\textbf{Ecology}\\
Environment: Anywhere\\
Organization: Group (8-12), warband (10-24), or tribe (50+, 1 sergeant 3rd level per 20 adults, 1 or 2 lieutenants 4th or 5th level, 1 leader 6th -8th level, 6-12 wild wolves and 1-4 Ogres or 1-2 Gablin Champion)\\
\textbf{Treasure}: Occasional\\
\textbf{Description}\\
The Gablin are the scum of the scum, it is said that a Gablin is born with every bad thought and surely there are so many.
The Gablin are small humanoids with dark skin, with green streaks initially generated by the will of Cattalm with the sole purpose of bringing destruction, death and suffering.
Gablins can hide anywhere as long as they are near a food source, usually preferring sewers or abandoned structures near villages.
A Gablin's sole purpose is to kill and perpetuate the species. Gablin are all male, and their fiendish nature makes them capable of impregnating any female humanoid.
Usually the gestation lasts only 3 weeks during which the women are tortured to strengthen the 1d6+2 babies she carries. The birth usually ends with Gablin's young disembowelling their mother and making it their first meal.
This method of procreation combined with their ravenous voracity for blood and flesh make them among the most hated and feared creatures.
Even if individually they are not particularly fearsome, the Gablin always move in groups and if this exceeds two dozen then there is almost always a Gablin Enchanter or even a Gablin Champion to lead them.


\

\index[Monsters]{Gablin, Champion}\textbf{Gablin Champion}

\textit{Large fiend, chaotic evil}

\textbf{STRENGTH} +4

\textbf{DEXTERITY} +2

\textbf{CONSTITUTION} +3

\textbf{INTELLIGENCE} +1

\textbf{WISDOM} +0

\textbf{CHARISMA} -1

\textbf{Initiative} +2 -- \textbf{Defence} 18

\textbf{Hit Points} 60 (7d10 + 25)

\textbf{Move} 12m

\textbf{Saving Throws}: Fortitude +9, Reflexes +6, Will +3

\textbf{Damage Resistance}: Void

\textbf{Senses} Darkvision 18m

\textbf{Languages} Common, Abyssal

\textbf{Challenge} 3 (700 XP)

\textbf{Actions}

\textit{\textbf{Club.} Melee Weapon Attack}: +7 to hit, reach 2m, one target.

\textit{Hit:} 11 (2d6 + 4) bludgeoning damage.

\textit{\textbf{Summon Gablin}}: 3 Actions. The Gablin spills its blood on the ground and 2d4 Gablins arise, loses 1 Hit Point

\textbf{Ecology}\\
Environment: Any\\
Organization: lead a group of Gablin\\
\textbf{Treasure}: Standard (Leather Armour, Greatclub)\\
\textbf{Description}\\
Gablin Champions come spontaneously when the number of Gablins present reaches 20. Hugely bigger, stronger and more intelligent than a Gablin, the Champions are the leaders of the group, those who plan battles and clashes.
They have no qualms about slaughtering the Gablin or killing anything that breathes. Imbued with the spirit of Cattalm their purpose is always and only to destroy and kill.

\

\index[Monsters]{Gablin, Paladin}\textbf{Paladin Gablin}

\textit{Large fiend, chaotic evil}

\textbf{STRENGTH} +5

\textbf{DEXTERITY} +2

\textbf{CONSTITUTION} +3

\textbf{INTELLIGENCE} +2

\textbf{WISDOM} +3

\textbf{CHARISMA} +3

\textbf{Initiative} +4 -- \textbf{Defence} 21

\textbf{Hit Points} 105 (10d10 + 50)

\textbf{Move} 12m

\textbf{Saving Throws}: Fortitude +12, Reflexes +11, Will +12

\textbf{Damage Resistance}: Void

\textbf{Senses} Darkvision 18m

\textbf{Languages} Common, Abyssal

\textbf{Challenge} 6 (2300 XP)

\textbf{Actions}

\textit{\textbf{Multiattack.}} Paladin Gablin attacks with 2 bastard sword strikes.

\textit{\textbf{Bastard sword.} Melee weapon attack}: +13 to hit, reach 2m, one target.

\textit{Hit:} 10 (1d10 + 5) bludgeoning damage, plus 1d6 void damage. If the affected creature is a Gradh follower or devotee, the damage increases by an additional 1d6.

\textit{\textbf{Summon Gablin}}: 3 Actions. The Gablin spills its blood on the ground and 3d4 Gablins arise to this.

\textbf{Fiendish Aura}: The Gablin Paladin exudes a 6 meters-radius aura around him that grants +2 on Attack and Damage rolls to all other Gablins, and imposes -2 on Attack Roll and Saving Throw to other non- Devotees or Followers of Cattalm.

\textbf{Ecology}\\
Environment: Any\\
Organization: Lead a Gablin army\\
\textbf{Treasure}: Standard (Field Armour, Bastard Sword)\\
\textbf{Description}\\
The Gablin Paladins are among the most powerful Gablin known, the truly chosen ones of Cattalm. Summoned by more powerful followers of Cattalm they can single-handedly lead hundreds of Gablins and with their wits prepare careful plans and wreak havoc and destruction across entire regions.


\

\index[Monsters]{Gargoyle}\textbf{Gargoyle}

\textit{Elemental Mean, Chaotic Evil}

\textbf{STRENGTH} +2

\textbf{DEXTERITY} +0

\textbf{CONSTITUTION} +3

\textbf{INTELLIGENCE} -2

\textbf{WISDOM} +0

\textbf{CHARISMA} -2

\textbf{Initiative} +0 -- \textbf{Defence} 16

\textbf{Hit Points} 52 (7d8 + 21)

\textbf{Move} 9m, fly 18m

\textbf{Saving Throws}: Fortitude +4, Reflexes +6, Will +4

\textbf{Damage Resistances} from non-magical or non-adamantine weapons

\textbf{Immunity to Damage} Poison

\textbf{Condition Immunity} poisoned, petrified, fatigued

\textbf{Senses} Darkvision 18m

\textbf{Languages} Tremun

\textbf{Challenge} 2 (450 XP)

\textit{\textbf{False Appearance.}} While the gargoyle remains motionless, it is indistinguishable from an inanimate statue.

\textit{\textbf{Elemental nature.}} A gargoyle needs no air, food, drink, or sleep.

\textbf{Actions}

\textit{\textbf{Multiattack.}} The gargoyle makes two attacks: one with its bite and one with its claws.

\textit{\textbf{Claws.} Melee Weapon Attack}: +5 to hit, reach 1m, one target.

\textit{Hit:} 5 (1d6 + 2) slashing damage, 1 bleed damage.

\textit{\textbf{Bite.} Melee Weapon Attack}: +5 to hit, reach 1m, one target.

\textit{Hit:} 5 (1d6 + 2) piercing damage.

\textbf{Ecology}
Environment: Any\\
Organization: Solitary, pair, or flock (3-12)\\
\textbf{Treasure}: Standard\\
\textbf{Description}\\
Gargoyles often appear to be winged statues of stone, as they can remain motionless indefinitely and then surprise enemies. Gargoyles are prone to obsessive-compulsive behaviors, as diverse as their species is abundant. Stolen books, trinkets, weapons, and trophies gleaned from fallen enemies are just a few examples of the types of items a gargoyle can collect to decorate its lair and territory.

Gargoyles tend to lead a solitary lifestyle, though they sometimes form fearsome flocks called "wings" for protection and amusement. Under certain conditions, a tribe of gargoyles can even ally with other creatures, but even the most stable of such alliances can fall apart for the tiniest of reasons; gargoyles are just treacherous, petty, and vengeful.

Gargoyles are known to inhabit the heart of larger cities, hunkered down among the stone decorations of cathedrals and buildings where they hide in plain sight by day, swooping down to feed on vagrants, beggars and other unfortunates by night.

The longer a gargoyle tribe dwells in an area of buildings or ruins, the more its members begin to resemble the architectural style of the area. The changes a gargoyle's appearance undergoes are slow and subtle, but over the years they can become radical.

An unusual variant of the gargoyle does not live among buildings and ruins but under the waves of the sea. These creatures are known as kapoacinths; they have the same base statistics as normal gargoyles, except they have the aquatic subtype and their wings grant them a swim speed of 13 meters (but are useless for flying). Kapoacinths inhabit shallow coastal regions where they can crawl out of the surf to prey on area residents. They are more likely to form flocks, as kapoacinths prefer group life to solitary life.

\

\index[Monsters]{Genes, Djinni}\textbf{Djinni}

\textit{Large elemental, chaotic good}

\textbf{STRENGTH} +5

\textbf{DEXTERITY} +2

\textbf{CONSTITUTION} +6

\textbf{INTELLIGENCE} +2

\textbf{WISDOM} +3

\textbf{CHARISMA} +5

\textbf{Initiative} +2 -- \textbf{Defence} 23

\textbf{Hit Points} 161 (14d10 + 84)

\textbf{Move} 9m, fly 27m

\textbf{Saving Throws} Fortitude +17, Reflexes +13, Will +14

\textbf{Damage Immunity} electricity, sound

\textbf{Senses} darkvision 40m

\textbf{Languages} Ictum

\textbf{Challenge} 11 (7200 XP)

\textit{\textbf{Elemental Death.}} If the djinni dies, his body disintegrates in a warm breeze, leaving behind only the equipment the djinni was wearing or carrying.

\textit{\textbf{Innate Spells.}} The djinni's innate spellcasting ability is Charisma 17, +9 to hit on spell attacks). He can innately cast the following spells, requiring no material components:

At will: \textit{detect good and evil, detect magic, thunderwave}

3/day each: \textit{wind walk, create food and water} (can create wine instead of water), \textit{languages}

1/day each: \textit{creation}, \textit{summon elementals} (air elemental only), \textit{gas form, major image}, \textit{invisibility,} \textit{shift planar}

\textbf{Actions}

\textit{\textbf{Multiattack.}} The djinni makes three attacks of
scimitar.

\textit{\textbf{Scimitar.} Melee Weapon Attack}: +15 to hit, reach 1m, one target.

\textit{Hit:} 12 (2d6 + 5) slashing damage plus 3 (1d6) electricity or sound damage (gin's choice).

\textit{\textbf{Create Whirlwind.}} A swirling cylinder of air with a 1 meter radius and 9 meters high magically forms in a point visible to the djinni within 36 meters of it. The whirlwind remains as long as the djinni maintains concentration (as if concentrating on a spell). Any creature other than the djinni that enters the whirlwind must succeed on a DC 18 Fortitude save or be restrained by it. The djinni can move the whirlwind up to 20 meters as an action, and creatures restrained by the whirlwind move with it. The whirlwind ends if the djinni loses sight of it.

A creature can use her action to free an entangled creature from the whirlwind, including itself, with a successful DC 18 Strength check. If the check succeeds, the creature is no longer entangled and moves to the nearest space outside the whirlwind. whirlwind.

\textbf{Ecology}
Environment any (Plane of Air)\\
Organization: Solitary, pair, company (3-6), or gang (7-10)\\
\textbf{Treasure}: Standard (masterwork scimitar, other treasure)\\
\textbf{Description}\\
Djinn (singular djinni) are Geniuses from the Elemental Plane of Air. They are said to be made of clouds and have the strength of the most powerful storms. A Djinni is about 3 meters tall and weighs about 500 kg.

Djinn disdain physical combat, preferring to use their magical powers and aerial abilities against enemies. A Djinni defeated in combat usually takes flight and becomes a whirlwind to harass those pursuing it. When faced with no choice but to fight in melee, most djinn prefer to wield scimitars with two masterful hands.

Towards other Genii, Djinn get along well with Janni and Marid. They are frequently at odds with the Shaitan, and are sworn enemies of the efreeti, despising these ferocious genies more than any other of the genie races. The conflict between the efreeti and the djinn is so legendary that many spellcasters attempt (with varying degrees of success) to secure a djinni's service by promising him help in his cause against his hated enemies.


\

\index[Monsters]{Genes, Efreeti}\textbf{Efreeti}

\textit{Large elemental, lawful evil}

\textbf{STRENGTH} +6

\textbf{DEXTERITY} +1

\textbf{CONSTITUTION} +7

\textbf{INTELLIGENCE} +3

\textbf{WISDOM} +2

\textbf{CHARISMA} +3

\textbf{Initiative} +3 -- \textbf{Defence} 23

\textbf{Hit Points} 200 (16d10 + 112)

\textbf{Move} 12m, fly 18m

\textbf{Saving Throws} Fortitude +18, Reflexes +12, Will +13

\textbf{Damage Immunity} Fire

\textbf{Senses} darkvision 36m

\textbf{Languages} Ignan

\textbf{Challenge} 11 (7200 XP)

\textit{\textbf{Elemental Death.}} If the efreeti dies, its body disintegrates in a flash of fire and a puff of smoke, leaving behind only the equipment the efreeti was wearing or carrying.

\textit{\textbf{Innate Spells.}} The efreeti's innate spellcasting ability is Charisma, +7 to hit on spell attacks). He can innately cast the following spells, requiring no material components:

At will: \textit{detect magic}

3/day each: \textit{enlarge/reduce, languages}

1/day each: \textit{summon elementals} (fire elemental only), \textit{gas form, major image}, \textit{invisibility, wall of fire, plane shift}

\textbf{Actions}

\textit{\textbf{Multiattack.}} The efreeti makes two scimitar attacks or uses Throw Flame twice.

\textit{\textbf{Scimitar.} Melee Weapon Attack}: +21 to hit, reach 1m, one target.

\textit{Hit:} 13 (2d6 + 6) slashing damage plus 7 (2d6) fire damage.

\textit{\textbf{Throw Flame.} Ranged weapon attack}: +16 to hit, range 36m, one target.

\textit{Hit:} 17 (5d6) fire damage.

\textbf{Ecology}
Environment any (Plane of Fire)\\
Organization: Solitary, pair, company (3-6), or gang (7-12)\\
\textbf{Treasure}: Standard (Masterwork Glaive, other treasure)\\
\textbf{Description}\\
Efreet (singular Efreeti) are Djinns from the Plane of Fire. They are 3.6 meters tall and weigh about 1000 kg.

The Efreet have few allies among other Geniuses: they hate the Djinni, and attack them on sight, they can't stand the Marid, and they see the Janni as weak and frail. The Efreet often cooperate well with the Shaitan, yet even these alliances are temporary.


\

\index[Monsters]{Ghost}\textbf{Ghost}

\textit{Medium Undead, any trait}

\textbf{STRENGTH} -2

\textbf{DEXTERITY} +1

\textbf{CONSTITUTION} +0

\textbf{INTELLIGENCE} +0

\textbf{WISDOM} +1

\textbf{CHARISMA} +3

\textbf{Initiative} +1 -- \textbf{Defence} 13

\textbf{Hit Points} 45 (10d8)

\textbf{Move} 0m, fly 12m (float)

\textbf{Saving Throws} Fortitude +7, Reflexes +6, Will +7

\textbf{Damage Resistances} acid, electricity, fire, sound, not magical weapon

\textbf{Immunity to Damage} Cold, Void, Poison

\textbf{Condition Immunity} charmed, grabbed, poisoned, restrained, paralyzed, petrified, prone, fatigued, frightened, bleeding

\textbf{Senses} Darkvision 18m

\textbf{Languages} any language known in life, Exspiram

\textbf{Challenge} 4 (1100 XP)

\textit{\textbf{Incorporeal Movement.}} The ghost can pass through other creatures and objects as if they were hindering terrain. He takes 5 (1d10) force damage if he ends his round inside an object.

\textit{\textbf{Undead nature.}} The ghost does not need air, food, drink, or sleep.

\textit{\textbf{Ethereal Sight.}} The ghost can see 20 meters in the Ethereal Plane when on the Material Plane, and vice versa.

\textbf{Actions}

\textit{\textbf{Withering Touch.} Melee Weapon Attack}: +6 to hit, reach 1m, one target.

\textit{Hit:} 17 (4d6 + 3) void damage. The target must make a DC 15 Fortitude save or become fatigued.

\textit{\textbf{Ethereality.}} The ghost enters the Ethereal Plane from the Material Plane, or vice versa. She is visible on the Material Plane while on the Ethereal Plane, and vice versa, but cannot interact with anything on the other plane.

\textit{\textbf{Possession (Cooldown 6).}} A humanoid, within 1 meter and visible to the ghost, must succeed at a DC 13 Will save or be possessed by the ghost; the ghost then disappears, and the target is incapacitated and loses control of its body. The ghost now controls the body but does not knock the target out of awareness of him. The ghost cannot be the target of attacks, spells, or other effects, except those that turn undead, and retains its traits, Intelligence, Wisdom, Charisma, and immunity to being charmed and frightened. It otherwise uses the possessed target's statistics, but does not access the target's lore and skills.

The possession lasts until the body drops to 0 Hit Points, the ghost ends it as a bonus action, or the ghost is turned or expelled by an effect such as the spell \textit{dispel good and evil}. When the possession ends, the ghost reappears in an unoccupied space within 1 meter of the body. The target is immune to this ghost's possession for 24 hours after a successful save or after the possession ends.

\textit{\textbf{Horrific Face.}} Any non-undead creature within 20 meters of the ghost that can see it must succeed at a DC 13 Will save or be frightened for 1 minute. On a failed save by 5 or more, the target also ages 1d4 x 10 years. A frightened target can repeat the Saving Throw at the end of each of its rounds, ending the effect for itself if it succeeds at the Saving Throw. If the target's Saving Throw succeeds and the effect ends, the target is immune to the ghost's Horrific Face for the next 24 hours. The aging effect can be reversed with the spell \textit{greater restoration}, but only if cast within 24 of the aging effect.

\textbf{Ecology}
Environment: any\\
Organization: solitaire\\
\textbf{Treasure}: NPC gear\\
\textbf{Description}\\
When a soul is not granted rest due to some serious injustice, real or imagined, it sometimes returns as a ghost. These beings are eternally distressed, insubstantial, and unable to put things right. While ghosts can have any Trait, many cling to the world of the living with a strong sense of hatred and anger, and become evil as a result; even a good creature after death can become a hateful and cruel ghost.\\

More than other monsters, the ghost must have a clear background. Why did this character become a ghost? What legends surround it? An encounter with a ghost should never be accidental - there are plenty of other incorporeal undead, such as Wraiths and Specters, for that. A proper encounter with a ghost should take place in a scene at the culmination of a long, tense period built up with lesser minions or manifestations of undead spirits. The ghost example above represents a human princess murdered by an unfaithful lover; after a confrontation, he bound her with chains and threw her into the castle well, where she drowned. The ghost's abilities have been selected based on the background, showing how a powerful antagonist can be created. By applying the template to creatures with levels and therefore Abilities of their own or with significant racial abilities, much more powerful ghosts can be created.\\

When a ghost is created, it gains "copies" of the objects it valued in life (provided the originals are not in the possession of other creatures). The equipment functions normally for the ghost but passes through material objects or creatures. A weapon +1 or higher, however, can damage material creatures, but such attacks do half damage (50\%) unless it is a ghost touch weapon. A ghost can use shields and Armour only if they have the Ghost Touch ability.\\

The original items are left behind, just like the ghost's physical remains. If another creature wields the original, the incorporeal copy vanishes. This loss inevitably infuriates the ghost, who will stop at nothing to return the item to its original location (and regain its use).


\

\index[Monsters]{Ghoul, Black}\textbf{Ghoul, Black}

\textit{Medium Undead, Chaotic Evil}

\textbf{STRENGTH} +4

\textbf{DEXTERITY} +2

\textbf{CONSTITUTION} +2

\textbf{INTELLIGENCE} +0

\textbf{WISDOM} +1

\textbf{CHARISMA} -2

\textbf{Initiative} +2 -- \textbf{Defence} 19

\textbf{Hit Points} 105 (15d8+30)

\textbf{Move} 12m

\textbf{Saving Throws}: Fortitude +11, Reflexes +11, Will +8

\textbf{Damage Immunity} poison, void, critical damage, bleed, non-magical or silver weapons

\textbf{Condition Immunity} charmed, poisoned, fatigue,

\textbf{Senses} Darkvision 18m

\textbf{Languages} Common, Exspiram

\textbf{Challenge} 6 (2300 XP)

\textbf{\textit{Nefarious Aura}}: The Black Ghoul constantly exudes an aura around him that weakens the Defences of all but other ghouls. Every two rounds of permanence in the aura of 12 meters of radius around the Black Ghoul, a -1 is accumulated to all ST, when one moves away from the Black Ghoul, one point is recovered per round.

\textbf{Actions}

\textit{\textbf{Claws.} Melee Weapon Attack}: +12 to hit, reach 1m, one target.

\textit{Hit:} 15 (2d10 + 4) slashing damage, 2 bleed damage. If the target is a creature, other than an elf or undead, it must succeed on a DC 16 Fortitude save or be paralyzed for 1 minute. The target can repeat the Saving Throw at the end of each of its rounds, ending the effect on a successful save.

\textit{\textbf{Bite.} Melee Weapon Attack}: +13 to hit, reach 1m, one creature.

\textit{Hit:} 18 (3d8 + 6) piercing damage, 1 damage Bleeding, Disease Ghoul Fever

\textit{Ghoul Fever:} 3 days, DC Fortitude save 18, 6 hours, 3 successes, you transform into a Ghoul

\textbf{Ecology}
Environment: Any terrain\\
Organization: Group (4-8) or pack (14-24)\\
\textbf{Treasure}: Standard\\
\textbf{Description}\\
The Black Ghoul represents one of the evolutionary elite of the Ghouls. Usually at least one rotting ghoul to about 18 ghouls leads a group.

\

\index[Monsters]{Ghoul, Ghast}\textbf{Ghast}

\textit{Medium Undead, Chaotic Evil}

\textbf{STRENGTH} +3

\textbf{DEXTERITY} +3

\textbf{CONSTITUTION} +0

\textbf{INTELLIGENCE} +0

\textbf{WISDOM} +0

\textbf{CHARISMA} -1

\textbf{Initiative} +3 -- \textbf{Defence} 14

\textbf{Hit Points} 36 (8d8)

\textbf{Move} 9m

\textbf{Saving Throws}: Fortitude +2, Reflexes +2, Will +5

\textbf{Damage Resistances} from Void

\textbf{Immunity to Damage} Poison

\textbf{Condition Immunity} charmed, poisoned, fatigue

\textbf{Senses} Darkvision 18m

\textbf{Languages} Common

\textbf{Challenge} 2 (450 XP)

\textit{\textbf{stink.}} Any creature that starts its round within 1 meter of the ghast must succeed on a DC 12 Fortitude save or be nauseated until the start of its next round. On a successful save, the creature is immune to the ghast's stench for the next 24 hours
hours.

\textit{\textbf{Rebellion against Turning.}} The ghast and all ghouls within 10 meters of it have +1d6 on Saving Throws against effects that turn undead.

\textbf{Actions}

\textit{\textbf{Claws.} Melee Weapon Attack}: +6 to hit, reach 1m, one target.

\textit{Hit:} 10 (2d6 + 3) slashing damage. If the target is a creature other than an undead, it must succeed on a DC 12 Fortitude save or be paralyzed for 1 minute. The target can repeat the Saving Throw at the end of each of its rounds, ending the effect on a successful save.

\textit{\textbf{Bite.} Melee Weapon Attack}: +6 to hit, reach 1 meter, one creature.

\textit{Hit:} 12 (2d8 + 3) piercing damage.

\textbf{Ecology}\\
Environment: Any terrain\\
Organization: Solitary, group (2-4), or herd (7-12)\\
\textbf{Treasure}: Standard\\
\textbf{Description}\\
Ghasts are Ghouls with a deeper connection to the void. A ghast's paralysis also affects Elves. Ghasts roam in packs or command groups of common ghouls. The stench of death and decay that surrounds these creatures is overwhelming.


\

\index[Monsters]{Ghoul, Ghoul}\textbf{Ghoul}

\textit{Medium Undead, Chaotic Evil}

\textbf{STRENGTH} +1

\textbf{DEXTERITY} +2

\textbf{CONSTITUTION} +0

\textbf{INTELLIGENCE} -2

\textbf{WISDOM} +0

\textbf{CHARISMA} -2

\textbf{Initiative} +2 -- \textbf{Defence} 13

\textbf{Hit Points} 22 (5d8)

\textbf{Move} 9m

\textbf{Saving Throws}: Fortitude +1, Reflexes +2, Will +4

\textbf{Immunity to Damage} Poison

\textbf{Condition Immunity} charmed, poisoned, fatigue

\textbf{Senses} Darkvision 18m

\textbf{Languages} Common

\textbf{Challenge} 1 (200 XP)

\textbf{Actions}

\textit{\textbf{Claws.} Melee Weapon Attack}: +4 to hit, reach 1m, one target.

\textit{Hit:} 7 (2d4 + 2) slashing damage, 1 bleed damage. If the target is a creature, other than an elf or undead, it must succeed on a DC 12 Fortitude save or be paralyzed for 1 minute. The target can repeat the Saving Throw at the end of each of its rounds, ending the effect on a successful save.

\textit{\textbf{Bite.} Melee Weapon Attack}: +4 to hit, reach 1 meter, one creature.

\textit{Hit:} 9 (2d6 + 2) piercing damage.

\textbf{Ecology}
Environment: Any terrain\\
Organization: Solitary, group (2-4), or pack (7-12)\\
\textbf{Treasure}: Standard\\
\textbf{Description}\\
Ghouls are undead who frequent graveyards and eat corpses. Legends hold that the earliest ghouls were either cannibalistic humans brought back from the dead by unnatural hunger, or humans who in life fed on the decaying remains of their kin and who died (and were reborn) of a hideous disease; the true origin of these scavenging undead is uncertain.

Ghouls lurk at the fringes of civilization (in or near graveyards or city sewers) where they can source ample supplies of their favorite food. Though they prefer rotting bodies and often bury their victims to enhance the flavor, they eat the dead fresh if hungry enough.


While many surface ghouls live primitively, rumors speak of ghoul cities deep underground ruled by priests who worship cruel ancient gods or strange demon lords of hunger. These "civilized" ghouls are no less horrific in their eating habits, and indeed their concept of a well-laid banquet table is perhaps even more horrific than the idea of a fresh meal taken from a coffin.

\

\index[Monsters]{Ghoul, Mother}\textbf{Ghoul, Mother}

\textit{Medium Undead, Chaotic Evil}

\textbf{STRENGTH} +0

\textbf{DEXTERITY} +3

\textbf{CONSTITUTION} +2

\textbf{INTELLIGENCE} +2

\textbf{WISDOM} +1

\textbf{CHARISMA} +2

\textbf{Initiative} +3 -- \textbf{Defence} 21

\textbf{Hit Points} 90 (10d10+45)

\textbf{Move} 9m

\textbf{Saving Throws}: Fortitude +9, Reflexes +11, Will +9

\textbf{Damage Immunity} poison, void, critical damage, bleed, weapons +1

\textbf{Condition Immunity} charmed, poisoned, fatigue

\textbf{Senses} Darkvision 18m

\textbf{Languages} Common, Exspiram

\textbf{Challenge} 5 (1800 XP)

\textbf{Actions}

\textit{\textbf{Claws.} Melee Weapon Attack}: +5 to hit, reach 1m, one target.

\textit{Hit:} 12 (2d6 + 6) slashing damage, 2 bleed damage. If the target is a creature other than an undead, she must succeed on a DC 15 Fortitude save or be paralyzed for 1 minute. The target can repeat the Saving Throw at the end of each of its rounds, ending the effect on a successful save. If the creature fails the Saving Throw then it falls under the ghoul's curse. Within 1d3+1 days it will transform into a ghoul. Remove Curse with DC 19 is needed for removing the curse.

\textit{\textbf{Bite.} Melee Weapon Attack}: +6 to hit, reach 3 ft., one creature.

\textit{Hit:} 8 (2d6 + 2) piercing damage.

\textbf{Ecology}
Environment: Any Terrain\\
Organization: Clans (7-12+)\\
\textbf{Treasure}: Standard\\
\textbf{Description}\\
The Ghoul Mother is usually the head of a ghoul clan that can number several dozen members. She is respected and feared, she is usually among the most intelligent evolved ghouls and much appreciated for her ability to transform the living into ghouls. Their tactic plans to injure and not kill several people so that when they come home and then transformed they can attack and kill the whole village.


\

\index[Monsters]{Ghoul, Rotting}\textbf{Ghoul, Rotting}

\textit{Large undead, chaotic evil}

\textbf{STRENGTH} +1

\textbf{DEXTERITY} +2

\textbf{CONSTITUTION} +3

\textbf{INTELLIGENCE} -1

\textbf{WISDOM} +0

\textbf{CHARISMA} -2

\textbf{Initiative} +2 -- \textbf{Defence} 15

\textbf{Hit Points} 82 (12d10+12)

\textbf{Move} 6m

\textbf{Saving Throws}: Fortitude +7, Reflexes +5, Will +4

\textbf{Damage Immunity} poison, bleed, critical, void, non-magical or silver weapons

\textbf{Condition Immunity} charmed, poisoned, fatigue

\textbf{Senses} darkvision 40m

\textbf{Languages} Common, Exspiram

\textbf{Challenge} 4 (110 XP)

\textbf{\textit{Regeneration}}. The Rotting Ghoul regenerates 5 HP per round unless it is in full sunlight or took damage from Light in the previous round. If the Rotting Ghoul is in a graveyard, it recovers 10 HP per round.

\textbf{Actions}

\textit{\textbf{Claws.} Melee Weapon Attack}: +5 to hit, reach 2m, one target.

\textit{Hit:} 12 (2d10 + 2) slashing damage, 1 bleed damage. If the target is a creature other than an undead, she must succeed on a DC 14 Fortitude save or be paralyzed for 1 minute.

\textit{\textbf{Bite.} Melee Weapon Attack}: +6 to hit, reach 3 ft., one creature.

\textit{Hit:} 10 (2d8 + 2) piercing damage.

\textit{\textbf{Aura of Suffering.}}: The Rotting Ghoul exudes a 6 meters aura around him in which every successful attack automatically deals critical damage. Activating this aura costs 1 Action and lasts until the start of the next round.

\textbf{Ecology}
Environment: Any Terrain\\
Organization: Group (4-8) or pack (10-18)\\
\textbf{Treasure}: Standard\\
\textbf{Description}\\
Rotting Ghouls are one of the many evolutions of Ghouls. The continuous contact with negative energy and the feeding of corpses of all kinds for centuries have made him bigger, stronger and capable of inflicting and causing the most dangerous wounds to be inflicted.


\

\textbf{Giant Ape}\index[Monsters]{Giant Ape}

\textit{Huge beast, unaligned}

\textbf{STRENGTH} +6

\textbf{DEXTERITY} +2

\textbf{CONSTITUTION} +4

\textbf{INTELLIGENCE} -2

\textbf{WISDOM} +1

\textbf{CHARISMA} -2

\textbf{Initiative} +2 -- \textbf{Defence} 16

\textbf{Hit Points} 157 (15d12 + 60)

\textbf{Movement} 12m, climb 12m

\textbf{Saving Throws}: Fortitude +7, Reflexes +6, Will +4

\textbf{Skills} Acrobatics +9, Awareness +4

\textbf{Languages} -

\textbf{Challenge} 7 (2900 XP)

\textbf{Actions}

\textit{\textbf{Multiattack.}} The ape makes two punch attacks.

\textit{\textbf{Punch.} Melee Weapon Attack}: +9 to hit, reach 3m, one target.

\textit{Hit:} 22 (3d10 + 6) bludgeoning damage.

\textit{\textbf{Rock.} Ranged Weapon Attack}: +9 to hit, range 15m, one target.

\textit{Hit:} 30 (7d6 + 6) bludgeoning damage.

\

\textbf{Giant Badger}\index[Monsters]{Giant Badger}

\textit{Medium beast, unaligned}

\textbf{STRENGTH} +1

\textbf{DEXTERITY} +0

\textbf{CONSTITUTION} +2

\textbf{INTELLIGENCE} -4

\textbf{WISDOM} +1

\textbf{CHARISMA} -3

\textbf{Initiative} +0 -- \textbf{Defence} 11

\textbf{Hit Points} 13 (2d8 + 4)

\textbf{Movement} 9m, digging 3m

\textbf{Saving Throws}: Fortitude +2, Reflexes +1, Will +2

\textbf{Senses} vision in the dark 9m

\textbf{Languages} -

\textbf{Challenge} 1/4 (50 XP)

\textit{\textbf{A keen sense of smell.}} The badger has +1d6 on Wisdom (Awareness) checks based on smell.

\textbf{Actions}

\textit{\textbf{Multiattack.}} The badger makes two attacks: one with its bite and one with its claws.

\textit{\textbf{Claws.} Melee Weapon Attack}: +3 to hit, reach 1m, one target.

\textit{Hit:} 6 (2d4 + 1) slashing damage.

\textit{\textbf{Bite.} Melee Weapon Attack}: +3 to hit, reach 1m, one target.

\textit{Hit:} 4 (1d6 + 1) piercing damage.

\

\textbf{Giant Billy Goat}\index[Monsters]{Giant Billy Goat}

\textit{Large beast, unaligned}

\textbf{STRENGTH} +3

\textbf{DEXTERITY} +0

\textbf{CONSTITUTION} +1

\textbf{INTELLIGENCE} -4

\textbf{WISDOM} +1

\textbf{CHARISMA} -2

\textbf{Initiative} +0 -- \textbf{Defence} 12

\textbf{Hit Points} 19 (3d10 + 3)

\textbf{Move} 12m

\textbf{Saving Throws}: Fortitude +4, Reflexes +1, Will +1

\textbf{Languages} -

\textbf{Challenge} 1/2 (100 XP)

\textit{\textbf{Charge.}} If the billy goat moves at least 6 meters directly towards the target and hits with a bill attack during the same turn, the target takes an additional 5 (2d4) bludgeoning damage. If the target is a creature, it must succeed on a DC 13 Fortitude save or fall prone.

\textit{\textbf{Steady Feet.}} The billy goat has +1d6 on Fortitude and Reflex saves made against effects that would knock it prone.

\textbf{Actions}

\textit{\textbf{Beak.} Melee Weapon Attack}: +5 to hit, reach 1m, one target.

\textit{Hit:} 8 (2d4 + 3) bludgeoning damage.

\

\textbf{Giant Boar}\index[Monsters]{Giant Boar}

\textit{Large beast, unaligned}

\textbf{STRENGTH} +3

\textbf{DEXTERITY} +0

\textbf{CONSTITUTION} +3

\textbf{INTELLIGENCE} -4

\textbf{WISDOM} -2

\textbf{CHARISMA} -3

\textbf{Initiative} +0 -- \textbf{Defence} 13

\textbf{Hit Points} 42 (5d10 + 15)

\textbf{Move} 12m

\textbf{Saving Throws}: Fortitude +4, Reflexes +2, Will +0

\textbf{Languages} -

\textbf{Challenge} 2 (450 XP)

\textit{\textbf{Charge.}} If the boar moves at least 6 meters directly towards the target and hits with a tusk attack during the same turn, the target takes an additional 7 (2d6) slashing damage. If the target is a creature, it must succeed on a DC 13 Fortitude save or be knocked prone.

\textit{\textbf{Relentless (Recharges after 1 hour).}} If the boar takes 10 damage or less that would reduce it to 0 Hit Points, it drops to 1 hit point instead.

\textbf{Actions}

\textit{\textbf{Fang.} Melee Weapon Attack}: +5 to hit, reach 1m, one target.

\textit{Hit:} 10 (2d6 + 3) slashing damage.

\

\textbf{Giant Constrictor Snake}\index[Monsters]{Giant Constrictor Snake}

\textit{Huge beast, unaligned}

\textbf{STRENGTH} +4

\textbf{DEXTERITY} +2

\textbf{CONSTITUTION} +1

\textbf{INTELLIGENCE} -5

\textbf{WISDOM} +0

\textbf{CHARISMA} -4

\textbf{Initiative} +2 -- \textbf{Defence} 13

\textbf{Hit Points} 60 (8d12 + 8)

\textbf{Movement} 9m, swim 9m

\textbf{Saving Throws}: Fortitude +3, Reflexes +2, Will +0

\textbf{Skills} Awareness +2

\textbf{Senses} blindsight 3 m

\textbf{Languages} -

\textbf{Challenge} 2 (450 XP)

\textbf{Actions}

\textit{\textbf{Bite.} Melee Weapon Attack}: +6 to hit, reach 3m, one creature.

\textit{Hit:} 11 (2d6 + 4) piercing damage.

\textit{\textbf{Constrict.} Melee Weapon Attack}: +6 to hit, reach 1 meter, one creature.

\textit{Hit:} 13 (2d8 + 4) bludgeoning damage, and the target is grappled (DC 16 to flee). Until the grab ends, the creature is restrained, and the snake can't constrict another target.

\

\textbf{Giant Crab}\index[Monsters]{Giant Crab}

\textit{Medium beast, unaligned}

\textbf{STRENGTH} +1

\textbf{DEXTERITY} +2

\textbf{CONSTITUTION} +0

\textbf{INTELLIGENCE} -5

\textbf{WISDOM} -1

\textbf{CHARISMA} -4

\textbf{Initiative} +2 -- \textbf{Defence} 16

\textbf{Hit Points} 13 (3d8)

\textbf{Movement} 9m, swim 9m

\textbf{Saving Throws}: Fortitude +5, Reflexes +2, Will +1

\textbf{Skills} Stealth +4

\textbf{Senses} blindsight 9 m

\textbf{Languages} -

\textbf{Challenge} 1/8 (25 XP)

\textit{\textbf{Amphibious.}} The crab can breathe air and water.

\textbf{Actions}

\textit{\textbf{Claw (Claw).} Weapon Attack Melee}: +3 to hit, reach 1m, one target.

\textit{Hit:} 4 (1d6 + 1) bludgeoning damage and the target is grabbed (DC 11 to flee). The crab has two claws, each of which can grip only one target.

\

\textbf{Giant Crocodile}\index[Monsters]{Giant Crocodile}

\textit{Huge beast, unaligned}

\textbf{STRENGTH} +5

\textbf{DEXTERITY} -1

\textbf{CONSTITUTION} +3

\textbf{INTELLIGENCE} -4

\textbf{WISDOM} +0

\textbf{CHARISMA} -2

\textbf{Initiative} -1 -- \textbf{Defence} 15

\textbf{Hit Points} 85 (9d12 + 27)

\textbf{Movement} 9m, swim 15m

\textbf{Saving Throws}: Fortitude +15, Reflexes +8, Will +8

\textbf{Skills} Stealth +5

\textbf{Languages} -

\textbf{Challenge} 5 (1800 XP)

\textit{\textbf{Hold Breath.}} The crocodile can hold its breath for 30 minutes.

\textbf{Actions}

\textit{\textbf{Multiattack.}} The crocodile makes two attacks: one with its bite and one with its tail.

\textit{\textbf{Tail.} Melee Weapon Attack}: +8 to hit, reach 3m, one target not grabbed by the crocodile.

\textit{Hit:} 14 (2d8 + 5) bludgeoning damage. If the target is a creature, it must succeed on a DC 16 Fortitude save or be knocked prone.

\textit{\textbf{Bite.} Melee Weapon Attack}: +8 to hit, reach 1m, one target.

\textit{Hit:} 21 (3d10 + 5) piercing damage, and the target is grappled (DC 16 to escape). Until the grab ends, the target is restrained, and the crocodile can't use the bite against another target.

\

\textbf{Giant Eagle}\index[Monsters]{Giant Eagle}

The giant eagle is a noble creature that speaks its own language and understands that of other races.

\textit{Large beast, neutral good}

\textbf{STRENGTH} +3

\textbf{DEXTERITY} +3

\textbf{CONSTITUTION} +1

\textbf{INTELLIGENCE} -1

\textbf{WISDOM} +2

\textbf{CHARISMA} +0

\textbf{Initiative} +3 -- \textbf{Defence} 14

\textbf{Hit Points} 26 (4d10 + 4)

\textbf{Move} 3m, fly 24m

\textbf{Saving Throws}: Fortitude +5, Reflexes +7, Will +3

\textbf{Skills} Awareness +4

\textbf{Languages} Giant Eagle, understands Common and Ictum but cannot speak them

\textbf{Challenge} 1 (200 XP)

\textit{\textbf{Honed Vision.}} The eagle has +1d6 on Wisdom (Awareness) checks based on sight.

\textbf{Actions}

\textit{\textbf{Multiattack.}} The eagle makes two attacks: one with its beak and one with its spurs.

\textit{\textbf{Beak.} Melee Weapon Attack}: +5 to hit, reach 1m, one target.

\textit{Hit:} 6 (1d6 + 3) piercing damage.

\textit{\textbf{Spurs.} Melee Weapon Attack}: +5 to hit, reach 1m, one target.

\textit{Hit:} 10 (2d6 + 3) slashing damage.

\

\textbf{Giant Fire Beetle}\index[Monsters]{Giant Fire Beetle}

A giant fire beetle is a nocturnal creature that possesses a pair of glow glands capable of emitting light for 1d6 days after the beetle's death.

\textit{Little beast, unaligned}

\textbf{STRENGTH} -1

\textbf{DEXTERITY} +0

\textbf{CONSTITUTION} +1

\textbf{INTELLIGENCE} -5

\textbf{WISDOM} -2

\textbf{CHARISMA} -4

\textbf{Initiative} +0 -- \textbf{Defence} 14

\textbf{Hit Points} 4 (1d6 + 1)

\textbf{Move} 9m

\textbf{Saving Throws}: Fortitude +2, Reflexes +0, Will +0

\textbf{Senses} blindsight 9 m

\textbf{Languages} -

\textbf{Challenge} 0 (10 XP)

\textit{\textbf{Illumination.}} The beetle sheds bright light in a 3m radius and dim light for an additional 3 meter.

\textbf{Actions}

\textit{\textbf{Bite.} Melee Weapon Attack}: +1 to hit, reach 1m, one target.

\textit{Hit:} 2 (1d6 - 1) slashing damage.

\

\textbf{Giant Frog}\index[Monsters]{Giant Frog}

\textit{Medium beast, unaligned}

\textbf{STRENGTH} +1

\textbf{DEXTERITY} +1

\textbf{CONSTITUTION} +0

\textbf{INTELLIGENCE} -4

\textbf{WISDOM} +0

\textbf{CHARISMA} -4

\textbf{Initiative} +1 -- \textbf{Defence} 12

\textbf{Hit Points} 18 (4d8)

\textbf{Movement} 9m, swim 9m

\textbf{Saving Throws}: Fortitude +2, Reflexes +2, Will +0

\textbf{Skills} Stealth +3, Awareness +2

\textbf{Senses} vision in the dark 9 m

\textbf{Languages} -

\textbf{Challenge} 1/4 (50 XP)

\textit{\textbf{Amphibian.}} The frog can breathe air and water.

\textit{\textbf{Standing Jump.}} A frog can jump up to 6 meters long and up to 3 meters high, with or without a running start.

\textbf{Actions}

\textit{\textbf{Bite.} Melee Weapon Attack}: +3 to hit, reach 1m, one target.

\textit{Hit:} 4 (1d6 + 1) piercing damage and the target is grappled (DC 11 to escape). Until the grab ends, the target is restrained, and the frog can't use the bite against another target.

\textit{\textbf{Swallow.}} The frog makes a bite attack against a Small or smaller target it is grabbing. If the attack hits, the target is engulfed, and the grab ends. The swallowed target is blinded and restrained, has full cover against attacks and other effects outside the frog, and takes 5 (2d4) acid damage at the start of each of the frog's rounds. The frog can only swallow one target at a time. If the frog dies, a swallowed creature is no longer restrained by it and can exit the corpse using 1 meter of movement, coming prone.

\

\textbf{Giant Hyena}\index[Monsters]{Giant Hyena}

\textit{Large beast, unaligned}

\textbf{STRENGTH} +3

\textbf{DEXTERITY} +2

\textbf{CONSTITUTION} +2

\textbf{INTELLIGENCE} -4

\textbf{WISDOM} +1

\textbf{CHARISMA} -2

\textbf{Initiative} +2 -- \textbf{Defence} 13

\textbf{Hit Points} 45 (6d10 + 12)

\textbf{Move} 15m

\textbf{Saving Throws}: Fortitude +6, Reflexes +6, Will +2

\textbf{Skills} Awareness +3

\textbf{Languages} -

\textbf{Challenge} 1 (200 XP)

\textit{\textbf{Rage.}} When the hyena reduces a creature to 0 Hit Points with a melee attack during its round, the hyena can take a bonus action to move up to half its movement and perform a bite attack.

\textbf{Actions}

\textit{\textbf{Bite.} Melee Weapon Attack}: +5 to hit, reach 1m, one target.

\textit{Hit:} 10 (2d6 + 3) piercing damage.

\

\textbf{Giant Lizard}\index[Monsters]{Giant Lizard}

Giant lizards are fearsome predators and are often used as mounts or draft animals by reptilian humanoids and underground residents.

\textit{Large beast, unaligned}

\textbf{STRENGTH} +2

\textbf{DEXTERITY} +1

\textbf{CONSTITUTION} +1

\textbf{INTELLIGENCE} -4

\textbf{WISDOM} +0

\textbf{CHARISMA} -3

\textbf{Initiative} +1 -- \textbf{Defence} 13

\textbf{Hit Points} 19 (3d10 + 3)

\textbf{Move} 9m, climb 9m

\textbf{Saving Throws}: Fortitude +11, Reflexes +8, Will +4

\textbf{Senses} vision in the dark 9m

\textbf{Languages} -

\textbf{Challenge} 1/4 (50 XP)

\textbf{Actions}

\textit{\textbf{Bite.} Melee Weapon Attack}: +4 to hit, reach 1m, one target.

\textit{Hit:} 6 (1d8 + 2) piercing damage.

\textbf{VARIANT}

Some giant lizards possess one or both of the following traits.

\textit{\textbf{Climb as Spider.}} The lizard can climb difficult surfaces, including standing upside down on ceilings, without needing to make an ability check.

\textit{\textbf{Hold Breath.}} The lizard can hold its breath for 15 minutes. (A lizard with this trait also has a swim speed of 10 meters.)

\

\textbf{Giant Moose}\index[Monsters]{Giant Moose}

\textit{Huge beast, unaligned}

\textbf{STRENGTH} +4

\textbf{DEXTERITY} +3

\textbf{CONSTITUTION} +2

\textbf{INTELLIGENCE} -2

\textbf{WISDOM} +2

\textbf{CHARISMA} +0

\textbf{Initiative} +3 -- \textbf{Defence} 15

\textbf{Hit Points} 42 (5d12 + 10)

\textbf{Move} 18m

\textbf{Saving Throws}: Fortitude +8, Reflexes +7, Will +2

\textbf{Skills} Awareness +4

\textbf{Languages} Giant Elk, includes Common, Elvish, and

Silvano but cannot speak to them

\textbf{Challenge} 2 (450 XP)

\textit{\textbf{Charge.}} If the moose moves at least 6 meters directly at the target and hits it with a bill attack during the same turn, the target takes an additional 7 (2d6) bludgeoning damage . If the target is a creature, it must succeed on a DC 14 Fortitude save or be knocked prone.

\textbf{Actions}

\textit{\textbf{Beak.} Melee Weapon Attack}: +6 to hit, reach 3m, one target.

\textit{Hit:} 11 (2d6 + 4) piercing damage.

\textit{\textbf{Hooves.} Melee Weapon Attack}: +6 to hit, reach 1m, one prone creature.

\textit{Hit:} 22 (4d4 + 4) bludgeoning damage.

\

\textbf{Giant Owl}\index[Monsters]{Giant Owl}

Giant owls are intelligent creatures that protect the woodland kingdoms.

\textit{Large beast, neutral}

\textbf{STRENGTH} +1

\textbf{DEXTERITY} +2

\textbf{CONSTITUTION} +1

\textbf{INTELLIGENCE} -1

\textbf{WISDOM} +1

\textbf{CHARISMA} +0

\textbf{Initiative} +2 -- \textbf{Defence} 13

\textbf{Hit Points} 19 (3d10 + 3)

\textbf{Move} 1m, fly 18m

\textbf{Saving Throws}: Fortitude +1, Reflexes +4, Will +1

\textbf{Skills} Stealth +4, Awareness +5

\textbf{Senses} vision in the dark 36 m

\textbf{Languages} Giant Owl, understands Common, Elven, and Sylvan but cannot speak them

\textbf{Challenge} 1/4 (50 XP)

\textit{\textbf{Flying over.}} The owl does not provoke attacks of opportunity when it flies out of an enemy's reach.

\textit{\textbf{Honed hearing and vision.}} The owl has +1d6 on Wisdom (Awareness) checks based on hearing or vision.

\textbf{Actions}

\textit{\textbf{Spurs.} Melee Weapon Attack}: +3 to hit, reach 1m, one target.

\textit{Hit:} 8 (2d6 + 1) piercing damage.

\

\textbf{Giant Rat}\index[Monsters]{Giant Rat}

\textit{Little beast, unaligned}

\textbf{STRENGTH} -2

\textbf{DEXTERITY} +2

\textbf{CONSTITUTION} +0

\textbf{INTELLIGENCE} -4

\textbf{WISDOM} +0

\textbf{CHARISMA} -3

\textbf{Initiative} +2 -- \textbf{Defence} 13

\textbf{Hit Points} 7 (2d6)

\textbf{Move} 9m

\textbf{Saving Throws}: Fortitude +3, Reflexes +5, Will +1

\textbf{Senses} vision in the dark 18 m

\textbf{Languages} -

\textbf{Challenge} 1/8 (25 XP)

\textit{\textbf{Enhanced sense of smell.}} The rat has +1d6 on Wisdom (Awareness) checks based on smell.

\textit{\textbf{Pack tactics.}} The rat has +1d6 on attack rolls against a creature if at least one of the rat's allies is within 1 meter of the creature and that ally isn't incapacitated.

\textbf{Actions}

\textit{\textbf{Bite.} Melee Weapon Attack}: +4 to hit, reach 1m, one target.

\textit{Hit:} 4 (1d4 + 2) piercing damage.

\textbf{VARIANT: GIANT SICK RAT}\index[Monsters]{Giant Sick Rat}

Some giant rats carry a terrible disease which they spread by biting. A sick giant rat has a challenge rating of 1/8 (25 XP) and the following action instead of its normal bite attack.

\textit{\textbf{Bite.} Melee Weapon Attack}: +4 to hit, reach 1m, one target.

\textit{Hit:} 4 (1d4 + 2) piercing damage. If the target is a creature, it must succeed on a DC 10 Fortitude save or contract a disease. Until the disease is cured, the target cannot regain Hit Points except through magical methods, and the target's maximum Hit Points decrease by 3 (1d6) every 24 hours. If the target's maximum Hit Points drop to 0 as a result of the disease, the target dies.

\

\textbf{Giant Scorpion}\index[Monsters]{Giant Scorpion}

\textit{Large beast, unaligned}

\textbf{STRENGTH} +2

\textbf{DEXTERITY} +1

\textbf{CONSTITUTION} +2

\textbf{INTELLIGENCE} -5

\textbf{WISDOM} -1

\textbf{CHARISMA} -4

\textbf{Initiative} +1 -- \textbf{Defence} 17

\textbf{Hit Points} 52 (7d10 + 14)

\textbf{Move} 12m

\textbf{Saving Throws}: Fortitude +7, Reflexes +1, Will +1

\textbf{Senses} blindsight 18 m

\textbf{Languages} -

\textbf{Challenge} 3 (700 XP)

\textbf{Actions}

\textit{\textbf{Multiattack.}} The scorpion makes three attacks: two with its claws and one with its sting.

\textit{\textbf{Claw.} Melee Weapon Attack}: +4 to hit, reach 1m, one target.

\textit{Hit:} 6 (1d8 + 2) bludgeoning damage and the target is grabbed (DC 12 to flee). The scorpion has two claws, each of which can only grip one target.

\textit{\textbf{Sting.} Melee Weapon Attack}: +4 to hit, reach 1 meter, one creature.

\textit{Hit:} 7 (1d10 + 2) piercing damage and the target must make a DC 12 Fortitude save, taking 22 (4d10) poison damage on a failed save, or half as much damage on a failed save. he succeeds.

\

\textbf{Giant Seahorse}\index[Monsters]{Giant Seahorse}

The giant seahorse is often used as a mount by aquatic humanoids.

\textit{Large beast, unaligned}

\textbf{STRENGTH} +1

\textbf{DEXTERITY} +2

\textbf{CONSTITUTION} +0

\textbf{INTELLIGENCE} -4

\textbf{WISDOM} +1

\textbf{CHARISMA} -3

\textbf{Initiative} +2 -- \textbf{Defence} 14

\textbf{Hit Points} 16 (3d10)

\textbf{Movement} 0m, swim 12m

\textbf{Saving Throws}: Fortitude +2, Reflexes +3, Will +1

\textbf{Languages} -

\textbf{Challenge} 1/2 (100 XP)

\textit{\textbf{Charge.}} If the seahorse moves at least 6 meters directly towards the target and hits with a ram attack during the same turn, the target takes an additional 7 (2d6) bludgeoning damage. If the target is a creature, it must succeed on a DC 11 Fortitude save or fall prone.

\textit{\textbf{Water Breathing.}} The seahorse can only breathe underwater.

\textbf{Actions}

\textit{\textbf{Beak.} Melee Weapon Attack}: +3 to hit, reach 1m, one target.

\textit{Hit:} 4 (1d6 + 1) bludgeoning damage.

\

\textbf{Giant Shark}\index[Monsters]{Giant Shark}

The giant shark is 9 meters long and you come across it

normally only in the deepest oceans.

\textit{Huge beast, unaligned}

\textbf{STRENGTH} +6

\textbf{DEXTERITY} +0

\textbf{CONSTITUTION} +5

\textbf{INTELLIGENCE} -5

\textbf{WISDOM} +0

\textbf{CHARISMA} -3

\textbf{Initiative} +0 -- \textbf{Defence} 16

\textbf{Hit Points} 126 (11d12 + 55)

\textbf{Movement} 0m, swim 15m

\textbf{Saving Throws}: Fortitude +7, Reflexes +2, Will +1

\textbf{Skills} Awareness +3

\textbf{Senses} blindsight 18 m

\textbf{Languages} -

\textbf{Challenge} 5 (1800 XP)

\textit{\textbf{Blood Frenzy.}} The shark has +1d6 on attack rolls

melee against any creature that isn't at full Hit Points.

\textit{\textbf{Water Breathing.}} The shark can only breathe underwater.

\textbf{Actions}

\textit{\textbf{Bite.} Melee Weapon Attack}: +9 to hit, reach 1m, one target.

\textit{Hit:} 22 (3d10 + 6) piercing damage.

\

\textbf{Giant Spider}\index[Monsters]{Giant Spider}

\textit{Large beast, unaligned}

\textbf{STRENGTH} +2

\textbf{DEXTERITY} +3

\textbf{CONSTITUTION} +1

\textbf{INTELLIGENCE} -4

\textbf{WISDOM} +0

\textbf{CHARISMA} -3

\textbf{Initiative} +2 -- \textbf{Defence} 15

\textbf{Hit Points} 26 (4d10 + 4)

\textbf{Move} 9m, climb 9m

\textbf{Saving Throws}: Fortitude +4, Reflexes +4, Will +1

\textbf{Skills} Stealth +7

\textbf{Senses} blindsight 3m, darksight 18m

\textbf{Languages} -

\textbf{Challenge} 1 (200 XP)

\textit{\textbf{Web Walk.}} The spider ignores movement restrictions caused by webs.

\textit{\textbf{Climb as Spider.}} The spider can climb difficult surfaces, including standing upside down on ceilings, without needing to make an ability check.

\textit{\textbf{Web Sense.}} While in contact with a web, the spider knows the exact location of any other creature in contact with the same web.

\textbf{Actions}

\textit{\textbf{Bite.} Melee Weapon Attack}: +5 to hit, reach 1 meter, one creature.

\textit{Hit:} 7 (1d8 + 3) piercing damage and the target must make a DC 11 Fortitude save, and suffer 9

(2d8) poison damage on a failed save, or half as much damage on a successful one. If the poison damage reduces the target to 0 Hit Points, the target is stable but poisoned for 1 hour, even after regaining Hit Points, and while poisoned in this way becomes paralysed.

\textit{\textbf{Web (Cooldown 5-6).} Ranged Weapon Attack}: +5 to hit, reach 9m, one creature.

\textit{Hit:} Target is entangled in the web. As an action, the entangled target can make a DC 12 Strength check and, if successful, break the web. The web can also be attacked and destroyed (Defense 10; HP 5; vulnerability to fire damage; immunity to bludgeoning and poison damage).

\

\textbf{Giant Toad}\index[Monsters]{Giant Toad}

\textit{Large beast, unaligned}

\textbf{STRENGTH} +2

\textbf{DEXTERITY} +1

\textbf{CONSTITUTION} +1

\textbf{INTELLIGENCE} -4

\textbf{WISDOM} +0

\textbf{CHARISMA} -4

\textbf{Initiative} +1 -- \textbf{Defence} 12

\textbf{Hit Points} 39 (6d10 + 6)

\textbf{Movement} 6m, swim 12m

\textbf{Saving Throws}: Fortitude +6, Reflexes +6, Will +0

\textbf{Senses} vision in the dark 9m

\textbf{Languages} -

\textbf{Challenge} 1 (200 XP)

\textit{\textbf{Amphibian.}} The toad can breathe air and water.

\textit{\textbf{Standing Jump.}} A toad can jump up to 6 meters long and up to 3 meters high, with or without a running start.

\textbf{Actions}

\textit{\textbf{Bite.} Melee Weapon Attack}: +4 to hit, reach 1m, one target.

\textit{Hit:} 7 (1d10 + 2) piercing damage plus 5 (1d10) poison damage, and the target is grappled (DC 13 to escape). Until the grab ends, the target is restrained, and the toad can't use the bite against another target.

\textit{\textbf{Swallow up.}} The toad makes a bite attack against a Medium or smaller target it is grappling. If the attack hits, the target is engulfed, and the grab ends. The swallowed target is blinded and restrained, has full cover against attacks and other effects outside the frog, and takes 10 (3d6) acid damage at the start of each of the toad's rounds. The toad can only swallow one target at a time.

If the toad dies, a swallowed creature is no longer restrained by it and can exit the corpse using 1 meter of movement, coming prone.

\

\textbf{Giant Venomous Snake}\index[Monsters]{Giant Venomous Snake}

\textit{Medium beast, unaligned}

\textbf{STRENGTH} +0

\textbf{DEXTERITY} +4

\textbf{CONSTITUTION} +1

\textbf{INTELLIGENCE} -4

\textbf{WISDOM} +0

\textbf{CHARISMA} -4

\textbf{Initiative} +4 -- \textbf{Defence} 15

\textbf{Hit Points} 11 (2d8 + 2)

\textbf{Movement} 9m, swim 9m

\textbf{Saving Throws}: Fortitude +1, Reflexes +5, Will +2

\textbf{Skills} Awareness +2

\textbf{Senses} blindsight 3 m

\textbf{Languages} -

\textbf{Challenge} 1/4 (50 XP)

\textbf{Actions}

\textit{\textbf{Bite.} Melee Weapon Attack}: +6 to hit, reach 3m, one target.

\textit{Hit:} 6 (1d4 + 4) piercing damage and the target must make a DC 11 Fortitude Saving Throw, taking 10 (3d6) poison damage on a failed save, or half as much damage on a failed save. he succeeds.

\

\textbf{Giant Vulture}\index[Monsters]{Giant Vulture}

The giant vulture possesses superior intelligence and a mischievous aptitude.

\textit{Large Beast, Neutral Evil}

\textbf{STRENGTH} +2

\textbf{DEXTERITY} +0

\textbf{CONSTITUTION} +2

\textbf{INTELLIGENCE} -2

\textbf{WISDOM} +1

\textbf{CHARISMA} -2

\textbf{Initiative} +0 -- \textbf{Defence} 11

\textbf{Hit Points} 22 (3d10 + 6)

\textbf{Move} 3m, fly 18m

\textbf{Saving Throws}: Fortitude +10, Reflexes +6, Will +3; +4 against disease

\textbf{Skills} Awareness +3

\textbf{Languages} understands Common but cannot speak

\textbf{Challenge} 1 (200 XP)

\textit{\textbf{Honed sense of smell and vision.}} The vulture has +1d6 on Wisdom (Awareness) checks based on smell or sight.

\textit{\textbf{Packing tactics.}} The vulture has +1d6 on attack rolls against a creature if at least one of the vulture's allies is within 1 meter of the creature and that ally isn't incapacitated.

\textbf{Actions}

\textit{\textbf{Multiattack.}} The Vulture makes two attacks: one with its beak and one with its talons.

\textit{\textbf{Beak.} Melee Weapon Attack}: +4 to hit, reach 1m, one target.

\textit{Hit:} 7 (2d4 + 2) piercing damage.

\textit{\textbf{Spurs.} Melee Weapon Attack}: +4 to hit, reach 1m, a target.

\textit{Hit:} 9 (2d6 + 2) slashing damage.

\

\textbf{Giant Wasp}\index[Monsters]{Giant Wasp}

\textit{Medium beast, unaligned}

\textbf{STRENGTH} +0

\textbf{DEXTERITY} +2

\textbf{CONSTITUTION} +0

\textbf{INTELLIGENCE} -5

\textbf{WISDOM} +0

\textbf{CHARISMA} -4

\textbf{Initiative} +2 -- \textbf{Defence} 13

\textbf{Hit Points} 13 (3d8)

\textbf{Move} 3m, fly 15m

\textbf{Saving Throws}: Fortitude +1, Reflexes +3, Will +0

\textbf{Languages} -

\textbf{Challenge} 1/2 (100 XP)

\textbf{Actions}

\textit{\textbf{Sting.} Melee Weapon Attack}: +4 to hit, reach 1 meter, one creature.

\textit{Hit:} 5 (1d6 + 2) piercing damage and the target must make a DC 11 Fortitude Saving Throw, taking 10 (3d6) poison damage on a failed save, or half as much damage on a failed save. he succeeds. If the poison damage reduces the target to 0 Hit Points, the target is stable but poisoned even after regaining Hit Points, and while poisoned in this way becomes paralysed.

\

\textbf{Giant Weasel}\index[Monsters]{Giant Weasel}

\textit{Medium beast, unaligned}

\textbf{STRENGTH} +0

\textbf{DEXTERITY} +3

\textbf{CONSTITUTION} +0

\textbf{INTELLIGENCE} -3

\textbf{WISDOM} +1

\textbf{CHARISMA} -3

\textbf{Initiative} +3 -- \textbf{Defence} 14

\textbf{Hit Points} 9 (2d8)

\textbf{Move} 12m

\textbf{Saving Throws}: Fortitude +6, Reflexes +7, Will +2

\textbf{Skills} Stealth +5, Awareness +3

\textbf{Senses} vision in the dark 18 m

\textbf{Languages} -

\textbf{Challenge} 1/8 (25 XP)

\textit{\textbf{Refined hearing and smell.}} The weasel has +1d6 on Wisdom (Awareness) checks based on hearing or smell.

\textbf{Actions}

\textit{\textbf{Bite.} Melee Weapon Attack}: +5 to hit, reach 1m, one target.

\textit{Hit:} 5 (1d4 + 3) piercing damage.

\

\textbf{Giant Wolf Spider}\index[Monsters]{Giant Wolf Spider}

A giant wolf spider hunts for prey in open ground or hides in burrows or crevices in the ground to ambush it.

\textit{Medium beast, unaligned}

\textbf{STRENGTH} +1

\textbf{DEXTERITY} +3

\textbf{CONSTITUTION} +1

\textbf{INTELLIGENCE} -4

\textbf{WISDOM} +1

\textbf{CHARISMA} -3

\textbf{Initiative} +3 -- \textbf{Defence} 14

\textbf{Hit Points} 11 (2d8 + 2)

\textbf{Movement} 12m, climb 12m

\textbf{Saving Throws}: Fortitude +2, Reflexes +4, Will +1

\textbf{Skills} Stealth +7, Awareness +3

\textbf{Senses} blindsight 3m, darksight 18m

\textbf{Languages} -

\textbf{Challenge} 1/4 (50 XP)

\textit{\textbf{Web Walk.}} The spider ignores movement restrictions caused by webs.

\textit{\textbf{Climb as Spider.}} The spider can climb difficult surfaces, including standing upside down on ceilings, without needing to make an ability check.

\textit{\textbf{Web Sense.}} While in contact with a web, the spider knows the exact location of any other creature in contact with the same web.

\textbf{Actions}

\textit{\textbf{Bite.} Melee Weapon Attack}: +3 to hit, reach 1 meter, one creature.

\textit{Hit:} 4 (1d6 + 1) piercing damage and the target must make a DC 11 Fortitude Saving Throw, taking 7 (2d6) poison damage on a failed save, or half as much damage on a failed save. he succeeds. If the poison damage reduces the target to 0 Hit Points, the target is stable but poisoned for 1 hour, even after regaining Hit Points, and while poisoned in this way becomes paralysed.

\

\index[Monsters]{Giant, Cloud}\textbf{Cloud Giant}

\textit{Huge giant, neutral good (50\%) or neutral evil (50\%)}

\textbf{STRENGTH} +8

\textbf{DEXTERITY} +0

\textbf{CONSTITUTION} +6

\textbf{INTELLIGENCE} +1

\textbf{WISDOM} +3

\textbf{CHARISMA} +3

\textbf{Initiative} +1 -- \textbf{Defence} 19

\textbf{Hit Points} 200 (16d12 + 96)

\textbf{Move} 12m

\textbf{Saving Throws} Fortitude +16, Reflexes +6, Will +10

\textbf{Skills} Sense Emotions +7, Awareness +7

\textbf{Languages} Common, Giant

\textbf{Challenge} 9 (5000 XP)

\textit{\textbf{Innate Spells.}} The giant's spellcasting ability is Charisma. The giant can cast these spells innately, requiring no material components:

At will: \textit{detect magic, light, Cloud of Mist}

3/day each: \textit{Feather Fall, veiled step, telekinesis}

1/day each: \textit{control weather, gaseous form}

\textit{\textbf{A keen sense of smell.}} The giant has +1d6 on Wisdom (Awareness) checks based on smell.

\textbf{Actions}

\textit{\textbf{Multiattack.}} The giant makes two attacks with the spiked mace.

\textit{\textbf{Spiked Mace.} Melee Weapon Attack}: +22 to hit, reach 3m, one target.

\textit{Hit:} 21 (3d8 + 8) piercing damage.

\textit{\textbf{Rock.} Ranged weapon attack}: +14 to hit, range 18m, one target.

\textit{Hit:} 30 (4d10 + 8) bludgeoning damage.

\textit{\textbf{Enraged}}: the Cloud Giant swings his weapon above his head summoning storm clouds and casting the spell Call Lightning. Cost 2 Actions.

\textbf{Ecology}\\
Environment: Temperate Mountains\\
Organization Solitary, party (2-5), family (2-5 plus 35\% noncombatants plus 1 4th-7th level wizard or Devotee and 2-5 Griffins), or tribe (6-20 plus 1 oracle mage or Devotee of 7th-12th level and 2-5 Griffins)\\
\textbf{Treasure}: Standard (Mailjack, Spiked Mace, other treasure)\\
\textbf{Description}\\
The skin color of cloud giants ranges from milky white to powder blue. Adult males stand about 5.4 meters tall and weigh approximately 2,500 kg. Females are slightly shorter and slimmer. Cloud giants can live up to 400 years, dressed in precious clothes and jewels. For many, appearance indicates status. The better the clothes and the finer the jewels, the more important the wearer. They also enjoy music, and most play one or more instruments (the harp is a favorite).

Cloud giants can have unusually varied Traits; about half are good and half are evil. Good cloud giants build roads that connect their settlements with human roads to promote trade. It is not unusual to see a good cloud giant walking among humans, for example, in a human city near a high mountain range. Evil cloud giants tend not to establish permanent settlements and instead prefer to live in crude shelters on high peaks, from which they descend only to rob villages of what they might need. These two philosophies often lead to the outbreak of violent and long-lasting wars between neighboring tribes.

There are many legends that speak of magical cloud giant cities located in the clouds themselves, floating on the winds and circumnavigating the world. While cloud giants acknowledge that most of them are fantasies, some claim to have seen them and have dedicated their entire lives to finding them.


\

\index[Monsters]{Giant, Fire}\textbf{Fire Giant}

\textit{Huge Giant, Lawful Evil}

\textbf{STRENGTH} +7

\textbf{DEXTERITY} -1

\textbf{CONSTITUTION} +6

\textbf{INTELLIGENCE} +0

\textbf{WISDOM} +2

\textbf{CHARISMA} +1

\textbf{Initiative} +0 -- \textbf{Defence} 22 (plate Armour)

\textbf{Hit Points} 162 (13d12 + 78)

\textbf{Move} 9m

\textbf{Saving Throws}: Fortitude +14, Reflexes +8, Will +10

\textbf{Skills} Acrobatics +11, Awareness +6

\textbf{Immunity to Damage} Fire

\textbf{Languages} Giant

\textbf{Challenge} 9 (5000 XP)

\textbf{Actions}

\textit{\textbf{Multiattack.}} The giant makes two attacks with his greatsword.

\textit{\textbf{Large Greatsword.} Melee Weapon Attack}: +20 to hit, reach 3m, one target.

\textit{Hit:} 28 (6d6 + 7) slashing damage.

\textit{\textbf{Rock.} Ranged weapon attack}: +12 to hit, range 18m, one target.

\textit{Hit:} 29 (4d10 + 7) bludgeoning damage.

\textit{\textbf{Enraged}}: the fire giant channels its energy into the weapon, which causes +2d6 fire damage.


\textbf{Ecology}
Environment: Hot Mountains\\
Organization: Solitary, party (2-5), warband (6-12 plus 35\% noncombatants and 1 adept or 1st-2nd level Devotee), raiding party (6-12 plus 1 adept or 3rd-5th level wizard, 2-5 Hellhounds and 2-3 Trolls or Ettins) or tribe (20-30 plus 1 6th-7th level Adept, wizard, or Devotee; 1 Warking or ranger of 8 8th-9th level; and 17-38 Hellhounds, 12-22 Trolls, 7-12 Ettins, and 1-2 Young Red Dragons)\\
\textbf{Treasure}: Standard (Half Armour, Greatsword, other treasure)\\
\textbf{Description}\\
Fire giants are the stiffest and most martial giants, ever ready for war and to treat anyone they encounter brutally. Their rigid command structure requires soldiers, officers and even generals, all obeying their king's orders without question.

Fire giants have bright orange hair that shimmers and sparkles as if on fire. An adult male is between 3.6 and 4.8 meters tall, with a ribcage of approximately 2.7 meters, and weighs approximately 3,500 kg. Females are slightly shorter and slimmer. Fire giants can live up to 350 years.

Fire giants wear clothing of sturdy fabrics or leather in orange, yellow, black, or red. The warriors wear helmets and half Armour of burnished steel and wield greatswords which they whirl across the battlefield. In large groups, fire giants fight with brutal and efficient group tactics, and they don't hesitate to sacrifice a few comrades to ambush the enemy.

Fire giants prefer warm places: the hotter the better. They can be found in deserts, volcanoes, hot springs and deep underground near lava vents. They live in castles, fortified settlements, or large caves, and the architecture of these places reflects their rigid, militaristic lifestyle, with officers living in better quarters than their subordinates.



\

\index[Monsters]{Giant, Frost}\textbf{Frost Giant}

\textit{Huge Giant, Neutral Evil}

\textbf{STRENGTH} +6

\textbf{DEXTERITY} -1

\textbf{CONSTITUTION} +5

\textbf{INTELLIGENCE} -1

\textbf{WISDOM} +0

\textbf{CHARISMA} +1

\textbf{Initiative} -1 -- \textbf{Defence} 19 (composite Armour)

\textbf{Hit Points} 138 (12d12 + 60)

\textbf{Move} 12m

\textbf{Saving Throws} Fortitude +14, Reflexes +3, Will +6

\textbf{Skills} Acrobatics +9, Awareness +3

\textbf{Immunity to Damage} Cold

\textbf{Languages} Giant

\textbf{Challenge} 8 (3900 XP)

\textbf{Actions}

\textit{\textbf{Multiattack.}} The giant makes two attacks with its great axe.

\textit{\textbf{Great Axe.} Melee Weapon Attack}: +18 to hit, reach 3m, one target.

\textit{Hit:} 25 (3d12 + 6) slashing damage.

\textit{\textbf{Rock.} Ranged weapon attack}: +11 to hit, range 18m, one target.

\textit{Hit:} 28 (4d10 + 6) bludgeoning damage.

\textit{\textbf{Enraged}}: the Frost Giant channels his energies through the weapon. The weapon deals an additional 2d6 points of cold damage.


\textbf{Ecology}\\
Environment: Cold Mountains\\
Organization: Solitary, gang (3-5), party (6-12 plus 35\% noncombatants and 1 1st-2nd level wizard or Devotee), raiding party (6-12 plus 35\ % non-combatants, 1 Devotee or wizard of 3rd-5th level, 1-4 Winter Wolves and 2-3 Ogres) or tribe (21-30 plus 1 adept, wizard or Devotee of 6th-7th level; 1 jarl Barbarian or ranger 7th-9th level; and 15-36 Winter Wolves, 13-22 Ogres, and 1-2 Young White Dragons)\\
\textbf{Treasure}: Standard (Mailcoat, Greataxe, other treasure)\\
\textbf{Description}\\
A frost giant has blue or dirty yellow hair, and eyes that are usually the same color. They dress in skins and furs, adorning themselves with whatever jewelry they own. Fighting frost giants also wear mail coats and metal helmets decorated with horns and feathers. An adult male is 5 meters tall and weighs about 1,400 kg. Females are slightly shorter and slimmer, but are otherwise identical to males. Frost giants can live up to 250 years.

Frost giants are greatly feared, as their lust for destruction and war and their reckless demeanor drives them to ever greater displays of brutality. Frost giants start out by attacking from a distance, hurling rocks until they run out of ammo or the opponent gets close, then deal them with their massive axes. A favorite tactic is to ambush them by hiding under the snow above an icy or snowy slope, where opponents will have a hard time reaching them, and then start by causing an avalanche before going into battle. Frost giants can hide very well in snowy environments and are masters of stealth in their domain.

Frost giants survive by hunting and raiding on their own, living in cold, desolate environments. Frost giant groups are split almost evenly between those who live in makeshift settlements or abandoned castles and those who roam the frozen north as nomads in search of loot and supplies. Frost giant leaders are called jarls and demand absolute obedience from their followers. At any time, a jarl may be challenged in combat for leadership of the tribe. These challenges typically end in the death of one of the contenders. A single jarl can often count on a dozen or more smaller tribes of frost giants as an extension of his tribe. In these cases, the leaders of minor tribes are known as captains or warlords.

Frost giants love to take captives and use them as both slaves and raw material. Typically, each group of frost giants keeps 1-2 humanoid slaves chained to a slave handler: the meanest and cruelest of the group after the jarl. They also have a thing for monstrous pets: White Dragons and Winter Wolves are popular choices, but Remorhaz and Yeti can also be found in a frost giant's lair.

\

\index[Monsters]{Giant, Hill}\textbf{Hill Giant}

\textit{Huge giant, chaotic evil}

\textbf{STRENGTH} +5

\textbf{DEXTERITY} -1

\textbf{CONSTITUTION} +4

\textbf{INTELLIGENCE} -3

\textbf{WISDOM} -1

\textbf{CHARISMA} -2

\textbf{Initiative} -1 -- \textbf{Defence} 16

\textbf{Hit Points} 105 (10d12 + 40)

\textbf{Move} 12m

\textbf{Saving Throws}: Fortitude +11, Reflexes +2, Will +3

\textbf{Skills} Awareness +2

\textbf{Languages} Giant

\textbf{Challenge} 5 (1800 XP)

\textbf{Actions}

\textit{\textbf{Multiattack.}} The giant makes two attacks with his heavyclub.

\textit{\textbf{Club.} Melee Weapon Attack}: +12 to hit, reach 3m, one target.

\textit{Hit:} 18 (3d8 + 5) bludgeoning damage.

\textit{\textbf{Rock.} Ranged weapon attack}: +6 to hit, range 18m, one target.

\textit{Hit:} 21 (3d10 + 5) bludgeoning damage.

\textbf{Ecology}\\
Environment: Temperate Hills\\
Organization Solitary, gang (2-5), warband (6-8), raiding party (9-12 plus 1d4 Dire Wolves), or tribe (13-30 plus 35\% noncombatant plus 1 combat leader of 4 8th-6th level, 11-16 dire wolves, 1-4 ogres, and 13-20 orc slaves)\\
\textbf{Treasure}: Standard (leather Armour, cudgel, other treasure)\\
\textbf{Description}\\
Hill giants have skin that ranges from tan to reddish brown, brown or black hair, and eyes of the same color. They wear layers of roughly tanned hides with the fur still on. They rarely wash or repair their own clothes, preferring to simply add new layers as the old ones wear out. Adults are about 3 meters tall and weigh more or less 550 kg. Hill giants can live to be 200 years old, although they rarely reach this age.

Hill giants prefer to fight from atop ledges and cliffs, where they can pel opponents with rocks and boulders, thus limiting personal risk. They like to make overrun attacks against smaller creatures at the start of combat, and only then do they take up position and begin swinging their massive clubs.

Hill giants are nomadic by nature and prefer to travel from place to place to raid and plunder. Though they prefer temperate climates, they don't mind traveling far from their favored environment if raiding is plentiful and prosperous. They are, on the whole, very selfish creatures, rarely facing battles they are not sure of winning. Hill giants are notorious for pushing each other when faced with formidable opponents, and they don't hesitate to sacrifice a mate to save their own skin. Roving bands of hill giants are widespread across the temperate hills, and their constant aggressiveness makes them one of the most feared dangers in this environment.


Solitary, nonevil hill giants are very rare, but are occasionally found in other humanoid societies, though they are almost never accepted in major cities or population centers. They are at home as workers and soldiers in remote frontier towns, and often serve as rudimentary diplomats negotiating with bands of marauding hill giants. Unfortunately, hill giants who abandon their racial way of life for civilization are laughed at and often killed on sight by their nomadic brethren. However, these "civilized" Hill Giants can find their place in society and many have managed to live a peaceful and tranquil existence.

\

\index[Monsters]{Giant, Stone}\textbf{Stone Giant}

\textit{Huge giant, neutral}

\textbf{STRENGTH} +6

\textbf{DEXTERITY} +2

\textbf{CONSTITUTION} +5

\textbf{INTELLIGENCE} +0

\textbf{WISDOM} +1

\textbf{CHARISMA} -1

\textbf{Initiative} +2 -- \textbf{Defence} 21

\textbf{Hit Points} 126 (11d12 + 55)

\textbf{Move} 12m

\textbf{Saving Throws} Fortitude +12, Reflexes +6, Will +7

\textbf{Skills} Acrobatics +12, Awareness +4

\textbf{Senses} Darkvision 18m

\textbf{Languages} Giant

\textbf{Challenge} 7 (2900 XP)

\textit{\textbf{Stone Camouflage.}} The giant has +1d6 on Dexterity (Hide) checks made to hide in rocky terrain.

\textbf{Actions}

\textit{\textbf{Multiattack.}} The giant makes two attacks with his heavyclub.

\textit{\textbf{Club.} Melee Weapon Attack}: +19 to hit, reach 5 meters, one target.

\textit{Hit:} 19 (3d8 + 6) bludgeoning damage.

\textit{\textbf{Rock.} Ranged weapon attack}: +15 to hit, range 18m, one target.

\textit{Hit:} 28 (4d10 + 6) bludgeoning damage. If the target is a creature, it must succeed on a DC 17 Fortitude save or fall prone.

\textbf{Reactions}

\textit{\textbf{Catch Rock.}} If a rock or similar object is hurled at the giant, the giant can, with a successful DC 10 Reflex save, catch the projectile and take no bludgeoning damage from it.

\textit{\textbf{Enraged}}: the Stone Giant concentrates his energies making his skin hard as stone. Until the end of the following round, it gains damage reduction equal to 13. It costs 2 Actions


\textbf{Ecology}
Environment: Temperate Mountains\\
Organization: Solitary, party (2-5), band (4-8), hunting party (9-12 plus 1 Elder), or tribe (13-30 plus 35\% noncombatants, 1-3 Elders and 4 -6 Dire Bears)\\
\textbf{Treasure}: Standard (cudgel, other treasure)\\
\textbf{Description}\\
Stone giants prefer thick leather garments, dyed in shades of brown and gray to blend in with the stone around them. Adults are about 3.6 meters tall, weigh about 750 kg and can live up to 800 years.

Stone giants fight from a distance if possible, but use gigantic stone clubs if they cannot avoid melee. One of the favorite tactics of the Stone giants is to stand still, blending into the landscape, then advance by hurling rocks and surprising the enemies.

Stone giants prefer to live in huge caves on rocky peaks. They rarely live more than a few days' travel from other bands of stone giants, and tend shared herds of goats and other livestock.

Older stone giants tend to drift away from the tribe for a long time, either to live in solitude somewhere or attempting to insert themselves into other humanoid civilizations. After decades of self-imposed exile, those who return are known as Elder Rock Giants.


\

\index[Monsters]{Giant, Storm}\textbf{Storm Giant}

\textit{Huge Giant, Chaotic Good}

\textbf{STRENGTH} +9

\textbf{DEXTERITY} +2

\textbf{CONSTITUTION} +5

\textbf{INTELLIGENCE} +3

\textbf{WISDOM} +4

\textbf{CHARISMA} +4

\textbf{Initiative} +3 -- \textbf{Defence} 23 (scale Armour)

\textbf{Hit Points} 230 (20d12 + 100)

\textbf{Movement} 15m, swim 15m

\textbf{Saving Throws} Fortitude +18, Reflexes +15, Will +17

\textbf{Skills} Arcanum +8, Acrobatics +14, Awareness +9, Story +8

\textbf{Damage Resistances} cold

\textbf{Damage Immunity} electricity, sound

\textbf{Languages} Common, Giant

\textbf{Challenge} 13 (10000 XP)

\textit{\textbf{Amphibious.}} The giant can breathe air and water.

\textit{\textbf{Innate Spells.}} The giant's spellcasting ability is Charisma. The giant can cast these spells innately, requiring no material components:

At will: \textit{controlled fall, detect magic,} \textit{levitation, light}

3/day each: \textit{control weather, breathe} \textit{underwater}

\textbf{Actions}

\textit{\textbf{Multiattack.}} The giant makes two attacks with his greatsword.

\textit{\textbf{Large Greatsword.} Melee Weapon Attack}: +29 to hit, reach 3m, one target.

\textit{Hit:} 30 (6d6 + 9) slashing damage.

\textit{\textbf{Rock.} Ranged weapon attack}: +22 to hit, range 18m, one target.

\textit{Hit:} 35 (4d12 + 9) bludgeoning damage.

\textit{\textbf{Lightning Strike (Cooldown 5-6).}} The giant hurls a magical thunderbolt at a visible point within 150 meters of itself. Each creature within 3 meter of that point must make a DC 17 Reflex save, taking 54 (12d8) electricity damage on a failed save, or half as much damage on a successful one.

\textit{\textbf{Enraged}}: the storm giant charges the area around him with electricity until the end of the fight. A creature that ends the round within 6 meter of the giant takes 13 (3d8) electricity damage. It costs 1 Action.


\textbf{Ecology}\\
Environment: Any warm\\
Organization Solitary or Family (2-5 plus 1 7th-10th level Mage or Devotee, 1-2 Rocs, 2-6 Griffins, and 2-8 Sharks)\\
\textbf{Treasure}: Standard (masterwork plate cuirass, masterwork composite longbow[+9 Strength] with 20 arrows, masterwork greatsword, other treasure)\\
\textbf{Description}\\
Storm giants tend to have tan skins, though rare specimens have purple skin, purple or dark blue hair, and silver gray or purple eyes. The color purple is considered auspicious among storm giants, and those who possess it tend to become leaders among their people. Adults are normally 6.3 meters tall and weigh 6000 kg. Storm giants can live up to 600 years.

When at rest, they prefer to wear short tunics and wide waistbands, sandals or bare feet, and a headband. They wear a few jewels of simple but excellent workmanship, the most common being anklets (preferred by barefoot giants), rings or tiaras. But when they gear up for battle, they wear breastplates of gleaming plate and wield huge greatswords and bows.

Storm giants tend to be solitary, preferring to inhabit remote coasts or tropical islands. As their name suggests, they are prone to violent mood swings. Storm giants are quick to anger in the face of evil and can be brutal and dangerous foes when insulted. In battle, they prefer to unleash a shower of arrows at their foes, drawing their greatswords only after their opponents are close.

Storm giants live in beautiful towers, castles, or walled settlements and enjoy farming. They have huge manicured gardens and manage hundreds of acres of crops per group. They often employ other humanoids, such as Elves or Humans, as support to run their huge farms. An enclave of storm giants often assumes responsibility for the safety of an entire island or shoreline.

\

\index[Monsters]{Globule}\textbf{Globule}

\textit{Little aberration, evil}

\textbf{STRENGTH} -2

\textbf{DEXTERITY} +2

\textbf{CONSTITUTION} +0

\textbf{INTELLIGENCE} +3

\textbf{WISDOM} +1

\textbf{CHARISMA} +3

\textbf{Initiative} +3 -- \textbf{Defence} 15

\textbf{Hit Points} 30 (5d10 + 5)

\textbf{Move} fly 18m

\textbf{Saving Throws}: Fortitude +4, Reflexes +6, Will +5

\textbf{Skills} -

\textbf{Senses} darkvision 40m

\textbf{Linguaggi} understands common but does not speak it

\textbf{Challenge} 1 (200 XP)

\textbf{Vulnerability} Fire

\textbf{Immunity} Void, Cold

\textbf{Immunity to conditions} poison, prone

\textbf{I hate birds} The Globule has +1d6 on attack rolls against birds. Attack the birds and flying creatures first

\textbf{Unusual nature} the Globule does not breathe

\textbf{I hate water} the Globule hates getting wet and every 5 liters of water splashed on it takes 1d4 damage

\textbf{Actions}

\textit{\textbf{Tentacle}}. Melee attack, +5 on hit, reach 3 meter, one target

\textit{\textbf{Hit}} 5 (1d6+2) Void damage. The target must make a DC 11 Fortitude save or increase the degree of fatigue by 1.

\textbf{\textit{Gleaming}} once per day the globule becomes extremely luminous, creatures within a 6 meters radius around it must make a DC 13 Fortitude save or go blind for 3 rounds.

\textbf{Ecology}
Environment: Any, desert, nocturnal\\
Organization solitary, groups 2d4\\
\textbf{Treasure}: None\\
\textbf{Description}\\
Globules are magical aberrations from some open portal to the Beyond. Creatures of cold and emptiness look like little stars that yearn only to suck the life out of the creatures they encounter.
Clever and cunning, they prefer to attack by staying airborne and wearing down their opponent until they are dead fatigued. Once a Globule is killed, all that remains is a small star-shaped creature with a large central eye, completely white.


\

\index[Monsters]{Gnoll}\textbf{Gnoll}

\textit{Medium humanoid (gnoll), chaotic evil}

\textbf{STRENGTH} +2

\textbf{DEXTERITY} +1

\textbf{CONSTITUTION} +0

\textbf{INTELLIGENCE} -2

\textbf{WISDOM} +0

\textbf{CHARISMA} -2

\textbf{Initiative} +1 -- \textbf{Defence} 16 (leather Armour, shield)

\textbf{Hit Points} 22 (5d8)

\textbf{Move} 9m

\textbf{Saving Throws}: Fortitude +4, Reflexes +0, Will +0

\textbf{Senses} Darkvision 18m

\textbf{Languages} Gnoll

\textbf{Challenge} 1/2 (100 XP)

\textit{\textbf{Rage.}} When the gnoll reduces a creature to 0 Hit Points with a melee attack during its round, it can take a bonus action to move up to half its movement and make an attack of bite.

\textbf{Actions}

\textit{\textbf{Bite.} Melee Weapon Attack}: +4 to hit, reach 3 ft., one creature.

\textit{Hit:} 4 (1d4 + 2) piercing damage, Disease Gnoll Rage

\textit{Gnoll Rage:} 1 day, save Fortitude DC 13, 12 hour, 1 success, -2 Wisdom

\textit{\textbf{Spear.} Melee or Ranged weapon attack}: +4 to hit, reach 1m or range 6m, one target.

\textit{Hit:} 5 (1d6 + 2) piercing damage, or 6 (1d8 + 2) piercing damage when used with two hands to make a melee attack.

\textit{\textbf{Longbow.} Ranged weapon attack}: +3 to hit, range 45m, one target.

\textit{Hit:} 5 (1d8 + 1) piercing damage.

\textbf{Ecology}\\
Environment: Hot plains, deserts\\
Organization: Solitary, pair, hunting party (2-5 and 1-2 Hyenas), band (10-100 adults plus 50\% young noncombatants, 1 3rd level sergeant for every 20 adults, 1 leader of 4 th-6th level and 5-8 Hyenas) or tribe (20-200 plus 1 3rd level sergeant for every 20 adults, 1 or 2 4th or 5th level lieutenants, 1 6th-8th level chief, 7-12 hyenas and 4-7 hyaenodonts)\\
\textbf{Treasure}: NPC gear (Leather Armour, Heavy Wooden Shield, Spear, other treasure)\\
\textbf{Description}\\
Gnolls are a race of large, hulking humanoids that resemble hyenas in more than just appearance; they show an evident affinity with these scavenger animals, so much so that they are kept as pets, and they reflect many of the behaviors of these animals.

Gnolls are skilled hunters, but they much prefer scavenging or stealing a carcass to hunting prey. This laziness drives them to obtain slaves of any available species to force them to dig lairs, gather supplies and water, and even hunt for their gnoll masters.

Other creatures than hyenas or gnolls become food or slaves, depending on the temperament of the tribe. Even a dead or fallen companion becomes a fresh meal for a gnoll, who may honor a famous member of the tribe with a short prayer or cook one who died of a devastating disease whole; other creature. More "civilized" gnolls do not eat their captives: instead, they keep them as slaves, to defend or improve their lair or to exchange them with other slave-holding tribes or bands.

Gnolls enjoy fighting, but only when outnumbered. In other situations, they prefer to avoid combat except as a means of obtaining a carcass from another hunter, or as an ingenious ambush to bring down a large meal. These hyenamen see no value in bravery or heroism and instead prefer to flee once it is clear that victory is not attainable, arguing that it is better to run away with your tail between your legs than to lose it altogether.

During combat, gnolls use an odd combination of herd tactics and individual strategies. If a gnoll is certain of victory, it attempts to knock down the weaker opponent rather than aid its comrades. If the gnolls are in trouble, they gang up on a powerful opponent and attempt to eliminate him, hoping to force his allies to flee.

Gnoll leaders have ranger skills, but it's not impossible to find gnolls devoted to some ravenous Patron, too. They rarely master magic effectively.


\

\textbf{Goat}\index[Monsters]{Goat}

\textit{Medium beast, unaligned}

\textbf{STRENGTH} +1

\textbf{DEXTERITY} +0

\textbf{CONSTITUTION} +0

\textbf{INTELLIGENCE} -4

\textbf{WISDOM} +0

\textbf{CHARISMA} -3

\textbf{Initiative} +0 -- \textbf{Defence} 11

\textbf{Hit Points} 4 (1d8)

\textbf{Move} 12m

\textbf{Saving Throws}: Fortitude +1, Reflexes +1, Will +0

\textbf{Languages} -

\textbf{Challenge} 0 (10 XP)

\textit{\textbf{Charge.}} If the Fase Dog moves at least 6 meters directly towards the target and hits with a bill attack during the same turn, the target takes an additional 2 (1d4) bludgeoning damage. If the target is a creature, it must succeed on a DC 10 Fortitude save
or fall prone.

\textit{\textbf{Steady Feet.}} The Fase Dog has +1d6 on Fortitude and Reflex saves made against effects that would knock it prone.

\textbf{Actions}

\textit{\textbf{Beak.} Melee Weapon Attack}: +3 to hit, reach 1m, one target.

\textit{Hit:} 3 (1d4 + 1) bludgeoning damage.

\

\index[Monsters]{Goblin}\textbf{Goblin}

\textit{Small humanoid (goblinoid), chaotic evil}

\textbf{STRENGTH} +0

\textbf{DEXTERITY} +0

\textbf{CONSTITUTION} +1

\textbf{INTELLIGENCE} -1

\textbf{WISDOM} -2

\textbf{CHARISMA} -1

\textbf{Initiative} +0 -- \textbf{Defence} 13

\textbf{Hit Points} 7 (2d6 + 1)

\textbf{Move} 9m

\textbf{Saving Throws}: Fortitude +1, Reflexes +1, Will -1

\textbf{Senses} Darkvision 18m

\textbf{Languages} Common, Goblin

\textbf{Challenge} 1/4 (50 XP)

\textbf{Actions}

\textit{\textbf{Short sword.} Melee weapon attack}: +1 to hit, reach 1m, one target.

\textit{Hit:} 4 (1d6 + 1) slashing damage

\textit{\textbf{Shortbow.} Ranged weapon attack}: +1 to hit, range 15m, one target.

\textit{Hit:} 3 (1d6) piercing damage.

\textbf{Ecology}\\
Environment Any Temperate\\
Organization: Group (4-9), warband (10-24), or tribe (50+ plus 50\% noncombatants\\
\textbf{Treasure}: 1d4 pieces of silver\\
\textbf{Description}\\
Goblins are wild, unpredictable, noisy.
Goblins prefer to live in caves, deep forests, and abandoned ancient structures when available. Goblins don't like to build but rather destroy and then complain that nothing is useful.

Goblins are highly superstitious, and view magic with a mixture of awe and fear. Everything they don't understand is magic to them and this leads them to be extremely suspicious of everything and to destroy everything, since what they don't understand must be destroyed.

Goblins are ravenous and can eat huge quantities of food. a goblin doesn't give up eating anything except maybe salad..


\

\index[Monsters]{Golem, Clay}\textbf{Clay Golem}

\textit{Large construct, unaligned}

\textbf{STRENGTH} +5

\textbf{DEXTERITY} -1

\textbf{CONSTITUTION} +4

\textbf{INTELLIGENCE} -4

\textbf{WISDOM} -1

\textbf{CHARISMA} -5

\textbf{Initiative} -1 -- \textbf{Defence} 19

\textbf{Hit Points} 133 (14d10 + 56)

\textbf{Move} 6m

\textbf{Saving Throws}: Fortitude +4, Reflexes +3, Will +4

\textbf{Damage Immunity} acid, poison; from non-magical weapons or that are not adamantium

\textbf{Condition Immunity} charmed, poisoned, paralyzed, petrified, fatigued, frightened

\textbf{Senses} Darkvision 18m

\textbf{Languages} understands its creator's languages but cannot speak

\textbf{Challenge} 9 (5000 XP)

\textit{\textbf{Berserk.}} Whenever the golem begins its round with 60 Hit Points or fewer, roll a d6. On a 6, the golem goes berserk. During each of its rounds while berserk, the golem attacks the closest creature it can see. If there is no creature close enough to move and attack it, the golem attacks one object, preferring objects smaller than itself. Once the golem is berserk, it will continue to be so until it is destroyed or it regains all of its Hit Points.

\textit{\textbf{Magic Weapons.}} The golem's weapon attacks are magical.

\textit{\textbf{Acid Absorption.}} Whenever the golem is the victim of acid damage, it takes no damage but instead regains an equal number of Hit Points.

\textit{\textbf{Form Unchangeable.}} The golem is immune to any spell or effect that would alter its form.

\textit{\textbf{Nature of construct.}} A golem needs no air, food, drink, or sleep.

\textit{\textbf{Resistance to Magic.}} The golem has +1d6 on Saving Throws against spells and other magical effects.

\textbf{Actions}

\textit{\textbf{Multiattack.}} The golem makes two slam attacks or one Cursed Fist

\textit{\textbf{Slam.} Melee Weapon Attack}: +18 to hit, reach 1m, one target.

\textit{Hit:} 16 (2d10 + 5) bludgeoning damage.
.
\textit{\textbf{Cursed Fist.}: Natural Weapon Attack}: +20 to hit, range 1m, one target

\textit{Hit:} 16 (2d6 + 5) bludgeoning damage. Cursed fist wounds heal at a rate of 1 hit point per day. Magical cures, spells or potions, heal 1 hit point per die + any fixed value.


\textit{\textbf{Haste (Cooldown 5-6).}} Until the end of its next round, the golem gains a +2 magical bonus to Defence, has +1d6 on Reflex saves, and can use slam attacks as a bonus action.

\textbf{Ecology}\\
Environment: Any\\
Organization: Solitary or group (2-4)\\
\textbf{Treasure}: None\\
\textbf{Description}\\
Clay golems wear no clothing, except for a garment of treated leather or metal around the hips. On average they are more than 2.4 meters tall and weigh 300 kilos.

\textbf{Construction}
A clay golem can be sculpted from a single block of clay weighing at least 500kg, treated with 1,500 gp worth of rare powders and oils.


\

\index[Monsters]{Golem, Flesh}\textbf{Flesh Golem}

\textit{Medium Construct, Neutral}

\textbf{STRENGTH} +4

\textbf{DEXTERITY} -1

\textbf{CONSTITUTION} +4

\textbf{INTELLIGENCE} -2

\textbf{WISDOM} +0

\textbf{CHARISMA} -3

\textbf{Initiative} -1 -- \textbf{Defence} 12

\textbf{Hit Points} 93 (11d8 + 44)

\textbf{Move} 9m

\textbf{Saving Throws}: Fortitude +8, Reflexes +2, Will +3

\textbf{Damage Immunity} Electricity, poison; from non-magical weapons or that are not adamantium

\textbf{Condition Immunity} charmed, poisoned, paralyzed, petrified, fatigued, frightened

\textbf{Senses} Darkvision 18m

\textbf{Languages} understands its creator's languages but cannot
speak

\textbf{Challenge} 5 (1800 XP)

\textit{\textbf{Berserk.}} Whenever the golem begins its round with 40 Hit Points or less, roll a d6. On a 6, the golem goes berserk. During each of its rounds while berserk, the golem attacks the closest creature it can see. If there is no creature close enough to move and attack it, the golem attacks one object, preferring objects smaller than itself. Once the golem is berserk, it will continue to be so until it is destroyed or it regains all of its Hit Points.

\textit{\textbf{Magic Weapons.}} The golem's weapon attacks are magical.

\textit{\textbf{Electricity Absorption.}} Whenever the golem is the victim of Electricity damage, it takes no damage but instead regains an equal number of Hit Points.

\textit{\textbf{Aversion to fire.}} If the golem takes fire damage, it has -1d6 on attack rolls and proficiency checks until the end of its next round.

\textit{\textbf{Form Unchangeable.}} The golem is immune to any spell or effect that would alter its form.

\textit{\textbf{Nature of construct.}} A golem needs no air, food, drink, or sleep.

\textit{\textbf{Resistance to Magic.}} The golem has +1d6 on Saving Throws against spells and other magical effects.

\textbf{Actions}

\textit{\textbf{Multiattack.}} The golem makes two slam attacks.

\textit{\textbf{Slam.} Melee Weapon Attack}: +11 to hit, reach 1m, one target.

\textit{Hit:} 13 (2d8 + 4) bludgeoning damage. The affected creature must make a DC 15 Fortitude save or become diseased. Each time that fails the saving throw, it takes one less action the next round. If it comes to lose 3 Actions, or fails the Saving Throw 3 times in a row, the creature dies. As soon as the saving throw succeeds, the disease is eradicated.


\textit{\textbf{Enraged}}: the flesh golem overloads. For 2d4 rounds, he can perform an additional Move or Attack Action. Cost 1 Actions.

\textbf{Ecology}\\
Environment: Any\\
Organization: Solitary or group (2-4)\\
\textbf{Treasure}: None\\
\textbf{Description}\\
A flesh golem is a monstrous collection of purloined humanoid body parts and stitched together. Its cadaverous flesh has a pale green or yellowish hue. A flesh golem wears any type of clothing its creator desires, usually just a ragged pair of pants. It has no Equipment and no weapons. A flesh golem stands more than 8 feet tall and weighs 250kg.

A flesh golem does not speak, although it can make a hoarse growl. Walks and moves with a jerky gait, as if not in full control of his body.

While many flesh golems lack reason, exceptional golems are said to have somehow retained memories of their previous life. The head (and therefore the brain) of these flesh golems must be the right combination of freshness and (in the previous life) decisiveness, but luck and chance also seem to be of utmost importance in order to preserve the intellect during their creation . Of course, those who build flesh golems prefer mindless slaves to slaves with a will of their own, so intelligent flesh golems are rare.


\

\index[Monsters]{Golem, Iron}\textbf{Iron Golem}

\textit{Large construct, unaligned}

\textbf{STRENGTH} +7

\textbf{DEXTERITY} -1

\textbf{CONSTITUTION} +5

\textbf{INTELLIGENCE} -4

\textbf{WISDOM} +0

\textbf{CHARISMA} -5

\textbf{Initiative} -1 -- \textbf{Defence} 28

\textbf{Hit Points} 210 (20d10 + 100)

\textbf{Move} 9m

\textbf{Saving Throws}: Fortitude +21, Reflexes +15, Will +16

\textbf{Damage Immunity} Fire, poison; from non-magical weapons or that are not adamantium

\textbf{Condition Immunity} charmed, poisoned, paralyzed, petrified, fatigued, frightened

\textbf{Senses} darkvision 40m

\textbf{Languages} understands its creator's languages but cannot speak

\textbf{Challenge} 16 (15000 XP)

\textit{\textbf{Magic Weapons.}} The golem's weapon attacks are magical.

\textit{\textbf{Absorb Fire.}} Whenever the golem is the victim of fire damage, it takes no damage but instead regains an equal number of Hit Points.

\textit{\textbf{Immutable Form.}} The golem is immune to any spell or effect that would alter its form.

\textit{\textbf{Nature of construct.}} A golem needs no air, food, drink, or sleep.

\textit{\textbf{Resistance to Magic.}} The golem has +1d6 on Saving Throws against spells and other magical effects.

\textbf{Actions}

\textit{\textbf{Multiattack.}} The golem makes two melee attacks.

\textit{\textbf{Slam.} Melee Weapon Attack}: +30 to hit, reach 1m, one target.

\textit{Hit:} 20 (3d8 + 7) bludgeoning damage.

\textit{\textbf{Sword.} Melee Weapon Attack}: +30 to hit, reach 3m, one target.

\textit{Hit:} 23 (3d10 + 7) slashing damage.

\textit{\textbf{Venom Breath (Cooldown 6).}} The golem exhales a poisonous gas in a 5-meter cone. Each creature in that area must make a DC 19 Fortitude save, taking 45 (10d8) poison damage on a failed save, or half as much damage on a successful one.

\textbf{Ecology}\\
Environment: Any\\
Organization: Solitary or group (2-4)\\
\textbf{Treasure}: None\\
\textbf{Description}\\
An iron golem has a humanoid body made of iron. The creator can give it any shape he wishes, but it almost always presents Armour of some sort, whether ceremonial and precious or simple and practical. Compared to a stone golem it has much more defined features. Iron golems sometimes carry a weapon with them, though more often than not they tend to favor their slam attacks.

An iron golem stands 10.5 meter tall and weighs approximately 2500kg. An iron golem cannot speak or utter a voice. Furthermore, it does not emit any recognizable odor.

Though the practice of building iron golems has gradually fallen into disuse, venerable members of some great civilizations of the past considered the ability to forge iron golems of staggering strength and size a point of pride. These golems (of size greater than or equal to Huge), in some remote corners of the world, still exist, and still mechanically carry out orders given to them by long-gone empires.

\textbf{Construction}
To build an iron golem requires 2,500 kg of iron, fused with rare dyes worth at least 10,000 gp.


\

\index[Monsters]{Golem, Stone}\textbf{Stone Golem}

\textit{Large construct, unaligned}

\textbf{STRENGTH} +6

\textbf{DEXTERITY} -1

\textbf{CONSTITUTION} +5

\textbf{INTELLIGENCE} -4

\textbf{WISDOM} +0

\textbf{CHARISMA} -5

\textbf{Initiative} -1 -- \textbf{Defence} 22

\textbf{Hit Points} 178 (17d10 + 85)

\textbf{Move} 9m

\textbf{Saving Throws}: Fortitude +14, Reflexes +9, Will +10

\textbf{Immunity to Damage} Poison; from non-magical weapons or that are not adamantium

\textbf{Condition Immunity} charmed, poisoned, paralyzed, petrified, fatigued, frightened

\textbf{Senses} darkvision 40m

\textbf{Languages} understands its creator's languages but cannot speak

\textbf{Challenge} 10 (5900 XP)

\textit{\textbf{Magic Weapons.}} The golem's weapon attacks are magical.

\textit{\textbf{Immutable Form.}} The golem is immune to any spell or effect that would alter its form.

\textit{\textbf{Nature of construct.}} A golem needs no air, food, drink, or sleep.

\textit{\textbf{Resistance to Magic.}} The golem has +1d6 on Saving Throws against spells and other magical effects.

\textbf{Actions}

\textit{\textbf{Multiattack.}} The golem makes two slam attacks.

\textit{\textbf{Slam.} Melee Weapon Attack}: +21 to hit, reach 1m, one target.

\textit{Hit:} 19 (3d8 + 6) bludgeoning damage.

\textit{\textbf{Slow (Cooldown 5-6).}} The golem targets one or more creatures within 3 meter of it that it can see. Each target must make a DC 18 Will save against this spell. On a failed save, the target can't use reactions, has its speed halved, and can't make more than one attack during its round. Also, during its round, the target can take either an action or a bonus action, but not both. These effects last for 1 minute. The target can repeat the Saving Throw at the end of each of its rounds, ending the effect on itself if successful.

\textbf{Ecology}\\
Environment: Any\\
Organization: Solitary or group (2-4)\\
\textbf{Treasure}: None\\
\textbf{Description}\\
A stone golem has a humanoid body made of stone, often stylized to suit its creator. For example, he may be sculpted to wear Armour, with particular symbols carved into his breastplate, or have designs inlaid into the stone of his limbs. The head is often carved to look like a helmet or the head of some beast. While he can be sculpted with a shield or stone weapon such as a sword, these cosmetic choices do not affect his combat capabilities.

As with most golems, a stone golem cannot speak and makes no sound other than that of stone rubbing against stone as it moves. A stone golem is 9 feet tall and weighs approximately 1000kg.

There are numerous variations of Stone Golems, depending on the materials they are made of but also as expressions of elemental spirits, i.e. it is possible that an elemental spirit inhabits a "rock" (or gem) and defines its appearance and animates it as own body.

\textbf{Construction}
A stone golem's body is carved from a single block of hard stone, such as granite, weighing at least 1500kg. The stone must be of exceptional quality, and cost 5,000 gp.

\

\index[Monsters]{Gorgon}\textbf{Gorgon}

\textit{Large monstrosity, unaligned}

\textbf{STRENGTH} +5

\textbf{DEXTERITY} +0

\textbf{CONSTITUTION} +4

\textbf{INTELLIGENCE} -4

\textbf{WISDOM} +1

\textbf{CHARISMA} -2

\textbf{Initiative} +0 -- \textbf{Defence} 22

\textbf{Hit Points} 114 (12d10 + 48)

\textbf{Move} 12m

\textbf{Saving Throws}: Fortitude +13, Reflexes +6, Will +7

\textbf{Skills} Awareness +4

\textbf{Condition Immunity} Petrified

\textbf{Senses} Darkvision 18m

\textbf{Languages} -

\textbf{Challenge} 5 (1800 XP)

\textit{\textbf{Rrumbling Charge.}} If the gorgon moves at least 6 meters in a straight line toward the target and hits it with a gore attack during the same turn, the target must succeed at a Fortitude save DC 16 or fall prone. If the target is prone, the gorgon can make a hoof attack against it as a bonus action.

\textbf{Actions}

\textit{\textbf{Gored.} Melee Weapon Attack}: +12 to hit, reach 1m, one target.

\textit{Hit:} 18 (2d12 + 5) piercing damage.

\textit{\textbf{Hooves.} Melee Weapon Attack}: +12 to hit, reach 1m, one target.

\textit{Hit:} 16 (2d10 + 5) bludgeoning damage.

\textit{\textbf{Petrifying Breath (Cooldown 5-6).}} The gorgon exhales a petrifying gas in a 10m cone. Each creature in that area must succeed on a DC 15 Fortitude save. On a failed save, the target begins to turn to stone and is restrained. The entangled target must repeat the Saving Throw at the end of its next round. If it succeeds, the effect on the target ends. On a failed save, the target is petrified until freed by the spell \textit{greater restoration} or similar magic.

\textit{\textbf{Enraged}}: the Gorgon concentrates all its petrifying poison in a single breath. She costs 2 actions. A creature within melee range must make a DC 15 Fortitude save or turn to stone for 24 hours.

\textbf{Ecology}\\
Environment: Temperate Plains, Rocky Hills, and Underground\\
Organization: Solitary, pair, herd (3-4) or herd (5-12)\\
\textbf{Treasure}: None\\
\textbf{Description}\\
Gorgons are magical, short-tempered creatures: while they may appear to be constructs at first glance, beneath the artificial-looking metal plates they are made of flesh and blood. Like aggressive bulls, they challenge any unfamiliar creature they encounter, often trampling over their opponent's corpse or shattering its petrified remains until the creature is no longer recognizable. Females are just as dangerous as males, and both sexes look alike. A typical gorgon is 1.8 meters tall and 2.4 meters long. It weighs about 2000 kg.

Gorgons derive their nourishment from consuming minerals, especially the stone of their petrified victims, and any statue they create is completely devoured. They cannot digest metal or gems, so their dung (which looks like acrid-smelling gray powder) often contains small, rough crystals and nuggets of gold. Their aggressiveness towards all other creatures means that there are few, if any, predators and prey in their pastures. Each herd is led by a dominant bull; solitary gorgons are generally adolescent bulls removed from the dominant bull's herd.

Their meat is tough and sinewy (once the Armour is removed), and to those who taste it, quite nutritious. Many tribes of stone giants believe that eating gorgon flesh increases their natural Armour. Pulverized gorgon horns are worth 250 gp as an alternate material component for magic items and spells that affect Strength or Stone.


\

\index[Monsters]{Green Hag}\textbf{Green Hag}

\textit{Medium Fey, Neutral Evil}

\textbf{STRENGTH} +4

\textbf{DEXTERITY} +1

\textbf{CONSTITUTION} +3

\textbf{INTELLIGENCE} +1

\textbf{WISDOM} +2

\textbf{CHARISMA} +2

\textbf{Initiative} +1 -- \textbf{Defence} 19

\textbf{Hit Points} 82 (11d8 + 33)

\textbf{Damage Vulnerability} cold iron

\textbf{Move} 9m

\textbf{Saving Throws}: Fortitude +6, Reflexes +7, Will +7

\textbf{Skills} Arcane +3, Stealth +3, Deceive +4, Awareness +4

\textbf{Senses} Darkvision 18m

\textbf{Languages} Common, Draconic, Sylvan

\textbf{Challenge} 3 (700 XP)

\textit{\textbf{Amphibious.}} The hag can breathe air and water.

\textit{\textbf{Imitation.}} The Hag can mimic animal sounds and humanoid voices. A creature hearing these noises can determine that it is an imitation with a successful DC 14 Wisdom check.

\textit{\textbf{Innate Spells.}} The hag's innate spellcasting ability is Charisma (DC 12 for spell saves). The hag can innately cast the following spells, requiring no material components.

At will: \textit{minor illusion, dancing lights, malevolent taunt}

\textbf{Actions}

\textit{\textbf{Claws.} Melee Weapon Attack}: +6 to hit, reach 1m, one target.

\textit{Hit:} 13 (2d8 + 4) slashing damage, 1 bleed damage.

\textit{\textbf{Illusionary Appearance.}} The hag coats herself and everything she is wearing or carrying in a magical illusion that gives her the appearance of another creature of roughly the same size and humanoid form. The illusion ends if the hag takes a bonus action to end it or if she dies.

Changes wrought by this effect are unable to pass physical inspections. For example, the hag might appear as a smooth-skinned creature, but touch would reveal her rough skin. Otherwise, a creature must take an action to visually inspect the illusion and succeed on a DC 20 Intelligence check to realize that it is a hag in disguise.

\textit{\textbf{Invisible passage.}} The hag can render herself invisible until she attacks or casts a spell, or until she ends her concentration (as if she were concentrating on a spell). While she is invisible, she leaves no physical trace of her passage, so her tracks can only be followed by magic. All equipment she is carrying or wearing becomes invisible with her.

\textbf{Ecology}
Environment: Temperate marshes\\
Organization: Solitary or coven (3 hags of any type)\\
\textbf{Treasure}: Standard\\
\textbf{Description}\\
Terrifying wrinkled Hag who haunt foul swamps and tangled forests, green hags harbor an intense hatred for all that is beautiful and pure. Making use of their various illusory abilities, these crones delight in killing the innocent, upsetting the noble souls, and degrading the pure hearts. They like to use Disguise Himself to take the form of young and attractive girls so as to seduce and snatch young men from their affections and relatives, and to corrupt noble and honest citizens with all sorts of depravities and scandals. Some green hags prefer to reveal their true nature to their loved ones in a carefully engineered moment to drive the man mad with horror and shame. Others prolong their flirtation and go to great lengths to completely ruin the lives of the men they seduce before showing them the truth. Finally, the luckiest of these unfortunates end up being devoured by their lover the green hag: for the unfortunate, the ultimate fate may be much worse, since the green hag's cruel imagination is immense. A typical green hag stands between 1.5 and 1.8 meters tall and weighs just under 80 kg.


\

\index[Monsters]{Grick}\textbf{Grick}

\textit{Medium Monstrosity, Neutral}

\textbf{STRENGTH} +2

\textbf{DEXTERITY} +2

\textbf{CONSTITUTION} +0

\textbf{INTELLIGENCE} -4

\textbf{WISDOM} +2

\textbf{CHARISMA} -3

\textbf{Initiative} +2 -- \textbf{Defence} 15

\textbf{Hit Points} 27 (6d8)

\textbf{Move} 9m, climb 9m

\textbf{Saving Throws}: Fortitude +3, Reflexes +3, Will +2

\textbf{Damage Resistance} from non-magical weapon

\textbf{Senses} Darkvision 18m

\textbf{Languages} -

\textbf{Challenge} 2 (450 XP)

\textit{\textbf{Stone camouflage.}} The grick has +1d6 on Dexterity (Hide) checks made to hide in rocky terrain.

\textbf{Actions}

\textit{\textbf{Multiattack.}} The grick makes an attack with its own tentacles. If the attack hits, the grick can make a peck attack against the same target.

\textit{\textbf{Tentacles.} Melee Weapon Attack}: +4 to hit, reach 1m, one target.

\textit{Hit:} 9 (2d6 + 2) slashing damage.

\textit{\textbf{Beak.} Melee Weapon Attack}: +4 to hit, reach 1m, one target.

\textit{Hit:} 5 (1d6 + 2) piercing damage.

\textbf{Ecology}: \\
Environment any dungeon\\
Organization: Solitary or Cluster (2-5)\\
\textbf{Treasure}: Accidental\\

\textbf{Description}
The vermiform grick is the terror of the caverns and burrows it inhabits, waiting in ambush near busy tunnels or subterranean cities, to leap out of the darkness and capture its prey. It is rare for such prey to be consumed on the spot. The grick prefers to carry fresh food to its lair, a narrow burrow or to the high ledge of a cave, where it can consume it in small bites in peace.

The origins of the grick are unknown. And while it has rudimentary intelligence, it has no society to speak of, and most of the time one encounters individual specimens. On occasions when the ill-fated travelers encounter more than one, the groups of gricks don't seem to communicate or work with each other—each instead attacks individual targets and retreats with its loot as soon as it manages to bring down an opponent. Capable predators, gricks also have a strange, weapon-resistant hide that makes them particularly dangerous. Many inexperienced adventurers have perished under a grick's attack simply because they were unable to harm the creature with their nonmagical weapons. Those familiar with gricks (especially Dwarves, Morlocks, and Troglodytes) know that the best strategy for dealing with them is to retreat and wait for more powerful or magical reinforcements.

Gricks rely on their dark color and ability to climb walls to keep out of sight until they're ready to charge. On more than one occasion when food is scarce in a given region, gricks make their way to the surface and roam the desert in search of prey, but these stays are almost always out of necessity, and eventually the grick quickly find entrances to new underground lairs . They prefer darkness and the comfort of a "roof" over their heads, avoiding the open sky and going to great lengths to stay covered by trees, low clouds, or buildings.


\

\index[Monsters]{Griffin}\textbf{Griffin}

\textit{Large monstrosity, unaligned}

\textbf{STRENGTH} +4

\textbf{DEXTERITY} +2

\textbf{CONSTITUTION} +3

\textbf{INTELLIGENCE} -4

\textbf{WISDOM} +1

\textbf{CHARISMA} -1

\textbf{Initiative} +2 -- \textbf{Defence} 13

\textbf{Hit Points} 59 (7d10 + 21)

\textbf{Move} 9m, fly 24m

\textbf{Saving Throws}: Fortitude +7, Reflexes +6, Will +4

\textbf{Skills} Awareness +5

\textbf{Senses} Darkvision 18m

\textbf{Languages} -

\textbf{Challenge} 2 (450 XP)

\textit{\textbf{Enhanced Sight.}} The griffin has +1d6 on Wisdom (Awareness) checks based on sight.

\textbf{Actions}

\textit{\textbf{Multiattack.}} The griffin makes two attacks: one with its beak and one with its claws.

\textit{\textbf{Claws.} Melee Weapon Attack}: +7 to hit, reach 1m, one target.

\textit{Hit:} 11 (2d6 + 4) slashing damage, 1 bleed damage.

\textit{\textbf{Beak.} Melee Weapon Attack}: +7 to hit, reach 1m, one target.

\textit{Hit:} 8 (1d8 + 4) piercing damage.

\textbf{Ecology}\\
Environment: Temperate Hills\\
Organization: Solitary, Pair, or Pack (6-10)\\
\textbf{Treasure}: Accidental\\
\textbf{Description}\\
Griffon vultures are powerful aerial predators, swooping down from their towering nests to snatch their prey with their beaks and claws. Aggressive and territorial, they are no mere beasts, but cunning fighters and loyal companions to those who earn their respect, fighting to the death to protect their friends and kin.

Weighing over 250 kg and 2.4 meters long, from its pointed beak to its crested tail, the griffin has an imposing profile that has long been used in heraldry and other iconography as a symbol of power, authority and justice. In reality, the griffon is less interested in abstract concepts and more in hunting for food and Defence. While they can sometimes be trained or befriended to serve as mounts, griffons lack an innate affinity for humanoids, and often enter bloody conflicts with civilized races in an attempt to obtain their favorite food: horsemeat. Townsfolk may marvel at a trained griffon's stately style and 21m wingspan, but those farmers forced to share land with its kind know it pays to hurry home and secure their flocks when they hear the hunting cries of the beasts.

Griffons mate for life, and often seek revenge for the killing of a mate or offspring for years. It was precisely this innate stubbornness and fierce loyalty that brought them into domestic use as mounts and guardians of treasures. Despite the inherent danger, the trade in captured griffons and stolen eggs is brisk, with eggs worth up to 2,000 gp each and live young up to 3,000. However, characters desiring a griffon mount should know that buying or forcibly taming intelligent creatures such as griffins is deemed slavery by most good deities, and earning a griffin's unwilling loyalty is no easy task. Reaching mutual agreement (or even friendship) is a much more elegant and safer route to securing a griffin mount.

Before a griffin can be ridden into combat, it must practice carrying the weight of its rider. To be well trained, a griffon must first have a friendly demeanor toward its handler (at a Handle Animal, Diplomacy, or Intimidate check). After that, 6 weeks of practice and a successful DC 20 Handle Animal check is enough for the beast to be comfortable with the load, and due to their intelligence, trained griffons can be assumed to know all the tricks listed in the description of the load. ability Handle Animals, and it is also possible that they learn new commands, issuing simple requests in Common.

Griffins can carry up to 150kg as a light load, 300kg as a medium load and 450kg as a heavy load. An exotic saddle is required to ride a griffin.


\

\index[Monsters]{Grimlock}\textbf{Grimlock}

\textit{Medium humanoid (grimlock), neutral evil}

\textbf{STRENGTH} +3

\textbf{DEXTERITY} +1

\textbf{CONSTITUTION} +1

\textbf{INTELLIGENCE} -1

\textbf{WISDOM} -1

\textbf{CHARISMA} -2

\textbf{Initiative} +1 -- \textbf{Defence} 12

\textbf{Hit Points} 11 (2d8 + 2)

\textbf{Move} 9m

\textbf{Saving Throws}: Fortitude +3, Reflexes +1, Will +0

\textbf{Skills} Acrobatics +5, Stealth +3, Awareness +3

\textbf{Condition Immunity} blinded

\textbf{Senses} blindsight 10m, or 3m if deafened (blind beyond this range)

\textbf{Languages} Language of the Depths

\textbf{Challenge} 1/4 (50 XP)

\textit{\textbf{Stone camouflage.}} The grimlock has +1d6 on Dexterity (Hide) checks made to hide in rocky terrain.

\textit{\textbf{Blindsense.}} The grimlock cannot use blindsight while deafened and can no longer sniff.

\textit{\textbf{Honed sense of smell and hearing.}} The grimlock has +1d6 on Wisdom (Awareness) checks based on hearing or smell.

\textbf{Actions}

\textit{\textbf{Pointed bonecud.} Melee weapon attack}: +5 to hit, reach 1m, one target.

\textit{Hit:} 5 (1d4 + 3) bludgeoning damage plus 2 (1d4) piercing damage.

\textit{\textbf{Longbow.} Ranged weapon attack}: +3 to hit, range 45m, one target.

\textit{Hit:} 5 (1d8 + 1) piercing damage.

\textbf{Ecology}\\
Grimlocks inhabit the abandoned settlements of other races and are often found enslaved by other more organized creatures, such as duergar and elves. They are believed to be an even more degenerate offshoot of the Morlocks, who travel from Sekamina to hunt Grimlocks for food and consider their flesh a delicacy.\\
\textbf{Description}\\
Grimlocks are blind, savage humans who inhabit the realm of the deep darklands, where they organize into small tribal groups.

\

\index[Monsters]{Guardian Protector}\textbf{Guardian Protector}

\textit{Large construct, unaligned}

\textbf{STRENGTH} +4

\textbf{DEXTERITY} -1

\textbf{CONSTITUTION} +4

\textbf{INTELLIGENCE} -2

\textbf{WISDOM} +0

\textbf{CHARISMA} -4

\textbf{Initiative} -1 -- \textbf{Defence} 21

\textbf{Hit Points} 142 (15d10 + 60)

\textbf{Move} 9m

\textbf{Saving Throws}: Fortitude +6, Reflexes +1, Will +2

\textbf{Immunity to Damage} Poison

\textbf{Condition Immunity} charmed, poisoned, paralyzed, fatigued, frightened

\textbf{Senses} Darkvision 18m, blindsight 3m

\textbf{Languages} understands commands given in any language but cannot speak

\textbf{Challenge} 7 (2900 XP)

\textit{\textbf{Accumulating Enchantments.}} A spellcaster wearing the protective guardian's amulet can cause the guardian to accumulate a spell of 4th level or lower. To do so, the caster must cast the spell on the guardian. The spell has no effect but is stored within the guardian. When commanded to do so by the wearer of the amulet or a situation predetermined by the caster arises, the guardian casts the accumulated spell with all parameters set by the original caster, requiring no components. When the spell is cast or any new spells are accumulated, all previously accumulated spells are lost.

\textit{\textbf{Nature of construct.}} The guardian does not need air, food, drink, or sleep.

\textit{\textbf{Regeneration.}} The shield guardian regains 10 Hit Points at the start of its round if it still has at least 1.

\textit{\textbf{Bound.}} The shield guardian is magically bound to an amulet. As long as the guardian and the amulet are on the same plane of existence, the wearer of the amulet can telepathically summon the guardian to come to him, and the guardian will know the distance and direction the amulet is. If the guardian is within 20 meters of the wearer, half the damage the wearer takes (rounded down) is transferred to the guardian. If the amulet is destroyed, the guardian is incapacitated until a replacement amulet is created. The guardian amulet can be subject to direct attack if it is not worn or carried by anyone. It has Defence 10, 10 Hit Points, and immunity to poison damage. Crafting an amulet takes 1 week and costs 10,000 gp in components.

\textbf{Actions}

\textit{\textbf{Multiattack.}} The golem makes two punch attacks.

\textit{\textbf{Punch.} Melee Weapon Attack}: +15 to hit, reach 1m, one target.

\textit{Hit:} 11 (2d6 + 4) bludgeoning damage.

\textbf{Reactions}

\textit{\textbf{Shield.}} When a creature attacks the wearer of the guardian's amulet, the guardian grants a +2 bonus to its Defence, if within 1 meter of its controller.

\

\index[Monsters]{Hag, Night}\textbf{Night Hag}

\textit{Medium fiend, neutral evil}

\textbf{STRENGTH} +4

\textbf{DEXTERITY} +2

\textbf{CONSTITUTION} +3

\textbf{INTELLIGENCE} +3

\textbf{WISDOM} +2

\textbf{CHARISMA} +3

\textbf{Initiative} +3 -- \textbf{Defence} 20

\textbf{Hit Points} 112 (15d8 + 45)

\textbf{Move} 9m

\textbf{Saving Throws}: Fortitude +14, Reflexes +8, Will +11

\textbf{Skills} Stealth +6, Deceive +7, Sense Emotions +6, Awareness +6,

\textbf{Damage Resistances} cold, fire; from a non-magical weapon or are not silvered

\textbf{Senses} darkvision 36m

\textbf{Languages} Abyssal, Common, Infernal, Druidic

\textbf{Challenge} 5 (1800 XP)

\textit{\textbf{Innate Spells.}} The hag's innate spellcasting ability is Charisma (DC 14 on spell saves, +6 to hit on spell attacks). The hag can innately cast the following spells, without needing to
material components.

At will: \textit{Arcane Dart, detect magic} 2/day each: \textit{ray of fatigue, sleep, shift} \textit{planar} (personal)

\textit{\textbf{Resistance to Magic.}} The hag has +1d6 on Saving Throws against spells and other magical effects.

\textbf{Actions}

\textit{\textbf{Claws (In hag Form Only).} Melee Weapon Attack}: +10 to hit, reach 1m, one target.

\textit{Hit:} 13 (2d8 + 4) slashing damage, 1 bleed damage.

\textit{\textbf{Ethereal Form.}} The hag magically enters the Ethereal Plane from the Material Plane, and vice versa. To do so, she must possess a \textit{heart of stone}.

\textit{\textbf{Haunting Nightmares (1 / Day).}} While on the Ethereal Plane, the hag magically contacts a sleeping humanoid on the Material Plane. The spell \textit{protection from good and evil} cast on the target prevents this contact, as does \textit{magic circle}. As long as the contact persists, the target suffers from horrific visions. If these visions last for at least 1 hour, the target gains no benefit from its rest, and its maximum Hit Points are reduced by 5 (1d10). If this effect reduces the target's maximum Hit Points to 0, the target dies, and if the target was evil, its soul becomes trapped in the hag's \textit{bag} \textit{of souls}. The reduction to the target's maximum Hit Points remains until removed by the \textit{restoration} \textit{greater} spell or similar magic.

\textit{\textbf{Shapeshift.}} The hag can magically transform into a Small or Medium humanoid female, or revert to her true form. Her stats are the same in any form. Any equipment she was carrying or wearing is not transformed. Upon death, she reverts to her true form of herself.



\

\index[Monsters]{Hag, Sea}\textbf{Sea Hag}

\textit{Medium Fey, Chaotic Evil}

\textbf{STRENGTH} +3

\textbf{DEXTERITY} +1

\textbf{CONSTITUTION} +3

\textbf{INTELLIGENCE} +1

\textbf{WISDOM} +1

\textbf{CHARISMA} +1

\textbf{Initiative} +1 -- \textbf{Defence} 15

\textbf{Hit Points} 52 (7d8 + 21)

\textbf{Damage Vulnerability} cold iron

\textbf{Movement} 9m, swim 12m

\textbf{Saving Throws}: Fortitude +5, Reflexes +7, Will +5

\textbf{Senses} Darkvision 18m

\textbf{Languages} Aquan, Common, Giant

\textbf{Challenge} 2 (450 XP)

\textit{\textbf{Amphibious.}} The hag can breathe air and water.

\textit{\textbf{Horrific Appearance.}} Any humanoid who begins its round within 10 meters of the hag and can see her true form must make a DC 11 Will save. On a failed save, the creature be scared for 1 minute. A creature can repeat the save at the end of each of its rounds, with a -1d6 if the hag is in line of sight, ending the effect on a successful save. If the creature's Saving Throw succeeds or the effect ends on it, the creature is immune to the horrific appearance for the next 24 hours.

Unless the target is surprised or the revelation of the hag's true form is sudden, the target can look away and avoid making the initial Saving Throw. Until the start of its next round, a creature that looks away she has -1d6 on attack rolls against the crone.

\textbf{Actions}

\textit{\textbf{Claws.} Melee Weapon Attack}: +5 to hit, reach 1m, one target.

\textit{Hit:} 10 (2d6 + 3) slashing damage, 1 bleed damage.

\textit{\textbf{Illusionary Appearance.}} The hag coats herself and everything she is wearing or carrying in a magical illusion that gives her the appearance of a loathsome creature of roughly the same size and shape humanoid. The illusion ends if the hag takes a bonus action to end it or if she dies.

Changes wrought by this effect are unable to pass physical inspections. For example, the hag might appear as a creature without claws, but a person in contact with her hands would feel them. Otherwise, a creature must take an action to visually inspect the illusion and succeed on a DC 16 Intelligence check to realize that the hag has disguised herself.

\textit{\textbf{Death Gaze.}} The hag targets one frightened creature visible within 10 meters of her. If the target can see the hag, she must succeed at a DC 11 Will save against this spell or drop to 0 Hit Points.

\textbf{Ecology}\\
Environment any aquatic\\
Organization: solitary or coven (3 hags of any species)\\
\textbf{Treasure}: standard\\
\textbf{Description}\\
These treacherous and monstrous hags possess terrifying traits which few dare gaze upon, take pleasure in the discord and death of sailors, and torment seafarers with inescapable calamities. Sea hags are always fearsome in appearance, and despite their ravenous nature, they are usually emaciated creatures that appear to be on the verge of starvation. They are 1.8 meters tall and weigh 75 kg.

Sea hags prefer to live close to shore where fishing and freighters are more common, and away from urban areas so that their actions don't draw too much attention from potential enemies, although it is not uncommon for a brave sea hag or greedy woman settles in a port city or at the mouth of a deep river.

Sea hags form covens similar to those of other hags, but their aquatic nature generally prompts them to refrain from forming mixed covens. Where a green hag dwells along the coast (often in a salt marsh or coastal marsh), a coven is made up of two sea hags who respect the green hag as mother and leader. Most commonly, a coven of sea hags consists of a group of sea hags who are particularly close and friendly.


\

\index[Monsters]{Harpy}\textbf{Harpy}

\textit{Medium Monstrosity, Chaotic Evil}

\textbf{STRENGTH} +1

\textbf{DEXTERITY} +1

\textbf{CONSTITUTION} +1

\textbf{INTELLIGENCE} -2

\textbf{WISDOM} +0

\textbf{CHARISMA} +1

\textbf{Initiative} +1 -- \textbf{Defence} 12

\textbf{Hit Points} 38 (7d8 + 7)

\textbf{Move} 6m, fly 12m

\textbf{Saving Throws}: Fortitude +2, Reflexes +2, Will +1

\textbf{Languages} Common

\textbf{Challenge} 1 (200 XP)

\textbf{Actions}

\textit{\textbf{Multiattack.}} The Armour makes two attacks: one with its claws and one with its club.

\textit{\textbf{Claws.} Melee Weapon Attack}: +3 to hit, reach 1m, one target.

\textit{Hit:} 5 (2d4 + 1) slashing damage, 1 bleed damage.

\textit{\textbf{Cudgeon.} Melee Weapon Attack}: +3 to hit, reach 1m, one target.

\textit{Hit:} 3 (1d4 + 1) bludgeoning damage.

\textit{\textbf{Song of Charm.}} The harpy sings a magical tune. Each humanoid and giant within 100 meters of the harpy who can hear the song must succeed at a DC 11 Will save or be charmed until the song ends. The harpy must take a bonus action on her next round to continue singing. He can stop singing at any time. The chant ends if the harpy is incapacitated.

While charmed by the harpy, a target is incapacitated and ignores the songs of other harpies. If the charmed target is more than 1 meter from the harpy, the target must move during its round to approach the harpy by the most direct route. Before moving into dangerous terrain, such as lava or a pit, and before taking damage from any source other than the harpy, the target can repeat the Saving Throw. A creature can repeat the Saving Throw at the end of each of its rounds. If the save succeeds, the effect ends for that target.

A target that succeeds at the Saving Throw is immune to that harpy's song for the next 24 hours.

\textbf{Ecology}\\
Environment: Temperate Marshes\\
Organization: Solitary, pair, or flock (3-12)\\
\textbf{Treasure}: Standard (Leather Armour, Spiked Mace, and other treasure)\\
\textbf{Description}\\
Often viewed as evil and corrupt creatures, harpies know how others think and act. This perceptual ability gives them an edge in finding their favorite meals. While wild creatures fall easily prey to the bewitching song, these wicked bird-women prefer meals laced with complex sentient thoughts. Easy prey makes the meal boring.

While ultimately savage and without any remorse for their actions, many harpies live among humanoid societies and enjoy exploiting the creatures they deem potential meals.

Harpies tend to wear trinkets and trinkets stolen from their victims, for they like to indulge in the brilliant adornments of men. Up close, these creatures exude the stench of their devoured victims, and they rarely let unenthralled creatures get too close so they don't smell the blood and rot on their feathers. For this reason, many harpies bathe themselves in perfumes and aromatic oils.

Harpies are markedly different depending on the region in which they live. Some resemble a mixture of vultures and women, while others bear the regal features of hawks and falcons on their feathers. Rare broods of harpies, in isolated, tropical places around the world, also have colorful feathers like parrots.

\

\index[Monsters]{Hellhound}\textbf{Hellhound}

\textit{Medium fiend, lawful evil}

\textbf{STRENGTH} +3

\textbf{DEXTERITY} +1

\textbf{CONSTITUTION} +2

\textbf{INTELLIGENCE} -2

\textbf{WISDOM} +1

\textbf{CHARISMA} -2

\textbf{Initiative} +1 -- \textbf{Defence} 17

\textbf{Hit Points} 45 (7d8 + 14)

\textbf{Move} 15m

\textbf{Saving Throws}: Fortitude +6, Reflexes +5, Will +1

\textbf{Skills} Awareness +5

\textbf{Damage Immunity} Fire

\textbf{Senses} Darkvision 18m

\textbf{Languages} understands Infernal but cannot speak

\textbf{Challenge} 3 (700 XP)

\textit{\textbf{Hearing and Fine Smell.}} The hound has +1d6 on Wisdom (Awareness) checks based on hearing or smell.

\textit{\textbf{Packing tactics.}} The hound has +1d6 on attack rolls against a creature if at least one of the hound's allies is within 1 meter of the creature and that ally isn't incapacitated.

\textbf{Actions}

\textit{\textbf{Bite.} Melee Weapon Attack}: +7 to hit, reach 1m, one target.

\textit{Hit:} 7 (1d6 + 3) piercing damage plus 7 (2d6) fire damage.

\textit{\textbf{Fiery Breath (Cooldown 5-6).}} The hound exhales fire in a 5-meter cone. Each creature in that area must make a DC 12 Reflex save, taking 21 (6d6) fire damage on a failed save, or half as much damage on a successful one.



\

\index[Monsters]{Hippogriff}\textbf{Hippogriff}

\textit{Large beast, unaligned}

\textbf{STRENGTH} +3

\textbf{DEXTERITY} +1

\textbf{CONSTITUTION} +1

\textbf{INTELLIGENCE} -4

\textbf{WISDOM} +1

\textbf{CHARISMA} -1

\textbf{Initiative} +1 -- \textbf{Defence} 12

\textbf{Hit Points} 19 (3d10 + 3)

\textbf{Move} 12m, fly 18m

\textbf{Saving Throws}: Fortitude +5, Reflexes +5, Will +2

\textbf{Skills} Awareness +5

\textbf{Languages} -

\textbf{Challenge} 1 (200 XP)

\textit{\textbf{Enhanced Sight.}} The hippogriff has +1d6 on Wisdom (Awareness) checks based on sight.

\textbf{Actions}

\textit{\textbf{Multiattack.}} The hippogriff makes two attacks: one with its beak and one with its claws.

\textit{\textbf{Claws.} Melee Weapon Attack}: +5 to hit, reach 1m, one target.

\textit{Hit:} 10 (2d6 + 3) slashing damage.

\textit{\textbf{Beak.} Melee Weapon Attack}: +5 to hit, reach 1m, one target.

\textit{Hit:} 8 (1d10 + 3) piercing damage.

\textbf{Ecology}\\
Environment Temperate Hills or Plains\\
Organization: Solitary, pair, or flock (7-12)\\
\textbf{Treasure}: None\\
\textbf{Description}\\
The hippogriff has the wings, front legs and head of a large raptor and the tail and body of a magnificent horse. Since horses are the favorite food of griffins, scholars say that a wizard with a sense of humor long ago created this unfortunate fusion of a horse and a falcon as a joke.

The hippogriff's feathers are similar in coloration to that of a hawk or eagle; however, some breeders have managed to produce specimens with completely white or charcoal colored feathers. A hippogriff's torso and hindquarters are most often bay, nut, or gray in color, with some coats displaying piebald or even palomino coloration. A hippogriff is 3.3 meters long and weighs up to 680 kg.

Territorial hippogriffs fiercely guard their domain. Hippogriffs also have to watch the skies for other predators, as they are a favorite food of griffons, wyverns, and young dragons. Hippogriffs nest in the vast grassy meadows, rugged hills and flowing prairies. Exceptionally hardy hippogriffs make their homes within niches or canyon walls, from which they scour the rocky deserts for coyotes, deer, and sometimes humanoids. Hippogriffs prefer mammals, however they do graze after any meat meal to aid in digestion. These dietary habits of theirs can be dangerous to both farmers and their herds, so often farming communities place rewards on them. The victims of these hunting parties are often stuffed, and stuffed hippogriffs frequently decorate frontier taverns and remote outposts.

Far easier to train than griffins and as intelligent as horses, hippogriffs are trained as stud animals by a few elite companies of mounted soldiers, who patrol the skies and swoop down on unsuspecting enemies. Although they are magical beasts, if hippogriffs are caught young, they can be trained with Handle Animals as if they were animals. An adult hippogriff is much more difficult to train, and training must follow the normal rules for training magical beasts using this skill. A hippogriff saddle must be made in such a way that it doesn't impede the movement of the creature's wings; these saddles are always exotic saddles.

Hippogriffs are oviparous: as a general rule, a hippogriff's nest contains only one egg at a time. The hippogriff egg is worth 200 gp, but a healthy young hippogriff is worth 500 gp. A fully trained hippogriff as a mount can see its value soar to 5,000 gp or more. A hippogriff can carry 90 kg as a light load, 180 kg as a medium load, and 270 kg as a heavy load.


\

\index[Monsters]{Hobgoblin}\textbf{Hobgoblin}

\textit{Medium humanoid (goblinoid), lawful evil}

\textbf{STRENGTH} +1

\textbf{DEXTERITY} +1

\textbf{CONSTITUTION} +1

\textbf{INTELLIGENCE} +0

\textbf{WISDOM} +0

\textbf{CHARISMA} -1

\textbf{Initiative} +1 -- \textbf{Defence} 19 (mail Armour, shield)

\textbf{Hit Points} 11 (2d8 + 2)

\textbf{Move} 9m

\textbf{Saving Throws}: Fortitude +5, Reflexes +2, Will +1

\textbf{Senses} Darkvision 18m

\textbf{Languages} Common, Goblin

\textbf{Challenge} 1/2 (100 XP)

\textit{\textbf{+1d6 Martial.}} Once per turn, the hobgoblin can deal an additional 7 (2d6) damage to a creature it hits with a weapon attack, if that creature is within 1 meter of one ally of the hobgoblin who is not incapacitated.

\textbf{Actions}

\textit{\textbf{Longsword.} Melee Weapon Attack}: +3 to hit, reach 1m, one target.

\textit{Hit:} 5 (1d8 + 1) slashing damage, or 6 (1d10 + 1) slashing damage when used with two hands.

\textit{\textbf{Longbow.} Ranged weapon attack}: +3 to hit, range 45m, one target.

\textit{Hit:} 5 (1d8 + 1) piercing damage.

\textbf{Ecology}\\
Environment: Temperate Hills\\
Organization: Group (4-9), warband (10-24), or tribe (25+ plus 50\% noncombatants, 1 sergeant 3rd level per 20 adults, 1 or 2 lieutenants 4th or 5th 1st level, 1 6th-8th level leader, 6-12 Leopards and 1-4 Ogres or 1-2 Trolls)\\
\textbf{Treasure}: NPC gear (studded leather breastplate, light metal shield, longsword, longbow w/20 arrows, other treasure)\\
\textbf{Description}\\
Hobgoblins are militaristic and prolific, a combination that makes them very dangerous in some regions. They procreate rapidly, replacing fallen members with new soldiers, keeping their numbers constant regardless of the tide of war. Generally it doesn't take much for them to declare war, but in most cases the reason is to capture new Slaves: Slave life in a Hobgoblin lair is brutal and short, and new Slaves are always needed to replace those who die or are eaten .

Of all the Goblinoid Races, the Hobgoblins are by far the most civilized.
They see the larger, more solitary Bugbears as tools to be hired and used where needed, usually for specific missions involving murder and theft, and they view the smaller Goblin species with a mixture of shame and frustration. Hobgoblins admire the tenacity of goblins, although the unpredictable nature and passion for fire of their diminutive kin makes them unwelcome additions to hobbgoblin tribes or settlements. However, most hobgoblin tribes include a small group of goblins, who normally lurk in the worst corners of the settlement.

Many hobgoblin tribes combine a love of war with a keen intellect. The science of siege engines, alchemy, and complex engineering feats fascinate most hobgoblins, and the particularly gifted ones are treated as heroes and always attain high-ranking positions in the tribe. Slaves with fine minds are prized, so raids on Dwarven cities are commonplace.

Hobgoblins have been known to distrust and despise Magic, especially Arcane Magic. Their Shamans are regarded with a mixture of fear and respect, and are usually forced to live alone on the fringes of the tribe's lair. Hobgoblins have never been heard of practicing Arcane Magic or, as the Hobgoblins say, "Elf Magic". This is the cause of their hatred of Magic: Hobgoblins hate Elves.

A Hobgoblin is 1 meter tall and weighs 80 kg.


\

\index[Monsters]{Homunculus}\textbf{Homunculus}

\textit{Tiny construct, neutral}

\textbf{STRENGTH} -3

\textbf{DEXTERITY} +2

\textbf{CONSTITUTION} +0

\textbf{INTELLIGENCE} +0

\textbf{WISDOM} +0

\textbf{CHARISMA} -2

\textbf{Initiative} +2 -- \textbf{Defence} 14

\textbf{Hit Points} 5 (2d4)

\textbf{Move} 6m, fly 12m

\textbf{Saving Throws}: Fortitude +0, Reflexes +4, Will +1

\textbf{Immunity to Damage} Poison

\textbf{Condition Immunity} charmed, poisoned

\textbf{Senses} Darkvision 18m, blindsight 3m

\textbf{Languages} understands its creator's languages but cannot speak

\textbf{Challenge} 0 (10 XP)

\textit{\textbf{Telepathic Bond.}} While the homunculus is on the same plane of existence as its master, it can magically communicate to its master what it senses, and the two can communicate telepathically.

\textbf{Actions}

\textit{\textbf{Bite.} Melee Weapon Attack}: +4 to hit, reach 3 ft., one creature.

\textit{Hit:} 1 piercing damage, and the target must succeed on a DC 10 Fortitude save or be poisoned for 1 minute. On a failed save by 5 or more, the target is instead poisoned for 5 (1d10) minutes and is also knocked unconscious while so poisoned.

\

\textbf{Hunter Shark}\index[Monsters]{Hunter Shark}

A hunter shark is 4 to 6 meters long and usually hunts alone in deeper waters.

\textit{Large beast, unaligned}

\textbf{STRENGTH} +4

\textbf{DEXTERITY} +1

\textbf{INSTITUTION} +2

\textbf{INTELLIGENCE} -5

\textbf{WISDOM} +0

\textbf{CHARISMA} -3

\textbf{Initiative} +1 -- \textbf{Defence} 13

\textbf{Hit Points} 45 (6d10 + 12)

\textbf{Movement} 0m, swim 12m

\textbf{Saving Throws}: Fortitude +4, Reflexes +2, Will +0

\textbf{Skills} Awareness +2

\textbf{Senses} blindsight 9 m

\textbf{Languages} -

\textbf{Challenge} 2 (450 XP)

\textit{\textbf{Blood Frenzy.}} The shark has +1d6 on melee attack rolls against any creature that is not at full Hit Points.

\textit{\textbf{Water Breathing.}} The shark can only breathe underwater.

\textbf{Actions}

\textit{\textbf{Bite.} Melee Weapon Attack}: +6 to hit, reach 1m, one target.

\textit{Hit:} 13 (2d8 + 4) piercing damage.

\

\index[Monsters]{Hydra}\textbf{Hydra}

\textit{Huge monstrosity, unaligned}

\textbf{STRENGTH} +5

\textbf{DEXTERITY} +1

\textbf{CONSTITUTION} +5

\textbf{INTELLIGENCE} -4

\textbf{WISDOM} +0

\textbf{CHARISMA} -2

\textbf{Initiative} +1 -- \textbf{Defence} 19

\textbf{Hit Points} 172 (15d12 + 75)

\textbf{Movement} 9m, swim 9m

\textbf{Saving Throws}: Fortitude +8, Reflexes +7, Will +3

\textbf{Skills} Awareness +6

\textbf{Senses} Darkvision 18m

\textbf{Languages} -

\textbf{Challenge} 8 (3900 XP)

\textit{\textbf{Multiple heads.}} The hydra has five heads. As long as she has more than one head, the hydra has +1d6 on Saving Throws against blinded, charmed, deafened, frightened, stunned, or unconscious conditions.

Whenever the hydra takes 25 or more damage in a single turn, one of its heads dies. If all heads die, the hydra dies as well.

At the end of its round, the hydra grows back two heads for each of its heads killed since its last turn, unless it has taken fire damage since its last turn. The hydra regains 10 Hit Points for each head regrown this way.

\textit{\textbf{Reactive Heads.}} For each possessed head beyond the first, the hydra receives an extra Reaction Action that can only be used to make Awareness checks.

\textit{\textbf{Hold Breath.}} The hydra can hold its breath for 1 hour.

\textit{\textbf{Awake.}} While the hydra sleeps, at least one of its heads remains awake.

\textbf{Actions}

\textit{\textbf{Multiattack.}} The hydra makes as many bite attacks as it has heads.

\textit{\textbf{Bite.} Melee Weapon Attack}: +13 to hit, reach 3m, one target.

\textit{Hit:} 10 (1d10 + 5) piercing damage.

\textbf{Ecology}\\
Environment: Temperate Marshes\\
Organization: Solitary\\
\textbf{Treasure}: Standard\\
\textbf{Description}\\
Hydra is a multi-headed, but stupid dragon.


\

\textbf{Hyena}\index[Monsters]{Hyena}

\textit{Medium beast, unaligned}

\textbf{STRENGTH} +0

\textbf{DEXTERITY} +1

\textbf{CONSTITUTION} +1

\textbf{INTELLIGENCE} -4

\textbf{WISDOM} +1

\textbf{CHARISMA} -3

\textbf{Initiative} +1 -- \textbf{Defence} 12

\textbf{Hit Points} 5 (1d8 + 1)

\textbf{Move} 15m

\textbf{Saving Throws}: Fortitude +5, Reflexes +5, Will +1

\textbf{Skills} Awareness +3

\textbf{Languages} -

\textbf{Challenge} 0 (10 XP)

\textit{\textbf{Packing tactics.}} The hyena has +1d6 on attack rolls against a creature if at least one of the hyena's allies is within 1 meter of the creature and that ally isn't incapacitated.

\textbf{Actions}

\textit{\textbf{Bite.} Melee Weapon Attack}: +2 to hit, reach 1m, one target.

\textit{Hit:} 3 (1d6) piercing damage.

\

\index[Monsters]{Imp}\textbf{Imp}

\textit{Tiny fiend (devil, shapeshifter), lawful evil}

\textbf{STRENGTH} -2

\textbf{DEXTERITY} +3

\textbf{CONSTITUTION} +1

\textbf{INTELLIGENCE} +0

\textbf{WISDOM} +1

\textbf{CHARISMA} +2

\textbf{Initiative} +3 -- \textbf{Defence} 14

\textbf{Hit Points} 10 (3d4 + 3)

\textbf{Movement} 6m, fly 12m (6m in rat form; 6m, fly 18m in crow form; 6m, climb 6m in spider form)

\textbf{Saving Throws} Fortitude +1, Reflexes +6, Will +4

\textbf{Skills} Stealth +5, Deceive +4, Sense Emotions +3

\textbf{Damage Resistances} cold; from a non-magical or non-silver weapon

\textbf{Damage Immunity} Fire, poison

\textbf{Condition Immunity} poisoned

\textbf{Senses} darkvision 40m

\textbf{Languages} Infernal, Common

\textbf{Challenge} 1 (200 XP)

\textit{\textbf{Shapeshift.}} The devil can use his action to transform into a beastly rat, crow, or spider form, or back to his true form. His stats are the same in all forms, although the attacks may vary for some forms. Any equipment he is wearing or carrying is not transformed. Upon death he reverts to his true form.

\textit{\textbf{Resistance to Magic.}} The devil has +1d6 on Saving Throws against spells and other magical effects.

\textit{\textbf{Devil's Sight.}} The devil's darkvision is not limited by magical darkness.

\textbf{Actions}

\textit{\textbf{Sting (Bite in Beast Form).} Melee Weapon Attack}: +5 to hit, reach 3 ft., one creature.

\textit{Hit:} 5 (1d4 + 3) piercing damage, and the target must make a DC 11 Fortitude save, taking 10 (3d6) poison damage on a failed save, or half as much damage on a successful one .

\textit{\textbf{Invisibility.}} The devil remains invisible until he attacks or ends his concentration. Anything the devil is carrying or wearing remains invisible as long as he remains in contact with the devil.

\textbf{Ecology}\\
Environment: Any (Hell)\\
Organization: Solitary, pair, or flock (3-10)\\
\textbf{Treasure}: Standard\\
\textbf{Description}\\
Born directly from the pits of Hell, imps are the least powerful devils, although these cruel and intrusive creatures play an important role in corrupting mortal souls. Free from the hierarchies and duties of infernal armies, imps revel in every opportunity to travel to the Material Plane and to cunningly tempt mortals into ever more depraved acts.

Willingly serving spellcasters as familiars, imps play the part of faithful servants, often offering their masters cunning advice and infernal insights. In reality, imps work to send souls to Hell, ensuring that their master's soul, along with many others, is destined for damnation after death.

Imps vary widely in appearance, spanning a wide spectrum of beastly and grotesque traits, though many are in the form of a reddish-skinned, winged humanoid with bulbous features. The typical imp is only 0.5 meter tall, has a wingspan of 1 meter, and weighs 5kg.

One in every thousand imps has the ability to telepathically communicate with creatures within 15 meters and the power to change their shape into that of a Small or Tiny animal, as if by the effect of a beast shape II spell. These consular imps are highly prized by powerful devils, who send them as servants to their favored followers or to bribe mortal heroes. An imp consular can be summoned with the Improved Familiar feat, but only by a spellcaster of 8th level or higher. Diabolists tell of other races of imps with similarly specialized abilities, but if such creatures actually exist, those are extremely rare.

Unlike other devils, imps often find themselves free and alone in the Material Plane, particularly after they have been summoned to serve as familiars and their masters have died (often, indirectly, through the machinations of the imp itself). Without any means of returning home, these imps, free from all ties to arcane masters, can become dangerous nuisances or even lead small tribes of gory humanoids, such as goblins or kobolds.


\

\index[Monsters]{Invisible Stalker}\textbf{Invisible Stalker}

\textit{Elemental Average, Neutral}

\textbf{STRENGTH} +3

\textbf{DEXTERITY} +4

\textbf{CONSTITUTION} +2

\textbf{INTELLIGENCE} +0

\textbf{WISDOM} +2

\textbf{CHARISMA} +0

\textbf{Initiative} +4 -- \textbf{Defence} 17

\textbf{Hit Points} 104 (16d8 + 32)

\textbf{Move} 15m, fly 15m (float)

\textbf{Saving Throws}: Fortitude +13, Reflexes +11, Will +4

\textbf{Skills} Stealth +10, Awareness +8

\textbf{Damage Resistances} from non-magical weapon

\textbf{Immunity to Damage} Poison

\textbf{Condition Immunity} grabbed, poisoned, restrained, paralyzed, petrified, unconscious, prone, fatigue

\textbf{Senses} Darkvision 18m

\textbf{Languages} Ictum, understands Common but does not speak it

\textbf{Challenge} 6 (2300 XP)

\textit{\textbf{Succeeding Hunter.}} The summoner assigns a quarry to the stalker. The stalker knows the direction and distance to which the prey is as long as both are on the same plane of existence. The stalker also knows the location of his summoner.

\textit{\textbf{Invisibility.}} The stalker is invisible.

\textit{\textbf{Elemental nature.}} An invisible stalker needs no air, food, drink, or sleep.

\textbf{Actions}

\textit{\textbf{Multiattack.}} The stalker makes two slam attacks.

\textit{\textbf{Slam.} Melee Weapon Attack}: +12 to hit, reach 1m, one target.

\textit{Hit:} 10 (2d6 + 3) bludgeoning damage.

\textit{\textbf{Enraged}}: the Invisible Stalker break the pact and leave for air elemental plane.

\textbf{Ecology}
Environment: Any\\
Organization: Solitary\\
\textbf{Treasure}: None\\
\textbf{Description}\\

Originally from the Plane of Air, these creatures move through the world following assignments for those who summon them. Invisible hunters usually act as guardians and assassins. Natural invisibility and stealth prevent them from following their prey unseen and give them an advantage when deciding to eliminate a target.

However, many invisible hunters regard these tasks as meager demands on mortals. If given a particularly complex or unwelcome task, an invisible hunter will try to find a loophole if the instruction is sparsely worded. For example, Wizards who summon an invisible hunter with the instruction "guard me from danger" might be escorted to a distant hidden location, or even taken to the Plane of Air.

Due to constant summoning, many invisible hunters oppose the denizens of the Material Plane. Those newly summoned into the mortal world know only the stories of their kin and maintain an open attitude toward those who call them back. Over time, or if they serve an evil master, they begin to form a negative opinion of these deadly creatures, leading them to deflect instructions and harm their masters. For older and more experienced invisible hunters, the only thing that protects their summoners is the magic that binds them. These creatures always attempt to use inconsistencies in the formulation of their tasks and literal distortions in intention to find a way to annoy, injure, or even kill those who brought them to this plane.


\textbf{Jackal}\index[Monsters]{Jackal}

\textit{Little beast, unaligned}

\textbf{STRENGTH} -1

\textbf{DEXTERITY} +2

\textbf{CONSTITUTION} +0

\textbf{INTELLIGENCE} -4

\textbf{WISDOM} +1

\textbf{CHARISMA} -2

\textbf{Initiative} +2 -- \textbf{Defence} 13

\textbf{Hit Points} 3 (1d6)

\textbf{Move} 12m

\textbf{Saving Throws}: Fortitude -1, Reflexes +3, Will +1

\textbf{Skills} Awareness +3

\textbf{Languages} -

\textbf{Challenge} 0 (10 XP)

\textit{\textbf{Pack tactics.}} The jackal has +1d6 on attack rolls against a creature if at least one of the jackal's allies is within 1 meter of the creature and that ally isn't incapacitated.

\textit{\textbf{Hearing and keen sense of smell.}} The jackal has +1d6 on Wisdom (Awareness) checks based on hearing or smell.

\textbf{Actions}

\textit{\textbf{Bite.} Melee Weapon Attack}: +1 to hit, reach 1m, one target.

\textit{Hit:} 1 (1d4 - 1) piercing damage.

\

\index[Monsters]{Kobold}\textbf{Kobold}

\textit{Small humanoid (kobold), lawful evil}

\textbf{STRENGTH} -2

\textbf{DEXTERITY} +2

\textbf{CONSTITUTION} -1

\textbf{INTELLIGENCE} -1

\textbf{WISDOM} -2

\textbf{CHARISMA} -1

\textbf{Initiative} +2 -- \textbf{Defence} 13

\textbf{Hit Points} 5 (2d6 - 2)

\textbf{Move} 9m

\textbf{Saving Throws}: Fortitude +0, Reflexes +2, Will -1

\textbf{Senses} Darkvision 18m

\textbf{Languages} Common, Draconic

\textbf{Challenge} 1/8 (25 XP)

\textit{\textbf{Sensitivity to Light}}. While in sunlight, the kobold has -1d6 on attack rolls, as well as Wisdom (Awareness) checks based on sight.

\textit{\textbf{Packing tactics.}} The kobold has +1d6 on attack rolls against a creature if at least one of the kobold's allies is within 1 meter of the creature and that ally isn't incapacitated.

\textbf{Actions}

\textit{\textbf{Dagger.} Melee Weapon Attack}: +4 to hit, reach 1m, one target.

\textit{Hit:} 4 (1d4 + 2) piercing damage.

\textit{\textbf{Slingshot.} Ranged weapon attack}: +4 to hit, range 9m, one target.

\textit{Hit:} 4 (1d4 + 2) bludgeoning damage.

\textbf{Ecology}\\
Environment: Temperate forests or underground\\
Organization: Solitary, group (2-4), nest (5-30 plus an equal number of noncombatants, 1 3rd-level sergeant for every 20 adults, and 1 4th-6th level leader), or tribe (31-300 more than 35\% non-combatants, 1 sergeant 3rd level for every 20 adults, 2 lieutenants 4th level, 1 leader 6th-8th level, and 5-16 dire rats)\\
\textbf{Treasure}: NPC gear (leather Armour, spear, sling, other treasure), 2d6 silver coins\\
\textbf{Description}\\
Kobolds are creatures of the dark, most likely to be encountered in huge underground mazes or in the dark corners of forests where the sun never shines. Because of their physical resemblance, kobolds loudly proclaim themselves heirs of draconic bloodline and destined to rule the land under the wing of their great godlike cousins, but most dragons regard them as little more than nuisance insects. But even as they claim divine lineage and the evidence of their destiny, kobolds are aware of their weakness. Cowardly and scheming, they never fight openly if they can help it, instead setting ambushes and traps, burrowing into their warrens behind a blanket of crude but ingenious pitfalls, or falling upon the enemy in vast howling hordes.

The hue of kobolds also varies among siblings of the same brood, ranging from the colors of Tàhil dragons, with red and purple predominating, and more rarely white, green, blue, and black.

Kobolds have a soft spot for silver but being bad miners they prefer to prey on adventurers for silver coins and eat them like butter cookies. Kobolds can digest silver quite quickly and the more they eat the brighter their scales are and the kobolds appear healthy.

\

\index[Monsters]{Kraken}\textbf{Kraken}

\textit{Gargantuan monstrosity (titan), chaotic evil}

\textbf{STRENGTH} +10

\textbf{DEXTERITY} +0

\textbf{CONSTITUTION} +7

\textbf{INTELLIGENCE} +6

\textbf{WISDOM} +4

\textbf{CHARISMA} +5

\textbf{Initiative} +6 -- \textbf{Defence} 30

\textbf{Hit Points} 472 (27x3d6 + 189)

\textbf{Movement} 6m, swim 18m

\textbf{Saving Throws}: Fortitude +30, Reflexes +23, Will +27

\textbf{Damage Immunity} Electricity, weapons +1

\textbf{Condition Immunity} paralyzed, frightened

\textbf{Senses} True Seeing 36 m

\textbf{Languages} understands Abyssal, Celestial, Infernal, and Druidic but cannot speak, telepathy 36m

\textbf{Challenge} 23 (50000 XP)

\textit{\textbf{Amphibious.}} The kraken can breathe air and water.

\textit{\textbf{Freedom of movement.}} The kraken ignores difficult terrain, and magical effects cannot reduce its speed or cause it to become entangled. It can expend 1 meter of movement to break free from nonmagical restraints or from being grappled.

\textit{\textbf{Siege Monster.}} Kraken deals double damage to objects and structures.

\textbf{Actions}

\textit{\textbf{Multiattack.}} The kraken makes three tentacle attacks, each of which can be replaced by one use of Slingshot.

\textit{\textbf{Bite.} Melee Weapon Attack}: +30 to hit, reach 6m, one target.

\textit{Hit:} 23 (3d8 + 10) piercing damage. If the target is a Large or smaller creature grabbed by the kraken, that creature is engulfed, and the grab ends. While engulfed, the creature is blinded and restrained, has full cover against attacks and other effects from outside the kraken, and takes 42 (12d6) acid damage at the start of each of the kraken's rounds.

If the kraken takes 50 or more damage in a single round from a creature within it, the kraken must succeed on a DC 25 Fortitude save or vomit all engulfed creatures, which fall prone in a space within 3 meter of the kraken. If the kraken dies, a swallowed creature is no longer entangled by it and can escape from the corpse using 5 meters of movement, coming prone.

\textit{\textbf{Tentacle.} Melee Weapon Attack}: +30 to hit, reach 9m, one target.

\textit{Hit:} 20 (3d6 + 10) bludgeoning damage, and the target is grappled (DC 18 to flee). Until the grab ends, the target is restrained. The kraken has ten tentacles, each of which can grip a target.

\textit{\textbf{Slingshot.}} A Large or smaller held item or creature grabbed by the kraken is thrown 20 meters in a random direction and knocked prone. If the thrown target hits a solid surface, it takes 3 (1d6) bludgeoning damage for every 3 meter it travels. If the target is thrown at another creature, that creature must succeed on a DC 25 Reflex save or take the same damage and fall prone.

\textit{\textbf{Lightning Storm.}} The kraken magically creates three bolts of energy, each of which can strike a target within 36 meters that the kraken can see. The target must make a DC 25 Reflex save, taking 22 (4d10) Electricity damage on a failed save, or half as much damage on a successful one.

\textbf{Additional Actions}

The kraken can perform 3 additional Actions, chosen from the options below. He can only use one Additional Action at a time, and only at the end of another creature's turn. The kraken regains expended additional Actions at the start of its round.

\textbf{Tentacle Attack or Slingshot.} The kraken makes a tentacle attack or uses Slingshot.

\textbf{Cloud of Ink (Costs 3 Actions).} While underwater, the kraken expels a cloud of ink with a radius of 20 meters. The cloud spreads around corners, and that area is heavily obscured to all creatures except the kraken. Each creature other than the kraken that ends its round in the area must succeed on a Fortitude 25 save, taking 16 (3d10) poison damage on a failed save, or half as much on a successful one. A strong current scatters the cloud, which otherwise vanishes at the end of the kraken's next round. \textbf{Storm of Lightning (Costs 2 Actions).} The kraken uses Storm of Lightning.

\textbf{Ecology}\\
Environment any ocean\\
Organization: Solitary\\
\textbf{Treasure}: Triple\\
\textbf{Description}\\
The legendary kraken is one of sailors' greatest fears, for it is a creature the size of a whale, can strike depths without being seen, can command the winds and weather conditions necessary for the ship to move, and possesses the cruel intellect of most of the most ruthless and creative criminals in the world. Some believe krakens are divine punishment, while others believe they are the true lords of the depths, who regard the air-breathing races as nothing more than cattle.

Many legends have arisen as to whether he understands the Druidic language.

A kraken is nearly 30 meters long and weighs 2000kg.


\

\index[Monsters]{Lamia}\textbf{Lamia}

\textit{Large monstrosity, chaotic evil}

\textbf{STRENGTH} +3

\textbf{DEXTERITY} +1

\textbf{CONSTITUTION} +2

\textbf{INTELLIGENCE} +2

\textbf{WISDOM} +2

\textbf{CHARISMA} +3

\textbf{Initiative} +2 -- \textbf{Defence} 15

\textbf{Hit Points} 97 (13d10 + 26)

\textbf{Move} 9m

\textbf{Saving Throws}: Fortitude +6, Reflexes +9, Will +11

\textbf{Skills} Stealth +3, Deceive +7, Sense Emotions +4,

\textbf{Senses} Darkvision 18m

\textbf{Languages} Abyssal, Common

\textbf{Challenge} 4 (1100 XP)

\textit{\textbf{Innate Spells.}} The lamia's innate spellcasting ability is Charisma—The lamia can innately cast the following spells, requiring no material components:

At will: \textit{disguise self} (any humanoid form)\textit{,} \textit{major image}

3/day each: \textit{charm person, mirror image,}

\textit{scan, suggestion}

1/Day: \textit{restriction}

\textbf{Actions}

\textit{\textbf{Multiattack.}} The lamia makes two attacks: one with her claws and one with her dagger or Intoxicating Touch.

\textit{\textbf{Claws.} Melee Weapon Attack}: +9 to hit, reach 1m, one target.

\textit{Hit:} 14 (2d10 + 3) slashing damage, 1 bleed damage.

\textit{\textbf{Dagger.} Melee Weapon Attack}: +9 to hit, reach 1m, one target.

\textit{Hit:} 5 (1d4 + 3) piercing damage.

\textit{\textbf{Intoxicating Touch.} Melee spell attack}: +5 to hit, reach 1 meter, one creature.

\textit{Hit:} The target is cursed for 1 hour by this spell. Until the curse ends, the target has -1d6 on Will saves and all proficiency checks.

\textbf{Ecology}\\
Environment: Temperate Deserts\\
Organization: Solitary, pair, or cult (3-12)\\
\textbf{Treasure}: Double (Dagger+1, other treasure)\\

\textbf{Description}\\
Hateful heirs to an ancient curse, lamias have the appearance of slender, attractive women from the waist up, with the body of a mighty lion below. Their humanoid features also bear feline traits, their eyes narrow and feral, and their teeth resemble the fangs of predators. A typical standing lamia stands 1.8 meters tall, is 2.4 meters long, and weighs more than 325 kg.

Lamias are drawn to ruined keeps, abandoned cities, and forgotten monuments that suit the crude aesthetics of these lethal hunters; especially those in arid or barren areas. However, lamias favor decrepit temples. They delight in seeing the temples of good deities in ruins, and they go out of their way to put these thriving sacred places in trouble.

Lamias view the elder females of their group as leaders, mothers, and shamans, latching onto them with fanatical reverence. Though lamias shun most religions, seeing them as the source of the curse that forced them into these bestial forms, elder lamias claim to hear the whispers of the desert wind and know the cold whims of the stars, and they rely on these mystical springs to guide their people.

The lamias presented here are only the most common and least powerful exponents of this accursed race; others have serpentine, flying and even more perverse forms.


\

\index[Monsters]{Lemur}\textbf{Lemur}

\textit{Medium fiend (devil), lawful evil}

\textbf{STRENGTH} +0

\textbf{DEXTERITY} -3

\textbf{CONSTITUTION} +0

\textbf{INTELLIGENCE} -5

\textbf{WISDOM} +0

\textbf{CHARISMA} -4

\textbf{Initiative} -3 -- \textbf{Defence} 8

\textbf{Hit Points} 13 (3d8)

\textbf{Movement} 5 meters

\textbf{Saving Throws} Fortitude +4, Reflexes +3, Will +0

\textbf{Damage Resistances} cold

\textbf{Damage Immunity} Fire, poison

\textbf{Condition Immunity} charmed, poisoned, frightened

\textbf{Senses} darkvision 40m

\textbf{Languages} understands Infernal but cannot speak

\textbf{Challenge} 0 (10 XP)

\textit{\textbf{Evil Rejuvenation.}} A lemur that dies in the Nine Hells returns to life with full Hit Points in 1d10 days unless killed by a creature with good traits that has been cast 'spell \textit{bless} or his remains come
sprinkled with holy water.

\textit{\textbf{Devil's Sight.}} The devil's darkvision is not limited by magical darkness.

\textbf{Actions}

\textit{\textbf{Punch.} Melee Weapon Attack}: +3 to hit, reach 1m, one target.

\textit{Hit:} 2 (1d4) bludgeoning damage.

\textbf{Ecology}\\
Environment: Any (Hell)\\
Organization: Solitary, Pair, Group (3-5), Swarm (6-17), or Host (10-40 or more)\\
\textbf{Treasure}: None\\
\textbf{Description}\\
The lowest of devils, lemurs originate from the ranks of condemned souls in hell, shapeless masses of quivering flesh. The spark of instinct or memory that survives in their sleeping consciousness usually shapes their features, which mimic those of her torturers or the tortured souls around them. Grotesque and useless, a lemur's features reveal nothing of what it once was. Many sport various hideous faces or are nothing more than seething columns of cancerous flesh. Only their lumpy, constantly flailing limbs seem to function properly, and they are only used to destroy any non-hell lifeforms that get too close.

Lemurs on the move consolidate into forms over 4 feet tall and weighing over 100kg, though these disgusting devils, when at rest, often have the indistinct appearance of loose, misshapen-featured masses of flesh.

Though among the most revolting creatures in existence, lemurs play a vital role in the perverse ecology of Hell. When, at the end of its mortal existence, a soul is damned, either because it worships diabolical forces or because it lacks faith in other deities, it joins the masses of suffering souls who fill the plains of Avernus, the first group of 'Hell. Here the torments begin, while minor devils propel them along with other spirits, preparing them for the arduous journey up to one of the deepest circles of hell, usually one suitable for the appropriate punishment for the crimes committed by the soul, or simply towards the dominion of a devil who needs new slaves. Once in the realm of their damnation, souls face countless centuries of torment at the hands of devils, other evil beings, and the lethal machinations of Hell itself. As the mortal essence slowly goes mad, these creatures forget their lives, first becoming savages and eventually little more than automatons driven by hatred and fear. After eons of this existence, Hell's cruel process either utterly destroys the soul or, in the case of more profane spirits, reconsecrates these forgotten beings in the form of lemurs, the most basic life form of devils, mindless hordes of rotting flesh. and diabolical. These foul beings gather in great masses, revolting waves formed by thousands and thousands of these creatures.

The major devils are able to recognize the most corrupt among them and, by means of mysterious tortures or thanks to the very powers of Hell, they reshape them into real devils, newly reborn and ready to serve obediently in the legions of the damned.



\

\index[Monsters]{Lesser Water Elemental}\textbf{Lesser Water Elemental}

\textit{Elemental Medium, Neutral}

\textbf{STRENGTH} +2

\textbf{DEXTERITY} +1

\textbf{CONSTITUTION} +2

\textbf{INTELLIGENCE} -3

\textbf{WISDOM} +0

\textbf{CHARISMA} -1

\textbf{Initiative} +4 -- \textbf{Defence} 15

\textbf{Hit Points} 16 (2d8 + 4)

\textbf{Movement} 9m, swim 27m

\textbf{Saving Throws} Fortitude +3, Reflexes +4, Will +0

\textbf{Damage Resistances} acid; from a non-magical weapon

\textbf{Immunity to Damage} Poison

\textbf{Condition Immunity} grabbed, poisoned, restrained, paralyzed, petrified, unconscious, prone, fatigue

\textbf{Senses} Darkvision 18m

\textbf{Languages} Aquan

\textbf{Challenge} 2

\textit{\textbf{Freezing.}} If the elemental takes cold damage, it is partially frozen; his movement is reduced by 6 meters until the end of his next round.

\textit{\textbf{Water form.}} The elemental can enter a hostile creature's space and stop there. He can move through a space as narrow as 3 centimeters without having to squeeze.

\textit{\textbf{Elemental nature.}} An elemental has no need for air, food, drink, or sleep.

\textbf{Actions}

\textit{\textbf{Multiattack.}} The elemental makes two slam attacks.

\textit{\textbf{Slam.} Melee Weapon Attack}: +5 to hit, reach 1m, one target.

\textit{Hit:} 6 (1d6 + 2) bludgeoning damage.

\textit{\textbf{Submerge (Cooldown 4-6).}} Each creature in the elemental's space must make a DC 13 Fortitude save. On a failed save, the target takes 8 (2d4 + 4) damage from hit. If it is Medium or smaller, the target is also grappled (DC 12 to escape). Until the grab ends, the target is restrained and can't breathe unless able to breathe water. On a successful save, the target is pushed out of the elemental's space.

The elemental can grab one Medium or up to two Small creatures at a time. At the start of each elemental's turn, each grabbed target takes 8 (2d4 + 4) bludgeoning damage. A creature within 1 meter of the elemental can pull a creature or object out of it, taking an action to attempt a DC 12 Strength check.

\textbf{Ecology}\\
Environment any (Plane of Water)\\
Organization: Solitary, pair, or group (3-8)\\
\textbf{Treasure}: None\\
\textbf{Description}\\
Water elementals are patient, unyielding creatures composed of living, fresh or salt water. They prefer to cover their opponents with water or drag them into it to gain an advantage.\\
Like other elementals, all water elementals have unique shapes and forms. Many are wave-like creatures with vaguely humanoid faces and smaller waves on the sides that act as arms. Another common form is that of some aquatic creature, such as a shark or octopus, but made entirely of water.\\
A large water elemental stands 2 meters tall and weighs 90kg.

\

\index[Monsters]{Lich}\textbf{Lich}

\textit{Medium Undead, Evil Traits}

\textbf{STRENGTH} +0

\textbf{DEXTERITY} +3

\textbf{CONSTITUTION} +3

\textbf{INTELLIGENCE} +5

\textbf{WISDOM} +2

\textbf{CHARISMA} +3

\textbf{Initiative} +5 -- \textbf{Defence} 28

\textbf{Hit Points} 135 (18d8 + 54)

\textbf{Move} 9m

\textbf{Saving Throws}: Fortitude +26, Reflexes +24, Will +23

\textbf{Damage Resistances} cold, Electricity, void

\textbf{Immunity to Damage} Poison; from a non-magical weapon

\textbf{Condition Immunity} charmed, poisoned, paralyzed, fatigued, frightened, bleeding

\textbf{Senses} True Seeing 36 m

\textbf{Languages} Common plus five other languages, Exspiram

\textbf{Challenge} 21 (33000 XP)

\textit{\textbf{Spells.}} The lich has MP 18. His spellcasting ability is Intelligence, +5 to hit on spell attacks). The lich knows the following spells:

Cantrips (at-will): \textit{mage hand, sleight of hand, ray} \textit{frost}

level 1 (4 slots): \textit{Arcane Dart, detect magic,} \textit{thunderwave, shield}

level 2 (3 slots): \textit{acid arrow, mirror image,} \textit{detect thoughts, invisibility}

level 3 (3 slots): \textit{animate dead, counterspell, dispel} \textit{spells, fireball}

level 4 (3 slots): \textit{wither, dimension door}

level 5 (3 slots): \textit{mortal cloud, scrying}

level 6 (1 slot): \textit{disintegrate, Orb of Invulnerability}

level 7 (1 slot): \textit{finger of death, planar shift}

level 8 (1 slot): \textit{dominate monster, power word stun}

level 9 (1 slot): \textit{kill word}

\textit{\textbf{Undead nature.}} The lich needs no air, food, drink, or sleep.

\textit{\textbf{Legendary Resistance (3/day).}} If the lich fails a Saving Throw, it can choose to succeed instead.

\textit{\textbf{Resistance to Turning.}} The lich has +1d6 on Saving Throws against effects that turn undead.

\textit{\textbf{Rejuvenation.}} If possessed by a phylactery, the destroyed lich gains a new body in 1d10 days, regaining all its Hit Points and returning to activity. The new body appears within 1 meter of the phylactery.

\textit{\textbf{Sacrifices of souls.}} A lich must periodically feed souls to his phylactery to sustain the magic that maintains his body and consciousness. To do this, use the spell \textit{imprison}. Instead of choosing one of the spell's normal options, the lich uses it to magically trap the target's body and soul within the phylactery. The phylactery must be on the same plane as the lich for this spell to work. A lich's phylactery can hold only one creature at a time, and \textit{dispel magic} cast as a level 9 spell on the phylactery frees any creature imprisoned within it. A creature imprisoned in the phylactery for 24 hours is consumed and destroyed, after which nothing short of divine intervention can bring it back to life.

A lich that forgets or fails to maintain its body with sacrificed souls begins to fall apart, and may eventually transform into a demilich.

\textbf{Actions}

\textit{\textbf{Crippling Touch.} Melee Spell Attack}: +18 to hit, reach 1 meter, one creature.

\textit{Hit:} 10 (3d6) cold damage. The target must succeed on a DC 25 Fortitude save or be paralyzed for 1 minute. The target can repeat the Saving Throw at the end of each of its rounds, ending the effect on itself on a success.

\textbf{Additional Actions}

The lich can perform 3 additional Actions, chosen from the options below. He can only use one Additional Action at a time, and only at the end of another creature's turn. The lich regains expended additional Actions at the start of its round.

\textit{\textbf{To destroy Life (Costs 3 Actions).}} Each creature other than undead within 6 meters of the lich must make a DC 25 Fortitude save against this spell, taking 21 (6d6) damage from Void on a failed save, or half as much damage on a successful one. Creatures become fatigued.

\textit{\textbf{Fearful Gaze (Costs 2 Actions).}} The lich fixes its gaze on a creature visible within 3 meter of it. The target must succeed on a DC 25 Will save against this spell or be frightened for 1 minute. The frightened target can repeat the Saving Throw at the end of each of its rounds, ending the effect on itself on a success. If the target's save is successful or the effect ends for it, the target is immune to the lich's gaze for the next 24 hours.

\textit{\textbf{Crippling Touch (Costs 2 Actions).}} The lich uses its Crippling Touch.

\textit{\textbf{Cantrip.}} The lich casts a cantrip.

\textbf{Ecology}\\
Environment: Any\\
Organization: Solitary\\
\textbf{Treasure}: NPC gear (Ring of Protection +2, Sash of Lore +2 (Awareness), Boots of Levitation, scroll of Dominate Person, scroll of Teleportation, potion of Invisibility)\\

\textbf{Description}
Few creatures are more feared than liches. The pinnacle of the necromantic arts, the lich is a spellcaster who has chosen to give up life and cheat death by becoming undead. While many who reach such heights of power would do anything to achieve immortality, the idea of becoming a lich is abhorrent by many creatures. The process involves extracting the caster's life force and imprisoning it in a specially prepared phylactery; the spellcaster surrenders his life, but remains caught between life and death, and as long as his phylactery remains intact he can continue his research and work without fear of the passage of time.



\

\textbf{Lion}\index[Monsters]{Lion}

\textit{Large beast, unaligned}

\textbf{STRENGTH} +3

\textbf{DEXTERITY} +2

\textbf{CONSTITUTION} +1

\textbf{INTELLIGENCE} -4

\textbf{WISDOM} +1

\textbf{CHARISMA} -1

\textbf{Initiative} +2 -- \textbf{Defence} 13

\textbf{Hit Points} 26 (4d10 + 4)

\textbf{Move} 15m

\textbf{Saving Throws}: Fortitude +6, Reflexes +7, Will +2

\textbf{Skills} Stealth +6, Awareness +3

\textbf{Languages} -

\textbf{Challenge} 1 (200 XP)

\textit{\textbf{Leap.}} If the lion moves at least 6 meters directly at a creature and hits it with a claw attack during the same turn, the target must succeed on a DC 13 Fortitude save or fall prone. If the target is prone, the lion can make a
bite attack as a bonus action.

\textit{\textbf{Enhanced sense of smell.}} The lion has +1d6 on Wisdom (Awareness) checks based on smell.

\textit{\textbf{Leap with Running.}} With a 3m run, the lion can jump up to 6 meters long.

\textit{\textbf{Pack tactics.}} The lion has +1d6 on attack rolls against a creature if at least one of the lion's allies is within 1 meter of the creature and that ally isn't incapacitated.

\textbf{Actions}

\textit{\textbf{Claw.} Melee Weapon Attack}: +5 to hit, reach 1m, one target.

\textit{Hit:} 6 (1d6 + 3) slashing damage, 1 bleed damage.

\textit{\textbf{Bite.} Melee Weapon Attack}: +5 to hit, reach 1m, one target.

\textit{Hit:} 7 (1d8 + 3) piercing damage.

\

\textbf{Lizard}\index[Monsters]{Lizard}

\textit{Tiny beast, unaligned}

\textbf{STRENGTH} -4

\textbf{DEXTERITY} +0

\textbf{CONSTITUTION} +0

\textbf{INTELLIGENCE} -5

\textbf{WISDOM} -1

\textbf{CHARISMA} -4

\textbf{Initiative} +0 -- \textbf{Defence} 11

\textbf{Hit Points} 2 (1d4)

\textbf{Movement} 6m, climb 6m

\textbf{Saving Throws}: Fortitude +1, Reflexes +4, Will +1

\textbf{Senses} vision in the dark 9m

\textbf{Languages} -

\textbf{Challenge} 0 (10 XP)

\textit{\textbf{Climb as Spider.}} The lizard can climb difficult surfaces, including standing upside down on ceilings, without needing to make an ability check.

\textbf{Actions}

\textit{\textbf{Bite.} Melee Weapon Attack}: +0 to hit, reach 1m, one target.

\textit{Hit:} 1 piercing damage.

\

\index[Monsters]{Lizardfolk}\textbf{Lizardfolk}

\textit{Medium humanoid (lizardfolk), neutral}

\textbf{STRENGTH} +2

\textbf{DEXTERITY} +0

\textbf{CONSTITUTION} +1

\textbf{INTELLIGENCE} -2

\textbf{WISDOM} +1

\textbf{CHARISMA} -2

\textbf{Initiative} +0 -- \textbf{Defence} 16 (natural Armour, shield)

\textbf{Hit Points} 22 (4d8 + 4)

\textbf{Movement} 9m, swim 9m

\textbf{Saving Throws}: Fortitude +4, Reflexes +0, Will +0

\textbf{Skills} Stealth +4, Awareness +3, Survival +5

\textbf{Languages} Draconic

\textbf{Challenge} 1/2 (100 XP)

\textit{\textbf{Hold Breath.}} The lizardfolk can hold its breath for 15 minutes.

\textbf{Actions}

\textit{\textbf{Multiattack.}} The lizardfolk makes two melee attacks, each with a different weapon.

\textit{\textbf{Javelin.} Melee or Ranged weapon attack}: +4 to hit, reach 1m or range 12m, one target. \textit{Hit:} 5 (1d6 + 2) piercing damage.

\textit{\textbf{Bite.} Melee Weapon Attack}: +4 to hit, reach 1m, one target.

\textit{Hit:} 5 (1d6 + 2) piercing damage.

\textit{\textbf{Club.} Melee Weapon Attack}: +4 to hit, reach 1m, one target.

\textit{Hit:} 5 (1d6 + 2) bludgeoning damage.

\textit{\textbf{Spiked Shield.} Melee Weapon Attack}: +4 to hit, reach 1m, one target.

\textit{Hit:} 5 (1d6 + 2) piercing damage.

\textbf{Ecology}\\
Environment: temperate marshes\\
Organization: solitary, pair, band (3-12) or tribe (13-60)\\
\textbf{Treasure}: NPC gear (Heavy Wooden Shield, Spiked Mace, 3 Javelins)\\

\textbf{Description}\\
Lizardfolk are proud and powerful predatory reptiles that make their communal homes in scattered villages in the recesses of marshes and marshes. With no interest in dryland colonization and content with their simple weapons and rituals that have served them well for millennia, the lizardfolk are viewed by many of the other races as retrograde savages, but within their isolated communities they are a vital people steeped in traditions and with an oral history dating back to before man walked upright.

Most lizardfolk stand 1.8 to 2.1 meters tall and weigh 100 to 125 kg, and have powerful muscles covered in gray, green, or brown scales. Some rays have small dorsal crests or brightly colored collars, and all swim well by moving with rapid movements of their mighty 1.2m long tail. Even if they are fully at ease in the water, they hold their breath and return to their homes on artificial hills to breed and sleep. Because their reptilian blood makes them slow in the cold, many lizardfolk hunt and work during the day and retreat to their abodes at night to huddle with others of their tribe to share the warmth of great turf fires.

While generally neutral, lizardfolk's standoffish demeanor, staunch rejection of the "gifts" of civilization, and legendary ferocity in battle cause them to be misjudged by most humanoids. These traits come with good reasons, however, as their low reproductive rate is unmatched among warm-blooded humanoids, and if tribes didn't defend their swampy territories to their last breath they would soon find themselves overwhelmed by hordes of mammals. As for their propensity for eating the bodies of the dead both friend and foe, practical lizardfolk are quick to point out that life is hard in the swamp, and nothing should go to waste.

The lizardfolk presented here live in swampy environments. Lizardfolk tribes can do just as well in other environments, but gain Climb 5 meters instead of Swim as a haste.


\

\index[Monsters]{Magma Man (Magmin)}\textbf{Magma Man (Magmin)}

\textit{Small elemental, chaotic neutral}

\textbf{STRENGTH} -2

\textbf{DEXTERITY} +2

\textbf{CONSTITUTION} +1

\textbf{INTELLIGENCE} -1

\textbf{WISDOM} +0

\textbf{CHARISMA} +0

\textbf{Initiative} +2 -- \textbf{Defence} 15

\textbf{Hit Points} 9 (2d6 + 2)

\textbf{Move} 9m

\textbf{Saving Throws}: Fortitude +6, Reflexes +4, Will +3

\textbf{Damage Resistances} from non-magical weapon

\textbf{Immunity to Damage} Fire

\textbf{Senses} Darkvision 18m

\textbf{Languages} Ignan

\textbf{Challenge} 1/2 (100 XP)

\textit{\textbf{Incendiary illumination.}} As a bonus action, the magma man can ignite or extinguish his own flames. While the flame is lit, the magma man sheds bright light in a 3m radius and dim light for an additional 3 meter.

\textit{\textbf{Deadly Blast.}} When the Magma Man dies, he explodes in a burst of fire and magma. Each creature within 3 meter of it must make a DC 11 Reflex save, taking 7 (2d6) fire damage on a failed save, or half as much damage on a successful one. Flammable items that are not being worn or carried and that are in the area catch fire.

\textbf{Actions}

\textit{\textbf{Touch.} Melee Weapon Attack}: +4 to hit, reach 1m, one target.

\textit{Hit:} 7 (2d6) fire damage. If the target is a flammable creature or object, it catches fire. As long as a creature takes an action to extinguish the flame, the creature takes 3 (1d6) fire damage at the end of each of its rounds.

\textbf{Ecology}\\
Environment any terrain (Plane of Fire)\\
Organization: Solitary or gang (2-8)\\
\textbf{Treasure}: Standard\\
\textbf{Description}\\
While Ignim inhabit the Plane of Fire, they sometimes slip into elemental fissures in the Material Plane. These cracks usually form in places of intense heat, such as volcanoes or underground rivers of magma, or in places of strong, unpredictable magic. The latter scenario usually ends in more complex events, as Ignim tend to unintentionally set fire to nearby flammable objects.

While not brave, these small outsiders are nevertheless fearsome foes to creatures without resistance to their intense heat. Their touch incinerates clothing, and creatures that strike their bodies with steel run the risk of reducing their weapons to slag. The Ignim's best Defence at home on the Plane of Fire is their numbers. The settlements, dotted with magma lakes and spurting geysers of molten rock, are teeming with staggering numbers of these creatures.

Magmins are paranoid and distrustful. Always fearful of the Plane of Fire's larger denizens, Ignim pepper intruders with thousands of questions, asking where they go, where they come from, and what they are doing near their precious magma lakes. If the travellers' answers are unsatisfactory, the Ignim attempt to get rid of the creatures as quickly as possible. Anyone who refuses to leave risks being thrown into a lake of liquid rock.

Magmins take great pride in the way they maintain their magma lakes. Each lake has a different purpose: for bathing, for cooking meals or for relaxing. Magmins put minerals and salts into these lakes to suit their purpose. Cooking lakes (sometimes called "killer lakes" by outsiders) are warmer than others, and recreation lakes are usually darker than swimming lakes.

At maturity, Ignims stand 1.2 meters tall, their dense composition making them weigh 150 kg.

\

\index[Monsters]{Major Water Elemental}\textbf{Major Water Elemental}

\textit{Huge elemental, neutral}

\textbf{STRENGTH} +6

\textbf{DEXTERITY} +3

\textbf{CONSTITUTION} +5

\textbf{INTELLIGENCE} -2

\textbf{WISDOM} +1

\textbf{CHARISMA} +0

\textbf{Initiative} +4 -- \textbf{Defence} 21

\textbf{Hit Points} 158

\textbf{Movement} 9m, swim 33m

\textbf{Saving Throws} Fortitude +12, Reflexes +12, Will +6

\textbf{Damage Resistances} acid; from a non-magical weapon

\textbf{Immunity to Damage} Poison

\textbf{Condition Immunity} grabbed, poisoned, restrained, paralyzed, petrified, unconscious, prone, fatigue

\textbf{Senses} Darkvision 18m

\textbf{Languages} Aquan

\textbf{Challenge} 9

\textit{\textbf{Freezing.}} If the elemental takes cold damage, it freezes partially; his movement is reduced by 6 meters until the end of his next round.

\textit{\textbf{Water form.}} The elemental can enter a hostile creature's space and stop there. It can move through a space as narrow as 3 centimeters without having to squeeze.

\textit{\textbf{Elemental nature.}} An elemental does not need air, food, drink, or sleep.

\textbf{Actions}

\textit{\textbf{Multiattack.}} The elemental makes two slam attacks.

\textit{\textbf{Slam.} Melee Weapon Attack}: +19 to hit, reach 3m, one target.

\textit{Hit:} 26 (4d8 + 10) bludgeoning damage.

\textit{\textbf{Submerge (Cooldown 4-6).}} Each creature in the elemental's space must make a DC 21 Fortitude save. On a failed save, the target takes 25 (5d8 + 5) damage from hit. If Huge or smaller, the target is also grappled (DC 19 to flee). Until the grab ends, the target is restrained and can't breathe unless able to breathe water. On a successful save, the target is pushed out of the elemental's space.

The elemental can grab hold of one Huge creature or up to two Large or smaller creatures at a time. At the start of each elemental's turn, each grabbed target takes 25 (4d8 + 5) bludgeoning damage. A creature within 3 meter of the elemental can pull a creature or object out of it, taking an action to attempt a DC 19 Strength check.

\textbf{Ecology}\\
Environment any (Plane of Water)\\
Organization: Solitary, pair, or group (3-8)\\
\textbf{Treasure}: None\\
\textbf{Description}\\
Water elementals are patient, unyielding creatures composed of living, fresh or salt water. They prefer to cover their opponents with water or drag them into it to gain an advantage.\\
Like other elementals, all water elementals have unique shapes and forms. Many are wave-like creatures with vaguely humanoid faces and smaller waves on the sides that act as arms. Another common form is that of some aquatic creature, such as a shark or octopus, but made entirely of water.\\
A large water elemental stands 10 meters tall and weighs 10000kg.

\

\textbf{Mammoth}\index[Monsters]{Mammoth}

The mammoth is an elephant-like creature with thick fur and long tusks.

\textit{Huge beast, unaligned}

\textbf{STRENGTH} +7

\textbf{DEXTERITY} -1

\textbf{CONSTITUTION} +5

\textbf{INTELLIGENCE} -4

\textbf{WISDOM} +0

\textbf{CHARISMA} -2

\textbf{Initiative} -1 -- \textbf{Defence} 16

\textbf{Hit Points} 126 (11d12 + 55)

\textbf{Move} 12m

\textbf{Saving Throws}: Fortitude +14, Reflexes +10, Will +7

\textbf{Languages} -

\textbf{Challenge} 6 (2300 XP)

\textit{\textbf{Rrumbling Charge.}} If the mammoth moves at least 6 meters directly toward a creature and hits it with a gore attack during the same turn, the target must succeed on a DC 18 Fortitude save or fall prone. If the target is prone, the mammoth can make a stomp attack against it as a bonus action.

\textbf{Actions}

\textit{\textbf{Gore.} Melee Weapon Attack}: +10 to hit, reach 3m, one target.

\textit{Hit:} 25 (4d8 + 7) piercing damage.

\textit{\textbf{Stamp.} Melee Weapon Attack}: +10 to hit, reach 1m, one prone creature.

\textit{Hit:} 29 (4d10 + 7) bludgeoning damage.

\

\index[Monsters]{Manticore}\textbf{Manticore}

\textit{Large Monstrosity, Lawful Evil}

\textbf{STRENGTH} +3

\textbf{DEXTERITY} +3

\textbf{CONSTITUTION} +3

\textbf{INTELLIGENCE} -2

\textbf{WISDOM} +1

\textbf{CHARISMA} -1

\textbf{Initiative} +3 -- \textbf{Defence} 16

\textbf{Hit Points} 68 (8d10 + 24)

\textbf{Move} 9m, fly 15m

\textbf{Saving Throws}: Fortitude +9, Reflexes +7, Will +3

\textbf{Senses} Darkvision 18m

\textbf{Languages} Common

\textbf{Challenge} 3 (700 XP)

\textit{\textbf{Regrow tail spines.}} The manticore has twenty-four tail spines. Used thorns grow back at dawn.

\textbf{Actions}

\textit{\textbf{Multiattack.}} The manticore makes three attacks: one with its bite and two with its claws, or three with its tail spikes.

\textit{\textbf{Claw.} Melee Weapon Attack}: +7 to hit, reach 1m, one target.

\textit{Hit:} 6 (1d6 + 3) slashing damage, 1 bleed damage.

\textit{\textbf{Bite.} Melee Weapon Attack}: +7 to hit, reach 1m, one target.

\textit{Hit:} 7 (1d8 + 3) piercing damage.

\textit{\textbf{Tail Thorns.} Ranged Weapon Attack}: +7 to hit, range 30m, one target.

\textit{Hit:} 7 (1d8 + 3) piercing damage.

\textbf{Ecology}
Environment: Hills and Warm Swamps\\
Organization: Solitary, pair, or herd (3-6)\\
\textbf{Treasure}: Standard\\

\textbf{Description}\\
Manticores are ferocious predators that survey large areas in search of fresh meat. A typical manticore is about 3 meters long and weighs about 500 kg. Some have heads similar to that of a human, usually bearded. Males and females are very similar.

Manticores eat all kinds of meat, even carrion, but prefer human meat and rarely pass up an opportunity to enjoy this delicacy. They are cunning and social enough to make pacts with evil humanoids to form alliances or compel them to offer tribute, and many powerful creatures charge them with policing or controlling a place or area. They prefer to make their dens in high places, such as hilltops and caves between cliffs.

While manticores are similar to magical creations, they have long been counted among the natural species. Curiously, manticores seem strangely fecund and can interbreed with numerous other similarly shaped species, including Lions, Tigers, Lamias, Sphinxes, and Chimeras.

\

\textbf{Mastiff}\index[Monsters]{Mastiff}

\textbf{The} mastiffs are impressive hounds prized by humanoids for their reality and heightened senses.

\textit{Medium beast, unaligned}

\textbf{STRENGTH} +1

\textbf{DEXTERITY} +2

\textbf{CONSTITUTION} +1

\textbf{INTELLIGENCE} -4

\textbf{WISDOM} +1

\textbf{CHARISMA} -2

\textbf{Initiative} +2 -- \textbf{Defence} 13

\textbf{Hit Points} 5 (1d8 + 1)

\textbf{Move} 12m

\textbf{Saving Throws}: Fortitude +3, Reflexes +3, Will +1

\textbf{Skills} Awareness +3, Track +3

\textbf{Languages} -

\textbf{Challenge} 1/8 (25 XP)

\textit{\textbf{Hearing and keen sense of smell.}} The mastiff has +1d6 on Wisdom (Awareness) checks based on hearing or smell.

\textbf{Actions}

\textit{\textbf{Bite.} Melee Weapon Attack}: +3 to hit, reach 1m, one target.

\textit{Hit:} 4 (1d6 + 1) piercing damage. If the target is a creature, it must succeed on a DC 11 Fortitude save or be knocked prone.

\

\index[Monsters]{Medusa}\textbf{Medusa}

\textit{Medium monstrosity, lawful evil}

\textbf{STRENGTH} +0

\textbf{DEXTERITY} +2

\textbf{CONSTITUTION} +3

\textbf{INTELLIGENCE} +1

\textbf{WISDOM} +1

\textbf{CHARISMA} +2

\textbf{Initiative} +2 -- \textbf{Defence} 18

\textbf{Hit Points} 127 (17d8 + 51)

\textbf{Move} 9m

\textbf{Saving Throws}: Fortitude +6, Reflexes +8, Will +7

\textbf{Skills} Stealth +5, Deceive +5, Sense Emotions +4, Awareness +4

\textbf{Senses} Darkvision 18m

\textbf{Languages} Common

\textbf{Challenge} 6 (2300 XP)

\textit{\textbf{Petrifying Gaze.}} If a creature begins its round within 10 meters of a medusa whose eyes it can see, the medusa, if it isn't incapacitated and can see the creature, he can force it to make a DC 16 Fortitude save. If the creature fails the save by 5 or more, it is instantly petrified, or else it magically begins to turn to stone and is restrained. The entangled creature must repeat the Saving Throw at the end of its next round. If she succeeds, the effect ends. On a failed save, the creature is petrified until freed by the spell \textit{greater restoration} or other magic.

A creature that isn't surprised can look away to avoid the Saving Throw at the start of its round. In that case, he can't see the jellyfish until the start of his next round, when he can look away again. If he looks at the jellyfish in the meantime, he must immediately make the Saving Throw.

If the medusa sees her reflection on a reflective surface within 10 meters of her in an area of bright light, she is affected by her own gaze as a result of her own curse.

\textbf{Actions}

\textit{\textbf{Multiattack.}} The medusa makes three attacks -- one with her snake hair and two with her short sword -- or two ranged attacks with her longbow.



\textit{\textbf{Serpentine Hair.} Melee Weapon Attack}: +9 to hit, reach 1m, one target.

\textit{Hit:} 4 (1d4 + 2) piercing damage plus 14 (4d6) poison damage.

\textit{\textbf{Short sword.} Melee weapon attack}: 9 to hit, reach 1m, one target.

\textit{Hit:} 5 (1d6 + 2) piercing damage.

\textit{\textbf{Longbow.} Ranged weapon attack}: +9 to hit, range 45m, one target.

\textit{Hit:} 6 (1d8 + 2) piercing damage plus 7 (2d6) poison damage.

\textbf{Ecology}\\
Environment: Temperate marshes and underground\\
Organization: Solitary\\
\textbf{Treasure}: Double (Dagger, masterwork longbow w/20 arrows, other treasure)\\
\textbf{Description}\\
Jellyfish are human-like creatures with snakes for hair. From a distance of 10 meters or more, a medusa can easily pass for a beautiful woman if she wears something that covers her serpentine hair; when she wears clothing that conceals her head and face she can be mistaken for a human even at close range. Jellyfish use lies and disguises to hide their faces until opponents are close enough to use their petrifying gaze, though they like to play with their prey and can use flaming arrows to ensnare enemies from a distance. Some enjoy creating intricate decorations out of their victims, using petrification to add some flair to their swampy hideouts, but many medusas take care to hide evidence of their previous encounters so their new enemies won't notice their dangerous attack. presence.

Accustomed to hiding, city jellyfish are generally thieves, while those of the wilderness often end up as rangers. The medusae of the best-known legends, however, are those that take caster levels. Charismatic and intelligent, urban jellyfish are often involved in thieves' guilds and other aspects of the criminal underworld. Jellyfish can form alliances with blind creatures or intelligent undead, both of which are immune to their petrifying gaze. Enchantress jellyfish often act as oracles or prophetesses, usually living in remote areas of legendary power or ominous history. These oracle jellyfish take great delight in their role, and when presented with the right gifts and flattery, the secrets they offer can be very useful. Naturally, the hideouts of these powerful creatures are decorated with statues of those who have offended them, as a reminder to exercise due caution during encounters.

All jellyfish are female. Rarely, a medusa decides to take a humanoid male as a mate, usually with the aid of a love elixir or some similar magic, and they always take care not to petrify their captive unless they are bored with the company of she.


\

\index[Monsters]{Mephit, Dust}\textbf{Dust Mephit}

\textit{Small elemental, neutral evil}

\textbf{STRENGTH} -3

\textbf{DEXTERITY} +2

\textbf{CONSTITUTION} +0

\textbf{INTELLIGENCE} -1

\textbf{WISDOM} +0

\textbf{CHARISMA} +0

\textbf{Initiative} +2 -- \textbf{Defence} 13

\textbf{Hit Points} 17 (5d6)

\textbf{Move} 9m, fly 9m

\textbf{Saving Throws}: Fortitude +2, Reflexes +5, Will +3

\textbf{Skills} Stealth +4, Awareness +2

\textbf{Vulnerability to Damage} fire

\textbf{Immunity to Damage} Poison

\textbf{Condition Immunity} poisoned

\textbf{Senses} Darkvision 18m

\textbf{Languages} Ictum, Tremun

\textbf{Challenge} 1/2 (100 XP)

\textit{\textbf{Innate Spells (1/day).}} The mephit can innately cast \textit{sleep} (spell save DC 10), without needing material components. His innate spellcasting ability is Charisma.

\textit{\textbf{Elemental nature.}} A mephit needs no food, drink, or sleep.

\textit{\textbf{Deadly Blast.}} When the mephit dies, it explodes in a blast of dust. Each creature within 1 meter of it must succeed on a DC 10 Fortitude save or be blinded for 1 minute. A blinded creature can repeat the Saving Throw during each of its rounds, ending the effect on itself on a success.

\textbf{Actions}

\textit{\textbf{Claws.} Melee Weapon Attack}: +4 to hit, reach 1m, one creature.

\textit{Hit:} 4 (1d4 + 2) slashing damage.

\textit{\textbf{Blinding Breath (Cooldown 6).}} The mephit exhales a 1m  cone of blinding dust. Each creature in the area must succeed on a DC 10 Reflex save or be blinded for 1 minute. A blinded creature can repeat the Saving Throw during each of its rounds, ending the effect on itself on a success.

\textbf{Ecology}\\
Environment any (elemental plane of air)\\
Organization: Solitary, pair, group (3-6), or flock (7-12)\\
\textbf{Treasure}: Standard\\
\textbf{Description}\\
Mephits are the servants of powerful elemental creatures. The key sites and clearings of the elemental planes are filled with mephits scrambling to perform some important duty or errand.

Dust mephits are commonly found on the Plane of Air. These mephits are annoying and insistent.

\

\index[Monsters]{Mephit, Ice}\textbf{Ice Mephit}

\textit{Small elemental, neutral evil}

\textbf{STRENGTH} -2

\textbf{DEXTERITY} +1

\textbf{CONSTITUTION} +0

\textbf{INTELLIGENCE} -1

\textbf{WISDOM} +0

\textbf{CHARISMA} +1

\textbf{Initiative} +1 -- \textbf{Defence} 12

\textbf{Hit Points} 21 (6d6)

\textbf{Move} 9m, fly 9m

\textbf{Saving Throws}: Fortitude +2, Reflexes +5, Will +3

\textbf{Skills} Stealth +3, Awareness +2

\textbf{Vulnerability to Damage} from slam, fire

\textbf{Immunity to Damage} Cold, Poison

\textbf{Condition Immunity} poisoned

\textbf{Senses} Darkvision 18m

\textbf{Languages} Aquan, Ictum

\textbf{Challenge} 1/2 (100 XP)

\textit{\textbf{False Appearance.}} While the mephit remains motionless, it is indistinguishable from an ordinary shard of ice.

\textit{\textbf{Innate Spells (1/day).}} The mephit can innately cast \textit{cloud of mist}, requiring no material components. His innate spellcasting ability is Charisma.

\textit{\textbf{Elemental nature.}} A mephit needs no food, drink, or sleep.

\textit{\textbf{Deadly Burst.}} When the mephit dies, it explodes in a burst of ice shards. Each creature within 1 meter of it must make a DC 10 Reflex save or take 4 (1d8) slashing damage on a failed save, or half as much damage on a failed save.
successfull.

\textbf{Actions}

\textit{\textbf{Claws.} Melee weapon attack}: +3 to hit, reach 1m, one creature.

\textit{Hit:} 3 (1d4 + 1) slashing damage plus 2 (1d4) cold damage.

\textit{\textbf{Icy Breath (Cooldown 6).}} The mephit exhales a 5-meter cone of cold air. Each creature in the area must make a DC 10 Reflex save, taking 5 (2d4) cold damage on a failed save, or half as much damage on a successful one.

\textbf{Ecology}\\
Environment any (elemental plane of air)\\
Organization: Solitary, pair, group (3-6), or flock (7-12)\\
\textbf{Treasure}: Standard\\
\textbf{Description}\\
Mephits are the servants of powerful elemental creatures. The key sites and clearings of the elemental planes are filled with mephits scrambling to perform some important duty or errand.

Ice mephits are commonly found on the Plane of Air. These mephits are aloof and cruel.


\

\index[Monsters]{Mephit, Magma}\textbf{Magma Mephit}

\textit{Small elemental, neutral evil}

\textbf{STRENGTH} -1

\textbf{DEXTERITY} +1

\textbf{CONSTITUTION} +1

\textbf{INTELLIGENCE} -2

\textbf{WISDOM} +0

\textbf{CHARISMA} +0

\textbf{Initiative} +1 -- \textbf{Defence} 12

\textbf{Hit Points} 22 (5d6 + 5)

\textbf{Move} 9m, fly 9m

\textbf{Saving Throws}: Fortitude +2, Reflexes +5, Will +3

\textbf{Skills} Stealth +3

\textbf{Vulnerability to Damage} cold

\textbf{Immunity to Damage} Fire, Poison

\textbf{Condition Immunity} poisoned

\textbf{Senses} Darkvision 18m

\textbf{Languages} Ignan, Tremun

\textbf{Challenge} 1/2 (100 XP)

\textit{\textbf{False Appearance.}} While the mephit remains motionless, it is indistinguishable from an ordinary pool of magma.

\textit{\textbf{Innate Spells (1/day).}} The mephit can innately cast \textit{heat metal} (spell save DC 10), without needing material components. His innate spellcasting ability is Charisma.

\textit{\textbf{Elemental nature.}} A mephit needs no food, drink, or sleep.

\textit{\textbf{Deadly Blast.}} When the mephit dies, it explodes in a burst of lava. Each creature within 1 meter of it must make a DC 11 Reflex save or take 7 (2d6) fire damage on a failed save, or half as much damage on a successful one.

\textbf{Actions}

\textit{\textbf{Claws.} Melee weapon attack}: +3 to hit, reach 1m, one creature.

\textit{Hit:} 3 (1d4 + 1) slashing damage plus 2 (1d4) fire damage.

\textit{\textbf{Fiery Breath (Cooldown 6).}} The mephit exhales a 5-meter cone of fire. Each creature in the area must make a DC 11 Reflex save, taking 7 (2d6) fire damage on a failed save, or half as much damage on a successful one.

\textbf{Ecology}\\
Environment any (elemental plane of fire)\\
Organization: Solitary, pair, group (3-6), or flock (7-12)\\
\textbf{Treasure}: Standard\\
\textbf{Description}\\
Mephits are the servants of powerful elemental creatures. The key sites and clearings of the elemental planes are filled with mephits scrambling to perform some important duty or errand.

Magma mephits are commonly found on the Plane of Fire. These mephits are stupid brutes.


\

\index[Monsters]{Mephit, Steam}\textbf{Steam Mephit}

\textit{Small Elemental, Neutral Evil}

\textbf{STRENGTH} -3

\textbf{DEXTERITY} +0

\textbf{CONSTITUTION} +0

\textbf{INTELLIGENCE} +0

\textbf{WISDOM} +0

\textbf{CHARISMA} +1

\textbf{Initiative} +0 -- \textbf{Defence} 11

\textbf{Hit Points} 21 (6d6)

\textbf{Move} 9m, fly 9m

\textbf{Saving Throws}: Fortitude +2, Reflexes +5, Will +3

\textbf{Immunity to Damage} Fire, Poison

\textbf{Condition Immunity} poisoned

\textbf{Senses} Darkvision 18m

\textbf{Languages} Aquan, Ignan

\textbf{Challenge} 1/4 (50 XP)

\textit{\textbf{Innate Spells (1/day).}} The mephit can innately cast \textit{blur}, requiring no material components. His innate spellcasting ability is Charisma.

\textit{\textbf{Elemental nature.}} A mephit needs no food, drink, or sleep.

\textit{\textbf{Deadly Blast.}} When the mephit dies, it explodes into a cloud of vapor. Each creature within 1 meter of it must succeed on a DC 10 Reflex save or take 4 (1d8) fire damage.

\textbf{Actions}

\textit{\textbf{Claws.} Melee Weapon Attack}: +2 to hit, reach 1 meter, one creature.

\textit{Hit:} 2 (1d4) slashing damage plus 2 (1d4) fire damage.

\textit{\textbf{Vaporous Breath (Cooldown 6).}} The mephit exhales a 5-meter cone of hot vapor. Each creature in the area must make a DC 10 Reflex save, taking 4 (1d8) fire damage on a failed save, or half as much damage on a successful one.

\textbf{Ecology}\\
Environment any (elemental plane of fire)\\
Organization: Solitary, pair, group (3-6), or flock (7-12)\\
\textbf{Treasure}: Standard\\
\textbf{Description}\\
Mephits are the servants of powerful elemental creatures. The key sites and clearings of the elemental planes are filled with mephits scrambling to perform some important duty or errand.

Steam mephits are commonly found on the Plane of Fire. These mephits are defiant and dismissive.



\

\index[Monsters]{Mimic}\textbf{Mimic}

\textit{Medium monstrosity (shapeshifter), neutral}

\textbf{STRENGTH} +3

\textbf{DEXTERITY} +1

\textbf{CONSTITUTION} +2

\textbf{INTELLIGENCE} -3

\textbf{WISDOM} +1

\textbf{CHARISMA} -1

\textbf{Initiative} +1 -- \textbf{Defence} 13

\textbf{Hit Points} 58 (9d8 + 18)

\textbf{Movement} 5 meters

\textbf{Saving Throws}: Fortitude +5, Reflexes +5, Will +6

\textbf{Skills} Stealth +5

\textbf{Damage Immunity} acid

\textbf{Condition Immunity} prone

\textbf{Senses} Darkvision 18m

\textbf{Languages} -

\textbf{Challenge} 2 (450 XP)

\textit{\textbf{Clinging (Item Form Only).}} The mimic clings to anything it comes into contact with. A Huge or smaller creature that the mimic adheres to is considered grabbed by it (DC 13 to flee). The ability checks you make to escape from
this grabs have -1d6.

\textit{\textbf{Grabber.}} The mimic has +1d6 on attack rolls against a creature it grapples.

\textit{\textbf{False Appearance (Item Form Only).}} While the mimic remains motionless, he is indistinguishable from an ordinary object.

\textit{\textbf{Shapeshift.}} The mimic can use his action to transform into an object, or back to its true amorphous form. His stats are the same in any form. Any equipment he is wearing or carrying does not transform. Upon death it returns to its true form.

\textbf{Actions}

\textit{\textbf{Bite.} Melee Weapon Attack}: +6 to hit, reach 1m, one target.

\textit{Hit:} 7 (1d8 + 3) piercing damage plus 4 (1d8) acid damage.

\textit{\textbf{Pseudopod.} Melee Weapon Attack}: +6 to hit, reach 1m, one target.

\textit{Hit:} 7 (1d8 + 3) bludgeoning damage. If the mimic is in item form, the target is affected by the Clinging trait.

\textbf{Ecology}
Environment: Any\\
Organization: Solitary\\
\textbf{Treasure}: Accidental\\
\textbf{Description}\\
Mimics are believed to be the result of an alchemist's attempt to bring an inanimate object to life through the application of a mystical reagent, the formula for which has been lost. Over the years, these strange but intelligent creatures have learned the ability to transform themselves into simulacra of manufactured objects, particularly in places rarely frequented by a small number of creatures, where they increase their chances of success with an attack on their victims.

While mimics aren't inherently evil, some sages suggest that they attack humans and other intelligent creatures more for entertainment than for food. The desire to deceive others is part of their being, and their surprise attacks are the culmination of this desire.

A typical mimic has a volume of 2 cubic meters (1m by 1m by 2m) and weighs around 450kg. Legends and tales speak of mimics of larger sizes, with the ability to take the form of houses, ships, or entire subterranean complexes that they garnish with treasures (both real and fake) to lure their unsuspecting food inside.


\

\index[Monsters]{Minotaur}\textbf{Minotaur}

\textit{Large monstrosity, chaotic evil}

\textbf{STRENGTH} +4

\textbf{DEXTERITY} +0

\textbf{CONSTITUTION} +3

\textbf{INTELLIGENCE} -2

\textbf{WISDOM} +3

\textbf{CHARISMA} -1

\textbf{Initiative} +0 -- \textbf{Defence} 16

\textbf{Hit Points} 76 (9d10 + 27)

\textbf{Move} 12m

\textbf{Saving Throws}: Fortitude +6, Reflexes +5, Will +5

\textbf{Skills} Awareness +7

\textbf{Senses} Darkvision 18m

\textbf{Languages} Abysmal

\textbf{Challenge} 3 (700 XP)

\textit{\textbf{Charge.}} If the minotaur moves at least 3 meter directly towards a target and hits it with a gore attack during the same turn, the target takes an additional 9 (2d8) piercing damage. If the target is a creature, it must succeed on a DC 14 Fortitude save or be pushed up to 3 meter away and knocked prone.

\textit{\textbf{Reckless.}} At the start of its round, the minotaur can gain +1d6 on all attack rolls with melee weapons made during that turn, but attack rolls against it have + 1d6 until the start of its next round.

\textit{\textbf{Remember Labyrinth.}} The minotaur can perfectly remember any path he has traveled.

\textbf{Actions}

\textit{\textbf{Great Axe.} Melee Weapon Attack}: +8 to hit, reach 1m, one target.

\textit{Hit:} 17 (2d12 + 4) slashing damage.

\textit{\textbf{Gored.} Melee Weapon Attack}: +8 to hit, reach 1m, one target.

\textit{Hit:} 13 (2d8 + 4) piercing damage.

\textbf{Ecology}\\
Environment Temperate Ruins and Dungeons\\
Organization: Solitary, pair or group (3-4)\\
\textbf{Treasure}: Standard (Great Axe, other treasure)\\
\textbf{Description}\\
No one holds a grudge like a minotaur. Despised by civilized races and born centuries ago of a divine curse, minotaurs have hunted, killed, and devoured lesser humanoids to punish real or perceived offenses for longer than they can remember. Most cultures have legends of how minotaurs were created by vengeful or offended gods who punished humans by deforming their appearance, robbing them of their beauty and intelligence, and endowing them with bull heads. Yet most modern minotaurs despise these legends and believe they are not the joke of some deity, but models of divine perfection created by the cruel and powerful demon lord Baphomet.

Traditional minotaur hiding places are labyrinths, either mazes built to confuse and baffle, or natural ones created by a maze of caves or other underground passageways. Thanks to their natural cunning, minotaurs use their labyrinthine hiding places to deter unwary enemies who seek to hunt them down or who simply stumble upon their hiding places and become lost, slowly chasing down intruders who futilely try to find their way out. Only when desperation has clearly taken the upper hand does the minotaur strike his lost victims. When dealing with a group, minotaurs often let one creature escape to spread its terrible tale and lure others, who hope to slay these beasts, into their labyrinths. Naturally, to minotaurs, these would-be heroes make for delicious food.

Minotaurs can also be found in the service of a more powerful monster or evil creature, serving it as long as they can hunt and eat as they please. Generally this means guarding some powerful item or valuable location, but it can also mean working as a mercenary, hunting down the master's enemies.

Minotaurs are relatively straightforward combatants, using their horns to horribly gore the closest living creatures when they engage in combat.


\

\index[Monsters]{Minotaur Skeleton}\textbf{Minotaur Skeleton}

\textit{Large Undead, Lawful Evil}

\textbf{STRENGTH} +4

\textbf{DEXTERITY} +0

\textbf{CONSTITUTION} +2

\textbf{INTELLIGENCE} -2

\textbf{WISDOM} -1

\textbf{CHARISMA} -3

\textbf{Initiative} +0 -- \textbf{Defence} 13

\textbf{Hit Points} 67 (9d10 + 18)

\textbf{Move} 12m

\textbf{Saving Throws}: Fortitude +6, Reflexes +3, Will +2

\textbf{Vulnerability to Damage} from bludgeoning

\textbf{Immunity to Damage} Poison

\textbf{Damage Resistances} piercing and slashing

\textbf{Condition Immunity} poisoned, fatigue, bleeding

\textbf{Senses} Darkvision 18m

\textbf{Languages} understands the Abyssal but cannot speak

\textbf{Challenge} 2 (450 XP)

\textit{\textbf{Charge.}} If the minotaur skeleton moves at least 3 meter in a straight line toward the target and then hits it with a gore attack during the same turn, the target takes 9 (2d8) additional piercing damage. If the target is a creature, it must succeed on a DC 14 Fortitude save or be knocked back 3 meter and knocked prone.

\textit{\textbf{Undead nature.}} The skeleton requires no air, food, drink, or sleep.

\textbf{Actions}

\textit{\textbf{Great Axe.} Melee Weapon Attack}: +6 to hit, reach 1m, one target.

\textit{Hit:} 17 (2d12 + 4) slashing damage.

\textit{\textbf{Gored.} Melee Weapon Attack}: +6 to hit, reach 1m, one target.

\textit{Hit:} 13 (2d8 + 4) piercing damage.

\

\textbf{Monkey}\index[Monsters]{Monkey}

\textit{Little beast, unaligned}

\textbf{STRENGTH} -3

\textbf{DEXTERITY} +2

\textbf{CONSTITUTION} +0

\textbf{INTELLIGENCE} -3

\textbf{WISDOM} +1

\textbf{CHARISMA} -2

\textbf{Initiative} +2 -- \textbf{Defence} 13

\textbf{Hit Points} 3 (1d6)

\textbf{Movement} 9m, climb 9m

\textbf{Saving Throws}: Fortitude +0, Reflexes +3, Will +1

\textbf{Skills} Acrobatics +6, Awareness +3

\textbf{Languages} -

\textbf{Challenge} 1/4 (10 XP)

\textbf{Actions}

\textit{\textbf{Scratch.} Melee weapon attack}: +1 to hit, reach 1m, one target.

\textit{Hit:} 1 (1d4 - 1) slashing damage.

\textit{\textbf{Bite.} Melee Weapon Attack}: +1 to hit, reach 1 meter, one target.

\textit{Hit:} 2 (1d4) piercing damage.

\

\textbf{Moose}\index[Monsters]{Moose}

\textit{Large beast, unaligned}

\textbf{STRENGTH} +3

\textbf{DEXTERITY} +0

\textbf{CONSTITUTION} +1

\textbf{INTELLIGENCE} -4

\textbf{WISDOM} +0

\textbf{CHARISMA} -2

\textbf{Initiative} +0 -- \textbf{Defence} 11

\textbf{Hit Points} 13 (2d10 + 2)

\textbf{Move} 15m

\textbf{Saving Throws}: Fortitude +4, Reflexes +1, Will +0

\textbf{Languages} -

\textbf{Challenge} 1/4 (50 XP)

\textit{\textbf{Charge.}} If the moose moves at least 6 meters directly at the target and hits it with a bill attack during the same turn, the target takes an additional 7 (2d6) bludgeoning damage . If the target is a creature, it must succeed on a Fortitude save
DC 13 or fall prone.

\textbf{Actions}

\textit{\textbf{Beak.} Melee Weapon Attack}: +5 to hit, reach 1m, one target.

\textit{Hit:} 6 (1d6 + 3) bludgeoning damage.

\textit{\textbf{Hooves.} Melee Weapon Attack}: +5 to hit, reach 1m, one prone creature.

\textit{Hit:} 8 (2d4 + 3) bludgeoning damage.

\

\textbf{Mule}\index[Monsters]{Mule}

\textit{Medium beast, unaligned}

\textbf{STRENGTH} +2

\textbf{DEXTERITY} +0

\textbf{CONSTITUTION} +1

\textbf{INTELLIGENCE} -4

\textbf{WISDOM} +0

\textbf{CHARISMA} -3

\textbf{Initiative} +0 -- \textbf{Defence} 11

\textbf{Hit Points} 11 (2d8 + 2)

\textbf{Move} 12m

\textbf{Saving Throws}: Fortitude +3, Reflexes +1, Will +1

\textbf{Languages} -

\textbf{Challenge} 1/8 (25 XP)

\textit{\textbf{Beast of Burden.}} The mule is considered a Large animal for the purposes of determining its carrying capacity.

\textit{\textbf{Steady Feet.}} The mule has +1d6 on Fortitude and Reflex saves made against effects that would knock it prone.

\textbf{Actions}

\textit{\textbf{Hooves.} Melee Weapon Attack}: +2 to hit, reach 1m, one target.

\textit{Hit:} 4 (1d4 + 2) bludgeoning damage.

\

\index[Monsters]{Mummy, Mummy}\textbf{Mummy}

\textit{Medium Undead, Lawful Evil}

\textbf{STRENGTH} +3

\textbf{DEXTERITY} -1

\textbf{CONSTITUTION} +2

\textbf{INTELLIGENCE} -2

\textbf{WISDOM} +0

\textbf{CHARISMA} +1

\textbf{Initiative} -1 -- \textbf{Defence} 13

\textbf{Hit Points} 58 (9d8 + 18)

\textbf{Move} 6m

\textbf{Saving Throws}: Fortitude +4, Reflexes +2, Will +8

\textbf{Damage Vulnerability} fire

\textbf{Damage Resistances} from non-magical weapon

\textbf{Damage Immunity} Void, poison

\textbf{Condition Immunity} charmed, poisoned, paralyzed, fatigued, frightened, bleeding

\textbf{Senses} Darkvision 18m

\textbf{Languages} the languages he knew in life

\textbf{Challenge} 3 (700 XP)

\textit{\textbf{Undead Nature.}} A mummy needs no air, food, drink, or sleep.

\textbf{Actions}

\textit{\textbf{Multiattack.}} The mummy can use its Dreadful Glance and make an attack with its rotting fist.

\textit{\textbf{Rotting Fist.} Melee Weapon Attack}: +7 to hit, reach 1m, one target.

\textit{Hit:} 10 (2d6 + 3) bludgeoning damage plus 10 (3d6) void damage. If the target is a creature, it must succeed at a Fortitude 13 save or be cursed with mummy rot. The cursed target cannot regain Hit Points, and its maximum Hit Points decrease by 10 (3d6) for every 24 hours of the curse's duration. If the curse reduces the target's maximum Hit Points to 0, the target dies, and her body turns to dust. The curse lasts until removed by the spell \textit{remove curse} or other magic.

\textit{\textbf{Dreadful Glance.}} The mummy targets one creature it can see and is within 20 meters of it. If the target can see the mummy, it must succeed at a DC 12 Will save against this spell or be frightened until the end of the mummy's next round. If the target fails its Saving Throw by 5 or more, it is also paralyzed for the same duration. A target that succeeds at the Saving Throw is immune to the dire stare from all mummies (but not mummies overlords) for the next 24 hours.

\

\index[Monsters]{Mummy, Sovereign}\textbf{Sovereign Mummy}

\textit{Medium Undead, Lawful Evil}

\textbf{STRENGTH} +4

\textbf{DEXTERITY} +0

\textbf{CONSTITUTION} +3

\textbf{INTELLIGENCE} +0

\textbf{WISDOM} +4

\textbf{CHARISMA} +3

\textbf{Initiative} +0 -- \textbf{Defence} 25

\textbf{Hit Points} 97 (13d8 + 39)

\textbf{Move} 6m

\textbf{Saving Throws}: Fortitude +18, Reflexes +15, Will +19

\textbf{Skills} Religion +5, History +5

\textbf{Damage Vulnerability} fire

\textbf{Damage Immunity} Void, poison; weapons +1

\textbf{Condition Immunity} charmed, poisoned, paralyzed, fatigued, frightened, bleeding

\textbf{Senses} Darkvision 18m

\textbf{Languages} the languages he knew in life

\textbf{Challenge} 15 (13000 XP)

\textit{\textbf{Heart of the Mummy Sovereign.}} As part of the ritual that creates a mummy sovereign, the heart and entrails of the creature are removed from the corpse and placed inside sealed containers. These containers are usually made of stone or ceramic, engraved or painted with religious hieroglyphics.

As long as her withered heart remains intact, the mummy ruler cannot be permanently destroyed. When it drops to 0 Hit Points, the mummy king crumbles to dust and reforms at full strength 24 hours later, emerging from the dust near the sealed jar containing her heart. To prevent a sovereign mummy from reforming and destroying it once and for all, its heart must be reduced to ashes. For this reason, the sovereign mummy usually keeps its heart and entrails hidden inside a hidden tomb.

The heart of the mummy ruler has Defence 5, 25 Hit Points, and immunity to all damage except fire.

\textit{\textbf{Enchantments.}} The mummy has MP 10. Her spellcasting characteristic is Wisdom, +9 to hit on spell attacks. The mummy has the following spells prepared: Cantrips (at will): \textit{holy flame, thaumaturgy}

level 1 (4 slots): \textit{command, tracer bolt, shield of faith}

level 2 (3 slots): \textit{spiritual weapon, hold person, silence}

level 3 (3 slots): \textit{animate dead, dispel magic}

level 4 (3 slots): \textit{divination, guardian of faith}

level 5 (2 slots): \textit{contagion, insect plague}

level 6 (1 slot): \textit{wound}

\textit{\textbf{Undead Nature.}} A mummy needs no air, food, drink, or sleep.

\textit{\textbf{Resistance to Magic.}} The mummy ruler has +1d6 on Saving Throws against spells or other magical effects.

\textit{\textbf{Rejuvenation.}} A sovereign mummy forms a new body within 24 hours if her heart remains intact, regaining all Hit Points and being able to act again. The new body appears within 1 meter of the mummy ruler's heart.

\textbf{Actions}

\textit{\textbf{Multiattack.}} The mummy can use its Dreadful Glance and make one attack with its rotting fist.

\textit{\textbf{Rotting Fist.} Melee Weapon Attack}: +22 to hit, reach 1m, one target.

\textit{Hit:} 14 (3d6 + 4) bludgeoning damage plus 21 (6d6) void damage. If the target is a creature, it must succeed at a Fortitude 25 save or be cursed with mummy rot. The cursed target cannot regain Hit Points, and its maximum Hit Points decrease by 10 (3d6) for every 24 hours of the curse's duration. If the curse reduces the target's maximum Hit Points to 0, the target dies, and her body turns to dust. The curse lasts until removed by the spell \textit{remove curse} or other magic.

\textit{\textbf{Dreadful Glance.}} The mummy targets one creature it can see and is within 20 meters of it. If the target can see the mummy, she must succeed at a DC 22 Will save against this spell or be frightened until the end of the mummy's next round. If the target fails its Saving Throw by 5 or more, you are also paralyzed for the same duration. A target that succeeds at the Saving Throw is immune to the dire stare from all mummies (but not mummies overlords) for the next 24 hours.

\textbf{Additional Actions}

The mummy ruler can perform 3 additional Actions, chosen from the options below. She can only use one Additional Action at a time, and only at the end of another creature's turn. The mummy ruler regains expended additional Actions at the start of its round.

\textit{\textbf{Attack.}} The mummy ruler makes a rotting fist attack or uses her Dreadful Glance.

\textit{\textbf{Channel Negative Energy (Costs 2 Actions).}} The mummy ruler can magically unleash negative energy. Creatures within 20 meters of the mummy ruler, including those behind barriers or corners, cannot regain Hit Points until the end of the mummy ruler's next round.

\textit{\textbf{Blasphemous Word (Costs 2 Actions).}} The mummy ruler utters a blasphemous word. Each creature, other than undead, within 3 meter of the mummy ruler that can hear this magical phrase must succeed on a DC 22 Fortitude save or be stunned until the end of the mummy ruler's next round.

\textit{\textbf{Blinding Dust.}} Blinding dust and sand swirl magically around the mummy ruler. Each creature within 1 meter of the mummy overlord must succeed on a DC 22 Fortitude save or be blinded until the end of the creature's next round.

\textit{\textbf{Sand Whirlwind (Costs 2 Actions).}} The mummy ruler can magically transform into a sand whirlwind, moving up to 20 meters, and then returning to her normal form. While in whirlwind form, the mummy ruler is immune to all damage, and cannot be grabbed, petrified, knocked prone, restrained, or stunned. Equipment worn or carried by the mummy ruler remains in her possession.

\textit{\textbf{Enraged}}: The Sovereign Mummy is hungry for life. She channels the energy of death and destruction in a 40-foot radius around her. Each creature must make a DC 22 Fortitude save to take half or take 22 damage. The mummy regains all Hit Points lost by other creatures.


\textbf{Ecology}
Environment: Any\\
Organization: Solitary, Cort (3-6), or Flock (7-12)\\
\textbf{Treasure}: Double\\
\textbf{Description}\\
Many cultures practice the sacred art of mummification, though the sinister magical techniques used to imbue corpses with undead vitality are far less common. In some ancient lands, such blasphemous techniques have been honed through centuries of ceremonies and countless deaths, resulting in mummies of terrible power. On rare occasions, if the deceased was of great rank and excessive wickedness, he might undergo such elaborate rituals, rising from his grave as a fearsome mummy lord. Likewise, a ruler known for his malice or dead in a moment of great anger might spontaneously present himself as a vengeful despot. Regardless of the exact circumstances of her resurrection, a Sovereign mumm retains the abilities she had in life, becoming a creature consumed by the desire to restore her dominion and rule over both the living and the dead.

\

\index[Monsters]{Murderer's Shroud}\textbf{Murderer's Shroud}

\textit{Large Aberration, Chaotic Neutral}

\textbf{STRENGTH} +3

\textbf{DEXTERITY} +2

\textbf{CONSTITUTION} +1

\textbf{INTELLIGENCE} +1

\textbf{WISDOM} +1

\textbf{CHARISMA} +2

\textbf{Initiative} +2 -- \textbf{Defence} 18

\textbf{Hit Points} 78 (12d10 + 12)

\textbf{Move} 3m, fly 12m

\textbf{Saving Throws}: Fortitude +6, Reflexes +5, Will +7

\textbf{Skills} Stealth +5

\textbf{Senses} Darkvision 18m

\textbf{Languages} Language of the Depths

\textbf{Challenge} 8 (3900 XP)

\textit{\textbf{False Appearance.}} While the cloaker stands motionless without exposing its lower body, it is indistinguishable from a cloak of black leather.

\textit{\textbf{Sensitivity to Light}}. While in sunlight, the cloaker has -1d6 on attack rolls, as well as Wisdom (Awareness) checks based on sight.

\textit{\textbf{Transfer of Damage.}} While attached to a creature, the cloaker takes only half the damage dealt to it (round down), and the creature targeted by the cloaker takes the The other half.

\textbf{Actions}

\textit{\textbf{Multiattack.}} The cloaker makes two attacks:
one with the bite and one with the tail.

\textit{\textbf{Bite.} Melee Weapon Attack}: +12 to hit, reach 1m, one creature.

\textit{Hit:} 10 (2d6 + 3) piercing damage, and if the target is Large or smaller, the cloaker sticks to it.
While the cloaker is stuck on, it has a +1d6 on attack rolls. When cloaker rolls to attack and has the +1d6 bonus and hits, the target is blinded and unable to breathe. The cloaker can detach by expending 1 move action. A creature, including the target, can take its action to peel off the cloaker by making a successful Fortitude save with a DC 16 Strength modifier.


\textit{\textbf{Tail.} Melee Weapon Attack}: +12 to hit, reach 3m, one creature.

\textit{Hit:} 7 (1d8 + 3) slashing damage.

\textit{\textbf{Appearances (Recharge after 1 hour).}} When not in bright light, the cloaker creates three illusory duplicates of itself, which move with it and mimic its actions, swapping positions to make it impossible to figure out what the real killer cloak is. If the original is in an area of bright light, the duplicates vanish.

Whenever a creature targets the cloaker with an attack or harmful spell while duplicates are still present, that creature randomly determines whether it targets the cloaker or one of the duplicates. A creature that can't see or that relies on senses other than sight ignores this magical effect.

A duplicate has Defence and uses cloaker Saving Throws. If an attack hits a duplicate, or if a duplicate fails a Saving Throw against an effect that deals damage, it vanishes.

\textit{\textbf{Moan.}} Each creature within 20 meters of the cloaker that can hear its moan and that is not an aberration must succeed at a DC 13 Will save or be frightened until the end of the next round of the cloaker. If the creature's Saving Throw succeeds, the creature is immune to the cloaker's wail for the next 24 hours.

\textit{\textbf{Enraged}}: the Cloaker recharge its Appearances ability. Cost 1 Action.

\textbf{Ecology}
Environment: Dungeons\\
Organization: Solitary, Pair, Host (3-6), or Flock (7-12)\\
\textbf{Treasure}: Standard\\
\textbf{Description}\\
Resembling horribly evil flying manta rays, cloakers are mysterious and paranoid creatures. A typical specimen has a wingspan of 2.4 meters and weighs 50 kg.

Their motivations are mysterious and confusing, and they distrust even their own kind. The strange shape allows them to be mistaken for cloaks, wall hangings, or other commonplace items, and some stories tell of cloakers teaming up with other creatures, being carried on their backs and helping to protect their allies for inscrutable reasons. Some specimens are priests of ancient deities, leading cults of cloakers and skums intent on performing horrific rites with sinister purposes.


\

\index[Monsters]{Mushrooms, Purple Mushroom}\textbf{Purple Mushroom}

\textit{Medium plan, unaligned}

\textbf{STRENGTH} -4

\textbf{DEXTERITY} -5

\textbf{CONSTITUTION} +0

\textbf{INTELLIGENCE} -5

\textbf{WISDOM} -4

\textbf{CHARISMA} -5

\textbf{Initiative} -5 -- \textbf{Defence} 6

\textbf{Hit Points} 18 (4d8)

\textbf{Move} 2 sts

\textbf{Saving Throws}: Fortitude -3, Reflexes -3, Will -3

\textbf{Condition Immunity} blinded, deafened, frightened

\textbf{Senses} blindsight 9m (blind beyond this range)

\textbf{Languages} -

\textbf{Challenge} 1/4 (50 XP)

\textit{\textbf{False Appearance.}} While the violet mushroom remains motionless, it is indistinguishable from a normal mushroom.

\textbf{Actions}

\textit{\textbf{Multiattack.}} The mushroom makes 1d4 Putrid Touch attacks.

\textit{\textbf{Rot Touch.} Melee Weapon Attack}: +2 to hit, reach 3m, one target.

\textit{Hit:} 4 (1d8) void damage.

\textbf{Ecology}\\
Environment: Any dungeon\\
Organization: Solitary, pair or scrub (3-12)\\
\textbf{Treasure}: Accidental\\
\textbf{Description}\\
Purple mushrooms are one of the best known and most feared cave hazards. A traveler can often notice the marks left by the purple mushroom on those who live or hunt where these carnivorous mushrooms lurk. These deep, horrific scars look like furrows carved into the flesh—the marks of a close encounter with a purple mushroom.

A purple mushroom feeds on putrefied organic matter, but unlike most mushrooms it is not a passive consumer. The tendrils of a purple mushroom can strike with unexpected speed and are coated with a virulent poison that causes flesh to rot with sickening speed. This potent poison, if neglected, can quickly rot an entire arm or leg, leaving behind only bones that will soon corrode as well.

While purple mushrooms can move, they only do so to attack or hunt prey. A purple mushroom with a steady stream of rot to feed on is content to stay in one place. Many underground dwellers, particularly Troglodytes and Vegepygmies, use this behavior to their advantage and place multiple purple mushrooms at key junctions and entrances of their caves as guardians, making sure to supply them with enough corpses to prevent them from entering the shelter in search of food.

Some species of Shrieker Boleto look somewhat similar to purple mushrooms, although they lack sprawling branches. It's not uncommon to find screechers and purple mushrooms in the same tangle, especially in areas where other creatures cultivate these mushrooms as guardians.

A purple mushroom is 4 feet tall and weighs 25kg.


\

\index[Monsters]{Mushrooms, Screeching Mushroom}\textbf{Screeching Mushroom}

\textit{Medium plan, unaligned}

\textbf{STRENGTH} -5

\textbf{DEXTERITY} -5

\textbf{CONSTITUTION} +0

\textbf{INTELLIGENCE} -5

\textbf{WISDOM} -4

\textbf{CHARISMA} -5

\textbf{Initiative} -5 -- \textbf{Defence} 6

\textbf{Hit Points} 13 (3d8)

\textbf{Move} 0m

\textbf{Saving Throws}: Fortitude -3, Reflexes +3, Will -4

\textbf{Condition Immunity} blinded, deafened, frightened

\textbf{Senses} blindsight 9m (blind beyond this range)

\textbf{Languages} -

\textbf{Challenge} 0 (10 XP)

\textit{\textbf{False Appearance.}} While the screeching mushroom remains motionless, it is indistinguishable from a normal mushroom.

\textbf{Actions}

\textit{\textbf{Screech.}} When a bright light or creature is within 10 meters of the shriek mushroom, it emits a shriek audible up to 100 meters away. The screeching mushroom continues to screech until the source of the disturbance has moved out of range and for another 1d4 rounds thereafter, or until the hat has deflated.

\textbf{Ecology}\\
Environment: Any dungeon\\
Organization: Solitary, pair or scrub (3-12)\\
\textbf{Treasure}: Accidental\\
\textbf{Description}\\
A screeching mushroom is about 50 cm tall, with a broad brown cap. Once the scream is emitted, the hat deflates.

It is said of Duergar cooks specialized in cooking these mushrooms in superfine dishes. The best ones also manage not to deflate the hat.

\

\index[Monsters]{Naga, Guardian}\textbf{Guardian Naga}

\textit{Large Monstrosity, Lawful Good}

\textbf{STRENGTH} +4

\textbf{DEXTERITY} +4

\textbf{CONSTITUTION} +3

\textbf{INTELLIGENCE} +3

\textbf{WISDOM} +4

\textbf{CHARISMA} +4

\textbf{Initiative} +4 -- \textbf{Defence} 23

\textbf{Hit Points} 127 (15d10 + 45)

\textbf{Move} 12m

\textbf{Saving Throws}: Fortitude +12, Reflexes +14, Will +14

\textbf{Immunity to Damage} Poison

\textbf{Condition Immunity} charmed, poisoned

\textbf{Senses} Darkvision 18m

\textbf{Languages} Celestial, Common

\textbf{Challenge} 10 (5900 XP)

\textit{\textbf{Enchantments.}} The naga has MP 11. His spellcasting ability is Wisdom (+8 to hit with spell attacks), and he needs only the verbal components to cast his spells. The naga prepares the following spells:

Cantrips (at will): \textit{holy flame, mend, thaumaturgy}

level 1 (4 slots): \textit{command, cure wounds, shield of faith}

level 2 (3 slots): \textit{block person, calm emotions}

level 3 (3 slots): \textit{clairvoyance, bestow curse}

level 4 (3 slots): \textit{exile, freedom of movement}

level 5 (2 slots): \textit{Flame Strike, geas}

level 6 (1 slot): \textit{true seeing}

\textit{\textbf{Rejuvenation.}} If he dies, the naga returns to life in 1d6 days and regains all its Hit Points. Only the spell \textit{wish} can stop this trait from working.

\textbf{Actions}

\textit{\textbf{Bite.} Melee Weapon Attack}: +14 to hit, reach 3m, one creature.

\textit{Hit:} 8 (1d8 + 4) piercing damage, and the target must make a DC 17 Fortitude Saving Throw, taking 45 (10d8) poison damage on a failed save, or half as much damage on a failed save. he succeeds.

\textit{\textbf{Spit poison.} Ranged weapon attack}: +14 to hit, range 5m, one creature.

\textit{Hit:} The target must make a DC 17 Fortitude save, taking 45 (10d8) poison damage on a failed save, or half as much damage on a successful one.

\textbf{Ecology}\\
Environment: Temperate Plains\\
Organization: Solitary, pair, or nest (3-6)\\
\textbf{Treasure}: Standard\\
\textbf{Description}\\
Though fierce in appearance, with gleaming scales, cobra-like hoods, and powerful serpentine bodies, guardian naga serve as conscientious protectors of places of exceptional power and sacredness. Their scales often feature elaborate designs similar to those of exotic jungle snakes. A typical naga guardian reaches lengths of 14 feet and an approximate weight of 150kg.

While some naga guardians adhere to the exotic practices of ancient or forgotten deities, others are simply drawn to sites of marked natural beauty, such as temples on towering waterfalls, natural pinnacles, and mountaintops, guarding them with the utmost reverence and a sense of duty. Often these nagas join active faiths, serving as protectors of shrines or ancient treasures. A pair of naga may settle near a site they deem worthy of protection, brood there and raise their offspring. When young people reach adulthood, they may choose to leave to find their homes or stay to protect the area guarded by their parents. Sometimes, a naga guardian guarding a ruin or temple is just the latest in a succession of sentinels that have come and gone over the centuries. These sentinels often go by the same name as their predecessors, appearing to be a single, exceptionally long-lived individual.


\

\index[Monsters]{Naga, Spirit}\textbf{Naga Spirit}

\textit{Large monstrosity, chaotic evil}

\textbf{STRENGTH} +4

\textbf{DEXTERITY} +3

\textbf{CONSTITUTION} +2

\textbf{INTELLIGENCE} +3

\textbf{WISDOM} +2

\textbf{CHARISMA} +3

\textbf{Initiative} +3 -- \textbf{Defence} 19

\textbf{Hit Points} 75 (10d10 + 20)

\textbf{Move} 12m

\textbf{Saving Throws}: Fortitude +8, Reflexes +10, Will +10

\textbf{Immunity to Damage} Poison

\textbf{Condition Immunity} charmed, poisoned

\textbf{Senses} Darkvision 18m

\textbf{Languages} Abyssal, Common

\textbf{Challenge} 8 (3900 XP)

\textit{\textbf{Spells.}} The naga has MP 10. His spellcasting ability is Intelligence (+6 to hit with spell attacks), and he needs only the verbal components to cast his spells . The naga prepares the following spells:

Cantrips (at-will): \textit{minor illusion, mage hand, ray of} \textit{frost}

level 1 (4 slots): \textit{charm person, detect magic,} \textit{sleep}

level 2 (3 slots): \textit{hold person, detect thoughts}

level 3 (3 slots): \textit{lightning bolt, water breathing}

level 4 (3 slots): \textit{wither, dimension door}

level 5 (2 slots): \textit{dominate people}

\textit{\textbf{Rejuvenation.}} If he dies, the naga returns to life in 1d6 days and regains all its Hit Points. Only the spell \textit{wish} can stop this trait from working.

\textbf{Actions}

\textit{\textbf{Bite.} Melee Weapon Attack}: +12 to hit, reach 3m, one creature.

\textit{Hit:} 7 (1d8 + 4) piercing damage, and the target must make a DC 16 Fortitude Saving Throw, taking 31 (7d8) poison damage on a failed save, or half as much damage on a failed save. he succeeds.


\

\index[Monsters]{Ogre}\textbf{Ogre}

\textit{Large giant, chaotic evil}

\textbf{STRENGTH} +4

\textbf{DEXTERITY} -1

\textbf{CONSTITUTION} +3

\textbf{INTELLIGENCE} -3

\textbf{WISDOM} -2

\textbf{CHARISMA} -2

\textbf{Initiative} -1 -- \textbf{Defence} 12 (leather Armour)

\textbf{Hit Points} 59 (7d10 + 21)

\textbf{Move} 12m

\textbf{Saving Throws}: Fortitude +6, Reflexes +0, Will +1

\textbf{Senses} Darkvision 18m

\textbf{Languages} Common, Giant

\textbf{Challenge} 2 (450 XP)

\textbf{Actions}

\textit{\textbf{Club.} Melee Weapon Attack}: +6 to hit, reach 1m, one target.

\textit{Hit:} 13 (2d8 + 4) bludgeoning damage.

\textit{\textbf{Javelin.} Melee or Ranged weapon attack}: +6 to hit, reach 1m or range 12m, one target.

\textit{Hit:} 11 (2d6 + 4) piercing damage.

\textbf{Ecology}\\
Environment: Cold or temperate hills\\
Organization: Solitary, couple, group (3-4) or family (5-16)\\
\textbf{Treasure}: Standard (Leather Armour, Cudgel, 4 Javelins, other)\\
\textbf{Description}\\
In the stories about ogres there are elements of horror: brutality and savagery, cannibalism and torture. Then rape, dismemberment, necrophilia, incest, mutilation and other examples of cruelty. Those who have never encountered ogres take these stories as a warning. Anyone who has survived such an encounter knows that stories are nothing compared to reality.

Ogres enjoy the suffering of others. If they don't have the smaller races at their disposal to crush between their fat hands or to violate in violent embraces, they enjoy each other. There are no taboos for ogres. One would think that, left to its own devices, an ogre tribe would tear itself apart and that only the strongest would survive, but if there's one thing ogres respect, it's family.

Ogre tribes are known as families, and many of their deformities are caused by the common practice of incest. The chieftain of the tribe is often the father, but in some cases a female ogre is able to claim the title of mother. Ogre tribes fight among themselves, which keeps them busy and prevents them from tormenting their neighbors. From time to time, however, a particularly violent or feared patriarch emerges, capable of uniting several families under his command.

The regions inhabited by ogres are sad and degraded places, as these giants live in squalor and feel no need to be in harmony with their surroundings. The border between the civilized lands and those of the ogres is a place of despair inhabited by outcasts, where the Ogremanni live, deformed offspring born from the raids that the ogres carry out on the lands of humans.

The games of the ogres are violent and cruel: the victims used as toys are lucky to die on the first day. The ogres' cruel sense of humor is the only time they show creativity: the methods and tools of ogre torture are straight out of nightmares.

Their great strength and lack of imagination make them particularly suitable for heavy work, in mines, blacksmithing or logging. The more powerful giants (especially those of the Hills and Rocks) often subjugate ogre families to serve them.

An adult ogre stands about 3 meter tall and weighs approximately 350kg.


\

\index[Monsters]{Oni}\textbf{Oni}

\textit{Large Giant, Lawful Evil}

\textbf{STRENGTH} +4

\textbf{DEXTERITY} +0

\textbf{CONSTITUTION} +3

\textbf{INTELLIGENCE} +2

\textbf{WISDOM} +1

\textbf{CHARISMA} +2

\textbf{Initiative} +2 -- \textbf{Defence} 20 (chainmail)

\textbf{Hit Points} 110 (13d10 + 39)

\textbf{Move} 9m, fly 9m

\textbf{Saving Throws}: Fortitude +7, Reflexes +4, Will +6

\textbf{Skills} Arcane +5, Deceive +8, Awareness +4

\textbf{Senses} Darkvision 18m

\textbf{Languages} Common, Giant

\textbf{Challenge} 7 (2900 XP)

\textit{\textbf{Magic Weapons.}} The oni's weapon attacks are magical.

\textit{\textbf{Innate Spells.}} The oni's spellcasting ability is Charisma. The oni can cast these spells innately, requiring no material components:

At will: \textit{invisibility, darkness}

1/day: \textit{charm person, cone of cold, gaseous form, sleep}

\textit{\textbf{Regeneration.}} If the oni has at least 1 hit point, it regains 10 Hit Points at the start of its round.

\textbf{Actions}

\textit{\textbf{Multiattack.}} The oni makes two attacks, with its claws or glaive.

\textit{\textbf{Claw (Oni Form Only).} Melee Weapon Attack}: +11 to hit, reach 1m, one target. \textit{Hit:} 8 (1d8 + 4) slashing damage.

\textit{\textbf{Glachion.} Melee Weapon Attack}: +7 to hit, reach 3m, one target.

\textit{Hit:} 15 (2d10 + 4) slashing damage, or 9 (1d10 + 4) slashing damage in Small or Medium form.

\textit{\textbf{Shapeshift.}} The oni can magically transform into a Small or Medium humanoid, into a Large giant, or revert to its true form. Aside from size, its stats are the same in each form. The only equipment that is transformed is the glaive, which shrinks to be wielded even in humanoid form. If the oni dies, it reverts to its true form, and the glaive reverts to its original size.

\textit{\textbf{Enraged}}: the Oni is filled with a murderous fury, until the end of the fight his Claw attacks cause Bleeding 2/10.


\

\index[Monsters]{Orc}\textbf{Orc}

\textit{Medium humanoid (orc), chaotic evil}

\textbf{STRENGTH} +3

\textbf{DEXTERITY} +1

\textbf{CONSTITUTION} +3

\textbf{INTELLIGENCE} -2

\textbf{WISDOM} +0

\textbf{CHARISMA} +0

\textbf{Initiative} +1 -- \textbf{Defence} 14 (leather Armour)

\textbf{Hit Points} 18 (3d8 + 6)

\textbf{Move} 9m

\textbf{Saving Throws}: Fortitude +3, Reflexes +1, Will +2

\textbf{Skills} Intimidate +2

\textbf{Senses} Darkvision 18m

\textbf{Languages} Common, Goblinoid

\textbf{Challenge} 1 (100 XP)

\textit{\textbf{Aggressive.}} As a bonus action, the orc can move up to half its move toward a hostile creature it can see.

\textbf{Actions}

\textit{\textbf{Great Axe.} Melee Weapon Attack}: +5 to hit, reach 1m, one target.

\textit{Hit:} 9 (1d12 + 3) slashing damage.

\textit{\textbf{Javelin.} Melee or Ranged weapon attack}: +5 to hit, reach 1m or range 12m, one target. \textit{Hit:} 6 (1d6 + 3) piercing damage.

\textbf{Ecology}\\
Environment: Temperate hills and mountains or underground\\
Organization: Solitary, group (2-4), squad (11-20 plus 2 sergeants 3rd level and 1 leader 3rd-6th level), or gang \\
\textbf{Treasure}: NPC gear (Studded Leather Armour, Glaive, 4 Javelins, other treasure)\\
\textbf{Description}\\
The main difference between orcs and civilized humanoids, other than their brute strength and inferior intelligence, is their tempers. As a culture, orcs are violent and aggressive, and the strong dominate the weak through fear and brutality. They take what they want by force and have no qualms about taking entire villages as slaves if they get the chance. They care little for comforts, and their villages and fields tend to be filthy and precarious places filled with drunken brawls, fighting arenas, and other sadistic amusements. Lacking the patience to farm and capable of raising only the hardiest and most self-sufficient animals, orcs find it easier to take the fruits of their labor from others. They are arrogant and quick to rage when challenged, but are concerned with honor only as long as doing so benefits them.

An adult male orc stands 2 meters tall and weighs approximately 115 kg. Orcs and humans can mate, though this usually occurs during raids, and not as a consensual union. Many orc tribes breed half-orcs on purpose, as they make excellent strategists and chieftains.

Although the vulgate says that the orcs were created by Cattalm to destroy and bring chaos, it is also true that very often they are the victim of prejudices and summary judgments. Not all ogres are the same and not only physically, individual orcs if not entire tribes live their existence in a "normal, civilized" way yet in no state of Yeru are there penalties for those who kill an ogre.

\

\textbf{Killer Whale (Orca)}\index[Monsters]{Orca}

\textit{Huge beast, unaligned}

\textbf{STRENGTH} +4

\textbf{DEXTERITY} +0

\textbf{CONSTITUTION} +1

\textbf{INTELLIGENCE} -4

\textbf{WISDOM} +1

\textbf{CHARISMA} -2

\textbf{Initiative} +0 -- \textbf{Defence} 14

\textbf{Hit Points} 90 (12d12 + 12)

\textbf{Movement} 0m, swim 18m

\textbf{Saving Throws}: Fortitude +9, Reflexes +8, Will +5

\textbf{Skills} Awareness +3

\textbf{Senses} blind sight 36 m

\textbf{Languages} -

\textbf{Challenge} 3 (700 XP)

\textit{\textbf{Echolocation.}} The whale cannot use blindsight when deafened.

\textit{\textbf{Hold Breath.}} Whale can hold its breath for 30 minutes

\textit{\textbf{Honed Hearing.}} The whale has +1d6 on Wisdom (Awareness) checks based on hearing.

\textbf{Actions}

\textit{\textbf{Bite.} Melee Weapon Attack}: +6 to hit, reach 1m, one target.

\textit{Hit:} 21 (5d6 + 4) piercing damage.

\

\index[Monsters]{Otyugh}\textbf{Otyugh}

\textit{Large Aberration, Neutral}

\textbf{STRENGTH} +3

\textbf{DEXTERITY} +0

\textbf{CONSTITUTION} +4

\textbf{INTELLIGENCE} +2

\textbf{WISDOM} +1

\textbf{CHARISMA} -2

\textbf{Initiative} +0 -- \textbf{Defence} 17

\textbf{Hit Points} 114 (12d10 + 48)

\textbf{Move} 9m

\textbf{Saving Throws}: Fortitude +3, Reflexes +2, Will +6

\textbf{Senses} darkvision 36m

\textbf{Languages} Otyugh

\textbf{Challenge} 5 (1800 XP)

\textit{\textbf{Limited telepathy.}} The otyugh can magically transmit simple messages and images to any creature within 16 meters of it that can understand a language. This form of telepathy does not allow the receiving creature to respond telepathically.

\textbf{Actions}

\textit{\textbf{Multiattack.}} The otyugh makes three attacks: one with its bite and two with its tentacles.

\textit{\textbf{Bite.} Melee Weapon Attack}: +9 to hit, reach 1m, one target.

\textit{Hit:} 12 (2d8 + 3) piercing damage. If the target is a creature, it must succeed on a DC 15 Fortitude save against disease. Every 24 hours thereafter, the target must repeat the Saving Throw, reducing its maximum Hit Points by 5 (1d10) on a failure. If the Saving Throw succeeds, the disease is gone. The target dies if the disease reduces its maximum Hit Points to 0.

This reduction to the character's maximum Hit Points lasts until the disease is cured.

\textit{\textbf{Tentacle.} Melee Weapon Attack}: +9 to hit, reach 3m, one target.

\textit{Hit:} 7 (1d8 + 3) bludgeoning damage plus 4 (1d8) piercing damage. If the target is Medium or smaller, it is grappled (DC 13 to flee) and restrained until the grapple ends. The otyugh has two tentacles, each of which can grip a different target.

\textit{\textbf{Tentacle Smash.}} The otyugh slams creatures caught in its tentacles, into each other or to the floor. Each creature must succeed on a DC 15 Fortitude save or take 10 (2d6 + 3) bludgeoning damage and be stunned until the end of the otyugh's next round. On a successful save, the target takes half damage from the bludgeon and is not stunned.

\textit{\textbf{Enraged}}: the otyugh emits a scent that dulls the senses. All creatures within a 20-foot radius must make a DC 18 Will save or act randomly, as  \hyperlink{incconfusione}{Confusion} spell (page \pageref{incconfusione}), till end of next round. Cost 2 Actions.


\textbf{Ecology}\\
Environment any dungeon\\
Organization: Solitary, pair or group (3-4)\\
\textbf{Treasure}: Standard\\
\textbf{Description}\\
Otyughs are particularly filthy and horrid creatures that live in places sane people tend to avoid. Their lairs are found in sewers, cesspools, garbage dumps, and the foulest swamps: the dirtier a place, the more it attracts otyughs. They love the role of scavenger, and wander through the underground caverns in search of new tidbits in the midst of waste. Once found, they gorge themselves and bring back to their lair what they cannot consume in one go. Otyughs spend considerable time in their filthy lairs, which they fill with carrion and dung, which release fetid fumes.

Intelligent creatures that dwell underground near otyughs sometimes form alliances of convenience with them. They supply waste and raw meat to the otyughs, making them quite a means of disposal. In return, the otyughs leave their benefactors alone, do not attack them, and can even act as guardians.

What most races find terrifying about otyughs isn't their diet or the smell of their burrows, but the fact that creatures with their tastes aren't just mindless scavengers. In fact, otyughs are surprisingly intelligent, and love to form alliances with those who supply them with foods more refined than manure and dirt. Most otyughs realize that other creatures find them revolting, but there are few who really care.

\

\textbf{Owl}\index[Monsters]{Owl}

\textit{Tiny beast, unaligned}

\textbf{STRENGTH} -4

\textbf{DEXTERITY} +1

\textbf{CONSTITUTION} -1

\textbf{INTELLIGENCE} -4

\textbf{WISDOM} +1

\textbf{CHARISMA} -2

\textbf{Initiative} +1 -- \textbf{Defence} 12

\textbf{Hit Points} 1 (1d4 - 1)

\textbf{Move} 1m, fly 18m

\textbf{Saving Throws}: Fortitude +2, Reflexes +5, Will +2

\textbf{Skills} Stealth +3, Awareness +3

\textbf{Senses} vision in the dark 36 m

\textbf{Languages} -

\textbf{Challenge} 0 (10 XP)

\textit{\textbf{Fly over.}} The owl does not provoke attacks of opportunity when flying out of an enemy's reach.

\textit{\textbf{Honed hearing and vision.}} The owl has +1d6 on Wisdom (Awareness) checks based on hearing or vision.

\textbf{Actions}

\textit{\textbf{Spurs.} Melee Weapon Attack}: +3 to hit, reach 1m, one target.

\textit{Hit:} 1 slashing damage.

\

\index[Monsters]{Owlbear}\textbf{Owlbear}

\textit{Large beast, unaligned}

\textbf{STRENGTH} +5

\textbf{DEXTERITY} +1

\textbf{CONSTITUTION} +3

\textbf{INTELLIGENCE} -4

\textbf{WISDOM} +1

\textbf{CHARISMA} -2

\textbf{Initiative} +1 -- \textbf{Defence} 15

\textbf{Hit Points} 59 (7d10 + 21)

\textbf{Move} 12m

\textbf{Saving Throws}: Fortitude +10, Reflexes +5, Will +2

\textbf{Skills} Awareness +3

\textbf{Senses} Darkvision 18m

\textbf{Languages} -

\textbf{Challenge} 3 (700 XP)

\textit{\textbf{Honed sense of smell and vision.}} The Owlbear has +1d6 on Wisdom (Awareness) checks based on smell or sight.

\textbf{Actions}

\textit{\textbf{Multiattack.}} The Owlbear makes two attacks: one with its beak and one with its claws.

\textit{\textbf{Claws.} Melee Weapon Attack}: +9 to hit, reach 1m, one target.

\textit{Hit:} 14 (2d8 + 5) slashing damage.

\textit{\textbf{Beak.} Melee Weapon Attack}: +9 to hit, reach 3 ft., one creature.

\textit{Hit:} 10 (1d10 + 5) piercing damage.

\textbf{Ecology}\\
\textbf{Environment: Temperate Forests}
Organization: Solitary, pair, or pack (3-8)\\
\textbf{Treasure}: Accidental\\
\textbf{Description}\\
The origins of the Owlbear are a matter of debate among scholars of monstrous creatures. Most of them agree that it was a wizard, in the past, who created the first specimen by combining a bear with a giant owl; perhaps as an experiment in some insane concept of the nature of life, but more likely due to his utter madness. Whatever the original purpose of a creation as insane as the Owlbear, the creature has begun to reproduce, and has become one of the best-known predators of the woodlands.

Owlbears are savage predators, known for their bad tempers, aggression and ferocity. They tend to attack anything that moves in front of them, even if this does not show warlike intentions. Many scholars who have encountered these creatures in the wilds have noted that they always have bloodshot eyes that whirl around just before an attack. This is generally seen as a sign of insanity, suggesting that all Owlbears are born with a pathological need to fight and kill, but more realistic researchers believe it is due to the structure of their sharp eyes.

Owlbears inhabit the innermost and most hidden areas of the woods, and make their lairs inside tangled forests or deep, dark caves. They can hunt both during the day and at night, depending on the habits of the preys that populate the territories surrounding their den.

Adult Owlbears live in pairs and hunt prey in packs, leaving the cubs in their dens. Typically 1d6 cubs can be found in a den, which can be worth up to 750 gp in city markets.

Although it is almost impossible to tame them due to their wild nature, Owlbears can be exploited as guardians of a specific territory, provided they are left free to move within it to hunt. Professional trainers charge up to 2,000 gp to train an Owlbear to become a guardian who obeys simple commands (DC 23 for a baby Owlbear, DC 30 for an adult Owlbear).\\

\textit{\textbf{Variant}}: \textbf{Polar Owlbear}\index{Polar Owlbear}\\
This Owlbear is found in snowy mountainous or arctic regions. Unlike the normal Owlbear he is sturdier and stronger. Has 85 HP, +10 to hit, 21 claw damage +1 from Bleeding, 15 beak damage. GS 4

\
\index[Monsters]{Panoptikhan}\textbf{Panoptikhan}

\textit{Large aberration, evil}

\textbf{STRENGTH} +0

\textbf{DEXTERITY} +1

\textbf{CONSTITUTION} +2

\textbf{INTELLIGENCE} +3

\textbf{WISDOM} +2

\textbf{CHARISMA} +2

\textbf{Initiative} +5 -- \textbf{Defence} 26

\textbf{Hit Points} 82 (11d8 + 38)

\textbf{Move} 1m, fly 10m (good)

\textbf{Saving Throws}: Fortitude +14, Reflexes +13, Will +14

\textbf{Resistance}: Acid, Electricity

\textbf{Senses} darkvision 36m, True Seeing 18m

\textbf{Languages} telepathy 50m

\textbf{Challenge} 12 (8400 XP)

\textbf{Actions}

\textit{\textbf{Multiattack.}} The Panoptikhan can attack with two short tentacles.

\textit{\textbf{Tentacle.} Melee Weapon Attack}: +12 to hit, reach 1m, one target.

\textit{Hit:} 6 (1d6 + 3) piercing slashing damage.

\textit{\textbf{He who sees all}}. The Panoptikhan can activate one of its eye tentacles (2 Actions).

\textit{The one that freezes}: The eye points to a target within 18 meters, a ray of frost is activated on them. 8d8 cold damage, DC 23 Reflex save to completely avoid the blow.

\textit{The one that melts}: The eye points to a target within 10 meters, a beam is activated on them that has acid effects. 4d8 acid damage, DC 23 Reflex save for half damage.

\textit{The one that burns}: The eye points to a target within 18 meters, a fiery beam is activated on them. 8d8 fire damage, DC 23 Reflex save to completely avoid the blow.

\textit{The one that paralyzes}: The eye targets a target within 10 meters, a beam is activated on that target that paralyzes the creature. Will save DC 23 to completely avoid the effects.

\textit{The one that slows down}: Eye points in a 10m cone. A slowing beam is shot at affected creatures. Will save DC 23 to completely avoid the effects. Duration 1 minute.

\textit{The one that confuses}: The eye points in a 20m cone. A beam is blasted at affected creatures causing confusion. Will save DC 23 to completely avoid the effects. Duration 1 minute, each round it is possible to make a new Saving Throw to recover from the effects.

\textit{The one that puts to sleep}: The eye points to a target within 36 meters, a beam is activated on this, putting the creature to sleep. Will save DC 23 to completely avoid the effects.

\textit{The one that moves}; this eye can manifest the spell Mage Hand or Telekinesis.

\textit{\textbf{One look.}} The Panoptikhan activates the focus eye. The central eye can be used as a reaction action to cast Counterspell to a spell he has seen cast.

\textit{\textbf{Enraged}}: Panoptikhan in blindest fury activates 1d6 random eyes on random targets. Cost 3 Actions.

\textbf{Ecology}\\
Environment any dungeon\\
Organization: Solitary, Pair\\
\textbf{Treasure}: Triple\\
\textbf{Description}\\
The Panoptikhans are xenophobic aberrations, balls of hard flying meat equipped with a large central eye, a large mouth and 7 tentacles about 1 meter long each with an eye (about 10 cm in diameter) of a different color.

Little is known of the origin of the Panoptikhans, they are thought to be an evolutionary experiment of Calicante, in an attempt to create a sentient and dominant race.

Unfortunately arrogance, arrogance, the desire to be the center of attention have wrecked these attempts at society and the Panoptikhans have often disappeared underground.

The Panoptikhans have a very long life, in the order of a thousand years but they are also creatures that have more than doubled this limit. Panoptikhans increase in size with age and so do the number of eyes. The statistics reported here refer to an adult specimen of about 300 years old.


\

\textbf{Panther}\index[Monsters]{Panther}

\textit{Medium beast, unaligned}

\textbf{STRENGTH} +2

\textbf{DEXTERITY} +2

\textbf{CONSTITUTION} +0

\textbf{INTELLIGENCE} -4

\textbf{WISDOM} +2

\textbf{CHARISMA} -2

\textbf{Initiative} +2 -- \textbf{Defence} 13

\textbf{Hit Points} 13 (3d8)

\textbf{Movement} 15m, climb 12m

\textbf{Saving Throws}: Fortitude +3, Reflexes +5, Will +3

\textbf{Skills} Stealth +6, Awareness +4

\textbf{Languages} -

\textbf{Challenge} 1/4 (50 XP)

\textit{\textbf{Leap.}} If the panther moves at least 6 meters directly towards a creature and hits it with a claw attack during the same turn, the target must succeed on a DC 12 Fortitude save or fall prone. If the target is prone, the panther can make a bite attack against it as a bonus action.

\textit{\textbf{A keen sense of smell.}} The panther has +1d6 on Wisdom (Awareness) checks based on smell.

\textbf{Actions}

\textit{\textbf{Claw.} Melee Weapon Attack}: +4 to hit, reach 1m, one target.

\textit{Hit:} 4 (1d4 + 2) slashing damage, 1 bleed damage.

\textit{\textbf{Bite.} Melee Weapon Attack}: +4 to hit, reach 1m, one target.

\textit{Hit:} 5 (1d6 + 2) piercing damage.


\

\index[Monsters]{Pegasus}\textbf{Pegasus}

\textit{Large Celestial, Chaotic Good}

\textbf{STRENGTH} +4

\textbf{DEXTERITY} +2

\textbf{CONSTITUTION} +3

\textbf{INTELLIGENCE} +0

\textbf{WISDOM} +2

\textbf{CHARISMA} +1

\textbf{Initiative} +2 -- \textbf{Defence} 13

\textbf{Hit Points} 59 (7d10 + 21)

\textbf{Move} 18m, fly 27m

\textbf{Saving Throws} Fortitude +7, Reflexes +6, Will +4

\textbf{Skills} Awareness +6

\textbf{Languages} understands Celestial, Common, Elven, and Sylvan but cannot speak

\textbf{Challenge} 2 (450 XP)

\textbf{Actions}

\textit{\textbf{Hooves.} Melee Weapon Attack}: +6 to hit, reach 1m, one target.

\textit{Hit:} 11 (2d6 + 4) bludgeoning damage.

\textbf{Ecology}
Environment: Temperate and Warm Plains\\
Organization: Solitary, pair, or herd (6-10)\\
\textbf{Treasure}: None\\
\textbf{Description}\\
Pegasus is a magnificent winged horse that sometimes serves the cause of good. While highly prized as flying mounts, pegasi are shy creatures that are unlikely to make friends. A typical pegasus stands 1.8 meters tall at the withers, weighs 750 kg, and has a wingspan of 6 meters. Most pegasi are white, but occasionally some pegasi have different colors.

Pegasus, despite appearances, is as intelligent as a human. Those who try to train one to act as a mount will find that the pegasus is recalcitrant and even violent. A pegasus cannot speak, but it understands Common and prefers the company of good creatures. The correct way to get a pegasus to act as a mount is to befriend it with Diplomacy, favors, and good deeds. A pegasus usually has an indifferent attitude towards good creatures, ill-disposed towards neutral ones, and hostile towards evil ones. Before it can serve as a mount, a pegasus must be made friendly with a Diplomacy check or otherwise. Riding a pegasus requires an exotic saddle or bareback riding, as a normal saddle interferes with its wings. A pegasus can fight while carrying a rider, but the rider cannot attack in round unless he succeeds on a Ride check. Trained pegasus do not fear combat, and the rider need not make a Ride check to master it.

Pegasi lay eggs that are worth 1,000 gp each on the market, while young ones cost 2,000 gp each. Being intelligent and good creatures, selling eggs and young is essentially slavery: in good societies whoever does it is despised or punished by law.

Pegasi mature like horses. Professional trainers charge 1000 to train a pegasus, which will serve a good or neutral rider faithfully for life.

A light load for a pegasus is up to 150 kg; an average load is 150-300 kg; a heavy load is 300-450kg.

In some pegasi, the blood of an ancestor who was a heroic stallion still runs strong. These champions have the lifespan of a human, the advanced template, perfect maneuverability, fire resistance 10, a +4 racial bonus on Saving Throws against poisons, and immunity to petrification. Some can speak a few words in Celestial or Common. They realize their superiority to other horses and pegasi, and are not to be trained to fly with a rider, but only allow the greatest heroes to ride them.


\

\textbf{Phase Spider}\index[Monsters]{Phase Spider}

The phase spider has the magical ability to enter and exit the Ethereal Plane. It seems to appear out of nowhere and quickly disappears after attacking.

\textit{Large monstrosity, unaligned}

\textbf{STRENGTH} +2

\textbf{DEXTERITY} +2

\textbf{CONSTITUTION} +1

\textbf{INTELLIGENCE} -2

\textbf{WISDOM} +0

\textbf{CHARISMA} -2

\textbf{Initiative} +2 -- \textbf{Defence} 15

\textbf{Hit Points} 32 (5d10 + 5)

\textbf{Movement} 9m, climb 9m

\textbf{Saving Throws}: Fortitude +8, Reflexes +8, Will +3

\textbf{Skills} Stealth +6

\textbf{Senses} vision in the dark 18 m

\textbf{Languages} -

\textbf{Challenge} 3 (700 XP)

\textit{\textbf{Web Walk.}} The spider ignores movement restrictions caused by webs.

\textit{\textbf{Ethereal Excursion.}} As a bonus action, the spider can magically move from the Material Plane to the Ethereal Plane, or vice versa.

\textit{\textbf{Climb as Spider.}} The spider can climb difficult surfaces, including standing upside down on ceilings, without needing to make an ability check.

\textbf{Actions}

\textit{\textbf{Bite.} Melee Weapon Attack}: +4 to hit, reach 1 meter, one creature.

\textit{Hit:} 7 (1d10 + 2) piercing damage and the target must make a DC 11 Fortitude Saving Throw, taking 18 (4d8) poison damage on a failed save, or half as much damage on a failed save. he succeeds. If the poison damage reduces the target to 0 Hit Points, the target is stable but poisoned for 1 hour, even after regaining Hit Points, and while poisoned in this way becomes paralysed.

\

\index[Monsters]{Phoenix}\textbf{Phoenix}

\textit{Celestial Gargantuan, Brave, Protective, Good}

\textbf{STRENGTH} +8

\textbf{DEXTERITY} +6

\textbf{CONSTITUTION} +5

\textbf{INTELLIGENCE} +5

\textbf{WISDOM} +6

\textbf{CHARISMA} +6

\textbf{Initiative} +11 -- \textbf{Defence} 28

\textbf{Hit Points} 210 (20d10 + 100)

\textbf{Vulnerability to Damage} magical cold

\textbf{Move} 9m, fly 27m (good)

\textbf{Saving Throws} Fortitude +20, Reflexes +21, Will +21

\textbf{Damage Immunity} Fire, Light, poison, weapons +1

\textbf{Condition Immunity} grabbed, poisoned, restrained, paralyzed, petrified, prone, unconscious, fatigued, bleeding

\textbf{Regeneration} a Phoenix regenerates 10 Hit Points at the start of each of its rounds

\textbf{Senses} Darkvision 18m, Low-light vision 18m

\textbf{Languages} Ictum, Celestial, Common, Ignan

\textbf{Challenge} 15 (13000 XP)

\textit{\textbf{Awareness of Light.}} The Phoenix always has the following spells active \textit{Detect Magic, Detect Disease and Poison, See Invisibility}

\textit{\textbf{Innate Spells.}} The Phoenix's spellcasting ability is Charisma. The Phoenix can innately cast the following spells, requiring no material components:

At will: \textit{cure critical wounds, dispel magic, eternal flame, remove curse, polymorph (humanoids only)}

3/day: \textit{Mass cure critical wounds, heal, wall of fire, greater restoration, firestorm}

1 time: \textit{Resurrection} the Phoenix by sacrificing her life in a definitive way can bring a creature back to life.

\textbf{Actions}

\textit{\textbf{Multiattack.}} The Phoenix can attack with two claws and its bite

\textit{\textbf{Bite.} Melee Weapon Attack}: +23 to hit, reach 6m, one creature.

\textit{Hit:} 19 piercing damage (2d8+8 + 1d6 Light)

\textit{\textbf{Claw.} Melee Weapon Attack}: +23 to hit, reach 6m, one creature.

\textit{Hit:} 17 slashing damage (2d6+8 + 1d6 Light)

\textbf{Special Abilities}

\textit{\textbf{Rebirth}}

A slain phoenix shrinks to a 3m cubic bonfire with a phoenix egg lying in the center. After 1d4+4 rounds this egg hatches into a perfectly healthy phoenix. The only way to avoid rebirth is to remove the egg from the bonfire (20d6 Light damage) or use a disintegrate spell on the egg.
A Phoenix can resurrect this way once a year, if she dies before this time has elapsed, the death is final. Killing a Phoenix incites the wrath of the Pupils of the Light and Sumkjr's Knights.

\textit{\textbf{Wings of Flame}}

The phoenix can turn her feathers to flame as a free reaction action. These feathers deal 1d6 fire damage + 1d6 light damage to all creatures within 6 meters at the start of its round.

\textit{\textbf{Enraged}}: only legends tell of an angry Phoenix and it is said that a Patron personally intervenes.


\textbf{Ecology}\\
Environment: Deserts and hot hills\\
Organization: Solitary\\
\textbf{Treasure}: Standard\\
\textbf{Description}\\
Legend has it that the Phoenixes are Ljust's pet birds, they are certainly majestic and beautiful creatures and they emanate a Light similar to that of the Patroness of Genesis. The movement of their wings makes no noise while their voice is singing. The phoenix is a legendary bird of fire and light that usually lives in deserts. They are very intelligent and wise creatures and sometimes using their metamorphosis ability they go to cities where they help those who fight against the darkness.

\

\textbf{Pirana}\index[Monsters]{Pirana}

Pirana is a sharp-toothed carnivorous fish.

\textit{Tiny beast, unaligned}

\textbf{STRENGTH} -4

\textbf{DEXTERITY} +3

\textbf{CONSTITUTION} -1

\textbf{INTELLIGENCE} -5

\textbf{WISDOM} -2

\textbf{CHARISMA} -4

\textbf{Initiative} +3 -- \textbf{Defence} 14

\textbf{Hit Points} 1 (1d4 - 1)

\textbf{Movement} 0m, swim 12m

\textbf{Saving Throws}: Fortitude -4, Reflexes +3, Will -2

\textbf{Senses} vision in the dark 18 m

\textbf{Languages} -

\textbf{Challenge} 0 (10 XP)

\textit{\textbf{Blood Frenzy.}} The pirana has +1d6 on melee attack rolls against any creature that is not at full Hit Points.

\textit{\textbf{Water Breathing.}} The pirana can only breathe underwater.

\textbf{Actions}

\textit{\textbf{Bite.} Melee Weapon Attack}: +5 to hit, reach 1m, one target.

\textit{Hit:} 1 piercing damage.

\

\textbf{Pirana Swarm}\index[Monsters]{Pirana Swarm}

\textit{Medium Tiny Beast Swarm, unaligned}

\textbf{STRENGTH} +1

\textbf{DEXTERITY} +3

\textbf{CONSTITUTION} -1

\textbf{INTELLIGENCE} -5

\textbf{WISDOM} -2

\textbf{CHARISMA} -4

\textbf{Initiative} +3 -- \textbf{Defence} 14

\textbf{Hit Points} 28 (8d8 -- 8)

\textbf{Movement} 0m, swim 12m

\textbf{Saving Throws}: Fortitude -3, Reflexes +4, Will -1

\textbf{Damage Resistances} slashing, piercing, slashing

\textbf{Condition Immunity} charmed, grabbed, restrained, paralyzed, petrified, prone, frightened, stunned

\textbf{Senses} vision in the dark 18 m

\textbf{Languages} -

\textbf{Challenge} 1 (200 XP)

\textit{\textbf{Blood Frenzy.}} The swarm has +1d6 on melee attack rolls against any creature that isn't at full Hit Points.

\textit{\textbf{Water Breathing.}} The swarm can only breathe underwater.

\textit{\textbf{Swarm.}} The swarm can occupy another creature's space and vice versa, and the swarm can move through any opening large enough for a Tiny Pirana. The swarm cannot regain Hit Points or gain temporary Hit Points.

\textbf{Actions}

\textit{\textbf{Bites.} Melee Weapon Attack}: +5 to hit, reach 0m, one creature in the swarm's space.

\textit{Hit:} 14 (4d6) piercing damage, or 7 (2d6) piercing damage if the swarm is at half or fewer Hit Points.

\

\textbf{Polar Bear}\index[Monsters]{Polar Bear}

\textit{Large beast, unaligned}

\textbf{STRENGTH} +5

\textbf{DEXTERITY} +0

\textbf{CONSTITUTION} +3

\textbf{INTELLIGENCE} -4

\textbf{WISDOM} +1

\textbf{CHARISMA} -2

\textbf{Initiative} +0 -- \textbf{Defence} 13

\textbf{Hit Points} 42 (5d10 + 15)

\textbf{Movement} 12m, swim 9m

\textbf{Saving Throws}: Fortitude +10, Reflexes +7, Will +4

\textbf{Skills} Awareness +3

\textbf{Languages} -

\textbf{Challenge} 2 (450 XP)

\textit{\textbf{Enhanced sense of smell.}} The bear has +1d6 on Wisdom (Awareness) checks based on smell.

\textbf{Actions}

\textit{\textbf{Multiattack.}} The bear makes two attacks: one with its bite and one with its claws.

\textit{\textbf{Claws.} Melee Weapon Attack}: +7 to hit, reach 1m, one target.

\textit{Hit:} 12 (2d6 + 5) slashing damage.

\textit{\textbf{Bite.} Melee Weapon Attack}: +7 to hit, reach 1m, one target.

\textit{Hit:} 9 (1d8 + 5) piercing damage.

\textbf{VARIANT: CAVE BEAR}\index[Monsters]{Cave Bear}

Some bears have adapted to life underground. They have the same stats as polar bears, but with 18m in the dark.

\

\textbf{Pony}\index[Monsters]{Pony}

\textit{Medium beast, unaligned}

\textbf{STRENGTH} +2

\textbf{DEXTERITY} +0

\textbf{CONSTITUTION} +1

\textbf{INTELLIGENCE} -4

\textbf{WISDOM} +0

\textbf{CHARISMA} -2

\textbf{Initiative} +0 -- \textbf{Defence} 11

\textbf{Hit Points} 11 (2d8 + 2)

\textbf{Move} 12m

\textbf{Saving Throws}: Fortitude +5, Reflexes +4, Will +0

\textbf{Languages} -

\textbf{Challenge} 1/8 (25 XP)

\textbf{Actions}

\textit{\textbf{Hooves.} Melee Weapon Attack}: +4 to hit, reach 1m, one target.

\textit{Hit:} 7 (2d4 + 2) bludgeoning damage.

\

\index[Monsters]{Pseudodragon}\textbf{Pseudodragon}

\textit{Tiny dragon, neutral good}

\textbf{STRENGTH} -2

\textbf{DEXTERITY} +2

\textbf{CONSTITUTION} +1

\textbf{INTELLIGENCE} +0

\textbf{WISDOM} +1

\textbf{CHARISMA} +0

\textbf{Initiative} +2 -- \textbf{Defence} 14

\textbf{Hit Points} 7 (2d4 + 2)

\textbf{Movement} 5m, flight 18m

\textbf{Saving Throws}: Fortitude +4, Reflexes +5, Will +4

\textbf{Skills} Stealth +4, Awareness +3

\textbf{Senses} Darkvision 18m, blindsight 3m

\textbf{Languages} understands Common and Draconic but does not speak

\textbf{Challenge} 1/4 (50 XP)

\textit{\textbf{Resistance to Magic.}} The pseudodragon has +1d6 on Saving Throws against spells and other magical effects.

\textit{\textbf{Fine senses.}} The pseudodragon has +1d6 on Wisdom (Awareness) checks based on sight, hearing, and smell.

\textit{\textbf{Limited telepathy.}} The pseudodragon can communicate simple ideas, emotions, and images telepathically with any creature within 30 meters of it that can understand a language.

\textbf{Actions}

\textit{\textbf{Bite.} Melee Weapon Attack}: +4 to hit, reach 1m, one target.

\textit{Hit:} 4 (1d4 + 2) piercing damage.

\textit{\textbf{Sting.} Melee Weapon Attack}: +4 to hit, reach 3 ft., one creature.

\textit{Hit:} 4 (1d4 + 2) piercing damage, and the target must succeed on a DC 11 Fortitude save or be poisoned for 1 hour. If the save result is 6 or less, the target falls unconscious for the same duration, or until it takes damage or another creature uses an action to awaken it.

\textbf{Ecology}\\
Environment: Temperate forests\\
Organization: Solitary, pair or nest (3-5)\\
\textbf{Treasure}: Standard\\
\textbf{Description}\\
Pseudodragons are small relatives of true dragons, playful and shy. They speak by chirping, hissing, growling, and purring, but can communicate telepathically with any intelligent creature. If approached peacefully with food offerings, they are willing to share information about what is in their territory, but threats and violence drive them away.

Pseudodragons are carnivores and eat insects, rodents, small birds, and snakes, though they eat eggs and love butter, cheese, and fish. Sometimes they hunt on the ground like lizards or flying like birds of prey. Intelligent like most humanoids, they dislike being treated as pets, and prefer to be treated as friends. They are wary of evil creatures, may join spellcasters and devotees as familiars, and some have befriended druids and rangers or partner with good dragons as sentinels. Pseudodragons become familiars only if they appreciate the caster's personality (and if he has the Familiar skill and Charisma at least 1), but they can also bond with people whose company they enjoy. A pseudodragon might follow a character in this fashion for days, weeks, years, or even a lifetime, provided they are well fed and treated with affection.

Upon reaching adulthood, a pseudodragon's body is 15 centimeters long with a 40cm tail, and it weighs about 4kg. A pseudodragon's eggs are as large as a hen's, but leathery in texture and speckled brown, and females lay them in batches of 2-5 each spring. A nest of pseudodragons (which form a family group, and have not hatched from the same group of eggs) usually consists of a pair of adults and several near-adult hatchlings.


\

\index[Monsters]{Purple Worm}\textbf{Purple Worm}

\textit{Colossal monstrosity, unaligned}

\textbf{STRENGTH} +9

\textbf{DEXTERITY} -2

\textbf{CONSTITUTION} +6

\textbf{INTELLIGENCE} -5

\textbf{WISDOM} -1

\textbf{CHARISMA} -3

\textbf{Initiative} -2 -- \textbf{Defence} 26

\textbf{Hit Points} 247 (15x3d6 + 90)

\textbf{Movement} 15m, digging 9m

\textbf{Saving Throws}: Fortitude +21, Reflexes +13, Will +14

\textbf{Senses} blind sight 9 m, telluric sense 18 m

\textbf{Languages} -

\textbf{Challenge} 15 (13000 XP)

\textit{\textbf{Tunnel Borer.}} The worm can burrow through solid rock at half burrowing speed and leaves a tunnel 3 meter in diameter behind it.

\textbf{Actions}

\textit{\textbf{Multiattack.}} The worm makes two attacks: one with its bite and one with its sting.

\textit{\textbf{Bite.} Melee Weapon Attack}: +30 to hit, reach 3m, one target.

\textit{Hit:} 22 (3d8 + 9) piercing damage. If the target is a Large creature, it must succeed on a DC 19 Reflex save or be swallowed by the worm. While engulfed, the creature is blinded and restrained, has full cover against attacks and other effects from outside the worm, and takes 21 (6d6) acid damage at the start of each of the worm's rounds.

If the worm takes 30 or more damage in a single round from a creature within it, the worm must succeed on a DC 21 Fortitude save at the end of its round or vomit all engulfed creatures, who fall prone in a space within 3 meters from the worm. If the worm dies, an engulfed creature is no longer restrained by it and can escape from the corpse using 6 meters of movement, coming prone.

\textit{\textbf{Sting.} Melee Weapon Attack}: +9 to hit, reach 3m, one creature.

\textit{Hit:} 19 (3d6 + 9) piercing damage, and the target must make a DC 19 Fortitude save, taking 42 (12d6) poison damage on a failed save, or half as much damage on a failed save. he succeeds.

\textbf{Ecology}\\
Environment: Any dungeon\\
Organization: Solitary\\
\textbf{Treasure}: Accidental\\
\textbf{Description}\\
Purple worms are gigantic scavengers that inhabit the deepest regions of the world, eating any organic material they encounter. They are known to swallow their prey whole. It is not uncommon to hear of a group of adventurers disappearing into the ravenous jaws of a purple worm, screaming in terror as its members vanish one by one.

While searching for living creatures to devour, the purple worms also gobble up huge amounts of dirt and minerals by burrowing underground. A purple worm's innards can contain a considerable number of gems and other objects capable of resisting the corrosive acid within its esophagus. In areas rich in precious minerals, such as those near dwarven mines, the natural tunnels created by the digging of the purple worms are often filled with large numbers of raw gold nuggets.

A purple worm generally claims a large underground cavern as its lair, and although it returns there to rest and digest its food, it spends most of its time prowling, burrowing through the endless darkness or slithering along pre-existing tunnels. constant search for food to satiate its immense hunger. Though nearly mindless, purple worms are rarely stupid. They are popular as guardians among those who can magically control them or have a room in their lair large enough to house them.


\

\textbf{Racehorse}\index[Monsters]{Racehorse}

\textit{Large beast, unaligned}

\textbf{STRENGTH} +3

\textbf{DEXTERITY} +0

\textbf{CONSTITUTION} +1

\textbf{INTELLIGENCE} -4

\textbf{WISDOM} +0

\textbf{CHARISMA} -2

\textbf{Initiative} +0 -- \textbf{Defence} 11

\textbf{Hit Points} 13 (2d10 + 2)

\textbf{Move} 18m

\textbf{Saving Throws}: Fortitude +3, Reflexes +1, Will +1

\textbf{Languages} -

\textbf{Challenge} 1/4 (50 XP)

\textbf{Actions}

\textit{\textbf{Hooves.} Melee Weapon Attack}: +5 to hit, reach 1m, one target.

\textit{Hit:} 8 (2d4 + 3) bludgeoning damage.

\

\index[Monsters]{Rakshasa}\textbf{Rakshasa}

\textit{Medium fiend, lawful evil}

\textbf{STRENGTH} +2

\textbf{DEXTERITY} +3

\textbf{CONSTITUTION} +4

\textbf{INTELLIGENCE} +1

\textbf{WISDOM} +3

\textbf{CHARISMA} +5

\textbf{Initiative} +3 -- \textbf{Defence} 23

\textbf{Hit Points} 110 (13d8 + 52)

\textbf{Move} 12m

\textbf{Saving Throws}: Fortitude +17, Reflexes +16, Will +16

\textbf{Skills} Deceive +10, Sense Emotions +8

\textbf{Vulnerability to Damage} piercing magical weapons wielded by
good creatures

\textbf{Damage Immunity} bludgeoning, weapons +1

\textbf{Senses} Darkvision 18m

\textbf{Languages} Common, Infernal

\textbf{Challenge} 13 (10000 XP)

\textit{\textbf{Limited Magic Immunity.}} The rakshasa is immune to affects or detection by spells of 6-level or lower unless he wishes to be affected by them. He has +1d6 on Saving Throws against all other spells and magical effects.

\textit{\textbf{Innate Spells.}} The rakshasa's spellcasting ability is Charisma (+10 to hit on spell attacks). The rakshasa can innately cast the following spells, requiring no material components:

At will: \textit{disguise self, minor illusion, detection of thoughts, mage hand}

3/Day each: \textit{charm person, major image,} \textit{detect magic, invisibility, suggestion} 1/Day: \textit{dominate person, plane shift, vision of true, fly}

\textbf{Actions}

\textit{\textbf{Multiattack.}} The rakshasa can make two claw attacks.

\textit{\textbf{Claw.} Melee Weapon Attack}: +13 to hit, reach 1m, one target.

\textit{Hit:} 9 (2d6 + 2) slashing damage, and if the target is a creature, it remains cursed. The magical curse takes effect whenever the target rests, filling the target's thoughts with horrific images and dreams. The cursed target gets no benefit from finishing a rest. The curse lasts until removed by the spell \textit{remove curse} or similar magic.

\textbf{Ecology}
Environment: Any\\
Organization: Solitary, pair, or cult (3-12)\\
\textbf{Treasure}: Double (Dagger+1, other treasure)\\
\textbf{Description}\\
The rakshasa is an evil spirit that disguises itself as a humanoid creature so that it can stalk its prey incognito. Embodying the taboos of most societies and capable of assuming the guise of those it seeks to corrupt, a rakshasa does a great many horrifying deeds. Were they human, their blasphemy, cannibalism, and even worse acts they commit would brand them criminals worthy of the cruellest of hells.

When otherwise in appearance, the rakshasa appears as a humanoid with the head of an animal. It often has the head of a large feline (such as tigers or panthers) or snake (such as cobras or vipers) and, although it is rarer, it can have the head of a gorilla, jackal, vulture, elephant, mantis, lizard, rhinoceros, wild boar and many still others. In many cases, the type of head a rakshasa possesses says something about its personality: a tiger-headed rakshasa is stealthy and ravenous, while a boar-headed one can be gluttonous and cruel. These differences rarely affect the rakshasa's base stats, although there are more powerful variants of the standard with multiple heads, more potent magical powers, and additional strange and deadly special abilities.

Rakshasas despise religions; they recognize the power of the gods, but see themselves as the only beings worthy of veneration by the mortal races. Rakshasa Devotees are therefore quite rare. While rakshasas are outsiders, they are also creatures of the Material Plane, and some believe that the first rakshasas chose this banishment over some other role offered them by a long-forgotten god. While they are usually solitary, it is not uncommon to find large families of rakshasas working together to bring about the downfall of a mortal civilization from within, over many generations.

A rakshasa is 1.8 meters tall and weighs 90 kg.

\

\textbf{Rat}\index[Monsters]{Rat}

\textit{Tiny beast, unaligned}

\textbf{STRENGTH} -4

\textbf{DEXTERITY} +0

\textbf{CONSTITUTION} -1

\textbf{INTELLIGENCE} -4

\textbf{WISDOM} +0

\textbf{CHARISMA} -3

\textbf{Initiative} +0 -- \textbf{Defence} 11

\textbf{Hit Points} 1 (1d4 - 1)

\textbf{Move} 6m

\textbf{Saving Throws}: Fortitude -4, Reflexes +0, Will +0

\textbf{Senses} vision in the dark 9m

\textbf{Languages} -

\textbf{Challenge} 0 (10 XP)

\textit{\textbf{Enhanced sense of smell.}} The rat has +1d6 on Wisdom (Awareness) checks based on smell.

\textbf{Actions}

\textit{\textbf{Bite.} Melee Weapon Attack}: +0 to hit, reach 1m, one target.

\textit{Hit:} 1 piercing damage.

\

\textbf{Raven}\index[Monsters]{Raven}

\textit{Tiny beast, unaligned}

\textbf{STRENGTH} -4

\textbf{DEXTERITY} +2

\textbf{CONSTITUTION} -1

\textbf{INTELLIGENCE} -4

\textbf{WISDOM} +1

\textbf{CHARISMA} -2

\textbf{Initiative} +2 -- \textbf{Defence} 13

\textbf{Hit Points} 1 (1d4 - 1)

\textbf{Move} 3m, fly 15m

\textbf{Saving Throws}: Fortitude +1, Reflexes +4, Will +2

\textbf{Skills} Awareness +3

\textbf{Languages} -

\textbf{Challenge} 0 (10 XP)

\textit{\textbf{Imitation.}} The raven can imitate simple sounds it has heard, such as a person's whisper, a child's cry, or an animal's cries. A creature that hears the sound can identify it as a mimic with a successful DC 10 Wisdom (Survival) check.

\textbf{Actions}

\textit{\textbf{Beak.} Melee Weapon Attack}: +4 to hit, reach 1m, one target.

\textit{Hit:} 1 piercing damage.

%\

\index[Monsters]{Remorhaz}\textbf{Remorhaz}

\textit{Huge monstrosity, unaligned}

\textbf{STRENGTH} +7

\textbf{DEXTERITY} +1

\textbf{CONSTITUTION} +5

\textbf{INTELLIGENCE} -3

\textbf{WISDOM} +0

\textbf{CHARISMA} -3

\textbf{Initiative} +1 -- \textbf{Defence} 23

\textbf{Hit Points} 195 (17d12 + 85)

\textbf{Movement} 9m, digging 6m

\textbf{Saving Throws}: Fortitude +16, Reflexes +12, Will +11

\textbf{Immunity to Damage} Cold, Fire

\textbf{Senses} Darkvision 18m, telluric sense 18m

\textbf{Languages} -

\textbf{Challenge} 11 (7200 XP)

\textit{\textbf{Heated Body.}} A creature that touches the remorhaz or hits it with a melee attack while within 1 meter of it takes 10 (3d6) fire damage.

\textbf{Actions}

\textit{\textbf{Bite.} Melee Weapon Attack}: +18 to hit, reach 3m, one target.

\textit{Hit:} 40 (6d10 + 7) piercing damage plus 10 (3d6) fire damage. If the target is a creature, it is grabbed (DC 17 to escape). Until the grab ends, the target is restrained, and the remorhaz can't bite another target.

\textit{\textbf{Swallow up.}} The remorhaz makes a bite attack against a Medium or smaller target it is grappling. If the attack hits, the creature takes bite damage and is swallowed, and the grab ends. The swallowed target is blinded and restrained, has full cover against attacks and other effects outside the remorhaz, and takes 21 (6d6) acid damage at the start of each remorhaz's turn.

If the remorhaz takes 30 or more damage in a single round from a creature within it, the remorhaz must succeed on a DC 15 Fortitude save at the end of that round or vomit all engulfed creatures, who fall prone in a space within 3 meters from the remorhaz. If the remorhaz dies, a swallowed creature is no longer restrained by it and can exit the corpse using 5 meters of movement, coming prone.

\textit{\textbf{Enraged}}: the Remorhaz heats up its body even more until the end of the fight bringing the fire damage to 18 (6d6) for those within 1 meter.


\textbf{Ecology}\\
Environment: Cold Deserts and Glaciers\\
Organization: Solitary\\
\textbf{Treasure}: None\\
\textbf{Description}\\
In a world of ice and snow, remorhazes are especially feared for the terrible fire that burns within their bodies. This inner fire causes the plates along its back to burn hot when the creature is particularly angry, excited, or panicked. Creatures that have adapted to arctic regions are often particularly vulnerable to fire, which makes the remorhaz's main Defence incredibly powerful and secures its role as a dangerous predator of icy areas. Remorhazes live in vast labyrinths carved into the heart of glaciers. These beasts use their heat to carve tunnels through the ice, tunnels whose smooth glass walls rapidly refreeze in their wake, creating numerous incredibly stable mazes.

While the remorhaz has much in common with smaller surface parasites, this beast is surprisingly intelligent. Though unable to speak, the typical remorhaz understands the Giant well, and giant tribes often take advantage of this to form alliances with these beasts. Frost Giants are especially obsessed with them; these giants face the cruel and lethal burns that a remorhaz can inflict to become "friends of the worm" obtaining a powerful weapon to use against their enemies, an assassin capable of burrowing through the floor of glacial fortifications to strike directly with the greatest weakness of a frost giant: fire. Other giants use these beasts as living forges, as their backs are hot enough to melt metal.

A remorhaz is 7 meters long and weighs 5000 kg.



\

\textbf{Rhinoceros}\index[Monsters]{Rhinoceros}

\textit{Large beast, unaligned}

\textbf{STRENGTH} +5

\textbf{DEXTERITY} -1

\textbf{CONSTITUTION} +2

\textbf{INTELLIGENCE} -4

\textbf{WISDOM} +1

\textbf{CHARISMA} -2

\textbf{Initiative} -1 -- \textbf{Defence} 12

\textbf{Hit Points} 45 (6d10 + 12)

\textbf{Move} 12m

\textbf{Saving Throws}: Fortitude +10, Reflexes +4, Will +2

\textbf{Languages} -

\textbf{Challenge} 2 (450 XP)

\textit{\textbf{Charge.}} If the rhino moves at least 6 meters directly towards a target and hits them with a gore attack during the same turn, the target takes an additional 9 (2d8) bludgeoning damage. If the target is a creature, it must succeed on a DC 15 Fortitude save or be knocked prone.

\textbf{Actions}

\textit{\textbf{Gored.} Melee Weapon Attack}: +7 to hit, reach 1m, one target.

\textit{Hit:} 14 (2d8 + 5) bludgeoning damage.

\

\index[Monsters]{Roper}\textbf{Roper}

\textit{Large Monstrosity, Neutral Evil}

\textbf{STRENGTH} +4

\textbf{DEXTERITY} -1

\textbf{CONSTITUTION} +3

\textbf{INTELLIGENCE} -2

\textbf{WISDOM} +3

\textbf{CHARISMA} -2

\textbf{Initiative} -1 -- \textbf{Defence} 23

\textbf{Hit Points} 93 (11d10 + 33)

\textbf{Movement} 3m, climb 3m

\textbf{Saving Throws}: Fortitude +13, Reflexes +5, Will +13

\textbf{Skills} Stealth +5, Awareness +6

\textbf{Senses} Darkvision 18m

\textbf{Languages} -

\textbf{Challenge} 5 (1800 XP)

\textit{\textbf{False Appearance.}} When the roper remains motionless, it is indistinguishable from a normal rock formation, such as a stalagmite.

\textit{\textbf{Climb as Spider.}} The roper can climb difficult surfaces, including standing upside down on ceilings, without needing to make an ability check.

\textit{\textbf{Grasping Tendrils.}} The roper can have up to six tendrils at a time. Each tendril can be attacked (Defense 20; 10 Hit Points; immunity to poison damage). Destroying a tendril deals no damage to the roper, who can produce a replacement tendril in his next round. A tendril can also be broken if a creature takes an action and succeeds on a DC 15 Strength check against it.

\textbf{Actions}

\textit{\textbf{Multiattack.}} The roper can make four attacks with its tendrils, use envelop, and make a bite attack.

\textit{\textbf{Bite.} Melee Weapon Attack}: +7 to hit, reach 2m, one target.

\textit{Hit:} 22 (4d8 + 4) piercing damage and Purulent Necrosis Disease

\textit{Purulent necrosis:} 1 day, save Fortitude DC 15, 12 hours, 1 success, -1 Constitution.

\textit{\textbf{Tender.} Melee Weapon Attack}: +7 to hit, reach 15m, one creature.

\textit{Hit:} Target is grappled (DC 15 to flee). Until the grab ends, the target is entangled and has -1d6 on Strength checks and Fortitude saves, while the roper cannot use the same vine against another target.

\textit{\textbf{Envelop.}} The roper drags grabbed creatures 6 meters toward him.

\textit{\textbf{Enraged}}: the roper emits a nauseating cacophonic wave. All creatures within a 20-foot radius must make a DC 18 Fortitude save or be nauseated until the end of the next round.


\textbf{Ecology}
Environment any dungeon\\
Organization: Solitary, pair or group (3-6)\\
\textbf{Treasure}: Standard\\
\textbf{Description}\\
The roper is an ambush hunter. Able to change the coloration and shape of its body, a hidden roper looks like a stalagmite of stone or ice (or in low-ceilinged locations, a column of stone or ice). In areas lacking these hiding traits, a roper can compress its body until it resembles a boulder. The lashes he can shoot out are not flesh but a thick, semi-liquid material similar to partially molten wax but with the strength of an iron chain and the ability to numb flesh and sap strength. The roper can wield these lashes with great Featand send them flying up to 15 meters to steal items that attract his attention.

Despite its alien and monstrous form, the roper is one of the most intelligent inhabitants of the underground. They do not form large societies (although they often coexist with other underground creatures such as the Brains Eaters, with whom they sometimes ally), but often congregate in small groups. Particularly interested in the philosophy of life and death, and the subtler aspects of the world's most sinister and cruel religions, a roper can talk or argue for hours with those that he initially simply tried to eat. Some stories tell of particularly gifted orators and philosophers who were kept for days or even years as pets or conversational partners by groups of floggers; ultimately, though, if they fail to escape, the roper's appetite eventually gets the better of their inquisitive intelligence, especially in cases where these companion animals consistently outwit the wit and patience of their guardians.
A roper is 2.7 meters tall and weighs 1,100 kg.


\

\index[Monsters]{Rust-eater}\textbf{Rust-eater}

\textit{Medium Monstrosity, unaligned}

\textbf{STRENGTH} +1

\textbf{DEXTERITY} +1

\textbf{CONSTITUTION} +1

\textbf{INTELLIGENCE} -4

\textbf{WISDOM} +1

\textbf{CHARISMA} -2

\textbf{Initiative} +1 -- \textbf{Defence} 15

\textbf{Hit Points} 27 (5d8 + 5)

\textbf{Move} 12m

\textbf{Saving Throws}: Fortitude +2, Reflexes +4, Will +5

\textbf{Senses} Darkvision 18m

\textbf{Languages} -

\textbf{Challenge} 1/2 (100 XP)

\textit{\textbf{Iron Sense.}} The rust monster can detect, by smell, the exact location of ferrous metals within 36 meters.

\textit{\textbf{Rusting Metal.}} Any nonmagical weapon made of metal, even Mithral and Adamatium, that strikes the rust monster corrodes after dealing damage. Non-magical ammunition made of metal that strikes the rust monster is destroyed after inflicting damage.

\textbf{Actions}

\textit{\textbf{Bite.} Melee Weapon Attack}: +3 to hit, reach 1m, one target.

\textit{Hit:} 5 (1d8 + 1) piercing damage.

\textit{\textbf{Antenna.}} The rust monster eats away at non-magical ferrous metal objects that it can see and are within 1 meter. If the item is not worn or carried, contact with the rust monster destroys a cube with a 120cm edge. If the item is worn or carried by a creature, the creature can make a DC 11 Reflex save to avoid contact with the rust monster.

If the object it comes into contact with is metal Armour or shields being worn or carried, they suffer a permanent and cumulative -2 penalty to the Defence they provide. Armour reduced to 0 Defence or shields that drop to a +0 bonus are destroyed. If the object it comes into contact with is a metal weapon being held by someone, it rusts it as described in the Rusting Metal trait.

\textbf{Ecology}
Environment any dungeon\\
Organization: Solitary, pair, or nest (3-10)\\
\textbf{Treasure}: Accidental (no metal treasure)\\
\textbf{Description}\\
Of all the terrifying beasts an explorer may encounter underground, only the rust monster targets the one the average adventurer values most: his treasure.

Typically 1 meter long and weighing at least 100 kg, the rust monster resembles a crustacean and would be scary enough even without the alien feeding process from which it takes its name. Rust-eaters eat metal objects, preferring those made of iron and ferrous alloys such as steel, but they also eat mithral, adamantium, and enchanted metals with equal ease. Any metal touched by the rust monster's delicate antennae or Armoured hide corrodes and crumbles to dust in seconds, making it one of the most feared beasts by subterranean adventurers and Dwarf miners who must defend their forges and compete with them for gold.

While rust monsters have no innate tendency for violence, their insatiable hunger drives them to charge anything that comes close with enough metal on them, and any resistance is met with unexpected ferocity. It is not uncommon for rust monsters in metal-poor areas to follow fleeing victims for days using their metal sniffing ability, provided they still have metal objects intact.\\
Fortunately, it is often possible to escape a rust monster's attentions by throwing a dense metal object, such as a shield, at it and running in the opposite direction. Those who frequent rust monster-infested areas quickly learn to keep wooden or stone weapons close at hand.


\

\textbf{Saber-toothed Tiger}\index[Monsters]{Saber-toothed Tiger}

\textit{Large beast, unaligned}

\textbf{STRENGTH} +4

\textbf{DEXTERITY} +2

\textbf{CONSTITUTION} +2

\textbf{INTELLIGENCE} -4

\textbf{WISDOM} +1

\textbf{CHARISMA} -1

\textbf{Initiative} +2 -- \textbf{Defence} 13

\textbf{Hit Points} 52 (7d10 + 14)

\textbf{Move} 12m

\textbf{Saving Throws}: Fortitude +5, Reflexes +3, Will +2

\textbf{Skills} Stealth +6, Awareness +3

\textbf{Languages} -

\textbf{Challenge} 2 (450 XP)

\textit{\textbf{Leap.}} If the tiger moves at least 6 meters directly towards a creature and hits it with a claw attack during the same turn, the target must succeed on a DC 14 Fortitude save or fall prone. If the target is prone, the tiger can make a bite attack against it as a bonus action.

\textit{\textbf{Enhanced sense of smell.}} The tiger has +1d6 on Wisdom (Awareness) checks based on smell.

\textbf{Actions}

\textit{\textbf{Claw.} Melee Weapon Attack}: +6 to hit, reach 1m, one target.

\textit{Hit:} 12 (2d6 + 5) slashing damage, 1 bleed damage.

\textit{\textbf{Bite.} Melee Weapon Attack}: +6 to hit, reach 1m, one target.

\textit{Hit:} 10 (1d10 + 5) piercing damage.

\

\index[Monsters]{Sahuagin}\textbf{Sahuagin}

\textit{Medium humanoid (sahuagin), lawful evil}

\textbf{STRENGTH} +1

\textbf{DEXTERITY} +0

\textbf{CONSTITUTION} +1

\textbf{INTELLIGENCE} +1

\textbf{WISDOM} +1

\textbf{CHARISMA} -1

\textbf{Initiative} +1 -- \textbf{Defence} 13

\textbf{Hit Points} 22 (4d8 + 4)

\textbf{Movement} 9m, swim 12m

\textbf{Saving Throws}: Fortitude +4, Reflexes +4, Will +4

\textbf{Skills} Awareness +5

\textbf{Senses} darkvision 40m

\textbf{Languages} Sahuagin

\textbf{Challenge} 1/2 (100 XP)

\textit{\textbf{Limited Amphibian.}} The sahuagin can breathe air and water, but must remain submerged at least once every 4 hours to avoid suffocation.

\textit{\textbf{Blood Frenzy.}} The sahuagin has +1d6 on melee attack rolls against any creature that is not at its full Hit Points.

\textit{\textbf{Telepathy with Sharks}}. The sahuagin can magically command any shark within 16 meters of itself, using a limited form of telepathy.

\textbf{Actions}

\textit{\textbf{Multiattack.}} The sahuagin can make two melee attacks: one with its bite and one with its claws or spear.

\textit{\textbf{Claws.} Melee Weapon Attack}: +3 to hit, reach 1m, one target.

\textit{Hit:} 3 (1d4 + 1) slashing damage.

\textit{\textbf{Spear.} Melee or Ranged weapon attack}: +3 to hit, reach 1m or range 6m, one target.

\textit{Hit:} 4 (1d6 + 1) piercing damage, or 5 (1d8 + 1) piercing damage when used with two hands to make a melee attack.

\textit{\textbf{Bite.} Melee Weapon Attack}: +3 to hit, reach 1m, one target.

\textit{Hit:} 3 (1d4 + 1) piercing damage.

\textbf{Ecology}\\
Environment: Temperate or Warm Oceans\\
Organization: Solitary, pair, squad (5-8), patrol (11-20 plus 1 3rd-level lieutenant and 1-2 Sharks), gang (20-80 plus 100\% noncombatants, 1 lieutenant of 3 1st level and 1 4th level captain for every 20 adults, and 1-2 Sharks) or tribe (70-160 plus 100\% noncombatants, 1 3rd level lieutenant for every 20 adults, 1 4th level captain per 40 adults, 9 4th level guards, 1-4 3rd-6th level novices, 1 7th level priestess, 1 6th-8th level baron, and 5-8 Sharks)
\textbf{Treasure}: NPC gear (Trident, Heavy crossbow with 10 bolts, other treasure)\\
\textbf{Description}\\
Ravenous and cruel, the sahuagin are, unfortunately, among the most prosperous oceanic races. Great cities have been built by this race in the dark depths of oceanic trenches, and some fortresses rise near the coasts from where they launch continuous assaults against air-breathing enemies who live close to shore. Prideful and warlike, sahuagin rarely ally with others, and view other aquatic races, such as aboleths, merfolk, and the like, as competitors. The only creatures they seem to respect other than their own kind are sharks; in these relentless predators, in fact, the sahuagin see a lot of themselves. A sahuagin stands 7 feet tall and weighs approximately 100kg.

Sahuagin are subject to genetic mutations, and when a mutant is born, he almost always rises to the ranks of nobility or leadership in society. The most common sahuagin mutation consists of an extra pair of arms (granting two additional claw attacks or the ability to wield more weapons). Some speak of the rare malenti, sahuagin who look not like sharkmen but water elves, though they share the bloodlust and cruel nature of their kin. Malenti often serve as spies or assassins for sahuagin rulers, but there are reports of entire tribes made up of malenti in remote areas of the sea.


\

\index[Monsters]{Salamander}\textbf{Salamander}

\textit{Large Elemental, Neutral Evil}

\textbf{STRENGTH} +4

\textbf{DEXTERITY} +2

\textbf{CONSTITUTION} +2

\textbf{INTELLIGENCE} +0

\textbf{WISDOM} +0

\textbf{CHARISMA} +1

\textbf{Initiative} +2 -- \textbf{Defence} 18

\textbf{Hit Points} 90 (12d10 + 24)

\textbf{Move} 9m

\textbf{Saving Throws}: Fortitude +10, Reflexes +7, Will +6

\textbf{Vulnerability to Damage} cold

\textbf{Damage Resistances} from non-magical weapon

\textbf{Immunity to Damage} Fire

\textbf{Senses} Darkvision 18m

\textbf{Languages} Ignan

\textbf{Challenge} 5 (1800 XP)

\textit{\textbf{Heated Weapons.}} Any metallic melee weapon the salamander wields deals an additional 3 (1d6) fire damage per hit (already included in the attack).

\textit{\textbf{Heated Body.}} A creature that touches the salamander or hits it with a melee attack while within 1 meter of it takes 7 (2d6) fire damage.

\textbf{Actions}

\textit{\textbf{Multiattack.}} The salamander makes two attacks: one with its spear and one with its tail.

\textit{\textbf{Tail.} Melee Weapon Attack}: +10 to hit, reach 3m, one target.

\textit{Hit:} 11 (2d6 + 4) bludgeoning damage plus 7 (2d6) fire damage, and the target is grappled (DC 14 to escape). Until the grab ends, the target is restrained, the salamander can automatically smack the target with its tail, and the salamander cannot make tail attacks against other targets.

\textit{\textbf{Spear.} Melee or Ranged weapon attack}: +9 to hit, reach 1m, range 6m, one target.

\textit{Hit:} 11 (2d6 + 4) piercing damage, or 13 (2d8 +4) piercing damage if used with two hands to make a melee attack, plus 3 (1d6) fire damage.

\textit{\textbf{Enraged}}: the Salamadra concentrates its flames in a ranged attack. A creature within 30 feet must make a DC 18 Reflex save to take half damage. The creature is hit by a ball of fire that does 4d6 fire damage. It costs 2 Actions.


\textbf{Ecology}
Environment any (Plane of Fire)\\
Organization: Solitary, pair or group (3-5)\\
\textbf{Treasure}: Standard (Spear, other nonflammable treasure)\\
\textbf{Description}\\
Salamanders are native to the Plane of Fire, where their legions of fierce fighters are greatly feared by other inhabitants of the Plane. Because many of the strongest fire elemental races enslave the salamanders for their metalworking prowess and fighting prowess, the salamanders hate the efreet and others with a fervor.

Even if their hiding places exceed 250 degrees C in temperature, Salamanders can tolerate lower temperatures. They generally do so when forced to, and are even more gruff and short-tempered than normal in these environments. Though hails from the Plane of Fire, the Salamander race identifies more with the Abyss, and has great respect for Demons (particularly those associated with fire, such as balors and certain flame-related demon lords). As such, it is not unusual to encounter a large group of Salamanders in the Abyss.

Salamanders are often summoned to the Material Plane to serve as guardians or, more commonly, as makers of armour, weapons, and other metallurgical items, as their prowess in this area is legendary. Salamanders also infest those areas of the Material Plane where the line between this world and the Plane of Fire has blurred, such as near and inside Volcanoes.

Inhabiting such extreme areas, Salamanders possess only treasures that can withstand high temperatures, such as swords, Armour, jewels, rods, and other items that have a high melting point. Salamander society is cruel and based on power and the ability to subjugate those below them. Beings less than Salamanders who cause trouble face a slow and painful death.



\

\index[Monsters]{Satyr}\textbf{Satyr}

\textit{Medium Fey, Chaotic Neutral}

\textbf{STRENGTH} +1

\textbf{DEXTERITY} +3

\textbf{CONSTITUTION} +0

\textbf{INTELLIGENCE} +1

\textbf{WISDOM} +0

\textbf{CHARISMA} +2

\textbf{Initiative} +3 -- \textbf{Defence} 15 (leather Armour)

\textbf{Hit Points} 31 (7d8)

\textbf{Damage Vulnerability} cold iron

\textbf{Move} 12m

\textbf{Saving Throws}: Fortitude +4, Reflexes +8, Will +8

\textbf{Skills} Stealth +5, Perform +6, Awareness +2

\textbf{Languages} Common, Elvish, Sylvan

\textbf{Challenge} 1/2 (100 XP)

\textit{\textbf{Resistance to Magic.}} The satyr has +1d6 on Saving Throws against spells and other magical effects.

\textbf{Actions}

\textit{\textbf{Gore.} Melee Weapon Attack}: +3 to hit, reach 1m, one target.

\textit{Hit:} 6 (2d4 + 1) bludgeoning damage.

\textit{\textbf{Short sword.} Melee weapon attack}: +5 to hit, reach 1m, one target.

\textit{Hit:} 6 (1d6 + 3) piercing damage.

\textit{\textbf{Shortbow.} Ranged weapon attack}: +5 to hit, range 24m, one target.

\textit{Hit:} 6 (1d6 + 3) piercing damage.

\textbf{Ecology}\\
Environment: Temperate Forests\\
Organization: Solitary, pair, band (3-6) or party (7-11)\\
\textbf{Treasure}: Standard (Dagger, Shortbow plus 20 Arrows, masterwork panpipe, other treasure)\\
\textbf{Description}\\
Satyrs, known in many regions as fauns, are debauched and hedonistic creatures from the deepest and most primeval parts of the forests. They adore wine, music and the pleasures of the flesh, they are renowned as libertines and dudes who court clueless maidens and shepherd boys and leave behind a trail of embarrassing explanations and unwanted pregnancies.

While their bodies are almost always those of handsome, well-proportioned men, the seductive powers of satyrs lie in their musical talent. With the aid of her flute, a satyr is capable of weaving a wide variety of melodic spells designed to charm others into complying with her whimsical desires.

In addition to constant frolicking, satyrs often act as guardians of their forests, and those who manage to turn the faun's lust into rage are likely to face the most dangerous of the animals surrounding the faun. Furthermore, while satyrs tend to put their amusement above the rights of others, they harbor no resentment against those they seduce.

The children born of these encounters are always full-blooded satyrs and are generally taken away from their wanton fathers soon after birth.


\

\textbf{Swarm of Wasps}\index[Monsters]{Sciame di Serpenti Velenosi}

\textit{Medium Tiny Beast Swarm, unaligned}

\textbf{STRENGTH} -4

\textbf{DEXTERITY} +1

\textbf{CONSTITUTION} +0

\textbf{INTELLIGENCE} -5

\textbf{WISDOM} -2

\textbf{CHARISMA} -5

\textbf{Initiative} +1 -- \textbf{Defence} 13

\textbf{Hit Points} 22 (5d8)

\textbf{Move} 1m, fly 9m

\textbf{Saving Throws}: Fortitude -3, Reflexes +2, Will -1

\textbf{Damage Resistances} slashing, piercing, slashing

\textbf{Condition Immunity} charmed, grabbed, restrained, paralyzed, petrified, prone, frightened, stunned

\textbf{Senses} blindsight 3 m

\textbf{Languages} -

\textbf{Challenge} 1/2 (100 XP)

\textit{\textbf{Swarm.}} The swarm can occupy another creature's space and vice versa, and the swarm can move through any opening large enough for a Tiny insect. The swarm cannot regain Hit Points or gain temporary Hit Points.

\textbf{Actions}

\textit{\textbf{Bites.} Melee Weapon Attack}: +3 to hit, reach 0m, one target in the swarm's space.

\textit{Hit:} 10 (4d4) piercing damage, or 5 (2d4) piercing damage if the swarm is at half or fewer Hit Points.


\

\textbf{Scorpio}\index[Monsters]{Scorpio}

\textit{Tiny beast, unaligned}

\textbf{STRENGTH} -4

\textbf{DEXTERITY} +0

\textbf{CONSTITUTION} -1

\textbf{INTELLIGENCE} -5

\textbf{WISDOM} -1

\textbf{CHARISMA} -4

\textbf{Initiative} +0 -- \textbf{Defence} 12

\textbf{Hit Points} 1 (1d4 - 1)

\textbf{Move} 3m

\textbf{Saving Throws}: Fortitude -3, Reflexes +2, Will -1

\textbf{Senses} blindsight 3 m

\textbf{Languages} -

\textbf{Challenge} 0 (10 XP)

\textbf{Actions}

\textit{\textbf{Sting.} Melee Weapon Attack}: +2 to hit, reach 1 meter, one creature.

\textit{Hit:} 1 piercing damage and the target must make a DC 9 Fortitude save, taking 4 (1d8) poison damage on a failed save, or half as much damage on a successful one.

\

\index[Monsters]{Shadow}\textbf{Shadow}

\textit{Medium Undead, Chaotic Evil}

\textbf{STRENGTH} -2

\textbf{DEXTERITY} +2

\textbf{CONSTITUTION} +1

\textbf{INTELLIGENCE} -2

\textbf{WISDOM} +0

\textbf{CHARISMA} -1

\textbf{Initiative} +2 -- \textbf{Defence} 13

\textbf{Hit Points} 16 (3d8 + 3)

\textbf{Move} 12m

\textbf{Saving Throws}: Fortitude +3, Reflexes +3, Will +4

\textbf{Skills} Stealth +4 (+6 in dim light or darkness)

\textbf{Vulnerability to Damage} from Light

\textbf{Damage Resistances} acid, cold, Electricity, fire, sound; from a non-magical weapon

\textbf{Damage Immunity} Void, Poison

\textbf{Condition Immunity} grabbed, poisoned, restrained, paralyzed, petrified, prone, fatigued, frightened, bleeding

\textbf{Senses} Darkvision 18m

\textbf{Languages} -

\textbf{Challenge} 1/2 (100 XP)

\textit{\textbf{Amorphous.}} The shadow can move through a space as narrow as 1 cm without constricting.

\textit{\textbf{Weakness to sunlight.}} While in sunlight, the shadow has -1d6 on attack rolls, proficiency checks, and Saving Throws.

\textit{\textbf{Spirit of the Shadow.}} While in an area of dim light the Shadow regenerates 5 Hit Points at the start of its round, if in an area of darkness it regenerates 10 Hit Points at the start of its round and can become invisible using 1 Action. Shadow Spirit increases the Shadow's Challenge Rating by 1.

\textit{\textbf{Shadow Stealth.}} When in dim light or darkness, the shadow can take the Hide action as a bonus action.

\textit{\textbf{Undead nature.}} A shadow needs no air, food, drink, or sleep.

\textbf{Actions}

\textit{\textbf{Force Drain.} Melee Weapon Attack}: +4 to hit, reach 3 ft., one creature.

\textit{Hit:} 9 (2d6 + 2) void damage, and the target's Strength score is reduced by 1. The target dies if this reduces its Strength to -5. Otherwise, the reduction lasts until the target rests 8 hours.

If a nonevil humanoid dies from this attack, a new shadow will animate within 1d4 hours of its corpse.

\textbf{Ecology}
Environment: Any\\
Organization: Solitary, pair, group (3–6), or swarm (7–12)\\
\textbf{Treasure}: Standard\\

\textbf{Description}\\
The evil shadow moves along the border between the darkness of darkness and the harsh truth of light. The shadow prefers to infest the ruins civilization leaves behind, where it hunts down living creatures foolish enough to stumble upon its territory. The shadow is a hideous undead, and as such has no apparent purpose or motivation other than to drain life force and vitality from living beings.


\

\index[Monsters]{Sibilant}\textbf{Sibilant}

\textit{Large monstrosity, chaotic}

\textbf{STRENGTH} +2

\textbf{DEXTERITY} +1

\textbf{CONSTITUTION} +1

\textbf{INTELLIGENCE} -3

\textbf{WISDOM} +0

\textbf{CHARISMA} -2

\textbf{Initiative} +1 -- \textbf{Defence} 14

\textbf{Hit Points} 32 (5d10+5)

\textbf{Move} 6m, climb 6m

\textbf{Saving Throws}: Fortitude +3, Reflexes +3, Will +2

\textbf{Skills} Stealth +4, Awareness +3

\textbf{Senses} Darkvision 18m

\textbf{Fine Senses}: The Sibilant has +1d6 on Awareness checks based on hearing or smell

\textbf{Languages}: -

\textbf{Challenge} 2 (450 XP)

\textbf{Actions}

\textit{\textbf{Multiattack.}} The Sibilant can make two claw attacks or one tail swipe.

\textit{\textbf{Claw.} Melee Weapon Attack}: +4 to hit, reach 1m, one target.

\textit{Hit:} 6 (1d8+2) slashing damage.

\textit{\textbf{Tail Whip}}: The Sibilant wags its long tail and strikes a target.

\textit{Hit:} 11 (2d8+2) bludgeoning damage and 7 (2d6) slashing damage, reach 3 meter. In the event of a critical roll, any Armour or shield is damaged by lowering the opponent's Defence by 1. Damage to Armour is not considered permanent.

\textbf{Reactions}

\textit{\textbf{Knockdown}}: When the Sibilant is attacked by a creature within range of its tail, it is whipped, forcing the attacker to make a Fortitude/Reflex save after the attack resolves at DC 12 or take 7 (2d6) bludgeoning damage and fall prone. On a successful save take only half damage and are not prone.

\textbf{Ecology}\\
Environment: Caverns\\
Organization: Solitary, pair or nest (2-4)\\
\textbf{Treasure}: random\\

\textbf{Description}

The Sibilant, so called because of the noise their tail makes as they wag, is a very particular creature. At first sight it resembles a crocodile, about 5 meters long, 4 of which are tail, but has 8 legs and a short and flattened snout. The extremely robust tail ends in a kind of hook that the Sibilant uses to strike, kill and grab enemies as if it were an additional paw.

Dark gray to brown in color, they prefer to hide in the dark and attack when hungry or to defend their territory. They try to keep their distance in combat and if seriously injured they escape by climbing walls.


\

\index[Monsters]{Skeleton}\textbf{Skeleton}

\textit{Medium Undead, Lawful Evil}

\textbf{STRENGTH} +0

\textbf{DEXTERITY} +2

\textbf{CONSTITUTION} +2

\textbf{INTELLIGENCE} -2

\textbf{WISDOM} -1

\textbf{CHARISMA} -3

\textbf{Initiative} +2 -- \textbf{Defence} 14 (Armour pieces)

\textbf{Hit Points} 13 (2d8 + 4)

\textbf{Move} 9m

\textbf{Saving Throws}: Fortitude +0, Reflexes +2, Will +2

\textbf{Vulnerability to Damage} from bludgeoning

\textbf{Damage Resistances} piercing and slashing

\textbf{Immunity to Damage} Poison

\textbf{Condition Immunity} poisoned, fatigue, bleeding

\textbf{Senses} Darkvision 18m

\textbf{Languages} understands all the languages he spoke in life but cannot speak

\textbf{Challenge} 1/4 (50 XP)

\textit{\textbf{Undead nature.}} The skeleton requires no air, food, drink, or sleep.

\textbf{Actions}

\textit{\textbf{Short sword.} Melee weapon attack}: +4 to hit, reach 1m, one target.

\textit{Hit:} 5 (1d6 + 2) piercing damage.

\textit{\textbf{Shortbow.} Ranged weapon attack}: +4 to hit, range 24m, one target.

\textit{Hit:} 5 (1d6 + 2) piercing damage.

\textbf{Ecology}\\
Environment: Any\\
Organization: Any\\
\textbf{Treasure}: None (Broken Chainmail, Broken Scimitar)\\
\textbf{Description}\\
Skeletons are bones of the animated dead, brought unto life by unholy magic. For the most part, skeletons are mindless automatons, but they possess an evil cunning bestowed upon them by the force that animates them: a cunning that allows them to bear weapons and wear Armour.

\

\index[Monsters]{Slime, Black Pudding}\textbf{Black Pudding}

\textit{Big slime, unaligned}

\textbf{STRENGTH} +3

\textbf{DEXTERITY} -3

\textbf{CONSTITUTION} +3

\textbf{INTELLIGENCE} -5

\textbf{WISDOM} -2

\textbf{CHARISMA} -5

\textbf{Initiative} -3 -- \textbf{Defence} 9

\textbf{Hit Points} 85 (10d10 + 30)

\textbf{Move} 6m, climb 6m

\textbf{Saving Throws}: Fortitude +9, Reflexes -2, Will -2

\textbf{Damage Immunity} Acid, cold, Electricity, slashing, critical

\textbf{Condition Immunity} blinded, charmed, deafened, prone, fatigued, frightened

\textbf{Senses} blindsight 18m (blind beyond this range)

\textbf{Languages} -

\textbf{Challenge} 4 (1100 XP)

\textit{\textbf{Amorphous.}} The black pudding can move through a gap up to 1 cm wide without squeezing.

\textit{\textbf{Corrosive Form.}} A creature that touches the black pudding or hits it with a melee attack while within 1 meter of it takes 4 (1d8) acid damage. Any nonmagical weapon made of metal or wood that strikes the black pudding corrodes. After dealing the damage, the weapon takes a permanent, cumulative -1 penalty on damage rolls. If the penalty reaches -5, the weapon is destroyed. Nonmagical ammunition made of metal or wood that strikes black pudding is destroyed after inflicting damage.

The black pudding can devour 2cm thick wood or nonmagical metal in 1 round.

\textit{\textbf{Jelly's nature}} Black pudding does not need sleep.

\textit{\textbf{Climb as Spider.}} The black pudding can climb difficult surfaces, including standing upside down on ceilings, without needing to make an ability check.

\textbf{Actions}

\textit{\textbf{Pseudopod.} Melee Weapon Attack}: +7 to hit, reach 1m, one target.

\textit{Hit:} 6 (1d6 + 3) bludgeoning damage plus 18 (4d8) acid damage. In addition, nonmagical Armour worn by the target is partially dispelled and suffers a permanent, cumulative -1 penalty to the Defence it affords. The Armour is destroyed if the penalty reduces its Defence to 10.

\textbf{Reactions}

\textit{\textbf{Split.}} When a Medium or larger black pudding takes Electricity or slashing damage, it splits into two new black puddings of at least 10 Hit Points each. Each new black pudding has a number of Hit Points equal to half the original black pudding, rounded down. The new black puddings are one size smaller than the original.

\textbf{Ecology}\\
Environment any dungeon\\
Organization: Solitary\\
\textbf{Treasure}: None\\
\textbf{Description}\\
Black puddings are the scavengers of the underworld, constantly on the lookout for food. They can sense organic or metallic bodies within 20 meters and instinctively attack such objects or beings until they dissolve, or until the ooze is slain. A black pudding reproduces by breaking off a piece of its own body and forming a new, smaller pudding that reaches adulthood within a month. Some of the smartest creatures in the underworld use black puddings to naturally dispose of garbage, creating stone caves to house the pudding, then dumping organic waste or enemies into them.
The largest specimens of black puddings have been sighted in the deepest regions of the world: Gargantuan individuals possessing up to 30 HD. It is said that there are also colored protoplasms: some white that live in the arctic areas, brown in the swamps and reddish in color that inhabit the desert.


\

\index[Monsters]{Slime, Grey Slime}\textbf{Grey Slime}

\textit{Medium slime, unaligned}

\textbf{STRENGTH} +1

\textbf{DEXTERITY} -2

\textbf{CONSTITUTION} +3

\textbf{INTELLIGENCE} -5

\textbf{WISDOM} -2

\textbf{CHARISMA} -4

\textbf{Initiative} -2 -- \textbf{Defence} 9

\textbf{Hit Points} 22 (3d8 + 9)

\textbf{Move} 3m, climb 3m

\textbf{Saving Throws}: Fortitude +9, Reflexes -4, Will -4

\textbf{Damage Resistances} acid, cold, fire

\textbf{Condition Immunity} blinded, charmed, deafened, prone, fatigued, frightened

\textbf{Senses} blindsight 18m (blind beyond this range)

\textbf{Languages} -

\textbf{Challenge} 1/2 (100 XP)

\textit{\textbf{Amorphous.}} The ooze can move through a space up to 1 cm wide without squeezing.

\textit{\textbf{Corrode Metal.}} Any non-magical weapon made of metal that strikes the ooze corrodes. After dealing the damage, the weapon takes a permanent, cumulative -1 penalty on damage rolls. If the penalty reaches -5, the weapon is destroyed. Nonmagical ammunition made of metal that strikes ooze destroys itself after inflicting damage.

The ooze can devour nonmagical metal 5 centimeters thick in 1 round.

\textit{\textbf{False Appearance.}} When the ooze stands still, it is indistinguishable from a puddle of oil or a wet stone.

\textit{\textbf{Jelly's nature.}} Ooze does not need sleep.

\textbf{Actions}

\textit{\textbf{Pseudopod.} Melee Weapon Attack}: +3 to hit, reach 1m, one target.

\textit{Hit:} 4 (1d6 + 1) bludgeoning damage plus 7 (2d6) acid damage, and if the target is wearing metal Armour, it is partially dispelled and suffers a permanent, cumulative penalty of - 1 to the Defence it bids. The Armour is destroyed if the penalty reduces its Defence to 10.

\textbf{Ecology}\\
Environment Cold swamps and dungeons\\
Organization: Solitary\\
\textbf{Treasure}: None\\
\textbf{Description}\\
Creeping through cold swamps and foggy marshes or, sometimes into dungeons and caverns, the gray oozes consume any organic matter they encounter. Although devoid of intelligence, the gray slime is one of the creatures that gives many problems for its transparency. While she cannot easily climb walls or swim, her habit of hiding in the thick mud along marshy banks or lying motionless in innocuous-looking pools on the gray dungeon floor make her very difficult to notice and avoid.

Some sages believe that the gray oozes are the result of a failed alchemical experiment, while others theorize that the first gray oozes arose spontaneously from a magical debris pit. Naturally, these theories that they are not living organisms, but the result of an unfortunate mixture of caustic fluids and magical residues, are derided by those who live in areas infested by these creatures, which have no history of magical pollution.


\

\index[Monsters]{Slime, Jelly Cube}\textbf{Jelly Cube}

\textit{Big slime, unaligned}

\textbf{STRENGTH} +2

\textbf{DEXTERITY} -4

\textbf{CONSTITUTION} +5

\textbf{INTELLIGENCE} -5

\textbf{WISDOM} -2

\textbf{CHARISMA} -5

\textbf{Initiative} -4 -- \textbf{Defence} 7

\textbf{Hit Points} 84 (8d10 + 40)

\textbf{Movement} 5 meters

\textbf{Saving Throws}: Fortitude +9, Reflexes -4, Will -4

\textbf{Damage Immunity} non-magical slashing weapons, from critical

\textbf{Condition Immunity} blinded, charmed, deafened, prone, fatigued, frightened

\textbf{Senses} blindsight 18m (blind beyond this range)

\textbf{Languages} -

\textbf{Challenge} 2 (450 XP)

\textit{\textbf{Cube of Slime.}} The cube takes up its entire space. Other creatures can enter the space, but fall victim to the cube's Submerge and have -1d6 on their Saving Throws.

Creatures inside the cube are visible but have full cover.

A creature within 1 meter of the cube can take an action to pull a creature or object off the cube. Doing so requires a successful DC 12 Strength check, and the creature making the attempt takes 10 (3d6) acid damage.

The cube can hold only one Large creature or up to four Medium or smaller creatures at a time.

\textit{\textbf{Jelly's nature.}} The cube does not need to sleep.

\textit{\textbf{Transparent.}} Even when in plain sight, you must succeed on a DC 15 Wisdom (Awareness) check to notice a cube that hasn't moved or attacked. A creature that tries to enter the cube's space while unaware of its presence is surprised by the cube.

\textbf{Actions}

\textit{\textbf{Pseudopod.} Melee Weapon Attack}: +4 to hit, reach 1m, one target.

\textit{Hit:} 10 (3d6) acid damage.

\textit{\textbf{Submerge.}} The cube moves up to its movement limit. In doing so, it can enter the space of a Large or smaller creature. Whenever the cube enters a creature's space, the creature must make a DC 12 Reflex Saving Throw.

On a successful save, the creature can choose to be pushed back or sideways 1 meter. A creature that decides not to be pushed suffers the consequences of a failed Saving Throw.

On a failed save, the cube enters the creature's space, which takes 10 (3d6) acid damage and is submerged. The submerged creature can't breathe, is restrained, and takes 21 (6d6) acid damage at the start of the cube's turn. When the cube moves, the submerged creature moves with it.

A submerged creature can attempt to escape by taking an action to make a DC 12 Strength check. On a successful one, the creature escapes and enters the space of its choice within 1 meter of the cube.

\textbf{Ecology}
Environment any dungeon\\
Organization: Solitary\\
\textbf{Treasure}: Accidental\\
\textbf{Description}\\
One of the dungeons' most unusual and peculiar predators, gelatinous cubes spend their existence aimlessly roaming underground tunnels and dark caverns, consuming organic materials such as plants, refuse, carrion, and even living creatures. Matter that the cube cannot digest, such as metal and stone, fills the creature's volume with debris, and it can sometimes expel some of it from its body. Often the treasure and possessions of past victims remain inside the gelatinous cube: ghostly image of their material remains.

Some sages believe these creatures evolved from Gray Ooze. Some beings use the gelatinous cubes as guardians of dungeons and underground fortifications, trapping these immense creatures in massive metal cases and transporting them with powers or magic to their final guard post. They are particularly effective waste disposal mechanisms; a tribe can trap a gelatinous cube in a pit or other area it cannot climb as a dunghill or even a deathtrap, depending on the ingenuity of the creatures that captured it.

The gelatinous cubes typically have a 3m edge and weigh more than 7,500kg, although some underground explorers claim that larger specimens exist underground. In areas where food is plentiful, jelly cubes can live for hundreds, if not thousands, of years. However, if organic matter is lacking for more than 6 months, a gelatinous cube begins to waste away, and its walls begin to leak, rapidly dissolving into liquid mucus until the entire body collapses and disappears completely.


\

\index[Monsters]{Slime, Straw-colored Amoeba}\textbf{Straw-colored Amoeba}

\textit{Big slime, unaligned}

\textbf{STRENGTH} +2

\textbf{DEXTERITY} -2

\textbf{CONSTITUTION} +2

\textbf{INTELLIGENCE} -4

\textbf{WISDOM} -2

\textbf{CHARISMA} -5

\textbf{Initiative} +2 -- \textbf{Defence} 9

\textbf{Hit Points} 45 (6d10 + 12)

\textbf{Movement} 3m, climb 3m

\textbf{Saving Throws}: Fortitude +8, Reflexes -3, Will -3

\textbf{Damage Resistances} acid

\textbf{Damage Immunity} Electricity, slashing

\textbf{Condition Immunity} blinded, charmed, deafened, prone, fatigued, frightened

\textbf{Senses} blindsight 18m (blind beyond this range)

\textbf{Languages} -

\textbf{Challenge} 2 (450 XP)

\textit{\textbf{Amorphous.}} The amoeba can move through a gap up to 3 centimeters wide without squeezing.

\textit{\textbf{Jelly's nature.}} The amoeba does not require sleep.

\textit{\textbf{Climb as Spider.}} The amoeba can climb difficult surfaces, including standing upside down on ceilings, without needing to make an ability check.

\textbf{Actions}

\textit{\textbf{Pseudopod.} Melee Weapon Attack}: +4 to hit, reach 1m, one target.

\textit{Hit:} 9 (2d6 + 2) bludgeoning damage plus 3 (1d6) acid damage.

\textbf{Reactions}

\textit{\textbf{Split.}} When a Medium or larger amoeba takes Electricity or slashing damage, it splits into two new amoebas that have at least 10 Hit Points. Each new amoeba has a number of Hit Points equal to half the original amoeba, rounded down. The new amoebas are one size smaller than the original.

\textbf{Ecology}
Environment dungeons or temperate swamps\\
Organization: Solitary\\
\textbf{Treasure}: None\\
\textbf{Description}\\
The Straw-colored Amoeba are animated masses of protoplasm similar in color to a repulsive amalgam of yellow, orange and brown. When at rest, their flat, pulsating body is about 15 centimeters tall and extends all the way around; in motion, they gather in a vaguely spherical shape and almost seem to move by rolling. Their malleable bodies allow them to pass through cracks and holes much smaller than the space they occupy. Creatures that live underground often seal all openings to defend themselves from the Straw Amoeba.

The highly specialized acid of Straw Amoeba only dissolves flesh. This discovery has led many master poisoners and alchemists to seek out specimens for study. From these experiments were born several specific weapons designed to destroy the bodies. It is told of the existence of a slow-acting poison that destroys the cells of living creatures one by one, the secret of which is well kept by its creator.

Some notes in a forgotten tome speak of a funeral ritual used in distant places. Instead of burning the body, it was sealed in a stone sarcophagus with a Straw Amoeba, which dissolved the body. Subsequently, the undertakers placed the jelly in an urn complete with a bronze plaque with the name of the deceased. This practice protects the buried objects with the dead (which is reduced in a short time to a shiny skeleton) and the essence of the creature, which was believed to still live inside the jelly.

The Straw-colored Amoeba are about 15 centimeters high, have a diameter that can reach 3 meters and weigh about 1,300 kilos. In combat, they gather together and produce long, moist pseudopodia to strike and grab anything that moves.

While the typical Straw Amoeba has the statistics presented here, deep underground these predators can grow to monstrous sizes. There is also talk of Amoeba Straw-colored who have developed other ways of catching their prey. For example, jellies that poison on touch and expel clouds of toxic gas that burn eyes and mouths, leaving you Defenceless but conscious as this protoplasmic beast glides over bodies and feeds on them.


\

\index[Monsters]{Sphinxes, Androsfinge}\textbf{Androsfinge}

\textit{Large monstrosity, lawful neutral}

\textbf{STRENGTH} +6

\textbf{DEXTERITY} +0

\textbf{CONSTITUTION} +5

\textbf{INTELLIGENCE} +3

\textbf{WISDOM} +4

\textbf{CHARISMA} +6

\textbf{Initiative} +3 -- \textbf{Defence} 26

\textbf{Hit Points} 199 (19d10 + 95)

\textbf{Move} 12m, fly 18m

\textbf{Saving Throws}: Fortitude +22, Reflexes +17, Will +21

\textbf{Skills} Arcanum +9, Awareness +10, Religion +15

\textbf{Immunity to Damage} from non-magical weapon

\textbf{Condition Immunity} fascinated, frightened

\textbf{Senses} True Seeing 36 m

\textbf{Languages} Common, Sphinx

\textbf{Challenge} 17 (18000 XP)

\textit{\textbf{Magic Weapons.}} The sphinx's weapon attacks are magical.

\textit{\textbf{Unfathomable.}} The sphinx is immune to any effect that can sense its emotions or read its thoughts, as well as any divination spell it rejects. Wisdom (Sense Emotion) checks to discern the sphinx's intentions or sincerity have -1d6.

\textit{\textbf{Enchantments.}} The sphinx has MP 12.
His spellcasting ability is Wisdom (spell save DC 22, +10 to hit on spell attacks). She needs no material components to cast her spells. The sphinx has the following spells prepared:

Cantrips (at will): \textit{holy flame, save the dying,} \textit{thaumaturgy}

level 1 (4 slots): \textit{command, detect magic,} \textit{detect evil and good}

level 2 (3 slots): \textit{lower restoration, zone of truth}

level 3 (3 slots): \textit{dispel magic, tongues}

level 4 (3 slots): \textit{exile, freedom of movement}

level 5 (2 slots): \textit{fire strike, greater restoration}

level 6 (1 slot): \textit{Heroes' Feast}

\textbf{Actions}

\textit{\textbf{Multiattack.}} The sphinx can make two claw attacks.

\textit{\textbf{Claw.} Melee Weapon Attack}: +17 to hit, reach 1m, one target.

\textit{Hit:} 17 (2d6 + 10) slashing damage, 1 bleed damage.

\textit{\textbf{Roar (3/Day).}} The sphinx lets out a magical roar. Each time he roars before a new dawn, the roar is louder and the effect is different, as detailed below. Each creature within 150 meters of the sphinx and able to hear its roar must make a Saving Throw.

\textbf{First roar.} Any creature that fails a DC 22 Will save is frightened for 1 minute. A frightened creature can repeat the Saving Throw at the end of each of its rounds, ending the effect on itself on a successful one.

\textbf{Second roar.} Any creature that fails a DC 22 Will save is deafened and frightened for 1 minute. A frightened creature is paralyzed and can repeat the Saving Throw at the end of each of its rounds, ending the effect on itself on a successful one.

\textbf{Third roar.} Each creature makes a DC 22 Fortitude save. Those who fail the save take 44 (8d10) sound damage and are knocked prone. On a successful save, the creature takes half as much damage and isn't knocked prone.

\textbf{Additional Actions}

The sphinx can perform 3 additional Actions, chosen from the options below. It can only use one Additional Action at a time, and only at the end of another creature's turn. The sphinx regains expended additional Actions at the start of its round.

\textbf{Claw Attack.} The sphinx makes a claw attack.

\textbf{Casting a Spell (Costs 3 Actions).} The sphinx casts a spell from the list of prepared spells, using up a spell slot as normal.

\textbf{Teleport (Costs 2 Actions).} The sphinx magically teleports itself, along with any equipment it is wearing or carrying, to an unoccupied space that it can see, up to 16 meters away.

\textit{\textbf{Enraged}}: the Sphinx asks a riddle. the creature must make a DC 31 Will or become paralyzed. Each round could retry the Saving Throw for giving the exact answer. Cost 1 Action.


\textbf{Ecology}\\
Environment: Hills or Hot Deserts\\
Organization: Solitary\\
\textbf{Treasure}: Standard\\
\textbf{Description}\\
The most powerful of the common sphinxes, androsphinxes believe they represent all that is worthy and noble about their species, and pose as if the weight of the world rests on their good example. They regard the Criosphinxes with patronizing smugness, the Hieracosphinxes with undisguised disgust, and the Gynosphinxes as the only other sphinxes worthy of their time.

Androsphinxes display a grumpy and resentful facade towards strangers. They make no effort to hide their annoyance when they are irritated. They also tend to be jealous of their territory, though less so than other sphinxes. They almost inevitably issue warnings and bombastic proclamations before attacking, and they almost always comply with a call to deal. Androsphinxes trade information and conversation, not treasure, for safe passage.

Androsphinxes are 3.6 meters tall and weigh 500 kg.


\

\index[Monsters]{Sphinxes, Ginosphinx}\textbf{Ginosphinx}

\textit{Large monstrosity, lawful neutral}

\textbf{STRENGTH} +4

\textbf{DEXTERITY} +2

\textbf{CONSTITUTION} +3

\textbf{INTELLIGENCE} +4

\textbf{WISDOM} +4

\textbf{CHARISMA} +4

\textbf{Initiative} +4 -- \textbf{Defence} 23

\textbf{Hit Points} 136 (16d10 + 48)

\textbf{Move} 12m, fly 18m

\textbf{Saving Throws}: Fortitude +14, Reflexes +13, Will +15

\textbf{Skills} Arcanum +14, Awareness +9, Religion +9, History +14

\textbf{Damage Resistances} from non-magical weapon

\textbf{Condition Immunity} fascinated, frightened

\textbf{Senses} True Seeing 36 m

\textbf{Languages} Common, Sphinx

\textbf{Challenge} 11 (7200 XP)

\textit{\textbf{Magic Weapons.}} The sphinx's weapon attacks are magical.

\textit{\textbf{Unfathomable.}} The sphinx is immune to any effect that can sense its emotions or read its thoughts, as well as any divination spell it rejects. Wisdom (Sense Emotions) checks to discern the sphinx's intentions or sincerity have -1d6.

\textit{\textbf{Spells.}} The sphinx has MP 9. Her spellcasting ability is Intelligence (spell save DC 17, +9 to hit with spell attacks). She does not need material components to cast her spells. The sphinx has the following spells prepared: Cantrips (at-will): \textit{minor illusion, mage hand,} \textit{prestidigitation}

level 1 (4 slots): \textit{identify, detect magic, shield}

level 2 (3 slots): \textit{locate object, darkness, suggestion}

level 3 (3 slots): \textit{dispel magic, tongues, remove curse}

level 4 (3 slots): \textit{exile, greater invisibility}

level 5 (2 slots): \textit{knowledge of legends}

\textbf{Actions}

\textit{\textbf{Multiattack.}} The sphinx can make two claw attacks.

\textit{\textbf{Claw.} Melee Weapon Attack}: +11 to hit, reach 1m, one target.

\textit{Hit:} 13 (2d8 + 4) slashing damage, 1 bleed damage.

\textbf{Additional Actions}

The sphinx can perform 3 additional Actions, chosen from the options below. She can only use one Additional Action at a time, and only at the end of another creature's turn. The sphinx regains expended additional Actions at the start of its round.

\textbf{Claw Attack.} The sphinx makes a claw attack.

\textbf{Casting a Spell (Costs 3 Actions).} The sphinx casts a spell from the list of prepared spells, using up a spell slot as normal.

\textbf{Teleport (Costs 2 Actions).} The sphinx magically teleports, along with any equipment she is wearing or carrying, to an unoccupied space that she can see, up to 16 meters away.

\textbf{Ecology}
Environment: Hot deserts and hills\\
Organization: Solitary, pair, or cult (3-6)\\
\textbf{Treasure}: Double\\
\textbf{Description}\\
While there are many different types of sphinx, the one referred to by scholars as the gynosphinx (a name many sphinxes find offensive) is a wise and majestic creature, yet terrifying when angry. Less moralistic than their male counterparts (the Androsphinxes, totally different creatures from the one presented here), the sphinxes are prudent and methodical when making decisions, and they pride themselves on their cool logic and impartiality. They have little patience with lower variants of sphinxes, considering them little more than animals. Sphinxes love tricky puzzles and riddles, and treasure unusual facts and arcane dilemmas much more than gold or gems.

While not great scholars in the traditional sense, sphinxes' great appreciation for puzzles leads them to research a wide variety of subjects, often making them a valuable source of information, especially when making use of their magical abilities. They are usually happy to have contact with other races, and regularly offer material goods in exchange for new and interesting information or riddles. They are excellent guardians of temples, tombs and other important places, as long as they are properly entertained. Sphinxes place great value on kindness, but they can be capricious: they may selflessly decide to share their latest puzzles with travelers but won't think twice about devouring them if they don't pay enough attention to them or don't provide any clues to solve them.

A typical sphinx is 3 meters long and weighs about 400 kg. While their wings can keep them in the air for long periods of time, they are poor flyers, preferring to land before engaging in battle, attacking with their powerful claws. Though extremely territorial, sphinxes tend to warn intruders several times before attacking.

\

\textbf{Spider}\index[Monsters]{Spider}

\textit{Tiny beast, unaligned}

\textbf{STRENGTH} 2 (-5)

\textbf{DEXTERITY} +2

\textbf{CONSTITUTION} -1

\textbf{INTELLIGENCE} -5

\textbf{WISDOM} +0

\textbf{CHARISMA} -4

\textbf{Initiative} +2 -- \textbf{Defence} 13

\textbf{Hit Points} 1 (1d4 - 1)

\textbf{Move} 6m, climb 6m

\textbf{Saving Throws}: Fortitude -4, Reflexes +2, Will -4

\textbf{Skills} Stealth +4

\textbf{Senses} vision in the dark 9m

\textbf{Languages} -

\textbf{Challenge} 0 (10 XP)

\textit{\textbf{Web Walk.}} The spider ignores movement restrictions caused by webs.

\textit{\textbf{Climb as Spider.}} The spider can climb difficult surfaces, including standing upside down on ceilings, without needing to make an ability check.

\textit{\textbf{Web Sense.}} While in contact with a web, the spider knows the exact location of any other creature in contact with the same web.

\textbf{Actions}

\textit{\textbf{Bite.} Melee Weapon Attack}: +4 to hit, reach 3 ft., one creature.

\textit{Hit:} 1 piercing damage and the target must succeed on a Fortitude 9 save or take 2 (1d4) poison damage.

\

\index[Monsters]{Spirite}\textbf{Spirite}

\textit{Tiny fey, neutral good}

\textbf{STRENGTH} -4

\textbf{DEXTERITY} +4

\textbf{CONSTITUTION} +0

\textbf{INTELLIGENCE} +2

\textbf{WISDOM} +1

\textbf{CHARISMA} +0

\textbf{Initiative} +4 -- \textbf{Defence} 16 (leather Armour)

\textbf{Hit Points} 2 (1d4)

\textbf{Damage Vulnerability} cold iron

\textbf{Move} 3m, fly 12m

\textbf{Saving Throws}: Fortitude +0, Reflexes +5, Will +2

\textbf{Skills} Stealth +8 (check is made with -1d6 if the imp is flying), Awareness +3

\textbf{Languages} Common, Elvish, Sylvan

\textbf{Challenge} 1/4 (50 XP)

\textbf{Actions}

\textit{\textbf{Longsword.} Melee Weapon Attack}: +2 to hit, reach 1 m, one target.

\textit{Hit:} 1 slashing damage.

\textit{\textbf{Shortbow.} Ranged weapon attack}: +6 to hit, range 12m, one target.

\textit{Hit:} 1 piercing damage. If the target is a creature, it must succeed on a DC 10 Fortitude save or be poisoned, -1 Strenght and Dexterity, for 1 minute. If the result of this save is 5 or less, the target falls unconscious for the same duration, or until it takes damage or another creature uses an action to awaken it.

\textit{\textbf{Invisibility.}} The imp remains invisible until it attacks or ends its concentration. Anything the imp is carrying or wearing remains invisible as long as it remains in contact with the imp.

\textit{\textbf{Heart Sight.}} The imp contacts a creature and learns its current emotional state. If the target fails a DC 10 Fortitude save, the imp also learns the creature's traits. Celestials, fiends, and undead automatically fail this Saving Throw.

\textbf{Description}\\
Sprites gather in groups deep in wooded regions, united in the cause of protecting nature. Entire tribes of sprites have declared themselves protectors of a particular person, place or creature of particular importance in their lands, even if the being does not want or need any protection.

A faerie's body is naturally luminous, though the creature can vary the color and intensity of the light emitted by its body at will. Upon his death, a sprite's body dissolves into a shimmering mist. Sprites are the smallest of the goblins, standing just over 22cm tall and weighing rarely more than 1kg.

In many ways, sprites are more primitive than most fey. They enjoy the company of their own kind, but tend to distrust other fey and assume that any humanoid or creature they have not expressly chosen to protect is out to harm them. Even animals are usually considered dangerous by them. The reason for this distrust is largely due to the tiny size of these creatures, which makes them easy prey for predators. Thus, a faerie's initial reaction to danger is to flee—usually using its magical abilities to slow or distract pursuers, and later relying on its flying speed and size to escape.

While sprites themselves have an uncultured and wild nature, they have a healthy curiosity for all things with innate magic. They are especially drawn to places of great latent magical power, such as the ruins of ancient temples. This curiosity also makes them unusually suited to being familiars. A 5th-level chaotic neutral spellcaster can gain a faerie as a familiar if he has the familiar skill.


\

\textbf{Strige}\index[Monsters]{Strige}

This hideous monster looks like a cross between a large bat and an oversized mosquito. Its legs end in long pincers, and its long, needle-like proboscis slices through the air as it seeks to feed on the blood of living creatures.

\textit{Tiny beast, unaligned}

\textbf{STRENGTH} -3

\textbf{DEXTERITY} +3

\textbf{CONSTITUTION} +0

\textbf{INTELLIGENCE} -4

\textbf{WISDOM} -1

\textbf{CHARISMA} -2

\textbf{Initiative} +3 -- \textbf{Defence} 15

\textbf{Hit Points} 2 (1d4)

\textbf{Move} 3m, fly 12m

\textbf{Saving Throws}: Fortitude -3, Reflexes +4, Will -1

\textbf{Senses} vision in the dark 18 m

\textbf{Languages} -

\textbf{Challenge} 1/8 (25 XP)

\textbf{Actions}

\textit{\textbf{Drain Blood.} Melee Weapon Attack}: +5 to hit, reach 1m, one creature.

\textit{Hit:} 5 (1d4 + 3) piercing damage and the striga sticks to the target. While attacked, the striga does not attack. Instead, at the start of each striga's turn, the target loses 5 (1d4 + 3) Hit Points due to blood loss.

The striga can detach itself by expending 1 meter of movement. He does this automatically after draining 10 Hit Points from the target or upon the target's death. A creature, including the target, can use its action to detach the striga.

\

\index[Monsters]{Stygian Bir (or Striga))}\textbf{Stygian Bird)}

\textit{Tiny beast, unaligned}

\textbf{STRENGTH} -3

\textbf{DEXTERITY} +3

\textbf{CONSTITUTION} +0

\textbf{INTELLIGENCE} -4

\textbf{WISDOM} -1

\textbf{CHARISMA} -2

\textbf{Initiative} +3 -- \textbf{Defence} 15

\textbf{Hit Points} 2 (1d4)

\textbf{Move} 3m, fly 12m

\textbf{Saving Throws}: Fortitude +2, Reflexes +6, Will +1

\textbf{Senses} Darkvision 18m

\textbf{Languages} -

\textbf{Challenge} 1/8 (25 XP)

\textbf{Actions}

\textit{\textbf{Drain blood.} Melee weapon attack}: +5 to hit, reach 1m, one creature.

\textit{Hit:} 5 (1d4 + 3) piercing damage and the striga sticks to the target. While attacked, the striga does not attack. Instead, at the start of each striga's turn, the target loses 5 (1d4 + 3) Hit Points due to blood loss.

The striga can detach itself by expending 1 meter of movement. He does this automatically after draining 10 Hit Points from the target or upon the target's death. A creature, including the target, can use its action to detach the striga.

\textbf{Ecology}
Environment: Temperate and warm swamps\\
Organization: Solitary, colony (2-4), flock (5-8), cloud (9-14), or swarm (15-40)\\
\textbf{Treasure}: None\\
\textbf{Description}\\
Striga are dangerous bloodsuckers who infest swamps and prey on wild animals, livestock, and unsuspecting travelers. While individually weak, swarms of these creatures are capable of draining a man in minutes, leaving behind only a desiccated corpse.

More mammal-like than insect-like, striga soar with their four fleshy wings, seeking out warm-blooded prey. They often hide near pools of drinkable water waiting for travelers to let their guard down before attacking them and drinking their fill, thrusting their trunks into exposed veins. After feeding, they fly off to hide in the mud and reeds to lay their eggs and rest until hunger prompts them to hunt again.

Striga are usually about 30 centimeters long, with a wingspan about twice that, and weigh less than 0.5 kg. They are rusty red or reddish brown, and have a dirty yellow underbelly, but those that have not fed properly are pale pink.

\

\index[Monsters]{Succubus}\textbf{Succubus}

\textit{Medium fiend (shapeshifter), neutral evil}

\textbf{STRENGTH} -1

\textbf{DEXTERITY} +3

\textbf{CONSTITUTION} +1

\textbf{INTELLIGENCE} +2

\textbf{WISDOM} +1

\textbf{CHARISMA} +5

\textbf{Initiative} +3 -- \textbf{Defence} 17

\textbf{Hit Points} 66 (12d8 + 12)

\textbf{Move} 9m, fly 18m

\textbf{Saving Throws}: Fortitude +7, Reflexes +9, Will +10

\textbf{Skills} Deceive +9, Sense Emotions +5, Awareness +5, Stealth 5

\textbf{Damage Resistances} cold, Electricity, fire, poison; from a non-magical weapon

\textbf{Senses} Darkvision 18m

\textbf{Languages} Abyssal, Common, Infernal, telepathy 18m

\textbf{Challenge} 4 (1100 XP)

\textit{\textbf{Telepathic bond.}} The fiend ignores the range restrictions of his telepathy when communicating with a creature he has charmed. The two are not even forced to be on the same plane of existence.

\textit{\textbf{Shapeshift.}} The fiend can use his action to transform into a Small or Medium humanoid, or back to his true form. Without wings, the fiend loses flying speed. Aside from size and speed, his stats are the same in all forms. Any equipment he is wearing or carrying is not transformed. Upon death he reverts to his true form.

\textbf{Actions}

\textit{\textbf{Claw (Fiendish form only).} Melee Weapon Attack}: +6 to hit, reach 3 ft., one target.

\textit{Hit:} 6 (1d6 + 3) slashing damage.

\textit{\textbf{Fascinate.}} A humanoid visible to the fiend within 10 meters of it must succeed on a DC 15 Will save or be magically fascinated for 1 day. The charmed target obeys the fiend's verbal or telepathic commands. If the target takes damage or receives a suicidal command, it can repeat the Saving Throw, ending the effect on a successful one. If the target succeeds at its Saving Throw against the effect, or if the effect ends, the target is immune to the fiend's charm for the next 24 hours.

The fiend can only have one target charmed at a time. If you charm another, the effect on the previous target ends.

\textit{\textbf{Sucking Kiss.}} The fiend kisses a charmed creature or a willing creature. The target must make a DC 15 Fortitude save against this spell, taking 32 (5d10 + 5) damage on a failed save, or half as much damage on a successful one. The target's maximum Hit Points are reduced by an amount equal to the damage taken. This reduction lasts until dawn breaks. The target dies if this effect reduces its maximum Hit Points to 0.

\textit{\textbf{Ethereal form.}} The fiend magically enters the Ethereal Plane from the Material Plane, and vice versa.

\textbf{Ecology}\\
Environment: Any (Abyss)\\
Organization: Solitary, pair, or harem (3-12)\\
\textbf{Treasure}: double\\
\textbf{Description}\\
Among the demonic hordes a succubus can often reach very high levels of power, using her manipulations and her sensual charms, and many demonic wars rage due to the devious machinations of these creatures. A succubus originates from the souls of particularly libidinous and greedy evil mortals.


\

\textbf{Swarm of Bats}\index[Monsters]{Swarm of Bats}

\textit{Medium Tiny Beast Swarm, unaligned}

\textbf{STRENGTH} -3

\textbf{DEXTERITY} +2

\textbf{CONSTITUTION} +0

\textbf{INTELLIGENCE} -4

\textbf{WISDOM} +1

\textbf{CHARISMA} -3

\textbf{Initiative} +2 -- \textbf{Defence} 13

\textbf{Hit Points} 22 (5d8)

\textbf{Move} 0m, fly 9m

\textbf{Saving Throws}: Fortitude -2, Reflexes +4, Will +2

\textbf{Damage Resistances} slashing, piercing, slashing

\textbf{Condition Immunity} charmed, grabbed, restrained, paralyzed, petrified, prone, frightened, stunned

\textbf{Senses} blindsight 18 m

\textbf{Languages} -

\textbf{Challenge} 1/4 (50 XP)

\textit{\textbf{Echolocation.}} The swarm cannot use blindsight when deafened.

\textit{\textbf{Swarm.}} The swarm can occupy another creature's space and vice versa, and the swarm can move through any opening large enough for a Tiny Bat. The swarm cannot regain Hit Points or gain temporary Hit Points.

\textit{\textbf{Honed hearing.}} The swarm has +1d6 on hearing-based Wisdom (Awareness) checks.

\textbf{Actions}

\textit{\textbf{Bites.} Melee Weapon Attack}: +4 to hit, reach 0m, one creature in the swarm's space.

\textit{Hit:} 5 (2d4) piercing damage, or 2 (1d4) piercing damage if the swarm is at half or fewer Hit Points.

\

\textbf{Swarm of Bugs}\index[Monsters]{Swarm of Bugs}

\textit{Medium Tiny Beast Swarm, unaligned}

\textbf{STRENGTH} -4

\textbf{DEXTERITY} +1

\textbf{CONSTITUTION} +0

\textbf{INTELLIGENCE} -5

\textbf{WISDOM} -2

\textbf{CHARISMA} -5

\textbf{Initiative} +1 -- \textbf{Defence} 13

\textbf{Hit Points} 22 (5d8)

\textbf{Movement} 6m, climb 6m

\textbf{Saving Throws}: Fortitude -3, Reflexes +2, Will -1

\textbf{Damage Resistances} slashing, piercing, slashing

\textbf{Immunity to Conditions} charmed, grabbed, restrained, paralyzed, petrified, prone, frightened, stunned

\textbf{Senses} blindsight 3 m

\textbf{Languages} -

\textbf{Challenge} 1/2 (100 XP)

\textit{\textbf{Swarm.}} The swarm can occupy another creature's space and vice versa, and the swarm can move through any opening large enough for a Tiny insect. The swarm cannot regain Hit Points or gain temporary Hit Points.

\textbf{Actions}

\textit{\textbf{Bites.} Melee Weapon Attack}: +3 to hit, reach 0m, one target in the swarm's space.

\textit{Hit:} 10 (4d4) piercing damage, or 5 (2d4) piercing damage if the swarm is at half or fewer Hit Points.

\

\textbf{Swarm of Crows}\index[Monsters]{Swarm of Crows}

\textit{Medium Tiny Beast Swarm, unaligned}

\textbf{STRENGTH} -2

\textbf{DEXTERITY} +2

\textbf{CONSTITUTION} -1

\textbf{INTELLIGENCE} -4

\textbf{WISDOM} +1

\textbf{CHARISMA} -2

\textbf{Initiative} +2 -- \textbf{Defence} 13

\textbf{Hit Points} 24 (7d8 -- 7)

\textbf{Move} 3m, fly 15m

\textbf{Saving Throws}: Fortitude -1, Reflexes +3, Will +2

\textbf{Skills} Awareness +5

\textbf{Damage Resistances} slashing, piercing, slashing

\textbf{Condition Immunity} charmed, grabbed, restrained, paralyzed, petrified, prone, frightened, stunned

\textbf{Languages} -

\textbf{Challenge} 1/4 (50 XP)

\textit{\textbf{Swarm.}} The swarm can occupy another creature's space and vice versa, and the swarm can move through any opening large enough for a Tiny Crow. The swarm cannot regain Hit Points or gain temporary Hit Points.

\textbf{Actions}

\textit{\textbf{Beaks.} Melee Weapon Attack}: +4 to hit, reach 1m, one target in the swarm's space.

\textit{Hit:} 7 (2d6) piercing damage, or 3 (1d6) piercing damage if the swarm is at half or fewer Hit Points.

\

\textbf{Swarm of Rats}\index[Monsters]{Swarm of Rats}

\textit{Medium Tiny Beast Swarm, unaligned}

\textbf{STRENGTH} -1

\textbf{DEXTERITY} +0

\textbf{CONSTITUTION} -1

\textbf{INTELLIGENCE} -4

\textbf{WISDOM} +0

\textbf{CHARISMA} -4

\textbf{Initiative} +0 -- \textbf{Defence} 11

\textbf{Hit Points} 24 (7d8 - 7)

\textbf{Move} 9m

\textbf{Saving Throws}: Fortitude +0, Reflexes +1, Will +1

\textbf{Damage Resistances} slashing, piercing, slashing

\textbf{Condition Immunity} charmed, grabbed, restrained, paralyzed, petrified, prone, frightened, stunned

\textbf{Senses} vision in the dark 9m

\textbf{Languages} -

\textbf{Challenge} 1/4 (50 XP)

\textit{\textbf{Enhanced sense of smell.}} The swarm has +1d6 on Wisdom (Awareness) checks based on smell.

\textit{\textbf{Swarm.}} The swarm can occupy another creature's space and vice versa, and the swarm can move through any opening large enough for a Tiny Rat. The swarm cannot regain Hit Points or gain temporary Hit Points.

\textbf{Actions}

\textit{\textbf{Bites.} Melee Weapon Attack}: +2 to hit, reach 0m, one target in the swarm's space.

\textit{Hit:} 7 (2d6) piercing damage, or 3 (1d6) piercing damage if the swarm is at half or fewer Hit Points.

\

\textbf{Swarm of Spiders}\index[Monsters]{Swarm of Spiders}

\textit{Medium Tiny Beast Swarm, unaligned}

\textbf{STRENGTH} -4

\textbf{DEXTERITY} +1

\textbf{CONSTITUTION} +0

\textbf{INTELLIGENCE} -5

\textbf{WISDOM} -2

\textbf{CHARISMA} -5

\textbf{Initiative} +1 -- \textbf{Defence} 13

\textbf{Hit Points} 22 (5d8)

\textbf{Movement} 6m, climb 6m

\textbf{Saving Throws}: Fortitude -3, Reflexes +2, Will -1

\textbf{Damage Resistances} slashing, piercing, slashing

\textbf{Condition Immunity} charmed, grabbed, restrained, paralyzed, petrified, prone, frightened, stunned

\textbf{Senses} blindsight 3 m

\textbf{Languages} -

\textbf{Challenge} 1/2 (100 XP)

\textit{\textbf{Web Walk.}} The swarm ignores movement restrictions caused by webs.

\textit{\textbf{Climb as Spider.}} The swarm can climb difficult surfaces, including standing upside down on ceilings, without needing to make an ability check.

\textit{\textbf{Web Sense.}} While in contact with a web, the swarm knows the exact location of any other creature in contact with the same web.

\textit{\textbf{Swarm.}} The swarm can occupy another creature's space and vice versa, and the swarm can move through any opening large enough for a Tiny insect. The swarm cannot regain Hit Points or gain temporary Hit Points.

\textbf{Actions}

\textit{\textbf{Bites.} Melee Weapon Attack}: +3 to hit, reach 0m, one target in the swarm's space.

\textit{Hit:} 10 (4d4) piercing damage, or 5 (2d4) piercing damage if the swarm is at half or fewer Hit Points.

\

\textbf{Swarms}\index[Monsters]{Swarms}

The swarms presented below are not ordinary or benign gatherings of small creatures. Instead, they form as a result of an external, often malignant influence. Even druids are unable to charm these swarms, and their aggression is almost unnatural.

\textbf{Centipede Swarm}\index[Monsters]{Centipede Swarm}

\textit{Medium Tiny Beast Swarm, unaligned}

\textbf{STRENGTH} -4

\textbf{DEXTERITY} +1

\textbf{CONSTITUTION} +0

\textbf{INTELLIGENCE} -5

\textbf{WISDOM} -2

\textbf{CHARISMA} -5

\textbf{Initiative} +1 -- \textbf{Defence} 13

\textbf{Hit Points} 22 (5d8)

\textbf{Move} 6m, climb 6m

\textbf{Saving Throws}: Fortitude -1, Reflexes +3, Will +1

\textbf{Damage Resistances} slashing, piercing, slashing

\textbf{Condition Immunity} charmed, grabbed, restrained, paralyzed, petrified, prone, frightened, stunned

\textbf{Senses} blindsight 3 m

\textbf{Languages} -

\textbf{Challenge} 1/2 (100 XP)

\textit{\textbf{Swarm.}} The swarm can occupy another creature's space and vice versa, and the swarm can move through any opening large enough for a Tiny insect. The swarm cannot regain Hit Points or gain temporary Hit Points.

\textbf{Actions}

\textit{\textbf{Bites.} Melee Weapon Attack}: +3 to hit, reach 0m, one target in the swarm's space.

\textit{Hit:} 10 (4d4) piercing damage, or 5 (2d4) piercing damage if the swarm is at half or fewer Hit Points. A creature reduced to 0 Hit Points by a swarm of centipedes and stable is poisoned for 1 hour, even after regaining Hit Points, and is paralyzed by the poison during this time.

\

\index[Monsters]{Tarrasque}\textbf{Tarrasque}

\textit{Colossal monstrosity (titan), unaligned}

\textbf{STRENGTH} +10

\textbf{DEXTERITY} +0

\textbf{CONSTITUTION} +10

\textbf{INTELLIGENCE} -2

\textbf{WISDOM} +0

\textbf{CHARISMA} +0

\textbf{Initiative} +0 -- \textbf{Defence} 35

\textbf{Hit Points} 676 (33x3d6 + 330)

\textbf{Move} 24m

\textbf{Saving Throws}: Fortitude +40, Reflexes +30, Will +30

\textbf{Immunity to Damage} Fire, Poison. Electricity; Weapons +2

\textbf{Condition Immunity} charmed, poisoned, paralyzed, frightened, fatigued

\textbf{Senses} blind sight 36 m

\textbf{Languages} -

\textbf{Challenge} 30 (155000 XP)

\textit{\textbf{Reflecting Carapace.}} Whenever the Tarrasque is the target of a \textit{Arcane Dart} spell, a line spell, or a spell that requires a ranged attack roll, roll a d6. On a 1 to 5, the Tarrasque ignores it. On a 6, the Tarrasque ignores it, and the effect is reflected back at the caster as if originating from the Tarrasque, transforming the caster into the target.

\textit{\textbf{Siege Monster.}} The Tarrasque deals double damage to objects and structures.

\textit{\textbf{Legendary Endurance (3 / Day).}} If the Tarrasque fails a Saving Throw, he can choose to succeed instead.

\textit{\textbf{Resistance to Magic.}} The Tarrasque has +1d6 on Saving Throws against spells or other magical effects.

\textit{\textbf{Regeneration.}} The Tarrasque regenerates 10 Hit Points at the start of its round if it took no acid damage in the previous round.


\textbf{Actions}

\textit{\textbf{Multiattack.}} The Tarrasque can use its Frightening Presence. He then makes five attacks: one with his bite, two with his claws, one with his horns, and one with his tail. Instead of biting, he can use Swallow. Tarrasque attack's are considerated +4 magical.

\textit{\textbf{Claw.} Melee Weapon Attack}: +30 to hit, reach 5 meters, one target.

\textit{Hit:} 28 (4d8 + 10) slashing damage, 3 bleed damage.

\textit{\textbf{Tail.} Melee Weapon Attack}: +30 to hit, reach 6m, one target.

\textit{Hit:} 24 (4d6 + 10) bludgeoning damage. If the target is a creature, it must succeed on a DC 32 Fortitude save or fall prone.

\textit{\textbf{Horns.} Melee Weapon Attack}: +30 to hit, reach 3m, one target.

\textit{Hit:} 32 (4d10 + 10) piercing damage.

\textit{\textbf{Bite.} Melee Weapon Attack}: +30 to hit, reach 3m, one target.

\textit{Hit:} 36 (4d12 + 10) piercing damage. If the target is a creature, she is grabbed (DC 20 to escape). Until the grab ends, the target is restrained, and the tarrasque can't use the bite against another target.

\textit{\textbf{Swallow Whole.}} The Tarrasque makes a bite attack against a Large or smaller target it is grappling. If the attack hits, the target is engulfed, and the grab ends. The swallowed target is blinded and restrained, has full cover against attacks and other effects outside the tarrasque, and takes 56 (16d6) acid damage at the start of each round of the tarrasque.

If the Tarrasque takes 60 or more damage in a single round from a creature within it, the Tarrasque must succeed on a DC 30 Fortitude save at the end of that round or vomit all engulfed creatures, who fall prone in a space within 3 meters from the Tarrasque. If the tarrasque dies, a swallowed creature is no longer restrained by it and can exit the corpse using 10 meters of movement, coming prone.

\textit{\textbf{Frightening Presence.}} Any creature of the Tarrasque's choice that is within 16 meters of it and aware of its presence must succeed at a DC 30 Will save or be frightened for 1 minute. A creature can repeat the Saving Throw at the end of each of its rounds, with a -1d6 if the tarrasque is in line of sight, ending the effect on itself on a successful one. If the creature's Saving Throw succeeds or the effect ends for it, the creature is immune to the tarrasque's frightening presence for the next 24 hours.

\textbf{Additional Actions}

The Tarrasque can perform 3 additional Actions, chosen from the options below. It can only use one Additional Action at a time, and only at the end of another creature's turn. The tarrasque regains expended additional actions at the start of its round.

\textbf{Attack.} The Tarrasque makes a claw or tail attack. \textbf{Chew (Costs 2 Actions).} The Tarrasque makes a bite attack or uses Swallow Whole.

\textbf{Move.} The Tarrasque moves up to half its movement.

\textbf{Ecology}\\
Environment: Any\\
Organization: Solitary\\
\textbf{Treasure}: None\\
\textbf{Description}\\
The legendary Tarrasque is among the most destructive monsters in the world. Fortunately, he spends most of his time in a sort of deep hibernation in an unfamiliar cave in a remote corner of the world. When he awakens, however, whole kingdoms perish.

While not particularly intelligent, the Tarrasque is intelligent enough to understand some words in the language of the Deep (while being unable to speak). Likewise, the fury isn't uncontrolled: it focuses on the creature that did it the most damage and is difficult to distract through trickery.

Legends say that Tarrasque is Cattalm's pet.

\

\textbf{Badger}\index[Monsters]{Badger}

\textit{Tiny beast, unaligned}

\textbf{STRENGTH} -3

\textbf{DEXTERITY} +0

\textbf{CONSTITUTION} +1

\textbf{INTELLIGENCE} -4

\textbf{WISDOM} +1

\textbf{CHARISMA} -3

\textbf{Initiative} +0 -- \textbf{Defence} 11

\textbf{Hit Points} 3 (1d4 + 1)

\textbf{Movement} 6m, digging 1m

\textbf{Saving Throws}: Fortitude -3, Reflexes +1, Will +1

\textbf{Senses} vision in the dark 9m

\textbf{Languages} -

\textbf{Challenge} 0 (10 XP)

\textit{\textbf{A keen sense of smell.}} The badger has +1d6 on Wisdom (Awareness) checks based on smell.

\textbf{Actions}

\textit{\textbf{Bite.} Melee Weapon Attack}: +2 to hit, reach 1m, one target.

\textit{Hit:} 1 piercing damage.

\

\textbf{Tiger}\index[Monsters]{Tiger}

\textit{Large beast, unaligned}

\textbf{STRENGTH} +3

\textbf{DEXTERITY} +2

\textbf{CONSTITUTION} +2

\textbf{INTELLIGENCE} -4

\textbf{WISDOM} +1

\textbf{CHARISMA} -1

\textbf{Initiative} +2 -- \textbf{Defence} 13

\textbf{Hit Points} 37 (5d10 + 10)

\textbf{Move} 12m

\textbf{Saving Throws}: Fortitude +4, Reflexes +4, Will +2

\textbf{Skills} Stealth +6, Awareness +3

\textbf{Senses} vision in the dark 18m

\textbf{Languages} -

\textbf{Challenge} 1 (200 XP)

\textit{\textbf{Leap.}} If the tiger moves at least 6 meters directly towards a creature and hits it with a claw attack during the same turn, the target must succeed on a DC 13 Fortitude save or fall prone. If the target is prone, the tiger can make a bite attack against it as a bonus action.

\textit{\textbf{Enhanced sense of smell.}} The tiger has +1d6 on Wisdom (Awareness) checks based on smell.

\textbf{Actions}

\textit{\textbf{Claw.} Melee Weapon Attack}: +5 to hit, reach 1m, one target.

\textit{Hit:} 7 (1d8 + 3) slashing damage, 1 bleed damage.

\textit{\textbf{Bite.} Melee Weapon Attack}: +5 to hit, reach 1m, one target.

\textit{Hit:} 8 (1d10 + 3) piercing damage.

\

\textbf{Topi, La}\\\index[Monsters]{Topi, La}
\textit{Tiny Fairy mice}\\
\textbf{Strength}: -1\\
\textbf{Dexterity}: +4\\
\textbf{Constitution}: +0\\
\textbf{Intelligence}: +6\\
\textbf{Wisdom}: +2\\
\textbf{Charisma}: +6\\
\textbf{Defence}: 17 -- \textbf{Initiative}: +15\\
\textbf{Hit Points}: 4 (1d10 - 1)\\
\textbf{Movement}: 6m\\
\textbf{Saving Throws}: Fortitude +40, Reflexes +40, Will +40 \\
\textbf{Senses}: Tremorsense 30m, Darkvision 30m, True Seeing 30m\\
\textbf{Languages}: all\\
\textbf{Challenge} 0 (10 XP)\\
\textbf{Immunity}: to damage from weapons with magic bonus less than +6\\
\textbf{Immunity}: Any effect that does not please the Topi\\
\textbf{Immunity}: To any spell the Topi doesn't want to be affected\\
\textbf{Immunity}: To suffer any type of critical roll\\
\textit{\textbf{It's La Topi}} La Topi has +3d6 (or +18) whenever she has to roll dice or count a value.
Any attack made by the Topi is considered magical +6 and is not resistible.\\
\textbf{Actions}\\
\textit{\textbf{Snout}} each creature of your choice, within 30 meters, suffers a Snout. The creature is knocked away 2d6 meters and takes 3d6 damage\\
\textit{\textbf{Mouse bite} Melee Weapon Attack}: +26 on hit, reach 1m, one target.\\
\textit{Hit:} 6 piercing damage.\\
\textit{\textbf{Scratch} up to 8 Melee Weapon Attacks}: hits automatically, reach 1m, up to 4 targets.\\
\textit{Hit:} 1 piercing damage.\\
\textit{\textbf{Enraged}}: the Topi does what she wants. Cost 1 Reaction.\\
\textbf{Ecology}\\
Environment: Anywhere\\
Organization: Solitary\\
\textbf{Treasure}: Special\\
\textbf{Description}\\
It could be mistaken for a little white mouse, but La Topi is much more. Smart, intelligent, beautiful, she loves going to the fair and buying handbags.


\

\index[Monsters]{Tortoise Dragon}\textbf{Tortoise Dragon}

\textit{Gargantuan Dragon, Neutral}

\textbf{STRENGTH} +7

\textbf{DEXTERITY} +0

\textbf{CONSTITUTION} +5

\textbf{INTELLIGENCE} +0

\textbf{WISDOM} +1

\textbf{CHARISMA} +1

\textbf{Initiative} +0 -- \textbf{Defence} 29

\textbf{Hit Points} 341 (22x3d6 + 110)

\textbf{Movement} 6m, swim 12m

\textbf{Saving Throws} Fortitude +22, Reflexes +17, Will +18

\textbf{Senses} Darkvision 18m

\textbf{Languages} Aquan, Draconic

\textbf{Challenge} 17 (18000 XP)

\textit{\textbf{Amphibian.}} The Dragon Tortoise can breathe air and water.

\textbf{Actions}

\textit{\textbf{Multiattack.}} The dragon can make three attacks: one with its bite and two with its claws. He can make a tail attack instead of two claw attacks.

\textit{\textbf{Claw.} Melee Weapon Attack}: +26 to hit, reach 3m, one target.

\textit{Hit:} 16 (2d8 + 7) slashing damage.

\textit{\textbf{Tail.} Melee Weapon Attack}: +26 to hit, reach 5 meters, one target.

\textit{Hit:} 26 (3d12 + 7) bludgeoning damage. If the target is a creature, it must succeed on a DC 22 Fortitude save or be knocked 3 meter away from the dragon tortoise and fall prone.

\textit{\textbf{Bite.} Melee Weapon Attack}: +26 to hit, reach 5 meters, one target.

\textit{Hit:} 26 (3d12 + 7) piercing damage.

\textit{\textbf{Blast of Steam (Cooldown 5-6).}} The Dragon Tortoise exhales hot steam in a 20m cone. Each creature in that area must make a DC 22 Fortitude save and take 52 (15d6) fire damage on a failed save, or half as much damage on a successful one. Being underwater gives no resistance against this type of damage.

\textbf{Ecology}
Environment: Temperate aquatic\\
Organization: Solitary\\
\textbf{Treasure}: Double\\
\textbf{Description}\\
Dragon tortoises inhabit both fresh and salt waters, where they are among the greatest dangers to mariners and those traveling by ship through the world's sea lanes. Experienced mariners know what the dragon tortoises of the area want and frequently make offerings of gold and magic to ensure safe passage or avoid the area altogether. For its part, a dragon tortoise appreciates and even expects such tolls and bribes, and a dragon tortoise that expects bribes but is ignored is a dangerous enemy indeed.

The shell color of a dragon tortoise varies from individual to individual. Some have dull brown and rusty-red shells, while others have deep blue-green shells with a silvery sheen at the rocky tips. The coloration of the head, tail and legs is slightly paler than the shell and includes golden streaks along the crest and spines.

Dragon tortoises claim huge territories in the open sea, which include regions often exceeding 75 square km. Here, these dangerous beasts capsize ships that do not respect their territories, adding sunken wrecks and their precious cargoes to their hiding places. Dragon tortoises generally make their lairs in deep caverns accessible only through water, and often decorate them not only with the riches looted from the ships they have sunk, but also with the wrecks of these unfortunate vessels. Their territorial nature and their preference for this type of lair put them in direct conflict with other underwater races such as merfolk and sahuagin.

Large fish, such as tuna, sturgeon and even sharks are among the favorite foods of dragon tortoises, but being omnivores, they sometimes also feed on large underwater fields of seaweed. They certainly do not disdain to integrate their diet with the passengers of sinking ships, even if this practice is not due to malice or cruelty. Dragon tortoises have shells 5 meters in diameter, with limbs extending a few meters outward, and measure 7 meters from the tip of their noses to the tip of their mighty tails.

\

\index[Monsters]{Treeman (Treant)}\textbf{Treeman (Treant)}

\textit{Huge Plant, Chaotic Good}

\textbf{STRENGTH} +6

\textbf{DEXTERITY} -1

\textbf{CONSTITUTION} +5

\textbf{INTELLIGENCE} +1

\textbf{WISDOM} +3

\textbf{CHARISMA} +1

\textbf{Initiative} +1 -- \textbf{Defence} 21

\textbf{Hit Points} 138 (12d12 + 60)

\textbf{Move} 9m

\textbf{Saving Throws}: Fortitude +13, Reflexes +3, Will +9

\textbf{Damage Resistances} bludgeoning, piercing

\textbf{Damage Vulnerability} fire

\textbf{Languages} Common, Druidic, Elvish, Sylvan

\textbf{Challenge} 9 (5000 XP)

\textit{\textbf{False Appearance.}} While the treeman stands motionless, he is indistinguishable from a normal tree.

\textit{\textbf{Siege Monster.}} Treeman deals double damage to objects and structures.

\textbf{Actions}

\textit{\textbf{Multiattack.}} The treeman makes two slam attacks.

\textit{\textbf{Slam.} Melee Weapon Attack}: +16 to hit, reach 1m, one target.

\textit{Hit:} 16 (3d6 + 6) bludgeoning damage.

\textit{\textbf{Rock.} Ranged weapon attack}: +16 to hit, range 18m, one target.

\textit{Hit:} 28 (4d10 + 6) bludgeoning damage.

\textit{\textbf{Animate Trees (1/day).}} The treeman magically animates one or two visible trees within 20 meters of him. These trees have the same stats as the ent, except that they have Intelligence and Charisma scores of -3, cannot speak, and have only the slam attack option. An animated tree acts as an ally of the tree man. The tree stays for 1 day or until he dies; until the treeman dies or is more than 16 meters away from the tree, or until the treeman takes a bonus action to transform it back into an inanimate tree. Then the tree will take root if possible.

\textbf{Ecology}\\
Environment Any Forest\\
Organization: Solitary or scrub (2-7)\\
\textbf{Treasure}: Standard\\
\textbf{Description}\\
Treants are guardians of forests and ambassadors of trees. As old as the forests themselves, they see themselves as parents and shepherds rather than gardeners: they are slow and methodical, but terrifying when forced to fight to defend their flock. While they rarely seek the companionship of short-lived races and have an innate distrust of change, they show tolerance for those who want to learn from their long, slow monologues, especially those whose eyes can read a desire to protect the wilds. Against those who threaten their forests, especially loggers gathering wood or those who would clear a forest to build a road or a fort, the treant's rage is unleashed swift and devastating. They are able to tear down what others build - a trait that helps them during their excesses of rampage.

Treants are primarily solitary creatures, and a single individual is often responsible for an entire forest, but they sometimes gather in groups called groves to exchange news and reproduce.

In times of grave danger, all the groves of a region unite for a months-long meeting called a council, but such events are very rare, and even millennia pass between councils.

A typical treant stands 10 meters tall, with a trunk diameter of 0.5 meter, and weighs about 2500kg. Treants resemble the more common trees of the territories where they live.

\

\index[Monsters]{Troll}\textbf{Troll}

\textit{Large giant, chaotic evil}

\textbf{STRENGTH} +5

\textbf{DEXTERITY} +1

\textbf{CONSTITUTION} +5

\textbf{INTELLIGENCE} -2

\textbf{WISDOM} -1

\textbf{CHARISMA} -2

\textbf{Initiative} +1 -- \textbf{Defence} 18

\textbf{Hit Points} 84 (8d10 + 40)

\textbf{Move} 9m

\textbf{Saving Throws}: Fortitude +11, Reflexes +4, Will +3

\textbf{Skills} Awareness +2

\textbf{Senses} Darkvision 18m

\textbf{Languages} Giant

\textbf{Challenge} 5 (1800 XP)

\textit{\textbf{A keen sense of smell.}} The troll has +1d6 on Wisdom (Awareness) checks based on smell.

\textit{\textbf{Regeneration.}} The troll regains 10 Hit Points at the start of its round. If the troll takes acid or fire damage, this trait does not function at the start of the troll's next round. The troll dies only if it starts its round at -5 Hit Points and cannot regenerate.

\textbf{Actions}

\textit{\textbf{Multiattack.}} The troll can make three attacks: one with its bite and two with its claws.

\textit{\textbf{Claw.} Melee Weapon Attack}: +11 to hit, reach 1m, one target.

\textit{Hit:} 12 (2d6 + 5) slashing damage, 1 bleed damage.

\textit{\textbf{Bite.} Melee Weapon Attack}: +11 to hit, reach 1m, one target.

\textit{Hit:} 8 (1d6 + 5) piercing damage.

\textbf{Ecology}\\
Environment: Cold Mountains\\
Organization: Solitary or gang (2-4)\\
\textbf{Treasure}: Standard\\
\textbf{Description}\\
Trolls possess sharp claws and incredible regenerative abilities that allow them to heal almost any wound. They are hunchbacked, ugly but very strong: combined with their claws, their strength allows them to tear flesh with their bare hands. Trolls stand around 11 meter tall, but their posture makes them appear shorter. An adult troll weighs about 500 kg.

A troll's appetite and regenerative abilities make it an indomitable fighter, charging head-on at the nearest living creature and attacking with all its fury. Only fire makes a troll hesitate, but even what is mortal danger to him does not stop his advance. Those who face trolls know to locate and burn any part of themselves after a fight, for even the smallest shred of their body can be reborn into a full troll over time. Fortunately, only the larger parts of a troll, such as limbs, grow back this way.

Despite their ferocity, trolls are extraordinarily tender and kind to their young. Female trolls work in groups, spending a lot of time teaching the pups how to hunt and defend themselves before sending them off to find their own territory. A male troll lives a solitary existence, meeting females briefly only to mate. All trolls spend their time foraging for food, as they must consume huge amounts of it every day or they will starve. Because of this, most trolls create their own hunting territory which is often defended by fighting with rivals. Such encounters are usually non-lethal, but trolls know their weaknesses well, using them to kill an opponent in lean times.

It is universally known that trolls can naturally mutate by acquiring for short periods the most peculiar characteristics of the creatures they feed on. You have no idea how funny a Pegasutroll can be...


\

\index[Monsters]{Tàhil}\textbf{Tàhil}

\textit{Colossal dragon}

\textbf{STRENGTH} +10

\textbf{DEXTERITY} +3

\textbf{CONSTITUTION} +10

\textbf{INTELLIGENCE} +8

\textbf{WISDOM} +8

\textbf{CHARISMA} +9

\textbf{Initiative} +11 -- \textbf{Defence} 40

\textbf{Hit Points} 615 (30x3d6+300)

\textbf{Movement} 20 meters, fly 40 meters

\textbf{Saving Throws}: Fortitude +40, Reflexes +33, Will +38

\textbf{Skills} all +18

\textbf{Damage Immunity} cold, electricity, fire, acid, poison, sound, weapons +3

\textbf{Condition Immunity} charmed, poisoned, paralyzed, fatigued, frightened

\textbf{Senses} Darkvision 60m, True Seeing 40m

\textbf{Languages} all

\textbf{Challenge} 30 (155,000 XP)

\textbf{Immortal on Yeru.} When Tàhil's body is slain on Yeru it reforms in 3d6 days in the lair made by Calicante.

\textit{\textbf{Spells.}} Tàhil has MP 20. Her spellcasting characteristic is Charisma, +9 to hit on spell attacks. Tàhil knows the following spells:

At will: Divine Word

\textit{\textbf{Divine Nature.}} Tàhil has no need for air, food, drink, or sleep. Spells of 5th level and lower have no effect on Tàhil except if he wishes it to.

\textit{\textbf{Master of Dragons.}} Every Tàhil's Dragon on Yeru is faithful and obedient to Tàhil's will.

\textit{\textbf{Voice of the Master.}} Tàhil can converse with every Tàhil dragon in Yeru, regardless of distance.

\textit{\textbf{Call of the Master.}} Tàhil opens a portal and 1d2+1 Tàhil's dragons of random age and color exit. The power is usable once per day.

\textit{\textbf{Legendary Endurance (5 / Day).}} If the Tàhil fails a Saving Throw, he may choose to succeed instead.

\textit{\textbf{More heads.}} Tàhil has +1d6 on Saving Throws against being blind, deaf, unconscious. Tàhil can perform up to 6 Reactions per round.

\textit{\textbf{Regeneration.}} Tàhil regenerates 30 Hit Points at the start of her round

\textbf{Actions}

\textit{\textbf{Multiattack.}} Tàhil can use her Dreadful Presence or make 3 attacks (2 with claws and one with her tail) or just one with her bite. Claw +30, reach 5 meters. Tail +30 reach 8 meters. Bite +30, reach 6 meters. All of Tàhil's attacks are considered magical +5.

\textit{Hit:} Claw, 24 (4d6 +10, 5 bleed damage, to a maximum of 40) slashing. Tail, 28 (4d8 +10) bludgeoning. Bite 48 (8d6 +10) on a critical roll of the bite cuts the creature's body in half if a DC 30 Fortitude save is not successful.

\textit{\textbf{Frightening Presence}} Any creature that can see Tàhil and is within 80 m must make a DC 32 Will save or be Frightened for 1 minute. Each round the creature can make a Saving Throw, if this succeeds it is immune to Tàhil's Dreadful Presence for the next 24 hours.

\textbf{Additional Actions}

The Tàhil can perform 3 additional actions, chosen from those below and one per round only at the end of another creature's round. Tàhil can change appearance of his head to simulate powers of other type of dragon. The actions depend on the chosen head. 

\textbf{Claw attack.}: +19, reach 6 meters, one objective. On hit, 32 (4d10 + 10, 3 from Bleed) slashing damage plus 14 (4d6) acid (Black head) or Electricity (Blue head) or Poison (Green head) or Fire (Red head) or Cold damage (White head) or by Fire (Yellow head) or by Sound (Violet head)

\textbf{Black head.}: Costs 2 legendary actions, Tàhil breathes acid into a 40-meter cone. DC 27 Reflex save or take 68 (15d8) acid damage or halve.

\textbf{Blue Head.}: Costs 2 legendary actions, Tàhil breathes Electricity into a 40-meter cone. DC 27 Reflex save or take 88 (16d10) electricity damage or halve.

\textbf{Green head.}: Costs 2 legendary actions, Tàhil breathes poison into a 30-meter cone. DC 27 Reflex save or take 77 (22d6) poison damage or halve.

\textbf{Red head.}: Costs 2 legendary actions, Tàhil breathes Fire in a 30-meter cone. DC 27 Reflex save or take 91 (26d6) fire damage or halve.

\textbf{White head.}: Costs 2 Legendary Actions, Tàhil blows Ice into a 30m cone. DC 27 Reflex save or take 72 (16d8) ice damage or halve.

\textbf{Purple Head.}: Costs 2 legendary actions, Tàhil breathes Sound into a 30-meter cone. DC 27 Reflex save or take 90 (18d8) sound damage or halve.

\textbf{Yellow head.}: Costs 2 legendary actions, Tàhil blows hot sand in a 60-meter cone. DC 27 Reflex save or take 72 (16d8) fire damage or halve.


\textbf{Ecology}\\
Environment: Unknown\\
Organization: Unique\\
\textbf{Treasure}: Special\\

\textbf{Description}
Tàhil is the Patron of Dragons incarnate. Nothing resists his fury, madness, rage and destruction. See chapter on Cosmology for details of his story.


\

\index[Monsters]{Unicorn}\textbf{Unicorn}

\textit{Large Celestial, Lawful Good}

\textbf{STRENGTH} +4

\textbf{DEXTERITY} +2

\textbf{CONSTITUTION} +2

\textbf{INTELLIGENCE} +0

\textbf{WISDOM} +3

\textbf{CHARISMA} +3

\textbf{Initiative} +2 -- \textbf{Defence} 15

\textbf{Hit Points} 67 (9d10 + 18)

\textbf{Move} 15m

\textbf{Saving Throws}: Fortitude +7, Reflexes +7, Will +6; +2 resistance against Void, Negative Energy

\textbf{Immunity to Damage} Poison

\textbf{Condition Immunity} charmed, poisoned, paralyzed

\textbf{Senses} Darkvision 18m

\textbf{Languages} Celestial, Elven, Sylvan, telepathy 18m

\textbf{Challenge} 5 (1800 XP)

\textit{\textbf{Magic Weapons.}} The unicorn's weapon attacks are magical.

\textit{\textbf{Charge.}} If the unicorn moves at least 6 meters in a straight line towards the target and hits it with a horn attack during the same turn, the target takes 9 (2d8) piercing damage additional. If the target is a creature, it must succeed on a DC 15 Fortitude save or be knocked prone.

\textit{\textbf{Innate Spells.}} The unicorn's innate spellcasting ability is Charisma (DC 14 for spell saves). The unicorn can innately cast the following spells, requiring no components:

At will: \textit{Druidic Artifice, detect good and evil, pass without a trace}

1/day each: \textit{calm emotions, dispel good and evil,} \textit{entangle}

\textit{\textbf{Resistance to Magic.}} The unicorn has +1d6 on Saving Throws against spells and other magical effects.

\textbf{Actions}

\textit{\textbf{Multiattack.}} The unicorn makes two attacks: one with its hooves and one with its horn.

\textit{\textbf{Horn.} Melee Weapon Attack}: +10 to hit, reach 1m, one target.

\textit{Hit:} 8 (1d8 + 4) piercing damage.

\textit{\textbf{Hooves.} Melee Weapon Attack}: +10 to hit, reach 1m, one target.

\textit{Hit:} 11 (2d6 + 4) bludgeoning damage.

\textit{\textbf{Teleport (1/day).}} The unicorn can magically teleport itself and up to three other willing creatures visible within 1 meter of it, along with any equipment they are wearing or carrying , in a place familiar to the unicorn, which is a maximum of 1.5 kilometers away.

\textit{\textbf{Healing Touch (3/Day).}} The unicorn makes horn contact with another creature. The target magically regains 11 (2d8 + 2) Hit Points. Additionally, the touch removes all disease and neutralizes all poison afflicting the target.

\textbf{Additional Actions}

The unicorn can perform 3 additional Actions, chosen from the options below. It can only use one Additional Action at a time, and only at the end of another creature's turn. The unicorn regains expended additional actions at the start of its round.

\textbf{Self-Heal (Costs 3 Actions).} The unicorn magically regains 11 (2d8 + 2) Hit Points.

\textbf{Shimmering Shield (Costs 2 Actions).} The unicorn creates a glittering magical field that surrounds itself or another creature visible to it within 20 meters. The target gains a +2 bonus to Defence until the end of the unicorn's next round.

\textbf{Hooves.} The unicorn makes a hoof attack.

\textbf{Ecology}\\
Environment: Temperate Forests\\
Organization: Solitary, pair, or blessing (3-6)\\
\textbf{Treasure}: None\\
\textbf{Description}\\
Unicorns are fierce, intelligent woodland creatures who prefer to remain isolated, appearing only to defend their homes from evil. They shun all creatures except good fey, good humanoid women, and animals native to their forest, but may team with other good creatures against common enemies. A typical unicorn is 2.4 meters long, 1 meter tall at the shoulder and weighs 600 kg.

Pairs of unicorns stay together for life and dwell in particular clearings or within the forests they defend. They allow good and neutral creatures to cross them, hunt or live there, but evil creatures or those who would like to disturb their ecosystem, for example by hunting for fun or cutting down their trees to sell their timber, are quickly driven away or killed. On some rare occasions, unicorns whose partners have been slain take young women of rare virtue as surrogates, allowing them to ride them and become their lifelong guardians. If the woman becomes attached to someone else, such as a son or a lover, the bond with the unicorn is lovingly dissolved, generating the legend that unicorns only befriend virgins.

A unicorn's horn is the source of her powers, and to use her magical abilities on other creatures they must touch them with it. Evil creatures place great value on unicorn horns as reagents for healing potions and for dark rites: a powdered unicorn horn is worth 800 gp when used to create a magic item of healing.


\

\index[Monsters]{Vampires, Vampire}\textbf{Vampire}

\textit{Medium Undead (Shapeshifter), Lawful Evil}

\textbf{STRENGTH} +4

\textbf{DEXTERITY} +4

\textbf{CONSTITUTION} +4

\textbf{INTELLIGENCE} +3

\textbf{WISDOM} +2

\textbf{CHARISMA} +4

\textbf{Initiative} +4 -- \textbf{Defence} 23

\textbf{Hit Points} 144 (17d8 + 68)

\textbf{Move} 9m

\textbf{Saving Throws}: Fortitude +17, Reflexes +17, Will +14

\textbf{Skills} Stealth +9, Awareness +17

\textbf{Damage Immunity} Void; from a non-magical weapon

\textbf{Condition Immunity} charmed, deafened, bleeding

\textbf{Senses} darkvision 40m

\textbf{Languages} the languages he knew in life, Exspiram

\textbf{Challenge} 13 (10000 XP)

\textit{\textbf{Shapeshift.}} If the vampire is not in sunlight or immersed in flowing water, she can use 1 action to transform into a Tiny bat, a Medium mist cloud, or return to her its true form.

While in bat form, the vampire cannot speak, its walking speed is 1 meter, and its flying speed is 10 meters. His stats, aside from size and speed, are unchanged. Whatever gear she's wearing transforms with it, but whatever she was carrying is knocked to the ground. Upon death he reverts to his true form.

While in mist form, the vampire cannot take actions, speak, or manipulate objects. He is weightless, has a flying speed of 6 meters, can hover, and can enter a hostile creature's space and stop there. Also, if air passes through a space, the mist can do so loosely, but it can't pass through water. He has +1d6 on Fortitude and Reflex saves and is immune to all nonmagical damage, except damage taken from
Sun.

\textit{\textbf{Weaknesses of the Vampire.}} The Vampire has the following flaws:

\textit{Damaged by flowing water.} The vampire takes 20 acid damage if he ends his round in flowing water.

\textit{Hypersensitivity to Light.} The vampire takes 20 Light damage when he begins his round in sunlight. While he is in the sunlight, he has -1d6 on attack rolls and proficiency checks.

\textit{Stake through Heart.} If a piercing weapon made of wood is driven into a vampire's heart while the vampire is incapacitated in its resting place, the vampire is paralyzed until the stake is removed.

\textit{Prohibition.} The vampire cannot enter a dwelling without invitation from its occupants.

\textit{\textbf{Escape into the Mist.}} When she drops to 0 Hit Points outside her resting place, the vampire transforms into a cloud of mist (as per the Shapeshifter trait) instead of falling free of senses, as long as it is not exposed to sunlight or running water. If he can't transform, he is destroyed.

While at 0 Hit Points in this form, he cannot revert to his vampire form, and must reach his resting place within 2 hours or be destroyed. Upon reaching his resting place, he reverts to his vampire form. He will then remain paralyzed until he has regained at least 1 hit point. After spending at least 1 hour in its resting place with 0 Hit Points, the vampire will regain 1 hit point.

\textit{\textbf{Undead nature.}} The vampire does not need air.

\textit{\textbf{Legendary Endurance (3 / Day).}} If the vampire fails a Saving Throw, he may choose to succeed instead.

\textit{\textbf{Regeneration.}} The vampire regains 20 Hit Points at the start of his round if he has at least 1 hit point and is not exposed to sunlight or running water. If the vampire takes damage from Light or damage from holy water, this trait does not function at the start of the vampire's next round.

\textit{\textbf{Climb as Spider.}} The vampire can climb difficult surfaces, including standing upside down on ceilings, without needing to make an ability check.

\textbf{Actions}

\textit{\textbf{Multiattack.}} The vampire can make two attacks, but only one of them can be a bite attack.

\textit{\textbf{Unarmed Strike (In Vampire Form Only).} Melee Weapon Attack}: +18 to hit, reach 1 m, one creature.

\textit{Hit:} 8 (1d8 + 4) bludgeoning damage. Instead of dealing damage, the vampire can grab the target (DC to flee 18).

\textit{\textbf{Bite (In Bat or Vampire Form Only).} Melee Weapon Attack}: +18 to hit, reach 1m, one willing creature or one grabbed, incapacitated, or entangled by the vampire.

\textit{Hit:} 7 (1d6 + 4) piercing damage plus 10 (3d6) void damage. The target's maximum Hit Points are reduced by an amount equal to the void damage taken, and the vampire regains a number of Hit Points equal to that amount, DC 23 Fortitude save to resist the loss of maximum Hit Points. The target becomes fatigued. The target dies if this effect reduces its maximum Hit Points to 0. A humanoid slain in this way and then buried in the ground revives the following night as a vampire spawn under the vampire's control.

\textit{\textbf{Fascinate.}} The vampire targets one humanoid within 10 meters that he can see. If the target can see the vampire, he must make a DC 22 Will save against this spell or be charmed by it. The charmed target considers the vampire a trusted friend to listen to and protect. While the target is not under the vampire's control, it takes the vampire's requests and actions as favorably as possible, and is a willing target of the vampire's bite attack.

Whenever the vampire or the vampire's companions do something harmful to the target, it can repeat the Saving Throw, ending the effect on itself on a success. Otherwise, the effect lasts 24 hours or until the vampire is destroyed, is on a different plane of existence than the target, or takes a bonus action to end the effect.

\textit{\textbf{Children of the Night (1/Day).}} The vampire magically summons 2d4 swarms of bats or rats, provided the sun has not risen. While he is outside, the vampire can summon 3d6 wolves instead. The summoned creatures arrive in 1d4 rounds, acting as allies to the vampire and obeying his commands. The beasts remain for 1 hour, until the vampire dies, or until he dismisses them as a bonus action.

\textbf{Additional Actions}

The vampire can perform 3 additional Actions, chosen from the options below. He can only use one Additional Action at a time, and only at the end of another creature's turn. The vampire recovers the additional Actions he has spent at the beginning of his round.

\textbf{Unarmed Strike.} The vampire makes an unarmed strike.

\textbf{Bite (Costs 2 Actions).} The vampire makes a bite attack.

\textbf{Move.} The vampire moves its movement without provoking attacks of opportunity.

\textbf{Ecology}
Environment: Any\\
Organization: Solitary or family (vampire plus 2-8 Spawn)\\
\textbf{Treasure}: NPC gear (Ring of Protection +2, Sash of Seduction +4, Cloak of Resistance +3)\\
\textbf{Description}\\
Vampires are undead humanoid creatures that feed on the blood of the living. They look much like they did in life, often becoming more attractive, although some appear tough and feral.


\

\index[Monsters]{Vampires, Vampiric Spawn}\textbf{Vampiric Spawn}

\textit{Medium Undead, Neutral Evil}

\textbf{Initiative} +0 -- \textbf{Defence} 18

\textbf{Hit Points} 82 (11d8 + 33)

\textbf{Move} 9m

\textbf{Saving Throws} Fortitude +3, Reflexes +2, Will +5

\textbf{STRENGTH} +3

\textbf{DEXTERITY} +3

\textbf{CONSTITUTION} +3

\textbf{INTELLIGENCE} +0

\textbf{WISDOM} +0

\textbf{CHARISMA} +1

\textbf{Skills} Stealth +6, Awareness +3

\textbf{Damage Resistances} from Void; from a non-magical weapon

\textbf{Senses} Darkvision 18m

\textbf{Languages} the languages he knew in life

\textbf{Challenge} 6 (1800 XP)

\textit{\textbf{Weaknesses of Vampiric Spawn.}} Vampiric Spawn has the following flaws:

\textit{Damaged by Running Water.} The Vampiric Spawn takes 20 acid damage if it ends its round in running water.

\textit{Hypersensitivity to Light.} The Vampiric Spawn takes 20 Light damage when it begins its round in sunlight. While in the sunlight, he has -1d6 on attack rolls and proficiency checks.

\textit{Stake through Heart.} The Vampiric Spawn is destroyed if a wooden piercing weapon is driven through its heart while it is incapacitated within its resting place.

\textit{Prohibition.} Vampiric Spawn cannot enter a dwelling without invitation from its occupants.

\textit{\textbf{Undead Nature.}} Vampiric Spawn does not need air.

\textit{\textbf{Regeneration.}} The Vampiric Spawn regains 10 Hit Points at the start of its round if it has at least 1 hit point and is not exposed to sunlight or running water. If the vampire spawn takes damage from light or damage from holy water, this trait does not function at the start of the vampire's next round.

\textit{\textbf{Climb as Spider.}} The Vampiric Spawn can climb difficult surfaces, including standing upside down on ceilings, without needing to make an ability check.

\textbf{Actions}

\textit{\textbf{Multiattack.}} The vampire spawn can make two attacks, but only one of them can be a bite attack.

\textit{\textbf{Claws.} Melee Weapon Attack}: +9 to hit, reach 1m, one creature.

\textit{Hit:} 8 (2d4 + 3) slashing damage. Instead of dealing damage, the vampire can grab the target (DC to flee 13).

\textit{\textbf{Bite.} Melee weapon attack}: +9 to hit, reach 1m, one creature grabbed by the vampire, incapacitated, or entangled.

\textit{Hit:} 6 (1d6 + 3) piercing damage plus 7 (2d6) void damage. The target's maximum Hit Points are reduced by an amount equal to the void damage taken, and the vampire regains a number of Hit Points equal to that amount, DC 20 Fortitude save to resist the loss of maximum Hit Points. The target dies if this effect reduces its maximum Hit Points to 0. The creature becomes fatigued.

\textbf{Ecology}\\
Environment: Any\\
Organization: Solitary, pair, group (3-6) or mob (7-12)\\
\textbf{Treasure}: Standard\\
\textbf{Description}\\
A Vampire can decide to create a vampire spawn from a victim instead of making a full vampire only when he uses his create spawn ability on a humanoid creature. This decision must be made as a free action as soon as a vampire kills an appropriate creature using its bite.

\

\textbf{Venomous Snake}\index[Monsters]{Venomous Snake}

\textit{Tiny beast, unaligned}

\textbf{STRENGTH} -4

\textbf{DEXTERITY} +3

\textbf{CONSTITUTION} +0

\textbf{INTELLIGENCE} -5

\textbf{WISDOM} +0

\textbf{CHARISMA} -4

\textbf{Initiative} +3 -- \textbf{Defence} 14

\textbf{Hit Points} 2 (1d4)

\textbf{Movement} 9m, swim 9m

\textbf{Saving Throws}: Fortitude +1, Reflexes +4, Will +1

\textbf{Senses} blindsight 3 m

\textbf{Languages} -

\textbf{Challenge} 1/8 (25 XP)

\textbf{Actions}

\textit{\textbf{Bite.} Melee Weapon Attack}: +5 to hit, reach 1m, one target.

\textit{Hit:} 1 piercing damage and the target must make a DC 10 Fortitude save, taking 5 (2d4) poison damage on a failed save, or half as much damage on a successful one.

\

\index[Monsters]{Wallcrawler Horror}\textbf{Wallcrawler Horror}

\textit{Large monstrosity, unaligned}

\textbf{STRENGTH} +4

\textbf{DEXTERITY} +0

\textbf{CONSTITUTION} +2

\textbf{INTELLIGENCE} -2

\textbf{WISDOM} +1

\textbf{CHARISMA} -2

\textbf{Initiative} +1 -- \textbf{Defence} 15

\textbf{Hit Points} 75 (10d10 + 25)

\textbf{Move} 9m, climb 9m

\textbf{Saving Throws}: Fortitude +5, Reflexes +3, Will +4

\textbf{Senses} darkvision 3m, blindsight 20m

\textbf{Languages} Wall Climbing Horror

\textbf{Challenge} 3 (700 XP)

\textit{\textbf{Radar Sense.}} The Wallcrawler Horror cannot use blindsight if deafened.

\textbf{Actions}

\textit{\textbf{Multiattack.}} The Wallcrawler Horror makes two attacks with its barbed claws.

\textit{\textbf{Claws.} Melee Weapon Attack}: +7 to hit, reach 1m, one target.

\textit{Hit:} 10 (2d6 + 4) piercing damage, 1 bleed damage.

\textbf{Ecology}\\
\textbf{Environment: Underground}
Organization: Solitary, pair, or pack (3-8)\\
\textbf{Treasure}: Accidental\\
\textbf{Description}\\
The Wallcrawler Horror is a ferocious predator of the underground, aggressively defending its hunting grounds. The subterranean caverns in which these creatures dwell thunder with the thumping and swishing of their hooks as these creatures clamber up rocky cliffs or cave walls.

A wall-crawler horror is a monstrous creature with a vulture-like head and a massive beetle-like thorax, protected by an exoskeleton studded with sharp bony protuberances. It takes its name as well as from the horrendous appearance from the fact that using the long and muscular limbs that end with the deadly curved hooked claws it is climbed on the walls.

\textit{Echoes in the Darkness}. Wallcrawler Horrors communicate by striking their exoskeleton or surrounding rock surfaces with their hooks. What appears to others as a random noise is actually an elaborate language that only Wall Climbing Horrors understand and the echoes of which carry for miles and miles underground.

\textit{Pack of Raiders}. Wallcrawler Horrors are omnivorous creatures: they feed on fungi, lichens, plants, and any creature they can catch. With their hooked limbs, horrors gain a good grip on rocky surfaces and use their climbing skills to ambush prey from above. They hunt in packs and work together to take on the biggest and most dangerous opponents. If a battle goes badly, a wall-climbing horror quickly scrambles up a cave wall to escape.

\textit{Solidarity Clans}. Hook horrors live in large family groups or clans. Each clan is led by the eldest female, who usually places her mate in charge of the clan's hunters. Wallcrawler Horrors spawn in a central, well-defended area of caves used as lairs.

\

\textbf{Warhorse}\index[Monsters]{Warhorse}

\textit{Large beast, unaligned}

\textbf{STRENGTH} +4

\textbf{DEXTERITY} +1

\textbf{CONSTITUTION} +1

\textbf{INTELLIGENCE} -4

\textbf{WISDOM} +1

\textbf{CHARISMA} -2

\textbf{Initiative} +1 -- \textbf{Defence} 12 (plus barding possible)

\textbf{Hit Points} 19 (3d10 + 3)

\textbf{Move} 18m

\textbf{Saving Throws}: Fortitude +4, Reflexes +2, Will +1

\textbf{Languages} -

\textbf{Challenge} 1/2 (100 XP)

\textit{\textbf{Rrumbling Charge.}} If the horse moves at least 6 meters directly at the target and hits it with a hoof attack during the same turn, the target must succeed at a DC 14 Fortitude save or fall prone. If the target is prone, the horse can make another hoof attack against it as a bonus action.

\textbf{Actions}

\textit{\textbf{Hooves.} Melee Weapon Attack}: +6 to hit, reach 1m, one target.

\textit{Hit:} 11 (2d6 + 4) bludgeoning damage.

\

\index[Monsters]{Warhorse Skeleton}\textbf{Warhorse Skeleton}

\textit{Large Undead, Lawful Evil}

\textbf{STRENGTH} +4

\textbf{DEXTERITY} +1

\textbf{CONSTITUTION} +2

\textbf{INTELLIGENCE} -4

\textbf{WISDOM} -1

\textbf{CHARISMA} -3

\textbf{Initiative} +1 -- \textbf{Defence} 14 (pieces of barding)

\textbf{Hit Points} 22 (3d10 + 6)

\textbf{Move} 18m

\textbf{Saving Throws}: Fortitude +4, Reflexes +3, Will +1

\textbf{Vulnerability to Damage} from bludgeoning

\textbf{Damage Resistances} piercing and slashing

\textbf{Immunity to Damage} Poison

\textbf{Condition Immunity} poisoned, fatigue, bleeding

\textbf{Senses} Darkvision 18m

\textbf{Languages} -

\textbf{Challenge} 1/2 (100 XP)

\textit{\textbf{Undead nature.}} The skeleton requires no air, food, drink, or sleep.

\textbf{Actions}

\textit{\textbf{Hooves.} Melee Weapon Attack}: +6 to hit, reach 1m, one target.

\textit{Hit:} 11 (2d6 + 4) bludgeoning damage.

\

\index[Monsters]{Water Elemental}\textbf{Water Elemental}

\textit{Large elemental, neutral}

\textbf{STRENGTH} +4

\textbf{DEXTERITY} +2

\textbf{CONSTITUTION} +4

\textbf{INTELLIGENCE} -3

\textbf{WISDOM} +0

\textbf{CHARISMA} -1

\textbf{Initiative} +4 -- \textbf{Defence} 17

\textbf{Hit Points} 114 (12d10 + 48)

\textbf{Movement} 9m, swim 27m

\textbf{Saving Throws} Fortitude +9, Reflexes +8, Will +2

\textbf{Damage Resistances} acid; from a non-magical weapon

\textbf{Immunity to Damage} Poison

\textbf{Condition Immunity} grabbed, poisoned, restrained, paralyzed, petrified, unconscious, prone, fatigue

\textbf{Senses} Darkvision 18m

\textbf{Languages} Aquan

\textbf{Challenge} 5 (1800 XP)

\textit{\textbf{Freezing.}} If the elemental takes cold damage, it freezes partially; his movement is reduced by 6 meters until the end of his next round.

\textit{\textbf{Water form.}} The elemental can enter a hostile creature's space and stop there. He can move through a space as narrow as 3 centimeters without having to squeeze.

\textit{\textbf{Elemental Nature.}} An elemental doesn't need air,
food, drink or sleep.

\textbf{Actions}

\textit{\textbf{Multiattack.}} The elemental makes two slam attacks.

\textit{\textbf{Slam.} Melee Weapon Attack}: +11 to hit, reach 1m, one target.

\textit{Hit:} 13 (2d8 + 4) bludgeoning damage.

\textit{\textbf{Submerge (Cooldown 4-6).}} Each creature in the elemental's space must make a DC 15 Fortitude save. On a failed save, the target takes 13 (2d8 + 4) damage from hit. If Large or smaller, the target is also grappled (DC 14 to escape). Until the grab ends, the target is restrained and can't breathe unless able to breathe water. On a successful save, the target is pushed out of the space
of the elemental.

The elemental can grab hold of one Large or up to two Medium or smaller creatures at a time. At the start of each elemental's turn, each grabbed target takes 13 (2d8 + 4) bludgeoning damage. A creature within 1 meter of the elemental can pull a creature or object out of it, taking an action to attempt a DC 14 Strength check.

\textbf{Ecology}\\
Environment any (Plane of Water)\\
Organization: Solitary, pair, or group (3-8)\\
\textbf{Treasure}: None\\
\textbf{Description}\\
Water elementals are patient, unyielding creatures composed of living, fresh or salt water. They prefer to cover their opponents with water or drag them into it to gain an advantage.\\
Like other elementals, all water elementals have unique shapes and forms. Many are wave-like creatures with vaguely humanoid faces and smaller waves on the sides that act as arms. Another common form is that of some aquatic creature, such as a shark or octopus, but made entirely of water.\\
A large water elemental stands 5 meters tall and weighs 2,250kg.


\

\index[Monsters]{Werecreatures, Werebear}\textbf{Werebear}

\textit{Medium humanoid (human, shapeshifter), neutral good}

\textbf{STRENGTH} +4

\textbf{DEXTERITY} +0

\textbf{CONSTITUTION} +3

\textbf{INTELLIGENCE} +0

\textbf{WISDOM} +1

\textbf{CHARISMA} +1

\textbf{Initiative} +0 -- \textbf{Defence} 13 in humanoid form, 14

in bear or hybrid form

\textbf{Hit Points} 135 (18d8 + 54)

\textbf{Move} 9m (12m, climb 9m in bear form or hybrid form)

\textbf{Saving Throws}: Fortitude +5, Reflexes +6, Will +2

\textbf{Skills} Awareness +7

\textbf{Immunity to Damage} from non-magical or non-silver weapon

\textbf{Languages} Common (can't speak in bear form)

\textbf{Challenge} 5 (1800 XP)

\textit{\textbf{Shapeshift.}} The werebear can use its action to transform into a bear-humanoid hybrid or bear, or back to its true form, which is humanoid. Its stats, aside from Defence, are the same in all forms. Any equipment he is wearing or carrying is not transformed. Upon death it returns to its true form.

\textit{\textbf{Enhanced sense of smell.}} The werebear has +1d6 on Wisdom (Awareness) checks based on smell.

\textbf{Actions}

\textit{\textbf{Multiattack.}} In bear form, the werebear makes two claw attacks. In humanoid form, it makes two greataxe attacks. In hybrid form, it can attack as a bear or humanoid.

\textit{\textbf{Claw (In Bear or Hybrid Form only).} Melee Weapon Attack}: +11 to hit, reach 1m, one target. \textit{Hit:} 13 (2d8 + 2) slashing damage.

\textit{\textbf{Great Ax (In Humanoid or Hybrid Form only).} Melee Weapon Attack}: +11 to hit, reach 1m, one target. \textit{Hit:} 10 (1d12 + 4) slashing damage.

\textit{\textbf{Bite (In Bear or Hybrid Form only).} Melee Weapon Attack}: +11 to hit, reach 1m, one target.

\textit{Hit:} 15 (2d10 + 4) piercing damage. If the target is a humanoid, it must succeed at a DC 14 Fortitude save or be cursed with werebear lycanthropy.


\textbf{Ecology}\\
Environment any forest\\
Organization: Solitary, pair, family (3-6), or troop (3-6 plus 1-4 Black or Gray bears)\\
\textbf{Treasure}: NPC gear (Mailjack, Masterwork Battleaxe, 2 Masterwork Throwing Axes, other treasure)\\
\textbf{Description}\\
In their humanoid forms, werebears tend to be muscular and broad-shouldered, with rugged features, and dark eyes. They have red, brown or black hair and seem accustomed to a life of hard work. Though the most benign of the werewolves, they are shunned by most normal people, who fear their bestial transformation. Most live in isolated wooded areas or in small family units of their own species. They shy away from dealing with outsiders, but they don't hesitate if they need to drive evil humanoids from their territories.

\

\index[Monsters]{Werecreatures, Wereboar}\textbf{Wereboar}

\textit{Medium humanoid (human, shapeshifter), neutral evil}

\textbf{STRENGTH} +3

\textbf{DEXTERITY} +0

\textbf{CONSTITUTION} +2

\textbf{INTELLIGENCE} +0

\textbf{WISDOM} +0

\textbf{CHARISMA} -1

\textbf{Initiative} +0 -- \textbf{Defence} 12 in humanoid form, 13 in boar or hybrid form

\textbf{Hit Points} 78 (12d8 + 24)

\textbf{Move} 9m (12m in boar form)

\textbf{Saving Throws}: Fortitude +7, Reflexes +1, Will +4

\textbf{Skills} Awareness +2

\textbf{Immunity to Damage} from non-magical or non-silver weapon

\textbf{Languages} Common (cannot speak in boar form)

\textbf{Challenge} 4 (1100 XP)

\textit{\textbf{Charge (Boar or Hybrid Form Only).}} If the wereboar moves in a straight line at least 5 meters towards a target and then fangs them during the same turn, the target takes an additional 7 (2d6) slashing damage. If the target is a creature, it must succeed on a DC 13 Fortitude save or be knocked prone.

\textit{\textbf{Relentless (Recharges after 1 hour).}} If the wereboar takes 14 damage or less that would reduce it to 0 Hit Points, it drops to 1 hit point instead.

\textit{\textbf{Shapeshift.}} The wereboar can use its action to transform into a boar-humanoid hybrid or a boar, or back to its true form, which is humanoid. Its stats, aside from Defence, are the same in all forms. Any equipment he is wearing or carrying is not transformed. Upon death it returns to its true form.

\textbf{Actions}

\textit{\textbf{Multiattack (In Humanoid or Hybrid Form only).}} The wereboar makes two attacks, only one of which can be with fangs.

\textit{\textbf{Maul (In Humanoid or Hybrid Form only).} Melee Weapon Attack}: +9 to hit, reach 1m, one target. \textit{Hit:} 10 (2d6 + 3) bludgeoning damage.

\textit{\textbf{Tusks (Boar or Hybrid Form Only).} Melee Weapon Attack}: +9 to hit, reach 1m, one target. \textit{Hit:} 10 (2d6 + 3) slashing damage. If the target is a humanoid, it must succeed at a DC 12 Fortitude save or be cursed with wereboar lycanthropy.

\textbf{Ecology}\\
Environment any Forest or Plains\\
Organization: Solitary, Pair, Family (3-8), or Troop (3-8 plus 1-4 Boars)\\
\textbf{Treasure}: NPC gear (Studded Leather Armour, 2 Daggers, other treasure)\\
\textbf{Description}\\
In their humanoid form, wereboars tend to be stocky, with upturned noses, shaggy fur, and prominent front teeth. They have red, brown or black hair but some are also blond, white-haired or bald. They usually have fuzz on their upper lip, and males usually cannot grow a beard. Because they are stubborn and aggressive, they have small communities of their own kind and do not mix with non-werewolves: they usually live on small, perfectly ordinary-looking farms. They tend to have large families and many children.


\

\index[Monsters]{Werecreatures, Wererat}\textbf{Wererat}

\textit{Medium humanoid (human, shapeshifter), lawful evil}

\textbf{STRENGTH} +0

\textbf{DEXTERITY} +2

\textbf{CONSTITUTION} +1

\textbf{INTELLIGENCE} +0

\textbf{WISDOM} +0

\textbf{CHARISMA} -1

\textbf{Initiative} +2 -- \textbf{Defence} 13

\textbf{Hit Points} 33 (6d8 + 6)

\textbf{Move} 9m

\textbf{Saving Throws}: Fortitude +2, Reflexes +5, Will +3

\textbf{Skills} Stealth +4, Awareness +2

\textbf{Immunity to Damage} from non-magical or non-silver weapon

\textbf{Senses} Darkvision 18m (in rat form only)

\textbf{Languages} Common (cannot speak in rat form)

\textbf{Challenge} 2 (450 XP)

\textit{\textbf{Shapeshift.}} The wererat can use its action to transform into a rat-humanoid hybrid or rat, or back to its true form, which is humanoid. Its stats, aside from Defence, are the same in all forms. Any equipment he is wearing or carrying is not transformed. Upon death it returns to its true form.

\textit{\textbf{A keen sense of smell.}} The wererat has +1d6 on Wisdom (Awareness) checks based on smell.

\textbf{Actions}

\textit{\textbf{Multiattack (In Humanoid or Hybrid Form only).}} The wererat makes two attacks, only one of which can be with a bite.

\textit{\textbf{Short Sword (In Humanoid or Hybrid Form only).} Melee Weapon Attack}: +4 to hit, reach 1m, one target. \textit{Hit:} 5 (1d6 + 2) piercing damage.

\textit{\textbf{Hand crossbow (In Humanoid or Hybrid Form only).} Ranged weapon attack}: +4 to hit, range 9m, one target.

\textit{Hit:} 5 (1d6 + 2) piercing damage.

\textit{\textbf{Bite (In Rat or Hybrid Form only).} Melee Weapon Attack}: +4 to hit, reach 1m, one target.

\textit{Hit:} 4 (1d4 + 2) piercing damage. If the target is a humanoid, it must succeed at a DC 11 Fortitude save or be cursed with the wererat's lycanthropy.

\textbf{Ecology}\\
Environment any Urban\\
Organization: Solitary, pair, pack (5-10), or guild (11-30 plus 5-12 dire rats)\\
\textbf{Treasure}: NPC gear (masterwork studded leather Armour, shortsword, light crossbow with 20 bolts, other treasure)\\
\textbf{Description}\\
Natural wererats are short, wiry, and muscular, with sharp, alert eyes, and nervous movements. Males often have thin shriveled whiskers.

\

\index[Monsters]{Werecreatures, Weretiger}\textbf{Weretiger}

\textit{Medium humanoid (human, shapeshifter), neutral}

\textbf{STRENGTH} +3

\textbf{DEXTERITY} +2

\textbf{CONSTITUTION} +3

\textbf{INTELLIGENCE} +0

\textbf{WISDOM} +1

\textbf{CHARISMA} +0

\textbf{Initiative} +2 -- \textbf{Defence} 14

\textbf{Hit Points} 120 (16d8 + 48)

\textbf{Movement} 9m (12m in tiger form)

\textbf{Saving Throws}: Fortitude +2, Reflexes +7, Will +4

\textbf{Skills} Stealth +4, Awareness +5

\textbf{Immunity to Damage} from non-magical weapons that aren't silver

\textbf{Senses} Darkvision 18m

\textbf{Languages} Common (cannot speak in tiger form)

\textbf{Challenge} 4 (1.1100 XP)

\textit{\textbf{Leap.}} If the weretiger moves at least 5 meters in a straight line towards a creature and hits it with a claw attack during the same turn, the target must succeed at a Fortitude save DC 14 or fall prone. If the target is prone, the weretiger can make a bite attack against it as a bonus action.

\textit{\textbf{Shapeshift.}} The weretiger can use its action to transform into a tiger-humanoid hybrid or tiger, or back to its true form, which is humanoid. Its stats, aside from Defence, are the same in all forms. Any equipment he is wearing or carrying is not transformed. Upon death it returns to its true form.

\textit{\textbf{Enhanced smell and hearing.}} The weretiger has +1d6 on Wisdom (Awareness) checks based on smell and hearing.

\textbf{Actions}

\textit{\textbf{Multiattack (in Humanoid or Hybrid Form only).}} In humanoid form, the weretiger makes two scimitar attacks or two longbow attacks. In hybrid form, it can attack as a humanoid or make two claw attacks.

\textit{\textbf{Claw (In Tiger or Hybrid Form only).} Melee Weapon Attack}: +9 to hit, reach 1m, one target. \textit{Hit:} 7 (1d8 + 3) slashing damage, 1 bleed damage.

\textit{\textbf{Bite (In Tiger or Hybrid Form only).} Melee Weapon Attack}: +9 to hit, reach 1m, one target.

\textit{Hit:} 8 (1d10 + 3) piercing damage. If the target is a humanoid, it must succeed at a DC 13 Fortitude save or be cursed with the weretiger's lycanthropy.

\textit{\textbf{Scimitar (In Humanoid or Hybrid Form only).} Melee Weapon Attack}: +9 to hit, reach 1m, one target. \textit{Hit:} 6 (1d6 + 3) slashing damage.

\textit{\textbf{Longbow (In Humanoid or Hybrid Form only).} Ranged weapon attack}: +8 to hit, range 45m, one target.

\textit{Hit:} 6 (1d8 + 2) piercing damage.

\textbf{Ecology}
Environment any plains or swamps\\
Organization: Solitary or pair\\
\textbf{Treasure}: NPC gear (Studded Leather Armour, Shortsword, 2 Daggers, other treasure)\\
\textbf{Description}\\
Weretigers in humanoid form have large eyes, elongated noses, prominent cheekbones, and brown or red, white, black, or blue-gray hair. Their movements are careful and graceful, and the viewer might mistake them for a fine cutpurse, a graceful dancer or a skilled courtesan.


\

\index[Monsters]{Werecreatures, Werewolf}\textbf{Werewolf}

\textit{Medium humanoid (human, shapeshifter), chaotic evil}

\textbf{STRENGTH} +2

\textbf{DEXTERITY} +1

\textbf{CONSTITUTION} +2

\textbf{INTELLIGENCE} +0

\textbf{WISDOM} +0

\textbf{CHARISMA} +0

\textbf{Initiative} +1 -- \textbf{Defence} 13 in humanoid form, 14 in wolf or hybrid form

\textbf{Hit Points} 58 (9d8 + 18)

\textbf{Movement} 9m (12m in wolf form)

\textbf{Saving Throws}: Fortitude +5, Reflexes +1, Will +2

\textbf{Skills} Stealth +3, Awareness +4

\textbf{Immunity to Damage} from non-magical or non-silver weapon

\textbf{Languages} Common (can't speak in wolf form)

\textbf{Challenge} 3 (700 XP)

\textit{\textbf{Shapeshift.}} The werewolf can use his action to transform into a wolf-humanoid hybrid or a wolf, or back to his true form, which is humanoid. Its stats, aside from Defence, are the same in all forms. Any equipment he is wearing or carrying is not transformed. Upon death it returns to its true form.

\textit{\textbf{Hearing and keen sense of smell.}} The werewolf has +1d6 on Wisdom (Awareness) checks based on hearing or smell.

\textbf{Actions}

\textit{\textbf{Multiattack (In Humanoid or Hybrid Form only).}} The werewolf makes two attacks: one with its bite and one with its claws or spear.

\textit{\textbf{Claws (In Hybrid Form only).} Melee Weapon Attack}: +6 to hit, reach 1m, one creature. \textit{Hit:} 7 (2d4 + 2) slashing damage.

\textit{\textbf{Spear (in humanoid form only).} Melee or Ranged weapon attack}: +4 to hit, reach 10 or range 6, one creature.

\textit{Hit:} 5 (1d6 + 2) piercing damage or 6 (1d8 + 2) piercing damage when used with two hands in a melee attack.

\textit{\textbf{Bite (In Wolf or Hybrid Form only).} Melee Weapon Attack}: +6 to hit, reach 1m, one target.

\textit{Hit:} 6 (1d8 + 2) piercing damage. If the target is a humanoid, it must succeed at a DC 12 Fortitude save or be cursed with werewolf lycanthropy.

\textbf{Ecology}\\
Environment any terrain\\
Organization: Solitary, pair, or pack (3-6)\\
\textbf{Treasure}: NPC gear (Chainmail, Longsword, Light Crossbow with 20 bolts, other treasure)\\
\textbf{Description}\\
In human form, werewolves resemble normal people, though some tend to have a feral appearance and unruly hair. Eyebrows growing together, an index finger longer than the middle finger, and strange birthmarks on the palm of the hand are all commonly accepted signs that a person is actually a werewolf. Of course, these telltale signs aren't always accurate, because these physical traits exist in regular people as well, but in areas where werewolves are a common problem, these traits can be considered overwhelming regardless.

\

\index[Monsters]{Wight}\textbf{Wight}

\textit{Medium Undead, Neutral Evil}

\textbf{STRENGTH} +2

\textbf{DEXTERITY} +2

\textbf{CONSTITUTION} +3

\textbf{INTELLIGENCE} +0

\textbf{WISDOM} +1

\textbf{CHARISMA} +2

\textbf{Initiative} +2 -- \textbf{Defence} 16 (studded Armour)

\textbf{Hit Points} 45 (6d8 + 18)

\textbf{Move} 9m

\textbf{Saving Throws}: Fortitude +3, Reflexes +2, Will +5

\textbf{Skills} Stealth +4, Awareness +3

\textbf{Damage Resistances} from Void; from a non-magical or non-silver weapon

\textbf{Immunity to Damage} Poison

\textbf{Condition Immunity} poisoned, fatigue, bleeding

\textbf{Senses} Darkvision 18m

\textbf{Languages} the languages knew in life, Exspiram

\textbf{Challenge} 3 (700 XP)

\textit{\textbf{Undead nature.}} The wight does not need air, food, drink, or sleep.

\textit{\textbf{Sensitivity to Light}}. While in sunlight, the wight has -1d6 on attack rolls, as well as Wisdom (Awareness) checks based on sight.

\textbf{Actions}

\textit{\textbf{Multiattack.}} The wight can make two longsword attacks or two longbow attacks. She can use Drain Life in place of one of his longsword attacks.

\textit{\textbf{Drain Life.} Melee Weapon Attack}: +5 to hit, reach 3 ft., one creature.

\textit{Hit:} 5 (1d6 + 2) void damage. The target must succeed on a DC 13 Fortitude save or have its maximum Hit Points reduced by an amount equal to the damage taken. The target becomes fatigued. The target dies if the effect reduces its maximum Hit Points to 0.

A humanoid killed by this attack reanimates 24 hours later as a zombie under the wight's control, unless the humanoid is first restored to life or the body is destroyed. The wight can control no more than twelve zombies at a time.

\textit{\textbf{Longsword.} Melee Weapon Attack}: +5 to hit, reach 1m, one target.

\textit{Hit:} 6 (1d8 + 2) slashing damage, or 7 (1d10 + 2) slashing damage when used with two hands.

\textit{\textbf{Longbow.} Ranged weapon attack}: +5 to hit, range 45m, one target.

\textit{Hit:} 6 (1d8 + 2) piercing damage.

\textbf{Ecology}\\
Environment: any\\
Organization: Solitary, pair, group (3-6), or herd (7-12)\\
\textbf{Treasure}: Standard\\
\textbf{Description}\\
Wights are humanoids resurrected as undead as a result of necromancy, a violent death, or an extremely malevolent personality. In some cases, a wight arises when an undead spirit permanently bonds with a corpse, often that of a warrior. They are barely recognizable to those who knew them in life: their flesh is corrupted by evil and undeath, their eyes burn with hatred and their teeth become those of a beast. In a sense, a wight is the link between ghouls and wraiths: a misshapen corpse that sucks life energy with touch.

Being undead, wights do not need to breathe, so they can sometimes be found underwater, though they are not particularly good swimmers unless they originated from swimming creatures such as water elves and merfolk. Underwater, wights prefer low-ceilinged caves where their poor swimming skills are not a limitation.

\

\textbf{Winter Wolf}\index[Monsters]{Winter Wolf}

Winter wolves inhabit the Arctic regions and are evil, intelligent creatures with snow-white coats and ice-colored eyes.

\textit{Large Monstrosity, Neutral Evil}

\textbf{STRENGTH} +4

\textbf{DEXTERITY} +1

\textbf{CONSTITUTION} +2

\textbf{INTELLIGENCE} -2

\textbf{WISDOM} +1

\textbf{CHARISMA} -1

\textbf{Initiative} +1 -- \textbf{Defence} 15

\textbf{Hit Points} 75 (10d10 + 20)

\textbf{Move} 15m

\textbf{Saving Throws}: Fortitude +9, Reflexes +6, Will +3

\textbf{Skills} Stealth +3, Awareness +5

\textbf{Damage Immunity} cold

\textbf{Languages} Common, Giant, Winter Wolf

\textbf{Challenge} 3 (700 XP)

\textit{\textbf{Snow camouflage.}} The wolf has +1d6 on Dexterity (Hide) checks made to hide in snowy terrain.

\textit{\textbf{Hearing and Fine Smell.}} The wolf has +1d6 on Wisdom (Awareness) checks based on hearing or smell.

\textit{\textbf{Pack tactics.}} The wolf has +1d6 on attack rolls against a creature if at least one of the wolf's allies is within 1 meter of the creature and that ally isn't incapacitated.

\textbf{Actions}

\textit{\textbf{Bite.} Melee Weapon Attack}: +6 to hit, reach 1m, one target.

\textit{Hit:} 11 (2d6 + 4) piercing damage. If the target is a creature, it must succeed on a DC 14 Fortitude save or fall prone.

\textit{\textbf{Icy Breath (Cooldown 5-6).}} The wolf exhales a blast of icy wind in a 5-meter cone. Each creature in that area must make a DC 12 Reflex save, taking 18 (4d8) cold damage on a failed save, or half as much damage on a successful one.

\

\index[Monsters]{Wise Owlbear}\textbf{Wise Owlbear}

\textit{Large monstrosity, neutral}

\textbf{STRENGTH} +3

\textbf{DEXTERITY} +1

\textbf{CONSTITUTION} +2

\textbf{INTELLIGENCE} +3

\textbf{WISDOM} +3

\textbf{CHARISMA} +1

\textbf{Initiative} +3 -- \textbf{Defence} 15

\textbf{Hit Points} 45 (7d10 + 10)

\textbf{Move} 12m

\textbf{Saving Throws}: Fortitude +10, Reflexes +5, Will +4

\textbf{Skills} Awareness +9

\textbf{Senses} Darkvision 18m

\textbf{Languages} understands and reads the following languages: Common, Celestial, Infernal, Dwarven, Elven, Orc, Giant, Exspiram, Elemental languages

\textbf{Challenge} 3 (700 XP)

\textit{\textbf{Honed sense of smell and vision.}} The wise Owlbear has +1d6 on Wisdom (Awareness) checks based on smell or sight.

\textit{\textbf{Innate Spells.}} The wise Owlbear's spellcasting ability is Intelligence. The wise Owlbear can innately cast the following spells, requiring no material components:

At will: \textit{Magic Hand}

\textbf{Actions}

\textit{\textbf{Multiattack.}} The wise Owlbear makes two attacks: one with its beak and one with its claws.

\textit{\textbf{Claws.} Melee Weapon Attack}: +7 to hit, reach 1m, one target.

\textit{Hit:} 14 (2d8 + 5) slashing damage.

\textit{\textbf{Beak.} Melee Weapon Attack}: +7 to hit, reach 3 ft., one creature.

\textit{Hit:} 10 (1d10 + 5) piercing damage.

\textbf{Ecology}\\
\textbf{Environment: Temperate Forests}
Organization: Solitary, Pair, or Pack (3-8)\\
\textbf{Treasure}: Standard + 10\% Textbooks and Tomes\\
\textbf{Description}\\
The wise Owlbear's origins are as mysterious as those of its unwise kin, but aficionados of these creatures claim it directly descended from Nethergal as a variant of the original Owlbear.
Usually the wise Owlbear loves to surround himself with books and adores the company of other wise men but does not disdain the tales of adventurers and the enthralling ballads of Arbiters. The wise Owlbear has a real talent for languages and although he cannot speak in a way understandable to a man he can understand many spoken and written languages and in a few days he is able to learn new ones (such as Universal Language Advantage) both spoken and written. Wise Owlbear is able to read any language or code if he has the opportunity to study it for 3 days.
Usually weaker and more fragile than the close relative, they are however fearsome beings in combat.
Preferably a wise Owlbear does not attack except in Defence and seeks as tactical and useful an approach as possible. A characteristic trait of wise Owlbears is a scarf worn around the absent neck. Killing a wise Owlbear is an affront to the Devotees and Followers of Nethergal, it has also happened that the Patron himself takes away the ability to communicate to those who have been guilty of cruelty with his favorite creatures.

Training a wise Owlbear is much easier than an Owlbear but the creature's high intelligence will prompt it to have an equal or familiar relationship instead.

The Mage Hand spell is usually used to leaf through the more delicate tomes and to write, even if extremely slowly.

\

\index[Monsters]{Wisp}\textbf{Wisp}

\textit{Tiny undead, chaotic evil}

\textbf{STRENGTH} -5

\textbf{DEXTERITY} +9

\textbf{CONSTITUTION} +0

\textbf{INTELLIGENCE} +1

\textbf{WISDOM} +2

\textbf{CHARISMA} +0

\textbf{Initiative} +9 -- \textbf{Defence} 20

\textbf{Hit Points} 22 (9d4)

\textbf{Move} 0m, fly 15m (float)

\textbf{Saving Throws}: Fortitude +3, Reflexes +12, Will +9

\textbf{Immunity to Damage} Electricity, Poison

\textbf{Damage Resistances} acid, cold, fire, void, sound; non-magical weapons

\textbf{Condition Immunity} grabbed, poisoned, restrained, paralyzed, unconscious, prone, fatigue, bleeding

\textbf{Senses} darkvision 40m

\textbf{Languages} the languages he knew in life

\textbf{Challenge} 2 (450 XP)

\textit{\textbf{Consume Life.}} As a bonus action, the will-o'-the-wisp can target a creature it can see within 1 meter of it that has 0 Hit Points and is still alive. The target must succeed on a DC 10 Fortitude save against this spell or die. If the target dies, the will-o'-the-wisp regains 10 (3d6) Hit Points.

\textit{\textbf{Ephemeral.}} The will-o'-the-wisp can't wear or carry anything.

\textit{\textbf{Variable Illumination.}} The will-o'-the-wisp sheds bright light in a radius of 1 to 6 meters and dim light for an additional number of meters equal to the chosen radius. The will-o'-the-wisp can modify this ray as a bonus action.

\textit{\textbf{Incorporeal Movement.}} The will-o'-the-wisp can move through other creatures and objects as if they were hindering terrain. It takes 5 (1d10) force damage if it ends its round inside an object.

\textit{\textbf{Undead nature.}} The will-o'-the-wisp needs no air, food, or drink.

\textbf{Actions}

\textit{\textbf{Shake.} Melee spell attack}: +9 to hit, reach 1 meter, one creature.

\textit{Hit:} 9 (2d8) electricity damage.

\textit{\textbf{Invisibility.}} The will-o'-the-wisp and its light magically become invisible until it attacks or uses Spend Life, or until its concentration ends (as if concentrating on a spell).

\textbf{Ecology}
Environment any swamp\\
Organization: Solitary, pair or sequence (3-4)\\
\textbf{Treasure}: Accidental\\
\textbf{Description}\\
Every hunter and farmer who lives near a marsh or marsh has named these orbs of dim light: jack lanterns, candles of the dead, walking fires, pine lights, ghost lights, rush lights; but everyone knows that they are dangerous predators and false guides in the dark.

Wicked creatures that feed on the strong psychic emanations of terrified creatures, will-o'-the-wisp delight in putting gullible travelers into dangerous situations. In the wilderness, where they are very common, will-o'-the-wisp prefer simple tactics such as positioning themselves on rocks or quicksand where they can easily be mistaken for lanterns (especially if they can set their trap near real signal lanterns), so as to attract travelers towards danger. On rare occasions, will-o'-the-wisps in search of an easy life move into a city and settle near the gallows or follow an army unseen, so as to feed on the fear of dying men; why the vast majority choose to stay in the swamps, where victims are scarce, remains a mystery.

Wisps rely only on their electric shock in dangerous situations, so they prefer to let other creatures or dangers deal with their victims while they float by and feast.

Wisps can glow any color they desire, but are most often yellow, white, green, or blue. They can also vary their brightness to create a pattern: many will-o'-the-wisps like to create shapes that vaguely resemble skulls in their luminescence to heighten terror in their victims. Their true bodies are barely visible globes of translucent spongy material about 30 centimeters in size that weigh 1.5 kg and can emit light across their entire surface. The light from the will-o'-the-wisp shines roughly like a torch, and while they don't appear to use sounds to communicate, they hear perfectly and can vibrate their bodies so rapidly that they mimic speech.

Despite being vilified by most sentient creatures, will-o'-the-wisp are actually quite intelligent, though they think in completely alien ways. Sometimes they organize themselves into groups called "sequences"; their society and purposes remain completely unknown, as do their origins, although they are sometimes known to make deals with those who offer them large numbers of suitably terrified victims.

Will-o'-the-wisps are ageless and virtually immortal, unless they die a violent death; older will-o'-the-wisps can be excellent repositories of knowledge of the past, though convincing one of these cruel creatures to cooperate can be quite tricky.


\

\textbf{Wolf}\index[Monsters]{Wolf}

\textit{Medium beast, unaligned}

\textbf{STRENGTH} +1

\textbf{DEXTERITY} +2

\textbf{CONSTITUTION} +1

\textbf{INTELLIGENCE} -4

\textbf{WISDOM} +1

\textbf{CHARISMA} -2

\textbf{Initiative} +2 -- \textbf{Defence} 14

\textbf{Hit Points} 11 (2d8 + 2)

\textbf{Move} 12m

\textbf{Saving Throws}: Fortitude +5, Reflexes +5, Will +1

\textbf{Skills} Stealth +4, Awareness +3

\textbf{Languages} -

\textbf{Challenge} 1/4 (50 XP)

\textit{\textbf{Hearing and Fine Smell.}} The wolf has +1d6 on Wisdom (Awareness) checks based on hearing or smell.

\textit{\textbf{Pack tactics.}} The wolf has +1d6 on attack rolls against a creature if at least one of the wolf's allies is within 1 meter of the creature and that ally isn't incapacitated.

\textbf{Actions}

\textit{\textbf{Bite.} Melee Weapon Attack}: +4 to hit, reach 1m, one target.

\textit{Hit:} 7 (2d4 + 2) piercing damage. If the target is a creature, it must succeed on a DC 11 Fortitude save or fall prone.

\

\textbf{Worg}\index[Monsters]{Worg}

Worgs are monstrous wolf-like predators who love to hunt and devour creatures weaker than themselves.

\textit{Large Monstrosity, Neutral Evil}

\textbf{STRENGTH} +3

\textbf{DEXTERITY} +1

\textbf{CONSTITUTION} +1

\textbf{INTELLIGENCE} -2

\textbf{WISDOM} +0

\textbf{CHARISMA} -1

\textbf{Initiative} +1 -- \textbf{Defence} 14

\textbf{Hit Points} 26 (4d10 + 4)

\textbf{Move} 15m

\textbf{Saving Throws}: Fortitude +3, Reflexes +2, Will +2

\textbf{Skills} Awareness +4

\textbf{Senses} vision in the dark 18m

\textbf{Languages} Goblin, Worg

\textbf{Challenge} 1/2 (100 XP)

\textit{\textbf{Hearing and keen sense of smell.}} The worg has +1d6 on Wisdom (Awareness) checks based on hearing or smell.

\textbf{Actions}

\textit{\textbf{Bite.} Melee Weapon Attack}: +5 to hit, reach 1m, one target.

\textit{Hit:} 10 (2d6 + 3) piercing damage. If the target is a creature, it must succeed on a DC 13 Fortitude save or fall prone.

\

\index[Monsters]{Wraith}\textbf{Wraith}

\textit{Medium Undead, Neutral Evil}

\textbf{STRENGTH} -2

\textbf{DEXTERITY} +3

\textbf{CONSTITUTION} +3

\textbf{INTELLIGENCE} +1

\textbf{WISDOM} +2

\textbf{CHARISMA} +2

\textbf{Initiative} +3 -- \textbf{Defence} 16

\textbf{Hit Points} 67 (9d8 + 27)

\textbf{Move} 0m, fly 18m (float)

\textbf{Saving Throws}: Fortitude +6, Reflexes +4, Will +6

\textbf{Damage Resistances} acid, cold, Electricity, fire, sound; from a non-magical or non-silver weapon

\textbf{Damage Immunity} Void, poison

\textbf{Condition Immunity} charmed, grabbed, poisoned, restrained, paralyzed, petrified, prone, fatigue, bleeding

\textbf{Senses} Darkvision 18m

\textbf{Languages} the languages he knew in life, Exspiram

\textbf{Challenge} 5 (1800 XP)

\textit{\textbf{Incorporeal movement.}} The wraith can pass through creatures and objects as if they were hindering terrain. He takes 5 (1d10) force damage if he ends his round inside an object.

\textit{\textbf{Undead nature.}} The wraith needs no air, food, drink, or sleep.

\textit{\textbf{Sensitivity to Light}}. While in sunlight, the wraith has -1d6 on attack rolls, as well as Wisdom (Awareness) checks based on sight.

\textbf{Actions}

\textit{\textbf{Drain Life.} Melee Weapon Attack}: +7 to hit, reach 3 ft., one creature.

\textit{Hit:} 21 (4d8 + 3) void damage. The target must succeed on a DC 15 Fortitude save or have its maximum Hit Points reduced by an amount equal to the damage taken. The target becomes fatigued. The target dies if the effect reduces its maximum Hit Points to 0.

\textit{\textbf{Create Wraith.}} The wraith targets one humanoid within 3 meter of it that has been dead for up to 1 minute and of violent causes. The target's spirit animates as a wraith in its corpse space and in the nearest unoccupied space. The wraith is under the wraith's control. The wraith can have no more than seven wraiths under its control at a time.

\textit{\textbf{Enraged}}: the Wraith channels its negative energies into a blast of Void around itself in a 6m radius. All creatures must make a DC 15 Fortitude save or take 3d6 void damage, on a successful save they are Slowed 1/3r.


\textbf{Ecology}\\
Environment: Any\\
Organization: Solitary, pair, group (3-6), or herd (7-12)\\
\textbf{Treasure}: None\\
\textbf{Description}\\
Wraiths are creatures born of evil and darkness. They detest light and living creatures, having lost most of the link to their previous life.


\

\index[Monsters]{Wyvern}\textbf{Wyvern}

\textit{Big dragon, unaligned}

\textbf{STRENGTH} +4

\textbf{DEXTERITY} +0

\textbf{CONSTITUTION} +3

\textbf{INTELLIGENCE} -3

\textbf{WISDOM} +1

\textbf{CHARISMA} -2

\textbf{Initiative} +0 -- \textbf{Defence} 16

\textbf{Hit Points} 110 (13d10 + 39)

\textbf{Move} 6m, fly 24m

\textbf{Saving Throws}: Fortitude +9, Reflexes +6, Will +8

\textbf{Skills} Awareness +4

\textbf{Senses} Darkvision 18m

\textbf{Languages} -

\textbf{Challenge} 6 (2300 XP)

\textbf{Actions}

\textit{\textbf{Multiattack.}} The wyvern can make two attacks: one with its bite and one with its sting. While flying, she can use her claws in place of one of her other attacks.

\textit{\textbf{Claws.} Melee Weapon Attack}: +13 to hit, reach 1m, one target.

\textit{Hit:} 13 (2d8 + 4) slashing damage, 1 bleed damage.

\textit{\textbf{Bite.} Melee Weapon Attack}: +13 to hit, reach 3m, one creature.

\textit{Hit:} 11 (2d6 + 4) piercing damage.

\textit{\textbf{Sting.} Melee Weapon Attack}: +13 to hit, reach 3m, one creature.

\textit{Hit:} 11 (2d6 + 4) piercing damage. The target must make a DC 15 Fortitude save, taking 24 (7d6) poison damage on a failed save, or half as much damage on a successful one.

\textit{\textbf{Enraged}}: the Wyvern points its tail at the enemy and generates a 3-metre cone of poison. You can make a DC 19 Reflex save to halve the 7d8 poison damage.


\textbf{Ecology}\\
Environment: Temperate or warm hills\\
Organization: Solitary, pair, or flock (3-6)\\
\textbf{Treasure}: standard\\
\textbf{Description}\\
Wyverns are brutal and violent reptiles related to dragons. They are always aggressive and impatient and prefer to achieve their goals using force. For this reason, dragons regard wyverns with superiority, regarding their distant relatives as primitive savages lacking style and intelligence.

In most cases, this generalization is spot on. While certainly not animal intellect and capable of speech, most wyverns do not care for diplomacy, preferring to fight first and argue later, only if faced with an opponent they cannot defeat or escape from.

Wyverns are territorial creatures. While they occasionally hunt larger prey in larger groups, they are solitary creatures whose hunting grounds range from 160 to 320 square km. Wyverns are known to often fight each other to the death over disputes over territory rich in prey.

Though constantly hungry and prone to attack, a wyvern can be made friendly through a careful combination of flattery, intimidation, food, and treasure, to make a powerful ally. They often serve Giants and Monstrous Humanoids as guardians, and some tribes of Boggards and Lizardfolk use them as mounts, though such arrangements often prove quite expensive in terms of food and gold, as few wyverns agree to serve creatures for long. similar as mounts.

A wyvern is about 5 meters long, with the tail alone accounting for about half the length. A wyvern weighs an average of 1000 kg.


\

\index[Monsters]{Xorn}\textbf{Xorn}

\textit{Elemental Average, Neutral}

\textbf{STRENGTH} +3

\textbf{DEXTERITY} +0

\textbf{CONSTITUTION} +6

\textbf{INTELLIGENCE} +0

\textbf{WISDOM} +0

\textbf{CHARISMA} +0

\textbf{Initiative} +0 -- \textbf{Defence} 22

\textbf{Hit Points} 73 (7d8 + 42)

\textbf{Movement} 6m, digging 6m

\textbf{Saving Throws}: Fortitude +8, Reflexes +2, Will +5

\textbf{Skills} Stealth +3, Awareness +6

\textbf{Damage Resistances} piercing and slashing non-magical or non-adamantine weapons

\textbf{Senses} Darkvision 18m, telluric sense 18m

\textbf{Languages} Tremun

\textbf{Challenge} 5 (1800 XP)

\textit{\textbf{Stone Camouflage.}} The xorn has +1d6 on Dexterity (Hide) checks made to hide in rocky terrain.

\textit{\textbf{Land Flow.}} The xorn can burrow through nonmagical, unwrought earth and stone. When it does, the xorn doesn't disturb the material it moves.

\textit{\textbf{Treasure sense.}} The xorn can pinpoint, by smell, the location of metals and precious stones, such as coins and gems, within 20 meters of it.

\textbf{Actions}

\textit{\textbf{Multiattack.}} The xorn makes three claw attacks and one bite attack.

\textit{\textbf{Claw.} Melee Weapon Attack}: +9 to hit, reach 1m, one target.

\textit{Hit:} 6 (1d6 + 3) slashing damage, 1 bleed damage.

\textit{\textbf{Bite.} Melee Weapon Attack}: +9 to hit, reach 1m, one target.

\textit{Hit:} 13 (3d6 + 3) piercing damage.

\textit{\textbf{Enraged}}: the Xorn erupts the last eaten gems and nuggets. In a 6m  cone, all creatures must make a DC 18 Reflex save to halve the damage, 3d8 bludgeoning, from thrown gems and ores.


\textbf{Ecology}\\
Environment any (Plane of Earth)\\
Organization: Solitary, pair or group (3-6)\\
\textbf{Treasure}: Standard (precious metals, gems and jewels, and magic gems only)\\
\textbf{Description}
Strange creatures as wide as they are tall, xorns have little interest in natives of the Material Plane, were it not for the gems and precious metals they might carry. Hidden beneath the surface of the ground for what might seem a very long time to a human, a xorn can wait months, even years, for the ideal prey, then attack those who carry their favorite food, such as a particular gem or a specific kind of silver. Adventurers who enter regions inhabited by xorns often bring with them small nuggets of ore or cheap gems and crystals to use as tribute. While its value is usually directly proportional to its flavor and how palatable it can be, most xorns are quite gluttonous, preferring quantity over quality.

The treasure a xorn carries with it or hides in its lair is a snack it has saved for the next day. Offering a particularly delicious (and expensive) jewel or precious metal to a xorn can cement a temporary alliance. Since xorns can traverse rock with ease, they are excellent guides in the subterranean regions.

Xorns are not terribly religious, but those among them who find faith are usually Devotees of Efrem (although it is rare, if not unlikely, for xorns to have Animal Companions, as they cannot follow them into the rock, and instead choose to rule the Earth ). Bards and xorn Devotees are not unknown: Bards usually choose Perform (song), and Devotees invariably have the elemental bloodline (earth).


\

\index[Monsters]{Zombi Ogre}\textbf{Zombi Ogre}

\textit{Large Undead, Neutral Evil}

\textbf{STRENGTH} +4

\textbf{DEXTERITY} -2

\textbf{CONSTITUTION} +4

\textbf{INTELLIGENCE} -4

\textbf{WISDOM} -2

\textbf{CHARISMA} -3

\textbf{Initiative} -2 -- \textbf{Defence} 9

\textbf{Hit Points} 85 (9d10 + 36)

\textbf{Move} 9m

\textbf{Saving Throws}: Fortitude +6, Reflexes +0, Will +3

\textbf{Immunity to Damage} Poison

\textbf{Condition Immunity} poisoned, bleeding

\textbf{Senses} Darkvision 18m

\textbf{Languages} understands Common and Giant but cannot speak

\textbf{Challenge} 2 (450 XP)

\textit{\textbf{Undead nature.}} The zombie does not need air, food, drink, or sleep.

\textit{\textbf{Fortitude undead.}} If the damage reduces the zombie to 0 Hit Points, the zombie must make a DC 5 Fortitude save + the damage taken, unless the damage is from Light or a critical roll. If successful, the zombie drops to 1 hit point instead.

\textbf{Actions}

\textit{\textbf{Shocked Mace.} Melee Weapon Attack}: +6 to hit, reach 1m, one target.

\textit{Hit:} 13 (2d8 + 4) bludgeoning damage.

\

\index[Monsters]{Zombies}\textbf{Zombies}

\textit{Medium Undead, Neutral Evil}

\textbf{STRENGTH} +1

\textbf{DEXTERITY} -2

\textbf{CONSTITUTION} +3

\textbf{INTELLIGENCE} -4

\textbf{WISDOM} -2

\textbf{CHARISMA} -3

\textbf{Initiative} -2 -- \textbf{Defence} 9

\textbf{Hit Points} 22 (3d8 + 9)

\textbf{Move} 6m

\textbf{Saving Throws} Fortitude +0, Reflexes +0, Will +3

\textbf{Immunity to Damage} Poison

\textbf{Condition Immunity} poisoned, bleeding

\textbf{Senses} Darkvision 18m

\textbf{Languages} understands all the languages he spoke in life but cannot speak

\textbf{Challenge} 1/4 (50 XP)

\textit{\textbf{Undead nature.}} The zombie does not need air, food, drink, or sleep.

\textit{\textbf{Fortitude undead.}} If the damage reduces the zombie to 0 Hit Points, the zombie must make a DC 5 Fortitude save + the damage taken, unless the damage is from Light or a critical roll. If successful, the zombie drops to 1 hit point instead.

\textbf{Actions}

\textit{\textbf{Slam.} Melee Weapon Attack}: +3 to hit, reach 1m, one target.

\textit{Hit:} 4 (1d6 + 1) bludgeoning damage.

\textbf{Ecology}\\
Environment: Any\\
Organization: Any\\
\textbf{Treasure}: None\\
\textbf{Description}\\
Zombies are the animated corpses of dead creatures, forced to move by necromantic spells such as animate dead. While the zombies encountered are usually slow and sturdy, others possess different traits, allowing them to spread a disease or move faster.

Zombies are mindless automatons and can do nothing but follow orders. If left to their own devices, they lie still or move around in search of living creatures to slaughter and devour. Zombies attack till destruction, not caring for their safety.

While capable of following orders, zombies are often let loose with orders to kill all living creatures. They are often encountered in packs that infest lands frequented by the living, looking for prey. Most zombies are created through animate dead. Such zombies are always standard, unless the creator also casts Haste or Remove Paralysis to create Swift Zombies or Contagion to create Plague Zombies.



\subsection{Appendix A: Miscellaneous Creatures}

This appendix contains statistics of various animals, parasites and
other creatures. The statistics are organized alphabetically.

\medskip\textbf{Awakened Tree}\index[Monsters]{Awakened Tree}

The awakened tree is a normal tree granted by ability magic
feeling and mobility.

\textit{Huge plant, unaligned}

\textbf{STRENGTH} +4

\textbf{DEXTERITY} -2

\textbf{CONSTITUTION} +2

\textbf{INTELLIGENCE} +0

\textbf{WISDOM} +0

\textbf{CHARISMA} -2

\textbf{Initiative} +0 -- \textbf{Defence} 14

\textbf{Hit Points} 59 (7d12 + 14)

\textbf{Move} 6m

\textbf{Saving Throws}: Fortitude +6, Reflexes -1, Will +1

\textbf{Damage Vulnerability} fire

\textbf{Damage Resistances} bludgeoning, piercing

\textbf{Languages} a language known to its creator

\textbf{Challenge} 2 (450 XP)

\textit{\textbf{False Appearance.}} While the tree remains motionless, it is indistinguishable from a normal tree.

\textbf{Actions}

\textit{\textbf{Slam.} Melee Weapon Attack}: +6 to hit, reach 3m, one target.

\textit{Hit:} 14 (3d6 + 4) bludgeoning damage.

\medskip\textbf{Moose}\index[Monsters]{Moose}

\textit{Large beast, unaligned}

\textbf{STRENGTH} +3

\textbf{DEXTERITY} +0

\textbf{CONSTITUTION} +1

\textbf{INTELLIGENCE} -4

\textbf{WISDOM} +0

\textbf{CHARISMA} -2

\textbf{Initiative} +0 -- \textbf{Defence} 11

\textbf{Hit Points} 13 (2d10 + 2)

\textbf{Move} 15m

\textbf{Saving Throws}: Fortitude +4, Reflexes +1, Will +0

\textbf{Languages} -

\textbf{Challenge} 1/4 (50 XP)

\textit{\textbf{Charge.}} If the moose moves at least 6 meters directly at the target and hits it with a bill attack during the same turn, the target takes an additional 7 (2d6) bludgeoning damage . If the target is a creature, it must succeed on a Fortitude save
DC 13 or fall prone.

\textbf{Actions}

\textit{\textbf{Beak.} Melee Weapon Attack}: +5 to hit, reach 1m, one target.

\textit{Hit:} 6 (1d6 + 3) bludgeoning damage.

\textit{\textbf{Hooves.} Melee Weapon Attack}: +5 to hit, reach 1m, one prone creature.

\textit{Hit:} 8 (2d4 + 3) bludgeoning damage.

\medskip\textbf{Giant Moose}\index[Monsters]{Giant Moose}

\textit{Huge beast, unaligned}

\textbf{STRENGTH} +4

\textbf{DEXTERITY} +3

\textbf{CONSTITUTION} +2

\textbf{INTELLIGENCE} -2

\textbf{WISDOM} +2

\textbf{CHARISMA} +0

\textbf{Initiative} +3 -- \textbf{Defence} 15

\textbf{Hit Points} 42 (5d12 + 10)

\textbf{Move} 18m

\textbf{Saving Throws}: Fortitude +8, Reflexes +7, Will +2

\textbf{Skills} Awareness +4

\textbf{Languages} Giant Elk, includes Common, Elvish, and

Silvano but cannot speak to them

\textbf{Challenge} 2 (450 XP)

\textit{\textbf{Charge.}} If the moose moves at least 6 meters directly at the target and hits it with a bill attack during the same turn, the target takes an additional 7 (2d6) bludgeoning damage . If the target is a creature, it must succeed on a DC 14 Fortitude save or be knocked prone.

\textbf{Actions}

\textit{\textbf{Beak.} Melee Weapon Attack}: +6 to hit, reach 3m, one target.

\textit{Hit:} 11 (2d6 + 4) piercing damage.

\textit{\textbf{Hooves.} Melee Weapon Attack}: +6 to hit, reach 1m, one prone creature.

\textit{Hit:} 22 (4d4 + 4) bludgeoning damage.

\medskip\textbf{Aquila}\index[Monsters]{Aquila}

\textit{Little beast, unaligned}

\textbf{STRENGTH} -2

\textbf{DEXTERITY} +2

\textbf{CONSTITUTION} +0

\textbf{INTELLIGENCE} -4

\textbf{WISDOM} +2

\textbf{CHARISMA} -2

\textbf{Initiative} +2 -- \textbf{Defence} 13

\textbf{Hit Points} 3 (1d6)

\textbf{Move} 3m, fly 18m

\textbf{Saving Throws}: Fortitude +3, Reflexes +4, Will +2

\textbf{Skills} Awareness +4

\textbf{Languages} -

\textbf{Challenge} 0 (10 XP)

\textit{\textbf{Honed Vision.}} The eagle has +1d6 on Wisdom (Awareness) checks based on sight.

\textbf{Actions}

\textit{\textbf{Spurs.} Melee Weapon Attack}: +4 to hit, reach 1m, one target.

\textit{Hit:} 4 (1d4 + 2) slashing damage.

\medskip\textbf{Giant Eagle}\index[Monsters]{Giant Eagle}

The giant eagle is a noble creature that speaks its own language and understands that of other races.

\textit{Large beast, neutral good}

\textbf{STRENGTH} +3

\textbf{DEXTERITY} +3

\textbf{CONSTITUTION} +1

\textbf{INTELLIGENCE} -1

\textbf{WISDOM} +2

\textbf{CHARISMA} +0

\textbf{Initiative} +3 -- \textbf{Defence} 14

\textbf{Hit Points} 26 (4d10 + 4)

\textbf{Move} 3m, fly 24m

\textbf{Saving Throws}: Fortitude +5, Reflexes +7, Will +3

\textbf{Skills} Awareness +4

\textbf{Languages} Giant Eagle, understands Common and Ictum but cannot speak them

\textbf{Challenge} 1 (200 XP)

\textit{\textbf{Honed Vision.}} The eagle has +1d6 on Wisdom (Awareness) checks based on sight.

\textbf{Actions}

\textit{\textbf{Multiattack.}} The eagle makes two attacks: one with its beak and one with its spurs.

\textit{\textbf{Beak.} Melee Weapon Attack}: +5 to hit, reach 1m, one target.

\textit{Hit:} 6 (1d6 + 3) piercing damage.

\textit{\textbf{Spurs.} Melee Weapon Attack}: +5 to hit, reach 1m, one target.

\textit{Hit:} 10 (2d6 + 3) slashing damage.

\medskip\textbf{Vulture}\index[Monstruarium]{Vulture}

\textit{Medium beast, unaligned}

\textbf{STRENGTH} -2

\textbf{DEXTERITY} +0

\textbf{CONSTITUTION} +1

\textbf{INTELLIGENCE} -4

\textbf{WISDOM} +1

\textbf{CHARISMA} -3

\textbf{Initiative} +0 -- \textbf{Defence} 11

\textbf{Hit Points} 5 (1d8 + 1)

\textbf{Move} 3m, fly 15m

\textbf{Saving Throws}: Fortitude +6, Reflexes +3, Will +1; +4 against disease

\textbf{Skills} Awareness +3

\textbf{Languages} -

\textbf{Challenge} 0 (10 XP)

\textit{\textbf{Honed sense of smell and vision.}} The vulture has +1d6 on Wisdom (Awareness) checks based on smell or sight.

\textit{\textbf{Packing tactics.}} The vulture has +1d6 on attack rolls against a creature if at least one of the vulture's allies is within 1 meter of the creature and that ally isn't incapacitated.

\textbf{Actions}

\textit{\textbf{Beak.} Melee Weapon Attack}: +2 to hit, reach 1m, one target.

\textit{Hit:} 2 (1d4) piercing damage.

\medskip\textbf{Giant Vulture}\index[Monsters]{Giant Vulture}

The giant vulture possesses superior intelligence and a mischievous aptitude.

\textit{Large Beast, Neutral Evil}

\textbf{STRENGTH} +2

\textbf{DEXTERITY} +0

\textbf{CONSTITUTION} +2

\textbf{INTELLIGENCE} -2

\textbf{WISDOM} +1

\textbf{CHARISMA} -2

\textbf{Initiative} +0 -- \textbf{Defence} 11

\textbf{Hit Points} 22 (3d10 + 6)

\textbf{Move} 3m, fly 18m

\textbf{Saving Throws}: Fortitude +10, Reflexes +6, Will +3; +4 against disease

\textbf{Skills} Awareness +3

\textbf{Languages} understands Common but cannot speak

\textbf{Challenge} 1 (200 XP)

\textit{\textbf{Honed sense of smell and vision.}} The vulture has +1d6 on Wisdom (Awareness) checks based on smell or sight.

\textit{\textbf{Packing tactics.}} The vulture has +1d6 on attack rolls against a creature if at least one of the vulture's allies is within 1 meter of the creature and that ally isn't incapacitated.

\textbf{Actions}

\textit{\textbf{Multiattack.}} The Vulture makes two attacks: one with its beak and one with its talons.

\textit{\textbf{Beak.} Melee Weapon Attack}: +4 to hit, reach 1m, one target.

\textit{Hit:} 7 (2d4 + 2) piercing damage.

\textit{\textbf{Spurs.} Melee Weapon Attack}: +4 to hit, reach 1m, a target.

\textit{Hit:} 9 (2d6 + 2) slashing damage.

\medskip\textbf{Baboon}\index[Monsters]{Baboon}

\textit{Little beast, unaligned}

\textbf{STRENGTH} -1

\textbf{DEXTERITY} +2

\textbf{CONSTITUTION} +0

\textbf{INTELLIGENCE} -3

\textbf{WISDOM} +1

\textbf{CHARISMA} -2

\textbf{Initiative} +2 -- \textbf{Defence} 13

\textbf{Hit Points} 3 (1d6)

\textbf{Move} 9m, climb 9m

\textbf{Saving Throws}: Fortitude +3, Reflexes +4, Will +1

\textbf{Languages} -

\textbf{Challenge} 0 (10 XP)

\textit{\textbf{Pack tactics.}} The baboon has +1d6 on attack rolls against a creature if at least one of the baboon's allies is within 1 meter of the creature and that ally isn't incapacitated.

\textbf{Actions}

\textit{\textbf{Bite.} Melee Weapon Attack}: +1 to hit, reach 1m, one target.

\textit{Hit:} 1 (1d4 - 1) piercing damage.


\medskip\textbf{Killer Whale (Orca)}\index[Monsters]{Orca}

\textit{Huge beast, unaligned}

\textbf{STRENGTH} +4

\textbf{DEXTERITY} +0

\textbf{CONSTITUTION} +1

\textbf{INTELLIGENCE} -4

\textbf{WISDOM} +1

\textbf{CHARISMA} -2

\textbf{Initiative} +0 -- \textbf{Defence} 14

\textbf{Hit Points} 90 (12d12 + 12)

\textbf{Movement} 0m, swim 18m

\textbf{Saving Throws}: Fortitude +9, Reflexes +8, Will +5

\textbf{Skills} Awareness +3

\textbf{Senses} blind sight 36 m

\textbf{Languages} -

\textbf{Challenge} 3 (700 XP)

\textit{\textbf{Echolocation.}} The whale cannot use blindsight when deafened.

\textit{\textbf{Hold Breath.}} Whale can hold its breath for 30 minutes

\textit{\textbf{Honed Hearing.}} The whale has +1d6 on Wisdom (Awareness) checks based on hearing.

\textbf{Actions}

\textit{\textbf{Bite.} Melee Weapon Attack}: +6 to hit, reach 1m, one target.

\textit{Hit:} 21 (5d6 + 4) piercing damage.

\medskip\textbf{Axe Beak}\index[Monsters]{Axe Beak}

The ax-bill is a large, slender wingless bird with powerful legs, a wedge-shaped beak, and a bad temper.

\textit{Large beast, unaligned}

\textbf{STRENGTH} +2

\textbf{DEXTERITY} +1

\textbf{CONSTITUTION} +1

\textbf{INTELLIGENCE} -4

\textbf{WISDOM} +0

\textbf{CHARISMA} -3

\textbf{Initiative} +1 -- \textbf{Defence} 12

\textbf{Hit Points} 19 (3d10 + 3)

\textbf{Move} 15m

\textbf{Saving Throws}: Fortitude +3, Reflexes +1, Will +1

\textbf{Languages} -

\textbf{Challenge} 1/4 (50 XP)

\textbf{Actions}

\textit{\textbf{Beak.} Melee Weapon Attack}: +4 to hit, reach 1m, one target.

\textit{Hit:} 6 (1d8 + 2) slashing damage.

\medskip\textbf{Camel}\index[Monsters]{Camel}

\textit{Large beast, unaligned}

\textbf{STRENGTH} +3

\textbf{DEXTERITY} -1

\textbf{CONSTITUTION} +2

\textbf{INTELLIGENCE} -4

\textbf{WISDOM} -1

\textbf{CHARISMA} -3

\textbf{Initiative} -1 -- \textbf{Defence} 10

\textbf{Hit Points} 15 (2d10 + 4)

\textbf{Move} 15m

\textbf{Saving Throws}: Fortitude +5, Reflexes +6, Will +0

\textbf{Languages} -

\textbf{Challenge} 1/8 (25 XP)

\textbf{Actions}

\textit{\textbf{Bite.} Melee Weapon Attack}: +5 to hit, reach 1m, one target.

\textit{Hit:} 2 (1d4) bludgeoning damage.

\medskip\textbf{Dog of Death}\index[Monsters]{Dog of Death}

The Dog of Death is a hideous two-headed hound that prowls plains, deserts, and dungeons.

\textit{Medium Monstrosity, Neutral Evil}

\textbf{STRENGTH} +2

\textbf{DEXTERITY} +2

\textbf{CONSTITUTION} +2

\textbf{INTELLIGENCE} -4

\textbf{WISDOM} +1

\textbf{CHARISMA} -2

\textbf{Initiative} +2 -- \textbf{Defence} 13

\textbf{Hit Points} 39 (6d8 + 12)

\textbf{Move} 12m

\textbf{Saving Throws}: Fortitude +4, Reflexes +5, Will +2

\textbf{Skills} Stealth +4, Awareness +5

\textbf{Senses} vision in the dark 36 m

\textbf{Languages} -

\textbf{Challenge} 1 (200 XP)

\textit{\textbf{Two-headed.}} The dog has +1d6 on Wisdom (Awareness) checks and on Saving Throws against blinded, charmed, deafened, frightened, stunned, or unconscious conditions.

\textbf{Actions}

\textit{\textbf{Multiattack.}} The dog makes two bite attacks.

\textit{\textbf{Bite.} Melee Weapon Attack}: +4 to hit, reach 1m, one target.

\textit{Hit:} 5 (1d6 + 2) piercing damage. If the target is a creature, it must succeed on a DC 12 Fortitude save against the disease. After every 24 hours, the creature must repeat the Saving Throw, reducing its maximum Hit Points by 5 (1d10) on a failure. This reduction lasts until the disease is cured. The creature dies if the disease reduces its maximum Hit Points to 0.

\medskip\textbf{Fase Dog}\index[Monsters]{Fase Dog}

The fase dog gets its name from its ability to slip in and out of reality, a talent it uses to attack and avoid being attacked.

\textit{Medium Fey, Lawful Good}

\textbf{STRENGTH} +1

\textbf{DEXTERITY} +3

\textbf{CONSTITUTION} +1

\textbf{INTELLIGENCE} +0

\textbf{WISDOM} +1

\textbf{CHARISMA} +0

\textbf{Initiative} +3 -- \textbf{Defence} 14

\textbf{Hit Points} 22 (4d8 + 4)

\textbf{Damage Vulnerability} cold iron

\textbf{Move} 12m

\textbf{Saving Throws}: Fortitude +5, Reflexes +5, Will +4

\textbf{Skills} Stealth +5, Awareness +3

\textbf{Languages} Fase Dog, understands Sylvan but cannot speak it

\textbf{Challenge} 1/4 (50 XP)

\textit{\textbf{Hearing and Fine Smell.}} The dog has +1d6 on Wisdom (Awareness) checks based on hearing or smell.

\textbf{Actions}

\textit{\textbf{Bite.} Melee Weapon Attack}: +3 to hit, reach 1m, one target.

\textit{Hit:} 4 (1d6 + 1) piercing damage.

\textit{\textbf{Teleport (Cooldown 4-6).}} The dog magically teleports, along with whatever he's wearing or carrying, up to 13 meters in an unoccupied space that he can see. Before or after teleporting, the dog can make a bite attack.

\medskip\textbf{Goat}\index[Monsters]{Goat}

\textit{Medium beast, unaligned}

\textbf{STRENGTH} +1

\textbf{DEXTERITY} +0

\textbf{CONSTITUTION} +0

\textbf{INTELLIGENCE} -4

\textbf{WISDOM} +0

\textbf{CHARISMA} -3

\textbf{Initiative} +0 -- \textbf{Defence} 11

\textbf{Hit Points} 4 (1d8)

\textbf{Move} 12m

\textbf{Saving Throws}: Fortitude +1, Reflexes +1, Will +0

\textbf{Languages} -

\textbf{Challenge} 0 (10 XP)

\textit{\textbf{Charge.}} If the Fase Dog moves at least 6 meters directly towards the target and hits with a bill attack during the same turn, the target takes an additional 2 (1d4) bludgeoning damage. If the target is a creature, it must succeed on a DC 10 Fortitude save
or fall prone.

\textit{\textbf{Steady Feet.}} The Fase Dog has +1d6 on Fortitude and Reflex saves made against effects that would knock it prone.

\textbf{Actions}

\textit{\textbf{Beak.} Melee Weapon Attack}: +3 to hit, reach 1m, one target.

\textit{Hit:} 3 (1d4 + 1) bludgeoning damage.

\medskip\textbf{Giant Billy Goat}\index[Monsters]{Giant Billy Goat}

\textit{Large beast, unaligned}

\textbf{STRENGTH} +3

\textbf{DEXTERITY} +0

\textbf{CONSTITUTION} +1

\textbf{INTELLIGENCE} -4

\textbf{WISDOM} +1

\textbf{CHARISMA} -2

\textbf{Initiative} +0 -- \textbf{Defence} 12

\textbf{Hit Points} 19 (3d10 + 3)

\textbf{Move} 12m

\textbf{Saving Throws}: Fortitude +4, Reflexes +1, Will +1

\textbf{Languages} -

\textbf{Challenge} 1/2 (100 XP)

\textit{\textbf{Charge.}} If the billy goat moves at least 6 meters directly towards the target and hits with a bill attack during the same turn, the target takes an additional 5 (2d4) bludgeoning damage. If the target is a creature, it must succeed on a DC 13 Fortitude save or fall prone.

\textit{\textbf{Steady Feet.}} The billy goat has +1d6 on Fortitude and Reflex saves made against effects that would knock it prone.

\textbf{Actions}

\textit{\textbf{Beak.} Melee Weapon Attack}: +5 to hit, reach 1m, one target.

\textit{Hit:} 8 (2d4 + 3) bludgeoning damage.

\medskip\textbf{Racehorse}\index[Monsters]{Racehorse}

\textit{Large beast, unaligned}

\textbf{STRENGTH} +3

\textbf{DEXTERITY} +0

\textbf{CONSTITUTION} +1

\textbf{INTELLIGENCE} -4

\textbf{WISDOM} +0

\textbf{CHARISMA} -2

\textbf{Initiative} +0 -- \textbf{Defence} 11

\textbf{Hit Points} 13 (2d10 + 2)

\textbf{Move} 18m

\textbf{Saving Throws}: Fortitude +3, Reflexes +1, Will +1

\textbf{Languages} -

\textbf{Challenge} 1/4 (50 XP)

\textbf{Actions}

\textit{\textbf{Hooves.} Melee Weapon Attack}: +5 to hit, reach 1m, one target.

\textit{Hit:} 8 (2d4 + 3) bludgeoning damage.

\medskip\textbf{Warhorse}\index[Monsters]{Warhorse}

\textit{Large beast, unaligned}

\textbf{STRENGTH} +4

\textbf{DEXTERITY} +1

\textbf{CONSTITUTION} +1

\textbf{INTELLIGENCE} -4

\textbf{WISDOM} +1

\textbf{CHARISMA} -2

\textbf{Initiative} +1 -- \textbf{Defence} 12 (plus barding possible)

\textbf{Hit Points} 19 (3d10 + 3)

\textbf{Move} 18m

\textbf{Saving Throws}: Fortitude +4, Reflexes +2, Will +1

\textbf{Languages} -

\textbf{Challenge} 1/2 (100 XP)

\textit{\textbf{Rrumbling Charge.}} If the horse moves at least 6 meters directly at the target and hits it with a hoof attack during the same turn, the target must succeed at a DC 14 Fortitude save or fall prone. If the target is prone, the horse can make another hoof attack against it as a bonus action.

\textbf{Actions}

\textit{\textbf{Hooves.} Melee Weapon Attack}: +6 to hit, reach 1m, one target.

\textit{Hit:} 11 (2d6 + 4) bludgeoning damage.

\medskip\textbf{Draft Horse}\index[Monsters]{Draft Horse}

\textit{Large beast, unaligned}

\textbf{STRENGTH} +4

\textbf{DEXTERITY} +0

\textbf{CONSTITUTION} +1

\textbf{INTELLIGENCE} -4

\textbf{WISDOM} +0

\textbf{CHARISMA} -2

\textbf{Initiative} +0 -- \textbf{Defence} 11

\textbf{Hit Points} 19 (3d10 + 3)

\textbf{Move} 12m

\textbf{Saving Throws}: Fortitude +5, Reflexes +1, Will +2

\textbf{Languages} -

\textbf{Challenge} 1/4 (50 XP)

\textbf{Actions}

\textit{\textbf{Hooves.} Melee Weapon Attack}: +6 to hit, reach 1m, one target.

\textit{Hit:} 9 (2d4 + 4) bludgeoning damage.

\medskip\textbf{Giant Seahorse}\index[Monsters]{Giant Seahorse}

The giant seahorse is often used as a mount by aquatic humanoids.

\textit{Large beast, unaligned}

\textbf{STRENGTH} +1

\textbf{DEXTERITY} +2

\textbf{CONSTITUTION} +0

\textbf{INTELLIGENCE} -4

\textbf{WISDOM} +1

\textbf{CHARISMA} -3

\textbf{Initiative} +2 -- \textbf{Defence} 14

\textbf{Hit Points} 16 (3d10)

\textbf{Movement} 0m, swim 12m

\textbf{Saving Throws}: Fortitude +2, Reflexes +3, Will +1

\textbf{Languages} -

\textbf{Challenge} 1/2 (100 XP)

\textit{\textbf{Charge.}} If the seahorse moves at least 6 meters directly towards the target and hits with a ram attack during the same turn, the target takes an additional 7 (2d6) bludgeoning damage. If the target is a creature, it must succeed on a DC 11 Fortitude save or fall prone.

\textit{\textbf{Water Breathing.}} The seahorse can only breathe underwater.

\textbf{Actions}

\textit{\textbf{Beak.} Melee Weapon Attack}: +3 to hit, reach 1m, one target.

\textit{Hit:} 4 (1d6 + 1) bludgeoning damage.

\medskip\textbf{Giant Centipede}\index[Monsters]{Centopiedi Gigante}

\textit{Little beast, unaligned}

\textbf{STRENGTH} -3

\textbf{DEXTERITY} +2

\textbf{CONSTITUTION} +1

\textbf{INTELLIGENCE} -5

\textbf{WISDOM} -2

\textbf{CHARISMA} -4

\textbf{Initiative} +2 -- \textbf{Defence} 14

\textbf{Hit Points} 4 (1d6 + 1)

\textbf{Move} 9m, climb 9m

\textbf{Saving Throws}: Fortitude -2, Reflexes +3, Will -2

\textbf{Senses} blindsight 9 m

\textbf{Languages} -

\textbf{Challenge} 1/4 (50 XP)

\textbf{Actions}

\textit{\textbf{Bite.} Melee Weapon Attack}: +4 to hit, reach 1 meter, one creature.

\textit{Hit:} 4 (1d4 + 2) piercing damage and the target must succeed on a DC 11 Fortitude save or take 10 (3d6) poison damage. If the poison damage reduces the target to 0 Hit Points, the target is stable but remains poisoned for 1 hour, even after regaining Hit Points, and while poisoned in this way becomes paralysed.

\medskip\textbf{Deer}\index[Monsters]{Deer}

\textit{Medium beast, unaligned}

\textbf{STRENGTH} +0

\textbf{DEXTERITY} +3

\textbf{CONSTITUTION} +0

\textbf{INTELLIGENCE} -4

\textbf{WISDOM} +2

\textbf{CHARISMA} -3

\textbf{Initiative} +3 -- \textbf{Defence} 14

\textbf{Hit Points} 4 (1d8)

\textbf{Move} 15m

\textbf{Saving Throws}: Fortitude +2, Reflexes +3, Will +2

\textbf{Languages} -

\textbf{Challenge} 0 (10 XP)

\textbf{Actions}

\textit{\textbf{Bite.} Melee Weapon Attack}: +2 to hit, reach 1m, one target.

\textit{Hit:} 2 (1d4) piercing damage.

\medskip\textbf{Boar}\index[Monsters]{Boar}

\textit{Medium beast, unaligned}

\textbf{STRENGTH} +1

\textbf{DEXTERITY} +0

\textbf{CONSTITUTION} +1

\textbf{INTELLIGENCE} -4

\textbf{WISDOM} -1

\textbf{CHARISMA} -3

\textbf{Initiative} +0 -- \textbf{Defence} 12

\textbf{Hit Points} 11 (2d8 + 2)

\textbf{Move} 12m

\textbf{Saving Throws}: Fortitude +2, Reflexes +1, Will -1

\textbf{Languages} -

\textbf{Challenge} 1/4 (50 XP)

\textit{\textbf{Charge.}} If the boar moves at least 6 meters directly towards the target and hits with a tusk attack during the same turn, the target takes an additional 3 (1d6) slashing damage. If the target is a creature, it must succeed on a Fortitude save
DC 11 or fall prone.

\textit{\textbf{Relentless (Recharges after 1 hour).}} If the boar takes 7 damage or less that would reduce it to 0 Hit Points, it drops to 1 hit point instead.

\textbf{Actions}

\textit{\textbf{Fang.} Melee Weapon Attack}: +3 to hit, reach 1m, one target.

\textit{Hit:} 4 (1d6 + 1) slashing damage.

\medskip\textbf{Giant Boar}\index[Monsters]{Giant Boar}

\textit{Large beast, unaligned}

\textbf{STRENGTH} +3

\textbf{DEXTERITY} +0

\textbf{CONSTITUTION} +3

\textbf{INTELLIGENCE} -4

\textbf{WISDOM} -2

\textbf{CHARISMA} -3

\textbf{Initiative} +0 -- \textbf{Defence} 13

\textbf{Hit Points} 42 (5d10 + 15)

\textbf{Move} 12m

\textbf{Saving Throws}: Fortitude +4, Reflexes +2, Will +0

\textbf{Languages} -

\textbf{Challenge} 2 (450 XP)

\textit{\textbf{Charge.}} If the boar moves at least 6 meters directly towards the target and hits with a tusk attack during the same turn, the target takes an additional 7 (2d6) slashing damage. If the target is a creature, it must succeed on a DC 13 Fortitude save or be knocked prone.

\textit{\textbf{Relentless (Recharges after 1 hour).}} If the boar takes 10 damage or less that would reduce it to 0 Hit Points, it drops to 1 hit point instead.

\textbf{Actions}

\textit{\textbf{Fang.} Melee Weapon Attack}: +5 to hit, reach 1m, one target.

\textit{Hit:} 10 (2d6 + 3) slashing damage.

\medskip\textbf{Crocodile}\index[Monsters]{Crocodile}

\textit{Large beast, unaligned}

\textbf{STRENGTH} +2

\textbf{DEXTERITY} +0

\textbf{CONSTITUTION} +1

\textbf{INTELLIGENCE} -4

\textbf{WISDOM} +0

\textbf{CHARISMA} -3

\textbf{Initiative} +0 -- \textbf{Defence} 13

\textbf{Hit Points} 19 (3d10 + 3)

\textbf{Movement} 6m, swim 9m

\textbf{Saving Throws}: Fortitude +6, Reflexes +4, Will +2

\textbf{Skills} Stealth +2

\textbf{Languages} -

\textbf{Challenge} 1/2 (100 XP)

\textit{\textbf{Hold Breath.}} The crocodile can hold its breath for 15 minutes.

\textbf{Actions}

\textit{\textbf{Bite.} Melee Weapon Attack}: +4 to hit, reach 1 meter, one creature.

\textit{Hit:} 7 (1d10 + 2) piercing damage, and the target is grappled (DC 12 to escape). Until the grab ends, the target is restrained, and the crocodile can't use the bite against another target.

\medskip\textbf{Giant Crocodile}\index[Monsters]{Giant Crocodile}

\textit{Huge beast, unaligned}

\textbf{STRENGTH} +5

\textbf{DEXTERITY} -1

\textbf{CONSTITUTION} +3

\textbf{INTELLIGENCE} -4

\textbf{WISDOM} +0

\textbf{CHARISMA} -2

\textbf{Initiative} -1 -- \textbf{Defence} 15

\textbf{Hit Points} 85 (9d12 + 27)

\textbf{Movement} 9m, swim 15m

\textbf{Saving Throws}: Fortitude +15, Reflexes +8, Will +8

\textbf{Skills} Stealth +5

\textbf{Languages} -

\textbf{Challenge} 5 (1800 XP)

\textit{\textbf{Hold Breath.}} The crocodile can hold its breath for 30 minutes.

\textbf{Actions}

\textit{\textbf{Multiattack.}} The crocodile makes two attacks: one with its bite and one with its tail.

\textit{\textbf{Tail.} Melee Weapon Attack}: +8 to hit, reach 3m, one target not grabbed by the crocodile.

\textit{Hit:} 14 (2d8 + 5) bludgeoning damage. If the target is a creature, it must succeed on a DC 16 Fortitude save or be knocked prone.

\textit{\textbf{Bite.} Melee Weapon Attack}: +8 to hit, reach 1m, one target.

\textit{Hit:} 21 (3d10 + 5) piercing damage, and the target is grappled (DC 16 to escape). Until the grab ends, the target is restrained, and the crocodile can't use the bite against another target.

\medskip\textbf{Raven}\index[Monsters]{Raven}

\textit{Tiny beast, unaligned}

\textbf{STRENGTH} -4

\textbf{DEXTERITY} +2

\textbf{CONSTITUTION} -1

\textbf{INTELLIGENCE} -4

\textbf{WISDOM} +1

\textbf{CHARISMA} -2

\textbf{Initiative} +2 -- \textbf{Defence} 13

\textbf{Hit Points} 1 (1d4 - 1)

\textbf{Move} 3m, fly 15m

\textbf{Saving Throws}: Fortitude +1, Reflexes +4, Will +2

\textbf{Skills} Awareness +3

\textbf{Languages} -

\textbf{Challenge} 0 (10 XP)

\textit{\textbf{Imitation.}} The raven can imitate simple sounds it has heard, such as a person's whisper, a child's cry, or an animal's cries. A creature that hears the sound can identify it as a mimic with a successful DC 10 Wisdom (Survival) check.

\textbf{Actions}

\textit{\textbf{Beak.} Melee Weapon Attack}: +4 to hit, reach 1m, one target.

\textit{Hit:} 1 piercing damage.

%\medskip\textbf{Donnola}\index[Monsters]{Donnola}

%\textit{Minuscola bestia, disallineato}

%\textbf{FORZA} -4

%\textbf{DESTREZZA} +3

%\textbf{COSTITUZIONE} -1

%\textbf{INTELLIGENZA} -4

%\textbf{SAGGEZZA} +1

%\textbf{CARISMA} -4

%\textbf{Iniziativa} +3 -- \textbf{Difesa} 14

%\textbf{Punti Ferita} 1 (1d4 - 1)

%\textbf{Movimento} 9 m

%\textbf{Tiri Salvezza}: Tempra +2, Riflessi +4, Volontà +1

%\textbf{Competenze} Muoversi Silenziosamente / Nascondersi +5, Consapevolezza +3

%\textbf{Lingue} -

%\textbf{Sfida} 0 (10 PX)

%\textit{\textbf{Udito e Olfatto Affinati.}} La donnola ha +1d6 nelle prove di Saggezza (Consapevolezza) basate su udito o olfatto.

%\textbf{Azioni}

%\textit{\textbf{Morso.} Attacco con Arma da Mischia}: +5 a colpire, portata 1 m, un bersaglio.

%\textit{Colpisce:} 1 danno perforante.

\medskip\textbf{Giant Weasel}\index[Monsters]{Giant Weasel}

\textit{Medium beast, unaligned}

\textbf{STRENGTH} +0

\textbf{DEXTERITY} +3

\textbf{CONSTITUTION} +0

\textbf{INTELLIGENCE} -3

\textbf{WISDOM} +1

\textbf{CHARISMA} -3

\textbf{Initiative} +3 -- \textbf{Defence} 14

\textbf{Hit Points} 9 (2d8)

\textbf{Move} 12m

\textbf{Saving Throws}: Fortitude +6, Reflexes +7, Will +2

\textbf{Skills} Stealth +5, Awareness +3

\textbf{Senses} vision in the dark 18 m

\textbf{Languages} -

\textbf{Challenge} 1/8 (25 XP)

\textit{\textbf{Refined hearing and smell.}} The weasel has +1d6 on Wisdom (Awareness) checks based on hearing or smell.

\textbf{Actions}

\textit{\textbf{Bite.} Melee Weapon Attack}: +5 to hit, reach 1m, one target.

\textit{Hit:} 5 (1d4 + 3) piercing damage.

\medskip\textbf{Elephant}\index[Monsters]{Elephant}

\textit{Huge beast, unaligned}

\textbf{STRENGTH} +6

\textbf{DEXTERITY} -1

\textbf{CONSTITUTION} +3

\textbf{INTELLIGENCE} -4

\textbf{WISDOM} +0

\textbf{CHARISMA} -2

\textbf{Initiative} -1 -- \textbf{Defence} 14

\textbf{Hit Points} 76 (8d12 + 24)

\textbf{Move} 12m

\textbf{Saving Throws}: Fortitude +13, Reflexes +7, Will +6

\textbf{Languages} -

\textbf{Challenge} 4 (1000 XP)

\textit{\textbf{Rrumbling Charge.}} If the elephant moves at least 6 meters directly towards a creature and hits it with a butting attack during the same turn, the target must succeed at a DC Fortitude save 12 or fall prone. If the target is prone, the elephant can make a stomp attack against it as a bonus action.

\textbf{Actions}

\textit{\textbf{Gore.} Melee Weapon Attack}: +8 to hit, reach 1m, one target.

\textit{Hit:} 19 (3d8 + 6) piercing damage.

\textit{\textbf{Stamp.} Melee Weapon Attack}: +8 to hit, reach 1m, one prone target.

\textit{Hit:} 22 (3d10 + 6) bludgeoning damage.

\medskip\textbf{Falco}\index[Monsters]{Falco}

\textit{Tiny beast, unaligned}

\textbf{STRENGTH} -3

\textbf{DEXTERITY} +3

\textbf{CONSTITUTION} -1

\textbf{INTELLIGENCE} -4

\textbf{WISDOM} +2

\textbf{CHARISMA} -2

\textbf{Initiative} +3 -- \textbf{Defence} 14

\textbf{Hit Points} 1 (1d4 - 1)

\textbf{Move} 3m, fly 18m

\textbf{Saving Throws}: Fortitude +2, Reflexes +5, Will +2

\textbf{Skills} Awareness +4

\textbf{Languages} -

\textbf{Challenge} 0 (10 XP)

\textit{\textbf{Enhanced Sight.}} The falcon has +1d6 on Wisdom (Awareness) checks based on sight.

\textbf{Actions}

\textit{\textbf{Spurs.} Melee Weapon Attack}: +5 to hit, reach 1m, one target.

\textit{Hit:} 1 slashing damage.

\medskip\textbf{Blood Hawk}\index[Monsters]{Blood Hawk}

Owing its name to its crimson feathers and aggressive nature, the bloodhawk attacks fearlessly using its pointed beak.

\textit{Little beast, unaligned}

\textbf{STRENGTH} -2

\textbf{DEXTERITY} +2

\textbf{CONSTITUTION} +0

\textbf{INTELLIGENCE} -4

\textbf{WISDOM} +2

\textbf{CHARISMA} -3

\textbf{Initiative} +2 -- \textbf{Defence} 13

\textbf{Hit Points} 7 (2d6)

\textbf{Move} 3m, fly 18m

\textbf{Saving Throws}: Fortitude +3, Reflexes +6, Will +3

\textbf{Skills} Awareness +4

\textbf{Languages} -

\textbf{Challenge} 1/8 (25 XP)

\textit{\textbf{Packing tactics.}} The falcon has +1d6 on attack rolls against a creature if at least one of the falcon's allies is within 1 meter of the creature and that ally isn't incapacitated.

\textit{\textbf{Enhanced Sight.}} The falcon has +1d6 on Wisdom (Awareness) checks based on sight.

\textbf{Actions}

\textit{\textbf{Beak.} Melee Weapon Attack}: +4 to hit, reach 1m, one target.

\textit{Hit:} 4 (1d4 + 2) piercing damage.

\medskip\textbf{Pirana}\index[Monsters]{Pirana}

Pirana is a sharp-toothed carnivorous fish.

\textit{Tiny beast, unaligned}

\textbf{STRENGTH} -4

\textbf{DEXTERITY} +3

\textbf{CONSTITUTION} -1

\textbf{INTELLIGENCE} -5

\textbf{WISDOM} -2

\textbf{CHARISMA} -4

\textbf{Initiative} +3 -- \textbf{Defence} 14

\textbf{Hit Points} 1 (1d4 - 1)

\textbf{Movement} 0m, swim 12m

\textbf{Saving Throws}: Fortitude -4, Reflexes +3, Will -2

\textbf{Senses} vision in the dark 18 m

\textbf{Languages} -

\textbf{Challenge} 0 (10 XP)

\textit{\textbf{Blood Frenzy.}} The pirana has +1d6 on melee attack rolls against any creature that is not at full Hit Points.

\textit{\textbf{Water Breathing.}} The pirana can only breathe underwater.

\textbf{Actions}

\textit{\textbf{Bite.} Melee Weapon Attack}: +5 to hit, reach 1m, one target.

\textit{Hit:} 1 piercing damage.

\medskip\textbf{Cat}\index[Monsters]{Cat}

\textit{Tiny beast, unaligned}

\textbf{STRENGTH} -4

\textbf{DEXTERITY} +2

\textbf{CONSTITUTION} +0

\textbf{INTELLIGENCE} -4

\textbf{WISDOM} +1

\textbf{CHARISMA} -2

\textbf{Initiative} +2 -- \textbf{Defence} 13

\textbf{Hit Points} 2 (1d4)

\textbf{Movement} 12m, climb 9m

\textbf{Saving Throws}: Fortitude +1, Reflexes +4, Will +1

\textbf{Skills} Stealth +4, Awareness +3

\textbf{Languages} -

\textbf{Challenge} 0 (10 XP)

\textit{\textbf{Enhanced sense of smell.}} The cat has +1d6 on Wisdom (Awareness) checks based on smell.

\textbf{Actions}

\textit{\textbf{Claws.} Melee Weapon Attack}: +0 to hit, reach 1m, one target.

\textit{Hit:} 1 slashing damage.

\medskip\textbf{Giant Crab}\index[Monsters]{Giant Crab}

\textit{Medium beast, unaligned}

\textbf{STRENGTH} +1

\textbf{DEXTERITY} +2

\textbf{CONSTITUTION} +0

\textbf{INTELLIGENCE} -5

\textbf{WISDOM} -1

\textbf{CHARISMA} -4

\textbf{Initiative} +2 -- \textbf{Defence} 16

\textbf{Hit Points} 13 (3d8)

\textbf{Movement} 9m, swim 9m

\textbf{Saving Throws}: Fortitude +5, Reflexes +2, Will +1

\textbf{Skills} Stealth +4

\textbf{Senses} blindsight 9 m

\textbf{Languages} -

\textbf{Challenge} 1/8 (25 XP)

\textit{\textbf{Amphibious.}} The crab can breathe air and water.

\textbf{Actions}

\textit{\textbf{Claw (Claw).} Weapon Attack Melee}: +3 to hit, reach 1m, one target.

\textit{Hit:} 4 (1d6 + 1) bludgeoning damage and the target is grabbed (DC 11 to flee). The crab has two claws, each of which can grip only one target.

\medskip\textbf{Owl}\index[Monsters]{Owl}

\textit{Tiny beast, unaligned}

\textbf{STRENGTH} -4

\textbf{DEXTERITY} +1

\textbf{CONSTITUTION} -1

\textbf{INTELLIGENCE} -4

\textbf{WISDOM} +1

\textbf{CHARISMA} -2

\textbf{Initiative} +1 -- \textbf{Defence} 12

\textbf{Hit Points} 1 (1d4 - 1)

\textbf{Move} 1m, fly 18m

\textbf{Saving Throws}: Fortitude +2, Reflexes +5, Will +2

\textbf{Skills} Stealth +3, Awareness +3

\textbf{Senses} vision in the dark 36 m

\textbf{Languages} -

\textbf{Challenge} 0 (10 XP)

\textit{\textbf{Fly over.}} The owl does not provoke attacks of opportunity when flying out of an enemy's reach.

\textit{\textbf{Honed hearing and vision.}} The owl has +1d6 on Wisdom (Awareness) checks based on hearing or vision.

\textbf{Actions}

\textit{\textbf{Spurs.} Melee Weapon Attack}: +3 to hit, reach 1m, one target.

\textit{Hit:} 1 slashing damage.

\medskip\textbf{Giant Owl}\index[Monsters]{Giant Owl}

Giant owls are intelligent creatures that protect the woodland kingdoms.

\textit{Large beast, neutral}

\textbf{STRENGTH} +1

\textbf{DEXTERITY} +2

\textbf{CONSTITUTION} +1

\textbf{INTELLIGENCE} -1

\textbf{WISDOM} +1

\textbf{CHARISMA} +0

\textbf{Initiative} +2 -- \textbf{Defence} 13

\textbf{Hit Points} 19 (3d10 + 3)

\textbf{Move} 1m, fly 18m

\textbf{Saving Throws}: Fortitude +1, Reflexes +4, Will +1

\textbf{Skills} Stealth +4, Awareness +5

\textbf{Senses} vision in the dark 36 m

\textbf{Languages} Giant Owl, understands Common, Elven, and Sylvan but cannot speak them

\textbf{Challenge} 1/4 (50 XP)

\textit{\textbf{Flying over.}} The owl does not provoke attacks of opportunity when it flies out of an enemy's reach.

\textit{\textbf{Honed hearing and vision.}} The owl has +1d6 on Wisdom (Awareness) checks based on hearing or vision.

\textbf{Actions}

\textit{\textbf{Spurs.} Melee Weapon Attack}: +3 to hit, reach 1m, one target.

\textit{Hit:} 8 (2d6 + 1) piercing damage.

\medskip\textbf{Hyena}\index[Monsters]{Hyena}

\textit{Medium beast, unaligned}

\textbf{STRENGTH} +0

\textbf{DEXTERITY} +1

\textbf{CONSTITUTION} +1

\textbf{INTELLIGENCE} -4

\textbf{WISDOM} +1

\textbf{CHARISMA} -3

\textbf{Initiative} +1 -- \textbf{Defence} 12

\textbf{Hit Points} 5 (1d8 + 1)

\textbf{Move} 15m

\textbf{Saving Throws}: Fortitude +5, Reflexes +5, Will +1

\textbf{Skills} Awareness +3

\textbf{Languages} -

\textbf{Challenge} 0 (10 XP)

\textit{\textbf{Packing tactics.}} The hyena has +1d6 on attack rolls against a creature if at least one of the hyena's allies is within 1 meter of the creature and that ally isn't incapacitated.

\textbf{Actions}

\textit{\textbf{Bite.} Melee Weapon Attack}: +2 to hit, reach 1m, one target.

\textit{Hit:} 3 (1d6) piercing damage.

\medskip\textbf{Giant Hyena}\index[Monsters]{Giant Hyena}

\textit{Large beast, unaligned}

\textbf{STRENGTH} +3

\textbf{DEXTERITY} +2

\textbf{CONSTITUTION} +2

\textbf{INTELLIGENCE} -4

\textbf{WISDOM} +1

\textbf{CHARISMA} -2

\textbf{Initiative} +2 -- \textbf{Defence} 13

\textbf{Hit Points} 45 (6d10 + 12)

\textbf{Move} 15m

\textbf{Saving Throws}: Fortitude +6, Reflexes +6, Will +2

\textbf{Skills} Awareness +3

\textbf{Languages} -

\textbf{Challenge} 1 (200 XP)

\textit{\textbf{Rage.}} When the hyena reduces a creature to 0 Hit Points with a melee attack during its round, the hyena can take a bonus action to move up to half its movement and perform a bite attack.

\textbf{Actions}

\textit{\textbf{Bite.} Melee Weapon Attack}: +5 to hit, reach 1m, one target.

\textit{Hit:} 10 (2d6 + 3) piercing damage.

\medskip\textbf{Lion}\index[Monsters]{Lion}

\textit{Large beast, unaligned}

\textbf{STRENGTH} +3

\textbf{DEXTERITY} +2

\textbf{CONSTITUTION} +1

\textbf{INTELLIGENCE} -4

\textbf{WISDOM} +1

\textbf{CHARISMA} -1

\textbf{Initiative} +2 -- \textbf{Defence} 13

\textbf{Hit Points} 26 (4d10 + 4)

\textbf{Move} 15m

\textbf{Saving Throws}: Fortitude +6, Reflexes +7, Will +2

\textbf{Skills} Stealth +6, Awareness +3

\textbf{Languages} -

\textbf{Challenge} 1 (200 XP)

\textit{\textbf{Leap.}} If the lion moves at least 6 meters directly at a creature and hits it with a claw attack during the same turn, the target must succeed on a DC 13 Fortitude save or fall prone. If the target is prone, the lion can make a
bite attack as a bonus action.

\textit{\textbf{Enhanced sense of smell.}} The lion has +1d6 on Wisdom (Awareness) checks based on smell.

\textit{\textbf{Leap with Running.}} With a 3m run, the lion can jump up to 6 meters long.

\textit{\textbf{Pack tactics.}} The lion has +1d6 on attack rolls against a creature if at least one of the lion's allies is within 1 meter of the creature and that ally isn't incapacitated.

\textbf{Actions}

\textit{\textbf{Claw.} Melee Weapon Attack}: +5 to hit, reach 1m, one target.

\textit{Hit:} 6 (1d6 + 3) slashing damage, 1 bleed damage.

\textit{\textbf{Bite.} Melee Weapon Attack}: +5 to hit, reach 1m, one target.

\textit{Hit:} 7 (1d8 + 3) piercing damage.

\medskip\textbf{Lizard}\index[Monsters]{Lizard}

\textit{Tiny beast, unaligned}

\textbf{STRENGTH} -4

\textbf{DEXTERITY} +0

\textbf{CONSTITUTION} +0

\textbf{INTELLIGENCE} -5

\textbf{WISDOM} -1

\textbf{CHARISMA} -4

\textbf{Initiative} +0 -- \textbf{Defence} 11

\textbf{Hit Points} 2 (1d4)

\textbf{Movement} 6m, climb 6m

\textbf{Saving Throws}: Fortitude +1, Reflexes +4, Will +1

\textbf{Senses} vision in the dark 9m

\textbf{Languages} -

\textbf{Challenge} 0 (10 XP)

\textit{\textbf{Climb as Spider.}} The lizard can climb difficult surfaces, including standing upside down on ceilings, without needing to make an ability check.

\textbf{Actions}

\textit{\textbf{Bite.} Melee Weapon Attack}: +0 to hit, reach 1m, one target.

\textit{Hit:} 1 piercing damage.

\medskip\textbf{Giant Lizard}\index[Monsters]{Giant Lizard}

Giant lizards are fearsome predators and are often used as mounts or draft animals by reptilian humanoids and underground residents.

\textit{Large beast, unaligned}

\textbf{STRENGTH} +2

\textbf{DEXTERITY} +1

\textbf{CONSTITUTION} +1

\textbf{INTELLIGENCE} -4

\textbf{WISDOM} +0

\textbf{CHARISMA} -3

\textbf{Initiative} +1 -- \textbf{Defence} 13

\textbf{Hit Points} 19 (3d10 + 3)

\textbf{Move} 9m, climb 9m

\textbf{Saving Throws}: Fortitude +11, Reflexes +8, Will +4

\textbf{Senses} vision in the dark 9m

\textbf{Languages} -

\textbf{Challenge} 1/4 (50 XP)

\textbf{Actions}

\textit{\textbf{Bite.} Melee Weapon Attack}: +4 to hit, reach 1m, one target.

\textit{Hit:} 6 (1d8 + 2) piercing damage.

\textbf{VARIANT}

Some giant lizards possess one or both of the following traits.

\textit{\textbf{Climb as Spider.}} The lizard can climb difficult surfaces, including standing upside down on ceilings, without needing to make an ability check.

\textit{\textbf{Hold Breath.}} The lizard can hold its breath for 15 minutes. (A lizard with this trait also has a swim speed of 10 meters.)

\medskip\textbf{Wolf}\index[Monsters]{Wolf}

\textit{Medium beast, unaligned}

\textbf{STRENGTH} +1

\textbf{DEXTERITY} +2

\textbf{CONSTITUTION} +1

\textbf{INTELLIGENCE} -4

\textbf{WISDOM} +1

\textbf{CHARISMA} -2

\textbf{Initiative} +2 -- \textbf{Defence} 14

\textbf{Hit Points} 11 (2d8 + 2)

\textbf{Move} 12m

\textbf{Saving Throws}: Fortitude +5, Reflexes +5, Will +1

\textbf{Skills} Stealth +4, Awareness +3

\textbf{Languages} -

\textbf{Challenge} 1/4 (50 XP)

\textit{\textbf{Hearing and Fine Smell.}} The wolf has +1d6 on Wisdom (Awareness) checks based on hearing or smell.

\textit{\textbf{Pack tactics.}} The wolf has +1d6 on attack rolls against a creature if at least one of the wolf's allies is within 1 meter of the creature and that ally isn't incapacitated.

\textbf{Actions}

\textit{\textbf{Bite.} Melee Weapon Attack}: +4 to hit, reach 1m, one target.

\textit{Hit:} 7 (2d4 + 2) piercing damage. If the target is a creature, it must succeed on a DC 11 Fortitude save or fall prone.

\medskip\textbf{Dinwolf (Direwolf)}\index[Monsters]{Dinwolf (Direwolf}

\textit{Large beast, unaligned}

\textbf{STRENGTH} +3

\textbf{DEXTERITY} +2

\textbf{CONSTITUTION} +2

\textbf{INTELLIGENCE} -2

\textbf{WISDOM} +1

\textbf{CHARISMA} -2

\textbf{Initiative} +2 -- \textbf{Defence} 15

\textbf{Hit Points} 37 (5d10 + 10)

\textbf{Move} 15m

\textbf{Saving Throws}: Fortitude +7, Reflexes +6, Will +2

\textbf{Skills} Stealth +4, Awareness +3

\textbf{Languages} -

\textbf{Challenge} 1 (200 XP)

\textit{\textbf{Hearing and Fine Smell.}} The wolf has +1d6 on Wisdom (Awareness) checks based on hearing or smell.

\textit{\textbf{Pack tactics.}} The wolf has +1d6 on attack rolls against a creature if at least one of the wolf's allies is within 1 meter of the creature and that ally isn't incapacitated.

\textbf{Actions}

\textit{\textbf{Bite.} Melee Weapon Attack}: +5 to hit, reach 1m, one target.

\textit{Hit:} 10 (2d6 + 3) piercing damage. If the target is a creature, it must succeed on a DC 13 Fortitude save or fall prone.

\medskip\textbf{Winter Wolf}\index[Monsters]{Winter Wolf}

Winter wolves inhabit the Arctic regions and are evil, intelligent creatures with snow-white coats and ice-colored eyes.

\textit{Large Monstrosity, Neutral Evil}

\textbf{STRENGTH} +4

\textbf{DEXTERITY} +1

\textbf{CONSTITUTION} +2

\textbf{INTELLIGENCE} -2

\textbf{WISDOM} +1

\textbf{CHARISMA} -1

\textbf{Initiative} +1 -- \textbf{Defence} 15

\textbf{Hit Points} 75 (10d10 + 20)

\textbf{Move} 15m

\textbf{Saving Throws}: Fortitude +9, Reflexes +6, Will +3

\textbf{Skills} Stealth +3, Awareness +5

\textbf{Damage Immunity} cold

\textbf{Languages} Common, Giant, Winter Wolf

\textbf{Challenge} 3 (700 XP)

\textit{\textbf{Snow camouflage.}} The wolf has +1d6 on Dexterity (Hide) checks made to hide in snowy terrain.

\textit{\textbf{Hearing and Fine Smell.}} The wolf has +1d6 on Wisdom (Awareness) checks based on hearing or smell.

\textit{\textbf{Pack tactics.}} The wolf has +1d6 on attack rolls against a creature if at least one of the wolf's allies is within 1 meter of the creature and that ally isn't incapacitated.

\textbf{Actions}

\textit{\textbf{Bite.} Melee Weapon Attack}: +6 to hit, reach 1m, one target.

\textit{Hit:} 11 (2d6 + 4) piercing damage. If the target is a creature, it must succeed on a DC 14 Fortitude save or fall prone.

\textit{\textbf{Icy Breath (Cooldown 5-6).}} The wolf exhales a blast of icy wind in a 5-meter cone. Each creature in that area must make a DC 12 Reflex save, taking 18 (4d8) cold damage on a failed save, or half as much damage on a successful one.

\medskip\textbf{Mastiff}\index[Monsters]{Mastiff}

\textbf{The} mastiffs are impressive hounds prized by humanoids for their reality and heightened senses.

\textit{Medium beast, unaligned}

\textbf{STRENGTH} +1

\textbf{DEXTERITY} +2

\textbf{CONSTITUTION} +1

\textbf{INTELLIGENCE} -4

\textbf{WISDOM} +1

\textbf{CHARISMA} -2

\textbf{Initiative} +2 -- \textbf{Defence} 13

\textbf{Hit Points} 5 (1d8 + 1)

\textbf{Move} 12m

\textbf{Saving Throws}: Fortitude +3, Reflexes +3, Will +1

\textbf{Skills} Awareness +3, Survival (Track) +3

\textbf{Languages} -

\textbf{Challenge} 1/8 (25 XP)

\textit{\textbf{Hearing and keen sense of smell.}} The mastiff has +1d6 on Wisdom (Awareness) checks based on hearing or smell.

\textbf{Actions}

\textit{\textbf{Bite.} Melee Weapon Attack}: +3 to hit, reach 1m, one target.

\textit{Hit:} 4 (1d6 + 1) piercing damage. If the target is a creature, it must succeed on a DC 11 Fortitude save or be knocked prone.

\medskip\textbf{Mule}\index[Monsters]{Mule}

\textit{Medium beast, unaligned}

\textbf{STRENGTH} +2

\textbf{DEXTERITY} +0

\textbf{CONSTITUTION} +1

\textbf{INTELLIGENCE} -4

\textbf{WISDOM} +0

\textbf{CHARISMA} -3

\textbf{Initiative} +0 -- \textbf{Defence} 11

\textbf{Hit Points} 11 (2d8 + 2)

\textbf{Move} 12m

\textbf{Saving Throws}: Fortitude +3, Reflexes +1, Will +1

\textbf{Languages} -

\textbf{Challenge} 1/8 (25 XP)

\textit{\textbf{Beast of Burden.}} The mule is considered a Large animal for the purposes of determining its carrying capacity.

\textit{\textbf{Steady Feet.}} The mule has +1d6 on Fortitude and Reflex saves made against effects that would knock it prone.

\textbf{Actions}

\textit{\textbf{Hooves.} Melee Weapon Attack}: +2 to hit, reach 1m, one target.

\textit{Hit:} 4 (1d4 + 2) bludgeoning damage.

\medskip\textbf{Brown Bear}\index[Monsters]{Brown Bear}

\textit{Large beast, unaligned}

\textbf{STRENGTH} +4

\textbf{DEXTERITY} +0

\textbf{CONSTITUTION} +3

\textbf{INTELLIGENCE} -4

\textbf{WISDOM} +1

\textbf{CHARISMA} -2

\textbf{Initiative} +0 -- \textbf{Defence} 12

\textbf{Hit Points} 34 (4d10 + 12)

\textbf{Movement} 12m, climb 9m

\textbf{Saving Throws}: Fortitude +6, Reflexes +2, Will +3

\textbf{Skills} Awareness +3

\textbf{Languages} -

\textbf{Challenge} 1 (200 XP)

\textit{\textbf{Enhanced sense of smell.}} The bear has +1d6 on Wisdom (Awareness) checks based on smell.

\textbf{Actions}

\textit{\textbf{Multiattack.}} The bear makes two attacks: one with its bite and one with its claws.

\textit{\textbf{Claws.} Melee Weapon Attack}: +5 to hit, reach 1m, one target.

\textit{Hit:} 11 (2d6 + 4) slashing damage.

\textit{\textbf{Bite.} Melee Weapon Attack}: +5 to hit, reach 1m, one target.

\textit{Hit:} 8 (1d8 + 4) piercing damage.

\medskip\textbf{Black Bear}\index[Monsters]{Black Bear}

\textit{Medium beast, unaligned}

\textbf{STRENGTH} +2

\textbf{DEXTERITY} +0

\textbf{CONSTITUTION} +2

\textbf{INTELLIGENCE} -4

\textbf{WISDOM} +1

\textbf{CHARISMA} -2

\textbf{Initiative} +0 -- \textbf{Defence} 12

\textbf{Hit Points} 19 (3d8 + 6)

\textbf{Movement} 12m, climb 9m

\textbf{Saving Throws}: Fortitude +4, Reflexes +1, Will +1

\textbf{Skills} Awareness +3

\textbf{Languages} -

\textbf{Challenge} 1/2 (100 XP)

\textit{\textbf{Enhanced sense of smell.}} The bear has +1d6 on Wisdom (Awareness) checks based on smell.

\textbf{Actions}

\textit{\textbf{Multiattack.}} The black bear makes two attacks: one with its bite and one with its claws.

\textit{\textbf{Claws.} Melee Weapon Attack}: +3 to hit, reach 1m, one target.

\textit{Hit:} 7 (2d4 + 2) slashing damage, 1 bleed damage.

\textit{\textbf{Bite.} Melee Weapon Attack}: +3 to hit, reach 1m, one target.

\textit{Hit:} 5 (1d6 + 2) piercing damage.

\medskip\textbf{Polar Bear}\index[Monsters]{Polar Bear}

\textit{Large beast, unaligned}

\textbf{STRENGTH} +5

\textbf{DEXTERITY} +0

\textbf{CONSTITUTION} +3

\textbf{INTELLIGENCE} -4

\textbf{WISDOM} +1

\textbf{CHARISMA} -2

\textbf{Initiative} +0 -- \textbf{Defence} 13

\textbf{Hit Points} 42 (5d10 + 15)

\textbf{Movement} 12m, swim 9m

\textbf{Saving Throws}: Fortitude +10, Reflexes +7, Will +4

\textbf{Skills} Awareness +3

\textbf{Languages} -

\textbf{Challenge} 2 (450 XP)

\textit{\textbf{Enhanced sense of smell.}} The bear has +1d6 on Wisdom (Awareness) checks based on smell.

\textbf{Actions}

\textit{\textbf{Multiattack.}} The bear makes two attacks: one with its bite and one with its claws.

\textit{\textbf{Claws.} Melee Weapon Attack}: +7 to hit, reach 1m, one target.

\textit{Hit:} 12 (2d6 + 5) slashing damage.

\textit{\textbf{Bite.} Melee Weapon Attack}: +7 to hit, reach 1m, one target.

\textit{Hit:} 9 (1d8 + 5) piercing damage.

\textbf{VARIANT: CAVE BEAR}\index[Monsters]{Cave Bear}

Some bears have adapted to life underground. They have the same stats as polar bears, but with 18m in the dark.

\medskip\textbf{Panther}\index[Monsters]{Panther}

\textit{Medium beast, unaligned}

\textbf{STRENGTH} +2

\textbf{DEXTERITY} +2

\textbf{CONSTITUTION} +0

\textbf{INTELLIGENCE} -4

\textbf{WISDOM} +2

\textbf{CHARISMA} -2

\textbf{Initiative} +2 -- \textbf{Defence} 13

\textbf{Hit Points} 13 (3d8)

\textbf{Movement} 15m, climb 12m

\textbf{Saving Throws}: Fortitude +3, Reflexes +5, Will +3

\textbf{Skills} Stealth +6, Awareness +4

\textbf{Languages} -

\textbf{Challenge} 1/4 (50 XP)

\textit{\textbf{Leap.}} If the panther moves at least 6 meters directly towards a creature and hits it with a claw attack during the same turn, the target must succeed on a DC 12 Fortitude save or fall prone. If the target is prone, the panther can make a bite attack against it as a bonus action.

\textit{\textbf{A keen sense of smell.}} The panther has +1d6 on Wisdom (Awareness) checks based on smell.

\textbf{Actions}

\textit{\textbf{Claw.} Melee Weapon Attack}: +4 to hit, reach 1m, one target.

\textit{Hit:} 4 (1d4 + 2) slashing damage, 1 bleed damage.

\textit{\textbf{Bite.} Melee Weapon Attack}: +4 to hit, reach 1m, one target.

\textit{Hit:} 5 (1d6 + 2) piercing damage.


\medskip\textbf{Pony}\index[Monsters]{Pony}

\textit{Medium beast, unaligned}

\textbf{STRENGTH} +2

\textbf{DEXTERITY} +0

\textbf{CONSTITUTION} +1

\textbf{INTELLIGENCE} -4

\textbf{WISDOM} +0

\textbf{CHARISMA} -2

\textbf{Initiative} +0 -- \textbf{Defence} 11

\textbf{Hit Points} 11 (2d8 + 2)

\textbf{Move} 12m

\textbf{Saving Throws}: Fortitude +5, Reflexes +4, Will +0

\textbf{Languages} -

\textbf{Challenge} 1/8 (25 XP)

\textbf{Actions}

\textit{\textbf{Hooves.} Melee Weapon Attack}: +4 to hit, reach 1m, one target.

\textit{Hit:} 7 (2d4 + 2) bludgeoning damage.

\medskip\textbf{Spider}\index[Monsters]{Spider}

\textit{Tiny beast, unaligned}

\textbf{STRENGTH} 2 (-5)

\textbf{DEXTERITY} +2

\textbf{CONSTITUTION} -1

\textbf{INTELLIGENCE} -5

\textbf{WISDOM} +0

\textbf{CHARISMA} -4

\textbf{Initiative} +2 -- \textbf{Defence} 13

\textbf{Hit Points} 1 (1d4 - 1)

\textbf{Move} 6m, climb 6m

\textbf{Saving Throws}: Fortitude -4, Reflexes +2, Will -4

\textbf{Skills} Stealth +4

\textbf{Senses} vision in the dark 9m

\textbf{Languages} -

\textbf{Challenge} 0 (10 XP)

\textit{\textbf{Web Walk.}} The spider ignores movement restrictions caused by webs.

\textit{\textbf{Climb as Spider.}} The spider can climb difficult surfaces, including standing upside down on ceilings, without needing to make an ability check.

\textit{\textbf{Web Sense.}} While in contact with a web, the spider knows the exact location of any other creature in contact with the same web.

\textbf{Actions}

\textit{\textbf{Bite.} Melee Weapon Attack}: +4 to hit, reach 3 ft., one creature.

\textit{Hit:} 1 piercing damage and the target must succeed on a Fortitude 9 save or take 2 (1d4) poison damage.

\medskip\textbf{Phase Spider}\index[Monsters]{Phase Spider}

The phase spider has the magical ability to enter and exit the Ethereal Plane. It seems to appear out of nowhere and quickly disappears after attacking.

\textit{Large monstrosity, unaligned}

\textbf{STRENGTH} +2

\textbf{DEXTERITY} +2

\textbf{CONSTITUTION} +1

\textbf{INTELLIGENCE} -2

\textbf{WISDOM} +0

\textbf{CHARISMA} -2

\textbf{Initiative} +2 -- \textbf{Defence} 15

\textbf{Hit Points} 32 (5d10 + 5)

\textbf{Movement} 9m, climb 9m

\textbf{Saving Throws}: Fortitude +8, Reflexes +8, Will +3

\textbf{Skills} Stealth +6

\textbf{Senses} vision in the dark 18 m

\textbf{Languages} -

\textbf{Challenge} 3 (700 XP)

\textit{\textbf{Web Walk.}} The spider ignores movement restrictions caused by webs.

\textit{\textbf{Ethereal Excursion.}} As a bonus action, the spider can magically move from the Material Plane to the Ethereal Plane, or vice versa.

\textit{\textbf{Climb as Spider.}} The spider can climb difficult surfaces, including standing upside down on ceilings, without needing to make an ability check.

\textbf{Actions}

\textit{\textbf{Bite.} Melee Weapon Attack}: +4 to hit, reach 1 meter, one creature.

\textit{Hit:} 7 (1d10 + 2) piercing damage and the target must make a DC 11 Fortitude Saving Throw, taking 18 (4d8) poison damage on a failed save, or half as much damage on a failed save. he succeeds. If the poison damage reduces the target to 0 Hit Points, the target is stable but poisoned for 1 hour, even after regaining Hit Points, and while poisoned in this way becomes paralysed.

\medskip\textbf{Giant Spider}\index[Monsters]{Giant Spider}

\textit{Large beast, unaligned}

\textbf{STRENGTH} +2

\textbf{DEXTERITY} +3

\textbf{CONSTITUTION} +1

\textbf{INTELLIGENCE} -4

\textbf{WISDOM} +0

\textbf{CHARISMA} -3

\textbf{Initiative} +2 -- \textbf{Defence} 15

\textbf{Hit Points} 26 (4d10 + 4)

\textbf{Move} 9m, climb 9m

\textbf{Saving Throws}: Fortitude +4, Reflexes +4, Will +1

\textbf{Skills} Stealth +7

\textbf{Senses} blindsight 3m, darksight 18m

\textbf{Languages} -

\textbf{Challenge} 1 (200 XP)

\textit{\textbf{Web Walk.}} The spider ignores movement restrictions caused by webs.

\textit{\textbf{Climb as Spider.}} The spider can climb difficult surfaces, including standing upside down on ceilings, without needing to make an ability check.

\textit{\textbf{Web Sense.}} While in contact with a web, the spider knows the exact location of any other creature in contact with the same web.

\textbf{Actions}

\textit{\textbf{Bite.} Melee Weapon Attack}: +5 to hit, reach 1 meter, one creature.

\textit{Hit:} 7 (1d8 + 3) piercing damage and the target must make a DC 11 Fortitude save, and suffer 9

(2d8) poison damage on a failed save, or half as much damage on a successful one. If the poison damage reduces the target to 0 Hit Points, the target is stable but poisoned for 1 hour, even after regaining Hit Points, and while poisoned in this way becomes paralysed.

\textit{\textbf{Web (Cooldown 5-6).} Ranged Weapon Attack}: +5 to hit, reach 9m, one creature.

\textit{Hit:} Target is entangled in the web. As an action, the entangled target can make a DC 12 Strength check and, if successful, break the web. The web can also be attacked and destroyed (Defense 10; HP 5; vulnerability to fire damage; immunity to bludgeoning and poison damage).

\medskip\textbf{Giant Wolf Spider}\index[Monsters]{Giant Wolf Spider}

A giant wolf spider hunts for prey in open ground or hides in burrows or crevices in the ground to ambush it.

\textit{Medium beast, unaligned}

\textbf{STRENGTH} +1

\textbf{DEXTERITY} +3

\textbf{CONSTITUTION} +1

\textbf{INTELLIGENCE} -4

\textbf{WISDOM} +1

\textbf{CHARISMA} -3

\textbf{Initiative} +3 -- \textbf{Defence} 14

\textbf{Hit Points} 11 (2d8 + 2)

\textbf{Movement} 12m, climb 12m

\textbf{Saving Throws}: Fortitude +2, Reflexes +4, Will +1

\textbf{Skills} Stealth +7, Awareness +3

\textbf{Senses} blindsight 3m, darksight 18m

\textbf{Languages} -

\textbf{Challenge} 1/4 (50 XP)

\textit{\textbf{Web Walk.}} The spider ignores movement restrictions caused by webs.

\textit{\textbf{Climb as Spider.}} The spider can climb difficult surfaces, including standing upside down on ceilings, without needing to make an ability check.

\textit{\textbf{Web Sense.}} While in contact with a web, the spider knows the exact location of any other creature in contact with the same web.

\textbf{Actions}

\textit{\textbf{Bite.} Melee Weapon Attack}: +3 to hit, reach 1 meter, one creature.

\textit{Hit:} 4 (1d6 + 1) piercing damage and the target must make a DC 11 Fortitude Saving Throw, taking 7 (2d6) poison damage on a failed save, or half as much damage on a failed save. he succeeds. If the poison damage reduces the target to 0 Hit Points, the target is stable but poisoned for 1 hour, even after regaining Hit Points, and while poisoned in this way becomes paralysed.

\medskip\textbf{Frog}\index[Monsters]{Frog}

\textit{Tiny beast, unaligned}

\textbf{STRENGTH} -5

\textbf{DEXTERITY} +1

\textbf{CONSTITUTION} -1

\textbf{INTELLIGENCE} -5

\textbf{WISDOM} -1

\textbf{CHARISMA} -4

\textbf{Initiative} +1 -- \textbf{Defence} 12

\textbf{Hit Points} 1 (1d4 - 1)

\textbf{Movement} 6m, swim 6m

\textbf{Saving Throws}: Fortitude -4, Reflexes +1, Will -2

\textbf{Skills} Stealth +3, Awareness +1

\textbf{Senses} vision in the dark 9m

\textbf{Languages} -

\textbf{Challenge} 0 (0 XP)

\textit{\textbf{Amphibian.}} The frog can breathe air and water.

\textit{\textbf{Standing Jump.}} A frog can jump up to 3 meters long and up to 1 meter high, with or without a running start.

A \textbf{frog} has no attacks. It feeds on small insects and usually lives near marshes, inside trees or underground.

\medskip\textbf{Giant Frog}\index[Monsters]{Giant Frog}

\textit{Medium beast, unaligned}

\textbf{STRENGTH} +1

\textbf{DEXTERITY} +1

\textbf{CONSTITUTION} +0

\textbf{INTELLIGENCE} -4

\textbf{WISDOM} +0

\textbf{CHARISMA} -4

\textbf{Initiative} +1 -- \textbf{Defence} 12

\textbf{Hit Points} 18 (4d8)

\textbf{Movement} 9m, swim 9m

\textbf{Saving Throws}: Fortitude +2, Reflexes +2, Will +0

\textbf{Skills} Stealth +3, Awareness +2

\textbf{Senses} vision in the dark 9 m

\textbf{Languages} -

\textbf{Challenge} 1/4 (50 XP)

\textit{\textbf{Amphibian.}} The frog can breathe air and water.

\textit{\textbf{Standing Jump.}} A frog can jump up to 6 meters long and up to 3 meters high, with or without a running start.

\textbf{Actions}

\textit{\textbf{Bite.} Melee Weapon Attack}: +3 to hit, reach 1m, one target.

\textit{Hit:} 4 (1d6 + 1) piercing damage and the target is grappled (DC 11 to escape). Until the grab ends, the target is restrained, and the frog can't use the bite against another target.

\textit{\textbf{Swallow.}} The frog makes a bite attack against a Small or smaller target it is grabbing. If the attack hits, the target is engulfed, and the grab ends. The swallowed target is blinded and restrained, has full cover against attacks and other effects outside the frog, and takes 5 (2d4) acid damage at the start of each of the frog's rounds. The frog can only swallow one target at a time. If the frog dies, a swallowed creature is no longer restrained by it and can exit the corpse using 1 meter of movement, coming prone.

\medskip\textbf{Rat}\index[Monsters]{Rat}

\textit{Tiny beast, unaligned}

\textbf{STRENGTH} -4

\textbf{DEXTERITY} +0

\textbf{CONSTITUTION} -1

\textbf{INTELLIGENCE} -4

\textbf{WISDOM} +0

\textbf{CHARISMA} -3

\textbf{Initiative} +0 -- \textbf{Defence} 11

\textbf{Hit Points} 1 (1d4 - 1)

\textbf{Move} 6m

\textbf{Saving Throws}: Fortitude -4, Reflexes +0, Will +0

\textbf{Senses} vision in the dark 9m

\textbf{Languages} -

\textbf{Challenge} 0 (10 XP)

\textit{\textbf{Enhanced sense of smell.}} The rat has +1d6 on Wisdom (Awareness) checks based on smell.

\textbf{Actions}

\textit{\textbf{Bite.} Melee Weapon Attack}: +0 to hit, reach 1m, one target.

\textit{Hit:} 1 piercing damage.

\medskip\textbf{Giant Rat}\index[Monsters]{Giant Rat}

\textit{Little beast, unaligned}

\textbf{STRENGTH} -2

\textbf{DEXTERITY} +2

\textbf{CONSTITUTION} +0

\textbf{INTELLIGENCE} -4

\textbf{WISDOM} +0

\textbf{CHARISMA} -3

\textbf{Initiative} +2 -- \textbf{Defence} 13

\textbf{Hit Points} 7 (2d6)

\textbf{Move} 9m

\textbf{Saving Throws}: Fortitude +3, Reflexes +5, Will +1

\textbf{Senses} vision in the dark 18 m

\textbf{Languages} -

\textbf{Challenge} 1/8 (25 XP)

\textit{\textbf{Enhanced sense of smell.}} The rat has +1d6 on Wisdom (Awareness) checks based on smell.

\textit{\textbf{Pack tactics.}} The rat has +1d6 on attack rolls against a creature if at least one of the rat's allies is within 1 meter of the creature and that ally isn't incapacitated.

\textbf{Actions}

\textit{\textbf{Bite.} Melee Weapon Attack}: +4 to hit, reach 1m, one target.

\textit{Hit:} 4 (1d4 + 2) piercing damage.

\textbf{VARIANT: GIANT SICK RAT}\index[Monsters]{Giant Sick Rat}

Some giant rats carry a terrible disease which they spread by biting. A sick giant rat has a challenge rating of 1/8 (25 XP) and the following action instead of its normal bite attack.

\textit{\textbf{Bite.} Melee Weapon Attack}: +4 to hit, reach 1m, one target.

\textit{Hit:} 4 (1d4 + 2) piercing damage. If the target is a creature, it must succeed on a DC 10 Fortitude save or contract a disease. Until the disease is cured, the target cannot regain Hit Points except through magical methods, and the target's maximum Hit Points decrease by 3 (1d6) every 24 hours. If the target's maximum Hit Points drop to 0 as a result of the disease, the target dies.

\medskip\textbf{Rhinoceros}\index[Monsters]{Rhinoceros}

\textit{Large beast, unaligned}

\textbf{STRENGTH} +5

\textbf{DEXTERITY} -1

\textbf{CONSTITUTION} +2

\textbf{INTELLIGENCE} -4

\textbf{WISDOM} +1

\textbf{CHARISMA} -2

\textbf{Initiative} -1 -- \textbf{Defence} 12

\textbf{Hit Points} 45 (6d10 + 12)

\textbf{Move} 12m

\textbf{Saving Throws}: Fortitude +10, Reflexes +4, Will +2

\textbf{Languages} -

\textbf{Challenge} 2 (450 XP)

\textit{\textbf{Charge.}} If the rhino moves at least 6 meters directly towards a target and hits them with a gore attack during the same turn, the target takes an additional 9 (2d8) bludgeoning damage. If the target is a creature, it must succeed on a DC 15 Fortitude save or be knocked prone.

\textbf{Actions}

\textit{\textbf{Gored.} Melee Weapon Attack}: +7 to hit, reach 1m, one target.

\textit{Hit:} 14 (2d8 + 5) bludgeoning damage.

\medskip\textbf{Giant Toad}\index[Monsters]{Giant Toad}

\textit{Large beast, unaligned}

\textbf{STRENGTH} +2

\textbf{DEXTERITY} +1

\textbf{CONSTITUTION} +1

\textbf{INTELLIGENCE} -4

\textbf{WISDOM} +0

\textbf{CHARISMA} -4

\textbf{Initiative} +1 -- \textbf{Defence} 12

\textbf{Hit Points} 39 (6d10 + 6)

\textbf{Movement} 6m, swim 12m

\textbf{Saving Throws}: Fortitude +6, Reflexes +6, Will +0

\textbf{Senses} vision in the dark 9m

\textbf{Languages} -

\textbf{Challenge} 1 (200 XP)

\textit{\textbf{Amphibian.}} The toad can breathe air and water.

\textit{\textbf{Standing Jump.}} A toad can jump up to 6 meters long and up to 3 meters high, with or without a running start.

\textbf{Actions}

\textit{\textbf{Bite.} Melee Weapon Attack}: +4 to hit, reach 1m, one target.

\textit{Hit:} 7 (1d10 + 2) piercing damage plus 5 (1d10) poison damage, and the target is grappled (DC 13 to escape). Until the grab ends, the target is restrained, and the toad can't use the bite against another target.

\textit{\textbf{Swallow up.}} The toad makes a bite attack against a Medium or smaller target it is grappling. If the attack hits, the target is engulfed, and the grab ends. The swallowed target is blinded and restrained, has full cover against attacks and other effects outside the frog, and takes 10 (3d6) acid damage at the start of each of the toad's rounds. The toad can only swallow one target at a time.

If the toad dies, a swallowed creature is no longer restrained by it and can exit the corpse using 1 meter of movement, coming prone.

\medskip\textbf{Giant Fire Beetle}\index[Monsters]{Giant Fire Beetle}

A giant fire beetle is a nocturnal creature that possesses a pair of glow glands capable of emitting light for 1d6 days after the beetle's death.

\textit{Little beast, unaligned}

\textbf{STRENGTH} -1

\textbf{DEXTERITY} +0

\textbf{CONSTITUTION} +1

\textbf{INTELLIGENCE} -5

\textbf{WISDOM} -2

\textbf{CHARISMA} -4

\textbf{Initiative} +0 -- \textbf{Defence} 14

\textbf{Hit Points} 4 (1d6 + 1)

\textbf{Move} 9m

\textbf{Saving Throws}: Fortitude +2, Reflexes +0, Will +0

\textbf{Senses} blindsight 9 m

\textbf{Languages} -

\textbf{Challenge} 0 (10 XP)

\textit{\textbf{Illumination.}} The beetle sheds bright light in a 3m radius and dim light for an additional 3 meter.

\textbf{Actions}

\textit{\textbf{Bite.} Melee Weapon Attack}: +1 to hit, reach 1m, one target.

\textit{Hit:} 2 (1d6 - 1) slashing damage.

\medskip\textbf{Jackal}\index[Monsters]{Jackal}

\textit{Little beast, unaligned}

\textbf{STRENGTH} -1

\textbf{DEXTERITY} +2

\textbf{CONSTITUTION} +0

\textbf{INTELLIGENCE} -4

\textbf{WISDOM} +1

\textbf{CHARISMA} -2

\textbf{Initiative} +2 -- \textbf{Defence} 13

\textbf{Hit Points} 3 (1d6)

\textbf{Move} 12m

\textbf{Saving Throws}: Fortitude -1, Reflexes +3, Will +1

\textbf{Skills} Awareness +3

\textbf{Languages} -

\textbf{Challenge} 0 (10 XP)

\textit{\textbf{Pack tactics.}} The jackal has +1d6 on attack rolls against a creature if at least one of the jackal's allies is within 1 meter of the creature and that ally isn't incapacitated.

\textit{\textbf{Hearing and keen sense of smell.}} The jackal has +1d6 on Wisdom (Awareness) checks based on hearing or smell.

\textbf{Actions}

\textit{\textbf{Bite.} Melee Weapon Attack}: +1 to hit, reach 1m, one target.

\textit{Hit:} 1 (1d4 - 1) piercing damage.

\medskip\textbf{Swarms}\index[Monsters]{Swarms}

The swarms presented below are not ordinary or benign gatherings of small creatures. Instead, they form as a result of an external, often malignant influence. Even druids are unable to charm these swarms, and their aggression is almost unnatural.

\textbf{Centipede Swarm}\index[Monsters]{Centipede Swarm}

\textit{Medium Tiny Beast Swarm, unaligned}

\textbf{STRENGTH} -4

\textbf{DEXTERITY} +1

\textbf{CONSTITUTION} +0

\textbf{INTELLIGENCE} -5

\textbf{WISDOM} -2

\textbf{CHARISMA} -5

\textbf{Initiative} +1 -- \textbf{Defence} 13

\textbf{Hit Points} 22 (5d8)

\textbf{Move} 6m, climb 6m

\textbf{Saving Throws}: Fortitude -1, Reflexes +3, Will +1

\textbf{Damage Resistances} slashing, piercing, slashing

\textbf{Condition Immunity} charmed, grabbed, restrained, paralyzed, petrified, prone, frightened, stunned

\textbf{Senses} blindsight 3 m

\textbf{Languages} -

\textbf{Challenge} 1/2 (100 XP)

\textit{\textbf{Swarm.}} The swarm can occupy another creature's space and vice versa, and the swarm can move through any opening large enough for a Tiny insect. The swarm cannot regain Hit Points or gain temporary Hit Points.

\textbf{Actions}

\textit{\textbf{Bites.} Melee Weapon Attack}: +3 to hit, reach 0m, one target in the swarm's space.

\textit{Hit:} 10 (4d4) piercing damage, or 5 (2d4) piercing damage if the swarm is at half or fewer Hit Points. A creature reduced to 0 Hit Points by a swarm of centipedes and stable is poisoned for 1 hour, even after regaining Hit Points, and is paralyzed by the poison during this time.

\medskip\textbf{Swarm of Crows}\index[Monsters]{Swarm of Crows}

\textit{Medium Tiny Beast Swarm, unaligned}

\textbf{STRENGTH} -2

\textbf{DEXTERITY} +2

\textbf{CONSTITUTION} -1

\textbf{INTELLIGENCE} -4

\textbf{WISDOM} +1

\textbf{CHARISMA} -2

\textbf{Initiative} +2 -- \textbf{Defence} 13

\textbf{Hit Points} 24 (7d8 -- 7)

\textbf{Move} 3m, fly 15m

\textbf{Saving Throws}: Fortitude -1, Reflexes +3, Will +2

\textbf{Skills} Awareness +5

\textbf{Damage Resistances} slashing, piercing, slashing

\textbf{Condition Immunity} charmed, grabbed, restrained, paralyzed, petrified, prone, frightened, stunned

\textbf{Languages} -

\textbf{Challenge} 1/4 (50 XP)

\textit{\textbf{Swarm.}} The swarm can occupy another creature's space and vice versa, and the swarm can move through any opening large enough for a Tiny Crow. The swarm cannot regain Hit Points or gain temporary Hit Points.

\textbf{Actions}

\textit{\textbf{Beaks.} Melee Weapon Attack}: +4 to hit, reach 1m, one target in the swarm's space.

\textit{Hit:} 7 (2d6) piercing damage, or 3 (1d6) piercing damage if the swarm is at half or fewer Hit Points.

\medskip\textbf{Pirana Swarm}\index[Monsters]{Pirana Swarm}

\textit{Medium Tiny Beast Swarm, unaligned}

\textbf{STRENGTH} +1

\textbf{DEXTERITY} +3

\textbf{CONSTITUTION} -1

\textbf{INTELLIGENCE} -5

\textbf{WISDOM} -2

\textbf{CHARISMA} -4

\textbf{Initiative} +3 -- \textbf{Defence} 14

\textbf{Hit Points} 28 (8d8 -- 8)

\textbf{Movement} 0m, swim 12m

\textbf{Saving Throws}: Fortitude -3, Reflexes +4, Will -1

\textbf{Damage Resistances} slashing, piercing, slashing

\textbf{Condition Immunity} charmed, grabbed, restrained, paralyzed, petrified, prone, frightened, stunned

\textbf{Senses} vision in the dark 18 m

\textbf{Languages} -

\textbf{Challenge} 1 (200 XP)

\textit{\textbf{Blood Frenzy.}} The swarm has +1d6 on melee attack rolls against any creature that isn't at full Hit Points.

\textit{\textbf{Water Breathing.}} The swarm can only breathe underwater.

\textit{\textbf{Swarm.}} The swarm can occupy another creature's space and vice versa, and the swarm can move through any opening large enough for a Tiny Pirana. The swarm cannot regain Hit Points or gain temporary Hit Points.

\textbf{Actions}

\textit{\textbf{Bites.} Melee Weapon Attack}: +5 to hit, reach 0m, one creature in the swarm's space.

\textit{Hit:} 14 (4d6) piercing damage, or 7 (2d6) piercing damage if the swarm is at half or fewer Hit Points.

\medskip\textbf{Swarm of Bugs}\index[Monsters]{Swarm of Bugs}

\textit{Medium Tiny Beast Swarm, unaligned}

\textbf{STRENGTH} -4

\textbf{DEXTERITY} +1

\textbf{CONSTITUTION} +0

\textbf{INTELLIGENCE} -5

\textbf{WISDOM} -2

\textbf{CHARISMA} -5

\textbf{Initiative} +1 -- \textbf{Defence} 13

\textbf{Hit Points} 22 (5d8)

\textbf{Movement} 6m, climb 6m

\textbf{Saving Throws}: Fortitude -3, Reflexes +2, Will -1

\textbf{Damage Resistances} slashing, piercing, slashing

\textbf{Immunity to Conditions} charmed, grabbed, restrained, paralyzed, petrified, prone, frightened, stunned

\textbf{Senses} blindsight 3 m

\textbf{Languages} -

\textbf{Challenge} 1/2 (100 XP)

\textit{\textbf{Swarm.}} The swarm can occupy another creature's space and vice versa, and the swarm can move through any opening large enough for a Tiny insect. The swarm cannot regain Hit Points or gain temporary Hit Points.

\textbf{Actions}

\textit{\textbf{Bites.} Melee Weapon Attack}: +3 to hit, reach 0m, one target in the swarm's space.

\textit{Hit:} 10 (4d4) piercing damage, or 5 (2d4) piercing damage if the swarm is at half or fewer Hit Points.

\medskip\textbf{Swarm of Bats}\index[Monsters]{Swarm of Bats}

\textit{Medium Tiny Beast Swarm, unaligned}

\textbf{STRENGTH} -3

\textbf{DEXTERITY} +2

\textbf{CONSTITUTION} +0

\textbf{INTELLIGENCE} -4

\textbf{WISDOM} +1

\textbf{CHARISMA} -3

\textbf{Initiative} +2 -- \textbf{Defence} 13

\textbf{Hit Points} 22 (5d8)

\textbf{Move} 0m, fly 9m

\textbf{Saving Throws}: Fortitude -2, Reflexes +4, Will +2

\textbf{Damage Resistances} slashing, piercing, slashing

\textbf{Condition Immunity} charmed, grabbed, restrained, paralyzed, petrified, prone, frightened, stunned

\textbf{Senses} blindsight 18 m

\textbf{Languages} -

\textbf{Challenge} 1/4 (50 XP)

\textit{\textbf{Echolocation.}} The swarm cannot use blindsight when deafened.

\textit{\textbf{Swarm.}} The swarm can occupy another creature's space and vice versa, and the swarm can move through any opening large enough for a Tiny Bat. The swarm cannot regain Hit Points or gain temporary Hit Points.

\textit{\textbf{Honed hearing.}} The swarm has +1d6 on hearing-based Wisdom (Awareness) checks.

\textbf{Actions}

\textit{\textbf{Bites.} Melee Weapon Attack}: +4 to hit, reach 0m, one creature in the swarm's space.

\textit{Hit:} 5 (2d4) piercing damage, or 2 (1d4) piercing damage if the swarm is at half or fewer Hit Points.

\medskip\textbf{Swarm of Spiders}\index[Monsters]{Swarm of Spiders}

\textit{Medium Tiny Beast Swarm, unaligned}

\textbf{STRENGTH} -4

\textbf{DEXTERITY} +1

\textbf{CONSTITUTION} +0

\textbf{INTELLIGENCE} -5

\textbf{WISDOM} -2

\textbf{CHARISMA} -5

\textbf{Initiative} +1 -- \textbf{Defence} 13

\textbf{Hit Points} 22 (5d8)

\textbf{Movement} 6m, climb 6m

\textbf{Saving Throws}: Fortitude -3, Reflexes +2, Will -1

\textbf{Damage Resistances} slashing, piercing, slashing

\textbf{Condition Immunity} charmed, grabbed, restrained, paralyzed, petrified, prone, frightened, stunned

\textbf{Senses} blindsight 3 m

\textbf{Languages} -

\textbf{Challenge} 1/2 (100 XP)

\textit{\textbf{Web Walk.}} The swarm ignores movement restrictions caused by webs.

\textit{\textbf{Climb as Spider.}} The swarm can climb difficult surfaces, including standing upside down on ceilings, without needing to make an ability check.

\textit{\textbf{Web Sense.}} While in contact with a web, the swarm knows the exact location of any other creature in contact with the same web.

\textit{\textbf{Swarm.}} The swarm can occupy another creature's space and vice versa, and the swarm can move through any opening large enough for a Tiny insect. The swarm cannot regain Hit Points or gain temporary Hit Points.

\textbf{Actions}

\textit{\textbf{Bites.} Melee Weapon Attack}: +3 to hit, reach 0m, one target in the swarm's space.

\textit{Hit:} 10 (4d4) piercing damage, or 5 (2d4) piercing damage if the swarm is at half or fewer Hit Points.

\medskip\textbf{Swarm of Rats}\index[Monsters]{Swarm of Rats}

\textit{Medium Tiny Beast Swarm, unaligned}

\textbf{STRENGTH} -1

\textbf{DEXTERITY} +0

\textbf{CONSTITUTION} -1

\textbf{INTELLIGENCE} -4

\textbf{WISDOM} +0

\textbf{CHARISMA} -4

\textbf{Initiative} +0 -- \textbf{Defence} 11

\textbf{Hit Points} 24 (7d8 - 7)

\textbf{Move} 9m

\textbf{Saving Throws}: Fortitude +0, Reflexes +1, Will +1

\textbf{Damage Resistances} slashing, piercing, slashing

\textbf{Condition Immunity} charmed, grabbed, restrained, paralyzed, petrified, prone, frightened, stunned

\textbf{Senses} vision in the dark 9m

\textbf{Languages} -

\textbf{Challenge} 1/4 (50 XP)

\textit{\textbf{Enhanced sense of smell.}} The swarm has +1d6 on Wisdom (Awareness) checks based on smell.

\textit{\textbf{Swarm.}} The swarm can occupy another creature's space and vice versa, and the swarm can move through any opening large enough for a Tiny Rat. The swarm cannot regain Hit Points or gain temporary Hit Points.

\textbf{Actions}

\textit{\textbf{Bites.} Melee Weapon Attack}: +2 to hit, reach 0m, one target in the swarm's space.

\textit{Hit:} 7 (2d6) piercing damage, or 3 (1d6) piercing damage if the swarm is at half or fewer Hit Points.

\medskip\textbf{Swarm of Beetles}\index{Sciame di Scarabei}

\textit{Medium Tiny Beast Swarm, unaligned}

\textbf{STRENGTH} -4

\textbf{DEXTERITY} +1

\textbf{CONSTITUTION} +0

\textbf{INTELLIGENCE} -5

\textbf{WISDOM} -2

\textbf{CHARISMA} -5

\textbf{Initiative} +1 -- \textbf{Defence} 13

\textbf{Hit Points} 22 (5d8)

\textbf{Move} 6m, climb 6m, dig 6m

\textbf{Saving Throws}: Fortitude -3, Reflexes +2, Will -1

\textbf{Damage Resistances} slashing, piercing, slashing

\textbf{Condition Immunity} charmed, grabbed, restrained, paralyzed, petrified, prone, frightened, stunned

\textbf{Senses} blindsight 3 m

\textbf{Languages} -

\textbf{Challenge} 1/2 (100 XP)

\textit{\textbf{Swarm.}} The swarm can occupy another creature's space and vice versa, and the swarm can move through any opening large enough for a Tiny insect. The swarm cannot regain Hit Points or gain temporary Hit Points.

\textbf{Actions}

\textit{\textbf{Bites.} Melee Weapon Attack}: +3 to hit, reach 0m, one target in the swarm's space.

\textit{Hit:} 10 (4d4) piercing damage, or 5 (2d4) piercing damage if the swarm is at half or fewer Hit Points.

\medskip\textbf{Swarm of Venomous Snakes}\index[Monsters]{Sciame di Serpenti Velenosi}

\textit{Medium Tiny Beast Swarm, unaligned}

\textbf{STRENGTH} -1

\textbf{DEXTERITY} +4

\textbf{CONSTITUTION} +0

\textbf{INTELLIGENCE} -5

\textbf{WISDOM} +0

\textbf{CHARISMA} -4

\textbf{Initiative} +4 -- \textbf{Defence} 15

\textbf{Hit Points} 36 (8d8)

\textbf{Movement} 9m, swim 9m

\textbf{Saving Throws}: Fortitude +0, Reflexes +5, Will +1

\textbf{Damage Resistances} slashing, piercing, slashing

\textbf{Condition Immunity} charmed, grabbed, restrained, paralyzed, petrified, prone, frightened, stunned

\textbf{Senses} blindsight 3 m

\textbf{Languages} -

\textbf{Challenge} 2 (450 XP)

\textit{\textbf{Swarm.}} The swarm can occupy another creature's space and vice versa, and the swarm can move through any opening large enough for a Tiny snake. The swarm cannot regain Hit Points or gain temporary Hit Points.

\textbf{Actions}

\textit{\textbf{Bites.} Melee Weapon Attack}: +6 to hit, reach 0m, one creature in the swarm's space.

\textit{Hit:} 7 (2d6) piercing damage, or 3 (1d6) piercing damage if the swarm is at half its Hit Points or less, and the target must make a DC 10 Fortitude Saving Throw, and suffer 14 (4d6) poison damage on a failed save, or half as much damage on a successful one.

\medskip\textbf{Swarm of Wasps}\index[Monsters]{Sciame di Serpenti Velenosi}

\textit{Medium Tiny Beast Swarm, unaligned}

\textbf{STRENGTH} -4

\textbf{DEXTERITY} +1

\textbf{CONSTITUTION} +0

\textbf{INTELLIGENCE} -5

\textbf{WISDOM} -2

\textbf{CHARISMA} -5

\textbf{Initiative} +1 -- \textbf{Defence} 13

\textbf{Hit Points} 22 (5d8)

\textbf{Move} 1m, fly 9m

\textbf{Saving Throws}: Fortitude -3, Reflexes +2, Will -1

\textbf{Damage Resistances} slashing, piercing, slashing

\textbf{Condition Immunity} charmed, grabbed, restrained, paralyzed, petrified, prone, frightened, stunned

\textbf{Senses} blindsight 3 m

\textbf{Languages} -

\textbf{Challenge} 1/2 (100 XP)

\textit{\textbf{Swarm.}} The swarm can occupy another creature's space and vice versa, and the swarm can move through any opening large enough for a Tiny insect. The swarm cannot regain Hit Points or gain temporary Hit Points.

\textbf{Actions}

\textit{\textbf{Bites.} Melee Weapon Attack}: +3 to hit, reach 0m, one target in the swarm's space.

\textit{Hit:} 10 (4d4) piercing damage, or 5 (2d4) piercing damage if the swarm is at half or fewer Hit Points.


\medskip\textbf{Monkey}\index[Monsters]{Monkey}

\textit{Little beast, unaligned}

\textbf{STRENGTH} -3

\textbf{DEXTERITY} +2

\textbf{CONSTITUTION} +0

\textbf{INTELLIGENCE} -3

\textbf{WISDOM} +1

\textbf{CHARISMA} -2

\textbf{Initiative} +2 -- \textbf{Defence} 13

\textbf{Hit Points} 3 (1d6)

\textbf{Movement} 9m, climb 9m

\textbf{Saving Throws}: Fortitude +0, Reflexes +3, Will +1

\textbf{Skills} Acrobatics +6, Awareness +3

\textbf{Languages} -

\textbf{Challenge} 1/4 (10 XP)

\textbf{Actions}

\textit{\textbf{Scratch.} Melee weapon attack}: +1 to hit, reach 1m, one target.

\textit{Hit:} 1 (1d4 - 1) slashing damage.

\textit{\textbf{Bite.} Melee Weapon Attack}: +1 to hit, reach 1 meter, one target.

\textit{Hit:} 2 (1d4) piercing damage.

\medskip\textbf{Ape}\index[Monsters]{Ape}

\textit{Medium beast, unaligned}

\textbf{STRENGTH} +3

\textbf{DEXTERITY} +2

\textbf{CONSTITUTION} +2

\textbf{INTELLIGENCE} -2

\textbf{WISDOM} +1

\textbf{CHARISMA} -2

\textbf{Initiative} +2 -- \textbf{Defence} 13

\textbf{Hit Points} 19 (3d8 + 6)

\textbf{Movement} 9m, climb 9m

\textbf{Saving Throws}: Fortitude +3, Reflexes +3, Will +2

\textbf{Skills} Acrobatics +5, Awareness +3

\textbf{Languages} -

\textbf{Challenge} 1/2 (100 XP)

\textbf{Actions}

\textit{\textbf{Multiattack.}} The ape makes two punch attacks.

\textit{\textbf{Punch.} Melee Weapon Attack}: +5 to hit, reach 1m, one target.

\textit{Hit:} 6 (1d6 + 3) bludgeoning damage.

\textit{\textbf{Rock.} Ranged Weapon Attack}: +5 to hit, range 8m, one target.

\textit{Hit:} 6 (1d6 + 3) bludgeoning damage.

\medskip\textbf{Giant Ape}\index[Monsters]{Giant Ape}

\textit{Huge beast, unaligned}

\textbf{STRENGTH} +6

\textbf{DEXTERITY} +2

\textbf{CONSTITUTION} +4

\textbf{INTELLIGENCE} -2

\textbf{WISDOM} +1

\textbf{CHARISMA} -2

\textbf{Initiative} +2 -- \textbf{Defence} 16

\textbf{Hit Points} 157 (15d12 + 60)

\textbf{Movement} 12m, climb 12m

\textbf{Saving Throws}: Fortitude +7, Reflexes +6, Will +4

\textbf{Skills} Acrobatics +9, Awareness +4

\textbf{Languages} -

\textbf{Challenge} 7 (2900 XP)

\textbf{Actions}

\textit{\textbf{Multiattack.}} The ape makes two punch attacks.

\textit{\textbf{Punch.} Melee Weapon Attack}: +9 to hit, reach 3m, one target.

\textit{Hit:} 22 (3d10 + 6) bludgeoning damage.

\textit{\textbf{Rock.} Ranged Weapon Attack}: +9 to hit, range 15m, one target.

\textit{Hit:} 30 (7d6 + 6) bludgeoning damage.

\medskip\textbf{Scorpio}\index[Monsters]{Scorpio}

\textit{Tiny beast, unaligned}

\textbf{STRENGTH} -4

\textbf{DEXTERITY} +0

\textbf{CONSTITUTION} -1

\textbf{INTELLIGENCE} -5

\textbf{WISDOM} -1

\textbf{CHARISMA} -4

\textbf{Initiative} +0 -- \textbf{Defence} 12

\textbf{Hit Points} 1 (1d4 - 1)

\textbf{Move} 3m

\textbf{Saving Throws}: Fortitude -3, Reflexes +2, Will -1

\textbf{Senses} blindsight 3 m

\textbf{Languages} -

\textbf{Challenge} 0 (10 XP)

\textbf{Actions}

\textit{\textbf{Sting.} Melee Weapon Attack}: +2 to hit, reach 1 meter, one creature.

\textit{Hit:} 1 piercing damage and the target must make a DC 9 Fortitude save, taking 4 (1d8) poison damage on a failed save, or half as much damage on a successful one.

\medskip\textbf{Giant Scorpion}\index[Monsters]{Giant Scorpion}

\textit{Large beast, unaligned}

\textbf{STRENGTH} +2

\textbf{DEXTERITY} +1

\textbf{CONSTITUTION} +2

\textbf{INTELLIGENCE} -5

\textbf{WISDOM} -1

\textbf{CHARISMA} -4

\textbf{Initiative} +1 -- \textbf{Defence} 17

\textbf{Hit Points} 52 (7d10 + 14)

\textbf{Move} 12m

\textbf{Saving Throws}: Fortitude +7, Reflexes +1, Will +1

\textbf{Senses} blindsight 18 m

\textbf{Languages} -

\textbf{Challenge} 3 (700 XP)

\textbf{Actions}

\textit{\textbf{Multiattack.}} The scorpion makes three attacks: two with its claws and one with its sting.

\textit{\textbf{Claw.} Melee Weapon Attack}: +4 to hit, reach 1m, one target.

\textit{Hit:} 6 (1d8 + 2) bludgeoning damage and the target is grabbed (DC 12 to flee). The scorpion has two claws, each of which can only grip one target.

\textit{\textbf{Sting.} Melee Weapon Attack}: +4 to hit, reach 1 meter, one creature.

\textit{Hit:} 7 (1d10 + 2) piercing damage and the target must make a DC 12 Fortitude save, taking 22 (4d10) poison damage on a failed save, or half as much damage on a failed save. he succeeds.

\medskip\textbf{Constrictor Serpent}\index[Monsters]{Constrictor Serpent}

\textit{Large beast, unaligned}

\textbf{STRENGTH} +2

\textbf{DEXTERITY} +2

\textbf{CONSTITUTION} +1

\textbf{INTELLIGENCE} -5

\textbf{WISDOM} +0

\textbf{CHARISMA} -4

\textbf{Initiative} +2 -- \textbf{Defence} 13

\textbf{Hit Points} 13 (2d10 + 2)

\textbf{Movement} 9m, swim 9m

\textbf{Saving Throws}: Fortitude +3, Reflexes +2, Will +0

\textbf{Senses} blindsight 3 m

\textbf{Languages} -

\textbf{Challenge} 1/4 (50 XP)

\textbf{Actions}

\textit{\textbf{Bite.} Melee Weapon Attack}: +4 to hit, reach 3 ft., one creature.

\textit{Hit:} 5 (1d6 + 2) piercing damage.

\textit{\textbf{Constrict.} Melee Weapon Attack}: +4 to hit, reach 3 ft., one creature.

\textit{Hit:} 6 (1d8 + 2) bludgeoning damage, and the target is grappled (DC 14 to flee). Until the grab ends, the creature is restrained, and the snake can't constrict another target.

\medskip\textbf{Giant Constrictor Snake}\index[Monsters]{Giant Constrictor Snake}

\textit{Huge beast, unaligned}

\textbf{STRENGTH} +4

\textbf{DEXTERITY} +2

\textbf{CONSTITUTION} +1

\textbf{INTELLIGENCE} -5

\textbf{WISDOM} +0

\textbf{CHARISMA} -4

\textbf{Initiative} +2 -- \textbf{Defence} 13

\textbf{Hit Points} 60 (8d12 + 8)

\textbf{Movement} 9m, swim 9m

\textbf{Saving Throws}: Fortitude +3, Reflexes +2, Will +0

\textbf{Skills} Awareness +2

\textbf{Senses} blindsight 3 m

\textbf{Languages} -

\textbf{Challenge} 2 (450 XP)

\textbf{Actions}

\textit{\textbf{Bite.} Melee Weapon Attack}: +6 to hit, reach 3m, one creature.

\textit{Hit:} 11 (2d6 + 4) piercing damage.

\textit{\textbf{Constrict.} Melee Weapon Attack}: +6 to hit, reach 1 meter, one creature.

\textit{Hit:} 13 (2d8 + 4) bludgeoning damage, and the target is grappled (DC 16 to flee). Until the grab ends, the creature is restrained, and the snake can't constrict another target.

\medskip\textbf{Venomous Snake}\index[Monsters]{Venomous Snake}

\textit{Tiny beast, unaligned}

\textbf{STRENGTH} -4

\textbf{DEXTERITY} +3

\textbf{CONSTITUTION} +0

\textbf{INTELLIGENCE} -5

\textbf{WISDOM} +0

\textbf{CHARISMA} -4

\textbf{Initiative} +3 -- \textbf{Defence} 14

\textbf{Hit Points} 2 (1d4)

\textbf{Movement} 9m, swim 9m

\textbf{Saving Throws}: Fortitude +1, Reflexes +4, Will +1

\textbf{Senses} blindsight 3 m

\textbf{Languages} -

\textbf{Challenge} 1/8 (25 XP)

\textbf{Actions}

\textit{\textbf{Bite.} Melee Weapon Attack}: +5 to hit, reach 1m, one target.

\textit{Hit:} 1 piercing damage and the target must make a DC 10 Fortitude save, taking 5 (2d4) poison damage on a failed save, or half as much damage on a successful one.

\medskip\textbf{Giant Venomous Snake}\index[Monsters]{Giant Venomous Snake}

\textit{Medium beast, unaligned}

\textbf{STRENGTH} +0

\textbf{DEXTERITY} +4

\textbf{CONSTITUTION} +1

\textbf{INTELLIGENCE} -4

\textbf{WISDOM} +0

\textbf{CHARISMA} -4

\textbf{Initiative} +4 -- \textbf{Defence} 15

\textbf{Hit Points} 11 (2d8 + 2)

\textbf{Movement} 9m, swim 9m

\textbf{Saving Throws}: Fortitude +1, Reflexes +5, Will +2

\textbf{Skills} Awareness +2

\textbf{Senses} blindsight 3 m

\textbf{Languages} -

\textbf{Challenge} 1/4 (50 XP)

\textbf{Actions}

\textit{\textbf{Bite.} Melee Weapon Attack}: +6 to hit, reach 3m, one target.

\textit{Hit:} 6 (1d4 + 4) piercing damage and the target must make a DC 11 Fortitude Saving Throw, taking 10 (3d6) poison damage on a failed save, or half as much damage on a failed save. he succeeds.

\medskip\textbf{Flying Serpent}\index[Monsters]{Flying Serpent}

A flying serpent is a richly colored, winged serpent found in remote jungles.

\textit{Tiny beast, unaligned}

\textbf{STRENGTH} -3

\textbf{DEXTERITY} +4

\textbf{CONSTITUTION} +0

\textbf{INTELLIGENCE} -4

\textbf{WISDOM} +1

\textbf{CHARISMA} -3

\textbf{Initiative} +4 -- \textbf{Defence} 15

\textbf{Hit Points} 5 (2d4)

\textbf{Move} 9m, swim 9m, fly 18m

\textbf{Saving Throws}: Fortitude -2, Reflexes +5, Will +1

\textbf{Senses} blindsight 3 m

\textbf{Languages} -

\textbf{Challenge} 1/8 (25 XP)

\textit{\textbf{Flying.}} The snake does not provoke attacks of opportunity when it flies out of an enemy's reach.

\textbf{Actions}

\textit{\textbf{Bite.} Melee Weapon Attack}: +6 to hit, reach 1m, one target.

\textit{Hit:} 1 piercing damage plus 7 (3d4) poison damage.

\medskip\textbf{Hunter Shark}\index[Monsters]{Hunter Shark}

A hunter shark is 4 to 6 meters long and usually hunts alone in deeper waters.

\textit{Large beast, unaligned}

\textbf{STRENGTH} +4

\textbf{DEXTERITY} +1

\textbf{INSTITUTION} +2

\textbf{INTELLIGENCE} -5

\textbf{WISDOM} +0

\textbf{CHARISMA} -3

\textbf{Initiative} +1 -- \textbf{Defence} 13

\textbf{Hit Points} 45 (6d10 + 12)

\textbf{Movement} 0m, swim 12m

\textbf{Saving Throws}: Fortitude +4, Reflexes +2, Will +0

\textbf{Skills} Awareness +2

\textbf{Senses} blindsight 9 m

\textbf{Languages} -

\textbf{Challenge} 2 (450 XP)

\textit{\textbf{Blood Frenzy.}} The shark has +1d6 on melee attack rolls against any creature that is not at full Hit Points.

\textit{\textbf{Water Breathing.}} The shark can only breathe underwater.

\textbf{Actions}

\textit{\textbf{Bite.} Melee Weapon Attack}: +6 to hit, reach 1m, one target.

\textit{Hit:} 13 (2d8 + 4) piercing damage.

\medskip\textbf{Coral Shark}\index[Monsters]{Coral Shark}

Reef sharks are 2 to 3 meters long and live in shallower waters and along coral reefs.

\textit{Medium beast, unaligned}

\textbf{STRENGTH} +2

\textbf{DEXTERITY} +1

\textbf{CONSTITUTION} +1

\textbf{INTELLIGENCE} -5

\textbf{WISDOM} +0

\textbf{CHARISMA} -3

\textbf{Initiative} +1 -- \textbf{Defence} 13

\textbf{Hit Points} 22 (4d8 + 4)

\textbf{Movement} 0m, swim 12m

\textbf{Saving Throws}: Fortitude +2, Reflexes +2, Will +1

\textbf{Skills} Awareness +2

\textbf{Senses} blindsight 9 m

\textbf{Languages} -

\textbf{Challenge} 1/2 (100 XP)

\textit{\textbf{Water Breathing.}} The shark can only breathe underwater.

\textit{\textbf{Polding tactics.}} The shark has +1d6 on attack rolls against a creature if at least one of the shark's allies is within 1 meter of the creature and that ally isn't incapacitated.

\textbf{Actions}

\textit{\textbf{Bite.} Melee Weapon Attack}: +4 to hit, reach 3 ft., one target.

\textit{Hit:} 6 (1d8 + 2) piercing damage.

\medskip\textbf{Giant Shark}\index[Monsters]{Giant Shark}

The giant shark is 9 meters long and you come across it

normally only in the deepest oceans.

\textit{Huge beast, unaligned}

\textbf{STRENGTH} +6

\textbf{DEXTERITY} +0

\textbf{CONSTITUTION} +5

\textbf{INTELLIGENCE} -5

\textbf{WISDOM} +0

\textbf{CHARISMA} -3

\textbf{Initiative} +0 -- \textbf{Defence} 16

\textbf{Hit Points} 126 (11d12 + 55)

\textbf{Movement} 0m, swim 15m

\textbf{Saving Throws}: Fortitude +7, Reflexes +2, Will +1

\textbf{Skills} Awareness +3

\textbf{Senses} blindsight 18 m

\textbf{Languages} -

\textbf{Challenge} 5 (1800 XP)

\textit{\textbf{Blood Frenzy.}} The shark has +1d6 on attack rolls

melee against any creature that isn't at full Hit Points.

\textit{\textbf{Water Breathing.}} The shark can only breathe underwater.

\textbf{Actions}

\textit{\textbf{Bite.} Melee Weapon Attack}: +9 to hit, reach 1m, one target.

\textit{Hit:} 22 (3d10 + 6) piercing damage.

\medskip\textbf{Strige}\index[Monsters]{Strige}

This hideous monster looks like a cross between a large bat and an oversized mosquito. Its legs end in long pincers, and its long, needle-like proboscis slices through the air as it seeks to feed on the blood of living creatures.

\textit{Tiny beast, unaligned}

\textbf{STRENGTH} -3

\textbf{DEXTERITY} +3

\textbf{CONSTITUTION} +0

\textbf{INTELLIGENCE} -4

\textbf{WISDOM} -1

\textbf{CHARISMA} -2

\textbf{Initiative} +3 -- \textbf{Defence} 15

\textbf{Hit Points} 2 (1d4)

\textbf{Move} 3m, fly 12m

\textbf{Saving Throws}: Fortitude -3, Reflexes +4, Will -1

\textbf{Senses} vision in the dark 18 m

\textbf{Languages} -

\textbf{Challenge} 1/8 (25 XP)

\textbf{Actions}

\textit{\textbf{Drain Blood.} Melee Weapon Attack}: +5 to hit, reach 1m, one creature.

\textit{Hit:} 5 (1d4 + 3) piercing damage and the striga sticks to the target. While attacked, the striga does not attack. Instead, at the start of each striga's turn, the target loses 5 (1d4 + 3) Hit Points due to blood loss.

The striga can detach itself by expending 1 meter of movement. He does this automatically after draining 10 Hit Points from the target or upon the target's death. A creature, including the target, can use its action to detach the striga.

\medskip\textbf{Badger}\index[Monsters]{Badger}

\textit{Tiny beast, unaligned}

\textbf{STRENGTH} -3

\textbf{DEXTERITY} +0

\textbf{CONSTITUTION} +1

\textbf{INTELLIGENCE} -4

\textbf{WISDOM} +1

\textbf{CHARISMA} -3

\textbf{Initiative} +0 -- \textbf{Defence} 11

\textbf{Hit Points} 3 (1d4 + 1)

\textbf{Movement} 6m, digging 1m

\textbf{Saving Throws}: Fortitude -3, Reflexes +1, Will +1

\textbf{Senses} vision in the dark 9m

\textbf{Languages} -

\textbf{Challenge} 0 (10 XP)

\textit{\textbf{A keen sense of smell.}} The badger has +1d6 on Wisdom (Awareness) checks based on smell.

\textbf{Actions}

\textit{\textbf{Bite.} Melee Weapon Attack}: +2 to hit, reach 1m, one target.

\textit{Hit:} 1 piercing damage.

\medskip\textbf{Giant Badger}\index[Monsters]{Giant Badger}

\textit{Medium beast, unaligned}

\textbf{STRENGTH} +1

\textbf{DEXTERITY} +0

\textbf{CONSTITUTION} +2

\textbf{INTELLIGENCE} -4

\textbf{WISDOM} +1

\textbf{CHARISMA} -3

\textbf{Initiative} +0 -- \textbf{Defence} 11

\textbf{Hit Points} 13 (2d8 + 4)

\textbf{Movement} 9m, digging 3m

\textbf{Saving Throws}: Fortitude +2, Reflexes +1, Will +2

\textbf{Senses} vision in the dark 9m

\textbf{Languages} -

\textbf{Challenge} 1/4 (50 XP)

\textit{\textbf{A keen sense of smell.}} The badger has +1d6 on Wisdom (Awareness) checks based on smell.

\textbf{Actions}

\textit{\textbf{Multiattack.}} The badger makes two attacks: one with its bite and one with its claws.

\textit{\textbf{Claws.} Melee Weapon Attack}: +3 to hit, reach 1m, one target.

\textit{Hit:} 6 (2d4 + 1) slashing damage.

\textit{\textbf{Bite.} Melee Weapon Attack}: +3 to hit, reach 1m, one target.

\textit{Hit:} 4 (1d6 + 1) piercing damage.

\medskip\textbf{Tiger}\index[Monsters]{Tiger}

\textit{Large beast, unaligned}

\textbf{STRENGTH} +3

\textbf{DEXTERITY} +2

\textbf{CONSTITUTION} +2

\textbf{INTELLIGENCE} -4

\textbf{WISDOM} +1

\textbf{CHARISMA} -1

\textbf{Initiative} +2 -- \textbf{Defence} 13

\textbf{Hit Points} 37 (5d10 + 10)

\textbf{Move} 12m

\textbf{Saving Throws}: Fortitude +4, Reflexes +4, Will +2

\textbf{Skills} Stealth +6, Awareness +3

\textbf{Senses} vision in the dark 18m

\textbf{Languages} -

\textbf{Challenge} 1 (200 XP)

\textit{\textbf{Leap.}} If the tiger moves at least 6 meters directly towards a creature and hits it with a claw attack during the same turn, the target must succeed on a DC 13 Fortitude save or fall prone. If the target is prone, the tiger can make a bite attack against it as a bonus action.

\textit{\textbf{Enhanced sense of smell.}} The tiger has +1d6 on Wisdom (Awareness) checks based on smell.

\textbf{Actions}

\textit{\textbf{Claw.} Melee Weapon Attack}: +5 to hit, reach 1m, one target.

\textit{Hit:} 7 (1d8 + 3) slashing damage, 1 bleed damage.

\textit{\textbf{Bite.} Melee Weapon Attack}: +5 to hit, reach 1m, one target.

\textit{Hit:} 8 (1d10 + 3) piercing damage.

\medskip\textbf{Saber-toothed Tiger}\index[Monsters]{Saber-toothed Tiger}

\textit{Large beast, unaligned}

\textbf{STRENGTH} +4

\textbf{DEXTERITY} +2

\textbf{CONSTITUTION} +2

\textbf{INTELLIGENCE} -4

\textbf{WISDOM} +1

\textbf{CHARISMA} -1

\textbf{Initiative} +2 -- \textbf{Defence} 13

\textbf{Hit Points} 52 (7d10 + 14)

\textbf{Move} 12m

\textbf{Saving Throws}: Fortitude +5, Reflexes +3, Will +2

\textbf{Skills} Stealth +6, Awareness +3

\textbf{Languages} -

\textbf{Challenge} 2 (450 XP)

\textit{\textbf{Leap.}} If the tiger moves at least 6 meters directly towards a creature and hits it with a claw attack during the same turn, the target must succeed on a DC 14 Fortitude save or fall prone. If the target is prone, the tiger can make a bite attack against it as a bonus action.

\textit{\textbf{Enhanced sense of smell.}} The tiger has +1d6 on Wisdom (Awareness) checks based on smell.

\textbf{Actions}

\textit{\textbf{Claw.} Melee Weapon Attack}: +6 to hit, reach 1m, one target.

\textit{Hit:} 12 (2d6 + 5) slashing damage, 1 bleed damage.

\textit{\textbf{Bite.} Melee Weapon Attack}: +6 to hit, reach 1m, one target.

\textit{Hit:} 10 (1d10 + 5) piercing damage.

\medskip\textbf{Giant Wasp}\index[Monsters]{Giant Wasp}

\textit{Medium beast, unaligned}

\textbf{STRENGTH} +0

\textbf{DEXTERITY} +2

\textbf{CONSTITUTION} +0

\textbf{INTELLIGENCE} -5

\textbf{WISDOM} +0

\textbf{CHARISMA} -4

\textbf{Initiative} +2 -- \textbf{Defence} 13

\textbf{Hit Points} 13 (3d8)

\textbf{Move} 3m, fly 15m

\textbf{Saving Throws}: Fortitude +1, Reflexes +3, Will +0

\textbf{Languages} -

\textbf{Challenge} 1/2 (100 XP)

\textbf{Actions}

\textit{\textbf{Sting.} Melee Weapon Attack}: +4 to hit, reach 1 meter, one creature.

\textit{Hit:} 5 (1d6 + 2) piercing damage and the target must make a DC 11 Fortitude Saving Throw, taking 10 (3d6) poison damage on a failed save, or half as much damage on a failed save. he succeeds. If the poison damage reduces the target to 0 Hit Points, the target is stable but poisoned for 1 hour, even after regaining Hit Points, and while poisoned in this way becomes paralysed.

\medskip\textbf{Worg}\index[Monsters]{Worg}

Worgs are monstrous wolf-like predators who love to hunt and devour creatures weaker than themselves.

\textit{Large Monstrosity, Neutral Evil}

\textbf{STRENGTH} +3

\textbf{DEXTERITY} +1

\textbf{CONSTITUTION} +1

\textbf{INTELLIGENCE} -2

\textbf{WISDOM} +0

\textbf{CHARISMA} -1

\textbf{Initiative} +1 -- \textbf{Defence} 14

\textbf{Hit Points} 26 (4d10 + 4)

\textbf{Move} 15m

\textbf{Saving Throws}: Fortitude +3, Reflexes +2, Will +2

\textbf{Skills} Awareness +4

\textbf{Senses} vision in the dark 18m

\textbf{Languages} Goblin, Worg

\textbf{Challenge} 1/2 (100 XP)

\textit{\textbf{Hearing and keen sense of smell.}} The worg has +1d6 on Wisdom (Awareness) checks based on hearing or smell.

\textbf{Actions}

\textit{\textbf{Bite.} Melee Weapon Attack}: +5 to hit, reach 1m, one target.

\textit{Hit:} 10 (2d6 + 3) piercing damage. If the target is a creature, it must succeed on a DC 13 Fortitude save or fall prone.

\subsection{Appendix B: Non-Player Characters}\index[Monsters]{Non-Player Characters}

This appendix contains statistics for various humanoid non-player characters (NPCs) that adventurers may encounter during the course of a campaign, from lowly commoners to powerful archwizards. These stats can be used to represent human and non-human NPCs.

Customize NPCs

There are several easy ways to customize the NPCs in this addendum for use in your home campaign.

\textit{\textbf{Change Enchantments.}} One way to customize an NPC spellcaster is to replace one or more of his enchantments. You can replace any spell from the list of
NPC spells with a different spell of the same level. Changing spells this way does not change the NPC's challenge rating.

\textbf{\textit{Change Weapons and Armour}.} You can improve or worsen the NPC's Armour or add or change weapons. Defence and damage changes can change the NPC's challenge rating.

\textit{\textbf{Magic Items}}. The more powerful an NPC is, the more likely he is to possess one or more magical items. A wizard, for example, might have a magical wand or staff, as well as one or more potions and scrolls. Equipping an NPC with a powerful magic item capable of dealing damage might change their challenge rating.

Some sample magic items are described later in this document.

\textbf{Fighters}

Fighters are individuals who make a living by putting their sword in the service of an individual or an ideal.

\medskip\textbf{Guard}

Guards include members of the city watch, sentries of a citadel or fortified city, and the bodyguards of nobles and merchants.

\textit{Medium humanoid (any race), any Trait}

\textbf{STRENGTH} +1

\textbf{DEXTERITY} +1

\textbf{CONSTITUTION} +1

\textbf{INTELLIGENCE} +0

\textbf{WISDOM} +0

\textbf{CHARISMA} +0

\textbf{Initiative} +1 -- \textbf{Defence} 17 (mail coat, shield)

\textbf{Hit Points} 11 (2d8 + 2)

\textbf{Move} 9m

\textbf{Saving Throws}: Fortitude +3, Reflexes +1, Will +1

\textbf{Skills} Awareness +2

\textbf{Languages} any language (usually Common)

\textbf{Challenge} 1/8 (25 XP)

\textbf{Actions}

\textit{\textbf{Spear.} Melee or Ranged Weapon Attack}: +3 to hit, reach 1m or range 6m, one target.

\textit{Hit:} 4 (1d6 + 1) piercing damage, or 5 (1d8 + 1) piercing damage when used with two hands to make a melee attack.

\medskip\textbf{Veteran}

Warriors who survived for a long time, gaining a great reputation as expert and skilled fighters.

\textit{Medium humanoid (any race), any Trait}

\textbf{STRENGTH} +3

\textbf{DEXTERITY} +1

\textbf{CONSTITUTION} +2

\textbf{INTELLIGENCE} +0

\textbf{WISDOM} +0

\textbf{CHARISMA} +0

\textbf{Initiative} +1 -- \textbf{Defence} 19 (Striped Armour)

\textbf{Hit Points} 58 (9d8 + 18)

\textbf{Move} 9m

\textbf{Saving Throws}: Fortitude +4, Reflexes +2, Will +3

\textbf{Skills} Acrobatics +5, Awareness +2

\textbf{Languages} any language (usually Common)

\textbf{Challenge} 3 (700 XP)

\textbf{Actions}

\textit{\textbf{Multiattack.}} The veteran makes two attacks with the longsword. If he has drawn a short sword, he can also make a short sword attack.

\textit{\textbf{Longsword.} Melee Weapon Attack}: +5 to hit, reach 1m, one target.

\textit{Hit:} 7 (1d8 + 3) slashing damage, or 8 (1d10 + 3) slashing damage when used with two hands.

\textit{\textbf{Short sword.} Melee Weapon Attack}: +5 to hit, reach 1m, one target.

\textit{Hit:} 6 (1d6 + 3) piercing damage.

\textit{\textbf{Heavy Crossbow.} Ranged Weapon Attack}: +3 to hit, range 30m, one target. \textit{Hit:} 6 (1d10 + 1) piercing damage.

\medskip\textbf{Knight}

Knights are fighters who swear allegiance to rulers, religious orders, and noble causes. The Knight's Traits determine to what extent he is willing to honor his oath.

\textit{Medium humanoid (any race), any Trait}

\textbf{STRENGTH} +3

\textbf{DEXTERITY} +0

\textbf{CONSTITUTION} +2

\textbf{INTELLIGENCE} +0

\textbf{WISDOM} +0

\textbf{CHARISMA} +2

\textbf{Initiative} +0 -- \textbf{Defence} 20 (plate Armour)

\textbf{Hit Points} 52 (8d8 + 16)

\textbf{Move} 9m

\textbf{Saving Throws}: Fortitude +4, Reflexes +1, Will +3

\textbf{Languages} any language (usually Common)

\textbf{Challenge} 3 (700 XP)

\textit{\textbf{Brave.}} The rider has +1d6 on Saving Throws against the frightened being.

\textbf{Actions}

\textit{\textbf{Multiattack.}} The knight makes two melee attacks.

\textit{\textbf{Great sword.} Melee Weapon Attack}: +5 to hit, reach 1m, one target.

\textit{Hit:} 10 (2d6 + 3) slashing damage.

\textit{\textbf{Heavy Crossbow.} Ranged Weapon Attack}: +2 to hit, range 30m, one target.

\textit{Hit:} 5 (1d10) piercing.

\textit{\textbf{Authority (Recharge after 1 hour)}}. For 1 minute, the cavalier can issue a special command or warning whenever a nonhostile creature within 10 meters of him that he can see makes an attack roll or Saving Throw. The creature can add a d4 to his roll as long as it can hear and understand the rider. A creature can benefit from only one Leadership die at a time. This effect ends if the knight is incapacitated.

\textbf{Reactions}

\textit{\textbf{Parry.}} The knight can add 2 to his Defence against a melee attack that would hit him. To do so, the cavalier must see the attacker and be wielding a melee weapon.

\medskip\textbf{Gladiator}

Trained to entertain crowds, they are some of the most dangerous fighters around.

\textit{Medium humanoid (any race), any Trait}

\textbf{STRENGTH} +4

\textbf{DEXTERITY} +2

\textbf{CONSTITUTION} +3

\textbf{INTELLIGENCE} +0

\textbf{WISDOM} +1

\textbf{CHARISMA} +2

\textbf{Initiative} +2 -- \textbf{Defence} 19 (studded leather Armour, shield)

\textbf{Hit Points} 112 (15d8 + 45)

\textbf{Move} 9m

\textbf{Saving Throws}: Fortitude +5, Reflexes +5, Will +3

\textbf{Skills} Acrobatics +10, Intimidate +5

\textbf{Languages} any language (usually Common)

\textbf{Challenge} 5 (1800 XP)

\textit{\textbf{Brute.}} Melee weapon deals an additional die of damage

when a gladiator hits with it (already included in the attack).

\textit{\textbf{Brave.}} The gladiator has +1d6 on Saving Throws against the frightened being.

\textbf{Actions}

\textit{\textbf{Multiattack.}} The gladiator makes three melee attacks or two ranged attacks.

\textit{\textbf{Spear.} Melee or Ranged Weapon Attack}: +7 to hit, reach 1m or range 6m, one target.

\textit{Hit:} 11 (2d6 + 4) piercing damage, or 13 (2d8 + 4) slashing damage when used with two hands.

\textit{\textbf{Shield Smash.} Melee Weapon Attack}: +7 to hit, reach 1m, one target.

\textit{Hit:} 9 (2d4 + 4) bludgeoning damage. If the target is a Medium or smaller creature, it must succeed on a DC 15 Fortitude save or be knocked prone.

\textbf{Reactions}

\textit{\textbf{Parry.}} The gladiator adds 3 to his Defence against a melee attack that would hit him. To do so, the gladiator must see the attacker and wield a melee weapon.

\medskip\textbf{Citizens}

This category includes those individuals who are concerned with running the world, carrying out the tasks necessary for the fields to be cultivated, the cities to be administered, the food to be grown and
new territories explored.

\medskip\textbf{Noble}

Nobles rule over the populace, by virtue of a birthright or accumulated wealth. Among them are also the courtiers who crowd the courts of the rich and powerful.

\textit{Medium humanoid (any race), any Trait}

\textbf{STRENGTH} +0

\textbf{DEXTERITY} +1

\textbf{CONSTITUTION} +0

\textbf{INTELLIGENCE} +1

\textbf{WISDOM} +2

\textbf{CHARISMA} +3

\textbf{Initiative} +1 -- \textbf{Defence} 16 (bib)

\textbf{Hit Points} 9 (2d8)

\textbf{Move} 9m

\textbf{Saving Throws}: Fortitude +1, Reflexes +1, Will +2

\textbf{Skills} Sense Emotions +4, Deceive +5

\textbf{Languages} any two languages

\textbf{Challenge} 1/8 (25 XP)

\textbf{Actions}

\textit{\textbf{rapier.} Melee Weapon Attack}: +3 to hit, reach 1m, one target.

\textit{Hit:} 5 (1d8 + 1) piercing damage.

\textbf{Reactions}

\textit{\textbf{Parry.}} The noble adds 2 to his Defence against a melee attack that would hit him. To do this, the noble must see

the attacker and wield a melee weapon.

\medskip\textbf{commoner}

Commoners include peasants, servants, slaves, retainers, pilgrims, merchants, artisans, and hermits.

\textit{Medium humanoid (any race), any Trait}

\textbf{STRENGTH} +0

\textbf{DEXTERITY} +0

\textbf{CONSTITUTION} +0

\textbf{INTELLIGENCE} +0

\textbf{WISDOM} +0

\textbf{CHARISMA} +0

\textbf{Initiative} +0 -- \textbf{Defence} 11

\textbf{Hit Points} 4 (1d8)

\textbf{Move} 9m

\textbf{Saving Throws}: Fortitude +0, Reflexes +0, Will +0

\textbf{Languages} any language (usually Common)

\textbf{Challenge} 0 (10 XP)

\textbf{Actions}

\textit{\textbf{Cudgeon.} Melee Weapon Attack}: +2 to hit, reach 1m, one target.

\textit{Hit:} 2 (1d4) bludgeoning damage.

\medskip\textbf{Criminals}

Criminals are individuals who live on the edge of the law, earning their bread by carrying out activities often considered illegal and immoral.

\medskip\textbf{Thunderman}

Thugs are ruthless criminals skilled at intimidating and perpetrating acts of violence. They work for money and have few scruples.

\textit{Medium humanoid (any race), any Trait}

\textbf{STRENGTH} +2

\textbf{DEXTERITY} +0

\textbf{CONSTITUTION} +2

\textbf{INTELLIGENCE} +0

\textbf{WISDOM} +0

\textbf{CHARISMA} +0

\textbf{Initiative} +0 -- \textbf{Defence} 12 (leather Armour)

\textbf{Hit Points} 32 (5d8 + 10)

\textbf{Move} 9m

\textbf{Saving Throws}: Fortitude +3, Reflexes +1, Will +0

\textbf{Skills} Intimidation +2

\textbf{Languages} any language (usually Common)

\textbf{Challenge} 1/2 (100 XP)

\textit{\textbf{Pack tactics.}} The thug has +1d6 on attack rolls against a creature if at least one of the thug's allies is within 1 meter of the creature and that ally is not
he is incapacitated.

\textbf{Actions}

\textit{\textbf{Multiattack.}} The bruiser makes two melee attacks.

\textit{\textbf{Mace.} Melee Weapon Attack}: +4 to hit, reach 3 ft., one creature.

\textit{Hit:} 5 (1d6 + 2) bludgeoning damage.

\textit{\textbf{Heavy Crossbow.} Ranged Weapon Attack}: +2 to hit, range 30m, one target. \textit{Hit:} 5 (1d10) piercing damage.

\medskip\textbf{Bandit/Pirate}

Whether they are men of the street or of the sea (pirates), they earn a living by plundering others.

\textit{Medium humanoid (any race), any illegal Trait}

\textbf{STRENGTH} +0

\textbf{DEXTERITY} +1

\textbf{CONSTITUTION} +1

\textbf{INTELLIGENCE} +0

\textbf{WISDOM} +0

\textbf{CHARISMA} +0

\textbf{Initiative} +1 -- \textbf{Defence} 13 (leather Armour)

\textbf{Hit Points} 11 (2d8 + 2)

\textbf{Move} 9m

\textbf{Saving Throws}: Fortitude +1, Reflexes +2, Will +1

\textbf{Languages} any language (usually Common)

\textbf{Challenge} 1/8 (25 XP)

\textbf{Actions}

\textit{\textbf{Scimitar.} Melee Weapon Attack}: +3 to hit, reach 1m, one target.

\textit{Hit:} 4 (1d6 + 1) slashing damage.

\textit{\textbf{Light crossbow.} Ranged Weapon Attack}: +3 to hit, range 24m, one target. \textit{Hit:} 5 (1d8 + 1) slashing damage.

\medskip\textbf{Spy}

A spy is an individual trained to obtain secrets for someone, or sometimes to sell them to the highest bidder.

\textit{Medium humanoid (any race), any Trait}

\textbf{STRENGTH} +0

\textbf{DEXTERITY} +2

\textbf{CONSTITUTION} +0

\textbf{INTELLIGENCE} +1

\textbf{WISDOM} +2

\textbf{CHARISMA} +3

\textbf{Initiative} +2 -- \textbf{Defence} 13

\textbf{Hit Points} 27 (6d8)

\textbf{Move} 9m

\textbf{Saving Throws}: Fortitude +2, Reflexes +3, Will +3

\textbf{Skills} Stealth +4, Sense Emotions +4, Investigation +5, Awareness +6, Deceive +5, Fey Hands +4

\textbf{Languages} any two languages

\textbf{Challenge} 1 (200 XP)

\textit{\textbf{Sneak Attack (1/turn).}} The spy deals an additional 7 (2d6) damage when hitting a target with a weapon attack and has +1d6 on the attack roll, or when the target is within 1 meter of an ally of the assassin that isn't incapacitated and the assassin doesn't have a -1d6 to attack roll.

\textit{\textbf{Cunning Action.}} During each of his rounds, the spy can use a bonus action to take the Retreat, Hide, or Dash action.

\textbf{Actions}

\textit{\textbf{Multiattack.}} The spy makes two melee attacks.

\textit{\textbf{Short sword.} Melee Weapon Attack}: +4 to hit, reach 1m, one target.

\textit{Hit:} 5 (1d6 + 2) piercing damage.

\textit{\textbf{Hand crossbow.} Ranged Weapon Attack}: +4 to hit, range 9m, one target. \textit{Hit:} 5 (1d6 + 2) piercing damage.


\medskip\textbf{Bandit Captain/Pirate}

Whether he lives on land or at sea, he is an individual with a great personality who can keep the rabble who do his bidding in line.

\textit{Medium humanoid (any race), any illegal Trait}

\textbf{STRENGTH} +2

\textbf{DEXTERITY} +3

\textbf{CONSTITUTION} +2

\textbf{INTELLIGENCE} +2

\textbf{WISDOM} +0

\textbf{CHARISMA} +2

\textbf{Initiative} +2 -- \textbf{Defence} 16 (studded leather Armour)

\textbf{Hit Points} 65 (10d8 + 8)

\textbf{Move} 9m

\textbf{Saving Throws}: Fortitude +5, Reflexes +5, Will +3

\textbf{Skills} Acrobatics +4, Raggiro +4

\textbf{Languages} any two languages

\textbf{Challenge} 2 (450 XP)

\textbf{Actions}

\textit{\textbf{Multiattack.}} The captain makes three melee attacks: two with his scimitar and one with his dagger. Or the captain makes two dagger attacks within range.

\textit{\textbf{Scimitar.} Melee Weapon Attack}: +5 to hit, reach 1m, one target.

\textit{Hit:} 6 (1d6 + 3) slashing damage.

\textit{\textbf{Dagger.} Melee or Ranged Weapon Attack}: +5 to hit, reach 1m or range 6m, one target. \textit{Hit:} 5 (1d4 + 3) piercing damage.

\textbf{Reactions}

\textit{\textbf{Block.}} The captain adds 2 to his Defence against a melee attack that would hit him. To do so, the captain must see the attacker and wield a melee weapon.

\medskip\textbf{Murderer}

Loners or members of a guild, assassins are paid to eliminate, often quietly and discreetly, their employers' rivals and enemies.

\textit{Medium humanoid (any race), any Trait not good}

\textbf{STRENGTH} +0

\textbf{DEXTERITY} +3

\textbf{CONSTITUTION} +2

\textbf{INTELLIGENCE} +1

\textbf{WISDOM} +0

\textbf{CHARISMA} +0

\textbf{Initiative} +3 -- \textbf{Defence} 19 (studded leather Armour)

\textbf{Hit Points} 78 (12d8 + 24)

\textbf{Move} 9m

\textbf{Saving Throws}: Fortitude +10, Reflexes +11, Will +8

\textbf{Skills} Acrobatics +6, Stealth +9, Awareness +3, Deceit +3


\textbf{Languages} Thieves' slang plus two other languages

\textbf{Challenge} 8 (3900 XP)

\textit{\textbf{Assassinate.}} On his first turn, the assassin has +1d6 on attack rolls against creatures that have not yet taken any rounds. Any hit the assassin lands against a surprised creature is a critical roll.

\textit{\textbf{Sneak Attack (1/turn).}} The assassin deals an additional 14 (4d6) damage when hitting a target with a weapon attack and has +1d6 on the attack roll, or when the target is within 1 meter of an ally of the assassin that isn't incapacitated, and the assassin has no -1d6 to attack roll.

\textit{\textbf{Evasion.}} If the assassin is the victim of an effect that allows him to make a Reflex Saving Throw for half damage, the assassin takes no damage on a successful save, and only the half if it fails.

\textbf{Actions}

\textit{\textbf{Multiattack.}} The assassin makes two attacks with his short swords.

\textit{\textbf{Short sword.} Melee Weapon Attack}: +6 to hit, reach 1m, one target.

\textit{Hit:} 6 (1d6 + 3) piercing damage, and the target must make a DC 15 Fortitude save, taking 24 (7d6) poison damage on a failed save, or half as much damage on a failed save. he succeeds.

\textit{\textbf{Light crossbow.} Ranged Weapon Attack}: +6 to hit, range 24m, one target.

\textit{Hit:} 7 (1d8 + 3) piercing damage, and the target must make a DC 15 Fortitude Saving Throw, taking 24 (7d6) poison damage on a failed save, or half as much damage on a failed save. he succeeds.

\medskip\textbf{Wizard}

The magician spends his life studying and practicing magic.

\textbf{VARIANT: FAMILIES}

Any spellcaster who can cast the spell \textit{find} \textit{familiar} is likely to have a familiar. The familiar can be one of the creatures described in the spell (see \textit{Basic Rules}) or some other Tiny monster, such as a slithering claw, imp, pseudodragon, or imp.

\medskip\textbf{Mage Adventurer}

A novice Wizard, who has successfully passed his first adventures and has begun to establish a reputation as a noble or notorious adventurer.

\textit{Medium humanoid (any race), any evil}

\textbf{STRENGTH} -1

\textbf{DEXTERITY} +2

\textbf{CONSTITUTION} +0

\textbf{INTELLIGENCE} +3

\textbf{WISDOM} +1

\textbf{CHARISMA} +0

\textbf{Initiative} +3 -- \textbf{Defence} 13

\textbf{Hit Points} 22 (5d8)

\textbf{Move} 9m

\textbf{Saving Throws}: Fortitude +0, Reflexes +3, Will +2

\textbf{Skills} Arcanum +5, Story +5

\textbf{Languages} any four languages

\textbf{Challenge} 1 (200 XP)

\textit{\textbf{Spells.}} The wizard has MP 4. His spellcasting ability is Intelligence (+5 on hit with spell attacks). The Wizard has the following spells prepared: Cantrips (at will):

\textit{light, mage hand, Shocking Grasp}

level 1 (4 slots): \textit{charm person, Arcane Dart}

level 2 (3 slots): \textit{block person, treadmill}

\textbf{Actions}

\textit{\textbf{Staff.} Melee Weapon Attack}: +1 to hit, reach 1m, one target.

\textit{Hit:} 3 (1d8 - 1) bludgeoning damage.

\medskip\textbf{Great Wizard}

A Magician who has established a good reputation in the area and who attracts students from all over the world.

\textit{Medium humanoid (any race), any Trait}

\textbf{STRENGTH} -1

\textbf{DEXTERITY} +2

\textbf{CONSTITUTION} +0

\textbf{INTELLIGENCE} +3

\textbf{WISDOM} +1

\textbf{CHARISMA} +0

\textbf{Initiative} +3 -- \textbf{Defence} 15 (18 with \textit{Mage's Armour})

\textbf{Hit Points} 40 (9d8)

\textbf{Move} 9m

\textbf{Saving Throws}: Fortitude +1, Reflexes +4, Will +3

\textbf{Skills} Arcanum +6, Story +6

\textbf{Languages} any four languages

\textbf{Challenge} 6 (2300 XP)

\textit{\textbf{Spells.}} The wizard has MP 9. His spellcasting ability is Intelligence (+6 on hit with spell attacks). The wizard has prepared the following spells:

Cantrips (at-will): \textit{fire bolt, light, mage hand}
\textit{prestidigitation}

level 1 (4 slots): \textit{Mage's Armour, Arcane Dart,}
\textit{detect magic, shield}

level 2 (3 slots): \textit{stealth, suggestion}

level 3 (3 slots): \textit{counterspell, fireball, Fly}

level 4 (3 slots): \textit{greater invisibility, ice storm}

level 5 (1 slot): \textit{cone of cold}

\textbf{Actions}

\textit{\textbf{Dagger.} Melee or Ranged Weapon Attack}: +5 to hit, reach 1m or range 6m, one target. \textit{Hit:} 4 (1d4 + 2) piercing damage.

\medskip\textbf{Archmage}

A very powerful (and also very old) wizard who studies the secrets of the multiverse.

\textit{Medium humanoid (any race), any Trait}

\textbf{STRENGTH} +0

\textbf{DEXTERITY} +2

\textbf{CONSTITUTION} +1

\textbf{INTELLIGENCE} +5

\textbf{WISDOM} +2

\textbf{CHARISMA} +3

\textbf{Initiative} +5 -- \textbf{Defence} 18 (21 with \textit{Mage's Armour})

\textbf{Hit Points} 99 (18d8 + 18)

\textbf{Move} 9m

\textbf{Saving Throws}: Fortitude +13, Reflexes +14, Will +14

\textbf{Skills} Arcanum +13, History +13

\textbf{Damage Resistances} spell damage; nonmagical bludgeoning, piercing, and slashing (from \textit{stoneskin})

\textbf{Languages} any six languages

\textbf{Challenge} 12 (8400 XP)

\textit{\textbf{Spells.}} The wizard has MP 18. His spellcasting ability is Intelligence (+9 on hit with spell attacks).

The archmage can cast \textit{disguise self} and \textit{invisibility} at will and has the following spells prepared: Cantrips (at will): \textit{fire bolt, light, mage hand}
\textit{prestidigitation, Shocking Grasp}

level 1 (4 slots): \textit{magic Armour*, Arcane Dart, identify, detect magic}

level 2 (3 slots): \textit{mirror image, detect thoughts, veiled step}

level 3 (3 slots): \textit{counterspell, lightning bolt}

level 4 (3 slots): \textit{banishment, stoneskin*, fire shield}

level 5 (3 slots): \textit{cone of cold, wall of force, scrying}

level 6 (1 slot): \textit{Orb of Invulnerability}

level 7 (1 slot): \textit{teleport}

level 8 (1 slot): \textit{Mind Shield*}

level 9 (1 slot): \textit{Time Stop}

The archmage casts these{*} spells on himself before combat.

\textbf{Actions}

\textit{\textbf{Dagger.} Melee or Ranged Weapon Attack}: +6 to hit, reach 1m or range 6m, one target. \textit{Hit:} 4 (1d4 + 2) piercing damage.


\medskip\textbf{Priests}

Priests are devotees of a deity or faith who care to impart divine teachings to their flock.

\medskip\textbf{Cultist}

Cultists swear allegiance to dark powers, and often show signs of madness in their beliefs and practices.

\textit{Medium humanoid (any race), any Trait not good}

\textbf{STRENGTH} +0

\textbf{DEXTERITY} +1

\textbf{CONSTITUTION} +0

\textbf{INTELLIGENCE} +0

\textbf{WISDOM} +0

\textbf{CHARISMA} +0

\textbf{Initiative} +0- \textbf{Defence} 13 (leather Armour)

\textbf{Hit Points} 9 (2d8)

\textbf{Move} 9m

\textbf{Saving Throws}: Fortitude +1, Reflexes +1, Will +2

\textbf{Skills} Deceit +2, Religion +2

\textbf{Languages} any language (usually Common)

\textbf{Challenge} 1/8 (25 XP)

\textit{\textbf{Dark Devotion.}} The cultist has +1d6 on Saving Throws against being charmed or frightened.

\textbf{Actions}

\textit{\textbf{Scimitar.} Melee Weapon Attack}: +3 to hit, reach 3 ft., one creature.

\textit{Hit:} 4 (1d6 + 1) slashing damage.

\medskip\textbf{Acolyte}

Acolytes are junior-ranking members of the clergy, and usually answerable to a higher-ranking priest. They perform various functions in a temple and are granted the ability to cast minor spells by their deity.

\textit{Medium humanoid (any race), any Trait}

\textbf{STRENGTH} +0

\textbf{DEXTERITY} +0

\textbf{CONSTITUTION} +0

\textbf{INTELLIGENCE} +0

\textbf{WISDOM} +2

\textbf{CHARISMA} +0

\textbf{Initiative} +0 -- \textbf{Defence} 11

\textbf{Hit Points} 9 (2d8)

\textbf{Move} 9m

\textbf{Saving Throws}: Fortitude +0, Reflexes +0, Will +3

\textbf{Skills} First Aid +4, Religion +2

\textbf{Languages} any language (usually Common)

\textbf{Challenge} 1/4 (50 XP)

\textit{\textbf{Spells.}} The acolyte has MP 1. His spellcasting ability is Wisdom (+4 on hitting with spell attacks). The acolyte has the following spells prepared: Cantrips (at will): \textit{holy flame, light, thaumaturgy} level 1 (3 slots): \textit{blessing}, \textit{cure wounds, sanctuary}

\medskip\textbf{Actions}

\textit{\textbf{Cudgeon.} Melee Weapon Attack}: +2 to hit, reach 1m, one target.

\textit{Hit:} 2 (1d4) bludgeoning damage.

\textbf{Cult Fanatic}

They are the leaders of a cult, using their charisma and dogmas to influence the weak of will.

\textit{Medium humanoid (any race), any Trait not good}

\textbf{STRENGTH} +0

\textbf{DEXTERITY} +2

\textbf{CONSTITUTION} +1

\textbf{INTELLIGENCE} +0

\textbf{WISDOM} +1

\textbf{CHARISMA} +2

\textbf{Initiative} +2 -- \textbf{Defence} 14 (leather Armour)

\textbf{Hit Points} 33 (6d8 + 6)

\textbf{Move} 9m

\textbf{Saving Throws}: Fortitude +2, Reflexes +2, Will +3

\textbf{Skills} Deceit +4, Deceit +4, Religion +2

\textbf{Languages} any language (usually Common)

\textbf{Challenge} 2 (450 XP)

\textit{\textbf{Spells.}} The priest has MP 4. His spellcasting ability is Wisdom (+3 on hitting with spell attacks). The priest has the following spells prepared: Cantrips (at will): \textit{holy flame, light, thaumaturgy}

level 1 (4 slots): \textit{command, inflict wounds, shield of faith}

level 2 (3 slots): \textit{spiritual weapon, hold person}

\textit{\textbf{Dark Devotion.}} The cultist has +1d6 on Saving Throws against being charmed or frightened.

\textbf{Actions}

\textit{\textbf{Multiattack.}} The fanatic makes two melee attacks.

\textit{\textbf{Dagger.} Melee or Ranged Weapon Attack}: +4 to hit, reach 1m or range 6m, a creature. \textit{Hit:} 4 (1d4 + 2) piercing damage.

\medskip\textbf{High Priest}

They are individuals in command of a temple or other sacred place and who have several acolytes at their disposal.

\textit{Medium humanoid (any race), any Trait}

\textbf{STRENGTH} +0

\textbf{DEXTERITY} +0

\textbf{CONSTITUTION} +1

\textbf{INTELLIGENCE} +1

\textbf{WISDOM} +3

\textbf{CHARISMA} +1

\textbf{Initiative} +1 -- \textbf{Defence} 14 (mail shirt)

\textbf{Hit Points} 27 (5d8 + 5)

\textbf{Move} 7m

\textbf{Saving Throws}: Fortitude +1, Reflexes +1, Will +4

\textbf{Skills} First Aid +7, Deceive +3, Religion +4

\textbf{Languages} any two languages

\textbf{Challenge} 2 (450 XP)

\textit{\textbf{Divine Eminence.}} As a bonus action, the priest can expend a spell slot to cause his melee weapon attack to deal an additional 10 (3d6) Light damage. The benefit lasts until the end of the turn.

\textit{\textbf{Spells.}} The priest has MP 5. His spellcasting ability is Wisdom (+5 on hitting with spell attacks). The priest has the following spells prepared: Cantrips (at will): \textit{holy flame, light, thaumaturgy}

level 1 (4 slots): \textit{heal wounds, tracer bolt, shrine}

level 2 (3 slots): \textit{spiritual weapon, lesser restoration}

level 3 (2 slots): \textit{dispel magic}, \textit{spiritual guardians}

\textbf{Actions}

\textit{\textbf{Mace.} Melee Weapon Attack}: +2 to hit, reach 1m, one target.

\textit{Hit:} 3 (1d6) bludgeoning damage.


\medskip\textbf{Savages}

These individuals live on the fringes of civilization, sometimes rarely coming into contact with it. Uncomfortably within walls and in civilized lands, they are at home when they can move through the wilds.

\medskip\textbf{Berserker}

Hailing from the wilds, the unpredictable berserkers gather in war companies and are always on the lookout for conflicts in which to fight.

\textit{Medium humanoid (any race), any chaotic Trait}

\textbf{STRENGTH} +3

\textbf{DEXTERITY} +1

\textbf{CONSTITUTION} +3

\textbf{INTELLIGENCE} -1

\textbf{WISDOM} +0

\textbf{CHARISMA} -1

\textbf{Initiative} +1 -- \textbf{Defence} 14 (leather Armour)

\textbf{Hit Points} 67 (9d8 + 27)

\textbf{Move} 9m

\textbf{Saving Throws}: Fortitude +4, Reflexes +3, Will +2

\textbf{Languages} any language (usually Common)

\textbf{Challenge} 2 (450 XP)

\textit{\textbf{Reckless.}} At the start of its round, the berserker can gain +1d6 on all melee weapon attack rolls made during that turn, but attack rolls against it have + 1d6 until the start of its next round.

\textbf{Actions}

\textit{\textbf{Big Axe.} Melee Weapon Attack}: +5 to hit, reach 1m, one target.

\textit{Hit:} 9 (1d12 + 3) slashing damage.\\

\textbf{Tribal Fighter}

They are the defenders of tribes living on the fringes of civilization.

\textit{Medium humanoid (any race), any Trait}

\textbf{STRENGTH} +1

\textbf{DEXTERITY} +0

\textbf{CONSTITUTION} +1

\textbf{INTELLIGENCE} -1

\textbf{WISDOM} +0

\textbf{CHARISMA} -1

\textbf{Initiative} +0 -- \textbf{Defence} 13 (leather Armour)

\textbf{Hit Points} 11 (2d8 + 2)

\textbf{Move} 9m

\textbf{Saving Throws}: Fortitude +2, Reflexes +1, Will +1

\textbf{Languages} any language

\textbf{Challenge} 1/8 (25 XP)

\textit{\textbf{Pack tactics.}} The tribal fighter has +1d6 on attack rolls against a creature if at least one of the thug's allies is within 1 meter of the creature and that ally isn't incapacitated.

\textbf{Actions}

\textit{\textbf{Spear.} Melee or Ranged Weapon Attack}: +3 to hit, reach 1m or range 6m, one target.

\textit{Hit:} 4 (1d6 + 1) piercing damage, or 5 (1d8 + 1) piercing damage when used with two hands to make a melee attack.

\medskip\textbf{Druid}

Druids protect the natural world from monsters and advancing civilization. Some are tribal shamans who heal the sick, pray to animal spirits and provide spiritual advice.

\textit{Medium humanoid (any race), any Trait}

\textbf{STRENGTH} +0

\textbf{DEXTERITY} +1

\textbf{CONSTITUTION} +1

\textbf{INTELLIGENCE} +1

\textbf{WISDOM} +2

\textbf{CHARISMA} +0

\textbf{Initiative} +1 -- \textbf{Defence} 12 (17 with \textit{Barkskin}*)

\textbf{Hit Points} 27 (5d8 + 5)

\textbf{Move} 9m

\textbf{Saving Throws}: Fortitude +1, Reflexes +2, Will +3 \\

\textbf{Skills} First Aid +4, Nature +3, Awareness +4

\textbf{Languages} Druidic plus two other languages

\textbf{Challenge} 2 (450 XP)

\textit{\textbf{Spells.}} The priest has MP 4. His spellcasting ability is Wisdom (+4 on hitting with spell attacks). The priest has the following spells prepared: Cantrips (at will): \textit{Druidic Artifice, staff, produce flame}

level 1 (4 slots): \textit{entangle, thunderwave, talk to people}
\textit{animals, fast pace}

level 2 (3 slots): \textit{messenger animal, barkskin}

\textbf{Actions}

\textit{\textbf{Combat Staff.} Melee Weapon Attack}: +2 to hit (+4 to hit with \textit{staff*}), reach 1m or range 6m, one target.

\textit{Hit:} 3 (1d6) bludgeoning damage, or 6 (1d8 + 2) bludgeoning damage with \textit{staff} or when wielded with two hands.

\medskip\textbf{Explorer}

Skilled hunters and track-beaters.

\textit{Medium humanoid (any race), any Trait}

\textbf{STRENGTH} +0

\textbf{DEXTERITY} +2

\textbf{CONSTITUTION} +1

\textbf{INTELLIGENCE} +0

\textbf{WISDOM} +1

\textbf{CHARISMA} +0

\textbf{Initiative} +2 -- \textbf{Defence} 14 (leather Armour)

\textbf{Hit Points} 16 (3d8 + 3)

\textbf{Move} 9m

\textbf{Saving Throws}: Fortitude +1, Reflexes +2, Will +3

\textbf{Skills} Stealth +6, Nature +4, Awareness +5, Survival +5

\textbf{Languages} any language (usually Common)

\textbf{Challenge} 1/2 (100 XP)

\textit{\textbf{Honed sense of smell and sight.}} The scout has +1d6 on Wisdom (Awareness) checks based on smell or sight.

\textbf{Actions}

\textit{\textbf{Multiattack.}} The scout makes two melee attacks or two ranged attacks.

\textit{\textbf{Short sword.} Melee Weapon Attack}: +4 to hit, reach 1m, one target.

\textit{Hit:} 5 (1d6 + 2) piercing damage.

\textit{\textbf{Longbow.} Melee Weapon Attack}: +4 to hit, range 45m, one target.

\textit{Hit:} 6 (1d8 + 2) piercing damage.




\end{multicols}

%{\scriptsize
%\printindex}
%\end{document}

\pagebreak

\subsection{List of Monsters by Challenge Rating}

\begin{multicols}{3}
{%\small
\flushleft{Eagle, Challenge 0 (10 XP)\\
Vulture, Challenge 0 (10 XP)\\
Baboon, Challenge 0 (10 XP)\\
Fase Dog, Challenge 0 (10 XP)\\
Deer, Challenge 0 (10 XP)\\
Raven, Challenge 0 (10 XP)\\
Weasel, Challenge 0 (10 XP)\\
Falcon, Challenge 0 (10 XP)\\
Screeching Mushroom, Challenge 0 (10 XP)\\
Cat, Challenge 0 (10 XP)\\
Owl, Challenge 0 (10 XP)\\
Hyena, Challenge 0 (10 XP)\\
Lemur, Challenge 0 (10 XP)\\
Lizard, Challenge 0 (10 XP)\\
Homunculus, Challenge 0 (10 XP)\\
Pirana, Challenge 0 (10 XP)\\
Commoner, Challenge 0 (10 XP)\\
Spider, Challenge 0 (10 XP)\\
Frog, Challenge 0 (10 XP)}\\
Rat, Challenge 0 (10 XP)\\
Giant Fire Beetle, Challenge 0 (10 XP)\\
Jackal, Challenge 0 (10 XP)\\
Scorpio, Challenge 0 (10 XP)\\
Badger, Challenge 0 (10 XP)\\
Mice, Challenge: 0 (10 XP)\\
Bandit/Pirate, Challenge 1/8 (25 XP)\\
Camel, Challenge 1/8 (25 XP)\\
Kobold, Challenge 1/8 (25 XP)\\
Cultist, Challenge 1/8 (25 XP)\\
Giant Weasel, Challenge 1/8 (25 XP)\\
Blood Hawk, Challenge 1/8 (25 XP)\\
Giant Crab, Challenge 1/8 (25 XP)\\
Guard, Challenge 1/8 (25 XP)\\
Hound, Challenge 1/8 (25 XP)\\
Mule, Challenge 1/8 (25 XP)\\
Noble, Challenge 1/8 (25 XP)\\
Pony, Challenge 1/8 (25 XP)\\
Giant Rat, Challenge 1/8 (25 XP)\\
Venomous Snake, Challenge 1/8 (25 XP)\\
Flying Serpent, Challenge 1/8 (25 XP)\\
Strix, Challenge 1/8 (25 XP)\\
Strix (Stygian Bird), Challenge 1/8 (25 XP)\\
Aquatic Man, Challenge 1/8 (25 XP)\\
Acolyte, Challenge 1/4 (50 XP)\\
Moose, Challenge 1/4 (50 XP)\\
Ax Beak, Challenge 1/4 (50 XP)\\
Intermittent Dog, Challenge 1/4 (50 XP)\\
Racehorse, Challenge 1/4 (50 XP)\\
Draft Horse, Challenge 1/4 (50 XP)\\
Giant Centipede, Challenge 1/4 (50 XP)\\
Boar, Challenge 1/4 (50 XP)\\
Dretch, Challenge 1/4 (50 XP)\\
Violet Mushroom, Challenge 1/4 (50 XP)\\
Gablin, Challenge 1/4 (50 XP)\\
Grimlock, Challenge 1/4 (50 XP)\\
Giant Owl, Challenge 1/4 (50 XP)\\
Giant Lizard, Challenge 1/4 (50 XP)\\
Wolf, Challenge 1/4 (50 XP)\\
Steam Mephith, Challenge 1/4 (50 XP)\\
Panther, Challenge 1/4 (50 XP)\\
Pseudodragon, Challenge 1/4 (50 XP)\\
Giant Wolf Spider , Challenge 1/4 (50 XP)\\
Giant Frog , Challenge 1/4 (50 XP)\\
Skeleton, Challenge 1/4 (50 XP)\\
Swarm of Crows , Challenge 1/4 (50 XP)\\
Swarm of Bats , Challenge 1/4 (50 PX)\\
Swarm of Rats, Challenge 1/4 (50 XP)\\
Constrictor Snake, Challenge 1/4 (50 XP)\\
Giant Venomous Snake, Challenge 1/4 (50 XP)\\
Flying Sword, Challenge 1/4 (50 XP)\\
Sprite, Challenge 1/4 (50 XP)\\
Giant Badger, Challenge 1/4 (50 XP)\\
Zombies, Challenge 1/4 (50 XP)\\
Giant Billy Goat, Challenge 1/2 (100 XP)\\
Warhorse, Challenge 1/2 (100 XP)\\
Giant Sea Horse, Challenge 1/2 (100 XP)\\
Crocodile, Challenge 1/2 (100 XP)\\
Cockatrice, Challenge 1/2 (100 XP)\\
Explorer, Challenge 1/2 (100 XP)\\
Gnoll, Challenge 1/2 (100 XP)\\
Deep Gnome, Challenge 1/2 (100 XP)\\
Hobgoblin, Challenge 1/2 (100 XP)\\
Lizardfolk, Challenge 1/2 (100 XP)\\
Dark hugger, Challenge 1/2 (100 XP)\\
Ice Mephith, Challenge 1/2 (100 XP)\\
Magma Mephit, Challenge 1/2 (100 XP)\\
Dust Mephit, Challenge 1/2 (100 XP)\\
Gray Slime, Challenge 1/2 (100 XP)\\
Shadow, Challenge 1/2 (100 XP)\\
Orc, Challenge 1/2 (100 XP)\\
Black Bear, Challenge 1/2 (100 XP)\\
Thumper, Challenge 1/2 (100 XP)\\
Rust Monster, Challenge 1/2 (100 XP)\\
Sahuagin, Challenge 1/2 (100 XP)\\
Satyr, Challenge 1/2 (100 XP)\\
Warhorse Skeleton, Challenge 1/2 (100 XP)\\
Swarm of Insects, Challenge 1/2 (100 XP)\\
Swarm of Spiders, Challenge 1/2 (100 XP)\\
Swarm of Scarabs , Challenge 1/2 (100 PX)\\
Swarm of Wasps, Challenge 1/2 (100 XP)\\
Swarms, Challenge 1/2 (100 XP)\\
Monkey, Challenge 1/2 (100 XP)\\
Reef Shark, Challenge 1/2 (100 XP)\\
Magma Man (Magmin), Challenge 1/2 (100 XP)\\
Giant Wasp, Challenge 1/2 (100 PX)\\
Worg, Challenge 1/2 (100 XP)\\
Giant Eagle, Challenge 1 (200 XP)\\
Animated Armor, Challenge 1 (200 XP)\\
Harpy, Challenge 1 (200 XP)\\
Giant Vulture, Challenge 1 (200 XP)\\
Bugbear, Challenge 1 (200 XP)\\
Dog of Death, Challenge 1 (200 XP)\\
Dinowolf (Diewolf), Challenge 1 (200 XP)\\
Baby Brass Dragon, Challenge 1 (200 XP)\\
Baby Copper Dragon, Challenge 1 (200 XP)\\
Dryad, Challenge 1 (200 XP)\\
Duergar, Challenge 1 (200 XP)\\
Ghoul, Challenge 1 (200 XP)\\
Globule, Challenge 1 (200 XP)\\
Giant Hyena, Challenge 1 (200 XP)\\
Imp, Challenge 1 (200 XP)\\
Hippogriff, Challenge 1 (200 XP)\\
Lion , Challenge 1 (200 XP)\\
Adventurer Magician, Challenge 1 (200 XP)\\
Orc, Challenge 1 (100 XP)\\
Brown Bear, Challenge 1 (200 XP)\\
Quasit, Challenge 1 (200 XP)\\
Giant Spider, Challenge 1 (200 XP)\\
Giant Toad, Challenge 1 (200 XP)\\
Swarm of Pirana, Challenge 1 (200 XP)\\
Spy Challenge 1 (200 XP)\\
Tiger, Challenge 1 (200 XP)\\
Awakened Tree, Challenge 2 (450 XP)\\
Giant Moose, Challenge 2 (450 XP)\\
Straw Amoeba, Challenge 2 (450 XP)\\
Ankheg, Challenge 2 (450 XP)\\
Azer, Challenge 2 (450 XP)\\
Berserker Challenge 2 (450 XP)\\
Explosive Cockroach, Challenge 2 (450 XP)\\
Bandit/Pirate Captain, Challenge 2 (450 XP)\\
Centaur, Challenge 2 (450 XP)\\
Giant Boar, Challenge 2 (450 XP)\\
Jelly Cube, Challenge 2 (450 XP)\\
Thorny Devil, Challenge 2 (450 XP)\\
White Dragon Baby, Challenge 2 (450 px)\\
Baby Silver Dragon, Challenge 2 (450 px)\\
Baby Bronze Dragon, Challenge 2 (450 PX)\\
Baby Black Dragon, Challenge 2 (450 px)\\
Green Baby Dragon, Challenge 2 (450 PX)\\
Druid, Challenge 2 (450 XP)\\
Lesser Water Elemental, Challenge 2 (450 XP)\\
Ettercap, Challenge 2 (450 XP)\\
Bubbling Maw, Challenge 2 (450 XP)\\
Wisp, Challenge 2 (450 XP)\\
Gargoyle, Challenge 2 (450 XP)\\
Ghast, Challenge 2 (450 XP)\\
Grick, Challenge 2 (450 XP)\\
Griffin, Challenge 2 (450 XP)\\
Sea Crone, Challenge 2 (450 XP)\\
Mimic, Challenge 2 (450 XP)\\
Ogre, Challenge 2 (450 XP)\\
Polar Bear, Challenge 2 (450 XP)\\
Pegasus, Challenge 2 (450 XP)\\
Plesiosaurus, Challenge 2 (450 XP)\\
Wererat, Challenge 2 (450 XP)\\
Rhino, Challenge 2 (450 XP)\\
Priest, Challenge 2 (450 XP)\\
Skeleton of Minotaur, Challenge 2 (450 XP)\\
Swarm of Venomous Snakes, Challenge 2 (450 XP)\\
Giant Constrictor Snake, Challenge 2 (450 XP)\\
Hissing, Challenge 2 (450 XP)\\
Shark Hunter, Challenge 2 (450 XP)\\
Carpet of Suffocation, Challenge 2 (450 XP)\\
Flaming Skull, Challenge 2 (200 XP)\\
Saber-toothed Tiger, Challenge 2 (450 XP)\\
Zombie Ogre, Challenge 2 (450 XP)\\
Killer Whale (Orca), Challenge 3 (700 XP)\\
Basilisk, Challenge 3 (700 XP)\\
Gablin Champion, Challenge 3 (700)\\
Knight, Challenge 3 (700 XP)\\
Nightmare Steed, Challenge 3 (700 XP)\\
Bearded Devil, Challenge 3 (700 XP)\\
Doppelganger, Challenge 3 (700 XP)\\
Baby Blue Dragon, Challenge 3 (700 px)\\
Golden Dragon Baby, Challenge 3 (700 XP)\\
Winter Wolf, Challenge 3 (700 XP)\\
Werewolf, Challenge 3 (700 XP)\\
Manticore, Challenge 3 (700 XP)\\
Green Crone, Challenge 3 (700 XP)\\
Minotaur, Challenge 3 (700 XP)\\
Mummy, Challenge 3 (700 XP)\\
Wall Climbing Horror, Challenge 3 (700 XP)\\
Owlbear, Challenge 3 (700 XP)\\
Wise Owlbear, Challenge 3 (700 XP)\\
Phase Spider, Challenge 3 (700 XP)\\
Giant Scorpion, Challenge 3 (700 XP)\\
Hellhound, Challenge 3 (700 XP)\\
Veteran, Challenge 3 (700 XP)\\
Wight, Challenge 3 (700 XP)\\
B.O.C., Challenge 4 (1100 XP)\\
Banshee, Challenge 4 (1100 XP)\\
Chuul, Challenge 4 (1100 XP)\\
Wereboar, Challenge 4 (1100 XP)\\
Couatl, Challenge 4 (1100 XP)\\
Baby Red Dragon, Challenge 4 (1100 XP)\\
Elephant, Challenge 4 (1100 XP)\\
Ettin, Challenge 4 (1100 XP)\\
Ghost, Challenge 4 (1100 XP)\\
Putrescent Ghoul, Challenge 4 (1100 XP)\\
Lamia, Challenge 4 (1100 XP)\\
Immortal Cursed, Challenge 4 (1100 XP)\\
Black Pudding, Challenge 4 (1100 XP)\\
Succubus, Challenge 4 (1100 XP)\\
Weretiger, Challenge 4 (1100 XP)\\
Darktorch, Challenge 4 (1100 XP)\\
Tentacled Crawling Worm, Challenge 4 (1100 XP)\\
Bulette, Challenge 5 (1800 XP)\\
Giant Crocodile, Challenge 5 (1800 XP)\\
Creeping Mound, Challenge 5 (1800 XP)\\
Fire Elemental, Challenge 5 (1800 XP)\\
Water Elemental, Challenge 5 (1800 XP)\\
Elemental of Air, Challenge 5 (1800 XP)\\
Earth Elemental, Challenge 5 (1800 XP)\\
Roper, Challenge 5 (1800 XP)\\
Ghoul, Mother, Challenge 5 (1800 XP)\\
Hill Giant, Challenge 5 (1800 XP)\\
Gladiator, Challenge 5 (1800 XP)\\
Flesh Golem, Challenge 5 (1800 XP)\\
Gorgon, Challenge 5 (1800 XP)\\
Night Hag, Challenge 5 (1800 XP)\\
Werebear, Challenge 5 (1800 XP)\\
Otyugh, Challenge 5 (1800 XP)\\
Salamander, Challenge 5 (1800 XP)\\
Giant Shark, Challenge 5 (1800 XP)\\
Triceratops, Challenge 5 (1800 XP)\\
Trolls, Challenge 5 (1800 XP)\\
Unicorn, Challenge 5 (1800 XP)\\
Wraith, Challenge, 5 (1800 XP)\\
Xorn, Challenge, 5 (1800 XP)\\
Bone Blossom, Challenge 6 (2300 XP)\\
Chimera, Challenge 6 (2300 XP)\\
Young White Dragon, Challenge 6 (2300 XP)\\
Young Brass Dragon, Challenge 6 (2300 XP)\\
Drider, Challenge 6 (2300 XP)\\
Ghoul, Black, Challenge 6 (2300 XP)\\
Great Wizard, Challenge 6 (2300 XP)\\
Mammoth, Challenge 6 (2300 XP)\\
Jellyfish, Challenge 6 (2300 XP)\\
Paladin Gablin, Challenge 6 (2300 XP)\\
Invisible Stalker, Challenge 6 (2300 XP)\\
Vampiric Spawn , Challenge 6 (1800 XP)\\
Wyvern, Challenge, 6 (2300 XP)\\
Vrock, Challenge, 6 (2300 XP)\\
Young Copper Dragon, Challenge 7 (2900 XP)\\
Young Black Dragon, Challenge 7 (2900 XP)\\
Stone Giant, Challenge 7 (2900 XP)\\
Guardian Protector, Challenge 7 (2900 XP)\\
Oni, Challenge 7 (2900 XP)\\
Giant Ape, Challenge 7 (2900 XP)\\
Assassin, Challenge 8 (3900 XP)\\
Chain Devil, Challenge 8 (3900 XP)\\
Young Bronze Dragon, Challenge 8 (3900 XP)\\
Young Green Dragon, Challenge 8 (3900 XP)\\
Frost Giant, Challenge 8 (3900 XP)\\
Hezrou, Challenge 8 (3900 XP)\\
Hydra, Challenge 8 (3900 XP)\\
Cloak, Challenge 8 (3900 XP)\\
Spirit Naga, Challenge 8 (3900 XP)\\
Tyrannosaurus, Challenge 8 (3900 XP)\\
Bone Devil, Challenge 9 (5000 XP)\\
Eat Brains, Challenge 9 (5000 XP)\\
Young Blue Dragon, Challenge 9 (5000 XP)\\
Young Silver Dragon, Challenge 9 (5000 XP)\\
Major Water Elemental, Challenge 9 (5000 XP)\\
Fire Giant, Challenge 9 (5000 XP)\\
Cloud Giant, Challenge 9 (5000 XP)\\
Glabrezu, Challenge 9 (5000 XP)\\
Treeman (Treant), Challenge 9 (5000 XP)\\
Aboleth, Challenge 10 (5900 XP)\\
Deva Angel, Challenge 10 (5900 XP)\\
Clay Golem, Challenge 9 (5000 XP)\\
Young Gold Dragon, Challenge 10 (5900 XP)\\
Young Red Dragon, Challenge 10 (5900 XP)\\
G.E.C., Challenge 10 (5900 XP)\\
Stone Golem, Challenge 10 (5900 XP)\\
Naga Guardian, Challenge 10 (5900 XP)\\
Behir, Challenge 11 (7200 XP)\\
Horned Devil, Challenge 11 (7200 XP)\\
Djinni, Challenge 11 (7200 XP)\\
Efreeti, Challenge 11 (7200 XP)\\
gynosphinx, Challenge 11 (7200 XP)\\
Remorhaz, Challenge 11 (7200 XP)\\
Archmage, Challenge 12 (8400 XP)\\
Erinyes, Challenge 12 (8400 XP)\\
Panoptikhan, Challenge 12 (8400 XP)\\
Adult White Dragon, Challenge 13 (10000 XP)\\
Adult Brass Dragon, Challenge 13 (10000 XP)\\
Storm Giant, Challenge 13 (10000 XP)\\
Nalfeshnee, Challenge 13 (10000 XP)\\
Rakshasa, Challenge 13 (10000 XP)\\
Vampire, Challenge 13 (10000 XP)\\
Ice Devil, Challenge 14 (11500 XP)\\
Adult Copper Dragon, Challenge 14 (11500 XP)\\
Adult Bronze Dragon, Challenge 15 (13000 PX)\\
Adult Green Dragon, Challenge 15 (13000 XP)\\
Phoenix, Challenge 15 (13000 XP)\\
Mummy Sovereign, Challenge 15 (13000 XP)\\
Purple Worm, Challenge 15 (13000 XP)\\
Planetar Angel, Challenge 16 (15000 XP)\\
Adult Blue Dragon, Challenge 16 (15000 XP)\\
Adult Silver Dragon, Challenge 16 (15000 PX)\\
Iron Golem, Challenge 16 (15000 XP)\\
Marilith, Challenge 16 (15000 XP)\\
Androsphinx, Challenge 17 (18000 XP)\\
Adult Gold Dragon, Challenge 17 (18000 PX)\\
Adult Black Dragon, Challenge 17 (18000 PX)\\
Adult Red Dragon, Challenge 17 (18000 PX)\\
Dragon Tortoise, Challenge 17 (18000 XP)\\
Black Knight, Challenge 18 (20000 XP)\\
Balor, Challenge 19 (22000 XP)\\
Pit Devil, Challenge 20 (25000 XP)\\
Ancient White Dragon, Challenge 20 (25000 XP)\\
Ancient Brass Dragon, Challenge 20 (25000 PX)\\
Angelo Solar, Challenge 21 (33000 XP)\\
Ancient Copper Dragon, Challenge 21 (33000 PX)\\
Ancient Black Dragon, Challenge 21 (33000 XP)\\
Lich, Challenge 21 (33000 XP)\\
Ancient Bronze Dragon, Challenge 22 (41000 PX)\\
Ancient Green Dragon, Challenge 22 (41000 PX)\\
Ancient Blue Dragon, Challenge 23 (50000 XP)\\
Ancient Silver Dragon, Challenge 23 (50000 PX)\\
Ancient Yellow Dragon, Challenge: 23 (50000 XP)\\
Kraken, Challenge 23 (50000 XP)\\
Ancient Gold Dragon, Challenge 24 (62000 PX)\\
Ancient Red Dragon, Challenge 24 (62000 PX)\\
Demogorgon, Challenge 26 (90000 XP)\\
Orcus, Challenge 26 (90000 XP)\\
Tàhil, Challenge 30 (155000 XP)\\
Tarrasque, Challenge 30 (155000 XP)\\
}


\end{multicols}

\pagebreak


\subsection{Monster conversion}\index{Monster conversion}

\bigskip

To add other monsters to OBSS I invite you to convert them from Pathfinder or from the 5ed of the famous RPG. OBSS is basically a d20 system heavily modified in dynamics but not in the foundations of numerical values.

\medskip

\textbf{Pathfinder conversion}

Take for example the common Orc from

https://www.d20pfsrd.com/bestiary/monster-listings/humanoids/orcs/orc/

let's skip the descriptive parts and focus on the numbers and values.

\bigskip

\textbf{Orc (Challenge rank 1/3)} this value remains the same in OBSS

\textbf{XP 135} this value can be traced back to the experience points table for CR (100 px in this case)

\textbf{Orc warrior 1} we are not interested in class, only in race.

\textbf{CE Medium humanoid} indicates that the creature is of medium size, humanoid and evil, for the purposes of the Traits the creature is not of such a level as to have attracted the attention of a Patron.

\textbf{Init +0} is the initiative, take the bonus to Dexterity or Intelligence

\textbf{Senses} Darkvision 18m; Awareness -1, remains the same, keep the same values ​​and skills. In this case 20 meters means the distance is 20 meters

\textbf{Weakness} light sensitivity the equivalent in OBSS is sought where possible, in this case light photophobia or the indicated penalties are applied directly.

\textbf{AC} 13, touch 10, flat-footed 13 (+3 Armor) this is Defense.

\textbf{Weapons Proficiency}: is equal to the indicated BAB

\textbf{Magic Proficiency}: by default it is half the degree of Challenge. Useful only if the creature has magical powers.

\textbf{hp} 6 (1d10+1) remains the same

\textbf{Strut} +3, Ref +0, Will -1 translate to Fortitude, Dexterity, and Will. The score remains the same

\textbf{Speed} 9m It's the movement, in this case it's 9 meters per move action.

\textbf{Melee} Glaive +5 (2d4+4/18--20) is my attack roll and damage. It stays the same

\textbf{Ranged} javelin +1 (1d6+3) is the attack and damage roll. It stays the same

\textbf{Str} 17, Dex 11, Con 12, Int 7, Wis 8, Cha 6. You have to take only the bonus part.

\textbf{Base Atk} +1; CMB +4; CMD 14 the first value determines the WP. I suggest using the hit bonuses listed in the melee directly.

\textbf{Feats} Weapon Focus (Glaive) Weapon Focus. The feat bonus is already calculated into the Melee values

\textbf{Skills} Intimidate +2 remains the same.

\textbf{Ferocity} (Ex): An orc remains conscious and can continue fighting even if its hit point total is below 0. It is still staggered and loses 1 hit point each round. A creature with ferocity still dies when its hit point total reaches a negative amount equal to its Constitution score. You take the feat as it is.

\bigskip

\textbf{Conversion from 5e} of the famous RPG:

- Increase Defense and Attack Rolls by 1/2 points the creature's Challenge Rating, rounded up.

- For Saving Throws, take the creature's Challenge Rating as a basis and then apply the characteristic modifier related to Dexterity (Reflexes), Constitution (Fortitude), Will (Wisdom).

\pagebreak


\section{Conditions}\index{Conditions}

\begin{changemargin}{0cm}{0.5cm}\begin{emphasis}{
The greatness of man is in the decision to be stronger than his condition. (Albert Camus)}
\end{emphasis}\end{changemargin}\medskip

%{\small
\begin{multicols}{2}

\label{condizioni}


\textbf{Bleeding}\index{Bleeding}\index{Bleeding}\hypertarget{bleeding}{}: A creature that is taking bleed damage takes the listed amount of damage at the start of its round. Bleeding can be reduced with a successful DC 12 First Aid check, 2 Actions.

For each Bleeding value above 1 the difficulty increases by 2. Cost \textbf{2 Actions}. A 1 minute treatment guarantees 1 success, no check. Each critical success reduces the bleeding by one additional point. Some bleed effects cause ability damage or even ability drain.
Bleeding is reduced by 1 for each Potion or Spell healing die. If the subject's Hit Points are restored to their maximum, the bleeding stops. \index{Bleeding and healing}

Unless otherwise indicated, the bleeding damage stacks with a maximum of 10 Hit Points per round. Bleeding damage is denoted as Bleeding value/max value, first value is the bleed score caused by the attack, and max value is the maximum bleeding value that can be achieved.

\textbf{Blind}:\index{Blind} The character can't see anything. he takes a –2 penalty to most Strength- and Dexterity-based proficiencies.

All checks or activities based on vision (such as reading, or any Awareness check based on vision) automatically fail. All opponents are treated as having invisibility towards the blinded character.

Blind characters always treat the terrain as difficult and must make a DC 12 Acrobatics check to move faster than their half speed. Creatures that fail this check are knocked prone. Characters who are blinded for a long time may become accustomed to some of these penalties and begin to overcome some, at the Arbiter's discretion.

Attacking a creature invisible to it has a -1d6 attack roll, an invisible creature attacking a blind creature has a +1d6 attack roll.

\textbf{Blocked}\index{Blocked}\hypertarget{Blocked}{}: A locked creature has its arms locked. He can move by trying to Push, he must use two Actions to free himself (see Grabbed). It get -4 to Defence and Reflex saving throws. A stranded spellcaster must make a Magic Check with Critical Magical Success or fail to cast spells. -1d6 to attack roll. Anyone who has Blocked a creature is still considered Grabbed.


\textbf{Broken}\index{Broken}: The broken condition has the following effects, depending on the object:

- If the object is a weapon, all attacks made with the object take a -2 penalty to attack and damage rolls. Such weapons only score a critical roll with a natural 3x6 and deal up to only 1x additional damage.

- If the item is Armour or a shield, the bonus it grants to Defence is halved, rounding up. Broken Armour doubles the Armour penalty to the Proficiency Check.

- If the item is a tool required for a Proficiency, all Proficiency checks made with it suffer a -2 penalty.

- If the item is a wand or staff, use twice as many charges each time it is used.

- If the item does not fit into any of the above categories, the broken condition has no effect on its use. Items with the broken condition, regardless of type, are worth 25\% of their normal cost. If the item is magical, it can only be repaired with the Craft spell used by a spellcaster of equal or higher level than the one who created the item.


\textbf{Charmed}:\index{Charmed}: A charmed creature cannot attack or target the charmer with special abilities or harmful magical effects.

Any potential threat, such as an approaching hostile creature, allows the charmed creature to make a new Saving Throw against the charm's effect. Any overt threat, such as someone drawing a weapon, casting a spell, or pointing a ranged weapon at the charmed creature, automatically ends the effect.

An ally of the charmed creature can shake it to allow it to make a new Saving Throw by spending 2 Actions.

The charmer has +1d6 on any ability checks to interact socially with the creature.

\textbf{Confused}\index{Confusion}: \index{Confused}A confused creature is mentally befuddled and cannot act normally. A confused creature cannot distinguish an ally from an enemy and treats all as enemies.

Allies who want to use a spell to benefit the confused creature must still touch it with a successful melee touch attack.

If a confused creature is attacked, it always attacks the last creature that attacked it, until that creature dies or goes out of sight.

Roll a die on the table below at the start of each confused creature's round each round to see what the creature does that round.

\textbf{d100 Behavior:}

01-25 Acts normally

26-50 All he does is stammer incoherently

51-75 Inflicts 1d8 + Strength modifier on self with weapon in hand

76-100 Attacks closest creature (for this purpose, a Familiar counts as part of the subject itself)

A confused creature unable to take the indicated action will do nothing but stammer incoherently. Attackers have no special advantage when attacking a confused creature. Any confused creature that is attacked automatically attacks its attacker on its next round, as long as it remains confused when its round comes.

\textbf{Coup de grace}:\index{Coup de grace} As its only action in the round, a creature can use a melee weapon to deal a coup de grace to a helpless character. He can also use a bow or crossbow, as long as it is adjacent to the target.

The attacker strikes automatically and inflicts three critical rolls. Creatures immune to critical rolls cannot be coup de grace.

\textbf{Damage Resistance}\index{Damage Resistance}: A creature that has Damage Resistance is considered to automatically halve damage from the specified source, e.g. Damage Resistance: Sound. Damage Resistance can also be indicated with a numerical value, e.g. Damage Resistance: Fire 10. In this case the protection works on the first 10 damage suffered, in the event of an effect that grants a Saving Throw to halve, first the amount of protection is removed from the total, then the Saving Throw is performed to halve the residual damage.

\textbf{Dead}:\index{Dead}\hypertarget{dead}{} The character's soul leaves his body permanently. Dead characters cannot benefit from normal or magical healing, and cannot be brought back to life by a spell. Only a Patron has enough power to return the soul to the body and bring the creature back to life. The School of Necromancy has spells for reanimating a body as undead.

\textbf{Deafened}:\index{Deafened}\index{Deaf} A deafened character cannot listen. He automatically fails all sound-based Awareness checks and is considered distracted when casting spells with at least verbal components.

Characters who are deafened for extended periods of time can get used to these penalties and overcome some of them, at the Arbiter's discretion.


\textbf{Dice Increase}\index{Dice Increase}\index{Dice Size}: when the rule tells you to increase the size of a die, follow this pattern.

1d4 > 1d6 > 1d8 > 1d10 > 2d6 > 2d8 > 2d10 > 3d6 > 3d8 > 3d10

\textbf{Distracted}\index{Distracted}: If the spellcaster is severely distracted, impeded, disturbed, bleeding, is under attack while trying to cast a spell, he must make a Magic Test.

If the check gets a critical failure the spell fails but suffers no consequences, if the check gets a critical success this is not considered.

A failed spell due to distraction does not subtract Spell Points.

\textbf{Dominated}:\index{Dominated} If you have a common language, you can generally force the subject to execute commands within the limits of his abilities. If you don't share any language, you can only issue basic commands like "come here", "go there", "fight" or "stand still". You are aware of what the subject is feeling but receive no direct sensory perceptions from it, nor can you communicate with it telepathically.

Once an order is given to the dominated creature, it continues to attempt to carry it out to the exclusion of all other activities except those necessary for day-to-day survival (such as eating, sleeping, and so on). Thanks to this limited spectrum of activity, an Awareness check with DC 15 (rather than DC 25) can determine whether the subject's behavior has been affected by an enchantment effect.

By concentrating fully on the spell (2 actions), one can receive sensory perceptions as interpreted by the subject's mind, though the subject still cannot communicate them. You can't actually see through your subject's eyes, so it's not as if you're present, but you can see what's going on.

Obviously blatantly self-destructive orders are not carried out. Once control is established, the range within which it can be maintained is unlimited as long as both subjects remain on the same footing. No need to see the subject to check it. If at least 1 minute is not spent concentrating on the spell each day, the subject receives a new Saving Throw to break free from the control.


\textbf{Drown/Hold Breath}: \index{Drown}\index{Hold Breath} Any character can hold his breath for a number of rounds equal to 6 rounds for his Constitution score, with a minimum of 3 rounds. For each Action you take, the remaining duration decreases by 1 round. After this period of time, the character must make a DC 12 Fortitude save each round to continue holding his breath. Each round, DC increases by 1.

\textbf{Dying} \index{Dying}: A dying character has -1 Hit Points. Each round he loses 1 hit point until he dies or is healed. See also \textbf{Helpless}.

\textbf{Entangled}:\index{Entangled}\hypertarget{entangled}{} An entangled character has difficulty moving, but can still attempt to move unless the restraining bonds are anchored to an immobile object or held by an opposing force.

An entangled creature treats terrain as difficult, cannot Run or Charge, and takes a –2 penalty on Defense and attack rolls.

An entangled character who tries to cast a spell is considered distracted.


\textbf{Fatigued}\index{Fatigued}\hypertarget{fatigued}{}\label{affaticato}\index{Exhausted}: A fatigued character cannot run or charge and takes a –1 penalty to Defense and attack rolls and saving throws. If he does anything normally tiring, his Fatigued rating increases and he also takes penalties on movement and proficiency checks.

If a character does not sleep at least 8 hours or sleeps with medium or heavy armor in the morning he is tired.

A Fatigued 2 character moves more slowly and takes a –2 penalty on attack rolls, defense, and saving throws. After 1 hour of complete rest (or Lesser Restoration spell), a Fatigued 2 character becomes Fatigued.

\medskip

\textbf{Table: Fatigue Levels}\index{Fatigue Levels Table}

\medskip

\begin{tabularx}{0.45\textwidth}{lcl}
	\textbf{Conditions}& \textbf{Pen./Mov/Comp.}&\textbf{Rec.}\\
	\hline
	Fatigued &1/-/-&1h\\
	Fatigued 2&2/2m/-4&1h\\
	Fatigued 3&4/3m/-6&8h\\
	Fatigued 4&6/6m/-8&12h\\
	Fatigued 5&Stunned&6h\\
	Fatigued 6&Death&--\\
\end{tabularx}


After 8 hours of rest a creature goes from Fatigued 3 to Fatigued 2 and after another hour it goes to Fatigued, as long as it suffers no further fatigue


\medskip


\textbf{Fear, Frightened}:\index{Fear}\index{Frightened}\hypertarget{conditionfear}{} Spells, Magic Items, and certain creatures can affect characters with fear. In most cases, the character must make a Will save to resist the effects, and a failed save indicates that the character is frightened.

A frightened creature has -1d6 on attack rolls, Saving Throws, and Skill checks as long as the source of its fear is visible. A frightened creature cannot voluntarily approach the source of its fear.

\textbf{Flanking}\index{Flanking}\index{Flanked}: a creature is flanked if it has two not side-by-side opponents around and a hypothetical line joining the opponents crosses the creature's square completely. The two creatures get +2 to attack or Defense.

\textbf{Flat-footed / Surprised}\index{Surprised}\index{Flat-footed}: 
A surprised/flat-footed creature has a -4 penalty to Defense and Reflex saves. You will not be able to react, you will not use Actions or Reactions unless explicitly permitted; from the following round you will be able to declare the initiative and act normally.

\textbf{Friendly}:\index{Friendly} A friendly creature will not attack the character unless explicitly threatened.

\textbf{Grabbed}\index{Grabbed}: A grabbed character cannot move but can attempt to Push. He must use two Actions to free himself (Fortitude save opposed with Strength bonus + 1d6 per Size difference).

He can attack with melee weapons if they are small (he will hardly be able to use a greatsword, halberd... a dagger or short sword is more likely). He has -2 Defense.

The conditions apply to whoever is grabbed and whoever grabs.

\textbf{Helpless}:\index{Helpless}\index{Unconscious}\index{Dying}\index{Sleeping}\hypertarget{helpless}{}\hypertarget{morente}{} A character asleep, unconscious, dying, or for some other reason completely at the mercy of his opponents, is considered helpless.

A Helpless creature cannot take Actions or Reactions or speak, melee attacks against it have a +2d6 bonus. He is not aware of what is happening around him. The creature drops whatever it is holding and falls prone.

The creature automatically fails Fortitude and Reflex saving throws.

The creature loses its Dexterity bonus to Defense.

\textbf{Holding breath}: CON*6 rounds. Each Action -1 round. Spells -3 rounds. When holding breath end make a Saving Throw Fortitude DC 12 +1 per round.


\textbf{Incapacitated}\index{Incapacitated}: An incapacitated creature is Flat footed and cannot make Actions.

\textbf{Invisible}:\index{Invisible} Invisible creatures cannot be seen.
Attacking a creature invisible to it has a -1d6 to attack, an invisible creature attacking a creature that does not see it has +1d6 to attack.

\textbf{Loss of characteristic points}\index{Loss of characteristic points}\index{Loss of characteristic points}: when the characteristic scores decrease, remember to remove any Hit Points 1 per point of Constitution lost per level, lower Saving Throws (Dex , Constitution, Wisdom), Attack Rolls (Strength and Dexterity), Defence (Defence). If not listed as permanent, you recover 1 point of Characteristics per day of rest.

\textbf{Maximum Hit Points}\index{Maximum Hit Points}: A creature that suffers an attack that lowers the maximum Hit Points must first decrease the current maximum Hit Points and then decrease the current maximum Hit Points by the same amount if not already removed. If the maximum Hit Points reach 0, the creature is dead. Maximum Hit Points are recovered at the rate of 1 per Constitution value per 8 hours of rest.

\textbf{Nauseated}\index{Nauseated}: If the penalty is not already stated, a nauseated creature has -1d6 on attack rolls, saving throws, and proficiency checks.

\textbf{Paralysed}: \index{Paralysed}A paralyzed character is frozen in place and unable to move or act, he is \textbf{Helpless} and can only perform mental actions.

When paralyzed, a winged creature in flight can no longer flap its wings and falls. A paralyzed swimmer can no longer Swim and may drown.

Terrain occupied by a paralyzed (or dead) creature is treated as hindering terrain.

\textbf{Petrified}: \index{Petrified}A petrified character has been turned to stone and is unconscious and \textbf{Helpless}.

If a petrified character cracks or breaks, but the broken pieces are joined to the body when it returns to flesh, the character is not injured or harmed. If the character's petrified body is incomplete when transformed back into flesh, the body remains incomplete and may have some permanent loss of Hit Points and/or other ailments.

The creature has resistance to all damage. The creature is immune to poisons and diseases, but any poisons and diseases already in its system are only suspended, not neutralized.

\textbf{Poisoned}\index{Poisoned}: any subject under the influence of a poison or potion is considered poisoned, regardless of whether this is already producing its effects or has yet to produce them given the time of onset. The Poisoned condition if isn't detailed cause -1 Strenght and Dexterity.

\textbf{Prone}\index{Prone}: whoever is prone has a -4 to attack and a -4 to Defense. Getting up from prone costs 2 Actions. You cannot become prone if you fly.
A prone creature's only movement option is to crawl (difficult terrain), unless it stands up and ends its condition.

The creature can perform an Acrobatics check, costing 1 Action, if it rolls 13 or more it stands up.

When the Acrobatics score reaches 6, getting up from prone costs 1 Action. With Acrobatics 8 it costs an Immediate Action.

If you roll three 1's on the test check, you cannot take any other actions that round and remain prone.

The terrain is considered difficult and you are still considered prone until you stand up.

\textbf{Restricted}\index{Restricted}\hypertarget{restricted}{} : Two medium or small creatures sharing the same map square are considered restricted. Both creatures take -1d6 to attack and Defence (-4) while sharing the space. A creature can share a square with a creature at least three times its size without penalty.


\textbf {Sleep} \index {Sleep}: Whenever a character ends a 24-hour period without sleeping for at least 8 hours, he must make a DC 12 Fortitude save or become fatigued. Each further missed rest will make him even more fatigued by cumulating the relative penalties.
If the character stays awake for more days, fighting sleep becomes more difficult. After the first 24 hours, DC increases by 4 for each consecutive 24-hour period without 8 hours of sleep. DC resets to 12 when character completes a rest of at least 8 hours.

\textbf{Slowed}\index{Slowed}: a slowed creature is unable to perform all 3 possible Actions in the round. Slowed down is always indicated with two values, the first indicates how many less Actions are done per round, the second the duration of the effect, if marked with a - then it has no indicated end. Ex. Slowed 1/3r, Slowed 2/- .

\textbf{Stunned/Knocked Out}:\index{Stunned}\index{Knocked Out} is considered \textbf{Helpless}.

\textbf{Unconscious}\index{Unconscious}: is considered to be \textbf{Helpless}.

\textbf{Vulnerability}\index{Vulnerability}: works the opposite of Resistance. The damage is doubled before any Saving Throw.

\end{multicols}
%}

\pagebreak


\section{Author}\index{Author}

\bigskip

\flushleft{

\textsc{Author and Creator}: Andres Zanzani - azanzani@gmail.com

\bigskip
\textsc{Contributions}: Federica Angeli - angelifdc@gmail.com

\bigskip

\bigskip

A special thanks to all my family who put up with and supported me in these desperate years!

\bigskip

Powered by \Large\LaTeX\ \normalfont\& \Large\textbf{GitHub}

\bigskip

Andres Zanzani}

\bigskip

\vfill

\section{Playing materials}\index{Character Sheet}\index{Manuals}\index{Storyterller Screen}\index{Keep track of character}

\label{scheda-e-manuale}
{\normalsize


You are invited to download from GitHub \textbf{Old Bell School System} freely and without constraints other than those expressed by the license.
The main site is \href{https://github.com/buzzqw/TUS}{https://github.com/buzzqw/TUS}

\medskip

* This is the link for the \textbf{OBSS Handbook}:
\href{https://github.com/buzzqw/TUS/blob/master/OBSS/BSS-eng-v3.pdf}{OBSS-eng-v3.pdf}

or url https://github.com/buzzqw/TUS/blob/master/OBSS/OBSS-eng-v3.pdf

\medskip

* This is the link for the \textbf{Character Sheet}:

\href{https://github.com/buzzqw/TUS/blob/master/OBSS/OBSS-sheet-eng.pdf}{OBSS-Scheda-eng.pdf}

or url  https://github.com/buzzqw/TUS/blob/master/OBSS-sheet-eng.pdf


\medskip

*  This is the link for the \textbf{Arbiter's Screen}:
\href{https://github.com/buzzqw/TUS/blob/master/OBSS/screen-eng.pdf}{screen-eng.pdf}

or url  https://github.com/buzzqw/TUS/blob/master/OBSS/screen-eng.pdf

\medskip

* This is the link to keep track of \textbf{character information}:
\href{https://github.com/buzzqw/TUS/blob/master/OBSS/OBSS-schema-arbiter-character-eng.pdf}{OBSS-schema-arbiter-character-eng.pdf}

or url  https://github.com/buzzqw/TUS/blob/master/OBSS/OBSS-schema-arbiter-character-eng.pdf

\medskip

This is the link to the \textbf{changelog} which is updated with every significant commit \href{https://github.com/buzzqw/TUS/blob/master/OBSS/changelog.md}{changelog.md}

or \url {https://github.com/buzzqw/TUS/blob/master/OBSS/changelog.md}

\medskip

In the Projects Version section of GitHub you will find the Manual, Character Sheet and Screen packages. Beware that they are certainly out of date.

Any report or advice you want to give me is more than welcome! Open an issue on GitHub or contact me on the OBSS group on Telegram \href{https://t.me/obssgdr}{https://t.me/obssgdr}


\bigskip

\vfill

\section{Materials for playing}\index{Card}\index{Manual}\index{Screen}\index{Character information}

\label{sheet-and-manual}
%{\normalsize


You are invited to download \textbf{Old Bell School System} from GitHub freely and without constraints other than those expressed in the license.
The main site is \href{https://github.com/buzzqw/TUS}{https://github.com/buzzqw/TUS}

\medskip

* \textbf{OBSS Manual}:
\href{https://github.com/buzzqw/TUS/blob/master/OBSS/OBSS.pdf}{OBSS.pdf} https://github.com/buzzqw/TUS/blob/master/OBSS/OBSS. pdf

\medskip

* \textbf{Tab}:
\href{https://github.com/buzzqw/TUS/blob/master/OBSS/OBSS-scheda.pdf}{OBSS-Scheda.pdf} https://github.com/buzzqw/TUS/blob/master/ OBSS/OBSS-scheda.pdf

\smallskip

3 page version
\href{https://github.com/buzzqw/TUS/blob/master/OBSS/OBSS-scheda-v3.pdf}{OBSS-scheda-v3.pdf} https://github.com/buzzqw/TUS/ blob/master/OBSS/OBSS-scheda-v3.pdf

\medskip

* \textbf{Narrator's Screen}:
\href{https://github.com/buzzqw/TUS/blob/master/OBSS/screen.pdf}{screen.pdf} https://github.com/buzzqw/TUS/blob/master/OBSS/screen. pdf

\medskip

* \textbf{character info}:
\href{https://github.com/buzzqw/TUS/blob/master/OBSS/OBSS-schema-narratore-personaggi.pdf}{OBSS-schema-narratore-personaggi.pdf}

https://github.com/buzzqw/TUS/blob/master/OBSS/OBSS-schema-narratore-personaggi.pdf

\medskip

* \textbf{changelog} \href{https://github.com/buzzqw/TUS/blob/master/OBSS/changelog.md}{changelog.md} url {https://github.com/buzzqw/TUS/ blob/master/OBSS/changelog.md}


\medskip

\begin{center}\index{Maps}
\textbf{Maps}:
\end{center}

All the maps (splendidly prepared by Lorenzo Caputo) can be downloaded from the \href{https://github.com/buzzqw/TUS/tree/master/OBSS/immagini}{Images} folder

\medskip

\small{

\href{https://github.com/buzzqw/TUS/blob/master/OBSS/immagini/Curyan.jpg}{Curyan} https://github.com/buzzqw/TUS/blob/master/OBSS/immagini/Curyan.jpg

\href{https://github.com/buzzqw/TUS/blob/master/OBSS/immagini/Tiya.jpg}{Tiya} https://github.com/buzzqw/TUS/blob/master/OBSS/immagini/Tiya.jpg

\href{https://github.com/buzzqw/TUS/blob/master/OBSS/immagini/Mappacompleta.jpg}{Complete map} https://github.com/buzzqw/TUS/blob/master/OBSS/immagini/Mappacompleta.jpg

\href{https://github.com/buzzqw/TUS/blob/master/OBSS/immagini/Curyan-zona0.jpg}{Partial map of Curyan} https://github.com/buzzqw/TUS/blob/master/OBSS/immagini/Curyan-zona0.jpg

\href{https://github.com/buzzqw/TUS/blob/master/OBSS/immagini/Tiya-zona0.jpg}{Partial map of Tiya} https://github.com/buzzqw/TUS/blob/master/OBSS/immagini/Tiya-zona0.jpg}

\bigskip

\section{Acknowledgments}\index{Acknowledgments}\index{EditoriFolli}

A huge thank you to \href{http://www.editorifolli.it/gdr/dnd5/srd5/}{EditoriFolli} ( http://www.editorifolli.it/gdr/dnd5/srd5/ ) and his collaborators for the translation work of the 5e SRD. Without their work this manual would not have been possible.

\bigskip

Lorenzo Caputo, for his precious collaboration and patience in preparing the maps of Curyan and Tiya.

You can contact him on his Instagram page \ href {https://www.instagram.com/galiosjourney/} {galiosjourney}

https://www.instagram.com/galiosjourney/ or by email Frank.thegamer@outlook.com

\bigskip

Thanks to \href{https://github.com/ThomasJockin/readexpro}{Readex} (https://github.com/ThomasJockin/readexpro) for the font used. Readex is a highly readable font even for people with reading difficulties.


The \emph{Italic} and \textbf{\emph{Italic Bold}} fonts are taken from \href{https://dejavu-fonts.github.io/}{Dejavu} ( https://dejavu-fonts.github .I/ )

\bigskip

For any report or advice, open an issue on GitHub, or send me an email azanzani@gmail.com or contact me on the OBSS group on Telegram \href{https://t.me/obssgdr}{https://t.me/obssgdr }

bigskip

\begin{center}
Powered by \Large\LaTeX\ {\normalsize {\&}} \Large\textbf{GitHub}
\end{center}


\vfill

\begin{changemargin}{0.3cm}{0.3cm}\begin{emphasis}{
And enjoy the game. (Players' Guide to Immortals. Frank Mentzer)
}\end{emphasis}\end{changemargin}\medskip



\thispagestyle{plain}
\begin{center}
\includepdf[pages={1,2},addtotoc={1,section,0,Character Sheet,incl:first},scale=0.85]{OBSS-sheet-eng.pdf}
%\includepdf[pages={1,2},scale=0.85]{OBSS-sheet-eng.pdf}
\end{center}

%\thispagestyle{plain}
%\begin{center}
%\begin{tikzpicture}[remember picture,overlay]
%\node[anchor=south west,inner sep=0pt] at ($(current page.south west)+(1cm,1cm)$) {
%	\includegraphics[scale=0.93]{OBSS-scheda-0.png}
%	};
%\end{tikzpicture}
%\end{center}
%\pagebreak
%\thispagestyle{plain}
%\begin{center}
%\begin{tikzpicture}[remember picture,overlay]
%	\node[anchor=south west,inner sep=0pt] at ($(current page.south west)+(1cm,1cm)$) {
%		\includegraphics[scale=0.93]{OBSS-scheda-1.png}
%	};
%\end{tikzpicture}
%\end{center}

\pagebreak


\section{My Options}\index{My Options}

\justify

I'm a Arbiter and as much as I've built OBSS around my preferences there are a few choices which I personally would do differently. Many of these choices are present in the manual as Options and they haven't become "official" in order not to deviate the game too much from the standard canons.
At my game table I usually use the following Options:

\begin{itemize}[leftmargin=*]

\item
\hyperlink{etadelpersonaggio}{Age of the Character} page \pageref{etadelpersonaggio}

\item
\hyperlink{successoparziale}{Partial Success}  page \pageref{successoparziale}

\item
\hyperlink{initiative variant}{Initiative variant}, if I have experienced players. Page \pageref{varianteiniziativa}


\item
\hyperlink{tirocriticovariante}{Critical Roll Variant} is up to the player whether to use it or not at the time of character creation. Page \pageref{tirocriticovariante}

\item \hyperlink{OpzionaleAzioniTiro}{Critical Shot Actions} suggested for experienced players, player choice. Pag. \pageref{OpzionaleAzioniTiro}

\item
\hyperlink{lunicaregola}{The Only Rule} I use if I am Arbiter of a group of beginners. Page \pageref{lunicaregola}

\item
\hyperlink{magiaspecialista}{Specialist} if I want to facilitate specialist high-level spellcasters.

\item
\hyperlink{abilitaiconiche}{Iconic Abilities} in case of long campaigns. Page \pageref{abilitaiconiche}

\item
 \hyperlink{droghe}{Drugs} only in the case of groups made up of mature and mentally grown-ups. Page \pageref{droghe}

\end{itemize}


\vfill
{\small
\begin{multicols}{2}

For me OBSS should be played in a straightforward way, without too many thoughts and bizarre projects. OBSS is not made to kill the characters but in the same way it does not facilitate their survival, everything is up to the Arbiter to decide how to play. The Arbiter, the style of the players and the interest of the group are the key to the game, OBSS wants to offer the framework, the tools, to play the adventure.

Try to emphasize the scenes, also be theatrical in the descriptions, remove the veneer from the clean and politically correct game. It always remains your world, your table and your game, try to give that immersiveness that is often a little lost in more modern systems.
When there is a fight let it be a fight! You must hear the clanging of weapons, the banging of Armour, the ozone in the air from lightning, the crackling burns of fireballs. An adventurer in OBSS is not a hero because he defeats the bad guys (maybe he is also for this) but he is a hero because he survived.

The first OBSS adventure is the equivalent of the initiation rite to challenge, fate and death. In the small village where a group of orcs has decided to raid the character who ends his first adventure and defeats the orcs will be remembered as a hero, one of the few who came back alive and not because he defeated the enemies. For the character it will be the spark for other adventures.

It will be the thirst for glory, wealth, power if not cruel fate that drags the characters into new dangerous adventures. As I have already said the characters in OBSS are not the chosen heroes, in fact, it is much more likely that they are scoundrels who want to survive and possibly get rich.

Create the party, and I don't mean just as a set of players, but as a set of characters as well. A group where people respect and trust each other (possibly...). Build adventures that involve everyone, where everyone can contribute. There must also be more \textit{sewn} adventures around a character but this should not \textbf{should} exclude others from participating, in the broadest term of the word.
Take advantage of these adventures to introduce the characters to each other, nothing unites more than the fear of dying!

Once the group is done, and it may even take time, then use the personal stories, clues and hypotheses created by the players to shape situations and events. Like a millstone this will continue to create situations, adventures and new plots to follow.

There could be some difficulties in creating the group, unfortunately it happens. Try to talk to the problem player. Try to figure out if it's his character that doesn't \textit{work} with the party or if it's the player who doesn't quite understand the mechanics of the party.

That's why I always suggest you do the so-called \textbf{Session Zero}, where as a Arbiter you will broadly outline what the cornerstones of the adventure are, what you expect from the characters, what are the basic moral rules to follow . There is nothing worse than a group of disjointed characters where everyone wants to do something different and doesn't care about the \textit{common goal}.

It is very important to understand what the players like, each person and group wants a certain style of play and it is correct to try to please them. If the group wants political adventures, romantic dramas try to make them find satisfaction in the adventure. If, on the other hand, they prefer to fight more then don't skimp on fights as long as they are consistent with the adventure itself.

Emphasize that you need to function as a set of players and characters in order to play well and enjoy it all. No player has to be on top of the others, only the Arbiter has the last word.

Finally, always be fair, for better or for worse. There will be more unfortunate sessions and others where the dice will find the right path, where the brilliant idea will save the group. Don't act as a Arbiter who \textbf{always and in any case} saves the characters, there can be help from time to time especially in the most unfortunate session, but respect the choices of the characters and the outcome of the dice. Remember that players have Fate Points to use unlike the poor monsters!.

And finally an obvious: \textbf{enjoy yourself}, everyone strive so that the session has that mixture of tension, fun and satisfaction. You are people who want to play, have fun and be together, never forget that.


\end{multicols}}

\pagebreak

\section{License}\index{License}

\medskip

\begin{center}
\textbf{Old Bell School System (OBSS)} is a trademark and property of Andres Zanzani, licensed under  \href{https://creativecommons.org/licenses/by-sa/4.0/legalcode}{Attribution-ShareAlike 4.0 International} - https://creativecommons.org/licenses/by-sa/4.0/legalcode
\end{center}

This is a human-readable summary of (and not a substitute for) the license.\medskip

\textbf{You are free to}:\medskip

\textbf{Share} — copy and redistribute the material in any medium or format\medskip

\textbf{Adapt} — remix, transform, and build upon the material for any purpose, even commercially. This license is acceptable for Free Cultural Works.\medskip

The \textbf{licensor} cannot revoke these freedoms as long as you follow the license terms.
Under the following terms:\medskip

\textbf{Attribution} — You must give appropriate credit, provide a link to the license, and indicate if changes were made. You may do so in any reasonable manner, but not in any way that suggests the licensor endorses you or your use.\medskip

\textbf{ShareAlike} — If you remix, transform, or build upon the material, you must distribute your contributions under the same license as the original.\medskip

\textbf{No additional restrictions} — You may not apply legal terms or technological measures that legally restrict others from doing anything the license permits.\medskip

\textbf{Notices}:
You do not have to comply with the license for elements of the material in the public domain or where your use is permitted by an applicable exception or limitation.
No warranties are given. The license may not give you all of the permissions necessary for your intended use. For example, other rights such as publicity, privacy, or moral rights may limit how you use the material.


\vfill

Any use of OBSS or its parts or ideas, please notify us in advance.

{\normalsize The images included in this manual are in the public domain or are unlicensed. If i mistakenly include copyrighted images, please report them for removal.}

\pagebreak

\invisiblesection{Mappe}\index{Mappe}


\begin{center}
	\includegraphics[width=\linewidth]{immagini/Curyan-zona0-reduced.png}\\

	\medskip

	\emph{Details of Curyan}
\end{center}

\pagebreak



\begin{center}
	\includegraphics[width=\linewidth]{immagini/Tiya-zona0-reduced.png}\\

	\medskip

	\emph{Details of Tiya}
\end{center}

You can download the complete maps of Curyan \href{https://github.com/buzzqw/TUS/blob/master/OBSS/immagini/Curyan.jpg}{Curyan} and Tiya \href{https://github.com/ buzzqw/TUS/blob/master/OBSS/images/Tiya.jpg}{Tiya} from github.

The complete map of the two continents is this \href{https://github.com/buzzqw/TUS/blob/master/OBSS/immagini/Mappacomplete.jpg}{Complete map}



{\scriptsize\printindex{Index}}

%\immediate\write18{./contaobss.sh > contaobssidx.txt}
%\immediate\openin\myscriptresult=./contaobssidx.txt
%\read\myscriptresult to \ScriptResult
%\immediate\closein\myscriptresult

\vfill

%Totale elementi nell'indice \ScriptResult

\TotalBox{OBSS}


{\scriptsize\printindex[Spells]}

%\immediate\write18{./contaspell.sh > contaspell.txt}
%\immediate\openin\myscriptresult=./contaspell.txt
%\read\myscriptresult to \ScriptResult
%\immediate\closein\myscriptresult

\vfill

%Totale elementi in questo indice \ScriptResult

\TotalBox{Spells}

{\scriptsize\printindex[MagicItem]}


%\immediate\write18{./contaomagici.sh > contaomagici.txt}
%\immediate\openin\myscriptresult=./contaomagici.txt
%\read\myscriptresult to \ScriptResult
%\immediate\closein\myscriptresult

\vfill

%Totale elementi in questo indice \ScriptResult

\TotalBox{MagicItem}

{\scriptsize\printindex[Monsters]}

\vfill

\TotalBox{Monsters}

%Totale elementi in questo indice \ScriptResult

%\immediate\write18{./contamostri.sh > contamostri.txt}
%\immediate\openin\myscriptresult=./contamostri.txt
%\read\myscriptresult to \ScriptResult
%\immediate\closein\myscriptresult

\end{document}
